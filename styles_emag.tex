\usepackage[usenames, dvipsnames]{xcolor}
\usepackage[utf8x]{inputenc}
\usepackage[T1]{fontenc}
\usepackage{tikz}
\usepackage{pgfplots}
\pgfplotsset{compat=1.9}
\usetikzlibrary{calc,arrows,fadings,decorations.pathreplacing,decorations.markings,patterns,shapes.geometric}
\usepackage{fp}
\usepackage{yhmath}% pour les longueurs d'arc
\usepackage{verbatim}
\usepackage[active,tightpage]{preview}
\PreviewEnvironment{tikzpicture}
\setlength\PreviewBorder{5pt}

%------ mes couleurs -------
\definecolor{fond}{rgb}{1,0.9,0.75}
\definecolor{filet}{rgb}{1,0.6,0}
\definecolor{monOrange}{RGB}{255,157,0}
\definecolor{monBleu}{RGB}{54,134,193}
\definecolor{monCyan}{RGB}{74,181,247}
\definecolor{monGris}{RGB}{100,100,100}


%------ styles tikz ---------------
\colorlet{darkblue}{blue!50!black} 
\tikzset{>=stealth,inner sep=0pt, outer sep=2pt,}
\tikzset{tiret/.style={gray,dashed}}
\tikzset{doublefleche/.style={|<->|,>=stealth,thin}}
\tikzset{titre/.style={inner sep=0pt, outer sep=0pt,above right,text justified,fill=orange!50}}
\tikzset{bloc/.style={rounded corners=4pt,color=white,ball color=purple,smooth}}
\tikzset{force/.style={->,ultra thick,rounded corners=4pt,color=blue,smooth,line cap=round}}
\tikzset{vecteur/.style={->,thick,color=black,smooth}}
\tikzset{verre/.style={draw=SkyBlue,fill=SkyBlue!30}}
\tikzset{axis/.style={thin,gray}}
\tikzset{figure/.style={thick,color=#1,fill=purple, opacity=0.5}}
\tikzset{ressort/.style={very thick,black,smooth}}
\tikzset{eau/.style={draw=black,fill=blue,opacity=0.5}}
\tikzset{trajet/.style={thick,color=blue}}
\tikzset{verre/.style={draw=SkyBlue,fill=SkyBlue!30}}
%-----------------------------------

%-------- patatoide ----------
\newcommand{\patate}[1][fill=white] 
{\draw [#1][preaction={fill=white}] (0,0) .. controls +(0.1,0.2) and +(0.3,0.3) .. (1,0) .. controls +(-0.3,-0.3) and +(-0.05,0.15) .. (1,-1) .. controls +(0.2,-0.6) and +(0.1,-0.2) .. (0,-1) .. controls +(-0.1,0.15) and +(-0.2,-0.4) .. (0,0);}

% --- commande \maSphere{options}{rayon} ---
\newcommand{\maSphere}[2]{
% \filldraw[ball color=Maroon!10,#1] (0,0) circle(#2);
\filldraw[ball color=monBleu!50,#1] (0,0) circle(#2);
\draw[dashed,#1] (0,0)circle({#2/4} and #2);
\draw[dashed,#1] (0,0)circle(#2 and {#2/4});
\draw[#1] (#2,0) arc(0:-180:#2 and {#2/4});
\draw[#1] (0,#2) arc(90:270:{#2/4} and #2);
}

%------ aimant -------
\newcommand{\aimant}[1]{
\draw[fill, color=red,ultra thick,#1] (0,0.2) rectangle(1,-0.2) node[color=white,midway]{\tiny S};
\draw[fill, color=black,ultra thick,#1] (1,0.2) rectangle(2,-0.2) node[color=white,midway]{\tiny N};}

% --- support{rayon}{hauteur} ---
\newcommand{\support}[2]{
 \path [top color=black!25,bottom color=white](0,.05*#1/3) ellipse [x radius = #1-.05,y radius = #1/3-.05*#1/3];
 \path [left color=black!25,right color=black!25,middle color= white](-#1,0) -- (-#1,-#2) arc (180:360:#1 and #1/3)  -- (#1,0) arc (360:180:#1 and #1/3);
\foreach \r in {225,315}
	\foreach \i [evaluate = {\s=30}] in {0,2,...,30}
	\fill [black, fill opacity = 1/50](0,0) -- (\r+\s-\i:#1 and #1/3) -- ++(0,-#2) arc (\r+\s-\i:\r-\s+\i:#1 and #1/3) -- ++(0,#2) -- cycle;
\foreach \r in {45,135}
	\foreach \i [evaluate = {\s=30}] in {0,2,...,30}
	\fill [black, fill opacity = 0.02](0,0) -- (\r+\s-\i:#1 and #1/3) arc (\r+\s-\i:\r-\s+\i:#1 and #1/3)  -- cycle;
}

% style des composants
\tikzset{composant/.style={decoration={coil,aspect=0.5,segment length=2mm,amplitude=2mm}}}

% commande \resistance  : résistance de longueur unité, centrée en (0,0).
\newcommand{\resistance}[2]
{
\draw[#1,composant] (-0.5,0.25)--++(1,0)--++(0,-0.5)--++(-1,0)--cycle;
\draw[#1] (0,0) node{#2};
}
% commande \resistanceH  : résistance de longueur unité, centrée en (0,0).
\newcommand{\resistanceH}[2]
{
\draw[#1,composant,fill=white] (-0.5,3pt)--++(1,0)--++(0,-6pt)--++(-1,0)--cycle;
\draw[#1] (0,0) node[above=3pt]{#2};
}
% commande \resistanceV  : résistance de longueur unité, centrée en (0,0).
\newcommand{\resistanceV}[2]
{
\draw[#1,composant,fill=white] (-3pt,0.5)--++(0,-1)--++(6pt,0)--++(0,1)--cycle;
\draw[#1] (0,0) node[right=3pt]{#2};
}
% \condoH{options}{nom}{valeur} : condensateur horizontal de longueur unité, centrée en (0,0)
\newcommand{\condoH}[3]%
{
\fill[#1,white] (-3pt,-0.5)rectangle(3pt,0.5);
\draw[#1,composant,ultra thick] (-3pt,-0.4)--++(0,.8);
\draw[#1,composant,ultra thick] (3pt,-0.4)--++(0,.8);
\draw[#1] (0,-0.4) node[below]{#3};
\draw[#1] (0,0.4) node[above]{#2};
}

% \condoV{options}{nom}{valeur} : condensateur vertical de longueur unité, centrée en (0,0)
\newcommand{\condoV}[3]%
{ 
\fill[#1,white] (-0.5,-3pt)rectangle(0.5,3pt);
\draw[#1,composant,ultra thick] (-0.4,-3pt)--++(.8,0);
\draw[#1,composant,ultra thick] (-0.4,3pt)--++(.8,0);
\draw[#1] (0.4,0) node[right=3pt]{#3};
\draw[#1] (-0.4,0) node[left]{#2};

}

% \bobineH{options}{nom}{valeur} : bobine horizontal de longueur unité, centrée en (0,0)
\newcommand{\bobineH}[3]%
{
\draw[#1,draw=none, fill=white] (-5mm,-3pt) rectangle (5mm,3pt);
\draw[#1,composant,decorate] (-0.5,0)--++(1,0)node[midway,above=5pt]{#2} node[midway,below=3pt]{#3};
}

% \GBFV{options}{valeur} : GBF vertical de longueur unité, centrée en (0,0)
\newcommand{\GBFV}[2]%
{ 
\draw[#1] (0,-0.5) node {\tiny $\bullet$} --++(0,1)node {\tiny $\bullet$};
\draw[#1,composant,fill=white] (0,0) circle(0.3)node{$\sim$} node[right=4mm]{#2};
}
% \FEMV{options}{valeur} : source de tension vertical de longueur unité, centrée en (0,0)
\newcommand{\FEMV}[2]%
{ 
\draw[#1] (0,-0.5) node {\tiny $\bullet$} --++(0,1)node {\tiny $\bullet$};
\draw[#1,composant,fill=white] (0,0) circle(0.3)node[left=3mm]{#2};
\draw[#1,composant,->] (0,-2mm)--(0,2mm);
}






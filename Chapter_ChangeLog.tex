Ceci est un journal de changement (log) détaillé du présent livre pour les personnes intéressées à voir comment ce dernier a évolué (pour rester informé des nouvelles versions, envoyez-nous un e-mail pour vous abonner à la newsletter):
	\begin{itemize}
		\item \textbf{Mai 2002}
		\begin{itemize}[noitemsep]
			\item Définitions (sciences, loi, théorème, postulat, axiome, corollaire,...)
			\item Opérations d'addition, soustraction, multiplication, division (puissance)
			\item Les nombres (entiers, relatifs, réels, fractionnaires, complexes, algébriques, abstraits,...)
			\item Domaine de définition d'une variable
			\item Polynômes arithmétiques
			\item Valeur absolue
			\item Relations binaires, relations d'ordre
			\item Fonction Gamma d'Euler, constante d'Euler
			\item Equation d'onde électromagnétique, vitesse de l'onde, vitesse de la lumière, énergie véhiculée
			\item Introduction à l'optique
			\item Règle de trois, pourcentages
			\item Quantités nettes, prix de revient d'achat, indices, prix de ventes, prix de ventre brut, bénéfice net, nombre clé, actifs
			\item Intérêts simples et composés, versements tardifs et précoces
			\item Gestion de portefeuilles (critères de décision, goodwill, return on investisment)
			\item Modèle de Markowitz (fonction d'utilité, élaboration du critère de choix)
			\item Bourse
			\item Fréquence Plasma
			\item Probabilités (univers, événements, axiomes)
			\item Analyse combinatoire 
			\item Statistiques (variables discrètes, variables continues, écart-type, variance et covariance)
			\item Fonctions de distribution (fonction discrète uniforme, de Bernoulli, binomiale, hypergéométrique, multinomiale, de Poisson, de Gauss-Laplace, de Cauchy, bêta, gamma, Khi-deux, loi de Student)
			\item Estimateurs, la corrélation
			\item Matrice des covariances
			\item Tests statistiques d'adéquation 
		\end{itemize}
		\item \textbf{Juin 2002}
			\begin{itemize}[noitemsep]
				\item Ensembles (théorie Zermelo-Fraenkel, l'inclusion, la complémentarité, l'intersection, la réunion, la différence, le produit, l'ensemble vide
				\item Moyennes arithmétique, harmonique, géométrique, quadratique
				\item Univocités
				\item Fonctions (logarithmes et exponentielles)
				\item Nombre d'or
				\item Introduction aux séries de Fourier
				\item Loi de Coulomb, champ électrostatique, potentiel électrostatique
				\item Théorème d'Ampère
				\item Loi de Biot et Savart, champ magnétique, induction magnétique
				\item Equations de Maxwell
				\item Dimensions
				\item Formes géométriques unidimensionelles, bidimensionelles, tridimensionelles
				\item Théorème de Pythagore
				\item Systèmes d'unités en physique
				\item Principe de moindre action et de conservation de l'énergie
				\item Positions, vitesse et accélération
				\item Equation de continuité
				\item Equation de Bernoulli
				\item L'effet Doppler
				\item Relativité Restreinte, principe d'invariance, transformations de Lorentz (temps, longueurs, addition des vitesses, augmentation de la masse, espace-temps de Minkowski
				\item Relativité générale (ligne d'univers, événement ponctuel, particules inertielles, cônes nuls, vecteurs d'espace-temps, espaces plans et courbes, tenseur métrique)
				\item Principes d'incertitudes de Heisenberg, équation de Schrödinger, densité de probabilité
				\item Introduction aux supercordes
			\end{itemize}
		\item \textbf{Juillet 2002}
			\begin{itemize}[noitemsep]
				\item Création d'une table des matières
				\item Représentation des fonctions
				\item Polynômes et zéros de polynômes du deuxième degré
				\item Opérateur de champs vectoriels et scalaires (gradient, nabla, divergence, rotationnel, Laplacien)
				\item Analyse vectorielle (notion de flèche, ensemble des vecteurs, multiplication par un scalaire, espace vectoriel, combinaisons linéaires, familles génératrices, bases d'un espace vectoriel)
				\item Calcul Tensoriel (convention d'Einstein, symbole de Kronecker, symbole d'antisymétrie)
				\item Notations de sommation (Sigma), notation de multiplication (Pi majuscule)
				\item Axiomes pour l'ensemble des réels
				\item Inégalités
				\item Mouvements circulaires et relatifs
				\item Forces d'inertie (Force de coriolis)
				\item Equation de Drake
			\end{itemize}
		\item \textbf{Août 2002}
			\begin{itemize}[noitemsep]
				\item Nouveaux personnages (Hilbert, Riemann, Legendre)
				\item Nouveau chapitre "Humour"
				\item Introduction à la Topologie
				\item Postulats d'Euclide
				\item Plan de Gauss
				\item Formule de Moivre
				\item Transformations dans le plan complexe
				\item Changement de base et produit scalaire tensoriel
				\item Composantes covariantes et contravariantes
				\item Transformations relativiste de la quantité de mouvement
				\item Préfixe des multiples et sous-multiples des unités
				\item Contrôles de qualité (probabilités), courbe d'efficacité, valeur de niveau de qualité acceptable (NQA)
				\item Lois de Kepler
				\item Démonstration de la force gravitationelle classique à partir des lois de Kepler
				\item Démonstration de la loi des Aires (deuxième loi de Kepler)
				\item Moment de force
				\item Moment cinétique
				\item Théorème du moment cinétique
				\item Equation de Newton-Poisson
				\item Modèle de l'atome de Thomson, de Bohr, de Bohr-Sommerfeld
				\item Spectre de l'hydrogène
				\item Hypothèse du Neutron
				\item Nombres quantiques
				\item Principes d'exclusion de Pauli
				\item Analyse classique de l'équation de Schrödinger pour le puits de potentiel rectangulaire idéal
				\item Lois de Newton
				\item Démonstration de la troisième loi de Newton à partir du principe de moindre action
				\item Vitesse de libération
				\item Définition du "Scientisme"
				\item Variation de l'accélération gravifique dans et à l'extérieur d'un astre homogène sphérique
				\item Principe de moindre action et physique quantique (limite semi-classique)
				\item Balistique (portée maximale, parabole de sûreté)
				\item Introduction à la physique nucléaire
				\item Nombre atomique, nombre de masse
				\item Radioactivité, activité, filiation, isotopes, isotones, nucléides, datation
				\item Système de masse atomique (UMA)
				\item Défaut de masse
				\item Fusion et fission nucléaire
				\item Désintégration alpha, beta (moins et plus), capture électronique, émission gamma
			\end{itemize}
		\item \textbf{Septembre 2002}	
			\begin{itemize}[noitemsep]
				\item Biographies de Dalton, Boltzmann et Broglie
				\item Principe du bon ordre
				\item Propriété Archimédienne
				\item Principe d'induction
				\item Divisibilité (division euclidienne)
				\item Nombres congrus
				\item Preuve par neuf
				\item Bases de nombres
				\item Définition de l'utilitié des séparateurs de milliers
				\item Priorité des parenthèses, crochets et accolades
				\item Priorité des opérandes
				\item Théorie de la démonstration (intro)
				\item Notions de termes, formules et démonstrations
				\item Définition des langages, symboles, relations et fonctions
				\item Définition des fonctions périodiques, composées, élémentaires, rationnelles entières, fractionnelles, irrationnelles, algébriques et transcendantes
				\item Généralisation de l'algèbre élémentaire
				\item Dimensions d'un espace vectoriel, pronlogement d'une famille libre, rang d'une famille finie
				\item Hyperplan vectoriel, sommes directes
				\item Définition rigoureuse des notions de ligne, surface (plan) et volume
				\item Droites sécantes, demi-droites, segments, partie aliquote
				\item Axiome de continuité de la droite
				\item Déplacement et retournement de plan
				\item Angles, unités, mesures, côtés de l'angle, angles saillants, angles plats, angles égaux
				\item Axiome de continuité du plan
				\item Angles droites, aigus, obtus, supplémentaires, complémentaires
				\item Droites perpendiculaires, bissectrice d'un angle
				\item Définition du travail et de l'énergie: Thèroème de l'énergie cinétique et potentielle (travail moteur et résistant)
				\item Notion de champ conservatif
				\item Conservation de l'énergie et de la quantité de mouvement
				\item Théorème du centre de masse
				\item Transformation relativiste de la force
				\item Transformées relativistes des champs électrique et magnétique
				\item Masse limite de Chandreskhar (limite d'effondrement de naines blanches)
				\item Défintions de l'optique, généralisation de la loi de la réfraction
				\item Condition de normalisation de de Broglie, état liés et non liés
				\item Oscillateur harmonique
				\item Chimie quantique, vibrations moléculaire
			\end{itemize}
		\item \textbf{Octobre 2002}
			\begin{itemize}[noitemsep]
				\item Ajout de biographies sur Cauchy, Neumann, Bessel et Archimède
				\item Nouvelle section sur les "Méthodes Numériques"
				\item Plus grand commun diviseur, plus petit commun multiple
				\item Règle des signes (...)
				\item Démonstration de l'irrationalité d'un nombre
				\item Introduction aux progressions arithmétiques, harmoniques et géométriques
				\item Limite et continuité des fonctions
				\item Définition des espaces affines
				\item Introduction aux tenseurs euclidiens et leurs propriétés
				\item Triangles et propriétés des triangles
				\item Définition des systèmes thermodynamiques
				\item Définition de la masse réduite
				\item Fonctions de Bessel
				\item Définition du moment d'inertie
				\item Introduction à la théorie quantique des champs
				\item Introduction à la radioprotection, formule de Bethe-Bloch
				\item Effet tunnel
				\item Introduction au formalisme de Dirac
				\item Algorithme d'Heron et d'Archimède
				\item Introduction aux ensembles fractals
				\item Introduction à la théorie des jeux (jeux coopératifs, gains, matrice des gains, formes extensives, optimums de Pareto, équilibre de Nash, jeux évolutionnaires)
			\end{itemize}
			\item \textbf{Novembre 2002}
				\begin{itemize}[noitemsep]
				\item Théorème fondamental de l'arithmétique
				\item Introduction à la crypthographie (RSA, DES, MD5, SHA-1)
				\item Fonction phi d'Euler
				\item Petit théorème de Fermat
				\item Introduction aux dimensions topologiques et d'homothétie
				\item Cosinus directeurs
			\end{itemize}
			\item \textbf{Décembre 2002}
				\begin{itemize}[noitemsep]
				\item Ajout de biographies sur Nash, Cartan, Lucas, et Lie
				\item Développement détaillé de la fonction "Entière"
				\item Superposition linéaire des états quantiques (cohérence quantique)
				\item Définition des fonctions Lipschitziennes et des fonctions contractantes
				\item Définition d'une suite convergente de Cauchy
				\item Théorème du point fixe (utilisé dans les fractals, méthodes de Newton et bcp d'autres)
				\item Définitions de la réalité, du problème, de la théorie et essai sur la réalité
				\item Définition de de l'espace euclidien et de l'espace affine euclidien
				\item Définition des notions de propriété dans le domaine de la chimie
			\end{itemize}
			\item \textbf{Janvier 2003}
			\begin{itemize}[noitemsep]
				\item Théorème de Pascal
				\item Poussée d'Archimède
				\item Introduction simple aux différentes symétries en physique (temporelle, spatiale)
				\item Introduction simple aux différentes transformation dans le plan (translation, homothétie, réflexion, isométrie, rotation)
				\item Définition d'une application réciproque et composée
				\item Trigonométrie (introduction, relations remarquables, trigonométrie sphérique)
				\item Signature d'un espace vectoriel
				\item Méthodes d'orthogonalisation de Schmidt, changements de bases, espaces associés de Fourier
				\item Tout (ou presque) sur la trigonométrie plane et sphérique
				\item Trajectoires d'orbitales képleriennes
				\item Introduction au modèle monétaire néo-classique (loi de say, postulat d'homogénéité)
				\item Algèbre de Boole (propriétés et théorèmes simples)
				\item Refonte du chapitre de physique quantique (ordre de présentation des sujets)
				\item Démonstration de l'équation d'évolution de Schrödinger
				\item Démonstration de l'équation d'évolution relativiste de Schrödinger
				\item Introduction à la théorie de l'anti-matière
			\end{itemize}
		\item \textbf{Février 2003}
				\begin{itemize}[noitemsep]
				\item Démonstration gradient, divergence, rotationnel et Laplacien en coordonnées cartésiennes, polaires, cylindriques et sphériques
				\item Développements mathématiques complets des expressions de la vitesse et de l'accélération en coordonnées cartésiennes, polaires, cylindriques et sphériques
				\item Démonstration de l'invariance relativiste de la charge électrique (équation de conservation de la charge électrique... je pense)
				\item Démonstration de l'existence d'anti-particule de charge opposées
				\item Introduction à la théorie de Jauges (quadripotentiel, jauge de Lorenz, jauge de Coulomb, d'Alembertien)
				\item Introduction au formalisme Lagrangien et Hamiltonien (coordonnées généralisées, espaces des configurations, équation d'Euler-Lagrange, formalisme canonique, transformation de Legendre, crochet de Poisson)
				\item Définition rigoureuse du principe de moindre action
				\item Espaces tensoriels et définition
				\end{itemize}
		\item \textbf{Mars 2003}
			\begin{itemize}[noitemsep]
				\item Ajout de biographies sur Lorentz, Hermann, Ricci-Curbastro et Levi-Civita
				\item Définition du produit cartésien et extension du cadre d'application des cardinaux
				\item Démonstration de l'inégalité de Cauchy-Schwarz
				\item Démonstration de l'inégalité triangulaire
				\item Définitions du produit vectoriel et du produit mixte
				\item Démonstration de la forme condensée la somme des n premiers nombres entiers
				\item Démonstration de la validité de l'intégration par partie
				\item Définition de la structure algébrique vectorielle et d'un algèbre
				\item Définition d'un homorphismes, isomorphisme, endomorphisme, automorphisme
			\end{itemize}
		\item \textbf{Avril 2003}
			\begin{itemize}[noitemsep]
				\item Théorie de l'équilibre de Cournot (concurrence)
				\item Modèle de Wilson (gestion de stocks)
				\item Mathématique des phaseurs
				\item Modèle relativiste de l'atome de Sommerfeld
				\item Résolution analytique de l'équation de Schrödinger
				\item Principes d'incertitudes quantiques de Heisenberg
				\item Formalisme Lagrangien de la physique quantique des champs
				\item Ajout de biographies sur Göpper-Meyer, Hideki, Nöther et Cournot 
				\item 10 nouveaux liens vers des pages web intéressantes (associations + mathématiques)
			\end{itemize}
		\item \textbf{Mai 2003}
			\begin{itemize}[noitemsep]
				\item Ajout de biographies sur Bell, Ramanujan et Landau 
				\item Démonstration de la précession du périhélie des orbites d'astres couplés et de charges couplées
				\item Définition et développements relatifs au théorème du Viriel
				\item Calcul de l'énergie potentielle d'un sphère de matière (température interne des étoiles)
				\item Définition des nombres premiers et démonstration qu'ils sont en nombre infini 
				\item Définition d'un anneau intégralement clos
				\item Démonstration qu'un nombre rationnel est un nombre algébrique si et seulement si c'est un entier relatif. 
				\item Définition d'une application multi-linéaire (ou morphisme d'espace vectoriel) 
				\item Définition d'une partition d'un ensemble et d'une classe d'équivalence. 
				\item Descriptions des opérations ensemblistes d'absorption et d'idempotence. 
				\item Définitions et exemples de diagrammes sagittals.
				\item Définition d'un magma et d'un monoïde
				\item Pseudo-démonstrations des structures aglébriques des ensembles fondamentaux de l'arithmétique
				\item Développement de la théorie du moment cinétique en physique quantique ondulatoire
				\item Définition d'une équation diophantienne et enoncé du grand théorème de Fermat
			\end{itemize}
		\item \textbf{Octobre 2003}
			\begin{itemize}[noitemsep]
				\item Ajout de biographies sur Abel, Banach, Boole, Bose, Luitzen Egbertus, Clausius, Cayley, Curie, Connes, Dirichlet, Frege, Gibbs, Picard, Erdös, Grothendieck,Hamilton, Hausdorf, Heaviside, Helmholtz, Hermite, Hoyle, Jacobi, Klein, Kronecker, Langevin, Lee, Lobatchevski, Möbius, Monge, Poisson, Schwartz, Shannon, Thom, Van Der Waals, Viète, Weinberg, Witten, Gamow, Sturm, Liouville, Clairaut, Teller
				\item Définition des espaces ponctuels en mécanique classique, des conventions d'écriture et des changements de repères.
				\item Théorie mathématique de la perspective projective à points de fuite et définition des perspectives techniques isométriques et cavalières.
				\item Définition simpliste du concept de "dérivée" en analyse fonctionnelle. Démonstration de quelques dérivées courantes (polynômes, fonctions composées, fonctions réciproques, cosinus, sinus, arcsinus, arcosinus, quotient de deux fonctions).
				\item Méthodes numériques: définition mathématique de la complexité d'un algorithme et recherches élémentaires d'optimisation
				\item Méthode de calcul du nombre e
				\item Méthode de résolution des systèmes de N équations linéaires à N inconues par la méthode du pivot 
				\item Recherche des racines de fonction par la méthodes des parties proportionnelles, de la bissection, de la sécante (regula falsi) et de la méthode de Newton - Calculs des aires et intégrales par la méthode des sommes de riemann 
				\item Exposition de la méthode de Monte-Carlo pour le calcul d'intégrales, de $\pi$ et de recherche de racines (dichotomie) 
				\item Développement mathématique de l'algorithme du simplexe utilisé dans le cadre des recherches opérationnelles (programmation linéaire) 
				\item Calcul tensoriel: la contraction des indices - de la définition et des propriétés de quelquens tenseurs particuliers (tenseurs symétriques, antisymétriques, tenseur fondamental) 
				\item Coordoonnées curvilignes (détermination de la métrique et élément linéaire d'un espace sphérique et cartésien et du plan en coordonnées polaires) 
				\item Symboles de christoffel (de première et de deuxième espèce).
				\item Démonstration du théorème de Cantor-Bernstein
				\item Détermination du lagrangien libre généralisé en Relativité Générale
			\end{itemize}
		\item \textbf{Décembre 2003}
			\begin{itemize}[noitemsep]
				\item Ajout des biographes sur Kirchhoff, Markowitz, Cox Merton, Sholes, Sharpe, Lindemann, Bachelier et Stefan 
				\item Démonstration d'une des formules de Stirling
				\item Calcul de la température de surface d'une étoile
				\item Calculs des propriétés temporelles des titres de valeurs
				\item Démonstration des séries de Taylor et Maclaurin (limité et non limité)
				\item Définition du reste de Lagrange et des critères de divergences d'Alembert, de Cauchy, Test de l'intégrale, Convergence absolue
				\item Développements relatifs aux définitions de la luminosité, éclat, magnitude apparente et absolue des étoiles et calcul de la distance des céphéides
				\item Définitions de l'angle solide, l'angle solide de révolution et l'angle solide élémentaire
				\item Introduction à la photométrie: définition des grandeurs photométriques, photoniques et S.I., définitions et développements relatifs à l'intensité lumineuse, le flux énergétique (avec démonstration de la loi de Beer-Lambert), l'émittance énergétique, luminance énergétique (avec démonstration de la loi de Lambert), loi de Kirchhoff (vois le chapitre d'optique dans la section d'électromagnétisme)
				\item Démonstration de la loi de Stefan et de la loi de Stefan-Boltzmann
				\item Résolutions des équations du troisième degré par radicaux (méthode de Cardan) et développements de la résolution des équations du second degré dans l'ensemble des complexes
				\item Définition du concept d'équations et inéquations
				\item Détermination de l'équation cartésienne du plan, d'un droite (dans l'espace), d'un cône, d'une sphère
				\item Détermination de l'équation de Black \& Sholes (définition du postulat d'efficience du marché) et présentation des processus de Wiener et du lemme d'Ito ainsi que du mouvement Brownien (marche au hasard)
			\end{itemize}
		\item \textbf{January 2004}
			\begin{itemize}[noitemsep]
				\item Définition de l'intégrale indéfinie
				\item Modèle cosmologique newtonien (sans la constante cosmologique)
				\item Enoncé des axiomes de Peano
				\item Introduction à l'algèbre linéaire (méthode de réduction de Gauss, opérations élémentaires entre matrices)
				\item Enoncé des 5 axiomes d'Euclide et des 5 groupes d'axiomes en géométrie
				\item Démonstration des relations de calcul du perimètre, de la surface et du barycentre du carré, du rectangle, et du triangle. Démonstration des relations du calcul des volumes et des surfaces du tore, de la sphère, de l'ellipsoïde, du cylindre et du cône
				\item Définition du barycentre (centre de gravité) et démonstration de quatre propriétés y relatives
				\item Démonstration de la décomposition en fonction impaire et paire de toute fonction 
				\item Définition des fonctions trigonométriques hyperboliques et énumération des relations et propriétés remarquables y relatives
				\item Introduction à la géométrie différentielle (définition d'une géométrie riemanienne, trièdre de Frenet, nappe paramétrée, ...)
				\item Introduction à la théorie des graphes (ponts de Königsberg)
				\item Définition d'un espace topologique et de Hausdorff, définition d'un espace métrique/ultra-métrique ainsi que des distances associées (hölderiennes, discrète, équivalentes, ...), définition d'une fonction lipschitzienne (et des isométries associées), définition des ensemble ouverts et fermés (boules ouvertes, fermées, sphères, adhérence, diamètre, excès de Hausdorff), définition des diamètres définition des distances ensemblistes (gap), définition d'une variété/carte/atlas et homéomorphisme différentiel.
			\end{itemize}
		\item \textbf{Avril 2004}
			\begin{itemize}[noitemsep]
				\item Démonstration du théorème de Guldin
				\item Démonstration du théorème de König de l'énergie cinétique et du moment cinétique
				\item Présentation et démonstration de techniques de calcul pour les moments d'inertie: théorème d'Huygens-Steiner, moment d'inertie polaire, tenseur d'inertie, théorème d'Huygens-Steiner généralisé
				\item Définition de la "Puissance" et du "Rendement" et démonstration du calcul de la puissance d'une machine tournante
				\item Démonstration de loi d'entropie thermodynamique de Boltzmann ainsi que des distributions statistique suivantes: vitesses de Maxwell, Maxwell-Boltzmann, Fermi-Dirac, Bose-Einstein
				\item Démonstration de quelques moments d'inertie principaux des corps suivants: tore, boule, boule creuse, cône, plaque rectangulaire, tube
				\item Introduction à l'optique ondulatoire: éconcé du principe de Huygens, démonstration de la loi de Malus, développement du modèle de la diffraction de Fraunhofer dans le cas d'un fente rectangulaire. Définition et détermination du pouvoir de résolution d'une fente rectangulaire simple.
				\item Démonstration de la provenance et de la solution de la non moins fameuse équation différentielle de Bessel d'ordre n
				\item Démonstration de la fonction d'onde d'une corde tendue et d'une membrane circulaire tendue
				\item Démonstration de la loi de Planck et démonstration des approximations connues: première loi de Wien, loi de Rayleigh-Jeans.
				\item Démonstration de la loi de déplacement (deuxième loi de Wien) et de la loi de Stefan-Boltzmann en passant par la loi de Planck et détermination de la constante de Stefan-Boltzmann
				\item Démonstration des dimensions de Planck: longueur de Planck, masse de Planck, densité de Planck, temps de Planck, énergie de Planck
			\end{itemize}
		\item \textbf{Juillet 2004}
			\begin{itemize}[noitemsep]
			\item Démonstration de l'origine physique de la chaleur
				\item Démonstration du théorème de Toricelli, de l'effet Venturi et de la loi de Poiseuille
				\item Démonstration de l'effet Compton relativiste
				\item Démonstration de l'existence du rayonnement fossile de l'Univers
				\item Introductions aux ensembles quotients (en l'occurrence $\mathbb{Z}/n\mathbb{Z}$)
				\item Démonstration des transformations de Lorentz des vitesses et des accélérations
				\item Détermination du lagrangien relativiste d'un système libre
				\item Démonstration et définition du théorème de Ricci, de la dérivée covariante, de l'identité de Ricci, du tenseur de Riemann-Christoffel, du tenseur de Ricci, du scalaire de Ricci, des deux identités de Bianchi et enfin du tenseur d'Einstein
				\item Définition mathématique de la capacité et détermination de l'expression d'une capacité plane parallèle 
				\item Détermination de l'énergie potentielle électrostatique
				\item Démonstration de la valeur du champ et potentiel électrique d'un fil rectiligne infini.
				\item Détermination des propriétés fondamentales des dipôles électriques tel que le moment dipolaire rigide, la présentation du moment dipolaire induite, les liaisons hydrogènes, les forces de Van der Waals, etc.
				\item Définition du principe de symétrie de Curie et énoncé des 6 propriétés en découlant
				\item Définition des pseudo-vecteurs
				\item Démonstration du calcul du volume de la pyramide, du paraboloïde, du tetraèdre et de l'ocataèdre ainsi que écritue du volume du cube et de parallélépipéde
				\item Détermination du champ magnétique produit par un solénoïde toroïdal, d'un solénoïde rectiligne infini et d'une boucle de courant
				\item Détermination du champ magnétique produit par un dipôle magnétique et définitions fondamentales des propriétés des matériaux magnétiques 
				\item Calcul du rayon de Larmor et de la pulsation gyromagnétique dans un cadre non relativiste
				\item Détermination du lagrangien du champ électromagnétique et par extension dans l'approximation non relativiste du tenseur du champ électromagnétique
				\item Introduction aux calculs du rayonnement émis par une charge accélérée (rayonnement synchrotron, potentiels retardés de Liénard-Wiechert)
				\item Calcul des valeurs des résistances et capacités en série. 
				\item Différence entre le potentiel électrique et le potentiel électromoteur
				\item Démonstration de la loi de Faraday et définition de la "self"
				\item Démonstrations des formules de Descartes pour les surfaces sphériques concaves et convexes réfringent et non réfrignent ainsi que pour les lentilles réfringentes. 
				\item Définition du stigmatisme et démonstration que la parabole est rigoureusement stigmatique
				\item Démonstration des formules de Descartes pour les lentilles minces et détermination de loi de conjugaison
				\item Définition de la dioptrie et explication des différentes handicaps visuels
			\end{itemize}
		\item \textbf{Septembre 2004}
			\begin{itemize}[noitemsep]
				\item Énonce du principe de Mach
				\item Présentation de l'effet photoélectrique et détermination de la loi physique le régissant. Démonstration par l'exemple que la lumière peut être vue à la fois comme un corpuscule ou comme une onde
				\item Démonstration du théorème de la classe monotone
				\item Détermination détaillée de l'équation de Klein-Gordon généralisée (particule relativiste dans un champ magnétique) ainsi que de l'équation de Dirac libre avec laquelle sont exposées les solutions explicites de Pauli (particules, antiparticules). 
				\item Détermination du rayon de l'atome en utilisant la diffusion de Rutherford-Coulomb (in extenso: détermination de la section efficace de Rutherford)
				\item Présentation des interactions macroscopiques et microscopiques du rayonnement X et gamma avec la matière (étude qui comporte les détails de la matérialisation du photon en une paire électron-positron). 
				\item Introduction aux spineurs en définissant.
				\item Définition des propriétés opératoires des matrices, des matrices remarquables, des déterminants, des vecteurs et valeurs propres dans le chapitre d'algèbre linéaire. 
				\item Enoncés des postulats de la physique quantique ondulatoire
- Détermination des orbitales de l'atome hydrogénoïde
			\end{itemize}
		\item \textbf{Novembre 2004}
			\begin{itemize}[noitemsep]
				\item Exposé et démonstration du théorème de Noether
- Enumération de quelques constantes physiques, chimiques, et astronomiques
				\item Ajout de biographies sur Smith, Say, Malthus, Keynes, Walras et Pareto
				\item Introduction à la théorie de la spéculation: calcul de l'espérance mathématique prévisionnelle d'un actif financier
				\item Introduction à la théorie de préférence (modèle d'Arrow-Debreu). En d'autres termes: étude de l'utilité des paniers des agents économiques et des courbes d'indifférences
				\item Exposé des solutions de l'équation de Black \& Scholes et remarques sur le delta
				\item Démonstration de l'équation de parité Call-Put
				\item Modèle de détermination du stock intial (optimal) dans la cadre des technique de gestion de production 
			\end{itemize}
		\item \textbf{Janvier 2005}
			\begin{itemize}[noitemsep]
				\item Ajout de biographies sur Penrose, Hawking, Turing et Marx
				\item Développement de la version "duaire" des équations de Maxwell et exposé (démonstration) de la provenance de l'expression (hypothèse) des monopôles magnétiques
				\item Définition des algorithmes de classe P, NP et NPC.
				\item Démonstration du théorème fondamental de l'analyse ou également appelé "théorème fondamental du calcul intégral et différentiel"
				\item Présentation du cube $cGh$ et de l'interprétation de Copenhague
				\item Introduction aux concepts des réseaux de neurones formels
				\item Introduction aux concepts des algorithmes génétiques
				\item Introduction rigoureuse aux espaces fractales
				\item Introduction à l'informatique quantique
				\item Introduction à la logique floue
				\item Démonstration du théorème de Shannon, des théorèmes de Morgan, théorèmes d'expansion, tables de Karnaugh, addionneur complet, soustracteur complet
			\end{itemize}
		\item \textbf{Avril 2005}
			\begin{itemize}[noitemsep]
				\item Définitions de base sur les codes en blocs, les codes linéaires, les codes systématiques
				\item Démonstration de la relation de la variation relativiste de la masse
				\item Introduction aux codes et codes préfixes
				\item Calcul du nombre de jours entre deux dates données
				\item Techniques de calculs d'arrondis
				\item Définition des rentes post et praenumerando rigides ou non à taux constant (avenir certain) ou variables
				\item Définition et étude des propriétés des emprunts à remboursement ou annuité constant. 
			\end{itemize}
		\item \textbf{Avril 2006}
			\begin{itemize}[noitemsep]
				\item résentation du critère de Schild par l'intermédiaire de l'effet Einstein (redshift gravitationnel)
				\item Développement de la l'approximation newtonienne de l'équation des géodésiques.
				\item Définitions et démonstrations des formes développés des quadrivecteurs déplacement, vitesse, courant, accélération et énergie-impulsion
				\item Démonstration de la provenance du tenseur du champ électromagnétique et calculs de transformation de référentiels.
				\item Définition d'une tautologie, principe de non-tautologie
				\item Descriptions, définitions et démonstrations nombreuses sur les quaternions + démonstration de l'irrationalité du nombre d'euler
				\item Définition de la loi log-normal et triangulaire et démonstration de leur espérance et écart-type
				\item Introduction au calcul d'erreurs (incertitudes absolues et relatives, propagation des erreurs, chiffres significatifs, etc.)
				\item Définition de la loi de Weibull, démonstration de son espérance et écart-type
				\item Démonstration de la déviation de la lumière au bord d'un astre avec le modèle newtonien
				\item Définition d'une matrice de rotation (et développements y relatifs)
				\item Démonstration de l'existence de la division euclidienne dans l'anneau des polynômes
				\item Définitions du MWRR (Money Weighted Time of Return) et du TWRR (Time Weighted Rate of Return)
				\item Démonstration du théorème de Gauss-Ostrogradsky 
			\end{itemize}
		\item \textbf{Juillet 2006}
			\begin{itemize}[noitemsep]
				\item Définition du concept d'espace dual
				\item Définition de la loi de Pareto et démonstration de l'espérance et de la variance de la loi
				\item Définition des quantiles (quartile, centile)
				\item Démonstration que le mode est la valeur qui minimise la dispersion absolue
				\item Démonstration de l'inégalité de Minkowski et Bienaymé-Tchebychev
				\item Démonstration de la loi faible des grands nombres
				\item Démonstration de la formule d'Euler pour les graphes planaires
				\item Introduction mathématique à la méthode de contrôles de processus Six Sigma
				\item Démonstration du théorème du calcul variationnel
				\item Démonstration du calcul de la vitesse superluminique apparente d'un astre à Redshift élevé
				\item Méthode de résolution de jeux à somme nul utilisant la recherche opérationnelle
				\item Définition des fonds de placement
				\item Démonstration de l'expression du bêta selon les modèle de régression linéaire simple (Sharpe)		
			\end{itemize}
		\item \textbf{Octobre 2006}
			\begin{itemize}[noitemsep]
				\item Démonstration des estimateurs et vraisemblances des lois de Poisson et Binomiale
				\item Présentation du concept de "couleur" et de la synthèse additive et soustractive
				\item Démonstration de l'équation d'Einstein des champs (approche par l'approximation des champs faibles)
				\item Différenciation du principe d'équivalence, principe d'équivalence faible et principe d'équivalence d'Einstein
				\item Démonstration de la durée de l'arc Diurne des planètes dans l'approximation de la précession et nutation nulle
				\item Étude numérique et formelle (approximative) des points de Lagrange d'un système binaire
				\item Méthode de résolution des équations polynomiales du 4èmes degré (méthode de Ferrari)
				\item Présentation du déterminant de Gram via le volume euclidien représenté par le produit mixte des vecteurs d'une base canonique
				\item Définition des fonctions monotones, strictement monotones, etc.. sans approche formelle pure
				\item Détermination approximative via le potentiel de Yukawa (champs massiques) de la masse des mésons de l'interactions faible et de l'interaction nucléaire forte.
				\item Introduction aux équation différentielles linéaires et du premier ordre
			\end{itemize}
		\item \textbf{Décembre 2007}
			\begin{itemize}[noitemsep]
				\item Nombres et polynômes de Bernoulli
				\item Limite de Roche
				\item Aplattisement des corps célestes
				\item Pression et température cinétique
				\item Force aimant ou électroaimant
				\item Espaces vectoriels hermitiens et de Hilbert
				\item Solution de Schwarzschild en relativité générale
				\item Précession du périhélie de Mercure en relativité générale
				\item Déflexion de la lumière en relativité générale
				\item Effet Shapiro en relativité générale
				\item Modèle d'évaluation des actifs financiers
				\item L'Univers Trou Noir
				\item Constantes de couplage des interactions fondamentales
				\item Hamiltonien de l'équation de Schrödinger pour une particule chargée dans un champ électromagnétique
				\item Chaînes de Markov
				\item Théorie des files d'attentes
				\item Introduction aux mathématiques du génie météo et marin
				\item Exemples pratiques dans MS Excel du modèle d'efficience de Markowitz
				\item Exemples pratiques dans MS Excel du modèle d'efficience de Sharp
				\item Démonstration (simpliste) du théorème de Green(-Riemann) et de Stokes
				\item Introduction à l'analyse par composantes principales
				\item Démonstration du théorème spectral/cas réel
				\item Ajout des biographies sur Heckman, McFadden et Tesla
				\item Introduction à la régression logistique
				\item Démonstration du théorème de Rolle et des accroissements finis
				\item Démonstration du théorème de l'Hopital et des accroissements finis généralisés
				\item Introduction et démonstration en statistique de la fonction géométrique, sa variance et son espérance
				\item Démonstration du calcul de la surface et du volume des 5 polyèdres réguliers platoniciens
				\item Introduction à l'algèbre et géométrie corporelle
				\item Masse critique (masse Jeans) et Rayon critique (Rayons de Jeans) d'effondrement d'un nuage interstellaire/pépinières d'étoiles
				\item Temps d'effondrements d'un nuage interstellaire
				\item Durée de vie nucléaire d'une étoile
				\item Introduction aux transformées de Fourier
				\item Résolution détaillée des E.D.L. homogènes à coefficients constants
				\item Spirale de Cornu
				\item Mathématique des fonctions biométriques
				\item Détermination et résolution simple de l'équation de Pauli
				\item Solution générale (transformée de Fourier) de l'équation d'onde électromagnétique
				\item Introduction à la resistance des matériaux
				\item Horizon visuel
				\item Taux de croissance d'une population en fonction de la température
				\item Tube de Pitot et Perte de charge
				\item Théories de jauge $U(1)$ en physique quantique des champs
				\item Courbes de Bézier
				\item Système d'équation différentielles avec exponentiation de matrices
			\end{itemize}
		\item \textbf{Septembre 2008}
			\begin{itemize}[noitemsep]
				\item Nouvelles primitives usuelles importantes
				\item Démonstration détaillée des fonctions arcsinh et arccosh
				\item Démonstration de l'équation de Laplace et de la relation de Mayer
				\item Démonstration de la propagation d'ondes de pression
				\item Introduction historique pour la physique nucléaire
				\item Démonstration des relations de Maxwell en thermodynamique et introduction à l'énergie et enthalpie libre
				\item Modèle d'atmosphère adiabatique
				\item Démonstration des équations de Lorenz et de l'effet papillon
				\item Prisme
				\item Conditions de cohérence/interférence d'ondes électromagnétique
				\item Sphère de Bloch
				\item Primitives supplémentaires pour le Génie Civil et la Mécanique analytique
				\item Démonstration du volume de révolution de surface minimale
				\item Traitement de la particule libre
				\item Traitement du qubit polarisé et de qubit de spin 1/2
				\item Fonction caractéristique et théorème central limite
				\item Quelques démonstrations sur les inégalités dans les triangles
				\item Démonstration du volume d'un tonneau à section circulaire
				\item Démonstration de l'origine de la variance et de l'espérance de la loi de Student et Fisher-Snedecor
				\item Introduction au coût marginal
				\item Test statistique de l'ANOVA à un facteur
				\item Test statistique d'ajustement du Khi-deux de Pearson
				\item Ajout des biographies sur Pearson, Gosset et Fisher
				\item Introduction à l'analyse de la variance de la régression
				\item Analyse Factorielle des Correspondances
				\item Développement des circuits linéaires RC, RL, RLC série libres et forcés
			\end{itemize}
		\item \textbf{Septembre 2009}
			\begin{itemize}[noitemsep]
				\item Analyse de systèmes à topologie simple ou complexe pour la maintenance préventive
				\item Introduction aux fractions continues finies et infinies
				\item Solutions détaillées de l'effet tunnel d'une barrière rectangulaire
				\item Modèle mathématique de la désintégration alpha via effet tunnel
				\item Introduction au mouvement brownien selon modèle de Langevin
				\item Introduction à la tribologie/frottement
				\item Introduction à l'analyse des séries temporelles
				\item Démonstrations supplémentaires sur les indices de capabilité et procédé long terme et court terme ainsi que des appareils de mesure en statistique des procédés, des PPM et démonstration de la relation de Taguchi
				\item Démonstration de l'expression des potentiels électrique et magnétique de Lienard-Wiechert
				\item Introduction à l'analyse complexe
				\item Démonstration de la deuxième équation de Friedmann en cosmologie
				\item Démonstration du "ralentissement" de la lumière aux abords d'un Trou Noir
				\item Démonstration de l'expression du développement de Taylor d'une fonction de deux variables réelles
				\item Introduction aux plans d'expérience
				\item Démonstration du théorème d'Ehrenfest
				\item Démonstration des estimateurs de la loi de Weibull à deux paramètres
				\item Ajout d'un exemple d'application de la théorie de la décision
				\item Théorie des bandes (approximation parabolique et semi-classique) dans le cadre des semi-conducteur
				\item Théorème des résidus et séries de Laurent
				\item Démonstration des valeurs du Lean Six Sigma pour les processus
			\end{itemize}
		\item \textbf{Octobre 2011}
			\begin{itemize}[noitemsep]
				\item Calcul de l'orbite géostationnaire
				\item Calculs sur les ballons sonde PVC en météorologie
				\item Développements sur le gyroscope symétrique pesant et de la toupie
				\item Indice de Gini
				\item Équilibre séculaire, transitoire et non-équilibre en filiation radioactive
				\item Détermination approximative du rayon d'étoiles en rotation rapide
				\item Présentation de la Value At Risk delta-normale historique et en variance-covariance
				\item Deux nouvelles histoires drôles dans la section Humour...
				\item Ajout d'une biographie sur Agner Krarup Erlang
				\item Modèle statistique empirique de contrôle des salaires
				\item Démonstration simplifiée de l'absence d'opportunité d'arbitrage en finance
				\item Présentation du concept de portefeuille autofinancé sur sous-jacent risqué
				\item  Introduction aux techniques mathématiques des assurances
				\item Développements mathématiques sur la puissance et l'intensité d'une onde sonore longitudinale
				\item  Test statistique $Z$ bilatéral sur la différence de deux moyennes
				\item Test statistique de Student sur deux moyennes d'échantillons appariés
				\item Démonstration de la relation de l'intervalle de confiance statistique de proportions d'échantillons de grande taille
				\item Test statistique de l'égalité de deux proportions d'échantillons de grande taille
				\item Application du théorème de Shannon au calcul d'un indice de diversité en statistiques
				\item Démonstration de la détermination des coefficients d'une régression linéaire multiple
				\item Démonstration de la détermination des coefficients d'une régression linéaire simple passant par l'origine
				\item Introduction à l'analyse de la sensibilité
				\item Introduction aux statistiques de rangs/ordres
				\item Démonstration de la provenance, de l'espérance et de la variance de la loi binomiale négative
				\item Introduction aux cartes de contrôles avec démonstrations mathématiques détaillées
				\item Présentation de l'approche mathématique du Google Page Rank à ses débuts
				\item Démonstration de l'identité de Beltrami pour la simplification de l'équation d'Euler-Lagrange
				\item Test statistique binomial exact pour l'équilibre d'une population ayant deux caractéristiques
				\item Développements et études des vagues de gravité dans un fluide
				\item Quelques développements simples sur les engrenages/arbres d'engrenages
				\item Démonstration mathématique de l'effet de peau
				\item Théorie de l'arc-en-ciel
				\item Théorie du pendule double
				\item Loi de distribution de Boltzmann
				\item Loi de Dalton et d'Amagat
				\item Écoulement de la chaleur
				\item Puissance moyenne en courant alternatif
				\item Présentation de quelques calculs sur le betatron
			\end{itemize}
		\item \textbf{Mai 2013}
			\begin{itemize}[noitemsep]
				\item Exemple détaillé de construction d'un réseau de neurones particulier avec MS Office Excel
				\item Résolution des équations différentielles linéaires homogènes d'ordre 1 à coefficients non constants
				\item Ajout de l'exemple d'une transformée de Fourier d'une fonction Gaussienne et propriété de dérivation de la transformée de Fourier d'une dérivée
				\item Introduction aux interactions dans les ANOVA à deux facteurs
				\item Intervalle de confiance et intervalle de prédiction d'une régression linéaire
				\item Test des signes (médiane) en statistiques
				\item Introduction aux probabilités conditionnelles et l'espérance conditionnelle avec la loi de Pareto
				\item Détermination des estimateurs de la loi Gamma selon les méthodes des moments
				\item Ajout d'une deuxième approche pour la mise en évidence des marées
				\item Test statistique d'ajustement de Kolmogorov-Smirnov avec approche de Lilliefors
				\item Démonstration du modèle de gestion des quotas d'exploitation de Scheafer
				\item Démonstration du calcul de la période synodique des planètes et du temps de rétrogradation
				\item Démonstration de la construction de ponts browniens
				\item Démonstration de la provenance de la carte de contrôle des événements rares
				\item Calcul du facteur d'actualisation d'une assurance retraite basé sur l'inflation et l'espérance de vie
				\item Démonstration de l'ANOVA à deux facteurs sans répétition et à deux facteurs avec répétition
				\item Démonstration mathématique et physique élémentaire du fonctionnement du LASER
				\item Ajout de petites biographies sur Neper, Wilcoxon, Born, Heisenberg, Jordan, Kolmogorov, Stokes, Ostrogradsky, Zeeman, Henry, Faraday,  Meitner, Biot, Debye, Drude et Ohm
				\item Théorème de la série de Taylor avec reste intégral
				\item Stabilité de la loi de Poisson
				\item Test statistique de Poisson à 1 et 2 échantillons
				\item Tests statistiques non paramétriques de Kruskal-Wallis et Friedman
				\item Test statistique de normalité de Ryan-Joiner
				\item Test statistique C de Cochran
				\item Séries de Taylor-Maclaurin usuelles
				\item Régression polynomiale par la méthode des moindres carrés
				\item M.D.F. spatio-temporelle avec équations de Maxwell
				\item Analyse du seuil de rentabilité
				\item Oscillateur harmonique mécanique
				\item Effet Doppler acoustique
				\item Superposition d'ondes périodiques
				\item Test statistique de l'étendue de Tukey
				\item Modèle prévisionnel par moyenne mobile, coefficients saisonniers, lissage simple, double de Brown, double de Holt (additif), double de Holt et Winter (multiplicatif)
				\item Introduction élémentaire aux processus autorégressifs AR, autorégressifs AM, ARMA et ARIMA
				\item Démonstration de l'expression du facteur de correction sur population finie
				\item Test statistique exact de Fisher
				\item Lissage exponentiel de Laplace
				\item Kappa d'agrément de Cohen et test statistique de McNemar
				\item Modèle d'analyse de survie de Kaplan-Meier
				\item V de Cramer
				\item Clustering avec algorithme des K-Means
				\item Clustering avec algorithme pour dendrogrammes
				\item Test statistique de la somme des rangs signés de Wilcoxon à 1 échantillon et à deux échantillons appariés
				\item Étude quantitative de l'énergie potentielle effective (modèle harmonique de la liaison atomique) de l'atome hydrogénoïde
				\item Démonstration de l'équation des poutres (équation d'Euler-Bernoulli)
				\item Calcul du taux de défaillance d'un système en utilisant la technique du maximum de vraisemblance
				\item Ajout d'une chronologie des sciences
				\item Démonstration du modèle d'Einstein (loi de Dulong-Petit) de la capacité calorifique des solides cristallins et dérivation du modèle de Debye
				\item Modèle de Langenvin du diamagnétisme et paramagnétisme
				\item Introduction au calcul d'intégrales curvilignes
				\item Détermination naïve de l'énergie d'un dipôle magnétique
				\item Modèle nucléaire en "goutte liquide"
				\item Modèle de la résonance magnétique de spin
				\item Méthode du facteur d'intégrant de résolution d'équations différentielles
				\item Méthode de variation de la constante pour la résolution d'équations différentielles
				\item Loi de Mendel
				\item Rente viagère temporaire et différée
				\item Cycle de Carnot
				\item Modèle d'évaluation des actions de Durand et Gordon-Shapiro
				\item Test statistique d'Anderson-Darling
				\item Optimisation non-linéaire par la méthode Newton-Quadratique et de Gauss-Newton
				\item Méthode d'interpolation polynomiale de Lagrange
				\item Test statistique de Cochran-Mantel-Heanzel
			\end{itemize}
		\item \textbf{Novembre 2016}
			\begin{itemize}[noitemsep]
				\item Décomposition en valeurs singulières SVD
			\end{itemize}
		\item \textbf{Décembre 2016}
			\begin{itemize}[noitemsep]
				\item Test de Fieller (rapport de deux moyennes)
			\end{itemize}
		\item \textbf{Janvier 2017}
			\begin{itemize}[noitemsep]
				\item Problème de la cheminée en chute
				\item Cartes de contrôle de Levey-Jennings
				\item Plans d'expérience de mélange en réseau avec variables de processus
				\item Taux de panne moyen (fiabilité)
				\item Modèle de fiabilité de chaîne de Markov
				\item Conception de tests de fiabilité (temps de test du khi-carré, taille de l'échantillon binomial, taille de l'échantillon bêta-binomial)
				\item Linéarisation de la distribution de Weibull
				\item Pendule inversé
				\item Test d'autocorrélation de Durbin-Watson
				\item Méthode de Fisher pour $p$-valeurs multiples
				\item Loupe
				\item Cartes de contrôle de Laney
				\item Classification des coniques par le déterminant	
				\item Classification des équations aux dérivées partielles
			\end{itemize}
		\item \textbf{Février 2017}
			\begin{itemize}[noitemsep]
				\item Bases de la loi Normale repliée (fonction de densité et cumulée)
				\item Fonction de densité et cumulée de la loi demi-Normal et variance, espérance et médiane correspondante
				\item Séries téléscopiques et de Gandi
				\item Somme de Césaro
				\item Différenciation implicite
				\item Dérivation composée bivariée
			\end{itemize}
		\item \textbf{Mars 2017}
			\begin{itemize}[noitemsep]
				\item Méthode d'intégration de Laplace
				\item Pseudo Erreur Standard (PSE) de Lenth pour les erreur margianles des plans factoriels non-répliqués
				\item Erreurs marginales de Pareto pour les plans d'expériences factoriels avec réplications
				\item Désirabilité des plans d'expérience
			\end{itemize}
		\item \textbf{Avril 2017}
			\begin{itemize}[noitemsep]
				\item Métrique de Friedmann–Lemaître–Robertson–Walker
				\item Inégalité de Jensen
			\end{itemize}
		\item \textbf{Mai 2017}
			\begin{itemize}[noitemsep]
				\item Introduction aux ondes gravitationnelles en champs faibles
				\item Une approche mathématique du "diviser pour mieux régner" en gouvernance
				\item Trois nouveaux gags dans la section d'humour
			\end{itemize}
		\item \textbf{Juillet 2017} (v3.7 $\rightarrow$ v3.8)
			\begin{itemize}[noitemsep]
				\item Photos d'un arc-en-ciel quasi-vertical et photo d'un appereil fonctionnement sur la base de la résonance magnétique nucléaire
				\item Détalis sur la méthode de rétro-propagation pour les réseaux de neurones
				\item Nouveaux détails mathématiques sur l'effet de levier en régression linéaire simple
				\item Applications numériques de quelques tests expérimentaux de la Relativité Générale
				\item Détermination de la géodésique de la sphère (comme exemple dans le section de Mécanique Analytique)
				\item Méthode de classification ZeroR
				\item Méthode des $K$ plus proches voisins
				\item Définition d'une fonction concave/convexe (pour la démonstration de l'inégalité de Jensen)
				\item Démonstration de l'orthogonalité des polynômes de Hermite
				\item Démonstration du modèle d'évaluation des options de Bachelier
				\item Démonstration de l'égalité de Cox-Ross-Ingersoll pour les Forward/Future
				\item Démonstration du tenseur énergie-impulsion pour un fluide non-relativiste
				\item Définition de la distance orthodromique
				\item Démonstration de l'aire (surface) s'une section d'ellipse
				\item Démonstration de l'orbite stable profonde de Schwarzschild
				\item Expérience Hafele–Keating avec traitement via le formalisme de la Relativité Générale
				\item Introduction des hyperparamètres de Machine Learning (apprentissage machine)
				\item Lissage par densité de noyaux (Kernel smoothing)
				\item Risque de défaut de crédit
				\item Ajout de nombreuses dates dans la section de chronologie
			\end{itemize}
		\item \textbf{Novembre 2017} (v3.8 $\rightarrow$ v3.9)
			\begin{itemize}[noitemsep]
				\item Création des références croisées dans le texte (mais à améliorer!)
				\item Ajout d'un nouveau point au règles de publiction scientifiques (citer les études équivalents, si existantes, pour méta-analyses ultérieures)
				\item Démonstration de la dérivée de $f(g)^{g(x)}$
				\item Arbre de décision OneR (One Rule) pour la Data Science avec matrix de confusion y relative
				\item Règles d'association pour le machine learning (apprentissage machine)
				\item Kurtosis (coefficient d'aplatissement) et Skewness (coefficient d'asymétrie)
			\end{itemize}
		\item \textbf{Janvier 2018} (v3.9 $\rightarrow$ v3.10)
			\begin{itemize}[noitemsep]
				\item Équation de continuité de Dirac
				\item Modèle CCR (Charnes, Cooper et Rhodes) du modèle Data Envelopment Analysis (DEA)
				\item Divergence de Kullback-Leibler
				\item Convolution discrète et linéaire
				\item Intégrale de Dirichlet
				\item Tests de permutation
				\item Ajout de références croisées supplémentaires
			\end{itemize}
			\item \textbf{June 2018} (v3.10 $\rightarrow$ v3.11)
			\begin{itemize}[noitemsep]
				\item Il y avait une erreur de frappe concernant le signe de l'angle retourné par la fonction $\mathrm{atan()}$ ($-$ au lieu d'un $+$)
				\item Ajout de quelques tables de Taguchi supplémentaires
				\item Démonstration de la loi du khi-deux non-central à un degré de liberté pour l'étude des intervalles de tolérance
				\item Ajout de détails sur la régression forcée à l'origine (particulièrement sur le problème du calcul du $R^2$ dans ce cas là!)
				\item Ajout de nouvelles images (sur la tromperie des données, un gag sur le chat de Schrödinger, comparaison des tailles des planètes, une illustration de la différence en le bagging et boosting en apprentissage machine)
				\item Ajout dans la section d'Astronomie de tables de données intéressantes sur les planètes du système solaire (incluant Pluton!)
				\item Ajout dans la section d'Astrophysique d'une table de données intéressante sur quelques étoiles fameuses
				\item Modèle de survie de Cox (modèle à hasard proportionnel de Cox)
				\item Nous avons rendu plus explicite le lien entre la méthode de Gauss-Newton et la méthode des gradients descendants
				\item Démonstration des intervalles de confiance pour l'estimateur de Kaplan-Meier et le modèle de hasard proportionnel de Cox (ration de hasard)
		\end{itemize}
		\item \textbf{June 2019} (v3.11 $\rightarrow$ v3.12)
			\begin{itemize}[noitemsep]
				\item Améliorations et corrections relativement au décalage vers le rouge
				\item Ajout de petites biographies sur Yates, Friedmann, Lemaître et Cramer
				\item Améliorations sur la partie concernant les modèles d'Univers de FRLW
				\item Ajout d'une dizaine d'harmoniques sphériques supplémentaires dans la section de Chimie Quantique (grâce a Wikipédia!)
				\item Brève introduction à l'analyse $XYZ$
				\item Introduction au concept d'avantage mécanique idéal et d'avantage mécanique actuel
		\end{itemize}
	\end{itemize}
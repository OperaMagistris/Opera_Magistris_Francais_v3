	Cet annuaire contient des liens répartis dans $7$ catégories tous ayant trait aux sciences et que nous trouvons intéressants. Nous tenons à préciser qu'en aucun cas nous avons été rétribué sous quelque forme que ce soit pour l'ajout de liens dans la liste ci-dessous! Vous pouvez aussi trouver des apps iPad/iPhone "scientifiques" en bas de la liste que nous recommandons fortement (nous n'avons pas contrôlé s'il existe un équivalent de chacune de ces applications pour Android ou Microsoft Windows Surface/Phone).
	
	\begin{itemize}	 
		\item[$-$] {\Large \ding{52}} Site de qualité tant du point de vue contenu que contenant
		\item[$-$] {\Large \ding{45}} Contenu avec développements et démonstrations détaillées
		\item[$-$] {\Large \ding{41}} Site avec forum de discussion
		\item[$-$] {\Large \ding{36}} Softwares, sharewares, freewares (logiciels) scientifiques à télécharger
		\item[$-$] {\Large \ding{229}} Ouvrages, publications, magazines, journaux scientifiques (à visualiser ou télécharger)
		\item[$-$] {\Large \ding{44}} Site sympathique
		\item[$-$] {\Large \ding{170}} Coup de coeur
		\item[$-$] {\Large \ding{73}} À voir absolument!
	\end{itemize}	
	S'il y avait par ailleurs, trois liens à mettre en évidence parmi tous ce serait \href{http://www.google.com}{\color{blue} Google}, \href{http://www.wikipedia.com}{\color{blue} Wikipedia} et \href{http://www.youtube.com}{\color{blue} YouTube} dont nous sommes redevables de nombreux emprunts!
		
	\begin{tcolorbox}[title=Remark,colframe=black,arc=10pt]
	Nous ne sommes pas responsables de la persistance ou de l'exactitude des URL des sites Internet externes ou tiers mentionnés dans ce livre, et ne garantissons pas que le contenu de ces sites est ou restera exact ou approprié.
	\end{tcolorbox}
	
	\pagebreak

	\section{Sciences Exactes}

	{\Large \ding{52}}{\Large \ding{45}}{\Large \ding{36}}{\Large \ding{44}}{\Large \ding{170}}{\Large \ding{73}}\bcdfrance{} ChronoMath, "petite" chronologie des mathématiques, est un site Internet pédagogique en perpétuel renouvellement que Serge Mehl fit connaître dès 1988 en tant que conseiller pédagogique de mathématiques en Afrique francophone où il fut coopérant pendant de nombreuses années. Plus de 450 mathématiciens (et leurs travaux) sont passés en revue...!!! À voir!\\
	\href{http://www.chronomath.com}{\color{blue} http://www.chronomath.com}
	
	{\Large \ding{41}}{\Large \ding{36}}{\Large \ding{229}}\bcdfrance{} Ce site propose une actualité des mathématiques, une encyclopédie avec parties dictionnaire, des biographies et formulaires, ainsi que différents dossiers sur des sujets mathématiques divers et un forum. Ce site est particulièrement impressionnant relativement à la quantité d'informations proposée en téléchargement. \\
	\href{http://www.bibmath.net}{\color{blue} http://www.bibmath.net} 
	
	{\Large \ding{52}}{\Large \ding{45}}\bcdfrance{} Un bon site traitant de quelques sujets pertinents en physique (synophysique). Le design pourrait être revu en ce qui concerne la navigation utilisant des cadres... mais il faut tout de même privilégier la qualité du contenu et celle-ci prédomine largement.\\
	\href{http://www.sciences.univ-nantes.fr/physique/perso/blanquet/frame3.htm}{\color{blue} http://www.sciences.univ-nantes.fr/physique/perso/blanquet/frame3.htm}
	
	{\Large \ding{52}}{\Large \ding{45}}{\Large \ding{36}}\bcdfrance{} Nombreux contenus PDFs sur l'algèbre, l'analyse et la géométrie.\\
	\href{http://c.caignaert.free.fr}{\color{blue} http://c.caignaert.free.fr}
	
	{\Large \ding{52}}{\Large \ding{45}}{\Large \ding{36}} arXiv est une archive de prépublications électroniques d'articles scientifiques. Le laps de temps s'écoulant entre le moment où un chercheur termine un projet et le moment où son travail est publié dans un journal peut être de l'ordre d'un an. À l'échelle du temps de la recherche c'est une durée longue. La mise en place de l'arXiv a donc été un moyen de pallier à ce problème temporel et de coût.\\
	\href{http://arxiv.org}{\color{blue} http://arxiv.org}
	
	{\Large \ding{52}}{\Large \ding{45}}{\Large \ding{36}}\bcdfrance{} L'archive ouverte pluridisciplinaire HAL est destinée au dépôt et à la diffusion gratuite d'articles scientifiques de niveau recherche, publiés ou non, et de thèses, émanant des établissements d'enseignement et de recherche français ou étrangers, des laboratoires publics ou privés. On y trouve même des livres scientifiques publiés récemment par des maisons d'édition françaises préstigieuses.\\
	\href{http://hal.archives-ouvertes.fr/}{\color{blue} http://hal.archives-ouvertes.fr/}
	
	\pagebreak
	\section{Éditions/Magazines}

	{\Large \ding{52}}{\Large \ding{41}}{\Large \ding{229}}{\Large \ding{170}}{\Large \ding{73}}\bcdfrance{} On peut considérer le site proposé ici comme l'équivalent du précédent, mais pour les ingénieurs francophones. Il est cependant qualitativement meilleur et on y trouve une quantité de dossiers toute aussi grande mais plus homogène. On regrettera juste peut-être l'accès à certains éléments qui n'est pas toujours aisé du premier coup.\\
	\href{http://www.techniques-ingenieur.fr}{\color{blue} http://www.techniques-ingenieur.fr}
	
	{\Large \ding{52}}{\Large \ding{229}}{\Large \ding{170}}{\Large \ding{73}}\bcdfrance{} Absolument excellent et à voir! De nombreux cours complets de l'École Polytechnique (France) sont publiées et disponibles gratuitement au téléchargement au format PDF (est-ce que cela va durer?). Environ $1'000$ ressources éducatives sont disponibles au total, dont la qualité est hautement variable, mais dont la pertinence et la rareté sont toujours égales.\\
	\href{http://catalogue.polytechnique.fr}{\color{blue} http://catalogue.polytechnique.fr}
	
	{\Large \ding{52}}{\Large \ding{229}}{\Large \ding{170}}{\Large \ding{73}}\bcdfrance{} Le site Internet en soi n'est pas excellent (ce qui est fort dommage) mais le magazine pour lequel il vous propose de vous abonner (contre payement) vaut largement le détour pour les passionnées de mathématiques et de leur actualité.\\
	\href{http://tangente.poleditions.com/}{\color{blue}http://tangente.poleditions.com/}
	
	{\Large \ding{52}}{\Large \ding{229}}{\Large \ding{170}}{\Large \ding{73}}\bcdfrance{} Un magazine de la même trempe que Tangente mais à un niveau technique et académique bien supérieur selon mes critères personnels. Le contenu est particulièrement orienté mathématiques pures et très rarement avec des applications directes et explicites à la physique, ingénierie ou à l'économie.\\
	\href{http://www.quadrature.info}{\color{blue}http://www.quadrature.info}
	
	{\Large \ding{52}}{\Large \ding{229}}{\Large \ding{170}}{\Large \ding{73}} Ce site en soi n'est pas excellent non plus (ce qui est fort dommage) mais certains des ouvrages proposés sont simplement historiques!!!\\
	\href{http://urss.ru}{\color{blue}http://urss.ru}
	
	{\Large \ding{52}}{\Large \ding{229}}{\Large \ding{170}}{\Large \ding{73}}\bcdfrance{} Excellent site avec une ressource considérable de documentation électronique traitant des mathématiques uniquement. Le site contient des liens vers les pages des auteurs des documents (sinon quoi il faudrait un serveur considérable...).\\
	\href{http://mathslinker.chez-alice.fr}{\color{blue}http://mathslinker.chez-alice.fr}
	
	{\Large \ding{229}}\bcdfrance{} Ce site contient une bibliothèque en ligne de publications de grands mathématiciens du 20ème et 19ème siècle. À voir absolument!\\
	\href{http://matwbn.icm.edu.pl/wyszukiwarka.php}{\color{blue}http://matwbn.icm.edu.pl/wyszukiwarka.php}
	
	{\Large \ding{52}}{\Large \ding{229}}{\Large \ding{170}}{\Large \ding{73}}\bcdfrance{} Reprints d'oeuvres fondamentales concernant la mathématique, la physique, l'histoire et la philosophie des Sciences.\\
	\href{http://www.gabay.com}{\color{blue}http://www.gabay.com}
	
	\pagebreak
	{\Large \ding{52}}{\Large \ding{229}}{\Large \ding{170}}{\Large \ding{73}}\bcdfrance{} Éditions Eyrolles - Excellent site contenant de nombreux ouvrages francophones et anglophones de qualité. Propose au fait des ouvrages de plusieurs maisons d'éditions dont Dunod, Springer, etc. Il vaut la peine d'aller jeter un coup d'oeil dans les sections "Mathématique" et "Physique" y'a du bon... \\
	\href{http://www.eyrolles.com}{\color{blue}http://www.eyrolles.com}
	
	{\Large \ding{52}}{\Large \ding{229}}{\Large \ding{170}}{\Large \ding{73}}\bcdfrance{} Éditions Dunod - Proposent des ouvrages scientifiques contemporains (niveau 1-3ème cycle et +). Les ouvrages de cette maison d'édition sont pour la plupart excellents. Personnellement ils sont techniquement (mais pas pédagogiquement) mes préférés, car souvent les développements y sont souvent très détaillés.\\
	\href{http://www.dunod.com}{\color{blue}http://www.dunod.com}
	
	{\Large \ding{229}}{\Large \ding{170}} Éditions Springer - Proposent des ouvrages scientifiques de niveau postdoc. La navigation sur le site n'est pas aisée car les choix sont disposés un peu n'importe comment mais sinon les ouvrages proposés sont très techniques et de haut niveau... très haut niveau...\\
	\href{http://www.springer.de}{\color{blue}http://www.springer.de}
	
	{\Large \ding{229}}{\Large \ding{170}}{\Large \ding{73}}\bcdfrance{} Le programme NUMDAM, piloté par la Cellule MathDoc (UMS 5638 CNRS - UJF) pour le compte du CNRS, propose la numérisation rétrospective des fonds mathématiques publiés en France.\\
	\href{http://www.numdam.org}{\color{blue}http://www.numdam.org}
	
	{\Large \ding{52}}{\Large \ding{229}}{\Large \ding{170}}{\Large \ding{73}} Éditions Wrox - Proposent des ouvrages développement informatique (niveau expert). D'après ce que je sais (et j'en pense de même), cette maison d'édition est la référence mondiale dans le domaine des ouvrages traitant des langages de programmation.\\
	\href{http://www.wrox.com}{\color{blue}http://www.wrox.com}
	
	{\Large \ding{52}}{\Large \ding{170}}{\Large \ding{73}} Éditions Cambridge University Press - Proposent des uvrages scientifiques niveau postdoc aussi. Un peu l'équivalent des éditions Springer mais la structure du site Internet est un peu mieux faite. On appréciera particulièrement la possibilité de s'abonner gratuitement pour recevoir régulièrement leur catalogue.\\
	\href{http://www.cambridge.org}{\color{blue}http://www.cambridge.org}
	
	
	{\Large \ding{45}}{\Large \ding{229}}{\Large \ding{170}}{\Large \ding{73}} La Bibliothèque Numérique de France (BNF) propose gratuitement en téléchargement des ouvrages scientifiques (entre autres) dont les droits d'auteur sont tombés après 75 années. Le système de téléchargement est peu convivial, mais on peut cependant tomber sur des ouvrages très pertinents. Les PDF font souvent plus de 10 MB donc attention à ceux qui ont du petit débit.\\
	\href{http://gallica.bnf.fr}{\color{blue} http://gallica.bnf.fr}
	
	{\Large \ding{45}}{\Large \ding{229}}{\Large \ding{170}}{\Large \ding{73}} Le Physical Review Letters (PRL) search machine (articles scientifiques) propose contre inscription et paiement (sic!) l'accès à des articles scientifiques de recherche en physique théorique et expérimentale publisée depuis plus de 100 ans. Il faut admettre que le niveau de détail des articles proposés n'est souvent pas très élevé mais là n'est pas l'objectif pour les spécialistes.\\
	\href{http://prola.aps.org/search}{\color{blue} http://prola.aps.org/search}
	
	\pagebreak
	{\Large \ding{45}}{\Large \ding{229}}{\Large \ding{73}} Éditions Lavoisier - Très bonne maison d'édition proposant des ouvrages très spécialisés dans le domaine de l'ingénierie électronique, électrotechnique, civile, informatique, etc. On appréciera aussi particulièrement la possibilité de s'abonner gratuitement pour recevoir régulièrement leur excellent catalogue. Voir plus particulièrement la partie "Hermès Sciences" et " Tech \& Doc".\\
	\href{http://www.lavoisier.fr}{\color{blue} http://www.lavoisier.fr} (\href{http://www.editions-hermes.fr}{\color{blue} http://www.editions-hermes.fr} + \href{http://www.tec-et-doc.com}{\color{blue} http://www.tec-et-doc.com})
	
	{\Large \ding{170}}{\Large \ding{45}}{\Large \ding{229}} Éditions World Scientific - Éditeur américain proposant de très nombreux ouvrages de haut niveau pour les sciences physiques et mathématiques (voir la large palette de sections scientifiques correspondantes dans leur site).\\
	\href{http://www.worldscibooks.com}{\color{blue}http://www.worldscibooks.com}
	
	\section{Associations}

		{\Large \ding{44}} myScience.ch donne une vue d'ensemble des sciences, de la recherche, des universités, des entreprises et d'autres centres de recherche en Suisse. Le site fournit des informations pratiques sur l'emploi, le financement et la vie en Suisse ainsi que des nouvelles scientifiques aux chercheurs, scientifiques, universitaires et à toute personne intéressée par les sciences. Il a une proportion plus grande d'articles en allemand... même dans la version francophone...\\
		\href{http://www.myscience.ch}{\color{blue}http://www.myscience.ch}
		
		{\Large \ding{229}}{\Large \ding{44}}{\Large \ding{73}}\bcdfrance{} L'Union Rationaliste a pour but de promouvoir le rôle de la raison dans le débat intellectuel comme dans le débat public, face à toutes les dérives irrationnelles. Elle s'emploie à mettre à la disposition de chacun la possibilité d'accéder à une conception intelligible du monde et de la vie. Elle lutte pour que l'État demeure laïque, assume sa fonction de protection contre toute forme d'endoctrinement.\\
		\href{http://www.union-rationaliste.org}{\color{blue}http://www.union-rationaliste.org}
		
		{\Large \ding{45}}{\Large \ding{229}} La Société Américaine de Mathématiques (en anglais: American Mathematical Society, AMS) est une association de mathématiciens professionnels, dédiée aux intérêts de la recherche et de l'enseignement des mathématiques, ce qu'elle fait sous forme de différentes publications gratuites et conférences, et de prix décernés à des mathématiciens.\\
		\href{http://www.ams.org}{\color{blue}http://www.ams.org}
		
		{\Large \ding{45}}{\Large \ding{229}} La Société Américaine de Physique (en anglais: American Physical Society, APS) est une société savante fondée le 20 mai 1899, basée aux États-Unis, très active dans le domaine de la recherche scientifique en physique.\\
		\href{http://www.aps.org}{\color{blue}http://www.aps.org}
		
		{\Large \ding{45}}{\Large \ding{229}}{\Large \ding{170}} L'Association Européenne de Physique (en anglais: Physical Society, EPS) est une organisation à but non lucratif dont le but est de promouvoir la physique et les physiciens en Europe. Cette société propose un abonnement à son magazine \textit{Europhysicsnews} et à bien d’autres choses..\\
		\href{http://www.eps.org}{\color{blue}http://www.eps.org}

	\pagebreak
	\section{Emploi}
	
	{\Large \ding{73}} Voici une liste non-exhaustive de sites d'offres d'emplois déstinée particulièrement aux physiciens et mathématiciens suisses... (cependant un lien est destiné uniquement au territoire U.S.!). Il en découle que souvent les descriptifs des offres sont en allemand (...) et que ces dernières sont souvent relatives au domaine bancaire (qui reprèsente 10-15\% de l'emploi en Suisse d'après ce que je sais...). Certaines offres sont reliées à des sites plus généraux comme le fameux jobup.ch bien connu en Suisse.
	
	\begin{itemize}	 
		\item[$\bullet$] \href{http://www.telejob.ch}{\color{blue}http://www.telejob.ch} 
	
		\item[$\bullet$] \href{http://www.jobs.myscience.ch}{\color{blue}http://www.jobs.myscience.ch}
	
		\item[$\bullet$] \href{http://www.math-jobs.ch}{\color{blue}http://www.math-jobs.ch}
	
		\item[$\bullet$] \href{http://www.analyticrecruiting.com}{\color{blue}http://www.analyticrecruiting.com}
	\end{itemize}
	
	\section{Télévision/Radio}

	{\Large \ding{73}} Excellent site expliquant avec des animations 3D de nombreux appareillages ultra-classiques de la vie de tous les jours conçus et développés par les méthodes de l'ingénieur (moteur, réfrigérateur, pompe, etc.).\\
	\href{http://www.learnengineering.org}{\color{blue}http://www.learnengineering.org}
	
	\bcdfrance{} C'est pas Sorcier: La célèbre émission de télévision française (version moderne de "Bill Nye the Science guy") sur toutes les sciences que presque tous les jeunes francophones connaissent ou découvrent encore!\\
	\href{https://www.youtube.com/user/cestpassorcierftv/}{\color{blue}https://www.youtube.com/user/cestpassorcierftv/}
	
	\bcdfrance{} Cité-Sciences: visionnement de vidéo-conférences, ou écoute d'enregistrements audio sur des sujets de physique et astronomie (+ d'autres) vulgarisés (requière RealOne Player au jour où nous écrivons ces lignes).\\
	\href{http://www.cite-sciences.fr}{\color{blue}http://www.cite-sciences.fr}
	
	\bcdfrance{} Sur Canal Académie, les académiciens, spécialistes en physique, mathématique, sciences humaines, philosophie, sociologie, droit et jurisprudence, économie, politique et finances, histoire, géographie et démographie et sciences politiques vous font partager leurs réflexions sur l'actualité et les évolutions de la société.\\
	\href{http://www.canalacademie.com}{\color{blue}http://www.canalacademie.com}
	
	\pagebreak
	Et voici une liste non-exhaustive de chaînes scientifiques YouTube produites par des particuliers passionnés qui sont scientifiquement relativement \underline{fiables} et dont la qualité de post-production est très bonne et qui peuvent s'avérer très très utiles pour mieux comprendre certains théorèmes complexes et sujets traités dans ce livre (la plupart sont de niveau scolaire secondaire):
	\begin{itemize}
		 \item[$\bullet$] \href{https://www.youtube.com/user/epenser1}{\color{blue}https://www.youtube.com/user/epenser1} (fr, Bruce Benamran)
		 
		 \item[$\bullet$] \href{https://www.youtube.com/user/ScienceEtonnante}{\color{blue}https://www.youtube.com/user/ScienceEtonnante} (en, David Louapre)
		 
		 \item[$\bullet$] \href{https://www.youtube.com/user/Micmaths}{\color{blue}https://www.youtube.com/user/Micmaths} (fr, Mickaël Launay)
		 
		 \item[$\bullet$]  \href{https://www.youtube.com/user/physicswoman}{\color{blue}https://www.youtube.com/user/physicswoman} (en, Dianna Cowern)
		 
		 \item[$\bullet$] \href{https://www.youtube.com/user/1veritasium}{\color{blue}https://www.youtube.com/user/1veritasium} (en, Derek Muller)
		 
		 \item[$\bullet$] \href{https://www.youtube.com/channel/UCYO_jab_esuFRV4b17AJtAw}{\color{blue}https://www.youtube.com/channel/UCYO\_jab\_esuFRV4b17AJtAw} (en, Grant Sanderson)
		 
		 \item[$\bullet$] \href{https://www.youtube.com/user/fauxsceptique}{\color{blue}https://www.youtube.com/user/fauxsceptique}(fr, Christophe Michel)
		 
		 \item[$\bullet$] \href{https://www.youtube.com/channel/UC7_gcs09iThXybpVgjHZ_7g}{\color{blue}https://www.youtube.com/channel/UC7\_gcs09iThXybpVgjHZ\_7g} (en, Matt O'Dowd)
		 
		 \item[$\bullet$] \href{https://www.youtube.com/user/mathdude2012}{\color{blue}https://www.youtube.com/user/mathdude2012} (en, Andrey Kopot)
		 
		 \item[$\bullet$] \href{https://www.youtube.com/user/LearnEngineeringTeam}{\color{blue}https://www.youtube.com/user/LearnEngineeringTeam} (en, Sabin Mathew)
	\end{itemize}

	\pagebreak
	\section{Divers Sciences}

	{\Large \ding{52}}{\Large \ding{45}}{\Large \ding{41}}{\Large \ding{44}}\bcdfrance{} Site comportant de nombreuses fiches de cours et surtout quantités d'exercices résolus intéressants. Malheureusement, le site est devenu payant avec accès pendant une période limitée... le prix étant peu élevé il peut être quand même pertinent de débourser la somme ad hoc pour la qualité.\\
	\href{http://www.web-sciences.com}{\color{blue}http://www.web-sciences.com}
	
	{\Large \ding{52}}{\Large \ding{41}}{\Large \ding{229}}{\Large \ding{44}}{\Large \ding{73}}\bcdfrance{} Futura-Sciences est un site à vocation de vulgarisation sur les sciences pures et exactes avec des informations quotidiennes sur les sciences et technologies, de nombreux dossiers dans toutes les thématiques et des forums de sciences de débats et de discussions ... (le forum de physique y est souvent très bien fréquenté).\\
	\href{http://www.futura-sciences.com}{\color{blue}http://www.futura-sciences.com}
	
	{\Large \ding{52}}{\Large \ding{41}}{\Large \ding{229}}{\Large \ding{44}}{\Large \ding{73}}\bcdfrance{} Physics Forum est considéré comme la plus grande communauté de physique au monde. Il y a beaucoup de discussions sur de nombreux sujets auxquels la communauté a répondu avec qualité. Ce forum est également considéré comme le forum partenaire de du présent livre car je n'ai plus le temps de répondre gratuitement aux questions qu'on m'envoie par courriel.\\
	\href{http://www.physicsforums.com}{\color{blue}http://www.physicsforums.com}
	
	{\Large \ding{52}}{\Large \ding{45}}{\Large \ding{41}}{\Large \ding{36}}{\Large \ding{229}}{\Large \ding{44}}{\Large \ding{170}}{\Large \ding{73}}\bcdfrance{} Astrosurf est un portail de liens pour les astronomes amateurs francophones. Il y est aussi proposé des forums d'astronomie, des petites annonces, l'hébergement gratuit de sites d'astronomie, des éphémérides et tout sur les clubs et associations d'astronomie francophones. On y trouve en particulier le site fameux de Thierry Lombry (\href{http://astrosurf.com/luxorion/}{\color{blue}ici}).\\
	\href{http://www.astrosurf.com}{\color{blue}http://www.astrosurf.com}
	
	{\Large \ding{52}}{\Large \ding{229}}\bcdfrance{} Le C.E.A. est le Commissariat Français à l'Énergie Atomique. Le site propose des dossiers et nouvelles intéressantes sur certains domaines particuliers de la physique de pointe. On peut également trouver des ressources pédagogiques gratuites pour les professeurs (diaporamas et posters).\\
	\href{http://www.cea.fr}{\color{blue}http://www.cea.fr}
	
	{\Large \ding{44}}{\Large \ding{73}} Voici un site qui, si j'étais encore enfant, aurait fait mon bonheur... et le malheur de mes parents. Ce n'est pas tant le contenu qui est intéressant mais surtout la boutique en ligne (eStore) qui propose une bonne centaine de gadgets et jeux ludo-éducatifs pour les passionnées de sciences. Attention au porte-monnaie quand même!\\
	\href{http://www.xump.com}{\color{blue}http://www.xump.com}
	
	{\Large \ding{44}}{\Large \ding{73}} Pendant des années, les dessins de S. Harris ont ajouté de l'humour à d'innombrables magazines, livres, bulletins d'information, publicités et sites Internet dans le domaine des sciences.\\
	\href{http://www.sciencecartoonsplus.com}{\color{blue}http://www.sciencecartoonsplus.com}
	
	\pagebreak
	\section{Logiciels/Applications}
	{\Large \ding{52}}{\Large \ding{170}}{\Large \ding{73}} Rosetta Code est un site de programmation chrestomathy. L'idée est de présenter des solutions à la même tâche dans autant de langues de programmation différents que possible, de montrer comment les languages sont similaires et différentes, et d'aider une personne ayant une approche unique à résoudre un problème pour en apprendre une autre. Le code Rosetta comporte actuellement $861$ tâches, $208$ tâches à l'état de brouillon et $671$ languages.\\
	\href{https://rosettacode.org/wiki/Category:Programming_Tasks}{\color{blue}https://rosettacode.org}
	
	{\Large \ding{52}}{\Large \ding{36}}{\Large \ding{170}}{\Large \ding{73}} Site Internet du logiciel TeXMaker utilisé pour écrire ce livre en \LaTeX{} (logiciel fonctionnant sur plusieurs systèmes d'exploitation).\\
	\href{http://www.xm1math.net/texmaker/index.html}{\color{blue}http://www.xm1math.net}
	
	{\Large \ding{52}}{\Large \ding{36}}{\Large \ding{170}}{\Large \ding{73}} Minitab est un des logiciels (payant) de référence pour les ingénieurs travaillant dans l'industrie et appliquant la maîtrise statistique des procédés (voir la section de Génie Industriel de ce livre) dans le cadre de leur travail ou faisant tout autre étude statistique dans le cadre de la R\&D.\\
	\href{http://www.minitab.com}{\color{blue}http://www.minitab.com}
	
	{\Large \ding{52}}{\Large \ding{36}}{\Large \ding{170}}{\Large \ding{73}} Isograph est une excellente suite de logiciels pour les ingénieurs travaillant dans l'industrie et appliquant les techniques de la maintenance préventive (fiabilité, AMDEC) et d'assistance à la décision (voir chapitre de Théorie Des Jeux Et De La Décision).\\
	\href{http://www.isograph-software.com}{\color{blue}http://www.isograph-software.com}
	
	{\Large \ding{52}}{\Large \ding{36}}{\Large \ding{170}}{\Large \ding{73}} L'entreprise ReliaSoft développe probablement les meilleurs logiciels du marché pour les ingénieurs en fiabilité et en assurance. Leurs logiciels sont utilisés par les entreprises étant à la pointe dans l'ingénierie mondiale. Globalement leurs logiciels sont de vrais petits bijoux (en particulier Weibull++).\\
	\href{http://ww.reliasoft.com}{\color{blue}http://ww.reliasoft.com}
	
	{\Large \ding{52}}{\Large \ding{36}}{\Large \ding{73}} JMP de la société SAS est probablement le meilleur logiciel (dans le sens de sa complétude) pour les plans d'expérience et l'analyse de la capabilité à ce jour. Son approche est pédagogique, structurée et il existe de très bonnes documentations avec les démonstrations mathématiques de la démarche utilisée par le logiciel.\\
	\href{http://www.jmp.com}{\color{blue}http://www.jmp.com}
	
	{\Large \ding{52}}{\Large \ding{36}}{\Large \ding{170}}{\Large \ding{73}} Site internet officiel de l'excellent logiciel de calcul formel Maple utilisé pour de nombreux exemples dans les différentes chapitres du présent livre. Le site propose aussi beaucoup de documentation et de compléments à télécharger.\\
	\href{http://www.mapleapps.com}{\color{blue}http://www.mapleapps.com}
	
	{\Large \ding{52}}{\Large \ding{36}}{\Large \ding{73}} Site internet officiel du logiciel de calcul et simulation numérique MATLAB™. Je ne suis pas un fan mais c'est un outil incontournable dans certaines entreprises et particulièrement pour la partie SimuLink. Il s'agit en quelque sorte d'un "must have" pour les ingénieurs au même titre que LabView.\\
	\href{http://www.mathworks.com}{\color{blue}http://www.mathworks.com}
	
	{\Large \ding{52}}{\Large \ding{36}}{\Large \ding{73}} Site officiel de la société Statistica. Une référence mondiale dans le domaine de l'analyse statistique, de l'exploration de données et du contrôle statistique des processus qui se situe à priori dans un mouchoir de poche avec IBM SPSS.\\
	\href{http://www.statsoft.com}{\color{blue}http://www.statsoft.com}
	
	{\Large \ding{52}}{\Large \ding{36}}{\Large \ding{170}}{\Large \ding{73}} COMSOL Multiphysics (anciennement FEMLAB) est un excellent environnement interactif pour la modélisation d'applications industrielles et scientifiques basées sur les équations aux dérivées partielles (EDP) utilisant les méthodes par éléments finis (plus simple à utiliser que ANSYS). À acheter pour les ingénieurs et chercheurs (pour ceux qui ont les moyens d'acheter la licence bien sûr..)!\\
	\href{http://www.comsol.fr}{\color{blue}http://www.comsol.fr}
	
	{\Large \ding{52}}{\Large \ding{36}}{\Large \ding{170}}{\Large \ding{73}} Palisade @Risk est une suite de compléments pour Microsoft Excel et Microsoft Project pour la simulation probabiliste (de Monte Carlo et Latin Hypercube). Son intégration conjointe avec Microsoft Project et Microsoft Excel (ie convivialité) ainsi que ses modules de calculs basés sur les algorithmes génétiques et réseaux de neurones et son module de théorie de la décision en fait un outil très convoité pas les hauts dirigeants et ingénieurs des grandes entreprises dans le domaine de la finance, de la qualité, la gestion de projets, l'audit et de la production.\\
	\href{http://www.palisade-europe.com}{\color{blue}http://www.palisade-europe.com}
	
	{\Large \ding{52}}{\Large \ding{36}}{\Large \ding{170}}{\Large \ding{73}} Le logiciel Decision Analysis de TreeAge (DATA) permet à l'utilisateur de construire, d'analyser et de distribuer des arbres d'analyse, des modèles de Markov et des diagrammes d'influence. Les modèles DATA intègrent visuellement les aspects quantitatifs et qualitatifs reliés aux décisions d'entreprise afin de fournir un outil d'assistance dans les projets lors d'analyse de risques complexes.\\
	\href{http://www.treeage.com}{\color{blue}http://www.treeage.com}
	
	{\Large \ding{52}}{\Large \ding{36}}{\Large \ding{170}}{\Large \ding{73}} MathType est un éditeur d'équations puissant (logiciel utilisé pour le site Internet companion au présent livre) pour importer / exporter du MathML ou TeX que vous pouvez l'exécuter dans une application autonome. Avec une barre d'outils, ce logiciel s'intègre également parfaitement aux programmes de la suite Microsoft Office 2003-2013 (Word, Excel, PowerPoint), mais aussi à 500 autres logiciels (en particulier des logiciels de mathématiques). À la différence de \ \LaTeX 2 $\varepsilon$, vous ne perdrez pas d’heures pour résoudre les problèmes de compatibilité et de compilation.\\ 
	\href{http://www.dessci.com/en/}{\color{blue}http://www.dessci.com/en/}
	
	{\Large \ding{52}}{\Large \ding{36}}{\Large \ding{73}} Scilab (contraction de Scientific Laboratory) est un logiciel libre, développé à l'INRIA Rocquencourt (France). C'est un environnement de calcul numérique qui permet d'effectuer rapidement de nombreuses résolutions et représentations graphiques couramment rencontrées en mathématiques appliquées pour les gens ou petites entreprises n'ayant pas les moyens financiers d'acquérir MATLAB™.\\
	\href{http://www.scilab.org}{\color{blue}http://www.scilab.org}
	
	{\Large \ding{36}}{\Large \ding{73}} Site officiel du fameux logiciel de géométrie dynamique Cabri (référence dans le domaine). destiné principalement à l'apprentissage de la géométrie en milieu scolaire. Il permet d'animer des figures géométriques, au contraire de celles dessinées au tableau. Il se décline pour la géométrie plane ou pour la géométrie en 3D. C'est l'ancêtre de tous les logiciels de géométrie dynamique.\\
	\href{http://www.cabri.com}{\color{blue}http://www.cabri.com}
	
	{\Large \ding{36}}{\Large \ding{170}}{\Large \ding{73}}\bcdfrance{}  	Site Internet d'un prof de math ayant développé un très bon petit logiciel gratuit pour faire de nombreux plots, calculs d'intégrales, simulations de statistiques et probabilités pertinents très simplement et ludiquement en classe (niveau BAC/Matu).\\
	\href{http://www.patrice-rabiller.fr}{\color{blue}http://www.patrice-rabiller.fr}
	
	{\Large \ding{36}}{\Large \ding{73}} ACDLabs développe l'excellent logiciel Chemsketch pour la modélisation et la conception de molécules (2D, 3D) avec accès à certaines propriétés physiques et chimiques. C'est un outil très utile également pour préparer des documents ou des articles scientifiques (publications) dans le domaine de la chimie.\\
	\href{http://www.acdlabs.com}{\color{blue}http://www.acdlabs.com}
	
	{\Large \ding{36}}{\Large \ding{73}}{\Large \ding{170}} MolView est une application web intuitive, open-source, pour rendre la science de la chimie moléculaire plus intuitive!\\
	\href{http://molview.org}{\color{blue}http://molview.org}
	
	{\Large \ding{36}}{\Large \ding{73}} Excellent logiciel de conception de structures de génie civil de haut niveau basé sur les méthodes par éléments finis (pas mal utilisé en Suisse pour des projets simples ou complexes et pendant les études d'ingénieurs).\\
	\href{http://www.scia-online.com}{\color{blue}http://www.scia-online.com}
	
	{\Large \ding{36}}{\Large \ding{73}} Maxima est un logiciel libre de calcul formel, descendant sous licence GNU GPL du package Macsyma, le logiciel de calcul symbolique développé à l'origine pour les besoins du Département de l'Énergie américain. Maxima permet de faire du calcul sur les polynômes, les matrices, de l'intégration, de la dérivation, du calcul de séries, de limites, résolutions de systèmes, d'équations différentielles, etc.\\
	\href{http://maxima.sourceforge.net}{\color{blue}http://maxima.sourceforge.net}
	
	{\Large \ding{36}}{\Large \ding{73}} SPSS (Statistical Package for the Social Sciences) est a priori le logiciel le plus utilisé dans les entreprises en Suisse pour l'analyse statistique de données car il comporte un nombre impressionnant de tests ainsi que de nombreux packages métier. De par son prix et son propriétaire (IBM) on peut le considérer comme la solution haut de gamme de la statistique.\\
	\href{http://www.ibm.com/software/fr/analytics/spss}{\color{blue}http://www.ibm.com/software/fr/analytics/spss}
	
	{\Large \ding{36}}{\Large \ding{170}}{\Large \ding{73}} R est un langage puissant et un environnement pour le calcul statistique et les graphiques. R fournit une très grande variété de statistiques (modélisation linéaire et non linéaire, tests statistiques classiques, analyse de séries chronologiques, la classification, clustering, ...) et de techniques graphiques qui par ailleurs sont très extensible (accès au code source de toutes les fonctions est disponible!). Une des forces de R est la facilité avec laquelle on peut rédiger correctement des publications scientifiques de qualité.\\
	\href{http://www.r-project.org}{\color{blue}http://www.r-project.org}
	
	{\Large \ding{36}}{\Large \ding{73}} LabVIEW (Laboratory Virtual Instrument Engineering Workbench) est un puissant outil visuel ( langage de programmation graphique) de contrôle d'instruments de mesures ou de robots depuis un PC utilisé par énormément d'entreprises à traver à le monde (particulièrement par les ingénieurs) et développé par National Instruments.\\
	\href{http://www.ni.com/labview}{\color{blue}http://www.ni.com/labview}
	
	Et enfin une petite liste de quelques applications "scientifiques" pour iPhone / iPad (nous ne donnons pas de liens puisqu'il suffit de les trouver via le iTunes Store en écrivant leur nom dans l'outil de recherche):
	\begin{itemize}
		\item[$\bullet$] \textbf{Microsoft OneNote}: Pour prendre facilement des notes (au clavier ou stylet) et écrire des équations en pseudo LaTeX et les synchroniser avec votre ordinateur et le cloud.
		
		\item[$\bullet$] \textbf{iAnnote}: Super lecteur PDF avec de nombreuses fonctions pour annoter et modifier vos manuels de cours et publications scientifiques et gérer des livres volumineux.
		
		\item[$\bullet$] \textbf{WeatherProHD}: Pour les personnes qui aiment la météorologie et les prévisions météorologiques présentées de manière belle et professionnelle avec beaucoup d'options.
		
		\item[$\bullet$] \textbf{Analyser}: Pour ceux qui veulent lancer R ou Python sur leur iPad et leur iPhone. Application incroyable à la science des données en voyage!
		
		\item[$\bullet$] \textbf{TeX Writer}: Jusqu'à présent, la meilleure application \LaTeX{} que nous connaissons vous permettant d'écrire des équations et des documents de qualité sur vos appareils mobiles!
		
		\item[$\bullet$] \textbf{Telesat}: Pour identifier ou anticiper dans le ciel nocturne le passage de certains satellites bien connus.
		
		\item[$\bullet$] \textbf{ISS Dectector}: Même idée que pour Telesat mais pour la Station Spatiale Internationale (ISS).
		
		\item[$\bullet$] \textbf{Fligthradar}: Mêmes idées que pour Telesat et ISS Detector mais pour presque tous les avions civils (mais la version gratuite montre les avions avec un petit décalage pour des raisons de sécurité évidentes...).
		
		\item[$\bullet$]  \textbf{Sun Info}: Une application qui vous aide à anticiper la position du Soleil et de la Lune pendant de nombreuses années, ainsi que le solstice été / hiver, les prochaines éclipses, etc.
		
		\item[$\bullet$] \textbf{Stellarium}: Comme son nom l'indique, c'est un magnifique stellarium en temps réel et complet pour regarder les étoiles, les galaxies, les constellations et aussi certains satellites.
		
		 \item[$\bullet$] \textbf{C. Anatomy 19}: Une application (assez chère avec un abonnement mensuel / annuel) pour découvrir en haute résolution 3D l'intérieur du corps humain.
		 
		 \item[$\bullet$] \textbf{EarthViewer}: Une application pour découvrir et jouer avec les dérives des continents de la Terre à travers le temps.
		 
		 \item[$\bullet$] \textbf{MathStudio}: Une calculatrice arithmétique et algébrique très puissante avec des options de calcul similaires à celle de Maple.
		 
		 \item[$\bullet$] \textbf{PCalc Lite}: Un calculateur arithmétique complet très intuitif (mais évidemment moins puissant que MathStudio!).
		 
		 \item[$\bullet$] \textbf{Molecules}: Un visualiseur de molécules 3D où vous pouvez télécharger des molécules à partir d'une base de données en ligne.
		 
		 \item[$\bullet$] \textbf{E Numbers}: E Numbers vous donnent les codes pour les substances pouvant être utilisées comme additifs alimentaires.
		 
		 \item[$\bullet$] \textbf{iCircuit}: Une belle application pour créer des circuits électroniques simples et les simuler! Très bon pour apprendre les bases de l'électronique pendant les vacances sans avoir besoin de composants physiques réels.
		 
		 \item[$\bullet$] \textbf{Electronic TB}: Une application qui fournit une liste très importante (exhaustive) et des calculatrices pour de nombreux composants électroniques, y compris plusieurs de leurs propriétés techniques!
	\end{itemize}
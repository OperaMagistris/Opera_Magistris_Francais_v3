	%to make section start on odd page
	\newpage
	\thispagestyle{empty}
	\mbox{}
	\section{Impressum}	
	\subsection{Utilisation du contenu}

	Le contenu de ce livre est élaboré par un processus de développement par lequel les volontaires parviennent à un consensus. Ce processus qui rassemble des bénévoles, recherche aussi le point de vue des personnes intéressées par les sujets de ce livre. Le responsable de ce livre administre le processus et établit des règles pour promouvoir l'équité dans l'approche consensuelle. Il est également responsable de la rédaction du texte, parfois pour tester/évaluer ou vérifier de manière indépendante l'exactitude ou l'exhaustivité de l'information présentée.

	Nous déclinons toute responsabilité pour toute blessure, dommage ou tout autre type, spécial, accessoire, consécutif ou compensatoire, découlant de la publication, l'application ou la confiance du contenu de ce livre. Nous n'offrons aucune garantie expresse ou implicite quant à l'exactitude ou l'exhaustivité des informations publiées dans ce livre, et ne garantissons pas que les informations contenues dans ce livre répondent à un besoin spécifique ou à un objectif du lecteur. Nous ne garantissons pas la performance des produits ou services d'un fabricant ou d'un fournisseur uniquement en vertu du contenu de ce livre.
	
	Les descriptions techniques, les procédures et les programmes informatiques de ce livre ont été développés sans précaution pointue, ils sont donc fournis sans garantie d'aucune sorte. Nous ne garantissons pas non plus que les équations, les programmes et les procédures de ce manuel ou de ses logiciels associés sont exempts d'erreurs, ou sont conformes à une norme particulière de qualité marchande, ou répondront à vos exigences pour une application particulière. Ils ne devraient pas être invoqués pour résoudre un problème dont la solution incorrecte pourrait entraîner des blessures à des personnes ou la perte de biens. Toute utilisation du contenu de ce livre est aux risques et périls du lecteur. Les auteurs, rédacteurs et éditeurs déclinent toute responsabilité pour les dommages directs, indirects ou consécutifs résultant de l'utilisation du contenu de ce livre ou des logiciels, codes  qui y sont associés.

	En publiant des textes, il n'est pas dans l'intention de ce livre de fournir des services au nom d'une personne ou d'une entité ou d'accomplir une tâche à accomplir par une personne ou une entité au profit d'un tiers. Quiconque utilise ce livre devrait s'appuyer sur son propre jugement indépendant ou, le cas échéant, demander l'avis d'un expert (ou comité d'experts) qualifié pour déterminer comment faire preuve de diligence raisonnable dans toutes les circonstances. Les informations et les normes sur les sujets couverts par ce livre peuvent être disponibles à partir d'autres sources que le lecteur peut souhaiter parcourir à la recherche de points de vue ou d'informations supplémentaires non couvertes par le contenu de ce livre.

	Nous n'avons aucun pouvoir pour faire respecter le contenu de ce livre, et nous ne nous engageons pas à surveiller ou à faire respecter cette conformité. Nous n'avons aucune activité de certification, d'essai ou d'inspection de produits, de conceptions ou d'installations pour la sécurité ou la santé des personnes et des biens. Toute certification ou autre déclaration de conformité concernant les informations relatives à la santé ou à la sécurité des personnes et des biens, mentionnée dans ce livre, ne peut être attribuée au contenu de ce livre et reste sous la responsabilité du centre de certification ou du déclarant concerné.

	\subsection{Comment utiliser ce livre}
	Au niveau universitaire, ce livre peut être utilisé pour un doctorat, un diplôme d'études supérieures ou un séminaire avancé de premier cycle dans de nombreux domaines des sciences exactes et pures. Les séminaires où nous utilisons ce matériel font partie du programme de \textbf{Scientific Evolution Sàrl}, où les stagiaires ont généralement déjà suivi des cours de premier ou deuxième cycle dans leur spécialisation respective. En réalité, ce livre vise également à couvrir le programme complet de la maternelle au doctorat.

	Parce que les méthodes de mathématiques appliquées sont apprises par la pratique et l'expérience, nous considérons un séminaire sur les mathématiques appliquées comme un séminaire d'apprentissage par la pratique (axé sur les projets). Nous structurons nos séminaires de modélisation mathématique autour d'un ensemble de problèmes qui nécessitent que le stagiaire construise des modèles qui aident à la planification et à la prise de décision. L'impératif est que les modèles doivent être compatibles avec la théorie et revérifiés. Pour remplir cet impératif, il est nécessaire que le stagiaire combine la théorie mathématique avec la modélisation. Le résultat est que le stagiaire apprend la théorie, et plus important encore, apprend comment cette théorie est appliquée et combinée dans le monde réel. La capacité de critiquer et d'identifier les limites des outils mathématiques dangereux est la caractéristique la plus importante de nos séminaires.

	Les problèmes avec solutions présentées dans ce livre fournissent l'occasion d'appliquer le matériel du texte à un ensemble complet de situations assez réalistes (bien que simplifiées). À la fin des séminaires ainsi qu'à la fin de la lecture de ce livre, les stagiaires/lecteurs auront amélioré leurs compétences et leurs connaissances des outils théoriques et informatiques les plus importants et auront fortement développé leur esprit analytique. Ce sont des compétences précieuses qui sont demandées par les entreprises et administrations au plus haut niveau.

	Il est très difficile de couvrir tout le matériel de ce livre en un semestre. Il faut beaucoup de temps pour expliquer les concepts aux stagiaires. Le lecteur est invité à choisir les sujets qui seront abordés pendant le trimestre. Il n'est pas strictement nécessaire de les couvrir dans l'ordre, mais cela peut aider de manière significative?

	En un mot, ce livre vous offre une grande variété de sujets qui peuvent être modélisés. Tous, sans exception, sont utiles dans la pratique.
	
	\pagebreak
	\subsubsection{Annexes}
	Nous offrons gratuitement une gamme d'annexes (ouvrages compagnons) pour les étudiants, les instructeurs et les praticiens.
	
	D'abord, il y a des livres et des outils gratuits en français et en anglais écrits par Vincent ISOZ et Daname KOLANI pour les personnes qui veulent mettre en pratique la théorie présentée dans ce livre.
	
	Voici la liste:
	\begin{itemize}
		\item MATLAB™ en anglais (1,361 pages):\\ \href{http://www.sciences.ch/htmlfr/php/cliccount/click.php?id=319}{http://www.sciences.ch/dwnldbl/divers/Matlab.pdf}
		
		\item Maple en français (99 pages):\\ \href{http://www.sciences.ch/dwnldbl/divers/Maple.pdf}{http://www.sciences.ch/dwnldbl/divers/Maple.pdf}
		
		\item \textsf{R} en français (2,100 pages):\\ \href{http://www.sciences.ch/htmlfr/php/cliccount/click.php?id=313}{http://www.sciences.ch/dwnldbl/divers/R.pdf}
		
		\item Minitab en français (1,118 pages):\\ \href{http://www.sciences.ch/htmlfr/php/cliccount/click.php?id=282}{http://www.sciences.ch/dwnldbl/divers/Minitab.pdf}
		
		\item Scientific Linux installation \& Configuration en anglais  (211 pages):\\ \href{http://www.sciences.ch/dwnldbl/divers/ScientificLinux.pdf}{http://www.sciences.ch/dwnldbl/divers/ScientificLinux.pdf}
	\end{itemize}
	\begin{center}
		\includegraphics[scale=0.75]{img/books/matlab.jpg}
		\includegraphics[scale=0.75]{img/books/maple.jpg}
		\includegraphics[scale=0.75]{img/books/r.jpg}
		\includegraphics[scale=0.75]{img/books/minitab.jpg}
		\includegraphics[scale=0.75]{img/books/scientificlinux.jpg} 
	\end{center}	
	En second lieu, nous vous proposons gratuitement quelques Quizz et Flashcards en français et en anglais pour défier vos étudiants ou vous-même avec le reste du monde:
	 \begin{itemize}
		\item Quizz MATLAB™ bases niveau L1 en français (100 questions)\\ \url{http://www.scientific-evolution.com/qcm/start_session/a73647cf3b/}
		
		\item Quizz Astonomie/Astrophysique niveau H1 en anglais (100 questions):\\ \url{http://www.scientific-evolution.com/qcm/start_session/ffd0810fa0/}
		
		\item Cartes de révision pour les lettres grecques (48 cartes):\\
		\url{http://www.scientific-evolution.com/qcm/fr/start_session/6d9f1fef90/}
		
		\item Cartes de révisions pour les dérivées les plus communes (29 cartes):\\
		\url{http://www.scientific-evolution.com/qcm/fr/start_session/c15a40f2c4/}
		
		\item Cartes de révisions pour les intégrales les plus communes (60 cartes):\\
		\url{http://www.scientific-evolution.com/qcm/fr/start_session/ccfc20fdef/}
		
		\item Cartes de révisions pour les identités trigonométriques les plus communes (68 cartes):\\
		\url{http://www.scientific-evolution.com/qcm/fr/start_session/882f9696cd/}
		
		\item Quizz \LaTeX{} niveau L3 en français (100 questions):\\ \url{http://www.scientific-evolution.com/qcm/fr/start_session/ff1e1d1b91/}
		
		\item Quizz R Software 3.1.2 niveau L3 en français (100 questions):\\ \url{http://www.scientific-evolution.com/qcm/fr/start_session/2a6fca7473/}
		
		\item Quizz C++ nivau en français (100 questions):\\
		\url{http://www.scientific-evolution.com/qcm/fr/start_session/e031ce4b43/}
	\end{itemize}
	Et comme tout livre technique devrait avoir un forum, le lecteur peut passer par ce lien pour toute discussion sur le contenu de ce livre: \url{https://www.physicsforums.com}.
	
	Pour ceux qui préfèrent les réseaux sociaux, nous avons aussi une page Facebook dédiée:
	\begin{center}
		\faFacebook{} \href{https://www.facebook.com/operamagistris/}{https://www.facebook.com/operamagistris/}
	\end{center}
	ou pour plus de plaisir (photos de science, citations, blagues, vidéos, etc.) il y a aussi un compte Instagram associé:
	\begin{center}
		\faInstagram{} \href{https://www.instagram.com/opera.magistris/}{https://www.instagram.com/opera.magistris/}
	\end{center}
	et une peite collection non-exhaustive d'une sélection de ce que nous considérons comme des vidéos scientifiques intéressantes sur notre chaîne YouTube (voir à la fin de cette livre une section de chaînes YouTube scientifiques intéressantes):
	\begin{center}
		\faYoutube{} \href{https://www.youtube.com/user/AdminSciences}{https://www.youtube.com/user/AdminSciences}
	\end{center}
	Comme pour le présent livre, les annexes et compléments ci-dessus ne sont que des échantillons. Les versions complètes avec \underline{mises à jour gratuites perpétuelles} sont disponible pour le prix de $299$.- chacun et pour $499$.. vous obtenez les fichiers d'exercice et les sources \LaTeX{} (pour les informations sur l'achat, vous pouvez simplement m'envoyer un e-mail au {\href{mailto:isoz@sciences.ch}{{\color{blue}email}}}).
	
	Pour les personnes qui veulent aider à traduire ce livre en d'autres langues, voici le lien avec les sources \ LaTeX {} pour les textes originaux (mais sans les équations\footnote{Les équations ne sont pas fournies car souhaitons contrôler qui fait quoi avec le livre!}!) sur GitHub:
	\begin{center}
		\faGithubSquare{} \href{https://github.com/OperaMagistris}{https://github.com/OperaMagistris}
	\end{center}
	Parce que ce livre se concentre principalement sur l'aspect mathématique (technique) des phénomènes physiques, nous ne pouvons que recommander fortement au lecteur un autre livre gratuit qui est de notre point de vue subjectif actuellement le meilleur sur l'aspect scientifique populaire des sujets que nous couvrirons:
	\begin{center}
	Motion Mountain par le Dr. Christoph Schiller: \url{http://www.motionmountain.net}
	\end{center}
	\begin{center}
		\includegraphics[scale=1]{img/books/motion_mountain.jpg}
	\end{center}

	\pagebreak
	\subsection{Protection des données}
	Lorsque vous consultez des informations sur le site Internet compagnon (Sciences.ch), certaines données sont automatiquement sauvegardées. Nous essayons de sauvegarder le moins de données possible et ce le plus rapidement possible. Partout où nous pouvons, nous n'avons que des données anonymes. Nous nous engageons à traiter les données que vous nous envoyez personnellement avec la plus grande diligence.

	Cependant, votre adresse IP et la page source qui vous emmène sur Sciences.ch et les mots-clés associés, sont disponibles gratuitement pour tout le monde \href{http://www.sciences.ch/htmlfr/php/cedstat/references.php}{ici} pour le mois en cours. Après quoi les données détaillées sont détruites. Vous pouvez vous opposer à tout moment à la publication de vos données en nous contactant.

	\subsection{Utilisation des données}
	Vos données ne sont utilisées que pour l'envoi du bulletin d'information (newletter) de Sciences.ch. La communication des données personnelles (à l'exception de l'adresse e-mail, du titre et du nom) est facultative. Lorsque vous vous inscrivez à la newsletter, vous pouvez bien sûr spécifier une autre adresse et / ou un nom fictif.

	\subsection{Transmission des données}
	Nous ne vendrons ni ne commercialiserons jamais les données de nos clients ou parties intéressées et n'affecterons jamais les droits de la personne. En outre, nous ne louerons pas de listes de diffusion et ne vous enverrons pas de publicité de la part de tiers ou en notre nom.

	\subsection{Accord}
	Lorsque vous nous fournissez des informations personnelles, vous nous autorisez à les sauvegarder et à les utiliser au sens de la loi fédérale suisse sur la protection des données. Si vous nous demandez de ne pas vous envoyer d'e-mails (courriels), nous sommes dans l'obligation, dans votre intérêt, d'enregistrer votre e-mail (courriel) dans une liste négative interne.
	
	\subsection{Errata}
	Bien que nous ayons pris toutes les précautions pour assurer l'exactitude de notre contenu, des erreurs humaines peuvent se produire. Si vous trouvez une erreur dans ce livre - peut-être une erreur dans le texte, les scripts ou les illustrations - nous vous serions reconnaissants de nous le signaler. Ce faisant, vous pouvez éviter la frustration d'autres lecteurs en nous aidant à améliorer les versions ultérieures de ce livre. Notre e-mail est donné sur le pied de page à chaque page de ce livre. Une fois vos errata vérifiés, vos soumissions seront acceptées et l'erreur sera visible dans le journal des modifications des versions de mise à jour.

	%to make section start on odd page
	\newpage
	\thispagestyle{empty}
	\mbox{}
	\section{Licence}
	Tout le contenu de ce livre est soumis à la licence de documentation libre GNU, ce qui signifie:
	\begin{itemize}
			\item[$\bullet$] que chacun a le droit d'utiliser librement les textes à des fins non commerciales (Google Ads ou tout équivalent étant considéré comme un usage commercial!)
			\item[$\bullet$] que toute personne est autorisée à diffuser des articles pour un usage non commercial (Google Ads ou tout équivalent étant considéré comme un usage commercial!)
			\item[$\bullet$] que n'importe qui peut éditer librement les textes pour un usage non commercial (Google Ads ou tout équivalent étant considéré comme un usage commercial!)
	\end{itemize}
	
	et bla bla bla...

	conformément à la licence décrite ci-dessous: 
	
	\begin{center}
	\url{https://creativecommons.org/licenses/by-nc-sa/4.0/}\\[2pt]
\includegraphics[width=1cm]{img/icons/share2.eps}\includegraphics[width=1cm]{img/icons/remix2.eps}
\includegraphics[width=1cm]{img/icons/by2.eps}
\includegraphics[width=1cm]{img/icons/nc-eu2.eps}
\includegraphics[width=1cm]{img/icons/sa2.eps}
	\end{center}

	\begin{center}
	Version 1.1, Mars 2000
		
	Copyright (C) 2000 Fondation du Logiciel Libre, Inc. 59 Temple Place, Suite 330, Boston, MA 02111-1307 USA Tout le monde est autorisé à copier et à distribuer des copies verbatim de ce document de licence, mais il n'est pas autorisé de le modifier. 
	\end{center}

	\subsection{Préambule} 
	Le but de cette Licence est de rendre un manuel, une référence ou un autre document écrit "libre" dans le sens de la liberté: assurer à chacun la liberté effective de le copier et de le redistribuer, avec ou sans le modifier à des fins non commerciales \footnote{L'avantage immédiat est que vous pouvez générer le livre, avec de nouveaux ensembles de problèmes, et le distribuer à vos étudiants simplement en format PDF (dans un courriel, par exemple). Mais plus généralement, si vous n'êtes pas intéressé par la façon dont nous avons expliqué (ou omis d'expliquer) quelque chose, alors vous êtes libre de le réécrire. Si vous souhaitez couvrir plus de sujets (ou moins), vous êtes libre d'ajouter (ou de supprimer) les chapitres / sections / paragraphes que vous souhaitez. Et puisque vous avez le code source, vous n'avez pas besoin de recréer la roue.} (Sources disponibles sur \ href {https://github.com/vincentisoz} {Github}). En second lieu, cette Licence conserve pour l'auteur et l'éditeur un moyen d'obtenir un crédit pour leur travail, tout en n'étant pas considéré comme responsable des modifications apportées par d'autres.

	Cette Licence est une sorte de "copyleft", ce qui signifie que les oeuvres dérivées du Document doivent elles-mêmes être libres dans le même sens. Il complète la licence publique générale GNU, qui est une licence copyleft conçue pour un logiciel libre.

	Nous avons conçu cette Licence afin de l'utiliser pour les manuels de logiciels libres, car le logiciel libre a besoin de documentation gratuite: un programme gratuit devrait être fourni avec des manuels offrant les mêmes libertés que le logiciel. Mais cette Licence n'est pas limitée aux manuels de logiciels; il peut être utilisé pour n'importe quel travail textuel, indépendamment du sujet ou s'il est publié comme un livre imprimé. Nous recommandons cette Licence principalement pour les travaux dont l'objectif est l'instruction ou la référence.

	\subsection{Applications et Définitions}
	Cette Licence s'applique à tout travail manuel ou autre contenant un avis placé par le détenteur des droits d'auteur indiquant qu'il peut être distribué selon les termes de cette Licence. Le "Document", ci-dessous, fait référence à un tel manuel ou travail. Tout membre du public est un titulaire de Licence et est appelé «vous».

	Une "Version Modifiée" du Document désigne tout travail contenant le Document ou une partie de celui-ci, soit copié textuellement, soit avec des modifications et / ou traduit dans une autre langue.

	Une «section secondaire» est une annexe nommée ou une section de première instance du Document qui traite exclusivement de la relation entre les éditeurs ou les auteurs du Document à la matière globale du Document (ou à des questions connexes) et ne contient rien qui pourrait tomber directement dans ce sujet global (par exemple, si le Document est en partie un manuel de mathématiques, une section secondaire pourrait ne pas expliquer les mathématiques). La relation pourrait être une question de connexion historique avec le sujet ou avec des questions connexes, ou juridique, commercial, philosophique, position éthique ou politique à leur égard.
	
	Les "Sections Invariantes" sont certaines Sections Secondaires dont les titres sont désignés, comme étant ceux des Sections Invariantes, dans l'avis qui dit que le Document est publié sous cette Licence.

	Les "Textes de couverture" sont de courts passages de texte qui sont listés, en tant que Textes d'ouverture ou Textes de clôture, dans l'avis indiquant que le Document est publié sous cette Licence.

	Une copie "Transparente" du Document désigne une copie lisible par machine, représentée dans un format dont la spécification est accessible au grand public, dont le contenu peut être visualisé et édité directement et directement avec des éditeurs de texte génériques ou (pour des images composées de pixels) programmes de peinture génériques ou (pour les dessins) un éditeur de dessin largement disponible, et qui est adapté pour l'entrée aux formateurs de texte ou pour la traduction automatique à une variété de formats appropriés pour l'entrée aux formateurs de texte. Une copie faite dans un format de fichier autrement transparent dont le balisage a été conçu pour contrecarrer ou décourager la modification ultérieure par les lecteurs n'est pas transparent. Une copie qui n'est pas "transparente" est appelée "opaque".

	Des exemples de formats appropriés pour les copies transparentes incluent l'ASCII simple sans balisage, le format d'entrée Texinfo, le format d'entrée LaTeX, SGML ou XML utilisant une DTD accessible au public et un HTML simple conforme aux normes conçu pour la modification humaine. Les formats opaques incluent PostScript, PDF, formats propriétaires qui ne peuvent être lus et édités que par des traitements de texte propriétaires, SGML ou XML pour lesquels la DTD et / ou les outils de traitement ne sont généralement pas disponibles, et HTML produit par certains traitements de texte pour fins de sortie seulement.

	La «page de titre» désigne, pour un livre imprimé, la page de titre elle-même, ainsi que les pages suivantes qui sont nécessaires pour contenir, de manière lisible, le matériel dont cette Licence a besoin pour apparaître dans la page de titre. Pour les œuvres dans des formats qui n'ont pas de page de titre en tant que telle, "Page de titre" signifie le texte près de l'apparence la plus visible du titre de l'oeuvre, précédant le début du corps du texte.

	\subsection{Copie Verbatim} 
	Vous pouvez copier et distribuer ce libre sur n'importe quel support, non commercial, à condition que cette Licence, les avis de copyright et l'avis de licence indiquant que cette Licence s'applique au livre soient reproduits dans toutes les copies et que vous n'ajoutez aucune autre condition à celles de cette Licence. Vous ne pouvez pas utiliser des mesures techniques pour entraver ou contrôler la lecture ou la copie ultérieure des copies que vous faites ou distribuez. Cependant, vous pouvez accepter une compensation en échange de copies. Si vous distribuez un nombre suffisant de copies, vous devez également respecter les conditions de la section 3.

	Vous pouvez également prêter des copies, dans les mêmes conditions que celles indiquées ci-dessus, et vous pouvez en afficher publiquement des copies. 

	\subsection{Copie en Quantité}
	Si vous publiez des copies imprimées du Document en quantité supérieurs à $100$ et que l'avis de Licence du Document exige des textes de couverture, vous devez joindre les copies dans des pochettes qui portent, clairement et lisiblement, tous ces textes de couverture: les textes de la première de couverture sur la page de couverture. les textes de la dernière de couverture sur la couverture arrière. Les deux couvertures doivent également vous identifier clairement et lisiblement en tant qu'éditeur de ces copies. La couverture doit présenter le titre complet avec tous les mots du titre tout aussi visible l'un que l'autre. Vous pouvez ajouter d'autres informations sur les couvertures en plus. La copie avec des modifications limitées aux couvertures, tant qu'elles conservent le titre du Document et satisfont à ces conditions, peut être considérée comme une reproduction textuelle à d'autres égards.

	Si les textes requis pour l'une ou l'autre des couvertures sont trop volumineux pour être lisibles, vous devez mettre les premiers inscrits (autant que possible) sur la couverture actuelle, et continuer le reste sur les pages suivantes.

	Si vous publiez ou distribuez des copies opaques en quantité supérieure à $100$, vous devez inclure une copie transparente lisible par machine avec chaque copie opaque, ou indiquer dans chaque copie opaque un emplacement réseau informatique accessible au public contenant une copie complète transparente du Document, sans matériel ajouté et de manière à ce que le public utilisant un un réseau général y ait accès de façon anonyme, sans frais et en utilisant des protocoles de réseau public ou standard. Si vous utilisez cette dernière option, vous devez prendre des mesures raisonnablement prudentes lorsque vous commencez la distribution de copies opaques en quantité, pour vous assurer que cette copie transparente restera accessible à l'emplacement indiqué jusqu'à au moins un an après la dernière distribution d'un copie opaque (directement ou par l'intermédiaire de vos agents ou détaillants) de cette édition au public.

	Il est demandé, mais pas obligatoire, que vous contactiez les auteurs du Document bien avant de redistribuer un grand nombre de copies, pour leur donner une chance de vous fournir une version mise à jour du Document.

	\subsection{Modifications}
	Vous pouvez copier et distribuer une version modifiée du Document dans les conditions des sections 2 et 3 ci-dessus, à condition de publier la Version Modifiée sous cette Licence, avec la version modifiée remplissant le rôle du Document, autorisant ainsi la distribution et la modification de la Version Modifiée à quiconque en possède une copie. En outre, vous devez effectuer ces opérations dans la version modifiée:
	
	\begin{itemize}
		\item Utilisez dans la page de titre (et sur les couvertures, le cas échéant) un titre distinct de celui du Document, et de ceux des versions précédentes (qui devraient, le cas échéant, figurer dans la section Historique du document). Vous pouvez utiliser le même titre que la version précédente si l'éditeur d'origine de cette version donne son autorisation.

		\item Lister sur la page de titre, en tant qu'auteurs, une ou plusieurs personnes ou entités responsables de la paternité des modifications de la Version Modifiée, ainsi qu'au moins cinq des auteurs principaux du document (tous ses auteurs principaux, s'il y en a moins de cinq).

		\item Indiquer sur la page de titre le nom de l'éditeur de la version modifiée, en tant qu'éditeur.

		\item Conserver tous les avis de droits d'auteur du document.

		\item Ajoutez un avis de droit d'auteur approprié pour vos modifications à côté des autres avis de droits d'auteur. Add an appropriate copyright notice for your modifications adjacent to the other copyright notices. 

		\item Inclure, immédiatement après les avis de droits d'auteur, un avis de licence autorisant le public à utiliser la Version Modifiée conformément aux conditions de la présente Licence, sous la forme indiquée dans l'addenda ci-dessous.

		\item Préservez dans cet avis de licence les listes complètes des sections invariantes et les textes de couverture requis indiqués dans l'avis de Licence du Document.

		\item Inclure une copie non modifiée de cette licence.

		\item Conservez la section intitulée «Historique des révisions» et son titre, et ajoutez-y un élément indiquant au moins le titre, l'année, les nouveaux auteurs et l'éditeur de la version modifiée comme indiqué sur la page de titre. S'il n'y a pas de section intitulée «Historique» dans le document, créez-en un en indiquant le titre, l'année, les auteurs et l'éditeur du document comme indiqué sur sa page de titre, puis ajoutez un élément décrivant la version modifiée.

		\item Conservez l'emplacement réseau, le cas échéant, indiqué dans le document pour l'accès public à une copie transparente du Document, ainsi que les emplacements réseau indiqués dans le document pour les versions précédentes sur lesquelles il était basé. Ceux-ci peuvent être placés dans la section "Historique". Vous pouvez omettre un emplacement réseau pour un travail publié au moins quatre ans avant le document lui-même ou si l'éditeur d'origine de la version à laquelle il fait référence donne son autorisation.

		\item Dans toute section intitulée «Remerciements» ou «Dédicaces», conserver le titre de la section et conserver dans la section toute la substance et le ton de chacun des remerciements et / ou dédicaces du contributeur qui y sont donnés.

		\item Conserver toutes les sections invariables du Document, inchangées dans leur texte et dans leurs titres. Les numéros de section ou l'équivalent ne sont pas considérés comme faisant partie des titres de section.

		\item  Supprimer toute section intitulée "Avenants". Une telle section peut ne pas être incluse dans la Version Modifiée.

		\item Ne retitrez aucune section existante en tant que "Avenants" ou tout autre section qui pourrait générer un conflit dans le titre avec une Section Invariante.

		\item Si la version modifiée inclut de nouvelles sections ou appendices qualifiées en tant que Sections Secondaires et ne contient aucun contenu copié à partir du Document original, vous pouvez, à votre discrétion, désigner certaines ou toutes ces sections comme invariantes. Pour ce faire, ajoutez leurs titres à la liste des sections invariantes dans l'avis de licence de la Version Modifiée. Ces titres doivent être distincts des autres titres de section.

		\item Vous pouvez ajouter une section intitulée «Avenants», à condition qu'elle ne contienne que des approbations de votre version modifiée par diverses parties - par exemple, des déclarations d'examen par les pairs ou que le texte a été approuvé par une organisation comme définition officielle d'une norme.

		\item Vous pouvez ajouter jusqu'à cinq mots comme texte de première de couverture et jusqu'à 25 mots comme texte de quatrième couverture à la fin de la liste des textes de ces deux couverture dans la Version Modifiée. Un seul passage de texte de premère de couverture et un de texte de quatrième de couverture peuvent être ajoutés par (ou par des arrangements faits par) une entité. Si le Document contient déjà un texte de couverture pour les couvertures sus-mentionnées, précédemment ajouté par vous ou par un arrangement conclu avec la même entité sous le compte de laquelle vous agissez, vous ne pouvez pas en ajouter un autre; mais vous pouvez remplacer l'ancien, avec l'autorisation explicite de l'éditeur précédent qui a ajouté l'ancien.

		\item Le(s) auteur(s) et le(s) éditeur(s) du document ne donnent pas la permission d'utiliser leur nom à des fins publicitaires, d'affirmer ou de sous-entendre l'approbation d'une Version Modifiée.
	\end{itemize} 

	\subsection{Combinaison de Documents}
	Vous pouvez combiner le Document avec d'autres documents publiés également sous cette même Licence, selon les termes définis dans la section 4 ci-dessus pour les Versions Modifiées, à condition d'inclure dans la combinaison toutes les Sections Invariantes de tous les documents originaux, non modifiés, et de les lister tous comme Sections Invariantes de votre travail combiné dans son avis de Licence.

	Le travail combiné doit seulement contenir une copie de cette Licence, et plusieurs Sections Invariantes identiques peuvent être remplacées par une seule copie. S'il y a plusieurs Sections invariantes avec le même nom mais des contenus différents, rendez le titre de chaque section unique en ajoutant à la fin, entre parenthèses, le nom de l'auteur original ou de l'éditeur de cette section s'il est connu, ou bien un nombre unique. Apportez le même ajustement aux titres de section dans la liste des sections invariables dans l'avis de Licence du travail combiné.
	
	Dans la combinaison, vous devez combiner toutes les sections intitulées «Historique» dans les divers documents originaux, en formant une section intitulée «Historique»; De même, vous devez combiner toutes les sections intitulées «Remerciements» et toutes les sections intitulées «Dédicaces». Vous devez supprimer toutes les sections intitulées "Avenants".

	\subsection{Collections de Documents}
	Vous pouvez faire une collection constituée du Document et d'autres documents publiés sous cette Licence, et remplacer les copies individuelles de cette Licence dans les divers documents par une seule copie qui est incluse dans la collection, à condition que vous respectiez les règles de cette Licence pour chaque copie textuelle de chacun des documents à tous égards.

	Vous pouvez extraire un seul document d'une telle collection et le distribuer individuellement sous cette Licence, à condition d'insérer une copie de cette Licence dans le document extrait, et de suivre cette Licence à tous égards concernant la copie textuelle de ce document.

	\subsection{Agrégation avec des œuvres indépendantes} 
	Une compilation du Document ou de ses dérivés avec d'autres documents ou travaux séparés ou indépendants, dans ou sur un volume d'un support de stockage ou de distribution, ne compte pas dans son ensemble comme Version Modifiée du Document, à condition qu'aucun droit de compilation ne soit réclamé pour la compilation. Une telle compilation est appelée un "agrégat", et cette Licence ne s'applique pas aux autres oeuvres autonomes ainsi compilées avec le Document, du fait qu'elles sont ainsi compilées, si elles ne sont pas elles-mêmes des oeuvres dérivées du Document.

	Si l'exigence du texte de couverture de la section 3 s'applique à ces copies du Document, alors si le document représente moins du quart de l'ensemble, les textes de couverture du Document peuvent être placés sur des couvertures qui entourent uniquement le Document à l'intérieur de l'aggrégat. Sinon, ils doivent apparaître sur les couvertures autour de l'ensemble des agrégats.
	
	\subsection{Compilation du Document}
	Vous aurez besoin de trois trois choses pour générer ce document pour vous-même:
	\begin{enumerate}
		\item Installer \href{https://miktex.org/}{MiKTeX}

		\item Installer \href{http://www.xm1math.net/texmaker/index_fr.html}{TeXMaker}

		\item Une connexion Internet

		\item Configurer MikTeX comme indiqué dans les remarques au début du fichier \textit{LaTeX\_SciencesCh\_FR.tex}
	\end{enumerate}

	\subsection{Traductions}
	La traduction est considérée comme une sorte de modification, vous pouvez donc distribuer des traductions du document sous les termes de la section correspondante sur la transformation. Le remplacement de sections invariables par des traductions nécessite une autorisation spéciale de leurs détenteurs de droits d'auteur, mais vous pouvez inclure des traductions de certaines ou de toutes les sections invariables en plus des versions originales de ces sections invariables.

	\subsection{Résiliation}
	Vous n'êtes pas autorisé à copier, modifier, sous-licencier ou distribuer le Document, sauf dans les cas expressément prévus par cette Licence. Toute autre tentative de copier, modifier, sous-licencier ou distribuer le document est annulée et met fin automatiquement à vos droits en vertu de cette Licence. Cependant, les parties qui ont reçu des copies ou des droits de votre part en vertu de cette Licence ne verront pas leur licence résiliée tant que ces parties resteront en pleine conformité.

	\subsection{Révisions futures de cette Licence}
	La Free Software Foundation peut publier de temps à autre de nouvelles versions révisées de la Licence de documentation libre GNU. Ces nouvelles versions seront similaires dans l'esprit à la présente version, mais peuvent différer dans le détail pour répondre à de nouveaux problèmes ou préoccupations. Voir \href{http://www.gnu.org/copyleft/}{{\color{blue} http://www.gnu.org/copyleft/}}.

	Chaque version de la Licence reçoit un numéro de version distinctif. Si le Document spécifie qu'une version numérotée particulière de cette Licence "ou toute version ultérieure" s'applique à elle, vous avez la possibilité de suivre les termes et conditions de cette version spécifiée ou de toute version ultérieure qui a été publiée (et non en tant que brouillon) par la Free Software Foundation. Si le Document ne spécifie pas le numéro de version de cette Licence, vous pouvez choisir n'importe quelle version jamais publiée (pas comme brouillon) par la Free Software Foundation.
	
	\begin{center}
	\color{ForestGreen}{{\Large \faTree} \textbf{S'il-vous-plaît, pensez à l'environnement avant d'imprimer}}
	\end{center}
	
	%to make section start on odd page
	\newpage
	\thispagestyle{empty}
	\mbox{}
	\section{Feuille de route}
	Ce livre a une règle de progression simple qui est: $1$ nouvelle page A4 par jour depuis Mai 2001 sur des sujets qui intéressent le superviseur de la distribution \textit{Originale} du livre \textit{Opera Magistris}. Les nouveaux sujets sont débloqués (publiés) en fonction des paliers de dons qui sont faits via note page \href{https://www.tipeee.com/elements-of-applied-mathematics}{Tipeee} ou \href{https://www.patreon.com/sciences}{Patreon}. Les sujets suivants sont déjà prévus pour un futur proche ou lointain avec toujours le même niveau de détails et d'approche pédagogique dans les preuves mathématiques que le reste du livre (tous les sujets ci-dessous devraient nécessiter environ $1'500$ à $2'500$ pages supplémentaires):
	\begin{itemize}
		\item Introduction:
			\begin{itemize}
				\item Ajouter des remerciements à tous les scientifiques qui ont contribué aux théorèmex et aux modèles présentés dans ce livre (y compris les auteurs de livres / articles)
				\item Ajouter des remerciements à toutes les personnes qui ont créé la distribution MikTeX / LaTeX et aux packages utilisés pour ce livre
				\item
			\end{itemize}
		\item Probabilités:
			\begin{itemize}
				\item Conjugaison Bayésienne pour la loi Normale et Binomiale
				\item Chaînes de Markove cachées
				\item Log-loss
			\end{itemize}
		\item Statistiques: 
			\begin{itemize}
				\item Mode et Médiane de lois statistiques
				\item Semi-variance	
				\item Corrélation partielle et Semi-partielle
				\item M-Estimateurs pour la localisation et la dispersion
				\item Maximum de vraisemblance pour données censurées
				\item Entropie de la loi Normale
				\item Score de Propension
				\item Test d'Équivalence
				\item Matrice de Quasi-corrélation
				\item Analyse Factorielle
				\item T-Test de Hotelling
				\item Résidus standardisés de Pearson
				\item Test the Welch avec équation de Welch-Satterhwaite
				\item ANCOVA
				\item ANOVA de type I, II, III et IV
				\item Test de Wald-Wolfowitz Test (des séquences binaires)
				\item Test de Levene-Wolfwitz\footnote{aussi appelé  "test de point de bifurcation" ou "test de tendance"} (séquences haussière/baissière)
				\item Odds Ratio (rapport des chances) et son intervalle de confiance
				\item Risk Ratio (ratio de risque) et son intervalle de confiance
				\item Ellipse de contrôle
				\item Mesures basées sur l'entropie des tables de contingence
				\item Test de tendance d'Armitage
				\item Test d'Ansari-Bradley
				\item Test régulier de Dickey-Fuller
				\item Modèle de Poisson pour la distance spatiale moyenne (2D)
				\item Corrélation canonique
				\item G-test de périodicité
				\item Copula Gaussien et de Student
				\item ANOVA à facteur fixe hiérarchique
				\item ANOVA carré latin sans réplication
				\item Introduction à la MANOVA
				\item Théorème des valeurs extrêmes
				\item Théorie des sondages
				\item Modèles linéaires généralisés (Gauss, Poissson, Binomial Négatif, Gamma)
				\item Régression PLS (moindres carrés partiels)
				\item Moindres carrés en deux étapes (2SLS)
				\item Régression logique
				\item Chi-carré ajusté
				\item Probabilité des fonctions génératrices
			\end{itemize}
		\item Géométrie:
			\begin{itemize}
				\item Volume de l'hypersphère
			\end{itemize}
		\item Calcul Différentiel
			\begin{itemize}
				\item Intégrale de Lebesgue avec application numérique dans MATLAB™
				\item Représentation intégrale des fonctions de Bessel
				\item Linéarisation des puissances des fonctions trigonométriques
				\item Intégrales elliptiques et fonction elliptiques
			\end{itemize}
		\item Analyse: 
			\begin{itemize}
				\item Transformée de Hilbert
			\end{itemize}
		\item Analyse Complexe: 
			\begin{itemize}
				\item Théorème des résidus pour des ratios de polynômes
				\item Théorème de la valeur moyenne de Gauss
			\end{itemize}
		\item Topologie: 
			\begin{itemize}
				\item Distance de Mahalanobis
			\end{itemize}			
		\item Géométrie Différentielle: 
			\begin{itemize}
				\item Coordonnées normales
				\item Courbure de Gauss
				\item Périmètre d'un circle sur le plan, sur la sphère et sur une surface hyperbolique
				\item Théorème isopérimètrique du plan
			\end{itemize}
		\item Mécanique: 
			\begin{itemize}
				\item Effet Magnus
				\item Critère de Lawson (plasmas)
				\item Écoulement à travers un orifice submergé
				\item Écoulement sur les encoches et les déversoirs
				\item Force due à l'écoulement autour d'un coude de tuyau
				\item Force sur une buse
				\item Impact d'un jet sur un avion
				\item Turbine Pelton
				\item Force due à un jet atteignant un plan incliné
				\item Effet bélier d'un fluide
				\item Équations de Saint-Venant
				\item Portance de Kutta-Joukowski
			\end{itemize}	
		\item Électrodynamique:
			\begin{itemize}		
				\item Modèle de l'oscillateur de Lorentz
				\item Champs électromagnétiques d'une sphère de charges en rotation
				\item Équation de Sellmeier
				\item Rayonnement des antennes
				\item Brehmstrahlung (commencé mais pas fini)
				\item Principe de l'interféromètre de Michelson
				\item Fibre optique
				\item Couple de rotor d'un électroaimant
				\item Diffusion de Rayleigh
			\end{itemize}
		\item Électrocinétique:
			\begin{itemize}		
				\item Convertisseur photoélectrique
				\item Jonctions PN
			\end{itemize}
		\item Optique Ondulatoire:
			\begin{itemize}		
				\item Diffraction de Fresnel
				\item Diffraction de Fraunhoffer
				\item Fibre optique (bases)
			\end{itemize}
		\item Astronomie:
			\begin{itemize}	
				\item Formule de MacCullagh's 
				\item Calcul indirect de l'aplattissement des corps célestes
				\item Verrouillage synchrone des satellites en révolution	
				\item Aplatissement des sphéroïdes
				\item Sphère d'influence
				\item Échappement planétaire
			\end{itemize}		
		\item Relativité Générale:
			\begin{itemize}
				\item Volume réel d'un objet en relativité générale
				\item Dérivation du rayon d'Einstein
				\item Théorème général de Birkhoff
				\item Global Positioning System (GPS)
				\item Métrique et solution de Kerr	
			\end{itemize}
		\item Cosmologie:
			\begin{itemize}
				\item Finir le texte sur les Univers de Friedman basés sur la Relativité Générale (débuté mais pas fini)
				\item Dérivation du rayon d'Einstein
				\item Température théorique du fond diffus cosmologique
				\item Théorie de Kaluza-Klein
			\end{itemize}
		\item Physique Nucléaire:
			\begin{itemize}
				\item Diffusion Rayleigh
				\item Transport de Neutrons
				\item Spin (hélicité) et relation de polarité du photon
				\item Les inégalités de Bell
				\item L'inégalité de Kennard des incertitudes de Heisenberg
				\item Calcul détaillé du déplacement Lamb
				\item Expérience de Davisson-Germer
				\item Formalisme du paradoxe EPR
			\end{itemize}
		\item Physique Quantique Ondulatoire:
			\begin{itemize}
				\item Temps de l'effet tunnel quantique
			\end{itemize}
		\item Physique Quantique Relativiste:
			\begin{itemize}
				\item Parité, conjugaison de charge et inversion temporelle (CPT)
			\end{itemize}
		\item Chimie Quantique:
			\begin{itemize}
				\item Théorie de la répulsion de la paire d'électrons de Valence
				\item Dérivation de l'équation de Sackur-Tetrode
			\end{itemize}
		\item Méthodes numériques: 
			\begin{itemize}
				\item Problème d'optimisation univariée par la méthode de substitution				
				\item Échantillonnage par acceptation/rejet
				\item Échantillonnage de Gibbs
				\item Indicateur de cohérence de Cronbach
				\item Analyse Discriminante Linéaire (ADL)
				\item Analyse Discriminant Quadratique (ADQ)
				\item Positionnement Multidimensionnel (MDS)
				\item Modèle Linéaire Mixte (MLM)
				\item Décalage moyen
				\item Analyse Factorielle (AF)
				\item Analyse factorielle des Correspondances (AFC)
				\item Méthode d'optimisation GRG (gradient réduit généralisé)
				\item Informations mutuelles normalisées
				\item Machines à vecteurs de support (MVS)
				\item Détection d'Interactions Automatiques par le Chi-2 (CHAID)
				\item Analyse sémantique latente(LSA)
				\item Échantillonnage préférentiel 
				\item Échantillonnage stratifié
				\item Monte Carlo avec variable de contrôle
				\item Classificateur Bayésien Naif binomial et Gaussien
				\item Réduction dimensionnelle par la corrélation
				\item Algorithmes ID3, PRISM, AQ, CN2 et C4.5
				\item Analyse procustéenne
				\item Analyses en Composantes Indépendantes (ICA)
				\item Modèle uplift
			\end{itemize}
		\item Informatique Quantique: 
			\begin{itemize}
				\item Impossibilité du clonage quantique
			\end{itemize}
		\item Cryptographie: 
			\begin{itemize}
				\item Courbes elliptiques
			\end{itemize}	
		\item Ingénierie:
			\begin{itemize}
				\item Domaines de Box
				\item Plans de criblage définitifs (DSD)
				\item Plans Split-plots
				\item Plans Composites Centraux
				\item Plans Cubiques Faces Centrées
				\item Modèle de Survie de Cox (modèle à hasard proportionnel de Cox)
				\item Modélisation par équations structurelles				
				\item Tests de veillissements accélérés
				\item Microélectronique (jonctions NPN/PNP, diodes, amplificateurs)
				\item Équations des télégraphistes
				\item Théorème de poussée de Kutta-Jukowski
			\end{itemize} 
		\item Théorie des jeux et de la décision: 
			\begin{itemize}
				\item Coalition
				\item Valeur de Shapley
				\item Critère de Kelly 
			\end{itemize}
		\item Économie: 
			\begin{itemize}
				\item Taux de rendement continu
				\item Courbe des taux zéro-coupons
				\item Équivalence du taux d'une obligation avec un bon du trésor
				\item Taux Spot et taux Forward
				\item Ajustement du beta d'un portefeuille avec des Futures
				\item Égalité du prix de Cox-Ingersoll des Future/Forward
				\item Solution de l'EDP de Black \& Scholes
				\item Duration de Macaulay
				\item Duration Modifiée
				\item Taux de Retour Interne Modifié (MIRR)
				\item Couverture de portefeuilles par options
				\begin{itemize}
					\item Option de Vente/Achat de protection (protective Put/Call)
					\item Écart d'option d'achat à la hausse (bull Spread/Call)
					\item Écart d'option d'achat à la vente (bear Spread/Call)
					\item Papillon (butterfly)
					\item Opération liée (straddle)
					\item Position combinée (strangle)
					\item Collar
					\item Écart de Box (Box spread)
					\item Écart de Calendrier (calendar spreads)
					\item Méthodes d'allocation de portefeuilles
					\begin{itemize}
						\item Portefeuille pondéré optimal pour un risque équilibré
						\item Portefeuille pondéré optimal pour le suivi des erreurs
						\item Portefeuille pondéré optimal de Sharp
						\item Portefeuille pondéré optimal avec une diversification maximale
						\item Portefeuille de Treynor-Black optimal pondéré en fonction du benchmark
					\end{itemize}
				\end{itemize}
				\item Grecques pour les arbres binomiaux
				\item Swaps
				\item Formule de Margrabe
				\item Fomrule (approxmative) de Kirk
				\item Risque de défaut de crédit (basé sur la notation de Standard \& Poor)
				\item CreditRisk+
				\item VaR Equity Coverage
				\item Perte conditionnelle de la VaR (CVaR)
				\item Approche de KMV-Merton pour le mesure de probabilité de défaut
				\item Distance jusqu'à défaut
				\item Équation de Fokker-Planck
				\item Processus stochastiques ARCH-GARCH
				\item Modèles autorégressifs vectoriels pour séries temporelles multivariées
				\item Test de Granger pour la causalité de deux séries temporelles
				\item Filtres de Karman
				\item Modèle de pricing d'options de Heston
				\item Pricing d'options de type Spread
			\end{itemize}	
		\item Management Quantitatif: 
			\begin{itemize}
				\item Algorithme de Gale-Shapley
				\item Modèles Pareto/NBD, BG/NBD and BG/BB de la valeur vie client
				\item Problème du vendeur de journaux
				\item Effet coup de fouet
				\item Paradoxe de Condorcet	
				\item Méthode CRAFT (Computerized Relative Allocation of Facilities Technique)
				\item Options réelles
				\item Capital différé en cas de survie (assurance vie)
				\item Indice de volatilité CBOE (commencé mais pas terminé)
				\item Décès différé temporaire (assurance vie)
			\end{itemize}
	\end{itemize}
	Rappelez-vous que les sources \ LaTeX {} de ce livre peuvent être obtenues en fonction de votre don sur Patreon, Paypal, Tipee ou via votre participation à la traduction de ce livre dans une autre langue.
	
	Comme chaque produit robuste a un cycle de vie, le cycle de vie commence lorsqu'un produit est mis à disposition du grand public et se termine lorsqu'il n'est plus pris en charge. Connaître les dates clés de ce cycle de vie vous aide à prendre des décisions éclairées sur la date de mise à niveau. Ce livre a le cycle de vie suivant: une nouvelle version majeure ou mineure est publiée chaque fois qu'un seuil donné de dons est atteint et peut être téléchargé en cliquant sur le bouton suivant (PDF de $410$ mégaoctets ...):
	\begin{center}
		\href{http://www.sciences.ch/dwnldbl/telecharger.php3}{\includegraphics[scale=0.6]{img/books/download.jpg}}
	\end{center}
	ou si ce lien ne fonctionne pas, une copie du fichier PDF est disponible sur les Archives Internet:
	\begin{center}
		\includegraphics[scale=0.1]{img/internet_archive.jpg}
	\end{center}
	\begin{center}
	\href{https://archive.org/details/OperaMagistris}{https://archive.org/details/OperaMagistris}
	\end{center}
	
	Pour citer ce livre:
	\begin{quote}
	\noindent @book\{OperaMagistris2014v3, \\
		  author =       \{Vincent Isoz et Léon Harmel\}, \\
		  title =        \{Opera Magistris - Éléments de Mathématiques Appliquées pour Ingénieurs\}, \\
		  year =         \{2014\}, \\
	      publisher=     \{Sciences.ch\}, \\
		  keywords =     \{science, physique, maths, ingénierie, finance, management\}, \\
		  isbn =          \{978239909327\},\\
	\}
	\end{quote}
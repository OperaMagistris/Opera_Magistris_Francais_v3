	%to make section start on odd page
	\newpage
	\thispagestyle{empty}
	\mbox{}
	\section{Impressum}	
	\subsection{Utilisation du contenu}

	Le contenu de ce livre est \'elabor\'e par un processus de d\'eveloppement par lequel les volontaires parviennent à un consensus. Ce processus qui rassemble des b\'en\'evoles, recherche aussi le point de vue des personnes int\'eress\'ees par les sujets de ce livre. Le responsable de ce livre administre le processus et \'etablit des règles pour promouvoir l'\'equit\'e dans l'approche consensuelle. Il est \'egalement responsable de la r\'edaction du texte, parfois pour tester/\'evaluer ou v\'erifier de manière ind\'ependante l'exactitude ou l'exhaustivit\'e de l'information pr\'esent\'ee.

	Nous d\'eclinons toute responsabilit\'e pour toute blessure, dommage ou tout autre type, sp\'ecial, accessoire, cons\'ecutif ou compensatoire, d\'ecoulant de la publication, l'application ou la confiance du contenu de ce livre. Nous n'offrons aucune garantie expresse ou implicite quant à l'exactitude ou l'exhaustivit\'e des informations publi\'ees dans ce livre, et ne garantissons pas que les informations contenues dans ce livre r\'epondent à un besoin sp\'ecifique ou à un objectif du lecteur. Nous ne garantissons pas la performance des produits ou services d'un fabricant ou d'un fournisseur uniquement en vertu du contenu de ce livre.
	
	Les descriptions techniques, les proc\'edures et les programmes informatiques de ce livre ont \'et\'e d\'evelopp\'es sans pr\'ecaution pointue, ils sont donc fournis sans garantie d'aucune sorte. Nous ne garantissons pas non plus que les \'equations, les programmes et les proc\'edures de ce manuel ou de ses logiciels associ\'es sont exempts d'erreurs, ou sont conformes à une norme particulière de qualit\'e marchande, ou r\'epondront à vos exigences pour une application particulière. Ils ne devraient pas être invoqu\'es pour r\'esoudre un problème dont la solution incorrecte pourrait entraîner des blessures à des personnes ou la perte de biens. Toute utilisation du contenu de ce livre est aux risques et p\'erils du lecteur. Les auteurs, r\'edacteurs et \'editeurs d\'eclinent toute responsabilit\'e pour les dommages directs, indirects ou cons\'ecutifs r\'esultant de l'utilisation du contenu de ce livre ou des logiciels, codes  qui y sont associ\'es.

	En publiant des textes, il n'est pas dans l'intention de ce livre de fournir des services au nom d'une personne ou d'une entit\'e ou d'accomplir une tâche à accomplir par une personne ou une entit\'e au profit d'un tiers. Quiconque utilise ce livre devrait s'appuyer sur son propre jugement ind\'ependant ou, le cas \'ech\'eant, demander l'avis d'un expert (ou comit\'e d'experts) qualifi\'e pour d\'eterminer comment faire preuve de diligence raisonnable dans toutes les circonstances. Les informations et les normes sur les sujets couverts par ce livre peuvent être disponibles à partir d'autres sources que le lecteur peut souhaiter parcourir à la recherche de points de vue ou d'informations suppl\'ementaires non couvertes par le contenu de ce livre.

	Nous n'avons aucun pouvoir pour faire respecter le contenu de ce livre, et nous ne nous engageons pas à surveiller ou à faire respecter cette conformit\'e. Nous n'avons aucune activit\'e de certification, d'essai ou d'inspection de produits, de conceptions ou d'installations pour la s\'ecurit\'e ou la sant\'e des personnes et des biens. Toute certification ou autre d\'eclaration de conformit\'e concernant les informations relatives à la sant\'e ou à la s\'ecurit\'e des personnes et des biens, mentionn\'ee dans ce livre, ne peut être attribu\'ee au contenu de ce livre et reste sous la responsabilit\'e du centre de certification ou du d\'eclarant concern\'e.

	\subsection{Comment utiliser ce livre}
	Au niveau universitaire, ce livre peut être utilis\'e pour un doctorat, un diplôme d'\'etudes sup\'erieures ou un s\'eminaire avanc\'e de premier cycle dans de nombreux domaines des sciences exactes et pures. En r\'ealit\'e, ce livre vise \'egalement à couvrir le programme complet de la maternelle au doctorat.

	Parce que les m\'ethodes de math\'ematiques appliqu\'ees sont apprises par la pratique et l'exp\'erience, nous consid\'erons un s\'eminaire sur les math\'ematiques appliqu\'ees comme un s\'eminaire d'apprentissage par la pratique (ax\'e sur les projets). Nous structurons nos s\'eminaires de mod\'elisation math\'ematique autour d'un ensemble de problèmes qui n\'ecessitent que le stagiaire construise des modèles qui aident à la planification et à la prise de d\'ecision. L'imp\'eratif est que les modèles doivent être compatibles avec la th\'eorie et rev\'erifi\'es. Pour remplir cet imp\'eratif, il est n\'ecessaire que le stagiaire combine la th\'eorie math\'ematique avec la mod\'elisation. Le r\'esultat est que le stagiaire apprend la th\'eorie, et plus important encore, apprend comment cette th\'eorie est appliqu\'ee et combin\'ee dans le monde r\'eel. La capacit\'e de critiquer et d'identifier les limites des outils math\'ematiques dangereux est la caract\'eristique la plus importante de nos s\'eminaires.

	Les problèmes avec solutions pr\'esent\'ees dans ce livre fournissent l'occasion d'appliquer le mat\'eriel du texte à un ensemble complet de situations assez r\'ealistes (bien que simplifi\'ees). À la fin des s\'eminaires ainsi qu'à la fin de la lecture de ce livre, les stagiaires/lecteurs auront am\'elior\'e leurs comp\'etences et leurs connaissances des outils th\'eoriques et informatiques les plus importants et auront fortement d\'evelopp\'e leur esprit analytique. Ce sont des comp\'etences pr\'ecieuses qui sont demand\'ees par les entreprises et administrations au plus haut niveau.

	Il est très difficile de couvrir tout le mat\'eriel de ce livre en un semestre. Il faut beaucoup de temps pour expliquer les concepts aux stagiaires. Le lecteur est invit\'e à choisir les sujets qui seront abord\'es pendant le trimestre. Il n'est pas strictement n\'ecessaire de les couvrir dans l'ordre, mais cela peut aider de manière significative?

	En un mot, ce livre vous offre une grande vari\'et\'e de sujets qui peuvent être mod\'elis\'es. Tous, sans exception, sont utiles dans la pratique.
	
	\subsubsection{Guide des styles}
	Nous utilisons quelques conventions standard tout au long de ce livre. Ils ont été préparés évidemment avec \LaTeX{} qui numérote automatiquement les sections et le package hyperref qui génère les liens dans la Table des matières du présent PDF ainsi que d'autres références croisées faites dans le corps du texte.

	Nous utilisons la couleur et certaines boîtes pour mettre en évidence certains points pour que le lecteur puisse s'y référer plus facilement. En particulier:
	\begin{itemize}
		\item Les liens cliquables et les références croisées sont la plupart du temps en magenta (seulement quelque cas particuliers sont en bleu)
	
		\item Les textes expliquant la cible des références croisées sont en gris clair
	
		\item Les nouveaux termes (la plupart du temps associés aux définitions) sont en rouge
	
		\item Les remarques sont dans des boîtes avec une arrière plan grisé
	
		\item Les preuves commencent par le mot "Démonstration" et se terminent par un $\blacksquare$
	
		\item Les xemples supplémentaires \footnote{Un exemple compagnon à un théorème n'a pas de symbole spécifique} commence par le symbole {\Large \ding {45}}
	\end{itemize}
	Sans aucun doute, nous n’avons pas réussi à certains endroits à respecter ces règles. Si le lecteur garde une liste de ces transgressions pour nous la communiquer à la fin de la lecture de ce livre, cela pourrait valoir la peine de la partager avec les rédacteurs.
	
	\subsubsection{Vidéos et Animations}
	Ce livre contient des dizaines de petites vidéos et animations intégrées dans les pages PDF elles-mêmes. Cependant, pour pouvoir les jouer, vous aurez besoin:
	\begin{enumerate}
		\item D'Adobe Flash Player pour Firefox - NPAPI (la dernière version)
		
		\item D'activer l'option suivante disponible dans la plupart des lecteurs Adobe PDF sous Microsoft Windows:
		\begin{center}
			\includegraphics[scale=0.75]{img/intro/adobe_reader_multimedia.jpg}
		\end{center}
		Cette option doit être réactivée chaque fois que vous fermez et rouvrez le lecteur en fonction de la politique de votre département IT.
	\end{enumerate}
	
	\pagebreak
	\subsubsection{Annexes}
	Nous offrons gratuitement une gamme d'annexes (ouvrages compagnons) pour les \'etudiants, les instructeurs et les praticiens.
	
	D'abord, il y a des livres et des outils gratuits en français et en anglais pour les personnes qui veulent mettre en pratique la th\'eorie pr\'esent\'ee dans ce livre.
	
	Voici la liste:
	\begin{itemize}
		\item MATLAB™ en anglais (1,361 pages):\\ \href{http://www.sciences.ch/htmlfr/php/cliccount/click.php?id=319}{http://www.sciences.ch/dwnldbl/divers/Matlab.pdf}
		
		\item Maple en français (99 pages):\\ \href{http://www.sciences.ch/dwnldbl/divers/Maple.pdf}{http://www.sciences.ch/dwnldbl/divers/Maple.pdf}
		
		\item \textsf{R} en français (2,100 pages):\\ \href{http://www.sciences.ch/htmlfr/php/cliccount/click.php?id=313}{http://www.sciences.ch/dwnldbl/divers/R.pdf}
		
		\item Minitab en français (1,118 pages):\\ \href{http://www.sciences.ch/htmlfr/php/cliccount/click.php?id=282}{http://www.sciences.ch/dwnldbl/divers/Minitab.pdf}
		
		\item Scientific Linux installation \& Configuration en anglais  (211 pages):\\ \href{http://www.sciences.ch/dwnldbl/divers/ScientificLinux.pdf}{http://www.sciences.ch/dwnldbl/divers/ScientificLinux.pdf}
	\end{itemize}
	\begin{center}
		\includegraphics[scale=0.75]{img/books/matlab.jpg}
		\includegraphics[scale=0.75]{img/books/maple.jpg}
		\includegraphics[scale=0.75]{img/books/r.jpg}
		\includegraphics[scale=0.75]{img/books/minitab.jpg}
		\includegraphics[scale=0.75]{img/books/scientificlinux.jpg} 
	\end{center}	
	En second lieu, nous vous proposons gratuitement quelques Quizz et Flashcards en français et en anglais pour d\'efier vos \'etudiants ou vous-même avec le reste du monde:
	 \begin{itemize}
		\item Quizz MATLAB™ bases niveau L1 en français (100 questions)\\ \url{http://www.scientific-evolution.com/qcm/start_session/a73647cf3b/}
		
		\item Quizz Astonomie/Astrophysique niveau H1 en anglais (100 questions):\\ \url{http://www.scientific-evolution.com/qcm/start_session/ffd0810fa0/}
		
		\item Cartes de r\'evision pour les lettres grecques (48 cartes):\\
		\url{http://www.scientific-evolution.com/qcm/fr/start_session/6d9f1fef90/}
		
		\item Cartes de r\'evisions pour les d\'eriv\'ees les plus communes (29 cartes):\\
		\url{http://www.scientific-evolution.com/qcm/fr/start_session/c15a40f2c4/}
		
		\item Cartes de r\'evisions pour les int\'egrales les plus communes (60 cartes):\\
		\url{http://www.scientific-evolution.com/qcm/fr/start_session/ccfc20fdef/}
		
		\item Cartes de r\'evisions pour les identit\'es trigonom\'etriques les plus communes (68 cartes):\\
		\url{http://www.scientific-evolution.com/qcm/fr/start_session/882f9696cd/}
		
		\item Quizz \LaTeX{} niveau L3 en français (100 questions):\\ \url{http://www.scientific-evolution.com/qcm/fr/start_session/ff1e1d1b91/}
		
		\item Quizz R Software 3.1.2 niveau L3 en français (100 questions):\\ \url{http://www.scientific-evolution.com/qcm/fr/start_session/2a6fca7473/}
		
		\item Quizz C++ nivau en français (100 questions):\\
		\url{http://www.scientific-evolution.com/qcm/fr/start_session/e031ce4b43/}
	\end{itemize}
	Et comme tout livre technique devrait avoir un forum, le lecteur peut passer par ce lien pour toute discussion sur le contenu de ce livre: \url{https://www.physicsforums.com}.
	
	Pour ceux qui pr\'efèrent les r\'eseaux sociaux, nous avons aussi une page Facebook d\'edi\'ee:
	\begin{center}
		\faFacebook{} \href{https://www.facebook.com/operamagistris/}{https://www.facebook.com/operamagistris/}
	\end{center}
	ou pour plus de plaisir (photos de science, citations, blagues, vid\'eos, etc.) il y a aussi un compte Instagram associ\'e:
	\begin{center}
		\faInstagram{} \href{https://www.instagram.com/opera.magistris/}{https://www.instagram.com/opera.magistris/}
	\end{center}
%	et une petite collection non-exhaustive d'une s\'election de ce que nous consid\'erons comme des vid\'eos scientifiques int\'eressantes sur notre chaîne YouTube (voir à la fin de cette livre une section de chaînes YouTube scientifiques int\'eressantes):
%	\begin{center}
%		\faYoutube{} \href{https://www.youtube.com/user/AdminSciences}{https://www.youtube.com/user/AdminSciences}
%	\end{center}
	Comme pour le pr\'esent livre, les annexes et compl\'ements ci-dessus ne sont que des \'echantillons. Les versions complètes avec \underline{mises à jour gratuites perp\'etuelles} sont disponible pour le prix de $299$.- chacun et pour $499$.. vous obtenez les fichiers d'exercice et les sources \LaTeX{} (pour les informations sur l'achat, le lecteur peut simplement envoyer un e-mail à {\href{mailto:info@sciences.ch}{{\color{blue}email}}}).
	
	Pour les personnes qui veulent aider à traduire ce livre en d'autres langues, voici le lien avec les sources \ LaTeX {} pour les textes originaux (mais sans les \'equations\footnote{Les \'equations ne sont pas fournies car souhaitons contrôler qui fait quoi avec le livre!}!) sur GitHub:
	\begin{center}
		\faGithubSquare{} \href{https://github.com/OperaMagistris}{https://github.com/OperaMagistris}
	\end{center}
	Parce que ce livre se concentre principalement sur l'aspect math\'ematique (technique) des ph\'enomènes physiques, nous ne pouvons que recommander fortement au lecteur un autre livre gratuit qui est de notre point de vue subjectif actuellement le meilleur sur l'aspect scientifique populaire des sujets que nous couvrirons:
	\begin{center}
	Motion Mountain par le Dr. Christoph Schiller: \url{http://www.motionmountain.net}
	\end{center}

	\pagebreak
	\subsection{Protection des donn\'ees}
	Lorsque vous consultez des informations sur le site Internet compagnon (Sciences.ch), certaines donn\'ees sont automatiquement sauvegard\'ees. Nous essayons de sauvegarder le moins de donn\'ees possible et ce le plus rapidement possible. Partout où nous pouvons, nous n'avons que des donn\'ees anonymes. Nous nous engageons à traiter les donn\'ees que vous nous envoyez personnellement avec la plus grande diligence.

	Cependant, votre adresse IP et la page source qui vous emmène sur Sciences.ch et les mots-cl\'es associ\'es, sont disponibles gratuitement pour tout le monde \href{http://www.sciences.ch/htmlfr/php/cedstat/references.php}{ici} pour le mois en cours. Après quoi les donn\'ees d\'etaill\'ees sont d\'etruites. Vous pouvez vous opposer à tout moment à la publication de vos donn\'ees en nous contactant.

	\subsection{Utilisation des donn\'ees}
	Vos donn\'ees ne sont utilis\'ees que pour l'envoi du bulletin d'information (newletter) de Sciences.ch. La communication des donn\'ees personnelles (à l'exception de l'adresse e-mail, du titre et du nom) est facultative. Lorsque vous vous inscrivez à la newsletter, vous pouvez bien sûr sp\'ecifier une autre adresse et / ou un nom fictif.

	\subsection{Transmission des donn\'ees}
	Nous ne vendrons ni ne commercialiserons jamais les donn\'ees de nos clients ou parties int\'eress\'ees et n'affecterons jamais les droits de la personne. En outre, nous ne louerons pas de listes de diffusion et ne vous enverrons pas de publicit\'e de la part de tiers ou en notre nom.

	\subsection{Accord}
	Lorsque vous nous fournissez des informations personnelles, vous nous autorisez à les sauvegarder et à les utiliser au sens de la loi f\'ed\'erale suisse sur la protection des donn\'ees. Si vous nous demandez de ne pas vous envoyer d'e-mails (courriels), nous sommes dans l'obligation, dans votre int\'erêt, d'enregistrer votre e-mail (courriel) dans une liste n\'egative interne.
	
	\subsection{Errata}
	Bien que nous ayons pris toutes les pr\'ecautions pour assurer l'exactitude de notre contenu, des erreurs humaines peuvent se produire. Si vous trouvez une erreur dans ce livre - peut-être une erreur dans le texte, les scripts ou les illustrations - nous vous serions reconnaissants de nous le signaler. Ce faisant, vous pouvez \'eviter la frustration d'autres lecteurs en nous aidant à am\'eliorer les versions ult\'erieures de ce livre. Notre e-mail est donn\'e sur le pied de page à chaque page de ce livre. Une fois vos errata v\'erifi\'es, vos soumissions seront accept\'ees et l'erreur sera visible dans le journal des modifications des versions de mise à jour.

	%to make section start on odd page
	\newpage
	\thispagestyle{empty}
	\mbox{}
	\section{Licence}
	Tout le contenu de ce livre est soumis à la licence de documentation libre GNU, ce qui signifie:
	\begin{itemize}
			\item[$\bullet$] que chacun a le droit d'utiliser librement les textes à des fins non commerciales (Google Ads ou tout \'equivalent \'etant consid\'er\'e comme un usage commercial!)
			\item[$\bullet$] que toute personne est autoris\'ee à diffuser des articles pour un usage non commercial (Google Ads ou tout \'equivalent \'etant consid\'er\'e comme un usage commercial!)
			\item[$\bullet$] que n'importe qui peut \'editer librement les textes pour un usage non commercial (Google Ads ou tout \'equivalent \'etant consid\'er\'e comme un usage commercial!)
	\end{itemize}
	
	et bla bla bla...

	conform\'ement à la licence d\'ecrite ci-dessous: 
	
	\begin{center}
	\url{https://creativecommons.org/licenses/by-nc-sa/4.0/}\\[2pt]
\includegraphics[width=1cm]{img/icons/share2.eps}\includegraphics[width=1cm]{img/icons/remix2.eps}
\includegraphics[width=1cm]{img/icons/by2.eps}
\includegraphics[width=1cm]{img/icons/nc-eu2.eps}
\includegraphics[width=1cm]{img/icons/sa2.eps}
	\end{center}

	\begin{center}
	Version 1.1, Mars 2000
		
	Copyright (C) 2000 Fondation du Logiciel Libre, Inc. 59 Temple Place, Suite 330, Boston, MA 02111-1307 USA Tout le monde est autoris\'e à copier et à distribuer des copies verbatim de ce document de licence, mais il n'est pas autoris\'e de le modifier. 
	\end{center}

	\subsection{Pr\'eambule} 
	Le but de cette Licence est de rendre un manuel, une r\'ef\'erence ou un autre document \'ecrit "libre" dans le sens de la libert\'e: assurer à chacun la libert\'e effective de le copier et de le redistribuer, avec ou sans le modifier à des fins non commerciales \footnote{L'avantage imm\'ediat est que vous pouvez g\'en\'erer le livre, avec de nouveaux ensembles de problèmes, et le distribuer à vos \'etudiants simplement en format PDF (dans un courriel, par exemple). Mais plus g\'en\'eralement, si vous n'êtes pas int\'eress\'e par la façon dont nous avons expliqu\'e (ou omis d'expliquer) quelque chose, alors vous êtes libre de le r\'e\'ecrire. Si vous souhaitez couvrir plus de sujets (ou moins), vous êtes libre d'ajouter (ou de supprimer) les chapitres / sections / paragraphes que vous souhaitez. Et puisque le lecteur peut faire la demande du code source, il n'a pas besoin de recr\'eer la roue.}. En second lieu, cette Licence conserve pour l'auteur et l'\'editeur un moyen d'obtenir un cr\'edit pour leur travail, tout en n'\'etant pas consid\'er\'e comme responsable des modifications apport\'ees par d'autres.

	Cette Licence est une sorte de "copyleft", ce qui signifie que les oeuvres d\'eriv\'ees du Document doivent elles-mêmes être libres dans le même sens. Il complète la licence publique g\'en\'erale GNU, qui est une licence copyleft conçue pour un logiciel libre.

	Nous avons conçu cette Licence afin de l'utiliser pour les manuels de logiciels libres, car le logiciel libre a besoin de documentation gratuite: un programme gratuit devrait être fourni avec des manuels offrant les mêmes libert\'es que le logiciel. Mais cette Licence n'est pas limit\'ee aux manuels de logiciels; il peut être utilis\'e pour n'importe quel travail textuel, ind\'ependamment du sujet ou s'il est publi\'e comme un livre imprim\'e. Nous recommandons cette Licence principalement pour les travaux dont l'objectif est l'instruction ou la r\'ef\'erence.

	\subsection{Applications et D\'efinitions}
	Cette Licence s'applique à tout travail manuel ou autre contenant un avis plac\'e par le d\'etenteur des droits d'auteur indiquant qu'il peut être distribu\'e selon les termes de cette Licence. Le "Document", ci-dessous, fait r\'ef\'erence à un tel manuel ou travail. Tout membre du public est un titulaire de Licence et est appel\'e «vous».

	Une "Version Modifi\'ee" du Document d\'esigne tout travail contenant le Document ou une partie de celui-ci, soit copi\'e textuellement, soit avec des modifications et / ou traduit dans une autre langue.

	Une «section secondaire» est une annexe nomm\'ee ou une section de première instance du Document qui traite exclusivement de la relation entre les \'editeurs ou les auteurs du Document à la matière globale du Document (ou à des questions connexes) et ne contient rien qui pourrait tomber directement dans ce sujet global (par exemple, si le Document est en partie un manuel de math\'ematiques, une section secondaire pourrait ne pas expliquer les math\'ematiques). La relation pourrait être une question de connexion historique avec le sujet ou avec des questions connexes, ou juridique, commercial, philosophique, position \'ethique ou politique à leur \'egard.
	
	Les "Sections Invariantes" sont certaines Sections Secondaires dont les titres sont d\'esign\'es, comme \'etant ceux des Sections Invariantes, dans l'avis qui dit que le Document est publi\'e sous cette Licence.

	Les "Textes de couverture" sont de courts passages de texte qui sont list\'es, en tant que Textes d'ouverture ou Textes de clôture, dans l'avis indiquant que le Document est publi\'e sous cette Licence.

	Une copie "Transparente" du Document d\'esigne une copie lisible par machine, repr\'esent\'ee dans un format dont la sp\'ecification est accessible au grand public, dont le contenu peut être visualis\'e et \'edit\'e directement et directement avec des \'editeurs de texte g\'en\'eriques ou (pour des images compos\'ees de pixels) programmes de peinture g\'en\'eriques ou (pour les dessins) un \'editeur de dessin largement disponible, et qui est adapt\'e pour l'entr\'ee aux formateurs de texte ou pour la traduction automatique à une vari\'et\'e de formats appropri\'es pour l'entr\'ee aux formateurs de texte. Une copie faite dans un format de fichier autrement transparent dont le balisage a \'et\'e conçu pour contrecarrer ou d\'ecourager la modification ult\'erieure par les lecteurs n'est pas transparent. Une copie qui n'est pas "transparente" est appel\'ee "opaque".

	Des exemples de formats appropri\'es pour les copies transparentes incluent l'ASCII simple sans balisage, le format d'entr\'ee Texinfo, le format d'entr\'ee LaTeX, SGML ou XML utilisant une DTD accessible au public et un HTML simple conforme aux normes conçu pour la modification humaine. Les formats opaques incluent PostScript, PDF, formats propri\'etaires qui ne peuvent être lus et \'edit\'es que par des traitements de texte propri\'etaires, SGML ou XML pour lesquels la DTD et / ou les outils de traitement ne sont g\'en\'eralement pas disponibles, et HTML produit par certains traitements de texte pour fins de sortie seulement.

	La «page de titre» d\'esigne, pour un livre imprim\'e, la page de titre elle-même, ainsi que les pages suivantes qui sont n\'ecessaires pour contenir, de manière lisible, le mat\'eriel dont cette Licence a besoin pour apparaître dans la page de titre. Pour les œuvres dans des formats qui n'ont pas de page de titre en tant que telle, "Page de titre" signifie le texte près de l'apparence la plus visible du titre de l'oeuvre, pr\'ec\'edant le d\'ebut du corps du texte.

	\subsection{Copie Verbatim} 
	Vous pouvez copier et distribuer ce libre sur n'importe quel support, non commercial, à condition que cette Licence, les avis de copyright et l'avis de licence indiquant que cette Licence s'applique au livre soient reproduits dans toutes les copies et que vous n'ajoutez aucune autre condition à celles de cette Licence. Vous ne pouvez pas utiliser des mesures techniques pour entraver ou contrôler la lecture ou la copie ult\'erieure des copies que vous faites ou distribuez. Cependant, vous pouvez accepter une compensation en \'echange de copies. Si vous distribuez un nombre suffisant de copies, vous devez \'egalement respecter les conditions de la section 3.

	Vous pouvez \'egalement prêter des copies, dans les mêmes conditions que celles indiqu\'ees ci-dessus, et vous pouvez en afficher publiquement des copies. 

	\subsection{Copie en Quantit\'e}
	Si vous publiez des copies imprim\'ees du Document en quantit\'e sup\'erieurs à $100$ et que l'avis de Licence du Document exige des textes de couverture, vous devez joindre les copies dans des pochettes qui portent, clairement et lisiblement, tous ces textes de couverture: les textes de la première de couverture sur la page de couverture. les textes de la dernière de couverture sur la couverture arrière. Les deux couvertures doivent \'egalement vous identifier clairement et lisiblement en tant qu'\'editeur de ces copies. La couverture doit pr\'esenter le titre complet avec tous les mots du titre tout aussi visible l'un que l'autre. Vous pouvez ajouter d'autres informations sur les couvertures en plus. La copie avec des modifications limit\'ees aux couvertures, tant qu'elles conservent le titre du Document et satisfont à ces conditions, peut être consid\'er\'ee comme une reproduction textuelle à d'autres \'egards.

	Si les textes requis pour l'une ou l'autre des couvertures sont trop volumineux pour être lisibles, vous devez mettre les premiers inscrits (autant que possible) sur la couverture actuelle, et continuer le reste sur les pages suivantes.

	Si vous publiez ou distribuez des copies opaques en quantit\'e sup\'erieure à $100$, vous devez inclure une copie transparente lisible par machine avec chaque copie opaque, ou indiquer dans chaque copie opaque un emplacement r\'eseau informatique accessible au public contenant une copie complète transparente du Document, sans mat\'eriel ajout\'e et de manière à ce que le public utilisant un un r\'eseau g\'en\'eral y ait accès de façon anonyme, sans frais et en utilisant des protocoles de r\'eseau public ou standard. Si vous utilisez cette dernière option, vous devez prendre des mesures raisonnablement prudentes lorsque vous commencez la distribution de copies opaques en quantit\'e, pour vous assurer que cette copie transparente restera accessible à l'emplacement indiqu\'e jusqu'à au moins un an après la dernière distribution d'un copie opaque (directement ou par l'interm\'ediaire de vos agents ou d\'etaillants) de cette \'edition au public.

	Il est demand\'e, mais pas obligatoire, que vous contactiez les auteurs du Document bien avant de redistribuer un grand nombre de copies, pour leur donner une chance de vous fournir une version mise à jour du Document.

	\subsection{Modifications}
	Vous pouvez copier et distribuer une version modifi\'ee du Document dans les conditions des sections 2 et 3 ci-dessus, à condition de publier la Version Modifi\'ee sous cette Licence, avec la version modifi\'ee remplissant le rôle du Document, autorisant ainsi la distribution et la modification de la Version Modifi\'ee à quiconque en possède une copie. En outre, vous devez effectuer ces op\'erations dans la version modifi\'ee:
	
	\begin{itemize}
		\item Utilisez dans la page de titre (et sur les couvertures, le cas \'ech\'eant) un titre distinct de celui du Document, et de ceux des versions pr\'ec\'edentes (qui devraient, le cas \'ech\'eant, figurer dans la section \textit{Historique des révisions} du document). Vous pouvez utiliser le même titre que la version pr\'ec\'edente si l'\'editeur d'origine de cette version donne son autorisation.

		\item Lister sur la page de titre, en tant qu'auteurs, une ou plusieurs personnes ou entit\'es responsables de la paternit\'e des modifications de la Version Modifi\'ee, ainsi qu'au moins cinq des auteurs principaux du document (tous ses auteurs principaux, s'il y en a moins de cinq).

		\item Indiquer sur la page de titre le nom de l'\'editeur de la version modifi\'ee, en tant qu'\'editeur.

		\item Conserver tous les avis de droits d'auteur du document.

		\item Ajoutez un avis de droit d'auteur appropri\'e pour vos modifications à côt\'e des autres avis de droits d'auteur. 

		\item Inclure, imm\'ediatement après les avis de droits d'auteur, un avis de licence autorisant le public à utiliser la Version Modifi\'ee conform\'ement aux conditions de la pr\'esente Licence, sous la forme indiqu\'ee dans l'addenda ci-dessous.

		\item Pr\'eservez dans cet avis de licence les listes complètes des sections invariantes et les textes de couverture requis indiqu\'es dans l'avis de Licence du Document.

		\item Inclure une copie non modifi\'ee de cette licence.

		\item Conservez la section intitul\'ee \textit{Historique des révisions} et son titre, et ajoutez-y un \'el\'ement indiquant au moins le titre, l'ann\'ee, les nouveaux auteurs et l'\'editeur de la version modifi\'ee comme indiqu\'e sur la page de titre. S'il n'y a pas de section intitul\'ee \textit{Historique des révisions} dans le document, cr\'eez-en un en indiquant le titre, l'ann\'ee, les auteurs et l'\'editeur du document comme indiqu\'e sur sa page de titre, puis ajoutez un \'el\'ement d\'ecrivant la version modifi\'ee.

		\item Conservez l'emplacement r\'eseau, le cas \'ech\'eant, indiqu\'e dans le document pour l'accès public à une copie transparente du Document, ainsi que les emplacements r\'eseau indiqu\'es dans le document pour les versions pr\'ec\'edentes sur lesquelles il \'etait bas\'e. Ceux-ci peuvent être plac\'es dans la section \textit{Historique des révisions}. Vous pouvez omettre un emplacement r\'eseau pour un travail publi\'e au moins quatre ans avant le document lui-même ou si l'\'editeur d'origine de la version à laquelle il fait r\'ef\'erence donne son autorisation.

		\item Dans toute section intitul\'ee «Remerciements» ou «D\'edicaces», conserver le titre de la section et conserver dans la section toute la substance et le ton de chacun des remerciements et / ou d\'edicaces du contributeur qui y sont donn\'es.

		\item Conserver toutes les sections invariables du Document, inchang\'ees dans leur texte et dans leurs titres. Les num\'eros de section ou l'\'equivalent ne sont pas consid\'er\'es comme faisant partie des titres de section.

		\item  Supprimer toute section intitul\'ee "Avenants". Une telle section peut ne pas être incluse dans la Version Modifi\'ee.

		\item Ne retitrez aucune section existante en tant que "Avenants" ou tout autre section qui pourrait g\'en\'erer un conflit dans le titre avec une Section Invariante.

		\item Si la version modifi\'ee inclut de nouvelles sections ou appendices qualifi\'ees en tant que Sections Secondaires et ne contient aucun contenu copi\'e à partir du Document original, vous pouvez, à votre discr\'etion, d\'esigner certaines ou toutes ces sections comme invariantes. Pour ce faire, ajoutez leurs titres à la liste des sections invariantes dans l'avis de licence de la Version Modifi\'ee. Ces titres doivent être distincts des autres titres de section.

		\item Vous pouvez ajouter une section intitul\'ee «Avenants», à condition qu'elle ne contienne que des approbations de votre version modifi\'ee par diverses parties - par exemple, des d\'eclarations d'examen par les pairs ou que le texte a \'et\'e approuv\'e par une organisation comme d\'efinition officielle d'une norme.

		\item Vous pouvez ajouter jusqu'à cinq mots comme texte de première de couverture et jusqu'à 25 mots comme texte de quatrième couverture à la fin de la liste des textes de ces deux couverture dans la Version Modifi\'ee. Un seul passage de texte de premère de couverture et un de texte de quatrième de couverture peuvent être ajout\'es par (ou par des arrangements faits par) une entit\'e. Si le Document contient d\'ejà un texte de couverture pour les couvertures sus-mentionn\'ees, pr\'ec\'edemment ajout\'e par vous ou par un arrangement conclu avec la même entit\'e sous le compte de laquelle vous agissez, vous ne pouvez pas en ajouter un autre; mais vous pouvez remplacer l'ancien, avec l'autorisation explicite de l'\'editeur pr\'ec\'edent qui a ajout\'e l'ancien.

		\item Le(s) auteur(s) et le(s) \'editeur(s) du document ne donnent pas la permission d'utiliser leur nom à des fins publicitaires, d'affirmer ou de sous-entendre l'approbation d'une Version Modifi\'ee.
	\end{itemize} 

	\subsection{Combinaison de Documents}
	Vous pouvez combiner le Document avec d'autres documents publi\'es \'egalement sous cette même Licence, selon les termes d\'efinis dans la section 4 ci-dessus pour les Versions Modifi\'ees, à condition d'inclure dans la combinaison toutes les Sections Invariantes de tous les documents originaux, non modifi\'es, et de les lister tous comme Sections Invariantes de votre travail combin\'e dans son avis de Licence.

	Le travail combin\'e doit seulement contenir une copie de cette Licence, et plusieurs Sections Invariantes identiques peuvent être remplac\'ees par une seule copie. S'il y a plusieurs Sections invariantes avec le même nom mais des contenus diff\'erents, rendez le titre de chaque section unique en ajoutant à la fin, entre parenthèses, le nom de l'auteur original ou de l'\'editeur de cette section s'il est connu, ou bien un nombre unique. Apportez le même ajustement aux titres de section dans la liste des sections invariables dans l'avis de Licence du travail combin\'e.
	
	Dans la combinaison, vous devez combiner toutes les sections intitul\'ees \textit{Historique des révisions} dans les divers documents originaux, en formant une section intitul\'ee \textit{Historique des révisions}; De même, vous devez combiner toutes les sections intitul\'ees \textit{Remerciements}.

	\subsection{Collections de Documents}
	Vous pouvez faire une collection constitu\'ee du Document et d'autres documents publi\'es sous cette Licence, et remplacer les copies individuelles de cette Licence dans les divers documents par une seule copie qui est incluse dans la collection, à condition que vous respectiez les règles de cette Licence pour chaque copie textuelle de chacun des documents à tous \'egards.

	Vous pouvez extraire un seul document d'une telle collection et le distribuer individuellement sous cette Licence, à condition d'ins\'erer une copie de cette Licence dans le document extrait, et de suivre cette Licence à tous \'egards concernant la copie textuelle de ce document.

	\subsection{Agr\'egation avec des œuvres ind\'ependantes} 
	Une compilation du Document ou de ses d\'eriv\'es avec d'autres documents ou travaux s\'epar\'es ou ind\'ependants, dans ou sur un volume d'un support de stockage ou de distribution, ne compte pas dans son ensemble comme Version Modifi\'ee du Document, à condition qu'aucun droit de compilation ne soit r\'eclam\'e pour la compilation. Une telle compilation est appel\'ee un "agr\'egat", et cette Licence ne s'applique pas aux autres oeuvres autonomes ainsi compil\'ees avec le Document, du fait qu'elles sont ainsi compil\'ees, si elles ne sont pas elles-mêmes des oeuvres d\'eriv\'ees du Document.

	Si l'exigence du texte de couverture de la section 3 s'applique à ces copies du Document, alors si le document repr\'esente moins du quart de l'ensemble, les textes de couverture du Document peuvent être plac\'es sur des couvertures qui entourent uniquement le Document à l'int\'erieur de l'aggr\'egat. Sinon, ils doivent apparaître sur les couvertures autour de l'ensemble des agr\'egats.
	
	\subsection{Compilation du Document}
	Vous aurez besoin de trois trois choses pour g\'en\'erer ce document pour vous-même:
	\begin{enumerate}
		\item Installer \href{https://miktex.org/}{MiKTeX}

		\item Installer \href{http://www.xm1math.net/texmaker/index_fr.html}{TeXMaker}

		\item Une connexion Internet

		\item Configurer MikTeX comme indiqu\'e dans les remarques au d\'ebut du fichier \textit{LaTeX\_SciencesCh\_FR.tex}
	\end{enumerate}

	\subsection{Traductions}
	La traduction est consid\'er\'ee comme une sorte de modification, vous pouvez donc distribuer des traductions du document sous les termes de la section correspondante sur la transformation. Le remplacement de sections invariables par des traductions n\'ecessite une autorisation sp\'eciale de leurs d\'etenteurs de droits d'auteur, mais vous pouvez inclure des traductions de certaines ou de toutes les sections invariables en plus des versions originales de ces sections invariables.

	\subsection{R\'esiliation}
	Vous n'êtes pas autoris\'e à copier, modifier, sous-licencier ou distribuer le Document, sauf dans les cas express\'ement pr\'evus par cette Licence. Toute autre tentative de copier, modifier, sous-licencier ou distribuer le document est annul\'ee et met fin automatiquement à vos droits en vertu de cette Licence. Cependant, les parties qui ont reçu des copies ou des droits de votre part en vertu de cette Licence ne verront pas leur licence r\'esili\'ee tant que ces parties resteront en pleine conformit\'e.

	\subsection{R\'evisions futures de cette Licence}
	La Free Software Foundation (FSF) peut publier de temps à autre de nouvelles versions r\'evis\'ees de la Licence de documentation libre GNU. Ces nouvelles versions seront similaires dans l'esprit à la pr\'esente version, mais peuvent diff\'erer dans le d\'etail pour r\'epondre à de nouveaux problèmes ou pr\'eoccupations. Voir \href{http://www.gnu.org/copyleft/}{{\color{blue} http://www.gnu.org/copyleft/}}.

	Chaque version de la Licence reçoit un num\'ero de version distinctif. Si le Document sp\'ecifie qu'une version num\'erot\'ee particulière de cette Licence "ou toute version ult\'erieure" s'applique à elle, vous avez la possibilit\'e de suivre les termes et conditions de cette version sp\'ecifi\'ee ou de toute version ult\'erieure qui a \'et\'e publi\'ee (et non en tant que brouillon) par la FSF. Si le Document ne sp\'ecifie pas le num\'ero de version de cette Licence, vous pouvez choisir n'importe quelle version jamais publi\'ee (pas comme brouillon) par la FSF.
	
	\begin{center}
	\color{ForestGreen}{{\Large \faTree} \textbf{S'il-vous-plaît, pensez à l'environnement avant d'imprimer}}
	\end{center}
	
	%to make section start on odd page
	\newpage
	\thispagestyle{empty}
	\mbox{}
	\section{Feuille de route}
	Ce livre a une règle de progression simple qui est: $1$ nouvelle page A4 par jour depuis Mai 2001 sur des sujets qui int\'eressent le superviseur de la distribution \textit{Originale} du livre \textit{Opera Magistris}. Les nouveaux sujets sont d\'ebloqu\'es (publi\'es) en fonction des paliers de dons qui sont faits via note page \href{https://www.tipeee.com/elements-of-applied-mathematics}{Tipeee}, \href{https://www.patreon.com/sciences}{Patreon} ou \href{https://www.paypal.me/scientificevolution}{Paypal}. Les sujets suivants sont d\'ejà pr\'evus pour un futur proche ou lointain avec toujours le même niveau de d\'etails et d'approche p\'edagogique dans les preuves math\'ematiques que le reste du livre (tous les sujets ci-dessous devraient n\'ecessiter environ $1'500$ à $2'500$ pages suppl\'ementaires):
	\begin{itemize}
		\item Introduction:
			\begin{itemize}
				\item Ajouter des remerciements à toutes les personnes qui ont cr\'e\'e la distribution MikTeX / LaTeX et aux packages utilis\'es pour ce livre
			\end{itemize}
		\item Probabilit\'es:
			\begin{itemize}
				\item Conjugaison Bay\'esienne pour la loi Normale et Binomiale
				\item Chaînes de Markove cach\'ees
				\item Log-loss
			\end{itemize}
		\item Statistiques: 
			\begin{itemize}
				\item Mode et M\'ediane de lois statistiques
				\item Maximum de vraisemblance pour donn\'ees censur\'ees
				\item Entropie de la loi Normale
				\item Score de Propension
				\item Test d'\'equivalence
				\item Matrice de Quasi-corr\'elation
				\item Analyse Factorielle
				\item T-Test de Hotelling
				\item R\'esidus standardis\'es de Pearson
				\item Test de Welch avec \'equation de Welch-Satterhwaite
				\item ANCOVA
				\item Test de Cramer-vos Mises
				\item Test de Wald-Wolfowitz Test (des s\'equences binaires)
				\item Test de Levene-Wolfwitz\footnote{aussi appel\'e  "test de point de bifurcation" ou "test de tendance"} (s\'equences haussière/baissière)
				\item Ellipse de contrôle
				\item Mesures bas\'ees sur l'entropie des tables de contingence
				\item Test de tendance d'Armitage
				\item Test d'Ansari-Bradley
				\item Test r\'egulier de Dickey-Fuller
				\item Modèle de Poisson pour la distance spatiale moyenne (2D)
				\item Corr\'elation canonique
				\item G-test de p\'eriodicit\'e
				\item Copula Gaussien et de Student
				\item ANOVA à facteur fixe hi\'erarchique
				\item Introduction à la MANOVA
				\item Th\'eorème des valeurs extrêmes
				\item Th\'eorie des sondages
				\item Modèles lin\'eaires g\'en\'eralis\'es (Gauss, Poissson, Binomial N\'egatif, Gamma)
				\item R\'egression PLS (moindres carr\'es partiels)
				\item Moindres carr\'es en deux \'etapes (2SLS)
				\item R\'egression logique
				\item Chi-carr\'e ajust\'e
				\item Probabilit\'e des fonctions g\'en\'eratrices
			\end{itemize}
		\item G\'eom\'etrie:
			\begin{itemize}
				\item Volume de l'hypersphère
			\end{itemize}
		\item Calcul Diff\'erentiel
			\begin{itemize}
				\item Int\'egrale de Lebesgue avec application num\'erique dans MATLAB™
				\item Repr\'esentation int\'egrale des fonctions de Bessel
				\item Lin\'earisation des puissances des fonctions trigonom\'etriques
				\item Int\'egrales elliptiques et fonction elliptiques
			\end{itemize}
		\item Analyse: 
			\begin{itemize}
				\item Transform\'ee de Hilbert
			\end{itemize}
		\item Analyse Complexe: 
			\begin{itemize}
				\item Th\'eorème des r\'esidus pour des ratios de polynômes
				\item Th\'eorème de la valeur moyenne de Gauss
			\end{itemize}
		\item Topologie: 
			\begin{itemize}
				\item Distance de Mahalanobis
			\end{itemize}			
		\item G\'eom\'etrie Diff\'erentielle: 
			\begin{itemize}
				\item Coordonn\'ees normales
				\item Courbure de Gauss
				\item P\'erimètre d'un circle sur le plan, sur la sphère et sur une surface hyperbolique
				\item Th\'eorème isop\'erimètrique du plan
			\end{itemize}
		\item M\'ecanique: 
			\begin{itemize}
				\item Effet Magnus
				\item Critère de Lawson (plasmas)
				\item \'ecoulement à travers un orifice submerg\'e
				\item \'ecoulement sur les encoches et les d\'eversoirs
				\item Force due à l'\'ecoulement autour d'un coude de tuyau
				\item Force sur une buse
				\item Impact d'un jet sur un avion
				\item Turbine Pelton
				\item Force due à un jet atteignant un plan inclin\'e
				\item Effet b\'elier d'un fluide
				\item \'equations de Saint-Venant
				\item Portance de Kutta-Joukowski
			\end{itemize}	
		\item \'Electrodynamique:
			\begin{itemize}		
				\item Champs \'electromagn\'etiques d'une sphère de charges en rotation
				\item \'equation de Sellmeier
				\item Brehmstrahlung (commenc\'e mais pas fini)
				\item Principe de l'interf\'eromètre de Michelson
				\item Couple de rotor d'un \'electroaimant
			\end{itemize}
		\item \'Electrocin\'etique:
			\begin{itemize}		
				\item Convertisseur photo\'electrique
				\item Jonctions PN
			\end{itemize}
		\item Astronomie:
			\begin{itemize}	
				\item Formule de MacCullagh's 
				\item Calcul indirect de l'aplattissement des corps c\'elestes
				\item Verrouillage synchrone des satellites en r\'evolution	
				\item Aplatissement des sph\'eroïdes
				\item Sphère d'influence
				\item \'Echappement plan\'etaire
			\end{itemize}		
		\item Relativit\'e G\'en\'erale:
			\begin{itemize}
				\item Volume r\'eel d'un objet en relativit\'e g\'en\'erale
				\item D\'erivation du rayon d'Einstein
				\item Th\'eorème g\'en\'eral de Birkhoff
				\item Global Positioning System (GPS)
				\item M\'etrique et solution de Kerr	
			\end{itemize}
		\item Cosmologie:
			\begin{itemize}
				\item Finir le texte sur les Univers de Friedman bas\'es sur la Relativit\'e G\'en\'erale (d\'ebut\'e mais pas fini)
				\item D\'erivation du rayon d'Einstein
				\item Temp\'erature th\'eorique du fond diffus cosmologique
				\item Th\'eorie de Kaluza-Klein
			\end{itemize}
		\item Physique Nucl\'eaire:
			\begin{itemize}
				\item Diffusion Rayleigh
				\item Transport de Neutrons
				\item Spin (h\'elicit\'e) et relation de polarit\'e du photon
				\item Les in\'egalit\'es de Bell
				\item L'in\'egalit\'e de Kennard des incertitudes de Heisenberg
				\item Calcul d\'etaill\'e du d\'eplacement Lamb
				\item Exp\'erience de Davisson-Germer
				\item Formalisme du paradoxe EPR
			\end{itemize}
		\item Physique Quantique Ondulatoire:
			\begin{itemize}
				\item Temps de l'effet tunnel quantique
			\end{itemize}
		\item Physique Quantique Relativiste:
			\begin{itemize}
				\item Parit\'e, conjugaison de charge et inversion temporelle (CPT)
			\end{itemize}
		\item Chimie Quantique:
			\begin{itemize}
				\item Th\'eorie de la r\'epulsion de la paire d'\'electrons de Valence
				\item D\'erivation de l'\'equation de Sackur-Tetrode
			\end{itemize}
		\item M\'ethodes num\'eriques: 
			\begin{itemize}
				\item Problème d'optimisation univari\'ee par la m\'ethode de substitution				
				\item \'echantillonnage par acceptation/rejet
				\item \'echantillonnage de Gibbs
				\item Indicateur de coh\'erence de Cronbach
				\item Analyse Discriminante Lin\'eaire (ADL)
				\item Analyse Discriminant Quadratique (ADQ)
				\item Positionnement Multidimensionnel (MDS)
				\item Modèle Lin\'eaire Mixte (MLM)
				\item D\'ecalage moyen
				\item Analyse Factorielle (AF)
				\item Analyse factorielle des Correspondances (AFC)
				\item M\'ethode d'optimisation GRG (gradient r\'eduit g\'en\'eralis\'e)
				\item Informations mutuelles normalis\'ees
				\item Machines à vecteurs de support (MVS)
				\item D\'etection d'Interactions Automatiques par le Chi-2 (CHAID)
				\item Analyse s\'emantique latente(LSA)
				\item \'echantillonnage pr\'ef\'erentiel 
				\item \'echantillonnage stratifi\'e
				\item Monte Carlo avec variable de contrôle
				\item Classificateur Bay\'esien Naif binomial et Gaussien
				\item R\'eduction dimensionnelle par la corr\'elation
				\item Algorithmes ID3, PRISM, AQ, CN2 et C4.5
				\item Analyse procust\'eenne
				\item Analyses en Composantes Ind\'ependantes (ICA)
				\item Modèle uplift
			\end{itemize}
		\item Informatique Quantique: 
			\begin{itemize}
				\item Impossibilit\'e du clonage quantique
			\end{itemize}
		\item Cryptographie: 
			\begin{itemize}
				\item Courbes elliptiques
			\end{itemize}	
		\item Ing\'enierie:
			\begin{itemize}
				\item Domaines de Box
				\item Plans de criblage d\'efinitifs (DSD)
				\item Plans Split-plots
				\item Plans Composites Centraux
				\item Plans Cubiques Faces Centr\'ees
				\item Modèle de Survie de Cox (modèle à hasard proportionnel de Cox)
				\item Mod\'elisation par \'equations structurelles				
				\item Tests de veillissements acc\'el\'er\'es
				\item Micro\'electronique (jonctions NPN/PNP, diodes, amplificateurs)
				\item \'Equations des t\'el\'egraphistes
				\item Th\'eorème de pouss\'ee de Kutta-Jukowski
			\end{itemize} 
		\item Th\'eorie des jeux et de la d\'ecision: 
			\begin{itemize}
				\item Coalition
				\item Valeur de Shapley
				\item Critère de Kelly 
			\end{itemize}
		\item \'Economie: 
			\begin{itemize}
				\item Taux de rendement continu
				\item Courbe des taux z\'ero-coupons
				\item \'equivalence du taux d'une obligation avec un bon du tr\'esor
				\item Taux Spot et taux Forward
				\item Ajustement du beta d'un portefeuille avec des Futures
				\item \'egalit\'e du prix de Cox-Ingersoll des Future/Forward
				\item Solution de l'EDP de Black \& Scholes
				\item Duration de Macaulay
				\item Duration Modifi\'ee
				\item Taux de Retour Interne Modifi\'e (MIRR)
				\item Couverture de portefeuilles par options
				\begin{itemize}
					\item Option de Vente/Achat de protection (protective Put/Call)
					\item \'ecart d'option d'achat à la hausse (bull Spread/Call)
					\item \'ecart d'option d'achat à la vente (bear Spread/Call)
					\item Papillon (butterfly)
					\item Op\'eration li\'ee (straddle)
					\item Position combin\'ee (strangle)
					\item Collar
					\item \'Ecart de Box (Box spread)
					\item \'Ecart de Calendrier (calendar spreads)
					\item M\'ethodes d'allocation de portefeuilles
					\begin{itemize}
						\item Portefeuille pond\'er\'e optimal pour un risque \'equilibr\'e
						\item Portefeuille pond\'er\'e optimal pour le suivi des erreurs
						\item Portefeuille pond\'er\'e optimal de Sharp
						\item Portefeuille pond\'er\'e optimal avec une diversification maximale
						\item Portefeuille de Treynor-Black optimal pond\'er\'e en fonction du benchmark
					\end{itemize}
				\end{itemize}
				\item Grecques pour les arbres binomiaux
				\item Swaps
				\item Formule de Margrabe
				\item Fomrule (approxmative) de Kirk
				\item Risque de d\'efaut de cr\'edit (bas\'e sur la notation de Standard \& Poor)
				\item CreditRisk+
				\item VaR Equity Coverage
				\item Perte conditionnelle de la VaR (CVaR)
				\item Approche de KMV-Merton pour le mesure de probabilit\'e de d\'efaut
				\item Distance jusqu'à d\'efaut
				\item \'Equation de Fokker-Planck
				\item Processus stochastiques ARCH-GARCH
				\item Modèles autor\'egressifs vectoriels pour s\'eries temporelles multivari\'ees
				\item Test de Granger pour la causalit\'e de deux s\'eries temporelles
				\item Filtres de Karman
				\item Modèle de pricing d'options de Heston
				\item Pricing d'options de type Spread
			\end{itemize}	
		\item Management Quantitatif: 
			\begin{itemize}
				\item Algorithme de Gale-Shapley
				\item Modèles Pareto/NBD, BG/NBD et BG/BB de la valeur vie client
				\item Problème du vendeur de journaux
				\item Effet coup de fouet
				\item Paradoxe de Condorcet	
				\item M\'ethode CRAFT (Computerized Relative Allocation of Facilities Technique)
				\item Options r\'eelles
				\item Capital diff\'er\'e en cas de survie (assurance vie)
				\item Indice de volatilit\'e CBOE (commenc\'e mais pas termin\'e)
				\item D\'ecès diff\'er\'e temporaire (assurance vie)
			\end{itemize}
		\item Biographies:
			\begin{itemize}
				\item Ajout de la religion de chaque individu pr\'esent\'e dans la biographie à la demande d'un lecteur
			\end{itemize}
	\end{itemize}
	De plus, nous souhaitons cr\'eer une version r\'esum\'ee de ce PDF (sans les d\'eveloppements d\'etaill\'es mais uniquement avec les r\'esultats principaux) et une pr\'esentation avec diapositives (r\'ealis\'ees avec Beamer) uniquement avec les r\'esultats, textes et images / graphiques principaux pouvant être utilis\'es par n'importe quel enseignant gratuitement dans le monde entier.
	
	Rappelez-vous que les sources \LaTeX{} de ce livre peuvent être obtenues en fonction de votre don sur Patreon, Paypal, Tipee ou via votre participation à la traduction de ce livre dans une autre langue.
	
	Comme chaque produit robuste a un cycle de vie, le cycle de vie commence lorsqu'un produit est mis à disposition du grand public et se termine lorsqu'il n'est plus pris en charge. Connaître les dates cl\'es de ce cycle de vie vous aide à prendre des d\'ecisions \'eclair\'ees sur la date de mise à niveau. Ce livre a le cycle de vie suivant: une nouvelle version majeure ou mineure est publi\'ee chaque fois qu'un seuil donn\'e de dons est atteint et peut être t\'el\'echarg\'e en cliquant sur le bouton suivant (PDF de $410$ m\'egaoctets ...):
	\begin{center}
		\href{http://www.sciences.ch/dwnldbl/telecharger.php3}{\includegraphics[scale=0.6]{img/books/download.jpg}}
	\end{center}
	ou si ce lien ne fonctionne pas, une copie du fichier PDF est disponible sur les Archives Internet:
	\begin{center}
		\includegraphics[scale=0.1]{img/internet_archive.jpg}
	\end{center}
	\begin{center}
	\href{https://archive.org/details/OperaMagistris}{https://archive.org/details/OperaMagistris}
	\end{center}
	
	Pour citer ce livre (non ce n'est pas une blague... la version anglaise sera probablement finie d'être traduite en français en 2040 selon nos estimations actuelles):
	\begin{quote}
	\noindent @book\{OperaMagistris2040v4, \\
		  title =        \{Opera Magistris - \'El\'ements de Math\'ematiques Appliqu\'ees pour Ing\'enieurs\}, \\
		  year =         \{2040\}, \\
		  keywords =     \{science, physique, maths, ing\'enierie, finance, management\}, \\
		  isbn =          \{978239909327\},\\
	\}
	\end{quote}
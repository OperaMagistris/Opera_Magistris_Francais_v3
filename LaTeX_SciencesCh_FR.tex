%---------------------------------
%Coordinateur: Vincent ISOZ
%Dernière mise à jour: 2019-02-04
%Contact: info@sciences.ch
%Compilé sur: TeXMaker 5.0.2 x64/MiKTeX 2.9 / Package Update 2018-10-25/Microsoft Windows 10 Creator Update x64/PC 16GB RAM (recommané sinon le docoument pourrait ne pas se compiler correctement sans déactiver les LOF et LOT!)

%Remarques: 

%R1. Les packages tkz-tab et etex doivent être installés manuellement depuis le Gestionnaire de Package MikTeX!!!

%R2. When you install first MiKTeX and after TeXMaker don't forget to reboot computer. Compile the document (probably will fail first time...). Do an update of packages using the MiKTeX package updater. Try compile again and use once again de MiKTeX package update. And it should work! If it still don't work i have notice that sometimes corrupted images cause the compilation to not be possible.

%R3. Merdi d'éviter d'utiliser le moins possible de styles et d'opérateurs mathématiques spéciaux définis dans le préamuble pour éviter toute confusion et complication de copier/coller (merci).

%R4. Since 2015-08-29 the ctable package makes conflicts and therefore you can't compile the document with it anymore. We have commented it.

%R5. When the MikTeX Updater updates the BibTex package, then you have to remove all temporary compilation files of the book to be able to compile it again.

%R6. Above 2,600 pages the MikTeX compilation reach TeX memory limit. You will get the message:
%TeX capacity exceeded, sorry [main memory size=3000000]
%To solve this open a command Window or the PowerShell and enter:
%initexmf --edit-config-file=pdflatex
%Add the following line in the new document that appears on the screen:
%pool_size=5000000
%main_memory=6000000
%extra_mem_bot=2000000
%font_mem_size=2000000
%and save it (ctrl+s) and quit the editor. After rebuild the format with:
%initexmf --dump=pdflatex
%that's it!!!!

%R7. To remove all equation in Notepad++ with a RegEx, use:
% (?s)\\begin\{gather\}.*?\\end\{gather\}
%don't forget to activate the field ". match newlines"
%---------------------------------
\documentclass[12pt,a4paper,twoside,openright]{report}
%solve problem of figures position
\usepackage{float}
%text styles and some special operators
\newcommand{\NewTerm}[1]{\textcolor{red}{#1}}
\newcommand{\SeeChapter}[1]{\textcolor{gray}{#1}}
\newcommand{\Ima}{\text{Im}}
\newcommand{\mi}{\mathrm{i}}
\newcommand{\card}[1]{\ensuremath{\left\|#1\right\|}}
%package to write TexMaker logo
\usepackage{hologo}
%package to write the reycling logo
\usepackage{recycle}
%package for tree logo (avoid to print)
\usepackage{fontawesome}
%package for beautiful quote font
\usepackage{calligra}
%package to control margine and paper size
\usepackage[paperheight=297mm,paperwidth=210mm,top=2.5cm,bottom=2.5cm,left=2cm,right=2cm,bindingoffset=1cm]{geometry}
%packages pour l'internationalisation de la police et son rendu
\usepackage[utf8]{inputenc}
\usepackage[T1]{fontenc}
\usepackage{emerald} %package for nice font for book title
\renewcommand{\rmdefault}{ptm}%to darken the text
\usepackage[french]{babel}
%package for specials chars (wingdings like)
\usepackage{pifont}
%package for per-thousand symbol
\usepackage{textcomp}
%package for version history
\usepackage{vhistory}
%packages for headers and footers
\usepackage{fancyhdr}
%package to force page notes to have no idents and to be at bottom of page
\usepackage[hang,flushmargin,bottom]{footmisc} 
%package to have more deep levels than by default
\usepackage{enumitem}
%package to have enumerations on multiple columsn
\usepackage{multicol}
%package for colors in tables, title page and anywhere else
\usepackage[usenames,dvipsnames]{xcolor}
\definecolor{BrickRed}{rgb}{0.72,0,0}
%change color of sections
\usepackage{sectsty}
\allsectionsfont{\color{black!50}}

%package to make captions small, bold and second package to make multiple captions
\definecolor{ocre}{RGB}{243,102,25}
\usepackage[font={scriptsize,bf},labelfont={color=ocre,bf},skip=0pt,justification=centering]{caption}

%\renewcommand{\thefigure}{\textcolor{ocre}{\bfseries\itshape\thechapter.\arabic{figure}}}
%\renewcommand{\figurename}{\textcolor{ocre}{\bfseries\itshape Fig.}}
%\renewcommand{\thetable}{\textcolor{ocre}{\bfseries\itshape\arabic{table}}}
%\renewcommand{\tablename}{\textcolor{ocre}{\bfseries\itshape Table}}

\usepackage{subcaption}
%definition of progress circles for sections
\newlength\charwidth
\newlength\chwidth
\newcommand*\circled[1]%
  \settototalheight\chwidth{#1\,\%}%
  \ifdim\chwidth>\charwidth\let\charwidth\chwidth\fi
  \addtolength\charwidth{15pt}% twice inner sep plus half line width
  \tikz[baseline=(char.base)];
    \draw [line width=2pt, color=basecol] (char.north) arc (90:90-#1*3.6:.5\charwidth) coordinate (a);
    \draw [line width=2pt, color=othercol]  (a) arc (90-#1*3.6:-270:.5\charwidth);
  }%
}
\colorlet{basecol}{purple}
\colorlet{othercol}{purple!25}
%end of definition of progress circles

%some colors for the title page
\definecolor{titlepagecolor}{cmyk}{0.3,0.3,0.3,0.3}
\definecolor{namecolor}{cmyk}{1,.50,0,.10}
%toc/lof/lot stuff
\usepackage{minitoc}
\setcounter{minitocdepth}{3}
\usepackage{tocloft} %this line and above two is to minimize vertical space in LOF and to increase number alignement in LOF
\addtolength{\cftfignumwidth}{1em}
\addtolength{\cfttabnumwidth}{1em}

%various packages for tables complications
\usepackage{hhline} %to draw correctyl double vertical/horizotnal lines
\usepackage{dcolumn} %to align decimals
\usepackage{multirow} %to merge cells
\usepackage{colortbl} %for background colors
\usepackage{array} %to center text horizontally in cells
\usepackage{variations} %to build variation tables
\usepackage{spreadtab} %for formulas in tables
\usepackage{booktabs} %for special borders
\usepackage{slashbox} %for diagonals in cells
\usepackage{fancybox} %for rows/columns with bck colors
\usepackage{hhline} %for double horizontal lines
\usepackage{diagbox}
%\usepackage{diagbox}
\usepackage{rotating} %rotation of text
\usepackage{makecell}
\setlength{\textfloatsep}{0.1cm} %space below table (between table and text)
%macro to vertically center images in cells
\newcommand\cincludegraphics[2][]{\raisebox{-0.6\height}{\includegraphics[#1]{#2}}}
%\usepackage{ctable} %for long tables
\usepackage{longtable} %to repeat title row on multiple pages
\usepackage{pbox} %for carriage return in cells
%package for charts and images
\usepackage{graphicx}
\usepackage{bclogo} %four countries flags
%package for videos (flash animations)
\usepackage{media9}
%package for lettrine
\usepackage{lettrine,oldgerm,yfonts}
%packages to include eps into final pdf (must be declared after graphicx package
\usepackage{epstopdf}
\epstopdfsetup{update} % only regenerate pdf files when eps file is newer
\usepackage{pdfpages}
%packages for *.eps images and others
\usepackage{graphicx}
\usepackage{epsfig}
\usepackage{transparent}
\usepackage{eso-pic}
%for background images
\newcommand\BackImage[2][scale=1]{%
\BgThispage
\backgroundsetup{
  contents={\includegraphics[#1]{#2}}
  }
}
%load packages for text canvas (borders)
\usepackage{bclogo,environ,wrapfig}
\usepackage[many]{tcolorbox}
\usepackage{pst-all}
%page to use footnotes in tables
\usepackage{footnote}
%to force big footnotes to stay on the same page
\interfootnotelinepenalty=10000

%numbering Definitions
\newcounter{def}
\newcommand{\mydef}{%
        \stepcounter{def}%
        \thedef}
        
%numbering lines
\usepackage[modulo,right]{lineno}
%put personal paragraphs number or simply margin notes
\usepackage{marginnote}
\newcounter{importantparagraph}[chapter]
\newcommand{\myparagraph}{%
        \stepcounter{importantparagraph}%
        \theimportantparagraph}

%style for equations boxes
\newcommand*{\boxcolor}{orange}
\makeatletter
\renewcommand{\boxed}[1]{\textcolor{\boxcolor}{%
\tikz[baseline={([yshift=-1ex]current bounding box.center)}] \node [rectangle, minimum width=1ex,rounded corners,draw] {\normalcolor\m@th$\displaystyle#1$};}}
 \makeatother
%end for equation boxes style

%post-it definition
\definecolor{myyellow}{RGB}{242,226,149}
\usetikzlibrary{shadows}
\NewDocumentCommand\StickyNote{O{6cm}mO{6cm}}{%
\begin{tikzpicture}
\node[
drop shadow={
  shadow xshift=2pt,
  shadow yshift=-4pt
},
inner xsep=7pt,
fill=myyellow,
xslant=-0.1,
yslant=0.1,
inner ysep=10pt
] {\parbox[t][#1][c]{#3}{#2}};
\end{tikzpicture}%
}
%end of post-it definition


%package for flowchart/workflows and definitions of some items
\usepackage{tikz}
\usetikzlibrary{backgrounds,arrows.meta,calc,arrows,fadings,mindmap}
\usetikzlibrary{decorations.pathreplacing,decorations.markings,patterns}
\usetikzlibrary{shapes.geometric,shapes.gates.logic.US,trees,positioning,arrows}
\usepackage{tkz-tab} %to draw tikz tables
%tikzlibrary for the standard model particle figure
\usetikzlibrary{calc,positioning,shadows,shadows.blur,decorations.pathreplacing}

%petit code pour désactiver une erreur tikz qui est apparue depuis les mise à jour des packages CTAN de Avril 2018
\makeatletter
\global\let\tikz@ensure@dollar@catcode=\relax
\makeatother

%for circle numbers (in section Automata Theory)
\newcommand*\circledtext[1]{\tikz[baseline=(char.base)]{
            \node[shape=circle,draw,inner sep=2pt] (char) {#1};}}
%package for karnaugh map
\usepackage{karnaughmap}
%\usepackage{karnaugh-map}
%for smileys
\usepackage{tikzsymbols}
%for bohr atoms
\usepackage{bohr}
%package for the font size of the morphisms venn diagram
\usepackage{scalefnt}
%to build matrix multiplication scheme
\usepackage{amsbsy}
\newcommand*{\clap}[1]{\hbox to 0pt{\hss#1\hss}}
\newcommand*{\mat}[1]{\boldsymbol{\mathrm{#1}}}
\newcommand*{\subdims}[3]{\clap{\raisebox{#1}[0pt][0pt]{$\scriptstyle(#2 \times #3)$}}}
\fboxrule=1pt
%to build star vote system
\usetikzlibrary{arrows,shapes.geometric,positioning,matrix}
\newcommand\score[2]{
\pgfmathsetmacro\pgfxa{#1+1}
\tikzstyle{scorestars}=[star, star points=5, star point ratio=2.25, draw,inner sep=1.3pt,anchor=outer point 3]
  \begin{tikzpicture}[baseline]
    \foreach \i in {1,...,#2} {
    \pgfmathparse{(\i<=#1?"yellow":"gray")}
    \edef\starcolor{\pgfmathresult}
    \draw (\i*1.75ex,0) node[name=star\i,scorestars,fill=\starcolor]  {};
   }
  \end{tikzpicture}
}
\usetikzlibrary{matrix}
\usetikzlibrary{shapes,arrows}
\usetikzlibrary{shapes.geometric, arrows}
\tikzstyle{startstop}=[rectangle,rounded corners,minimum width=3cm,minimum height=1cm,text centered,draw=black,fill=red!30]
\tikzstyle{io}=[trapezium,trapezium left angle=70,trapezium right angle=110,minimum width=3cm,minimum height=1cm, text centered,draw=black,fill=blue!30]
\tikzstyle{process}=[rectangle,minimum width=3cm,minimum height=1cm,text centered,text width=3cm,draw=black,fill=orange!30]
\tikzstyle{decision}=[diamond,minimum width=3cm,minimum height=1cm,text centered,draw=black,fill=green!30]
\tikzstyle{arrow}=[thick,->,>=stealth]
\tikzstyle{decision}=[diamond, draw,fill=blue!20,text width=4.5em,text badly centered,node distance=3cm, inner sep=0pt]
\tikzstyle{block}=[rectangle,draw,fill=blue!20,text width=5em,text centered,rounded corners,minimum height=4em]
\tikzstyle{line}=[draw,-latex']
\tikzstyle{cloud}=[draw,ellipse,fill=red!20,node distance=3cm,minimum height=2em]
%paragraphe identation by default
\setlength\parindent{0pt}
%package of report visual style
\usepackage{background}
\backgroundsetup{ contents= {\begin{tikzpicture}[remember picture, overlay] \draw [line width=0.3pt,color=gray,step=0.5cm] (current page.south west) grid (current page.north east); \end{tikzpicture} } scale=1, angle=0} 
\usepackage{atbegshi}% http://ctan.org/pkg/atbegshi
%for codes
\usepackage{listings,lstautogobble,xcolor}
%color definition for matlab script
\definecolor{mygreen}{RGB}{28,172,0} % color values Red, Green, Blue
\definecolor{mylilas}{RGB}{170,55,241}
\lstset{language=Matlab,%
    %basicstyle=\color{red},
    breaklines=true,%
    morekeywords={matlab2tikz},
    keywordstyle=\color{blue},%
    morekeywords=[2]{1}, keywordstyle=[2]{\color{black}},
    identifierstyle=\color{black},%
    stringstyle=\color{mylilas},
    commentstyle=\color{mygreen},%
    showstringspaces=false,%without this there will be a symbol in the places where there is a space
    numbers=left,%
    numberstyle={\tiny \color{black}},% size of the numbers
    numbersep=9pt, % this defines how far the numbers are from the text
    emph=[1]{for,end,break},emphstyle=[1]\color{red}, %some words to emphasise
    %emph=[2]{word1,word2}, emphstyle=[2]{style}, 
    autogobble=true   
}
%for algorithms
\usepackage[linesnumbered,ruled]{algorithm2e}
%load americal mathematical society packages for equations
\usepackage{amsmath,amsthm,amssymb,amsfonts}
\DeclareMathOperator{\sgn}{sgn} %must be after package ams!
\usepackage{gauss} %package for gauss elimination in linear algebra
\usepackage{xfrac} %package for small fractions like spin 1/2
\usepackage{gensymb} %for degree symbole
\usepackage{arcs} %for arc symbol over letters (overarc)
\usepackage{empheq} %for multiple equations box
\usepackage{etoolbox} %to create matrices with outbounds legends using bbordermatrix
\let\bbordermatrix\bordermatrix
\patchcmd{\bbordermatrix}{8.75}{4.75}{}{}
\patchcmd{\bbordermatrix}{\left(}{\left[}{}{}
\patchcmd{\bbordermatrix}{\right)}{\right]}{}{}
\AtBeginDocument{\AtBeginShipoutNext{\AtBeginShipoutDiscard}}
%\package to automatically resize equation to page width
\usepackage{resizegather}
\usepackage{mathtools} %for text above arrows
%package for linear system of equations
\usepackage{systeme}
%packages for special integral
\usepackage{wasysym} %closed inetegrals
\usepackage{esint} %intégrales de contour orientées (analyse complexe)
\usepackage{bigints} %write huge integral
%to have an arc over 2 letters as the arc package and the yhmath package have issues
\def\wideparen#1{\overset{\;\rotatebox{90}{)}}{#1}}
%package for unitary matrix
\usepackage{dsfont}
%definition for theorems
\theoremstyle{definition}
\newtheorem{theorem}{Théorème}[chapter]
\newtheorem{dem}{Démonstration}[theorem]
\newtheorem{corollary}{Corollaire}[theorem]
\newtheorem{lemma}{Lemme}[theorem]
\newtheoremstyle{itexmp}
 {\topsep}
 {\topsep}
 {\normalfont}
 {0pt}
 {\itshape\bfseries}{.}
 { }
 {#1 \textit{#2}}
\theoremstyle{itexmp}
\newtheorem{exmp}{Example}[chapter]
\newcommand{\statedefn}[2]{
 \definecolor{shadethmcolor}{cmyk}{0.1,0.05,0,0}
 \definecolor{shaderulecolor}{cmyk}{0.73,0.19,0,0}
 \begin{defn}\label{#1}{#2}\end{defn}
}
%for annuities symboles
\usepackage{lifecon}
%for euro symbol
\usepackage{eurosym}
%for tensor calculus AND annuities notations
\usepackage{tensor}
%for quantum bra-ket symbols
\usepackage{braket}
%numbering of equations restart at each section
\numberwithin{equation}{section} 
%package for terms simplifications in equations
\usepackage[makeroom]{cancel}
%package for watermark
\usepackage{draftwatermark}
\SetWatermarkLightness{0.85}
\SetWatermarkAngle{25}
\SetWatermarkScale{2}
\SetWatermarkFontSize{1cm}
\SetWatermarkText{Brouillon 4ème Édition}
%packages for chapters color margin markers and definitions
\usepackage{background}
\usepackage{xifthen}
\usepackage{totcount}
\regtotcounter{chapter}

\backgroundsetup
{   contents={
        \begin{tikzpicture}[overlay]
            \pgfmathtruncatemacro{\mytotalchapters}{\totvalue{chapter} > 0 ? \totvalue{chapter} : 20}
            \pgfmathsetmacro{\mypaperheight}{\paperheight/28.453}
            \pgfmathsetmacro{\mytop}{-(\thechapter-1)/\mytotalchapters*\mypaperheight}
            \pgfmathsetmacro{\mybottom}{-\thechapter/\mytotalchapters*\mypaperheight}
            \ifcase\thechapter
                \xdef\mycolor{white}
                \or \xdef\mycolor{red}
                \or \xdef\mycolor{orange}
                \or \xdef\mycolor{yellow}
                \or \xdef\mycolor{green}
                \or \xdef\mycolor{blue}
                \or \xdef\mycolor{violet}
                \or \xdef\mycolor{Apricot}
                \or \xdef\mycolor{Magenta}
                \or \xdef\mycolor{GreenYellow}
                \or \xdef\mycolor{Sepia}
                \or \xdef\mycolor{SkyBlue}
                \or \xdef\mycolor{Aquamarine}
                \or \xdef\mycolor{LimeGreen}
                \or \xdef\mycolor{WildStrawberry}
                \or \xdef\mycolor{RawSienna}
                \or \xdef\mycolor{Purple}
                \or \xdef\mycolor{RedOrange}
                \or \xdef\mycolor{CadetBlue}
                \else \xdef\mycolor{black}
            \fi
            \ifthenelse{\isodd{\value{page}}}
            {\fill[\mycolor] ($(current page.north east)+(0,\mytop)$) rectangle ($(current page.north east)+(-0.5,\mybottom)$);}
            {\fill[\mycolor] ($(current page.north west)+(0,\mytop)$) rectangle ($(current page.north west)+(0.5,\mybottom)$);}
        \end{tikzpicture}
    },
    scale=1,
    angle=0
}
%package for image positions
\usepackage{wrapfig}
\usepackage{picins}
%package lipsum and blindtext for various tests
\usepackage{lipsum,blindtext}
%definition for numbering sections
\setcounter{secnumdepth}{5}
\setcounter{tocdepth}{5}
%definition for main TOC depth
%\setcounter{\tocdepth}{5}
%package to get lastpage number
\usepackage{lastpage}
\AtBeginShipout{
\ifnum\value{page}=\number\numexpr\getpagerefnumber{LastPage}-0\relax
\phantomsection\label{preLastPage}
\fi}
%package for chapter styles formatting (gray box) and keept it here otherwise it will make a render bug
\usepackage[Bjornstrup]{fncychap}
\ChNameVar{\Huge}
%for electronic circuit
\usepackage[compatibility]{circuitikz}
\input{electronic_symbols_macros.tex}
%package for small TOC
\usepackage{shorttoc}
%was supposed to be use for Creative Common logo but doesn't work since August 2015 update
%\usepackage[type={CC},modifier={by},version={3.0},]{doclicense}
%part for bibtex
\usepackage[backend=bibtex]{biblatex}
\bibliography{biblio}
%package to make index
\usepackage{makeidx,imakeidx}
%to make work the options below first do an index with the options commented and after compile again with the option uncommented!
\makeindex[options=-s my_index_style.ist -c]

%decoration for title page (if \titlepacagedecoration uncommented further below
\newcommand\titlepagedecoration{
\begin{tikzpicture}[remember picture,overlay,shorten >= -10pt]

\coordinate (aux1) at ([yshift=-15pt]current page.north east);
\coordinate (aux2) at ([yshift=-410pt]current page.north east);
\coordinate (aux3) at ([xshift=-4.5cm]current page.north east);
\coordinate (aux4) at ([yshift=-150pt]current page.north east);

\begin{scope}[black!40,line width=12pt,rounded corners=12pt]
\draw
  (aux1) -- coordinate (a)
  ++(225:5) --
  ++(-45:5.1) coordinate (b);
\draw[shorten <= -10pt]
  (aux3) --
  (a) --
  (aux1);
\draw[opacity=0.6,black,shorten <= -10pt]
  (b) --
  ++(225:2.2) --
  ++(-45:2.2);
\end{scope}
\draw[black,line width=8pt,rounded corners=8pt,shorten <= -10pt]
  (aux4) --
  ++(225:0.8) --
  ++(-45:0.8);
\begin{scope}[black!70,line width=6pt,rounded corners=8pt]
\draw[shorten <= -10pt]
  (aux2) --
  ++(225:3) coordinate[pos=0.45] (c) --
  ++(-45:3.1);
\draw
  (aux2) --
  (c) --
  ++(135:2.5) --
  ++(45:2.5) --
  ++(-45:2.5) coordinate[pos=0.3] (d);   
\draw 
  (d) -- +(45:1);
\end{scope}
\end{tikzpicture}
}

%dedication page
\newenvironment{dedication}
  {\clearpage           % we want a new page
   \thispagestyle{empty}% no header and footer
   \vspace*{\stretch{1}}% some space at the top 
   \itshape             % the text is in italics
   \raggedleft          % flush to the right margin
  }
  {\par % end the paragraph
   \vspace{\stretch{3}} % space at bottom is three times that at the top
   \clearpage           % finish off the page
  }

%for start page
\usepackage{authblk}
%package for international date format
\usepackage[yyyymmdd]{datetime}

%package hyperref for links and pdf metadata, Must be the last package!
\usepackage[unicode,pdfencoding=auto,hidelinks,pdfinfo={pdfauthor={ISOZ Vincent},pdftitle={Elementary Applied Mathematics - Sciences.ch},pdfsubject={Elementary Applied Mathematics},pdfkeywords={Arithmetics;Statistics;Geometry;Trigonometry;Quantitative Finance;Engineering;Theoretical Computing},pdfproducer={Scientifc Evolution Sàrl}, pdfcreator={ISOZ Vincent}}]{hyperref}
\definecolor{wine-stain}{rgb}{0.5,0,0}
\hypersetup{colorlinks,linkcolor=wine-stain,linktoc=all}

%\usepackage[author={Opera Magistris}]{pdfcomment}

\renewcommand{\dateseparator}{--}
%package to control globally space between items
\usepackage{enumitem}
\setlist[1]{itemsep=6pt}
%three packages for old-style background
\usepackage{eso-pic}
\usepackage{changepage}
\strictpagecheck

%*****************************************fancy quotes starts here
\definecolor{quotemark}{gray}{0.7}
\makeatletter
\def\fquote{%
    \@ifnextchar[{\fquote@i}{\fquote@i[]}%]
           }%
\def\fquote@i[#1]{%
    \def\tempa{#1}%
    \@ifnextchar[{\fquote@ii}{\fquote@ii[]}%]
                 }%
\def\fquote@ii[#1]{%
    \def\tempb{#1}%
    \@ifnextchar[{\fquote@iii}{\fquote@iii[]}%]
                      }%
\def\fquote@iii[#1]{%
    \def\tempc{#1}%
    \vspace{1em}%
    \noindent%
    \begin{list}{}{%
         \setlength{\leftmargin}{0.1\textwidth}%
         \setlength{\rightmargin}{0.1\textwidth}%
                  }%
         \item[]%
         \begin{picture}(0,0)%
         \put(-15,-5){\makebox(0,0){\scalebox{3}{\textcolor{quotemark}{''}}}}%
         \end{picture}%
         \begingroup\itshape}%
 \def\endfquote{%
 \endgroup\par%
 \makebox[0pt][l]{%
 \hspace{0.8\textwidth}%
 \begin{picture}(0,0)(0,0)%
 \put(15,15){\makebox(0,0){%
 \scalebox{3}{\color{quotemark}''}}}%
 \end{picture}}%
 \ifx\tempa\empty%
 \else%
    \ifx\tempc\empty%
       \hfill\rule{100pt}{0.5pt}\\\mbox{}\hfill\tempa\ \emph{\tempb}%
   \else%
       \hfill\rule{100pt}{0.5pt}\\\mbox{}\hfill\tempa\ \emph{\tempb}\ \tempc%
   \fi\fi\par%
   \vspace{0.5em}%
 \end{list}%
 }%
 \makeatother
 %*****************************************fancy quotes ends here
%package for some advanced plots
\usepackage{pgfplots}
\usetikzlibrary{patterns}

\usepackage{titlesec}
%this is a test to make chapter title bigger
%\titleformat{\chapter}{\bf\huge}{\thechapter}{1em}{}

%to make beautiful section titles
\newcommand\titlebar{%
\tikz[baseline,trim left=3.1cm,trim right=3cm] {
    \fill [BrickRed!25] (4cm,-1ex) rectangle (\textwidth+3.1cm,2.5ex);
    \node [
        fill=BrickRed,
        anchor= base east,
        rounded rectangle,
        minimum height=3.5ex] at (4.5cm,0) {
        \textbf{\thesection.}
        %before is it was (but the compiler don't managed same anymore)
        %\textbf{\arabic{chapter}.\thesection.}
    };
}%
}
\titleformat{\section}{\LARGE}{\titlebar}{2cm}{} %distance from section text to number
%to have chapter number in front of sections and subsections
%\renewcommand*{\thesection}{\arabic{section}}

%push memory limits of LaTeX
%\usepackage{morewrites}
%if the above "morewrites" does not work...
\usepackage{scrwfile}

%Pour renommer les titres de la liste des figures et tables et algorithmes en français

\begin{document}
	%Pour les traductions de plein de termes de la version française
	\def\refname{R\'ef\'erences}%
   \def\abstractname{R\'esum\'e}%
   \def\bibname{Bibliographie}%
   \def\prefacename{Pr\'eface}%
   \def\chaptername{Chapitre}%
   \def\appendixname{Annexe}%
   \def\contentsname{Table des mati\`eres}%
   \def\listfigurename{Table des figures}%
   \def\listtablename{Liste des tableaux}%
   \def\indexname{Index}%
   \def\figurename{{\scshape Fig.}}%
   \def\tablename{{\scshape Tab.}}%
   \def\partname{\protect\@Fpt partie}%
   \def\pagename{page}%
   \def\seename{{\emph{voir}}}%
   \def\alsoname{{\emph{voir aussi}}}%
   \def\enclname{P.~J. }%
   \def\ccname{Copie \`a }%
   \def\proofname{D\'emonstration}% for AMS-LaTeX
   
	\sloppy %to force texts and equations to respects pages margins
	\pageref*{LastPage} %the star is to avoid the page number to be a link
	\raggedbottom
	\pagenumbering{roman}
	\newgeometry{margin=2.5cm}
%page de couverture
	\begin{titlepage}
	\newgeometry{left=2cm,bottom=2cm,top=2cm} %defines the geometry for the titlepage
	\BackImage[scale=1]{img/quantum.jpg}
	%\pagecolor{titlepagecolor}
	\noindent
	\color{gray}
	{\normalsize Le livre de niveau 1er cycle ultra-portable, ultra-facile, ultra rapide de 6'000 pages}\\
	{\normalsize sur les Éléments de Mathématiqus Appliquées pour Ingénieurs (EMAI). Rédigé pour la culture, et non pour le profit!!}
	\color{white}
	\makebox[0pt][l]{\rule{1.3\textwidth}{1pt}}\\
	\par
	\noindent
	\scalebox{2.2}{\fontsize{32pt}{0pt}\selectfont \textbf{Opera Magistris}} \\ \textcolor{namecolor}	{\textsf{Distribution originale}}
	\vfill
	\noindent
	{\huge \textsf{3ème Édition}}
	\vskip\baselineskip
	\noindent
	\textsf{\{ISBN 978-2-8399-0932-7\} Janvier 2014}
	%DOI: 10.13140/RG.2.2.31296.02562
	\vskip\baselineskip
	Compilé avec {\huge \LaTeXe} sur \TeX maker
	\vskip\baselineskip
	{\small
EMAI-3 est une oeuvre sous license Creative Commons "Attribution
3.0 Unported".\\ paternité; usage commerical interdit; conditions de partage identiques\\
\url{https://creativecommons.org/licenses/by-nc-sa/4.0/}}\\[2pt]
\includegraphics[width=1cm]{img/icons/share2.eps}\includegraphics[width=1cm]{img/icons/remix2.eps}
\includegraphics[width=1cm]{img/icons/by2.eps}
\includegraphics[width=1cm]{img/icons/nc-eu2.eps}
\includegraphics[width=1cm]{img/icons/sa2.eps}\\
2001-2018: \textit{Une livre pour les gouverner (presque) tous!} \href{https://www.facebook.com/sharer/sharer.php?u=http://www.sciences.ch/htmlen/latex/LaTeX_SciencesCh.pdf}{\faThumbsOUp{}}
\\\\
	
	%\titlepagedecoration
	\end{titlepage}
	\restoregeometry %restores the geometry
	\nopagecolor %use this to restore the color pages to white (or other color depending on your choice with \pagecolor{yellow!20} for example)
%end of title page
	\newpage\null\thispagestyle{empty}\newpage
	\restoregeometry
	%old style background
	%\AddToShipoutPicture{\checkoddpage
	%\ifoddpage
    %\put(0,0){\includegraphics[width=\paperwidth,height=\paperheight]{img/odd_page.jpg}}
	%\else
	%     \put(0,0){\includegraphics[width=\paperwidth,height=\paperheight]{img/even_page.jpg}}
	%\fi
 	%}r
	\begin{titlepage}

\newcommand{\HRule}{\rule{\linewidth}{0.5mm}} % Defines a new command for the horizontal lines, change thickness here

\center % Center everything on the page
 
%----------------------------------------------------------------------------------------
%	HEADING SECTIONS
%----------------------------------------------------------------------------------------

\textsc{\Large Distribution Originale}\\[1.5cm] % Name of your university/college
\textsc{\Huge \textbf{Opera Magistris}}\\[0.5cm] % Major heading such as course name
\textsc{\large 3ème Édition}\\[0.5cm] % Minor heading such as course title

%----------------------------------------------------------------------------------------
%	TITLE SECTION
%----------------------------------------------------------------------------------------

\HRule \\[0.4cm]
{ \Large \bfseries Compendium sur les \\ Éléments de Mathématiques Appliquées pour Ingénieurs}\\[0.4cm] % Title of your document
\HRule \\[1.5cm]
 
%----------------------------------------------------------------------------------------
%	AUTHOR SECTION
%----------------------------------------------------------------------------------------

%\begin{minipage}{0.4\textwidth}
%\begin{flushleft} \large
%\emph{Co-editors:}\\
%Léon \textsc{HARMEL}\\
%Vincent \textsc{ISOZ}\\
%\end{flushleft}
%\end{minipage}
%~
%\begin{minipage}{0.4\textwidth}
%\begin{flushright} \large
%\emph{Superviseurs:} \\
%F.D.C. Tigrou % Supervisor's Name Felis domesticus catus Tigrou
%\end{flushright}
%\end{minipage}\\[2cm]

% If you don't want a supervisor, uncomment the two lines below and remove the section above
%\Large \emph{Author:}\\
%John \textsc{Smith}\\[3cm] % Your name

%----------------------------------------------------------------------------------------
%	DATE SECTION
%----------------------------------------------------------------------------------------

{\large \today}\\[2cm] % Date, change the \today to a set date if you want to be precise

\vspace{120px}
\begin{center}
\textit{La Méthode Scientifique pour un meilleur Monde et réinformer le peuple!}
\end{center}

%----------------------------------------------------------------------------------------
%	LOGO SECTION
%----------------------------------------------------------------------------------------

\BackImage[scale=1.75]{img/god.jpg}
%----------------------------------------------------------------------------------------

\vfill % Fill the rest of the page with whitespace

\end{titlepage}
	\begin{versionhistory}
		\vhEntry{3.10}{2018-01-20}{VI}{Nouveaux sujets ajoutés et erreurs de traductions corrigées (voir le journal des changements)}
		\vhEntry{3.9}{2017-11-13}{VI}{Quelques mises à jour mineures pour la version gratuite du livre (voir le journal des changements)}
		\vhEntry{3.8}{2017-08-01}{VI}{Quelques mises à jour majeures et critiques et corrections diverses (voir le journal des changements)}
		\vhEntry{2.0}{2005-03-10}{VI}{2ème Édition francophone publiée sur Internet en tant que PDF ($2,001$ pages).}
		\vhEntry{1.0}{2002-05-01}{VI}{1ère Édition francophone publiée sur Internet en tant que pages HTML seulement.}
	\end{versionhistory}
	\begin{fquote}[Toni Morrison]S'il y a un livre que tu rêves de lire, mais qui n'a pas encore été écrit, alors c'est à toi de l'écrire.
 	\end{fquote}
	\newpage\null\thispagestyle{empty}\newpage %création d'une nouvelle page en forcant la disparition du numéro de page
	\dominitoc
	\shorttoc{Contents}{0} % Only chapters
	\pagebreak
	\renewcommand{\contentsname}{Table of Contents}
	\tableofcontents
	\newpage\null\thispagestyle{empty}\newpage
	\setlength{\parskip}{12pt}
	\pagenumbering{arabic}
	\clearpage %eliminate headers and footer of table of contents page
	\pagestyle{fancy} %du package fancyhdr!!!
	\renewcommand{\chaptermark}[1]{\markboth{\thechapter.\space#1}{}}
	\renewcommand{\sectionmark}[1]{\markright{#1}}
	\fancyhead[LE,RO]{\nouppercase\leftmark~(\nouppercase\rightmark)} %LE=Left Even,RO=Right Odd
	\fancyhead[LO,RE]{EMAI v3.11-2014}
	\renewcommand{\footrulewidth}{1pt}
	\fancyfoot[LE,RO]{{\thepage}/\pageref*{LastPage} (\Acrobatmenu{GoBack}{retour})}
	\fancyfoot[LO,RE]{\href{mailto:info@sciences.ch}{info@sciences.ch}}
	\fancyfoot[C]{}
	\let\cleardoublepage\clearpage

	\begin{dedication}
	{\LARGE Dédié à \textbf{Mère Nature}}
	\end{dedication}

	\chapter{Avertissements}
	\minitoc
		%to make section start on odd page
	\newpage
	\thispagestyle{empty}
	\mbox{}
	\section{Impressum}	
	\subsection{Utilisation du contenu}

	Le contenu de ce livre est \'elabor\'e par un processus de d\'eveloppement par lequel les volontaires parviennent à un consensus. Ce processus qui rassemble des b\'en\'evoles, recherche aussi le point de vue des personnes int\'eress\'ees par les sujets de ce livre. Le responsable de ce livre administre le processus et \'etablit des règles pour promouvoir l'\'equit\'e dans l'approche consensuelle. Il est \'egalement responsable de la r\'edaction du texte, parfois pour tester/\'evaluer ou v\'erifier de manière ind\'ependante l'exactitude ou l'exhaustivit\'e de l'information pr\'esent\'ee.

	Nous d\'eclinons toute responsabilit\'e pour toute blessure, dommage ou tout autre type, sp\'ecial, accessoire, cons\'ecutif ou compensatoire, d\'ecoulant de la publication, l'application ou la confiance du contenu de ce livre. Nous n'offrons aucune garantie expresse ou implicite quant à l'exactitude ou l'exhaustivit\'e des informations publi\'ees dans ce livre, et ne garantissons pas que les informations contenues dans ce livre r\'epondent à un besoin sp\'ecifique ou à un objectif du lecteur. Nous ne garantissons pas la performance des produits ou services d'un fabricant ou d'un fournisseur uniquement en vertu du contenu de ce livre.
	
	Les descriptions techniques, les proc\'edures et les programmes informatiques de ce livre ont \'et\'e d\'evelopp\'es sans pr\'ecaution pointue, ils sont donc fournis sans garantie d'aucune sorte. Nous ne garantissons pas non plus que les \'equations, les programmes et les proc\'edures de ce manuel ou de ses logiciels associ\'es sont exempts d'erreurs, ou sont conformes à une norme particulière de qualit\'e marchande, ou r\'epondront à vos exigences pour une application particulière. Ils ne devraient pas être invoqu\'es pour r\'esoudre un problème dont la solution incorrecte pourrait entraîner des blessures à des personnes ou la perte de biens. Toute utilisation du contenu de ce livre est aux risques et p\'erils du lecteur. Les auteurs, r\'edacteurs et \'editeurs d\'eclinent toute responsabilit\'e pour les dommages directs, indirects ou cons\'ecutifs r\'esultant de l'utilisation du contenu de ce livre ou des logiciels, codes  qui y sont associ\'es.

	En publiant des textes, il n'est pas dans l'intention de ce livre de fournir des services au nom d'une personne ou d'une entit\'e ou d'accomplir une tâche à accomplir par une personne ou une entit\'e au profit d'un tiers. Quiconque utilise ce livre devrait s'appuyer sur son propre jugement ind\'ependant ou, le cas \'ech\'eant, demander l'avis d'un expert (ou comit\'e d'experts) qualifi\'e pour d\'eterminer comment faire preuve de diligence raisonnable dans toutes les circonstances. Les informations et les normes sur les sujets couverts par ce livre peuvent être disponibles à partir d'autres sources que le lecteur peut souhaiter parcourir à la recherche de points de vue ou d'informations suppl\'ementaires non couvertes par le contenu de ce livre.

	Nous n'avons aucun pouvoir pour faire respecter le contenu de ce livre, et nous ne nous engageons pas à surveiller ou à faire respecter cette conformit\'e. Nous n'avons aucune activit\'e de certification, d'essai ou d'inspection de produits, de conceptions ou d'installations pour la s\'ecurit\'e ou la sant\'e des personnes et des biens. Toute certification ou autre d\'eclaration de conformit\'e concernant les informations relatives à la sant\'e ou à la s\'ecurit\'e des personnes et des biens, mentionn\'ee dans ce livre, ne peut être attribu\'ee au contenu de ce livre et reste sous la responsabilit\'e du centre de certification ou du d\'eclarant concern\'e.

	\subsection{Comment utiliser ce livre}
	Au niveau universitaire, ce livre peut être utilis\'e pour un doctorat, un diplôme d'\'etudes sup\'erieures ou un s\'eminaire avanc\'e de premier cycle dans de nombreux domaines des sciences exactes et pures. En r\'ealit\'e, ce livre vise \'egalement à couvrir le programme complet de la maternelle au doctorat.

	Parce que les m\'ethodes de math\'ematiques appliqu\'ees sont apprises par la pratique et l'exp\'erience, nous consid\'erons un s\'eminaire sur les math\'ematiques appliqu\'ees comme un s\'eminaire d'apprentissage par la pratique (ax\'e sur les projets). Nous structurons nos s\'eminaires de mod\'elisation math\'ematique autour d'un ensemble de problèmes qui n\'ecessitent que le stagiaire construise des modèles qui aident à la planification et à la prise de d\'ecision. L'imp\'eratif est que les modèles doivent être compatibles avec la th\'eorie et rev\'erifi\'es. Pour remplir cet imp\'eratif, il est n\'ecessaire que le stagiaire combine la th\'eorie math\'ematique avec la mod\'elisation. Le r\'esultat est que le stagiaire apprend la th\'eorie, et plus important encore, apprend comment cette th\'eorie est appliqu\'ee et combin\'ee dans le monde r\'eel. La capacit\'e de critiquer et d'identifier les limites des outils math\'ematiques dangereux est la caract\'eristique la plus importante de nos s\'eminaires.

	Les problèmes avec solutions pr\'esent\'ees dans ce livre fournissent l'occasion d'appliquer le mat\'eriel du texte à un ensemble complet de situations assez r\'ealistes (bien que simplifi\'ees). À la fin des s\'eminaires ainsi qu'à la fin de la lecture de ce livre, les stagiaires/lecteurs auront am\'elior\'e leurs comp\'etences et leurs connaissances des outils th\'eoriques et informatiques les plus importants et auront fortement d\'evelopp\'e leur esprit analytique. Ce sont des comp\'etences pr\'ecieuses qui sont demand\'ees par les entreprises et administrations au plus haut niveau.

	Il est très difficile de couvrir tout le mat\'eriel de ce livre en un semestre. Il faut beaucoup de temps pour expliquer les concepts aux stagiaires. Le lecteur est invit\'e à choisir les sujets qui seront abord\'es pendant le trimestre. Il n'est pas strictement n\'ecessaire de les couvrir dans l'ordre, mais cela peut aider de manière significative?

	En un mot, ce livre vous offre une grande vari\'et\'e de sujets qui peuvent être mod\'elis\'es. Tous, sans exception, sont utiles dans la pratique.
	
	\subsubsection{Guide des styles}
	Nous utilisons quelques conventions standard tout au long de ce livre. Ils ont été préparés évidemment avec \LaTeX{} qui numérote automatiquement les sections et le package hyperref qui génère les liens dans la Table des matières du présent PDF ainsi que d'autres références croisées faites dans le corps du texte.

	Nous utilisons la couleur et certaines boîtes pour mettre en évidence certains points pour que le lecteur puisse s'y référer plus facilement. En particulier:
	\begin{itemize}
		\item Les liens cliquables et les références croisées sont la plupart du temps en magenta (seulement quelque cas particuliers sont en bleu)
	
		\item Les textes expliquant la cible des références croisées sont en gris clair
	
		\item Les nouveaux termes (la plupart du temps associés aux définitions) sont en rouge
	
		\item Les remarques sont dans des boîtes avec une arrière plan grisé
	
		\item Les preuves commencent par le mot "Démonstration" et se terminent par un $\blacksquare$
	
		\item Les xemples supplémentaires \footnote{Un exemple compagnon à un théorème n'a pas de symbole spécifique} commence par le symbole {\Large \ding {45}}
	\end{itemize}
	Sans aucun doute, nous n’avons pas réussi à certains endroits à respecter ces règles. Si le lecteur garde une liste de ces transgressions pour nous la communiquer à la fin de la lecture de ce livre, cela pourrait valoir la peine de la partager avec les rédacteurs.
	
	\subsubsection{Vidéos et Animations}
	Ce livre contient des dizaines de petites vidéos et animations intégrées dans les pages PDF elles-mêmes. Cependant, pour pouvoir les jouer, vous aurez besoin:
	\begin{enumerate}
		\item D'Adobe Flash Player pour Firefox - NPAPI (la dernière version)
		
		\item D'activer l'option suivante disponible dans la plupart des lecteurs Adobe PDF sous Microsoft Windows:
		\begin{center}
			\includegraphics[scale=0.75]{img/intro/adobe_reader_multimedia.jpg}
		\end{center}
		Cette option doit être réactivée chaque fois que vous fermez et rouvrez le lecteur en fonction de la politique de votre département IT.
	\end{enumerate}
	
	\pagebreak
	\subsubsection{Annexes}
	Nous offrons gratuitement une gamme d'annexes (ouvrages compagnons) pour les \'etudiants, les instructeurs et les praticiens.
	
	D'abord, il y a des livres et des outils gratuits en français et en anglais pour les personnes qui veulent mettre en pratique la th\'eorie pr\'esent\'ee dans ce livre.
	
	Voici la liste:
	\begin{itemize}
		\item MATLAB™ en anglais (1,361 pages):\\ \href{http://www.sciences.ch/htmlfr/php/cliccount/click.php?id=319}{http://www.sciences.ch/dwnldbl/divers/Matlab.pdf}
		
		\item Maple en français (99 pages):\\ \href{http://www.sciences.ch/dwnldbl/divers/Maple.pdf}{http://www.sciences.ch/dwnldbl/divers/Maple.pdf}
		
		\item \textsf{R} en français (2,100 pages):\\ \href{http://www.sciences.ch/htmlfr/php/cliccount/click.php?id=313}{http://www.sciences.ch/dwnldbl/divers/R.pdf}
		
		\item Minitab en français (1,118 pages):\\ \href{http://www.sciences.ch/htmlfr/php/cliccount/click.php?id=282}{http://www.sciences.ch/dwnldbl/divers/Minitab.pdf}
		
		\item Scientific Linux installation \& Configuration en anglais  (211 pages):\\ \href{http://www.sciences.ch/dwnldbl/divers/ScientificLinux.pdf}{http://www.sciences.ch/dwnldbl/divers/ScientificLinux.pdf}
	\end{itemize}
	\begin{center}
		\includegraphics[scale=0.75]{img/books/matlab.jpg}
		\includegraphics[scale=0.75]{img/books/maple.jpg}
		\includegraphics[scale=0.75]{img/books/r.jpg}
		\includegraphics[scale=0.75]{img/books/minitab.jpg}
		\includegraphics[scale=0.75]{img/books/scientificlinux.jpg} 
	\end{center}	
	En second lieu, nous vous proposons gratuitement quelques Quizz et Flashcards en français et en anglais pour d\'efier vos \'etudiants ou vous-même avec le reste du monde:
	 \begin{itemize}
		\item Quizz MATLAB™ bases niveau L1 en français (100 questions)\\ \url{http://www.scientific-evolution.com/qcm/start_session/a73647cf3b/}
		
		\item Quizz Astonomie/Astrophysique niveau H1 en anglais (100 questions):\\ \url{http://www.scientific-evolution.com/qcm/start_session/ffd0810fa0/}
		
		\item Cartes de r\'evision pour les lettres grecques (48 cartes):\\
		\url{http://www.scientific-evolution.com/qcm/fr/start_session/6d9f1fef90/}
		
		\item Cartes de r\'evisions pour les d\'eriv\'ees les plus communes (29 cartes):\\
		\url{http://www.scientific-evolution.com/qcm/fr/start_session/c15a40f2c4/}
		
		\item Cartes de r\'evisions pour les int\'egrales les plus communes (60 cartes):\\
		\url{http://www.scientific-evolution.com/qcm/fr/start_session/ccfc20fdef/}
		
		\item Cartes de r\'evisions pour les identit\'es trigonom\'etriques les plus communes (68 cartes):\\
		\url{http://www.scientific-evolution.com/qcm/fr/start_session/882f9696cd/}
		
		\item Quizz \LaTeX{} niveau L3 en français (100 questions):\\ \url{http://www.scientific-evolution.com/qcm/fr/start_session/ff1e1d1b91/}
		
		\item Quizz R Software 3.1.2 niveau L3 en français (100 questions):\\ \url{http://www.scientific-evolution.com/qcm/fr/start_session/2a6fca7473/}
		
		\item Quizz C++ nivau en français (100 questions):\\
		\url{http://www.scientific-evolution.com/qcm/fr/start_session/e031ce4b43/}
	\end{itemize}
	Et comme tout livre technique devrait avoir un forum, le lecteur peut passer par ce lien pour toute discussion sur le contenu de ce livre: \url{https://www.physicsforums.com}.
	
	Pour ceux qui pr\'efèrent les r\'eseaux sociaux, nous avons aussi une page Facebook d\'edi\'ee:
	\begin{center}
		\faFacebook{} \href{https://www.facebook.com/operamagistris/}{https://www.facebook.com/operamagistris/}
	\end{center}
	ou pour plus de plaisir (photos de science, citations, blagues, vid\'eos, etc.) il y a aussi un compte Instagram associ\'e:
	\begin{center}
		\faInstagram{} \href{https://www.instagram.com/opera.magistris/}{https://www.instagram.com/opera.magistris/}
	\end{center}
%	et une petite collection non-exhaustive d'une s\'election de ce que nous consid\'erons comme des vid\'eos scientifiques int\'eressantes sur notre chaîne YouTube (voir à la fin de cette livre une section de chaînes YouTube scientifiques int\'eressantes):
%	\begin{center}
%		\faYoutube{} \href{https://www.youtube.com/user/AdminSciences}{https://www.youtube.com/user/AdminSciences}
%	\end{center}
	Comme pour le pr\'esent livre, les annexes et compl\'ements ci-dessus ne sont que des \'echantillons. Les versions complètes avec \underline{mises à jour gratuites perp\'etuelles} sont disponible pour le prix de $299$.- chacun et pour $499$.. vous obtenez les fichiers d'exercice et les sources \LaTeX{} (pour les informations sur l'achat, le lecteur peut simplement envoyer un e-mail à {\href{mailto:info@sciences.ch}{{\color{blue}email}}}).
	
	Pour les personnes qui veulent aider à traduire ce livre en d'autres langues, voici le lien avec les sources \ LaTeX {} pour les textes originaux (mais sans les \'equations\footnote{Les \'equations ne sont pas fournies car souhaitons contrôler qui fait quoi avec le livre!}!) sur GitHub:
	\begin{center}
		\faGithubSquare{} \href{https://github.com/OperaMagistris}{https://github.com/OperaMagistris}
	\end{center}
	Parce que ce livre se concentre principalement sur l'aspect math\'ematique (technique) des ph\'enomènes physiques, nous ne pouvons que recommander fortement au lecteur un autre livre gratuit qui est de notre point de vue subjectif actuellement le meilleur sur l'aspect scientifique populaire des sujets que nous couvrirons:
	\begin{center}
	Motion Mountain par le Dr. Christoph Schiller: \url{http://www.motionmountain.net}
	\end{center}

	\pagebreak
	\subsection{Protection des donn\'ees}
	Lorsque vous consultez des informations sur le site Internet compagnon (Sciences.ch), certaines donn\'ees sont automatiquement sauvegard\'ees. Nous essayons de sauvegarder le moins de donn\'ees possible et ce le plus rapidement possible. Partout où nous pouvons, nous n'avons que des donn\'ees anonymes. Nous nous engageons à traiter les donn\'ees que vous nous envoyez personnellement avec la plus grande diligence.

	Cependant, votre adresse IP et la page source qui vous emmène sur Sciences.ch et les mots-cl\'es associ\'es, sont disponibles gratuitement pour tout le monde \href{http://www.sciences.ch/htmlfr/php/cedstat/references.php}{ici} pour le mois en cours. Après quoi les donn\'ees d\'etaill\'ees sont d\'etruites. Vous pouvez vous opposer à tout moment à la publication de vos donn\'ees en nous contactant.

	\subsection{Utilisation des donn\'ees}
	Vos donn\'ees ne sont utilis\'ees que pour l'envoi du bulletin d'information (newletter) de Sciences.ch. La communication des donn\'ees personnelles (à l'exception de l'adresse e-mail, du titre et du nom) est facultative. Lorsque vous vous inscrivez à la newsletter, vous pouvez bien sûr sp\'ecifier une autre adresse et / ou un nom fictif.

	\subsection{Transmission des donn\'ees}
	Nous ne vendrons ni ne commercialiserons jamais les donn\'ees de nos clients ou parties int\'eress\'ees et n'affecterons jamais les droits de la personne. En outre, nous ne louerons pas de listes de diffusion et ne vous enverrons pas de publicit\'e de la part de tiers ou en notre nom.

	\subsection{Accord}
	Lorsque vous nous fournissez des informations personnelles, vous nous autorisez à les sauvegarder et à les utiliser au sens de la loi f\'ed\'erale suisse sur la protection des donn\'ees. Si vous nous demandez de ne pas vous envoyer d'e-mails (courriels), nous sommes dans l'obligation, dans votre int\'erêt, d'enregistrer votre e-mail (courriel) dans une liste n\'egative interne.
	
	\subsection{Errata}
	Bien que nous ayons pris toutes les pr\'ecautions pour assurer l'exactitude de notre contenu, des erreurs humaines peuvent se produire. Si vous trouvez une erreur dans ce livre - peut-être une erreur dans le texte, les scripts ou les illustrations - nous vous serions reconnaissants de nous le signaler. Ce faisant, vous pouvez \'eviter la frustration d'autres lecteurs en nous aidant à am\'eliorer les versions ult\'erieures de ce livre. Notre e-mail est donn\'e sur le pied de page à chaque page de ce livre. Une fois vos errata v\'erifi\'es, vos soumissions seront accept\'ees et l'erreur sera visible dans le journal des modifications des versions de mise à jour.

	%to make section start on odd page
	\newpage
	\thispagestyle{empty}
	\mbox{}
	\section{Licence}
	Tout le contenu de ce livre est soumis à la licence de documentation libre GNU, ce qui signifie:
	\begin{itemize}
			\item[$\bullet$] que chacun a le droit d'utiliser librement les textes à des fins non commerciales (Google Ads ou tout \'equivalent \'etant consid\'er\'e comme un usage commercial!)
			\item[$\bullet$] que toute personne est autoris\'ee à diffuser des articles pour un usage non commercial (Google Ads ou tout \'equivalent \'etant consid\'er\'e comme un usage commercial!)
			\item[$\bullet$] que n'importe qui peut \'editer librement les textes pour un usage non commercial (Google Ads ou tout \'equivalent \'etant consid\'er\'e comme un usage commercial!)
	\end{itemize}
	
	et bla bla bla...

	conform\'ement à la licence d\'ecrite ci-dessous: 
	
	\begin{center}
	\url{https://creativecommons.org/licenses/by-nc-sa/4.0/}\\[2pt]
\includegraphics[width=1cm]{img/icons/share2.eps}\includegraphics[width=1cm]{img/icons/remix2.eps}
\includegraphics[width=1cm]{img/icons/by2.eps}
\includegraphics[width=1cm]{img/icons/nc-eu2.eps}
\includegraphics[width=1cm]{img/icons/sa2.eps}
	\end{center}

	\begin{center}
	Version 1.1, Mars 2000
		
	Copyright (C) 2000 Fondation du Logiciel Libre, Inc. 59 Temple Place, Suite 330, Boston, MA 02111-1307 USA Tout le monde est autoris\'e à copier et à distribuer des copies verbatim de ce document de licence, mais il n'est pas autoris\'e de le modifier. 
	\end{center}

	\subsection{Pr\'eambule} 
	Le but de cette Licence est de rendre un manuel, une r\'ef\'erence ou un autre document \'ecrit "libre" dans le sens de la libert\'e: assurer à chacun la libert\'e effective de le copier et de le redistribuer, avec ou sans le modifier à des fins non commerciales \footnote{L'avantage imm\'ediat est que vous pouvez g\'en\'erer le livre, avec de nouveaux ensembles de problèmes, et le distribuer à vos \'etudiants simplement en format PDF (dans un courriel, par exemple). Mais plus g\'en\'eralement, si vous n'êtes pas int\'eress\'e par la façon dont nous avons expliqu\'e (ou omis d'expliquer) quelque chose, alors vous êtes libre de le r\'e\'ecrire. Si vous souhaitez couvrir plus de sujets (ou moins), vous êtes libre d'ajouter (ou de supprimer) les chapitres / sections / paragraphes que vous souhaitez. Et puisque le lecteur peut faire la demande du code source, il n'a pas besoin de recr\'eer la roue.}. En second lieu, cette Licence conserve pour l'auteur et l'\'editeur un moyen d'obtenir un cr\'edit pour leur travail, tout en n'\'etant pas consid\'er\'e comme responsable des modifications apport\'ees par d'autres.

	Cette Licence est une sorte de "copyleft", ce qui signifie que les oeuvres d\'eriv\'ees du Document doivent elles-mêmes être libres dans le même sens. Il complète la licence publique g\'en\'erale GNU, qui est une licence copyleft conçue pour un logiciel libre.

	Nous avons conçu cette Licence afin de l'utiliser pour les manuels de logiciels libres, car le logiciel libre a besoin de documentation gratuite: un programme gratuit devrait être fourni avec des manuels offrant les mêmes libert\'es que le logiciel. Mais cette Licence n'est pas limit\'ee aux manuels de logiciels; il peut être utilis\'e pour n'importe quel travail textuel, ind\'ependamment du sujet ou s'il est publi\'e comme un livre imprim\'e. Nous recommandons cette Licence principalement pour les travaux dont l'objectif est l'instruction ou la r\'ef\'erence.

	\subsection{Applications et D\'efinitions}
	Cette Licence s'applique à tout travail manuel ou autre contenant un avis plac\'e par le d\'etenteur des droits d'auteur indiquant qu'il peut être distribu\'e selon les termes de cette Licence. Le "Document", ci-dessous, fait r\'ef\'erence à un tel manuel ou travail. Tout membre du public est un titulaire de Licence et est appel\'e «vous».

	Une "Version Modifi\'ee" du Document d\'esigne tout travail contenant le Document ou une partie de celui-ci, soit copi\'e textuellement, soit avec des modifications et / ou traduit dans une autre langue.

	Une «section secondaire» est une annexe nomm\'ee ou une section de première instance du Document qui traite exclusivement de la relation entre les \'editeurs ou les auteurs du Document à la matière globale du Document (ou à des questions connexes) et ne contient rien qui pourrait tomber directement dans ce sujet global (par exemple, si le Document est en partie un manuel de math\'ematiques, une section secondaire pourrait ne pas expliquer les math\'ematiques). La relation pourrait être une question de connexion historique avec le sujet ou avec des questions connexes, ou juridique, commercial, philosophique, position \'ethique ou politique à leur \'egard.
	
	Les "Sections Invariantes" sont certaines Sections Secondaires dont les titres sont d\'esign\'es, comme \'etant ceux des Sections Invariantes, dans l'avis qui dit que le Document est publi\'e sous cette Licence.

	Les "Textes de couverture" sont de courts passages de texte qui sont list\'es, en tant que Textes d'ouverture ou Textes de clôture, dans l'avis indiquant que le Document est publi\'e sous cette Licence.

	Une copie "Transparente" du Document d\'esigne une copie lisible par machine, repr\'esent\'ee dans un format dont la sp\'ecification est accessible au grand public, dont le contenu peut être visualis\'e et \'edit\'e directement et directement avec des \'editeurs de texte g\'en\'eriques ou (pour des images compos\'ees de pixels) programmes de peinture g\'en\'eriques ou (pour les dessins) un \'editeur de dessin largement disponible, et qui est adapt\'e pour l'entr\'ee aux formateurs de texte ou pour la traduction automatique à une vari\'et\'e de formats appropri\'es pour l'entr\'ee aux formateurs de texte. Une copie faite dans un format de fichier autrement transparent dont le balisage a \'et\'e conçu pour contrecarrer ou d\'ecourager la modification ult\'erieure par les lecteurs n'est pas transparent. Une copie qui n'est pas "transparente" est appel\'ee "opaque".

	Des exemples de formats appropri\'es pour les copies transparentes incluent l'ASCII simple sans balisage, le format d'entr\'ee Texinfo, le format d'entr\'ee LaTeX, SGML ou XML utilisant une DTD accessible au public et un HTML simple conforme aux normes conçu pour la modification humaine. Les formats opaques incluent PostScript, PDF, formats propri\'etaires qui ne peuvent être lus et \'edit\'es que par des traitements de texte propri\'etaires, SGML ou XML pour lesquels la DTD et / ou les outils de traitement ne sont g\'en\'eralement pas disponibles, et HTML produit par certains traitements de texte pour fins de sortie seulement.

	La «page de titre» d\'esigne, pour un livre imprim\'e, la page de titre elle-même, ainsi que les pages suivantes qui sont n\'ecessaires pour contenir, de manière lisible, le mat\'eriel dont cette Licence a besoin pour apparaître dans la page de titre. Pour les œuvres dans des formats qui n'ont pas de page de titre en tant que telle, "Page de titre" signifie le texte près de l'apparence la plus visible du titre de l'oeuvre, pr\'ec\'edant le d\'ebut du corps du texte.

	\subsection{Copie Verbatim} 
	Vous pouvez copier et distribuer ce libre sur n'importe quel support, non commercial, à condition que cette Licence, les avis de copyright et l'avis de licence indiquant que cette Licence s'applique au livre soient reproduits dans toutes les copies et que vous n'ajoutez aucune autre condition à celles de cette Licence. Vous ne pouvez pas utiliser des mesures techniques pour entraver ou contrôler la lecture ou la copie ult\'erieure des copies que vous faites ou distribuez. Cependant, vous pouvez accepter une compensation en \'echange de copies. Si vous distribuez un nombre suffisant de copies, vous devez \'egalement respecter les conditions de la section 3.

	Vous pouvez \'egalement prêter des copies, dans les mêmes conditions que celles indiqu\'ees ci-dessus, et vous pouvez en afficher publiquement des copies. 

	\subsection{Copie en Quantit\'e}
	Si vous publiez des copies imprim\'ees du Document en quantit\'e sup\'erieurs à $100$ et que l'avis de Licence du Document exige des textes de couverture, vous devez joindre les copies dans des pochettes qui portent, clairement et lisiblement, tous ces textes de couverture: les textes de la première de couverture sur la page de couverture. les textes de la dernière de couverture sur la couverture arrière. Les deux couvertures doivent \'egalement vous identifier clairement et lisiblement en tant qu'\'editeur de ces copies. La couverture doit pr\'esenter le titre complet avec tous les mots du titre tout aussi visible l'un que l'autre. Vous pouvez ajouter d'autres informations sur les couvertures en plus. La copie avec des modifications limit\'ees aux couvertures, tant qu'elles conservent le titre du Document et satisfont à ces conditions, peut être consid\'er\'ee comme une reproduction textuelle à d'autres \'egards.

	Si les textes requis pour l'une ou l'autre des couvertures sont trop volumineux pour être lisibles, vous devez mettre les premiers inscrits (autant que possible) sur la couverture actuelle, et continuer le reste sur les pages suivantes.

	Si vous publiez ou distribuez des copies opaques en quantit\'e sup\'erieure à $100$, vous devez inclure une copie transparente lisible par machine avec chaque copie opaque, ou indiquer dans chaque copie opaque un emplacement r\'eseau informatique accessible au public contenant une copie complète transparente du Document, sans mat\'eriel ajout\'e et de manière à ce que le public utilisant un un r\'eseau g\'en\'eral y ait accès de façon anonyme, sans frais et en utilisant des protocoles de r\'eseau public ou standard. Si vous utilisez cette dernière option, vous devez prendre des mesures raisonnablement prudentes lorsque vous commencez la distribution de copies opaques en quantit\'e, pour vous assurer que cette copie transparente restera accessible à l'emplacement indiqu\'e jusqu'à au moins un an après la dernière distribution d'un copie opaque (directement ou par l'interm\'ediaire de vos agents ou d\'etaillants) de cette \'edition au public.

	Il est demand\'e, mais pas obligatoire, que vous contactiez les auteurs du Document bien avant de redistribuer un grand nombre de copies, pour leur donner une chance de vous fournir une version mise à jour du Document.

	\subsection{Modifications}
	Vous pouvez copier et distribuer une version modifi\'ee du Document dans les conditions des sections 2 et 3 ci-dessus, à condition de publier la Version Modifi\'ee sous cette Licence, avec la version modifi\'ee remplissant le rôle du Document, autorisant ainsi la distribution et la modification de la Version Modifi\'ee à quiconque en possède une copie. En outre, vous devez effectuer ces op\'erations dans la version modifi\'ee:
	
	\begin{itemize}
		\item Utilisez dans la page de titre (et sur les couvertures, le cas \'ech\'eant) un titre distinct de celui du Document, et de ceux des versions pr\'ec\'edentes (qui devraient, le cas \'ech\'eant, figurer dans la section \textit{Historique des révisions} du document). Vous pouvez utiliser le même titre que la version pr\'ec\'edente si l'\'editeur d'origine de cette version donne son autorisation.

		\item Lister sur la page de titre, en tant qu'auteurs, une ou plusieurs personnes ou entit\'es responsables de la paternit\'e des modifications de la Version Modifi\'ee, ainsi qu'au moins cinq des auteurs principaux du document (tous ses auteurs principaux, s'il y en a moins de cinq).

		\item Indiquer sur la page de titre le nom de l'\'editeur de la version modifi\'ee, en tant qu'\'editeur.

		\item Conserver tous les avis de droits d'auteur du document.

		\item Ajoutez un avis de droit d'auteur appropri\'e pour vos modifications à côt\'e des autres avis de droits d'auteur. 

		\item Inclure, imm\'ediatement après les avis de droits d'auteur, un avis de licence autorisant le public à utiliser la Version Modifi\'ee conform\'ement aux conditions de la pr\'esente Licence, sous la forme indiqu\'ee dans l'addenda ci-dessous.

		\item Pr\'eservez dans cet avis de licence les listes complètes des sections invariantes et les textes de couverture requis indiqu\'es dans l'avis de Licence du Document.

		\item Inclure une copie non modifi\'ee de cette licence.

		\item Conservez la section intitul\'ee \textit{Historique des révisions} et son titre, et ajoutez-y un \'el\'ement indiquant au moins le titre, l'ann\'ee, les nouveaux auteurs et l'\'editeur de la version modifi\'ee comme indiqu\'e sur la page de titre. S'il n'y a pas de section intitul\'ee \textit{Historique des révisions} dans le document, cr\'eez-en un en indiquant le titre, l'ann\'ee, les auteurs et l'\'editeur du document comme indiqu\'e sur sa page de titre, puis ajoutez un \'el\'ement d\'ecrivant la version modifi\'ee.

		\item Conservez l'emplacement r\'eseau, le cas \'ech\'eant, indiqu\'e dans le document pour l'accès public à une copie transparente du Document, ainsi que les emplacements r\'eseau indiqu\'es dans le document pour les versions pr\'ec\'edentes sur lesquelles il \'etait bas\'e. Ceux-ci peuvent être plac\'es dans la section \textit{Historique des révisions}. Vous pouvez omettre un emplacement r\'eseau pour un travail publi\'e au moins quatre ans avant le document lui-même ou si l'\'editeur d'origine de la version à laquelle il fait r\'ef\'erence donne son autorisation.

		\item Dans toute section intitul\'ee «Remerciements» ou «D\'edicaces», conserver le titre de la section et conserver dans la section toute la substance et le ton de chacun des remerciements et / ou d\'edicaces du contributeur qui y sont donn\'es.

		\item Conserver toutes les sections invariables du Document, inchang\'ees dans leur texte et dans leurs titres. Les num\'eros de section ou l'\'equivalent ne sont pas consid\'er\'es comme faisant partie des titres de section.

		\item  Supprimer toute section intitul\'ee "Avenants". Une telle section peut ne pas être incluse dans la Version Modifi\'ee.

		\item Ne retitrez aucune section existante en tant que "Avenants" ou tout autre section qui pourrait g\'en\'erer un conflit dans le titre avec une Section Invariante.

		\item Si la version modifi\'ee inclut de nouvelles sections ou appendices qualifi\'ees en tant que Sections Secondaires et ne contient aucun contenu copi\'e à partir du Document original, vous pouvez, à votre discr\'etion, d\'esigner certaines ou toutes ces sections comme invariantes. Pour ce faire, ajoutez leurs titres à la liste des sections invariantes dans l'avis de licence de la Version Modifi\'ee. Ces titres doivent être distincts des autres titres de section.

		\item Vous pouvez ajouter une section intitul\'ee «Avenants», à condition qu'elle ne contienne que des approbations de votre version modifi\'ee par diverses parties - par exemple, des d\'eclarations d'examen par les pairs ou que le texte a \'et\'e approuv\'e par une organisation comme d\'efinition officielle d'une norme.

		\item Vous pouvez ajouter jusqu'à cinq mots comme texte de première de couverture et jusqu'à 25 mots comme texte de quatrième couverture à la fin de la liste des textes de ces deux couverture dans la Version Modifi\'ee. Un seul passage de texte de premère de couverture et un de texte de quatrième de couverture peuvent être ajout\'es par (ou par des arrangements faits par) une entit\'e. Si le Document contient d\'ejà un texte de couverture pour les couvertures sus-mentionn\'ees, pr\'ec\'edemment ajout\'e par vous ou par un arrangement conclu avec la même entit\'e sous le compte de laquelle vous agissez, vous ne pouvez pas en ajouter un autre; mais vous pouvez remplacer l'ancien, avec l'autorisation explicite de l'\'editeur pr\'ec\'edent qui a ajout\'e l'ancien.

		\item Le(s) auteur(s) et le(s) \'editeur(s) du document ne donnent pas la permission d'utiliser leur nom à des fins publicitaires, d'affirmer ou de sous-entendre l'approbation d'une Version Modifi\'ee.
	\end{itemize} 

	\subsection{Combinaison de Documents}
	Vous pouvez combiner le Document avec d'autres documents publi\'es \'egalement sous cette même Licence, selon les termes d\'efinis dans la section 4 ci-dessus pour les Versions Modifi\'ees, à condition d'inclure dans la combinaison toutes les Sections Invariantes de tous les documents originaux, non modifi\'es, et de les lister tous comme Sections Invariantes de votre travail combin\'e dans son avis de Licence.

	Le travail combin\'e doit seulement contenir une copie de cette Licence, et plusieurs Sections Invariantes identiques peuvent être remplac\'ees par une seule copie. S'il y a plusieurs Sections invariantes avec le même nom mais des contenus diff\'erents, rendez le titre de chaque section unique en ajoutant à la fin, entre parenthèses, le nom de l'auteur original ou de l'\'editeur de cette section s'il est connu, ou bien un nombre unique. Apportez le même ajustement aux titres de section dans la liste des sections invariables dans l'avis de Licence du travail combin\'e.
	
	Dans la combinaison, vous devez combiner toutes les sections intitul\'ees \textit{Historique des révisions} dans les divers documents originaux, en formant une section intitul\'ee \textit{Historique des révisions}; De même, vous devez combiner toutes les sections intitul\'ees \textit{Remerciements}.

	\subsection{Collections de Documents}
	Vous pouvez faire une collection constitu\'ee du Document et d'autres documents publi\'es sous cette Licence, et remplacer les copies individuelles de cette Licence dans les divers documents par une seule copie qui est incluse dans la collection, à condition que vous respectiez les règles de cette Licence pour chaque copie textuelle de chacun des documents à tous \'egards.

	Vous pouvez extraire un seul document d'une telle collection et le distribuer individuellement sous cette Licence, à condition d'ins\'erer une copie de cette Licence dans le document extrait, et de suivre cette Licence à tous \'egards concernant la copie textuelle de ce document.

	\subsection{Agr\'egation avec des œuvres ind\'ependantes} 
	Une compilation du Document ou de ses d\'eriv\'es avec d'autres documents ou travaux s\'epar\'es ou ind\'ependants, dans ou sur un volume d'un support de stockage ou de distribution, ne compte pas dans son ensemble comme Version Modifi\'ee du Document, à condition qu'aucun droit de compilation ne soit r\'eclam\'e pour la compilation. Une telle compilation est appel\'ee un "agr\'egat", et cette Licence ne s'applique pas aux autres oeuvres autonomes ainsi compil\'ees avec le Document, du fait qu'elles sont ainsi compil\'ees, si elles ne sont pas elles-mêmes des oeuvres d\'eriv\'ees du Document.

	Si l'exigence du texte de couverture de la section 3 s'applique à ces copies du Document, alors si le document repr\'esente moins du quart de l'ensemble, les textes de couverture du Document peuvent être plac\'es sur des couvertures qui entourent uniquement le Document à l'int\'erieur de l'aggr\'egat. Sinon, ils doivent apparaître sur les couvertures autour de l'ensemble des agr\'egats.
	
	\subsection{Compilation du Document}
	Vous aurez besoin de trois trois choses pour g\'en\'erer ce document pour vous-même:
	\begin{enumerate}
		\item Installer \href{https://miktex.org/}{MiKTeX}

		\item Installer \href{http://www.xm1math.net/texmaker/index_fr.html}{TeXMaker}

		\item Une connexion Internet

		\item Configurer MikTeX comme indiqu\'e dans les remarques au d\'ebut du fichier \textit{LaTeX\_SciencesCh\_FR.tex}
	\end{enumerate}

	\subsection{Traductions}
	La traduction est consid\'er\'ee comme une sorte de modification, vous pouvez donc distribuer des traductions du document sous les termes de la section correspondante sur la transformation. Le remplacement de sections invariables par des traductions n\'ecessite une autorisation sp\'eciale de leurs d\'etenteurs de droits d'auteur, mais vous pouvez inclure des traductions de certaines ou de toutes les sections invariables en plus des versions originales de ces sections invariables.

	\subsection{R\'esiliation}
	Vous n'êtes pas autoris\'e à copier, modifier, sous-licencier ou distribuer le Document, sauf dans les cas express\'ement pr\'evus par cette Licence. Toute autre tentative de copier, modifier, sous-licencier ou distribuer le document est annul\'ee et met fin automatiquement à vos droits en vertu de cette Licence. Cependant, les parties qui ont reçu des copies ou des droits de votre part en vertu de cette Licence ne verront pas leur licence r\'esili\'ee tant que ces parties resteront en pleine conformit\'e.

	\subsection{R\'evisions futures de cette Licence}
	La Free Software Foundation (FSF) peut publier de temps à autre de nouvelles versions r\'evis\'ees de la Licence de documentation libre GNU. Ces nouvelles versions seront similaires dans l'esprit à la pr\'esente version, mais peuvent diff\'erer dans le d\'etail pour r\'epondre à de nouveaux problèmes ou pr\'eoccupations. Voir \href{http://www.gnu.org/copyleft/}{{\color{blue} http://www.gnu.org/copyleft/}}.

	Chaque version de la Licence reçoit un num\'ero de version distinctif. Si le Document sp\'ecifie qu'une version num\'erot\'ee particulière de cette Licence "ou toute version ult\'erieure" s'applique à elle, vous avez la possibilit\'e de suivre les termes et conditions de cette version sp\'ecifi\'ee ou de toute version ult\'erieure qui a \'et\'e publi\'ee (et non en tant que brouillon) par la FSF. Si le Document ne sp\'ecifie pas le num\'ero de version de cette Licence, vous pouvez choisir n'importe quelle version jamais publi\'ee (pas comme brouillon) par la FSF.
	
	\begin{center}
	\color{ForestGreen}{{\Large \faTree} \textbf{S'il-vous-plaît, pensez à l'environnement avant d'imprimer}}
	\end{center}
	
	%to make section start on odd page
	\newpage
	\thispagestyle{empty}
	\mbox{}
	\section{Feuille de route}
	Ce livre a une règle de progression simple qui est: $1$ nouvelle page A4 par jour depuis Mai 2001 sur des sujets qui int\'eressent le superviseur de la distribution \textit{Originale} du livre \textit{Opera Magistris}. Les nouveaux sujets sont d\'ebloqu\'es (publi\'es) en fonction des paliers de dons qui sont faits via note page \href{https://www.tipeee.com/elements-of-applied-mathematics}{Tipeee}, \href{https://www.patreon.com/sciences}{Patreon} ou \href{https://www.paypal.me/scientificevolution}{Paypal}. Les sujets suivants sont d\'ejà pr\'evus pour un futur proche ou lointain avec toujours le même niveau de d\'etails et d'approche p\'edagogique dans les preuves math\'ematiques que le reste du livre (tous les sujets ci-dessous devraient n\'ecessiter environ $1'500$ à $2'500$ pages suppl\'ementaires):
	\begin{itemize}
		\item Introduction:
			\begin{itemize}
				\item Ajouter des remerciements à toutes les personnes qui ont cr\'e\'e la distribution MikTeX / LaTeX et aux packages utilis\'es pour ce livre
			\end{itemize}
		\item Probabilit\'es:
			\begin{itemize}
				\item Conjugaison Bay\'esienne pour la loi Normale et Binomiale
				\item Chaînes de Markove cach\'ees
				\item Log-loss
			\end{itemize}
		\item Statistiques: 
			\begin{itemize}
				\item Mode et M\'ediane de lois statistiques
				\item Maximum de vraisemblance pour donn\'ees censur\'ees
				\item Entropie de la loi Normale
				\item Score de Propension
				\item Test d'\'equivalence
				\item Matrice de Quasi-corr\'elation
				\item Analyse Factorielle
				\item T-Test de Hotelling
				\item R\'esidus standardis\'es de Pearson
				\item Test de Welch avec \'equation de Welch-Satterhwaite
				\item ANCOVA
				\item Test de Cramer-vos Mises
				\item Test de Wald-Wolfowitz Test (des s\'equences binaires)
				\item Test de Levene-Wolfwitz\footnote{aussi appel\'e  "test de point de bifurcation" ou "test de tendance"} (s\'equences haussière/baissière)
				\item Ellipse de contrôle
				\item Mesures bas\'ees sur l'entropie des tables de contingence
				\item Test de tendance d'Armitage
				\item Test d'Ansari-Bradley
				\item Test r\'egulier de Dickey-Fuller
				\item Modèle de Poisson pour la distance spatiale moyenne (2D)
				\item Corr\'elation canonique
				\item G-test de p\'eriodicit\'e
				\item Copula Gaussien et de Student
				\item ANOVA à facteur fixe hi\'erarchique
				\item Introduction à la MANOVA
				\item Th\'eorème des valeurs extrêmes
				\item Th\'eorie des sondages
				\item Modèles lin\'eaires g\'en\'eralis\'es (Gauss, Poissson, Binomial N\'egatif, Gamma)
				\item R\'egression PLS (moindres carr\'es partiels)
				\item Moindres carr\'es en deux \'etapes (2SLS)
				\item R\'egression logique
				\item Chi-carr\'e ajust\'e
				\item Probabilit\'e des fonctions g\'en\'eratrices
			\end{itemize}
		\item G\'eom\'etrie:
			\begin{itemize}
				\item Volume de l'hypersphère
			\end{itemize}
		\item Calcul Diff\'erentiel
			\begin{itemize}
				\item Int\'egrale de Lebesgue avec application num\'erique dans MATLAB™
				\item Repr\'esentation int\'egrale des fonctions de Bessel
				\item Lin\'earisation des puissances des fonctions trigonom\'etriques
				\item Int\'egrales elliptiques et fonction elliptiques
			\end{itemize}
		\item Analyse: 
			\begin{itemize}
				\item Transform\'ee de Hilbert
			\end{itemize}
		\item Analyse Complexe: 
			\begin{itemize}
				\item Th\'eorème des r\'esidus pour des ratios de polynômes
				\item Th\'eorème de la valeur moyenne de Gauss
			\end{itemize}
		\item Topologie: 
			\begin{itemize}
				\item Distance de Mahalanobis
			\end{itemize}			
		\item G\'eom\'etrie Diff\'erentielle: 
			\begin{itemize}
				\item Coordonn\'ees normales
				\item Courbure de Gauss
				\item P\'erimètre d'un circle sur le plan, sur la sphère et sur une surface hyperbolique
				\item Th\'eorème isop\'erimètrique du plan
			\end{itemize}
		\item M\'ecanique: 
			\begin{itemize}
				\item Effet Magnus
				\item Critère de Lawson (plasmas)
				\item \'ecoulement à travers un orifice submerg\'e
				\item \'ecoulement sur les encoches et les d\'eversoirs
				\item Force due à l'\'ecoulement autour d'un coude de tuyau
				\item Force sur une buse
				\item Impact d'un jet sur un avion
				\item Turbine Pelton
				\item Force due à un jet atteignant un plan inclin\'e
				\item Effet b\'elier d'un fluide
				\item \'equations de Saint-Venant
				\item Portance de Kutta-Joukowski
			\end{itemize}	
		\item \'Electrodynamique:
			\begin{itemize}		
				\item Champs \'electromagn\'etiques d'une sphère de charges en rotation
				\item \'equation de Sellmeier
				\item Brehmstrahlung (commenc\'e mais pas fini)
				\item Principe de l'interf\'eromètre de Michelson
				\item Couple de rotor d'un \'electroaimant
			\end{itemize}
		\item \'Electrocin\'etique:
			\begin{itemize}		
				\item Convertisseur photo\'electrique
				\item Jonctions PN
			\end{itemize}
		\item Astronomie:
			\begin{itemize}	
				\item Formule de MacCullagh's 
				\item Calcul indirect de l'aplattissement des corps c\'elestes
				\item Verrouillage synchrone des satellites en r\'evolution	
				\item Aplatissement des sph\'eroïdes
				\item Sphère d'influence
				\item \'Echappement plan\'etaire
			\end{itemize}		
		\item Relativit\'e G\'en\'erale:
			\begin{itemize}
				\item Volume r\'eel d'un objet en relativit\'e g\'en\'erale
				\item D\'erivation du rayon d'Einstein
				\item Th\'eorème g\'en\'eral de Birkhoff
				\item Global Positioning System (GPS)
				\item M\'etrique et solution de Kerr	
			\end{itemize}
		\item Cosmologie:
			\begin{itemize}
				\item Finir le texte sur les Univers de Friedman bas\'es sur la Relativit\'e G\'en\'erale (d\'ebut\'e mais pas fini)
				\item D\'erivation du rayon d'Einstein
				\item Temp\'erature th\'eorique du fond diffus cosmologique
				\item Th\'eorie de Kaluza-Klein
			\end{itemize}
		\item Physique Nucl\'eaire:
			\begin{itemize}
				\item Diffusion Rayleigh
				\item Transport de Neutrons
				\item Spin (h\'elicit\'e) et relation de polarit\'e du photon
				\item Les in\'egalit\'es de Bell
				\item L'in\'egalit\'e de Kennard des incertitudes de Heisenberg
				\item Calcul d\'etaill\'e du d\'eplacement Lamb
				\item Exp\'erience de Davisson-Germer
				\item Formalisme du paradoxe EPR
			\end{itemize}
		\item Physique Quantique Ondulatoire:
			\begin{itemize}
				\item Temps de l'effet tunnel quantique
			\end{itemize}
		\item Physique Quantique Relativiste:
			\begin{itemize}
				\item Parit\'e, conjugaison de charge et inversion temporelle (CPT)
			\end{itemize}
		\item Chimie Quantique:
			\begin{itemize}
				\item Th\'eorie de la r\'epulsion de la paire d'\'electrons de Valence
				\item D\'erivation de l'\'equation de Sackur-Tetrode
			\end{itemize}
		\item M\'ethodes num\'eriques: 
			\begin{itemize}
				\item Problème d'optimisation univari\'ee par la m\'ethode de substitution				
				\item \'echantillonnage par acceptation/rejet
				\item \'echantillonnage de Gibbs
				\item Indicateur de coh\'erence de Cronbach
				\item Analyse Discriminante Lin\'eaire (ADL)
				\item Analyse Discriminant Quadratique (ADQ)
				\item Positionnement Multidimensionnel (MDS)
				\item Modèle Lin\'eaire Mixte (MLM)
				\item D\'ecalage moyen
				\item Analyse Factorielle (AF)
				\item Analyse factorielle des Correspondances (AFC)
				\item M\'ethode d'optimisation GRG (gradient r\'eduit g\'en\'eralis\'e)
				\item Informations mutuelles normalis\'ees
				\item Machines à vecteurs de support (MVS)
				\item D\'etection d'Interactions Automatiques par le Chi-2 (CHAID)
				\item Analyse s\'emantique latente(LSA)
				\item \'echantillonnage pr\'ef\'erentiel 
				\item \'echantillonnage stratifi\'e
				\item Monte Carlo avec variable de contrôle
				\item Classificateur Bay\'esien Naif binomial et Gaussien
				\item R\'eduction dimensionnelle par la corr\'elation
				\item Algorithmes ID3, PRISM, AQ, CN2 et C4.5
				\item Analyse procust\'eenne
				\item Analyses en Composantes Ind\'ependantes (ICA)
				\item Modèle uplift
			\end{itemize}
		\item Informatique Quantique: 
			\begin{itemize}
				\item Impossibilit\'e du clonage quantique
			\end{itemize}
		\item Cryptographie: 
			\begin{itemize}
				\item Courbes elliptiques
			\end{itemize}	
		\item Ing\'enierie:
			\begin{itemize}
				\item Domaines de Box
				\item Plans de criblage d\'efinitifs (DSD)
				\item Plans Split-plots
				\item Plans Composites Centraux
				\item Plans Cubiques Faces Centr\'ees
				\item Modèle de Survie de Cox (modèle à hasard proportionnel de Cox)
				\item Mod\'elisation par \'equations structurelles				
				\item Tests de veillissements acc\'el\'er\'es
				\item Micro\'electronique (jonctions NPN/PNP, diodes, amplificateurs)
				\item \'Equations des t\'el\'egraphistes
				\item Th\'eorème de pouss\'ee de Kutta-Jukowski
			\end{itemize} 
		\item Th\'eorie des jeux et de la d\'ecision: 
			\begin{itemize}
				\item Coalition
				\item Valeur de Shapley
				\item Critère de Kelly 
			\end{itemize}
		\item \'Economie: 
			\begin{itemize}
				\item Taux de rendement continu
				\item Courbe des taux z\'ero-coupons
				\item \'equivalence du taux d'une obligation avec un bon du tr\'esor
				\item Taux Spot et taux Forward
				\item Ajustement du beta d'un portefeuille avec des Futures
				\item \'egalit\'e du prix de Cox-Ingersoll des Future/Forward
				\item Solution de l'EDP de Black \& Scholes
				\item Duration de Macaulay
				\item Duration Modifi\'ee
				\item Taux de Retour Interne Modifi\'e (MIRR)
				\item Couverture de portefeuilles par options
				\begin{itemize}
					\item Option de Vente/Achat de protection (protective Put/Call)
					\item \'ecart d'option d'achat à la hausse (bull Spread/Call)
					\item \'ecart d'option d'achat à la vente (bear Spread/Call)
					\item Papillon (butterfly)
					\item Op\'eration li\'ee (straddle)
					\item Position combin\'ee (strangle)
					\item Collar
					\item \'Ecart de Box (Box spread)
					\item \'Ecart de Calendrier (calendar spreads)
					\item M\'ethodes d'allocation de portefeuilles
					\begin{itemize}
						\item Portefeuille pond\'er\'e optimal pour un risque \'equilibr\'e
						\item Portefeuille pond\'er\'e optimal pour le suivi des erreurs
						\item Portefeuille pond\'er\'e optimal de Sharp
						\item Portefeuille pond\'er\'e optimal avec une diversification maximale
						\item Portefeuille de Treynor-Black optimal pond\'er\'e en fonction du benchmark
					\end{itemize}
				\end{itemize}
				\item Grecques pour les arbres binomiaux
				\item Swaps
				\item Formule de Margrabe
				\item Fomrule (approxmative) de Kirk
				\item Risque de d\'efaut de cr\'edit (bas\'e sur la notation de Standard \& Poor)
				\item CreditRisk+
				\item VaR Equity Coverage
				\item Perte conditionnelle de la VaR (CVaR)
				\item Approche de KMV-Merton pour le mesure de probabilit\'e de d\'efaut
				\item Distance jusqu'à d\'efaut
				\item \'Equation de Fokker-Planck
				\item Processus stochastiques ARCH-GARCH
				\item Modèles autor\'egressifs vectoriels pour s\'eries temporelles multivari\'ees
				\item Test de Granger pour la causalit\'e de deux s\'eries temporelles
				\item Filtres de Karman
				\item Modèle de pricing d'options de Heston
				\item Pricing d'options de type Spread
			\end{itemize}	
		\item Management Quantitatif: 
			\begin{itemize}
				\item Algorithme de Gale-Shapley
				\item Modèles Pareto/NBD, BG/NBD et BG/BB de la valeur vie client
				\item Problème du vendeur de journaux
				\item Effet coup de fouet
				\item Paradoxe de Condorcet	
				\item M\'ethode CRAFT (Computerized Relative Allocation of Facilities Technique)
				\item Options r\'eelles
				\item Capital diff\'er\'e en cas de survie (assurance vie)
				\item Indice de volatilit\'e CBOE (commenc\'e mais pas termin\'e)
				\item D\'ecès diff\'er\'e temporaire (assurance vie)
			\end{itemize}
		\item Biographies:
			\begin{itemize}
				\item Ajout de la religion de chaque individu pr\'esent\'e dans la biographie à la demande d'un lecteur
			\end{itemize}
	\end{itemize}
	De plus, nous souhaitons cr\'eer une version r\'esum\'ee de ce PDF (sans les d\'eveloppements d\'etaill\'es mais uniquement avec les r\'esultats principaux) et une pr\'esentation avec diapositives (r\'ealis\'ees avec Beamer) uniquement avec les r\'esultats, textes et images / graphiques principaux pouvant être utilis\'es par n'importe quel enseignant gratuitement dans le monde entier.
	
	Rappelez-vous que les sources \LaTeX{} de ce livre peuvent être obtenues en fonction de votre don sur Patreon, Paypal, Tipee ou via votre participation à la traduction de ce livre dans une autre langue.
	
	Comme chaque produit robuste a un cycle de vie, le cycle de vie commence lorsqu'un produit est mis à disposition du grand public et se termine lorsqu'il n'est plus pris en charge. Connaître les dates cl\'es de ce cycle de vie vous aide à prendre des d\'ecisions \'eclair\'ees sur la date de mise à niveau. Ce livre a le cycle de vie suivant: une nouvelle version majeure ou mineure est publi\'ee chaque fois qu'un seuil donn\'e de dons est atteint et peut être t\'el\'echarg\'e en cliquant sur le bouton suivant (PDF de $410$ m\'egaoctets ...):
	\begin{center}
		\href{http://www.sciences.ch/dwnldbl/telecharger.php3}{\includegraphics[scale=0.6]{img/books/download.jpg}}
	\end{center}
	ou si ce lien ne fonctionne pas, une copie du fichier PDF est disponible sur les Archives Internet:
	\begin{center}
		\includegraphics[scale=0.1]{img/internet_archive.jpg}
	\end{center}
	\begin{center}
	\href{https://archive.org/details/OperaMagistris}{https://archive.org/details/OperaMagistris}
	\end{center}
	
	Pour citer ce livre (non ce n'est pas une blague... la version anglaise sera probablement finie d'être traduite en français en 2040 selon nos estimations actuelles):
	\begin{quote}
	\noindent @book\{OperaMagistris2040v4, \\
		  title =        \{Opera Magistris - \'El\'ements de Math\'ematiques Appliqu\'ees pour Ing\'enieurs\}, \\
		  year =         \{2040\}, \\
		  keywords =     \{science, physique, maths, ing\'enierie, finance, management\}, \\
		  isbn =          \{978239909327\},\\
	\}
	\end{quote}
	
	
	\chapter{Remerciements}
	Les concepts de ce livre ont été développés et renforcés par de nombreuses personnes. J'ai beaucoup bénéficié de mes interactions régulières avec des étudiants (universités, écoles d'ingénieurs) et les cadres de tous horizons, y compris les PDG, directeurs financiers, chefs de projets de nombreuses entreprises à travers le monde, sessions d'enseignement, développement de programmes spécifiques, conseil et même conversations informelles . Je leur suis reconnaissant d'avoir partagé leur sagesse tout en supportant mes nombreuses questions dans des échanges calmes et constructifs et donc d'inspiré de nombreuses idées de le livre.

	Ce livre et son site Internet d'accompagnement n'auraient pas été possible sans le précieux soutien des personnes mentionnées ci-dessous. Ils trouvent ici l'expression de ma gratitude la plus sincère (et à coup sûr si quelques erreurs subsistent dans ce livre c'est évidemment leur faute...):
	
	\begin{itemize}		
		\item \textbf{HARMEL Léon }(†2012), Ingénieur électricien diplômé avec une spécialisation en électronique et automatisation, responsable dans le laboratoire de recherche physique à l'AFIC à Charleroi (BEL), pour la rédection détaillée de nombreux développements disponibles dans les sections de Physique Quantique Corpusculaire, Physique Quantique Ondulatoire, Physique Quantique des Champs, Calcul Spinoriel et Relativité Générale.
		
		\item \textbf{KOLANI Daname}, Doctorant en Gestion de Portefeuilles à UM5-Rabat (MAR) pour son aide à la rédaction des livres d'accompagnement R \& MATLAB™, pour sa relecture et compléments de la sous-section de ce livre sur la Fouille de Données (Data Mining) et l'Apprentissage Machine (Machine Learning) et la conversion de quelques centaines de pages du site Internet francophone original (Sciences.ch) en ce livre \LaTeX.
		
		\item \textbf{LEGRAND Mathias}, Ph.D. de l'École Centrale de Nantes (FRA) pour son aide à la conversion des premières $550$ pages du site Internet francophone original (Sciences.ch) en ce livre \LaTeX.
				
		\item \textbf{RICCHIUTO Ruben}, Ingénieur HES en Physique (B.Sc.) de l'Ecole d'Ingénieurs de Genève et mathématicien de l'Université de Genève pour son aide précieuse en physique des plasmas, électromagnétisme, physique quantique, statistiques, topologie, chimie quantique, théorie des fractales, analyse fonctionnelle et de nombreux autres domaines touchant les mathématiques pures et l'informatique théorique.
		
		\item Habitués des forums \href{http://www.les-mathematiques.net}{{\color{blue} Les Mathematiques.net}} et \href{http://www.futura-sciences.com}{{\color{blue} Futura-Sciences.com}}, pour leur précieuse aide dans de nombreux domaines de la mathématique et de la physique. Les débats et discussions qui y ont lieu permettent de constamment améliorer l'aspect pédagogique et accessible du présent site.
		
		\item  Les sites Internet \href{http://www.wikipedia.com}{{\color{blue} Wikipédia}} et \href{http://www.planetmath.com}{{\color{blue} PlanetMath}} à qui je suis redevable de nombreux emprunts presque au mot par mot (et c'est réciproque...). 
		
		\item Les professeurs Gabriel Nagy de la Michigan State University, Liam Revell de la Massachusetts Boston University, Jorge Cham (de PhD Comics) et l'équipe OpenStax qui ont reçu \underline{et} lu mes multiples e-mails de requêtes sur les crédits/droits d'images et de textes mais n'ont jamais daigné répondre (pas tout à fait de bons exemples de ce que devrait être la "collaboration et l'esprit scientifique"...)
	\end{itemize}
	
	Et encore merci à tous les internautes, webmasters et professeurs anonymes pour leurs sites et documents de qualité mis à disposition gratuitement et anonymement sur la toile et aux intervenants réguliers des divers forum existant sur Internet. J'ai parfois récupéré mot à mot leurs explications et réponses qui ne nécessitaient ni compléments ni corrections.

	Inutile de dire qu'il ne faut pas supposer que ces personnes soient en accord total avec les vues scientifiquese exprimées sur dans présent livre; et elles ne sont nullement responsables des erreurs ou des obscurités qui pourraient s'y trouver par erreur.

	Pour terminer, je souhaiterais remercier ici les quelques collègues et clients qui ont bien voulu me faire part de leurs remarques pour améliorer le contenu de ce livre. Il est cependant certain qu'il est encore perfectible sur de nombreux points.
	
	Je voudrais enfin remercier tout particulièrement ma famille pour leur soutien continu et mes amis pour leur patience car l'écriture de ce livre absorbe tout mon temps depuis plus de dix ans, mais je voudrais envoyer un remerciement spécial à mon père et à ma mère! Merci aussi à ma conjointe d'être toujours là pour prendre soin de moi quand j'oublie de le faire...
	
	Pour tout retour ou commentaire/critique public, vous pouvez utiliser le livre d'or associé à ce PDF (pour les questions veuillez s'il vous plaît utiliser le forum!):
	\begin{center}
	\url{http://www.sciences.ch/htmlfr/guestbook.php}
	\end{center}
	ou si vous voulez me communiquer un retour ou commentaire/critique en privé, vous pouvez me contacter par {\href{mailto:info@sciences.ch}{{\color{blue}email}}}.
	
  \chapter{Introduction}
	%To add line numbers on all paragraphes each 5 lines if necessary	
	%\linenumbers
	
	\minitoc
	\pagebreak
		%to make section start on odd page
	\newpage
	\thispagestyle{empty}
	\mbox{}
	\parpic[l][t]{%
	  \begin{minipage}{30mm}
	    \fbox{\includegraphics[width=80px,height=100px]{img/einstein.eps}}
	  \end{minipage}
	}		
	Ce livre dont la première \'edition a \'et\'e publi\'ee en 2001 est conçu pour que les connaissances requises pour le lire soient aussi simples que possible. Il n'est pas n\'ecessaire d'avoir un doctorat pour le consulter, il suffit de savoir raisonner, penser de façon critique, observer et avoir le temps...
	\begin{flushright}
	\textit{"La simplicit\'e est le sceau de la v\'erit\'e et elle rayonne de beaut\'e"} \\
	 Albert EINSTEIN
	\end{flushright}
	
	\section{Avant-Propos}
	\lettrine[lines=4]{\color{BrickRed}A}ucune entreprise humaine n'a eu plus d'impact que la Science\footnote{Du latin \textit{scientia} "connaissance, savoir, expertise". Lui-même venant de \textit{sciens} (g\'enitif scientis) qui signifie "intelligent, talentueux", participe pr\'esent de \textit{scire} qui signifie "connaître" vient probablement de "s\'eparer une chose d'une autre, distinguer" li\'e à \textit{scindere} "couper, diviser".} sur nos vies et notre conception de notre Monde et de nous-mêmes. Ses th\'eories, conquêtes et r\'esultats sont tout autour de nous dans le quotidien de la majorit\'e des habitants de cette planète.

	Omnipr\'esents dans l'industrie (a\'erospatiale, imagerie, cryptographie, transport, chimie, algorithmique, etc.) ou dans les services (banque, fintech, assurance, ressources humaines, projets, logistique, architecture, communication, etc.), les math\'ematiques appliqu\'ees apparaîssent \'egalement dans d'autres domaines: enquêtes, mod\'elisation des risques, protection des donn\'ees, politique, etc. Les math\'ematiques appliqu\'ees (parfois aussi appel\'ees "ing\'enierie math\'ematique") influencent nos vies (t\'el\'ecommunications, transport, m\'edecine, m\'et\'eorologie, musique, gestion de projet) et contribuent à la r\'esolution des enjeux d'actualit\'e: \'energie, sant\'e, environnement, climat, optimisation, d\'eveloppement durable, etc. bien plus que toute technique ou m\'ethodologie de soft skills! Leur grand succès est leur fabuleuse dispersion dans le monde r\'eel et leur int\'egration croissante dans toutes les activit\'es d'intelligence humaine et artificielle qui n\'ecessitent transparence et l'\'evitement de biais cognitifs. Nous allons donc à une situation où les math\'ematiciens et les ing\'enieurs n'auront plus le monopole des math\'ematiques, mais où presque tous les postes de "cols blancs" devront faire des math\'ematiques avanc\'ees.

	En tant qu'ancien \'etudiant dans le domaine de l'ing\'enierie, j'ai souvent regrett\'e l'absence d'un seul livre assez complet, d\'etaill\'e (sans aller à l'extrême ...) et \'educatif si possible gratuit (!) et portable (êtant personnellement fan des eBooks ...) contenant au moins une id\'ee non exhaustive du programme g\'en\'eral de math\'ematiques appliqu\'ees dans les \'ecoles d'ing\'enieurs avec un aperçu de ce qui est utilis\'e dans les entreprises avec des preuves plus intuitives que rigoureuses, mais avec suffisamment de d\'etails pour \'eviter tout effort inutile au lecteur. Aussi un livre qui n'exige pas que le lecteur adopte à chaque fois une nouvelle notation ou une terminologie sp\'ecifique à l'auteur - quand il n'est pas carr\'ement n\'ecessaire parfois de changer la lecture dans une langue \'etrangère ... - et où n'importe qui peut sugg\'erer des am\'eliorations ou des ajouts (à travers un forum, uen livre d'or ou un simple courriel).

	J'ai aussi \'et\'e frustr\'e pendant mes \'etudes d'avoir souvent à avaler des «formules» ou des «lois» cens\'ees être (et à tort) non prouvables ou trop compliqu\'ees comme le disaient mes professeurs ou même d'être d\'eçu par des livres d'auteurs de renom (où les d\'eveloppements sont laiss\'es au lecteur ou comme exercice et où aucune application r\'eelle est mentionn\'e ...). Dans ce pr\'esent livre pr\'edomine la volont\'e de ne jamais confondre le lecteur avec des phrases vides comme "il est \'evident que ...", "il est facile de prouver que ...", "nous laissons cela au lecteur comme un exercice ... ", puisque tous les d\'eveloppements sont pr\'esent\'es en d\'etail. Mais je ne suis pas un puriste des maths! Je n'ai qu'une ambition: expliquer le plus facilement possible.
	
	Bien que je doive admettre que certaines relations math\'ematiques pr\'esent\'ees dans le cursus des \'ecoles d'ing\'enieurs ne peuvent être r\'ealis\'ees faute de temps dans le programme officiel ou à cause de la taille limite d'un livre papier, je ne peux accepter qu'un enseignant ou un auteur en racontent à ses \'elèves (respectivement, ses lecteurs) que certaines lois et relations ne sont pas prouvables (parce que la plupart du temps ce n'est pas vrai!) ou que telle ou telle preuve est trop compliqu\'ee sans donner de r\'ef\'erence (où l'\'etudiant pourrait trouver l'information n\'ecessaire pour satisfaire sa curiosit\'e) ou au moins une preuve simplifi\'ee mais satisfaisante.
	
	De plus, je pense qu'il est totalement archaïque aujourd'hui que certains enseignants continuent à demander à leurs \'elèves de prendre une quantit\'e massive de notes pendant les cours. Il serait beaucoup plus favorable et optimal de distribuer un document de cours contenant tous les d\'etails afin de pouvoir se concentrer sur les points essentiels avec les \'elèves, c'est-à-dire les explications orales, les interpr\'etations, la compr\'ehension, le raisonnement et la pratique. \'evidemment en fournissant un support de cours complet, certains \'elèves seront brillants par leur absence mais ... c'est probablement mieux ainsi! Ainsi, ceux qui sont passionn\'es peuvent approfondir des sujets à la maison ou à la bibliothèque universitaire, les non-int\'eress\'es et je m'en foutistes feront ce qu'ils ont à faire (...) et le reste (\'etudiants en difficult\'e mais travailleurs) suivra le cours dispens\'e par l'enseignant pour profiter de poser des questions plutôt que de suivre un cours sans r\'efl\'echir à copier un tableau noir dans une salle avec un effectif d'\'elève surnum\'eaire.
	
	Inspir\'e d'un modèle d'apprentissage d'un \'erudit am\'ericain, dont j'ai oubli\'e le nom (...), ce livre propose et impose au lecteur les caract\'eristiques suivantes: d\'ecouvrir, m\'emoriser, citer, int\'egrer, expliquer, reformuler, d\'eduire, s\'electionner, utiliser, d\'ecomposer, comparer, interpr\'eter, juger, argumenter, mod\'eliser, d\'evelopper, cr\'eer, rechercher, raisonner, d\'evelopper dans un  enseignement progressif limpide pour d\'evelopper les comp\'etences analytiques et l'ouverture d'esprit.

	Donc, dans mon esprit, ce livre non exhaustif (et ses PDF compagnons associ\'es) doit être un substitut, gratuit pour tous les \'etudiants et employ\'es à travers le monde, à de nombreuses r\'ef\'erences et lacunes du système scolaire, permettant à tout \'etudiant curieux de ne pas être frustr\'e pendant de nombreuses ann\'ees pendant son cursus acad\'emique. Sinon, la science de l'ing\'enieur pourrait avoir l'aspect d'une science fig\'ee, mis à part les d\'eveloppements scientifiques et techniques, une accumulation h\'et\'eroclite de connaissances et surtout de formules qui la ferait consid\'erer comme un sous-produit insipide des math\'ematiques et qui amène les entreprises et les gouvernements à beaucoup de faux r\'esultats et de mauvaises d\'ecisions ...
	
	Ce livre a \'egalement \'et\'e conçu pour r\'epondre aux besoins des dirigeants, aussi bien financiers que non financiers. Tout dirigeant qui veut approfondir et comprendre les fondamentaux de la finance strat\'egique, du marketing strat\'egique ou de l'ing\'enierie de gestion de projet, et de l'ing\'enierie de la gestion d'entreprise/administrations et des questions li\'ees à la chaîne d'approvisionnement b\'en\'eficiera de la lecture de ce livre.
	
	Ce livre a \'egalement pour but de d\'ecrire et d'expliquer comment notre Univers et notre Monde (\'egalement probablement d'autres "mondes" de notre Univers) fonctionne d'une manière beaucoup plus pr\'ecise, plus complète et plus d\'etaill\'ee que n'importe quel livre "Saint". Il donne des modèles et des m\'ethodes de quantification pour l'origine des espèces, des galaxies, des planètes, des ph\'enomènes quantiques, des mouvements physiques, de la physique stellaire, des \'ev\'enements extrêmes observables et aussi des \'ev\'enements extrêmement rares et explique les strat\'egies sociales et technologiques de manière prouvable que tout tout à chacun peut v\'erifier par lui-même et ce en exposant chaque fois les hypothèses que toute entit\'e raisonnable devrait prendre a priori en charge dans l'\'etat actuel de nos connaissances!
	
	De toute \'evidence, les math\'ematiques appliqu\'ees sont un sujet si vast qu'un livre de cette envergure ne peut qu'aborder la base. Les lecteurs sont certainement encourag\'es à aller au-delà (voir la bibliographie à la fin du livre).

	Maintenant, ceux qui voient les Math\'ematiques Appliqu\'ees seulement comme un outil (ce qu'elles sont aussi), ou comme l'ennemi des croyances religieuses, ou comme une \'ecole de campagne ennuyeuse, sont l\'egion. Cependant, il est peut-être utile de rappeler que, comme l'a dit Galil\'ee, «\textit{le livre de la nature est \'ecrit dans le langage des math\'ematiques}» (sans vouloir faire de scientisme!). Quand vous allez en Chine, vous apprenez le chinois. Quand vous voulez aller à l'Univers, vous apprenez la math\'ematique parce que cette dernière est la langue de l'Univers et de la Nature perceptible. C'est pourquoi la math\'ematique est fondamentale et si incroyable car elle s'applique à tout l'Univers et ce à l'\'etat de nos connaissances actuelles, aussi à travers le temps. C'est dans cet esprit que ce livre traite des math\'ematiques appliqu\'ees pour les \'etudiants en sciences naturelles, terrestres et de la vie, ainsi que pour tous ceux qui ont une occupation li\'ee aux diff\'erents sujets, y compris la philosophie ou pour toute personne int\'eress\'ee par la science dans la vie de tous les jours.

	Le choix d'\'etudier l'ing\'enierie dans ce livre comme une branche des Math\'ematiques Appliqu\'ees vient du fait que les diff\'erences entre tous les domaines de la physique (anciennement connu comme «philosophie naturelle») et les math\'ematiques sont si difficilement distinguable que la m\'edaille Fields (la plus haute distinction aujourd'hui dans le domaine des math\'ematiques) a \'et\'e d\'ecern\'e en 1990 au physicien Edward Witten, qui a utilis\'e des id\'ees de la physique pour prouver un th\'eorème math\'ematique. Cette tendance n'est certainement pas fortuite, car on peut observer que toute la science, puisqu'elle cherche à approfondir la compr\'ehension du sujet qu'elle \'etudie, finit toujours ses essais et erreurs dans le domaines math\'ematiques pures (le chemin absolu par excellence!). Ainsi, nous pouvons pr\'edire dans un futur lointain, la convergence de toutes les sciences (pures, exactes ou sociales) vers les math\'ematiques pour les techniques de mod\'elisation (voir par exemple le PDF français "\textit{L'explosion des math\'ematiques}" disponible sur la page de t\'el\'echargement du site Web compagnon).

	Il peut parfois nous sembler difficile (en raison de la peur irrationnelle et injustifi\'ee des sciences pures d'une partie significative de la population) de transmettre le sentiment de la beaut\'e math\'ematique de la Nature, son harmonie la plus profonde et la m\'ecanique bien huil\'ee de l'Univers à ceux qui ne connaissent que les bases de l'algèbre. Le physicien Richard Feynman a parl\'e d'une une fois de «deux cultures»: les gens qui ont et ceux qui n'ont pas une compr\'ehension suffisante des math\'ematiques pour appr\'ecier la structure scientifique de la Nature. Il est dommage que les math\'ematiques soient n\'ecessaires pour comprendre profond\'ement la Nature et qu'elles aient \'egalement une mauvaise r\'eputation. Pour l'anecdote, on pr\'etend qu'un roi qui a demand\'e à Euclide de lui enseigner la g\'eom\'etrie se plaignait de la difficult\'e de cette dernière. Euclide r\'epondit: "Il n'y a pas de voie royale". Les physiciens et les math\'ematiciens ne peuvent pas se convertir à une langue diff\'erente. Si vous voulez en apprendre davantage sur la Nature, pour en appr\'ecier la vraie valeur, vous devez comprendre sa langue. La Nature n'est r\'ev\'el\'ee que sous cette forme et nous ne pouvons être pr\'etentieux au point de lui demander de changer ce fait.
	
	De même, aucune discussion intellectuelle ne vous permettra de communiquer avec une personne sourde ce que vous ressentez en \'ecoutant de la musique. De même, toute discussion sur le monde reste vaine pour  transmettre une compr\'ehension intime de la Nature de ceux de «l'autre culture». Les philosophes et les th\'eologiens peuvent essayer de vous donner des id\'ees qualitatives sur l'Univers. Le fait que la m\'ethode scientifique (au sens propre du terme) ne puisse pas convaincre le Monde de ses forces à travers son processus it\'eratif (le fait que la science est r\'e\'ecrite, r\'e\'ecrite et r\'e\'ecrite par am\'eliorations incr\'ementales comme dans la m\'ethodologie DMAIC Six Sigma\footnote{DMAIC (acronyme anglais pour D\'efinir, Mesurer, Analyser, Am\'eliorer et Contrôler) se r\'efère à un cycle d'am\'elioration pilot\'e par les donn\'ees utilis\'e pour am\'eliorer, optimiser et stabiliser les processus et la conception. Le cycle d'am\'elioration DMAIC est l'outil de base utilis\'e pour conduire les projets Six Sigma que nous \'etudierons en profondeur dans la section de G\'enie Industriel. Cependant, DMAIC n'est pas exclusif à Six Sigma et peut être utilis\'e comme cadre pour d'autres applications d'am\'elioration.}) est peut-être le fait de l'horizon limit\'e de certaines personnes qui imaginent que l'humain, ou un autre concept intuitif, sentimental ou arbitraire est le centre de l'Univers (principe anthropocentrique).
	\begin{figure}[H]
		\centering
		\includegraphics[width=1.0\textwidth]{img/intro/scientific_method.jpg}
		\caption[Processus cyclique de la méthode scientifique]{Processus cyclique de la méthode scientifique (source: ?)}
	\end{figure}
	
	\begin{tcolorbox}[title=Remarque,colframe=black,arc=10pt]
	Si vous êtes un scientifique honnête, la grande majorité de vos idées, même de bonnes idées, seront exclues, non pas par de nouvelles expériences mais déjà par incohérence avec d'anciennes expériences. C'est ce qui rend vraiment la science très différente, et ce qui nous donne une notion interne du bien et du mal avant de nouvelles expériences. Donc, contrairement à ce que certains profanes sceptiques pensent, l'idée que vous pouvez simplement inventer de la merde est fausse!
	\end{tcolorbox}
	
	\subsection{Motivations et objectifs}
	Bien sûr, afin de partager cette connaissance math\'ematique, il peut sembler paradoxal d'augmenter, avec notre ouvrage, la longue liste de livres d\'ejà disponibles dans les bibliothèques, dans le commerce et sur Internet. N\'eanmoins, je dois être capable de pr\'esenter des arguments qui justifient la cr\'eation d'un tel ouvrage (et de son site Internet compagnon) par rapport à des livres tels que les Feynman, Landau ou Bourbaki et Wikipedia / Wolfram eux-mêmes ou Khan Academy ou OpenStax. Alors qu'est-ce que je pense que je peux ajouter à une telle richesse de mat\'eriels?
	\begin{enumerate}
		\item Le grand plaisir que nous prenons à \'ecrire ce livre ("garder la main" et am\'eliorer nos comp\'etences) et à avoir un compendium d\'etaill\'e de haute qualit\'e des outils math\'ematiques pour nos clients et nos \'etudiants (et aussi tous ceux du Monde entier) et ce gratuitement!

		\item La passion pour le partage de connaissances gratuitement (bataille contre la "folie des droits d'auteur" (RIP Aaron Swartz!)) et sans frontières avec un outil de haute qualit\'e comme \LaTeX{} (à l'oppos\'e de Wikipedia qui m\'elange \LaTeX{} et le contenu horrible et honteux de Khan Academy \footnote{OpenStax a de bons PDF de premier cycle - en particulier les exemples dans leurs livres - mais il y a entre $40$-$60\%$ de d\'emonstrations math\'ematiques manquantes dans leurs PDF et la table des matières et l'index de leur PDF ne sont à ce jour pas interactifs ... et problème majeur ...: le contenu est limit\'e uniquement aux sujets de premier cycle}).
		
		\item Soutenir l'\'education scientifique gratuite, la pens\'ee critique et la compr\'ehension fond\'ee sur les preuves. En outre, il est clair qu'il existe un app\'etit insatiable pour certains gens à comprendre les choses (même si cela semble être actuellement une minorit\'e) et ce livre a \'et\'e \'ecrit à cette fin.
		
		\item Nous voulons pr\'esenter les Math\'ematiques Appliqu\'ees d'une manière agr\'eable et facile à apprendre (typiquement à l'oppos\'e des livres des $9$ livres de Landau), parce que nous pensons que la maîtrise des Math\'ematiques Appliqu\'ees changent la façon dont nous comprenons l'Univers et am\'eliore la compr\'ehension et tol\'erance mutuelle entre humains et nos interactions avec la Nature.
		
		\item Cet ouvrage a \'et\'e \'ecrit avant (ann\'ee 2001) que la version française de Wikip\'edia ait un contenu math\'ematique satisfaisant et longtemps avant que Khan Academy ou OpenStax n'existent.

		\item Les possibilit\'e d'effectuer rapidement des mises à jour ou corrections critiques  (à l'oppos\'e des vid\'eos de Khan Academy) ainsi que de  collaboration que permettent un e-book gratuit (avec en plus les opportunit\'es qu'offrent les outils de recherche et d'annotation des lecteurs de PDF).

		\item Le contenu peut facilement être adapt\'e en fonction des demandes / commentaires des lecteurs et de nos int\'erêts (à l'oppos\'e des vid\'eos de Khan Academy ou des livres de OpenStax ou Landau)!
		
		\item A l'oppos\'e des publications scientifiques (PRL ou autre) qui sont discutables car ne donnent pas de preuves d\'etaill\'ees et tournent parfois dans une boucle infinie de r\'ef\'erences bibliographiques, nous fournissons toujours les d\'emonstration les plus d\'etaill\'ees possible.
		
		\item L'accès aux sources \LaTeX{} est disponible au Monde entier gratuit, donc personne (\'elève ou prof) n'a besoin de recr\'eer la roue et perdre des centaines ou des milliers d'heures de r\'edaction au lieu de faire de l'innovation (à l'oppos\'e des livres de Landau)!

		\item Une pr\'esentation rigoureuse avec des preuves d\'etaill\'ees simplifi\'ees de tous les concepts pr\'esent\'es (à l'oppos\'e de Wikipedia, Khan Academy et OpenStax qui se concentrent uniquement sur les d\'emonstrations math\'ematiques des concepts de premier cycle et souvent avec des d\'emonstrations très lacunaires).

		\item La pr\'esentation de nombreux outils math\'ematiques avanc\'es et d\'etaill\'es utilis\'es dans les affaires et la R\&D en gardant à l'esprit que le langage math\'ematique semble \'eternel et d'être l'un des seuls d\'enominateurs culturel commun entre tous les pays de notre Planète.

		\item L'occasion pour les \'etudiants et les enseignants de pouvoir r\'eutiliser un contenu par copier/coller (à l'oppos\'e de Khan Academy ou des livres de Landau Books).

		\item Proposer une notation constante (à l'oppos\'e de Wikip\'edia, Khan Academy et OpenStax) tout au long du livre pour les op\'erateurs math\'ematiques, un langage clair sur tous les sujets (critère 3.C.: clair, complet et concis) et se concentrer sur les bases pour faire un important travail p\'edagogique sur les sujets (à l'oppos\'e des livres de Landau).

		\item Compiler autant d'informations que possible sur les sciences pures et exactes dans un unique livre \'electronique (portable), homogène et rigoureux et de haute qualit\'e visuelle (mais en allant pas aussi loin que les livres de Landau).

		\item Distinguer de toutes les pseudo-v\'erit\'es, seulement les v\'erit\'es qui peuvent être prouv\'ees par la d\'emarche math\'ematique.

		\item B\'en\'eficier du d\'eveloppement des m\'ethodes d'enseignement qui utilisent Internet pour rechercher la solution de problèmes math\'ematiques.

		\item L'am\'elioration spectaculaire des logiciels de traduction automatique et de la puissance de calcul qui fera de ce livre, du moins on l'espère, une r\'ef\'erence internationale dans le domaine des sciences.
		
		\item Un PDF est meilleur qu'un site Internet car d'abord tous ceux qui utilisent Internet depuis 1990 savent que la grande majorit\'e des sites disparaissent après environ $10$ ans et deuxièmement il est bien connu que certains pays bloquent Wikip\'edia et autres sites de connaissances pour garder leur population dans l'ignorance (et bloquer un PDF qui peut être partag\'e dans un e-mail est beaucoup plus difficile).
		
		\item et ... parce que les Math\'ematiques Appliqu\'ees sont belles et surtout lorsqu'elles sont \'ecrites en \LaTeX{} et illustr\'ees (à l'oppos\'e des livres de Landau dont les illustrations sont assez anciennes et pauvres en couleurs).
\end{enumerate}

	Et aussi ... Je tends à penser que les r\'esultats de la recherche sont la propri\'et\'e de l'humanit\'e et devraient être accessibles gratuitement à tous ceux qui explorent les ph\'enomènes de la Nature. De cette façon, le travail de chacun b\'en\'eficie à tous, et c'est pour toute l'humanit\'e que nos connaissances se cumulent et c'est la tendance que permet Internet.

	Je ne cache pas que ma contribution est largement limit\'ee à ce jour à celle d'un collectionneur qui glane ses informations dans les oeuvres de maîtres ou de publications ou de pages web anonymes et qui complète et argumente les d\'eveloppements math\'ematiques et les am\'eliore quand c'est possible. Par cons\'equent, certains \'el\'ements de ce livre sont originaux et certains proviennent de la litt\'erature de r\'ef\'erence. Cependant la grande majorit\'e de ce que nous avons \'ecrit est une reformulation des r\'esultats pr\'esent\'es dans la vaste bibliothèque d'existences de quelques livres fantastiques (et rares). Pour ceux qui m'accuseraient de plagiat, ils devraient penser que les th\'eorèmes pr\'esent\'es dans la plupart des livres non libres et disponibles dans le commerce ont \'et\'e d\'ecouverts et \'ecrits par leurs pr\'ed\'ecesseurs des auteurs de ces mêmes livres et que leur contribution personnelle a \'et\'e, comme la mienne, de mettre toutes ces informations sous une forme claire et moderne quelques centaines d'ann\'ees plus tard. En outre, on peut douter que nous demandions à payer pour l'accès à une culture qui est certainement la seule vraiment valable et juste dans le Monde et où il n'y a pas de brevets ou de droits de propri\'et\'e intellectuelle.
	
	\begin{fquote}[Wilson Mizner]Si vous volez d'un auteur, c'est du plagiat; si vous volez de beaucoup d'auteurs, c'est de la recherche!
 	\end{fquote}
	
	Ce livre reflète \'egalement en grande partie mes propres limites intellectuelles relativement aux sciences dures et pures. Bien que j'essaie d'\'etudier autant de domaines scientifiques et math\'ematiques que possible, il est impossible de les maîtriser tous. Ce livre ne montre clairement que mes propres int\'erêts et exp\'eriences en tant que consultant et professeur, mais aussi mes forces et mes faiblesses. Je suis responsable de la s\'election des intrants et, bien sûr, en grande partie des erreurs et des imperfections possibles.

	Après avoir tent\'e un ordre de pr\'esentation strictement lin\'eaire du sujet de la Math\'ematique Appliqu\'ee, j'ai d\'ecid\'e d'organiser ce livre d'une manière plus p\'edagogique (th\'ematique) et toujours avec des exemples pratiques d'applications. Il est à mon avis très difficile de parler d'un sujet aussi vaste dans un ordre lin\'eaire dans une seule vie humaine, c'est-à-dire lorsque les concepts sont introduits un à un, parmi ceux d\'ejà connus (où chaque th\'eorie, op\'erateur, etc. n'apparaîtrait pas avant sa d\'efinition). Ainsi, avec ce choix th\'ematique, le lecteur pourra probablement se rendre compte de l'extrême complexit\'e du sujet.

	Les cons\'equences de ce choix sont les suivantes:
	\begin{enumerate}
		\item Parfois il faudra admettre certains concepts, quitte à les comprendre plus tard.
	
		\item Il sera probablement n\'ecessaire au lecteur de parcourir au moins deux fois dans le livre. En première lecture, il appr\'ehendra l'essentiel et à la deuxième lecture, il comprendrea les d\'etails (je f\'elicite dans tous les cas le lecteur d\'ebutant qui comprend toutes les subtilit\'es la première fois!).
	
		\item Vous devez accepter le fait que certains sujets (et très faible nombre) sont parfois r\'ep\'et\'es pour le confort de lecture et l'assimilation du cerveau et qu'il existe de nombreuses r\'ef\'erences crois\'ees et des remarques compl\'ementaires en pied de page.
	\end{enumerate}
	
	Certains savent que pour chaque th\'eorème et modèle math\'ematique, il existe presque toujours plusieurs approches pour la d\'emonstration math\'ematique. J'ai toujours essay\'e de choisir l'approche qui semblait lla plus simple (par exemple, en relativit\'e et en physique quantique, il y a le formalisme alg\'ebrique et matriciel). L'objectif \'etant d'arriver au même r\'esultat de toute façon. Dans certains cas int\'eressants, nous pr\'esentons même plusieurs approches d'une d\'emonstration car nous consid\'erons les diff\'erentes approches alors comme \'etant très "formatrices".
	
	Ce livre \'etant dans sa version brouillon, il manque forc\'ement des contrôles de convergence, de continuit\'e, de grammaire et autres ... (qui vont horrifier certains lecteurs et math\'ematiciens ...)! Cependant, j'ai \'evit\'e (ou, sinon, je l'indique) les approximations habituelles de la physique et l'utilisation de l'analyse dimensionnelle, en l'utilisant le moins possible. J'essaie aussi d'\'eviter autant que possible les sujets avec des outils math\'ematiques qui n'ont pas \'et\'e pr\'esent\'es auparavant et d\'emontr\'es rigoureusement (selon les critères de l'ing\'enieur!).
	
	Enfin, cette pr\'esentation des Math\'ematiques Appliqu\'ees, qui peut encore être am\'elior\'ee, n'est pas une r\'ef\'erence absolue et contient des erreurs probablement par endroits. Tout commentaire est donc la bienvenue par e-mail ou via le forum du site compagnon d\'ejà indiqu\'es plus haut. Je m'efforcerai, dans la mesure du possible, de corriger les faiblesses et d'apporter les changements n\'ecessaires le plus rapidement possible (cela prend quelques mois en g\'en\'eral).
	
	Cependant, alors que les math\'ematiques sont exactes et indiscutables, la physique th\'eorique (ses modèles) est toujours interpr\'et\'ee dans le vocabulaire commun (mais pas dans le vocabulaire math\'ematique) et ses conclusions sont toutes relatives d'un individu à l'autre. Je ne peux que conseiller, en lisant ce livre, de lire par vous-même et de ne pas subir d'influences ext\'erieures dans un premier temps. Vous devez avoir un esprit très (très) critique, ne rien prendre pour acquis et tout remettre en question sans h\'esitation. En outre, le mot cl\'e du bon scientifique devrait être: "Doute, doute, doute ... doute encore, et v\'erifie toujours.". Nous rappelons aussi que «rien de ce que nous pouvons voir, entendre, sentir, toucher ou goûter, n'est ce qu'il semble être», ne comptez donc pas sur votre exp\'erience quotidienne pour tirer des conclusions hâtives, soyez critique, cart\'esien, rationnel et rigoureux dans vos d\'eveloppements, raisonnements et conclusions et confrontez-les avec d'autres en admettant vos erreurs et remises en question avec sagesse!
	
	\begin{tcolorbox}[title=Remarque,colframe=black,arc=10pt]
	L'une des causes du syndrôme de l'anti-intellectualisme r\'eside probablement dans l'incapacit\'e des \'etablissements d'enseignement à inculquer la pens\'ee critique au lyc\'ee. En particulier les universit\'es, incapables de d\'evelopper la pens\'ee critique de leurs \'etudiants. Internet et son amalgame de vrai et de faux ont aussi leur part de responsabilit\'e ainsi que les mass-m\'edias qui manquent de rigueur dans la manière de citer leurs sources et leurs m\'ethodes d'investigations. En outre, certains experts amplifient le problème en parlant s'exprimant publiquement et de manière non quantifi\'ee et non sourc\'ee sur des questions en dehors de leur domaine d'expertise !!!!
	\end{tcolorbox}
	Je veux dire à ceux qui essaieraient de retrouver par eux-mêmes les r\'esultats de certains d\'eveloppements de ce livre, de ne pas s'inqui\'eter s'ils ne r\'eussissent pas ou s'ils doutent de leurs comp\'etences à cause du temps pass\'e à r\'esoudre une \'equation ou un problème donn\'e. Effectivement, quelques th\'eories qui semblent parfois \'evidents ou faciles aujourd'hui, ont parfois n\'ecessit\'e plusieurs semaines, mois, voire ann\'ees, pour être d\'evelopp\'es par des math\'ematiciens ou des physiciens de premier plan dans le pass\'e!
	
	J'ai aussi essay\'e de faire en sorte que ce livre soit agr\'eable à consulter et lire en y ajoutant de nombreuses illustrations en couleurs et \'egalement en choisissant un style d'\'ecriture peu formel.
	
	Enfin, j'ai choisi d'\'ecrire ce travail à la première personne du pluriel: «nous». En effet, la physique math\'ematique n'est pas une science qui a \'et\'e faite ou qui a \'evolu\'e à travers un travail individuel, mais avec une collaboration intense entre des personnes li\'ees par la même passion et d\'esir de la connaissance. Ainsi, en utilisant le «nous», je voudrais rendre hommage aux scientifiques d\'ec\'ed\'es, bien \'evidemment aux autres co-auteurs de ce livre mais aussi aux chercheurs contemporains et futurs pour le travail qu'ils vont effectuer afin d'approcher la v\'erit\'e et la sagesse.
	
	\begin{center}
	\includegraphics[scale=1]{img/humour/pure_math_vs_applied_math.jpg}
	\end{center}

	%to make section start on odd page
	\newpage
	\thispagestyle{empty}
	\mbox{}
	\section{M\'ethodes}	
	\lettrine[lines=4]{\color{BrickRed}L}a Science est l'ensemble de tous les efforts syst\'ematiques (observations scrupuleuses et hypothèses plausibles jusqu'à la preuve du contraire) pour acqu\'erir des connaissances sur notre environnement, pour les organiser et les synth\'etiser en lois et th\'eories testables, dont le but principal est d'expliquer les choses (et PAS le pourquoi!) souvent par une approche en quatre \'etapes:
	
	\begin{itemize}
		\item[$-$] Qu'est-ce que l'on a?
		\item[$-$] Où est-ce que l'on ira?
		\item[$-$] Quel est notre objectif?	
		\item[$-$] Est-ce que c'est conforme aux donn\'ees exp\'erimentales?
	\end{itemize}
	
	Les scientifiques doivent soumettre leurs id\'ees et r\'esultats à une v\'erification ind\'ependante et à la r\'eplication par leurs pairs ("\NewTerm{peer-review}\index{peer-review}"). Ils doivent abandonner ou modifier leurs conclusions lorsqu'ils sont confront\'es à des preuves exp\'erimentales et calculatoires plus complètes ou diff\'erentes. La cr\'edibilit\'e de la Science repose donc sur ce m\'ecanisme d'autocorrection et c'est ce qui fait encore qu'au XXIe siècle, la Science n'est peut-être pas LE meilleur outil (car nous ne savons pas ce qui existera dans le futur ...) mais elle a montré qu'elle était la meilleure m\'ethode d'investigation des mécanismes de la Nature et de l'Univers par rapport à toutes les autres m\'ethodes ou croyances existantes à ce jour. L'histoire de la science montre que ce système fonctionne depuis très longtemps et très bien par rapport à tous les autres puisqu'il permet d'\'eviter de nombreux biais cognitifs et culturels. Grâce à cela, dans chaque domaine, les progrès ont \'et\'e spectaculaires. Cependant, le système a parfois \'echou\'e (et la d\'etection de ses \'echecs est elle aussi une r\'eussite!) et doit \'egalement être corrig\'e avant que de petites d\'erives ne s'accumulent.

	Le problème majeur est et reste que les scientifiques sont des humains. Ils ont les imperfections de tous les humains, et surtout, la vanit\'e, la fiert\'e, la colère, la vanit\'e et les biais cognitifs. De nos jours, il arrive que beaucoup de personnes travaillant sur le même sujet pendant un temps donn\'e d\'eveloppent une foi commune et croient qu'elles d\'etiennent la v\'erit\'e ou qu'elles d\'eveloppent des biais à cause de leur environnement quotidien. Le Chef de la foi est alors le Pape et distille son opinion. Le Pape qui joue le jeu prend sa mitre et son bâton de pèlerin pour \'evang\'eliser ses compagnons h\'er\'etiques. Jusque-là, cela fait sourire. Mais, comme dans les religions r\'eelles, ils sont parfois agaçants de vouloir \'etendre leur opinion à ceux qui ne croient pas et qui basent leurs opinions uniquement sur les donn\'ees exp\'erimentales. Certaines de ces "\'eglises" n'h\'esitent pas à se comporter comme l'Inquisition. Ceux qui osent exprimer une opinion diff\'erente sont brûl\'es à chaque occasion, lors de conf\'erences ou sur leur lieu de travail ou sur les r\'eseaux sociaux. Certains jeunes chercheurs, peu inspir\'es, pr\'efèrent se convertir à la religion dominante, devenir des clercs ce qui est bien \'evidemment une voie plus rapide et plus simple que celle des chercheurs innovants ou même des iconoclastes. Le grand Pape \'ecrit sa Bible pour diffuser ses id\'ees, l'impose à lire aux \'etudiants et aux nouveaux venus. Il met alors en forme la pens\'ee des jeunes g\'en\'erations et assure son trône. C'est une attitude m\'edi\'evale qui peut bloquer le progrès. Certains Papes vont si loin qu'ils croient être le Pape dans leur domaine de sp\'ecialisation leur donne automatiquement le même droit d'avoir le même trône dans tous les autres domaines (bias cognitif à nouveau)...
	

	Ce dernier avertissement, et les rappels qui vont suivre, doivent servir le scientifique ou tout lecteur en faisant bon usage de ce que nous consid\'erons aujourd'hui comme les bonnes pratiques de travail / raisonnement (nous discuterons en d\'etail des principes de la m\'ethode Descartes plus tard) à r\'esoudre des problèmes ou d\'evelopper des modèles th\'eoriques de façon rigoureuse et non biais\'ee. 

	À cette fin, voici un tableau r\'ecapitulatif qui fournit les \'etapes qui devraient être suivies par un scientifique qui travaille en math\'ematiques ou en physique th\'eorique (pour les d\'efinitions, voir ci-dessous) ou toute personne souhaitent avoir une d\'emarche intellectuelle un tant soit peu honnnête et rigoureuse:

	\begin{table}[H]
		\centering
		\definecolor{gris}{gray}{0.85}
			\begin{tabular}{|p{7.5cm}|p{7.5cm}|}
				\hline
				\multicolumn{1}{c}{\cellcolor{black!30}\textbf{Math\'ematiques}} & 
  \multicolumn{1}{c}{\cellcolor{black!30}\textbf{Physique}} \\ \hline
				\textbf{1.} Exposer formellement ou en langage commun le ou les "hypothèses", "conjectures", "propri\'et\'es" à prouver (les hypothèses sont not\'ees H1., H2., etc. les conjectures CJ1., CJ2., etc. et les propri\'et\'es P1., P2 ., etc.). & \textbf {1.} Exposer correctement dans un langage formel ou commun tous les d\'etails des "problèmes" à r\'esoudre (les problèmes sont not\'es P1., P2., etc.). \\ \hline
				\textbf{2.} D\'efinir les "axiomes" (non d\'emontrables, ind\'ependants et non contradictoires) qui donneront les points de d\'epart et \'etabliront des restrictions aux d\'eveloppementx (les axiomes sont not\'es A1., A2, etc.)\footnotemark. \newline\newline
Dans la même veine, les math\'ematiciens d\'efinissent le vocabulaire sp\'ecialis\'e li\'e aux op\'erateurs math\'ematiques qui sera not\'e D1., D2., etc. & \textbf{2.} D\'efinir (ou \'enoncer) les "postulats" ou les "principes" ou les "hypothèses" (suppos\'ees improuvables ...) qui donneront le point de d\'epart et \'etabliront des restrictions sur les d\'eveloppements et raisonnements (typiquement, les hypothèses et les principes sont not\'es P1 ., P2., etc., les hypothèses H1., H2., etc. en essayant d'\'eviter la confusion entre les postulats et les principes)\footnotemark. \\ \hline
				\textbf{3.} Une fois les axiomes pos\'es, en extraire directement les "lemmes" ou "propri\'et\'es" dont la validit\'e en d\'ecoule et pr\'eparer le d\'eveloppement du th\'eorème suppos\'e valider l'hypothèse  ou les conjectures de d\'epart (les lemmes \'etant not\'es L1., L2., etc. et les  propri\'et\'es P1., P2. , etc.). & \textbf{3.} Une fois le "modèle th\'eorique" d\'evelopp\'e, v\'erifier les unit\'es dimensionnelles des \'equations pour identifier des erreurs possibles triviales dans les d\'eveloppements (ces contrôles \'etant marqu\'es VA1., VA2., etc.).\\ \hline
				\textbf{4.} Une fois les "th\'eorèmes" (not\'es T1., T2., etc.) d\'emontr\'es on peut conclure sur les "cons\'equences" (not\'ees C1., C2., Etc.) et même les propri\'et\'es (not\'ees P1., P2., Etc.). & \textbf{4.} Recherchez les cas limites (y compris les "singularit\'es") du modèle pour v\'erifier intuitivement la validit\'e (ces contrôles borderline sont not\'es CL1., CL2., Etc.).\\ \hline
				\textbf{5.} Tester la robustesse ou l'utilit\'e des conjectures ou des hypothèses en prouvant l'inverse (r\'eciproque) du th\'eorème ou en les comparant avec d'autres exemples de th\'eories math\'ematiques bien connues pour voir si l'ensemble forume une structure coh\'erente (les exemples \'etant not\'es E1., E2. , etc.). & \textbf{5.} Tester le modèle th\'eorique obtenu exp\'erimentalement  et soumettre le travail à comparer avec d'autres \'equipes de recherche ind\'ependantes. Le nouveau modèle devrait fournir des r\'esultats exp\'erimentaux et jamais observ\'es (pr\'edictions à falsifier). Si le modèle est valid\'e, alors il obtient le statut officiel de "th\'eorie" jusqu'à ce qu'il soit mis en d\'efaut. \\ \hline
				\textbf{6.} De possibles remarques peuvent être ajout\'ee dans un ordre hi\'erarchiquement structur\'e et not\'ees R1., R2., etc. & \textbf{6.} De possibles remarques peuvent être ajout\'ee dans un ordre hi\'erarchiquement structur\'e et not\'ees R1., R2., etc.			
				\\ \hline
		\end{tabular}
		\caption{M\'ethodologie pour les d\'evelopements en Math\'ematique \& Physique}
	\end{table}	
	\footnotetext[4]{Parfois, les «propriétés», les «conditions» et les «axiomes» sont confondus alors que le concept d'axiome est beaucoup plus précis et profond.}
	\footnotetext[5]{Il ne faut cependant pas oublier que la validité d'un modèle ne dépend pas du réalisme de ses hypothèses mais de la conformité de ses implications avec la réalité.}
	
	\begin{tcolorbox}[title=Remark,colframe=black,arc=10pt]
	Le fait qu'une question puisse être formulée dans une phrase dans un français grammaticalement correcte ne la rend pas significative, ni ne lui donne droit à une quelconque attention sérieuse. Ni, le fait qu'un mot existe dans le dictionnaire (comme le mot "âme") ne rend réel le concept sous-jacent...
	\end{tcolorbox}
	
	Proc\'eder comme dans le tableau ci-dessus est une base de travail possible pour les personnes actives dans le domaine des math\'ematiques ou de la physique, ou toute personne int\'eress\'ee à avoir une d\'emarche intelectuelle rigoureuse\footnote{Pour vous éduquer et vous exercer ou vos élèves sur le sujet de la pensée critique, nous recommandons fortement la lecture de \cite{parker2016looseleaf} et des tests de pensée critique comme le test Ennis-Weir (test d'écriture libre dans lequel le candidat évalue, paragraphe par paragraphe, un cas raisonné) ou le California Critical Thinking Disposition Inventory (CCTDI) qui mesure les dispositions à utiliser les compétences de pensée critique ainsi que le Watson-Glaser Critical Thinking Appraisal (WGCTA).}. \'evidemment, proc\'eder proprement et traditionnellement comme ci-dessus prend un peu plus de temps que de faire les choses n'importe comment (c'est pourquoi la plupart des enseignants ne suivent pas ces règles, ils n'ont pas assez de temps pour couvrir tout le programme), c'est aussi la raison pour laquelle la science amène une majorit\'e de personnes à l'ext\'erieur de leur zone de confort intelectuelle (la plupart des gens cherchant à r\'esoudre des problèmes et à trouver des r\'eponses à leurs questions en moins d'une heure...).

\begin{center}
\includegraphics[scale=0.9]{img/intro/hypothesis_definitions.jpg}
\end{center}
Le lecteur doit aussi savoir que nous insistons sur le fait que les vrais scientifiques ne devraient pas avoir d'\'emotions derrière les sujets qu'ils \'etudient ou dont ils parlent, ce afin simplement d'\'eviter les biais cognitifs. Ils doivent uniquement utiliser des preuves (faits bas\'es sur des donn\'ees, \'evaluation par des pairs, exp\'eriences reproductibles, consensus de la communaut\'e scientifique) plutôt que des analyses individuelles \'emotionnelles, biais\'ees\footnote{Parmi tous les principaux biais que nous présenterons plus tard, le Biais de Confirmation - la tendance à rechercher, interpréter, favoriser et rappeler des informations qui confirment ou soutiennent ses croyances ou valeurs personnelles a priori - est très probablement le plus courant chez les personnes illétrées scientifiquement.}, subjectives et \'educatives qui ne sont pas bas\'ees sur des donn\'ees sourc\'ees, reproductibles et r\'efutables.

Signalons aussi une forme amusante scientifique des $10$ commandements:
\begin{enumerate}
\item Les ph\'enomènes tu observeras\\
Et jamais mesure tu ne falsifieras \\
(attention à l'erreur de confirmation: \'etudier que des ph\'enomènes qui valident ses convictions)

\item Des hypothèses tu formuleras\\
Que par l'exp\'erimentation ou la simulation tu testeras

\item L'exp\'erience pr\'ecis\'ement tu d\'ecriras, tes donn\'ees et algorithmes tu fourniras\\
Car ton collègue la reproduira\\
(attention au piège de la discipline narrative: coller les faits aux r\'esultats d\'esir\'es)

\item Fort de tes r\'esultats\\
Une th\'eorie tu bâtiras

\item De parcimonie tu useras\\
Et l'hypothèse la plus simple tu retiendras

\item Jamais v\'erit\'e d\'efinitive ne sera (humilit\'e \'epist\'emique)\\
Et toujours tu chercheras

\item D'une thèse non r\'efutable tu t'abstiendras\\
Car hors de la science elle restera

\item Tout \'echec sera pris comme une r\'eussite\\
Car la science doit confirmer mais aussi infirmer

\item Mon autorit\'e je n'utiliserai pas (argument d'autorit\'e)\\
Pour biaiser les opinions des gens dans des domaines où je n'ai pas d'expertise prouv\'ee

\item Je respecterai le serment d'Archimède et les Règles de Publication Scientifique\\
Car la science doit être transparente et responsable
\end{enumerate}

	
	\begin{tcolorbox}[colback=red!5,borderline={1mm}{2mm}{red!5},arc=0mm,boxrule=0pt]
	\bcbombe Attention! Il est très facile de faire de nouvelles th\'eories physiques en alignant simplement des mots. Ce type de d\'emarche est nomm\'e "\NewTerm{philosophie}\index{philosophie}" et les Grecs ont d\'eduit l'existance des atomes avec cette m\'ethode d'investigation. Cela peut donc conduire avec beaucoup de chance à une vraie th\'eorie. Par contre il est beaucoup plus difficile de faire une "\NewTerm{th\'eorie pr\'edictive}"\index{th\'eorie pr\'edictive}, c'est-à-dire avec des \'equations qui pr\'edisent le r\'esultat d'une exp\'erience.\\
	
	De plus, de nombreux philosophes réinventent des arguments que les physiciens savent depuis longtemps être faux. Nous avons entendu des philosophes s'inquiéter de paradoxes résolus il y a des siècles par les physiciens, et nous avons entendu des philosophes déduire comment les lois naturelles devraient être tout en ignorant comment les lois naturelles sont. Bref, il y a malheureusement beaucoup de philosophes qui ne remarquent pas quand ils sont hors de leur domaine d'expertise. On peut en dire autant des physiciens. Certains physiciens s'appuient plus fréquemment sur des arguments philosophiques qu'ils n'aiment l'admettre. Il est cependant assez facile pour nous de rejeter la philosophie comme utile -
parce que c'est inutile.
	\end{tcolorbox}

	\begin{tcolorbox}[title=Remarques,colframe=black,arc=10pt]
	Ce qui s\'epare les math\'ematiques et la physique, c'est qu'en math\'ematiques, l'hypothèse est toujours vraie. Le discours math\'ematique n'est pas une preuve d'une v\'erit\'e de recherche externe, mais une cible de coh\'erence. Ce qui devrait être correct est juste le raisonnement.
	\end{tcolorbox}
	
	Lorsque ces règles ne sont pas respect\'ees, on parle de "\NewTerm {fraude scientifique}"\index{fraude scientifique}" ou de "\NewTerm {fraude intellectuelle}" (qui conduit souvent à être renvoy\'e de son travail mais malheureusement nous ne retirons toujours pas les diplômes quand cela arrive). En g\'en\'eral, la fraude scientifique elle-même se pr\'esente sous quatre formes principales: le plagiat, la fabrication de donn\'ees et l'alt\'eration des r\'esultats d\'efavorables à l'hypothèse, l'omission d'hypothèses de travail claires et de donn\'ees recolt\'ees, l'usage d'arguments fallacieux et biais\'es. A ces fraudes on peut aussi ajouter des comportements qui posent des problèmes de qualit\'e de travail ou plus sp\'ecifiquement d'\'ethique, tels que ceux visant à soumettre par exemple plusieurs fois la même publication avec seulement quelques modifications, l'omission de conflit d'int\'erêt, les exp\'eriences dangereuses, la non-conservation des donn\'ees primaires, etc.
	\begin{figure}[H]
		\centering
		\includegraphics[scale=0.7]{img/intro/peer_review.jpg}
		\caption[]{Source: \url{http://cartoonsbyjosh.co.uk}}
	\end{figure}	

	\subsection{M\'ethode de Descartes}
	Pr\'esentons maintenant les quatre principes de la m\'ethode de Descartes qui, rappelons-le, est consid\'er\'e comme le premier scientifique de l'histoire de par sa m\'ethode d'analyse:
	\begin{itemize}
		\item[P1.] Ne recevoir jamais aucune chose pour vraie que je ne la connusse \'evidemment être telle. C'est-à-dire, d'\'eviter soigneusement la pr\'ecipitation et de ne comprendre rien de plus en mes jugements que ce qui se pr\'esenterait si clairement et si distinctement à mon esprit, que je n'eusse aucune occasion de le mettre en doute.
		
		\item[P2.] De diviser chacune des difficult\'es que j'examinerais, en autant de parcelles qu'il se pourrait (observations scrupuleuses et hypothèses vraisemblables jusqu'à preuve du contraire), et qu'il serait requis pour les mieux r\'esoudre.
		
		\item[P3.] De conduire par ordre mes pens\'ees, en commençant par les objets les plus simples et les plus ais\'es à connaître, pour monter peu à peu comme par degr\'es jusqu'à la connaissance des plus compos\'es, et supposant même de l'ordre entre ceux qui ne se pr\'ecèdent point naturellement les uns les autres.
		
		\item[P4.] Faire partout des d\'enombrements si entiers et des revues si g\'en\'erales, que je fusse assur\'e de ne rien omettre.
	\end{itemize}	
	
	\begin{figure}[H]
		\centering
		\includegraphics[scale=0.3]{img/intro/nullius_in_verba.jpg}
	\end{figure}
	\textit{Nullius in verba} (en latin "ne croire personne sur parole") est la devise de la Royal Society. C'est une expression de la détermination des membres de la Royal Society à résister à la domination de l'autorité et à vérifier toutes les déclarations par un appel à des faits déterminés par des expériences reproductibles et par un examen scientifique précautionneux par les pairs\footnote{Le processus d'examen par les pairs s'applique généralement aux articles de revues scientifiques, mais il est possible qu'un livre soit également évalué par des pairs. Bien que de nombreux livres passent par une sorte de processus d'édition ou de révision, il n'y a pas de méthode facile pour déterminer si un livre est évalué par des pairs. Une méthode pour trouver des livres évalués par des pairs consiste à jeter un oeil aux publications de livres des presses universitaires. Les livres publiés par les presses universitaires sont presque toujours soumis à un processus d'examen par les pairs. Les livres des presses universitaires sont généralement écrits par des membres du corps professoral qui sont soumis à une immense pression pour produire une littérature savante faisant autorité. Le processus d'examen par les pairs pour les presses universitaires implique généralement deux ou trois arbitres indépendants qui examineront initialement le manuscrit. Si le manuscrit reçoit une évaluation positive, la presse universitaire l'enverra à son comité de rédaction, qui sont tous membres du corps professoral, pour examen final.} de toute information.

	\subsubsection{\'Etudes en aveugle}
	Les exp\'eriences scientifiques\footnote{Ce texte est un copier/coller d'un article \'ecrit par Manuel Gnida à \url{http://www.symmetrymagazine.org/article/the-facts-and-nothing-but-the-facts}} sont conçues pour d\'eterminer des faits sur notre monde en utilisant soit des "\NewTerm {\'etudes r\'etrospectives}\index{\'etudes r\'etrospectives}" bas\'ees sur la recherche de corr\'elations en exploitant des bases de donn\'ees existantes ou sur des "\NewTerm{\'etudes prospectives}\index{\'etudes prospectives}" bas\'ees sur la recherche de causalit\'es en utilisant des exp\'eriences contrôl\'ees, randomis\'ees et en double aveugle. Mais dans des analyses compliqu\'ees, il y a un risque que les chercheurs faussent involontairement les r\'esultats pour qu'ils correspondent à ce qu'ils s'attendaient à trouver. Pour r\'eduire ou \'eliminer ce biais potentiel, les scientifiques appliquent une m\'ethode appel\'ee "\NewTerm{l'analyse aveugle} \index{analyse aveugle}".
	
	Les \'etudes à l'aveugle sont probablement mieux connues par leur utilisation dans les essais cliniques de m\'edicaments (le terme «triple aveugle se r\'efère parfois à ces derniers), dans lequel les patients sont tenus dans l'ignorance - ou dans l'aveugle - quand à savoir s'ils reçoivent un m\'edicament ou un placebo (pour faire simple car dans la r\'ealit\'e c'est plus subtile!). Cette approche aide les chercheurs à juger si leurs r\'esultats proviennent du traitement lui-même ou de la croyance des patients qu'ils le reçoivent. Mais la m\'ethode est \'egalement utilis\'ee dans la d\'egustation gastronomique ou dans les laboratoires m\'edico-l\'egaux.
	
	Les physiciens des particules et les astrophysiciens font \'egalement des \'etudes "en aveugle". L'approche est particulièrement utile lorsque les scientifiques recherchent des effets extrêmement faibles cach\'es dans le bruit de fond qui indiquent l'existence de quelque chose de nouveau, non pris en compte dans le modèle actuel. Parmi les exemples, citons les d\'ecouvertes largement diffus\'ees du boson de Higgs par des exp\'eriences men\'ees au Large Hadron Collider du CERN (Centre Europ\'een de la Recherche Nucl\'eaire) et des ondes gravitationnelles par le d\'etecteur Advanced LIGO (Laser Interferometer Gravitational-Wave Observatory).
	\begin{figure}[H]
		\centering
		\includegraphics[scale=0.7]{img/intro/scientific_evidence_02.jpg}
		\caption{\'Evidence Scientifique Hi\'erarchis\'ee (pyramide des données probantes)}
	\end{figure}
	
	La figure ci-dessus illustre au plus bas niveau d'évidence le fait trivial que nous ne devrions pas faire confiance aux scientifiques (leurs opinions et croyances) et encore moins aux médecins, chirurgiens ou ingénieurs (qui ne sont pas des scientifiques !) - la science n'est pas dirigée par les croyances et opinions des humains (!) - mais uniquement par les évidence fournies par les données d'expériences indépendantes reproductibles et l'analyse statistique à un niveau de méta-analyse dans des journaux soumis à la revue par les pairs à haut niveau d'indice H!
	
	\begin{tcolorbox}[title=Remarque,colframe=black,arc=10pt]
	Les témoignages ou anecdotes personnelles n'ont presque aucune valeur si la taille de l'échantillon est petite et biaisée et s'il n'y a pas de moyen direct de mesurer directement l'événement associé ! C'est pourquoi les évidences historiques basées sur seulement quelques témoignages écrits dans un livre unique vieux de plusieurs milliers d'années n'ont aucune valeur scientifique (même s'il existe des dizaines de tels livres avec des témoignages concordants).
	\end{tcolorbox}
	
	Ceux qui citent Nietzche :
	\begin{fquote}[Friedrich Nietzsche]Il n'y a pas de faits, que des interprétations !
 	\end{fquote}
 	ne comprennent pas non plus qu'ils sont au niveau EL01 parce que cette affirmation est elle elle-même un fait... ????

	"\textit{Les analyses scientifiques sont des processus it\'eratifs, dans lesquels nous effectuons une s\'erie de petits ajustements aux modèles th\'eoriques jusqu'à ce que les modèles d\'ecrivent avec pr\'ecision les donn\'ees exp\'erimentales}", explique Elisabeth Krause, postdoc à l'Institut Kavli d'astrophysique des particules et de cosmologie, institut qui est conjointement exploit\'e conjointement par l'Universit\'e Stanford et le D\'epartement de l'\'energie SLAC National Accelerator Laboratory. "\textit{À chaque \'etape d'une analyse, il y a le danger que les connaissances ant\'erieures nous guident dans la façon dont nous proc\'edons aux ajustements, tandis que les analyses aveugles nous aident à prendre des d\'ecisions ind\'ependantes et meilleures}".
	
	Le Retour sur EXp\'erience (REX) montre comme attendu que les analyses en aveugle doivent être conçues individuellement pour chaque exp\'erience. Effectivement, la façon dont "l'aveuglement" est fait doit laisser aux chercheurs suffisamment d'informations pour permettre une analyse significative, et cela d\'epend du type de donn\'ees provenant d'une exp\'erience sp\'ecifique.

	Une approche commune consiste à baser l'analyse uniquement sur certaines donn\'ees, à l'exclusion de la partie dans laquelle une anomalie est cens\'ee se cacher. Les donn\'ees exclues sont dites être dans une "boîte noire" ou une "boîte de signalisation cach\'ee".

	Prenons le cas de la recherche du boson de Higgs. En utilisant les donn\'ees recueillies avec le Large Hadron Collider (LHC) jusqu'à la fin de 2011, les chercheurs ont vu des signes d'une bosse dans les statistiques comme un signe potentiel d'une nouvelle particule avec une masse d'environ $ 125 $ gigaelectronvolts. Donc, quand ils ont regard\'e de nouvelles donn\'ees, ils ont d\'elib\'er\'ement mis en quarantaine la plage de masse autour de cette bosse et se sont concentr\'es sur les donn\'ees restantes à la place.
	
	Ils ont utilis\'e ces donn\'ees pour s'assurer qu'ils travaillaient avec un modèle suffisamment pr\'ecis. Puis ils ont "ouvert la boîte" et appliqu\'e ce même modèle à la r\'egion pr\'ealabement mise en quarantaine. La bosse s'est r\'ev\'el\'ee finalement être la particule de Higgs tant recherch\'ee.

	Cela a bien fonctionn\'e pour les chercheurs du boson de Higgs. Cependant, comme les scientifiques impliqu\'es dans l'exp\'erience du Large Underground Xenon (LUX) ont rapport\'e après cette approche, la m\'ethode de la "boîte noire" de l'analyse aveugle peut bien \'evidemment poser des problèmes si les donn\'ees que nous mettons dans la boîte noire contiennent des \'ev\'enements cruciaux.
	
	LUX a r\'ecemment effectu\'e l'une des recherches les plus sensibles au monde sur les WIMP (Weakly Interacting Massive Particles) - des particules hypoth\'etiques de matière noire, une forme invisible de la matière qui est cinq fois plus r\'epandue que la matière ordinaire. Les scientifiques du LUX ont fait beaucoup de travail pour prot\'eger le LUX contre les particules du bruit de fond: construction du d\'etecteur dans une salle blanche, le remplir de liquide complètement purifi\'e, l'entourer de blindage et l'installer sous $1,600$ mètres de roche. Mais quelques particules errantes le traversent n\'eanmoins, et les scientifiques ont besoin de regarder toutes leurs donn\'ees pour les trouver et les \'eliminer.

	Pour cette raison, les chercheurs du LUX ont choisi une approche diff\'erente pour leurs analyses. Au lieu d'utiliser une "boîte noire", ils utilisent un processus appel\'e "salage".

	Les scientifiques du LUX qui n'\'etaient pas impliqu\'es dans l'analyse LUX la plus r\'ecente ont ajout\'e de faux \'ev\'enements aux signaux simul\'es par des donn\'ees qui ressemblent à des signaux r\'eels. Tout comme les patients dans un essai de m\'edicament à l'aveugle, les scientifiques du LUX ne savaient pas s'ils analysaient des donn\'ees r\'eelles ou placebo. Une fois qu'ils ont termin\'e leur analyse, les scientifiques qui ont fait le "salage" ont r\'ev\'el\'e quels \'ev\'enements \'etaient faux.

	Une technique similaire a \'et\'e utilis\'ee par les scientifiques de LIGO (Laser Interferometer Gravitational-Wave Observatory ), qui ont finalement fait la première d\'etection de très petites ondulations dans l'espace-temps appel\'ees "ondes gravitationnelles".

	Pas tout le monde dans la communaut\'e scientifique est convaincu que les analyses en aveugle soient n\'ecessaires. Les analyses en aveugle sont bien \'evidemment plus compliqu\'ees à concevoir que les analyses non-aveugles et prennent plus de temps à compl\'eter et sont donc plus on\'ereuses. Certains scientifiques participant à des analyses en aveugle passent in\'evitablement du temps à regarder de fausses donn\'ees, ce qui peut donner l'impression d'un certain gaspillage.
	
	Typiquement certains médecins et ingénieurs assez célèbres (tous détestant les mathématiques car ils étaient très mauvais dans ce domaine pendant leurs études et ne comprennent pas comment appliquer des statistiques avancées ni comment lire les résultats analytiques correspondants!) fustigent la "religion du tout randomisé en double aveugle" et une médecine/ingénierie qui est passée des mains humanistes des soignants/inventeurs à celles froides des statisticiens et des "méthodologues". Certes (...) nous savons effectivement depuis longtemps que la règle du pouce, les sentiments et les croyances fonctionnent bien mieux que la méthode statistique...
	\begin{center}
		\includegraphics[scale=0.5]{img/intro/evidence_truth.jpg}
	\end{center}
	Le lecteur doit également garder à l'esprit que nous n'avons jamais prétendu dans ce livre que les méta-analyses ou les ECR (essais contrôlés randomisés) sont la règle d'or de la science fondée basée sur l'évidence. Nous prétendons simplement que ce sont les outils qui semblent factuellement - au moment où nous écrivons ces lignes - donner les meilleurs résultats. De toute évidence, ils ne sont pas parfaits (car les humains qui mènent les expériences ne sont pas non plus des êtres parfaits...) et parfois ils ont échoué, mais quiconque critique les méta-analyses et les ECR devrait fournir des preuves quantitatives qu'il existe d'autres méthodes (en mentionnant laquelle) qui fonctionnent statistiquement significativement mieux !
	
	\begin{fquote}[L. Aron Nelson]La plupart des gens ne veulent pas vraiment la vérité ; ils veulent juste être rassurés que ce qu'ils croient déjà, est la vérité.
 	\end{fquote}
	
	\pagebreak
	\subsection{Serment d'Archimède}
	Sur le modèle du serment d'Hippocrate, un groupe d'\'etudiants de l'\'ecole Polytechnique F\'ed\'erale de Lausanne (E.P.F.L.) a \'elabor\'e en 1990 un serment d'Archimède exprimant les responsabilit\'es et les devoirs de l'ing\'enieur et du technicien. Il a \'et\'e repris sous diverses versions par d'autres \'ecoles d'ing\'enieurs europ\'eennes et pourrait servir d'inspiration de base comme serment pour les chercheurs en science (il manque cependant quelques points comme, dans la m\'edecine\footnote{Rappelons au passage que contrairement à une idée faussement répandue que les médecins et les chirurgiens ou tout personnel médical ne sont bien évidemment pas des scientifiques (nous laissons le soin au lecteur de faire des recherches approfondies si cela l'étonne)!}, d'être radi\'e de l'ordre des scientifiques en cas de tromperie grave).

	« Consid\'erant la vie d'Archimède de Syracuse qui illustra dès l'Antiquit\'e le potentiel ambivalent de la technique, consid\'erant la responsabilit\'e croissante des ing\'enieurs et des scientifiques à l'\'egard des hommes et de la nature, consid\'erant l'importance des problèmes \'ethiques que soulèvent la technique et ses applications, aujourd'hui, je prends les engagements suivants et m'efforcerai de tendre vers l'id\'eal qu'ils repr\'esentent:\\
	\begin{enumerate}[label=\protect\circledbullet{\arabic*},leftmargin=15mm]
		\item Je pratiquerai ma profession pour le bien des personnes, dans le respect des Droits de l'Homme et de l'environnement.

		\item Je reconnaîtrai, m'\'etant inform\'e au mieux, la responsabilit\'e de mes actes et ne m'en d\'echargerai en aucun cas sur autrui.

		\item Je comprends que mon travail peut avoir des impacts consid\'erables sur la soci\'et\'e et l'\'economie et ce bien au delà de ma compr\'ehension.

		\item Je m'appliquerai à parfaire mes comp\'etences professionnelles.

		\item Dans le choix et la r\'ealisation de mes projets, je resterai attentif à leur contexte et à leurs cons\'equences, notamment des points de vue technique, \'economique, social, \'ecologique...

		\item Je contribuerai, dans la mesure de mes moyens, à promouvoir des rapports \'equitables entre les hommes et à soutenir le d\'eveloppement des pays \'economiquement faibles.

		\item Je transmettrai, avec rigueur et honnêtet\'e, à des interlocuteurs choisis avec discernement, toute information importante, si elle repr\'esente un acquis pour la soci\'et\'e ou si sa r\'etention constitue un danger pour autrui. Dans ce dernier cas, je veillerai à ce que l'information d\'ebouche sur des dispositions concrètes.

		\item Je ne me laisserai pas dominer par la d\'efense de mes int\'erêts ou ceux de ma profession.

		\item Je m'efforcerai, dans la mesure de mes moyens, d'amener mon entreprise à prendre en compte les pr\'eoccupations du pr\'esent Serment.

		\item Je pratiquerai ma profession en toute honnêtet\'e intellectuelle, avec conscience et dignit\'e.
		
		\item Je le promets solennellement, librement et sur mon honneur. »\\ 
\end{enumerate}
	Malheureusement, ce serment devrait être compl\'et\'e par la "\NewTerm{D\'eclaration de Munich sur les droits et devoirs des journalistes (1971)}\index{D\'eclaration de Münich sur les devoirs et les droits des journalistes}". C'est-à-dire les tâches essentielles du scientifique dans la collecte, la communication et le commentaire des donn\'ees consistent en:
\begin{itemize}
	\item Respecter la v\'erit\'e, quelles qu’en puissent être les cons\'equences pour lui-même, et ce, en raison du droit que le public a de connaître la v\'erit\'e.

	\item D\'efendre la libert\'e de l’information, du commentaire et de la critique.

	\item Publier seulement les informations dont l’origine est connue ou les accompagner, si c’est n\'ecessaire, des r\'eserves qui s’imposent ; ne pas supprimer les informations essentielles et ne pas alt\'erer les textes et les documents.

	\item Ne pas user de m\'ethodes d\'eloyales pour obtenir des informations, des photographies et des documents.

	\item S'obliger à respecter la vie priv\'ee des personnes.

	\item Rectifier toute information publi\'ee qui se r\'evèle inexacte.

	\item Garder le secret professionnel et ne pas divulguer la source des informations obtenues confidentiellement.

	\item S'interdire le plagiat, la calomnie, la diffamation, les accusations sans fondement ainsi que de recevoir un quelconque avantage en raison de la publication ou de la suppression d'une information.

	\item Ne jamais confondre le m\'etier de journaliste avec celui du publicitaire ou du propagandiste ; n’accepter aucune consigne, directe ou indirecte, des annonceurs.

	\item Refuser toute pression et n’accepter de directives r\'edactionnelles que des responsables de la r\'edaction.
\end{itemize}

	\begin{center}
		\includegraphics[scale=0.7]{img/intro/serment_archimede.jpg}
	\end{center}

	\pagebreak
	\subsection{Règles de Publication Scientifique (RPS)}
	Il est impossible d'avoir un d\'ebat ou une analyse constructive si le mat\'eriel de base est inutilisable ou indisponible. Malheureusement, au XXIe siècle, il est assez facile de trouver des publications de Prix Nobel qui ont fait l'objet d'un examen par les pairs\footnote{Certaines \'etudes sont publi\'ees sans aucun examen par les pairs, même des \'etudes de Prix Nobel (...), ces "\'editeurs pr\'edateurs" inondent la litt\'erature scientifique avec des journaux qui sont essentiellement fallacieux, tout auteur peut y être publi\'e, il suffit qu'il paie pour cela!} et qui sont scientifiquement inutilisables (sans compter le fait qu'une proportion importante de revues scientifiques priv\'ees r\'echignent à publier les r\'eplications exp\'erimentales car trop ennuyantes selon elles, c'est-à-dire ne rapportant pas assez d'argent...). C'est pourquoi nous rappelons ici les règles de publication scientifique fondamentales pour qu'une publication soit accept\'ee par un v\'eritable comit\'e d'\'evaluation scientifique:
	\begin{enumerate}[label=\protect\circledbullet{\arabic*},leftmargin=15mm]
		\item Utilisation de \LaTeX{} pour la r\'edaction de la publication
		
		\item Tous les fichiers de r\'edaction (*.tex) et les fichiers de donn\'ees brutes doivent avoir des noms conformes à la norme ISO 9660
		
		\item La publication doit avoir un GUID (un code unique semblable à l'identificateur d'objet num\'erique DOI)
		
		\item Mettre les dates de publication et d'\'evaluation par les pairs (format de date/heure ISO 8601)
		
		\item Mettre la version majeure et mineure de la publication (ex: v3.6 r58) et les mots-clés relatifs à la toxonomie du domaine
		
		\item Mettre la date de la p\'eriode d'exp\'erimentation/d\'eveloppement (format de date/heure ISO) 
		
		\item R\'ediger un r\'esum\'e ou "abstract" (bref r\'ecaptiluatif des objectifs, de l'exp\'erience, des hypothèses, du protocole et des conclusions)
		
		\item \'Ecrire une introduction
		
		\item Toutes les unit\'es de mesure\footnote{Mesures qui au passage doivent être enregistrables, reproductibles et r\'epondre à une ou plusieurs causes identifi\'ees.} et notations math\'ematiques doivent respecter les normes ISO 80000
		
		\item Utiliser le "principe de pr\'ecaution\footnote{L'usage des expression "...nous pensons", "nous croyons" et "...nous avons la foi" ou toute expression similaire est bien \'evidemment interdite.}" (utilisation de conditionnel)
		
		\item Utiliser des "r\'eponses r\'eactives", c'est-à-dire faire les confrontations entre hypothèses / donn\'ees, hypothèses / faits, hypothèses / observations
		
		\item Utiliser, si possible, des "facteurs de levier" pour donner du contenu et du cr\'edit à la publication en faisant r\'ef\'erence à une autre publication correspondante sur le même sujet \footnote{Ceci est aussi l'\'etape très importante de la "revue personnelles", c'est-à-dire de plusieurs dizaines / centaines de publications scientifiques dont vous avez fait une analyse critique que vous utilisez pour construire votre propre argumentation.}
		
		\item Le mat\'eriel et les m\'ethodes doivent être d\'ecrits en d\'etails. Pour les articles th\'eoriques, ils doivent fournir un lien (URL) ou une r\'ef\'erence où la preuve d\'etaill\'ee peut être trouv\'ee (si la preuve d\'etaill\'ee est omise dans la publication originale!). Pour les exp\'eriences, le protocole d\'etaill\'e randomis\'e en double aveugle doit être fourni\footnote{Pour \'eviter typiquement un scandale, exemple parmi d'autres connus, comme l'affaire Jacques Benveniste...}
		
		\item Inclure des captures d'\'ecran (ou exports) haute r\'esolution de graphiques (incluant obligatoirement les erreurs de mesures et les intervalles de confiance/pr\'ediction visibles sur les graphiques avec le code source pour reproduire ces derniers!) ou de photos
		
		\item \'Ecrire les r\'esultats et pour les donn\'ees exp\'erimentales toujours fournir une analyse statistique pour montrer si l'effet semble significatif ou non (tailles d'effets, intervalles de fluctuations, moyennes, m\'edianes, \'ecarts-types, erreurs standards, tailles d'\'echantillons, kurtosis, skewness et si la $p$-value est communiqu\'ee \underline{toujours} communiquer avec la puissance du test!)
		
		\item Calculer la propagation des erreurs des instruments de mesure
		
		\item \'Ecrire la conclusion pr\'eliminaire pour éviter le HARKing\index{HARKing} (Hypothesizing After Results Are Known)\footnote{La conclusion pour les r\'esultats exp\'erimentaux (rejeter l'hypothèse nulle ou non) doit être \'ecrite avant (!) que l'exp\'erience soit ex\'ecut\'ee et non modifi\'ee par la suite pour \'eviter les biais cognitifs humains.}
		
		\item Donner accès aux donn\'ees brutes dans un format non propri\'etaire à la communaut\'e scientifique
		
		\item Donner accès aux scripts / codes utilis\'es pour l'analyse des donn\'ees et la reproductabilit\'e des calculs par la communaut\'e scientifique\footnote{Pour éviter des situations comme la fameuse (et honteuse) erreur de Reinhart-Rogoff...}
		
		\item Donner accès aux sources \LaTeX{} de la publication à la communaut\'e scientifique
		
		\item Fournir la version exacte (avec version mineure!) des logiciels utilis\'es pour l'expérimentation (calculs, r\'edaction et autres)
		
		\item Inclure la bibliographie avec les r\'ef\'erences à la fin du document aux normes ISO 690 (numérique) et rendre disponible le fichier BibTeX correspondant
		
		\item Citer des \'etudes \'equivalentes pour la m\'eta-analyse \footnote{S'il n'y a pas d'\'etudes \'equivalentes, alors aucune m\'eta-analyse n'est possible et les r\'esultats ainsi que les conclusions ne peuvent amener à aucun consensus scientifique pour rappel!}
		
		\item Mettre le \% de soutien financier de chaque sponsor de l'\'etude (conflits d'int\'erêts\footnote{Pas uniquement industriels et \'economiques, mais aussi religieux comme le fait de travailler pour une universit\'e non-laïque!}, sources de financement)
		
		\item Soumettre la publication au comit\'e à l'examen par les pairs (en simple ou en double aveugle\footnote{Le "simple aveugle" consiste à ce que les pairs ne connaissent pas le nom des auteurs de la publication, le"double aveugle" est que ni les auteurs ni les pairs connaissent l'identit\'e des uns et des autres.})
		
		\item Dresser la liste de tous les intervenants (avec fonctions, diplôme de quel universit\'e et e-mail\footnote{Et si possible avec le genre, le pays d'origine et l'ann\'ee de naissance pour des objectifs statistiques. Par exemple: Albert Einstein (Chercheur, Dr ès sc. Physique ETHZ, a.einstein@ethz.ch, M, CHE, 1879)}) et des pairs (seulement le noms de famille pour ces derniers) de la publication
	\end{enumerate}
	Toute publication ne respectant pas au moins une de ces règles ne peut être consid\'er\'ee comme une publication "scientifique"! Un grand nombre des points ci-dessus s'appliquent \'egalement aux contenus vid\'eos (vid\'eos TEDx ou vid\'eos YouTube où l'intervenant ne cite pas les sources et les m\'eta-analyses lorsqu'il argumente ou expose son "exp\'erience personnelle", ses "opinions" ou son "expertise").
	
	Le fait d’être publi\'e dans une revue à comit\'e de lecture ne garantit cependant pas la qualit\'e d’un article. C’est effectivement une donn\'ee int\'eressante qui m\'erite d’être à nouveau soulign\'ee et \'etudi\'ee. Est-elle surprenante pour autant ? \'evidemment pas! Tous les « publiants » savent que les « rapporteurs » jugeant de la qualit\'e des articles soumis aux revues sont eux-mêmes des chercheurs, pris dans la spirale de leurs propres travaux et enseignements, qui n’ont ni le d\'esir ni la possibilit\'e mat\'erielle de v\'erifier point par point tous les calculs d’un article de physique th\'eorique ou toutes les r\'ef\'erences d’un article de sciences humaines. Ce n’est d’ailleurs pas leur mission. L’immense majorit\'e des articles clairement erron\'es sont rejet\'es par le système (les auteurs des canulars rat\'es ne s’en ventent pas et personne ne saura combien ont \'et\'e d\'ejou\'es). Quelques-uns passent pourtant à travers les mailles du filet. C’est \'evidemment regrettable mais parfaitement connu de chaque communaut\'e concern\'ee. Celle des sciences dures n’est pas \'epargn\'ee et de notoires charlatans sont parvenus à publier dans des revues respectables et reconnues. Les sciences dures ne s’en sont naturellement pas trouv\'ee globalement disqualifi\'ees ! Ces articles ont tout simplement \'et\'e ignor\'es pour la grande majorit\'e (donc malheureusement pas tous et en particulier ceux qui ne d\'etaillent pas tous les d\'eveloppements math\'ematiques!): ni lus, ni cit\'es.
	
	\begin{tcolorbox}[title=Remarque,colframe=black,arc=10pt]
	Même s'il existe un consensus entre les scientifiques, une \'etude orient\'ee unique (qui peut être très importante) peut être utilis\'ee pour influencer l'opinion des principaux m\'edias, gouvernements et individus. C'est pourquoi une \'etude doit toujours être r\'ep\'et\'ee, \'evalu\'ee par des pairs et m\'eta-analys\'ee par des \'equipes et des laboratoires ind\'ependants.
	\end{tcolorbox}
	
	\begin{tcolorbox}[colback=red!5,borderline={1mm}{2mm}{red!5},arc=0mm,boxrule=0pt]
	\bcbombe Attention! Un certain nombre de gens pensent qu'un "\NewTerm{consensus scientifique}" fait r\'ef\'erence à un grand groupe de scientifiques qui sont tous d'accord sur le fait que quelque chose est vrai. En r\'ealit\'e, un consensus scientifique est un vaste corpus d'\'etudes scientifiques qui s'accordent et se soutiennent mutuellement ("consensus de donn\'ees"). L'accord entre les scientifiques eux-mêmes est simplement un sous-produit de la preuve coh\'erente.
	\end{tcolorbox}
	
	Un exemple bien connu de consensus inexistant est celui des religions. En effet, si quelqu'un pr\'etend que comme les statistiques ne mentent pas, alors le Dieu chr\'etien doit exister car c'est la religion la plus suivie au monde avec $2$ milliards de chr\'etiens et comme $2$ milliards de personnes ne peuvent pas avoir tort, vous pouvez rappeler à cette même personne que comme il y a $7$ milliards d'individus dans le Monde, les autres $5$ milliards qui ne croient pas au Dieu chr\'etien ne peuvent pas se tromper car... justement les statistiques ne mentent pas... Le même raisonnement s'applique si vous fusionnez les musulmans et les chr\'etiens, alors seulement $55\%$ des personnes dans le monde croient en un Dieu unique et $55\% $ ce n'est statistiquement pas assez pour atteindre le consensus scientifique qui lui est à un seuil de $95\%$...
	
	\begin{fquote}Si une religion n’avait ne serait-ce qu’une seule vraie évidence au-delà de tout doute raisonnable, il n’y aurait pas d’autres religions!
 	\end{fquote}
	
	\begin{tcolorbox}[title=Remarque,colframe=black,arc=10pt]
	Il est possible de réfuter logiquement l'existence de dieux ayant certains attributs, en montrant une incohérence entre ces attributs et la définition du dieu ou d'autres faits établis. Pour de nombreux exemples de cela, voir  \cite{martin2003impossibility}.
	\end{tcolorbox}
	
	\begin{center}
		\includegraphics[scale=2.5]{img/intro/scientific_papers.jpg}
	\end{center}
	Il est alors facile de comprendre pourquoi les pages Internet et les vid\'eos YouTube (ou toute autre plateforme similaire) ne sont pas des sources scientifiques fiables selon le protocole ci-dessus puisque:
	\begin{enumerate}
	   \item Les noms des pairs sont la grande majorit\'e du temps non indiqu\'e (à ce jour du moins!)
	   
	   \item Les contributeurs / \'editeurs sont anonymes en grande majroit\'e ne peuvent donc pas être identifi\'es (typiquement un problème de Wikip\'edia)
	   
	   \item Les d\'etails math\'ematiques ne sont pas fournis (ou même pire, il n'y a pas du tout d'\'equations!). Donc il est difficile, voire impossible de v\'erifier par vous-même si le raisonnement pr\'esent\'e est exact
	   
	   \item Le protocole exact de l'exp\'erience n'est pas communique\'e, il est donc impossible de savoir si les r\'esultats sont faux ou r\'eels ou même de les reproduire
	   
	   \item Aucune source ou r\'ef\'erence crois\'ee donn\'ee
	   
	   \item Le contenu est dans un format non fiable (une vid\'eo ou une page web ne sont pas des sources p\'erennes et prot\'eg\'ees \footnote{Au 21ème siècle un PDF par exemple devrait être prot\'eg\'e contre l'\'edition et sign\'e \'electroniquement})
	   
	   \item Les nouveaux modèles th\'eoriques pr\'esent\'es pr\'edisent bien ce que fait le pr\'ec\'edent, mais ne pr\'edisent rien de nouveau et ne sont donc pas falsifiables (r\'efutables)
	   
	   \item L'orateur sur la vid\'eo fait des hypothèses qui ne sont pas falsifiables (r\'ef\'erence à des dieux divers et vari\'es ou à des th\'eories dont les d\'etails math\'ematiques ne sont pas fournis)
	   
	   \item etc.
	\end{enumerate}
	
	\begin{center}
		\includegraphics[scale=0.5]{img/intro/fake_science.jpg}
	\end{center}
	
	La solidit\'e des preuves produites par de diff\'erents types d’\'etudes (par exemple revues syst\'ematiques, m\'eta-analyses, essais contrôl\'es randomis\'es, recherche par observation, \'etudes sur animaux, \'etudes sur cellules et avis d’experts) peut varier. Cette infographie vous aidera à comprendre les avantages et les restrictions de diff\'erents types de preuves (n'oubliez pas ce terme est abusif et que en toute rigueur on doit parler "d'évidences"!) scientifiques:
	\begin{figure}[H]
		\centering
		\includegraphics[width=1\textwidth]{img/intro/how_strong_is_the_scientific_evidence.pdf}
		\caption[Diff\'erents types de preuves scientifiques]{Diff\'erents types de preuves scientifiques (source: EUFIC)}
	\end{figure}
	
	\begin{tcolorbox}[title=Remarque,colframe=black,arc=10pt]
	Le monde n’est pas systématiquement en adéquation avec nos désirs, nos espoirs, nos préjugés, nos aspirations philosophiques, ni même avec nos appréhensions ou nos angoisses. Ce n’est pas parce qu’il serait supercoolde pouvoir déplacer des objets par la seule pensée que cela est nécessairement possible. De même, ce n’est pas parce qu’il serait heureux qu’un traitement contre une maladie rare existe qu’une allégation thérapeutique relative à cette pathologie est nécessairement vraie. Nos rêves, nos espoirs et notre imagination sont des attributs précieux de notre humanité, qui peuvent nous mettre en grand danger si nous oublions que les productions intellectuelles associées ne nous sont pas dues par la réalité. Dans le même ordre d’idée, une chose n’est pas nécessairement vraie parce qu’il est envisageable qu’elle le soit.
	\end{tcolorbox}

	\pagebreak
	\subsection{Communication des M\'edias de Masse en Sciences}
	Le lecteur de m\'edias grand public ou de r\'eseaux sociaux ne doit jamais faire confiance à une \'etude scientifique si le document de r\'ef\'erence et revu par les pairs n'est pas donn\'e en lien (et tout en gardant à l'esprit qu'en plus le document en question se doit respecter les règles de publication scientifique que nous avons \'enum\'er\'ees plus bas!). L'\'etude ne doit pas non plus être consid\'er\'ee comme "v\'erit\'e absolue" par le lecteur s'il y a un consensus de la communaut\'e scientifique seulement sur ... UNE SEULE ET UNIQUE ... publication/article\footnote{Gardez à l'esprit que même une pendule cass\'ee affiche l'heure juste deux fois par jour...}. La seule façon d'être \underline{presque} sûr est alors est de lire l'\'etude elle-même et v\'erifier si elle respecte les règles pr\'ec\'edemment \'enum\'er\'ees.

	Un premier exemple typique est une nouvelle qui a \'et\'e faussement et mal reprise par de nombreux m\'edias grand public à travers le monde sur la borr\'eliose de Lyme dont voici une capture d'\'ecran:
	\begin{figure}[H]
		\centering
		\includegraphics[scale=0.25]{img/intro/lyme_borreliose.jpg}
		\caption[Publication de la T\'el\'evision Suisse sur le traîtement de la borr\'eliose de Lyme]{Publication de la TV Suisse à propos du traîtement de la\\ borr\'eliose de Lyme le 2017-01-08 (source: App RTS/ATS)}
	\end{figure}
	En r\'esum\'e ce que le "journaliste scientifique" d'une des principales chaînes nationales suisses (donc une t\'el\'evision qui a assez d'argent pour enquêter correctement sur toute information avant de la relayer ... au moins en th\'eorie ... dans un pays qui estime être le num\'ero un dans presque tout...), a publi\'e est une très mauvaise interpr\'etation de la publication scientifique originale. L'article ci-dessus rapporte que: "\textit{... un traitement appliqu\'e pendant $3$ jours au plus tard $72$ heures après après la morsure de la tique a r\'ev\'el\'e être efficacit\'e de $100\%$...}. La chose est que cet article est fourni par l'Agence T\'el\'egraphique Suisse (et relay\'e par la suite par la t\'el\'evision suisse) qui se targue bêtement d'être $100 \%$ fiable (nous d\'etectons donc un manque de pr\'ecaution scientifique de la part de leurs journalistes ou d'une formation journalistique lacunaire...).
	
	En r\'ealit\'e (si les m\'edias avaient pris soin de lire la publication originale jusqu'à la fin ...) l'\'etude a \'et\'e arrêt\'ee après $8$ semaines et il a \'et\'e d\'emontr\'e que le traitement n'a pas de meilleur effet qu'un placebo...
	
	Une deuxième erreur typique et r\'ecurrente des m\'edias grand public est le biais de confirmation (nous verrons l'\'etude des biais plus tard) dont voici un exemple lassant et honteux tellement il se r\'epète (à croire qu'ils font exprès...):
	\begin{figure}[H]
		\centering
		\includegraphics[scale=0.25]{img/intro/miracle_lourdes.jpg}
		\caption[Publication de la T\'el\'evision Suisse sur un miracle à Lourdes]{Publication de la TV Suisse à propos d'un miracle à\\ Lourdes le 2018-02-12 (source: App RTS/AFP)}
	\end{figure}
	Bien \'evidemment n'importe quelle personne ayant un minimum de culture g\'en\'erale peut v\'erifier assez simplement via des m\'eta-analyses existantes que des "miracles\footnote{Un "miracle" est le terme utilis\'e par de nombreuses personnes lorsqu'elles ne savent pas expliquer un ph\'enomène observ\'e. Les \'eclipses \'etaient un exemple de "miracle" il n'y a pas si longtemps... Si quelque chose peut arriver et se produit, ce n'est guère miraculeux. Donc, un miracle doit être quelque chose qui ne peut pas arriver. Mais, si un miracle se produit, alors clairement, cela peut arriver. Et donc ce n'était pas un miracle au départ. Donc, si la vie est pleine de miracles, alors vous ne comprenez pas la Nature et les probabilités! Les miracles ont toujours été l'écueil des ignorants et l'asile des ambitieux...}" ont aussi lieu dans les hôpitaux et qu'en termes de "taux" de r\'emission, Lourdes ne fait pas mieux que le simple hasard en comparaison aux hôpitaux dispers\'es sur tout le planète relativement à ce type d'observation.
	
	Il est donc à nouveau honteux qu'une des principales chaînes nationales suisses (donc une t\'el\'evision qui a assez d'argent pour enquêter correctement sur toute information avant de la relayer ... au moins en th\'eorie ... dans un pays qui estime être le num\'ero un dans presque tout...) ait publi\'e une information biais\'ee et c'est d'autant plus grave que ce type d'information vient de l'AFP (Agence France Presse).
	
	\begin{center}
		\includegraphics[width=1.0\textwidth]{img/intro/deformation_medias.jpg}
	\end{center}

	\subsubsection{R\'eseaux sociaux}		
	À propos des r\'eseaux sociaux et de la communication scientifique... A priori on pourrait constater aussi bien sur Facebook, YouTube, Twitter, Instagram et TikTok que lors d'\'echanges:
	\begin{itemize}
		\item Si on simplifie le discours scientifique (en pensant bien faire...) sur certains sujets d\'elicats\footnote{Sujets où typiquement les gens vous diront que: "les faits sont d\'ementis par leur opinion"...} on peut se faire assez vite accuser de d\'eformer la r\'ealit\'e (c'est le problème effectivement de la simplification...!). Et si vous pr\'ecisez que vous avez simplifi\'e pour la compr\'ehension des illettr\'es scientifiques, vous serez probablement accus\'e d’attaque ad hominem. C'est une situation par la suite où il est difficile voire impossible de r\'etablir la confiance.
	
		\item Si on ne simplifie pas le discours scientifique (en utilisant le vocabulaire et les m\'ethodes quantitative du domaine), ou qu'on communique les liens vers les \'etudes ou th\'eories scientifiques elles-mêmes\footnote{A noter qu'il semble arriver r\'egulièrement que lorsque des liens vers les \'etudes ou m\'eta-analyses sont fournies, il y a presque toujours des personnes pour dire que soient elles sont financ\'ees par des lobbistes, soient elles ont \'et\'e choisies dans le sens des arguments d\'efendus, soient il n'y pas tous les liens vers toutes les \'etudes du monde et que pour le coup elles ne sont pas repr\'esentatives...}, on se fait accuser de cacher la v\'erit\'e sous un vocabulaire abscont et des termes et outils techniques n'ayant ni queue ni tête ou de faire dire aux statistiques ce qu'on l'on veut. Le r\'esultat est encore pire lorsque l'accès aux \'etudes est payant!
	
		\item Si on parle d'un sujet sur lequel on pas d'expertise ou de diplôme, ou qui n'est pas notre domaine d'activit\'e, on se fait remballer rapidement (à juste titre!)
	
		\item Sur certaines groupes et comptes de r\'eseaux sociaux, certains messages sont supprim\'es par les administrateurs, ce qui ne donne plus la possibilit\'e d'\'etayer des arguments ou contre-arguments ou biaise compl\'etement les \'echanges car certaines informations disparaîssent  ou n'apparaîssent jamais (certains intervenants ou messages sont typiquement masqu\'es ou supprim\'es par l'administrateur).
		
		\item Un petit pourcentage de personnes sont très bien éduquées mais peuvent être biaisées car elles sont endoctrinées depuis l'enfance et même pire (car assez difficile à détecter) ... certains ne sont que des trolls (avec un niveau scolaire de troisième cycle) qui aiment provoquer les gens sur les réseaux sociaux juste pour le plaisir de lire les réponses ou ... pour le plaisir d'analyser par curiosité les réponses des gens à leurs trollages.
		
		\item Statistiquement, $2.2\%$ des utilisateurs de réseaux sociaux ouverts comme Facebook, TikTok, Snapchat, Instagram (cela exclut automatiquement les réseaux sociaux comme ResearchGate évidemment!) ont un QI inférieur à $70$ points (et même certains avec un QI plus élevé souffrent d'une sorte de pédomorphose psychologique ou d'endoctrinement infantile). Dans un pays comme la France qui compte en 2021 quarante millions d'utilisateurs actifs quotidiens, cela fait $880'000$ utilisateurs... et donc pas mal de gens sans instruction, illétrés scientifiquement et non rationnels (sans oublier qu'il faut prendre en compte le fait que ce type de personnes est la majorité du temps sans emploi et a donc beaucoup plus à consacrer aux réseaux sociaux que les personnes en possession d'un Doctorat).
	
		\item Gardez finalement en-tête que la totalit\'e des r\'eseaux sociaux ne supportent par \LaTeX{}, il est donc impossible d'y avoir des \'echanges scientifiques (c'est-à-dire faisant usage d'\'equations math\'ematiques ou formules chimiques).
	\end{itemize}
	Le but ici n'est pas de donner une solution scientifique à ces problèmes (ce serait toutefois pertinent que des \'etudes soient men\'ees sur le sujet...!). Toutefois... des pistes qui semblent assez bien fonctionner sont de bloquer les commentaires sur les contenus publi\'es par les scientifiques ou par la communaut\'e scientifique. D'intervenir dans des \'echanges que si et seulement si on y est invit\'e à y communiquer (et non pas d'y intervenir de son propre chef).
	\begin{center}
		\includegraphics[scale=0.18]{img/intro/opinions.jpg}
	\end{center}
	Citons enfin pour clore les techniques fallacieuses d'argumentation courantes particulièrement flagrantes sur les r\'eseaux sociaux (et dans les autres m\'edias en g\'en\'eral aussi...) de la part de ceux qui sont imperm\'eables à la m\'ethode scientifique et aux analyses et simulations statistiques et qui s'inspirent volontairement (ou pas?) des principes de la propagande de guerre que l'historienne Anne Morelle a \'enonc\'es:
	\begin{enumerate}
		\item Nous ne voulons pas la guerre (ie nous ne voulons pas le "changement")
		
		\item Le camp adverse est le seul responsable de la guerre (c'est lui qui force le "changement non-naturel")
		
		\item Le chef du camp adverse a le visage du diable (ou "l'affreux de service")
		\item C'est une cause noble que nous d\'efendons (par exemple un "service public") et non des int\'erêts particuliers
		
		\item Le camp adverse provoque sciemment des atrocit\'es (meurtres, licenciements, d\'ecès,...) et si nous nous commettons des erreurs c'est involontairement
		
		\item Le camp adverse  utilise des armes non autoris\'ees (ie nous ne comprenons pas les arguments de l'autre)
		
		\item Nous subissons très peu de pertes, les pertes du camp adverse sont \'enormes (le système actuel est bon il ne faut pas le changer car il sera pire)
		
		\item Les artistes et intellectuels soutiennent notre cause
		
		\item Notre cause a un caractère sacr\'e (chercheur de v\'erit\'e, service public, etc.)
		
		\item Ceux (et celles) qui mettent en doute notre propagande sont des traîtres (ie ils mettent en pu\'eril la coh\'esion sociale, la coh\'esion ou nationale, cherchent à faire du profit sur les plus pauvres, etc.)
	\end{enumerate}
	Il s'agit généralement de circonstances dites de "post-vérité" telles que définies par le dictionnaire Oxford\footnote{Post-vérité: Fait référence à des circonstances dans lesquelles des faits objectifs influencent moins l'opinion publique que les appels à l'émotion et aux croyances personnelles.}.
	
	On peut également ajouter à cette liste le fameux: \og \textit{J'ai fait mes propres recherches} \fg{}. Les individus utilisant cet argument on souvent fait 200 heures de recherche sur Twitter, YouTube et des blogs amateurs pour trouver des preuves scientifiques en éliminant soigneusement les évidences qui allaient à l'encontre de leurs croyances, et ce en ignorant en plus les méta-analyses et ce sans les compétences techniques pour comprendre les protocoles scientifiques et les indicateurs statistiques (littéralement, ils récupèrent les premiers résultats sur Google en pensant que n'importe quel texte sur Internet ou dans un livre de fiction sacré a une valeur scientifique...).
	\begin{center}
		\includegraphics[width=0.7\textwidth]{img/intro/mes_recherches.jpg}
	\end{center}
	Il ne faut pas s’étonner, si le monde est rempli d’opinions vaines et ridicules, rien n’étant plus capable de leur donner cours, que l’ignorance!
	
	De plus, gardons à l'esprit "\NewTerm{l'effet Dunning-Kruger}\index{effet Dunning-Kruger}\label{Dunning-Kruger effect}" qui est un type de biais cognitif dans lequel les gens croient qu'ils sont plus intelligents et plus capables qu'ils ne le sont vraiment. Essentiellement, les personnes à faibles capacités ne possèdent pas les compétences nécessaires pour reconnaître leur propre incompétence.
	\begin{center}
		\includegraphics[width=0.8\textwidth]{img/intro/dunning_kruger_effect.jpg}
	\end{center}
	\begin{fquote}[Charles Darwin]L'ignorance engendre plus souvent la confiance que la connaissance : ce sont ceux qui savent peu, non ceux qui savent beaucoup, qui affirment si positivement que tel ou tel problème ne sera jamais résolu par la science.
 	\end{fquote}
	
	\pagebreak
	\pagebreak
	\subsubsection{Opinions d'Experts}
	Nous devons également être prudents avec la méthode choisie par les médias traditionnels pour interviewer des "experts".

	En effet, en science, cela n'a pas vraiment de sens d'inviter un expert à parler surtout si ce dernier:
	\begin{itemize}
		\item Utilise des arguments sans fournir d'évidences détaillées (nom de la méta-analyse évaluée par les pairs)
		
		\item Ne fait pas la différence entre  "opinions" et "évidences scientifiques"
		
		\item Parle de sujets en dehors de son domaine de spécialisation
		
		\item Ne travaille plus depuis de nombreuses années dans les laboratoires
		
		\item Utilise ses récompenses et ses livres pour se donner une position d'expert légitime
		
		\item Utiliser le travail de son équipe de laboratoire pour se promouvoir
		
		\item Est seul à faire un monologue\footnote{Les gens ne doivent pas faire confiance aux monologues - que ce soit à la radio, à la télévision ou sur les réseaux sociaux - car il n'y a pas d'experts pour contrer les éventuels mauvais arguments ou concepts mal définis qui peuvent conduire une partie de l'auditoire à de fausses interprétations spéculatives! De plus, les humains sous le stress de savoir qu'ils sont enregistrés sont naturellement sujets aux erreurs de vocabulaire et c'est sans compter les plus de 200 biais cognitifs du cerveau qui conduisent parfois à la simplification erronée de pensées complexes ...!} (typique de TEDx comme déjà mentionné)
		
		\item Est interviewé par un journaliste illétré scientifiquement
		
		\item Est plein de certitudes (les idiots sont pleins de certitudes, les gens intelligents pleins de doutes)
		
		\item À une maladie mentale (à ne pas confondre avec une maladie physique)
	\end{itemize}
	Il y a beaucoup d'exemples célèbres (comme certains prix Nobel qui sont devenus obsédés par des sujets irrationnels en dehors de leur domaine de compétence). Cependant, donnons deux exemples que nous rencontrerons plus tard dans ce livre dans différentes sections.
	
	Le premier exemple est celui de l'équipe de Nosek qui a invité des chercheurs à participer à un projet d'analyse de données. La configuration était simple. Les participants ont tous reçu le même ensemble de données et la même question: les arbitres de football donnent-ils plus de cartons rouges aux joueurs à la peau foncée qu'aux joueurs à la peau claire? Ils ont ensuite été invités à soumettre leur approche analytique pour obtenir les commentaires des autres équipes avant de se plonger dans l'analyse.
	
	Vingt-neuf équipes avec un total de $61$ analystes y ont participé. Les chercheurs ont utilisé une grande variété de méthodes, allant - pour ceux d'entre vous intéressés par le gore méthodologique - des techniques de régression linéaire simples aux régressions multiniveaux complexes et aux approches bayésiennes. Ils ont également pris différentes décisions concernant les variables secondaires à utiliser dans leurs analyses.

	Malgré l'analyse des mêmes données, les chercheurs ont obtenu une variété de résultats. Vingt équipes ont conclu que les arbitres de football donnaient plus de cartons rouges aux joueurs à la peau foncée, et neuf équipes n'ont trouvé aucune relation significative entre la couleur de la peau et les cartons rouges.
	\begin{figure}[H]
		\centering
		\includegraphics[width=0.8\textwidth]{img/arithmetics/repetability.jpg}
		\caption{Mêmes données, conclusions différentes (objectifs des méta-analyses)}
	\end{figure}
	La variabilité des résultats visible ci-dessus n'est pas due à une quelconque fraude ou à un travail bâclé. Ce sont des analystes très compétents qui étaient motivés pour trouver la vérité, a déclaré Eric Luis Uhlmann, psychologue à l'école de commerce Insead à Singapour et l'un des chefs de projet. Même les chercheurs les plus qualifiés doivent faire parfois des choix subjectifs qui ont un impact énorme sur le résultat qu'ils trouvent. C'est pourquoi, encore une fois, il est tout à fait insensé de ne faire parler qu'un seul expert dans les médias grand public dans un monologue ...
	
	Le deuxième exemple récent (évidemment c'est un cas particulier qui ne doit pas être généralisé!) a été la propagation du COVID-19 au premier trimestre de l'année 2020 aux USA. Il a été demandé à quelques experts américains quel est leur pronostic pour le nombre de cas positifs de COVID-19 au 29 mars aux USA. Cela a été résumé dans la figure suivante:
	\begin{figure}[H]
		\centering
		\includegraphics[width=1\textwidth]{img/intro/covid19_experts.jpg}
		\caption{Problèmes d'opinions/estimations d'experts}
	\end{figure}
	Pour information le nombre de cas positifs connus au 29 mars aux USA aura été en fait un peu supérieur à $100'000$ . C'est là encore un bon exemple pour lequel une opinion ou une estimation d'un experts unique - quel que soit le domaine - n'est pas forcément toujours très fiable.
	
	Évidemment, le lecteur doit garder à l'esprit que la variabilité existe dans les deux exemples ci-dessus car les scientifiques n'étaient pas autorisés à utiliser la "\NewTerm{méthode Delphes}\index{méthode de Delphes}". Fondamentalement, la méthode Delphes est un processus utilisé pour parvenir à une opinion ou à une décision de groupe en interrogeant un panel d'experts. Les experts répondent à plusieurs séries de questionnaires, et les réponses sont agrégées et partagées avec le groupe après chaque série à l'aide de techniques statistiques (nous reviendrons sur cette technique dans la section Théorie des Jeux et de la Décision page \pageref{Delphi method}).
	
	%to make section start on odd page
	\newpage
	\thispagestyle{empty}
	\mbox{}
	\section{Vocabulaire}
	\lettrine[lines=4]{\color{BrickRed}L}a physique-math\'ematique, comme tout domaine de sp\'ecialisation, a son vocabulaire propre. Afin que le lecteur ne soit pas perdu dans la compr\'ehension de certains textes qu'il pourra lire sur ce site (et son PDF associ\'e), nous avons choisi d'exposer ici les quelques termes, abr\'eviations et d\'efinitions fondamentaux à connaître. 

	Ainsi, le math\'ematicien aime bien terminer ses d\'emonstrations (quand il pense qu'elles sont justes) par l'abr\'eviation "C.Q.F.D" qui signifie "Ce Qu'il Fallait D\'emontrer" ou encore dans les hautes \'ecoles par souci d'esth\'etisme et de traditions certains professeurs (et mêmes \'elèves) notent cela en latin "Q.E.D" qui signifie "Quod Erat Demonstrandum" (cela en jette...).

	Et lors de d\'efinitions (elles sont nombreuses en math\'ematique et physique...) le scientifique fait souvent usage des terminologies suivantes:
	
	\begin{itemize}
	\item ... il suffit que  ...
	
	\item ... si et seulement si ...
	
	\item ... n\'ecessaire et suffisant ...
	
	\item ... signifie que ...
	
	\item ... prouve que ...
	\end{itemize}
	Les cinq ne sont pas \'equivalentes (identiques au sens strict). Car "il suffit que" correspond à une condition suffisante, mais pas à une condition n\'ecessaire. Il faut aussi noter que ces cinq terminologies doivent être planc\'es dans le contexte de l'analyse des donn\'ees, de l'exactitude des donn\'ees, de la reproduction et de l'\'evaluation par les pairs et non sur une croyance personnelle ou commune ou même \'emotionnelle d'un groupe de personnes (même si ce groupe à une taille de plusieurs milliards d'individus...)!
	
	De plus, il est peut être pertinent de noter que de nombreuses discussions ou d\'ebats dans la vie en g\'en\'eral (en priv\'e ou en public dans les m\'edias) sont souvent st\'eriles juste par le fait que le vocabulaire de base utilis\'e, les hypothèses de travail ou de raisonnement, ou l'objectif du d\'ebat (et des questions y relatives) n'ont pas \'et\'e correctement d\'efinis dès le d\'ebut. Même si cela est acceptable pour le citoyen lambda, ce type de situations n'est pas acceptable en science!
	\begin{center}
		\includegraphics[scale=0.30]{img/intro/an_old_age_argument.jpg}
	\end{center}

	\subsection{Sur les "sciences"}	
	Il est important que nous d\'efinissions rigoureusement les diff\'erents types de sciences auxquelles l'être humain fait souvent r\'ef\'erence. Effectivement, il semble qu'au 21ème siècle un abus de langage malsain s'instaure et qu'il ne devienne plus possible pour la population de distinguer la "qualit\'e intrinsèque" d'une science d'une autre.

	\begin{tcolorbox}[title=Remarque,colframe=black,arc=10pt]
	Etymologiquement le mot "science" vient du latin "scienta" (connaissance) dont la racine est le verbe "scire" qui veut dire "savoir".
	\end{tcolorbox}
	
	Cet abus de langage vient probablement du fait que les sciences pures et exactes perdent leurs illusions d'universalit\'e et d'objectivit\'e, dans le sens où elles s'auto-corrigent. Ceci ayant pour cons\'equence que certaines sciences sont rel\'egu\'ees au second plan et tentent d'en emprunter les m\'ethodes, les principes et les origines pour cr\'eer une confusion. Il faut ainsi être très prudent au sujet des pr\'etentions de scientificit\'e en sciences humaines, et cela vaut \'egalement (ou surtout) pour les courants dominants en \'economie, en sociologie et en psychologie. Tout simplement, les problèmes trait\'es par les sciences humaines sont extrêmement complexes, peu reproductibles, et les arguments empiriques \'etayant leurs th\'eories sont souvent assez faibles.

	\marginnote{\textcolor{NavyBlue}{{\footnotesize \textbf{~\thechapter:\myparagraph}}}}En soi, la science cependant ne produit pas de v\'erit\'e absolue. Par principe, une th\'eorie scientifique est valable tant qu'elle permet de pr\'edire des r\'esultats mesurables statistiquement et reproductibles\footnote{Donc, ce que nous pouvons lire dans tous les différents livres religieux existant dans le monde, ce ne sont pas des "théories scientifiques" mais des "théories spéculatives"!}. Mais les problèmes inhérents aux interpr\'etations des données que certaines veulent faire de ces r\'esultats statistiques font partie de la philosophie naturelle.
	
	\begin{center}
		\NewTerm{\textbf{Aucune th\'eorie scientifique n'est prouv\'ee ou prouvable. Elle n'est simplement pas r\'efut\'ee tant qu'une exp\'erience ne vient pas dire le contraire.}}
	\end{center}
	Cependant, la m\'ethodologie scientifique est suffisamment fiable pour que le pouvoir juridique ne soit pas l\'egitime à prendre position sur des \'evidences scientifiques.

	\'etant donn\'e la diversit\'e des ph\'enomènes à \'etudier, au cours des siècles s'est constitu\'e un nombre grandissant de disciplines comme la chimie, la biologie, la thermodynamique, etc. Toutes ces disciplines a priori h\'et\'eroclites ont pour socle commun la physique, pour langage la math\'ematique et comme principe \'el\'ementaire la m\'ethode scientifique.
	\begin{tcolorbox}[title=Remarque,colframe=black,arc=10pt]
	Par cons\'equent, gardez à l'esprit qu'un t\'emoignage, ou une simple phrase dans un livre ou dans un s\'eminaire même dite par un scientifique, n'a AUCUNE VALEUR SCIENTIFIQUE ou n'a aucune valeur pour tirer des conclusions, s'il n'est pas accompagn\'e de donn\'ees exp\'erimentales et d'un modèle math\'ematique d\'etaill\'e! Cependant, les t\'emoignages restent utiles pour construire des hypothèses sp\'eculatives et concevoir des exp\'eriences et des \'etudes. Si les hypothèses sp\'eculatives ne sont cependant pas accompagn\'ees d'une m\'ethode exp\'erimentale pour la v\'erifier ou la r\'efuter, alors ce n'est toujours pas de le science mais juste de la pseudo-science de comptoir!!!
	\end{tcolorbox}
	Ainsi, un petit rafraîchissement de m\'emoire peut être utile:

	\textbf{D\'efinitions (\#\mydef):}
	
	\begin{itemize}
		\item[D1.] Nous d\'efinissons par "\NewTerm{science pure}"\index{science pure}, tout ensemble de connaissances fond\'ees sur un raisonnement rigoureux valable quel que soit le facteur (arbitraire) \'el\'ementaire choisi (nous disons alors "ind\'ependant de la r\'ealit\'e sensible") et restreint au minimum n\'ecessaire. Il n'y a que la math\'ematique (appel\'ee souvent "reine des sciences") qui peut être classifi\'ee dans cette cat\'egorie.
	
		\item[D2.] Nous d\'efinissons par "\NewTerm{science exacte}"\index{science exacte} ou "\NewTerm{science dure}"\index{science dure}, tout ensemble de connaissances fond\'ees sur l'\'etude d'une observation, observation qui aura \'et\'e transcrite sous forme symbolique (physique th\'eorique par exemple). Principalement, le but des sciences exactes est non d'expliquer le "pourquoi" mais le "comment".
		
		Et n'oubliez jamais ... La science (en particulier la physique) n'a pas à "faire sens", elle doit juste faire toutes les bonnes pr\'edictions testables (instrumentalisme)! Selon le philosophe Karl Popper, une th\'eorie est scientifiquement acceptable si, comme pr\'esent\'ee, elle peut être "\NewTerm{falsifiable}\index{falsiable}\footnote{C'est pourquoi la philosophie et la logique humaine naïve elles-mêmes ne peuvent rien "prouver". Parce qu'elles dépendent fortement et varient d'un contexte socioculturel où les gens sont nés à un autre (les gens de New-York ont une logique et une philosophie assez différentes de la tribu des Sentinelles...). Sans la falsification, la Science serait une anarchie de modèles logiquement cohérents mais toutefois inutiles et qui conviendraient simplement à la fantaisie de quelqu'un.}" (les synonymes sont "\NewTerm{r\'efutable}\index{r\'efutable}" ou "\NewTerm{testable}\index{testable}"), c'est-à-dire qu'elle peut être soumises à des tests exp\'erimentaux (ou s'il est possible de concevoir une observation ou un argument qui nie l'\'enonc\'e en question). La "connaissance scientifique" est alors par d\'efinition l'ensemble des th\'eories qui ont r\'esist\'e à la falsification (r\'efutation). La science est donc de par sa nature sujette à un questionnement continu.
		
		Attention! Il n'y a aucun doute que les sciences exactes jouissent pour l'instant d'un prestige \'enorme, y compris parmi leur d\'etracteurs, à cause de leurs succès th\'eoriques et pratiques. Il est certain que certains scientifiques abusent parfois de ce prestige en exhibant un sentiment de sup\'eriorit\'e non n\'ecessairement justifi\'e. De plus, il arrive assez souvent que des scientifiques exposent, dans la litt\'erature de vulgarisation, des id\'ees fort sp\'eculatives comme si elles \'etaient bien \'etablies, ou extrapolent leurs r\'esultats en dehors du contexte où ils ont \'et\'e v\'erifi\'es (et encore... à condition qu'elles aient \'et\'e v\'erifi\'ees un jour...).
	
		\begin{tcolorbox}[title=Remarque,colframe=black,arc=10pt]
		Les deux d\'efinitions pr\'ec\'edentes sont souvent incluses dans la d\'efinition de "\NewTerm{sciences d\'eductives}"\index{sciences d\'eductives} ou encore de "\NewTerm{sciences ph\'enom\'enologiques}"\index{sciences ph\'enom\'enologiques}.
		\end{tcolorbox}
		
		\item[D3.] Nous d\'efinissons par "\NewTerm{science de l'ing\'enieur}"\index{science de l'ing\'enieur}, tout ensemble de connaissances th\'eoriques ou pratiques appliqu\'ees aux besoins de la soci\'et\'e humaine tels que: l'\'electronique, la chimie, l'informatique, les t\'el\'ecommunications, la robotique, l'a\'erospatiale, biotechnologies...
	
		\item[D4.] Nous d\'efinissons par "\NewTerm{science}"\index{science} tout ensemble de connaissances fond\'ees sur des \'etudes ou observations de faits dont l'interpr\'etation n'a pas encore \'et\'e retranscrite ni v\'erifi\'ee avec la rigueur math\'ematique, caract\'eristique des sciences qui pr\'ecèdent, mais qui applique des raisonnements comparatifs statistiques. Nous incluons dans cette d\'efinition: la m\'edecine (il faut cependant prendre garde au fait que certaines parties de la m\'edecine \'etudient des ph\'enomènes descriptifs sous forme math\'ematique tels que les r\'eseaux de neurones ou autres ph\'enomènes associ\'es à des causes physiques connues), la sociologie, la psychologie, l'histoire, la biologie...
		
		\item[D5.] Nous définissons un "\NewTerm {scientifique}"\index{scientifique} (dans le sens moderne du terme) comme un professionnel (un actif et donc non un retraité!) qui détient un Doctorat et qui recueille et utilise systématiquement des recherches et des preuves expérimentales reproductibles, pour faire des hypothèses et les tester à l'aide de méthodes scientifiques de pointe (des méthodes statistiques de niveau PhD ou des simulations numériques reproductibles avancées), pour acquérir et partager une compréhension et des connaissances évaluées par les pairs dans des revues scientifiques de référence dans un domaine d'expertise très spécialisé.

		La définition exclut donc l'immense majorité des ingénieurs (qui ne publient pas des articles évalués par les pairs et ne savent pas analyser les données à l'aide de méthodes statistiques de niveau Doctorat), cela exclut les assistants de laboratoire qui ne font que des manipulations expérimentales mais ne font pas d'hypothèses ni analysent les mesures résultantes de leurs expériences mais aussi les médecins (voir \cite{smith2004doctors} et \cite{freed2004doctors}) qui n'appliquent que les résultats de la recherche scientifique mais ne font pas de recherche comme spécifié ci-dessus. Les mathématiciens ne sont pas exclus de cette définition car leurs théories mathématiques (preuves) sont publiées dans des articles évalués par des pairs et sont reproduites (vérifiées) indépendamment par d'autres mathématiciens.
	
		\item[D6.] Nous d\'efinissons par "\NewTerm{science molle}"\index{science molle}, "\NewTerm{para-science}"\index{para-science} ou "\NewTerm{pseudo-science}"\index{pseudo-science} tout ensemble de connaissances ou de pratiques qui sont actuellement bas\'ees sur des faits non v\'erifiables et non r\'efutables (non reproductibles scientifiquement) par l'exp\'erience ou par les math\'ematiques. Nous incluons typiquement dans cette d\'efinition: l'astrologie, la th\'eologie, le paranormal (qui a \'et\'e d\'emoli par la science z\'et\'etique), la graphologie, la justice \footnote{En effet, en Suisse, par exemple, le juge cantonal et le juge f\'ed\'eral ne donnent pas le même jugement puisque ce dernier est non scientifique mais plutôt bas\'e sur l'exp\'erience subjective de la vie du juge et de ses biais cognitifs}, etc.
		
		Comme le disent certains scientifiques: «\textit{Cela ressemble à de la science, cela utilise le vocabulaire de la science ... mais ce n'est pas du tout de la science.}»
		
		Les pseudo-sciences sont particulièrement caract\'eris\'ees par:
		\begin{itemize}
			\item Elles commencent par une conclusion (croire), puis travaillent en arrière pour essayer de confirmer les croyances
	
			\item Elles sont hostiles à la critique et la remise en question
	
			\item Elles utilisent des raisonnements / arguments circulaires
	
			\item Elles utilisent un jargon vague pour cr\'eer de la confusion
	
			\item Elles utilisent des strat\'egies subtiles pour changer l'esprit des gens (en particulier les enfants)
	
			\item Elles font de la "cueillette des cerises" sur des preuves favorables
	
			\item Elles utilisent des m\'ethodes non reproductibles / non r\'efutables avec des r\'esultats non r\'ep\'etables
	
			\item Elles utilisent un langage al\'eatoire pseudo-scientifique pour impressionner le public
	
			\item Elles utilisent une logique incoh\'erente (\'emotionnelle) et invalide
	
			\item Les gens qui travaillent dans le domaine sont dogmatiques et inflexibles
		\end{itemize}
	
		\item[D7.] Nous d\'efinissons par "\NewTerm{sciences ph\'enom\'enologiques}" ou "\NewTerm{sciences naturelles}", toute science qui n'est pas inclue dans les d\'efinitions pr\'ec\'edentes (histoire, sociologie, psychologie, zoologie, biologie,...)
	
		\item[D8.] Le "\NewTerm{scientisme}"\index{scientisme} est une id\'eologie selon laquelle la science exp\'erimentale est le seul mode de connaissance valable, ou, du moins, sup\'erieur à toutes les autres formes d'interpr\'etation du monde. Dans cette perspective, il n'existe pas de v\'erit\'es philosophiques, religieuses ou morales sup\'erieures aux th\'eories scientifiques. Seul compte ce qui est scientifiquement d\'emontr\'e.
	
		\item[D9.] Le "\NewTerm{positivisme}"\index{positivisme} d\'esigne un ensemble de courants qui considère que seules l'analyse et la connaissance des faits r\'eels v\'erifi\'es par l'exp\'erience peuvent expliquer les ph\'enomènes du monde sensible. La certitude en est fournie exclusivement par l'exp\'erience scientifique. Il rejette l'introspection, l'intuition et toute approche m\'etaphysique pour expliquer la connaissance des ph\'enomènes.\\
		
		Ce qui est int\'eressant dans cette doctrine, c'est que c'est certainement une des seules qui demande aux gens de devoir r\'efl\'echir par eux-mêmes et de comprendre l'environnement qui les entoure en remettant continuellement tout en question et sans ne jamais rien accepter comme acquis (...). De plus, les vraies sciences ont ceci d'extraordinaire qu'elles permettent de comprendre au-delà de ce que nous pouvons voir.
	\end{itemize}

Mais enfin, la science, c'est la science, et rien de plus: une certaine mise en ordre, pas trop mal r\'eussie, des choses qui ne conduisent plus à la m\'etaphysique comme du temps d'Aristote, mais qui n'a pas le pr\'etention de nous livrer toute la r\'ealit\'e ni même le fond des choses visibles et cela même si c'est la meilleure m\'ethode d'investigation intellectuelle existante actuellement à notre \'epoque et aussi depuis plusieurs mill\'enaires.

	\pagebreak
	\subsection{Terminologie}

Le tableau m\'ethodique que nous avons pr\'esent\'e plus haut contient des termes qui peuvent peut-être vous sembler inconnus ou barbares. C'est la raison pour laquelle il nous semble fondamental de pr\'esenter les d\'efinitions de ces derniers, ainsi que de quelques autres tout aussi importants qui peuvent \'eviter des confusions malheureuses.

\textbf{D\'efinitions (\#\mydef):}

\begin{itemize}
	\item[D1.] Au-delà de son sens n\'egatif, l'id\'ee de "\NewTerm{problème}"\index{problème} renvoie à la première \'etape de la d\'emarche scientifique. Formuler un problème est ainsi essentiel à sa r\'esolution et permet de comprendre correctement ce qui fait problème et de voir ce qui doit être r\'esolu.\\
	
	Le concept de problème est intimement reli\'e au concept "d'hypothèse" dont nous allons voir la d\'efinition ci-dessous.

	\item[D2.] Une "\NewTerm{hypothèse}"\index{hypothèse} est toujours, dans le cadre d'une th\'eorie d\'ejà constitu\'ee ou sous-jacente, une supposition en attente de confirmation ou d'infirmation qui tente d'expliquer un groupe de faits ou de pr\'evoir l'apparition de faits nouveaux.\\
	
	Ainsi, une hypothèse peut être à l'origine d'un problème th\'eorique qu'il faudra formellement r\'esoudre. 

	\item[D3.] Le "\NewTerm{postulat}"\index{postulat} en physique correspond fr\'equemment à un principe (voir d\'efinition ci-dessous) dont l'admission est n\'ecessaire pour \'etablir une d\'emonstration (nous sous-entendons que cela est une proposition non-d\'emontrable).\\
	
	Ainsi, une hypothèse peut être à l'origine d'un problème th\'eorique qu'il faudra formellement r\'esoudre. 

	\item[D4.] Un "\NewTerm{principe}"\index{principe} (parent proche du "postulat") est donc une proposition admise comme base d'un raisonnement ou une règle g\'en\'erale th\'eorique qui guide la conduite des raisonnements qu'il faudra effectuer. En physique, il s'agit \'egalement d'une loi g\'en\'erale r\'egissant un ensemble de ph\'enomènes et v\'erifi\'ee par l'exactitude de ses cons\'equences.\\
	
	Le mot "principe" est utilis\'e avec abus dans les petites classes ou \'ecoles d'ing\'enieurs par les professeurs ne sachant (ce qui est très rare), ou ne voulant (plutôt fr\'equent), ou ne pouvant faute de temps (quasi exclusivement) d\'emontrer une relation.\\

	L'\'equivalent du postulat ou du principe en math\'ematiques est "l'axiome" que nous d\'efinissons ainsi:

	\item[D5.] Un "\NewTerm{axiome}"\index{axiome} est une v\'erit\'e ou proposition \'evidente par elle-même dont l'admission est n\'ecessaire pour \'etablir une d\'emonstration.\label{axiom} 
\end{itemize}

	\begin{tcolorbox}[title=Remarques,colframe=black,arc=10pt]
	\textbf{R1.} Nous pourrions dire que c'est quelque chose que nous posons comme une v\'erit\'e pour le discours que nous nous proposons de tenir, comme une règle du jeu, et qu'elle n'a pas forc\'ement par ailleurs une valeur de v\'erit\'e universelle dans le monde sensible qui nous entoure.\\

	\textbf{R2.} Les axiomes doivent toujours être ind\'ependants (on ne doit pas pouvoir d\'emontrer l'un à partir de l'autre) et non contradictoires (nous disons \'egalement parfois qu'ils doivent être "consistants").
	\end{tcolorbox}	
	
\begin{itemize}
	\item[D6.]  Le "\NewTerm{corollaire}"\index{corollaire} est un terme malheureusement quasi inexistant en physique (à tort !) et qui est en fait une proposition r\'esultant d'une v\'erit\'e d\'ejà d\'emontr\'ee. Nous pouvons \'egalement dire qu'un corollaire est une cons\'equence n\'ecessaire et \'evidente d'un th\'eorème (ou parfois d'un postulat en ce qui concerne la physique).

	\item[D7.] Un "\NewTerm{lemma}"\index{lemma} constitue une proposition d\'eduite d'un ou de plusieurs postulats ou axiomes et dont la d\'emonstration pr\'epare celle d'un th\'eorème.
\end{itemize}

	\begin{tcolorbox}[title=Remarque,colframe=black,arc=10pt]
	Le concept de "lemme" est lui aussi (et c'est malheureux) quasi r\'eserv\'e aux math\'ematiques.
	\end{tcolorbox}	

\begin{itemize}
	\item[D8.] Une "\NewTerm{conjecture}"\index{conjecture} constitue une supposition ou opinion fond\'ee sur la vraisemblance d'un r\'esultat math\'ematique.
	
	Beaucoup de conjectures jouent un rôle un peu comparable à des lemmes, car elles sont des passages oblig\'es pour obtenir d'importants r\'esultats.
	
	\item[D9.] Par-delà son sens faible de conjecture, une "\NewTerm{th\'eorie}"\index{th\'eorie} ou "\NewTerm{th\'eorème}"\index{th\'eorème} est un ensemble articul\'e autour d'une hypothèse et \'etay\'e par un ensemble de faits ou d\'eveloppements qui lui confèrent un contenu positif et rendent l'hypothèse bien fond\'ee (ou tout au moins plausible dans le cas de la physique th\'eorique).

	\item[D10.]  Une "\NewTerm{singularit\'e}"\index{singularit\'e} est une ind\'etermination d'un calcul qui intervient par l'apparition d'une division par le nombre z\'ero. Ce terme est aussi bien utilis\'e en math\'ematique qu'en physique.  

	\item[D11.] Une "\NewTerm{d\'emonstration}"\index{d\'emonstration} constitue un ensemble de proc\'edures math\'ematiques à suivre pour d\'emontrer le r\'esultat d\'ejà connu ou non d'un th\'eorème.

	\item[D12.] Si le mot "\NewTerm{paradoxe}"\index{paradoxe} signifie \'etymologiquement: contraire à l'opinion commune, ce n'est cependant pas par pur goût de la provocation, mais bel et bien pour des raisons solides. Le "\NewTerm{sophisms}"\index{sophisms} quant à lui, est un \'enonc\'e volontairement provocateur, une proposition fausse reposant sur un raisonnement apparemment valide. Ainsi parle-t-on du fameux "paradoxe de Z\'enon", alors qu'il ne s'agit que d'un sophisme. Le paradoxe ne se r\'eduit pas à de la fausset\'e, mais implique la coexistence de la v\'erit\'e et de la fausset\'e, au point qu'on ne parvient plus à discriminer le vrai et le faux. Le paradoxe apparaît alors problème insoluble ou "\NewTerm{aporia}"\index{aporia}. 
	
\end{itemize}

	\begin{tcolorbox}[title=Remarque,colframe=black,arc=10pt]
	Ajoutons que les grands paradoxes, par les interrogations qu'ils ont suscit\'ees, ont fait progresser la science et amen\'e des r\'evolutions conceptuelles de grande ampleur, en math\'ematique comme en physique th\'eorique (les paradoxes sur les ensembles et sur l'infini en math\'ematique, ceux à la base de la relativit\'e et de la physique quantique).
	\end{tcolorbox}	

	%to make section start on odd page
	\newpage
	\thispagestyle{empty}
	\mbox{}
	\section{Science et Foi}
	\lettrine[lines=4]{\color{BrickRed}N}ous verrons qu'en science, une th\'eorie est normalement incomplète, car elle ne peut d\'ecrire exhaustivement la complexit\'e du monde r\'eel (except\'e pour la Physique Quantique ou la Relativit\'e G\'en\'erale). Il en est ainsi de toutes les th\'eories, comme celle du Big Bang ((\SeeChapter{voir section d'Astrophyque page \pageref{astrophysics}}) ou de l'\'evolution des espèces (\SeeChapter{voir sections de Dynamique des Populations page \pageref{population dynamics} ou de Th\'eorie des Jeux et de la D\'ecision page \pageref{game and decision theory}}) ne serait-ce que parce qu'elles ne sont pas reproductibles dans des conditions identiques.
	
	\begin{center}
		\includegraphics[scale=0.3]{img/intro/science_we_trust.jpg}
	\end{center}
	
	Il convient de distinguer diff\'erents courants scientifiques majeurs: 
	\begin{itemize}
		\item Le "\NewTerm{r\'ealisme}"\index{r\'ealisme} est une doctrine où les th\'eories physiques ont pour objectif de d\'ecrire la r\'ealit\'e telle qu'elle est en soi, dans ses composantes inobservables.
	
		\item L'\NewTerm{Instrumentalisme}"\index{instrumentalisme} est une doctrine où les th\'eories sont des outils servant à pr\'edire des observations mais qui ne d\'ecrivent pas la r\'ealit\'e en soi.
	
		\item Le "\NewTerm{fictionalisme}"\index{fictionalisme} est le courant où le contenu r\'ef\'erentiel (principes et postulats) des th\'eories est un leurre, utile seulement pour assurer l'articulation linguistique des \'equations fondamentales.
	\end{itemize}

	Même si aujourd'hui les th\'eories scientifiques ont le soutien de beaucoup de sp\'ecialistes, les th\'eories alternatives ont des arguments valables et nous ne pouvons totalement les \'ecarter. Pour autant, la cr\'eation du monde en 7 jours d\'ecrite par la Bible ne peut plus être perçue comme un possible, et bien des croyants reconnaissent qu'une lecture litt\'erale est peu compatible avec l'\'etat actuel de nos connaissances et qu'il est plus sage de l'interpr\'eter comme une parabole (même s'il est écrit dans leurs propres livre que ce type de d'exercice intellectuel est interdit...). Si la science ne fournit jamais de r\'eponse d\'efinitive, il n'est plus possible de ne pas en tenir compte.
	
	La foi (qu'elle soit religieuse, superstitieuse, pseudo-scientifique ou autre non pilot\'ee par les donn\'ees) a au contraire pour objectif de donner des v\'erit\'es absolues d'une toute autre nature puisqu'elle relève d'une conviction personnelle inv\'erifiable et non reproductibles (par exemple, la science n\'ecessite des preuves ou des \'evidences exp\'erimentales pour être valable alors que les religions n\'ecessitent que la foi pour être consid\'er\'ees comme valable). C'est pourquoi un certain nombre de gens disent que \textit{la Science s'ajuste en fonction des observations alors que la foi est le rejet de l'observation afin que les croyances puissent être conserv\'ees}... En fait, l'une des fonctions des religions est de fournir du sens à des ph\'enomènes qui ne sont pas explicables rationnellement\footnote{Ce fut avec la pluie, le tonnerre, les maladies, les \'etoiles, les comètes, les tremblements de terre, les \'eruptions volcaniques, etc. il y a quelques centaines d'ann\'ees et est souvent d\'esign\'e par les scientifiques sous le nom d'argument d'ignorance. \index{argument d'ignorane}}. Les progrès de la connaissance entraînent donc parois une remise en cause des dogmes religieux par la science\footnote{Quand les théistes disent qu'un dieu doit exister parce que la Terre est "parfaitement" placée pour la vie, ils supposent que leur dieu est limité à placer la vie là où elle pourrait se produire de toute façon naturellement...}. 
	\begin{fquote}La SCIENCE n'est pas définie par ce que nous croyons. Elle N'EST PAS un système de croyance. Elle n'a pas de doctrine religieuse! Il s'agit simplement d'un système intellectuel qui rejette une affirmation qui n'a aucune évidence reproductible expérimentale à l'appui.
 	\end{fquote}
 	\begin{center}
		\includegraphics[scale=0.6]{img/intro/science_like_religion.jpg}
	\end{center}
	A contrario, sauf à pr\'etendre imposer sa foi (qui n'est autre qu'une conviction intimement personnelle et subjective) aux autres, il faut se d\'efier de la tentation naturelle de qualifier de fait scientifiquement prouv\'e les extrapolations des modèles scientifiques au-delà de leur champ d'application.
	
	Le mot "Science" est, comme nous l'avons d\'ejà mentionn\'e plus haut, de plus en plus utilis\'e pour soutenir qu'il existe des preuves scientifiques là où il n'y a que croyance (certaines pages web de ce genre prolifèrent de plus en plus). Selon ses d\'etracteurs c'est le cas, par exemple, du mouvement de scientologie (mais il y en a beaucoup d'autres). Selon ces derniers, nous devrions plutôt parler de "\NewTerm{sciences occultes}"\index{sciences occultes}.

	Les sciences occultes et sciences traditionnelles existent depuis l'Antiquit\'e; elles consistent en un ensemble de connaissances et de pratiques myst\'erieuses ayant pour but de p\'en\'etrer et dominer les secrets de la nature. Au cours des derniers siècles, elles ont \'et\'e progressivement exclues du champ de la science. Le philosophe Karl Popper s'est longuement interrog\'e sur la nature de la d\'emarcation entre science et pseudoscience. Après avoir remarqu\'e qu'il est possible de trouver des observations pour confirmer à peu près n'importe quelle th\'eorie, il propose une m\'ethodologie fond\'ee sur la r\'efutabilit\'e. Une th\'eorie doit selon lui, pour m\'eriter le qualificatif de "scientifique", pouvoir garantir l'impossibilit\'e de certains \'ev\'enements. Elle devient dès lors r\'efutable, donc (et alors seulement) apte à int\'egrer la science. Il suffirait en effet d'observer un de ces \'ev\'enements pour invalider la th\'eorie, et s'orienter par cons\'equent sur une am\'elioration de celle-ci.
	
	Et notons aussi que la diff\'erence majeure entre les livres de sciences et les livres de religions est que si vous d\'etruisiez ces derniers, dans un millier d'ann\'ees, la probabilit\'e qu'ils soient r\'e\'ecrits naturellement à l'identique est très faible. Alors que si nous prenions chaque livre de science et chaque note de chaque observation exp\'erimentale et les d\'etruisions tous, dans mille ans ils seraient refaits de manière identiques et ce pour la simple raison que la majorit\'e des tests exp\'erimentaux redonneraient les mêmes r\'esultats (observations) et que la math\'ematique est ind\'ependante de l'endroit où nous sommes n\'es sur Terre et du choix religieux que nous impose la famille dans la majorit\'e des cas.
	\begin{fquote}La raison pour laquelle il y a un conflit entre la science et les religions est que la science ne cesse de réfuter les choses que les religions prétendent être vraies ou exactes!
	\end{fquote}
	\begin{tcolorbox}[title=Remarque,colframe=black,arc=10pt]
	Si la science et, par-dessus tout, les MATHÉMATIQUES sont le langage d'un quelconque Dieu, alors pourquoi tous les livres saints sont-ils écrits sans mathématiques? Oh, euh ... nous savons pourquoi!
	\end{tcolorbox}
	
	\begin{center}
		\includegraphics[scale=0.55]{img/intro/methode_scientifique_vs_methode_religieuse.jpg}
	\end{center}
	
	Ceci est important car chacune des séries interminables de pseudo-preuves de l'existence de divers dieux qui a été proposée, de l'Antiquité à nos jours, est automatiquement un échec car, comme déjà mentionné, une déduction logique ne nous dit rien qui soit pas déjà intégré dans ses locaux. Tout ce que la logique peut faire pour nous est de tester l'auto-cohérence de ces prémisses. Il n'y a qu'un seul moyen fiable que les humains ont découvert jusqu'à présent pour obtenir des connaissances qu'ils ne possèdent pas déjà: l'observation! Et la science est la collecte méthodique d'observations et la construction et le test de modèles pour décrire ces observations. Sans la falsification, la science serait une anarchie de modèles logiquement cohérents mais toujours inutiles qui conviennent tout simplement à la fantaisie de quelqu'un !
	
	\begin{fquote}Un scientifique lira des centaines de livres au cours de sa vie, mais sera toujours persuadé qu'il lui reste beaucoup à apprendre. Un fanatique religieux n'en lira qu'un et sera persuadé d'avoir tout compris.
 	\end{fquote}
	
	Notons que les religions (et particulièrement les scientifiques croyants) ont eu des milliers d'années pour prouver que n'importe quel dieu existe. Pourtant, les croyants ne peuvent pas faire mieux à ce jour que: \og \textit{VOUS VERREZ QUAND VOUS SEREZ MORT} \fg{}. Cela en dit long ... d'autant plus que si les dieux étaient réels, ils n'auraient pas besoin de mortels pour défendre leur existence! Et si les croyants accordent vraiment plus d'importance à leur foi qu'en la science, ils peuvent le prouver facilement: la prochaine fois qu'ils tombent malades, ils peuvent aller dans leur lieu culte (temple, mosquée ou église) au lieu d'aller à l'hôpital!
	
	\begin{fquote}La religion est l'art d'utiliser des arguments absurdes pour expliquer l'ignorance.
 	\end{fquote}
	
	Le lecteur doit également savoir qu'à l'opposé d'un mythe commun auprès des personnes scientifiquement illétrées, tout phénomène ou hypothèse (même surnaturel!) s'il est bien défini peut être testé par la méthode scientifique. C'est pourquoi nous avons (parmi beaucoup d'autres) des évidences scientifiques que les prières n'ont pas d'effet thérapeutique meilleur que celui d'un placebo. Voir par exemple sur un sujet considéré comme surnaturel l'étude suivante \textit{Study of the Therapeutic Effects of Intercessory Prayer (STEP) in cardiac bypass patients: a multicenter randomized trial of uncertainty and certainty of receiving intercessory prayer} \cite{benson2006study} pour un assez bon protocole et analyse scientifique ou l'article suivant \textit{Positive therapeutic effects of intercessory prayer in a coronary care unit population} \cite{byrd1988positive} pour un mauvais protocole et analyse scientifique. Il existe également quelques méta-analyses sur ce sujet surnaturel...
	
	\begin{fquote} Peu importe si un grand scientifique était religieux. Ce qui compte, c'est qu'aucun d'eux n'a jamais prouvé que son dieu existe! \end{fquote}
	
	Les faits exaspèrent les croyants, qui les poussent au solipsisme. Leur seul moyen de défendre alors leurs croyances qui ne sont pas concordantes avec les données scientifiques est de remettre en cause la réalité elle-même (c'est pourquoi avant de débattre avec eux les scientifiques doivent se mettre d'accord avec eux sur une définition de la "réalité" sinon tout débat est inutile) ! Donc à chaque fois que vous vous disputez avec les croyants, parce qu'ils ne peuvent pas attaquer les faits alors ils attaquent l'épistomologie. Comme c'est comme si leur position était si faible que leur dieu ne pourrait pas être réel à moins que la réalité ne le soit pas...
	
	Enfin citons un passage de \cite{2014traité}:
	
	\og Le peuple (j'entends par ce mot le vulgaire ramassé, la tourbe et lie populaire, gens, sous quelque couvert que ce soit de basse, servile et mécanique condition) est une bête à plusieurs têtes, vagabonde, errante, folle, étourdie, sans conduite, sans esprit, ni jugement. 
	
	Que Postel lui persuade que Jésus-Christ n'a sauvé que les hommes et que la mère Jeanne doit sauver les femmes, il le croira soudain. Que David George se dise fils de Dieu, il l'adorera, qu'un tailleur enthousiaste et fanatique contrefasse le roi dans Munster et dise que Dieu l'a destiné pour châtier toutes les puissances de la Terre, il lui obéira et le respectera comme le plus grand monarque du monde. Que le Père Domptius lui annonce la venue de l'Antéchrist, qu'il est àgé de dix ans [et] qu'il a des cornes, il témoignera de s'en effrayer. Que des imposteurs et charlatans se qualifient Frères de la Rose-Croix, il courra après eux. Qu'on lui rapporte que Paris doit bientôt s'abîmer, il s'enfuira. Que tout le monde doit être submergé, il bâtira des arches et des bateaux de bonne heure pour n'être pas surpris. Que la mer doit sécher et que des chariots pourront aller de Gênes à Jérusalem, il se préparera pour faire le voyage.
	
	Qu'on lui conte les fables de Mélusine, du Sabbat des Sorciers, des Loups-garous, des Lutins, des Fées, des Parèdres, il les admirera. Que la matrice tourmente quelque pauvre fille, il dira qu'elle était possédée, ou croira à quelque prêtre ignorant ou méchant, qui la fait passer pour telle. Que quelque alchimiste, magicien, astrologue, lulliste, cabaliste, commencent un peu à le cajoler, il les prendra pour les plus savants et pour plus honnêtes gens du monde. Qu'un Pierre l'hermite vienne prêcher la croisade, il fera des reliques du poil de son mulet.
Qu'on lui dise en riant qu'une cane ou un oiseau sont inspirés du Saint-Esprit, il le croira sérieusement. Que la peste ou la tempête ruine une province, il en accusera soudain des graisseurs ou magiciens. Bref, si on le trompe aujourd'hui, il se laissera encore surprendre demain, ne faisant jamais profit des rencontres passées, pour se gouverner dans les présentes ou futures; et en ces choses consistent les principaux signes de sa grande faiblesse et imbécibilité.\fg{}
	
	\pagebreak
	\subsection{Kit de D\'etection de Balivernes }
	Grâce à leur formation, la majorit\'e des scientifiques sont \'equip\'es de ce que Carl Sagan appelait "\NewTerm{kit de d\'etection de balivernes }\index{kit de d\'etection de balivernes}" ou "\NewTerm{kit de d\'etection de conneries}\index{kit de d\'etection de conneries}" qui sont des outils cognitifs et des techniques qui fortifient l'esprit contre les arguments fallacieux et biais\'es et qui permettent de d\'efinir des limites entre la science et la pseudoscience ou plus simplement entre le rationnel et l'\'emotionnel. Ce n'est pas seulement un outil de la science, il contient des outils inestimables de scepticisme sain qui s'appliquent tout aussi \'el\'egamment, et tout aussi n\'ecessairement, à la vie quotidienne. En adoptant le kit, nous pouvons tous nous prot\'eger contre la ruse et la manipulation d\'elib\'er\'ee.
	
	Il existe de nombreuses versions de ces outils de d\'etection mais en voici une assez complète (mais encore incomplète par construction) propos\'ee par Michael Shermer (\'editeur fondateur de \href{http://www.skeptic.com}{Skeptic Magazine} et auteur de \textit{The Borderlands of Science}):
	
	\begin{enumerate}[label=\protect\circledbullet{\arabic*},leftmargin=15mm]
		\item \textit{\textbf{Quelle est la fiabilit\'e de la source de l'information?}}

		Les pseudoscientifiques semblent souvent assez fiables, mais lorsqu'on les examine de près, les faits et les chiffres qu'ils citent sont d\'eform\'es, non sourc\'es, pris hors contexte ou parfois même fabriqu\'es. Bien sûr, tout le monde fait des erreurs. Et en tant qu'historien de la science, Daniel Kevles a montr\'e assez efficacement dans son livre \textit{The Baltimore Affair}, qu'il peut être difficile de d\'etecter un signal frauduleux dans un bruit de fond et que la n\'egligence peut alors être parfois consid\'er\'ee comme une partie normale du processus scientifique. La question est: Est-ce que les donn\'ees et les interpr\'etations montrent des signes de distorsion intentionnelle? Lorsqu'un comit\'e ind\'ependant \'etabli pour enquêter sur une fraude potentielle a examin\'e un ensemble de notes de recherche dans le laboratoire du laur\'eat du prix Nobel David Baltimore, il a r\'ev\'el\'e un nombre surprenant d'erreurs. Baltimore a \'et\'e exon\'er\'e parce que les erreurs de son laboratoire \'etaient al\'eatoires et non directionnelles ... Donc, en science, il n'y a pas d'autorit\'es. Tout au plus, il y a des experts!

		\item \textit{\textbf{Est-ce que cette source fait souvent des affirmations similaires?}}

		Les pseudoscientifiques ont l'habitude d'aller bien au-delà des faits. Les g\'eologues des inondations (les cr\'eationnistes qui croient que les inondations de No\'e peuvent repr\'esenter de nombreuses formations g\'eologiques de la Terre) font constamment des affirmations qui n'ont aucun rapport avec la science g\'eologique. Bien sûr, certains grands penseurs vont souvent au-delà des donn\'ees dans leurs sp\'eculations cr\'eatives. Thomas Gold de l'Universit\'e Cornell est connu pour ses id\'ees radicales, mais il a souvent eu raison de dire que d'autres scientifiques \'ecoutaient ce qu'il avait à dire. Gold propose, par exemple, que le p\'etrole n'est pas du tout un combustible fossile mais le sous-produit d'une biosphère chaude et profonde (micro-organismes vivant à des profondeurs inattendues dans la croûte). Presque tous les scientifiques de la terre avec qui j'ai parl\'e pensent que Gold a raison, et pourtant ils ne le considèrent pas comme une r\'ef\'erence intelectuelle dans le domaine de la g\'eologie. M\'efiez-vous donc des tendances marginales qui ignorent ou d\'eforment constamment les donn\'ees exp\'erimentales!

		\item \textit{\textbf{Est-ce que la source a \'et\'e v\'erifi\'ee par par d'autres experts ind\'ependents?}}
		
		Typiquement, les pseudoscientifiques font des d\'eclarations qui ne sont v\'erifi\'ees ou v\'erifiables que par une source dans leur propre cercle de croyance. Nous devons alors nous demander qui v\'erifie la source et même qui v\'erifie les v\'erificateurs? Le plus gros problème avec la d\'ebâcle de la fusion froide, par exemple, n'\'etait pas que Stanley Pons et Martin Fleischman avaient tort. C'est qu'ils ont annonc\'e leur d\'ecouverte spectaculaire lors d'une conf\'erence de presse avant que d'autres laboratoires ne le v\'erifient. Pire, lorsque la fusion à froid n'a pas \'et\'e reproduite, ils ont continu\'e à s'accrocher à leurs arguments. La v\'erification ext\'erieure est essentielle à la bonne science et à un esprit sain.

		\item \textit{\textbf{Comment la source correspond-elle à ce que nous savons de la façon scientifique dont le monde fonctionne?}}

		Une revendication extraordinaire par une source pseudoscientifique doit être plac\'ee dans un contexte plus large pour voir comment elle s'intègre. Quand les gens pr\'etendent que les pyramides \'egyptiennes et le Sphinx ont \'et\'e construits il y a plus de 10'000 ans par une race inconnue et avanc\'ee, ils ne pr\'esentent aucun contexte pour cette civilisation ant\'erieure. Où sont les autres artefacts de ces autres races? Où sont leurs oeuvres d'art, leurs armes, leurs vêtements, leurs outils, leurs d\'echets? L'arch\'eologie ne fonctionne tout simplement pas de cette façon!

		\item \textit{\textbf{Est-ce qu'il y a des experts qui r\'efutent la source, ou a-t-on seulement des experts qui la supportent?}}

		Cette situation correspond au "biais de confirmation" (nous reviendrons sur les biais cognitif dans la section sur la Th\'eorie de la D\'ecision \pageref{cognitive bias}), ou la tendance à rechercher des preuves confirmatives et à rejeter ou ignorer les preuves allant dans le sens contraire. Le biais de confirmation est puissant, omnipr\'esent et presque impossible à \'eviter pour chacun d'entre nous. C'est pourquoi les m\'ethodes scientifiques qui mettent l'accent sur la v\'erification et la rev\'erification, la r\'eplication à l'identique ou par d'autres approces alternatives, et en particulier les tentatives de falsification d'une revendication, sont si essentielles!

		\item \textit{\textbf{La pr\'epond\'erance de la preuve va-t-elle dans le sens de la source ou mène-t-elle à une conclusion diff\'erente?}}

		La th\'eorie de l'\'evolution, par exemple, est appuy\'ee à travers une convergence d'\'evidences provenant d'un certain nombre d'exp\'eriences ind\'ependantes sur des sujets diff\'erents. Aucun fossile, aucune pièce biologique ou pal\'eontologique ne porte \'ecrit sur elle le mot "\'evolution"; Au lieu de cela, des dizaines de milliers d'\'el\'ements probants s'ajoutent à l'histoire de l'\'evolution de la vie. Les cr\'eationnistes ignorent commod\'ement cette confluence, se concentrant plutôt sur des anomalies triviales ou des ph\'enomènes actuellement inexpliqu\'es dans l'histoire de la vie et de la Terre en g\'en\'eral.

		\item \textit{\textbf{La source utilise-t-elles les règles de la m\'ethode scientifique, ou les a-t-on abandonn\'ees au profit d'autres (typiquement \'emotionnelles) qui mènent à la conclusion souhait\'ee?}} 

		Une distinction claire peut être faite entre les scientifiques SETI (Search for Extraterrestrial Intelligence) et les ufologues. Les scientifiques du SETI partent de l'hypothèse nulle que les ETI n'existent pas et qu'elles doivent fournir des preuves concrètes avant de faire l'affirmation extraordinaire que nous ne sommes pas seuls dans l'Univers. Les ufologues commencent par l'hypothèse positive que les ETI existent et nous ont visit\'e, puis utilisent des techniques de recherche douteuses pour soutenir cette croyance, comme la r\'egression hypnotique (r\'ev\'elations d'exp\'eriences d'abduction), le raisonnement anecdotique (innombrables histoires d'observations d'OVNIS), la pens\'ee conspiratrice (le gouvernement nous ment sur les rencontres extraterrestres), des preuves visuelles de mauvaise qualit\'e (photographies floues et vid\'eos granuleuses), et la pens\'ee anomaliste (anomalies atmosph\'eriques et perceptions erron\'ees visuelles par des t\'emoins oculaires). Ce type de d\'emarche intellectuelle est aussi celle des religions.

		\item \textit{\textbf{Le source fournit-elle une explication pour les ph\'enomènes observ\'es ou simplement nie-t-il l'explication existante?}}
	
		C'est une strat\'egie de d\'ebat classique: critiquez votre adversaire et n'affirmez jamais ce que vous croyez pour \'eviter les critiques. Il est presque impossible d'amener les cr\'eationnistes à offrir une explication à la vie (autre que "Dieu l'a fait"). Les cr\'eationnistes de la conception intelligente (ID) n'ont rien fait de mieux, se d\'ebarrassant des faiblesses des explications scientifiques pour les problèmes difficiles et l'offre à leur place. "IL l'a fait." Ce stratagème est inacceptable en science mais pourtant tellement courant dans les d\'ebats qui ont lieu même en dehors de la sphère des sciences (politique, religions, etc.).

		\item \textit{\textbf{Si la source pr\'esente une nouvelle explication, est-ce que cela tient compte d'autant de ph\'enomènes que l'ancienne explication?}}
	
		Beaucoup de sceptiques du VIH / SIDA affirment que le mode de vie cause le SIDA. Pourtant, leur th\'eorie alternative n'explique pas autant de donn\'ees que la th\'eorie du VIH. Pour faire valoir leur argument, ils doivent ignorer les diverses preuves à l'appui du VIH comme vecteur causal du sida tout en ignorant la corr\'elation significative entre l'augmentation du SIDA chez les h\'emophiles peu de temps après l'introduction du VIH dans l'approvisionnement en sang par inadvertance.

		\item \textit{\textbf{Les convictions personnelles et les pr\'ejug\'es (biais) de la source conduisent-elles aux conclusions, ou vice versa?}}

		Tous les scientifiques (et plus particulièrement les non-scientifiques qui ne sont pas form\'es pendant de nombreuses ann\'ees) ont des croyances sociales, politiques et id\'eologiques qui pourraient biaiser leurs interpr\'etations des donn\'ees et de la situation (c'est un "biais de confirmation" \'egalement appel\'e "biais de s\'election de cerises" qui est aussi la principale cause de rejet des r\'esultats et des outils scientifiques par les non-scientifiques), mais comment ces pr\'ejug\'es et ces croyances affectent-ils leur recherche dans la pratique? Habituellement, pendant le système d'\'evaluation par les pairs, ces pr\'ejug\'es et croyances sont extirp\'es, ou le papier ou le livre est rejet\'e. 
	\end{enumerate}
	
	Un certain nombre d'humains et particulièrement les croyants (pas que dans les religions mais sur tout sujet pour lequel il n'y a aucune évidence scientifique) sont doués pour ce que l'on appelle le "concordisme". L'idée sous-jacente du concordisme est d'interpréter des textes ou trouver des analogies aléatoires dans la Nature (en ayant évidemment zéro connaissances en combinatoire et probabilités) pour justifier que toutes les inventions ou découvertes des sciences modernes ont été annoncées dans leurs livres "saints" ou créés par des gens appartenant au même groupe de croyance qu'eux des centaines ou milliers d'années avant les autres. Évidemmment tout cela masqué sous un rhétorique fumeuse, des analogies qui font zéro sens (comparer les études statistiques avec des textes simples...), aucune méthodologie statistique, en omettant de citer des faits ou conclusions contradictoires, et avec une absence totale de croisement de sources et de vérifications par les pairs! C’est le triomphe de l’obscurantisme!!!
	
	On note majoritairement que cette démarche concordiste est l'apanage des gens peu éduqués ou endoctrinés depuis l'enfance (plus ceux souffrant de troubles mentaux pour diverses raisons) qui ne sont pas capables d'identifier leurs propres biais,  qui sont fermés à toute contradiction et qui ne sont pas capables d'itendifier la complexité réelle (multifactorialité) et les nuances d'un sujet même simple (ils ont tendance à tout binariser en Vrai/Faux).
	
	\begin{fquote}Vous pouvez gagner un débat contre 100'000 savants avec un unique fait, mais vous ne battrez très probablement jamais un idiot même avec 100 faits.
 	\end{fquote}
	
	\begin{figure}[H]
		\centering
		\includegraphics[width=1.0\textwidth]{img/intro/baloney_detection_toolkit.jpg}
	\end{figure}
	En ajustant, nous pouvons aller plus loin sur les sophismes de raisonnement. Voici une liste plus exhaustive avec des cas ultra ultra classiques (qui \'enervent \'enorm\'ement par ailleurs dans les d\'ebats les gens auxquels on indique clairement qu'ils tombent dans un ou plusieurs de ces sophismes):
	\begin{enumerate}
		\item Ad hominem: Un argument ad hominem attaque le messager, pas le message lui-même.

		\item Argument d'autorit\'e: Argument qui repose sur l'identit\'e d'une autorit\'e plutôt que sur les composants de l'argument lui-même.

		\item Argument des cons\'equences n\'egatives: Dire que parce que les implications d'une d\'eclaration \'etant vraie cr\'eeraient des r\'esultats n\'egatifs, cela ne doit pas être vrai.

		\item Appel à l'ignorance: Si quelque chose n'est pas connu pour être faux, cela doit être vrai.

		\item Plaidoyer sp\'ecial: \'enoncer un principe universel, en insistant sur le fait qu'il ne s'applique pas à vos affirmations pour une raison quelconque.

		\item Exposer la question / supposer la r\'eponse: Cela se produit quand une d\'eclaration a une pr\'emisse non prouv\'ee. Ce sophisme est \'egalement appel\'e "raisonnement circulaire" ou "logique circulaire".

		\item S\'election observationnelle: En regardant seulement des preuves positives tout en ignorant les n\'egatifs et vice versa.

		\item Statistiques de petits nombres: Utilisation de petits nombres pour signaler de fortes augmentations en pourcentage.

		\item M\'econnaissance de la nature des statistiques: L'ignorance à propos des hypothèses statistiques centrales et la d\'efinition des paramètres (la confusion entre la corr\'elation et la causalit\'e, la taille de l'\'echantillon et la haine du biais math\'ematique sont des exemples bien connus).

		\item Post hoc, ergo propter hoc: Fonder un effet sur une cause uniquement sur la base de la chronologie.

		\item Fausse dichotomie: Repr\'esenter un problème ou un argument comme n'ayant que deux options et aucun spectre entre les deux (biais d'estimation ponctuelle).

		\item Court terme vs Long terme: En supposant qu'une tendance actuelle demeurera constante tout au long de son histoire et continuera de le faire à l'avenir, même si rien ne laisse croire qu'une telle extrapolation soit justifi\'ee.

		\item Pente glissante, li\'ee à la fausse dichotomie: Dire que quelque chose ne va pas parce que c'est à côt\'e ou vaguement li\'e à quelque chose de mal.

		\item Preuves r\'eprim\'ees et demi-v\'erit\'es: tirer une conclusion injustifi\'ee de pr\'emisses qui sont au moins en partie correctes.

		\item Paroles en l'air (manque de franchise): L'utilisation de r\'ef\'erences vagues et non sp\'ecifiques.
	\end{enumerate}
	
	En plus de nous enseigner ce qu'il faut faire au minimum lors de l'\'evaluation d'une nouvelle source de connaissances, tout bon kit de d\'etection de balivernes doit \'egalement nous apprendre ce qu'il ne faut pas faire. Cela nous aide à reconnaître les erreurs les plus communes et les plus p\'erilleuses de la logique et de la rh\'etorique. Beaucoup de bons exemples peuvent être trouv\'es dans la religion et la politique, parce que leurs praticiens sont souvent oblig\'es de justifier deux propositions contradictoires (sans parler des d\'ebats sur les r\'eseaux sociaux...).

	Enfin, citons Lavoisier: «\textit{Le physicien peut aussi, dans le silence de son laboratoire et de son cabinet, exercer des fonctions patriotiques; il peut esp\'erer par ses travaux diminuer la masse des maux qui affligent bonheur et, n'eût-il contribu\'e, par les routes nouvelles qu'il s'est ouvertes, qu'à prolonger de quelques ann\'ees, de quelques jours, la vie moyenne des hommes, il pourrait aspirer aussi au titre glorieux de bienfaiteur de l'humanit\'e.}»
	\begin{center}
		\includegraphics[scale=0.30]{img/humour/evidence_based.jpg}	
	\end{center}
	
	\pagebreak
	\section{L'effet retour de flamme en Sciences}
	\lettrine[lines=4]{\color{BrickRed}U}n autre point important qu'il est important de souligner au sujet de la communication scientifique: Scientifiques, arrêtez de penser qu'expliquer la science va r\'esoudre les problèmes et \'eviter les pr\'ejug\'es, surtout si vous vous trouvez dans un \'etat d'incr\'edulit\'e ou d'\'evidence qui vous rend fou relativement aux nombreux complots comme celui de la Terre plate. les vaccins, le changement climatique, etc., car les m\'edias traditionnels (ou "pseudo-journalistes") ne savent pas comment communiquer sur les sujets scientifiques!!!

	Les raisons sont majoritairement les suivantes et s'appliquent en dehors du cas où les gens viennent vous \'ecouter ou \'ecouter d'autres scientifiques dans le cadre d'une conf\'erence ou d'un s\'eminaire:
	\begin{enumerate}
		\item La plupart des gens ne veulent pas \'ecouter quoi que ce soit à propos de la m\'ethode scientifique surtout quand ils ne vous ont jamais demand\'e «est-ce vrai?», «Est-ce la meilleure m\'ethode?», «N'est-ce pas un biais?». Si vous utilisez "votre science" juste pour souligner qu'ils ont tort sur ce qu'ils disent ou en argumentant vous allez juste les sortir de leur zone de confort et les faire en plus hair la science et les scientifiques (anti-intellectualisme!).
		
		\begin{figure}[H]
			\centering
			\includegraphics[scale=0.35]{img/intro/scientists_arent_arrogant.jpg}	
		\end{figure}
		
		\item La plupart des humains sont pleins de pr\'ejug\'es, de biais et opportunistes, ils n'aiment alors pas admettre que cela est vrai car ils supposent que l'humain est au sommet de l'\'evolution et ne peut donc pas avoir de tels d\'efauts. Donc, quand vous leur expliquez qu'ils ont des biais, vous faites simplement remarquer qu'ils ne sont pas fiables. Parlez donc de partialit\'e seulement si les gens vous demandent de le faire.
		
		\item La grande majorit\'e des humains croient que leur exp\'erience personnelle est plus fiable et significative que les centaines d'ann\'ees de revue par les pairs, de tests exp\'erimentaux, de contrôles de la "m\'ethode scientifique" qui semble jusqu'ici, sinon LA meilleure, au moins être la meilleure m\'ethode d'investigation intellectuelle que l'on connaisse à ce jour.
	\end{enumerate}
	
	\begin{fquote}Aucune quantité d'évidences revues par les pairs ne persuadera jamais un idiot, d'autant plus si ce dernier est scientifiquement analphabète et ne maîtrise pas les calculs statistiques de haut niveau et les subtilités de la méthode scientifique.
 	\end{fquote}
	
	Maintenant, citons quelques paragraphes d'un excellent \href{http://www.slate.com/articles/health_and_science/science/2017/04/explaining_science_won_t_fix_information_illiteracy.html}{{\color{blue} article}} de Tim Requarth puisque ceux-ci sont quasiment parfaits pour illustrer nos propos:
	
	«La th\'eorie que de nombreux scientifiques semblent supporter est techniquement connue sous le nom de "modèle de d\'eficit", qui stipule que les opinions des gens diffèrent du consensus scientifique parce qu'ils manquent de connaissances scientifiques. En 2010, Dan Kahan, un psychologue de Yale, semble avoir montr\'e que cette th\'eorie \'etait significativement fausse. Il a \href{http://www.nature.com/nclimate/journal/v2/n10/full/nclimate1547.html}{{\color{blue} sond\'e}} plus de 1'500 Am\'ericains, classant chaque personne dans une «vision du monde culturelle» d'une \'echelle qui correspond grossièrement avec une tendance politique lib\'erale ou conservatrice. Il a ensuite \'evalu\'e la culture scientifique de chaque personne avec des questions telles que «Vrai ou Faux: les \'electrons sont plus petits que les atomes». Enfin, il les a interrog\'es sur le changement climatique. Si le modèle de d\'eficit \'etait correct alors les gens avec une culture scientifique sup\'erieure, ind\'ependamment de leur vision du monde, devraient être d'accord avec les consensus scientifique que le changement climatique repr\'esente un risque s\'erieux pour l'humanit\'e.
  
	Ce n'est pas ce Dan Kahan a observ\'e. Au lieu de cela, les donn\'ees mettent en \'evidence que l'augmentation des connaissances scientifiques avait en r\'ealit\'e un petit effet n\'egatif: les r\'epondants conservateurs qui en savaient le plus sur la science semblent penser que le changement climatique pose le moins de risques. La culture scientifique, semble-t-il, augmente la polarisation. Dans une \'etude ult\'erieure, Dan Kahan a ajout\'e une vice dans le questionnaire: Il a demand\'e aux r\'epondants ce que selon eux les climatologues croyaient (\'eviement l'usage du verbe "croire" est vicieux dans ce contexte!). Les r\'epondants qui en savaient plus sur la science en g\'en\'eral, quelle que soit leur orientation politique, \'etaient plus à même d'identifier le consensus scientifique - en d'autres termes, la polarisation avait disparu. Pourtant, quand on a demand\'e aux mêmes personnes leurs opinions sur le changement climatique, la polarisation est revenue. Il a donc \'et\'e montr\'e sur cette unique exp\'erience que même lorsque les gens comprennent le consensus scientifique, ils peuvent souvent ne pas l'accepter.
	
	\begin{fquote}[Utilisateur lambda des réseaux sociaux]N'ayant aucune formation postdoctorale sur ce sujet, ni aucun biais, et n'ayant fait aucune recherche dans un laboratoire à un niveau professionnel que ce soit, je ne le comprends pas. Par conséquent, cela n'a pas de sens et c'est faux!
 	\end{fquote}

	Le point à retenir est clair: l'augmentation de la culture scientifique seule ne changera pas les esprits et le biais cognitifs. En fait, les tentatives bien intentionn\'ees des scientifiques pour informer le public pourraient même se retourner contre eux. Pr\'esenter des faits qui entrent en conflit avec la vision du Monde d'un individu peut, en fait, inciter les gens à aller plus loin. Les psychologues, à juste titre, ont surnomm\'e cela "l'effet de retour de flamme"\index{effet retour de flamme}.
	\begin{figure}[H]
		\centering
		\includegraphics[scale=0.4]{img/intro/explain_science.jpg}
		\caption[]{Source: Dr. Jones, https://www.ratbotcomics.com}
	\end{figure}
	Si les scientifiques veulent simplement expliquer la science à un public curieux, diffuser plus largement leur recherche ou \'ecrire pour s'amuser, cela n'a pas beaucoup d'importance. Mais si les scientifiques sont motiv\'es à faire changer les opionons - et nombreux sont les scientifiques inscrits à des ateliers de communication scientifique qui semblent avoir cet objectif - ils seront a priori très d\'eçus.

	Cela ne veut pas dire que les scientifiques devraient retourner à la laboratoire et rester muets. Ils devraient juste se rendre compte que la combler le "manque d'information (ou de culture)" n'est pas l'objectif r\'eel. Au lieu de cela, les scientifiques devraient apprendre à communiquer la science de façon strat\'egique!

	Il y a des raisons \'evidentes pour lesquelles la communication scientifique est une entreprise n\'ecessaire et utile, mais une particulièrement importante est: qu'il y a dans certains pays un politique organis\'ee faite d\'ecridibiliser la science et att\'enuer son efficacit\'e. Lors d'une conf\'erence au Heartland Institute en Mars 2017, Lamar Smith, le pr\'esident r\'epublicain du comit\'e scientifique a d\'eclar\'e aux participants qu'il qualifierait maintenant la «science du climat» de «science politiquement correcte» et ce afin de minimiser les critiques. Cela cat\'egorse donc implicitement les scientifiques comme faisant partie de la "gauche" politique et, comme Daniel Engber l'a soulign\'e dans le magazine Slate à propos de la prochaine Marche Pour La Science, à red\'efinir l'autorit\'e scientifique comme une forme d'\'elitisme.

	Est-il surprenant, alors que des conf\'erences de scientifiques construites sur le principe qu'ils en savent simplement plus (même si c'est vrai) n'arrivent pas à convaincre leur public? Plutôt que de combler le d\'eficit d'information en construisant un arsenal de faits et d'\'evidences exp\'erimentales, les scientifiques devraient plutôt envisager comment d\'eployer leurs connaissances. Ils peuvent avoir plus de chance de communiquer si, en plus de pr\'esenter des faits et des chiffres, ils font appel aux \'emotions (biais \'emotionnels). Cela pourrait signifier non seulement expliquer la science de la façon dont quelque chose fonctionne mais passer du temps sur les raisons pour lesquelles cela compte pour l'auditeur et pourquoi cela devrait avoir de l'importance pour le conf\'erencier. La recherche montre \'egalement que les communicateurs scientifiques peuvent être plus efficaces après avoir gagn\'e la confiance du public. Dans cette optique, il serait peut-être plus utile de comprendre comment parler de la science avec des gens qui la connaissent d\'ejà, par le biais d'interactions locales et communautaires, que d'essayer de publier des explications sur des sites d'information nationaux. Et les scientifiques pourraient envisager de r\'ediger des \'editoriaux pour leurs journaux locaux, en se concentrant sur les raisons pour lesquelles la science compte pour leurs communaut\'es respectives.

	Les scientifiques peuvent \'egalement apprendre à \'eviter certains pièges. J'ai parl\'e avec Gretchen Goldman, directrice de l'Union des Centres Scientifiques Concern\'es pour la Science et la D\'emocratie, qui propose des ateliers de communication et de plaidoyer. Une leçon contre-intuitive qu'elle a apprise est que r\'efuter des histoires qui nient le changement climatique en abordant chaque revendication et en expliquant pourquoi c'est faux n'est pas très productif. En fait, cela pourrait être contre-productif: «Si vous r\'ep\'etez le mythe, c'est la partie que les gens se souviennent même si vous la d\'emystifiez imm\'ediatement», dit-elle. Selon elle, une meilleure approche consiste à recadrer le problème. Ne continuez pas à expliquer pourquoi le changement climatique est r\'eel, expliquez comment le changement climatique va nuire à la sant\'e publique ou à l'\'economie locale. La communication qui fait appel à des valeurs, pas seulement à l'intellect, montre la recherche, peut être beaucoup plus efficace (biais \'emotionnel).

	[...] Mais les obstacles rencontr\'es par les communicateurs scientifiques ne sont pas \'epist\'emologiques mais culturels. Les comp\'etences requises ne sont pas celles d'un professeur d'universit\'e mais d'un rh\'eteur.

	C'est donc un but probablement admirable de communiquer sur la science, mais presque certainement destin\'e à \'echouer. C'est parce que la façon dont la plupart des scientifiques pensent à la communication scientifique - qui simplement qu'en expliquant la vraie science aidera - est tout à fait fausse. En fait, c'est tellement faux que c'est souvent l'effet inverse de ce qu'ils essaient d'accomplir qui est r\'ecolt\'e. [...]»
	
	\begin{fquote}Le silence est parfois la meilleure réponse aux imbéciles.
 	\end{fquote}
	
	\begin{tcolorbox}[title=Remarque,colframe=black,arc=10pt]
	Dans le cadre d’un échange cordial, adopter une position de juge est une mauvaise stratégie. Il y a déjà de fortes chances pour que nos arguments soient perçus comme une agression, ce n’est donc pas la peine d’en rajouter. Il semble plus sage d’adopter la stratégie de l’entretien épistémique. Comparable à la maïeutique de Socrate, elle consiste à aider notre interlocuteur à présenter sa pensée, à la synthétiser pour pouvoir mieux, avec lui, la scruter et mettre en évidence ses failles éventuelles.
	\end{tcolorbox}
	
	\pagebreak
	\section{La Science est-elle dogmatique?}
	Nous allons ici répéter principalement des choses que nous avons déjà mentionnées plus haut. Cependant, comme certaines personnes scientifiquement illetrées pensent encore en ce début du 21e siècle que les vidéos YouTube contenant un monologue rhétorique\footnote{Les gens ne doivent pas faire confiance aux monologues - que ce soit à la radio, à la télévision ou sur les réseaux sociaux - car il n'y a pas d'experts pour contrer les éventuels mauvais arguments ou concepts mal définis qui peuvent conduire une partie de l'auditoire à de fausses interprétations spéculatives! De plus, les humains sous le stress de savoir qu'ils sont enregistrés sont naturellement sujets aux erreurs de vocabulaire et c'est sans compter les plus de 200 biais cognitifs du cerveau qui conduisent parfois à la simplification erronée de pensées complexes ...!} ou des livres sans preuves et sans analyse statistiques de données  constituent une sorte de "d'évidence", il est peut-être nécessaire de revenir sur quelques sujets mais avec une perspective différente\footnote {Et n'oubliez pas que le but de la science n'est pas la "vérité" mais c'est seulement un outil qui explique assez bien les choses que nous voyons ou ressentons en suivant les meilleurs modèles actuels. Et n'oubliez pas non plus que citer un livre, un scientifique célèbre, un blog ou une vidéo YouTube n'est au mieux qu'une évidence de niveau 2 ou 3!}!
	
	Pour la Science, si quelque chose existe avec des évidences au-delà de tout doute raisonnable (ADDTDR), dans son état actuel de connaissance, cela signifie qu'elle peut être mesurée! Si elle ne peut pas être mesurée ou est mal définie, eh bien, alors la Science ne peut pas fournir l'évidence qu'elle n'existe pas (n'oubliez pas que le but de la Science est de réfuter les modèles et si elle échoue à le faire pendant assez longtemps alors un modèle peut prendre le statut de théorie!). Dans une métaphore simpliste et légèrement hors contexte, cela équivaut à dire que pour les aveugles (en leur enlevant aussi tous leurs autres sens), le monde n’existe pas avec des preuves solides parce qu’ils ne peuvent ni le voir ni le sentir. Certains disent alors que la science est matérialiste! Cependant, cela ne signifie pas du tout que la Science a échoué en tant que méthode d'investigation de la Nature, mais cela signifie simplement que la Science sait qu'elle ne sait pas tout sinon elle s'arrêterait et que la méthode scientifique doit être constamment corrigée en fonction des nouvelles évidences disponibles.
	
	Typiquement, la science ne rejette pas complètement la parapsychologie ou l'existence de l'une des mille divinités créées dans le monde par les humains, même si en réalité tous les tests expérimentaux menés jusqu'à ce jour les ont rejetées avec des évidences solides (mais pas définitivement!). Les partisans de la parapsychologie ou des religions (ou de certaines médecines alternatives) peuvent faire valoir que la science ne peut ni prouver ni réfuter ce qu'elle ne peut pas mesurer directement ou indirectement. Et ils ont très vraisemblablement absolument raison! La science ne peut \underline{échouer à rejeter avec un niveau d'évidence donné} que si quelque chose existe ou non \ \underline{à son état actuel de connaissances}. Donc, si quelqu'un soutient que les licornes volantes existent sans fournir de protocole afin que des milliers d'autres personnes puissent vérifier ce fait de manière reproductible et indiscutable... alors la science (scientifiques) nous dit humblement que: \textit{nous n'avons aucune évidence au-delà doute raisonnable à notre état actuel de connaissance que les licornes existent ou non}!
	
	Ce n'est donc pas la Science qui est dogmatique, ni l'immense majorité de la communauté scientifique. Mais seuls quelques scientifiques mal éduqués et biaisés (personne n'est parfait...).
	
	\begin{figure}[H]
		\centering
		\includegraphics[width=1\textwidth]{img/intro/science_dogmatic.jpg}
	\end{figure}
	
	Certains humains n'aiment pas que les scientifiques ne sachent pas tout sur quelque chose, ou que les scientifiques font des erreurs, et que la Science prend du temps et que pire encore, ces mêmes personnes n'auront peut-être jamais de réponse à leurs principales questions avant leur mort. Mais c'est ainsi que fonctionne la Science!!! Si quelqu'un touve un moyen plus fiable et plus rapide d'étudier l'Univers et ses phénomènes que l'état actuel de la Méthode Scientifique, alors la majorité de la communauté scientifique - en particulier la communauté académique - le vérifiera et si vraiment cela fonctionne mieux alors l'adoptera avec plaisir!
	
	\begin{fquote}[Carl Sagan]Les affirmations extraordinaires nécessitent des preuves extraordinaires.
 	\end{fquote}
 	
 	Ainsi, certaines personnes peuvent demander à la Science d'être vraiment scientifique, c'est-à-dire qu'elle devrait s'interroger sur la certitude de ses propres postulats et instruments. Mais comme nous l'avons déjà dit à plusieurs reprises dans les sections précédentes ci-dessus, c'est justement ce que font les chercheurs dans leur travail quotidien!!! Ils essaient de trouver de nouvelles évidences pour rejeter les théories, les modèles, les méthodes, les postulats ou les mauvais instruments actuels... sinon la science s'arrêterait et les scientifiques perdraient leur travail! Cependant, le processus est lent et prend des jours, des semaines, des mois, des années, des décennies, des siècles et parfois même des millénaires.
	 
	\begin{tcolorbox}[title=Remarque,colframe=black,arc=10pt]
	Certaines personnes soulignent que la version de l'Univers dont disposent actuellement les scientifiques n'est que ce que leurs instruments, et surtout leur imagination leur permet de comprendre. Et ils aussi très vraisemblablement raison! Nous - tous les chercheurs académiques professionnels - savons ce fait depuis des siècles en science et c'est pourquoi nous essayons chaque jour de repousser les limites de la connaissance (donc aussi de l'imagination) et de développer de nouveaux instruments et méthodes pour mesurer des choses qui n'étaient même pas connues quelques décennies auparavant! Le lecteur doit garder à l'esprit que c'est pour cela que nous sommes payés! S'il n'y a rien de nouveau à découvrir, nous perdrons tous nos emplois!\\
	
	Alors oui dans son état actuel, la Science ne peut pas rendre compte des phénomènes de conscience, de la synchronicité, ou même des expériences de mort imminente ou des épiphanies spontanées, etc. Mais les scientifiques professionnels correctement éduqués ne prétendront pas qu'ils existent ou n'existent pas parce que nous n'avons pas encore actuellement d'évidences expérimentales reproductible pour soutenir ou rejeter l'une de ces positions! Les scientifiques attendent juste de ceux qui prétendent que de telles choses existent - en faisant même des affaires très lucratives avec - de leur fournir des évidences expérimentales reproductibles. Malheureusement, jusqu'à présent, toutes les personnes qui ont fait de telles affirmations et qui ont des activités lucratives avec n'ont fourni aucune évidence expérimentale et même pire ils ont été systématiquement identifiés comme charlatans.\\
	
	N'importe qui devrait se demander pourquoi les pasteurs évangélistes ou tout type de guérisseur faiseur de miracles (...) ne pratiquent pas leur art dans l'environnement scientifique contrôlé des hôpitaux plutôt que dans leur église (devant un public en grande majorité biaisé, sans instruction et scientifiquement illetrés) ou avec un petit groupe de personnes inconnues et suspectes derrière une caméra..........
	\end{tcolorbox}
	 
	Des gens comme Rupert Sheldrake, titulaire d'un doctorat (de Cambridge) en biochimie et chercheur à la retraite dans le domaine de la parapsychologie qui a proposé le concept de résonance morphique ... et également auteur de nombreux livres enrichissants (au point figuré du terme!) ...... explore dix dogmes de la science qui devraient être reconsidérés en fonction de son opinion personnelle et subjective...
	
	Présentons ci-dessous ces dix dogmes et ... nous les commenterons car le Dr Sheldrake n'a pas de doctorat en physique, ni en cosmologie mais a un bon sens de la rhétorique surtout quand il fait un monologue devant un public de néophytes (le lecteur et le Dr Sheldrake pourront trouver les preuves mathématiques de nos réponses dans les pages à $7000$ de ce livre s'ils sont curieux - contrairement au Dr Sheldrake, nous aimons plus les preuves et les évidences expérimentales que les monologues rhétoriques...):
	\begin{enumerate}
		\item \og La nature est mécanique: toutes les créatures et systèmes de la nature sont des robots faits pour suivre un programme génétique donné. \fg{}
		
		$\vartriangleright$ Notre commentaire: Nous ne sommes pas sûrs de la définition du mot «Mécanique» dans cette phrase (les bases d'un débat sont de s'entendre sur le sens des mots ...) mais évidemment un système biologique ne peut être comparé à un système mécanique (un système mécanique n'évolue pas). Cependant, comme nous le prouverons dans ce livre, la nature est basée sur l'Information à tous les niveaux, basée sur les Probabilités au niveau microscoptique, et se comporte Statistiquement et Mécaniquement au niveau macroscopique. Donc comme la plupart du temps en Science ..., ce n'est pas aussi facile qu'il y paraît (il semble que le Dr Sheldrake ait une vision binaire du Monde et de l'Univers assez surprenante compte tenu de son niveau d'éducation supposé).
		
		\item \og La matière est inconsciente: Les plantes, les étoiles, les animaux et les éléments sont des choses matérielles qui sont et ne peuvent pas avoir une conscience d'elles-mêmes.\fg{}
		
		$\vartriangleright$ Notre commentaire: Nous devons d'abord nous demander comment la «conscience» est définie par le Dr Sheldrake. Deuxièmement, qui a affirmé cela dans la communauté scientifique? Existe-t-il un consensus scientifique écrit noir sur blanc sur ce sujet ou cela est-t-il issu de l'imagination du Dr Sheldrake?

		\item \og Les lois de la nature sont fixes: au moment du Big Bang, toutes les constantes nécessaires jusqu'à la fin des temps ont été établies. Les habitudes de la nature n'évoluent pas.\fg{}
		
		$\vartriangleright$ Notre commentaire: Quel consensus scientifique a déclaré cela? C'est très probablement faux car nous avons en fait quelques modèles en physique théorique qui prouvent que les constantes de l'Univers peuvent tout à fait avoir changé et nous avons aussi des évidences expérimentales que les lois de la Nature ont aussi peut-être changé sur de longues périodes de temps. Mais une chose est presque sûre: il n'y a pas encore de consensus scientifique écrit noir sur blanc sur ce sujet au jour où nous écrivons ces lignes!

		\item \og La quantité de nature et d'énergie dans l'Univers est toujours la même. \fg{}
		
		$\vartriangleright$ Notre commentaire: La communauté scientifique a de solides évidences au-delà de tout doute raisonnable de cette affirmation pour l'Univers \underline{observable}. Cependant, la dynamique de l'Univers réfute la conservation de l'énergie à grande échelle (cela est dérivé des théorèmes de Noether!). Notez que nous n'avons cependant aucune preuve de cette déclaration de conservation d'énergie pour l'Univers dans son entier (l'observable et non observable).

		\item \og La Nature n'a pas de buts: il n'y a pas de conception dans la Nature en termes d'intention et le processus d'évolution est mécanique.\fg{}
		
		$\vartriangleright$ Notre commentaire: Nous avons des évidences solides au-delà de tout doute raisonnable en effet qu'il n'y a pas de conception intelligente car la conception observable actuelle est défectueuse à bien des égards. Et le processus d'évolution tel que prouvé mathématiquement dans ce livre et expérimentalement (au-delà de tout doute raisonnable) en laboratoire n'est pas mécanique mais stochastique.
		
		\item \og Patrimoine biologique: Les plans pour produire un être vivant sont composés dans la matière physique logée dans leurs gènes. \fg{}
		
		$\vartriangleright$ Notre commentaire: Ce n'est pas tout à fait exact. Les observations expérimentales nous montrent que certaines bases des plans sont aléatoires et influencées par des modifications externes et internes. Encore une fois, le Dr Sheldrake donne une vue binaire d'un phénomène bien plus complexe. Mais c'est quelque chose que nous savons comme typique des scientifiques qui parlent de sujets qu'ils ont mal étudiés: ils réduisent quelque chose de complexe à quelque chose de simple parce qu'ils ne peuvent pas saisir la complexité du Monde et de l'Univers.

		\item \og La mémoire est conservée dans le cerveau sous forme d'empreintes matérielles: la mémoire est constituée de protéines et de terminaisons nerveuses organisées comme des tiroirs.\fg{}
		
		$\vartriangleright$ Notre commentaire: Si c'était le cas, nous n'oublierions pas des choses ... C'est parce que le cerveau est beaucoup plus complexe et implique des probabilités, des processus bayésiens et stochastiques que nous savons pourquoi le cerveau humain oublie des choses et a parfois des problèmes de construction...

		\item \og L'esprit est dans la tête: l'esprit a une connexion physique avec la tête et le cerveau, reléguant la subordination intellectuelle sur le reste du corps.\fg{}
		
		$\vartriangleright$ Notre commentaire: Qu'est-ce que «l'esprit» pour le Dr Sheldrake? À notre connaissance, il n'y a pas de consensus scientifique sur sa définition. De plus, l'esprit est-il ce que nous observons dans les scanners à RMN?

		\item \og Les phénomènes comme la télépathie sont impossibles: les pensées n'ont aucun effet sur le monde à cause du numéro 8 de la liste (l'esprit est dans la tête). \fg{}
		
		$\vartriangleright$ Notre commentaire: Quel consensus scientifique ou communauté a déclaré cela? En fait, la science n'a aucune évidence que la télépathie fonctionne ou existe, oui (!) - mais aucun scientifique bien éduqué ne dirait que c'est "impossible" (d'ailleurs tout scientifique bien éduqué sait qu'il vaut mieux éviter d'utiliser le mot "impossible" à propos d'un futur inconnu ou de phénomènes non mesurables).

		\item \og Seule la médecine mécanique fonctionne: c’est simplement par hasard ou par effet placebo que les pratiques de guérison traditionnelles ou les remèdes naturels ont un effet sur la santé des gens. \fg{}
		
		$\vartriangleright$ Notre commentaire: Encore une fois ... quel consensus scientifique ou quelle communauté a déclaré cela? Ce n'est pas exact. Peut-être que les scientifiques sont d'accord sur le fait qu'aucune autre méthode que la médecine scientifique n'a fourni un meilleur rapport de cotes que d'autres médicaments. Mais si un jour certaines personnes fournissent des évidences solides que les pratiques de guérison traditionnelles ou les remèdes naturels fonctionnent, alors presque sûrement ils seront promus par la communauté académique scientifique.
	\end{enumerate}
	
	Si ce sont là les meilleures évidences que le Dr Sheldrake a pour sa défendre sa position, alors cela signifie qu'il n'y a pas d'évidence du tout (au mieux du niveau 1). Ses arguments sont basés essentiellement sur de la rhétorique pseudo-scientifique et la spéculation, usant d'idées populaires fausses et d'opinions plutôt que des évidences et des exemples de "dogmes" en science.

	Nous pouvons comprendre pourquoi quelqu'un qui lit ou écoute ce genre rhétorique pseudo-scientifique pourrait penser qu'il existe une base d'évidence. Ainsi, quand quelqu'un prétend que «l'establishment» est dogmatique, immoral ou quoi que ce soit, nous pouvons sincèrement espérer que n'importe quel lecteur ou auditeur évaluera de manière critique les affirmations de cette personne.
	
	\begin{figure}[H]
		\centering
		\includegraphics[width=0.7\textwidth]{img/intro/fausses_equivalences.jpg}
	\end{figure}
	
	\begin{flushright}
	Note de qualit\'e de la section: \score{4}{5} 151 votes, 75.23\%
	\end{flushright}
	

 \chapter{Arithmétique}

	\textit{\textbf{Mathematics is the ultimate form of forced art.}} (unknown)
	\minitoc
	%to make section start on odd page
	\newpage
	\thispagestyle{empty}
	\mbox{}
	\section{Théorie de la Démonstration}\label{proof theory}
	\lettrine[lines=4]{\color{BrickRed}N}ous avons choisi de commencer l'étude de la mathématique appliquée par la théorie qui nous semble la plus fondamentale et la plus importante dans le domaine des sciences pures et exactes: La théorie de la démonstration. La théorie de la démonstration et du calcul propositionnel (logique) a cinq objectifs majeurs dans le cadre de ce site:
	\begin{enumerate}
		\item  Apprendre au lecteur comment raisonner et à d\'emontrer et cela ind\'ependamment de la sp\'ecialisation \'etudi\'ee.
		
		\item Montrer que le processus d'une d\'emonstration est ind\'ependant du langage utilis\'e.
		
		\item  Se pr\'eparer à la th\'eorie de la logique et au th\'eorème d'incompl\'etude de Gödel ainsi qu'aux automates (\SeeChapter{voir section de Systèmes Logiques page \pageref{logical systems}}).
		
		\item Pr\'eparer le chemin au th\'eorème d'incompl\'etude de Gödel (objectif principal de cette section!).
		
		\item Pr\'eparer le lecteur à la Th\'eorie des Afoutomates (\SeeChapter{voir section sur la Th\'eorie des Automates page \pageref{automata theory}}).
	\end{enumerate}
	
	Le th\'eorème d'incompl\'etude Gödel est le point le plus passionnant car si nous d\'efinissons une "religion" comme un système de pens\'ee qui contient des affirmations ind\'emontrables, alors elle contient des \'el\'ements de foi, et Gödel nous enseigne que la math\'ematique est non seulement une religion, mais que c'est alors la seule religion capable de prouver qu'elle en est une!
	
	\begin{tcolorbox}[colback=red!5,borderline={1mm}{2mm}{red!5},arc=0mm,boxrule=0pt]
	\bcbombe Le th\'eorème d'incompl\'etude de Gödel stipule que, dans tout système axiomatique, certaines d\'eclarations ne peuvent pas être d\'etermin\'ees comme vraies ou fausses. Ce th\'eorème d'incompl\'etude semble donc être quelque chose d'horrible, et pour les math\'ematiques, c'est le cas! Cependant, cela ne s'applique pas à la physique (ni à aucune autre science exp\'erimentale), car la physique ne repose pas sur la preuve logique dans un systèmes axiomatique, mais sur des \'evidences scientifiques exp\'erimentales!
	\end{tcolorbox}
	
	\begin{tcolorbox}[title=Remarques,colframe=black,arc=10pt]
	\textbf{R1.} Il est (très) fortement conseill\'e de lire en parallèle à cette section celle sur la Th\'eorie des Automates (page \pageref{automata theory} des Systèmes Logiques (page \pageref{logical systems}), inclus l'Algèbre de Boole (page \pageref{boolean algebra}), disponible dans le chapitre d'Informatique Th\'eorique (page \pageref{theoretical computing}) de ce livre.\\

	\textbf{R2.} Il faut la Th\'eorie de la D\'emonstration comme une curiosit\'e sympathique mais qui n'amène fondamentalement pas grand-chose except\'e des m\'ethodes de travail/raisonnement. Par ailleurs, son objectif n'est pas de d\'emontrer que tout est d\'emontrable mais que toute d\'emonstration peut se faire sur un langage commun à partir d'un certain nombre de règles.
	\end{tcolorbox}

	Souvent, quand un \'etudiant arrive dans une classe sup\'erieure, il a surtout appris à calculer, à utiliser des algorithmes mais relativement peu voire pas du tout à raisonner. Pour tous les raisonnements, le support visuel est un outil puissant, et les personnes qui ne voient pas qu'en traçant telle ou telle courbe ou droite la solution apparaît ou qui ne voient pas dans l'espace sont très p\'enalis\'ees.
	
	Lors des \'etudes secondaires, nous manipulons d\'ejà des objets inconnus, mais c'est surtout pour faire des calculs, et quand nous raisonnons sur des objets repr\'esent\'es par des lettres, nous pouvons remplacer ceux-ci visuellement par un nombre r\'eel, un vecteur, etc. A partir d'un certain niveau, nous demandons aux personnes de raisonner sur des structures plus abstraites, et donc de travailler sur des objets inconnus qui sont des \'el\'ements d'un ensemble lui-même inconnu, par exemple les \'el\'ements d'un groupe quelconque (\SeeChapter{voir section Théorie des Ensembles page \pageref{groups}}). Ce support visuel n'existe alors plus!

	Nous demandons ainsi souvent aux \'etudiants de raisonner, de d\'emontrer des propri\'et\'es, mais personne ne leur a jamais appris à raisonner convenablement, à \'ecrire des preuves. Si nous demandons à un \'etudiant de licence ce qu'est une d\'emonstration, il a très probablement quelque difficult\'e à r\'epondre. Il peut dire que c'est un texte dans lequel on trouve des mots-cl\'es comme: "donc", "parce que", "si", "si et seulement si", "prenons un $x$ tel que", "supposons que", "cherchons une contradiction", etc. Mais il est incapable de donner la grammaire de ces textes ni même leurs rudiments, et d'ailleurs, ses enseignants, s'ils n'ont pas suivi de cours sur la Th\'eorie de la D\'emonstration, en seraient probablement incapables aussi.

	Pour comprendre cette situation, rappelons que pour parler un enfant n'a pas besoin de connaître la grammaire. Il imite son entourage et cela marche très bien: un enfant de six ans sait utiliser des phrases d\'ejà compliqu\'ees quant à la structure grammaticale sans avoir jamais fait de grammaire. La plupart des enseignants ne connaissent pas non plus la grammaire du raisonnement mais, chez eux, le processus d'imitation a bien march\'e et ils raisonnent correctement. L'exp\'erience de la majorit\'e des enseignants d'universit\'e montre que ce processus d'imitation marche bien chez les très bons \'etudiants, et alors il est suffisant, mais il marche beaucoup moins bien, voire pas du tout, chez beaucoup d'autres.

	Tant que le degr\'e de complexit\'e est faible (notamment lors d'un raisonnement de type "\'equationnel"), la grammaire ne sert à rien, mais quand il augmente ou quand on ne comprend pas pourquoi quelque chose est faux, il devient n\'ecessaire de faire un peu de grammaire pour pouvoir progresser. Les enseignants et les \'etudiants connaissent bien la situation suivante: dans un devoir, le correcteur a barr\'e toute une page d'un grand trait rouge et mis "faux" dans la marge. Quand l'\'etudiant demande ce qui est faux, le correcteur ne peut que dire des choses du genre "ça n'a aucun rapport avec la d\'emonstration demand\'ee", "rien n'est juste", ..., ce qui n'aide \'evidemment pas l'\'etudiant à comprendre. Cela vient en partie, du fait que le texte r\'edig\'e par l'\'etudiant utilise les mots voulus mais dans un ordre plus ou moins al\'eatoire et qu'on ne peut donner de sens à l'assemblage de ces mots. De plus, l'enseignant n'a pas les outils n\'ecessaires pour pouvoir expliquer ce qui ne va pas. Il faut donc les lui donner!

	Ces outils existent mais sont assez r\'ecents. La th\'eorie de la d\'emonstration est une branche de la logique math\'ematique dont l'origine est la crise des fondements: il y a eu un doute sur ce que nous avions le "droit" de faire dans un raisonnement math\'ematique (voir la "crise des fondements" plus loin). Des paradoxes sont apparus, et il a alors \'et\'e n\'ecessaire de pr\'eciser les règles de d\'emonstration et de v\'erifier que ces règles ne sont pas contradictoires. Cette th\'eorie est apparue au d\'ebut du 20ème siècle, ce qui est très peu puisque l'essentiel des math\'ematiques enseign\'ees en première moiti\'e de l'universit\'e est connu depuis le 16ème-17ème siècle.

	\subsection{La Crise des Fondements}
	Pour les premiers Grecs, la g\'eom\'etrie \'etait consid\'er\'ee comme la forme la plus haute du savoir, une puissante cl\'e pour les mystères m\'etaphysiques de l'Univers. Elle \'etait plutôt une croyance mystique, et le lien entre le mysticisme et la religion \'etait rendu explicite dans des cultes comme ceux des Pythagoriciens. Aucune culture n'a depuis d\'efi\'e un homme pour avoir d\'ecouvert un th\'eorème de g\'eom\'etrie! Plus tard, la math\'ematique fut consid\'er\'ee comme le modèle d'une connaissance a priori dans la tradition aristot\'elicienne du rationalisme.

	L'\'etonnement des Grecs pour la math\'ematique ne nous a pas quitt\'es, on le retrouve sous la traditionnelle m\'etaphore des math\'ematiques comme "Reine des Science". Il s'est renforc\'e avec les succès spectaculaires des modèles math\'ematiques dans la science, succès que les Grecs (ignorant même la simple algèbre) n'avaient pas pr\'evus. Depuis la d\'ecouverte par Isaac Newton du calcul int\'egral et de la loi du carr\'e inverse de la gravit\'e, à la fin des ann\'ees 1600, les sciences ph\'enom\'enales et les plus hautes math\'ematiques \'etaient rest\'ees en \'etroite symbiose - au point qu'un formalisme math\'ematique pr\'edictif \'etait devenu le signe distinctif d'une "science dure". 

	Après Newton, pendant les deux siècles qui suivirent, la science aspira à ce genre de rigueur et de puret\'e qui semblaient inh\'erentes aux math\'ematiques. La question m\'etaphysique semblait simple: la math\'ematique poss\'edait une connaissance a priori parfaite, et parmi les sciences, celles qui \'etaient capables de se math\'ematiser le plus parfaitement \'etaient les plus efficaces pour la pr\'ediction des ph\'enomènes. La connaissance parfaite consistait donc dans un formalisme math\'ematique qui, une fois atteint par la science et embrassant tous les aspects de la r\'ealit\'e, pouvait fonder une connaissance empirique a post\'eriori sur une logique rationnelle a priori. Ce fut dans cet esprit que Marie Jean-Antoine Nicolas de Caritat, marquis de Condorcet (philosophe et math\'ematicien français), entreprit d'imaginer la description de l'Univers entier comme un ensemble d'\'equations diff\'erentielles partielles se r\'esolvant les unes après les autres. 

	La première faille dans cette image inspiratrice apparut dans la seconde moiti\'e du 19ème siècle, quand Riemann et Lobatchevsky prouvèrent s\'epar\'ement que l'axiome des parallèles d'Euclide pouvait être remplac\'e par d'autres qui produisaient des g\'eom\'etries "consistantes" (nous reviendrons sur ce terme plus loin). La g\'eom\'etrie de Riemann prenait modèle sur une sphère, celle de Lobatchevsky, sur la rotation d'un hyperboloïde.

	L'impact de cette d\'ecouverte a \'et\'e obscurci plus tard par de grands chamboulements, mais sur le moment, elle fit un coup de tonnerre dans le monde intellectuel. L'existence de systèmes axiomatiques mutuellement inconsistants, et dont chacun pouvait servir de modèle à l'Univers ph\'enom\'enal, remettait entièrement en question la relation entre la math\'ematique et la th\'eorie physique.
	
	Quand on ne connaissait qu'Euclide, il n'y avait qu'une g\'eom\'etrie possible. On pouvait croire que les axiomes d'Euclide (\SeeChapter{voir la section de G\'eom\'etrie Euclidienne page \pageref{euclid's postulates}}) constituaient un genre de connaissance parfaite a priori sur la g\'eom\'etrie dans le monde ph\'enom\'enal. Mais soudain, nous avons eu trois g\'eom\'etries, embarrassantes pour les subtilit\'es m\'etaphysiques.

	Pourquoi aurions-nous à choisir entre les axiomes de la g\'eom\'etrie plane, sph\'erique et hyperbolique comme descriptions de la g\'eom\'etrie du r\'eel? Parce que toutes les trois sont consistantes, nous ne pouvons en choisir aucune comme fondement a priori - le choix doit devenir empirique, bas\'e sur leur pouvoir pr\'edictif dans une situation donn\'ee.

	Bien sûr, les th\'eoriciens de la physique ont longtemps \'et\'e habitu\'es à choisir des formalismes pour poser un problème scientifique. Mais il \'etait admis largement, si ce n'est inconsciemment, que la n\'ecessit\'e de proc\'eder ainsi \'etait fonction de l'ignorance humaine, et qu'avec de la logique ou des math\'ematiques assez bonnes, on pouvait d\'eduire le bon choix à partir de principes premiers, et produire des descriptions a priori de la r\'ealit\'e, qui devaient être confirm\'ees après coup par une v\'erification empirique.

	Cependant, la g\'eom\'etrie euclidienne, consid\'er\'ee pendant plusieurs centaines d'ann\'ees comme le modèle de la perfection axiomatique des math\'ematiques, avait \'et\'e d\'etrôn\'ee. Si l'on ne pouvait connaître a priori quelque chose d'aussi fondamental que la g\'eom\'etrie dans l'espace, quel espoir restait-il pour une pure th\'eorie rationnelle qui embrasserait la totalit\'e de la nature ? Psychologiquement, Riemann et Lobatchevsky avaient frapp\'e au coeur de l'entreprise math\'ematique telle qu'elle avait \'et\'e conçue jusqu'alors.

	De plus, Riemann et Lobatchevsky remettaient la nature de l'intuition math\'ematique en question. Il avait \'et\'e facile de croire implicitement que l'intuition math\'ematique \'etait une forme de perception - une façon d'entrevoir le monde platonicien derrière la r\'ealit\'e. Mais avec deux autres g\'eom\'etries qui bousculaient celle d'Euclide, personne ne pouvait plus être sûr de savoir à quoi le monde ressemblait.

	Les math\'ematiciens r\'epondirent à ce double problème avec un excès de rigueur, en essayant d'appliquer la m\'ethode axiomatique à toutes la math\'ematique. Dans la p\'eriode pr\'e-axiomatique, les preuves reposaient souvent sur des intuitions commun\'ement admises de la "r\'ealit\'e" math\'ematique, qui ne pouvaient plus être consid\'er\'ees automatiquement comme valides.

	La nouvelle façon de penser la math\'ematique conduisait à une s\'erie de succès spectaculaires. Pourtant cela avait aussi un prix. La m\'ethode axiomatique rendait la connexion entre la math\'ematique et la r\'ealit\'e ph\'enom\'enale toujours plus \'etroite. En même temps, des d\'ecouvertes sugg\'eraient que les axiomes math\'ematiques qui semblaient être consistants avec l'exp\'erience ph\'enom\'enale pouvaient entraîner de vertigineuses contradictions avec cette exp\'erience.

	La majorit\'e des math\'ematiciens devinrent rapidement des "formalistes", soutenant que la math\'ematique pure ne pouvait qu'être consid\'er\'ees philosophiquement comme une sorte de jeu \'elabor\'e qui se jouait avec des signes sur le papier (c'est la th\'eorie qui sous-tend la proph\'etique qualification des math\'ematiques de "système à contenu nul" par Robert Heinlein). La croyance "platonicienne" en la r\'ealit\'e des objets math\'ematiques, à l'ancienne manière, semblait bonne pour la poubelle, malgr\'e le fait que les math\'ematiciens continuaient à se sentir comme les platoniciens durant le processus de d\'ecouverte des math\'ematiques.

	Philosophiquement, donc, la m\'ethode axiomatique conduisait la plupart des math\'ematiciens à abandonner les croyances ant\'erieures en la sp\'ecificit\'e m\'etaphysique des math\'ematiques. Elle produisit aussi la rupture contemporaine entre la math\'ematique pure et appliqu\'ee. La plupart des grands math\'ematiciens du d\'ebut de la p\'eriode moderne - Newton, Leibniz, Fourier, Gauss et les autres - s'occupaient aussi de science ph\'enom\'enale. La m\'ethode axiomatique avait couv\'e l'id\'ee moderne du math\'ematicien pur comme un super esthète, insoucieux de la physique. Ironiquement, le formalisme donnait aux purs math\'ematiciens un mauvais penchant à l'attitude platonicienne. Les chercheurs en math\'ematiques appliqu\'ees cessèrent de côtoyer les physiciens et apprirent à se mettre à leur traîne. 

	Ceci nous emmène au d\'ebut du 20ème siècle. Pour la minorit\'e assi\'eg\'ee des platoniciens, le pire \'etait encore à venir. Cantor, Frege, Russell et Whitehead montrèrent que toute la math\'ematique pure pouvait être construite sur le simple fondement axiomatique de la th\'eorie des ensembles. Cela convenait parfaitement aux formalistes: la math\'ematique se r\'eunifiait, du moins en principe, à partir d'un faisceau de petits jeux d\'etach\'es d'un grand. Les platoniciens aussi \'etaient satisfaits, s'il en survenait une grande structure, cl\'e de voûte consistante pour toute la math\'ematique, la sp\'ecificit\'e m\'etaphysique des math\'ematiques pouvait encore être sauv\'ee.

	D'une façon n\'egative, pourtant, un platonicien eut le dernier mot. Kurt Gödel mit son grain de sable dans le programme formaliste d'axiomatisation quand il d\'emontra que tout système d'axiomes assez puissant pour inclure les entiers devait être soit inconsistant (contenir des contradictions) soit incomplet (trop faible pour d\'ecider de la justesse ou de la fausset\'e de certaines affirmations du système). Et c'est plus ou moins où en sont les choses aujourd'hui. Les math\'ematiciens savent que de nombreuses tentatives pour faire avancer la math\'ematique comme une connaissance a priori de l'Univers doivent se heurter à de nombreux paradoxes et à l'impossibilit\'e de d\'ecider quel système axiomatique d\'ecrit la math\'ematique r\'eelle. Ils ont \'et\'e r\'eduits à esp\'erer que les axiomatisations standards ne soient pas inconsistantes mais incomplètes, et à se demander anxieusement quelles contradictions ou quels th\'eorèmes ind\'emontrables attendent d'être d\'ecouverts ailleurs.

	Cependant, sur le front de l'empirisme, la math\'ematique \'etait toujours un succès spectaculaire en tant qu'outil de construction th\'eorique. Les grands succès de la physique du 20ème siècle (la relativit\'e g\'en\'erale et la physique quantique) poussaient si loin hors du royaume de l'intuition physique, qu'ils ne pouvaient être compris qu'en m\'editant profond\'ement sur leurs formalismes math\'ematiques, et en prolongeant leurs conclusions logiques, même lorsque ces conclusions semblaient sauvagement bizarres. Quelle ironie! Au moment même où la perception math\'ematique en venait à paraître toujours moins fiable dans la math\'ematique pure, elle devenait toujours plus indispensable dans les sciences ph\'enom\'enales. 

	À l'oppos\'e de cet arrière-plan, l'applicabilit\'e des math\'ematiques à la science ph\'enom\'enale pose un problème plus \'epineux qu'il n'apparaît d'abord. Le rapport entre les modèles math\'ematiques et la pr\'ediction des ph\'enomènes est complexe, pas seulement dans la pratique mais dans le principe. D'autant plus complexe que, comme nous le savons maintenant, il y a des façons d'axiomatiser la math\'ematique qui s'excluent!

	Mais pourquoi existe-t-il seulement de bons choix de modèle math\'ematique ? C'est à dire, pourquoi y a-t-il un formalisme math\'ematique, par exemple pour la physique quantique, si productif qu'il pr\'edit r\'eellement la d\'ecouverte de nouvelles particules observables ?

	Pour r\'epondre à cette question nous observerons qu'elle peut, aussi bien, fonctionner comme une sorte de d\'efinition. Pour beaucoup de systèmes ph\'enom\'enaux, de tels formalismes pr\'edictifs exacts n'ont pas \'et\'e trouv\'es, et aucun ne semble plausible. Les poètes aiment marmonner sur le coeur des hommes, mais on peut trouver des exemples plus ordinaires: le climat, où le comportement d'une \'economie sup\'erieure à celle d'un village, par exemple - systèmes si chaotiquement interd\'ependants que la pr\'ediction exacte est effectivement impossible (pas seulement dans les faits mais en principe).

	\pagebreak
	\subsection{Paradoxes}
	Dès l'antiquit\'e, certains logiciens avaient constat\'e la pr\'esence de nombreux paradoxes au sein de la rationalit\'e. En fait, nous pouvons dire que malgr\'e leur nombre, ces paradoxes ne sont que les illustrations d'un petit nombre de structures paradoxales. Attardons-nous à exposer à titre de culture g\'en\'erale les plus connus qui constituent la classe des "\NewTerm{propositions ind\'ecidables}"\index{propositions ind\'ecidables}.

	\begin{tcolorbox}[colframe=black,colback=white,sharp corners]
	\textbf{{\Large \ding{45}}Exemple:}\\\\
	Le paradoxe de la classe des classes (Russell)\label{russell paradox} consiste en l'existence de deux types de classes: celles qui se contiennent elles-mêmes (ou classes r\'eflexives: la classe des ensembles non-vides, la classe des classes,...) et celles qui ne se contiennent pas elles-mêmes (ou classes irr\'eflexives: la classe des travaux à rendre, la classe des oranges sanguines, ...). La question pos\'ee est la suivante: la classe des classes irr\'eflexives est-elle elle-même r\'eflexive ou irr\'eflexive? Si elle est r\'eflexive, elle se contient et se trouve rang\'ee dans la classe des classes irr\'eflexives qu'elle constitue, ce qui est contradictoire. Si elle est irr\'eflexive, elle doit figurer dans la classe des classes irr\'eflexives qu'elle constitue et devient ipso facto r\'eflexive, nous sommes face à une nouvelle contradiction.\\
	
	Ce paradoxe de Russel est souvent connu principalement sous les deux variantes suivantes:
	\begin{itemize}
		\item Est-ce que l'ensemble de tous les ensembles qui ne se contiennent eux-mêmes se contient-il lui-même?\\
		
		La r\'eponse \'etant: Si "Oui", alors "Non" et si "Non" alors "Oui"...
		\begin{tcolorbox}[title=Remarque,colframe=black,arc=10pt]
		Une version religieuse amusante de ce paradoxe est la suivante: Le Tout-Puissant peut-il cr\'eer une entit\'e plus puissante que lui? Si non, alors il n'est pas tout-puissant! Si oui, alors il n'est pas tout-puissant...
		\end{tcolorbox}
	
		\item Ceux qui ne rasent pas eux-mêmes sont ras\'es par le barbier. Donc qui rase le barbier?\\
		
		La r\'eponse \'etant: Si le barbier se rase lui-même, il entre dans la cat\'egorie des personnes qui se rasent eux-même dès lors il ne rase pas lui-même parce qu'il est le barbier... Mais s'il ne se rase lui-même pas, il entre dans la cat\'egorie des personnes ras\'ees par le barbier. .. La r\'eponse est ind\'ecidable ...
	\end{itemize}
	\end{tcolorbox}

	Le paradoxe de Russel remet en cause la notion d'ensemble comme collection d\'efinie par une propri\'et\'e commune!! En un seul coup il d\'etruit la logique (proposition ind\'ecidable) et la th\'eorie des ensembles... car le concept d'ensemble de tous les ensembles est une impossibilit\'e. L'autor\'ef\'erence est au centre de ce problème de logique!

Ce paradoxe revient aussi à se poser la question si une question math\'ematique correctement formul\'ee (logique) admet n\'ecessairement une r\'eponse? Autrement, dit, tout \'enonc\'e math\'ematique est-il prouvable... et c'est Gödel qui bien des ann\'ees après l'\'enonc\'e du paradoxe de Russel prouva math\'ematiquement que la r\'eponse est Non!!!!!! En d'autres termes, il y aura toujours des questions sans r\'eponses car tout système (langue vivant ou outil math\'ematique) bas\'e sur lui-même est n\'ecessairement incomplet! C'est le fameux Th\'eorème d'Incompl\'etude de Gödel! C'est la fameuse cons\'equence du th\'eorème d'incompl\'etude de Gödel qui est techniquement \'ecrit comme suit (et que nous devrons prouver):
	
	cela signifie que pour une proposition $P$, quel que soit le jeu d'axiomes, il existe une proposition que nous ne pourrons jamais prouver comme \'etant vraie ou fausse.
	
	Voyons une autre application du paradoxe de Russell:
	\begin{tcolorbox}[colframe=black,colback=white,sharp corners]
	\textbf{{\Large \ding{45}}Exemple:}\\\\
	Le paradoxe du biblioth\'ecaire (Gonseth) consiste en l'existence d'une bibliothèque où il existe deux types de catalogues. Ceux qui se mentionnent eux-mêmes et ceux qui ne se mentionnent pas. Un biblioth\'ecaire doit dresser le catalogue de tous les catalogues qui ne se mentionnent pas eux-mêmes. Arriv\'e au terme de son travail, notre biblioth\'ecaire se demande s'il convient ou non de mentionner le catalogue qu'il est pr\'ecis\'ement en train de r\'ediger. A ce moment, il est frapp\'e de perplexit\'e. S'il ne le mentionne pas, ce catalogue sera un catalogue qui ne se mentionne pas et qui devra dès lors figurer dans la liste des catalogues ne se mentionnant pas eux-mêmes. D'un autre côt\'e, s'il le mentionne, ce catalogue deviendra un catalogue qui se mentionne et qui ne doit donc pas figurer dans ce catalogue, puisque celui-ci est le catalogue des catalogues qui ne se mentionnent pas.
	\end{tcolorbox}
	
	Une variance de ce paradoxe est le bien connu "paradox du menteur":

	\begin{tcolorbox}[colframe=black,colback=white,sharp corners]
	\textbf{{\Large \ding{45}}Exemple:}\\\\
	D\'efinissons provisoirement le mensonge comme l'action de formuler une proposition fausse. Le poète cr\'etois Epim\'enide affirme: "Tous les Cr\'etois sont des menteurs", soit la proposition $P$. Comment d\'ecider de la valeur de v\'erit\'e de $P$ ? Si $P$ est vraie, comme Epim\'enide est Cr\'etois, $P$ doit être fausse. Il faut donc que $P$ soit fausse pour pouvoir être vraie, ce qui est contradictoire. $P$ est donc fausse. Remarquons qu'on ne peut pas en d\'eduire, comme dans le v\'eritable paradoxe du menteur, que $P$ doit aussi être vraie.
	\end{tcolorbox}

	Comme l'aurait fait comporendre Ludwig Wittgenstein aux logicien, ces paradoxes montrent qu finalement les math\'ematiques sont un assez bon outil pour montrer la logique mais pas pour en parler. Donner avec les math\'ematiques un existence ind\'ependante à ses entit\'es alg\'ebriques est de la folie et c'est ce qui produit des monstres comme l'ensemble de tous les ensembles... La logique est vide et elle ne peut dire la r\'ealit\'e, elle se restreindre à être juste une image de celle-ci.

	\pagebreak
	\subsubsection{Raisonnement Hypoth\'etico-D\'eductif}
	Le raisonnement hypoth\'etico-d\'eductif est, nous le savons, la capacit\'e qu'a l'apprenant de d\'eduire des conclusions à partir de pures hypothèses et pas seulement d'une observation r\'eelle. C'est un processus de r\'eflexion qui tente de d\'egager une explication causale d'un ph\'enomène quelconque (nous y reviendrons lors de nos premiers pas en physique). L'apprenant qui utilise ce type de raisonnement commence par formuler une hypothèse et essaie ensuite de confirmer ou d'infirmer son hypothèse selon le sch\'ema synoptique ci-dessous:

	\begin{figure}[H]
		\begin{center}
			\includegraphics[scale=0.75]{img/intro/hypothesis_definitions.jpg}
		\end{center}	
		\caption{Diagramme synoptique du raisonnement hypoth\'etico-d\'eductif}
	\end{figure}
	La proc\'edure d\'eductive consiste à tenir pour vrai, à titre provisoire, cette proposition première que nous appelons, en logique "le pr\'edicat" (voir plus bas) et à en tirer toutes les cons\'equences logiquement n\'ecessaires, c'est-à-dire à en rechercher les implications.

	\begin{tcolorbox}[colframe=black,colback=white,sharp corners]
	\textbf{{\Large \ding{45}}Exemple:}\\\\
	Soit la proposition $P$: "$X$ est un homme", elle implique la proposition suivante $Q$: "$X$ est mortel".\\

	L'expression $P\Rightarrow Q$ (si c'est un homme il est n\'ecessairement mortel) est un implication pr\'edicative (d'où le terme "pr\'edicat"). Il n'y a pas dans cet exemple de cas où nous puissions \'enoncer $P$ sans $Q$. Cet exemple est celui d'une implication stricte, telle que nous la trouvons dans le "syllogisme" (figure logique du raisonnement).
	\end{tcolorbox}
	Beaucoup d'humains même en ce début du 3ème millénaire ont une dissonance cognitive intéressante relativement aux propositions. Typiquement ils vont affirmer que \textit{Toute chose a un créateur}, \textit{Dieu est CE créateur} mais que Dieu n'a pas de créateur. 

	\begin{tcolorbox}[title=Remarque,colframe=black,arc=10pt]
	Des sp\'ecialistes auraient montr\'e que le raisonnement  hypoth\'etico-d\'eductif  s'\'elabore progressivement chez l'enfant, à partir de 6-7ans, et que ce type de raisonnement n'est utilis\'e syst\'ematiquement, en partant d'une fonction propositionnelle stricte qu'à partir de 11-12 ans.
	\end{tcolorbox}
	
	\subsection{Calcul propositionnel}
	Le "\NewTerm{calcul propositionnel}"\index{calcul propositionnel} (ou "logique propositionnelle"\index{logique propositionnelle}) est un pr\'eliminaire absolument indispensable pour aborder une formation en sciences, philosophie, droit, politique, \'economie, etc. Ce type de calcul autorise des proc\'edures de d\'ecisions ou tests. Ceux-ci permettent de d\'eterminer dans quel cas une expression (proposition) logique est vraie et en particulier si elle est toujours vraie.

\textbf{D\'efinitions (\#\mydef):}

	\begin{enumerate}
		\item[D1.] Une expression toujours vraie quel que soit le contenu linguistique des variables qui la composent est appel\'ee une "\NewTerm{expression valide}"\index{expression valide}, une "\NewTerm{tautologie}"\index{tautologie} ou une "\NewTerm{loi de la logique propositionnelle}"\index{loi de la logique propositionnelle}.
		
		\item[D2.] Une expression toujours fausse est appel\'ee une "\NewTerm{contradiction}"\index{contradiction} ou "{antilogie}"\index{antilogie}.
		
		\item[D3.] Une expression qui est parfois vraie, parfois fausse est appel\'ee une "\NewTerm{expression contingente}"\index{expression contingente}.

		\item[D4.] Nous appelons "\NewTerm{assertion}"\index{assertion} une expression dont nous pouvons dire sans ambiguït\'e si elle est vraie ou fausse.

		\item[D5.] Le "\NewTerm{langage objet}"\index{langage objet} est le langage utilis\'e pour \'ecrire les expressions logiques.

		\item[D6.] Le  "\NewTerm{m\'etalangage}"\index{m\'etalangage} est le langage utilis\'e pour parler du langage objet dans la langue courante.
	\end{enumerate}

	\begin{tcolorbox}[title=Remarques,colframe=black,arc=10pt]
	\textbf{R1.}  Il existe des expressions qui ne sont effectivement pas des assertions. Par exemple, l'\'enonc\'e: "cet \'enonc\'e est faux", est un paradoxe qui ne peut être ni vrai, ni faux.\\

	\textbf{R2.} Soit une expression logique $A$. Si celle-ci est une tautologie, nous la notons fr\'equemment $\models A$ et si l'expression est une contradiction, nous la notons $A \models$.\\

	\textbf{R3.} En math\'ematiques on peut chercher à d\'emontrer de façon g\'en\'erale qu'une assertion est vraie mais pas qu'elle est fausse (le cas \'ech\'eant on donne juste un exemple).
	\end{tcolorbox}

	\pagebreak
	\subsubsection{Propositions (pr\'emisses)}

	\textbf{D\'efinition (\#\mydef):}  En logique, une "\NewTerm{proposition}"\index{proposition} est une affirmation qui a un sens. Cela veut dire que nous pouvons dire sans ambiguït\'e si cette affirmation est vraie ($V$) ou fausse ($F$). C'est ce que nous appelons le "\NewTerm{principe du tiers exclu}"\index{principe du tiers exclu}.


	\begin{tcolorbox}[colframe=black,colback=white,sharp corners]
	\textbf{{\Large \ding{45}}Exemples:}\\\\
	E1. "Je mens" n'est pas une proposition. Si nous supposons que cette affirmation est vraie, elle est une affirmation de sa propre invalidit\'e, donc nous devrions conclure qu'elle est fausse. Mais si nous supposons qu'elle est fausse, alors l'auteur de cette affirmation ne ment pas, donc il dit la v\'erit\'e, aussi la proposition serait vraie.\\
	
	E2. Un autre exemple rigolo est:
	\begin{itemize}
		\item Tout a \'et\'e cr\'e\'e par un cr\'eateur
		\item Dieu est un cr\'eateur
		\item Dieu n'a pas de cr\'eateur
	\end{itemize}
	C'est une solution qui \'echoue car elle viole ses propres pr\'emisses...
	\end{tcolorbox}

	\textbf{D\'efinition (\#\mydef):} Une proposition en logique binaire (où les propositions sont soit vraies, soit fausses) n'est donc jamais vraie et fausse à la fois. C'est que nous appelons le "\NewTerm{principe de non-contradiction}"\index{principe de non-contradiction}.

	Ainsi, une propri\'et\'e sur l'ensemble $E$ des propositions est une application $P$ de $E$ dans l'ensemble des "\NewTerm{valeurs de v\'erit\'e Vraie, Faux}\index{valeurs de v\'erit\'e (Vrai, Faux)}" $\left\lbrace T,F\right\rbrace$:
	
	Nous parlons de "\NewTerm{sous-ensemble associ\'e}"\index{sous-ensemble associ\'e}, lorsque la proposition engendre uniquement une partie $E'$ de $E$ et inversement.

	\begin{tcolorbox}[colframe=black,colback=white,sharp corners]
	\textbf{{\Large \ding{45}}Exemple:}\\\\
	Dans $E=\mathbb{N}$, si $P(x)$ s'\'enonce "$x$ est pair", alors $P=\{0,2,4,\ldots,2k,\ldots\}$ ce qui est bien seulement un sous-ensemble associ\'e de $E$ mais de même Cardinal (\SeeChapter{voir la section de Th\'eorie des Ensembles page \pageref{cardinal}}).	
	\end{tcolorbox}
	
	\pagebreak
	\textbf{D\'efinition (\#\mydef):} Soit $P$ une propri\'et\'e sur l'ensemble $E$. Une propri\'et\'e $Q$ sur $E$ est une "\NewTerm{n\'egation}"\index{n\'egation} de $P$ si et seulement si, pour tout $ x \in E$:
	\begin{itemize}
		\item $Q(x)$ est $F$ (faux) si $P(x)$ est $V$ (vrai)
		\item $Q(x)$ est $V$ (vrai) si $P(x)$ est $F$ (faux)
	\end{itemize}
	Nous pouvons rassembler ces conditions dans une table dite "\NewTerm{table de v\'erit\'e}"\index{table de v\'erit\'e}:
	\begin{table}[H]
	\begin{center}
		\definecolor{gris}{gray}{0.85}
			\begin{tabular}{|p{2cm}|p{2cm}|}
				\hline
				\multicolumn{1}{c}{\cellcolor{black!30}\textbf{$P$}} & 
  \multicolumn{1}{c}{\cellcolor{black!30}\textbf{$Q$}} \\ \hline
				\centering\arraybackslash\ $V$ & \centering\arraybackslash\ $F$ \\ \hline
				\centering\arraybackslash\ $F$ & \centering\arraybackslash\ $V$  \\ \hline
		\end{tabular}
	\end{center}
	\caption{Table de v\'erit\'e des valeurs}
	\end{table}	
	Table que nous pouvons aussi trouver ou donc aussi \'ecrire sous la forme plus explicite suivante:
	\begin{table}[H]
	\begin{center}
		\definecolor{gris}{gray}{0.85}
			\begin{tabular}{|p{2cm}|p{2cm}|}
				\hline
				\multicolumn{1}{c}{\cellcolor{black!30}\textbf{$P$}} & 
  \multicolumn{1}{c}{\cellcolor{black!30}\textbf{$Q$}} \\ \hline
				\centering\arraybackslash\ Vrai & \centering\arraybackslash\ Faux \\ \hline
				\centering\arraybackslash\ Faux & \centering\arraybackslash\ Vrai \\ \hline
		\end{tabular}
	\end{center}
	\caption{Table de vérité des valeurs explicites}
	\end{table}
	ou encore sous forme binaire:
	\begin{table}[H]
	\begin{center}
		\definecolor{gris}{gray}{0.85}
			\begin{tabular}{|p{2cm}|p{2cm}|}
				\hline
				\multicolumn{1}{c}{\cellcolor{black!30}\textbf{$P$}} & 
  \multicolumn{1}{c}{\cellcolor{black!30}\textbf{$Q$}} \\ \hline
				\centering\arraybackslash\ 1 & \centering\arraybackslash\ 0 \\ \hline
				\centering\arraybackslash\ 0 & \centering\arraybackslash\ 1 \\ \hline
		\end{tabular}
	\end{center}
	\caption{Table de vérité des valeurs binaires}
	\end{table}	
	En d'autres termes, $P$ et $Q$ ont toujours des valeurs de v\'erit\'e contraires. Nous noterons ce genre d'\'enonc\'e "$Q$ est une n\'egation de $P$":	
	où le symbole $\neg$ est le "\NewTerm{connecteur de n\'egation}"\index{connecteur de n\'egation}.

	\begin{tcolorbox}[title=Remarque,colframe=black,arc=10pt]
	Les expressions doivent être des expressions bien form\'ees (souvent abr\'eg\'e "EBF"). Par d\'efinition, toute variable est une expression bien form\'ee, alors $\neg P$ est aussi une expression bien form\'ee. Si $P, Q$ sont des expressions bien form\'ees, alors $P \Rightarrow Q$ est une expression bien form\'ee (l'expression "je mens" n'est pas bien form\'ee car elle se contredit elle-même).
	\end{tcolorbox}

	\pagebreak
	\subsubsection{Connecteurs}
	Il y a d'autres types de connecteurs en logique:

\textbf{D\'efinition (\#\mydef):} Soient $P$ et $Q$ deux propri\'et\'es d\'efinies sur le même ensemble $E$. $P\vee Q $ (lire "$P$ \texttt{OU} $Q$") est une propri\'et\'e sur $E$ d\'efinie par:

	\begin{itemize}
		\item $P\vee Q$ est vraie si au moins l'une des propri\'et\'es $P$ ou $Q$ est vraie
		\item $P\vee Q$ est fausse sinon
	\end{itemize}

	Nous pouvons cr\'eer la table de v\'erit\'e du "\NewTerm{connecteur \texttt{OU}}"\index{connecteur OU} ou "\NewTerm{connecteur de disjonction}"\index{connecteur de disjonction} $\vee$:
	\begin{table}[H]
	\begin{center}
		\definecolor{gris}{gray}{0.85}
			\begin{tabular}{|p{2cm}|p{2cm}|p{2cm}|}
				\hline
				\multicolumn{1}{c}{\cellcolor{black!30}\textbf{$P$}} & 
  \multicolumn{1}{c}{\cellcolor{black!30}\textbf{$Q$}}  & \multicolumn{1}{c}{\textbf{$P \vee Q$}} \\ \hline
				\centering\arraybackslash\ $V$ & \centering\arraybackslash\ $V$ & \centering\arraybackslash\ $V$ \\ \hline
				\centering\arraybackslash\ $V$ & \centering\arraybackslash\ $F$ & \centering\arraybackslash\ $V$  \\ \hline
				\centering\arraybackslash\ $F$ & \centering\arraybackslash\ $V$ & \centering\arraybackslash\ $V$  \\ \hline
				\centering\arraybackslash\ $F$ & \centering\arraybackslash\ $F$ & \centering\arraybackslash\ $F$  \\ \hline
		\end{tabular}
	\end{center}
	\caption{Table de v\'erit\'e \texttt{OU}}
	\end{table}	
	Il devrait être facile de se convaincre que, si les parties $P$, $Q$ de $E$ sont respectivement associ\'ees aux propri\'et\'es $P$, $Q$ que $P \cup Q$ (\SeeChapter{voir section de Th\'eorie des Ensembles page \pageref{union}}) est associ\'e à $P \vee Q$:
	
	Le connecteur  $\vee$ est associatif. Pour s'en convaincre, il suffit de faire une table de v\'erit\'e où nous v\'erifions que:
	
	\textbf{D\'efinition (\#\mydef):} Il y a aussi le "\NewTerm{connecteur \texttt{AND}}"\index{connecteur ET} aussi nomm\'e "\NewTerm{connecteur de conjonction}"\index{connecteur de conjonction} $\wedge$ pour quel que soient  $P$, $Q$   deux propri\'et\'es d\'efinies sur $E$, $P \wedge Q$ est une propri\'et\'e sur $E$ d\'efinie par $E$:
	\begin{itemize}
		\item $P \wedge Q$ est vraie si toutes les deux propri\'et\'es $P$, $Q$ sont vraies (le syllogisme très connu: Tous les hommes sont mortels, Socrate est un homme donc Socrate est mortel en est un exemple fameux).
		
		\item $P \wedge Q$ est fausse sinon
	\end{itemize}
	Nous pouvons cr\'eer la table de v\'erit\'e du "\NewTerm{connecteur \texttt{ET}}"\index{connecteur ET} ou "\NewTerm{connecteur de conjonction}"\index{connecteur de conjonction} $\wedge$:
	\begin{table}[H]
	\begin{center}
		\definecolor{gris}{gray}{0.85}
			\begin{tabular}{|p{2cm}|p{2cm}|p{2cm}|}
				\hline
				\multicolumn{1}{c}{\cellcolor{black!30}\textbf{$P$}} & 
  \multicolumn{1}{c}{\cellcolor{black!30}\textbf{$Q$}}  & \multicolumn{1}{c}{\textbf{$P \wedge Q$}} \\ \hline
				\centering\arraybackslash\ $V$ & \centering\arraybackslash\ $V$ & \centering\arraybackslash\ $V$ \\ \hline
				\centering\arraybackslash\ $V$ & \centering\arraybackslash\ $F$ & \centering\arraybackslash\ $F$  \\ \hline
				\centering\arraybackslash\ $F$ & \centering\arraybackslash\ $V$ & \centering\arraybackslash\ $F$  \\ \hline
				\centering\arraybackslash\ $F$ & \centering\arraybackslash\ $F$ & \centering\arraybackslash\ $F$  \\ \hline
		\end{tabular}
	\end{center}
	\caption{Table de v\'erit\'e \texttt{ET}}
	\end{table}
	Il est \'egalement facile de se convaincre que, si les parties  $P, Q$ de $E$ sont respectivement associ\'ees aux propri\'et\'es $P$, $Q$ que $P \cap Q$ (\SeeChapter{voir section de Th\'eorie des Ensembles page \pageref{intersection}}) est associ\'e à $P \wedge Q$:
	
	Le connecteur $\wedge$  est associatif. Pour s'en convaincre, il suffit aussi de faire une table de v\'erit\'e où nous v\'erifions que:
	
	Les connecteurs $\vee,\wedge$ sont distributifs l'un sur l'autre. A l'aide d'une simple table de v\'erit\'e, nous prouvons que:
	
	ainsi que:
	
	
	\textbf{D\'efinition (\#\mydef):} L'op\'erateur de "\NewTerm{n\'egation}"\index{n\'egation}    $\lnot$ transforme une valeur Vraie en une valeur Fausse tel que:
	
	Ainsi, en logique, la n\'egation, \'egalement appel\'ee "\NewTerm{compl\'ement logique}"\index{compl\'ement logique},est une op\'eration qui transforme une proposition $P$ en une autre proposition "non $P$", not\'ee $\lnot P$ ou parfois $\bar{P}$, qui est interpr\'et\'ee intuitivement \'etant Vraie lorsque $P$ est fausse et Faux lorsque $P$ est Vraie. La n\'egation est alors une op\'eration logique unaire (c'est-à-dire avec avec une seul argument).
	
	Comme nous le prouverons en d\'etails dans la section de Systèmes Logiques à la page \pageref{de morgan theorem} (en utilisant une simple table de v\'erit\'e), les  "\NewTerm{loi de De Morgan}"\index{loi de De Morgan} fournissent un moyen de changer une disjonction et une conjonction:
	
	\begin{tcolorbox}[title=Remarque,colframe=black,arc=10pt]
	Pour voir les d\'etails de tous les op\'erateurs logiques, le lecteur devra se rendre dans la section de Systèmes Logiques à la page \pageref{logical systems} où l'identit\'e, la double n\'egation, l'idempotence, l'associativit\'e, la distributivit\'e, les relations de De Morgan sont pr\'esent\'ees plus formellement.
	\end{tcolorbox}
	Revenons maintenant sur le "\NewTerm{connecteur d'implication logique}" \index{connecteur d'implication logique} appel\'e aussi parfois le "\NewTerm{conditionnel}"\index{conditionnel} not\'e $\Rightarrow$.

	\begin{tcolorbox}[title=Remarque,colframe=black,arc=10pt]
	Dans certains ouvrages sur le calcul propositionnel, ce connecteur est not\'e " $\supset$" et dans le cadre de la th\'eorie de la d\'emonstration nous lui pr\'ef\'erons souvent le symbole "$\rightarrow$".
	\end{tcolorbox}

	Soient $P, Q$ deux propri\'et\'es sur $E$. $P \Rightarrow Q$ est une propri\'et\'e sur $E$ d\'efinie par:
	\begin{enumerate}
		\item[P1.] $P \Rightarrow Q$ est Faux si $P$ est Vraie et $Q$ est Faux.
		\item[P2.] $P \Rightarrow Q$ est Vraie sinon.
	\end{enumerate}
	En d'autres termes, $P$ implique logiquement $Q$ signifie que $Q$ est vraie pour toute \'evaluation pour laquelle $P$ est vraie. L'implication repr\'esente donc le "si... alors.."
	
	Si nous \'ecrivons la table de v\'erit\'e de l'implication (attention à l'avant-dernière ligne !!!):
	\begin{table}[H]
	\begin{center}
		\definecolor{gris}{gray}{0.85}
			\begin{tabular}{|p{2cm}|p{2cm}|p{2cm}|}
				\hline
				\centering\arraybackslash\ \cellcolor{black!30}\textbf{$P$} & 
  \centering\arraybackslash\ \cellcolor{black!30}\textbf{$Q$}  & \centering\arraybackslash\ \textbf{$P \Rightarrow Q$} \\ \hline
				\centering\arraybackslash\ $V$ & \centering\arraybackslash\ $V$ & \centering\arraybackslash\ $V$ \\ \hline
				\centering\arraybackslash\ $V$ & \centering\arraybackslash\ $F$ & \centering\arraybackslash\ $F$  \\ \hline
				\centering\arraybackslash\ $F$ & \centering\arraybackslash\ $V$ & \centering\arraybackslash\ $V$  \\ \hline
				\centering\arraybackslash\ $F$ & \centering\arraybackslash\ $F$ & \centering\arraybackslash\ $V$  \\ \hline
		\end{tabular}
	\end{center}
	\caption{Table de v\'erit\'e de l'implication}
	\end{table}
	Si $P\Rightarrow Q$, nous pouvons dire que pour que $Q$ soit vraie, il suffit que $P$ soit vraie (effectivement l'implication sera vraie si $P$ est vraie ou fausse selon la table de v\'erit\'e). Donc $P$ est une condition suffisante de $Q$ (mais non n\'ecessaire!). D'un autre côt\'e, $P\Rightarrow Q$ est \'equivalent à $\neq Q\Rightarrow \neq P$ (nous parlons alors de "contrapos\'ee"). Donc, si $Q$ est fausse, il est impossible que $P$ soit vraie (pour que l'implication reste vraie bien sûr!). Donc finalement $Q$ est une condition n\'ecessaire de $P$.
	
	En d'autres termes, une proposition fausse implique que toute conclusion sera toujours vraie. Ceci est nomm\'e un "\NewTerm{sophisme informel}\index{sophisme informel}" et est un concept très important dans les tests statistiques d'hypothèses nulles (\SeeChapter{voir la section Statistiques page \pageref{p value}}). Si la proposition est vraie, l'implication ne peut être vraie que si le r\'esultat est vrai!
	
	\begin{tcolorbox}[title=Remarque,colframe=black,arc=10pt]
	À proprement parler, la table ci-dessus doit utiliser le symbole $\rightarrow$ et le symbole $\Rightarrow $ doit être r\'eserv\'e uniquement pour les implications vraies!
	\end{tcolorbox}
	
	\begin{tcolorbox}[colframe=black,colback=white,sharp corners]
	\textbf{{\Large \ding{45}}Exemple:}\\\\
	Soit la proposition: "Si tu obtiens ton diplôme, je t'achète un ordinateur".\\
	
	Parmi tous les cas, un seul correspond à une promesse non tenue: celui où l'enfant à son diplôme, et n'a toujours pas d'ordinateur (deuxième ligne dans le tableau ci-dessus).\\
	
	Et le cas où il n'a pas le diplôme, mais reçoit quand même un ordinateur? Il est possible qu'il ait \'et\'e longtemps malade et a rat\'e un semestre, et le père a le droit d'être bon.\\
	
	Que signifie cette promesse, que nous \'ecrirons aussi:
	\begin{center}
	"Tu as ton diplôme" $\Rightarrow$ " je t'achète un ordinateur"?
	\end{center}
	
	Exactement ceci:
	\begin{itemize}
		\item Si tu as ton diplôme, c'est sûr, je t'achète un ordinateur (je ne peux pas ne pas l'acheter)
		
		\item Si tu n'as pas ton diplôme, je n'ai rien dit
	\end{itemize}
	\end{tcolorbox}
	De toute proposition fausse nous pouvons d\'eduire toute proposition (deux dernières lignes)!
	\begin{tcolorbox}[title=Remarque,colframe=black,arc=10pt]
	L'erreur informelle est très int\'eressante à appliquer dans le cas de d\'ebats sur les r\'eseaux sociaux. En effet, si quelqu'un vous demande de prouver votre pr\'emisse (hypothèse nulle) qu'il suppose fausse, il ne comprend rien à la logique, car il doit savoir que, de toute fausse pr\'emisse (hypothèse nulle), vous pouvez toujours avoir une conclusion vraie (vous ne rejetez pas l'hypothèse nulle) quel que soit le raisonnement! Ce que la personne doit faire, c'est supposer que votre hypothèse nulle est vraie et avoir trouv\'e au moins un exemple - ou plusieurs dans le meilleur des cas - la rendant significativement fausse en ayant suivi la m\'ethode scientifique (nous rejetons l'hypothèse nulle en faveur de l'hypothèse alternative). C’est ainsi que la science fonctionne et le fait que cela n’est pas \'evident explique pourquoi tant de gens perdent leur temps à d\'ebattre dans les r\'eseaux sociaux en utilisant des arguments biais\'es (erreur informelle).
	\end{tcolorbox}

	\begin{tcolorbox}[colframe=black,colback=white,sharp corners]
	\textbf{{\Large \ding{45}}Exemple:}\\\\
	C'est un exemple plutôt anecdotique: dans un cours de Russell portant sur le fait que d'une proposition fausse, toute proposition peut être d\'eduite, un \'etudiant lui posa la question suivante:
	\begin{itemize}
		\item "Pr\'etendez-vous que de $2 + 2 = 5$, il s'ensuit que vous êtes le pape ? "
		\item "Oui", r\'epondit Russell.
		\item ""Et pourriez-vous le prouver ?!", demanda l'\'etudiant sceptique-
		\item "Certainement", r\'epliqua Russell, qui proposa sur le champ la d\'emonstration suivante.
			\begin{enumerate}
				\item Supposons que  $2 + 2 = 5$.
				\item Soustrayons $2$ de chaque membre de l'\'egalit\'e, nous obtenons $2 = 3$.
				\item Par sym\'etrie $2=1$.
				\item Maintenant le pape et moi sommes deux. Puisque $2 = 1$, le pape et moi sommes un. Par suite, je suis le pape!
			\end{enumerate}
	\end{itemize}
	Sur ce ...
	\end{tcolorbox}
	Le connecteur d'implication est essentiel en math\'ematiques, philosophie, etc. C'est un des fondements de toute d\'emonstration, preuve ou d\'eduction. Le connecteur d'implication a comme propri\'et\'es (v\'erifiables à l'aide de la table de v\'erit\'e ci-dessous):
	
	Cons\'equence de la dernière propri\'et\'e (à nouveau v\'erifiable par une table de v\'erit\'e):
	
	Le "\NewTerm{connecteur logique d'\'equivalence}"\index{connecteur logique d'\'equivalence}  ou "\NewTerm{connecteur biconditionnel}"\index{connecteur biconditionnel} not\'e la majorit\'e du temps "$\Leftrightarrow$" ou parfois "$\leftrightarrow$" signifie par d\'efinition:
	
	En d'autres termes, la première expression a la même valeur pour toute \'evaluation de la deuxième. Il est de même de la relation suivante qui est plus "atomique" puisque l'\'equivalence logique se r\'eduit uniquement à l'utilisation et $\wedge,\vee$ et de la n\'egation $\lnot$  (combinaison de ce que nous avons vu plus haut):
	
	Lorsque nous d\'emontrons une telle \'equivalence de deux expressions nous disons alors que: "nous prouvons que l'\'equivalence est une tautologie".

	La table de v\'erit\'e de l'\'equivalence est donn\'ee logiquement par:
	\begin{table}[H]
	\begin{center}
		\definecolor{gris}{gray}{0.85}
			\begin{tabular}{|p{2cm}|p{2cm}|p{2cm}|p{2cm}|p{2cm}|}
				\hline
				\centering\arraybackslash\ \cellcolor{black!30}\textbf{$P$} & 
  \centering\arraybackslash\ \cellcolor{black!30}\textbf{$Q$}  & \centering\arraybackslash\ \textbf{$P \Rightarrow Q$}  & \centering\arraybackslash\ \textbf{$Q \Rightarrow P$} & \centering\arraybackslash\ \textbf{$Q \Leftrightarrow P$}\\ \hline
				\centering\arraybackslash\ $V$ & \centering\arraybackslash\ $V$ & \centering\arraybackslash\ $V$ & \centering\arraybackslash\ $V$ & \centering\arraybackslash\ $V$ \\ \hline
				\centering\arraybackslash\ $V$ & \centering\arraybackslash\ $F$ & \centering\arraybackslash\ $F$ & \centering\arraybackslash\ $V$ & \centering\arraybackslash\ $F$ \\ \hline
				\centering\arraybackslash\ $F$ & \centering\arraybackslash\ $V$ & \centering\arraybackslash\ $V$ & \centering\arraybackslash\ $F$ & \centering\arraybackslash\ $F$\\ \hline
				\centering\arraybackslash\ $F$ & \centering\arraybackslash\ $F$ & \centering\arraybackslash\ $V$ & \centering\arraybackslash\ $V$ & \centering\arraybackslash\ $V$ \\ \hline
		\end{tabular}
	\end{center}
	\caption{Table de v\'erit\'e de l'\'equivalence logique}
	\end{table}
	$P \Leftrightarrow Q$ signifie bien (lorsqu'il est vrai!) que "$P$ et  $Q$ ont toujours la même valeur de v\'erit\'e" ou encore  "$P$ et  $Q$ sont \'equivalents". C'est vrai si $P$ et $Q$ ont même valeur, faux dans tout cas contraire.

Bien \'evidemment (c'est une tautologie):
	
	La relation $P \Leftrightarrow Q$ \'equivaut donc à ce que $P$ soit une condition n\'ecessaire et suffisante de $Q$ et à ce que $Q$ soit une condition n\'ecessaire et suffisante de $P$.
	
	\begin{tcolorbox}[title=Remarque,colframe=black,arc=10pt]
	À proprement parler, la table ci-dessus devrait utiliser le symbole $\leftrightarrow$ et le symbole $\Leftrightarrow$ ne devrait être r\'eserv\'e que pour les \'equivalences vraies.
	\end{tcolorbox}
	La conclusion, est que les conditions de type "n\'ecessaire, suffisant, n\'ecessaire et suffisant" peuvent être reformul\'ees avec les termes "seulement si", "si", "si et seulement si".

	Ainsi:
	\begin{enumerate}
		\item $P \Rightarrow Q$ traduit le fait que $Q$ est une condition \NewTerm{n\'ecessaire}\index{condition!n\'ecessaire} pour $P$ ou dit autrement, $P$ est Vraie \NewTerm{seulement si}\index{condition!seulement si} $Q$ est Vraie (dans la table de v\'erit\'e, lorsque $P \Rightarrow Q$ prend la valeur $1$ on constate bien que $P$ vaut $1$ \NewTerm{seulement si} $Q$ vaut $1$ aussi). On dit aussi, \NewTerm{si} $P$ est Vraie \NewTerm{alors} $Q$ est Vraie.
				
		\item $P \Longleftarrow Q$ ou ce qui revient au même $Q \Rightarrow P$ traduit le fait que $Q$ est une condition \NewTerm{suffisante}\index{condition!suffisante} pour $P$ ou dit autrement, $P$ est Vraie \NewTerm{si} $Q$ est Vraie (dans la table de v\'erit\'e, lorsque $Q \Rightarrow P$ prend la valeur $1$ on constate bien que $P$ vaut $1$ \NewTerm{si} $Q$ vaut $1$ aussi).
		
		\item $P \Leftrightarrow Q$ traduit le fait que $Q$ est une condition \NewTerm{n\'ecessaire et suffisante}\index{n\'ecessaire et suffisante}\index{condition!n\'ecessaire et suffisante} pour $P$ ou dit autrement, $P$ est Vraie \NewTerm{si et seulement si}\index{condition: si et seulement si} $Q$ est Vraie (dans la table de v\'erit\'e, lorsque equation prend la valeur $1$ on constate bien que $P$ vaut $1$ si $Q$ vaut $1$ et \NewTerm{si et seulement si} $Q$ vaut $1$).
	\end{enumerate}

	\begin{tcolorbox}[title=Remarque,colframe=black,arc=10pt]
	 L'expression "si et seulement si" correspond donc a une \'equivalence logique et ne peut être utilis\'ee que pour d\'ecrire une bi-implication!!
	\end{tcolorbox}
	
	La première \'etape du calcul propositionnel est donc la formalisation des \'enonc\'es du langage naturel. Pour r\'ealiser ce travail, le calcul propositionnel fournit finalement trois types d'outils:
	\begin{enumerate}
		 \item Les "\NewTerm{variables propositionnelles}\index{variables propositionnelles}" ($P$, $Q$, $R$,...) symbolisent des propositions simples quelconques. Si la même variable apparaît plusieurs fois, elle symbolise chaque fois la même proposition.
		 
		 \item Les Les cinq op\'erateurs logiques: $\neg , \wedge, \vee, \Leftrightarrow, \Rightarrow$.
		 
		 \item Les signes de ponctuation se r\'eduisent aux seules parenthèses ouvrante et fermante qui organisent la lecture de manière à \'eviter toute ambiguït\'e.
	\end{enumerate}
	Voici un tableau r\'ecapitulatif:
	\begin{table}[H]
	\begin{center}
		\definecolor{gris}{gray}{0.85}
			\begin{tabular}{|p{12cm}|p{1cm}|p{1.5cm}|}
				\hline
				\multicolumn{1}{c}{\cellcolor{black!30}\textbf{Description}} & 
  \multicolumn{1}{c}{\cellcolor{black!30}\textbf{Symbole}} & 
  \multicolumn{1}{c}{\cellcolor{black!30}\textbf{Utilit\'e}}  \\ \hline
				La "\NewTerm{n\'egation}" est un op\'erateur qui ne porte que sur une proposition; il est unaire et monadique. "\textit{Il ne pleut pas}" s'\'ecrit $\neg P$. Cet \'enonc\'e est vrai si et seulement si $P$ est faux (dans ce cas s'il est faux qu'il pleut!). L'usage classique de la n\'egation est caract\'eris\'e par la loi de double n\'egation: $\neg \neg P$ est \'equivalent à $P$. & $\neg$ & $\neg P$ \\ \hline
				
				La "\NewTerm{conjection}" ou "\NewTerm{produit logique}" est un op\'erateur binaire; elle met en relation deux propositions. "\textit{Tout homme est mortel ET Ma voiture perd de l'huile}" s'\'ecrit $P \wedge Q$. Cette dernière epression est vrai si et seulement si $P$ vraie et $Q$ est vraie. & $\wedge$ & $P \wedge Q$ \\ \hline
				
				La "\NewTerm{disjonction}\index{disjonction}" ou "\NewTerm{somme logique}\index{somme logique}" est, elle aussi, un op\'erateur binaire. $P \vee Q$ est vraie si et seulement si $P$ est vraie OU $Q$ est vraie ou les deux sont vraies. Nous pouvons comprendre le OU de deux façons: soit de manière inclusive, soit de manière exclusive. Dans le premier cas $P \vee Q$ est vrai si $P$ est vraie, si $Q$ est vraie ou si $P$ et $Q$ sont toutes deux vraies. Dans le second cas, $P \vee Q$ est vraie si $P$ est vraie ou si $Q$ vraie mais pas si les deux le sont. La disjonction du calcul propositionnel est le OU inclusif et on donne au OU exclusif, c'est-à-dire le XOR, le nom "\NewTerm{alternative}\index{alternative}". &  $\vee$ & $P \vee Q$ \\ \hline
				
				"\NewTerm{L'implication}\index{implication}" ou "\NewTerm{inf\'erence}\index{inf\'erence}" est \'egalement un op\'erateur binaire. Elle correspond, en gros, au sch\'ema linguistique  "\textit{Si ... alors ...}". "\textit{Si j'ai le temps, j'irai au cin\'ema}" sera not\'e $P \Rightarrow Q$. Cette dernière relation est faussi $P$ est vraie et $Q$ est fausse. Si le cons\'equent (ici $Q$) est vraie, l'implication $P \Rightarrow Q$ est vraie. Lorsque l'ant\'ec\'edent (ici $P$) est faux, l'implication est toujours vraie! Cette dernière remarque peut être comprise si l'on se r\'efère à des \'enonc\'es de type: "\textit{Si on pouvait mettre Paris en bouteille, on utiliserait la tour Eiffel comme bouchon}". En r\'esum\'e, une implication est fausse si et seulement si son ant\'ec\'edente est vraie et son cons\'equent est fausse. & $\Rightarrow$ & $P \Rightarrow Q$ \\ \hline
				
				La "\NewTerm{bi-implication}\index{bi-implication}" ou "\NewTerm{equivalence}\index{equivalence}" $\Rightarrow$ est, elle aussi, binaire: elle symbolise les expressions "... si et seulement si..." et "... est \'equivalent à..." L'\'equivalence entre deux propositions est vraie si celles-ci ont la même valeur de v\'erit\'e. La bi-implication exprime donc aussi une forme d'identit\'e et c'est pourquoi elle est souvent utilis\'ee dans les d\'efinitions. & $\Leftrightarrow$ & $P \Leftrightarrow Q$\\ \hline
		\end{tabular}
	\end{center}
	\caption{R\'ecapitulatif des op\'erateurs}
	\end{table}	
	Et la table de v\'erit\'e r\'esum\'ee correspondante:
	\begin{table}[H]
		\centering
		\begin{tabular}{|c|c|c|c|c|c|c|}
		\hline
		\rowcolor[HTML]{9B9B9B} 
		$P$ & $Q$ & $P\wedge Q$ & $P\vee Q$ & $P\Rightarrow Q$ & $P\Leftrightarrow Q$ & $\neg P$ \\ \hline
		$V$ & $V$ & $V$ & $V$ & $V$ & $V$ & $F$ \\ \hline
		$V$ & $F$ & $F$ & $V$ & $F$ & $F$ & $F$ \\ \hline
		$F$ & $V$ & $F$ & $V$ & $V$ & $F$ & $V$ \\ \hline
		$F$ & $F$ & $F$ & $F$ & $V$ & $V$ & $V$ \\ \hline
		\end{tabular}
		\caption{Table de v\'erit\'e des op\'erateurs logiques principaux}
	\end{table}
	Le lecteur peut trouver parfois chez certains auteurs qui n'aiment pas trop utiliser le langage naturel dans leurs livres (...) le symbole "$\therefore$" plac\'e parfois devant une cons\'equence logique, telle que la conclusion d'un syllogisme. Le symbole se compose de trois points plac\'es dans un triangle à l'endroit et qui doit être lu comme le mot "\textit{donc}". Nous pouvons \'egalement utiliser le symbole "$\because$", qui doit être lu comme le mot "\textit{parce que}".
	
	\begin{tcolorbox}[colframe=black,colback=white,sharp corners]
	\textbf{{\Large \ding{45}}Exemple:}\\\\
	$\because$ Tous les humains sont mortels.\\
	$\because$  Socrates est un humain.\\
	$\therefore$ Socrates est mortel.
	\end{tcolorbox}
	Dans ce livre, nous \'eviterons toutefois d'utiliser cette notation car les ing\'enieurs ne l'utilisent pas beaucoup.
	
	Il est possible d'\'etablir des \'equivalences entre les op\'erateurs pr\'esent\'es plus haut. Nous avons d\'ejà vu comment le biconditionnel pouvait se d\'efinir comme un produit de conditionnels r\'eciproques, voyons maintenant d'autres \'equivalences:
	
	\begin{tcolorbox}[title=Remarque,colframe=black,arc=10pt]
	Les op\'erateurs classiques $\wedge, \vee, \Leftrightarrow$ peuvent donc être d\'efinis à l'aide des op\'erateurs canoniques $\neg, \Rightarrow$ grâce aux lois d'\'equivalences entre op\'erateurs.
	\end{tcolorbox}
	Sont à noter \'egalement les deux relations de De Morgan (voir la section d'Algèbre Bool\'eenne page \pageref{de morgan theorem} pour la d\'emonstration):
	
	qui permettent de transformer la disjonction en conjonction et vice-versa:
	
	
	\pagebreak
	\subsubsection{Proc\'edures de d\'ecision}
	Nous avons introduit pr\'ec\'edemment les \'el\'ements de base nous permettant d'op\'erer sur des expressions à partir de propri\'et\'es (variables propositionnelles) sans toutefois dire grand-chose quant à la manipulation de ces expressions. Alors, il convient maintenant de savoir qu'en calcul propositionnel il existe deux manières d'\'etablir qu'une proposition est une loi de la logique propositionnelle. Nous pouvons soit:
	
	\begin{enumerate}
		\item Employer des proc\'edures non axiomatis\'ees
		
		\item Recourir à des proc\'edures axiomatiques et d\'emonstratives
	\end{enumerate}
	\begin{tcolorbox}[title=Remarque,colframe=black,arc=10pt]
	 Dans de nombreux ouvrages ces proc\'edures sont pr\'esent\'ees avant même la structure du langage propositionnel. Nous avons choisi de faire le contraire pensant que l'approche serait plus ais\'ee.
	\end{tcolorbox}
	
	\paragraph{Proc\'edures de d\'ecisions non axiomatis\'ees}\mbox{}\\\\
	Plusieurs de ces m\'ethodes existent mais nous nous limiterons ici à la plus simple et à la plus parlante d'entre elles, celle du calcul matriciel, souvent appel\'ee aussi "\NewTerm{m\'ethodes des tables de v\'erit\'e}\index{"m\'ethodes des tables de v\'erit\'e}".
	
	La proc\'edure de construction est comme nous l'avons vu pr\'ec\'edemment assez simple. Effectivement, la valeur de v\'erit\'e d'une expression complexe est fonction de la valeur v\'erit\'e des \'enonc\'es plus simples qui la composent, et finalement fonction de la valeur de v\'erit\'e des variables propositionnelles qui la composent. En envisageant toutes les combinaisons possibles des valeurs de v\'erit\'e des variables propositionnelles, nous pouvons d\'eterminer les valeurs de v\'erit\'e de l'expression complexe.
	
	Les tables de v\'erit\'e, comme nous l'avons vu, permettent donc de d\'ecider, à propos de toute proposition, si celle-ci est une tautologie (toujours vraie), une contradiction (toujours fausse) ou une expression contingente (parfois vraie, parfois fausse).

	Nous pouvons ainsi distinguer quatre façons de combiner les variables propositionnelles, les parenthèses et les connecteurs:
	\begin{table}[H]
	\begin{center}
		\definecolor{gris}{gray}{0.85}
			\begin{tabular}{|c|l|c|c|}
				\hline
				\multicolumn{1}{c}{\cellcolor{black!30}\textbf{}} & 
  \multicolumn{1}{c}{\cellcolor{black!30}\textbf{Nom}}  & \multicolumn{1}{c}{\cellcolor{black!30}\textbf{Description}} & \multicolumn{1}{c}{\cellcolor{black!30}\textbf{Exemple}} \\ \hline
				\cellcolor{black!30}\textbf{1} & Affirmation mal formul\'ee  & Absurdit\'e. Ni vrai, ni faux & \centering\arraybackslash\ $(\vee P)Q$ \\ \hline
				\cellcolor{black!30}\textbf{2} & Tautologie & Affirmation toujours vraie & \centering\arraybackslash\ $P \vee \neg P$  \\ \hline
				\cellcolor{black!30}\textbf{3} & Contradiction & Affirmation toujours fausse & \centering\arraybackslash\ $P \wedge \neg P$  \\ \hline
				\cellcolor{black!30}\textbf{4}  & Affirmation contingente & Affirmation parfois vraie, parfois fausse & \centering\arraybackslash\ $P\vee Q$  \\ \hline
		\end{tabular}
	\end{center}
	\caption{Combinaison de variables propositionnelles}
	\end{table}
	La m\'ethode des tables de v\'erit\'e permet de d\'eterminer le type d'expression bien form\'ee face auquel nous nous trouvons. Elle n'exige en principe aucune invention, c'est une proc\'edure m\'ecanique. Les proc\'edures axiomatis\'ees, en revanche, ne sont pas entièrement m\'ecaniques. Inventer une d\'emonstration dans le cadre d'un système axiomatis\'e demande parfois de l'habilit\'e, de l'habitude ou de la chance. Pour ce qui est des tables de v\'erit\'e, voici la marche à suivre:

	Lorsqu'on se trouve face à une expression bien form\'ee, ou fonction de v\'erit\'e, nous commençons par d\'eterminer à combien de variables propositionnelles distinctes nous avons affaire. Ensuite, nous examinons les diff\'erents arguments qui constituent cette expression. Nous construisons alors un tableau comprenant $2^n$ rang\'ees ($n$ \'etant le nombre de variables) et un nombre de colonnes \'egal au nombre d'arguments plus des colonnes pour l'expression elle-même et ses autres composantes. Nous attribuons alors aux variables les diff\'erentes combinaisons de v\'erit\'e et de fausset\'e qui peuvent leur être conf\'er\'ees (la v\'erit\'e est exprim\'ee dans la table par un $1$ et la fausset\'e par un $0$). Chacune des rang\'ees correspond à un monde possible et la totalit\'e des rang\'ees constitue l'ensemble des mondes possibles. Il existe, par exemple, un monde possible dans lequel $P$ est une proposition vraie tandis que $Q$ est fausse.
	
	\paragraph{Proc\'edures de d\'ecisions axiomatis\'ees}\mbox{}\\\\
	L'axiomatisation d'une th\'eorie implique, outre la formalisation de celle-ci, que nous partions d'un nombre fini d'axiomes et que, grâce à la transformation r\'egl\'ee de ces derniers, nous puissions obtenir tous les th\'eorèmes de cette th\'eorie. Nous partons donc de quelques axiomes dont la v\'erit\'e est pos\'ee (et non d\'emontr\'ee). Nous d\'eterminons des règles de d\'eduction permettant de manipuler les axiomes ou toute expression obtenue à partir de ceux-ci. L'enchaînement de ces d\'eductions est une d\'emonstration qui conduit à un th\'eorème, à une loi.
	
	Nous allons sommairement pr\'esenter deux systèmes axiomatiques, chacun \'etant constitu\'e d'axiomes utilisant deux règles dites "\NewTerm{règles d'inf\'erence}\index{règles d'inf\'erence}" (règles intuitives) particulières:
	\begin{enumerate}
		\item Le "\NewTerm{modus ponens}\index{modus ponens}": Si nous avons prouv\'e $A$ et $A\Rightarrow B$, alors nous pouvons d\'eduire $B$. $A$ est appel\'e la "\NewTerm{pr\'emisse mineure}\index{pr\'emisse mineure}" et $A\Rightarrow B$ la "\NewTerm{pr\'emisse majeure}\index{pr\'emisse majeure}" de la règle du modus ponens.
		\begin{tcolorbox}[colframe=black,colback=white,sharp corners]
		\textbf{{\Large \ding{45}}Exemple:}\\\\
		De:
		
		et:
		
		nous pouvons d\'eduire:
		
		\end{tcolorbox}
		\begin{tcolorbox}[title=Remarque,colframe=black,arc=10pt]
		Les humains communiquent g\'en\'eralement d'une manière qui r\'esiste à l'analyse logique superficielle. Dans une conversation r\'eelle, les gens utilisent des mots plutôt que des termes, font des \'enonc\'es plutôt que des phrases et emploient une plus grande vari\'et\'e de m\'ethodes d'inf\'erence que le modus ponens. Une grande partie de ce qui est communiqu\'e et d\'eduit dans une conversation d\'epend du contexte, des orateurs et du public, de leur histoire, de leurs connaissances et confidences partag\'ees, des sens qu'ils mettent en place pour \'etablir une relation de confiance mutuelle.
		\end{tcolorbox}
		
		\item La "\NewTerm{substitution}\index{substitution}": nous pouvons dans un sch\'ema d'axiomes remplacer une lettre par une formule quelconque, pourvu que toutes les lettres identiques soient remplac\'ees par des formules identiques.
		
		Donnons à titre d'exemple, deux systèmes axiomatiques: le système axiomatique de Whitehead et Russell, le système axiomatique de Lukasiewicz.
		\begin{enumerate}
			\item Le système axiomatique de Whitehead et Russel adopte comme symboles primitifs  $\neg, \vee$ et d\'efinit  $\Rightarrow, \wedge ,\Leftrightarrow$ à partir de ces derniers de la manière suivante (relations facilement v\'erifiables à l'aide de tables de v\'erit\'e):
			
						
			Ce système comprend $5$ axiomes, assez \'evidents en soi plus les deux règles d'inf\'erence. Les axiomes sont donn\'es ici en utilisant des symboles non primitifs, comme le faisaient Whitehead et Russel:
			\begin{enumerate}
				\item[A1.] $(A \vee A)\Rightarrow A$
				\item[A2.] $B \Rightarrow (A\vee B)$
				\item[A3.] $(A\vee B) \Rightarrow (B\vee A)$
				\item[A4.] $(A\vee (B\vee C)) \Rightarrow (B\vee (A\vee C))$
				\item[A5.] $(B\Rightarrow C)\Rightarrow \left((A\vee B)\Rightarrow(A\vee C)\right)$
			\end{enumerate}
			Nous avions d\'ejà pr\'esent\'e plus haut quelques-uns de ces \'el\'ements.
			\begin{tcolorbox}[title=Remarque,colframe=black,arc=10pt]
			Ces cinq axiomes ne sont pas ind\'ependants les uns des autres. Le quatrième peut être obtenu à partir des quatre autres.
			\end{tcolorbox}
			Par exemple, pour justifier que $\neg A\vee A$ a un sens\footnote{Cela s'appelle la "\NewTerm{loi du tiers exclu}" : une proposition est soit vraie, soit fausse. Cela a été largement considéré comme allant de soi depuis des milliers d'années}, nous pouvons procéder comme suit (dans ce système, nous ne prouvons que des vérités logiques, ce qui signifie que chaque ligne doit être une vérité logique en soi):
			\begin{table}[H]
			\begin{center}
				\definecolor{gris}{gray}{0.85}
					\begin{tabular}{ccr}
						(1) &  $B\Rightarrow (A\vee B)$ & Axiom A2 \\ 
						(2) &  $A\Rightarrow (A\vee A)$  & (1) and substitution \\
						(3) &  $(B\Rightarrow C)\Rightarrow \left((A\vee B)\Rightarrow (A\vee C)\right)$  & Axiom A5 (necessary)\\
						(4) &  $(B\Rightarrow C)\Rightarrow ((\neg A\vee B)\Rightarrow (\neg A\vee C))$ & (3) and substitution \\
						(5) &  $(B \Rightarrow C)\Rightarrow((A\Rightarrow B)\Rightarrow (A\Rightarrow C))$  & (4) and property of $\Rightarrow$ \\
						(6) &  $((A\vee A)\Rightarrow A)\Rightarrow((A\Rightarrow (A\vee A))\Rightarrow (A\Rightarrow A))$  & (5) and substitution \\
						(7) &  $(A\Rightarrow (A\vee A)) \Rightarrow (A\Rightarrow A)$  & (6) (modus ponens)\\
						(8) &  $(A \Rightarrow A)\Rightarrow (A\Rightarrow A)$  & (7) and axiom A1 \\
						(9) &  $A \Rightarrow A$  & (8) and modus ponens \\
						(10) &  $\neg A \vee A$  & (9) and property of $\Rightarrow$ \\
				\end{tabular}
			\end{center}
			\end{table}
			
			\item Le système axiomatique de Lukasiewicz comprend les trois axiomes suivants, plus les deux règles d'inf\'erences (modus ponens et substitution):
			\begin{enumerate}
				\item[A1.] $(A\Rightarrow B) ((B\Rightarrow C) \Rightarrow (A\Rightarrow C))$
				\item[A2.] $A\Rightarrow  (\neg A\Rightarrow B)$
				\item[A3.] $(\neg A \Rightarrow A)\Rightarrow  A$
			\end{enumerate}
			Voici des preuves des deux premiers axiomes, dans le système de Whitehead et Russel. Ce sont les formules (6) et (16) de la d\'erivation suivante:
			\begin{table}[H]
			\begin{center}
				\definecolor{gris}{gray}{0.85}
					\begin{tabular}{ccr}
						(1) &  $(A\vee (B\vee C))\Rightarrow (B\vee (A\vee C))$ & Axiome A4 \\ 
						(2) &  \parbox{6cm}{$(\neg (B\Rightarrow C)\vee (\neg(A\vee B)\vee (A\vee C)))$\\ $\Rightarrow (\neg (A\vee B)\vee (\neg(B\Rightarrow C)\vee(A\vee C)))$}  & (1) et substitution \\
						(3) &  $\neg(A\vee B)\vee (\neg(B\Rightarrow C)\vee (A\vee C))$  & A4 sur (2) et modus ponens\\
						(4) &  $(A\vee B)\Rightarrow ((B\Rightarrow C)\Rightarrow (A\vee C))$ & (3) par la propriété de $\Rightarrow$ \\
						(5) & $(\neg A\vee B)\Rightarrow ((B\Rightarrow C)\Rightarrow(\neg A\vee C))$ & (4) et substitution \\ 
						(6) & $(A\Rightarrow B)\Rightarrow ((B\Rightarrow C)\Rightarrow (A\Rightarrow C))$ & (5) et propriété de $\Rightarrow$\\
						(7) & $(B\Rightarrow (A\vee B))\Rightarrow \begin{pmatrix}((A\vee B)\Rightarrow (B\vee A))\\ \Rightarrow (B\Rightarrow (B\vee A))\end{pmatrix}$ & (6) et substitution\\
						(8) & $((A\vee B)\Rightarrow (B\vee A))\Rightarrow (B\Rightarrow (B\vee A))$ & (7) modus ponens\\
						(9) & $B\Rightarrow (B\vee A)$ & (8) modus ponens\\
						(10) & $\neg B\Rightarrow (\neg B\vee A)$ & (9) et substitution\\
						(11) & $\neg\neg B\vee (\neg B \vee A)$ & (10) et propriété de $\Rightarrow$\\
						(12) & $\neg\neg B\vee (\neg B \vee A) \Rightarrow (\neg B\vee (\neg\neg B\vee A))$ & A4 + substitution avec (11)\\
						(13) & $\neg B\vee (\neg\neg B\vee A)$ & (12) et modus ponens\\
						(14) & $B\Rightarrow (\neg\neg B\vee A)$ & (19) et propriété de $\Rightarrow$\\
						(15) & $B\Rightarrow (\neg B\vee A)$ & (14) et propriété de $\Rightarrow$\\
						(16) & $A\Rightarrow (\neg A\Rightarrow B)$ & (15) et substitution\\
				\end{tabular}
			\end{center}
			\end{table}
		\end{enumerate}
	\end{enumerate}
	Ces axiomatisations permettent de retrouver comme th\'eorèmes toutes les tautologies ou lois de la logique propositionnelle. De par tout ce qui a \'et\'e dit jusqu'à maintenant, nous pouvons tenter de d\'efinir ce qu'est une preuve!!!
	
	\textbf{D\'efinition (\#\mydef):} Une suite finie de formules $B_1,B_2,\ldots,B_m$ est appel\'ee  "\NewTerm{preuve}\index{preuve}" à partir des hypothèses $A_1,A_2,\ldots,A_n$ si pour chaque $i$:
	\begin{itemize}
		\item $B_i$ est l'une des hypothèses $A_1,A_2,\ldots,A_n$
		
		\item ou $B_i$ est une variante d'un axiome
		
		\item ou $B_i$ est inf\'er\'ee (par application de la règle du modus ponens) à partir de la pr\'emisse majeure $B_j$ et de la pr\'emisse mineure $B_k$ où $j,k<i$
		
		\item ou $B_i$ est inf\'er\'ee (par application de la règle de substitution) à partir d'une pr\'emisse ant\'erieure $B_j$, la variable remplac\'ee n'apparaissant pas dans $A_1,A_2,\ldots,A_n$
	\end{itemize}
	Une telle suite de formules, $B_m$ \'etant la formule finale de la suite, est appel\'ee plus explicitement "\NewTerm{preuve de $B_m$}" à partir des hypothèses (axiomes) $A_1,A_2,\ldots,A_n$, ce que nous notons par:
	
	Plus explicitement, une preuve est un argument d\'eductif pour un \'enonc\'e math\'ematique. Dans l'argument, d'autres instructions pr\'ec\'edemment \'etablies, telles que des th\'eorèmes ou lemmes, peuvent être utilis\'ees. En principe, une preuve peut être reli\'ee à des d\'eclarations \'evidentes ou suppos\'ees, connues sous le nom "d'axiomes".

	Les preuves utilisent la logique, mais incluent g\'en\'eralement une certaine quantit\'e de langage naturel qui admet g\'en\'eralement une certaine ambiguït\'e. En fait, la grande majorit\'e des preuves en math\'ematiques \'ecrites peuvent être consid\'er\'ees comme des applications d'une logique informelle rigoureuse. Les preuves purement formelles, \'ecrites dans un langage symbolique plutôt que dans un langage naturel, sont consid\'er\'ees dans la th\'eorie de la d\'emonstration.
	\begin{tcolorbox}[title=Remarque,colframe=black,arc=10pt]
	Il faut noter que lorsque nous essayons de prouver un r\'esultat à partir d'un certain nombre d'hypothèses, nous n'essayons pas de prouver les hypothèses elles-mêmes.
	\end{tcolorbox}
	
	\pagebreak	
	\subsubsection{Quantificateurs}
	Nous devons compl\'eter l'utilisation des connecteurs du calcul propositionnel par ce que nous appelons des "\NewTerm{quantificateurs}\index{quantificateurs}" si nous souhaitons pouvoir r\'esoudre certains problèmes. Effectivement, le calcul propositionnel ne nous permet pas d'affirmer des choses g\'en\'erales sur les \'el\'ements d'un ensemble par exemple. Dans ce sens, la logique propositionnelle ne reflète qu'une partie du raisonnement. Le "calcul des pr\'edicats" au contraire permet de manipuler formellement des affirmations telles que "il existe un $x$ tel que [$x$ a une voiture am\'ericaine]" ou "pour tous les $x$ [si $x$ est un teckel, alors $x$ est petit]"; en somme, nous \'etendons les formules compos\'ees afin de pouvoir affirmer des quantifications existentielles ("il existe...") et des quantifications universelles ("pour tout...."). Les exemples que nous venons de donner font intervenir des propositions un peu particulières comme "$x$ a une voiture am\'ericaine". Il s'agit ici de propositions comportant une variable. Ces propositions sont en fait l'application d'une fonction à $x$. Cette fonction, c'est celle qui associe "$x$ a une voiture am\'ericaine" à $x$. Nous d\'enoterons cette fonction par "... a une voiture am\'ericaine" et nous dirons que c'est une fonction propositionnelle, car c'est une fonction dont la valeur est une proposition. Ou encore un "pr\'edicat".

	Les quantificateurs existentiels et universels vont donc de pair avec l'emploi de fonctions propositionnelles. Le calcul des pr\'edicats est cependant limit\'e dans les formules existentielles et universelles. Ainsi, nous nous interdisons des formules comme "il existe une affirmation de $x$ telle que...". En fait, nous ne nous autorisons à quantifier que des "individus". C'est pour cela que la logique des pr\'edicats est dite une "\NewTerm{logique du premier ordre}\index{logique du premier ordre}\label{first order predicate}" ou "\NewTerm{logique de pr\'edicats premier ordre}\index{logique de pr\'edicats du premier ordre}\label{first order predicate}"  (LPPO) car elle utilise comme variables des objets math\'ematiques \'el\'ementaires (tandis que dans la logique du deuxième ordre elles peuvent aussi être des ensembles).
	\begin{center}
	\includegraphics[scale=0.5]{img/arithmetics/first_vs_second_order_logic.jpg}
	\end{center}
	
	Avant de passer à l'\'etude du calcul des pr\'edicats nous devons d\'efinir:
	\begin{enumerate}
		\item[D1.] Le "\NewTerm{quantificateur universel}\index{quantificateur universel}": $\forall$ (pour tout)

		\item[D2.] Le "\NewTerm{quantificateur universel}\index{quantificateur universel}": $\exists$ (il existe)
	\end{enumerate}
	\begin{tcolorbox}[colframe=black,colback=white,sharp corners]
	\textbf{{\Large \ding{45}}Exemple:}\\\\
	Si n'importe quel nombre complexe est le produit d'un nombre non n\'egatif et d'un nombre de module $1$, nous \'ecrirons:
	
	\end{tcolorbox}
	
	L'ordre des quantificateurs est essentiel à la signification, comme l'illustrent les deux propositions suivantes:
	\begin{center}
	Pour tout entier naturel $n$, il existe un entier $s$ tel que $s$ = $n^2$.
	\end{center}

	Ceci est clairement vrai! Cette phrase affirme simplement que chaque nombre naturel a un carr\'e. La signification de l'assertion dans laquelle les quantificateurs sont invers\'es est diff\'erente:

	\begin{center}
	Il existe un nombre naturel $s$ tel que pour tout entier naturel $n$, $s = n^2$.
	\end{center}
	Ceci est clairement faux! Il affirme qu'il existe un seul nombre naturel $s$ qui est en même temps le carr\'e de chaque nombre naturel.
	
	Une question fr\'equente en physique et en math\'ematiques est de savoir si les quantificateurs universels doivent être avant ou après les pr\'edicats auxquels ils se rapportent. En fait, strictement en termes de logique formelle, les quantificateurs sont toujours au d\'ebut de toute formule. Cependant, presque personne ne donne une preuve \'ecrite dans le langage formel. Même des preuves simples seraient très longues et illisibles. Mais quiconque, quelle que soit sa langue naturelle parl\'ee, interpr\'etera une phrase dans le langage formel de la même manière. Le prix pour cette clart\'e est bien sûr la lisibilit\'e. Les langues naturelles, en raison de leur ambiguït\'e inh\'erente, sont soumises à de nombreuses autres limitations.
		
	De toute \'evidence, l'utilisation correcte d'une notation formelle ou d'une notation plus informelle d\'epend particulièrement du contexte de la pr\'esentation. Il est essentiel d'essayer de nous adapter à qui nous communiquons (sans toutefois perdre en rigueur scientifique!) un concept et cela devrait nous guider pour utiliser un niveau appropri\'e de notation formelle.
	
	Nous utilisons aussi parfois le quantificateur $\exists !$ pour dire brièvement: "il existe un et un seul". 
	\begin{tcolorbox}[colframe=black,colback=white,sharp corners]
	\textbf{{\Large \ding{45}}Exemple:}\\\\
	Un exemple c\'elèbre est le moyen d'expliciter que le logarithme est une fonction bijective:
	
	\end{tcolorbox}
	Avant de passer au calcul des pr\'edicats, \'etablissons un r\'esum\'e non exhaustif, courant dans la litt\'erature et utilisant la n\'egation:
	\begin{table}[H]
		\centering
		\begin{tabular}{|c|c|c|}
			\hline 
			Expression & & Est \'equivalent à\\ \hline
			 $\neg\neg A $& $\Leftrightarrow$ & A\\ \hline
			 $\neg A \Rightarrow \neg B$&$\Leftrightarrow$ &$B\Rightarrow A$ \\ \hline
			$\neg A \Rightarrow B$&$\Leftrightarrow$ &$\neg B\Rightarrow A$ \\ \hline
			$A\Rightarrow \neg B$ & $\Leftrightarrow$ & $(A \wedge B )\Rightarrow F$\\ \hline
			$\neg A \wedge \neg B$ & $\Leftrightarrow$ & $\neg(A\vee B)$ \\ \hline
			$A \wedge\neg B$ & $\Leftrightarrow$ & $\neg(\neg A \vee B)$ \\ \hline
			$\wedge A \neg B$ &$\Leftrightarrow$ & $\neg(A\vee\neg B)$\\ \hline
			$\neg A \vee\neg B$ &$\Leftrightarrow$ & $\neg(A\wedge B)$\\ \hline
			$A \vee \neg B$ & $\Leftrightarrow$ & $\neg(\neg A\wedge
			 B)$\\ \hline
			$\neg A \vee B$ &$\Leftrightarrow$& $\neg(A\wedge\neg B)$\\ \hline
			$\neg (\exists x \in X: \neg P(x))$ &$\Leftrightarrow$& $(\forall x \in X: P(x))$\\ \hline
			$(\exists x \in X:\neg P(x))$ &$\Leftrightarrow$& $\neg(\forall x \in X: P(x))$ \\ \hline
			$(\exists x \in X: P(x))$ &$\Leftrightarrow$& $(\forall x \in X:\neg P(x))$ \\ \hline
			$(\exists x \in X: P(x))$ &$\Leftrightarrow$& $\neg(\forall x \in X: \neg P(x))$\\ \hline
		\end{tabular}
	\end{table}
	Nous verrons maintenant que la th\'eorie de la d\'emonstration et la th\'eorie des ensembles est la transcription exacte des principes et r\'esultats de la Logique (celle avec un "L" majuscule !!!).
	
	\pagebreak
	\subsection{Calcul des Pr\'edicats}
	Dans un cours de math\'ematiques (d'algèbre, d'analyse, de g\'eom\'etrie, ...), nous d\'emontrons les propri\'et\'es de diff\'erents types d'objets (entiers, r\'eels, matrices, suites, fonctions continues, courbes, ...).  Pour pouvoir prouver ces propri\'et\'es, il faut bien sûr que les objets sur lesquels nous travaillons soient clairement d\'efinis (qu'est-ce qu'un entier, un r\'eel, ...?).

En logique du premier ordre et, en particulier, en th\'eorie de la d\'emonstration, les objets que nous \'etudions sont les formules et leurs d\'emonstrations. Il faut donc donner une d\'efinition pr\'ecise de ce que sont ces notions. Les termes et les formules forment la grammaire d'une langue, simplifi\'ee à l'extrême et calcul\'ee exactement pour dire ce que nous voulons sans ambiguït\'e et sans d\'etour inutile.
	
	\subsubsection{Grammaire}\label{grammar}
	\textbf{D\'efinitions (\#\mydef):}
	\begin{enumerate}
		\item[D1.] Les "\NewTerm{termes}\index{termes}"", d\'esignent les objets dont nous voulons prouver des propri\'et\'es (nous reviendrons un peu plus loin beaucoup plus en d\'etail sur ces derniers):
		\begin{itemize}
			\item En algèbre, les termes d\'esignent les \'el\'ements d'un groupe (ou anneau, corps, espace vectoriel, etc.). Nous manipulons aussi des ensembles d'objets (sous-groupe, sous-espace vectoriel, etc). Les termes qui d\'esignent ces objets, d'un autre type, seront appel\'es "\NewTerm{termes du second ordre}\index{termes du second ordre}".

			\item En analyse, les termes d\'esignent les r\'eels ou (par exemple, si nous nous plaçons dans des espaces fonctionnels) des fonctions.
		\end{itemize}
		
		\item[D2.] Les "\NewTerm{formules}\index{formules}", repr\'esentent les propri\'et\'es des objets que nous \'etudions (nous reviendrons \'egalement beaucoup plus en d\'etail sur ces dernières):
		\begin{itemize}
			\item En algèbre, nous pourrons \'ecrire des formules pour exprimer que deux \'el\'ements commutent, qu'un sous-espace vectoriel est de dimension $3$, etc.

			\item En analyse, nous \'ecrirons des formules pour exprimer la continuit\'e d'une fonction, la convergence d'une suite, etc.

			\item En th\'eorie des ensembles, les formules pourront exprimer l'inclusion de deux ensembles, l'appartenance d'un \'el\'ement à un ensemble, etc.
		\end{itemize}
		
		\item[D3.] Les "\NewTerm{d\'emonstration}\index{d\'emonstration}", elles permettent d'\'etablir qu'une formule est vraie. Le sens pr\'ecis de ce mot aura lui aussi besoin d'être d\'efini. Plus exactement, elles sont des d\'eductions sous hypothèses, elles permettent de "mener du vrai au vrai", la question de la v\'erit\'e de la conclusion \'etant alors renvoy\'ee à celle des hypothèses, laquelle ne regarde pas la logique mais repose sur la connaissance que nous avons des choses dont nous parlons.
	\end{enumerate}
	
	\pagebreak
	\subsubsection{Langages}
	En math\'ematique, nous utilisons, suivant le domaine, diff\'erents langages qui se distinguent par les symboles utilis\'es. La d\'efinition ci-dessous exprime simplement qu'il suffit de donner la liste de ces symboles pour pr\'eciser le langage.
	
	\textbf{D\'efinition (\#\mydef):} Un "\NewTerm{langage}\index{langage}" est la donn\'ee d'une famille (pas n\'ecessairement finie) de symboles. Nous en distinguons de trois sortes: symboles, termes et formules.
	
	\begin{tcolorbox}[title=Remarques,colframe=black,arc=10pt]
	\textbf{R1.} Nous utilisons quelques fois le mot "vocabulaire" ou le mot "signature" à la place du mot "langage".\\
	
	\textbf{R2.} Le mot "pr\'edicat" peut être utilis\'e à la place du mot "relation". Nous parlons alors de "calcul des pr\'edicats" au lieu de "logique du premier ordre" (ce que nous avons \'etudi\'e pr\'ec\'edemment).
	\end{tcolorbox}
	
	\paragraph{Symboles}\mbox{}\\\\
	Il existe diff\'erents types de symboles que nous allons tâcher de d\'efinir!
	
	\textbf{D\'efinitions (\#\mydef):}
	\begin{enumerate}
		\item[D1.] Les "\NewTerm{symboles de constante}\index{symboles de constante}" (voir remarque plus bas)
		\begin{tcolorbox}[colframe=black,colback=white,sharp corners]
		\textbf{{\Large \ding{45}}Exemple:}\\\\
		Le $n$ pour l'\'el\'ement neutre en th\'eorie des ensembles (\SeeChapter{voir section de Th\'eorie des Ensembles page \pageref{neutral element}})
		\end{tcolorbox}
	
		\item[D2.] Les "\NewTerm{symboles de fonctions}\index{symboles de fonctions}" ou "\NewTerm{foncteurs}\index{foncteurs}". A chaque symbole de fonction est associ\'e un entier strictement positif que nous appelons son "\NewTerm{arit\'e}": c'est le nombre d'arguments de la fonction. Si l'arit\'e est 1 (resp. $2$, ..., $n$), nous disons que la fonction est unaire (resp. binaire, ..., $n$-aire)
		\begin{tcolorbox}[colframe=black,colback=white,sharp corners]
		\textbf{{\Large \ding{45}}Exemple:}\\\\
		Le foncteur de la multiplication $\times$ ou $\cdot$ dans la th\'eorie des groupes (\SeeChapter{voir section de Th\'eorie des Ensembles page \pageref{multiplication binary operator}}) est $2$-aire ou plus simplement un op\'erateur "binaire".
		\end{tcolorbox}
	
		\item[D3.] Les "\NewTerm{symboles de relation}\index{symboles de relation}". De la même manière, à chaque symbole de relation est associ\'e un entier positif ou nul (son arit\'e) qui correspond à son nombre d'arguments et nous parlons de relation unaire, binaire, $n$-aire
		
		\begin{tcolorbox}[colframe=black,colback=white,sharp corners]
		\textbf{{\Large \ding{45}}Exemple:}\\\\
		La relation $=$ est une relation binaire (\SeeChapter{voir section Op\'erateurs page \pageref{equality}})
		\end{tcolorbox}
	
		\item[D4.] Les "\NewTerm{variables individuelles}\index{variables individuelles}". Dans toute la suite, nous nous donnerons un ensemble infini $V$ de variables. Les variables seront not\'ees comme il l'est de tradition: $x, y, z$ (\'eventuellement index\'ees: $x_1,x_2,x_3$)
	
		\item[D5.] A cela il faut rajouter les connecteurs  $\neg, \Rightarrow, \vee,\wedge$ et quantificateurs $\forall,\exists,\exists!$ que nous avons longuement pr\'esent\'es plus haut et sur lesquels il est pour l'instant inutile de revenir.
	\end{enumerate}
	
	\begin{tcolorbox}[title=Remarques,colframe=black,arc=10pt]
	\textbf{R1.} Un symbole de constante peut être vu comme un symbole de fonction à 0 argument (d'arit\'e nulle).\\
	
	\textbf{R2.} Nous consid\'erons (sauf mention contraire) que chaque langage contient le symbole de relation binaire $=$ (lire "\'egal") et le symbole de relation à z\'ero argument d\'enot\'e $\perp$ (lire "bottom" ou "absurde") qui repr\'esente le faux. Dans la description d'un langage, nous omettrons donc souvent de les mentionner. Le symbole $\perp$ est souvent redondant. Nous pouvons en effet, sans l'utiliser, \'ecrire une formule qui est toujours fausse. Il permet cependant de repr\'esenter le faux d'une manière canonique et donc d'\'ecrire des règles de d\'emonstration g\'en\'erales.\\
	
	\textbf{R3.}  Le rôle des fonctions et des relations est très diff\'erent. Comme nous le verrons plus loin, les symboles de fonction sont utilis\'es pour construire les termes (les objets du langage) et les symboles de relation pour construire les formules (les propri\'et\'es de ces objets).
	\end{tcolorbox}
	
	\paragraph{Termes}\mbox{}\\\\
	Les termes (nous disons aussi "termes du premier ordre") repr\'esentent les objets associ\'es au langage. 
	
	\textbf{D\'efinitions (\#\mydef):}
	\begin{enumerate}
		\item[D1.]  L'ensemble $\mathcal{T}$ des termes sur un langage $\mathcal{L}$ est le plus petit ensemble contenant les variables, les constantes et stable (on ne sort pas de l'ensemble) par l'application des symboles de fonction de $\mathcal{L}$  à des termes.

		\item[D2.] Un "\NewTerm{terme clos}\index{terme clos}" est un terme qui ne contient pas de variables (donc par extension, seulement des constantes).

		\item[D3.] Pour obtenir une d\'efinition plus formelle, nous pouvons \'ecrire:
		
		où $t$ est une variable ou un symbole de constante et, pour tout $k\in \mathbb{N}$:
		
		où $f$ est une fonction d'arit\'e $n$ (rappelons que l'arit\'e est le nombre d'arguments de la fonction). Ainsi, pour chaque arit\'e, il y a un degr\'e d'ensemble de termes. Nous avons finalement:
		

		\item[D4.] Nous appellerons "\NewTerm{hauteur d'un terme}\index{hauteur d'un terme}" $t$ le plus petit $k$ tel que $t\in \mathcal{T}_k$.
		
		Cette d\'efinition signifie que les variables et les constantes sont des termes et que si $f$ est un symbole de fonction $n$-aire et $t_1,\ldots,t_n$ sont des termes alors $f(t_1,\ldots,t_n)$ est un terme en soi aussi. L'ensemble $\mathcal{T}$ des termes est d\'efini par la grammaire:
		
		Cette expression se lit de la manière suivante: un \'el\'ement de l'ensemble $\mathcal{T}$ que nous sommes en train de d\'efinir est soit un \'el\'ement de $V$ (variables), soit un \'el\'ement de $S_c$ (l'ensemble des symboles de constantes), soit l'application d'un symbole de fonction $f\in S_f$ à $n$ \'el\'ements (constantes ou variables) de  $\mathcal{T}$.
		
		Attention: le fait que $f$ soit de la bonne arit\'e est seulement implicite dans cette notation. De plus, l'\'ecriture $S_f(\mathcal{T},\ldots,\mathcal{T})$ ne signifie pas que tous les arguments d'une fonction sont identiques mais simplement que ces arguments sont des \'el\'ements de $\mathcal{T}$.
		\begin{tcolorbox}[title=Remarque,colframe=black,arc=10pt]
		Il est souvent commode de voir un terme (expression) comme un arbre dont chaque noeud est \'etiquet\'e par un symbole de fonction (op\'erateur ou fonction) et chaque feuille par une variable ou une constante.
		\end{tcolorbox}
	\end{enumerate}
	Dans la suite, nous allons sans cesse d\'efinir des notions (ou prouver des r\'esultats) "par r\'ecurrence\label{proof by recurrence}" sur la structure ou la taille d'un terme.

	\textbf{D\'efinitions (\#\mydef):}
	\begin{enumerate}
		\item[D1.] Pour prouver une propri\'et\'e $P$ sur les termes, il suffit de prouver $P$ pour les variables et les constantes et de prouver $P(f(t_1,\ldots,t_n))$ à partir de $P(t_1),\ldots,P(t_n)$. Nous faisons ainsi ici une "\NewTerm{preuve par induction sur la hauteur d'un terme}\index{preuve par induction sur la hauteur d'un terme}". C'est une technique que nous retrouverons dans les chapitres suivants.

		L'induction math\'ematique en tant que règle d'inf\'erence peut être formalis\'ee en tant qu'axiome de second ordre. L'axiome de l'induction est, en symboles logiques:
		
		En termes clairs, la base $P(0)$ et l'\'etape d'induction (explicitement, l'hypothèse inductive $P(k)$ implique $P(k+1)$) ensembles impliquent $P(n)$ pour tout entier naturel $n$. L'axiome d'induction affirme que la validit\'e d'inf\'erence que $P(n)$ est vrai pour tout entier naturel $n$ tient du principe de base de l'\'etape d'induction elle-même!
		
		On peut comparer l’induction à des dominos en chute: chaque fois qu'un domino tombe, le suivant tombe \'egalement. La première \'etape, prouvant (ou donnant l'\'evidence) que $P(1)$ est vraie, commence la r\'eaction en chaîne infinie.
	\begin{center}
	\includegraphics[scale=0.4]{img/arithmetics/dominoes.jpg}
	\end{center}

		\item[D2.]  Pour d\'efinir une fonction $\Phi$ sur les termes, il suffit de la d\'efinir sur les variables et les constantes et de dire comment nous obtenons $\Phi(f(t_1,\ldots,t_n))$ à partir de $\Phi(t_1),\ldots,\Phi(t_n)$. Nous faisons ici encore une "\NewTerm{d\'efinition par induction sur la hauteur d'un terme}\index{d\'efinition par induction sur la hauteur d'un terme}".
	\end{enumerate}
	\begin{tcolorbox}[colframe=black,colback=white,sharp corners]
	\textbf{{\Large \ding{45}}Exemple:}\\\\
	La taille (nous disons aussi la "longueur") d'un terme $t$ (not\'ee $\tau(t)$) est le nombre de symboles de fonction apparaissant dans $t$. Formellement:
	\begin{itemize}
		\item $\tau(x)=\tau(c)=0$ si $x$ est une variable $c$ une constante.

		\item  $\tau(f(t_1,\ldots,t_n))=1+\displaystyle\sum_{i\leq i\leq n}\tau(t_i)$
	\end{itemize}
	où le $1$ dans la dernière relation relation repr\'esente le terme $f$.
	\end{tcolorbox}
	\begin{tcolorbox}[title=Remarques,colframe=black,arc=10pt]
	La preuve par induction sur la hauteur d'un terme sera souvent insuffisante. Nous pourrons alors prouver une propri\'et\'e $P$ sur les termes en supposant la propri\'et\'e vraie pour tous les termes de taille $p<n$ et en la d\'emontrant ensuite pour les termes de taille $n$. Il s'agira alors d'une "\NewTerm{preuve par r\'ecurrence sur la taille du terme}" (voir de tels exemples dans le chapitre de Th\'eorie Des Nombres).
	\end{tcolorbox}
	
	\paragraph{Formules}\mbox{}\\\\
	\textbf{D\'efinition (\#\mydef):} Une "\NewTerm{formule bien form\'ee}\index{formule bien form\'ee}", (FBF) ou simplement "\NewTerm{formule}" est une mot (par extension une s\'equence finie de symboles d'un alphabet donn\'e) qui fait partie d'un langage formel. Un langage forme peut être consid\'er\'e comme identique à l'ensemble contenant seulements toutes ses propres formules.

	Les formules du calcul propositionnel, \'egalement nomm\'ees "\NewTerm{formules propositionnelles}\index{formules propositionnelles}", sont des expressions du type $(A \land (B \lor C))$.

	Un formule atomatique est une formule qui ne contient ni connecteurs logiques ni quantificateurs, ou de manière \'equivalente, une formule qui n'a strictement aucune sous-formule. La forme pr\'ecise des formules atomiques d\'epend du système formel consid\'er\'e; pour la logique propositionnelle par exemple, les formules atomiques sont les variables propositionnelles. Pour la logique des pr\'edicats, les atomes sont les symboles de pr\'edicats avec leurs arguments, chaque argument \'etant un terme.
	\begin{figure}[H]
		\centering
		\includegraphics{img/arithmetics/theorems.jpg}
	\end{figure}

	\textbf{D\'efinition (\#\mydef):} Les formules sont construites à partir de "\NewTerm{formules atomiques}\index{formules atomiques}" en utilisant des connecteurs et des quantificateurs. Nous utiliserons les connecteurs et les quantificateurs suivants (qui nous sont d\'ejà connus):
	\begin{itemize}
		\item Connecteur unaire de n\'egation $\neg$
		
		\item Connecteurs binaires de conjonction et disjonction ainsi que d'implication: $\wedge, \vee, \rightarrow$
		
		\item Quantificateurs: $\exists$ qui se lit "il existe" et $\forall$ qui se lit "pour tout"
	\end{itemize}
	Cette notation des connecteurs est standard (elle devrait du moins). Elle est utilis\'ee pour \'eviter les confusions entre les formules et le langage courant (le m\'etalangage).
	
	\textbf{D\'efinitions (\#\mydef):}
	\begin{enumerate}
		\item[D1.]  Soit $\mathcal{L}$ un langage, les "\NewTerm{formules atomiques}" de $\mathcal{L}$ sont les formules de la forme $R(t_1,\ldots,t_n$ où $R$ est un symbole de relation $n$-aire de $\mathcal{L}$ et $t_1,\ldots,t_n$ sont des termes de $\mathcal{L}$. Nous notons "Atom" l'ensemble des formules atomiques. Si nous notons par $S_r$ l'ensemble des symboles de relation, nous pouvons \'ecrire l'ensemble des termes mis en relations par l'expression:
		
		L'ensemble $\mathcal{F}$  des formules de la logique du premier ordre de $\mathcal{L}$ est donc d\'efini par la grammaire (où $x$ est une variable):
		
		où il faut lire: l'ensemble des formules est le plus petit ensemble contenant les formules et tel que si $F_1$ et $F_2$ sont des formules alors $F_1\vee F_2$, etc. sont des formules et qu'elles peuvent être en relation entre elles.
		\begin{tcolorbox}[colframe=black,colback=white,sharp corners]
		\textbf{{\Large \ding{45}}Exemple:}\\\\
		Les symboles de relation du langage propositionnel sont des relations d'arit\'e $0$ (même le symbole "$=$" est absent), les quantificateurs sont alors inutiles (puisqu'une formule propositionnelle ne peut pas contenir des variables). Nous obtenons alors le calcul propositionnel $\mathcal{C}_P$ d\'efini par:
		
		Remarquons la pr\'esence du symbole $\bot$ signifiant le "faux" que nous n'avions pas mentionn\'e lors de notre \'etude de la logique propositionnelle.
		\end{tcolorbox}
		Nous ferons attention à ne pas confondre termes et formules. $\sin(x)$ est un terme (fonction), $x=3$ est une formule. Mais $\sin(x)\wedge (x=3)$ n'est rien: nous ne pouvons, en effet, mettre un connecteur entre un terme et une formule (cela n'a aucun sens).
		\begin{tcolorbox}[title=Remarques,colframe=black,arc=10pt]
		\textbf{R1.}  Pour d\'efinir une fonction $\Phi$ sur les formules, il suffit de d\'efinir $\Phi$ sur les formules atomiques.\\

		\textbf{R2.}  Pour prouver une propri\'et\'e $P$ sur les formules, il suffit de prouver $P$ pour les formules atomiques.\\
		
		\textbf{R3.} Pour prouver une propri\'et\'e $P$ sur les formules, il suffit de supposer la propri\'et\'e vraie pour toutes les formules de taille $p<n$ et de la d\'emontrer pour les formules de taille $n$.
		\end{tcolorbox}
		
		\item[D2.]  Une "\NewTerm{sous-formula}\index{sous-formula}" d'une formule (ou expression) $F$ est l'un de ses composants, in extenso une formule à partir de laquelle $F$ est construite. Formellement, nous d\'efinissons l'ensemble $\mathrm{SF}(F)$ des sous-formules de $F$ par:
		\begin{itemize}
			\item Si $F$ est atomique: 
			
			
			\item Si $F=F_1\oplus F_2$ (soit une composition!) avec $\oplus\in\{\vee,\wedge,\rightarrow\}$:
			
			
			\item Si $F=\neg F$ ou $\mathrm{Q}\in F_1$ avec $\mathrm{Q}\in \{\forall,\exists\}$:
			
		\end{itemize}

		\item[D3.] Une formule $F$ de $\mathcal{L}$ n'utilise qu'un nombre fini de symboles de  $\mathcal{L}$. Ce sous-ensemble est appel\'e le "\NewTerm{langage de la formule}\index{langage de la formule}" et not\'e $\mathcal{L}(F)$.

		\item[D4.] La  "\NewTerm{taille ou (longueur) d'une formule}\index{taille ou (longueur) d'une formule}" $F$, not\'ee $\tau(F)$ est le nombre de connecteurs ou de quantificateurs apparaissant dans $F$. Formellement:
		\begin{itemize}
			\item $\tau(F)=0$ si $F$ est une formule atomique

			\item $\tau(F_1\oplus F_2)=1+\tau(F_1)+\tau(F_2)$ où encore une fois  $\oplus\in\{\vee,\wedge,\rightarrow\}$

			\item $\tau(\neg F_1)=\tau(\mathrm{Q}xF_1)=1\tau(F_1)$ avec encore une fois $\mathrm{Q}\in \{\forall,\exists\}$
		\end{itemize}

		\item[D5.] "\NewTerm{L'op\'erateur principal}\index{op\'erateur princiapl}" (nous disons aussi le "connecteur principal") d'une formule est d\'efini par:
		\begin{itemize}
			\item Si $A$ est atomique, alors elle n'a pas d'op\'erateur principal
	
			\item Si $A=\neg B$, alors $\neg$  est l'op\'erateur principal de $A$
	
			\item Si $A=B\oplus C$ où encore une fois $\oplus\in\{\vee,\wedge,\rightarrow\}$ alors $\oplus$ est l'op\'erateur principal de $A$
	
			\item Si $A=\mathrm{Q}xB$ où encore une fois $\mathrm{Q}\in \{\forall,\exists\}$, alors $\mathrm{Q}$ est l'op\'erateur principal de $A$
		\end{itemize}

		\item[D6.] Soit $F$ une formule. L'ensemble $\mathrm{VL}(F)$ des variables libres de $F$ l'ensemble $\mathrm{VM}(F)$ des variables muettes (ou li\'ees) de $F$ sont d\'efinies par r\'ecurrence sur $\tau(F)$.
		
		Une occurrence d'une variable donn\'ee est dite "\NewTerm{variable li\'ee}\index{variable li\'ee}" ou "\NewTerm{variable muette}\index{variable muette}" dans une formule $F$ si dans cette même formule, un quantificateur y fait r\'ef\'erence. Dans le cas contraire, nous disons avoir une "\NewTerm{variable libre}\index{variable libre}".
		
		\begin{tcolorbox}[title=Remarque,colframe=black,arc=10pt]
		Une occurrence d'une variable $x$ dans une formule $F$ est une position de cette variable dans la formule $F$. Ne pas confondre avec l'objet qu'est la variable elle-même!
		\end{tcolorbox}
		Pour pr\'eciser les variables libres possibles d'une formule $F$, nous noterons $F[x_1,\ldots,x_n]$. Cela signifie que les variables libres de $F$ sont parmi $x_1,\ldots,x_n$ in extenso si $y$ est libre dans $F$, alors $y$ est l'un des $x_i$ mais les $x_i$ n'apparaissent pas n\'ecessairement dans $F$.
		
		Nous pouvons d\'efinir les variables muettes ou libres de manière plus formelle:
		\begin{enumerate}
			\item Si $F=R(t_1,\ldots,t_n)$ est atomique alors $\mathrm{VL}(F)$ est l'ensemble des variables libres apparaissant dans les equation et nous avons alors pour les variables muettes $\mathrm{VM}(F)=\varnothing$.
	
			\item Si $F=F_1\oplus F_2$ où $\oplus \in\{\vee,\wedge,\rightarrow\}:\mathrm{FV}(F)=\mathrm{FV}(F_1)\cup \mathrm{FV}(F_2)$ alors $\mathrm{DV}(F)=\mathrm{DV}(F_1)\cup \mathrm{DV}(F_2)$.
	
			\item Si $F=\neg F_1$ où $\mathrm{FV}(F)=\mathrm{FV}(F_1)$ alors $\mathrm{DV}(F)=\mathrm{DV}(F_1)$.
	
			\item Si $F=\mathrm{Q}xF_1$ avec $\mathrm{Q}\in\{\forall,\exists\}:\mathrm{FV}(F)=\mathrm{FV}(F_1)-\{x\}$ et $\mathrm{DV}(F)=\mathrm{DV}(F_1)\cup\{x\}$.
		\end{enumerate}
		\begin{tcolorbox}[colframe=black,colback=white,sharp corners]
		\textbf{{\Large \ding{45}}Exemples:}\\\\
		E1. Soit $F$: $\forall x\;(x\cdot y=y\cdot x)$ alors $\mathrm{FV}(F)=\{y\}$ et $\mathrm{DV}(F)=\{x\}$\\
		
		E2. Soit $G$: $\{\forall x\exists y(x\cdot z=z\cdot y)\}\wedge \{x=z\cdot z\}$ alors $\mathrm{FV}(G)=\{x,z\}$ et $\mathrm{DV}(G)=\{x,y\}$.
		\end{tcolorbox}
		
		\item[D7.] Nous disons que les formules $F$ et $G$ sont "\NewTerm{$\alpha$-\'equivalentes}\index{formules $\alpha$-\'equivalentes}" si elles sont (syntaxiquement) identiques à un renommage près des occurrences li\'ees des variables.

		\item[D8.] Une "\NewTerm{formule close}\index{formule close}"  est une formule sans variables libres.

		\item[D9.]  Soit $F$ une formule,  $x$ une variable et  $t$ un terme. $F[x:=t]$ est la formule obtenue en remplaçant dans $F$ toutes les occurrences libres de $x$ par $t$, après renommage \'eventuel des occurrences li\'ees de $F$ qui apparaissent libres dans $t$.

	\end{enumerate}
	\begin{tcolorbox}[title=Remarques,colframe=black,arc=10pt]
	\textbf{R1.} Nous noterons dans les exemples vus qu'une variable peut avoir à la fois des occurrences libres et des occurrences li\'ees. Nous n'avons donc pas toujours  $\mathrm{VL}(F)\cap \mathrm{VM}(F)=\varnothing$.\\
	
	\textbf{R2.}  Nous ne pouvons pas renommer $y$ en $x$ dans la formule $\forall y\;(x\cdot y=y\cdot x)$ et obtenir la formule $\forall x\;(x\cdot x=x\cdot x)$: la variable $x$ serait "captur\'ee". Nous ne pouvons donc pas renommer des variables li\'ees sans pr\'ecautions: il faut \'eviter de capturer des occurrences libres!
	\end{tcolorbox}
	
	\subsection{D\'emonstrations (preuves)}
	Les d\'emonstrations que l'on trouve dans les ouvrages de math\'ematiques sont des assemblages de symboles math\'ematiques et de phrases contenant des mots-cl\'es tels que: "donc", "parce que", "si", "si et seulement si", "il est n\'ecessaire que", "il suffit de", "prenons un $x$ tel que", "supposons que", "cherchons une contradiction", etc. Ces mots sont suppos\'es être compris par tous de la même manière, ce qui n'est en fait, pas toujours le cas.
	
	Dans tout ouvrage, le but d'une d\'emonstration est de convaincre le lecteur de la v\'erit\'e de l'\'enonc\'e. Suivant le niveau du lecteur, cette d\'emonstration sera plus ou moins d\'etaill\'ee: quelque chose qui pourra être consid\'er\'e comme \'evident dans un cours de licence pourrait ne pas l'être dans un cours de niveau inf\'erieur.

	Dans un devoir, le correcteur sait que le r\'esultat demand\'e à l'\'etudiant est vrai et il en connaît la d\'emonstration. L'\'etudiant doit d\'emontrer (correctement) le r\'esultat demand\'e. Le niveau de d\'etail qu'il doit donner  d\'epend donc de la confiance qu'aura le correcteur: dans une bonne copie, une "preuve par une r\'ecurrence \'evidente" passera bien, alors que dans une copie où il y eu auparavant un "\'evident", qui \'etait \'evidemment... faux, ça ne passera pas!

	Pour pouvoir g\'erer convenablement le niveau de d\'etail, il faut savoir ce qu'est une d\'emonstration complète. Ce travail de formalisation n'a \'et\'e fait qu'au d\'ebut de 20ème siècle!!

	Plusieurs choses peuvent paraître surprenantes:
	\begin{enumerate}
		\item Il n'y a qu'un nombre fini de règles: deux pour chacun des connecteurs (et l'\'egalit\'e) plus trois règles g\'en\'erales. Il n'\'etait pas du tout \'evident a piori qu'un nombre fini de règles soit suffisant pour d\'emontrer tout ce qui est vrai. Nous montrerons ce r\'esultat (c'est essentiellement, le th\'eorème de compl\'etude). La preuve n'en est pas du tout triviale.
	
		\item Ce sont les mêmes règles pour toute la math\'ematique et la physique: algèbre, analyse, g\'eom\'etrie, etc. Cela veut dire que nous avons r\'eussi à isoler tout ce qui est g\'en\'eral dans un raisonnement. Nous verrons plus loin qu'une d\'emonstration est un assemblage de couples $(\Gamma, A)$, où $\Gamma$ est un ensemble de formules (les hypothèses) et $A$ une formule (la conclusion). Quand nous faisons de l'arithm\'etique, de la g\'eom\'etrie ou de l'analyse r\'eelle, nous utilisons, en plus des règles, des hypothèses que l'on appelle des "axiomes". Ceux-ci expriment les propri\'et\'es particulières des objets que nous manipulons (pour plus de d\'etails sur les axiomes voir la section Introduction  page \pageref{axiom}).
	\end{enumerate}
	Nous d\'emontrons donc, en g\'en\'eral, des formules en utilisant un ensemble d'hypothèses, et cet ensemble peut varier au cours de la d\'emonstration: quand nous disons "supposons $F$ et montrons $G$", $F$ est alors une nouvelle hypothèse que nous pourrons utiliser pour montrer $G$. Pour formaliser cela, nous introduisons le concept de "s\'equent".
	
	\textbf{D\'efinitions (\#\mydef):}
	\begin{enumerate}
		\item[D1.] Un "\NewTerm{s\'equent}\index{s\'equent}" est un couple (not\'e $\Gamma\vdash F$) où: 
		\begin{enumerate}
			\item $\Gamma$ est un ensemble fini de formules qui repr\'esente les hypothèses que nous pouvons utiliser. Cet ensemble s'appelle aussi le "\NewTerm{contexte du s\'equent}\index{contexte du s\'equent}".
	
			\item $F$ est une formule. C'est la formule que nous voulons montrer. Nous dirons que cette formule est la "\NewTerm{conclusion du s\'equent}\index{conclusion of a sequent}". 
		\end{enumerate}
		Le symbole "$\vdash$" doit être lu comme "\NewTerm{thèse}\index{symbole!thèse}" ou "\NewTerm{d\'emontre}\index{symbole!d\'emontre}".
		\begin{tcolorbox}[title=Remarques,colframe=black,arc=10pt]
		\textbf{R1.} Si $\Gamma=\{A_1,\ldots,A_n$ nous pourrons noter $A_1,\ldots,A_n \vdash F$ au lieu de $\Gamma \vdash F$. \\
		
		\textbf{R2.} Nous noterons  $\vdash F$  un s\'equent dont l'ensemble d'hypothèses est vide et  $\Gamma_1,\ldots,\Gamma_n \vdash F$ un s\'equent dont l'ensemble d'hypothèses est $\bigcup_i \Gamma_i$.\\
		
		\textbf{R3.} Nous \'ecrirons $\Gamma \not\vdash F$ pour dire que "$F$ est non d\'emontrable".
		\end{tcolorbox}
		
		\item[D2.] Un s\'equent $\Gamma\vdash F$ est dit "\NewTerm{prouvable}\index{provuable}" (ou \NewTerm{d\'emontrable}) s'il peut être obtenu par une application finie de règles. Une formule $F$ est prouvable si le s\'equent $\vdash F$ est prouvable.
	\end{enumerate}

	\subsubsection{Règles de d\'emonstration}
	Les règles de d\'emonstration sont les briques qui permettent de construire les d\'erivations. Une d\'erivation formelle est un assemblage fini (et correct!) de règles. Cet assemblage n'est pas lin\'eaire (ce n'est pas une suite) mais un "arbre". Nous sommes en effet souvent amen\'es à faire des branchements.

	Nous allons pr\'esenter un choix de règles. Nous aurions pu en pr\'esenter d'autres (à la place ou en plus) qui donneraient la même notion de prouvabilit\'e. Celles que l'on a choisies sont "naturelles" et correspondent aux raisonnements que l'on fait habituellement en math\'ematique. Dans la pratique courante nous utilisons, en plus des règles ci-dessous, beaucoup d'autres règles mais celles-ci peuvent se d\'eduire des pr\'ec\'edentes. Nous les appellerons "\NewTerm{règles d\'eriv\'ees}\index{règles d\'eriv\'ees}".

	Il est de tradition d'\'ecrire la racine de l'arbre (le s\'equent conclusion) en bas, les feuilles en haut: la nature est ainsi faite... Comme il est \'egalement de tradition d'\'ecrire, sur une feuille de papier, de haut en bas, il ne serait pas d\'eraisonnable d'\'ecrire la racine en haut et les feuilles en bas. Il faut faire un choix !
	\begin{itemize}
		\item d'un ensemble de "pr\'emisses": chacune d'elles est un s\'equent. Il peut y en avoir z\'ero, un ou plusieurs.

		\item du s\'equent conclusion de la règle

		\item d'une barre horizontale s\'eparant les pr\'emisses (en haut) de la conclusion (en bas). Sur la droit de la barre, nous indiquerons le nom de la règle.
	\end{itemize}
	\begin{tcolorbox}[colframe=black,colback=white,sharp corners]
	\textbf{{\Large \ding{45}}Exemple:}\\\\
	
	Cette règle a deux pr\'emisses ($\Gamma\vdash A \rightarrow$ et $\Gamma\vdash A$) et une conclusion ($\Gamma\vdash B$) et se note de manière abr\'eg\'ee sous la forme: $\rightarrow_e$. Elle peut se lire de deux manières:
	\begin{itemize}
		\item de bas en haut: si nous voulons prouver la conclusion, il suffit par utilisation de la règle de prouver les pr\'emisses. C'est ce qu'on fait quand nous cherchons une d\'emonstration. Cela correspond à "l'analyse".

		\item de haut en bas: si nous avons prouv\'e les pr\'emisses, alors nous avons aussi prouv\'e la conclusion. C'est ce que nous faisons quand nous r\'edigeons une d\'emonstration. Cela correspond à la "synthèse".
	\end{itemize}
	\end{tcolorbox}
	
	\pagebreak
	Pour les d\'emonstrations il existe un nombre fini de règles au nombre de 17 que nous allons d\'efinir ci-après:
	\begin{enumerate}
		\item Axiome:
		
		De bas en haut: si la conclusion du s\'equent est une des hypothèses, alors le s\'equent est prouvable.
		
		\item Affaiblissement:
		
		Explications:
		\begin{itemize}
		\item De haut en bas: si nous d\'emontrons $A$ sous les hypothèses $\Gamma$, en ajoutant d'autres hypothèses on peut encore d\'emontrer $A$.

		\item De bas en haut: il y a des hypothèses qui peuvent ne pas servir
		\end{itemize}

		\item Introduction de l'implication:
		
		De bas en haut: pour montrer que $A \rightarrow B$ nous supposons $A$ (c'est-à-dire que nous l'ajoutons aux hypothèses) et nous d\'emontrons $B$.
		
		\item \'elimination de l'implication:
		
		De bas en haut: pour d\'emontrer $B$, si nous connaissons un th\'eorème de la forme $A\rightarrow B$ et si nous pouvons d\'emontrer le lemme $A\rightarrow B$, il suffit de d\'emontrer $A$.
		
		\item Introduction à la conjonction:
		
		De bas en haut: pour montrer $A\wedge B$, il suffit de montrer $A$ et de montrer $B$.
		
		\item \'elimination de la conjonction:
		
		De haut en bas: de $A\wedge B$, nous pouvons d\'eduire $A$ (\'elimination gauche) et $B$ (\'elimination droite).
		
		\item Introduction de la disjonction:
		
		De bas en haut: pour d\'emontrer $A\vee B$, il suffit de d\'emontrer $A$ (disjonction gauche) ou de d\'emontrer $B$ (disjonction droite).
		
		\item Elimination de la disjonction:
		
		De bas en haut: si nous voulons montrer $C$ et que nous savons que nous avons $A\vdash B$, il suffit de le montrer d'une part en supposant $A$, d'autre part en supposant $B$. C'est un raisonnement par cas.
		
		\item Introduction de la n\'egation:
		
		De bas en haut: pour montrer $\neg A$, nous supposons $A$  et nous d\'emontrons l'absurde ($\perp$).
		
		\item \'elimination de la n\'egation:
		
		De haut en bas: si nous avons montr\'e  $\neg A$ et $A$, alors nous avons d\'emontr\'e l'absurde ($\perp$).

		\item Absurdit\'e classique (reductio ad absurdum):
		
		De bas en haut: pour d\'emontrer $A$, il suffit de d\'emontrer l'absurde en supposant $\neg A$.
		
		Cette règle, est \'equivalente à dire: $A$ est vraie si et seulement si il est faux que $A$ soit fausse. Cette règle ne va pas de soi: elle est n\'ecessaire pour prouver certains r\'esultats (il y a des r\'esultats que nous ne pouvons pas prouver si nous n'avons pas cette règle). Contrairement, à beaucoup d'autres, cette règle peut par ailleurs être appliqu\'ee à tout moment. Nous pouvons, en effet, toujours dire: pour prouver $A$ je suppose $\neg A$ et je vais chercher une contradiction.
		
		\item Introduction du quantificateur universel:
		
		 De bas en haut: pour d\'emontrer $\forall x\; A$, il suffit de montrer $A$ en ne faisant aucune hypothèse sur $x$.
		\begin{tcolorbox}[title=Remarque,colframe=black,arc=10pt]
		Pour des d\'emonstrations cette v\'erification (aucune hypothèse sur $x$) est souvent source d'erreurs.
		\end{tcolorbox}
		
		\item \'elimination du quantificateur universel:
		
		De haut en bas: de  $\forall x\; A$, nous pouvons d\'eduire $A[x:0t]$ pour n'importe quel terme $t$. Ce que nous pouvons dire aussi sous la forme: si nous avons prouv\'e $A$ pour tout $x$, alors nous pouvons utiliser $A$ avec n'importe quel objet $t$ (!!).
		
		\item Introduction du quantificateur existentiel:
		
		De bas en haut: pour d\'emontrer $\exists x\; A$, il suffit de trouver un objet (in extenso un terme $t$) pour lequel nous savons montrer $A[x:=t]$.
		
		\item \'elimination du quantificateur existentiel:
		
		De bas en haut: nous d\'emontrons qu'il existe bien un ensemble d'hypothèses tel que $\exists x \; A$ et partant de ce r\'esultat comme nouvelle hypothèse, nous d\'emontrons $C$. Cette formule $C$ h\'erite alors de la formule $\exists x \; A$  et dès lors $x$ n'est pas libre dans $C$ car il ne l'\'etait d\'ejà pas dans $\exists x \; A$.
		
		\item Introduction de l'\'egalit\'e:
		
		De bas en haut: nous pouvons toujours montrer $t=t$. Cette règle signifie que l'\'egalit\'e est r\'eflexive (\SeeChapter{voir section Op\'erateurs page \pageref{reflexive}}).
		
		\item \'elimination de l'\'egalit\'e:
		
		De haut en bas: si nous avons d\'emontr\'e $\Gamma\vdash A[x:=t]$ et $t = u$, alors nous avons d\'emontr\'e $A[x:=u]$. Cette règle exprime que les objets \'egaux ont les mêmes propri\'et\'es. Nous noterons cependant que les formules (ou relations) $t =u$ et $u = t$ ne sont pas, formellement, identiques. Il nous faudra d\'emontrer que l'\'egalit\'e est sym\'etrique (nous en profiterons aussi pour d\'emontrer que l'\'egalit\'e est transitive).
	\end{enumerate}
	Voyons maintenant trois exemples en les introduisant sous forme de th\'eorèmes comme il se doit dans la th\'eorie de la preuve!
	
	\pagebreak
	\begin{theorem}
	 l'\'egalit\'e est sym\'etrique (un petit peu non trivial mais bon pour commencer):
	\end{theorem}
	\begin{dem}
	
	De haut en bas: nous introduisons l'\'egalit\'e $=_i$ et prouvons à partir de l'hypothèse $x_1=x_2$ la formule $x_1=x_1$. En même temps, nous d\'efinissons l'axiome comme quoi $x_1=x_2$. Ensuite à partir de ces pr\'emisses, nous \'eliminons l'\'egalit\'e $=_e$ en substituant les termes de façon à ce qu'à partir de la supposition $x_1=x_2$ (venant de l'axiome) nous obtenions $x_2=x_1$. Ensuite, l'\'elimination de l'\'egalit\'e implique automatiquement sans aucune hypothèse que $x_1=x_2\rightarrow x_2=x-1$. Dès lors, il nous suffit d'introduire le quantificateur universel pour chacune des variables (donc deux fois) sans aucune hypothèse afin d'obtenir que l'\'egalit\'e est sym\'etrique.
	\begin{flushright}
		$\blacksquare$  Q.E.D.
	\end{flushright}
	\end{dem}
	
	\begin{theorem}
	L'\'egalit\'e est transitive (c'est-à-dire si $x_1=x_2$ et $x_2=x_3$ alors $x_1=x_3$) . En notant $F$ la formule $(x_1=x_2)\wedge (x_2=x_3)$:
	\end{theorem}

	\begin{dem}
	
	Que faisons, nous ici ? Nous introduisons d'abord la formule $F$ deux fois en tant qu'axiome afin de la "d\'ecortiquer" plus tard à gauche et à droite (nous n'introduisons pas l'\'egalit\'e suppos\'ee d\'ejà introduite en tant que règle). Une fois ceci fait, nous \'eliminons à gauche et à droite la conjonction sur la formule pour travailler sur les termes gauches et droites seuls et introduisons l'\'egalit\'e sur les deux termes ce qui fait qu'à partir de la formule nous avons l'\'egalit\'e transitive. Il s'ensuit que sans aucune hypothèse cela implique automatiquement que l'\'egalit\'e est transitive et finalement nous disons que ceci est valable pour toute valeur des diff\'erentes variables (si la formule est vraie, alors l'\'egalit\'e est transitive).
	\begin{flushright}
		$\blacksquare$  Q.E.D.
	\end{flushright}
	\end{dem}
	Et maintenant le dernier gros exemple sous la forme d'un th\'eorème:
	\begin{theorem}
	Toute involution est une bijection  (\SeeChapter{voir section de Th\'eorie des Ensembles page \pageref{bijection}}).
	\end{theorem}

	\begin{dem}
	 Soit $f$ un symbole de fonction unaire (à une variable), nous notons (pour plus de d\'etails voir le chapitre de Th\'eorie Des Ensembles page \pageref{functions and applications}):
	\begin{itemize}
		\item $\text{Inj}[f]$ la formule:
		
		qui signifie que $f$ est injective.
		
		\item $\text{Surj}[f]$ la formule:
		
		qui signifie que $f$ est surjective
		
		\item  $\text{Bij}[f]$ la formule:
		
		qui signifie que $f$ est bijective.
		
		\item $\text{Inv}[f]$ la formule:
		
	\end{itemize}
	qui signifie que $f$ est une involution (nous notons \'egalement cela $f\circ f=\text{Id}$\label{identity application}) c'est-à-dire que la composition de $f$ est l'identit\'e).
	
	Nous souhaiterions savoir si:\\
	
	
	Nous allons pr\'esenter (en essayant que ce soit au plus clair) cette d\'emonstration de quatre manières diff\'erentes: classique (informelle), classique (pseudo-formelle), formelle en arbre et formelle en ligne.
	
	\begin{itemize}
		\item \textbf{M\'ethode classique (informelle):}
		
		Nous devons montrer que si $f$ est involutive alors elle est donc bijective. Nous avons donc deux choses à montrer (et les deux doivent être satisfaites en même temps): que la fonction est injective et surjective.
		
		\begin{enumerate}
			\item Montrons dans un premier temps que l'involution est injective. 
			
			Nous supposons pour cela, puisque $f$ est involutive elle est donc injective, tel que:
			
			implique:
			
			Or, cette supposition d\'ecoule automatiquement de la d\'efinition de l'involution que:
			
			et de l'application de $f$ à la relation:
			
			(soit trois \'egalit\'es $=_e \times 2$) tel que:
			
			nous avons donc:
			
			
			\item Montrons que l'involution est surjective.
			
			Si elle est surjective, alors nous devons avoir:
			
			
			Or, d\'efinissons la variable $x$ par d\'efinition de l'involution elle-même:
			
			(puisque $y=f(x)$...) un changement de variables après nous obtenons:
			
			et donc la surjectivit\'e est assur\'ee.
			\end{enumerate}
			
			\item \textbf{M\'ethode pseudo-formelle:}
			
			Nous reprenons la même chose et nous y injectons les règles de la th\'eorie de la d\'emonstration:
			
			Nous devons montrer que $f$ involutive est donc bijective. Nous avons donc deux choses à montrer ($\wedge_i$) (et les deux doivent être satisfaites en même temps): que la fonction est injective et surjective:
			
			
			\begin{enumerate}
			\item Montrons d'abord que l'involution est injective. Nous supposons pour cela, puisque $f$ est involutive et donc injective, que:
			
			implique:
			
			Or, cette supposition d\'ecoule automatiquement de la d\'efinition de l'involution que:
			
			et de l'application de $f$ à la relation:
			
			(soit trois \'egalit\'es $=_e \times 2$) tel que:
			
			Nous avons donc:
			
			
			\item Montrons que l'involution est surjective. Si elle est surjective, alors nous devons avoir:
			
			Or, d\'efinissons la variable $x$ par d\'efinition de l'involution elle-même:
			
			puisque $y=f(x)$....,  un changement de variables après nous obtenons:
			
			et donc:
			
			la surjectivit\'e est assur\'ee.
		\end{enumerate}
		
		\item \textbf{M\'ethode formelle en arbre:}
			
		aisons cela avec la m\'ethode graphique que nous avons d\'ejà pr\'esent\'ee plus haut.
		\begin{enumerate}
			\item Montrons que l'involution est injective:
			
			Pour cela, montrons d'abord que:
			
			Donc:
			
			
			Ce qui nous amène à \'ecrire:
			
			\begin{tcolorbox}[title=Remarque,colframe=black,arc=10pt]
			Cette dernière relation est abr\'eg\'ee $=_c$ et appel\'ee (comme d'autres existantes) "règle d\'eriv\'ee" car c'est un raisonnement qui est très souvent fait lors de d\'emonstrations et un peu long à d\'evelopper à chaque fois...
			\end{tcolorbox}
			Dès lors:
			
			
			\item Montrons que l'involution est surjective:
			
			Il s'ensuit:
			
		\end{enumerate}
		
		\item \textbf{M\'ethode formelle en ligne:}
		
		Nous pouvons faire la même chose sous une forme un peu moins... large... et plus tabul\'ee... (cela n'en est pas moins indigeste):
		\begin{subequations}
		\begin{align}
			&\text{Inv}[f]\vdash \text{Bij}[f]&& \vee i\\
			&\quad 1:=\text{Inv}[f]\vdash \text{Inj}[f]&&\nonumber\\
			&\quad 2:=\text{Inv}[f]\vdash \text{Surj}[f]&&\nonumber\\[5pt]
			&(1)\text{Inv}[f]\vdash \text{Inj}[f]&& \forall i\\
			&\quad\text{Inv}[f]\vdash f(x)=f(y)\rightarrow x=y &&\rightarrow_i\nonumber\\
			&\quad\text{Inv}[f], f(x)=f(y)\vdash x=y&&=_e\times 2 (i)(ii)(iii)\nonumber\\
			&\quad(i)\quad\text{Inv}[f]\vdash f(f(x))=x &&\forall_e\nonumber\\
			&\quad\quad\quad\text{Inv}[f]\vdash \text{Inv}[f] &&\text{ax}\nonumber\\
			&\quad(ii)\quad\text{Inv}[f]\vdash f(f(y))=y &&\forall_e\nonumber\\
			&\quad\quad\quad\text{Inv}[f]\vdash \text{Inv}[f] &&\text{ax}\nonumber\\
			&\quad(iii)\quad f(x)=f(y)\vdash f(f(x))=f(f(y))&&=_c\nonumber\\
			&\quad\quad(1')f(x)=f(y)\vdash f(x)=f(y) &&\text{ax}\nonumber\\[5pt]
			&(2)\text{Inv}[f]\vdash \text{Surj}[f]&& \forall i\\
			&\quad\text{Inv}[f]\vdash \exists x \{f(x)=y\}&&\exists_i\nonumber\\
			&\quad\text{Inv}[f]\vdash \{f(x)=y\}[x:=f(y)]&&\forall_e\nonumber\\
			&\quad\text{Inv}[f]\vdash \forall x \{f(f(x))=x\} &&\text{ax}\nonumber
			\end{align}
		\end{subequations}
	\end{itemize}

	\begin{flushright}
		$\blacksquare$  Q.E.D.
	\end{flushright}
	\end{dem}
	Attention! Cependant, tout ce formalisme hautement technique ne permet pas toujours de savoir où se trouve l'erreur dans la "pseudo-preuve" suivante:
	Assume:
	
	Multiplions les deux côt\'es par $a$:
	
	Soustrayons par $b^2$:
	
	Nous factorisons:
	
	Nous divisons par $(a-b)$:
	
	Comme $a=b$, nous avons:
	
	Dès lors:
	
	Si vous ne voyez pas l'erreur, voici l'analyse:
	\begin{table}[H]
		\centering
		\begin{tabular}{|l|l|c|}
		\hline
		\rowcolor[HTML]{9B9B9B} 
		\multicolumn{1}{|c|}{\cellcolor[HTML]{9B9B9B}\textbf{\'ecrit}} & \multicolumn{1}{c|}{\cellcolor[HTML]{9B9B9B}\textbf{R\'ealit\'e}} & \textbf{Jugement} \\ \hline
		$a=b$ & $a=a$ & Vrai \\ \hline
		$a^2=ab$ & $a^2=aa$ & Vrai \\ \hline
		$a^2-b^2=ab-b^2$ & $a^2-a^2=aa-a^2$ & Vrai ($=0$) \\ \hline
		$(a+b)(a-b)=b(a-b)$ & $(a+a)(a-a)=a(a-a)$ & Vrai ($=0$) \\ \hline
		$a+b=b$ & $a+a=a$ & \begin{tabular}[c]{@{}c@{}}Faux (except\'e si $a=0$)\\ et interdit comme diviser par $a-b$ \\ est \'equivalent à diviser par $0$\end{tabular} \\ \hline
		$b+b=b$ & $a+a=a$ & Faux \\ \hline
		$2b=b$ & $2a=a$ & Faux \\ \hline
		$2=1$ & $2=1$ & Faux \\ \hline
		\end{tabular}
	\end{table}
	
	Bien évidemment il existe de nombreux autres types de preuves (...) comme :
	\begin{itemize}
		\item \textbf{Preuve par intimidation:} Trivial!
	
		\item \textbf{Preuve par notation lourde:} Le théorème découle immédiatement du fait que $\left|\oplus_{k \in S}\left(\mathfrak{r}^{F^{a}(i)}\right)_{i \in U_{k}}\right| \leqslant \aleph_{1}$ when $[\mathfrak{R}]_{w} \cap$ $\mathbb{F}^{\alpha}(\mathbb{N}) \neq \emptyset$
	
		\item \textbf{Preuve par la littérature inaccessible:} Le théorème est un corollaire facile d'un résultat prouvé dans une note manuscrite remise lors d'une conférence par la Société mathématique yougoslave en 1973.
	
		\item \textbf{Preuve fantôme:} La preuve peut être trouvée à la page 478 dans un manuel qui s'avère avoir 396 pages.
	
		\item \textbf{Preuve par argument circulaire:} La proposition $5.18$ dans $[\mathrm{BL}]$ est un corollaire facile du théorème $7.18$ dans $[\mathrm{C}]$, qui est encore basé sur le corollaire $2.14$ dans $[K]$. Ceci, d'autre part, est dérivé en référence à la proposition $5.18$ dans $[\mathrm{BL}]$.
	
		\item \textbf{Preuve par autorité:} Mon bon collègue Andrew a dit qu'il pensait qu'il aurait pu en apporter une preuve il y a quelques années... 
	
		\item \textbf{Preuve par référence Internet:} Pour ceux que ça intéresse, le résultat est affiché sur la page web de ce livre (ou wikipedia) et qui n'est malheureusement plus accessible.
	
		\item \textbf{Preuve par évitement:} Chapitre 3 : La preuve de ceci est retardée jusqu'au chapitre 7, lorsque nous avons développé la théorie encore plus loin. Chapitre 7 : Pour simplifier les choses, nous ne le démontrons que pour le cas $z=0$, mais le cas général est traité dans l'annexe $C$. Annexe $C$ : La preuve formelle dépasse le cadre de ce livre, mais bien sûr, notre intuition sait que cela est vrai.
	\end{itemize}
	
	\subsubsection{La logique est-elle une science?}
	Rappelons que cette question au sens large du mot "science" a évidemment une réponse positive. Cependant, comme nous l'avons vu dans l'introduction de ce livre, il existe différentes sciences!

	Si l'on désigne par "science" tout domaine centré autour de la "méthode scientifique": faire des hypothèses, faire des expériences dans le but de les réfuter (c'est-à-dire de les falsifier) et ainsi de suite, alors la Logique n'est rien de tel (la Logique n'a pas pour objectif de réfuter ses prémisses). La science expérimentale est intrinsèquement basée sur le monde physique et observable. Bien qu'il y ait une réflexion approfondie, des hypothèses a posteriori, l'arbitre ultime de toute idée scientifique expérimentale est le monde sensible, à travers l'expérimentation.
	
	La logique, quant à elle, est entièrement autonome. C'est l'étude des systèmes formels : des ensembles d'axiomes et de règles sur la façon de dériver des théorèmes. La logique est un jeu où toutes les règles sont inventées.
	
	Bien que la logique ne soit pas une science empirique ou une science physique, elle n'en est pas moins une science au sens large du terme, c'est-à-dire un corpus systématique de connaissances rationnelles.
	
	Alors que toute utilisation de la logique peut probablement retomber dans le monde réel, la Logique elle-même ne le fait pas. Elle est entièrement autonome. Lorsque nous prouvons quelque chose de manière logique, nous ne faisons pas une déclaration directe sur l'univers : nous faisons une affirmation cohérente avec les hypothèses de notre système logique et qui peut en outre ne pas être réfutable (c'est-à-dire non falsifiable) - ou pire encore - le défaut majeur historique connu de la Logique (nous avons plus de 2500 ans de preuves de tels défauts) est que la conclusion de tout raisonnement logique est toujours dans les prémisses. La relation entre la logique et l'univers est séparée : c'est ce qui rend la logique utile, mais ce n'est pas ce qui rend la logique logique.

	L'un des exemples les plus connus dans la science "moderne" de niveau universitaire de logique fallacieuse est \textit{l'argument cosmologique de Kalâm} (un argument en faveur de l'existence d'une divinité qui aurait créé notre Univers) le second est très probablement le pari de Pascal. La prémisse de l'argument cosmologique de Kalâm, « \textit{Tout ce qui commence a une cause} », peut être un peu logique pour un individu de premier cycle universitaire ou un personne scientifiquement illétrée ou un physicien non professionnel. Mais pour les physiciens professionnels de niveau postgrade, cela est prouvé mathématiquement et expérimentalement comme faux au moins depuis les années 1950 (nous en verrons les preuves dans la section Physique Quantique Ondulatoire \pageref{wave quantum physics} et de Gravité Quantique page \pageref{newton quantum gravitation } beaucoup plus loin dans ce livre). Donc, quiconque utilise cet argument comme une affirmation logique ne fait que prouver son ignorance de la physique moderne\footnote{Un autre exemple typique est qu'en raison de l'existence d'une table de vérité, certaines personnes soutiennent à nouveau que la logique peut conduire à de vraies conclusions. Mais les tables de vérité dépendent encore de la prémisse et si la prémisse est fausse... devinez quoi !} et aussi de la logique elle-même (puisque la conclusion est dans la prémisse... facepalm). Et de plus comme nous l'avons également vu lors de notre introduction de ce livre, une affirmation ou une citation est au maximum une évidence de niveau 02 (voir page \pageref{evidence level chart})...

	Très probablement le seul exemple bien connu dans la science "moderne" de niveau post-universitaire de logique fallacieuse (où encore une fois la conclusion est dans les prémisses\footnote{Un argument est "circulaire" juste au cas où il y aurait une prémisse, implicite ou explicite, qui est logiquement équivalente à la conclusion. Dans de nombreux cas, la circularité est un problème et nous appelons une telle circularité une "\NewTerm{circularité vicieuse}".}) est la \textit{preuve ontologique de Gödel} pour l'existence de (très probablement... ) la divinité chrétienne... comme cela convient bien aux croyances de Gödel... La plupart des théistes partagent le résumé de preuve suivant sur les réseaux sociaux ceci sans évidemment en reconnaître tous les biais car ils sont pour l'immense majorité d'entre eux scientifiquement illétrés:
	
	\begin{flushright}
	\begin{tabular}{l c}
	\circled{80} & \pbox{20cm}{\score{4}{5} \\ {\tiny 64 votes, 80.94\%}} 
	\end{tabular} 
	\end{flushright}
	\begin{flushright}
	{\setlength{\parskip}{0pt}{\tiny Version: 3.1 Revision 5 | Last update: 2015-09-06 16:33}}
	\end{flushright}
	
	%to make section start on odd page
	\newpage
	\thispagestyle{empty}
	\mbox{}
	\section{Nombres}

\lettrine[lines=4]{\color{BrickRed}L}a base des math\'ematiques, mis à part le raisonnement (\SeeChapter{voir section Th\'eorie de la D\'emonstration page \pageref{proof theory}}), est sans nul doute pour le commun des personnes l'arithm\'etique. Il est donc obligatoire que nous y fassions \'etape pour \'etudier sa provenance, quelques-unes de ses propri\'et\'es et cons\'equences.

Les nombres, comme les figures g\'eom\'etriques, constituent les bases de l'arithm\'etique. Ce sont aussi les bases historiques car la math\'ematique a certainement commenc\'e par l'\'etude de ces objets, mais aussi les bases p\'edagogiques, car c'est en apprenant à compter que nous entrons dans le monde de la math\'ematique.

L'histoire des nombres, appel\'es \'egalement parfois "\NewTerm{scalaires}"\index{scalar}\label{scalaires}, est beaucoup trop longue pour être relat\'ee ici, mais nous ne pouvons que vous conseiller un des meilleurs ouvrages francophones sur le sujet:\textit{ Histoire Universelle des chiffres} (~2'000 pages), Georges Ifrah, ISBN: 2221057791.

Cependant voici une petite bride de cette dernière qui nous semble fondamentale:

Notre système d\'ecimal actuel, de base $10$, utilise les chiffres de $0$ à $9$, dits "chiffres arabes", mais au fait d'origine indienne (hindous). Effectivement, les chiffres arabes (d'origine indienne...) dans le tableau ci-dessous sont la première ligne et nous voyons qu'ils sont nettement diff\'erents des "chiffres indiens" de la deuxième ligne:

\begin{figure}[H]
\centering
\includegraphics{img/arithmetics/numbers.eps}
\caption{Nombres Indo-Arabes}
\end{figure}

Il faut lire dans ce tableau: $0$ "\NewTerm{z\'ero}", $1$ "\NewTerm{un}", $2$ "\NewTerm{deux}", $3$ "\NewTerm{trois}", $4$ "\NewTerm{quatre}", $5$ "\NewTerm{cinq}", $6$ "\NewTerm{six}", $7$ "\NewTerm{sept}", $8$ "\NewTerm{huit}", $9$ "\NewTerm{neuf}". Ce système est beaucoup plus efficace que les chiffres romains (essayez de faire un calcul avec le système de notation romain vous allez voir...).

Ces chiffres ne furent introduits en Europe que vers l'an 1000. Utilis\'es en Inde, ils furent transmis par des Arabes au monde occidental par le pape Gerbert d'Aurillac lors de son s\'ejour en Andalousie à la fin du 9ème siècle. 

	\begin{tcolorbox}[title=Remarque,colframe=black,arc=10pt]
	Le mot français "\NewTerm{chiffre}\index{chiffre}" est une d\'eformation du mot arabe "sifr" d\'esignant "z\'ero". En italien, "z\'ero" se dit "zero", et serait une contraction de "zefiro", on voit là encore la racine arabe mais le z\'ero serait aussi d'origine indienne... Ainsi nos termes "chiffre" et "z\'ero" ont la même origine.
	\end{tcolorbox}
	
	L'usage pr\'ecoce d'un symbole num\'erique d\'esignant "rien", au sens de "aucune quantit\'e" ou "absence de quantit\'e", c'est à dire notre "\NewTerm{z\'ero}"\index{z\'ero}, provient du fait que les indiens utilisèrent un système dit "\NewTerm{système positionnel}"\index{système positionnel}. Dans un tel système, la position d'un chiffre dans l'\'ecriture d'un nombre exprime la puissance de $10$ et le nombre de fois qu'elle intervient... et l'absence d'une position dans ce système posait d'\'enormes problèmes de relecture et pouvait engendrer de grosses erreurs de calculs. L'introduction r\'evolutionnaire et pourtant simple du concept de rien permettait un relecture sans erreur des nombres.

L'absence d'une puissance est not\'ee par un petit rond...: c'est le z\'ero. Notre système actuel est donc le "\NewTerm{système d\'ecimal et positionnel}"\index{système d\'ecimal et positionnel}.


	\begin{tcolorbox}[colframe=black,colback=white,sharp corners]
	\textbf{{\Large \ding{45}}Exemple:}\\\\
	\begin{figure}[H]
	\centering
	\includegraphics{img/arithmetics/decimal_system.jpg}
	\caption{Description système d\'ecimal et positionnel}
	\end{figure}
	
	Le nombre $324$ s'\'ecrit de gauche à droite comme \'etant trois centaines: $3$ fois $100$, deux dizaines: $2$ fois $10$ et quatre unit\'es: $4$ fois $1$.
	\end{tcolorbox}

Donc un "\NewTerm{nombre d\'ecimal}\index{nombre d\'ecimal}" est donc u nombre qui a une \'ecriture finie en base $10$.

Nous voyons parfois (et c'est conseill\'e) un s\'eparateur de milliers repr\'esent\'e par une apostrophe $'$ en Suisse (pos\'e tous les trois chiffres à partir du premier en partant de la droite pour les nombres entiers). Ainsi, nous \'ecrirons $1'034$ au lieu de $1034$ ou encore $1'344'567'569$ au lieu de $1344567569$. Les s\'eparateurs de milliers permettent de rapidement quantifier l'ordre de grandeur des nombres lus.

	Ainsi: 
	\begin{itemize}
		\item Si nous voyons uniquement une apostrophe nous saurons que le nombre est de l'ordre du millier
		\item Si nous voyons deux apostrophes nous saurons que le nombre est de l'ordre du million
		\item Si nous voyons trois apostrophes nous saurons que le nombre est de l'ordre du milliard
		\item etc.
	\end{itemize}
et ainsi de suite... :
\begin{figure}[H]
\centering
\includegraphics{img/arithmetics/numbers_scale.jpg}
\caption{Principe de construction du système positionnel}
\end{figure}

	Au fait, tout nombre entier, autre que l'unit\'e, peut être pris pour base d'un système de num\'erotation. Nous avons ainsi les systèmes de num\'erotation binaire, ternaire, quaternaire,..., d\'ecimal, duod\'ecimal qui correspondent respectivement aux bases deux, trois, quatre,..., dix, douze.

Une g\'en\'eralisation de ce qui a \'et\'e vu pr\'ec\'edemment, peut s'\'ecrire sous la forme suivante:

\label{number power decomposition}Tout nombre entier positif $N$ peut être repr\'esent\'e dans une base $b$ sous forme de somme, où les coefficients $a_i$ sont multipli\'es chacun par leur poids respectif $b^i$. Tel que:
	
Plus \'el\'egamment \'ecrit:
	
avec $a_i \in \left[0,b-1\right]$ et $b_i \in \left[1,b^{n-1}\right]$.

	\begin{tcolorbox}[title=Remarques,colframe=black,arc=10pt]
	\textbf{R1.} Comme très fr\'equemment en math\'ematique, nous remplacerons l'\'ecriture des chiffres ou des nombres par des lettres latines ou grecques afin de g\'en\'eraliser leur repr\'esentation. Ainsi, lorsque nous parlons d'une base $b$ la valeur $b$ peut prendre n'importe quelle valeur entière $1$, $2$, $3$, ...\\
	
	\textbf{R2.} Lorsque nous prenons la valeur $2$ pour $b$, $N$ aura pour valeur maximale $2^n-1$. Les nombres qui s'\'ecrivent sous cette forme s'appellent les "\NewTerm{nombres de Mersenne}\index{nombres de Mersenne}". Ces nombres ne peuvent être premiers (voir plus bas ce qu'est un nombre premier) que si $n$ est premier.\\
	
	Effectivement, si nous prenons (par exemple) $b=10$ et $n=3$, la plus grande valeur que nous pourrons obtenir sera:
	
	\textbf{R3.} Lorsque qu'un nombre est le même lu de gauche à droite ou de droite à gauche, nous parlons de nombre "\NewTerm{palindrome}\index{palindrome}".
	\end{tcolorbox}

\subsection{Bases Num\'eriques}
Pour \'ecrire un nombre dans un système en base $b$, nous devons commencer par adopter $b$ caractères destin\'es à repr\'esenter les $b$ premiers nombres, par exemple dans le système d\'ecimal: $\left\lbrace 0, 1, 2, 3, 4, 5, 6, 7, 8, 9\right\rbrace $. Ces caractères sont comme nous les avons d\'ejà d\'efinis, les "chiffres" que nous \'enonçons comme à l'ordinaire: $\left\lbrace \text{z\'ero}, \text{un}, \text{deux}, \text{trois}, \text{quatre}, \text{cinq}, \text{six}, \text{sept}, \text{huit}, \text{neuf}\right\rbrace $.

Pour la num\'erotation \'ecrite, nous faisons cette convention, qu'un chiffre, plac\'e à gauche d'un autre repr\'esente des unit\'es de l'ordre imm\'ediatement sup\'erieur, ou $b$ fois plus grandes. Pour tenir la place des unit\'es qui peuvent manquer dans certains ordres, nous nous servons du z\'ero  "$0$" et par suite, le nombre de chiffres employ\'es peut varier.

\textbf{D\'efinition (\#\mydef):} Pour la num\'erotation parl\'ee, nous convenons d'appeler "\NewTerm{unit\'e simple}\index{unit\'e simple}", "\NewTerm{dizaine}", "\NewTerm{centaine}", "\NewTerm{millier}", etc., les unit\'es du premier ordre, du second, du troisième, du quatrième, etc. Ainsi les nombres $10$, $11$, ..., $19$ se liront de même dans tous les systèmes de num\'erotation. Les nombres $1a, 1b, a0, b0, ...$ se liront dix-$a$, dix-$b\'e$, $a$-dix, $b\'e$-dix, etc. Ainsi, le nombre 5b6a71c se lira:
\begin{center}
cinq millions $b\'e$-cent soixante-$a$ mille sept cent dix-$c\'e$
\end{center}
Cet exemple est pertinent car il nous montre l'expression g\'en\'erale de la langue parl\'ee que nous utilisons quotidiennement et intuitivement en base dix (faute à notre \'education).

	\begin{tcolorbox}[title=Remarques,colframe=black,arc=10pt]
\textbf{R1.}  Les règles des op\'erations d\'efinies pour les nombres \'ecrits dans le système d\'ecimal sont les mêmes pour les nombres \'ecrits dans un système quelconque de num\'erotation.\\

\textbf{R2.} Pour op\'erer rapidement dans un système quelconque de num\'erotation, il est indispensable de savoir par coeur toutes les sommes et tous les produits de deux nombres d'un seul chiffre.\\

\textbf{R3.} Le choix de la base d\'ecimale semblerait dû au fait que l'humain a dix doigts.
	\end{tcolorbox}
Voyons comment nous convertissons un système de num\'erotation dans un ordre:

	\begin{tcolorbox}[colframe=black,colback=white,sharp corners]
\textbf{{\Large \ding{45}}Exemples:}\\\\
E1. En base dix nous savons que $142'713$ s'\'ecrit:
	
E2. En base deux (base binaire) le nombre $0110$ s'\'ecrirait en base $10$:
	
	\end{tcolorbox}
	et ainsi de suite... 
	
	L'inverse (pour l'exemple de la base deux) est toujours un peu plus d\'elicat:
	\begin{tcolorbox}[colframe=black,colback=white,sharp corners]
	\textbf{{\Large \ding{45}}Exemples:}\\\\
	E1.  la conversion du nombre d\'ecimal $1'492$ en base deux se fait par divisions successives par 2 des restes et donne (le principe est à peu près identique pour toutes les autres bases):
	\begin{figure}[H]
	\centering
		\includegraphics[scale=1]{img/arithmetics/decimal_to_binary.jpg}
		\caption{Conversion d\'ecimal en binaire}
	\end{figure}
	E2. Pour convertir le nombre $142'713$ (base d\'ecimale) en base duod\'ecimale (base douze) nous avons (notation: $q$ est le "quotient", et $r$ le "reste"):
		
	Ainsi nous avons les restes $6$, $10$, $7$, $0$, $9$ ce qui nous amène à \'ecrire:
		
	Nous avons choisi pour ce cas particulier la symbolique que nous avions d\'efinie pr\'ec\'edemment ($a$-dix) pour \'eviter toute confusion.
	\end{tcolorbox}

	\pagebreak
	\subsection{Types de Nombres}\label{type of numbers}
	Il existe en math\'ematiques une très grande vari\'et\'e de nombres (naturels, rationnels, r\'eels, irrationnels, complexes, p-adiques, quaternions, transcendants, alg\'ebriques, constructibles...) puisque le math\'ematicien peut à loisirs en cr\'eer en ayant uniquement à poser les axiomes (règles) de manipulations de ceux-ci
	
	Cependant, il y en a quelques-uns que nous retrouvons plus souvent que d'autres et certains qui servent de base de construction à d'autres et qu'il conviendrait de d\'efinir suffisamment rigoureusement (sans aller dans les extrêmes) pour pouvoir savoir de quoi nous parlerons lorsque nous les utiliserons.

	\subsubsection{Nombres Entiers Naturels}\label{natural numbers}
	L'id\'ee du "\NewTerm{nombre entier}\index{nombre entier}" (nombre pour lequel il n'y a pas de chiffres après la virgule) est le concept fondamental de la math\'ematique et nous vient à la vue d'un groupement d'objets de même espèce (un mouton, un autre mouton, encore un autre, etc.). 
	
	Lorsque la quantit\'e d'objets d'un groupe est diff\'erente de celle d'un autre groupe nous parlons alors de groupe num\'eriquement sup\'erieur ou inf\'erieur quel que soit l'espèce d'objets contenus dans ces groupes. Lorsque la quantit\'e d'objets d'un ou de plusieurs groupes est \'equivalente, nous parlons alors "\NewTerm{d'\'egalit\'e}\index{\'egalit\'e}". 
	
	À chaque objet correspond le nombre "\NewTerm{un}" ou "\NewTerm{unit\'e}\index{unit}" not\'e "$1$" dans le système d\'ecimal.

	Pour former des groupements d'objets, nous pouvons op\'erer ainsi: à un objet, ajouter un autre objet, puis encore un et ainsi de suite... chacun des groupements, au point de vue de sa collectivit\'e, est caract\'eris\'e par un nombre. Il r\'esulte de là qu'un nombre peut être consid\'er\'e comme repr\'esentant un groupement d'unit\'es tel que chacune de ces unit\'es corresponde à un objet de la collection.

\textbf{D\'efinition (\#\mydef):} Deux nombres sont dits "\NewTerm{\'egaux}\index{\'egaux}" si à chacune des unit\'es de l'un nous pouvons faire correspondre une unit\'e de l'autre et inversement. Si ceci ne se v\'erifie pas alors nous parlons "\NewTerm{d'in\'egalit\'e}\index{in\'egalit\'e}".

Prenons un objet, puis un autre, puis au groupement form\'e, ajoutons encore un objet et ainsi de suite. Les groupements ainsi constitu\'es sont caract\'eris\'es par des nombres qui, consid\'er\'es dans le même ordre que les groupements successivement obtenus, constituent la "\NewTerm{suite naturelle}\index{suite naturelle}", aussi parfois appel\'ee  "\NewTerm{nombres entiers}\index{nombres entiers}", et not\'ee:
	
	Pour être sans ambiguït\'e quant à l'inclusion ou non du $0$, un index (ou un exposant) est parfois ajout\'e dans le dernier cas:
	
	\begin{tcolorbox}[title=Remarque,colframe=black,arc=10pt]
	La pr\'esence du $0$ (z\'ero) dans la d\'efinition ci-dessus de $\mathbb{N}$ est d\'ebatable $0$ puisqu'il est ni positif ni n\'egatif. C'est pourquoi dans de nombreux ouvrages vous trouverez la d\'efinition de $\mathbb{N}$ sans le $0$ (correspondant alors aussi à $\mathbb{Z}^+$) et alors l'ensemble $\mathbb{N}\cup \{0\}$ (aussi parfois not\'e $\mathbb{Z}_{\geq 0}$ ou $\mathbb{N}_0$) est nomm\'e "l'ensemble calculable" ou des "nombres entiers". Mais gardez à l'esprit qu'il ne semble pas y avoir d'accord g\'en\'eral et que ce n'est pas d\'efini dans la s\'erie de normes ISO 80000 !!!
	\end{tcolorbox}

Les constituants de cet ensemble peuvent être d\'efinis par (nous devons cette d\'efinition au math\'ematicien Frege Gottlob) les propri\'et\'es (avoir lu au pr\'ealable le chapitre de Th\'eorie des Ensembles est recommand\'e... à la page \pageref{set theory}) suivantes:
	\begin{enumerate}
		\item[P1.] $0$ (lire "z\'ero") est le nombre d'\'el\'ements (d\'efini comme une relation d'\'equivalence) de tous les ensembles \'equivalents à (en bijection avec) l'ensemble vide.
		
		\item[P2.] $1$ (lire "un") est le nombre d'\'el\'ements de tous les ensembles \'equivalents à l'ensemble dont le seul \'el\'ement est $1$.
		
		\item[P3.] $2$ (lire "deux") est le nombre d'\'el\'ements de tous les ensembles \'equivalents à l'ensemble dont tous les \'el\'ements sont $0$ et $1$.
		
		\item[P4.]  En g\'en\'eral, un nombre entier est le nombre d'\'el\'ements de tous les ensembles \'equivalents à l'ensemble des nombres entiers le pr\'ec\'edant!
	\end{enumerate}
La construction de l'ensemble des entiers naturels s'est faite de la manière la plus naturelle et coh\'erente qui soit. Les naturels doivent leur nom à ce qu'ils avaient pour objet, aux pr\'emices de leur existence, de d\'enombrer des quantit\'es et des choses de la nature ou qui intervenaient dans la vie de l'homme. L'originalit\'e de l'ensemble r\'eside dans la manière empirique dont il s'est construit car il ne r\'esulte pas r\'eellement d'une d\'efinition math\'ematique, mais davantage d'une prise de conscience par l'homme du concept de quantit\'e d\'enombrable, de nombre et de lois qui traduisent des relations entre eux.

La question de l'origine de $\mathbb{N}$ est dès lors la question de l'origine des math\'ematiques. Et de tout temps des d\'ebats confrontant les pens\'ees des plus grands esprits philosophiques ont tent\'e d'\'elucider ce profond mystère, à savoir si la math\'ematique est une pure cr\'eation de l'esprit humain ou si au contraire l'homme n'a fait que red\'ecouvrir une science qui existait d\'ejà dans la nature. Outre les nombreuses questions philosophiques que cet ensemble peut susciter, il n'en est pas moins int\'eressant d'un point de vue exclusivement math\'ematique. Du fait de sa structure, il pr\'esente des propri\'et\'es remarquables qui peuvent se r\'ev\'eler d'une grande utilit\'e lorsque l'on pratique certains raisonnements ou calculs.

Remarquons imm\'ediatement que la suite naturelle des nombres entiers est illimit\'ee $\mathbb{N}$ (\SeeChapter{voir section Th\'eorie des Nombres page \pageref{natural numbers}}) mais d\'enombrable (nous verrons cela plus bas), car, à un groupement d'objets qui se trouve repr\'esent\'e par un certain nombre $n$, il suffira d'ajouter un objet pour obtenir un autre groupement qui sera d\'efini par un nombre entier imm\'ediatement sup\'erieur $n + 1$.

\textbf{D\'efinition (\#\mydef):} Deux nombres entiers qui diff\'erent d'une unit\'e positive sont dits  "\NewTerm{cons\'ecutifs}\index{cons\'ecutifs}".

	\paragraph{Axiomes de Peano}\label{peano axioms}\mbox{}\\\\
	Lors de la crise des fondements des math\'ematiques, les math\'ematiciens ont bien \'evidemment cherch\'e à axiomatiser l'ensemble $\mathbb{N}$ et nous devons l'axiomatisation actuelle à Peano et à Dedekind.

	Les axiomes de ce système comportent les symboles $<$ et $=$ pour repr\'esenter les relations "plus petit" et "\'egal" (\SeeChapter{voir section Op\'erateurs page \pageref{comparators}}). Ils comprennent d'autre part les symboles $0$ pour le nombre z\'ero et $s$ pour repr\'esenter le nombre "successeur". Dans ce système, $1$ est not\'e:
	
	dit "successeur de z\'ero", $2$ est not\'e:
	
	Les axiomes de Peano qui construisent $\mathbb{N}$ sont les suivants (voir le chapitre de la Th\'eorie de la D\'emonstration page \pageref{proof theory} pour certains symboles):
	\begin{enumerate}
		\item[A1.] $0$ est un entier naturel (permet de poser que $\mathbb{0}$ n'est pas vide).

		\item[A2.] Tout entier naturel $n$ a un successeur, not\'e $s(n)$.
		
		Donc $s$ est une application injective (\SeeChapter{voir section de Th\'eorie des Ensembles page \pageref{injective}}), c'est-à-dire:
		
		C'est-à-dire que si deux successeurs sont \'egaux, ils sont les successeurs d'un même nombre.
		
		\item[A3.] Le successeur d'un entier naturel n'est jamais \'egal à z\'ero (ainsi $\mathbb{N}$ à un premier \'el\'ement):
		
		
		\item[A4.] Si nous d\'emontrons un propri\'et\'e $\varphi$ qui est vraie pour $x$ et son successeur $s(x)$, alors la propri\'et\'e est vraie pour tout $x$ (c'est "\NewTerm{l'axiome de r\'ecurrence}\index{axiome de r\'ecurrence}"):
		
		Ainsi, l'ensemble de tous les nombres satisfaisant les quatre axiomes ci-dessus est not\'e:
	\end{enumerate}
	Donc l'ensemble de tous les nombres v\'erifiant ces 4 axiomes est: 
	
	\begin{tcolorbox}[title=Remarque,colframe=black,arc=10pt]
	Les axiomes de Peano permettent de construire très rigoureusement les deux op\'erations de base de l'arithm\'etique que sont l'addition et la multiplication (\SeeChapter{voir la section Op\'erateurs page \pageref{addition} et page \pageref{addition}}) et ainsi tous les autres ensembles que nous verrons par la suite.
	\end{tcolorbox}
	
	\pagebreak
	\paragraph{Nombres Pairs, Impairs et Parfaits}\mbox{}\\\\
	En arithm\'etique, \'etudier la parit\'e d'un entier, c'est d\'eterminer si cet entier est ou non un multiple de $2$. Un entier multiple de $2$ est un entier pair, les autres sont les entiers impairs.
	
	\textbf{D\'efinitions (\#\mydef):}
	\begin{enumerate}
		\item[D1.] Les nombres obtenus en comptant par deux à partir de z\'ero, (soit $0$, $2$, $4$, $6$, $8$, ...) dans cette suite naturelle sont appel\'es "\NewTerm{nombres pairs}\index{nombres pairs}".
		
		Le $n^{\text{th}}$  nombre pair est donn\'e par la relation:
		
		Ou plus esth\'etiquement:
		
		
		\item[D2.]  Les nombres que nous obtenons en comptant par $2$ à partir de $1$ (soit $1$, $3$, $5$, $7$,... ) dans cette suite naturelle s'appellent "\NewTerm{nombres impairs}\index{nombres impairs}".
		
		Le  $(n+1)^{\text{th}}$ nombre impair est donn\'e par:
		
		Ou plus esth\'etiquement:
		
	\end{enumerate}
	
	\begin{tcolorbox}[title=Remarque,colframe=black,arc=10pt]
	Nous appelons "\NewTerm{nombres parfaits}\index{nombres parfaits}", les nombres \'egaux à la somme de leurs diviseurs entiers strictement plus petits qu'eux mêmes (concept que nous verrons en d\'etail plus tard) comme par exemple: $6=1+2+3$ et $28=1+2+4+7+14$.
	\end{tcolorbox}
	
	\paragraph{Nombres Premiers}\mbox{}\label{prime number}\\\\
	\textbf{D\'efinition (\#\mydef):} Un "\NewTerm{nombre premier}\index{nombre premier}" est un entier poss\'edant exactement $2$ diviseurs (ces deux diviseurs sont donc "$1$" et le nombre lui-même). Dans le cas où il y a plus de deux diviseurs on parle de"\NewTerm{nombre compos\'e}\index{nombre compos\'e}". La propri\'et\'e d'être premier est ce que l'on appelle la "\NewTerm{primalit\'e}\index{primalit\'e}".
	
	L’\'etude des nombres premiers est un sujet vate en math\'ematiques (voir, pour un petit exemple, la section de la page Th\'eorie des Nombres  \pageref{fundamental theorem of arithmetic} ou la section de Cryptographie \pageref{cryptography}). Il existe des livres de milliers de pages sur le sujet et probablement des centaines d'articles de recherche par mois, même en ce d\'ebut de 21ème siècle! La plupart des th\'eorèmes sont en grande partie hors du niveau du pr\'esent livre (et de l'int\'erêt de son auteur principal ...)!
	
	Voici la liste des nombres premiers inf\'erieurs à $1000$:
	
	2, 3, 5, 7, 11, 13, 17, 19, 23, 29, 31, 37, 41, 43, 47, 53, 59, 61,
	 67, 71, 73, 79, 83, 89, 97, 101, 103, 107, 109, 113, 127, 131, 137, 
	 139, 149, 151, 157, 163, 167, 173, 179, 181, 191, 193, 197, 199, 211, 
	 223, 227, 229, 233, 239, 241, 251, 257, 263, 269, 271, 277, 281, 283, 
	 293, 307, 311, 313, 317, 331, 337, 347, 349, 353, 359, 367, 373, 379, 
	 383, 389, 397, 401, 409, 419, 421, 431, 433, 439, 443, 449, 457, 461, 
	 463, 467, 479, 487, 491, 499, 503, 509, 521, 523, 541, 547, 557, 563, 
	 569, 571, 577, 587, 593, 599, 601, 607, 613, 617, 619, 631, 641, 643, 
	 647, 653, 659, 661, 673, 677, 683, 691, 701, 709, 719, 727, 733, 739, 
	 743, 751, 757, 761, 769, 773, 787, 797, 809, 811, 821, 823, 827, 829, 
	 839, 853, 857, 859, 863, 877, 881, 883, 887, 907, 911, 919, 929, 937, 
	 941, 947, 953, 967, 971, 977, 983, 991, 997
	
	L'ensemble des nombres premiers est traditionnelement not\'e $\mathbb{P}$.
	
	\begin{tcolorbox}[title=Remarque,colframe=black,arc=10pt]
	 A noter que la d\'efinition de nombre premier exclut le chiffre "$1$" de l'ensemble des nombres premiers car il a un unique diviseur (lui-même) et pas deux comme le veut la d\'efinition.
	\end{tcolorbox}
	
	Nous pouvons nous demander s'il existe une infinit\'e de nombres premiers ? La r\'eponse est positive et en voici une d\'emonstration (parmi tant d'autres) par l'absurde.
	
	\begin{dem}
	Supposons qu'il n'existe qu'un nombre fini de nombres premiers qui seraient:
	
	Nous formons un nouveau nombre à partir du produit de tous les nombres premiers auquel nous ajoutons "$1$":
	
	Selon notre hypothèse initiale et le th\'eorème fondamental de l'arithm\'etique (\SeeChapter{voir Section de Th\'eorie des Nombres page \pageref{fundamental theorem of arithmetic}}) ce nouveau nombre $N$ devrait être divisible par l'un des nombres premiers $p_i$ existants selon:
	
	où $q$ est un nombre entier. Nous pouvons effectuer la division:
	
	Le premier terme se simplifie, car $p_i$ est dans le produit. Nous notons $E$ cet entier:
	
	Or, $q$ et $E$  sont deux entiers, donc $1/p_i$ doit être un entier. Mais $p_i$ est par d\'efinition sup\'erieur à $1$. Donc $1/p_i$ n'est pas un entier et il en est de même pour $q$.
	
	Il y a alors contradiction et nous en concluons que les nombres premiers ne sont pas en nombre fini, mais infini.
	\begin{flushright}
		$\blacksquare$  Q.E.D.
	\end{flushright}
	
	\end{dem}
	\begin{tcolorbox}[title=Remarques,colframe=black,arc=10pt]
	\textbf{R1.} Le produit  $p_n=p_1p_2...p_n$ (le produit des $n$ premiers nombres premiers inf\'erieurs ou \'egaux à $n$) est appel\'e "\NewTerm{$n$ primorielle}\index{$n$ primorielle}".\\
	
	\textbf{R2.} Nous renvoyons le lecteur au chapitre de Cryptographie de la section d'Informatique Th\'eorique à la page  \pageref{cryptography} pour \'etudier quelques propri\'et\'es remarquables des nombres premiers dont la non moins fameuse fonction phi d'Euler (ou appel\'e aussi "fonction indicatrice") et une application industrielle/\'economique type au 20-21ème sciècle.\\
	\end{tcolorbox}
	
	\subsubsection{Nombres Entiers Relatifs}
	L'ensemble $\mathbb{N}$ à quelques d\'efauts que nous n'avons pas \'enonc\'es tout à l'heure. Par exemple, la soustraction de deux nombres dans $\mathbb{N}$ n'a pas toujours un r\'esultat dans $\mathbb{N}$ (les nombres n\'egatifs n'y existent pas). Autre d\'efaut..., la division de deux nombres dans $\mathbb{N}$ n'a \'egalement pas toujours un r\'esultant dans $\mathbb{N}$ (les nombres fractionnaires - rationnels ou irrationnels - n'y existent pas). Nous idosn alors dans le language de la th\'eorie des ensembles que: la soustraction et la division ne sont pas des op\'erations internes dans $\mathbb{N}$.

	Nous pouvons dans un premier temps r\'esoudre le problème de la soustraction en ajoutant à l'ensemble des entiers naturels, les entiers n\'egatifs (concept r\'evolutionnaire pour ceux qui en sont à l'origine) nous obtenons "\NewTerm{l'ensemble des entiers relatifs}\index{l'ensemble des entiers relatifs}" not\'e $\mathbb{Z}$ (pour "Zahl" de l'allemand):
	
	L'ensemble des entiers naturels est donc inclus dans l'ensemble des entiers relatifs. C'est ce que nous notons sous la forme (\SeeChapter{voir section de Th\'eorie des Ensembles page \pageref{subset}}):
	
	et nous avons par d\'efinition (ce sont des notations qu'il faut apprendre): 
	
	Cet ensemble a \'et\'e cr\'e\'e à l'origine pour faire de l'ensemble des entiers naturels $\mathbb{N}$ un objet que nous appelons un "groupe" (\SeeChapter{voir section de Th\'eorie des Ensembles page \pageref{groups}}) par rapport à l'addition.
	
	\textbf{D\'efinition (\#\mydef):}  Nous disons qu'un ensemble $E$ est un "\NewTerm{ensemble d\'enombrable}\index{ensemble d\'enombrable}", s'il est \'equipotent à $\mathbb{N}$. C'est-à-dire s'il existe une bijection de (\SeeChapter{voir section de Th\'eorie des Ensembles page \pageref{bijection}}) $S$ sur $E$. Ainsi, grosso modo, deux ensembles \'equipotents ont "autant" d'\'el\'ements au sens de leurs cardinaux (\SeeChapter{voir section de Th\'eorie des Ensembles page \pageref{cardinal}}), ou tout au moins la même infinit\'e.
	
	L'objectif de cette concept est de faire comprendre que les ensembles $\mathbb{N}$ et $\mathbb{Z}$ sont d\'enombrables.
	
	\begin{dem}
		Montrons que $\mathbb{Z}$ est d\'enombrable en posant:
		
		pour tout entier $k\geq 0$. Ceci donne l'\'enum\'eration suivante:
		
		de tous les entiers relatifs à partir des entiers naturels seuls.
		\begin{flushright}
			$\blacksquare$  Q.E.D.
		\end{flushright}
	\end{dem}
	
	\pagebreak
	\subsubsection{Nombres Rationnels}
	L'ensemble $\mathbb{Z}$ a aussi un d\'efaut. Ainsi, la division de deux nombres dans $\mathbb{Z}$ n'a \'egalement pas toujours un r\'esultat dans $\mathbb{Z}$ (tous les nombres fractionnaires n'y existent pas). Nous disons alors dans le langage de la th\'eorie des ensembles que: la division n'est pas une loi interne dans $\mathbb{Z}$.

Nous pouvons ainsi d\'efinir un nouvel ensemble qui contient tous les nombres qui peuvent s'\'ecrire sous forme de "fraction", c'est-à-dire du rapport d'un dividende (num\'erateur) et d'un diviseur (d\'enominateur). Quand un nombre peut se mettre sous cette forme, nous disons que c'est une  "\NewTerm{nombre fractionnaire}\index{nombre fractionnaire}":
	
	Une fraction peut être employ\'ee pour exprimer une partie, ou une part, de quelque chose
(d'un objet, d'une distance, d'un terrain, d'une somme d'argent...).
	
	Pour mieux comprendre le nombre rationnel (fractions), consid\'erons deux individus: Andy et Bobby qui adorent la pizza. Le lundi soir, ils partagent une pizza à parts \'egales. Quelle quantit\'e de pizza chaque personne reçoit-elle? Vous pensez que chaque garçon reçoit la moiti\'e de la pizza? C'est juste! Il y a une pizza entière, divis\'ee \'egalement en deux parties, de sorte que chaque garçon reçoit l'une des deux parties \'egales. En maths, nous \'ecrivons $\dfrac{1}{2}$ comme une partie sur deux:
	\begin{figure}[H]
		\centering
		\includegraphics[scale=0.75]{img/arithmetics/pizza_fraction_example_01.jpg}
		\caption[Exemple d'une $1/2$ fraction de pizza]{Une $1/2$ fraction de pizza (source: OpenStax)}
	\end{figure}
	Mardi, Andy et Bobby partagent une pizza avec leurs parents, Fred et Christy, chaque personne obtenant une quantit\'e \'egale de la pizza entière. Quelle quantit\'e de pizza chaque personne reçoit-elle? Il y a une pizza entière, divis\'ee \'egalement en quatre parties \'egales. Chaque personne a l'une des quatre parties \'egales, donc chacune a $ \dfrac{1}{4}$ de la pizza:
	\begin{figure}[H]
		\centering
		\includegraphics[scale=0.75]{img/arithmetics/pizza_fraction_example_02.jpg}
		\caption[Exemple d'une $1/4$ fraction de pizza]{Une $1/4$ fraction de pizza (source: OpenStax)}
	\end{figure}
	Mercredi, la famille invite des amis pour un dîner de pizza. Il y a un total de $12$ personnes. S'ils partagent la pizza à parts \'egales, chaque personne recevra $\dfrac{1}{12}$ de la pizza:
	\begin{figure}[H]
		\centering
		\includegraphics[scale=0.75]{img/arithmetics/pizza_fraction_example_03.jpg}
		\caption[Exemple d'une $1/12$ fraction de pizza]{Une $1/12$ fraction de pizza (source: OpenStax)}
	\end{figure}
	
	Par d\'efinition, "\NewTerm{l'ensemble des rationnels}\index{ensemble des rationnels}\label{rational numbers}" est donn\'e par:
	
	En d'autres termes, tout nombre rationnel est un nombre pouvant être exprim\'e sous la forme du quotient ou de la fraction $ p / q $ de deux entiers, $ p $ et $ q $, le d\'enominateur $ q $ n'\'etant pas \'egal à z\'ero. Puisque $ q $ peut être \'egal à $ 1 $, chaque entier est pour le coup aussi un nombre rationnel!
	\begin{tcolorbox}[title=Remarque,colframe=black,arc=10pt]
	Les purs math\'ematiciens diront que la d\'efinition ci-dessus est fausse. Oui, en effet, ce n’est pas exact, mais suffisant pour les ing\'enieurs et les physiciens. Alors voici la bonne d\'efinition:
	
	où $ q> 0 $ \'evite la division par z\'ero ET la r\'ep\'etition du signe "$-$" au dessus ou en dessous de la fraction. Nous avons \'egalement ajout\'e une propri\'et\'e importante: le plus grand commun diviseur (\SeeChapter{voir section de Th\'eorie des Nombres page \pageref{greatest common divisor}}), not\'e pgcd, entre les deux nombres doit être \'egal à z\'ero. Ainsi, dans le but que nous consid\'erons, par exemple, $3/6$ et $1/2$ ne sont pas deux nombres distincts. En fait, nous savons que nous pouvons toujours amener une fraction à sa forme irr\'eductible, où dans ce dernier cas nous aurons $\text{pgcd}(p,q)=1$.
	\end{tcolorbox}
	Nous supposerons par ailleurs comme \'evident que:
	
	La logique de la cr\'eation de l'ensemble des nombres rationnels $\mathbb{Q}$ est similaire à celle des entiers relatifs  $\mathbb{Z}$. Effectivement, les math\'ematiciens ont souhait\'e faire de l'ensemble des nombres relatifs un "groupe" par rapport à la loi de multiplication et de division (\SeeChapter{voir section de Th\'eorie des Ensembles page \pageref{multiplication} et page \pageref{division}}).
	
	De plus, contrairement à l'intuition, l'ensemble des nombres entiers $\mathbb{N}$ et nombres rationnels $\mathbb{Q}$ sont \'equipotents\label{natural and rational numbers equipotence}. Nous pouvons nous persuader de cette \'equipotence en rangeant comme le fit Cantor, les rationnels dans un premier temps de la façon suivante:
	
	\begin{figure}[H]
	\centering
	\begin{tikzpicture}
\matrix(m)[matrix of math nodes,column sep=0.5cm,row sep=0.5cm]{
    1/1 & 2/1 & 3/1 & 4/1 & 5/1 & 6/1 & 7/1 & \cdots \\
    1/2 & 2/2 & 3/2 & 4/2 & 5/2 & 6/2 & 7/2 & \cdots \\
    1/3 & 2/3 & 3/3 & 4/3 & 5/3 & 6/3 & 7/3 & \cdots \\
    1/4 & 2/4 & 3/4 & 4/4 & 5/4 & 6/4 & 7/4 & \cdots \\
    1/5 & 2/5 & 3/5 & 4/5 & 5/5 & 6/5 & 7/5 & \cdots \\
    1/6 & 2/6 & 3/6 & 4/6 & 5/6 & 6/6 & 7/6 & \cdots \\
    1/7 & 2/7 & 3/7 & 4/7 & 5/7 & 6/7 & 7/7 & \cdots \\
    \vdots & \vdots & \vdots & \vdots & \vdots & \vdots & \vdots & \cdots \\
};

\draw[->]
         (m-1-1)edge(m-1-2)
         (m-1-2)edge(m-2-1)
         (m-2-1)edge(m-3-1)
         (m-3-1)edge(m-2-2)
         (m-2-2)edge(m-1-3)
         (m-1-3)edge(m-1-4)
         (m-1-4)edge(m-2-3)
         (m-2-3)edge(m-3-2)
         (m-3-2)edge(m-4-1)
         (m-1-5)edge(m-1-6)
         (m-1-7)edge(m-1-8)
         
         (m-2-4)edge(m-1-5)
         (m-3-3)edge(m-2-4)
         (m-4-2)edge(m-3-3)
         (m-5-1)edge(m-4-2)
         
         (m-5-2)edge(m-6-1)
         (m-4-3)edge(m-5-2)
         (m-3-4)edge(m-4-3)
         (m-2-5)edge(m-3-4)
         (m-1-6)edge(m-2-5)
         
         (m-2-6)edge(m-1-7)
         (m-3-5)edge(m-2-6)
         (m-4-4)edge(m-3-5)
         (m-5-3)edge(m-4-4)
         (m-6-2)edge(m-5-3)
         (m-7-1)edge(m-6-2)
         
         (m-1-8)edge(m-2-7)
         (m-2-7)edge(m-3-6)
         (m-3-6)edge(m-4-5)
         (m-4-5)edge(m-5-4)
         (m-5-4)edge(m-6-3)
         (m-6-3)edge(m-7-2)
         (m-7-2)edge(m-8-1)
         
         (m-3-7)edge(m-2-8)
         (m-4-6)edge(m-3-7)
         (m-5-5)edge(m-4-6)
         (m-6-4)edge(m-5-5)
         (m-7-3)edge(m-6-4)
         (m-8-2)edge(m-7-3)
         
         (m-3-8)edge(m-4-7)
         (m-4-7)edge(m-5-6)
         (m-5-6)edge(m-6-5)
         (m-6-5)edge(m-7-4)
         (m-7-4)edge(m-8-3)
         
         (m-5-7)edge(m-4-8)
         (m-6-6)edge(m-5-7)
         (m-7-5)edge(m-6-6)
         (m-8-4)edge(m-7-5)
         
         (m-5-8)edge(m-6-7)
         (m-6-7)edge(m-7-6)
         (m-7-6)edge(m-8-5)
         
         (m-7-7)edge(m-6-8)
         (m-8-6)edge(m-7-7)
         
         (m-7-8)edge(m-8-7)
         
         (m-4-1)edge(m-5-1)
         (m-6-1)edge(m-7-1);
\end{tikzpicture}
	\caption{M\'etode diagonale de Cantor}
	\end{figure}
	Ce tableau est construit de telle manière que chaque rationnel n'apparaît qu'une seule fois (au sens de sa valeur d\'ecimale) par diagonale d'où le nom de la m\'ethode: "\NewTerm{diagonale de Cantor}\index{diagonale de Cantor}\label{Cantor's diagonal}".
	
	Si nous \'el\'eminons de chaque diagonale les rationnels qui apparaissent plus d'une fois (les "fractions \'equivalentes") pour ne garder plus que ceux qui sont irr\'eductibles (donc ceux dont le PGCD du num\'erateur et d\'enominateur est \'egal à $1$), nous pouvons alors ainsi grâce à cette distinction d\'efinir une application $f:\mathbb{N} \Rightarrow \mathbb{Q}$ qui est injective (deux rationnels distincts admettent des rangs distincts) et surjective (à toute place sera inscrit un rationnel). 
	

	L'application $f$ est donc bijective: $\mathbb{N}$ et $\mathbb{Q}$  sont donc bien \'equipotents !
	
	La d\'efinition un peu plus rigoureuse (et donc moins sympathique) de $\mathbb{Q}$ se fait à partir de $\mathbb{Z}$ en proc\'edant comme suit (il est int\'eressant d'observer les notations utilis\'ees):
	
	Sur l'ensemble $\mathbb{Z}\times \mathbb{Z} \ \lbrace 0 \rbrace$, qu'il faut lire comme \'etant l'ensemble construit à partir de deux \'el\'ements entiers relatifs dont on exclut le z\'ero pour le deuxième, on considère la relation $R$ entre deux couples d'entiers relatifs d\'efinie par:
	
	Nous v\'erifions facilement ensuite que $\mathcal{R}$ est une relation d'\'equivalence (\SeeChapter{voir la section Op\'erateurs page \pageref{equivalence relation}} sur $\mathbb{Z}\times \mathbb{Z} \setminus \lbrace 0 \rbrace$.
	
	L'ensemble des classes d'\'equivalences pour cette relation $\mathcal{R}$ not\'e alors $\left(\mathbb{Z}\times \mathbb{Z} \setminus \lbrace 0 \rbrace\right) / \mathcal{R}$ est par d\'efinition $\mathbb{Q}$. C'est-à-dire que nous posons alors plus rigoureusement:
	
	La classe d'\'equivalence de $(a,b) \in \mathbb{Z}\times\mathbb{Z}\setminus \{0\}$ est explicitement not\'ee par:
	
	conform\'ement à la notation que tout le monde a l'habitude d'employer.

	Nous v\'erifions facilement que l'addition et la multiplication qui \'etaient des op\'erations d\'efinies sur $\mathbb{Z}$ passent sans problèmes à  $\mathbb{Q}$  en posant:
	
	De plus ces op\'erations munissent $\mathbb{Q}$ d'une structure de corps (\SeeChapter{voir section Th\'eorie des Ensembles page \pageref{field (set)}}) avec $\dfrac{0}{1}$ comme \'el\'ement neutre additif et $\dfrac{1}{1}$ comme \'el\'ement neutre multiplicatif. Ainsi, tout \'el\'ement non nul de $\mathbb{Q}$ est inversible, en effet:
	
	ce qui s'\'ecrit aussi plus techniquement:
	
	\begin{tcolorbox}[title=Remarque,colframe=black,arc=10pt]
	Même si nous aurions envie de d\'efinir  $\mathbb{Q}$ comme \'etant l'ensemble $\mathbb{Z}\times \mathbb{Z}\setminus{0}$ où  $\mathbb{Z}$ repr\'esente les num\'erateurs et $\mathbb{Z}\setminus{0}$ les d\'enominateurs des rationnels, ceci n'est pas possible car autrement nous aurions par exemple $(1,2)\neq(2,4)$ tandis que nous nous attendons à une \'egalit\'e.\\
	
	D'où le besoin d'introduire une relation d'\'equivalence qui nous permet d'identifier, pour revenir à l'exemple pr\'ec\'edent, $(1,2)$ et $(2,4)$. La relation $\mathcal{R}$ que nous avons d\'efinie ne tombe pas du ciel, en effet le lecteur qui a manipul\'e les rationnels jusqu'à pr\'esent sans jamais avoir vu leur d\'efinition formelle sait que:
	
	Il est donc naturel de d\'efinir la relation $\mathcal{R}$ comme nous l'avons fait. En particulier, en ce qui concerne l'exemple ci-dessus, $\dfrac{1}{2}=\dfrac{2}{4}$ car $(1,2) \mathcal{R} (2,4)$ et le problème est r\'esolu.
	\end{tcolorbox}
	Outre les circonstances historiques de sa mise en place, ce nouvel ensemble se distingue des ensembles d'entiers relatifs car il induit la notion originale et paradoxale de quantit\'e partielle. Cette notion qui a priori n'a pas de sens, trouvera sa place dans l'esprit de l'homme notamment grâce à la g\'eom\'etrie où l'id\'ee de fraction de longueur, de proportion s'illustre plus intuitivement.
	
	\pagebreak
	\subsubsection{Nombres Irrationnels}
	Il peut sembler \'evident de pr\'esenter les nombres irrationnels avant des nombres r\'eels (voir plus bas), mais cela peut s’expliquer par le fait que c'est l'ordre de la d\'ecouverte dans l'histoire humaine et qu’il nous semble donc plus p\'edagogique de les pr\'esenter donc aussi dans ce même ordre.
	
	L'ensemble des rationnels  $\mathbb{Q}$ est limit\'e et non suffisant lui aussi. Effectivement, nous pourrions penser que tout calcul math\'ematique num\'erique avec les op\'erations commun\'ement connues se r\'eduisent à cet ensemble mais ce n'est pas le cas.
	
	\begin{tcolorbox}[colframe=black,colback=white,sharp corners]
	\textbf{{\Large \ding{45}}Exemples:}\\\\
	E1. Prenons le calcul de la racine carr\'ee de deux que nous noterons $\sqrt{2}$. Supposons que cette dernière racine soit un rationnel. Alors s'il s'agit bien d'un rationnel, nous devrions pouvoir l'exprimer comme $a/b$, où par de par la d\'efinition d'un rationnel $a$ et $b$ sont des entiers sans facteurs communs. Pour cette raison, $a$ et $b$ ne peuvent tous les deux être pairs. Il y a trois possibilit\'es restantes:
	\begin{enumerate}
		\item $a$ est impair (alors $b$ est pair)
		\item $a$ est pair (alors $b$ est impair)
		\item $a$ est impair (alors $b$ est impair)
	\end{enumerate}
	En mettant au carr\'e, nous avons:
	
	Ce qui peut s'\'ecrire: 
	
	Puisque le carr\'e d'un nombre impair est impair et le carr\'e d'un nombre pair est pair, le cas (1) est impossible, car $a^2$ serait impair et $2b^2$  serait pair.\\
	
	Le cas (2) est aussi impossible, car alors nous pourrions \'ecrire  $a=2c$, où $c$ est un entier quelconque, et donc si nous le portons au carr\'e nous avons $a^2=4c^2$ où nous avons un nombre pair des deux côt\'es de l'\'egalit\'e. En remplaçant dans $2b^2=a^2$ nous obtenons après simplification que $b^2=2c^2$. Alors $b^2$ serait impair alors que $2c^2$ serait pair.\\
	
	Le cas (3) est aussi impossible, car $a^2$ est donc alors impair et $2b^2$ est pair (que $b$ soit pair ou impair!).\\
	
	Il n'y a pas de solutions! C'est donc que l'hypothèse de d\'epart est fausse et qu'il n'existe pas deux entiers $a$ et $b$ tels que $\sqrt{2}=a/b$.
	\end{tcolorbox}

	\pagebreak
	\begin{tcolorbox}[colframe=black,colback=white,sharp corners]
	E2. D\'emontrons, aussi par l'absurde, que le fameux nombre d'Euler $e$ est irrationnel. Pour cela, rappelons que $e$ (\SeeChapter{voir section d'Analyse Fonctionnelle page \pageref{Euler number}}) peut aussi être d\'efini par la s\'erie de Taylor (\SeeChapter{viur section S\'equences et S\'eries page \pageref{euler maclaurin expansion}}):
	
	Alors si $e$ est rationnel, il doit pouvoir s'\'ecrire sous la forme $p/q$ (avec $q>1$, car nous savons que $e$ n'est pas entier). Multiplions les deux côt\'es de l'\'egalit\'e par $q!$:
	
	Le premier membre $q!e$ serait alors un entier, car par d\'efinition de la factorielle:
	
	est un entier.\\
	
	Les premiers termes du second membre de la relation ant\'epr\'ec\'edente, jusqu'au terme $q!/q!=1$ sont aussi des entiers car $q!/m!$ se simplifie si $q>m$. Donc par soustraction nous trouvons:
	
	où la s\'erie à droite devrait aussi être un entier!\\
	
	Après simplification, le second membre de l'\'egalit\'e devient:
	
	le premier terme de cette somme est strictement inf\'erieur à $1/2$, le deuxième inf\'erieur à $1/4$, le troisième inf\'erieur à $1/8$, etc.\\
	
	Donc, vu que chaque terme est strictement inf\'erieur aux termes de la s\'erie harmonique suivante qui converge vers $1$:
	
	alors par cons\'equent, la s\'erie n'est pas un entier puisque \'etant strictement inf\'erieure à $1$. Ce qui constitue une contradiction!
	\end{tcolorbox}
	Ainsi, les nombres rationnels ne satisfont pas à l'expression num\'erique de $\sqrt{2}$ comme de $e$ (pour citer seulement ces deux exemples particuliers).
	
	Il faut donc les compl\'eter par l'ensemble de tous les nombres qui ne peuvent s'\'ecrire sous forme de fraction (rapport d'un dividende et d'un diviseur entiers sans facteurs communs) et que nous appelons des "nombres irrationnels".
	
	Finallement nous pouvons dire:
	
	\textbf{D\'efinition (\#\mydef):} En math\'ematiques, un "\NewTerm{nombre irrationnel}\index{nombre irrationnel}" est un nombre r\'eel qui ne peut pas être exprim\'e sous forme de ratio d'entiers. Les nombres irrationnels ne peuvent pas être repr\'esent\'es comme des nombres d\'ecimaux finaux ou r\'ep\'et\'es.
	
	\subsubsection{Nombres R\'eels}
	\textbf{D\'efinition (\#\mydef):} La r\'eunion des nombres rationnels et irrationnels donne l'ensemble des "\NewTerm{nombres r\'eels}\index{nombres r\'eels} que nous notons:
	
	\begin{tcolorbox}[title=Remarque,colframe=black,arc=10pt]
	Les math\'ematiciens dans leur rigueur habituelle ont diff\'erentes techniques pour d\'efinir les nombres r\'eels. Ils utilisent pour cela des propri\'et\'es de la topologie (entre autres) et en particulier les suites de Cauchy mais c'est une autre histoire qui d\'epasse le cadre formel du pr\'esent chapitre. Pour une d\'efinition de $\mathbb{R}$ du point de vue de la th\'eorie des ensembles (les nombres r\'eels sont souvent d\'ecrits comme le "corps complet ordonn\'e") le lecteur doit se r\'ef\'erer à la section de Th\'eorie des Ensembles page \pageref{field (set)}.
	\end{tcolorbox}
	\begin{figure}[H]
		\centering
		\includegraphics{img/arithmetics/real_numbers.jpg}
		\caption{R\'esum\'e simplifi\'e d'ensemble de nombres}
	\end{figure}
	Nous sommes \'evidemment amen\'es à nous poser la question si $\mathbb{R}$ est d\'enombrable ou non. La d\'emonstration est assez simple.
	\begin{dem}
	Par d\'efinition, nous avons vu plus haut qu'il doit y avoir une bijection entre $\mathbb{Q}$ et $\mathbb{R}$ pour dire que $\mathbb{R}$ soit d\'enombrable.
	
	Pour simplifier, nous allons montrer que l'intervalle $[0,1[$ n'est alors pas d\'enombrable. Ceci impliquera bien sûr par extension que $\mathbb{R}$ ne l'est pas!
	
	Les \'el\'ements de cet intervalle sont repr\'esent\'es par des suites infinies entre $0$ et $9$ (dans le système d\'ecimal):
	\begin{itemize}
		\item Certaines de ces suites sont nulles à partir d'un certain rang, d'autres non.
		
		\item Nous pouvons donc identifier $[0,1[$ à l'ensemble de toutes les suites (finies ou infinies) d'entiers compris entre $0$ et $9$.
		\begin{table}[H]\centering
		\begin{tabular}{cccccccc}\hline
		$n^\circ 1 $ & $x_{11}$ & $x_{12}$ & $x_{13}$ & $x_{14}$ & $\cdots$ & $x_{1p}$ & $\cdots$ \\
		$n^\circ 2 $ & $x_{21}$ & $x_{22}$ & $x_{23}$ & $x_{24}$ & $\cdots$ & $x_{2p}$ & $\cdots$ \\
		$n^\circ 3 $ & $x_{31}$ & $x_{32}$ & $x_{33}$ & $x_{34}$ & $\cdots$ & $x_{3p}$ & $\cdots$ \\
		$n^\circ 4 $ & $x_{41}$ & $x_{42}$ & $x_{43}$ & $x_{44}$ & $\cdots$ & $x_{4p}$ & $\cdots$ \\
		$n^\circ 5 $ & $x_{51}$ & $x_{52}$ & $x_{53}$ & $x_{54}$ & $\cdots$ & $x_{5p}$ & $\cdots$ \\
		$n^\circ 6 $ & $x_{61}$ & $x_{62}$ & $x_{63}$ & $x_{64}$ & $\cdots$ & $x_{6p}$ & $\cdots$ \\
		 & $\cdots$ & &  &  &  &  &  \\
		 &  & $\cdots$ &  &  &  &  &  \\
		 &  &  & $\cdots$ &  &  &  &  \\
		$n^\circ k $ & $x_{k1}$ & $x_{k2}$ & $x_{k3}$ & $x_{k4}$ & $\cdots$ & $x_{kp}$ & $\cdots$ \\
		 &  &  &  &  &  & $\cdots$ & \\ \hline
		\end{tabular}
		\end{table}
		Si cet ensemble \'etait d\'enombrable, nous pourrions classer ces suites (avec une première, une deuxième, etc.). Ainsi, la suite $x_{11}x_{12}x_{13}x_{14}...x_{1p}...$ serait class\'ee première et ainsi de suite... comme le propose le tableau ci-dessus.
		
		Nous pourrions alors modifier cette matrice infinie de la manière suivante: à chaque \'el\'ement de la diagonale, rajouter $1$, selon la règle: $0 + 1 = 1, 1 + 1 = 2, 8 + 1 = 9$ et $9 + 1 = 0$:

		\begin{table}[H]\centering
		\begin{tabular}{cccccccc}\hline
		$n^\circ 1 $ & $x_{11}+1$ & $x_{12}$ & $x_{13}$ & $x_{14}$ & $\cdots$ & $x_{1p}$ & $\cdots$ \\
		$n^\circ 2 $ & $x_{21}$ & $x_{22}+1$ & $x_{23}$ & $x_{24}$ & $\cdots$ & $x_{2p}$ & $\cdots$ \\
		$n^\circ 3 $ & $x_{31}$ & $x_{32}$ & $x_{33}+1$ & $x_{34}$ & $\cdots$ & $x_{3p}$ & $\cdots$ \\
		$n^\circ 4 $ & $x_{41}$ & $x_{42}$ & $x_{43}$ & $x_{44}+1$ & $\cdots$ & $x_{4p}$ & $\cdots$ \\
		$n^\circ 5 $ & $x_{51}$ & $x_{52}$ & $x_{53}$ & $x_{54}$ & $\cdots$ & $x_{5p}$ & $\cdots$ \\
		$n^\circ 6 $ & $x_{61}$ & $x_{62}$ & $x_{63}$ & $x_{64}$ & $\cdots$ & $x_{6p}$ & $\cdots$ \\
		 & $\cdots$ & &  &  &  &  &  \\
		 &  & $\cdots$ &  &  &  &  &  \\
		 &  &  & $\cdots$ &  &  &  &  \\
		$n^\circ k $ & $x_{k1}$ & $x_{k2}$ & $x_{k3}$ & $x_{k4}$ & $\cdots$ & $x_{kp}$ & $\cdots$ \\
		 &  &  &  &  &  & $\cdots$ & \\ \hline
		\end{tabular}
		\end{table}
		Alors consid\'erons la suite infinie qui se trouve sur la diagonale:
		\begin{itemize}
			\item Elle ne peut être \'egale à la première car elle s'en distingue au moins par le premier \'el\'ement.
			
			\item Elle ne peut être \'egale à la deuxième car elle s'en distingue au moins par le deuxième \'el\'ement.
			
			\item Elle ne peut être \'egale à la troisième car elle s'en distingue au moins par le troisième \'el\'ement.

et ainsi de suite... Elle ne peut donc être \'egale à aucune des suites contenues dans ce tableau!
		\end{itemize}
		et ainsi de suite... Elle ne peut donc être \'egale à aucune des suites contenues dans ce tableau!
	\end{itemize}
	Donc, quel que soit le classement choisi des suites infinies de $0 \ldots 9$, il y en a toujours une qui \'echappe à ce classement! C'est donc qu'il est impossible de les num\'eroter... tout simplement parce qu'elles ne forment pas un ensemble d\'enombrable!
	\begin{flushright}
		$\blacksquare$  Q.E.D.
	\end{flushright}	
	\end{dem}
	La technique qui nous a permis d'arriver à ce r\'esultat est connue sous le nom de "\NewTerm{proc\'ed\'e diagonal de Cantor}\index{proc\'ed\'e diagonal de Cantor}" (car similaire à celle utilis\'ee pour l'\'equipotence entre ensemble naturel et rationnel) et l'ensemble des nombres r\'eels est dit avoir la "\NewTerm{puissance du continu}\index{puissance du continu}\label{power of the continuum}" de par le fait qu'il est ind\'enombrable.
	
	Nous pouvons supposer qu'il est intuitif pour le lecteur que tout nombre r\'eel puisse être approch\'e à l'infini par un nombre rationnel (pour les nombres irrationnels, nous nous arrêtons simplement à un nombre de d\'ecimales donn\'e et trouvons le rationnel correspondant). Les math\'ematiciens disent donc que $\mathbb{Q}$ est "\NewTerm{dense}\index{ensemble dense}" dans $\mathbb{R}$ et le notent par:
	
	Pour voir cela (dans le style des math\'ematicien ...), prenons un nombre r\'eel $x$. Pour tout $ n \in \mathbb{N}$ on note:
	
	la partie entière de $x$ pr\'ec\'edemment multipli\'ee par $10^n$.
	
	Par d\'efinition de la partie entière d'un nombre r\'eel, on a:
	
	Cette in\'egalit\'e est \'equivalente à:
	
	\'ecrivons pour tout $n \in \mathbb{N}$:
	
	où $a_n$ et $b_n$ sont respectivement les valeurs d\'ecimales approch\'ees à $x \cdot 10^{-n}$ par d\'efaut ou par excès.
	
	Notez que $\forall n\in\mathbb{N}$, nous avons:
	
	et que $\forall n\in\mathbb{N}$ nous avons $\{a_n,b_n\}\in \mathbb{Q}$.
	\begin{tcolorbox}[colframe=black,colback=white,sharp corners]
\textbf{{\Large \ding{45}}Exemple:}\\\\
	Nous voulons calculer $\sqrt{2}$. Alors:
	\begin{table}[H]
	\centering
		\definecolor{gris}{gray}{0.85}
			\begin{tabular}{|c|c|c|c|c|}
				\hline
				\multicolumn{1}{c}{\cellcolor{black!30}\pmb{$n$}} & 
\multicolumn{1}{c}{\cellcolor{black!30}} & \multicolumn{1}{c}{\cellcolor{black!30}\pmb{$a_n$}} & \multicolumn{1}{c}{\cellcolor{black!30}\pmb{$b_n$}} & \multicolumn{1}{c}{\cellcolor{black!30}\textbf{error}}\\ \hline
		1 & $1<\sqrt{2}<2$ & $1$ & $2$ & $1$ \\ \hline
		2 & $1.4<\sqrt{2}<1.5$ & $1.4$ & $1.5$ & $0.1$ \\ \hline
		3 & $1.41<\sqrt{2}<1.42$ & $1.41$ & $1.42$ & $0.01$ \\ \hline
		4 & $1.414<\sqrt{2}<1.415$ & $1.414$ & $1.415$  & $0.001$ \\ \hline
	\end{tabular}
	\end{table}	
	\end{tcolorbox}
	\begin{theorem}
	Si les s\'equences $(a_n)$ et $(b_n)$ sont adjacentes, c'est-à-dire que $(a_n)$ est ascendante et $(b_n)$ descendante, nous avons alors:
	
	\end{theorem}
	
	\begin{dem}
	Donn\'e $n\in \mathbb{N}$, nous avons pour rappel:
	
	où $p_n=[10^nx]$. Multiplions des deux côt\'es par $10$, nous avons alors:
	
	Alors:
	\begin{itemize}
		\item Nous avons de manière triviale:
		
		Il vient alors:
		
		Dès lors $a_n\leq a_{n+1}$. La s\'equence $(a_n)$ est alors bien ascendante.
		
		\item Et nous avons aussi de manière triviale:
		
		Il vient alors:
		
		Dès lors $b_{n+1}\leq b_n$. La s\'equence $(b_n)$  est alors bien descendante.
	\end{itemize}
	Comme:
	
	Nous avons bien:
	
	et alors nous avons prouv\'e que les deux s\'equences $(a_n)$ et $(b_n)$ sont bien adjacentes. Elles convergent toutes deux vers la même limite $L\in\mathbb{R}$. 
	
	Comme $\forall n\in\mathbb{N}$:
	
	Nous avons n\'ecessairement $L=x$.
	\begin{flushright}
		$\blacksquare$  Q.E.D.
	\end{flushright}
	\end{dem}
	
	\begin{corollary}
	Un nombre r\'eel est la limite d'une s\'equence de nombres rationnels.
	\end{corollary}
	\begin{dem}
	Ceci est une cons\'equence imm\'ediate du th\'eorème pr\'ec\'edent. En effet, les s\'equences $(a_n)$ et $(b_n)$ sont des s\'equences de rationnels.
	\begin{flushright}
		$\blacksquare$  Q.E.D.
	\end{flushright}
	\end{dem}
	
	\paragraph{Pourcentages}\mbox{}\\\\
	Dans le monde des affaires\footnote{Pour les taux d'int\'erêts, les rabais, les bonus, etc.} il est commun avec les nombres r\'eels de communiquer en pourcentage ou en pourmille\label{percentage}.
	
	\textbf{D\'efinitions (\#\mydef):}
	\begin{itemize}
		\item[D1.] \'etant donn\'e un scalaire $x \in \mathbb{R}$, alors exprim\'e en "\NewTerm{pourcentage}\index{pourcentage}" il sera not\'e:
			
			
		\item[D2.] \'etant donn\'e un scalaire $x \in \mathbb{R}$, alors exprim\'e en  "\NewTerm{pourmille}\index{pourmille}" il sera not\'e:
			
	\end{itemize}
	Mais nous avons aussi un autre pourcentage bien connu (autre que dans le monde des affaires!) par presque tous ceux qui prennent parfois la route en v\'elo on en voiture...:
	\begin{figure}[H]
		\centering
		\includegraphics[width=1.0\textwidth]{img/arithmetics/percentages.jpg}
		\caption{Illustration commune du concept de pourcentage}
	\end{figure}
	Vous devez être particulièrement prudent lorsque vous manipulez des pourcentages (qui sont des valeurs relatives pour rappel!) et encore plus lorsque vous comparez cela à des valeurs absolues. Une c\'elèbre "fake news" r\'ecente (en juin 2018) faite par un militant socialiste (suppos\'ement diplôm\'e en ing\'enierie ...) en France sur Twitter (un m\'edia social r\'eput\'e pour ses fausses nouvelles) a fait rire beaucoup de gens ... mais en même temps, un nombre consid\'erable de personnes pensaient que le "fait" \'etait vrai \footnote{Les journaux s\'erieux, comme  textit{Le Monde}, ont \'egalement relay\'e les informations comme \'etant manifestement fausses}. Voici la communication (essayez de deviner où se trouve l'erreur, et notez que l'auteur du message dit que le gouvernement français communique des "fake new", mais en r\'ealit\'e ... c'est lui qui communique les "fake news" pour le coup... ):
	\begin{figure}[H]
		\centering
		\includegraphics[scale=0.6]{img/arithmetics/percentage_fake_news.jpg}
	\end{figure}
	Ok, voyons si le bien-être gouvernemental a effectivement diminu\'e en France alors que la population et le produit int\'erieur brut (PIB) ont augment\'e en même temps, comme le pr\'etend l'auteur de ce Tweet...
	
	D'abord nous avons (si nous faisons les calculs correctement):
	
	Deuxièmement:
	
	Finalement:
	
	Vous voyez l'erreur maintenant ...? L’auteur du tweet calcule pour les prestations sociales l’augmentation de ce que nous appelons en finance "\NewTerm{points de base}\index{points de base}" "et non pas en "pourcentages" comme il le fait pour les deux premiers indicateurs. C'est un très bel exemple d'utilisation de statistiques pour mentir ... (le mensonge ne fonctionne \'evidemment qu'avec des illettr\'es scientifiques!)
	
	Inutile de dire que ce "tweet" a \'et\'e supprim\'e quelques jours après par son auteur sur Twitter...
	
	Un autre exemple int\'eressant de pseudo-tromperie avec les pourcentages est le cas de Toyota qui communiquait la performance de ses voitures en mode \'electrique en temps plutôt qu'en distance. Une \'etude ind\'ependante a donn\'e:
	\begin{figure}[H]
		\centering
		\includegraphics[width=1.0\textwidth]{img/arithmetics/pourcentages_temps_distance.jpg}
	\end{figure}
	Enfin, vous devez \'egalement faire extrêmement attention aux communications scientifiques dans les m\'edias de masse (et même davantage sur les r\'eseaux sociaux...) lorsque vous lisez quelque chose comme "\textit{... le produit $X$ augmente le syndrome mortel $Y$ de $400\%$ ...}". Parce que parfois, l'augmentation pourrait être (exemple prix au hasard) en fait de $2$ individus sur $100'000 $ à $8$ individus (toujours sur $100'000$ bien \'evidemment!). Il s'agit là d'une façon de biaiser notre cerveau en associant un pourcentage \'enorme à des quantit\'es de petites amplitudes ... Dans de telles situations, recherchez toujours les valeurs absolues originales et faites y correspondre le ratio de risque (\SeeChapter{voir la section M\'ethodes Num\'eriques page \pageref{rapport de risque}}).
	
	N'oubliez pas non plus qu'en pratique, il est fr\'equent de normaliser une s\'erie de valeurs $x_1, \ldots, x_n $ avec $ x_i \in \mathbb{R} $ (consid\'er\'ees comme un vecteur $\vec{x}$) en pourcentages. , tel que $x_{p, i} \in [0,1] $ en utilisant la transformation \'evidente suivante \footnote{Le lecteur doit \'eviter la confusion entre "normaliser une variable" et "centrer une variable al\'eatoire" pour cr\'eer une variable statistique $ Z = \ mathcal{N}(0,1) $ comme nous le verrons plus loin!}:
	
	Cette transformation de normalisation est particulièrement bien connue dans les tableurs tels que Microsoft Excel lorsque vous utilisez des diagrammes à barres normalis\'es, ou des indicateurs de performance cl\'es normalis\'es (mise en forme conditionnelle), ou \'egalement dans la Data Science pour l'utilisation de certains algorithmes de r\'eseaux neuronaux.
		
	
	\subsubsection{Nombres Transfinis}
	Nous nous retrouvons donc avec un "infini" des nombres r\'eels qui est diff\'erent de celui des nombres naturels. Cantor osa alors ce que personne n'avait (a priori) os\'e depuis Aristote: la suite des entiers positifs est infinie, l'ensemble  $\mathbb{N}$ , est donc un ensemble qui a une infinit\'e d\'enombrable d'\'el\'ements, alors il affirma que le cardinal (\SeeChapter{voir section Th\'eorie des Ensembles page \pageref{cardinal}}) de cet ensemble \'etait un nombre qui existait comme tel sans que l'on utilise le symbole fourre tout $\infty$, il le nota:
	
	Ce symbole est la première lettre de l'alphabet h\'ebreu (\SeeChapter{voir section Th\'eorie des Ensembles \pageref{aleph}}), qui se prononce "aleph z\'ero". Cantor allait appeler ce nombre \'etrange, un "\NewTerm{nombre transfini}\index{nombre transfini}".
	
	L'acte d\'ecisif est d'affirmer qu'il y a, après le fini, un transfini, c'est-à-dire une \'echelle illimit\'ee de modes d\'etermin\'es qui par nature sont infinis, et qui cependant peuvent être pr\'ecis\'es, tout comme le fini, par des nombres d\'etermin\'es, bien d\'efinis et distinguables les uns des autres !!

	Après ce premier coup d'audace allant à l'encontre de la plupart des id\'ees reçues depuis plus de deux mille ans, Cantor allait poursuivre sur sa lanc\'ee et \'etablir des règles de calcul, paradoxales à première vue, sur les nombres transfinis. Ces règles se basaient, comme nous l'avons pr\'ecis\'e tout à l'heure, sur le fait que deux ensembles infinis sont \'equivalents s'il existe une bijection entre les deux ensembles.

	Ainsi, nous pouvons facilement montrer que l'infini des nombres pairs est \'equivalent à l'infini des nombres entiers: pour cela, il suffit de montrer qu'à chaque nombre entier, nous pouvons associer un nombre pair, son double, et inversement.
	
	Ainsi, même si les nombres pairs sont inclus dans l'ensemble des nombres entiers, il y en a une infinit\'e $\alpha_0$ \'egaux, les deux ensembles sont donc \'equipotents. En affirmant qu'un ensemble peut être \'egal à une de ses parties, Cantor va à l'encontre ce qui semblait être une \'evidence pour Aristote et Euclide: l'ensemble de tous les ensembles est infini ! Cela va \'ebranler la totalit\'e des math\'ematiques et va amener à l'axiomatisation de Zermelo-Fraenkel que nous verrons dans le chapitre de Th\'eorie des Ensembles à la page \pageref{zermelo fraenkel axiomatic}.

	A partir de ce qui pr\'ecède, Cantor \'etablit les règles de calculs suivants sur les cardinaux:
	
	À première vue ces règles semblent non intuitives mais en fait elles le sont bien! En effet, Cantor d\'efinit l'addition de deux nombres transfinis comme le cardinal de l'union disjointe des ensembles correspondants.
	
	\begin{tcolorbox}[colframe=black,colback=white,sharp corners]
	\textbf{{\Large \ding{45}}Exemples:}\\\\
	E1. En notant donc $\aleph_0$ le cardinal de $\mathbb{N}$ nous avons $\aleph_0+\aleph_0$ qui est \'equivalent à dire que nous sommons le cardinal de $\mathbb{N}$ union disjointe $\mathbb{N}$. Ou $\mathbb{N}$ union disjointe $\mathbb{N}$ est \'equipotent à $\mathbb{N}$ donc $\aleph_0+\aleph_0=\aleph_0$ (il suffit pour s'en convaincre de prendre l'ensemble des entiers pairs et impairs tout deux d\'enombrables dont l'union disjointe est d\'enombrable).\\
	
	E2.  Autre exemple trivial: $\aleph_0+1$ correspond au cardinal de l'ensemble $\mathbb{N}$ union un point. Ce dernier ensemble est encore \'equipotent à  $\mathbb{N}$ donc $\aleph_0+1=\aleph_0$.
	\end{tcolorbox}
	Nous verrons \'egalement lors de notre \'etude de la section de Th\'eorie des Ensembles (page \pageref{cartesian product}) que le concept de produit cart\'esien de deux ensembles d\'enombrables est tel que nous ayons:
	
	et donc:
	
	De même  (\SeeChapter{voir section Th\'eorie des Ensembles page \pageref{union}}) puisque $\mathbb{Z}=\mathbb{Z}^{+}\cup\mathbb{Z}^{-}$  nous avons:
	
	et en identifiant $\mathbb{Q}$ à $\mathbb{Z}\times\mathbb{Z}$ (rapport d'un num\'erateur sur un d\'enominateur), nous avons imm\'ediatement:
	
	Nous pouvons d'ailleurs d\'emontrer un \'enonc\'e int\'eressant: si nous consid\'erons le cardinal de l'ensemble de tous les cardinaux, il est n\'ecessairement plus grand que tous les cardinaux, y compris lui-même (il vaut mieux avoir lu la section de Th\'eorie Des Ensembles au pr\'ealable à la page \pageref{set theory})! En d'autres termes: le cardinal de l'ensemble de tous les ensembles de $A$ est plus grand que le cardinal de $A$ lui-même.
	
	Ceci implique qu'il n'existe aucun ensemble qui contient tous les ensembles puisqu'il en existe toujours un qui est plus grand (c'est une forme \'equivalente du fameux ancien paradoxe de Cantor).

	Dans un langage technique cela revient à consid\'erer un ensemble non vide $A$ et alors d'\'enoncer que:
		
	où $\mathcal{P}(A)$ est l'ensemble des parties de $A$ (voir le chapitre de Th\'eorie des Ensembles pour le calcul g\'en\'eral du cardinal de l'ensemble des parties d'un ensemble d\'enombrable).

	C'est-à-dire par d\'efinition de la relation d'ordre $<$ (strictement inf\'erieur), qu'il suffit de montrer qu'il n'existe pas d'application surjective $f:A\mapsto \mathcal{P}(A)$, en d'autres termes qu'à chaque \'el\'ement de l'ensemble des parties de A il ne correspond pas au moins une pr\'e-image dans $A$.
	
	\begin{tcolorbox}[title=Remarque,colframe=black,arc=10pt]
	$\mathcal{P}(\mathbb{N})$ est par exemple constitu\'e de l'ensemble des nombres impairs, pairs, premiers, et l'ensemble des naturels, ainsi que l'ensemble vide lui-même, etc. $\mathcal{P}(\mathbb{N})$ est donc l'ensemble de toutes les "patates" (pour emprunter le vocabulaire de la petite \'ecole...) possibles qui forment $\mathbb{N}$.
	\end{tcolorbox}
	\begin{dem}
	L'id\'ee maintenant est de supposer que nous pouvons num\'eroter chacune des patates de $\mathcal{P}(A)$ avec au moins un \'el\'ement de $A$ (imaginez cela avec $\mathbb{N}$ ou allez voir l'exemple dans le chapitre de Th\'eorie des Ensembles). En d'autres termes cela revient à supposer que $f:A\mapsto \mathcal{P}(A)$ est surjective et consid\'erons un sous-ensemble $E$ de $A$ tel que:
	
	c'est-à-dire l'ensemble d'\'el\'ements $x$ de $A$ qui n'appartiennent pas à l'ensemble num\'ero $x$ (l'\'el\'ement $x$ n'appartient pas à la patate qu'il num\'erote... en d'autres termes).
	
	Or, si $f$ est surjective il doit alors exister aussi un $y \in A$ pour ce sous-ensemble $E$ tel que:
	
	puisque $E$ est aussi une partie de $A$.
	
	Suàpposons que $y$ appartienne  $E$. Dans ce cas, par d\'efinition de $E$, $y \notin f(y)=E$ (par d\'efinition de $E$ qui s'applique pour chaque $x$ et $x$ peut bien \'evidemment être aussi $y$ ou $z$ peu importe!). Par cons\'equence, $y\notin E$, mais dans ce seconde cas, toujours par d\'efinition de $E$, $y\in f(y)=E$ (comme $y$ n'est pas dans $E$). Nous voyons alors que l'\'el\'ement $y$ ne peut existe et dès lors que $f$ ne peut être surjective.
	
	Nous recommandons fortement au lecteur au besoin de relire plusieurs fois le paragraphe pr\'ec\'edent si n\'ecessaire...
	\begin{flushright}
		$\blacksquare$  Q.E.D.
	\end{flushright}
	\end{dem}
	
	\pagebreak
	\subsubsection{Nombres Complexes}\label{complex numbers}
	Invent\'es au 16ème siècle entre autres par J\'erôme Cardan et Rafaello Bombelli, les "\NewTerm{nombres complexes}\index{nombres complexes}" (aussi souvent nomm\'es "\NewTerm{nombres imaginaires numbers}\index{nombres imaginaires}") permettent de r\'esoudre des problèmes n'ayant pas de solutions dans $\mathbb{R}$ ainsi que de formaliser math\'ematiquement certaines transformations dans le plan telles que la rotation, la similitude, la translation, etc et aussi de g\'en\'eraliser certains th\'eorèmes restreints a priori à $\mathbb{R}$ et alors en faire \'emerger certaines propri\'et\'es subtiles très utiles en ing\'enierie. Pour les physiciens, les nombres complexes constituent surtout un moyen très commode de simplifier les notations. Il est ainsi très difficile d'\'etudier les ph\'enomènes ondulatoires, la relativit\'e g\'en\'erale ou la m\'ecanique quantique sans recourir aux nombres et expressions complexes.

Il existe plusieurs manières de construire les nombres complexes. La première est typique de la construction telle que les math\'ematiciens en ont l'habitude dans le cadre de la th\'eorie des ensembles. Ils d\'efinissent un couple de nombres r\'eels et d\'efinissent des op\'erations entre ces couples pour arriver enfin à une signification du concept de nombre complexe. La deuxième est moins rigoureuse mais son approche est plus simple et consiste à d\'efinir le nombre imaginaire pur unitaire $\mathrm{i}$ et ensuite de construire les op\'erations arithm\'etiques à partir de sa d\'efinition. Nous allons opter pour cette deuxième m\'ethode.
	
	\textbf{D\'efinition (\#\mydef):}
	\begin{enumerate}
		\item Nous d\'efinissons le "\NewTerm{nombre imaginaire unitaire pur}\index{nombre imaginaire unitaire pur}" que nous notons $\mathrm{i}$ par la propri\'et\'e suivante\label{unit pure imaginary number}:
		
			
		\item Un "\NewTerm{nombre complexes}\index{nombre complexes}" est un couple d'un nombre r\'eel $a$ et d'un nombre imaginaire $\mathrm{i}b$  et s'\'ecrit g\'en\'eralement sous la forme suivante:
		
		$a$ et $b$ \'etant des nombres appartenant à $\mathbb{R}$.
		
		\item Nous notons l'ensemble des nombres complexes  $\mathbb{C}$ et avons donc par construction:
		
	\end{enumerate}
	\begin{tcolorbox}[title=Remarque,colframe=black,arc=10pt]
	L'ensemble $\mathbb{C}$ est identifi\'e au plan euclidien orient\'e $E$ (\SeeChapter{voir section de Calcul Vectoriel page \pageref{oriented Euclidean space}}) grâce au choix d'une base orthonorm\'ee directe (nous obtenons ainsi le \NewTerm{plan d'Argand-Cauchy}\index{plan d'Argand-Cauchy}", appel\'e aussi "\NewTerm{plane de Gauss-Argand plane}\index{plane de  Gauss-Argand}" ou encore plus commun\'ement  "\NewTerm{plan de Gauss}\index{plan de Gauss}" que nous verrons un peu plus loin et qui aurait propos\'e pour la première fois en 1806).
	\end{tcolorbox}
	L'ensemble des nombres complexes qui constitue un corps (\SeeChapter{voir section Th\'eorie des Ensembles page \pageref{field (set)}}), et not\'e $\mathbb{C}$, est d\'efini (de manière simple pour commencer) dans la notation de la th\'eorie des ensembles par:
	
	En d'autres termes nous disons que le corps $\mathbb{C}$ est le corps $\mathbb{R}$  auquel nous avons "adjoint" le nombre imaginaire $\mathrm{i}$. Ce qui se note formellement:
	
	L'addition et la multiplication de nombres complexes sont des op\'erations internes à l'ensemble des complexes (nous reviendrons beaucoup plus en d\'etail sur certaines propri\'et\'es des nombres complexes dans le chapitre traitant de la Th\'eorie des Ensembles page \pageref{groups}) et d\'efinies par:
	
	La "\NewTerm{partie r\'eelle}\index{nombre complexe!partie r\'eelle}" de $z$ est traditionnellement not\'ee:
	
	La "\NewTerm{partie imaginaire}\index{nombre complexe!partie imaginaire}" de $z$ est traditionnellement not\'ee:
	
	Le "\NewTerm{conjugu\'e}\index{nombre complexe!conjugu\'e}\label{complex conjugate}" ou "\NewTerm{nombre complexe!conjugaison}\index{conjugaison}" de $z$ est d\'efini par:
	
	et est aussi parfois not\'e $z^{*}$ (en particulier en physique quantique dans certains ouvrages!).
	
	À partir d'un complexe et de son conjugu\'e, il est possible de trouver ses parties r\'eelles et imaginaires. Ce sont les relations \'evidentes suivantes:
	
	Le "module" de $z$ (ou "norme") repr\'esente la longueur par rapport au centre du plan de Gauss (voir un peu plus bas ce qu'est le plan de Gauss) et est simplement calcul\'e avec l'aide du th\'eorème de Pythagore: 
	
	Le "\NewTerm{module}\index{module}\label{complex module}" de $z$ (ou "norme") repr\'esente la longueur par rapport au centre du plan de Gauss (voir un peu plus bas ce qu'est le plan de Gauss) et est simplement calcul\'e avec l'aide du th\'eorème de Pythagore:
		
	et est donc toujours un nombre positif ou nul.
	
	Nous consid\'ererons comme \'evident que le module satisfait toutes les propri\'et\'es d'une distance (\SeeChapter{voir la section de Topologie \pageref{topology} et de Calcul Vectoriel \pageref{vector calculus}}).
	
	\begin{tcolorbox}[title=Remarque,colframe=black,arc=10pt]
	La notation $|z|$ pour le module n'est pas innocente puisque $|z|$ coïncide avec la valeur absolue de $z$ lorsque $z$ est r\'eel.
	\end{tcolorbox}
	La division entre deux complexes se calcule comme (le d\'enominateur \'etant \'evidemment non nul):
	
	L'inverse d'un complexe se calculant de façon similaire:
	
	Nous pouvons aussi \'enum\'erer $8$ importantes propri\'et\'es du module et du conjugu\'e complexe:
	\begin{itemize}
		\item[P1.] Nous affirmons que:
		
		\begin{dem}
		Par d\'efinition du module $|z|=\sqrt{x^2+y^2}$ , pour que la somme $x^2+y^2$ soit nulle, la condition n\'ecessaire est que comme  $(x,y)\in \mathbb{R}$:
		
		\begin{flushright}
			$\blacksquare$  Q.E.D.
		\end{flushright}
		\end{dem}
		
		\item[P2.] Nous affirmons que:
		
		\begin{dem}
		Ceci est imm\'ediat par:
		
		\begin{flushright}
			$\blacksquare$  Q.E.D.
		\end{flushright}
		\end{dem}
		
		\item[P3.] Nous affirmons que:
		
		\begin{dem}
		Les deux in\'egalit\'es ci-dessus peuvent être \'ecrites:
		
		Dès lors \'equivalentes respectivement à:
		
		qui sont triviales. La suite est alors triviale...!
		\begin{flushright}
			$\blacksquare$  Q.E.D.
		\end{flushright}
		\end{dem}
		
		\item[P4.] Nous avons:
		
		et si $z_2\neq 0$:
		
		\begin{dem}
		Premièrement:
		
		(nous allons prouver un peu plus bas qu'en toute g\'en\'eralit\'e $\overline{z_1z_2}=\bar{z}_1\bar{z}_2$) et pour $z_2\neq 0$:
		
		et en prenant la racine carr\'ee, cela termine la d\'emonstration.
		\begin{flushright}
			$\blacksquare$  Q.E.D.
		\end{flushright}
		\end{dem}
		
		\item[P5.] Nous affirmons que:
		
		\begin{dem}
		C'est imm\'ediate:
		
		\begin{flushright}
			$\blacksquare$  Q.E.D.
		\end{flushright}
		\end{dem}
		
		\item[P6.] Nous affirmons que:
		
		\begin{dem}
		La première \'egalit\'e est imm\'ediate:
		
		et:
		
		et\label{module product complex numbers}:
		
		\begin{flushright}
			$\blacksquare$  Q.E.D.
		\end{flushright}
		\end{dem}
		\begin{tcolorbox}[title=Remarque,colframe=black,arc=10pt]
		\textbf{R1.} En des termes math\'ematiques, la première d\'emonstration permet de montrer que la conjugaison complexe est ce que l'on appelle "\NewTerm{involutive}\index{involutive}"  (dans le sens qu'elle ne fait rien \'evoluer...).\\
		
		\textbf{R2.} En des termes tout aussi math\'ematiques (ce n'est que du vocabulaire!), la deuxième d\'emonstration montre que la conjugaison de la somme de deux nombres complexes est ce que nous appelons un "automorphisme du groupe" $(\mathbb{C},+)$ (\SeeChapter{voir section Th\'eorie des Ensembles page \pageref{automorphism}}).\\
		
		 \textbf{R3.} Encore une fois, pour le vocabulaire..., la troisième d\'emonstration montre que la conjugaison du produit de deux nombres complexes est ce que nous appelons un "automorphisme du corps"  $(\mathbb{C},+,\times)$ (\SeeChapter{voir section Set Theory page \pageref{automorphism}}).
		\end{tcolorbox}
		
		\item[P7.] Nous affirmons que pour $z$ diff\'erent de z\'ero:
		
		\begin{dem}
		\label{module ration complex numbers}Nous nous restreindrons à la d\'emonstration de la seconde relation qui est un cas g\'en\'eral de la première (pour $z=1$).
		
		\begin{flushright}
			$\blacksquare$  Q.E.D.
		\end{flushright}
		\end{dem}
		
		\item[P8.] Nous avons:
		
		pour tous complexes $z_1,z_2$ (rigoureusement non nuls car sinon le concept d'argument du nombre complexe que nous verrons plus loin est alors ind\'etermin\'e!). De plus l'\'egalit\'e a lieu si et seulement si $z_1$ et $z_2$ sont colin\'eaires (les vecteurs sont "sur la même droite") et de même sens, autrement dit .... s'il existe  $\lambda \in \mathbb{R}$ tel que  $\lambda z_1=z_2$.
		\begin{dem}
		Directement, nous avons:
		
	Cette in\'egalit\'e peut ne pas paraître \'evidente à tout le monde alors d\'eveloppons un peu et supposons-la vraie:
		
		Après simplification:
		
		et encore après simplification:
		
		Donc comme la parenthèse au carr\'e est forc\'ement positive ou nulle il s'ensuit:
		
		cette dernière relation d\'emontre donc que l'in\'egalit\'e est vraie.
		\begin{flushright}
			$\blacksquare$  Q.E.D.
		\end{flushright}
		\end{dem}
		\begin{tcolorbox}[title=Remarque,colframe=black,arc=10pt]
		 Il existe une forme plus g\'en\'erale de cette in\'egalit\'e appel\'ee "\NewTerm{in\'egalit\'e de Minkowski}\index{in\'egalit\'e de Minkowski}" pr\'esent\'ee dans le chapitre de Calcul Vectoriel  à la page3  page \pageref{triangle inequality} (les nombres complexes peuvent effectivement s'\'ecrire sous la forme de vecteurs comme nous allons le voir de suite).
		\end{tcolorbox}
	\end{itemize}
	
	
	\paragraph{Interpr\'etation g\'eom\'etrique des nombres complexes}\mbox{}\\\\
	Nous pouvons aussi repr\'esenter un nombre complexe $a+\mathrm{i}b$ ou  $a-\mathrm{i}b$ dans un plan d\'elimit\'e par deux axes (deux dimensions) de longueur infinie et orthogonaux entres eux. L'axe vertical repr\'esentant la partie imaginaire d'un nombre complexe et l'axe horizontal la partie r\'eelle (voir figure ci-après).
	
	Il y a donc bijection entre l'ensemble des nombres complexes $\mathbb{C}$ et l'ensemble des vecteurs du plan de Gauss $\mathbb{R}^2$ (notion d'affixe).
	
	Nous nommons parfois ce type de repr\'esentation "\NewTerm{plan de Gauss}\index{plan de Gauss}\index{gauss plane}" et nous nommons "\NewTerm{affixe}\index{affixe}" le point de coordonn\'ees cart\'esiennes $(a,b)$ qui est identifi\'e avec le nombre complexe $a+\mathrm{i}b$ lui-même.
	
	\begin{figure}[H]
		\centering
		\includegraphics{img/arithmetics/gauss_plane.jpg}
		\caption{Plan de Gauss}
	\end{figure}
	et alors nous \'ecrivons:
	
	Nous voyons sur ce diagramme qu'un nombre complexe a donc une interpr\'etation vectorielle (\SeeChapter{voir section de Calcul Vectoriel page \pageref{vector}}) donn\'ee par:
	
	où la base canonique est d\'efinie telle que:
	
	avec:
	
	Ainsi, $\begin{pmatrix}1\\0\end{pmatrix}$ est le vecteur de la base unitaire port\'e par l'axe horizontal $\mathbb{R}$ et $\begin{pmatrix}0\\1\end{pmatrix}$ est le vecteur de la base unitaire port\'e par l'axe imaginaire $\mathbb{R}_{\text{i}}$ et $r$ est le module (la norme) positif ou nul.
	
	Ceci est à comparer avec les vecteurs de $\mathbb{R}^2$ (\SeeChapter{voir section de Calcul Vectoriel page \pageref{canonical basis}}):
	
	avec:
	
	ce qui fait que nous pouvons identifier le plan complexe avec le plan euclidien.
	
	À l'aide de l'interpr\'etation g\'eom\'etrique du plan de Gauss, l'\'egalit\'e ci-dessous est par exemple imm\'ediate et \'evite de faire quelques d\'eveloppements:
	
	Par ailleurs, les d\'efinitions du cosinus et sinus (\SeeChapter{voir section de Trigonometrie page \pageref{cyclometrics functions}}) nous donnent:
	
	Finalement:
	
	Ainsi:
	
	complexe qui est toujours \'egal à lui-même modulo $2\pi$ de par les propri\'et\'es des fonctions trigonom\'etriques:
	
	avec $k\in \mathbb{N}$ et où $\varphi$ est appel\'e "\NewTerm{argument de $z$}\index{argument}" et est not\'e traditionnellement:
	
	Les propri\'et\'es du cosinus et du sinus (\SeeChapter{voir section Trigonometrie page \pageref{remarkable angles}}) nous amènent directement à \'ecrire pour l'argument:
	
	Nous d\'emontrons entre autres avec les s\'eries de Taylor (\SeeChapter{voir section Suites et S\'eries page \pageref{cosine maclaurin dev}}) que:
	
	et:
	
	dont la somme est semblable à\label{taylor expansion complex exponential}:
	
	mais par contre parfaitement identique au d\'eveloppement de Taylor de $e^{\mathrm{i}x}$:
	
	Donc finalement, nous pouvons \'ecrire:
	
	relation nomm\'ee  "\NewTerm{formule d'Euler}\index{formule d'Euler}\label{euler formula}".
	\begin{tcolorbox}[title=Remarque,colframe=black,arc=10pt]
	Le lecteur noter que:
	
	où $\delta$ est nomm\'e le "\NewTerm{d\'ecalage de phase}\index{d\'ecalage de phase}\label{phase shift}" et $e^{\mathrm{i}\delta}$ est nomm\'e le "\NewTerm{facteur de d\'ecalage de phase}\index{facteur de d\'ecalage de phase}":
	\begin{figure}[H]
	\centering
		\begin{tikzpicture}
	  	\begin{axis}[
	    trig format plots=rad,
	    axis lines = middle,
	    enlargelimits,
	    clip=false
	    ]
	    \addplot[domain=-2*pi:2*pi,samples=200,blue] {sin(x)};
	    \addplot[domain=-2*pi:2*pi,samples=200,red] {sin(x-2)};
	    \draw[dotted,blue!40] (axis cs: 0.5*pi,1.1) -- (axis cs: 0.5*pi,0);
	    \draw[dotted,red!40] (axis cs: 0.5*pi+2,1.1) -- (axis cs: 0.5*pi+2,0);
	    \draw[dashed,olive,<->] (axis cs: 0.5*pi,1.05) -- node[above,text=black,font=\footnotesize]{$\delta$}(axis cs: 0.5*pi+2,1.05);
	  	\end{axis}
		\end{tikzpicture}
	\end{figure}
	\end{tcolorbox}
	En utilisant les propri\'et\'es des fonctions trigonom\'etriques:
	
	Suivant que nous sommons ou soustrayons cela nous donne les "\NewTerm{formules d'Euler}" ou "\NewTerm{formules de Moivre et Euler}\index{formules de Moivre et Euler}\label{de Moivre and Euler formulas}":
	
	Remarquons que l'angle peut être un nombre purement complexe et dans ce cas les deux formules d'Euler donnent un nombre r\'eel. Si l'angle est un nombre complexe avec une partie r\'eelle plus imaginaire alors dans ce cas les fonctions trigonom\'etriques redonnent un nombre complexe en sortie. Ceci pour dire qu'en toute g\'en\'eralit\'e les fonctions trigonom\'etriques peuvent être consid\'er\'ees commes des fonctions qui vont de $\mathbb{C}$ dans $\mathbb{C}$.
	\begin{tcolorbox}[title=Remarque,colframe=black,arc=10pt]
	Notez que ces relations sont très utiles pour lin\'eariser des expressions telles que $\cos^k(\theta)$ ou $\sin^k(\theta)$. Effectivement:
	
	\end{tcolorbox}
	Un cas particulier bien connu pour les formules ci-dessus est le cas où $\varphi=\pi$ et $r=1$. Nous avons alors:
	
	après r\'earrangement, nous obtenons la fameuse relation que de nombreux "geek" connaissent:
	
	Un autre cas c\'elèbre est celui où nous prenons encore $r=1 $ mais avec $\varphi=\pi/2$ et que nous \'elevons à la puissance de $\mathrm{i}$. Nous obtenons alors:
	
	et si nous \'elevons cela à la puissance de $\mathrm{i}$:
	
	Certaines personnes considèrent alors comme un fait curieux que $\mathrm{i}$  \'elev\'e à la puissance $\mathrm{i}$ donne un nombre r\'eel.
	
	Grâce à la forme exponentielle d'un nombre complexe, très fr\'equemment utilis\'ee dans de nombreux domaines de la physique et de l'ing\'enierie, nous pouvons très facilement tirer des relations telles que ($\text{cis}$ est une vieille notation qui est l'abr\'eviation du $\cos(\varphi)+\mathrm{i}\sin(\varphi)$ se trouvant dans la parenthèse):
	
	et en supposant connues les relations trigonom\'etriques de bases (\SeeChapter{voir section de Trigonom\'etrie page \pageref{remarkable trigonometric identities}}) nous avons les relations suivantes pour la multiplication de deux nombres complexes:
	
	dès lors:
	
	et donc si $n$ est un entier positif:
	
	Pour le module de la multiplication (nous changeons de notation pour la lisibilit\'e): 
	
	d'où:
	
	Pour la division de deux nombres complexes:
	
	Le module de leur division vient alors imm\'ediatement:
	
	dès lors nous avons pour l'argument:
	
	ainsi il vient imm\'ediatement:
	
	Pour la mise en puissance d'un nombre complexe (ou la racine):
	
	ce qui nous donne imm\'ediatement un r\'esultat d\'ejà mentionn\'e plus haut: 
	
	et pour l'argument:
	
	Dans le cas où nous avons un module unit\'e (norme \'egale à $1$)  tel que $z=\cos(\varphi)+\mathrm{i}\sin(\varphi)$ nous avons alors la relation:
	
	appel\'ee "\NewTerm{formule de Moivre}\index{formule de Moivre}".

	Pour le logarithme n\'ep\'erien d'un nombre complexe, nous avons trivialement la relation suivante sur laquelle nous reviendrons dans le chapitre d'Analyse Complexe (page \pageref{logarithms}):
	
	où $\ln(z)$ est souvent dans le cas complexe \'ecrit $\mathrm{Log}(z)$ avec un "L" majuscule.
	 
	Toutes les relations pr\'ec\'edentes pourraient bien sûr être obtenues avec la forme trigonom\'etrique des nombres complexes mais n\'ecessiteraient alors quelques lignes suppl\'ementaires de d\'eveloppements.
	
	\begin{tcolorbox}[title=Remarques,colframe=black,arc=10pt]
	Notez que les d\'eveloppements pr\'ec\'edents nous autorisent à calculer des choses (..) du genre $\cos(\bar{z})$. Effectivement, posons $z=x+\mathrm{i}y$ et consid\'erons:
	
	Similairement, nous avons:
	
	Maintenant consid\'erons:
	
	\end{tcolorbox}
	
	\subparagraph{Vecteurs de Fresnel (phaseurs)}\mbox{}\\\\
	Une variation sinusoïdale $f(t)=r\sin(\omega t)$ peut être repr\'esent\'ee comme la projection (\SeeChapter{voir section de Trigonom\'etrie page \pageref{trigonometry}}) sur l'axe vertical $y$ (axe des imaginaires de l'ensemble $\mathbb{C}$) d'un vecteur $\vec{r}$ tournant à vitesse angulaire $\omega$ autour de l'origine dans le plan $x\text{O}y$:
	\begin{figure}[H]
		\centering
		\includegraphics{img/arithmetics/fresnel_representation.jpg}
		\caption{Repr\'esentation de Fresnel}
	\end{figure}
	Un tel vecteur tournant s'appelle "\NewTerm{vecteur de Fresnel}\index{vecteur de Fresnel}" et peut très bien être interpr\'et\'e comme la partie imaginaire d'un nombre complexe donn\'e par:
	
	C'est-à-dire:
	\begin{figure}[H]
		\centering
		\includegraphics{img/arithmetics/fresnel_rotating_vector.jpg}
		\caption{Vecteur tournant de Fresnel}
	\end{figure}
	Nous retrouverons les vecteurs tournants de façon explicite lors de notre \'etude de la m\'ecanique ondulatoire et optique g\'eom\'etrique (dans le cadre de la diffraction) dans les sections correspondantes aux pages respectives \pageref{wave mechanics} et \pageref{geometrical optics}.
	
	\paragraph{Transformations dans le plan}\mbox{}\\\\
	Il est habituel de repr\'esenter les nombres r\'eels comme points d'une droite gradu\'ee. Les op\'erations alg\'ebriques y ont leur interpr\'etation g\'eom\'etrique: l'addition est une translation, la multiplication une homoth\'etie centr\'ee à l'origine.
	
	En particulier nous pouvons parler de la "racine carr\'ee d'une transformation". Une translation d'amplitude $T$ peut être obtenue comme l'it\'eration d'une translation d'amplitude  $T / 2$. De même une homoth\'etie de rapport $S$ peut être obtenue comme l'it\'er\'ee d'une homoth\'etie de rapport $\sqrt{S}$. En particulier une homoth\'etie de rapport $9$ est la compos\'ee de deux homoth\'eties de rapport $3$ (ou $-3$).
	
	La racine carr\'ee prend alors un sens g\'eom\'etrique. Mais qu'en est-il de la racine carr\'ee de nombres n\'egatifs?  En particulier de la racine carr\'ee de $-1$?
	
	Une homoth\'etie de rapport $-1$ peut être vue comme une sym\'etrie par rapport à l'origine. Toutefois si nous voulons voir cette transformation d'une manière continue, force nous est de placer la droite dans un plan. Dès lors une homoth\'etie de rapport $-1$ peut être vue comme une rotation de $\pi$ radians autour de l'origine.
	
	Du coup, le problème de la racine carr\'ee n\'egative se simplifie. En effet, il n'est guère difficile de d\'ecomposer une rotation de $\pi$ radians en deux transformations: nous pouvons r\'ep\'eter soit une rotation de $\pi/2$ soit une rotation de $-\pi/2$. L'image de $1$ sera la racine carr\'ee de $-1$ et $\mathrm{i}$ est situ\'ee sur une perpendiculaire à l'origine à une distance $1$ soit vers le haut soit vers le bas.
	
	Ayant r\'eussi à positionner le nombre  $\mathrm{i}$ il n'est plus guère difficile de disposer les autres nombres complexes dans un plan de Gauss. Nous pouvons ainsi associer à $2\mathrm{i}$ le produit de l'homoth\'etie (\SeeChapter{voir section de G\'eom\'etrie Euclidenne page \pageref{scaling}}) de rapport $2$ par la rotation de centre O et d'angle $\pi/2$, soit une similitude centr\'ee à l'origine. C'est ce que nous allons nous efforcer à montrer maintenant.
	
	Soient:
	
	Nous avons les propri\'et\'es de transformations g\'eom\'etriques suivantes pour les nombres complexes (voir le chapitre de Trigonom\'etrie pour les propri\'et\'es du sinus et cosinus) que nous pouvons joyeusement combiner selon notre bon vouloir:
	\begin{enumerate}
		\item[P1.] La multiplication de $z_1$ par un r\'eel $\lambda$ dans le plan de Gauss correspond (trivial) à une homoth\'etie (agrandissement) de centre O (l'intersection des axes imaginaires et r\'eels), de rapport $\lambda$.
		
		Effectivement:
		
		
		\item[P2.] La multiplication de $z_1$  par un nombre complexe de module unitaire correspond à une rotation de centre 0 et à un angle correspondant à l'argument de $z_1$. Effectivement:
		
		\begin{tcolorbox}[title=Remarque,colframe=black,arc=10pt]
		Nous voyons alors imm\'ediatement, par exemple, que multiplier un nombre complexe par $\mathrm{i}$ (c'est-à-dire $ \sin(\omega)=1,\cos(\omega)=0$) correspond à une rotation de $\pi/2$.
		\end{tcolorbox}
		\begin{theorem}
		Il est int\'eressant d'observer que sous forme vectorielle la rotation de centre O de $z_1$ par  $z_0$ peut s'\'ecrire à l'aide de la matrice suivante:
		
		\end{theorem}
		\begin{dem}
		Nous savons que $z_0z_1$ est une rotation de centre O et d'angle $\omega$. Il suffit de l'\'ecrire à l'ancienne:
		
		ce qui donne sous forme vectorielle:
		
		donc l'application lin\'eaire est \'equivalente à:
		
		ou encore (nous retombons sur la matrice de rotation dans le plan que nous avons dans le chapitre de G\'eom\'etrie Euclidienne page \pageref{rotation matrix in the plane} ce qui est un r\'esultat remarquable!) en utilisant:
		
		dans le cas particulier et arbitraire où $r$ serait unitaire (afin d'avoir une rotation pure!):
		
		nous avons imm\'ediatement (nous avons repris les notations de l'angle tel que nous l'avons dans le chapitre de G\'eom\'etrie):
		
		Remarquons que la matrice de rotation 2D peut aussi s'\'ecrire sous la formes\label{2d rotation matrix}:
		
		de même:
		
		\begin{flushright}
			$\blacksquare$  Q.E.D.
		\end{flushright}
		\end{dem}
		Ainsi nous remarquons que ces matrices de rotation ne sont pas que des applications mais sont des nombres complexes aussi (bon c'\'etait \'evident dès le d\'ebut mais fallait le montrer de manière esth\'etique et simple).
		
		Ainsi, nous avons pour habitude de poser que:
		
		ou avec une autre notation fr\'equente en alègbre lin\'eaire:
		
		Le corps des nombres complexes est donc isomorphe au corps des matrices r\'eelles carr\'ees de dimension $2$ du type:
		
		C'est un r\'esultat que nous r\'eutiliserons de nombreuses fois dans divers chapitres de ce site pour des \'etudes particulières en algèbre, g\'eom\'etrie et en physique quantique relativiste.
		
		\item[P3.] La multiplication de deux complexes correspond à une homoth\'etie ajout\'ee à une rotation. En d'autres termes, d'une "\NewTerm{similitude directe}\index{similitude directe}".
		\begin{dem}
		
		il s'agit donc bien d'une similitude de rapport $b$ et d'angle  $\beta$.
		
		Au contraire, l'op\'eration suivante:
		
		sera appel\'ee une "\NewTerm{similitude lin\'eaire r\'etrograde}\index{similitude lin\'eaire r\'etrograde}".
		
		Par ailleurs, il en retourne trivialement la relation d\'ejà connue suivante:
		
		\begin{tcolorbox}[title=Remarques,colframe=black,arc=10pt]
		\textbf{R1.} La somme de deux nombres $z_1+z_2$ complexes ne pouvant avoir une \'ecriture math\'ematique simplifi\'ee sous quelque forme que ce soit, nous disons alors que la somme \'equivaut à une "\NewTerm{translation d'amplitude}\index{translation d'amplitude}".\\
		
		\textbf{R2.}  La combinaison d'une similitude lin\'eaire (multiplication de deux nombres complexes) directe et d'une translation d'amplitude (sommation par un troisième nombre complexe) correspond à ce que nous appelons une "\NewTerm{similitude lin\'eaire directe}\index{similitude lin\'eaire directe}".
		\end{tcolorbox}
		\begin{flushright}
			$\blacksquare$  Q.E.D.
		\end{flushright}
		\end{dem}
		
		\item[P4.] Le conjugu\'e d'un nombre complexe est g\'eom\'etriquement son sym\'etrique par rapport à l'axe r\'e\'el tel que:
		
		sans oublier que (trigonom\'etrie de base):
		
		Ce qui nous donne un r\'esultat d\'ejà connu:
		
		D'où nous pouvons tirer la propri\'et\'e suivante:
		
		d'où:
		
		
		\item[P5.] La n\'egation du conjugu\'e d'un nombre complexe est g\'eom\'etriquement son sym\'etrique par rapport à l'axe des imaginaires tel que:
		
		\begin{tcolorbox}[title=Remarques,colframe=black,arc=10pt]
		\textbf{R1.} La combinaison des propri\'et\'es P4, P5 est appel\'ee une "\NewTerm{similitude r\'etrograde}\index{similitude r\'etrograde}".\\
		
		\textbf{R2.} L'op\'eration g\'eom\'etrique qui consiste à prendre l'inverse du conjugu\'e d'un nombre complexe (soit  $\bar{z}^{-1}$) est appel\'ee une "\NewTerm{inversion de pôle}\index{inversion de pôle}".
		\end{tcolorbox}
		
		\item[P6.] La rotation de centre $c$ et d'angle $\varphi$ est donn\'ee par:
		
		Explications:
		
		Le complexe $c$ donne un point dans le plan de Gauss qui sera le centre de rotation. La diff\'erence $z_1-c$ donne le rayon $r$ choisi. La multiplication par $e^{\mathrm{i}\varphi}$ est la rotation du rayon par rapport à l'origine du plan de Gauss dans le sens inverse des aiguilles d'une montre. Finalement, l'addition par $c$ est la translation n\'ecessaire pour ramener le rayon $r$ tourn\'e à l'origine du centre $c$. Ce qui donne sch\'ematiquement:
		\begin{figure}[H]
			\centering
			\includegraphics{img/arithmetics/complex_rotation.jpg}
			\caption{Repr\'esentation de la rotation complexe}
		\end{figure}
	
		\item[P7.] Sur la même id\'ee, nous obtenons une homoth\'etie de centre $c$, de rapport $\lambda$ par l'op\'eration:
		
		Explications:
		
		La diff\'erence $z_1-c$ donne toujours le rayon $r$ et $c$ un point dans le centre de Gauss. L'expression $\lambda(z_1-c)$ donne l'homoth\'etie du rayon par rapport à l'origine du plan de Gauss et finalement l'addition par $c$ la translation n\'ecessaire pour que l'homoth\'etie soit vue comme \'etant faite de centre $c$.
	\end{enumerate}
	
	\subsubsection{Nombres Quaternions}\label{quaternions}
	Appel\'es aussi "\NewTerm{hypercomplex}\index{hypercomplex}", les nombres quaternions ont \'et\'e invent\'es en 1843 par William Rowan Hamilton pour g\'en\'eraliser les nombres complexes.
	
	\textbf{D\'efinition (\#\mydef):} Un "\NewTerm{quaternion}\index{quaternion}" est un \'el\'ement $(a,b,c,d)\in \mathbb{R}^4$ et dont nous notons $\mathbb{H}$ l'ensemble qui le contient et que nous appelons "\NewTerm{ensemble des quaternions}\index{ensemble des quaternions}".
	
	Un quaternion peut aussi bien être repr\'esent\'e en ligne ou en colonne tel que:
	
	Nous d\'efinissons la somme de deux quaternions $(a, b, c, d)$ et $(a ', b', c ', d')$ par:
	
	Il est \'evident (du moins nous l'esp\'erons pour le lecteur) que $(\mathbb{H},+)$ est un groupe commutatif (\SeeChapter{voir section Th\'eorie des Ensembles page \pageref{commutative field}}), d'\'el\'ement neutre $(0,0,0,0)$, l'oppos\'e d'un \'el\'ement $(a,b,c,d)$ \'etant $(-a,-b,-c,-d)$. 
	
	L'associativit\'e se v\'erifie en appliquant les propri\'et\'es correspondantes des op\'erations sur $\mathbb{R}$.
	
	Nous d\'efinissons \'egalement la multiplication:
	
	de deux quaternions $(a, b, c, d)$ et $(a', b', c', d')$ par l'expression:
	
	C'est peut-être difficile à accepter mais nous verrons un peu plus loin qu'il y a un air de famille avec les nombres complexes.

	Nous pouvons remarquer que la loi de multiplication n'est pas commutative. Effectivement, en prenant la d\'efinition de la multiplication ci-dessus, nous avons:
	
	Mais nous pouvons aussi remarquer que:
	
	
	\begin{tcolorbox}[title=Remarque,colframe=black,arc=10pt]
	C'est l'addition naturelle dans $\mathbb{R}^4$ vu comme un $\mathbb{R}$-espace vectoriel ((\SeeChapter{voir section Th\'eorie des Ensembles page \pageref{vector space}}).
	\end{tcolorbox}
	
	La loi de multiplication est distributive avec la loi d'addition mais c'est un excellent exemple où il faut quand même prendre garde à d\'emontrer la distributivit\'e à gauche et à droite, puisque le produit n'est pas commutatif !
	
	La multiplication a pour \'el\'ement neutre:
	
	Effectivement:
	
	Tout \'el\'ement:
	
	est inversible.
	
	En effet, si $(a,b,c,d)$ est un quaternion non nul, nous avons alors n\'ecessairement:
	
	sinon les quatre nombres $a, b, c, d$ sont de carr\'e nul, donc tous nuls. Soit alors le quaternion $(a_1,b_1,c_1,d_1)$ d\'efini par:
	
	alors en appliquant machinalement la d\'efinition de la multiplication des quaternions, nous v\'erifions que:
	
	ce dernier quaternion est donc l'inverse pour la multiplication!
	
	Montrons maintenant (pour la culture g\'en\'erale) que le corps des complexes equation est un sous-corps de $(\mathbb{H},+,\times)$.
	\begin{tcolorbox}[title=Remarque,colframe=black,arc=10pt]
	Nous aurions pu mettre cette d\'emonstration dans le chapitre de Th\'eorie Des Ensembles car nous faisons usage de beaucoup de concepts qui y sont vus mais il nous a sembl\'e un peu plus pertinent de la mettre ici.
	\end{tcolorbox}
	Soit $\mathbb{H}'$ l'ensemble des quaternions de la forme $(a, b, 0,0)$. Si $\mathbb{H}'$ est non vide, et si $(a, b, 0,0)$, $(a ', b', 0.0)$ sont des \'el\'ements de $\mathbb{H}'$ alors $(\mathbb{H}',+\times)$ est un corps. Effectivement:
	\begin{enumerate}
		\item[P1.] Pour la soustraction (et donc l'addition):
		

		\item[P2.] La multiplication:
			

		\item[P3.] L'\'el\'ement neutre:
		

		\item[P4.] Et finalement l'inverse:
		
		de $(a,b,0,0)$ est encore dans $\mathbb{H}'$.
	\end{enumerate}
	Donc $(\mathbb{H}',+,\times)$ est un sous-corps de $\mathbb{H}$. Soit alors l'application:
	
	$f$ est bijective, et nous v\'erifions ais\'ement que pour tous complexes $z_1,z_2$, nous avons:
	
	Donc $f$ est un isomorphisme de $(\mathbb{C},+,\times)$ sur $(\mathbb{H}',+,\times)$.
	
	Cet isomorphisme a pour int\'erêt (provoqu\'e) d'identifier $\mathbb{C}$ à $\mathbb{H}'$ et d'\'ecrire $\mathbb{C} \subset\mathbb{H}$, les lois d'addition et de soustraction sur $\mathbb{H}$ prolongeant les op\'erations d\'ejà connues sur $\mathbb{C}$.
	
	Ainsi, par convention, nous \'ecrirons tout \'el\'ement de $(a, b, 0,0)$ de $\mathbb{H}'$ sous la forme complexe $a + \mathrm{i}b$. En particulier $0$ est l'\'el\'ement $(0,0,0,0)$, $1$ est l'\'el\'ement $(1,0,0,0)$ et $\mathrm{i}$ est l'\'el\'ement $(0,1,0,0)$.
	
	Nous notons par analogie et par extension $\mathrm{j}$ l'\'el\'ement  $(0,0,1,0)$ et $\mathrm{k}$ l'\'el\'ement $(0,0,0,1)$. La famille $\{1, \mathrm{i}, \mathrm{j}, \mathrm{k}\}$ forme une base de l'ensemble des quaternions vu comme un espace vectoriel sur $\mathbb{R}$, et nous \'ecrirons:
	
	le quaternion $(a, b, c, d)$.

	La notation des quaternions sous forme d\'efinie ci-avant est parfaitement adapt\'ee à l'op\'eration de multiplication. Pour le produit de deux quaternions nous obtenons en d\'eveloppant l'expression:
	
	$16$ termes que nous devons identifier à la d\'efinition d'origine de la multiplication des quaternions pour obtenir les relations suivantes:
	
	Ce qui peut se r\'esumer dans un tableau:
	

	Ou dans une repr\'esentation 3D de ce qu'il se passe dans la sphère 4D correspondante (rouge, vert et bleu pour $\mathrm{i}$, $\mathrm{j}$ et $\mathrm{k}$ respectivement):
	\begin{figure}[H]
		\centering
		\includegraphics{img/arithmetics/quaternions_3D_4D_sphere_representation.jpg}
		\caption[]{Situation de d\'epart pour rotation des quaternions}
	\end{figure}
	Nous pouvons constater que l'expression de la multiplication de deux quaternions ressemble en partie beaucoup à un produit vectoriel (not\'e $\circ$ dans ce livre) et scalaire (not\'e equation sur ce site):
	
	Si ce n'est pas \'evident (ce qui serait tout à fait compr\'ehensible), faisons un exemple concret:
	\begin{tcolorbox}[colframe=black,colback=white,sharp corners]
	\textbf{{\Large \ding{45}}Exemple:}\\\\
	Soient deux quaternions sans partie r\'eelle:
	
	et $\vec{u},\vec{v}$ les vecteurs de $\mathbb{R}^3$  de coordonn\'ees respectives $(x,y,z)$ et $(x',y',z')$. Alors le produit:
	
	est \'egal à:
	
	Nous pouvons aussi par curiosit\'e nous int\'eresser au cas g\'en\'eral... Soient pour cela deux quaternions:
	
	Nous avons alors:
	
	\end{tcolorbox}
	\textbf{D\'efinition (\#\mydef):} Le centre du corps non-commutatif $(\mathbb{H},+,\times)$ est l'ensemble des \'el\'ements de $\mathbb{H}$ commutant pour la loi de multiplication avec tous les \'el\'ements de $\mathbb{H}$.

	\begin{theorem}
	Le centre de $(\mathbb{H},+,\times)$ est l'ensemble des r\'eels!
	\end{theorem}
	\begin{dem}
	Soit $\mathbb{H}_1$ le centre de $(\mathbb{H},+,\times)$, et $(x, y, z, t)$ un quaternion. Nous devons avoir les conditions suivantes qui soient satisfaites:

	Soit  $(x,y,z,t)\in \mathbb{H}_1 $  alors pour tout $(a,b,c,d)\in \mathbb{H}$ nous cherchons:
	
	ce qui donne en d\'eveloppant:
	
	après simplification (la première ligne du système pr\'ec\'edent est nulle des deux côt\'es de l'\'egalit\'e):
	
	la r\'esolution de ce système, nous donne:
	
	Donc pour que le quaternion $(x, y, z, t)$ soit le centre de $\mathbb{H}$ il doit être r\'eel (sans parties imaginaires)!
	\begin{flushright}
		$\blacksquare$  Q.E.D.
	\end{flushright}
	\end{dem}
	Au même titre que pour les nombres complexes, nous pouvons d\'efinir un conjugu\'e des quaternions:
	
	\textbf{D\'efinition (\#\mydef):} Le conjugu\'e d'un quaternion $Z=(a,b,c,d)$ est le quaternion $\bar{Z}=(a,-b,-c,-d)$.

	Au même titre que pour les complexes, nous remarquons que:
	\begin{enumerate}
		\item D'abord de manière \'evidente que si $Z=\bar{Z}$ alors cela signifie que $Z\in \mathbb{R}$.

		\item Que $Z+\bar{Z}\in \mathbb{R}$

		\item Qu'en d\'eveloppant le produit $Z\bar{Z}$ nous avons:
		
		que nous adopterons, par analogie avec les nombres complexes, comme une d\'efinition de la norme (ou module) des quaternions tel que:
		
		Dès lors nous avons aussi imm\'ediatement (relation qui nous sera utile plus tard):
		
	\end{enumerate}
	Comme pour les nombres complexes (voir plus loin), il est ais\'e de montrer que la conjugaison est un automorphisme du groupe $(\mathbb{H},+)$.
	
	Effectivement, soient $Z=(a,b,c,d)$ et $Z'=(a',b',c',d')$ alors:
	
	Il est aussi ais\'e de d\'emontrer qu'elle est involutive. Effectivement:
	
	La conjugaison n'est par contre pas un automorphisme multiplicatif du corps $(\mathbb{H},+,\times)$. En effet, si nous consid\'erons la multiplication de $Z$ et $Z'$ et en prenons le conjugu\'e:
	
	nous voyons imm\'ediatement (ne serait-ce que pour la deuxième ligne) que nous avons:
	
	Revenons maintenant sur notre norme (ou module).... Pour cela, calculons le carr\'e de la norme de $|ZZ'|$:
	
	Nous savons (par d\'efinition) que:
	
	notons ce produit de manière telle que:
	
	Nous avons alors:
	
	en substituant il vient:
	
	après un d\'eveloppement alg\'ebrique \'el\'ementaire (honnêtement ennuyeux), nous trouvons:
	
	Donc:
	
	\begin{tcolorbox}[title=Remarque,colframe=black,arc=10pt]
	La norme est donc un homomorphisme de $(\mathbb{H},\times)$ dans $(\mathbb{R},\times)$. Par la suite, nous noterons $\mathbb{G}$ l'ensemble des quaternions de norme unitaire.
	\end{tcolorbox}
	
	\paragraph{Interpr\'etation matricielles des quaternions}\mbox{}\\\\
	Soient $q$ et $p$ deux quaternions donn\'es, soit l'application:
	
	La multiplication (à gauche) peut être faite avec une application lin\'eaire (\SeeChapter{voir section d'Algèbre Lin\'eaire page \pageref{linear application}}) sur $\mathbb{H}$.
	
	Si $q$ s'\'ecrit:
	
	cette application a pour matrice, dans la base $(1,\mathrm{i},\mathrm{j},\mathrm{k})$:
	
	Ce que nous v\'erifions bien:
	
	En fait, nous pouvons alors d\'efinir les quaternions comme l'ensemble des matrices ayant la structure visible ci-dessus si nous le voulions. Cela les r\'eduirait alors à un sous espace vectoriel de $M_4(\mathbb{R})$.
	
	En particulier, la matrice de $1$  (la partie r\'eelle du quaternion $q$) n'est alors rien d'autre que la matrice identit\'e:
	
	de même:
	
	
	\paragraph{Rotations avec les quaternions}\mbox{}\\\\
	Nous allons maintenant voir que la conjugaison par un \'el\'ement du groupe $\mathbb{G}$ des quaternions de norme unit\'e peut s'interpr\'eter comme une rotation pure dans l'espace!
	
	\textbf{D\'efinition (\#\mydef):} La "\NewTerm{conjugaison}\index{conjugaison}" par un quaternion $q$ non nul et de norme unit\'e est l'application $S_q$ d\'efinie sur $\mathbb{H}$ par:
	
	et nous affirmons que cette application est une rotation.
	
	\begin{tcolorbox}[title=Remarques,colframe=black,arc=10pt]
	\textbf{R1.}  Comme $q$ est de norme $1$, nous avons bien \'evidemment $|q|=q\bar{q}=1$ donc $q^{-1}=\bar{q}$. Ce quaternion peut être vu comme la valeur propre (unitaire) de l'application (matricielle) $p$ sur le vecteur  $\bar{q}$ (on se retrouve avec un concept en tout point similaire aux matrices orthogonales de rotation vues dans la section d'Algèbre Lin\'eaire page \pageref{orthogonal matrix}).\\
	
	\textbf{R2.} $S_q$ est une application lin\'eaire (donc si c'est bien une rotation, la rotation peut être d\'ecompos\'ee en plusieurs rotations). Effectivement, prenons deux quaternions $p_1,p_2$  et deux r\'eels $\lambda_1,\lambda_2$, alors nous avons:
	
	\end{tcolorbox}
	V\'erifions maintenant que l'application est bien une rotation pure. Comme nous l'avons vu lors de notre \'etude de l'algèbre lin\'eaire et en particulier des matrices orthogonales (\SeeChapter{voir section d'Algèbre Lin\'eaire page \pageref{orthogonal matrix}}), une première condition est que l'application conserve la norme.
	
	V\'erifions cela:
	
	Par ailleurs, nous pouvons v\'erifier qu'une rotation d'un quaternion purement complexe (tel qu'alors nous nous restreignons à $\mathbb{R}^3$) et la même rotation inverse somm\'ees est nulle (le vecteur somm\'e à son oppos\'e s'annulent):
	
	nous v\'erifions trivialement que si nous avons deux quaternions $q, p$ alors $\overline{p\cdot q}=\bar{q}\bar{p}$ dès lors:
	
	pour que cette op\'eration soit nulle, nous voyons imm\'ediatement que nous devons restreindre $p$ aux quaternions purement complexes. Dès lors:
	
	Nous en d\'eduisons alors que $p$ doit être purement complexe pour que l'application $S_q$ soit une rotation et que $S_q(p)$  est un quaternion pur. En d'autres termes, cette application est stable (en d'autres termes: un quaternion pur par cette application reste un quaternion pur).
	
	$S_q$ restreint à l'ensemble des quaternions purement complexes est donc une isom\'etrie vectorielle, c'est-à-dire une sym\'etrie ou une rotation.
	
	Nous avons vu \'egalement lors de notre \'etude des matrices de rotation dans la section d'Algèbre Lin\'eaire page \pageref{rotation matrix in linear algebra} que d telles matrices devaient avoir un d\'eterminant de $+1$ pour que nous ayons une rotation. Voyons si c'est le cas de $S_q$:
	
	Pour cela, nous calculons explicitement en fonction de:
	
	la matrice (dans la base canonique $(\mathrm{i},\mathrm{j},\mathrm{k})$) de $S_q$ et nous en calculons le d\'eterminant. Ainsi, nous obtenons les coefficients des colonnes en se rappelant que:
	
	et ensuite en calculant:
	
	
	
	
	
	Il faut alors calculer le d\'eterminant de la matrice (pfff...):
	
	en se souvenant que (ce qui permet aussi de simplifier l'expression des termes de la diagonale comme nous pouvons le voir dans certains ouvrages):
	
	nous trouvons que le d\'eterminant vaut bien $1$. Sinon, nous pouvons v\'erifier cela avec Maple 4.00b:
	
	\texttt{>with(linalg):\\
	>A:=linalg[matrix](3,3,[a\string^2+b\string^2-c\string^2-d\string^2,2*(a*d+b*c),\\
	2*(b*d-a*c),2*(b*c-a*d),a\string^2-b\string^2+c\string^2-d\string^2,2*(a*b+c*d),\\
	2*(a*c+b*d),2*(c*d-a*b),a\string^2-b\string^2-c\string^2+d\string^2]);\\
	>factor(det(A));}
	
	Montrons maintenant que cette rotation est un demi-tour d'axe (l'exemple qui peut sembler particulier est g\'en\'eral!):
	
	D'abord, si:
	
	nous avons:
	
	ce qui signifie que l'axe de rotation $(x, y, z)$ est fix\'e par l'application $S_q$ elle-même !
	
	D'autre part, nous avons vu que si $q$ est un quaternion purement complexe de norme $1$ alors:
	
	Ce qui nous donne la relation:
	
	Ce r\'esultat nous amène à calculer la rotation d'une rotation:
	
	Conclusion: Puisque la rotation d'une rotation est un tour complet, alors $S_q$ est n\'ecessairement un demi-tour:
	
	relativement (!) aux axes $(x, y, z)$.
	
	À ce stade, nous pouvons affirmer que toute rotation de l'espace peut se repr\'esenter par $S_q$ (la conjugaison par un quaternion $q$ de norme $1$). En effet, les demi-tours engendrent le groupe des rotations, c'est-à-dire que toute rotation peut s'exprimer comme le produit d'un nombre fini de demi-tours, et donc comme la conjugaison par un produit de quaternions de norme unitaire (produit qui est lui-même un quaternion de norme unitaire ...).
	
	Nous allons tout de même donner une forme explicite reliant une rotation et le quaternion qui la repr\'esente, au même titre que nous l'avons fait pour les nombres complexes.
	\begin{theorem}
	Soit $\vec{u}(x,y,z)$ un vecteur unitaire et $\theta \in [0,2\pi]$ un angle. Alors nous affirmons que la rotation d'axe $\vec{u}$ et d'angle $\theta$ correspond à l'application $S_q$, où $q$ est le quaternion:
	
	Pour que cette affirmation soit v\'erifi\'ee, nous savons qu'il faut que:
	\begin{itemize}
		\item La norme de $q$ soit unitaire $1$

		\item Le d\'eterminant de l'application $S_q$ soit \'egal à l'unit\'e $1$

		\item Que l'application $S_q$ conserve la norme
	
		\item Que l'application $S_q$ renvoie tout vecteur colin\'eaire à l'axe de rotation sur l'axe de rotation
	\end{itemize}
	\end{theorem}
	\begin{dem}
	Ok contrôlons chacun de ces points!
	\begin{enumerate}
		\item La norme du quaternion propos\'e pr\'ec\'edemment vaut effectivement $1$:
		
		et comme $\vec{u}(x,y,z)$ est unitaire alors nous avons:
		
		Donc:
		
		
		\item Le fait que $q$ soit un quaternion de norme unitaire amène imm\'ediatement à ce que le d\'eterminant de l'application equation soit unitaire. Nous l'avons d\'ejà montr\'e plus haut dans le cas g\'en\'eral de n'importe quel quaternion de norme $1$ (condition n\'ecessaire et suffisante).
		
		\item Il en est de même pour la conservation de la norme. Nous avons d\'ejà montr\'e plus haut que c'\'etait de toute façon le cas dès que le quaternion $q$ \'etait de norme $1$ (condition n\'ecessaire et suffisante).

		\item Voyons maintenant que tout vecteur colin\'eaire à l'axe de rotation est projet\'e sur l'axe de rotation. Notons $q'$ le quaternion purement imaginaire et unitaire $x\mathrm{i}+y\mathrm{j}+z\mathrm{k}$. Nous avons alors:
		
		Alors:
		
		mais comme $q'$ est la restriction de $q$ à ces \'el\'ements purs qui le constituent, cela revient à \'ecrire:
		
		Montrons maintenant le choix de l'\'ecriture $\theta/2$. Si $\vec{v}=(x_1,y_1,z_1)$ d\'esigne un vecteur unitaire orthogonal à equation (perpendiculaire à l'axe de rotation donc), et $p$  le quaternion $x\mathrm{i}+y\mathrm{j}+z\mathrm{k}$ alors nous avons:	
		
		Nous avons montr\'e lors de la d\'efinition de la multiplication de deux quaternions que:
		
		nous obtenons alors:
		
		Nous avons \'egalement d\'emontr\'e plus haut que:
		
		dès lors:
		
		(le demi-tour d'axe $(x, y, z)$). Donc:
		
		\begin{tcolorbox}[title=Remarque,colframe=black,arc=10pt]
		Nous commençons à entrevoir ici d\'ejà l'utilit\'e d'avoir \'ecrit dès le d\'ebut $\theta/2$ pour l'angle!
		\end{tcolorbox}
	\end{enumerate}
	\begin{flushright}
		$\blacksquare$  Q.E.D.
	\end{flushright}
	\end{dem}
	Nous savons que $p$ est le quaternion pur assimil\'e à un vecteur unitaire $\vec{v}$ orthogonal à l'axe de rotation $\vec{u}$, lui-même assimil\'e à la partie purement imaginaire de $q'$. Nous remarquons alors de suite que la partie imaginaire du produit (d\'efini!) des quaternions $q'p$ est alors \'egal au produit vectoriel $\vec{u}\times\vec{v}=\vec{w}$. Ce produit vectoriel engendre donc un vecteur perpendiculaire à $\vec{u},\vec{v}$.
	
	Le couple $(\vec{v},\vec{w})$ forme donc un plan perpendiculaire à l'axe de rotation $\vec{u}$ (c'est comme pour les nombres complexes simples $\mathbb{C}$ dans lequel nous avons le plan de Gauss et perpendiculairement à celui-ci un axe de rotation!).
	
	Alors finalement:
	
	Nous nous retrouvons avec une rotation bas\'ee sur un plan (mais qui a donc lieu dans l'espace!) identique à celle pr\'esent\'ee plus haut avec les nombres complexes standards dans le plan de Gauss. Pour plus de d\'etails le lecteur peut se r\'ef\'erer à la section sur le Calcul Spinoriel à la page \pageref{spinors}.
	
	Nous savons donc maintenant comment faire n'importe quel type de rotation dans l'espace en une seule op\'eration math\'ematique et ce en plus par rapport à un libre choix de l'axe !

	Nous pouvons aussi maintenant mieux comprendre pourquoi l'algèbre des quaternions n'est pas commutative. Effectivement, les rotations vectorielles du plan sont commutatives mais celles de l'espace ne le sont pas comme nous le montre l'exemple ci-dessous:

	Soit la configuration initiale:
	\begin{figure}[H]
		\centering
		\includegraphics{img/arithmetics/quaternion_initial_configuration.jpg}
		\caption[]{Situation initiale pour rotations quaternions}
	\end{figure}
	Alors une rotation autour de l'axe $X$ suivie d'une rotation autour de l'axe $Y$:
	\begin{figure}[H]
		\centering
		\includegraphics{img/arithmetics/quaternion_x_y_rotation.jpg}
		\caption[]{Exemple rotation quaternion $X-Y$}
	\end{figure}
	n'est pas \'egale à une rotation autour de l'axe $Y$ suivie d'une rotation autour de l'axe $X$:
	\begin{figure}[H]
		\centering
		\includegraphics{img/arithmetics/quaternion_y_x_rotation.jpg}
		\caption[]{Exemple de non-\'equivalence pour la rotation des quaternions}
	\end{figure}
	Les r\'esultats obtenus seront fondamentaux pour notre compr\'ehension des spineurs (\SeeChapter{voir section de Calcul Spinoriel page \pageref{spinors}})!
	
	 \begin{tcolorbox}[title=Remarque,colframe=black,arc=10pt]
	Notez que nous avons (où $\mathbb{K}$ est un type de nombre appel\'e "nombres de Cayley" que nous n'as pas planifi\'e de traiter dans ce livre):
	
	et qu'à chaque \'etape certaines propri\'et\'es de la structure abstraite sont perdues. L'ordre est perdu lors du passage de $\mathbb{R}$ à $\mathbb{C}$. La commutativit\'e disparaît de $\mathbb{C}$ à $\mathbb{H}$. Les nombres de Cayley $\mathbb{K}$ perdent eux l'associativit\'e.
	\end{tcolorbox}
	
	\subsubsection{Nombres Alg\'ebriques et Transcendants}
	\textbf{D\'efinitions (\#\mydef):}
	\begin{enumerate}
		\item[D1.] Nous appelons "\NewTerm{nombre entier alg\'ebrique de degr\'e $n$}\index{nombre entier alg\'ebrique de degr\'e $n$}", tout nombre qui est solution d'une \'equation alg\'ebrique de degr\'e $n$, à savoir: un polynôme de degr\'e $n$ (concept que nous aborderons dans la section d'Algèbre) dont les coefficients sont des entiers relatifs et dont le coefficient dominant vaut $1$.

		\item[D2.] Nous appelons  "\NewTerm{nombre alg\'ebrique de degr\'e $n$}\index{nombre alg\'ebrique de degr\'e $n$}", tout nombre qui est solution d'une \'equation alg\'ebrique de degr\'e $n$, à savoir: un polynôme de degr\'e $n$ dont les coefficients sont des rationnels.
	\end{enumerate}
	
	L'ensemble des nombres alg\'ebriques est parfois not\'e $\overline{\mathbb{Q}}$ ou encore $\mathbb{A}$.
	
	\pagebreak
	\begin{theorem}
	
	Un premier r\'esultat int\'eressant et particulier dans ce domaine d'\'etude (curiosit\'e math\'ematique...) est qu'un nombre rationnel est un "nombre entier alg\'ebrique de degr\'e $n$" si et seulement si c'est un entier relatif (lisez plusieurs fois au besoin...). En termes savants, nous disons alors que l'anneau $\mathbb{Z}$ est "\NewTerm{int\'egralement clos}\index{anneau int\'egralement clos}".
	\end{theorem}
	\begin{dem}
	Nous supposons que le nombre $p/q$, où $p$ et $q$ sont deux entiers premiers entre eux (c'est-à-dire dont le rapport ne donne pas un entier ou plus rigoureusement... que le plus grand commun diviseur est 1!), est une racine du polynôme (\SeeChapter{voir la section de Calcul Alg\'ebrique page page \pageref{polynomial}}) suivant à coefficients entiers relatifs ($\in\mathbb{Z}$) et dont le coefficient dominant est unitaire:
	
	où l'\'egalit\'e avec z\'ero du polynôme est implicite.

	Dans ce cas:
	
	Puisque les coefficients sont par d\'efinition tous entiers et leurs multiples aussi dans la parenthèse, alors la parenthèse à n\'ecessairement une valeur dans $\mathbb{Z}$.
	
	Ainsi, $q$ (à droite de la parenthèse) divise une puissance de $p$ (à gauche de l'\'egalit\'e), ce qui n'est possible, dans l'ensemble $\mathbb{Z}$ (car notre parenthèse a une valeur dans cet ensemble pour rappel...), que si $q$ vaut $\pm 1$ (puisqu'ils \'etaient premiers entre eux).
	
	Donc parmi tous les nombres rationnels, les seuls qui sont solutions d'\'equations polynômiales à coefficients entiers relatifs $(\in \mathbb{Z}$) et dont le coefficient dominant est unitaire sont des entiers relatifs!
	\begin{flushright}
		$\blacksquare$  Q.E.D.
	\end{flushright}
	\end{dem}
	Pour prendre un autre cas int\'eressant et particulier, il est facile de montrer qu'absolument tout nombre rationnel est un "nombre alg\'ebrique". Effectivement, si nous prenons le plus simple polynôme suivant:
	
	où $q$ et $p$ sont premiers entre eux et où $q$ est diff\'erent de $1$. Alors comme il s'agit d'une polynôme à coefficients rationnels simples ($\in\mathbb{Q}$), après remaniement nous avons:
	
	Donc puisque $q$ et $p$ sont premiers entre eux et que $q$ est diff\'erent de l'unit\'e, nous avons bien que tout nombre rationnel est un "nombre alg\'ebrique de degr\'e $1$".
	
	Nous avons aussi le nombre r\'eel (et irrationnel) $\sqrt{2}$ qui est un "nombre entier alg\'ebrique de degr\'e $2$", car il est racine de:
	
	et le nombre complexe $\mathrm{i}$ qui est aussi un "nombre entier alg\'ebrique de degr\'e $2$", car il est racine de l'\'equation:
	
	etc.
	
	\textbf{D\'efinition (\#\mydef):} Un "\NewTerm{nombre transcendant}\index{nombre transcendant}" est un nombre r\'eel ou complexe qui n'est pas alg\'ebrique. C'est-à-dire qui n'est pas la racine d'une \'equation polynômiale non-nulle avec coefficient rationnels.
	
	\begin{theorem}
	L'ensemble de tous les nombres transcendants est non d\'enombrable. La preuve est simple et ne n\'ecessite aucun d\'eveloppement math\'ematique difficile.
	\end{theorem}
	\begin{dem}
	Effectivement, puisque les polynômes à coefficients entiers sont d\'enombrables, et puisque chacun de ces polynômes possède un nombre fini de z\'eros (voir le th\'eorème de factorisation dans la section de Calcul Alg\'ebrique page \pageref{factorization theorem}), l'ensemble des nombres alg\'ebriques est d\'enombrable! Mais l'argument de la diagonale de Cantor (\SeeChapter{vkjr la section de Th\'eorie des Ensembles pages \pageref{Cantor's diagonal}}) \'etablit que les nombres r\'eels (et par cons\'equent les nombres complexes aussi) sont non d\'enombrables, donc l'ensemble de tous les nombres transcendants doit être non d\'enombrable.

	En d'autres termes, il y a beaucoup plus de nombres transcendants que de nombres alg\'ebriques.
	\begin{flushright}
		$\blacksquare$  Q.E.D.
	\end{flushright}
	\end{dem} 
	Les transcendants les plus connus sont  $\pi$ et $e$. Les d\'emonstrations de leur transcendance est en cours de r\'edaction. Nous sommes toujours entrain de regarder pour en faire une d\'emonstration pour ce livre mais si possible plus simple et intuitive que celle donn\'ees par Hilbert ou Lindemann–Weierstrass.
	
	Voici un petit r\'esum\'e de tout ce que nous avons vu jusqu'à maintenant:
	\begin{figure}[H]
		\centering
		\includegraphics{img/arithmetics/numbers_type.jpg}
		\caption{Types de nombres $\mathbb{N},\mathbb{Z},\mathbb{Q},\mathbb{R},\mathbb{C},$...}
	\end{figure}
	
	\pagebreak
	\subsubsection{Nombres Univers (nombres normaux)}
	\textbf{D\'efinition (\#\mydef):} Un "\NewTerm{nombres Univers}\index{nombre Univers}" aussi appel\'e "\NewTerm{nombre normal}\index{nombre normal}" est un nombre r\'eel dont la s\'equence infinie de chiffres dans chaque base  $b$ est r\'epartie uniform\'ement dans le sens que chacune des valeurs du chiffre $b$ a la même densit\'e naturelle $1/b$. Intuitivement, cela signifie qu’aucun chiffre, ni aucune combinaison de chiffres (finie), ne se produisent plus fr\'equemment que les autres. L'ensemble des num\'eros d'Univers est parfois not\'e $\mathbb{U}$.

	Bien qu'une preuve g\'en\'erale puisse être donn\'ee que presque tous les nombres purement r\'eels sont des nombres Univers \cite{filip2010elementary} cette preuve n'est pas constructive et très peu de nombres sp\'ecifiques ont \'et\'e montr\'es comme \'etant des nombres Univers. Il est largement admis que les nombres (calculables) comme $\sqrt{2}$, $\pi$ et $e$ sont des nombres Univers, mais une preuve demeure insaisissable encore en cette ann\'ee 2016 où nous \'ecrivons ces lignes. Tous sont cependant fortement conjectur\'es à être des nombres Univers en raison de certaines preuves empiriques. On ne sait même pas si tous les chiffres apparaissent infiniment souvent dans les d\'eveloppements d\'ecimaux de ces constantes. En particulier, la revendication populaire "chaque chaîne de nombres se produit finalement dans $\pi$" ou "tout le livre sacr\'e est contenu dans $\pi$ n'est pas connue pour être vraie. On a suppos\'e que tout nombre alg\'ebrique irrationnel est un Num\'ero d'univers, bien qu'aucun contre-exemple ne soit connu, il n'existe pas non plus de nombre alg\'ebrique qui s'est av\'er\'e être un num\'ero Univers dans aucune base.
	
	Plus formellement, posons que $\sum $ est un alphabet fini de $b$ digits et $\sum^\infty$ l'ensemble des s\'equences pouvant être tir\'ees de cet alphabet. Soit $S \in \sum^\infty $ une telle s\'equence. Pour chaque $a$ dans $\sum$ laissons $ N_S (a, n) $ indiquer le nombre de fois que la lettre $a$ apparaît dans les premiers $ n $ chiffres de la s\'equence $ S $. Nous disons que $ S $ est un "\NewTerm{num\'ero Univers simple}\index{num\'ero Univers simple}" si la limite:
	
	pour chaque $a$ est v\'erifi\'ee. 
	
	Maintenant, supposons que $w$ soit une chaîne finie dans $\sum^{*}$ et que $N_S(w, n)$ soit le nombre de fois que la chaîne $w$ apparaît comme une sous-chaîne dans les premiers $ n $ digits de la s\'equence $ S $ (par exemple, si $ S = 01010101 $ ..., alors $ N_S (010, 8) = 3 $). Alors $ S $ est un "\NewTerm{num\'ero Univers}\index{num\'ero Univers}" si, pour toutes les chaînes finies $ w \in \sum^{*}$:
	
	$S$ est donc un num\'ero Univers si toutes les chaînes de longueur \'egale se produisent avec une fr\'equence asymptotique \'egale. Une s\'equence infinie donn\'ee est un num\'ero Univers ou non, alors qu'un nombre pur, ayant une expansion de base $b$ diff\'erente  pour chaque entier $ b \geq 2 $, peut être un num\'ero Univers dans une base mais pas dans une autre. Une "\NewTerm{s\'equence disjonctive}\index{s\'equence disjonctive}" est une s\'equence dans laquelle chaque chaîne finie apparaît. Une s\'equence de num\'ero Univers est une "\NewTerm{s\'equence disjonctive}\index{s\'equence disjonctive}" mais une s\'equence disjonctive n'est pac n\'ecessairement un num\'ero Univers.
	
	Il est possible de prouver (avec le "th\'eorème des nombres Univers" que nous ne souhaitons pas pr\'esenter  dans un livre de math\'ematiques appliqu\'ees) que presque tous les nombres r\'eels purs sont des nombres Univers. L'ensemble des nombres non-Univers, bien que "petit" au sens d'un ensemble nul, est "grand" au sens d'être ind\'enombrable (par exemple, aucun nombre rationnel est normal à toute base, car les s\'equences de chiffres des nombres rationnels sont finalement p\'eriodiques!). Par exemple, il existe d'innombrables nombres dont le d\'eveloppement d\'ecimal ne contient pas le chiffre $ 5 $, et aucun de ceux-ci n'est un num\'ero Univers.
	
	\subsubsection{Nombres Abstraits (variables)}
	\textbf{D\'efinition (\#\mydef):} Le nombre peut être envisag\'e en faisant abstraction de la nature des objets qui constituent le groupement qu'il caract\'erise et ainsi qu'à la façon de codifier (chiffre arabe, romain, ou autre système universel). Nous disons alors que le nombre est un "\NewTerm{nombre abstrait}\index{nombre abstrait}".
	\begin{tcolorbox}[title=Remarque,colframe=black,arc=10pt]
	Arbitrairement, l'être humain a adopt\'e un système num\'erique majoritairement utilis\'e de par le monde et repr\'esent\'e par les symboles $1, 2, 3, 4, 5, 7, 8, 9$ du système d\'ecimal et qui seront suppos\'es connus aussi bien en \'ecriture qu'oralement par le lecteur (apprentissage du langage).
	\end{tcolorbox}
	Pour les math\'ematiciens, il n'est pas avantageux de travailler avec ces symboles car ils repr\'esentent uniquement des cas particuliers. Ce que cherchent les physiciens th\'eoriciens ainsi que les math\'ematiciens, ce sont des "\NewTerm{relations litt\'erales}\index{relations litt\'erales}" applicables dans un cas g\'en\'eral et que les ing\'enieurs puissent en fonction de leurs besoins changer ces nombres abstraits par les valeurs num\'eriques qui correspondent au problème qu'ils ont besoin de r\'esoudre.
	
		Ces nombres abstraits appel\'es aujourd'hui commun\'ement "\NewTerm{variables}\index{variables}" ou "\NewTerm{inconnues}", utilis\'ees dans le cadre du "\NewTerm{calcul litt\'eral}\index{calcul litt\'eral}", sont très souvent repr\'esent\'es par:
	\begin{enumerate}
		\item Les lettres de l'alphabet latin:
		
		où les lettres minuscules du d\'ebut l'alphabet latin ($a, b, c, d, e\ldots$) sont souvent utilis\'ees pour repr\'esenter de manière abstraite des constantes, alors que les lettres minuscules de la fin de l'alphabet latin (...x, y, z) sont utilis\'ees pour repr\'esenter des entit\'es (variables ou inconnues) dont nous recherchons la valeur. Les lettrees majuscules sont souvent r\'eserv\'ees pour repr\'esenter des matrices ou des variables al\'eatoires.
		
		\item L'alphabet grec:
		\begin{table}[H]\centering\small
			\begin{tabular}{clcl}\hline
			A$\alpha$ & Alpha & $\Lambda \lambda$ & Lambda \\
			B$\beta$  & Beta  & M$\mu$ & Mu \\
			$\Gamma\gamma$ & Gamma & N$\nu$ & Nu \\
			$\Delta\delta$ & Delta & $\Xi\xi$ & Xi\\
			E$\epsilon\varepsilon$ & Epsilon & O$o$ & Omicron\\
			Z$\zeta$ & Zeta & $\Pi\pi$ & Pi\\
			H$\eta$ & Eta & P$\rho$ & Rho \\
			$\Theta\theta\vartheta$ & Theta & $\Sigma\sigma$ & Sigma\\
			I$\iota$ & Iota & T$\tau$ & Tau \\
			K$\kappa$ & Kappa & $\Upsilon\upsilon$ & Upsilon \\
			$\Phi\phi\varphi$ & Phi & X$\chi$ & chi \\
			$\Psi\psi$ & Psi & $\Omega\omega$ & Omega \\ \hline
			\end{tabular}
			\caption{Alphabet Grec}
		\end{table}
		qui est particulièrement utilis\'e pour repr\'esenter soit des op\'erateurs math\'ematiques plus ou moins complexes (comme la somme index\'ee  $\Sigma$, le produit index\'e $\Pi$, le variationnel $\delta$, l'\'el\'ement infinit\'esimal $\varepsilon$, le diff\'erentiel partiel  $\partial$, etc.) soit des variables dans le domaine de la physique (comme $\omega$ pour la pulsation, la fr\'equence $\nu$, la densit\'e $\rho$, etc.).
		
		\item L'alphabet h\'ebraïque modernis\'e (à moindre mesure...) 
		
		Comme nous l'avons vu, les nombres transfinis sont par exemples donn\'es par la lettre $\mathcal{N}_0$ dite "aleph".
	\end{enumerate}
	Bien que ces symboles puissent repr\'esenter n'importe quel nombre il en existe quelques-uns qui peuvent repr\'esenter en physique des valeurs dites "\NewTerm{constantes Universelles}\index{constantes Universelles}" comme la vitesse de la lumière $c$, la constante gravitationnelle $G$, la constante de Planck $h$, etc.
	
	Nous utilisons très souvent encore d'autres symboles que nous introduirons et d\'efinirons au fur et à mesure.
	\begin{tcolorbox}[title=Remarque,colframe=black,arc=10pt]
	Les lettres pour repr\'esenter les nombres auraient \'et\'e employ\'ees pour la première fois par Viète au $16$ème siècle.
	\end{tcolorbox}
	
	\paragraph{Domaine de d\'efinition de variables}\mbox{}\\\\
	Une variable est un nombre abstrait susceptible de prendre des valeurs num\'eriques diff\'erentes. L'ensemble de ces valeurs peut varier suivant le caractère du problème consid\'er\'e.
	
	Soient $a$ et $b$ deux nombres tel que $a<b$. Alors:
	
	\textbf{D\'efinitions (\#\mydef):}\label{domain of definition}
	\begin{enumerate}
		\item[D1.] Nous appelons "\NewTerm{domaine de d\'efinition}\index{domaine de d\'efinition}" d'une variable, l'ensemble des valeurs num\'eriques qu'elle est susceptible de prendre entre deux valeurs finies ou infinies appel\'ees "\NewTerm{bornes}\index{bornes}" sur un ensemble donn\'ee (comme $\mathbb{N}, \mathbb{R},\mathbb{R}^+,$ etc.).
		
		\item[D2.] Nous appelons "\NewTerm{intervalle ferm\'e d'extr\'emit\'es $a$ et $b$}\index{intervalle ferm\'e}", l'ensemble de tous les nombres $x$ compris entre ces deux valeurs incluses et nous le d\'esignons de la façon suivante:
		
		Le notation à gauche est appel\'ee trivialement "\NewTerm{notation d'intervalle}\index{intervalle}", celle est à droite est appel\'ee "\NewTerm{notation ensemblisste}".
		
		\item[D3.] Nous appelons "\NewTerm{intervalle ouvert d'extr\'emit\'es $a$ et $b$}\index{intervalle ouvert}", l'ensemble de tous les nombres x compris entre ces deux valeurs non incluses et nous le d\'esignons de la façon suivante: 
		
		
		\item[D4.] Nous appelons "\NewTerm{intervalle ferm\'e à gauche, ouvert à droite}\index{semi-intervalle}" l'ensemble suivant:
		
		
		\item[D5.] Nous appelons "\NewTerm{intervalle ouvert à gauche, ferm\'e à droite}\index{semi-interval}" l'ensemble suivant:
		
	\end{enumerate}
	Soit sous forme r\'esum\'ee et imag\'ee telle que souvent not\'ee en Suisse:
	\begin{table}[H]
		\begin{center}
			\definecolor{gris}{gray}{0.85}
				\begin{tabular}{|c|c|c|p{6cm}|}
					\hline
					\multicolumn{1}{c}{\cellcolor{black!30}\textbf{Type}} & 
	  \multicolumn{1}{c}{\cellcolor{black!30}\textbf{Visuel}} & 
	  \multicolumn{1}{c}{\cellcolor{black!30}\textbf{Notation}} & 
	  \multicolumn{1}{c}{\cellcolor{black!30}\textbf{Explicitement}} \\ \hline
					$[a,b]$ & \cincludegraphics{img/arithmetics/domain_interval_1.jpg} & $a\leq x \leq b$ & Intervalle ferm\'e born\'e\label{closed bounded interval} \\ \hline
					$[a,b[$ & \cincludegraphics{img/arithmetics/domain_interval_2.jpg} & $a\leq x < b$ & Intervalle born\'e semi-ferm\'e en $a$ et semi-ouvert en $b$ (ou semi-ferm\'e à gauche et semi-ouvert à droite)\\ \hline
					$]a,b]$ & \cincludegraphics{img/arithmetics/domain_interval_3.jpg} & $a< x \leq b$ & Intervalle born\'e semi-ouvert en $a$ et semi-ferm\'e en $b$ (ou semi-ouvert à gauche et semi-ferm\'e à droite)\\ \hline
					$]a,b[$ & \cincludegraphics{img/arithmetics/domain_interval_4.jpg} & $a< x < b$ & Intervalle ouvert born\'e\\ \hline
					$]-\infty,b]$ & \cincludegraphics{img/arithmetics/domain_interval_5.jpg} & $ x \leq b$ & Intervalle non born\'e ferm\'e en $b$ (ou ferm\'e à droite) \\ \hline
					$]-\infty,b[$ & \cincludegraphics{img/arithmetics/domain_interval_6.jpg} & $ x \leq b$ & Intervalle non born\'e ouvert en $b$ (ou ouvert à droite) \\ \hline
					$[a,+\infty[$ & \cincludegraphics{img/arithmetics/domain_interval_7.jpg} & $a\leq x$ & Intervalle non born\'e ferm\'e en $a$ (ou ferm\'e à gauche) \\ \hline
					$]a,+\infty[$ & \cincludegraphics{img/arithmetics/domain_interval_8.jpg} & $a< x $ & Intervalle non born\'e ouvert en $a$ (ou ouvert à gauche) \\ \hline
			\end{tabular}
		\end{center}
		\caption{Types d'intervalles et de bornes tels que not\'es en Suisse}
	\end{table}
	et selon la norme internationale ISO 80000-2:2009 (car les Suisses ont l'art de ne pas respecter les normes...):
	\begin{table}[H]
		\begin{center}
			\definecolor{gris}{gray}{0.85}
				\begin{tabular}{|c|c|c|p{6cm}|}
					\hline
					\multicolumn{1}{c}{\cellcolor{black!30}\textbf{Type}} & 
	  \multicolumn{1}{c}{\cellcolor{black!30}\textbf{Visuel}} & 
	  \multicolumn{1}{c}{\cellcolor{black!30}\textbf{Notation}} & 
	  \multicolumn{1}{c}{\cellcolor{black!30}\textbf{Explicitement}} \\ \hline
					$[a,b]$ & \cincludegraphics{img/arithmetics/domain_interval_1.jpg} & $a\leq x \leq b$ & Intervalle ferm\'e born\'e \\ \hline
					$[a,b)$ & \cincludegraphics{img/arithmetics/domain_interval_2.jpg} & $a\leq x < b$ & Intervalle born\'e semi-ferm\'e en $a$ et semi-ouvert en $b$ (ou semi-ferm\'e à gauche et semi-ouvert à droite)\\ \hline
					$(a,b]$ & \cincludegraphics{img/arithmetics/domain_interval_3.jpg} & $a< x \leq b$ & Intervalle born\'e semi-ouvert en $a$ et semi-ferm\'e en $b$ (ou semi-ouvert à gauche et semi-ferm\'e à droite)\\ \hline
					$(a,b)$ & \cincludegraphics{img/arithmetics/domain_interval_4.jpg} & $a< x < b$ & Intervalle ouvert born\'e\\ \hline
					$(-\infty,b]$ & \cincludegraphics{img/arithmetics/domain_interval_5.jpg} & $ x \leq b$ & Intervalle non born\'e ferm\'e en $b$ (ou ferm\'e à droite) \\ \hline
					$(-\infty,b[$ & \cincludegraphics{img/arithmetics/domain_interval_6.jpg} & $ x \leq b$ & Intervalle non born\'e ouvert en $b$ (ou ouvert à droite) \\ \hline
					$[a,+\infty)$ & \cincludegraphics{img/arithmetics/domain_interval_7.jpg} & $a\leq x$ & Intervalle non born\'e ferm\'e en $a$ (ou ferm\'e à gauche) \\ \hline
					$(a,+\infty)$ & \cincludegraphics{img/arithmetics/domain_interval_8.jpg} & $a< x $ & Intervalle non born\'e ouvert en $a$ (ou ouvert à gauche) \\ \hline
			\end{tabular}
		\end{center}
		\caption{Types d'intervalles et de bornes tels que not\'es selon les normes}
	\end{table}

	\begin{tcolorbox}[title=Remarques,colframe=black,arc=10pt]
	\textbf{R1.} La notation $\{x\text{ tel que } a<x<b\}$ d\'esigne l'ensemble des r\'eels $x$ strictement plus grands que $a$ et strictement inf\'erieurs à $b$.\\
	
	\textbf{R2.}  Le fait de dire qu'un intervalle est par exemple ouvert en $b$ signifie que le r\'eel $b$ ne fait pas partie de celui-ci. Par contre, s'il avait \'et\'e ferm\'e, alors $b$ en aurait fait partie.\\
	
	\textbf{R3.} Si la variable peut prendre toutes les valeurs n\'egatives et positives possibles nous \'ecrivons dès lors: $\left] -\infty,+\infty \right[$ où le symbole "$\infty$" signifie une "infinit\'e". \'evidemment il peut y avoir des combinaisons d'intervalles ouverts et infinis à droite, ferm\'e et limit\'e gauche et r\'eciproquement.\\
	
	\textbf{R4.} Nous rappellerons ces concepts avec une autre approche lorsque nous \'etudierons l'Algèbre (calcul litt\'eral).
	\end{tcolorbox}	

	Nous disons que la variable $x$ est une "\NewTerm{variable ordonn\'ee}\index{variable ordonn\'ee}" si en repr\'esentant son domaine de d\'efinition par un axe horizontal où chaque point de l'axe repr\'esente une valeur de $x$, alors pour chaque couple de valeurs, nous pouvons indiquer celle qui est "\NewTerm{ant\'ec\'edente}\index{ant\'ec\'edente}" (qui pr\'ecède) et celle qui est "\NewTerm{cons\'equente}\index{cons\'equente}" (qui suit). Ici la notion d'ant\'ec\'edente ou de cons\'equente n'est pas li\'ee au temps, elle exprime juste la façon d'ordonner les valeurs de la variable!
	
	\textbf{D\'efinitions (\#\mydef):}
	\begin{enumerate}
		\item[D1.] Une variable est dite "\NewTerm{croissante}\index{variable croissante}" si chaque valeur cons\'equente est plus grande que chaque valeur ant\'ec\'edente.

		\item[D2.] Une variable est dite "\NewTerm{d\'ecroissante}\index{variable d\'ecroissante}" si chaque valeur cons\'equente est plus petite que chaque valeur ant\'ec\'edente. 
		
		\item[D3.] Les variables croissantes et les variables d\'ecroissantes sont appel\'ees "\NewTerm{variables à variations monotones}\index{variables à variations monotones}" ou simplement "\NewTerm{variables monotones}".
	\end{enumerate}

	
	\begin{flushright}
	\begin{tabular}{l c}
	\circled{90} & \pbox{20cm}{\score{4}{5} \\ {\tiny 31 votes, 69.68\%}} 
	\end{tabular} 
	\end{flushright}
	
	%to make section start on odd page
	\newpage
	\thispagestyle{empty}
	\mbox{}
	\section{Op\'erateurs Arithm\'etiques}
	\lettrine[lines=4]{\color{BrickRed}P}arler des nombres comme nous l'avons fait dans le chapitre pr\'ec\'edent amène naturellement à consid\'erer les op\'erations de calculs. Il est donc logique que nous fassions une description non exhaustive des op\'erations qui peuvent exister entre les nombres. Ce sera l'objectif de ce chapitre.
	
	Nous consid\'ererons dans ce livre qu'il existe deux types d'outils fondamentaux en arithm\'etique (nous ne parlons pas de l'Algèbre mais de l'Arithm\'etique!):
	
	\begin{itemize}
		\item Les op\'erateurs arithm\'etiques:
		
		Il existe deux op\'erateurs de base (addition  "$+$" et soustraction "$-$") à partir desquels nous pouvons construire d'autres op\'erateurs: la "multiplication" $\times$ (dont le symbole contemporain aurait \'et\'e introduit en 1574 par William Oughtred) et la "division" (dont le vieux symbole est "$\div$" mais depuis la fin du 20ème siècle on utilise plutôt le symbole $/$).
		
		Ces quatre op\'erateurs ($+$, $-$, $\times$, $/$) sont couramment appel\'es "\NewTerm{op\'erateurs rationnels}\index{op\'erateurs rationnels}". Nous verrons ces derniers plus en d\'etails après avoir d\'efini les relations binaires.
		
		\begin{tcolorbox}[title=Remarque,colframe=black,arc=10pt]
		Rigoureusement l'addition suffirait si nous consid\'erons l'ensemble commun des r\'eels $\mathbb{R}$ car dès lors la soustraction n'est que l'addition d'un nombre n\'egatif.
		\end{tcolorbox}
	
		\item Les op\'erateurs (relations) binaires:
		
		Il existe six relations binaires fondamentales (\'egal  $=$, diff\'erent de $\neq$, plus grand que $>$, plus petit que $\leq$, plus grand ou \'egal $\geq$, plus petit ou \'egal) qui permettent de comparer des grandeurs d'\'el\'ements se trouvant à gauche et à droite (donc au nombre de deux, d'où leur nom) afin d'en tirer certaines conclusions. La majorit\'e des symboles de relations binaires auraient \'et\'e introduites par Viète et Harriot au 16ème siècle).
	\end{itemize}

	Il est bien \'evidemment essentiel de connaître au mieux ces deux outils et leurs propri\'et\'es avant de se lancer dans des calculs plus ardus.
	\begin{figure}[H]
		\centering
		\includegraphics[scale=0.4]{img/arithmetics/operators.jpg}
	\end{figure}
	
	\subsection{Relations binaires}
	Le concept de "\NewTerm{relation}\index{relation}" est la base de toute la math\'ematique dont le but est d'\'etudier - par observation et d\'eduction (raisonnement), calcul et comparaison - des configurations ou relations abstraites ou concrètes de ses objets (nombres, formes, structures) en cherchant à \'etablir les liens logiques, num\'eriques ou conceptuels entre ces objets.
	
	\textbf{D\'efinitions (\#\mydef):}
	\begin{enumerate}
		\item[D1.]  Consid\'erons deux ensembles non vides $E$ et $F$  (\SeeChapter{voir section de Th\'eorie des Ensembles page \pageref{empty set}}) non n\'ecessairement identiques. Si à certains \'el\'ements $x$ de $E$ nous pouvons associer par une règle math\'ematique pr\'ecise $\mathcal{R}$ (non ambiguë) un \'el\'ement $y$ de $F$, nous d\'efinissons ainsi une "\NewTerm{relation fonctionnelle}\index{relation fonctionnelle}" de $E$ vers $F$ et qui s'\'ecrit:
		
		Ainsi, de façon plus g\'en\'erale, une relation fonctionnelle $\mathcal{R}$ peut être d\'efinie comme une règle math\'ematique qui associe à certains \'el\'ements $x$ de $E$, certains \'el\'ements $y$ de $F$.
		
		Alors, dans ce contexte plus g\'en\'eral, si $x\mathcal{R}y$, nous disons que $y$ est une "image" de $x$ par $\mathcal{R}$ et que $x$ est un "\NewTerm{ant\'ec\'edent}" ou "\NewTerm{pr\'eimage}\index{pr\'eimage}" de $y$.
		
		L'ensemble des couples $(x, y)$ tels que $x\mathcal{R}y$ soit une assertion vraie forme un "graphe" ou une "repr\'esentation" de la relation $\mathcal{R}$. Nous pouvons repr\'esenter ces couples dans un repère ad\'equatement choisi pour faire une repr\'esentation graphique de la relation $R$.
		
		Il s'agit d'un type de relations sur lequel nous reviendrons dans le chapitre d'Analyse Fonctionnelle (page \pageref{composite function}) sous la forme: $\mathcal{R}:f(x)=y\circ f$ et qui ne nous int\'eresse pas directement dans cette section.
		
		\item[D2.] Consid\'erons un ensemble $E$ non vide, si nous associons à cet ensemble (et à celui-ci uniquement!) des outils permettant de comparer les \'el\'ements le composant alors nous parlons de "\NewTerm{relation binaire}\index{relation binaire}" ou "\NewTerm{relation de comparaison}\index{relation de comparaison}" et qui s'\'ecrit pour tout \'el\'ement $x$ et $y$ composant $E$:
		
		Ces relations peuvent aussi être repr\'esent\'ees sous forme graphique. Dans le cas des op\'erateurs binaires classiques de comparaisons où $E$ est l'ensemble des nombres naturels $\mathbb{N}$, relatifs $\mathbb{Z}$, rationnels $\mathbb{Q}$ ou r\'eels $\mathbb{R}$, cette forme graphique est repr\'esent\'ee par une droite horizontale (le plus souvent...); dans le cas de la congruence (\SeeChapter{voir section Th\'eorie des Nombres page \pageref{congruence}}) elle est repr\'esent\'ee par des droites dans le plan dont les points sont donn\'es par la contrainte de la congruence.
	\end{enumerate}
	Comme nous l'avons d\'ejà mentionn\'e, il existe $6$ relations binaires fondamentales (\'egal $=$, diff\'erent de $\neq $, plus grand que $>$, plus petit que $<$, plus grand ou \'egal $\geq$, plus petit ou \'egal $\leq$). Mais nous verrons un peu plus loin que la d\'efinition rigoureuse des relations binaires permet donc de construire des outils plus abstraits (comme par exemple la congruence bien connue par les \'elèves de petites classes et que nous \'etudierons dans le chapitre de Th\'eorie des Nombres).
	
	\subsubsection{\'egalit\'es}
	Il est fort difficile de d\'efinir la notion "\NewTerm{d'\'egalit\'e}\index{d'\'egalit\'e}" dans un cas g\'en\'eral applicable à toute situation. Pour notre part, nous nous permettrons pour cette d\'efinition de nous inspirer du th\'eorème d'extensionalit\'e de la Th\'eorie des Ensembles (que nous verrons plus tard page \pageref{extensionality axiom}).
	
	\textbf{D\'efinitions (\#\mydef):}
	\begin{enumerate}
		\item[D1.]  Deux \'el\'ements sont "\NewTerm{\'egaux}\index{\'egalit\'e}" si, et seulement si, ils ont les mêmes valeurs. L'\'egalit\'e est d\'ecrite par le symbole $=$ \label{equality} qui signifie "\'egal à" (ce symbole aurait \'et\'e introduit par Robert Rocorde en 1557).
		
		Propri\'et\'e (triviale): Si nous avons $a=b$, et $c$ un nombre et $\star$ une op\'eration quelconque (telle que l'addition, la soustraction, la multiplication ou la division) alors:
		
		Cette propri\'et\'e est très utilis\'ee pour r\'esoudre ou simplifier des \'equations de type quelconque.
		
		Evidemment nous avons (propri\'et\'e de r\'eflexivit\'e):
		
		Et aussi (propri\'et\'e de transitivit\'e):
		
		Nous n'\'enum\'ererons pas les autres propri\'et\'es de l'\'egalit\'e dans cette section (pour plus de d\'etails, voir la section de Th\'eorie des Ensembles page \pageref{equality}).
		
		\item[D2.] Si deux \'el\'ements ne sont pas \'egaux (donc sont in\'egaux...), nous les relions par le symbole $\neq$ et nous disons qu'ils sont "\NewTerm{non \'egaux}\index{non \'egaux}" ou simplement "\NewTerm{diff\'erents}\index{diff\'erents}".
		
		Si nous avons $a>b$ ou $a<b$ alors :
		
	\end{enumerate}
	Il existe encore d'autres symboles d'\'egalit\'es, qui sont une extension des deux que nous avons d\'efinis pr\'ec\'edemment. Malheureusement, ils sont assez souvent mal utilis\'es (disons plutôt qu'ils sont utilis\'es aux mauvais endroits) dans la plupart des ouvrages disponibles sur le march\'e:
	\begin{enumerate}
		\item $\cong$: Devrait être utilis\'e pour la congruence mais est en fait principalement utilis\'e pour indiquer une approximation.
		
		\item $\approx$: Devrait être utilis\'e pour des approximations mais en fait, $\cong$ est souvent utilis\'e (à tort!) à la place.
		
		\item $\equiv$: Devrait être utilis\'e pour dire que deux \'el\'ements sont \'equivalents mais en pratique la plupart des gens utilisent (à tort!) $=$.
		
		\item $:=$: Devrait utilis\'e pour dire qu'un \'el\'ement est "\'egal par d\'efinition à" un autre.
		
		\item $\doteq$: Est aussi utilis\'e pour dire "\'egal par d\'efinition à" mais en fait la plupart des gens utilisent plutôt $:=$.
		
		\item $\sim$: Est utilis\'e le plus souvent dans les statistiques pour dire "suit la loi ..." mais certains praticiens utilisent plutôt $ = $ ou pour dire "asymptotiquement \'egal".
	\end{enumerate}
	
	\subsubsection{Comparateurs}\label{comparators}
	Les comparateurs sont des outils qui nous permettent de comparer et d'ordonner tout couple de nombres (et in extenso aussi des ensembles!).
	
	La possibilit\'e d'ordonner des nombres est presque fondamentale en math\'ematique. Dans le cas contraire (s'il n'\'etait pas possible ou non impos\'e d'ordonner), il y aurait des tas de choses qui choqueraient nos habitudes, par exemple (certains des concepts pr\'esent\'es dans la phrase qui suit n'ont pas encore \'et\'e vus mais nous souhaitons quand même y faire r\'ef\'erence): plus de fonctions monotones (en particulier de suites) et li\'e à cela la d\'erivation n'indiquerait donc rien sur un "sens de variation", plus d'approche de z\'eros d'un polynôme par dichotomie (algorithme classique de recherche dans un ensemble ordonn\'e partag\'e en deux à chaque it\'eration), en g\'eom\'etrie, plus de segments ni de demi-droites, plus de demi-espace, plus de convexit\'e, nous ne pouvons plus orienter l'espace, etc. C'est donc important de pouvoir ordonner les choses comme vous l'aurez compris.
	
	Ainsi, pour tout  $a,b,c\in \mathbb{R}$ nous \'ecrivons lorsque $a$ est plus grand ou \'egal à $b$:
	
	et lorsque $a$ est plus petit ou \'egal à $b$:
	
	\begin{tcolorbox}[title=Remarque,colframe=black,arc=10pt]
	 Il est utile de rappeler que l'ensemble des r\'eels $\mathbb{R}$ est un groupe totalement ordonn\'e (\SeeChapter{voir section Th\'eorie des Ensembles page \pageref{groups}}), sans quoi nous ne pourrions pas d\'efinir des relations d'ordre entre ses \'el\'ements (ce qui n'est pas le cas des nombres complexes que nous ne pouvons pas ordonner!).
	\end{tcolorbox}
	
	\textbf{D\'efinition (\#\mydef):} Le symbole $\leq$ est une "\NewTerm{relation d'ordre}\index{relation d'ordre}" (voir la d\'efinition rigoureuse plus bas!) qui signifie "\NewTerm{plus petit ou \'egal à}" et inversement le symbole $\geq$ est aussi une relation d'ordre qui signifie "\NewTerm{plus grand ou \'egal à}\index{plus grand ou \'egal à}".
	
	Nous avons \'egalement concernant la comparaison stricte les propri\'et\'es suivantes qui sont relativement intuitives:
	
	et:
	
	si:
	
	si:
	
	et vice versa:
	
	Nous avons aussi:
	
	et vice versa:
	
	Nous pouvons bien \'evidemment multiplier, diviser, additionner ou soustraire un terme de chaque côt\'e de la relation telle que celle-ci soit toujours vraie. Petite remarque cependant, si vous multipliez les deux membres par un nombre n\'egatif il faudra bien \'evidemment changer le comparateur tel que si:
	
	et vice versa:
	
	Nous avons aussi:
	
	Consid\'erez maintenant $b<a<0$ et $p\in \mathbb{N}^{*}$. Alors $p$ est un nombre entier pair:
	
	sinon si $p$ est impair:
	
	Ce r\'esultat provient simplement de la multiplication des signes puisque la puissance lorsqu'elle est non fractionnaire n'est qu'une multiplication.

	Finalement:
	
	Les relations d'ordre:
	
	correspondent donc respectivement à: (strictement) plus grand que, (strictement) plus petit que, plus petit ou \'egal à, plus grand ou \'egal à, beaucoup plus grand que et enfin beaucoup plus petit que.
	
	Ces relations peuvent être d\'efinies de façon un peu plus subtile et rigoureuse et ne s'appliquent pas seulement aux comparateurs (voir par exemple la relation de congruence dans le chapitre de Th\'eorie Des Nombres page \pageref{congruence})!
	
	Voyons cela de suite (le vocabulaire qui va suivre est aussi d\'efini dans la section de Th\'eorie Des Ensembles page \pageref{surjective application}):
	
	\textbf{D\'efinition (\#\mydef):} Soit une relation binaire $\mathcal{R}$ d'un ensemble $A$ vers lui-même, une relation $\mathcal{R}$ dans $A$ est un sous-ensemble du produit cart\'esien $\mathcal{R}\subseteq A\times A$ (c'est-à-dire que la relation binaire engendre un sous-ensemble de par les contraintes qu'elle impose aux \'el\'ements de $A$ qui satisfont la relation) avec la propri\'et\'e d'être:
	\begin{enumerate}\label{strict order}
		\item[P1.] Une "\NewTerm{relation r\'eflexive}\index{relation r\'eflexive}\label{reflexive}" si $\forall x \in A$:
		
		
		\item[P2.] Une "\NewTerm{relation sym\'etrique}\index{relation sym\'etrique}" si $\forall x,y \in A$:
		
		
		\item[P3.] Une "\NewTerm{relation antisym\'etrique}\index{relation antisym\'etrique}" si $\forall x,y \in A$:
		
		
		\item[P4.] Une "\NewTerm{relation transitive}\index{relation transitive}" si $\forall x,y,z \in A$:
		
		
		\item[P5.] Une "\NewTerm{relation connexe}\index{relation connexe}" si $\forall x,y \in A$:
		
	\end{enumerate}
	Les math\'ematiciens ont donn\'e des noms particuliers aux familles de relations satisfaisant certaines de ces propri\'et\'es.
	
	\textbf{D\'efinitions (\#\mydef):}
	\begin{enumerate}
		\item[D1.] Une relation est appel\'ee "\NewTerm{relation d'ordre stricte}\index{relation d'ordre stricte}" si et seulement si elle est uniquement transitive (certains sp\'ecifient alors qu'elle est donc forc\'ement antir\'eflexive mais on s'en doute...).
		
		\item[D2.] Une relation est appel\'ee un "\NewTerm{pr\'e-ordre}\index{pr\'e-ordre}" si et seulement si elle est r\'eflexive et transitive.
		
		\item[D3.] Une relation est appel\'ee "\NewTerm{relation d'\'equivalence}\index{relation d'\'equivalence}\label{equivalence relation}" si et seulement si elle est r\'eflexive, sym\'etrique et transitive.
		
		\item[D4.] Une relation est appel\'ee "\NewTerm{relation d'ordre}\index{relation d'ordre}\label{order relation}" si et seulement si elle est r\'eflexive, transitive et antisym\'etrique (donc les relations $>$, $<$ ne sont pas des relations d'ordre car non r\'eflexives!).
		
		\item[D5.] Une relation est appel\'ee "\NewTerm{relation d'ordre total}\index{relation d'ordre total}\label{total order relation}" si et seulement si elle est r\'eflexive, transitive, connexe et antisym\'etrique.
	\end{enumerate}
	Pour les autres combinaisons il semblerait (?) qu'il n'y ait pas de d\'esignations particulières chez les math\'ematiciens...

	\begin{tcolorbox}[title=Remarque,colframe=black,arc=10pt]
	Les relations d'ordre binaire ont toutes des propri\'et\'es similaires dans les ensembles naturels $\mathbb{N}$, rationnels $\mathbb{Q}$, relatifs $\mathbb{Z}$ et r\'eels $\mathbb{R}$ (il n'y a pas de relation d'ordre naturelle sur l'ensemble des nombres complexes).
	\end{tcolorbox}
	\begin{tcolorbox}[colframe=black,colback=white,sharp corners]
	\textbf{{\Large \ding{45}}Exemple:}\\\\
	Sur l'ensemble ci-dessous de huit livres, la relation "... a le même num\'ero ISBN que ..." est une relationd d'\'equivalence:
	\begin{figure}[H]
		\centering
		\includegraphics[width=0.8\textwidth]{img/arithmetics/equivalence_relation.jpg}
		\caption[]{Illustration d'une relation d'\'equivalence (source: Wikip\'edia, auteur: Stephan Kulla)}
	\end{figure}
	\end{tcolorbox}
	Si nous r\'esumons:
	\begin{table}[H]\centering\small
		\renewcommand{\arraystretch}{1.2}
		\begin{tabular}{lcccccc}\hline
		\textsc{Relation binaire} & = & $\neq$ & > & <& $\leqslant$ & $\geqslant$ \\ \hline
		r\'eflexive & oui & non & non & non & oui &  oui \\
		sym\'etrique 	& oui  & oui & non & non & non & non \\
		transitive 	& oui & non & oui & oui & oui & oui \\
		connexe & non & non & non & non & oui & oui \\
		antisym\'etrique 	& oui & non & non & non & oui & oui \\ \hline
		\end{tabular}
		\caption{Relation binaires}
	\end{table}
	Ainsi, nous voyons que les relations binaires  $\leq, \geq$ forment avec les ensembles pr\'ecit\'es, des relations d'ordre total et qu'il est très facile de voir quelles relations binaires sont des relations d'ordre partiel, total ou d'\'equivalence.
	
	\textbf{D\'efinition (\#\mydef):} Si $\mathcal{R}$ est une relation d'\'equivalence sur $A$. Pour  $\forall x\in A$, la "\NewTerm{classe d'\'equivalence}\index{classe d'\'equivalence}\label{equivalence class}" de $x$ est par d\'efinition l'ensemble:
	
	$[x]$ est donc un sous-ensemble de $A$ ($x \subseteq A$) que nous noterons aussi... par la suite $\mathcal{R}$.
	
	Nous disposons ainsi d'un nouvel ensemble qui est "\NewTerm{l'ensemble des classes d'\'equivalences}\index{l'ensemble des classes d'\'equivalences}" ou "\NewTerm{ensemble quotient}\index{ensemble quotient}\label{quotient set}" not\'e $A/\mathcal{R}$. Ainsi:
	
	Il faut savoir que dans $A/\mathcal{R}$ nous ne regardons plus $[x]$ comme un sous-ensemble de $A$ mais comme un \'el\'ement!
	
	Une relation d'\'equivalence, de manière vulgaris\'ee sert donc à coller une seule \'etiquette à des \'el\'ements qui v\'erifient une même propri\'et\'e, et à les confondre avec ladite \'etiquette (en sachant ce que nous faisons avec cette \'etiquette).
	
	Une relation d'\'equivalence, de manière vulgaris\'ee sert donc à coller une seule \'etiquette à des \'el\'ements qui v\'erifient une même propri\'et\'e, et à les confondre avec ladite \'etiquette (en sachant ce que nous faisons avec cette \'etiquette).
	
	\begin{tcolorbox}[colframe=black,colback=white,sharp corners]
	\textbf{{\Large \ding{45}}Exemples:}\\\\
	E1. Ensemble des classes d'\'equivalences de l'illustration pr\'ec\'edente avec les num\'eros ISBN:
	\begin{figure}[H]
		\centering
		\includegraphics[width=0.8\textwidth]{img/arithmetics/class_of_equivalence.jpg}
		\caption[]{Illustration des ensembles de classes d'\'equivalences (source: Wikip\'edia, auteur: Stephan Kulla)}
	\end{figure}
	E2. Dans l'ensemble des entiers relatifs $\mathbb{Z}$, si nous \'etudions les restes de la division par $2$, nous avons que ceux-ci valent toujours soit $0$ soit $1$.\\
	\end{tcolorbox}
	
	\begin{tcolorbox}[colframe=black,colback=white,sharp corners]
	La classe d'\'equivalence de z\'ero est alors appel\'ee "l'ensemble des nombres entiers pairs", la classe d'\'equivalence de $1$ est appel\'ee "l'ensemble des entiers impairs". Nous avons donc deux classes d'\'equivalences pour deux partitions de $\mathbb{Z}$ (gardez toujours cet exemple simple en tête pour les \'el\'ements th\'eoriques qui suivront cela aide \'enorm\'ement).\\
	
	Si nous nommons la première $0$ et la deuxième $1$, nous retrouvons les règles d'op\'erations entre nombres pairs et impairs:
	
	ce qui signifie respectivement que la somme de deux entiers pairs est paire, que la somme d'un pair et d'un impair est impaire et que la somme de deux impairs est paire.\\
	
	Et pour la multiplication:
	
	ce qui signifie respectivement que le produit de deux pairs est pair, le produit d'un pair et d'un impair est pair et que le produit de deux impairs est impair.\\
	
	Et hop, nous avons d\'eplac\'e les op\'erations de $\mathbb{Z}$ sur cet ensemble quotient not\'e $\mathbb{Z}/2\mathbb{Z}$.\\
	
	Maintenant, pour v\'erifier que nous avons bien affaire à une relation d'\'equivalence, il faudrait encore v\'erifier qu'elle est r\'eflexive ($x\mathcal{R}x$), sym\'etrique (si $x\mathcal{R}y$ alors $y\mathcal{R}x$) et transitive (si $x\mathcal{R}y$  et $y\mathcal{R}z$ alors $x\mathcal{R}z$). Nous verrons comment v\'erifier cela quelques paragraphes plus loin car cet exemple constitue un cas très particulier de relation de congruence.
	\end{tcolorbox}
	
	\textbf{D\'efinition (\#\mydef):}  L'application $f:A\mapsto A/\mathcal{R}$ d\'efinie par $x\mapsto [x]$ est appel\'ee "\NewTerm{projection canonique}\index{projection canonique}". Tout \'el\'ement $z\in [x]$ est alors appel\'e "\NewTerm{repr\'esentant de la classe}\index{repr\'esentant de la classe}" $[x]$.
	
	\begin{theorem}
	Consid\'erons maintenant un ensemble $E$. Alors nous proposons de d\'emontrer qu'il y a bijection entre l'ensemble des relations d'\'equivalence sur $E$ et l'ensemble des partitions de $E$. En d'autres termes cette proposition dit qu'une relation d'\'equivalence sur $E$ n'est rien d'autre qu'une partition de $E$.
	\end{theorem}
	\begin{dem}
	Soit $R$ une relation d'\'equivalence sur $E$. Nous choisissons  $I=E/\mathcal{R}$ comme ensemble d'indexation des partitions et nous posons pour tout $[x]\in E/\mathcal{R}$, $E_{[x]}=[x]$.
	
	Il suffit de v\'erifier les deux propri\'et\'es suivantes de la d\'efinition des partitions pour montrer que la famille $\left(E_{[x]}\right)_I$ est une partition de $E$:
	
	\begin{enumerate}
		\item[P1.] Soient  $[x],[y]\in E/\mathcal{R}$  tels que $[x]\neq [y]$ alors (trivial)  $E_{[x]}\cap E_{[y]}=\varnothing$.
		
		\item[P2.] $E=\displaystyle\bigcup_{[x]\in E/\mathcal{R}}$ est \'evident car si $x\in E$ alors $x\in [x]=E_{[x]}$.
	\end{enumerate}
	\begin{flushright}
		$\blacksquare$  Q.E.D.
	\end{flushright}
	\end{dem}
	Encore une fois, il est ais\'e de v\'erifier avec l'exemple pratique de la division par 2 donn\'e plus haut que la partition des nombres pairs et impairs satisfait ces deux propri\'et\'es.

	Nous avons donc associ\'e à la relation d'\'equivalence $R$ une partition de $E$. R\'eciproquement si $(E_i)_I$ est une partition de $E$ alors nous v\'erifions facilement que la relation $R$ d\'efinie par $xRy$ si et seulement s'il existe $i \in I$ tel que $x,y \in E_i$ est une relation d'\'equivalence! Les deux applications ainsi d\'efinies sont bijectives et r\'eciproques l'une de l'autre.	
	\begin{tcolorbox}[colframe=black,colback=white,sharp corners]
	\textbf{{\Large \ding{45}}Exemple:}\\\\
	Nous allons à pr\'esent appliquer sur un exemple un peu moins trivial que le pr\'ec\'edent ce que nous venons de voir à la construction des anneaux  $\mathbb{Z}/d\mathbb{Z}$ après quelques rappels (pour le concept d'anneau voir la section de Th\'eorie Des Ensembles page \pageref{ring}).
	
	Rappels:
	\begin{enumerate}
		\item Soit deux nombres $n,m\in \mathbb{Z}$. Nous disons que "\NewTerm{$n$ divise $m$}\index{divise}" et nous \'ecrivons $n|m$ si et seulement si il existe un entier $k\in \mathbb{Z}$ tel que $m=k\cdot n$ (\SeeChapter{voir section Th\'eorie des Nombres page \pageref{division}}).
		
		\item Soit $d\geq 1$ un entier. Nous d\'efinissons la relation $\mathcal{R}$ par $n\mathcal{R}m$ si et seulement si $d|(n-m)$ ou dit autrement $n\mathcal{R}m$ si et seulement s'il existe $d\in\mathbb{Z}$ tel que $n=m+kd$. G\'en\'eralement nous \'ecrivons ceci aussi  $n\equiv m\; (\text{modulo } d)$ au lieu de $n\mathcal{R}m$ et nous disons que "\NewTerm{$n$ est congru à $m$ modulo $d$}\index{congruent}". Rappelons aussi que $n\equiv 0\; (\text{modulo } d)$ si et seulement si  $d$ divise  $n$ (\SeeChapter{voir section Th\'eorie des Nombres page \pageref{congruence}}).
	\end{enumerate}
	Nous allons maintenant introduire une relation d'\'equivalence sur $\mathbb{Z}$ . D\'emontrons que pour tout entier $d\geq 1$, la congruence modulo $d$ est une relation d'\'equivalence sur equation (nous avons d\'ejà d\'emontr\'e cela dans la section de Th\'eorie des Nombres lors de notre \'etude de la congruence mais refaisons le travail pour le plaisir...).\\
	
	Pour prouver cela nous avons simplement besoin de contrôler que les trois propri\'et\'es de la relation d'\'equivalence existent:
	\begin{enumerate}
		\item[P1.] R\'eflexivit\'e: $n\equiv n$ puisque $n=n+0d$.
		
		\item[P2.] Sym\'etrie: Si $n \equiv m$ alors $n=m+kd$ et donc $m=n+(-k)d$ c'est-à-dire $m\equiv n$.
		
		\item[P3.] Transitivit\'e: Si $n\equiv m$ et $m\equiv j$ alors $n=m+kd$ et $m=j+k'd$ dès lors $n=j+(k+k')d$ c'est-à-dire $n\equiv j$.
	\end{enumerate}
	Dans la situation ci-dessus, nous notons $\mathbb{Z}/d\mathbb{Z}$ l'ensemble des classes d'\'equivalence et noterons $[n]_d$ la classe d'\'equivalence de la congruence d'un entier $n$ donn\'ee par:
	
	(chaque diff\'erence de deux valeurs se trouvant dans les accolades est divisible par $d$ et c'est bien ainsi une classe d'\'equivalence) et ainsi:
	
	En particulier (trivial car nous obtenons ainsi tout $\mathbb{Z}$):
	
	\end{tcolorbox}
	Ainsi, nous voyons que le premier exemple que nous avions donn\'e avec les nombres pairs et impairs est un cas particulièrement simple des classes d'\'equivalence de congruence modulo $2$ car elles se r\'eduisent toutes à seulement deux classes.
	
	\begin{tcolorbox}[title=Remarque,colframe=black,arc=10pt]
	 Les op\'erations d'addition et de multiplication d\'efinies sur $\mathbb{Z}$ d\'efinissent des op\'erations d'addition et de multiplication sur $\mathbb{Z}/d\mathbb{Z}$. Nous disons alors que ces op\'erations sont compatibles avec la relation d'\'equivalence et forment alors un anneau (\SeeChapter{voir section Th\'eorie des Ensembles page \pageref{ring}}).
	\end{tcolorbox}
	
	\pagebreak
	\subsection{Lois Fondamentales de l'Arithm\'etique}
	Comme nous l'avons d\'ejà dit pr\'ec\'edemment, il existe un op\'erateur de base (addition) à partir duquel il possible de d\'efinir la multiplication, la soustraction (à condition que l'ensemble de nombres soit ad hoc) et la division (même remarque que pour la soustraction) et autour desquels nous pouvons construire toute la Math\'ematique Analytique.

	Bien \'evidemment il y a certaines subtilit\'es à prendre en compte lorsque le niveau de rigueur augmente. Le lecteur peut alors se reporter au chapitre de Th\'eorie Des Ensembles (page \pageref{set theory}) où ses lois fondamentales sont red\'efinies avec plus de justesse.
	
	\begin{figure}[H]
		\centering
		\includegraphics{img/arithmetics/fundamentals_delucq.jpg}
	\end{figure}
	
	\subsubsection{Addition}
	\textbf{D\'efinition (\#\mydef):}  "\NewTerm{L'addition}\index{addition}\label{addition}" de nombres entiers est une op\'eration not\'ee "+" qui a pour seul but de r\'eunir en un seul nombre toutes les unit\'es contenues dans plusieurs autres. Le r\'esultat de l'op\'eration se nomme "\NewTerm{sum}\index{sum}" ou "\NewTerm{total}" ou encore "\NewTerm{cumul}". Les nombres à additionner sont appel\'es "\NewTerm{termes de l'addition}\index{termes de l'addition}".
	
	\begin{tcolorbox}[title=Remarque,colframe=black,arc=10pt]
	Les signes d'addition "$+$" et de soustraction "$-$" seraient dus au math\'ematicien allemand Johannes Widmann (1489)
	\end{tcolorbox}
	Ainsi, $A + B + C ...$ sont les termes de l'addition et le r\'esultat est la somme des termes de l'addition.
	
	Ou sous forme sch\'ematique d'un cas particulier:
	
	\begin{figure}[H]
		\centering
		\includegraphics{img/arithmetics/addition.jpg}
		\caption{Exemple sch\'ematique d'un cas particulier d'addition}
	\end{figure}
	Voici une liste de quelques propri\'et\'es intuitives que nous admettrons sans d\'emonstrations de l'op\'eration de l'addition:
	\begin{enumerate}
		\item[P1.] La somme de plusieurs nombres ne d\'epend pas de l'ordre des termes. Nous disons alors que l'addition est une "\NewTerm{op\'eration commutative}\index{op\'eration commutative}". Ce qui signifie concr\'etement pour deux nombres quelconques:
		
		
		\item[P2.]  La somme de plusieurs nombres ne change pas si nous remplaçons deux ou plusieurs d'entre eux par leur r\'esultat interm\'ediaire. Nous disons alors que l'addition est "\NewTerm{op\'eration associative}\index{op\'eration associative}".
		
		
		\item[P3.] Le z\'ero est l'\'el\'ement neutre de l'addition car tout nombre additionn\'e à z\'ero donne ce même nombre:
		
		
		\item[P4.] Suivant l'ensemble dans lequel nous travaillons ($\mathbb{Z},\mathbb{Q},\mathbb{R},...$), l'addition peut comporter un terme de telle façon à ce que le total soit nul. Nous disons alors qu'il existe un "\NewTerm{oppos\'e}\index{oppos\'e}" pour l'addition:
		
	\end{enumerate}
	Nous allons d\'efinir plus rigoureusement l'addition en utilisant l'axiomatique de Peano dans le cas particulier de l'ensemble des nombres entiers naturels comme nous en avons d\'ejà fait mention dans la section traitant des Nombres. Ainsi, avec ces axiomes il est possible de d\'emontrer qu'il existe (existence) une et une seule application (unicit\'e), not\'ee "$+$", de $\mathbb{N}\times \mathbb{N}$ dans $\mathbb{N}$ v\'erifiant:
	
	où $s$ signifie "successeur".
	\begin{tcolorbox}[title=Remarque,colframe=black,arc=10pt]
	Ce site n'ayant pas pour vocation de s'adresser à des math\'ematiciens, nous nous passerons de la d\'emonstration (relativement longue) et admettrons intuitivement que l'application "$+$" existe et est unique... et qu'il en d\'ecoule les propri\'et\'es susmentionn\'ees.
	\end{tcolorbox}
	Soient  $x_1,x_2,...,x_n$ des nombres quelconques alors nous pouvons noter \'egalement la somme ainsi:
	
	en d\'efinissant des bornes sup\'erieure et inf\'erieure à la somme index\'ee (au-dessus et en-dessous de la lettre grecque majuscule "sigma").

	Voici quelques rappels des propri\'et\'es relatives à cette notation condens\'ee:
	
	où $k$ est une constante et:
	
	Voyons maintenant quelques cas concrets d'additions de diff\'erents nombres simples afin de mettre en pratique les bases.
	\begin{tcolorbox}[colframe=black,colback=white,sharp corners]
	\textbf{{\Large \ding{45}}Exemples:}\\\\
	L'addition de deux nombres relativement petits est assez facile dès que nous avons appris par coeur à compter jusqu'au nombre r\'esultant de cette op\'eration. Ainsi (exemples pris sur la base d\'ecimale):
	
	et:
	
	et:
	
	Pour les plus grands nombres il faut adopter une autre m\'ethode qu'il s'agit d'apprendre par coeur. Ainsi par exemple:
	
	L'algorithme (processus) est le suivant: nous additionnons les colonnes ($4$ colonnes dans cet exemple) de droite à gauche. Pour la première colonne nous avons donc $4+5=9$ ce qui nous donne:
	\end{tcolorbox}
	
	\begin{tcolorbox}[colframe=black,colback=white,sharp corners]
	
	et nous continuons ainsi pour la deuxième $4+7=11$ mais à la diff\'erence que comme nous avons un nombre sup\'erieur à $>10$, nous reportons le premier chiffre (de gauche) sur la colonne suivante de l'addition. Ainsi:
	
	La troisième colonne se calcule dès lors comme $4+2+1=7$ ce qui nous donne:
	
	Pour la dernière colonne nous avons $9+5=14$ et à nouveau nous reportons le premier chiffre (de gauche) sur la colonne suivante de l'addition. Ainsi:
	
	Et finalement la dernière colonne donne:
	
	\end{tcolorbox}
	Cet exemple montre comment nous pouvons proc\'eder pour l'addition de nombres quelconques: nous faisons une addition par colonne de droite à gauche et si le r\'esultat d'une addition est sup\'erieur à la dizaine, nous reportons une unit\'e sur la colonne suivante.

	Suite à une demande d'une lecteur, voici \'egalement un exemple illustr\'e sur la façon dont nous enseignons l’addition de fractions simples au niveau du collège, d’abord dans le cas où les d\'enominateurs sont les mêmes:
	\begin{figure}[H]
		\centering
		\includegraphics[width=1.0\textwidth]{img/arithmetics/fraction_add_common_denominator.jpg}
		\caption{Addition de fractions avec d\'enominateurs communs}
	\end{figure}
	Ou si les d\'enominateurs ne sont pas communs, nous recherchons des fractions communes \'equivalentes à chaque terme de l'addition, de manière à ce que toutes les fractions aient un d\'enominateur communm et nous ajoutons ensuite comme nous le faisions auparavant:
	\begin{figure}[H]
		\centering
		\includegraphics[width=1.0\textwidth]{img/arithmetics/fraction_add_not_common_denominator.jpg}
		\caption{Addition de fractions sans d\'enominateurs communs}
	\end{figure}
	Nous voyons ensuite ci-dessus que pour ajouter une fraction à autre, nous la mettons d'abord au même d\'enominateur, puis nous ajoutons les num\'erateurs et nous conservons le d\'enominateur s'il n'y a pas de simplification possible:
	
	\'evidemment, lorsque nous additionnons des fractions, nous pourrions \'egalement obtenir un nombre repr\'esentant un autre nombre sup\'erieur à $1$, comme par exemple:
	
	Ceci peut être illustr\'e comme suit:
	\begin{figure}[H]
		\centering
		\includegraphics[width=1.0\textwidth]{img/arithmetics/fraction_addition_greater_unit_parts_01.jpg}
	\end{figure}
	Si vous voulez assembler ces quarts de disques, voici ce que nous obtenons:
	\begin{figure}[H]
		\centering
		\includegraphics{img/arithmetics/fraction_addition_greater_unit_parts_02.jpg}
	\end{figure}
	Vous comprenez alors pourquoi on peut dire que nous avons $5/4$ de disque!

	Cet algorithme (processus ou m\'ethodologie) d'addition est assez simple à comprendre et à ex\'ecuter. Nous n'irons pas plus loin sur ce sujet du moins à ce jour!
	
	\pagebreak
	\subsubsection{Soustraction}
	\textbf{D\'efinition (\#\mydef):}  La "\NewTerm{soustraction}\index{soustraction}" est une op\'eration math\'ematique qui repr\'esente le fait de retirer des entit\'es d'une collection. Plus formellement, la soustraction du nombre entier $A$ par le nombre entier $B$ not\'ee par le symbole "$-$", c'est trouver le nombre $C$ qui, ajout\'e à $B$, redonne $A$. 

	\begin{tcolorbox}[title=Remarque,colframe=black,arc=10pt]
	Comme nous l'avions vu dans la section de Th\'eorie des Ensembles (page \pageref{internal composition law}), la soustraction dans l'ensemble $\mathbb{N}$ est possible seulement si $A>B$.
	\end{tcolorbox}
	
	Nous \'ecrivons la soustraction sous la forme:
	
	qui doit v\'erifier:
	
	Ou sous forme sch\'ematique avec un cas particulier:
	
	\begin{figure}[H]
		\centering
		\includegraphics{img/arithmetics/subtraction.jpg}
		\caption{Un sch\'ema possible pour la soustraction}
	\end{figure}
	Voici quelques propri\'et\'es intuitives que nous admettrons sans d\'emonstrations de l'op\'eration de soustraction (bon cela d\'ecoule de l'addition de nombres n\'egatifs en r\'ealit\'e...):
	\begin{enumerate}
		\item[P1.] La soustraction de plusieurs nombres d\'epend de l'ordre des termes. Nous disons alors que la soustraction est une "\NewTerm{op\'eration non-commutative}\index{op\'eration non-commutative}". Effectivement:
		
		
		\item[P2.] La soustraction de plusieurs nombres change si nous remplaçons deux ou plusieurs d'entre eux par leur r\'esultat interm\'ediaire. Nous disons alors que la soustraction est une "\NewTerm{op\'eration non-associative}\index{op\'eration non-associative}". Effectivement:
		
		
		\item[P3.] Le z\'ero n'est pas l'\'el\'ement neutre de la soustraction. Effectivement, tout nombre à qui nous soustrayons z\'ero donne ce même nombre, donc le z\'ero est neutre à droite... mais pas à gauche car tout nombre que nous soustrayons à z\'ero ne donne pas z\'ero! Nous disons alors que le z\'ero est seulement "\NewTerm{neutre à droite}\index{neutre à droite}" dans le cas de la soustraction. Effectivement:
		
		
		\item[P4.] Suivant l'ensemble dans lequel nous travaillons, la soustraction peut comporter un terme de telle façon à ce que le total soit nul. Nous disons alors qu'il existe un "\NewTerm{oppos\'e}\index{oppos\'e}" pour la soustraction.
	\end{enumerate}
	
	Dans les cas les plus compliqu\'es, nous avons un vocabulaire sp\'ecial:
	
	Le "\NewTerm{diminuende}\index{diminuende}" est $704$, le "\NewTerm{diminuteur}\index{diminuteur}" est $512$. Les digits du diminuende sont $m_3= 7$, $m_2 = 0$ et $m_1 = 4$. Les digits du diminuteur sont $s_3 = 5$, $s_2 = 1$ et $s_1 = 2$. À la position des unit\'es, $4$ n'est pas inf\'eriereur $2$, la diff\'erence $2$ est donc inscrite à la ligne du reste. À la position des dizaines, $0$ est inf\'erieur à $1$, de sorte que le $0$ est augment\'e de $10$ et que la diff\'erence avec $1$, soit $9$, est report\'ee dans la reste à la position des dizaines. La m\'ethode de soustraction am\'ericaine corrige l'augmentation de dix en r\'eduisant le chiffre de la place des centaines dans la diminuend de $1$. Autrement dit, le $7$ est ray\'e et remplac\'e par la valeur $6$. La soustraction continue ensuite à la position des centaines, où $6$ n'est pas inf\'erieur à $5$, de sorte que la diff\'erence est inscrite à la position des centaiens du reste. Nous avons maintenant termin\'e, le r\'esultat est $192$.
	
	Voyons maintenant quelques exemples concrets de soustraction de divers nombres simples dans le but de pratiquer la base:
	\begin{tcolorbox}[colframe=black,colback=white,sharp corners]
	\textbf{{\Large \ding{45}}Exemple:}\\\\
	La soustraction de deux nombres relativement petits est assez facile dès que nous avons appris par coeur à compter jusqu'à au moins le nombre r\'esultant de cette op\'eration. Ainsi:
	
	et:
	
	et:
	
	Pour les plus grands nombres il faut adopter une autre m\'ethode qu'il s'agit d'apprendre par coeur (au même titre que l'addition). Ainsi par exemple:
	
	\end{tcolorbox}
	\begin{tcolorbox}[colframe=black,colback=white,sharp corners]
	
	nous soustrayons les colonnes ($4$ colonnes dans cet exemple) de droite à gauche. Pour la première colonne nous avons $4-5=-1<0$ ce qui fait que nous reportons $-1$ sur la colonne suivante (deuxième) et \'ecrivons $10-1=9$ en bas de la barre d'\'egalit\'e de la première colonne:
	
	et nous continuons ainsi pour la deuxième $7-8=-1<0$ ce qui fait que nous reportons $-1$ sur la colonne suivante (troisième) et comme $-1-1=-2$ nous reportons $10-2=8$ en bas de la barre d'\'egalit\'e de la deuxième colonne:
	
	La troisième colonne se calcule dès lors comme $5-7=-2<0$ et nous reportons $-1$ sur la colonne suivante (quatrième) et comme $-1-2=-3$ nous reportons $10-3=7$ en bas de la barre d'\'egalit\'e de la troisième colonne:
	
	Pour la dernière colonne nous avons $4-3=1>0$ nous reportons donc rien sur la colonne suivante et comme $1-1=0$ nous reportons $0$  en bas de la barre d'\'egalit\'e de la quatrième colonne:
	
	\end{tcolorbox}
	Voilà comment nous proc\'edons donc pour la soustraction de nombres quelconques. Nous faisons une soustraction par colonne de droite à gauche et si le r\'esultat d'une soustraction est inf\'erieur à z\'ero nous faisons reporter $-1$ sur la colonne suivante et l'addition du dernier report sur la soustraction obtenue en bas de la barre d'\'egalit\'e.
	
	
	Suite à une demande d'un lecteur, voici \'egalement un exemple illustr\'e sur la façon dont nous enseignons à soustraire des fractions simples du niveau du collège (en France):
	 \begin{figure}[H]
		\centering
		\includegraphics[width=1.0\textwidth]{img/arithmetics/fraction_subtract_not_common_denominator.jpg}
		\caption{Soustraction de fractions sans d\'enominateur commun}
	\end{figure}
	Nous voyons ensuite ci-dessus que pour soustraire une fraction à une autre, nous mettons d'abord les fractions au même d\'enominateur, puis nous soustrayons les num\'erateurs et nous conservons le d\'enominateur s'il n'y a pas de simplification possible:
	
	Nous avons lorsque nous m\'elangeons l'addition et la soustraction les relations suivantes qui en d\'ecoulent:
	
	La m\'ethodologie utilis\'ee pour la soustraction se basant sur exactement le même principe que l'addition nous ne nous \'etendrons pas plus sur le sujet. Cette m\'ethode est très simple et n\'ecessite bien sûr une certaine habitude à travailler avec les chiffres pour être totalement appr\'ehend\'ee.
	
	\subsubsection{Multiplication}
	\textbf{D\'efinition (\#\mydef):} La "\NewTerm{multiplication}\index{multiplication}\label{multiplication}" de nombres est une op\'eration qui a pour but, \'etant donn\'e deux nombres, l'un appel\'e "\NewTerm{multiplicateur}\index{multiplicateur}" $m$, et l'autre "\NewTerm{multiplicande}\index{multiplicande}" $M$, d'en trouver un troisième appel\'e "\NewTerm{produit}\index{produit}" $P$ qui soit la somme (donc la multiplication d'\'ecoule de la somme!) d'autant de nombres \'egaux au multiplicande qu'il y a d'unit\'es au multiplicateur:
	
	Le multiplicande et le multiplicateur sont appel\'es les "\NewTerm{facteurs du produit}\index{facteurs du produit}".
	
	La multiplication s'indique à l'aide du signe "$\times$" (anciennement) ou du point de ponctuation sur\'elev\'e "$\cdot$" (notation moderne) ou sans aucun symbole s'il n'y pas de confusion possible... tel que:
	
	Nous pouvons d\'efinir la multiplication en utilisant l'axiomatique de Peano dans le cas particulier des nombres entiers naturels $\mathbb{N}$ comme nous en avons d\'ejà fait mention dans la section traitant des Nombres (page \pageref{peano axioms}). Ainsi, avec ces axiomes il est possible de d\'emontrer qu'il existe (existence) une et une seule application (unicit\'e), not\'ee "$\times$" ou plus souvent "$\cdot$", de $\mathbb{N}^2$ dans $\mathbb{N}$ v\'erifiant:
	
	\begin{tcolorbox}[title=Remarque,colframe=black,arc=10pt]
	Ce site n'ayant pas pour vocation de s'adresser à des math\'ematiciens, nous nous passerons de la d\'emonstration (relativement longue) et admettrons intuitivement que l'application "$\times $" existe et est unique...
	\end{tcolorbox}
	La "\NewTerm{puissance}\index{puissance}" est une notation particulière d'un cas pr\'ecis de multiplications. Lorsque le(s) multiplicateur(s) et multiplicande(s) sont identique(s) en valeur num\'erique, nous notons la multiplication (par exemple):
	
	c'est ce que nous nommons la "\NewTerm{notation en puissance}\index{notation en puissance}" ou "\NewTerm{l'exponentiation}\index{exponentation}". Le nombre en exposant est ce que nous nommons la "\NewTerm{puissance}\index{puissance}" ou "\NewTerm{l'exposant}\index{exposant}" du nombre ($n$ en l'occurrence). La notation en exposants se trouve pour la première fois dans l'ouvrage de Chuquet intitul\'e "\textit{Triparty en la science des nombres}" (1484).
	
	Vous pouvez v\'erifier par vous-même que ses propri\'et\'es sont les suivantes (par exemple):
	
	de même:
	
	Voici quelques propri\'et\'es intuitives que nous admettrons sans d\'emonstrations de l'op\'eration de multiplication:
	\begin{enumerate}
		\item[P1.] La multiplication de plusieurs nombres ne d\'epend pas de l'ordre des termes. Nous disons alors que la multiplication est une "\NewTerm{op\'eration commutative}\index{op\'eration commutative}".
		
		\item[P2.] La multiplication de plusieurs nombres ne change pas si nous remplaçons deux ou plusieurs d'entre eux par leur r\'esultat interm\'ediaire. Nous disons alors que la multiplication est "\NewTerm{op\'eration associative}\index{op\'eration associative}".
		
		\item[P3.] L'unit\'e est l'\'el\'ement neutre de la multiplication car tout multiplicande multipli\'e par le multiplicateur $1$ est \'egal au multiplicande.
		
		\item[P4.] La multiplication peut comporter un terme de telle façon à ce que le produit soit \'egal à l'unit\'e (l'\'el\'ement neutre). Nous disons alors qu'il existe un "\NewTerm{inverse pour la multiplication}\index{inverse pour la multiplication}" (mais cela d\'epend rigoureusement dans quel ensemble de nombres nous travaillons puisque dans certains ensemble le concept de nombre d\'ecimal n'existe pas!).
		
		\item[P5.] La multiplication est "\NewTerm{op\'eration distributive}\index{op\'eration distributive}", c'est-à-dire que:
		
		l'op\'eration inverse s'appelant la "\NewTerm{factorisation}\index{factorisation}".
		
		\item[P6.] Tout nombre à la puissance zero est égal $1$ étant donné que:
		
		De ce résultat certains déduisent trop vite (et même certains professeurs l'enseignent !) que $0^0=1$. Mais en réalité, $0^0$ n'est tout simplement pas défini.
		\begin{tcolorbox}[title=Remarque,colframe=black,arc=10pt]
		Il y a deux raisons principales pour lesquelles certains professeurs enseignent $0^0=1$. La première est une application du théorème du binôme (\SeeChapter{voir section Calcul Algébrique page \pageref{binomial theorem}}):
		$$
		(1+x)^{n}=\sum_{k=0}^{n}\left(\begin{array}{l}
		n \\
		k
		\end{array}\right) 1^{n-k} x^{k}
		$$
		Quand $x=0$, cette expression devient:
		$$
		\begin{aligned}
		(1+0)^{n} &=\sum_{k=0}^{n}\left(\begin{array}{l}
		n \\
		k
		\end{array}\right) 1^{n-k} 0^{k} \\
		1^{n} &=1 \cdot 0^{0} \\
		1 &=1 \cdot 0^{0}
		\end{aligned}
		$$
		Dans ce contexte, $0^{0}$ doit être égal à $1$.\\
		
		La seconde raison part du raisonnement selon lequel $1^1$ vaut $1$, $0.5^{0.5}$ vaut $0.70$ etc. $x^x$ descend diminue jusqu'à ce qu'il atteigne $0.4$, puis il remonte et approche la limite de un mais sans jamais l'atteindre. C'est-à-dire que $0.000000001^{0.000000001}$ équivaut à presque $1$, donc certaines personnes arrondissent $0^0$ à $1$.\\
		
		Pour résumer, il semble qu'il n'y ait pas de convention internationale à notre connaissance.
		\end{tcolorbox}
	\end{enumerate}
	Introduisons encore quelques notations particulières relatives à la multiplication:
	\begin{enumerate}
		\item  Soient $x_1,x_2,...,x_n$ des nombres quelconques (non n\'ecessairement \'egaux) alors nous pouvons noter le produit ainsi:
		
		en d\'efinissant des bornes sup\'erieure et inf\'erieure au produit index\'e (au-dessus et en-dessous de la lettre grecque majuscule "Pi").
		
		Nous avons triviallement relativement à cette notation (sur demande nous pouvons d\'etailler plus...):
		
		pour tout nombre $k$ tel que:
		
		Nous avons aussi par exemple:
		
		
		\item  Nous d\'efinissons \'egalement la "\NewTerm{factorielle}\index{factorielle}" simplement (car il existe aussi une manière complexe de la d\'efinir en passant par la fonction Gamma d'Euler comme cela est fait dans la section de Calcul Diff\'erentiel Et Int\'egral page \pageref{gamma euler function}) par:
		
		avec le fait particulier que (seule la d\'efinition complexe mentionn\'ee pr\'ec\'edemment peut rendre ce fait \'evident ...):
		
	\end{enumerate}
	Voyons quelques exemples simples de multiplications:
	\begin{tcolorbox}[colframe=black,colback=white,sharp corners]
	\textbf{{\Large \ding{45}}Exemples:}\\\\
	E1. La multiplication de deux nombres relativement petits est assez facile dès que nous avons appris par coeur à compter jusqu'à au moins le nombre r\'esultant de cette op\'eration. Ainsi:
	
	E2. Pour les beaucoup plus grands nombres il faut adopter une autre m\'ethode qu'il s'agit d'apprendre par coeur.\\
	
	 Ainsi par exemple:
	
	Cette m\'ethodologie est très logique si vous comprenez comment nous construisons un nombre en base dix. Ainsi nous avons (nous supposerons que la propri\'et\'e distributive est maîtris\'ee):
	
	Pour ne pas surcharger l'\'ecriture dans la multiplication par la m\'ethode "verticale", nous ne repr\'esentons pas les z\'eros qui surchargeraient inutilement les calculs (et ce d'autant plus si le multiplicateur et/ou le multiplicande sont de très grands nombres).
	\end{tcolorbox}
	Suite à la demande d'un lecteur, voici \'egalement un exemple illustr\'e sur la façon dont nous enseignons (une manière parmi d'autres!) à multiplier des fractions simples ensembles au niveau du collège:
	 \begin{figure}[H]
		\centering
		\includegraphics[width=1.0\textwidth]{img/arithmetics/multiplication_of_fractions.jpg}
		\caption{Multiplication de fractions avec d\'enominateurs non-communs}
	\end{figure}
	Nous avons alors:
	
	Cet exemple montre que pour multiplier deux fractions, il suffit simplement de multiplier les num\'erateurs et les d\'enominateurs.
	
	\subsubsection{Division}\label{division}
	\textbf{D\'efinition (\#\mydef):} La "\NewTerm{division}\index{division}\label{division}" de nombres entiers (pour commencer par le cas le plus simple...) est une op\'eration, qui a pour but, \'etant donn\'e deux nombres entiers, l'un appel\'e "\NewTerm{dividende}\index{dividende}" $D$, l'autre appel\'e "\NewTerm{diviseur}\index{diviseur}" $d$, d'en trouver un troisième appel\'e "\NewTerm{quotient}\index{quotient}" $Q$ qui soit le plus grand nombre dont le produit par le diviseur puisse se retrancher (donc la division d\'ecoule de la soustraction!) du dividende (la diff\'erence \'etant nomm\'ee le  "\NewTerm{reste}\index{reste}" $R$ ou la "\NewTerm{congruence}\index{congruence}"). 
	\begin{tcolorbox}[title=Remarque,colframe=black,arc=10pt]
	Dans les cas des nombre r\'eels il n'y a jamais de reste à la fin de l'op\'eration de division (car le quotient multipli\'e par le diviseur donne exactement le dividende)!
	\end{tcolorbox}
	D'une façon g\'en\'erale dans le cadre des nombres entiers $\mathbb{N}$ (ou les divions d'\'equations alg\'ebriques), si nous notons $D$ le dividende, $d$ le diviseur, $Q$ le quotient et $R$ le reste nous avons la relation:
	
	en sachant que la division \'etait initialement not\'ee de la manière suivante:
	
	Nous indiquons l'op\'eration de division en plaçant entre les deux nombres, le dividende et le diviseur, un "$:$" ou une barre de division "$/$" ou même parfois au jardin d'enfant par le symbole "$\div$".
	
	Nous d\'esignons \'egalement souvent par "\NewTerm{fraction}\index{fraction}" (au lieu de "quotient"), le rapport de deux nombres ou autrement dit, la division du premier par le deuxième.
	
	\begin{tcolorbox}[title=Remarque,colframe=black,arc=10pt]
	Le signe de la division "$:$" est dû à Leibniz. La barre de fraction se trouve elle pour la première fois dans les ouvrages de Fibonacci (1202) et elle est probablement due aux Hindous.
	\end{tcolorbox}
	Si nous divisons deux nombres entiers et que nous souhaitons un entier comme quotient et comme reste (s'il y en a un...), alors nous parlons de "\NewTerm{division euclidienne}\index{division euclidienne}".
	
	Par exemple, la division d'un gâteau, n'est pas un division euclidienne car le quotient n'est pas un entier, except\'e si l'on en prend les quatre quarts...:
	\begin{figure}[H]
		\centering
		\includegraphics{img/arithmetics/division_cake.jpg}
		\caption{Exemple sch\'ematique de la division et du fractionnement (fraction)}
	\end{figure}
	Si nous avons:
	
	nous notons $i_D$ l'inverse du dividende. A tout nombre est associ\'e un inverse qui satisfait cette condition.
	
	De cette d\'efinition il vient la notation (avec $x$ \'etant un nombre quelconque diff\'erent de z\'ero): 
	
	Dans le cas de deux nombres fractionnaires, nous disons qu'ils sont "\NewTerm{inverses}\index{inverses}" ou "\NewTerm{r\'eciproques}\index{r\'eciproques}", lorsque leur produit est \'egal à l'unit\'e (comme la relation pr\'ec\'edente).
	\begin{tcolorbox}[title=Remarques,colframe=black,arc=10pt]
	\textbf{R1.} Une division par z\'ero est ce que nous nommons une "\NewTerm{singularit\'e}\index{singularit\'e}". C'est-à-dire que le r\'esultat de la division est ind\'etermin\'e. Mais en analyse, on peut essayer de donner un sens à la limite de la fonction inverse en $0$, c'est-à-dire du quotient $1/x$ lorsque $x$ tend vers z\'ero. \\
	
	\textbf{R2.}  Lorsque nous multiplions le dividende et le diviseur d'une division (fraction) par un même nombre, le quotient ne change pas (il s'agit d'une "\NewTerm{fraction \'equivalente}\index{fraction \'equivalente}"), mais le reste est multipli\'e par ce nombre!\\
	
	\textbf{R3.} Diviser un nombre par un produit effectu\'e de plusieurs facteurs revient à diviser ce nombre successivement par chacun des facteurs du produit et r\'eciproquement.\\
	
	\textbf{R4.} Les fractions sup\'erieures à $0$ mais inf\'erieures à $1$ sont parfois nomm\'ees "\NewTerm{fractions propres}\index{fractions propres}". Dans les fractions propres, le num\'erateur est inf\'erieur au d\'enominateur. Lorsqu'une fraction a un num\'erateur sup\'erieur ou \'egal au d\'enominateur, la fraction est une "\NewTerm{fraction impropre}\index{fraction impropre}". Une fraction impropre est toujours \'egale ou sup\'erieure à $1$. Et, enfin, un "\NewTerm{nombre mixte}\index{nombre mixte}" est la combinaison d'un nombre entier et d'une fraction propre.
	\end{tcolorbox}
	Les propri\'et\'es des divisions avec les notations condens\'ees de puissances (exponentiation) sont les suivantes (nous laisserons le soin au lecteur de le v\'erifier avec des valeurs num\'eriques):
	
	ou un autre exemple trivial:
	
	Nous en d\'eduisons donc:
	
	Rappelons qu'un nombre premier (entier relatif) est un nombre qui n'a d'autres diviseurs que lui-même et l'unit\'e (rappelons que $1$ n'est pas un nombre premier!). Donc tout nombre qui n'est pas premier a au moins un nombre premier comme diviseur (except\'e $1$ par d\'efinition!). Le plus petit des diviseurs d'un nombre entier est donc un nombre premier (nous d\'etaillerons les propri\'et\'es des nombres premiers relativement au sujet de la division dans la section de Th\'eorie des Nombres \pageref{prime number}).
	
	Voyons quelques propri\'et\'es de la division (certaines nous sont d\'ejà connues car elles d\'ecoulent d'un raisonnement logique des propri\'et\'es de la multiplication):
	
	où: 
	
	est ce que nous appelons une "\NewTerm{amplification des termes}\index{amplification des termes}" et: 
	
	est une op\'eration consistant à tout mettre avec un "\NewTerm{d\'emoninateur commun}\index{d\'emoninateur commun}".
	
	Nous avons aussi les propri\'et\'es suivantes:
	\begin{enumerate}
		\item[P1.] La division de plusieurs nombres d\'epend de l'ordre des termes. Nous disons alors que la division est une "\NewTerm{op\'eration non-commutative}". Ce qui signifie que nous avons quand $a$ est diff\'erent de $b$:
		
		
		\item[P2.] Le r\'esultat de la division de plusieurs nombres change si nous remplaçons deux ou plusieurs d'entre eux par leur r\'esultat interm\'ediaire. Nous disons alors que la division est "\NewTerm{op\'eration non-associative}":
		
		
		\item[P3.]  L'unit\'e est l'\'el\'ement neutre à droite de la division car tout dividende divis\'e par le diviseur $1$ est \'egal au dividende mais l'unit\'e n'est par contre pas neutre à gauche:
		
		
		\item[P4.] La division peut comporter un diviseur de telle façon à ce que la division soit \'egale à l'unit\'e (l'\'el\'ement neutre $1$). Nous disons alors qu'il existe un "\NewTerm{sym\'etrique pour la division}\index{sym\'etrique pour la division}" qui est trivialement quand le diviseur est \'egal au num\'erateur (dividende) lui-même!
		
		\item[P5.] L’incr\'ementation du num\'erateur et du d\'enominateur d’une valeur constante n’est pas \'egale au rapport initial dans le cas g\'en\'eral où $a\neq b$:
		
	\end{enumerate}
	Maintenant que nous connaissons la multiplication (et donc la notation de puissance) et la division, si nous consid\'erons que $a$ et $b$ sont deux nombres r\'eels positifs, diff\'erents de z\'ero (ie $(a,b)\in\mathbb{R}_+^*$ ), nous avons:
	
	et (nomm\'e parfois la "\NewTerm{règle des exposants z\'ero}\index{règle des exposants z\'ero}"):
	
	et:
	
	Nous avons aussi trivialement:
	
	Ainsi que:
	
	
	\subsubsection{Racine $n$-ème}
	Maintenant que nous avons introduit de manière simple et pas trop formelle les op\'erations de multiplication (et de notation de puissance) et de division, nous pouvons introduire le concept de racine $n$-ème.
	
	Comme nous savons par exemple que:
	
	nous pouvons alors par inf\'erence aussi \'ecrire:
	
	et donc cela signifie que la racine fractionnaire d'un nombre existe! Ce ce que nous appelons la racine $n$-ème (donc dans l'exemple ci-dessus on parlera de racine $2$-ème mais plus couramment de "\NewTerm{racine carr\'ee}\index{racine carr\'ee}").
	
	Nous pouvons maintenant d\'efinir la racine $n$-ème principale de n'importe quel nombre!
	
	\textbf{D\'efinition (\#\mydef):} En math\'ematiques, la "\NewTerm{racine $n$-ème}\index{racine $n$-ème}" d'un nombre $a$, où $n$ est un entier positif, est un nombre $r$ qui, lorsqu'il est \'elev\'e à la puissance $n$, donne $x$. C'est-à-dire tel que: $r^n=x$, où $n$ est le degr\'e de la racine. Par convention nous \'ecrivons:
	
	Les racines sont habituellement not\'ees en utilisant le symbole du "\NewTerm{radical}\index{radical}" qui est $\sqrt[n]{\ldots}$. Le nombre $n\in\mathbb{N}^*$ est appel\'e le "\NewTerm{radicande}\index{radicande}" ou très rarement appel\'e "\NewTerm{index}\index{index}".
	
	De ce qui a d\'ejà \'et\'e dit pour les puissances, nous pouvons conclure ais\'ement que la racine $n$-ème d'un produit de plusieurs facteurs est \'egale au produit des racines $n$-ème de chacun des facteurs:
	
	et (comme vu pr\'ec\'edemment):
	
	Et dès lors:
	
	Il en ressort que:
	
	Nous avons aussi si $a<0$:
	
	si $n\in \mathbb{N}^{*}$ est impair et:
	
	si $n\in \mathbb{N}^{*}$ est pair.

	Si $x<0$ et $n\in \mathbb{N}^{*}$ est impair, alors:
	
	est le nombre r\'eel n\'egatif $y$ tel que:
	
	Si $n\in \mathbb{N}^{*}$ est pair alors bien sûr, comme nous l'avons d\'ejà vu, la racine est complexe (\SeeChapter{voir section Nombres page \pageref{unit pure imaginary number}}).
	
	Si le d\'enominateur d'un quotient contient un facteur de la forme $\sqrt[n]{a^k}$ avec  $a\neq 0$, en multipliant le num\'erateur et le d\'enominateur par $\sqrt[n]{a^{n-}}$, nous supprimerons la racine au d\'enominateur, puisque:
	
	\begin{tcolorbox}[colframe=black,colback=white,sharp corners]
	\textbf{{\Large \ding{45}}Exemple:}\\\\
	Voyons un exemple mondialement connu de l'application de la racine qui concerne l'origine des formats papier A6, A5, A4, A3, A2, A1, A0 etc. Ce format a au fait la propri\'et\'e (c'est un objectif à l'origine) de conserver ses proportions lorsque nous plions ou coupons la feuille en deux dans sa grande dimension. Ainsi, si nous notons $L$ la longueur et $W$ la largeur de la feuille, nous avons:
	
	Il en ressort que:
	
	Comme le format A0 à par d\'efinition une superficie de $1\;[\text{m}^2]$. Pour ce format nous avons alors:
	
	Nous en d\'eduisons donc:
	
	et donc:
	
	d'où nous tirons aussi:
	
	Les autres formats de d\'eduisant donc pour rappel en divisant par deux la feuille dans sa grande dimension.
	\end{tcolorbox}
	
	\pagebreak
	\subsection{Polynômes Arithm\'etiques}
	\textbf{D\'efinition (\#\mydef):} Un "\NewTerm{polynôme arithm\'etique}\index{polynôme arithm\'etique}" (à ne pas confondre avec les polynômes alg\'ebriques qui seront \'etudi\'es dans la section d'Algèbre page \pageref{polynomial}) est un ensemble de nombres s\'epar\'es les uns des autres par les op\'erations d'addition ou de soustraction ($+$ ou $-$).
	
	Les composants enferm\'es dans le polynôme sont appel\'es "\NewTerm{termes}" du polynôme. Lorsque le polynôme contient un unique terme, nous parlons alors de "\NewTerm{monôme}\index{monôme}", s'il y a deux termes nous parlons de "\NewTerm{binôme}\index{binôme}", et ainsi de suite...
	
	\begin{theorem}
	La valeur d'un polynôme arithm\'etique est \'egale à l'excès de la somme des termes pr\'ec\'ed\'es du signe $+$ sur la somme des termes pr\'ec\'ed\'es du signe $-$.
	\end{theorem}
	\begin{dem}
	
	quelles que soient les valeurs des termes.
	\begin{flushright}
		$\blacksquare$  Q.E.D.
	\end{flushright}
	\end{dem}
	Mettre en \'evidence l'unit\'e n\'egative $-1$ est ce que nous appelons une "\NewTerm{factorisation}" ou "\NewTerm{mise en facteurs}". L'op\'eration inverse, s'appelant une "\NewTerm{distribution}" ou "\NewTerm{development}".
	
	Le produit de plusieurs polynômes peut toujours être remplac\'e par un polynôme unique que nous appelons le "\NewTerm{produit effectu\'e}". Nous op\'erons habituellement comme  suit: nous multiplions successivement tous les termes du premier polynôme, en commençant par la gauche, par le premier, le second, ..., le dernier terme du second polynôme. Nous obtenons ainsi un premier produit partiel. Nous faisons, s'il y a lieu, la r\'eduction des termes semblables. Nous multiplions ensuite chacun des termes du produit partiel successivement par le premier, le second, ..., le dernier terme du troisième polynôme en commençant par la gauche et ainsi de suite.
	
	\begin{tcolorbox}[colframe=black,colback=white,sharp corners]
	\textbf{{\Large \ding{45}}Exemple:}\\\\
	
	\end{tcolorbox}
	Le produit des polynômes $P_1$, $P_2$, $P_3$, ..., $P_k$, ... est la somme de tous les produits de $n_i$ facteurs form\'es avec un terme de $P_i$, un terme de $P_2$, ..., et un terme de $P_k$ et ainsi de suite. S'il n'y a aucune r\'eduction, le nombre de termes du produit est alors \'egal au produit des nombres de termes des facteurs tel que le nombre final de termes soit \'egal à:
	

	\subsection{Valeur absolue}\label{absolute value}
	\textbf{D\'efinition (\#\mydef):} En math\'ematiques, le "\NewTerm{valeur absolue}\index{valeur absolue}" $|x|$ d'un nombre r\'eel $x$ est la valeur non n\'egative de $x$ sans tenir compte de son signe. À savoir, $| x | = x $ pour un nombre positif $x$, $ | x | = -x $ pour un nombre $ x $ n\'egatif (auquel cas $ -x $ est positif) et $ | 0 | = 0 $. Par exemple, la valeur absolue de $3$ est de $-3$ et la valeur absolue de $-3 $ est \'egalement de $3$. La valeur absolue d'un nombre peut être consid\'er\'ee comme sa distance à z\'ero.
	
	\begin{tcolorbox}[title=Remarques,colframe=black,arc=10pt]
	\textbf{R1.} Le terme "valeur absolue" est utilis\'e dans ce sens depuis au moins 1806. La notation $ | x | $, avec une barre verticale de chaque côt\'e, semble avoir \'et\'e introduit par Karl Weierstrass en 1841.\\
	
	\textbf{R2.} Pour les trac\'es relatifs à la valeur absolue, le lecteur est invit\'e à se reporter à la section Analyse fonctionnelle de ce livre (page \pageref{absolute value plot}).
	\end{tcolorbox}
	Pour tout nombre r\'eel $x$, la "valeur absolue" de $x$, not\'ee $|x|$ est donc donn\'ee par:
	
	A l'origine, la valeur absolue aurait \'etait d\'efinie comme suit:
	
	Nous remarquons aussi la notation possible suivante qui est math\'ematiquement \'equivalente:
	
	Et les expressions d'\'equivalences:
	
	ainsi que:
	
	ces dernières \'etant souvent utilis\'ees dans le cadre de la r\'esolution des in\'equations que nous verrons plus tard en d\'etails.
	
	\begin{tcolorbox}[colframe=black,colback=white,sharp corners]
	\textbf{{\Large \ding{45}}Exemple:}\\\\
	R\'esoudre l'in\'egalit\'e suivante telle que: 
	
	est alors r\'esolue simplement en utilisant le concept intuitif de distance. La solution est l'ensemble des nombres r\'eels dont la distance par rapport au nombre r\'eel $ 3 $ est inf\'erieure ou \'egale à $ 9 $. Ceci est la plage de valeurs de centre $ 3 $ et de rayon $ 9 $ ou formellement:
	
	\end{tcolorbox}
	
	Indiquons qu'il est aussi utile d'interpr\'eter l'expression: 
	
	comme la distance entre les deux nombres $x$ et $y$ sur la droite r\'eelle. Ainsi, en munissant l'ensemble des nombres r\'eels de la distance valeur absolue, il devient un espace m\'etrique (voir la section de Topologie page \pageref{distance} pour une d\'efinition assez rigoureuse du concept de distance)!!
	
	La valeur absolue a quatre propri\'et\'es fondamentales que nous \'enoncerons sans d\'emonstrations:
	\begin{enumerate}
		\item[P1.] Non-negativit\'e:
			
		
		\item[P2.] Positif d\'efini:
			
			
		\item[P3.]  La valeur absolue du produit (multiplication) est \'egale au produit des valeurs absolues:
			

		\item[P4.] La valeur absolue de la somme alg\'ebrique de plusieurs nombres r\'eels est inf\'erieure ou \'egale à la somme des valeurs absolues des composantes de la somme:
			
			ce que les math\'ematiciens appellent parfois la "\NewTerm{première in\'egalit\'e triangulaire}".
	\end{enumerate}
	
	Les autres propri\'et\'es importantes de la valeur absolue incluent:
	\begin{enumerate}
		\item[P5.] 	L'idempotence (la valeur absolue de la valeur absolue est la valeur absolue):
			
		
		\item[P6.] 	Parit\'e (sym\'etrie r\'eflexive de la fonction):
			
			
		\item[P7.] Pr\'eservation de la division (\'equivalent à la propri\'et\'e P3) si $y\neq 0$:
			

		\item[P8.] La valeur absolue de la diff\'erence est sup\'erieure ou \'egale à la valeur absolue de la diff\'erence des valeurs absolues des composantes de la diff\'erence:
			
			ce que les math\'ematiciens appellent parfois la "\NewTerm{deuxième in\'egalit\'e triangulaire}".
	\end{enumerate}
	La fonction valeur absolue est fortement li\'ee à d'autres fonctions que nous utiliserons assez souvent dans diff\'erentes sections de ce livre. Un exemple important bien connu est la "\NewTerm{fonction signum}\index{fonction signum}" ou \NewTerm{fonction signe}\index{fonction signe}" qui est une fonction math\'ematique \'etrange mais à la fois simple qui extrait le signe d'un r\'eel nombre et d\'efini par:
	
	Alternativement (la fonction signum est donc la d\'eriv\'ee de la fonction valeur absolue, à l'ind\'etermination de z\'ero près):
	
	Tout nombre r\'eel peut être exprim\'e comme le produit de sa valeur absolue et de sa fonction de signe:
	

	\pagebreak
	\subsection{Règles de Calcul (priorit\'e des op\'erateurs)}
	Fr\'equemment en informatique (dans le d\'eveloppement de code en particulier), nous parlons de "\NewTerm{priorit\'e des op\'erateurs}\index{priorit\'e des op\'erateurs}". En math\'ematiques nous parlons plutôt de "\NewTerm{priorit\'e des ensembles d'op\'erations et des règles des signes}\index{r\'egles des signes}". De quoi s'agit-il exactement?
	
	Nous avons d\'ejà vu quelles \'etaient les propri\'et\'es des op\'erations d'addition, soustraction, multiplication, mise en puissance et division. Nous tenons donc à ce que le lecteur diff\'erencie la notion de "propri\'et\'e" de celle de "priorit\'e" (que nous allons tout de suite voir) qui sont deux notions complètement diff\'erentes!
	
	En math\'ematiques, en particulier, nous d\'efinissons d'abord les priorit\'es des symboles $\left\lbrace\left[\left(\right)\right]\right\rbrace$:
	\begin{enumerate}
		\item Les op\'erations qui sont entre parenthèses $( )$ doivent être effectu\'ees en premier dans le polynôme.
	
		\item  Les op\'erations qui sont entre crochets $[\,]$ doivent être effectu\'ees en second à partir des r\'esultats obtenus des op\'erations qui se trouvaient entre les parenthèses $( )$.
	
		\item Finalement, à partir des r\'esultats interm\'ediaires des op\'erations qui se trouvaient entre parenthèses $( )$ et crochets $[\,]$, nous calculons les op\'erations qui se situent entre les accolades $\left\lbrace \right\rbrace$.
	\end{enumerate}	

	Faisons un exemple, ce sera plus \'eloquent.
	\begin{tcolorbox}[colframe=black,colback=white,sharp corners]
	\textbf{{\Large \ding{45}}Exemple:}\\\\
	Soit à calculer le polynôme arithm\'etique:
	
	Selon les règles que nous avons d\'efinies tout à l'heure, nous calculons d'abord tous les \'el\'ements qui sont entre parenthèses $( )$, c'est-à-dire:
	
	ce qui nous donne:
	
	Toujours selon le règles que nous avons d\'efinies tout à l'heure, nous calculons maintenant tous les \'el\'ements entre crochets en commençant toujours à calculer les termes qui sont dans les crochets $[\,]$ au plus bas niveau des autres crochets $[\,]$. Ainsi, nous commençons par calculer l'expression $[4+14\cdot 2]$  qui se trouve dans le crochet de niveau sup\'erieur: $[5\cdot 10+3\cdot ...]$.\\
	
	Cela nous donne $[4+14\cdot 2]=32$ et donc:
	
	\end{tcolorbox}
	
	\pagebreak
	\begin{tcolorbox}[colframe=black,colback=white,sharp corners]
	Il nous reste à calculer maintenant $[5\cdot 10+3\cdot 32]=146$ et donc:
	
	Nous calculons maintenant l'unique terme entre accolades, ce qui nous donne:
	
	Finalement il nous reste:
	
	\'evidemment il s'agit d'un cas particulier... Mais le principe est toujours le même.
	\end{tcolorbox}
	La priorit\'e des op\'erateurs arithm\'etiques est une notion sp\'ecifique aux langages informatiques (comme nous en avons d\'ejà fait mention) du fait qu'on ne peut dans ces derniers \'ecrire des relations math\'ematiques que sur une ligne unique.
	
	Ainsi, en informatique l'expression: 
	
	s'\'ecrit (à peu de choses près dans la majorit\'e des languages de programmation): 
	
	Un non-initi\'e pourrait lire cette dernière expression de multiples façons:
	
	ce qui vous en conviendrez, est fort dangereux car nous arriverons à des r\'esultats diff\'erents à chaque fois (cas particuliers mis à part...) !
	
	Ainsi, il a logiquement \'et\'e d\'efini un ordre de priorit\'e des op\'erandes tel que les op\'erations soient effectu\'ees dans l'ordre suivant:
	\begin{enumerate}
		\item $-$ N\'egation
		
		\item $\string^$ Puissance (exponentiation)

		\item $*$ Multiplication et $/$ division\footnote{La revue \textit{Physical Review}, par exemple, pr\'ecise que la multiplication est prioritaire à la division dans ses règles de publication!}

		\item $\backslash$ Division entière (sp\'ecifique à l'informatique)
		
		\item $\mathrm{mod}$ Modulo (\SeeChapter{voir la section de Th\'eorie des Nombres page \pageref{congruence}})
		
		\item $+,-$ Addition et soustraction
	\end{enumerate}
	\'evidemment les règles des parenthèses $()$, crochets $[\,]$, et accolades $\left\lbrace \right\rbrace$ qui ont \'et\'e d\'efinies en math\'ematiques s'appliquent à l'informatique.
	
	Ainsi, nous obtenons dans l'ordre (nous remplaçons chaque op\'eration effectu\'ee par un symbole):
	
	D'abord les termes entre parenthèses:
	
	Ensuite les règles de priorit\'e des op\'erateurs s'appliquent dans l'ordre d\'efini pr\'ec\'edemment:
	\begin{enumerate}
		\item D'abord la n\'egation (rège 1):
		
		
		\item Ensuite la puissance (règle 2):
		
		
		\item Nous appliquons la multiplication (règle 3):
		
		
		\item Nous appliquons la division (règle 3 encore):
		
		Les règles (4) et (5) ne s'appliquent pas à cet exemple particulier.
		
		\item Et finalement (règle 6):
		
	\end{enumerate}
	Ainsi, en suivant ces règles, ni l'ordinateur, ni l'être humain ne peuvent (ne devraient) se tromper lors de l'interpr\'etation d'une \'equation \'ecrite sur une ligne unique (d'où l'importance d'avoir des normes ISO dans l'industrie):
	\begin{figure}[H]
		\centering
		\includegraphics{img/arithmetics/operators_priorities_calculators.jpg}
	\end{figure}
	\begin{tcolorbox}[title=Remarque,colframe=black,arc=10pt]
	Des mn\'emoniques sont souvent utilis\'es pour aider les \'elèves du primaire et sconedaire à se rappeler les règles les plus \'el\'ementaires. Aux \'etats-Unis, l'acronyme "\NewTerm{PEMDAS}\index{PEMDAS}" est courant. Il repr\'esente respectivement les Parenthèses, les Exposants, la Multiplication, la Division, l'Addition et enfin la Soustraction. Le Canada et la Nouvelle-Z\'elande utilisent PEDMAS, acronyme de Parenthèses, Exposants, Division, Multiplication, Addition, Soustraction. Les plus courantes au Royaume-Uni, en Inde et en Australie sont PODMAS, qui est respectivement l'acronyme de Parenthèses, Ordre, Division, Addition et enfin Soustraction. Le Nigeria et certains autres pays d'Afrique de l'Ouest utilisent PIDMAS, etc.
	\end{tcolorbox}
	
	En informatique, il existe cependant plusieurs op\'erateurs que nous ne retrouvons pas en math\'ematiques et qui changent souvent de propri\'et\'es d'un langage informatique à un autre. Nous ne nous attarderons pas trop là-dessus cependant, nous avons mis ci-dessous un petit descriptif:
	\begin{itemize}
		\item L'op\'erateur de concat\'enation "\&" est \'evalu\'e avant les op\'erateurs de comparaisons
		
		\item Les op\'erateurs de comparaison ($=, <,>, \ldots$) possèdent tous une priorit\'e identique
	\end{itemize}
	Cependant, les op\'erateurs les plus à gauche dans une expression, d\'etiennent une priorit\'e plus \'elev\'ee.
	
	Les op\'erateurs logiques (not\'es en anglais quelle que soit la langue de travail dans la très grande majorit\'e des cas) sont \'evalu\'es dans l'ordre de priorit\'e suivant:
	\begin{enumerate}
		\item Not ($\neg$)
		\item And ($\wedge$)
		\item Or ($\vee$)
		\item Xor ($\oplus$)
		\item Eqv ($\Leftrightarrow$)
		\item Imp ($\Rightarrow$)
	\end{enumerate}
	Maintenant que nous avons vu les priorit\'es des op\'erateurs, quelles sont les règles des signes en vigueur en math\'ematiques?


	D'abord, il faut savoir que ces dernières ne s'appliquent que dans le cas de la multiplication et de la division. Soient deux nombres positifs $(+x),(+y)$. Nous avons:
	
	Autrement dit, la multiplication de deux nombres positifs est un nombre positif et ceci est g\'en\'eralisable à la multiplication de $n$ nombres positifs.

	Nous avons:
	
	Autrement dit, la multiplication d'un nombre positif par un nombre n\'egatif est n\'egative. Ce qui est g\'en\'eralisable à un r\'esultat positif de la multiplication s'il y a un nombre pair de nombres n\'egatifs et à un r\'esultat n\'egatif pour un nombre impair de nombres n\'egatifs sur la totalit\'e $n$ des nombres de la multiplication.

	Nous avons:
	
	Autrement dit, la multiplication de deux nombres n\'egatifs est positive. Ce qui est g\'en\'eralisable à un r\'esultat positif de la multiplication s'il y a un nombre pair de nombre n\'egatifs et à un r\'esultat n\'egatif pour un nombre impair de nombres n\'egatifs.

	Pour ce qui est des divisions, le raisonnement est identique:
	
	Autrement dit, si le num\'erateur et le d\'enominateur sont positifs, alors le r\'esultat de la division sera positif.

	Nous avons:
	
	Autrement dit, si soit le num\'erateur ou le d\'enominateur est n\'egatif, alors le r\'esultat de la division sera forc\'ement n\'egatif.

	Nous avons:
	
	Autrement dit, si le num\'erateur et le d\'enominateur sont positifs, alors le r\'esultat de la division, sera forc\'ement positif.

	\'evidemment, si nous avons une soustraction de termes, il est possible de la r\'e\'ecrire sous la forme:
	 
	
	\begin{flushright}
	\begin{tabular}{l c}
	\circled{80} & \pbox{20cm}{\score{4}{5} \\ {\tiny 16 votes, 70.00\%}} 
	\end{tabular} 
	\end{flushright}
	
	
	%to make section start on odd page
	\newpage
	\thispagestyle{empty}
	\mbox{}
	\section{Th\'eorie des Nombres}
	\lettrine[lines=4]{\color{BrickRed}T}raditionnellement, la th\'eorie des nombres est une branche des math\'ematiques qui s'occupe des propri\'et\'es des nombres entiers, qu'ils soient entiers naturels ou entiers relatifs. Plus g\'en\'eralement, le champ d'\'etude de cette th\'eorie concerne une large classe de problèmes qui proviennent naturellement de l'\'etude des entiers. La th\'eorie des nombres peut être divis\'ee en plusieurs branches d'\'etude (th\'eorie alg\'ebrique des nombres, th\'eorie calculatoire des nombres, etc.) en fonction des m\'ethodes utilis\'ees et des questions trait\'ees.
	
	\begin{tcolorbox}[title=Remarque,colframe=black,arc=10pt]
	Le terme "arithm\'etique" \'etait aussi utilis\'e pour faire r\'ef\'erence à la th\'eorie des nombres mais c'est un terme assez ancien, qui n'est plus aussi populaire que par le pass\'e.
	\end{tcolorbox}
	Nous avons choisi de ne pr\'esenter dans cet expos\'e que les sujets qui sont indispensables à l'\'etude de la math\'ematique et de la physique th\'eorique ainsi que ceux devant faire absolument partie de la culture g\'en\'erale de l'ing\'enieur (certains r\'esultats ont aussi des applications en  Biostatistiques!).
		
	\subsection{Principe du bon ordre}
	Nous tiendrons pour acquit ce principe qui dit que tout ensemble non vide  $S \subset \mathbb{N}$ contient un plus petit \'el\'ement.

	Nous pouvons utiliser ce th\'eorème pour d\'emontrer une propri\'et\'e importante des nombres appel\'ee " \NewTerm{propri\'et\'e archim\'edienne}\index{propri\'et\'e archim\'edienne}" ou "\NewTerm{axiome d'Archimède}\index{axiome d'Archimède}" qui s'\'enonce ainsi:
	
	Pour $\forall a,b \in \mathbb{N}$ où $a$  est non nul, il existe au moins un entier positif $n$ tel que:
	
	En d'autres termes, pour deux grandeurs in\'egales, il existe toujours un multiple entier de la plus petite, sup\'erieur à la plus grande. Nous appelons "\NewTerm{archim\'ediennes}\index{structure archim\'ediennes}" des structures dont les \'el\'ements v\'erifient une propri\'et\'e comparable (\SeeChapter{voir section Th\'eorie des Ensembles page \pageref{comparison relations}}).

	Même si cela est trivial à comprendre dans le cas des nombres entiers faisons la d\'emonstration car elle permet de voir le type de d\'emarches utilis\'ees par les math\'ematiciens quand ils doivent d\'emontrer des \'el\'ements triviaux de ce genre...
	
	\begin{dem}
	Supposons le contraire en disant que pour $\forall n \in \mathbb{N}$ nous avons:
	
	Si nous d\'emontrons que cela est absurde pour tout $n$ alors nous aurons d\'emontr\'e la propri\'et\'e archim\'edienne (et ce \'egalement si $a, b$ sont r\'eels).
	
	Consid\'erons alors l'ensemble:
	
	En utilisant le principe du bon ordre, nous en d\'eduisons qu'il existe $s_0  \in S$ tel que $s_0 \leq s$ pour tout $s \in S$. Posons donc que ce plus petit \'el\'ement est:
	
	et nous avons donc aussi:
	
	Comme par hypothèse  $na\leq b$ nous devons alors avoir:
	
	et si nous r\'earrangeons et simplifions:
	
	et que nous simplifions le signe n\'egatif nous devions donc avoir...:
	
	d'où une contradiction \'evidente!
	
	ette contradiction amène que l'hypothèse initiale comme quoi $na < b$ pour tout $n$ alors est fausse et donc que la propri\'et\'e archim\'edienne est d\'emontr\'ee par l'absurde.
	\begin{flushright}
		$\blacksquare$  Q.E.D.
	\end{flushright}
	\end{dem}
	
	\subsection{Principe d'induction}
	Soit $S$ un ensemble de nombres naturels qui possède les deux propri\'et\'es suivantes:
	\begin{enumerate}
		\item[P1.] $1\in S$

		\item[P2.] So $k\in S$, alors $k+1\in S$
	\end{enumerate}
	Alors:
	
	Nous construisons ainsi l'ensemble des nombres naturels (se r\'ef\'erer à la section de Th\'eorie des Ensembles page \pageref{zermelo fraenkel axiomatic} pour voir la construction rigoureuse de l'ensemble des nombres entiers avec les axiomes de Zermelo-Fraenkel).
	
	\begin{theorem}
	Soit maintenant:
	
	le symbole "$\setminus$" signifiant "excluant". Nous voulons d\'emontrer que:
	
	\end{theorem}
	A nouveau, même si cela est trivial à comprendre faisons la d\'emonstration car elle permet de voir le type de d\'emarches utilis\'ees par les math\'ematiciens quand ils doivent d\'emontrer des \'el\'ements triviaux de ce genre...
	\begin{dem}
	Supposons le contraire, c'est-à-dire:
	
	Par le principe du bon ordre, puisque $B\subset \mathbb{N}$, $B$ doit poss\'eder un plus petit \'el\'ement que nous noterons $b_0$.
	
	Mais puisque $1\in S$ de par (P1), nous avons que $b_0>1$ et bien \'evidemment aussi que $1\in B$, c'est-à-dire aussi $b_0-1\in S$. En faisant appel à (P2), nous avons finalement que $b_0\in S$, c'est-à-dire que $b_0\not\in B$, donc une contradiction!
	\begin{flushright}
		$\blacksquare$  Q.E.D.
	\end{flushright}
	\end{dem}
	\begin{tcolorbox}[colframe=black,colback=white,sharp corners]
	\textbf{{\Large \ding{45}}Exemple:}\\\\
	Nous souhaitons montrer à l'aide du principe d'induction, que la somme des $n$ premiers carr\'es est \'egale à $n(n+1)(2n+1)/6$, c'est-à-dire que pour $n\geq 1$ nous aurions (\SeeChapter{voir la section Suites Et S\'eries page \pageref{sum of squares integers}}):
	
	D'abord la relation ci-dessus est facilement v\'erifi\'ee pour $n=1$ nous allons montrer que equation v\'erifie aussi cette relation. En vertu de l'hypothèse d'induction:
	
	nous retrouvons bien l'hypothèse de la validit\'e de la première relation mais avec $n=k+1$, d'où le r\'esultat.
	\end{tcolorbox}
	Ce proc\'ed\'e de d\'emonstration est donc d'une très grande importance dans l'\'etude de l'arithm\'etique; souvent l'observation et l'induction ont permis de soupçonner des lois qu'il eût \'et\'e plus difficile de trouver par a priori. Nous nous rendons compte de l'exactitude des formules par la m\'ethode pr\'ec\'edente qui a donn\'e naissance à l'algèbre moderne par les \'etudes de Fermat et de Pascal sur le triangle de Pascal (\SeeChapter{voir la section d'Algèbre page \pageref{Pascal's triangle}})).
	
	\pagebreak
	\subsection{Divisibilit\'e}
	\textbf{D\'efinition (\#\mydef):} Soit $A,B\in \mathbb{Z}$  avec $A\neq 0$. Nous disons que "$A$ divise $B$ (sans reste)" s'il existe un entier $q$ (le quotient) tel que: 
	
	auquel cas nous \'ecrivons pour ne pas confondre avec la division classique:
	
	Sinon nous \'ecrivons (si $A$ ne divise pas $B$ sans reste):
	
	et nous disons donc que "\NewTerm{$A$ ne divise pas $B$}".
	\begin{tcolorbox}[title=Remarques,colframe=black,arc=10pt]
	\textbf{R1.} Se rappeler que le symbole $|$ est une relation alors que le symbole $/$  est une op\'eration!\\
	
	\textbf{R2.}  Il ne faut pas confondre l'expression "$A$ divise $B$" qui signifie que le reste est obligatoirement nul et "$A$ est le diviseur de la division de $B$" qui indique que le reste n'est pas forc\'ement nul!
	\end{tcolorbox}
	Par ailleurs, si  $A|B$, nous dirons aussi que "\NewTerm{$B$ est divisible par $A$}" ou "\NewTerm{$B$ est un multiple de $A$}".
	
	Dans le cas où $A|B$ et que $1\geq A <B$, nous dirons que $A$ est un "\NewTerm{diviseur propre}\index{diviseur propre}" de $B$.
	
	De plus, il est clair que $A|0$ quel que soit  $A\in \mathbb{Z}\setminus \{0\}$  sinon quoi nous avons une singularit\'e.

	Voici maintenant quelques th\'eorèmes \'el\'ementaires se rattachant à la divisibilit\'e:
	\begin{theorem}
	Si $A|B$, alors $A|BC$  quel que soit $C\in \mathbb{Z}$. Ou plus formellement:
	
	\end{theorem}
	\begin{dem}
	Si $A|B$, alors il existe un entier $q$ tel que:
	
	Alors:
	
	et finalement:
	
	\begin{flushright}
		$\blacksquare$  Q.E.D.
	\end{flushright}
	\end{dem}
	\begin{theorem}
	Si $A|B$ et $B|C$, alors $A|C$ ou plus formellement:
	
	\end{theorem}
	\begin{dem}
	Si $A|B$ et $B|C$ alors, il existe des entiers $q$ et $r$ tel que $B=Aq$ et $C=Br$. Plus formellement:
	
	Dès lors:
	
	et ainsi:
	
	\begin{flushright}
		$\blacksquare$  Q.E.D.
	\end{flushright}
	\end{dem}
	\begin{theorem}
	Si $A|B$ et $A|C$ alors:
	
	\end{theorem}
	\begin{dem}
	Si $A|B$ et $A|C$ alors, il existe des entiers $q$ et $r$ tel que $B=Aq$ et $C=Ar$. Il en d\'ecoule:
	
	et dès lors:
	
	\begin{flushright}
		$\blacksquare$  Q.E.D.
	\end{flushright}
	\end{dem}
	\begin{theorem}
	Si $A|B$ et $B|A$ alors:
	
	\end{theorem}
	\begin{dem}
	Si $A|B$ et $B|A$ alors, il existe des entiers $q$ et $r$ tel que $B=Aq$ et $A=Br$.

	Nous avons alors
	 
	et donc $qr=1$. C'est pourquoi nous pouvons avoir $q=\pm 1$ et $r=\pm 1$ et dès lors:
	
	\begin{flushright}
		$\blacksquare$  Q.E.D.
	\end{flushright}
	\end{dem}
	\begin{theorem}
	Si $A|B$ et $B\neq 0$ alors:
	
	\end{theorem}
	\begin{dem}
	Si $A|B$ alors il existe un entier $q\neq 0$ tel que $B=Aq$. Mais alors:
	
	puisque $|q|\geq 1$.
	\begin{flushright}
		$\blacksquare$  Q.E.D.
	\end{flushright}
	\end{dem}
	
	\subsubsection{Division Euclidienne}\label{euclidean division}
	La division euclidienne est une op\'eration qui, à deux entiers naturels appel\'es "\NewTerm{dividende}\index{dividende}" et "\NewTerm{diviseur}\index{diviseur}", associe deux entiers appel\'es "\NewTerm{quotient}\index{quotient}" et "\NewTerm{reste}\index{reste}". Initialement d\'efinie aux entiers naturels non nuls, elle se g\'en\'eralise aux entiers relatifs et aux polynômes, par exemple.
	
	\textbf{D\'efinition (\#\mydef):} Nous appelons "\NewTerm{division euclidienne}\index{division euclidienne}" ou "\NewTerm{division entière}\index{division entière}" de deux nombres $A$ et $B$ l'op\'eration consistant à diviser $B$ par $A$ en s'arrêtant quand le reste devient strictement inf\'erieur à $A$.
	
	Rappelons (\SeeChapter{voir section Nombres page \pageref{prime number}}) que tout nombre qui admet exactement les deux diviseurs euclidiens (dont la division donne un reste nul) que sont $1$ et le nombre lui-même est dit "\NewTerm{nombre premier}\index{nombre premier}" (ce qui exclut le nombre $1$ de la liste des nombres premiers) et que tout couple de nombres qui n'ont que $1$ comme diviseur euclidien commun sont dits "\NewTerm{premiers entre eux}\index{premiers entre eux}".
	\begin{theorem}
	Soient $A,B\in \mathbb{Z}$ avec $A>0$. Le "\NewTerm{th\'eorème de la division euclidienne}\index{th\'eorème de la division euclidienne}" affirme qu'il existe des entiers uniques $q$ et $r$ tels que:
	
	où $0\leq r <A$. De plus, si $A\nmid B$, alors $0<r<A$.
	\end{theorem}
	\begin{tcolorbox}[colframe=black,colback=white,sharp corners]
	\textbf{{\Large \ding{45}}Exemple:}\\\\
	Nous avons un gâteau avec $9$ parts ($B$), nous devons le diviser entre $4$ personne ($A$) avec une part restante ($r=1$) tel que $q=2$:
	\begin{figure}[H]
		\centering
		\includegraphics{img/arithmetics/euclidean_division.jpg}
		\caption[]{Le gâteau a $9$ parts, donc chacune des $4$ personnes recevra $2$ parts et il en restera $1$.}
	\end{figure}
	et dès lors:
	
	\end{tcolorbox}
	\begin{dem}
	Consid\'erons l'ensemble:
	
	Il est relativement facile de voir que  $S\subset \mathbb{N}^{*}\cup \{0\}$ et que $S\neq \varnothing$, d'où, d'après le principe du bon ordre, nous concluons que $S$ contient un plus petit \'el\'ement $r\geq 0$.
	
	Soit $q$ l'entier satisfaisant donc à:
	
	Nous voulons d'abord montrer que $r<A$  en supposant le contraire (d\'emonstration par l'absurde), c'est-à-dire que $r\geq A$. Alors, dans ce cas, nous avons:
	
	ce qui est \'equivalent à:
	
	mais  $B-(q+1)A\in S$ et:
	
	ce qui contredit le fait que:
	
	est le plus petit \'el\'ement de $S$. Donc, $r<A$. Enfin, il est clair que si $r=0$, nous avons $A|B$, d'où la seconde affirmation du th\'eorème.
	\begin{flushright}
		$\blacksquare$  Q.E.D.
	\end{flushright}
	\end{dem}
	\begin{tcolorbox}[title=Remarque,colframe=black,arc=10pt]
	 Dans l'\'enonc\'e de la division euclidienne, nous avons suppos\'e que $A>0$. Qu'obtenons-nous lorsque $A<0$? Dans cette situation, $-A$ est positif, et alors nous pouvons appliquer la division euclidienne à $B$ et $-A$. Par cons\'equent, il existe des entiers $q$ et $r$ tels que:
	
	où $0\geq r <|A|$. Or, cette relation peut s'\'ecrire:
	
	où bien sûr, $-q$ est un entier. La conclusion est que la division euclidienne peut s'\'enoncer sous une forme plus g\'en\'erale!\\
	
	Soient  $A,B\in \mathbb{Z}$, alors il existe des entiers $q$ et $r$ tels que:
	
	où  $0\geq r <|A|$. De plus, si $A\nmid B$, alors  $0<r<|A|$
	\end{tcolorbox}
	Les entiers $q$ et $r$ sont uniques dans la division euclidienne. En effet, s'il existe deux autres entiers $q'$ et $r'$ tels que:
	
	avec toujours $0\leq r'<A$, alors:
	
	et ainsi :
	
	En vertu du th\'eorème 4.17 nous avons si $r-r'\neq 0$ que $|r-r'|\geq A$.

	Or, cette dernière in\'egalit\'e est impossible puisque par construction $-A<r-r'$. Donc $r=r'$ et puisque $A\neq 0$, alors $q'=q$  d'où l'unicit\'e.
	
	\paragraph{Plus grand commun diviseur}\label{greatest common divisor}\mbox{}\\\\
	Le "\NewTerm{plus grand commun diviseur}\index{plus grand diviseur commun}" (gpcd) (\'egalement appel\'e "\NewTerm{plus grand facteur commun}" (gcf), "\NewTerm{plus grande mesure commune}" (gcm)) de deux ou plusieurs entiers, quand au moins l'un d'entre eux n'est pas nul, est le plus grand entier positif qui divise les nombres sans un reste.
	
	\textbf{D\'efinition (\#\mydef):} Soient $a,b\in\mathbb{Z}$  tels que $ab\neq 0$. Le  "\NewTerm{plus grand commun diviseur}\index{plus grand diviseur commun}" (gpcd) de $a$ et $b$, not\'e:
	
	est l'entier naturel $d$ non nul qui satisfait aux deux propri\'et\'es suivantes:
	\begin{enumerate}
		\item[P1.] $d|a$ et $d|b$ (donc sans reste $r$ dans la division!)

		\item[P2.] Si $c|a$ et $c|b$ alors $c\leq d$ et $c|d$ (par d\'efinition!)
	\end{enumerate}
	Notons que $1$ est toujours un diviseur commun de deux entiers arbitraires.
	\begin{tcolorbox}[colframe=black,colback=white,sharp corners]
	\textbf{{\Large \ding{45}}Exemple:}\\\\
	Consid\'erons les entiers positifs $36$ et $54$. Un diviseur commun de $36$ et $54$ est un entier positif qui divise $36$, et aussi $54$. Par exemple, $1$ et $2$ sont des diviseurs communs de $36$ et $54$.

	
	Nous avons alors l'intersection repr\'esent\'ee par le diagramme de Venn suivant: (\SeeChapter{voir section Th\'eorie des Ensembles page \pageref{Venn diagrams}}):
	\begin{figure}[H]
		\centering
		\includegraphics{img/arithmetics/gcd_set.jpg}
		\caption{Diagramme de Venn des diviseurs communs}
	\end{figure}
	avec l'ensemble des diviseurs communs suivant:
	
	et donc le PGCD est:
	
	et nous constatons que l'ensemble des diviseurs communs de $36$ et $54$ est aussi l'ensemble des diviseurs de $18$.
	\end{tcolorbox}
	Cependant, il n'est pas forc\'ement \'evident que le PGCD autre qu'unitaire (c'est-à-dire diff\'erent $1$) de deux entiers $a$ et $b$ qui ne sont pas premiers entre eux existe toujours. Ce fait est d\'emontr\'e dans le th\'eorème suivant (cependant, si le PGCD existe, il est de par sa d\'efinition unique!) dit "\NewTerm{th\'eorème de B\'ezout}\index{th\'eorème de B\'ezout}\label{bezout theorem}" qui permet aussi de d\'emontrer d'autres propri\'et\'es int\'eressantes de deux nombres comme nous le verrons plus tard.
	\begin{theorem}
	Soient $a,b\in \mathbb{Z}$ tels que $ab\neq 0$. Si $d$ divise $a$ et $d$ divise $b$ (pour les deux sans reste $r$!) il existe alors obligatoirement des entiers relatifs $x$ et $y$ tels que:
	
	Cette relation est appel\'ee "\NewTerm{identit\'e de B\'ezout}\index{identit\'e de B\'ezout}" et il s'agit d'une \'equation diophantienne lin\'eaire (\SeeChapter{voir section Calcul Alg\'ebrique page \pageref{diophantine equation}}).
	\end{theorem}
	\begin{dem}
	\'evidemment, si $a$ et $b$ sont premiers entre eux nous savons que $d$ vaut alors $1$.
	
	Pour d\'emontrer l'identit\'e de B\'ezout consid\'erons d'abord l'ensemble:
	
	Comme $S\subset \mathbb{N}$ et $S\neq \varnothing$ , nous pouvons utiliser le principe du bon ordre et conclure que $S$ possède un plus petit \'el\'ement $d$. Nous pouvons alors \'ecrire:
	
	pour un certain choix $x_0,y_0\in\mathbb{Z}$. Il suffit donc de montrer que $d=(a,b)$ pour d\'emontrer l'identit\'e de B\'ezout!
	
	Proc\'edons via une d\'emonstration par l'absurde en posant supposant $d\nmid a$. Alors si c'est le cas, d'après la division euclidienne, il existe  $q,r\in\mathbb{Z}$ tels que $a=qd+r$, où $0<r<d$. Mais alors:
	
	Ainsi, nous avons que $r\in S$ et $r<d$, ce qui contredit le fait que $d$ est le plus petit \'el\'ement possible de $S$. Donc nous avons d\'emontr\'e ainsi non seulement que $d | a$ mais qu'en plus $d$ existe toujours et, de la même façon, nous d\'emontrons que $d | b$.
	\begin{flushright}
		$\blacksquare$  Q.E.D.
	\end{flushright}
	\end{dem}
	\begin{corollary}
	Comme corollaire important montrons maintenant que si $a,b\in\mathbb{Z}$ tels que  $ab\neq 0$, alors:
	
	constitue l'ensemble de tous les multiples de $d(a,b)$.
	\end{corollary}
	\begin{dem}
	Comme $d | a$ et $d | b$, alors nous avons forc\'ement $dax+by|$ pour tout $x,y\in \mathbb{Z}$. Soit $M=\{nd|n\in\mathbb{Z}\}$. Notre problème se r\'eduit au fait à montrer que $S=M$.
	
	Soit d'abord $s\in S$ ce qui signifie que $d|s$ et qui implique $s\in M$.
	
	Soit un $m\in M$, cela voudrait donc dire que $m=nd$ pour un certain $n\in\mathbb{Z}$.
	
	Comme $d=ax_0+by_0$ pour un choix d'entiers quelconques $x_0,y_0\in\mathbb{Z}$, alors:
	
	\begin{flushright}
		$\blacksquare$  Q.E.D.
	\end{flushright}
	\end{dem}
	Les hypothèses peuvent sembler compliqu\'ees mais portez plutôt votre attention un certain temps sur la dernière relation. Vous allez tout de suite comprendre!
	\begin{tcolorbox}[title=Remarque,colframe=black,arc=10pt]
	 Si au lieu de d\'efinir le PGCD de deux entiers non nuls, nous permettons à l'un d'entre eux d'être \'egal à $0$, disons: $a\neq b$, $b=0$. Dans ce cas, nous avons $a|b$ et , selon notre d\'efinition du PGCD, il est clair que $(a,0)=|a|$.
	\end{tcolorbox}
	Soit  $d=(a,b)$ et soit $m\in\mathbb{Z}$,alors nous avons les propri\'et\'es suivantes du PGCD (sans prevues, mais si un lecteur le souhaite nous pourrons donner les d\'etails):
	\begin{enumerate}
		\item[P1.] $(a,b+ma)=(a,b)=(a,-b)$
		\item[P2.] $(am,bm)=|m|(a,b)$ où $m\neq 0$
		\item[P3.] $\left(\dfrac{a}{d},\dfrac{b}{d}\right)=1$
		\item[P4.] Si $g\in\mathbb{Z}\setminus \{0\}$ tel que $g|a$ et $g|b$ alors $\left(\dfrac{a}{g},\dfrac{b}{g}\right)=\dfrac{1}{|g|}(a,b)$
	\end{enumerate}
	Dans certains ouvrages, ces quatre propri\'et\'es sont d\'emontr\'ees en utilisant intrinsèquement la propri\'et\'e elle-même. Personnellement nous nous en abstiendrons car faire cela est plus ridicule qu'autre chose à notre goût car la propri\'et\'e est une d\'emonstration en elle-même.

	\'elaborons maintenant une m\'ethode (algorithme) qui s'av\'erera très importante pour calculer (d\'eterminer) le plus grand commun diviseur de deux entiers (utile en informatique parfois).
	
	\subsubsection{Algorithme d'Euclide}
	L'algorithme d'Euclide est un algorithme permettant donc de d\'eterminer le plus grand commun diviseur de deux entiers.

	Pour aborder cette m\'ethode de manière intuitive, il faut savoir que vous devez comprendre un nombre entier comme une longueur, un couple d'entiers comme un rectangle (côt\'es) et leur PGCD est la taille du plus grand carr\'e permettant de carreler (paver) ce rectangle par d\'efinition (oui si vous r\'efl\'echissez un petit moment c'est assez logique!).

	L'algorithme d\'ecompose le rectangle initial en carr\'es, de plus en plus petits, par divisions euclidiennes successives, de la longueur par la largeur, puis de la largeur par le reste, jusqu'à un reste nul. Il faut bien comprendre cette d\'emarche g\'eom\'etrique pour comprendre ensuite l'algorithme.
	\begin{tcolorbox}[colframe=black,colback=white,sharp corners]
	\textbf{{\Large \ding{45}}Exemple:}\\\\
	Consid\'erons que nous cherchons le PGCD de $(a,b)$ où $b$ vaut $21$ et $a$ vaut $15$ et gardons à l'esprit que le PGCD, outre le fait qu'il divise $a$ et $b$, doit laisser un reste nul! En d'autres termes il doit pouvoir diviser le reste de la division de $b$ par $a$ aussi!\\

	Nous avons donc le rectangle de $21$ par $15$ suivant:
	\begin{figure}[H]
		\centering
		\includegraphics[scale=0.7]{img/arithmetics/euclids_algorithm_step1.jpg}
		\caption[]{Première \'etape de l'algorithme PGCD}
	\end{figure}
	\end{tcolorbox}
	
	\begin{tcolorbox}[colframe=black,colback=white,sharp corners]
	D'abord nous regardons si 15 est le PGCD (on commence toujours par le plus petit). Nous divisons alors $21$ par $15$ ce qui \'equivaut g\'eom\'etriquement à:
	\begin{figure}[H]
		\centering
		\includegraphics{img/arithmetics/euclids_algorithm_step2.jpg}
		\caption[]{Deuxième \'etape de l'algorithme PGCD}
	\end{figure}
	$15$ n'est donc pas le PGCD (on s'en doutait...). Nous voyons imm\'ediatement que nous n'arrivons pas à paver le rectangle avec un carr\'e de $15$ par $15$.\\
	
	Nous avons donc un reste de $6$ (rectangle de gauche). Le PGCD comme nous le savons doit, s'il existe, par d\'efinition pouvoir diviser ce reste et laisser un reste nul.\\

	Il nous reste donc un rectangle de $15$ par $6$. Nous cherchons donc maintenant à paver ce nouveau rectangle car nous savons que le PGCD est par construction inf\'erieur ou \'egal à $6$. Nous avons alors:
	\begin{figure}[H]
		\centering
		\includegraphics{img/arithmetics/euclids_algorithm_step3.jpg}
		\caption[]{ Troisième \'etape de l'algorithme PGCD}
	\end{figure}
	Et nous divisons donc $15$ par le reste $6$ (ce r\'esultat sera inf\'erieur à 6 et permet imm\'ediatement de tester si le reste sera le PGCD). Nous obtenons:
	\begin{figure}[H]
		\centering
		\includegraphics{img/arithmetics/euclids_algorithm_step4.jpg}
		\caption[]{Quatrième \'etape de l'algorithme PGCD}
	\end{figure}
	A nouveau, nous n'arrivons pas à paver ce rectangle rien qu'avec des carr\'es. En d'autres termes, nous avons un reste non nul qui vaut $3$. Soit un rectangle de $6$ par $3$. Nous cherchons donc maintenant à paver ce nouveau rectangle car nous savons que le PGCD est par construction inf\'erieur ou \'egal à $3$ et qu'il laissera un reste nul s'il existe. Nous avons alors g\'eom\'etriquement:
	\end{tcolorbox}
	
	\begin{tcolorbox}[colframe=black,colback=white,sharp corners]
	\begin{figure}[H]
		\centering
		\includegraphics{img/arithmetics/euclids_algorithm_step5.jpg}
		\caption[]{Cinquième \'etape de l'algorithme PGCD}
	\end{figure}
	Nous divisons $6$ par $3$ (ce qui sera inf\'erieur à $3$ et permet imm\'ediatement de tester si le reste sera le PGCD):
	\begin{figure}[H]
		\centering
		\includegraphics{img/arithmetics/euclids_algorithm_step6.jpg}
		\caption[]{Sixième \'etape de l'algorithme PGCD}
	\end{figure}
	et c'est tout bon! Nous avons $3$ qui laisse donc un reste nul et divise le reste $6$ il s'agit donc du PGCD. Nous avons donc au final:
	\begin{figure}[H]
		\centering
		\includegraphics{img/arithmetics/euclids_algorithm_step7.jpg}
		\caption{R\'esum\'e de l'algorithme PGCD}
	\end{figure}
	\end{tcolorbox}
	Maintenant, voyons l'approche formelle \'equivalente:
	
	Soient $a,b\in\mathbb{Z}$, où $a>0$. En appliquant successivement la division euclidienne (avec $b> a$), nous obtenons la suite d'\'equations:
	
	si $d=(a,b)$, alors $d=r_j$. Avec le pseudo-code correspondant:
	
	\begin{algorithm}[H]
	 \KwData{$a$,$ b$}
	 \KwResult{$b$}
	 initialization\;
	$r=a\mod b$\;
	 \While{$r\neq 0$}{
	  $a:=b$\;
	  $b:=r$\;
	   $f(b):=f(a)$\;
	   $a:=x_1$\;
	   $f(a):=f(x_1)$\;
	   }
	  Display $x_1$\;
	 \caption{Algorithme en pseudo-code du PGCM}
	\end{algorithm}
	Sinon de manière plus formelle:
	\begin{dem}
	Nous voulons d'abord montrer que $r_j=(a,b)$. Or, d'après la propri\'et\'e P1:
	
	nous avons:
	
	Pour d\'emontrer la deuxième propri\'et\'e de l'algorithme d'Euclide, nous \'ecrivons l'avant-dernière \'equation du système sous la forme:
	
	Or, en utilisant l'\'equation qui pr\'ecède cette avant-dernière \'equation du système, nous avons:
	
	En continuant ce processus, nous arrivons à exprimer $r_j$ comme une combinaison lin\'eaire de $a$ et $b$.
	\begin{flushright}
		$\blacksquare$  Q.E.D.
	\end{flushright}
	\end{dem}
	\begin{tcolorbox}[colframe=black,colback=white,sharp corners]
	\textbf{{\Large \ding{45}}Exemple:}\\\\
	Calculons le plus grand commun diviseur de $(429,966)$ et exprimons ce nombre comme une combinaison lin\'eaire de $429$ et de $966$.\\

	Nous appliquons bien \'evidemment l'algorithme d'Euclide:
	
	Nous en d\'eduisons donc que:
	
	et, de plus, que:
	
	Donc le PGCD est bien exprim\'e comme une combinaison lin\'eaire de $a$ et $b$ et constitue à ce titre le PGCD.
	\end{tcolorbox}
	\textbf{D\'efinition (\#\mydef):} Nous disons que les entiers $a_1,a_2,\ldots,a_n$ sont "\NewTerm{relativement premiers}\index{"relativement premiers"}" ou "\NewTerm{premiers entre eux}" si:
	
	
	\subsubsection{Plus petit commun multiple (PPCM)}
	Le "\NewTerm{plus petit commun multiple}\index{plus petit commun multiple}" de deux entiers $a$ et $b$, g\'en\'eralement d\'esign\'e par $\text{PPCM}(a, b) $, est le plus petit entier positif divisible par $a$ et $b$. Puisque la division des nombres entiers par z\'ero n'est pas d\'efinie, cette d\'efinition n'a de sens que si $a$ et $b$ sont tous deux diff\'erents de z\'ero.
	
	Le PPCM est bien connu dans l'arithm\'etique des \'ecoles primaires comme le "\NewTerm{plus petit d\'enominateur commun}\index{plus petit d\'enominateur commun}" qui doit être d\'etermin\'e avant que les fractions puissent être ajout\'ees, soustraites ou compar\'ees. Le PPCM de plus de deux entiers est \'egalement bien d\'efini: il s'agit du plus petit entier positif divisible par chacun d'eux.
	
	\textbf{D\'efinitions (\#\mydef):}
	\begin{enumerate}
		\item[D1.] Soient  $a_1,a_2,\ldots,a_n\in \mathbb{Z}\setminus \{0\}$, nous disons que  $m$ est "\NewTerm{multiple commun}\index{multiple commun}" de $a_1,a_2,\ldots,a_n$ si $a_i|m$ pour $i=1,2,\ldots,n$

		\item[D2.] Soient $a_1,a_2,\ldots,a_n\in \mathbb{Z}\setminus \{0\}$, nous appelons "\NewTerm{plus petit commun multiple}\index{plus petit commun multiple}" (PPCM) de $a_1,a_2,\ldots,a_n$ si $a_i|m$ pour $i=1,2,\ldots,n$ not\'e traditionnellement par:
		
		le plus petit entier commun multiple positif à tous les communs multiples de $a_1,a_2,\ldots,a_n$.
	\end{enumerate}
	\begin{tcolorbox}[colframe=black,colback=white,sharp corners]
	\textbf{{\Large \ding{45}}Exemples:}\\\\
	E1. Consid\'erons les entiers positifs $3$ et $5$. Un multiple commun de $3$ et $5$ est un entier positif qui est à la fois un multiple de $3$, et un multiple de $5$. Autrement dit, qui est divisible par $3$ et aussi par $5$. Nous avons donc:
	
	Nous avons alors l'intersection repr\'esent\'ee par le diagramme de Venn suivant: (\SeeChapter{voir section Th\'eorie des Ensembles \pageref{Venn diagrams}}):
	\begin{figure}[H]
		\centering
		\includegraphics{img/arithmetics/lcm_binary_venn_diagram.jpg}
	\end{figure}
	avec l'ensemble des communs multiples suivants:
	
	et le PPCM est alors:
	
	Soit sous une autre forme sch\'ematique:
	\begin{figure}[H]
		\centering
		\includegraphics{img/arithmetics/lcm_binary_multiples_line.jpg}
	\end{figure}
	Nous constatons que l'ensemble des multiples communs de $3$ et $5$ est aussi l'ensemble des multiples de $15$.
	\end{tcolorbox}
	
	\begin{tcolorbox}[colframe=black,colback=white,sharp corners]
	E2. Wikip\'edia donne \'egalement un bel exemple de diagramme de Venn (\SeeChapter{voir la section de Th\'eorie des Ensembles page \pageref{diagrammes de Venn}}) lorsque nous recherchons un PPCM de multiples de $5$ à l'aide d'un outil visuel:
	\begin{figure}[H]
		\centering
		\includegraphics[scale=0.5]{img/arithmetics/lcm_5ary_venn_diagramm.jpg}
	\end{figure}
	\end{tcolorbox}
	\begin{tcolorbox}[title=Remarque,colframe=black,arc=10pt]
	Soient $a_1,a_2,\ldots,a_n\in\mathbb{Z}\setminus \{0\}$. Alors, le plus petit commun multiple existe. En effet, consid\'erons l'ensemble $E$ des entiers naturels $m$ qui pour tout $i$ divisent $a_i$. Ce que nous noterons:
	
	Puisque nous avons forc\'ement $|a_1a_2\ldots a_n|\in E$, alors l'ensemble est non vide et, d'après l'axiome du bon ordre, l'ensemble $E$ contient un plus petit \'el\'ement positif.
	\end{tcolorbox}
	Voyons maintenant quelques th\'eorèmes relatifs au PPCM:
	\begin{theorem}
	 Si $m$ est un commun multiple quelconque de $a_1,a_2,\ldots,a_n$ alors  $[a_1,a_2,\ldots,a_n]|m$, c'est-à-dire que $m$ divise chacun des $a_i$.
	\end{theorem}
	\begin{dem}
	Soit $M=[a_1,a_2,\ldots,a_n]$. Alors, d'après la division euclidienne, il existe des entiers  $q$ et $r$ tels que:
	
	Il suffit de montrer que $r=0$. Supposons $r\neq 0$ (d\'emonstration par l'absurde). Puisque $a_i|m$  et  $a_i|M$, alors on a $a_i|r$ et cela pour $i=1,2,\ldots,n$. Donc, $r$ est un commun multiple de $a_1,a_2,\ldots,a_n$ plus petit que le PPCM. On vient d'obtenir une contradiction, ce qui prouve le th\'eorème.
	\begin{flushright}
		$\blacksquare$  Q.E.D.
	\end{flushright}
	\end{dem}
	\begin{theorem}
	Si $k>0$, alors $[ka_1,ka_2,\ldots,ka_n]=k[a_1,a_2,\ldots,a_n]$

	La d\'emonstration sera suppos\'ee \'evidente (dans le cas contraire contactez-nous et cela sera d\'etaill\'e!).
	\end{theorem}
	\begin{theorem}
		$[a,b]\cdot(a,b)=|ab|$
	\end{theorem}
	\begin{dem}
	\begin{lemma}
	Pour la d\'emonstration, nous allons utiliser le "\NewTerm{lemme d'Euclide}\index{lemme d'Euclide}\label{euclid lemma}" qui dit que si $a|bc$ et $(a,b)=1$ alors $a|c$.
	
	En d'autres termes, le lemme d'Euclide capture une propri\'et\'e fondamentale des nombres premiers, à savoir: Si un nombre premier divise le produit de deux nombres, il doit diviser au moins un de ces nombres. Il s'appelle \'egalement du "\NewTerm{premier th\'eorème d'Euclide}\index {premier th\'eorème d'Euclide}". Ce lemme est la cl\'e de la preuve du th\'eorème fondamental de l'arithm\'etique que nous verrons plus loin! 
	
	Effectivement cela se v\'erifie ais\'ement car nous avons vu qu'il existe $x,y\in\mathbb{Z}$ tels que $1=ax+by$ et alors $c=acx+bcy$. Mais $a|ac$ et $a|bc$ impliquent que $a|(acx+bcy)$, c'est-à-dire \'egalement que $a|c$.
	\end{lemma}
	
	Revenons à notre th\'eorème:
	
	Puisque $(a,b)=(a,-b)$ et $[a,b]=[a,-b]$, il suffit de prouver le r\'esultat pour des entiers positifs $a$ et $b$. 
	
	En tout premier lieu, consid\'erons le cas où $(a,b)=1$. L'entier $[a, b]$ \'etant un multiple de $a$, nous pouvons \'ecrire $[a,b]=ma$. Ainsi, nous avons $b|ma$ et, puisque $(a,b)=01$, il s'ensuit, d'après le lemme d'Euclide, que $b | m$. Donc, $b\leq m$ et alors $ab\leq am$. Mais $ab$ est un commun multiple de $a$ et $b$ qui ne peut être plus petit que le PPCM. c'est pourquoi $ab=ma=[a,b]$.

	Pour le cas g\'en\'eral, c'est-à-dire $(a,b)=d>1$, nous avons, d'après la propri\'et\'e:
	
	et avec le r\'esultat obtenu pr\'ec\'edemment que:
	
	Lorsque nous multiplions des deux côt\'es de l'\'equation par $d^2$, le r\'esultat suit et la d\'emonstration est effectu\'ee.
	\begin{flushright}
		$\blacksquare$  Q.E.D.
	\end{flushright}
	\end{dem}
	
	\pagebreak
	\subsubsection{Th\'eorème fondamental de l'arithm\'etique}\label{fundamental theorem of arithmetic}
	Le th\'eorème fondamental de l'arithm\'etique dit que tout nombre naturel $n>1$ peut s'\'ecrire comme un produit de nombres premiers, et cette repr\'esentation est unique, à part l'ordre dans lequel les facteurs premiers sont dispos\'es.

	Le th\'eorème \'etablit l'importance des nombres premiers. Essentiellement, ils sont les briques \'el\'ementaires de construction des entiers positifs, chaque entier positif contenant des nombres premiers d'une manière unique.
	
	\begin{tcolorbox}[title=Remarque,colframe=black,arc=10pt]
	Ce th\'eorème est parfois appel\'e "th\'eorème de factorisation" (un peu à tort... car d'autres th\'eorèmes portent le même nom...).
	\end{tcolorbox}
	Ok, alors allons-y:
	\begin{theorem}
	Chaque nombre entier sup\'erieur à $ 1 $ est soit lui-même nombre premier, soit le produit de nombres premiers, et ce produit est unique, à l'ordre près des facteurs.
	\begin{tcolorbox}[title=Remarque,colframe=black,arc=10pt]
	Ce th\'eorème est l’une des principales raisons pour lesquelles $1$ n’est pas consid\'er\'e comme un nombre premier: si $1$ \'etait premier, la factorisation ne serait pas unique.
	\end{tcolorbox}
	\end{theorem}
	\begin{dem}
	La preuve utilise le lemme d'Euclide: si un nombre premier $p$ divise le produit de deux nombres naturels $a$ et $b$, alors $p$ divise $a$ ou $p$ divise $b$ (ou les deux).
	
	Si $n$ est premier, et donc produit d'un unique entier premier, à savoir lui-même le r\'esultat est vrai et la d\'emonstration est termin\'ee (dire qu'un nombre premier est produit de lui-même est bien \'evidemment un abus de langage!). Supposons que $n$ n'est pas premier et donc strictement sup\'erieur à $1$ et consid\'erons l'ensemble:
	
	Alors,  $D\subset \mathbb{N}$ et , puisque $n$ est compos\'e, nous avons que $D\neq \varnothing$. D'après le principe du bon ordre, $D$ possède un plus petit \'el\'ement $p_1$ qui est premier, sans quoi le choix minimal de $p_1$ serait contredit. Nous pouvons donc \'ecrire $n=p_1n_1$. Si $n_1$ est premier, alors la preuve est termin\'ee. Si  $n_1$ est aussi compos\'e, alors nous r\'ep\'etons le même argument que pr\'ec\'edemment et nous en d\'eduisons l'existence d'un nombre premier $p_2$ et d'un entier $n_2<n_1$ tels que $n=p_1p_2n_2$. En poursuivant ainsi nous arrivons forc\'ement à la conclusion que  $n_k$ sera premier.
	
	Donc finalement nous avons bien d\'emontr\'e qu'un nombre quelconque est d\'ecomposable en facteurs de nombres premiers à l'aide du principe du bon ordre.
	\begin{flushright}
		$\blacksquare$  Q.E.D.
	\end{flushright}
	\end{dem}
	Nous ne connaissons pas à ce jour de loi simple qui permette de calculer le $n$-ième facteur premier $p_n$. Ainsi, pour savoir si un entier $m¨$ est premier, il est pratiquement plus facile à ce jour de v\'erifier sa pr\'esence dans une table de nombres premiers.

	En fait, nous utilisons aujourd'hui la m\'ethode suivante:

	Soit un nombre $m$, si nous voulons d\'eterminer s'il est premier ou non, nous calculons s'il est divisible par les nombres premiers $p_n$ qui appartiennent à l'ensemble:
	

	\begin{tcolorbox}[colframe=black,colback=white,sharp corners]
	\textbf{{\Large \ding{45}}Exemple:}\\\\
	L'entier $223$ n'est ni divisible par $2$, ni par $3$, ni par $5$, ni par $7$, ni par $11$, ni par $13$. Il est inutile de continuer avec le prochain nombre premier, car $17^2=289>223$. Nous en d\'eduisons dès lors que le nombre $223$ est premier.
	\end{tcolorbox}
	
	\subsubsection{Congruences (arithm\'etique modulaire)}\label{congruence}
	L'arithm\'etique modulaire est un système d'arithm\'etique pour les nombres entiers, dans lequel les nombres se "retournent" lorsqu'ils atteignent une certaine valeur, appel\'ee le "\NewTerm{module}" (plural moduli).
	
	Une utilisation familière de l'arithm\'etique modulaire est l'horloge à $12$ heures (ainsi que le calendrier), dans laquelle le jour est divis\'e en deux p\'eriodes de $12$ heures. Si l'heure est maintenant $7:00$ heures, $8$ heures plus tard, il sera $3:00$ heures. Les ajouts habituels sugg\'ereraient que l'heure ult\'erieure devrait être de $7 + 8 = 15 $, mais ce n’est pas la r\'eponse, car le temps de l’horloge se "retourne" toutes les $12$ heures; dans un système à $12$ heures, il n'y a pas de "$15$ heures". De même, si l'horloge commence à $12:00$ (midi) et que $21$ heures s'\'ecoulent s'\'ecoulent, l'heure sera alors $9:00$ le lendemain, au lieu de $33:00$. Parce que le nombre d'heures recommence après avoir atteint la valeur $12$, il s'agit d'un calcul arithm\'etique modulo $12$. Selon la d\'efinition ci-dessous, $12$ est congruent non seulement à $12$ lui-même, mais \'egalement à $0$, de sorte que l'heure "$12:00$" pourrait \'egalement être \'ecrite "$0:00$", puisque $12$ est congru à $0$ modulo $12$.
	\begin{figure}[H]
		\centering
		\includegraphics{img/arithmetics/modular_arithmetics_clock.jpg}
		\caption[Mesure de l'heure sur une horloge utilisant l'arithm\'etique modulo $12$]{Mesure de l'heure sur une horloge utilisant l'arithm\'etique modulo $12$ (source: Wikipédia)}
	\end{figure}
	\textbf{D\'efinition (\#\mydef):} Soit $m\in\mathbb{Z}\setminus 0$. Si $a$ et $b$ ont même reste dans la division euclidienne par $m$ nous disons que "$a$ est congru à $b$ modulo $m$", et nous \'ecrivons:
	
	ou de manière \'equivalente il existe (au moins) un nombre entier relatif $k$ tel que:
	
	Nous appelons aussi le nombre $b$  "\NewTerm{r\'esidu}\index{r\'esidu}". Ainsi, un r\'esidu est un entier congru à un autre, modulo un entier $m$ donn\'e.
	
	Une autre manière de dire tout cela si ce n'est pas clair...: L'\'etude des propri\'et\'es qui relient trois nombres entre eux est appel\'ee commun\'ement "\NewTerm{l'arithm\'etique modulaire}\index{l'arithm\'etique modulaire}".
	
	\begin{tcolorbox}[title=Remarques,colframe=black,arc=10pt]
	\textbf{R1.} Que nous soyons bien d'accord, la congruence implique un reste nul pour la division!\\
	
	\textbf{R2.} Nous excluons en plus de $0$ aussi $1$ et $-1$ pour les valeurs que peut prendre $m$ dans la d\'efinition de la congruence dans certains ouvrages.\\
	
	\textbf{R3.} Derrière le terme de congruence se cachent des notions semblables mais de niveaux d'abstraction diff\'erents:
	\begin{itemize}
		\item En arithm\'etique modulaire, nous disons donc que "deux entiers relatifs $a$ et $b$ sont congrus modulo $m$ s'ils ont même reste dans la division euclidienne par $m$". Nous pouvons aussi dire qu'ils sont congrus modulo $m$ si leur diff\'erence est un multiple de $m$.

		\item Dans la mesure des angles orient\'es, nous disons que "deux mesures sont congrues modulo $2\pi$ [rad] si et seulement si leur diff\'erence est un multiple de $2\pi$ [rad]". Cela caract\'erise deux mesures d'un même angle (\SeeChapter{voir section Trigonom\'etrie page \pageref{periodicity trigonometric functions}}).

		\item En algèbre, nous parlons de congruence modulo $I$ dans un anneau commutatif (\SeeChapter{voir section Th\'eorie des Ensembles page \pageref{ring}}) dont $I$ est un id\'eal: "$x$ est congru à $y$ modulo $I$ si et seulement si leur diff\'erence appartient $I$". Cette congruence est une relation d'\'equivalence, compatible avec les op\'erations d'addition et multiplication et permet de d\'efinir un anneau quotient de l'ensemble parent avec son id\'eal $I$.

		\item Nous trouvons parfois, dans l'\'etude de la g\'eom\'etrie (\SeeChapter{voir section G\'eom\'etrie Euclidienne page \pageref{congruence axioms}}) le terme de congru mis à la place de semblable. Il s'agit alors d'une simple relation d'\'equivalence sur l'ensemble des figures planes.
	\end{itemize}
	\end{tcolorbox}
	La relation de congruence $\equiv$ est une relation d'\'equivalence (\SeeChapter{voir section sur les Op\'erateurs \pageref{equivalence relation}}), en d'autres termes , soient $a,b,c,m\in\mathbb{Z},m>1$ alors la relation de congruence est:
	\begin{enumerate}
		\item[P1.] R\'eflexive:
		
		\item[P2.] Sym\'etrique:
		
		\item[P3.] Transitive:
		
	\end{enumerate}
	Les propri\'et\'es P1 et P2 sont \'evidentes (si ce n'est pas le cas faites-le nous savoir nous d\'evelopperons!). Nous allons d\'emontrer maintenant P3.  equation.
	\begin{dem}
	Les hypothèses impliquent que:
	
	 Mais alors:
	
	ce qui montre que $a$ et $c$ sont congrus modulo $m$.
	\begin{flushright}
		$\blacksquare$  Q.E.D.
	\end{flushright}
	\end{dem}
	La relation de congruence $\equiv$ est compatible avec la somme et le produit (se rappeler que la puissance n'est finalement qu'une extension du produit!).
	
	Effectivement, soient $(a,b,a',b',m)\in\mathbb{Z},m>1$ tel que $a\equiv \mod(m)$ et $a'\equiv b'$ alors:
	\begin{enumerate}
		\item[P1.] $a+a'\equiv b+b'\mod(m)$

		\item[P2.] $aa'\equiv bb'\mod(m)$
	\end{enumerate}
	\begin{dem}
	Nous avons:
	
	par hypothèse. Mais alors:
	
	ce qui d\'emontre P1. Nous avons \'egalement:
	
	ce qui d\'emontre P2.
	\begin{flushright}
		$\blacksquare$  Q.E.D.
	\end{flushright}
	\end{dem}
	\begin{tcolorbox}[title=Remarque,colframe=black,arc=10pt]
	La relation de congruence se comporte sur de nombreux points comme la relation d'\'egalit\'e. N\'eanmoins une propri\'et\'e de la relation d'\'egalit\'e n'est plus vraie pour celle de congruence, à savoir la simplification: si $ab\equiv \mod(m)$, nous n'avons pas n\'ecessairement $b\equiv c \mod(m)$.
	\end{tcolorbox}
	\begin{tcolorbox}[colframe=black,colback=white,sharp corners]
	\textbf{{\Large \ding{45}}Exemple:}\\
	
	\end{tcolorbox}
	Jusqu'ici, nous avons vu des propri\'et\'es des congruences faisant intervenir un seul modulus. Nous allons maintenant \'etudier le comportement de la relation de congruence lors d'un changement de modulus.
	\begin{enumerate}
		\item[P1.] Si $a\equiv b \mod(m)$ et $d|m$, alors $a\equiv b \mod(d)$
		\item[P2.] Si $a\equiv b$ et $a\equiv b \mod(s)$ alors $a$ et $b$ sont congrus modulo $[r,s]$
	\end{enumerate}
	Nous pensons que ces deux propri\'et\'es sont \'evidentes. Inutile d'aller dans les d\'etails pour P1. Pour P2, puisque $b-a$ est un multiple de $r$ et de $s$ puisque par hypothèse:
	
	$b-a$ st donc un multiple du PPCM de $r$ et $s$, ce qui d\'emontre P2.
	
	De ces propri\'et\'es il vient que si nous d\'esignons par $f(x)$ un polynôme à coefficient entiers (positifs ou n\'egatifs):
	
	La congruence $a \equiv b \mod(m)$ donnera aussi $f(a)\equiv f(b) \mod(m)$.

	Si nous remplaçons $x$ successivement  par tous les nombres entiers dans un polynôme $f(x)$ à coefficients entiers, et si nous prenons les r\'esidus pour le module $m$, ces r\'esidus se reproduisent  de $m$ en $m$ (dans le sens où la congruence se v\'erifie), puisque nous avons, quel que soit l'entier $m$ et $x$:
	
	Nous en d\'eduisons alors l'impossibilit\'e de r\'esoudre la congruence suivante:
	
	en nombres entiers, si $r$ d\'esigne l'un quelconque des non-r\'esidus (un r\'esidu qui ne satisfait pas la congruence).
	
	\paragraph{Classes de congruences}\mbox{}\\\\
	\textbf{D\'efinition (\#\mydef):} Nous appelons "\NewTerm{classe de congruence modulo $m$}\index{classe de congruence modulo}", le sous-ensemble de l'ensemble $\mathbb{Z}$ d\'efini par la propri\'et\'e que deux \'el\'ements $a$ et $b$ de $\mathbb{Z}$ sont dans la même classe si et seulement si $a\equiv b \mod(m)$ ou qu'un ensemble d'\'el\'ements entre eux sont congrus par ce même modulo.
	\begin{tcolorbox}[title=Remarque,colframe=black,arc=10pt]
	Nous avons vu dans le section traitant des Op\'erateurs (page \pageref{equivalence class}) qu'il s'agit en fait d'une classe d'\'equivalence car la congruence modulo $m$ est, comme nous l'avons d\'emontr\'e plus haut, une relation d'\'equivalence!!!
	\end{tcolorbox}
	
	\begin{tcolorbox}[colframe=black,colback=white,sharp corners]
	\textbf{{\Large \ding{45}}Exemple:}\\\\
	Soit $m=3$. Nous divisons l'ensemble des entiers en classes de congruence modulo $3$. Exemple de trois ensembles dont tous les \'el\'ements sont congrus entre eux sans reste (observez bien ce que donne l'ensemble des classes!):
	
	Ainsi, nous voyons que pour chaque couple d'\'el\'ement d'une classe de congruence, la congruence modulo 3 existe. Cependant, nous voyons que nous ne pouvons pas prendre $-9\equiv -8 \mod(3)$ où $-9$ se trouve dans la première classe et $-8$ dans la seconde.\\
	
	Le plus petit nombre non n\'egatif de la première classe est $0$, celui de la deuxième est $1$ et celui de la dernière est $2$. Ainsi, nous noterons ces trois classes respectivement $[0]_3,[1]_3,[2]_3$, le chiffre $3$ en indice indiquant le modulus.\\
	
	Il est int\'eressant de noter que si nous prenons un nombre quelconque de la première classe et un nombre quelconque de la deuxième, alors leur somme est toujours dans la deuxième classe. Ceci se g\'en\'eralise et permet de d\'efinir une somme sur les classes modulo $3$ en posant:
	
	et aussi:
	
	\end{tcolorbox}
	Ainsi, pour tout $m>1$, la classe de congruence de:
	
	est l'ensemble des entiers congrus à a modulo $m$ (et congrus entre eux modulo $m$). Cette classe est not\'ee:
	
	\begin{tcolorbox}[title=Remarque,colframe=black,arc=10pt]
	Le fait d'avoir mis entre parenthèses l'expression "et congrus entre eux modulo $m$" est dû au fait que la congruence, \'etant une relation d'\'equivalence nous avons comme nous l'avons d\'emontr\'e plus haut que si $b\equiv a \mod(m)$, $c\equiv \mod(m)$, alors $b\equiv a\mod(m)$.
	\end{tcolorbox}
	
	\textbf{D\'efinition (\#\mydef):} L'ensemble des classes de congruence $[a]_m$ (qui forment par le fait que la congruence est une relation d'\'equivalence des: "classes d'\'equivalence"), pour un $m$ fixe donne ce que nous appelons un "\NewTerm{ensemble quotient}\index{ensemble quotient}" (\SeeChapter{voir section Op\'erateurs page \pageref{quotient set}}). Plus rigoureusement, nous parlons de "l'ensemble quotient de $\mathbb{Z}$ par la relation de congruence" dont les \'el\'ements sont les classes de congruence (ou: classes d'\'equivalence) et qui forment alors l'anneau $\mathbb{Z}/m\mathbb{Z}$.
	
	Nous d\'eduisons de la d\'efinition les deux propri\'et\'es triviales suivantes:
	\begin{enumerate}
		\item Le nombre $b$ est dans la classe $[a]_m$ si et seulement si $a\equiv b\mod(m)$
		\item Les classes $[a]_m$ et $[b]_m$ sont \'egales si et seulement $a\equiv b\mod(m)$
	\end{enumerate}
	\begin{theorem}
	Montrons maintenant qu'il y a exactement $m$ diff\'erentes classes de congruence modulo $m$, à savoir $[0]_m,[1]_m,\ldots,[m-1]_m$.
	\end{theorem}
	\begin{dem}
	Soit  $m>1$, alors tout nombre entier $a$ est congru modulo $m$ à un et un seul entier $r$ de l'ensemble $\{0,1,2,\ldots,m-1\}$ (remarquez bien, c'est important, que nous nous restreignons aux entiers positifs ou nuls sans prendre en compte les n\'egatifs!). De plus, cet entier $r$ est exactement le reste de la division de $a$ par $m$. En d'autres termes, si $0\leq r<m$, alors:
	
	si et seulement si $a=qm+r$ où $q$ est le quotient de $a$ par $m$ et $r$ le reste. La d\'emonstration est donc une cons\'equence imm\'ediate de la d\'efinition de la congruence et de la division euclidienne.
	\begin{flushright}
		$\blacksquare$  Q.E.D.
	\end{flushright}
	\end{dem}
	\textbf{D\'efinition (\#\mydef):}  Un entier $b$ dans une classe de congruence modulo $m$ est appel\'e "\NewTerm{repr\'esentant de cette classe}\index{repr\'esentant de cette classe}" (il est clair que par la relation d'\'equivalence que deux repr\'esentants d'une même classe sont donc congrus entre eux modulo $m$).
	
	Nous allons pouvoir maintenant d\'efinir une addition et une multiplication sur les classes de congruences. Pour d\'efinir la somme de deux classes $[a]_m,[b]_m$, il suffit de prendre un repr\'esentant de chaque classe, de faire leur somme et de prendre la classe de congruence du r\'esultat. Ainsi (voir les exemples plus haut):
	
	et de même pour la multiplication:
	
	Par construction de la somme et du produit, nous constatons que la classe de $0$ (z\'ero) est l'\'el\'ement neutre pour l'addition:
	
	et la classe de l'entier $1$ est l'\'el\'ement neutre pour la multiplication:
	
	\textbf{D\'efinition (\#\mydef):} Un \'el\'ement $[a]_m$ de $\mathbb{Z}/m\mathbb{Z}$ est une "\NewTerm{unit\'e}" s'il existe un \'el\'ement  $[b]_m\in \mathbb{Z}/m\mathbb{Z}$ tel que $[a]_m\cdot[b]_m$.
	
	Le th\'eorème suivant permet de caract\'eriser les classes modulo $m$ qui sont des unit\'es dans $\mathbb{Z}/m/\mathbb{Z}$:
	\begin{theorem}
		Soit $[a]$ un \'el\'ement de $\mathbb{Z}/m/\mathbb{Z}$. Alors $[a]$ est une unit\'e si et seulement si $(a,m)=1$.
	\end{theorem}
	\begin{dem}
	Supposons d'abord que $(a,m)=1$. Alors par B\'ezout, nous avons son identit\'e:
	
	Autrement dit, $as$ est congru à $1$ modulo $m$. Mais ceci est \'equivalent à \'ecrire par d\'efinition que $[a][s]=1$ ce qui montre que $[a]$ est une unit\'e. R\'eciproquement, si $[a]$ est une unit\'e, ceci implique qu'il existe une classe $[s]$ telle que $[a][s]=1$.
	
	Ainsi, nous venons de d\'emontrer que $\mathbb{Z}/\mathbb{Z}$ constitue bien un anneau puisqu'il possède une addition, une multiplication, un \'el\'ement neutre et un inverse!!
	\begin{flushright}
		$\blacksquare$  Q.E.D.
	\end{flushright}
	\end{dem}
	
	\paragraph{Système complet de r\'esidus}\mbox{}\\\\
	\textbf{D\'efinition (\#\mydef):} Un ensemble de nombres $a_0, a_1, ..., a_{(m-1)} \mod (m)$ forme un "\NewTerm{système complet de r\'esidus}\index{système complet de r\'esidus}", aussi nomm\'e un "\NewTerm{système de couverture}\index{système de couverture}", s'il satisfait $a_i=i \mod (m) $ pour $i=0, 1, ..., m-1$. 

	Ce type de système nous aidera à introduire dans la section Cryptographie (page \pageref{cryptography}) une fonction importante utilis\'ee pour s\'ecuriser les dispositifs de communication à la fin du 20ème siècle et au d\'ebut du 21ème siècle.

	Pour introduire ce système, consid\'erons le système fini suivant de congruences modulo $6$:
	
	où comme le lecteur l'aura peut-être remarqu\'e: aucun r\'esidu ne se r\'epète et les r\'esidus pris deux à deux ne sont pas congrus modulo $m$ (c'est ce dernier point qui fait que nous nous arrêtons à $5$ dans l'exemple). Nous disons alors que "\NewTerm{mutuellement non-congrus}\index{mutuellement non-congrus}".

	Si ces conditions sont satisfaites, nous disons alors que l'ensemble ordonn\'e $\{6, 13, 2, -3, 22, 11\}$ est un "système complet de r\'esidus modulo $m$". Un tel ensemble n'est pas unique pour un module donn\'e. Ainsi, l'ensemble $\{0, 1, 2, 3, 4, 5\}$ est aussi un système complet (trivial) de r\'esidus modulo $6$.
	
	Si nous \'eliminons de ce sytème complet tous les nombres qui ne sont pas premier avec $m$, nous avons alors un "\NewTerm{système r\'eduit de r\'esidus modulo $m$}\index{système r\'eduit de r\'esidus}\label{system of reduced residue}". Ainsi dans la cas ci-dessus, le système r\'eduit de r\'esidus module $6$ sera $\{13, 11\}$.
	
	Les systèmes r\'eduits nous seront utiles dans la section de Cryptographie (page \pageref{cryptography}) pour d\'emontrer un r\'esultat important dans les systèmes asym\'etriques à cl\'e publique.

	Nous verrons aussi dans la section de Cryptographie (page \pageref{euler indicator function cryptography}), que la "\NewTerm{fonction indicatrice d'Euler}\index{fonction indicatrice d'Euler}\label{euler indicator function}" lorsque $m$ est premier (ce qui n'est pas le cas de l'exemple pr\'ec\'edent) donne le cardinal du système r\'eduit de r\'esidus modulo $m$ comme \'etant:
	
	Donc sous couvert que m est premier, le système r\'eduit de r\'esidu s'\'ecrira bien \'evidemment:
	
	
\paragraph{Th\'eorème des restes chinois}\mbox{}\\\\
	Dans sa forme de base, le th\'eorème des r\'esidus chinois d\'eterminera un nombre $n$ qui, divis\'e par certains diviseurs, laissera certains r\'esiuds. Dans l'exemple de Sun Tzu (\'enonc\'e une terminologie moderne):  quel est le plus petit nombre $n$ qui, divis\'e par $3$, laisse un reste de $2$, divis\'e par $5$, laisse un reste de $3$ et divis\'e par $7$ laisse un reste de $2$?
	
	Le th\'eorème des restes chinois peut être vu comme la r\'esolution d'un système lin\'eaire mais dans un ensemble modulaire. Pour beaucoup d'\'etudiants et futurs ing\'enieurs, ce th\'eorème ne servira jamais dans la pratique, mais certains le retrouveront dans le domaine de la cryptographie (dans le cadre du d\'ecryptage en particulier).
	
	Il existe comme toujours plusieurs d\'emonstrations possibles mais nous avons opt\'e pour celle qui nous semblait comme toujours la plus p\'edagogique.

	Soient $m$ et $n$ deux entiers premiers entre eux. Le système particulier de congruence (voir plus bas pour un exemple de r\'esolution d'un système de trois congruences):
	
	admet une unique solution.
	\begin{dem}
	Comme $m$ et $n$ sont impos\'es comme \'etant premiers entre eux, il existe alors $u$ et $v$ deux entiers relatifs tels que (application de l'identit\'e de B\'ezout d\'emontr\'ee plus haut):
	
	Nous avons donc:
	
	C'est-à-dire:
	
	Nous avons aussi in extenso:
	
	C'est-à-dire:
	
	Donc afin d'être clair, nous avons donc pour l'instant:
	
	Nous avons donc pour rappel:
	
	Mais nous pouvons aussi \'ecrire avec $k\in\mathbb{Z}$:
	
	Dès lors:
	
	De même, nous avons:
	
	Mais nous pouvons aussi \'ecrire avec $k\in\mathbb{Z}$:
	
	Dès lors:
	
	Donc afin d'être toujours clair, nous avons donc pour l'instant:
	
	Donc, il vient finalement que:
	
	est une solution particulière du système. Mais nous avons aussi pour $\forall i,j\in\mathbb{Z}$ de par la d\'efinition de la congruence:
	
	Afin que $x$ soit toujours solution du système, il faut avoir:
	
	et donc:
	
	Dès lors, une solution un peu plus g\'en\'erale est:
	
	mais in extenso, nous avons la solution g\'en\'erale:
	
	avec $z\in\mathbb{Z}$. Nous disons alors parfois que la solution est "$x$ modulo $nm$".
	\begin{flushright}
		$\blacksquare$  Q.E.D.
	\end{flushright}
	\end{dem}
	\begin{tcolorbox}[colframe=black,colback=white,sharp corners]
	\textbf{{\Large \ding{45}}Exemple:}\\\\
	A titre d'exemple, consid\'erons le problème de la recherche d'un entier $x$ tel que:
	
	Une approche par force brute convertit ces congruences en ensembles et \'ecrit les \'el\'ements jusqu'auproduit de $ 3 \cdot 4 \cdot 5 = 60 $ (les solutions modulo $ 60 $ pour chaque congruence):
	
	Pour trouver un $ x $ qui satisfait les trois congruences, intersectez les trois ensembles pour obtenir:
	
	Cette solution est modulo $60$, donc toutes les solutions sont exprim\'ees comme suit:
	
	Une autre façon de trouver une solution consiste à utiliser l’algèbre de base, l’arithm\'etique modulaire et la substitution par \'etapes.\\
	\end{tcolorbox}
	
	\begin{tcolorbox}[colframe=black,colback=white,sharp corners]
	Nous commençons par traduire ces congruences en \'equations pour certains $t$, $s$ et $u$:
	
	Après nous substituons le $ x $ de la première \'equation dans la deuxième congruence:
	
	C'est-à-dire:
	
	Dès lors:
	
	ce qui signifie que $ t = 3 + 4s $ pour un entier $ s $. Remplaçons maintenant $ t $ par la première \'equation:
	
	Substituons ce $ x $ à la troisième congruence:
	
	C'est-à-dire:
	
	Dès lors:
	
	signifiant que $s = 0 + 5u$ pour un entier donn\'e $u$. Finalement:
	
	Donc nous avons pour solution $\{11,71,131,191,\ldots\}$.
	\end{tcolorbox}
	
	\pagebreak
	\subsubsection{Fractions continues}\label{continued fraction}
	
	\textbf{D\'efinition (\#\mydef):} Une "\NewTerm{fraction continue}\index{fraction continue}" est une expression obtenue par un processus it\'eratif consistant à repr\'esenter un nombre comme la somme de sa partie entière et l'inverse d'un autre nombre, puis à \'ecrire cet autre nombre comme la somme de sa partie entière et l'invere d'une autre nombre, etc. Dans une fraction continue finie (ou "fraction continue termin\'ee"), l'it\'eration / r\'ecursivit\'e est termin\'ee après un nombre fini d'\'etapes en utilisant un entier au lieu d'une autre fraction continue. En revanche, une fraction continue infinie est une expression infinie. Dans les deux cas, tous les entiers de la s\'equence, autres que le premier, doivent être positifs. Les entiers $ a_i $ sont nomm\'es les "coefficients" ou "termes" de la fraction continue.
	
	La notion de fraction continue remonte à l'\'epoque de Fermat et atteint son apog\'ee avec les travaux de Lagrange et Legendre vers la fin du 18ème siècle. Ces fractions sont importantes en physique car nous les retrouvons en acoustique ainsi que dans la d\'emarche intellectuelle qui a amen\'e Galois à cr\'eer sa th\'eorie des groupes et aussi dans l'\'etude des engrenages (pour les complications horlogères comme discut\'e dans la section d'Ing\'enierie M\'ecanique \pageref{gears association}).

	Pour comprendre la motivation des fractions continues, introduisons un exemple \'el\'ementaire.
	
	Consid\'erons un nombre rationnel typique:
	
	qui est à peu près \'egal à $4.4624$.

	En première approximation, commençons avec $4$, qui est la partie entière:
	
	Notez que la partie d\'ecimale est l'inverse de $93/43$, ce qui correspond à environ $2.1628$. Utilisons la partie entière, $2$, comme approximation de l'inverse, pour obtenir une seconde approximation de:
	
	Nous avons donc jusqu'ici:
	
	Notez que la partie d\'ecimale est maintenant l'inverse de $43/7$, soit environ $6.1429$. Utilisons la partie entière, $6$, comme approximation de l'inverse, pour obtenir une seconde approximation de:
	
	Ainsi:
	
	Notez que la partie fractionnaire $1/7$ est l'inverse de $7$, ce qui correspond à ... $7$ Utilisons la partie entière, $7$, comme approximation de l'inverse, pour obtenir une seconde approximation de:
	
	Ainsi, nous avons:
	
	Cette expression est nomm\'ee comme nous le savons la "repr\'esentation en fraction continue du nombre".
	
	En abandonnant certaines des parties les moins essentielles de l'expression:
	
	nous donne la notation condens\'ee:
	 
	Notez qu'il est habituel de remplacer uniquement la première virgule par un point-virgule.

	Comme g\'en\'eralisation de l'exemple pr\'ec\'edent, consid\'erons dans un premier temps le nombre rationnel $a/b$ avec $(a,b)=1$ avec $b>0$ et $a>b$. Nous savons que tous les quotients $q_i$ et les restes $r_i$ sont dans le cadre de la division euclidienne des entiers positifs.

	Rappelons l'algorithme d'Euclide vu plus haut (mais not\'e de manière un peu diff\'erente):
	
	Par substitutions successives, nous obtenons:
	
	Ce qui est aussi parfois not\'e:
	
	Ainsi, tout nombre rationnel positif peut s'exprimer comme une fraction continue finie où  $q_n\in\mathbb{N}$.
	
	Reprenant notre dernier exemple:
	
	nous notons qu'effectivement que $q_n\in\mathbb{N}$ et que par construction:
	
	où les crochets repr\'esentent la partie entière et nous avons aussi:
	 
	Le d\'eveloppement du nombre $a/b$ est nomm\'e le "\NewTerm{d\'eveloppement du nombre $a / b$ en fraction continue finie}\index{fraction infinie}" et est condens\'e dans la notation suivante:
	
	Voyons un autre exemple:
	\begin{tcolorbox}[colframe=black,colback=white,sharp corners]
	\textbf{{\Large \ding{45}}Exemple:}\\\\
	Voyons comment extraire la racine carr\'ee d'un nombre $A$ (par exemple $A=2$ tel que nous voulons extraire $\sqrt{2}$) par la m\'ethode des fractions continues.\\
	
	Soit $a$ le plus grand nombre entier dont le carr\'e $a^2$ est plus petit que $A$. On le soustrait de $A$. Il y a donc un reste de (pour $A=2$, nous avons $a=1$):
	
	où nous avons utilis\'e une des identit\'es remarquables que nous prouverons dans la section d'Algèbre plus loin (page \pageref{calculus remarkable identities}). D'où en divisant les deux membres par la deuxième parenthèse, nous avons:
	
	Soit:
	
	Dans le d\'enominateur, nous remplaçons $\sqrt{A}$ par:
	
	Cela donne:
	
	etc.... on voit ainsi que le système est simple pour d\'eterminer l'expression d'une racine en termes de fraction continue.
	\end{tcolorbox}
	Nous consid\'erons maintenant comme intuitif que chaque nombre rationnel peut être exprim\'e en fraction continue finie et inversement que toute fraction continue finie repr\'esente un nombre rationnel. Par extension, un nombre irrationnel est repr\'esent\'e par une fraction continue infinie!

	Consid\'erons maintenant $[q_1;q_2,q_3,\ldots,q_n]$ une fraction continue finie. La fraction continue:
	
	où  $k=1,2,\ldots,n$  est appel\'ee la "\NewTerm{$k$-ème r\'eduite}\index{$k$-r\'eduite}" ou "\NewTerm{$k$-ème convergente}\index{$k$-convergente}" ou encore le "\NewTerm{$k$-ème quotient partiel}\index{$k$-ème quotient partiel}".

	Avec cette notation, nous avons:
	
	Pour simplifier les expressions ci-dessus, nous introduisons les suites $\{n_i\},\{d_i\}$ ($n$ pour \textbf{n}um\'erateur et $d$ pour \textbf{d}\'enominateur) d\'efinies par:
	
	à l'aide de cette construction, nous avons une petite in\'egalit\'e imm\'ediate int\'eressante pour un peu plus loin:
	
	Avec la d\'efinition ci-dessus, nous constatons que:
	
	Soit en g\'en\'eralisant:
	
	Maintenant, montrons pour un usage ult\'erieur que pour $i\geq 1$, nous avons:
	
	Le r\'esultat est imm\'ediat pour $i=1$. En supposant que le r\'esultat est vrai pour $i$ montrons qu'il est aussi vrai pour $i+1$. Puisque:
	
	alors en utilisant l'hypothèse d'induction, nous obtenons le r\'esultat!
	
	Nous pouvons maintenant \'etablir une relation indispensable pour la suite. 
	\begin{theorem}
	Montrons que si $C_k$ est la $k$-ème r\'eduite de la fraction continue simple finie $[q_1;q_2,\ldots,q_n]$ alors:
	
	\end{theorem}
	\begin{dem}
	
	puisque:
	
	donc:
	
	ce qui nous indique que le signe $C_{k+1}-C_k$ est le même que celui de $(-1)^{k+1}$.

	Il en r\'esulte que $C_{k+2}>C_k$ pour $k$ impair, et que $C_{k+2}<C_k$ pour $k$ pair. Il s'ensuit que:
	
	Ensuite, puisque:
	
	Donc pour $k$ pair, nous avons $C_k>C_{k-1}$, nous en d\'eduisons donc:
	
	\begin{flushright}
		$\blacksquare$  Q.E.D.
	\end{flushright}
	\end{dem}
	Montrons maintenant que toute fraction continue infinie peut repr\'esenter un nombre irrationnel quelconque.
	
	En des termes formels, si $\{q_n\}$ est une suite d'entiers tous positifs et que nous consid\'erons $C_n=[q_1;q_2,\ldots,q_n]$ alors celui-ci converge n\'ecessairement vers un nombre r\'eel si $n\rightarrow +\infty$.

	Effectivement il n'est pas difficile d'observer (c'est assez intuitif) avec un exemple pratique que nous avons:
	
	lorsque $k\rightarrow +\infty$.
	
	Maintenant, notons $x$ un nombre r\'eel quelconque et $q_1=[x]$ la partie entière de ce nombre r\'eel. Alors nous avons vu tout au d\'ebut de notre \'etude des fractions continues que:
	
	Il vient donc que:
	
	Attardons-nous pour les n\'ecessit\'es de la section d'Acoustique (page \pageref{acoustic}) sur le calcul d'une fraction continue d'un logarithme en utilisant la relation pr\'ec\'edente!

	D'abord rappelons que:
	
	Soit (relation d\'emontr\'ee dans la section d'Analyse Fonctionnelle  page \pageref{logarithms}):
	
	avec  $1<a<u$ et $(a,u)=1$.
	
	Soit  $y_n$ d\'efini par:
	
	Alors d\'emontrons que:
	
	En effet, pour $n=1$ nous avons:
	
	pour $n=2$ nous utilisons d'abord le fait que:
	
	donc:
	
	et puisque nous avions montr\'e que:
	
	etc... par r\'ecurrence ce qui d\'emontre notre droit d'utiliser ce changement d'\'ecriture.
	
	\begin{tcolorbox}[colframe=black,colback=white,sharp corners]
	\textbf{{\Large \ding{45}}Exemple:}\\\\
	Cherchons l'expression de la fraction continue de:
	
	Nous savons en jouant avec la d\'efinition du logarithme que:
	
	donc:
	
	donc  $q_1=1$. Nous avons alors:
	
	et puisque:
	
	il vient:
	
	Donc nous avons le premier quotient partiel:
	
	Et in extenso nous avons d\'ejà:
	
	Simplifions:
	
	Donc le premier quotient partiel peut s'\'ecrire:
	\end{tcolorbox}
	
	\begin{tcolorbox}[colframe=black,colback=white,sharp corners]
	
	et passons au deuxième quotient partiel. Nous savons d\'ejà pour cela que:
	
	donc il est imm\'ediat que $q_2=1$ et alors comme:
	
	nous avons:
	
	Il vient finalement:
	
	etc... etc.
	\end{tcolorbox}
	
	\begin{flushright}
	\begin{tabular}{l c}
	\circled{90} & \pbox{20cm}{\score{4}{5} \\ {\tiny 21 votes, 68.57\%}} 
	\end{tabular} 
	\end{flushright}
	
	
	%to make section start on odd page
	\newpage
	\thispagestyle{empty}
	\mbox{}
	\section{Th\'eorie des Ensembles}\label{set theory}
	\lettrine[lines=4]{\color{BrickRed}L}ors de notre \'etude des nombres, des op\'erateurs, et de la th\'eorie des nombres (dans les chapitres du même nom), nous avons assez souvent utilis\'e les termes "groupes", "anneaux", "corps", "homomorphisme", etc. et continuerons par la suite à le faire encore de nombreuses fois. Outre le fait que ces concepts soient d'une extrême importance, permettant de faire des d\'emonstrations ou de construire des concepts math\'ematiques indispensables à l'\'etude de la physique th\'eorique contemporaine (physique quantique des champs, th\'eories des cordes, modèle standard,...), ils permettent de comprendre les composants et les propri\'et\'es de base de la math\'ematique et de ses op\'erateurs en rangeant ceux-ci par cat\'egories distinctes. Ainsi, choisir de mettre la th\'eorie des ensembles en tant que cinquième chapitre de ce site est un choix tout à fait discutable puisque rigoureusement c'est par là que tout commence... Cependant, nous avions besoin d'exposer quand même la th\'eorie de la d\'emonstration ne serait-ce que pour les notations et les m\'ethodes dont il sera fait usage ici.
	
	Par ailleurs, lors de l'enseignement des math\'ematiques modernes dans le secondaire, voire primaire (ann\'ees 1970), on introduisit le langage des ensembles et l'\'etude pr\'ealable des relations binaires pour une approche plus rigoureuse de la notion de fonctions et d'applications (voir la d\'efinition plus loin) et de la math\'ematique en g\'en\'eral.

	\textbf{D\'efinition (\#\mydef):} Nous parlons de "\NewTerm{diagramme sagittal}\index{diagramme sagittal}" (ou de "\NewTerm{sch\'ema sagittal}\index{sch\'ema sagittal}" du latin sagitta = flèche) pour tout sch\'ema repr\'esentant une correspondance entre les composantes de deux ensembles reli\'es totalement ou partiellement par un ensemble de flèches. Quand il n'y a de pas de flèches nous parlons de "\NewTerm{Diagramme de Venn}\index{Diagramme de Venn}\label{Venn diagrams}".

La repr\'esentation graphique d'une fonction d\'efinie de l'ensemble $E=\left\lbrace -3,-2,-1,0,1,2,3\right\rbrace $ vers l'ensemble $F=\left\lbrace 0,1,2,4,9\right\rbrace $ conduirait au diagramme sagittal ci-dessous:

\begin{figure}[H]
\centering
\includegraphics{img/arithmetics/figure1.eps}
\caption{Exemple de diagramme sagittal d'un ensemble de d\'efinition à un autre ensemble d'arriv\'ee}
\end{figure}

Une relation de $E$ dans $E$ (usuellement not\'e $E\mapsto E$) fournirait un diagramme sagittal du type:

\begin{figure}[H]
\centering
\includegraphics{img/arithmetics/figure2.eps}
\caption{Fonction renvoyant dans son propre ensemble de d\'efinition}
\end{figure}

Le bouclage de chaque \'el\'ement montrant une "\NewTerm{relation r\'eflexive}\index{relation r\'eflexive}" et la pr\'esence syst\'ematique d'une flèche retour indiquant une "\NewTerm{relation sym\'etrique}\index{relation sym\'etrique}".

\textbf{D\'efinition (\#\mydef):} Si l'ensemble d'arriv\'ee est identique à l'ensemble de d\'epart, nous disons que nous avons une "\NewTerm{relation binaire}\index{relation binaire}".

Cependant le choix d'introduire la th\'eorie des ensembles dans les classes d'\'ecole a une raison aussi un peu autre. Au fait, dans un souci de rigueur interne (in extenso: non li\'ee à la r\'ealit\'e), une très grande partie des math\'ematiques a \'et\'e reconstruite à l'int\'erieur d'un seul cadre axiomatique, d\'enomm\'e donc "\NewTerm{th\'eorie des ensembles}\index{th\'eorie des ensembles}", dans le sens où chaque concept math\'ematique (autrefois ind\'ependant des autres) est ramen\'e à une d\'efinition dont tous les constituants logiques proviennent de ce même cadre: elle est consid\'er\'ee comme fondamentale. Ainsi, la rigueur d'un raisonnement effectu\'e au sein de la th\'eorie des ensembles est garantie par le fait que le cadre est "non-contradictoire" ou "consistant". Voyons les d\'efinitions qui construisent ce cadre.

\textbf{D\'efinitions (\#\mydef):}

\begin{itemize}
	\item[D1.] Nous appelons "\NewTerm{ensemble}\index{ensemble}" toute liste, collection ou rassemblement d'objets bien d\'efinis, explicitement ou implicitement.
	
	\item[D2.] Un "\NewTerm{Univers}\index{Univers (math\'ematiques)}" $U$ est un objet dont les constituants sont des ensembles.\\
	
	Il faut noter que ce que les math\'ematiciens appellent "Univers" n'est pas un ensemble! En fait il s'agit d'un modèle qui satisfait aux axiomes des ensembles.\\
	
	Effectivement, nous verrons que nous ne pouvons pas parler de l'ensemble de tous les ensembles (ce n'est pas un ensemble), pour d\'esigner l'objet qui est constitu\'e de tous les ensembles ainsi, nous parlons "d'Univers".

	\item[D3.] Nous appelons  "\NewTerm{\'el\'ements}\index{\'el\'ements (math\'ematiques)}" ou "\NewTerm{membres de l'ensemble}\index{membres de l'ensemble}" les objets appartenant à l'ensemble et nous notons:
	
	si $p$  est un \'el\'ement de l'ensemble $A$ et dans le cas contraire:
	
	Si $B$ est une "\NewTerm{partie}\index{partie d'un ensemble}" de $A$, ou "\NewTerm{sous-ensemble}\index{sous-ensemble}" de $A$, nous notons cela:
	
	Dès lors:
	
	
	\begin{tcolorbox}[colframe=black,colback=white,sharp corners]
	\textbf{{\Large \ding{45}}Exemples:}\\\\
	E1. $A=\lbrace 1,2,3 \rbrace$\\
	
	E2. $X=\lbrace X \mid x\:\text{est un entier positif} \rbrace$
	\end{tcolorbox}
	
	\item[D4.] Nous pouvons munir les ensembles d'un certain nombre de relations qui permettent de comparer leurs \'el\'ements (c'est utile parfois...) ou de comparer certaines de leurs propri\'et\'es. Ces relations sont appel\'ees "\NewTerm{relations de comparaisons}\index{relations de comparaisons}\label{comparison relations}" ou "\NewTerm{relations d'ordre}\index{relations d'ordre}" (\SeeChapter{voir section Op\'erateurs page \pageref{order relation}}).

\end{itemize}

	\begin{tcolorbox}[title=Remarques,colframe=black,arc=10pt]
	\textbf{R1.} La structure d'ensemble ordonn\'e a \'et\'e mise en place à la base dans le cadre de la th\'eorie des Nombres par Cantor et Dedekind.\\
	
	\textbf{R2.} Comme nous l'avons d\'emontr\'e dans la section sur les Op\'erateurs, $\mathbb{N},\mathbb{Z},\mathbb{Q},\mathbb{R}$ sont totalement ordonn\'es par les relations usuelles $\leq,\geq$. La relation $<$, souvent dite "\NewTerm{d'ordre strict}\index{ordre strict}", n'est pas une relation d'ordre car non r\'eflexive et non antisym\'etrique (\SeeChapter{voir section Op\'erateurs page \pageref{strict order}}). Par exemple, dans $\mathbb{N}$, la relation "$a$ divise $b$", souvent not\'ee par le symbole "|", est un ordre partiel.\\
	
	\textbf{R3.} Si $\mathcal{R}$ est un ordre sur $E$ et $F$ est une partie de $E$, la restriction à $F$ de la relation $\mathcal{R}$ est un ordre sur $F$, dit "\NewTerm{ordre induit par $\mathcal{R}$ dans $F$}".\\
	
	\textbf{R4.} Si $\mathcal{R}$ est un ordre sur $E$, la relation $\mathcal{R}'$ d\'efinie par:
	
	est un ordre sur $E$, dit "\NewTerm{ordre r\'eciproque}\index{ordre r\'eciproque}" de $\mathcal{R}$. L'ordre r\'eciproque de l'ordre usuel  $\leq$ est l'ordre not\'e $\geq$ ainsi que l'ordre r\'eciproque de l'ordre "$a$ divise $b$" dans $\mathbb{N}$ est l'ordre "$b$ est multiple de $a$".
	\end{tcolorbox}

	L'ensemble est l'entit\'e math\'ematique de base, dont l'existence est pos\'ee: il n'est pas d\'efini en tant que tel, mais par ses propri\'et\'es, donn\'ees par les axiomes. Il fait appel à une proc\'edure humaine: une sorte de fonction de cat\'egorisation, qui permet à la pens\'ee de distinguer plusieurs \'el\'ements qualifi\'es d'ind\'ependants.

\begin{theorem}
Nous pouvons d\'emontrer à partir de ces concepts, que le nombre de sous-ensembles d'un ensemble de cardinal $n$ est $2^n$.
\end{theorem}

\begin{dem}
Il y a d'abord l'ensemble vide $\varnothing$, soit $0$ \'el\'ement choisi parmi $n$, in extenso $C_{0}^{n}$ (notation du coefficient binomial non conforme à la norme ISO 31-11!) conform\'ement à ce que nous avons vu dans le section de Probabilit\'es:

et ainsi de suite...

	Le nombre de sous-ensembles (cardinal) de $E$ correspond donc à la sommation de tous les coefficients binomiaux:
	

	Or, nous avons  (\SeeChapter{voir section Calcul Alg\'ebrique page \pageref{binomial theorem}}):
	
	Donc:
	
	\begin{flushright}
		$\blacksquare$  Q.E.D.
	\end{flushright}
	\end{dem}

	\begin{tcolorbox}[colframe=black,colback=white,sharp corners]
	\textbf{{\Large \ding{45}}Exemple:}\\\\
	Consid\'erons l'ensemble $S=\left\lbrace x_1,x_2,x_3\right\rbrace$, nous avons l'ensemble des parties $P(S)$ constitu\'e par:
	\begin{itemize}
		\item[$-$] L'ensemble vide: $\left\lbrace \right\rbrace =\varnothing$
		\item[$-$] Les singletons: ${x_1},{x_2},{x_3}$
		\item[$-$] Les duets: ${x_1,x_2},{x_1,x_3},{x_2,x_3}$	
		\item[$-$] Lui-même: $\left\lbrace x_1,x_2,x_3\right\rbrace $
	\end{itemize}
	Tel que:
	
	Ce qui fait bien $8$ \'el\'ements!
	\end{tcolorbox}

	\begin{tcolorbox}[title=Remarque,colframe=black,arc=10pt]
	L'ordre dans lequel sont diff\'erenci\'es les \'el\'ements ne rentre pas en compte lors du comptage des parties de l'ensemble de d\'epart.
	\end{tcolorbox}

	En math\'ematique appliqu\'ee, nous travaillerons presque exclusivement avec des ensembles de nombres. Nous nous restreindrons donc à l'\'etude des d\'efinitions et propri\'et\'es de ces derniers.

Maintenant, formalisons les concepts de base permettant de travailler avec les ensembles les plus courants que nous rencontrons dans les cursus scolaires de base.

	\subsection{Axiomatique de Zermelo-Fraenkel}	\label{zermelo fraenkel axiomatic}

	L'axiomatique de Zermelo-Fraenkel, abr\'eg\'ee "\NewTerm{axiomatique ZF-C}\index{axiomatique ZF-C}", pr\'esent\'ee ci-dessous a \'et\'e formul\'ee par Ernst Zermelo puis pr\'ecis\'ee par Adolf Abraham Fraenkel au d\'ebut du 20ème siècle et compl\'et\'ee par l'axiome du choix (d'où le C majuscule dans ZF-C). Elle est consid\'er\'ee comme la plus naturelle dans le cadre de la th\'eorie des ensembles.

	\begin{tcolorbox}[title=Remarque,colframe=black,arc=10pt]
	 Il existe bien d'autres axiomatiques, bas\'ees sur le concept plus g\'en\'eral de "classe", comme celle d\'evelopp\'ee par von Neumann, Bernays et Gödel (pour les notations, voir le section traitant de la Th\'eorie De La D\'emonstration page \pageref{proof theory}).
	\end{tcolorbox}
	
	Strictement et techniquement parlant, les axiomes de ZF sont des \'enonc\'es du calcul des pr\'edicats du premier ordre (\SeeChapter{voir section Th\'eorie De La D\'emonstration page \pageref{first order predicate}}) \'egalitaire dans un langage ayant un seul symbole primitif pour l'appartenance (relation binaire). Ce qui suit doit donc seulement être perçu comme une tentative d'exprimer en français la signification attendue de ces axiomes.

	\begin{itemize}
		\item[A1.] Axiome d'extensionnalit\'e:\label{extensionality axiom}
		
		Deux ensembles sont \'egaux si, et seulement si ils ont les mêmes \'el\'ements. C'est ce que nous notons:
		
		Donc $A$ et $B$ sont \'egaux si tout \'el\'ement $x$ de $A$ appartient aussi à $B$ et tout \'el\'ement $x$ de $B$ appartient aussi à $A$.
		
		\item[A2.] Axiome de l'ensemble vide \label{empty set}:
		
		L'ensemble vide existe, nous le notons:
		
		et il n'a aucun \'el\'ement, son cardinal vaut donc $0$.
		
		En r\'ealit\'e cet axiome peut être d\'eduit à partir d'un autre axiome que nous verrons un peu plus loin mais il est pratique à introduire en tant que tel par commodit\'e p\'edagogique dans les petites classes.
		
		\item[A3.] Axiome de la paire:
		
		Si $A$ et $B$ sont deux ensembles, alors, il existe un ensemble $C$ contenant $A$ et $B$ et eux seuls comme \'el\'ements. Cet ensemble $C$ se note alors $\{A, B\}$.
		
		Du point de vue des ensembles consid\'er\'es comme des \'el\'ements cela donne:
		
		Cet axiome montre aussi l'existence du "\NewTerm{singleton}\index{singleton}" (single=seul) d'un ensemble not\'e:
		
		qui est un ensemble dont le seul \'el\'ement est $X$ (donc de cardinal unitaire). Il suffit pour cela d'appliquer l'axiome en posant l'\'egalit\'e entre $A$ et $B$.
		
		\item[A4.]  Axiome de la somme (dit aussi "axiome de l'union" ou encore "axiome de la r\'eunion"):
		
		Cet axiome permet de construire la r\'eunion (ou "union"\label{union}) d'un ensemble. Dit de façon plus commune: la r\'eunion d'une famille quelconque d'un ensemble, est un... ensemble.
		
		La r\'eunion d'une famille quelconque d'un ensemble est souvent not\'e:
		
		ou si nous prenons certains de ses \'el\'ements:
		
		
		\item[A5.] Axiome des parties (dit aussi "axiome de l'ensemble des parties"): 
		
		Il exprime que pour tout ensemble $A$, l'ensemble de ses parties $P(A)$ existe (ne pas confondre la notation "$P$" avec celle de la probabilit\'e!).
		
		Donc à tout ensemble $A$, nous pouvons associer un ensemble $B$ qui contient exactement les parties (in extenso les sous-ensembles) $C$ du premier:
			
		
		\item[A6.] Axiome de l'infini:
		
		Cet axiome exprime le fait qu'il existe un ensemble infini. Pour le formaliser, nous disons qu'il existe un ensemble, dit "\NewTerm{ensemble autosuccesseur}\index{ensemble autosuccesseur}" $A$ contenant $\varnothing$ (l'ensemble vide) tel que si $x$ appartient à $A$, alors $x \cup \lbrace x\rbrace$ appartient \'egalement à $A$:
		
		Cet axiome exprime donc que l'ensemble des entiers existe. Effectivement, $\mathbb{N}$ est ainsi le plus petit ensemble autosuccesseur, au sens de l'inclusion $\mathbb{N}=\lbrace \varnothing ,\lbrace \varnothing , \lbrace \varnothing , ...\rbrace \rbrace \rbrace$  et par convention nous notons (où nous construisons l'ensemble des Naturels):
		
			
		\item[A7.] Axiome de r\'egularit\'e (dit aussi "axiome de fondation"): 
		
		Le but principal de cet axiome est d'\'eliminer la possibilit\'e d'avoir A comme \'el\'ement de lui-même.
		
		Ainsi, pour tout ensemble non vide $A$, il existe un ensemble $B$ qui est \'el\'ement de $A$ tel qu'aucun \'el\'ement de $A$ ne soit \'el\'ement de $B$ (il faut bien diff\'erencier le niveau du langage utilis\'e, un ensemble et ses \'el\'ements n'ont pas le même statut) ce que nous notons:
		
		et en cons\'equence nous obtenons ce que nous voulons. c'est-à-dire:
		
		\begin{dem}
		En effet, soit $A$ un ensemble tel que $A \in A$. Consid\'erons le singleton $\lbrace A \rbrace$, ensemble dont le seul \'el\'ement est $A$. D'après l'axiome de fondation, nous devons avoir un \'el\'ement de ce singleton qui n'a aucun \'el\'ement en commun avec lui. Mais le seul \'el\'ement possible est A lui-même, c'est-à-dire que nous devons avoir:
		
		Or par hypothèse, $A \in A$ et par construction $A \in \lbrace A \rbrace$. Donc:
		
		ce qui contredit l'assertion pr\'ec\'edente. Dès lors
		
		\end{dem}
		\begin{flushright}
			$\blacksquare$  Q.E.D.
		\end{flushright}
		
		\item[A8.] Axiome de remplacement (dit aussi "sch\'ema de remplacement"):
		
		Cet axiome exprime le fait que si une formule $f$ est une fonctionnelle alors pour tout ensemble $A$, il existe un ensemble $B$ constitu\'e exactement des images des \'el\'ements $A$ par cette fonction.

		Soient, de manière un peu plus formelle, l'ensemble $A$ d'\'el\'ements $a$ et la relation binaire $f$ (qui est donc en toute g\'en\'eralit\'e une fonctionnelle), il existe un ensemble $B$ constitu\'e des \'el\'ements $b$ tel que $f(a,b)$ soit vraie. Si $f$ est une fonction où $b$ est non libre cela signifie alors que:
		
		De manière technique nous \'ecrivons cet axiome sous la forme:
		
		Donc pour tout ensemble $A$ et tout \'el\'ement qu'il contient, il existe un et un seul $b$ d\'efini par la fonctionnelle $f$ tel qu'il existe un ensemble $B$ où pour tout \'el\'ement $a$ appartenant à l'ensemble $A$ il existe un $b$ appartenant à l'ensemble $B$ d\'efini par la fonctionnelle $f$.
	
		Voyons un exemple avec le pr\'edicat binaire suivant qui pour la valeur de tout a de A d\'etermine la valeur de tout $b$ de $B$:
		
		Donc de la connaissance que $a$ vaut $1$ nous en d\'erivons que $b$ vaut $2$ et de manière similaire (in extenso par remplacement) si $a$ vaut $3$, nous en d\'erivons que $b$ vaut $4$.
	
		Nous voyons bien au travers de ce petit exemple la relation forte qu'il y a à consid\'erer le pr\'edicat $P$ comme une fonction naïve! Par ailleurs, comme il y une infinit\'e possible de fonctions $f$, le sch\'ema de remplacement est consid\'er\'e comme une infinit\'e d'axiomes.
	
		\item[A9.] Axiome de s\'election (dit aussi "sch\'ema de compr\'ehension"):
		
		Cet axiome exprime simplement que pour tout ensemble $A$ et toute propri\'et\'e $P$ exprimable dans le langage de la th\'eorie des ensembles, l'ensemble des \'el\'ements de $A$ qui satisfont la propri\'et\'e $P$ existe.

		Donc de manière plus formelle, à tout ensemble $A$ et toute condition ou proposition $P(x)$, il correspond un ensemble $B$ dont les \'el\'ements sont exactement les \'el\'ements $x$ de $A$ pour lesquels $P(x)$ est vraie. C'est ce que nous notons:
		
		De manière plus complète et rigoureuse nous avons en r\'ealit\'e pour toute fonctionnelle $f$ ne comportant pas $a$ comme variable libre:
		
		C'est typiquement l'axiome qui nous sert à construire l'ensemble des nombres pairs:
		
		ou à d\'emontrer l'existence de l'ensemble vide (et qui rend caduc l'axiome de l'ensemble vide) car il suffit de poser qu'il existe un ensemble satisfaisant la propri\'et\'e:
		
		et ce quel que soit l'ensemble $A$. Et seulement l'ensemble vide satisfait cette propri\'et\'e de par l'axiome de s\'election.
		
		Le respect des conditions très strictes de cet axiome permet d'\'eliminer les paradoxes de la "\NewTerm{th\'eorie naïve des ensembles}\index{th\'eorie naïve des ensembles}", comme le paradoxe de Russel ou le paradoxe de Cantor qui ont invalid\'e la th\'eorie naïve des ensembles.
		
		Consid\'erons par exemple l'ensemble $R$ de Russell de tous les ensembles qui ne s'auto-contiennent pas (notez bien que nous donnons une propri\'et\'e de $R$ sans expliciter quel est cet ensemble):
		
		Le problème est de savoir si $R$ se contient ou non. Si $R \in R$, alors, $R$ s'auto-contient, et, par d\'efinition $R \not\in R$ et inversement. Chaque possibilit\'e est donc contradictoire.
	
		Si maintenant nous d\'esignons par $C$ l'ensemble de tous les ensembles (l'Universel de Cantor), nous avons en particulier:
		
		ce qui est impossible (c.-à-d. par exemple avec la puissance du continu de l'ensemble de r\'eels), d'après le th\'eorème de Cantor (\SeeChapter{voir section Th\'eorie des Nombres page \pageref{Cantor's diagonal}}).
	
		Ces "paradoxes" (ou "antinomies syntaxiques") proviennent d'un non-respect des conditions d'application de l'axiome de s\'election: pour d\'efinir $E$ (dans l'exemple de Russel), il doit exister une proposition P qui porte sur l'ensemble $R$, qui doit être explicit\'ee. La proposition d\'efinissant l'ensemble de Russell ou celui de Cantor n'indique pas quel est l'ensemble $E$. Elle est donc invalide!
		
		Un exemple fort sympathique et fort connu (c'est la raison pour laquelle nous le pr\'esentons) permet peut-être de mieux comprendre (il s'agit du paradoxe de Russel dont nous avons d\'ejà parl\'e plus longuement dans la sectopm de Th\'eorie De La D\'emonstration page \pageref{russell paradox}):
	
		Un jeune \'etudiant se rendit un jour chez son barbier. Il engagea la conversation et lui demanda s'il avait de nombreux concurrents dans sa jolie cit\'e. De manière apparemment innocente, le barbier lui r\'epondit: "Je n'ai aucune concurrence. En effet, de tous les hommes de la cit\'e, je ne rase \'evidemment pas ceux qui se rasent eux-mêmes, mais j'ai le bonheur de raser tous ceux qui ne se rasent pas eux-mêmes."

	En quoi donc, une telle affirmation si simple put-elle mettre en d\'efaut la logique de notre jeune \'etudiant si malin ?

	La r\'eponse est en effet innocente, jusqu'au moment où nous d\'ecidons de l'appliquer au cas du barbier: Se rase-t-il lui-même, Oui ou Non?

	Supposons qu'il se rase lui-même: il entre dans la cat\'egorie de ceux qui se rasent eux-mêmes, dont le barbier a pr\'ecis\'e qu'il ne les rasait \'evidemment pas.... Donc il ne rase pas lui-même........

	Très bien! Supposons alors qu'il ne se rase pas lui-même: il entre alors dans la cat\'egorie de ceux qui ne se rasent pas eux-mêmes, dont le barbier a pr\'ecis\'e qu'il les rasait tous. Donc il se rase lui-même.

	Finalement, ce malheureux barbier est dans une position \'etrange: s'il se rase lui-même, il ne se rase pas, et s'il ne se rase pas lui-même, il se rase. Cette logique est autodestructrice, stupidement contradictoire, rationnellement irrationnelle.

	Vient alors l'axiome de s\'election: Nous excluons le barbier de l'ensemble des personnes auxquelles s'applique la d\'eclaration. Car en r\'ealit\'e, le problème vient du fait que le barbier est un \'el\'ement de l'ensemble de tous les hommes de la cit\'e. Ainsi, ce qui s'applique à tous les hommes ne s'applique pas au cas individuel du barbier.
	
		\item[A10.] Axiome du choix:
		
		\'etant donn\'e un ensemble  $A$ d'ensembles non vides mutuellement disjoints, il existe un ensemble $B$ (l'ensemble de choix pour $A$) contenant exactement un \'el\'ement pour chaque membre de $A$.
		
		Indiquons cependant que la question de l'axiomatisation et donc des fondements se trouva quand même \'ebranl\'ee de deux questions à l'\'epoque de leur construction: quels axiomes valides doivent être choisis et dans un système d'axiomes la math\'ematique est-elle coh\'erente (ne risque-t-on pas de voir apparaître une contradiction)?
		
		La première question fut soulev\'ee d'abord par l'hypothèse du continu: si nous pouvons mettre deux ensembles de nombres en correspondance terme à terme, ils ont le même nombre d'\'el\'ements (cardinal). Nous pouvons mettre en correspondance les entiers  $\mathbb{N}$ avec les rationnels $\mathbb{Q}$ comme nous l'avons d\'emontr\'e dans le section sur les Nombres (page \pageref{natural and rational numbers equipotence}), ils ont donc même cardinal, nous ne pouvons par contre mettre en correspondance les entiers avec les r\'eels. La question est alors de savoir s'il y a un ensemble dont le nombre d'\'el\'ements serait situ\'e entre les deux ou pas? Cette question est importante pour construire la th\'eorie classique de l'analyse et les math\'ematiciens choisissent en g\'en\'eral de dire qu'il n'y en a pas, mais nous pouvons aussi dire le contraire.
		
		En fait l'hypothèse du continu est li\'ee de manière plus profonde à l'axiome du choix qui peut aussi être formul\'e de la manière suivante: si $C$ est une collection d'ensembles non vides alors nous pouvons choisir un \'el\'ement de chaque ensemble de la collection. Si $C$ a un nombre fini d'\'el\'ements ou un nombre d\'enombrable d'\'el\'ements, l'axiome semble assez \'evident: nous pouvons ranger les ensembles de $C$ en les num\'erotant, et le choix d'un \'el\'ement dans chaque ensemble est simple. Là où ça se complique c'est lorsque l'ensemble $C$ a la puissance du continu: comment choisir des \'el\'ements s'il n'y pas la possibilit\'e de les num\'eroter?
	
		Finalement en 1938 Kurt Gödel montre que la th\'eorie des ensembles est coh\'erente sans l'axiome du choix et sans l'hypothèse du continu aussi bien qu'avec! Et pour clore tout ça Paul Cohen montre en 1963 que l'axiome du choix et l'hypothèse du continu ne sont pas li\'es.
	\end{itemize}
	
	Ok pour faire un r\'esum\'e p\'edagogique de tout ça, consid\'erons la figure suivante (en excluant l’axiome de choix):
	\begin{figure}[H]
		\label{continuous distributions}
		\centering
		\includegraphics{img/arithmetics/zf_axioms.jpg}	
		\caption[R\'esum\'e visuel des axiomes de Zermelo-Fraenkel]{R\'esum\'e visuel des axiomes de Zermelo-Fraenkel (source:?)}
	\end{figure}

\subsubsection{Cardinaux}\label{cardinal}

	\textbf{D\'efinition (\#\mydef):} Des ensembles sont dits "\NewTerm{\'equipotents}\index{\'equipotents}" s'il existe une bijection (correspondance biunivoque) entre ces ensembles. Nous disons qu'ils ont alors même "\NewTerm{cardinal}\index{cardinal}" que la norme ISO 3111 pr\'econise d'\'ecrire $card(S)$ mais dans le pr\'esent livre nous utiliserons aussi la notation $\mathrm{Card}(S)$ (de nombreux livres am\'ericains utilisent la notation qui ressemble en tout point à la valeur absolue $\mid S \mid$ ou $\# S$).
	
	Ainsi, plus rigoureusement, un cardinal (qui quantifie le nombre d'\'el\'ements contenus dans l'ensemble) est une classe d'\'equivalence (\SeeChapter{voir la section Opr\'eateurs page \pageref{equivalence class}}) pour la relation d'\'equipotence.

	\begin{tcolorbox}[title=Remarque,colframe=black,arc=10pt]
	 Cantor est le principal cr\'eateur de la th\'eorie des ensembles, sous une forme que nous qualifions aujourd'hui de "th\'eorie naïve des ensembles". Mais, à côt\'e de consid\'erations \'el\'ementaires, sa th\'eorie comportait des niveaux d'abstraction \'elev\'es. La vraie nouveaut\'e de la th\'eorie de Cantor, c'est qu'elle permet de parler de l'infini. Par exemple, une id\'ee importante de Cantor a justement \'et\'e de d\'efinir "l'\'equipotence".
	\end{tcolorbox}
	
	Si nous \'ecrivons $c_1=c_2$ en tant qu'\'egalit\'e de cardinaux, nous entendons alors par là qu'il existe deux ensembles \'equipotents $A$ et $B$ tels que:
	
	Les cardinaux peuvent donc être compar\'es. L'ordre ainsi d\'efini est une relation d'ordre total (\SeeChapter{voir section Op\'erateurs page \pageref{total order relation}}) entre les cardinaux (la preuve que la relation d'ordre est totale utilise l'axiome du Choix et la preuve qu'elle soit antisym\'etrique est connue sous le nom de th\'eorème de Cantor-Bernstein que nous d\'emontrons d'ailleurs plus bas).

	Dire que  $c_1<c_2$ signifie dans un vocabulaire simple que $A$ est \'equipotent à une partie propre de  $B$, mais  $B$ n'est \'equipotent à aucune partie propre de $A$. Les math\'ematiciens diraient que le $\mathrm{Card}(A)$ est plus petit ou \'egal au $\mathrm{Card}(B)$ s'il existe une injection de $A$ dans $B$.
	
	Nous avons vu lors de notre \'etude des nombres (\SeeChapter{voir section Nombres page \pageref{countable set}}), en particulier des nombres transfinis, qu'un ensemble \'equipotent (ou en bijection) à $\mathbb{N}$ \'etait dit "\NewTerm{ensemble d\'enombrable}\index{ensemble d\'enombrable}".
	
	Voyons cette notion un petit peu plus dans les d\'etails:
	
	Soit $A$ un ensemble, s'il existe un entier $n$ tel qu'il y ait au moins à chaque \'el\'ement de $A$ un correspondant dans l'ensemble $\left\lbrace 1,2, ...,n\right\rbrace $ (au fait rigoureusement il s'agit d'une bijection... concept que nous d\'efinirons plus tard) nous disons alors que le cardinal de $\mathrm{Card}(A)$, not\'e $\mathrm{Card}(A)$ ou $\mathrm{card}(A)$, est un "\NewTerm{cardinal fini}\index{cardinal fini}" et vaut $n$.
	
	Dans le cas contraire, nous disons que l'ensemble $A$  est de  "\NewTerm{cardinal infini}\index{cardinal infini}" et nous posons:
	
	Un ensemble $A$ est donc "\NewTerm{d\'enombrable}\index{ensemble d\'enombrable}\label{countable set}" s'il existe une bijection entre $A$ et $\mathbb(N)$. Un ensemble de nombre $A$ est "au plus d\'enombrable" s'il existe une bijection entre $A$ et une partie  $\mathbb(N)$. Un ensemble au plus d\'enombrable est donc soit de cardinal fini, soit d\'enombrable.

Nous v\'erifions dès lors les propositions suivantes:
\begin{itemize}
	\item[P1.] Une partie d'un ensemble d\'enombrable est au plus d\'enombrable.

	\item[P2.] Un ensemble contenant un ensemble non-d\'enombrable n'est lui aussi pas d\'enombrable.

	\item[P3.] Le produit de deux ensembles d\'enombrables est d\'enombrable.
\end{itemize}

	\begin{tcolorbox}[title=Remarque,colframe=black,arc=10pt]
	Nous pouvons restreindre un ensemble de nombres par rapport à l'\'el\'ement nul et aux \'el\'ements n\'egatifs ou positifs qu'il contient et dès lors nous notons (exemple pour l'ensemble des r\'eels):
	
	Ces notions \'etant analogues pour $\mathbb{N},\mathbb{Z},\mathbb{Q}$ (l'ensemble des nombres complexes $\mathbb{C}$ n'\'etant pas ordonn\'e, la deuxième et troisième ligne ne s'y appliquent pas).
	\end{tcolorbox}
	Donc tout sous-ensemble infini de $\mathbb{N}$ est \'equipotent à $\mathbb{N}$ lui-même, ce qui peut sembler contre-intuitif au premier abord...!
	
	En particulier, il y a autant d'entiers naturels pairs que d'entiers naturels quelconques (utiliser la bijection $f(n)=2n$) de $\mathbb{N}$ vers $P$, où $P$ d\'esigne l'ensemble des entiers naturels pairs. Autant d'entiers relatifs que d'entiers naturels, autant d'entiers relatifs que de nombres rationnels (voir la section Nombres page \pageref{natural and rational numbers equipotence} pour la d\'emonstrations).

	Nous pouvons donc \'ecrire:\label{aleph}:
	
	et plus g\'en\'eralement, toute partie infinie de $\mathbb{Q}$ est d\'enombrable.
	
	Nous avons donc un r\'esultat important: tout ensemble infini possède donc une partie infinie d\'enombrable.
	
	Puisque nous avons d\'emontr\'e dans la section traitant des nombres (page \pageref{power of the continuum}) que l'ensemble des r\'eels $\mathbb{R}$  avait la "\NewTerm{puissance du continu}\index{puissance du continu}" et que l'ensemble des nombres naturels  $\mathbb{N}$ \'etait de cardinal transfini $\aleph_0$, Cantor souleva la question s'il existait un cardinal transfini entre $\aleph_0$ et le cardinal de $\mathbb{R}$? Autrement dit, nous avons donc une quantit\'e infinie de nombres entiers, et une quantit\'e encore plus grande de nombres r\'eels. Alors, existe-t-il un infini qui soit à la fois plus grand que celui des entiers et plus petit que celui des nombres r\'eels?

	Le problème apparaît en \'ecrivant $\aleph_0$ le cardinal de $\mathbb{N} $ et $\aleph_1$ (nouveau) le cardinal de $\mathbb{R}$ et en proposant de d\'emontrer ou de contredire ce qui suit:
	
	selon la loi combinatoire qui donne le nombre d'\'el\'ements que l'on peut obtenir de tous les sous-ensembles d'un ensemble (comme nous l'avons d\'ejà prouv\'e).

	Ce problème se r\'esout d'une façon assez \'etonnante. D'abord, en 1938, un des plus grands logiciens du 20ème siècle, Kurt Gödel, d\'emontra que l'hypothèse de Cantor n'\'etait pas r\'efutable, c'est-à-dire qu'on ne pourrait jamais d\'emontrer qu'elle \'etait fausse. Puis en 1963, le math\'ematicien Paul Cohen boucla la boucle. Il d\'emontra qu'on ne pourrait jamais non plus d\'emontrer qu'elle \'etait vraie!!! Nous pouvons conclure à juste raison que Cantor avait perdu la raison à chercher à d\'emontrer un problème qui ne pouvait pas l'être.Le reste de sa vie, Cantor essaya, en vain, de d\'emontrer ce r\'esultat que l'on nomma "\NewTerm{l'hypothèse du continu}\index{l'hypothèse du continu}". Il n'y r\'eussit pas et sombra dans la folie. En 1900, au congrès international des math\'ematiciens, Hilbert estima qu'il s'agissait là d'un des 23 problèmes majeurs qui devraient être r\'esolus au 20ème siècle.

	Ce problème se r\'esout d'une façon assez \'etonnante. D'abord, en 1938, un des plus grands logiciens du 20ème siècle, Kurt Gödel, d\'emontra que l'hypothèse de Cantor n'\'etait pas r\'efutable, c'est-à-dire qu'on ne pourrait jamais d\'emontrer qu'elle \'etait fausse. Puis en 1963, le math\'ematicien Paul Cohen boucla la boucle. Il d\'emontra qu'on ne pourrait jamais non plus d\'emontrer qu'elle \'etait vraie!!! Nous pouvons conclure à juste raison que Cantor avait perdu la raison à chercher à d\'emontrer un problème qui ne pouvait pas l'être.

\subsubsection{Produit cart\'esien}\label{cartesian product}

	\textbf{D\'efinition (\#\mydef):} Si $E$ et $F$ sont deux ensembles, nous appelons "\NewTerm{produit cart\'esien de E par F}\index{produit cart\'esien} l'ensemble not\'e $E \times F$ (à ne pas confondre avec le produit vectoriel!) form\'e de tous les couples possibles $(e,f)$ où $e$ est un \'el\'ement de $E$ et $f$ un \'el\'ement de $F$.


	Autrement \'ecrit:
	
	Nous notons le produit cart\'esien de $E$ par lui-même: 
	
	et nous disons alors $E^2$ est "\NewTerm{l'ensemble des couples d'\'el\'ements de $E$}".

	Nous pouvons effectuer le produit cart\'esien d'une suite d'ensembles $E_1 \times E_2 \times ... \times E_n$ et ainsi obtenir l'ensemble des $n$-uplets $(e_1,e_2,...,e_n)$ où $e_1 \in E_1,e_2 \in E_2,...,e_n \in E_n$.

	Dans le cas où tous les ensembles $E_i$ sont identiques à $E$, le produit cart\'esien $E_1 \times E_2 \times ... \times E_n$ se note bien \'evidemment $E^n$. Nous disons alors que $e^n$ est "\NewTerm{l'ensemble des $n$-uplets d'\'el\'ements de $E$}".

	Si $E$ et $F$ sont finis alors le produit cart\'esien $E \times F$ est fini. De plus:
	
	De là, nous voyons que si les ensembles $E_1, E_2, ..., E_n$ sont finis alors le produit cart\'esien $E_1 \times E_2 \times ... \times E_n$ est aussi fini et nous avons:
	
	En particulier:
	
	si $E$ est un ensemble fini.

	\begin{tcolorbox}[colframe=black,colback=white,sharp corners]
	\textbf{{\Large \ding{45}}Exemples:}\\\\
	E1. Si $\mathbb{R}$ est l'ensemble des nombres r\'eels,  $\mathbb{R}^2$ est alors l'ensemble des couples de r\'eels. Dans le plan rapport\'e à un repère, tout point $M$ admet des coordonn\'ees qui sont un \'el\'ement de $\mathbb{R}^2$.\\
	
	E2. Lorsque nous lançons deux d\'es dont les faces sont num\'erot\'ees de $1$ à $6$, chaque d\'e peut être symbolis\'e par l'ensemble $E=\left\lbrace 1,2,3,4,5,6\right\rbrace $.  Le r\'esultat d'un lancer est alors un \'el\'ement de $E^2=E \times E$. Le cardinal de $E \times E$ est alors $36$.  Il y a donc 36 r\'esultats possibles quand nous lançons 2 d\'es dont les faces sont num\'erot\'ees de $1$ à $6$.
	\end{tcolorbox}

	\begin{tcolorbox}[title=Remarque,colframe=black,arc=10pt]
	La th\'eorie des ensembles ainsi que le concept de cardinal constituent la base th\'eorique des logiciels de bases de donn\'ees relationnelles.
	\end{tcolorbox}
	
\subsubsection{Intervalles (bornes)}
	Soit $M$ un ensemble de nombres quelconques de façon à ce que $M \subset \mathbb{R}$ (exemple particulier mais fr\'equent). Nous avons comme d\'efinitions:
	
	\textbf{D\'efinitions (\#\mydef):}
	\begin{itemize}
		\item[D1.] $x \in \mathbb{R}$ est appel\'e "\NewTerm{borne sup\'erieure}\index{borne sup\'erieure}" ou "\NewTerm{majorant}\index{majorant}" de l'ensemble $M$, si $x \geq m$ pour $x \geq m$ pour $\forall m \in M$. Inversement, nous parlons de "\NewTerm{borne inf\'erieure}\index{borne inf\'erieure}" ou de "\NewTerm{minorant}\index{minorant}" (il ne faut donc pas confondre le concept de borne avec le concept d'intervalle!).
		
		\item[D2.]  Soit $M \subset \mathbb{R},M \neq \varnothing$.  $x \in \mathbb{R}$ est appel\'e "\NewTerm{plus petite borne sup\'erieure}\index{plus petite borne sup\'erieure}" not\'e:
		
		de $M$ si $x$ est une borne sup\'erieure de $M$ et si pour toute borne sup\'erieure $y \in \mathbb{R}$ nous avons $x \leq y$. Inversement, nous parlons de "\NewTerm{plus petite borne inf\'erieure}\index{plus petite borne inf\'erieure}" que nous notons:
				
	\end{itemize}
	Les d\'efinitions sont \'equivalentes dans le cadre de l'analyse fonctionnelle (voir du même nom page \pageref{functional analysis}) puisque les fonctions sont d\'efinies sur des ensembles.

	Effectivement, soit $f$  une fonction dont le domaine de d\'efinition I balaie tout equation. Ce que nous notons:
		
		et soit $x_0 \in \mathbb{R}$.

\textbf{D\'efinitions (\#\mydef):}
	\begin{itemize}
		\item[D1.] Nous disons que $f$ pr\'esente un "\NewTerm{maximum global}\index{maximum global}" en $x_0$ si:
		
		
		\item[D2.] Nous disons que $f$ pr\'esente un "\NewTerm{minimum global}\index{minimum global}" en $x_0$ si:
		
		Dans chacun de ces deux cas, nous disons que $f$ pr\'esente un "\NewTerm{extremum global}\index{extremum global}" en $x_0$ (c'est un concept que nous retrouverons souvent dans section de M\'ecanique Analytique page \pageref{lagrangian mechanics} et M\'ethodes Num\'eriques page \pageref{numerical methods}!).
		\begin{figure}[H]
			\centering
			\includegraphics{img/arithmetics/global_local_maximum_minimum.jpg}	
			\caption[Exemple de maximum Global/Local]{Exemple de maximum Global/Local (source: Wikip\'edia)}
		\end{figure}
		
		\item[D3.] $f$ est "\NewTerm{major\'ee}" s'il existe un r\'eel $M$ tel que $\forall x \in I, f(x) \leq M$. Dans ce cas, la fonction possède une borne sup\'erieure de $f$ sur son domaine de d\'efinition $I$ not\'ee traditionnellement:
		
		et elle repr\'esente donc la plus petite borne sup\'erieure (le plus petit majorant).
		
		\item[D4.] $f$ est "\NewTerm{minor\'ee}" s'il existe un r\'eel $M$ tel que $\forall x \in I, f(x) \geq M$. Dans ce cas, la fonction possède une borne inf\'erieure de $f$ sur son domaine de d\'efinition $I$ not\'ee traditionnellement:
		
		et elle repr\'esente la plus grande borne inf\'erieure (le plus grand minorant).
		
		\item[D5.] Nous disons que $f$ est "\NewTerm{born\'ee}\index{fonctione born\'ee}" si elle est à la fois major\'ee et minor\'ee (c'est le cas des fonctions trigonom\'etriques).
	\end{itemize}

	\subsection{Op\'erations ensemblistes}\label{set operations}
	Nous pouvons construire à partir d'au moins trois ensembles $A, B, C$ l'ensemble des op\'erations (dont nous devons les notations à Richard Dedekind) existant dans la th\'eorie des ensembles (très utiles dans l'\'etude des probabilit\'es et statistiques).

	\begin{tcolorbox}[title=Remarque,colframe=black,arc=10pt]
	Certaines des notations pr\'esentes ci-dessous se retrouveront fr\'equemment dans des th\'eorèmes complexes, il est donc n\'ecessaire de bien comprendre de quoi il en retourne.
	\end{tcolorbox}	

	Ainsi, nous pouvons construire les op\'erations ensemblistes suivantes:

	\subsubsection{Inclusion}
	Dans le cas le plus simple, nous d\'efinissons "\NewTerm{l'inclusion}\index{inclusion}" par:
	
	En langage non sp\'ecialis\'e voici ce qu'il faut lire: $A$ est "inclus" (ou "fait partie", ou encore est un "sous-ensemble") dans $B$ alors pour tout $x$ appartenant à $A$ chacun de ces $x$ appartient aussi à $B$:
	\begin{figure}[H]
		\begin{center}
			\includegraphics{img/arithmetics/inclusion.eps}
		\end{center}	
		\caption{Exemple visuel (diagramme d'Euler) de l'inclusion}
	\end{figure}
	où le $U$ dans le coin inf\'erieur droit de la figure repr\'esente l'univers (de Cantor).

	De ceci il en d\'ecoule les propri\'et\'es suivantes:
	\begin{itemize}
		\item[P1.] Si $A \in B$ et $B \in A$ alors cela implique $A=B$ et r\'eciproquement.
		
		\item[P2.] Si $A \in B$ et $B \in C$ alors cela implique $A \in C$.
	\end{itemize}

\subsubsection{Intersection}

	Dans le cas le plus simple, nous d\'efinissons "\NewTerm{l'intersection}\index{intersection}\label{intersection}" par:
	
	En langage non sp\'ecialis\'e voici ce qu'il faut lire: "L'intersection" des ensembles $A$ et $B$ consiste en l'ensemble des \'el\'ements qui se trouvent à la fois dans $A$ et dans $B$:
	\begin{figure}[H]
		\begin{center}
			\includegraphics{img/arithmetics/intersection.eps}
		\end{center}	
		\caption{Exemple visuel (diagramme d'Euler) de l'intersection}
	\end{figure}
	Plus g\'en\'eralement, si $(A_i)$ est une famille d'ensembles index\'es par $i \in I$, l'intersection des $(A_i),i \in I$ est not\'ee:
	
	Cette intersection est donc d\'efinie explicitement par:
	
	C'est-à-dire que l'intersection de la famille d'ensembles index\'es comprend tous les $x$ qui se trouvent dans chaque ensemble de tous les ensembles de la famille.
	
	Soient deux ensembles $A$ et $B$, nous disons qu'ils sont "\NewTerm{disjoints}\index{disjoint}" si et seulement si:
	
	Par ailleurs, si:
	
	Les math\'ematiciens notent cela:
	
	et l'appellent "\NewTerm{union disjointe}\index{union disjointe}".
	
	On plaisante parfois en disant que la connaissance se construit sur la disjonction... (ceux qui comprendront appr\'ecieront...).
	
	\textbf{D\'efinition (\#\mydef):}  Une collection $S=\{S_i\}$ d'ensembles non vides forment une "\NewTerm{partition}\index{partition}" d'un ensemble $A$ si les propri\'et\'es suivantes sont v\'erifi\'ees:
	\begin{itemize}
		\item[P1.] $\forall S_i,S_j \in S$ et $i \neq j \Rightarrow S_i \cap S_j = \varnothing$
		
		\item[P2.] $A= \displaystyle\bigcup_{S_i \in S} S_i$
	\end{itemize}
	
	\begin{tcolorbox}[colframe=black,colback=white,sharp corners]
	\textbf{{\Large \ding{45}}Exemple:}\\\\
	 L'ensemble des nombres pairs et l'ensemble des nombres impairs forment une partition de  $\mathbb{Z}$.
	\end{tcolorbox}
	
	La loi d'intersection $\cap$ est une loi commutative (voir plus loin la d\'efinition du concept de "loi") telle que:
	

	\subsubsection{Union}
	Dans le cas le plus simple, nous d\'efinissons "\NewTerm{l'union}\index{union}" (aussi parfois nom\'ee "fusion" ou "r\'eunion") par:
	
	En langage non sp\'ecialis\'e voici ce qu'il faut lire: "L'union" des ensembles $A$ et $B$ consiste en l'ensemble des \'el\'ements qui se trouvent dans $A$ et en plus dans $B$:
	\begin{figure}[H]
		\begin{center}
			\includegraphics{img/arithmetics/union.eps}
		\end{center}	
		\caption{Exemple visuel (diagrammer d'Euler) de l'union}
	\end{figure}
	Plus g\'en\'eralement, si $(A_i)$ est une famille d'ensembles index\'es par $i \in I$, l'union des $(A_i),i \in I$ est not\'ee:
	
	Cette r\'eunion est explicitement d\'efinie par:
	
	C'est-à-dire que la r\'eunion de la famille d'ensembles index\'es comprend tous les $x$ pour lesquels il existe un ensemble index\'e par $i$ tel que $x$ soit inclus dans cet ensemble $A_i$.

	Nous avons les propri\'et\'es de distributivit\'e suivantes:
	
	La loi de r\'eunion $\cup$ est une loi commutative (voir plus loin la d\'efinition du concept de "loi") telle que:
	
	Nous appelons par ailleurs "\NewTerm{lois d'idempotences}\index{lois d'idempotences}" les relations (pr\'ecisons cela pour la culture g\'en\'erale):
	
	et "\NewTerm{lois d'absorptions}\index{lois d'absorptions}" les lois:
	
	Les lois de r\'eunion et d'intersection sont associatives telles que:
	
et distributives telles que:
	

	Si nous rappelons le concept de "cardinal" (voir ci-dessus), nous avons avec les op\'erations d\'efinies pr\'ec\'edemment, la relation suivante:
	
	Donc si $A\cap B=\varnothing$:
	

\subsubsection{Diff\'erence}

	Dans le cas le plus simple, nous d\'efinissons la "\NewTerm{diff\'erence}\index{diff\'erence}" par:
	
	En langage non sp\'ecialis\'e voici ce qu'il faut lire: La "diff\'erence" des ensembles $A$ et $B$ consiste en l'ensemble des \'el\'ements qui se trouvent uniquement dans $A$ (et qui excluent donc les \'el\'ements de $B$):
	\begin{figure}[H]
		\begin{center}
			\includegraphics{img/arithmetics/difference.eps}
		\end{center}	
		\caption{ Exemple visuel (diagramme d'Euler) de la diff\'erence}
	\end{figure}
	
\subsubsection{Diff\'erence sym\'etrique}
	Soit $U$ un ensemble. Pour tout $A,B\subseteq U$ nous d\'efinissons la "\NewTerm{diff\'erence sym\'etrique}\index{diff\'erence sym\'etrique}\label{symmetric difference}" entre $A$ et $B$ par:
	
	En langage non sp\'ecialis\'e voici ce qu'il faut lire: La "diff\'erence sym\'etrique" des ensembles $A$ et $B$ consiste en l'ensemble des \'el\'ements qui se trouvent uniquement dans $A$ et de ceux se trouvant uniquement dans $B$ (nous laissons donc de côt\'e les \'el\'ements qui sont communs):
	\begin{figure}[H]
		\begin{center}
			\includegraphics{img/arithmetics/symetric_difference.eps}
		\end{center}	
		\caption{- Exemple visuel (diagramme d'Euler) de la diff\'erence sym\'etrique}
	\end{figure}
Donc comme nous pouvons le voir, nous avons:

Les propri\'et\'es triviales sont les suivantes:
\begin{itemize}
	\item[P1.] Commutativit\'e: $A \bigtriangleup B= B \bigtriangleup A$
	\item[P2.] Compl\'ementarit\'e (voir d\'efinition plus bas): $A^c \bigtriangleup B^c = A \bigtriangleup B$
\end{itemize}

\subsubsection{Produit}

Dans le cas le plus simple, nous d\'efinition le "\NewTerm{produit ensembliste}\index{produit ensembliste}" ou "\NewTerm{produit cart\'esien}\index{produit cart\'esien}" comme:

En langage non sp\'ecialis\'e voici ce qu'il faut lire: "l'ensemble produit" (à ne pas confondre avec la multiplication ou le produit vectoriel) de deux ensembles $A$ et $B$ est l'ensemble des couples tels que chaque \'el\'ement de chaque ensemble est combin\'e avec chaque \'el\'ement de l'autre ensemble.

L'ensemble de produits de nombres r\'eels, par exemple, g\'enère le plan dans lequel chaque \'el\'ement est d\'efini par les axes $ X $ et $ Y $.

 Nous retrouvons souvent les ensembles produits en math\'ematiques et en physique lors que nous travaillons avec des fonctions. Par exemple, une fonction de deux variables r\'eelles qui donne un r\'eel en sortie sera not\'e:
	
	ou plus simplement:
	
	
\subsubsection{Compl\'ementarit\'e}

Dans le cas le plus simple, nous d\'efinissons la "\NewTerm{compl\'ementarit\'e}\index{compl\'ementarit\'e}" par:

En langage non sp\'ecialis\'e voici ce qu'il faut lire: Le "compl\'ementaire" est d\'efini comme en prenant $U$ un ensemble et $A$ un sous-ensemble de $U$ alors le compl\'ementaire de $A$ dans $U$ est l'ensemble des \'el\'ements qui sont dans $U$ mais pas dans $A$.
	\begin{figure}[H]
		\centering
		\includegraphics{img/arithmetics/complementarity.eps}	
		\caption{Exemple visuel (diagramme d'Euler) de la compl\'ementarit\'e}
	\end{figure}
	Une autre notation très importante de la compl\'ementarit\'e qu'on retrouve parfois dans la litt\'erature est la suivante:
	
	où dans le cas particulier à droite ci-dessus, nous pourrions aussi \'ecrire $U \setminus A$ (la notation $_{U}A^c$ serait rarement utilis\'ee car elle peut prêter à confusion dans certaines situations).

	Nous avons comme propri\'et\'es pour tout $A_i$ inclus dans $B$:
	
	Voici quelques lois triviales relatives aux compl\'ements:
	
	Il existe d'autres lois très importantes en logique bool\'eenne (\SeeChapter{voir la section de Systèmes Logiques page \pageref{logical systems}}). Si nous consid\'erons trois ensembles $A$, $B$, $C$ comme repr\'esent\'es ci-dessous:
	\begin{figure}[H]
		\centering
		\begin{tikzpicture}
		  \tikzset{venn circle/.style={draw,circle,minimum width=6cm,fill=#1,opacity=0.4}}
		
		  \node [venn circle = red] (A) at (0,0) {$A$};
		  \node [venn circle = blue] (B) at (60:4cm) {$B$};
		  \node [venn circle = green] (C) at (0:4cm) {$C$};
		  \node[left] at (barycentric cs:A=1/2,B=1/2 ) {$A \cap B$}; 
		  \node[below] at (barycentric cs:A=1/2,C=1/2 ) {$A \cap C$};   
		  \node[right] at (barycentric cs:B=1/2,C=1/2 ) {$B \cap C$};   
		  \node[below] at (barycentric cs:A=1/3,B=1/3,C=1/3 ){$A \cap B \cap C$};
		\end{tikzpicture} 
	\end{figure}
	Nous avons donc:
	
	et les fameuses "\NewTerm{lois de De Morgan}\index{lois de De Morgan}" sous forme ensembliste (\SeeChapter{voir section de Systèmes Logiques page \pageref{de morgan theorem}}) et qui sont donn\'ees par les relations:
	
	Indiquons avant de passer à un autre sujet, qu'un nombre significatif d'adultes en emploi (souvent des cadres) en entreprise ayant oubli\'e ces notions après leur sortie de l'\'ecole obligtoire doivent les \'etudier à nouveau lorsqu'ils apprennent le language SQL (Structured Query Language) qui est le plus r\'epandu à travers le monde pour interroger les serveurs de bases de donn\'ees des entreprises au 20ème et 21ème siècles. Ils apprennent alors très en formation continue professionnelle le sch\'ema suivant pour construire leurs requêtes avec des jointures:
	\begin{figure}[H]
		\begin{center}
			\includegraphics[scale=0.33]{img/arithmetics/sql_joins.jpg}
		\end{center}	
		\caption{Jointures courantes du langage SQL bas\'e sur la th\'eorie des ensembles}
	\end{figure}

	\pagebreak
	\subsection{Fonctions et Applications}	\label{functions and applications}

	\textbf{D\'efinition (\#\mydef):} En math\'ematiques, une "\NewTerm{application}\index{application}" (ou "\NewTerm{fonction}\index{fonction}") not\'ee $f$ - en analyse fonctionnelle - ou $A$ - en algèbre lin\'eeaire - est la donn\'ee de deux ensembles, l'ensemble de d\'epart $E$ et l'ensemble d'arriv\'ee $F$ (ou d'image de $E$), et d'une relation associant à chaque \'el\'ement $x$ de l'ensemble de d\'epart un et un seul \'el\'ement de l'ensemble d'arriv\'ee, que nous appelons "\NewTerm{image de $x$ par $f$}" et que nous notons dans le domaine de l'analyse fonctionnelle par $f(x)$ ou $f(E)$ pour expliciter l'ensemble de d\'epart. Nous appelons "\NewTerm{images}\index{images}" les \'el\'ements de $f(E)$ et les \'el\'ements de $E$ sont appel\'es les "\NewTerm{ant\'ec\'edents}\index{ant\'ec\'edents}".

	Nous disons alors que $f$ est une application de $E$ dans $F$ not\'ee:
	
	(se rappeler du premier diagramme sagittal pr\'esent\'e au d\'ebut de ce chapitre), ou encore une application à arguments dans $E$ et valeurs dans $F$.

	\begin{tcolorbox}[title=Remarque,colframe=black,arc=10pt]
	Le terme "fonction" est souvent utilis\'e pour les applications à valeurs num\'eriques, r\'eelles ou complexes, c'est-à-dire lorsque l'ensemble d'arriv\'ee est $\mathbb{R}$ ou $\mathbb{C}$. Nous parlons alors de "fonction r\'eelle", ou de "fonction complexe".
	\end{tcolorbox}	
	
	\textbf{D\'efinitions (\#\mydef):}

	\begin{enumerate}
		\item[D1.] Le  "\NewTerm{graphe}\index{graphe}" (ou encore "\NewTerm{graphique}\index{graphique}" ou "repr\'esentative") d'une application $f:E \mapsto F$ est le sous-ensemble du produit cart\'esien $E \times F$ constitu\'e des couples $(x, f (x))$ pour $x$ variant dans $E$. La donn\'ee du graphe de $f$ d\'etermine son ensemble de d\'epart (par projection sur la première composante souvent not\'ee $x$) et son image (par projection sur la seconde composante souvent not\'ee $y$).
		
		\item[D2.]  Si le triplet $f(E,F,\Gamma)$ est une fonction où $E$ et $F$ sont deux ensembles et $\Gamma \subset (E \times F)$ est un graphe, $E$ et $F$ sont respectivement la source et le but de $f$. Le "\NewTerm{domaine de d\'efinition}\index{domaine de d\'efinition}" ou "\NewTerm{ensemble de d\'epart}\index{ensemble de d\'epart}" de $f$ est:
		
		
		\item[D3.] \'etant donn\'es trois ensembles $E, F$ et $G$ (non vides), toute fonction de $E \times F$ vers $G$ est appel\'ee "\NewTerm{loi de composition}\index{loi de composition}" de $E \times F$ à valeurs dans $G$.
		
		\item[D4.] Une "\NewTerm{loi de composition interne}\index{loi de composition interne}\label{internal composition law}" (ou simplement "\NewTerm{loi interne}\index{loi interne}") dans $E$ est une loi de composition de equation à valeurs dans $E$ (cas $E=F=G$). 

		\begin{tcolorbox}[title=Remarque,colframe=black,arc=10pt]
		La soustraction dans $\mathbb{N}$ n'est pas une loi de composition interne bien qu'elle fasse partie des quatre op\'erations \'el\'ementaires apprises à l'\'ecole. Par contre l'addition sur $\mathbb{N}$ en est bien une!
		\end{tcolorbox}
			
		\item[D5.] Une "\NewTerm{loi de composition externe}\index{loi de composition externe}" (ou simplement "\NewTerm{loi externe}\index{loi externe}") dans $E$ est une loi de composition de $F \times E$ à valeurs dans $E$, où $F$ est un ensemble distinct de $E$. En g\'en\'eral, $F$ est un corps, dit "\NewTerm{corps de scalaires}\index{corps de scalaires}".
	
		\begin{tcolorbox}[colframe=black,colback=white,sharp corners]
		\textbf{{\Large \ding{45}}Exemple:}\\\\
		Dans le cas d'un espace vectoriel (voir d\'efinition beaucoup plus bas) la multiplication d'un vecteur (dont les composantes se basent sur un ensemble donn\'e) par un r\'eel est une loi de composition externe.
		\end{tcolorbox}
	
		\begin{tcolorbox}[title=Remarque,colframe=black,arc=10pt]
		Une loi de composition externe à valeurs dans $E$ est aussi appel\'ee "\NewTerm{action de $F$ sur $E$}". L'ensemble $F$ est alors le domaine d'op\'erateurs. On dit aussi que $F$ opère sur $E$ (ayez en tête l'exemple des vecteurs pr\'ec\'edemment cit\'e).
		\end{tcolorbox}	
		
		\item[D5.] Nous appelons "\NewTerm{image de $f$}", et nous notons $\Ima(f)$, le sous-ensemble d\'efini par:
		
		Ainsi, "l'image" d'une application $f:E \mapsto F$ est la collection des $f(x)$ pour $x$ parcourant $E$, c'est un sous-ensemble de $F$.
		
		Et nous appelons "\NewTerm{noyau de $f$}\index{kerr}\index{noyau}\label{kerr}", et nous notons $\ker (f)$, le sous-ensemble très important en math\'ematiques d\'efini par:
		
		Selon la figure (il faut bien comprendre ce concept de noyau car nous le r\'eutiliserons de nombreuses fois pour d\'emontrer des th\'eorèmes ayant des applications pratiques importantes):
		\begin{figure}[H]
			\centering
			\includegraphics[scale=1]{img/arithmetics/ker.jpg}	
			\caption{Repr\'esentation du concept de noyau d'une fonction}
		\end{figure}
		\begin{tcolorbox}[title=Remarque,colframe=black,arc=10pt]
		\textbf{R1.} $\ker (f)$ provient de l'allemand "Kern", signifiant tout simplement "noyau". En anglais, le noyau se dit aussi "kernel", signifiant "amande" dans le civil.\\
		
		\textbf{R2.} Normalement les notations $\Ima$ et $\ker$ sont r\'eserv\'ees aux homomorphismes de groupes, d'anneaux, de corps et aux applications lin\'eaires entre espaces vectoriels ou modules etc.... (voir plus loin). Nous n'avons normalement pas l'habitude de les utiliser pour des applications quelconques entre ensembles quelconques. Mais bon...ça ne fait rien.
		\end{tcolorbox}	
	\end{enumerate}	
	\begin{tcolorbox}[colframe=black,colback=white,sharp corners]
	\textbf{{\Large \ding{45}}Exemple:}\\\\
	La fonction sinus a de son argument un noyau qui est $2\pi \mathbb{Z}$.
	\end{tcolorbox}
	Les applications peuvent avoir une quantit\'e ph\'enom\'enale de propri\'et\'es dont voici celles qui font partie des connaissances g\'en\'erales du physicien (pour plus de renseignements sur ce qu'est une fonction, voir le chapitre traitant de l'Analyse Fonctionnelle page \pageref{functional analysis}).

	Soit $f$ une application d'un ensemble $E$ à un ensemble $F$ alors nous avons les propri\'et\'es suivantes:

	\begin{enumerate}
		\item[P1.] Une application est dite "\NewTerm{surjective}\index{application surjective}\label{surjective application}" si:\\
		
		Tout \'el\'ement $y$ de $F$ est l'image par $f$ d'au moins (nous insistons sur le "au moins") un \'el\'ement de $E$. Nous disons encore que c'est une "surjection" de $E$ dans $F$. Il d\'ecoule de cette d\'efinition, qu'une application ou fonction $f:E\mapsto F$ est surjective et not\'ee:
		
		si et seulement si $F= \Ima (f)$. En d'autres termes, nous \'ecrivons aussi cette d\'efinition ainsi:
		
		\begin{figure}[H]
			\begin{center}
				\includegraphics[scale=0.75]{img/arithmetics/surjective.eps}
			\end{center}	
			\caption{Repr\'esentation d'une fonction surjective}
		\end{figure}
		
		\item[P2.] Une application est dite "\NewTerm{injective}\index{application injective}\label{injective}" si:\\
	
		Tout \'el\'ement $y$ de $F$ est l'image par $f$ d'au plus (nous insistons sur le "au plus") un seul \'el\'ement de $E$. Nous disons encore que $f$ est une injection de $E$ dans $F$. Il r\'esulte de cette d\'efinition, qu'une application $f:E\mapsto F$ est injective et not\'ee:
		
		si et seulement si les relations $x_1,x_2 \in E$  et $f(x_1)=f(x_2)$ impliquent  $x_1=x_2$ autrement dit: une application pour laquelle deux \'el\'ements distincts ont des images distinctes est dite "injective". Ou encore, une application est injective si l'une aux moins des propri\'et\'es \'equivalentes suivantes est v\'erifi\'ee:
		\begin{enumerate}
			\item[P2.1] $\forall x,y\in E^2:\;f(x)=f(y)\Rightarrow x=y$
			\item[P2.2] $\forall x,y:\;  x\neq y \Rightarrow f(x) \neq f(y)$
			\item[P2.3] $\forall y \in F$ l'\'equation en  $x$, $y=f(x)$ a au plus une solution dans $E$
		\end{enumerate}
		Tout cela s'illustrant par:
		\begin{figure}[H]
			\centering
			\includegraphics[scale=0.75]{img/arithmetics/injective.eps}	
			\caption{Repr\'esentation d'une fonction injective}
		\end{figure}
		
		\item[P3.] Une application est dite "\NewTerm{bijective}\index{application bijective}" ou "\NewTerm{application/fonction totale}\label{bijection}" si:
		
		Une application $f$ de $E$ dans $F$ est à la fois surjective et injective. Dans ce cas, nous avons que pour tout \'el\'ement $y$ de $F$, l'\'equation $y=f(x)$ admet dans $E$ une unique (ni "au plus", ni "au moins") pr\'e-image $x$ et not\'e:
		 
		Ce que nous \'ecrivons aussi plus explicitement:
		
		ce qui s'illustre par:
		\begin{figure}[H]
			\begin{center}
				\includegraphics[scale=0.75]{img/arithmetics/bijective.eps}
			\end{center}	
			\caption{Repr\'esentation d'une fonction bijective}
		\end{figure}
		Nous sommes ainsi tout naturellement amen\'es à d\'efinir une nouvelle application de $F$ dans $E$, appel\'ee "\NewTerm{fonction r\'eciproque}\index{fonction r\'eciproque}" de $f$ et not\'ee $f^{-1}$ (certaines auteurs et professeurs notent cela aussi $^rf$ parfois...), qui a tout \'el\'ement $y$ de $F$, fait correspondre l'\'el\'ement $x$ de $E$ pr\'e-image (ou aussi appel\'e "solution") unique de l'\'equation $y=f(x)$. Autrement dit:
		
		L'existence d'une application r\'eciproque implique que le graphique d'une application bijective (dans l'ensemble des r\'eels...) et celui de son application r\'eciproque sont sym\'etriques par rapport à la droite d'\'equation $y=x$.
		
		Effectivement, nous remarquons que si $y=f(x)$ est \'equivalent à $x=f^{-1}(y)$, alors ces \'equations impliquent que le point $(x, y)$ est sur le graphique de $f$ si et seulement si le point $(y, x)$ est sur le graphique de $f^{-1}$.
	
	Comme vous pouvez le voir par exemple dans la figure ci-dessous avec la fonction sinus (\SeeChapter{voir section de Trigonom\'etrie page \pageref{trigonometry}}):
	
	\begin{figure}[H]
		\begin{center}
			\includegraphics[scale=2]{img/arithmetics/bijective_example.jpg}
		\end{center}	
		\caption{Exemple de fonction bijective}
	\end{figure}

	\begin{tcolorbox}[colframe=black,colback=white,sharp corners]
	\textbf{{\Large \ding{45}}Exemple:}\\\\
	Prenons le cas d'une station de vacances où un groupe de touristes doit être log\'e dans un hôtel. Chaque façon de r\'epartir ces touristes dans les chambres de l'hôtel peut être repr\'esent\'ee par une application de l'ensemble des touristes vers l'ensemble des chambres (à chaque touriste est associ\'ee une chambre).
	\begin{itemize}
		\item Les touristes souhaitent que l'application soit injective, c'est-à-dire que chacun d'entre eux ait une chambre individuelle. Cela n'est possible que si le nombre de touristes ne d\'epasse pas le nombre de chambres.
		
		\item  L'hôtelier souhaite que l'application soit surjective, c'est-à-dire que chaque chambre soit occup\'ee. Cela n'est possible que s'il y a au moins autant de touristes que de chambres.
		
		\item S'il est possible de r\'epartir les touristes de telle sorte qu'il y en ait un seul par chambre, et que toutes les chambres soient occup\'ees: l'application sera alors à la fois injective et surjective nous dirons qu'elle est bijective.
	\end{itemize}
	\end{tcolorbox}
		
		\begin{tcolorbox}[title=Remarques,colframe=black,arc=10pt]
		\textbf{R1.}  Il vient des d\'efinitions ci-dessus qu'une application f est bijective (ou "biunivoque") dans l'ensemble des r\'eels si et seulement si toute droite horizontale coupe la repr\'esentation graphique de la fonction en un seul point. Nous sommes donc amen\'es à faire la seconde remarque suivante:\\
	
		\textbf{R2.} Une application qui v\'erifie le test de la droite horizontale est continument croissante ou d\'ecroissante en tout point de son domaine de d\'efinition.
		\end{tcolorbox}
		
		\item[P4.] Une application est dite "\NewTerm{fonction compos\'ee}\index{fonction compos\'ee}" ou "\NewTerm{fonction composite}" si:
		
		Soit $\varphi$ une application de $E$ dans $F$ et $\psi$ une fonction de $F$ dans $G$. L'application qui associe à chaque \'el\'ement $x$ de l'ensemble de $E$, un \'el\'ement $\psi(\varphi(x))$ de $G$ s'appelle "\NewTerm{application compos\'ee}\index{application compos\'ee}" de $\varphi$ et de $\psi$ et se note:
		
		où symbole "$\circ$" est appel\'e "\NewTerm{rond}\index{rond}" (à ne pas confondre avec la notation du produit scalaire que nous verrons dans la section de Calcul Vectoriel!). Donc, la relation pr\'ec\'edente s'\'ecrit "psi rond phi" mais se lit "phi rond psi"... Ainsi:
		
		Soit, de plus,  $\chi$ une application (pas une fonction!) de $G$ dans $H$. Nous v\'erifions aussitôt que l'op\'eration de composition est associative pour les applications (pous plus de d\'etails voir la section d'Algèbre Lin\'eaire page \pageref{non-commutativity matrices}):
		
		Cela nous permet d'omettre les parenthèses et d'\'ecrire plus simplement:
		
		Dans le cas particulier où $\varphi$ serait une application de $E$ dans $E$, nous notons $\varphi^k$ l'application compos\'ee $\varphi \circ \varphi \circ ... \circ \varphi$ ($k$ fois).
	\end{enumerate}
	
	Ce qui est important dans ce que nous venons de voir dans ce chapitre, c'est que toutes les propri\'et\'es d\'efinies et \'enonc\'ees ci-dessus sont applicables aux ensembles de nombres.

	Voyons en un exemple très concret et très puissant:


	\subsubsection{Th\'eorème de Cantor-Bernstein}
	\begin{tcolorbox}[colback=red!5,borderline={1mm}{2mm}{red!5},arc=0mm,boxrule=0pt]
	\bcbombe Attention! Ce th\'eorème, dont le r\'esultat peut sembler \'evident, n'est pas forc\'ement simple à aborder (son formalisme math\'ematique n'est pas très esth\'etique...). Nous vous conseillons de lire très lentement et de vous imaginer les diagrammes sagittaux dans la tête lors de la d\'emonstration.
	\end{tcolorbox}
	
	Voici l'hypothèse à d\'emontrer:
	\begin{theorem}
	Soient $X$ et $Y$ deux ensembles. S'il existe une injection (voir la d\'efinition d'une fonction injective ci-dessus) de $X$ vers $Y$ et une autre de $Y$ vers $X$, alors les deux ensembles sont en bijection (voir la d\'efinition d'une fonction bijective ci-dessus). Il s'agit donc aussi d'une relation antisym\'etrique!
	
	Ce qui s'illustre par:
	\begin{figure}[H]
		\begin{center}
			\includegraphics[scale=0.75]{img/arithmetics/cantor_bernstein.jpg}
		\end{center}	
		\caption{Repr\'esentation d'une relation antisym\'etrique}
	\end{figure}
	\end{theorem}
	Formellement ce th\'eorème est parfois \'ecrit:
	
	ou plus techniquement:
	
	Pour la d\'emonstration, nous avons besoin en toute rigueur de d\'emontrer au pr\'ealable un lemme (\'evident intuitivement mais pas formellement...) dont l'\'enonc\'e est le suivant:

	\begin{lemma}
	Soient $X, Y, Z$ trois ensembles tels que $X \subseteq Z \subseteq Y$. Si $X$ et $Y$ sont en bijection à travers une fonction $f$, alors $X$ et $Z$ sont en bijection à travers une fonction $g$.

	Techniquement, nous \'ecrivons ceci:
	
	
	Un exemple d'application de ce lemme est l'ensemble des nombres naturels $\mathbb{N}$ et des nombres rationnels $\mathbb{Q}$ qui sont en bijection. Dès lors, l'ensemble des entiers relatifs est en bijection avec l'ensemble des nombres naturels puisque $\mathbb{N} \subseteq \mathbb{Z} \subseteq \mathbb{Q}$.
	\end{lemma}

	\begin{dem}
	D'abord, au niveau formel, cr\'eons une fonction $f$ de $Y$ à $X$ telle quelle soit bijective:
	
	Nous avons besoin pour la suite d'un ensemble $A$ qui sera d\'efini par l'union des images des fonctions des fonctions $f$ (du genre $f(f(f\ldots)))$ et consturire un tel outil est là où r\'eside l'astuce!) des pr\'e-images de l'ensemble $Z$ (rappelez-vous que $Z \subseteq Y$) dont nous excluons les \'el\'ements de $X$ (ce que nous noterons pour cette d\'emonstration: $Z-X$):
	\begin{figure}[H]
		\centering
		\includegraphics[scale=1]{img/arithmetics/cantor_bernstein_construction_lemma.jpg}
	\end{figure}
	En d'autres termes (si la première forme n'est pas claire...) nous d\'efinissons l'ensemble $A$ comme \'etant l'union des images de $(Z-X)$ par les applications $f \circ f \circ ... \circ f$ Ce que nous noterons :
	
	Parce que $f:Y \mapsto X$ et que $(Z-X)\subseteq Y$ nous avons par construction $A \subseteq X$ et donc:
	
 	Remarquons que nous avons aussi:
	
	et en r\'eindexant:
	
	Nous avons alors (faire un sch\'ema de tête des diagrammes sagittaux peut aider à ce niveau-là...) quel que soit $A$:
	
	Nous pouvons d\'emontrer \'el\'egamment cette dernière relation (\'etant donn\'e que c'est aussi un des r\'esultats les plus importants!):
	
	Comme $Z$ peut être partitionn\'e (rien nous en empêche!) en les deux sous-ensembles disjoints suivants:
		
	 et sans oublier que $X \subseteq Z \subseteq Y$ et $A \subseteq X$, nous introduisons maintenant par d\'efinition la fonction $g$ (pour laquelle nous ne donne pas plus d'informations pour l'instant) telle que:
	
	tel que pour toute pr\'e-image $a$ de $g$ de la partition $((Z-X)\cup A) \subseteq Z$ nous ayons:
	
	Ce qui signifie que parce que $((Z-X)\cup A) \subseteq Z$ et $Z \subseteq Y$ nous pouvons alors appliquer (associer) la fonction bijective $f$ (rappelez-vous que $f:Y \rightarrow X$) comme \'equivalent de la fonction $g$ à tout \'el\'ement de $((Z-X)\cup A)$.

	Nous avons aussi pour tout pr\'e-image $a$ de $g$ de la partition $(X-A)$ (rappelez-vous que $A \subseteq X$):
	
	ce qui signifie que nous appliquons juste la fonction identit\'e que nous pouvons aussi associer à la fonction $g$.
	
	Donc pour r\'esumer, nous avons construit une fonction $g$ qui est alors bijective parce que ses restrictions à $((Z-X)\cup A)$ est $f$ et  $(X-A)$ is l'identiti\'e qui sont toutes deux bijectives, la première par d\'efinition, la deuxième par construction!

	Finalement il existe bien, par construction, une bijection entre $X$ et $Z$ et nous avons d\'emontr\'e le lemme que:
	
	\begin{flushright}
		$\blacksquare$  Q.E.D.
	\end{flushright}
\end{dem}

	Maintenant que nous avons prouv\'e le lemme, rappelons les hypothèses du th\'eorème de Cantor-Bernstein en utilisant le r\'esultat du lemme:
	
	Soit $\varphi$ une injection de $X$ à $Y$ et $\psi$ une injection de $Y$ à $X$ avec $X \subseteq Y$. 
	
	Nous avons alors:
	
	donc:
	
	Dès lors, nous avons jusqu'ici:
	\begin{itemize}
		\item Comme $\varphi$ est injective, alors $X$ et $\varphi(X)$ sont pas d\'efinition bijectifs (oui! effectivement, une fonction injective est par d\'efinition bijective lorsque nous r\'eduisons son ensemble image à son ensemble de pr\'e-image\footnote{mais il faut noter que cela ne fonction que pour des ensembles infinis!})
		
		\item Comme $\psi$ est injective, $\psi(\varphi(X))$ et $\varphi(X)$ est bijective.
	\end{itemize}

	Donc, du pr\'ec\'edent lemme prouv\'e, $X$ et $\psi(\varphi(X))$ sont alors aussi en bijection!
	
	En utilisant le lemme sur  $\psi(\varphi(X))$, $\psi(Y)$ et $X$ , il vient donc que $\psi(\varphi(X))$ est en bijection avec $\psi(Y)$ ce qui nous donne avec ce que nous avons vu juste pr\'ec\'edemment, que puisque $\psi(\varphi(X))$ et aussi $\varphi(X)$ sont en bijection, que  $\psi(Y)$ est en bijection avec $\varphi(X)$, et alors que $X$ et $Y$ sont reli\'es par une application bijective (ouf! c'est beau mais c'est aussi vicieux que simple) et que nous avons:
	
	Ce th\'eorème s'interprète de la manière suivante: Si nous pouvons compter une partie d'un ensemble avec la totalit\'e des \'el\'ements d'un autre ensemble, et r\'eciproquement, alors ils ont le même nombre d'\'el\'ements. Ce qui est \'evident pour des ensembles finis. Ce th\'eorème g\'en\'eralise alors cette notion pour des ensembles infinis et c'est là sa force!

	À partir de là, ce th\'eorème repr\'esente l'une des briques de base pour g\'en\'eraliser la notion de tailles d'ensembles à des ensembles infinis.
	
	\pagebreak
	\subsection{Structures}\label{structures}
	L'algèbre dite "\NewTerm{algèbre moderne}\index{algèbre moderne}" commence avec la th\'eorie des structures alg\'ebriques due en partie à Carl F. Gauss et surtout à \'evariste Galois. Ces structures existent en un très grand nombre mais seulement les fondamentales nous int\'eresseront ici. Avant de les d\'etailler, voici un diagramme synoptique de ces principales structures et de leur hi\'erarchie:
	\begin{figure}[H]
		\centering
		\includegraphics[width=1.0\textwidth]{img/arithmetics/structures_common.jpg}
		\caption{Diagramme synoptique des structures alg\'ebriques courantes}
	\end{figure}
	\begin{tcolorbox}[title=Remarque,colframe=black,arc=10pt]
	Tout en haut du diagramme, la structure au nombre minimal de contraintes, en bas, un maximum. Soit, plus nous descendons, plus la structure est en quelque sorte sp\'ecialis\'ee.
	\end{tcolorbox}	
	Soit pour simplifier les \'ecritures, supposons que $\star$ et  $\circ$ repr\'esentent des lois de compositions\footnote{Cette notation g\'en\'eralis\'ee est souvent appel\'ee "\NewTerm{notation stellaire}\index{notation stellaire}".} (comme l'addition, la soustraction, la multiplication ou encore la division,...), alors:
	
	\textbf{D\'efinitions (\#\mydef):} Soit $\star$ et $\circ$ des symboles de lois internes à un ensemble $E$ (cela pourrait être l'addition et la multiplication pour prendre les cas les plus connus) alors:
	\begin{enumerate}
		\item[D1.] $\star$ est une "\NewTerm{loi commutative}\index{loi commutative}" si: 
		
		
		\item[D2.] $\star$ est une "\NewTerm{loi associative}\index{loi associative}" si:
		
		
		\item[D3.] $n$ est "\NewTerm{l'\'el\'ement neutre}\index{l'\'el\'ement neutre}\label{neutral element}" pour $\star$ si:
		
		Nous admettrons par ailleurs sans d\'emonstration (c'est intuitif) que s'il existe un \'el\'ement neutre, il est unique.
		
		\item[D4.] $a'$ est "\NewTerm{l'\'el\'ement sym\'etrique}\index{l'\'el\'ement sym\'etrique}\label{symmetrical element}" (dans le sens g\'en\'eral de l'oppos\'e par exemple pour l'addition et l'inverse pour la multiplication) de $a$ pour $\star$  si:
		
		Nous admettrons \'egalement et sans d\'emonstration que le sym\'etrique de tout \'el\'ement est unique.		
		
		\item[D5.] $\circ$ est une "\NewTerm{loi distributive}\index{loi distributive}" par rapport à $\star$ si:
		
	
		\item[D6.] $b$ est "\NewTerm{l'\'el\'ement absorbant}\index{\'el\'ement absorbant}" si pour tout $a$ et une loi $\star$ nous avons:
		
	\end{enumerate}
	\begin{tcolorbox}[title=Remarques,colframe=black,arc=10pt]
	\textbf{R1.} Si $a$ est son propre sym\'etrique par rapport à la loi $\star$, les math\'ematiciens disent que $a$ est "\NewTerm{involutif}\index{involutif}".\\

	\textbf{R2.} Si un \'el\'ement $b$ de $E$ v\'erifie:
	
	alors $b$ est dit  "\NewTerm{\'el\'ement absorbant}\index{\'el\'ement absorbant}" pour la loi $\star$.\\

	\textbf{R3.} Il faut toujours v\'erifier que les neutres et les sym\'etriques le soient "à gauche" \underline{et} "à droite". Ainsi, par exemple, dans $(\mathbb{Z},-)$, l'\'el\'ement $0$ n'est un neutre qu'à droite car $x-0=x$ mais $0-x=-x$.
	\end{tcolorbox}	
	
	\subsubsection{Magma}
	\textbf{D\'efinition (\#\mydef):} Nous d\'esignons un ensemble par le terme "\NewTerm{magma}\index{magma}" $M$, si les composants le constituant sont op\'erables par rapport à une loi interne $\star$:
	
	Il est donc important de se rappeler que si nous d\'esignons une structure alg\'ebrique par le terme "magma" tout court, cela ne signifie en aucun cas que la loi interne est commutative, associative ou même qu'elle possède un \'el\'ement neutre !
	
	\begin{tcolorbox}[title=Remarques,colframe=black,arc=10pt]
	\textbf{R1.}  Si de plus la loi interne $\star$ est commutative, nous parlons de "\NewTerm{magma commutatif}\index{magma commutatif}".\\

	\textbf{R2.} Si de plus la loi interne $\star$  est associative, nous parlons de "\NewTerm{magma associatif}\index{magma associatif}".\\

	\textbf{R3.} Si de plus la loi interne $\star$ possède un \'el\'ement neutre $n$, nous parlons de "magma unitaire associatif" ou respectivement de "magma unitaire commutatif" mais nous verrons just pluse bas que les deux ont officiellement d\'enomm\'es autrement par les sp\'ecialistes.
	\end{tcolorbox}	
	
	\textbf{D\'efinition (\#\mydef):} Dans un magma $(M,\star)$, un \'el\'ement $x$ est nomm\'e "\NewTerm{\'el\'ement r\'egulier}\index{\'el\'ement r\'egulier}" (ou "\NewTerm{\'el\'ement simplifiable}\index{\'el\'ement simplifiable}") à gauche si pour tout couple $(a,b)\in M$ nous avons:
	
	\begin{tcolorbox}[title=Remarque,colframe=black,arc=10pt]
	Nous d\'efinissons sur la même logique un \'el\'ement r\'egulier à droite.
	\end{tcolorbox}	
	Ainsi, un \'el\'ement est dit "\NewTerm{r\'egulier}" s'il est r\'egulier à droite et à gauche. Si $\star$ est commutatif (comme dans le cas d'un magma commutatif), les notions d'\'el\'ement r\'egulier à gauche ou à droite coïncident.
	
	Un magma $(M,\star)$ est donc une structure alg\'ebrique \'el\'ementaire. Il existe des structures plus subtiles (monoïdes, groupes, anneaux, champs, espace vectoriel, etc.) dans lesquelles un ensemble est dot\'e de plusieurs lois et de diff\'erentes propri\'et\'es. Nous les verrons tout de suite et les utiliserons tout au long de ce livre de manière explicite ou implicite.
	
	\pagebreak
	\subsubsection{Monoïde}\label{monoid}
	\textbf{D\'efinition (\#\mydef):} Si la loi $\star$ est associative et possède un \'el\'ement neutre nous disons alors que le "\NewTerm{magma associatif unitaire}\index{magma associatif unitaire}" est un "\NewTerm{monoïde}\index{monoïde}":
	
	\begin{tcolorbox}[title=Remarques,colframe=black,arc=10pt]
	\textbf{R1.} Si de plus la loi interne $\star$ est commutative alors nous disons alors que la structure forme un "\NewTerm{monoïde ab\'elien}\index{monoïde ab\'elien}" (ou simplement "\NewTerm{monoïde commutatif}\index{monoïde commutatif}").\\

	\textbf{R2.} Dans certains ouvrages nous trouvons aussi comme d\'efinition que le monoïde est un "\NewTerm{semi-groupe}" (avec une loi associative $\star$) muni d'un \'el\'ement neutre $n$.
	\end{tcolorbox}
	Montrons  tout de suite que l'ensemble des entiers naturels $\mathbb{N}$ est un monoïde ab\'elien totalement ordonn\'e (comme nous l'avons partiellement vu dans la section sur les Op\'erateurs \pageref{comparators}) par rapport aux lois d'addition et de multiplication:
	
	La loi d'addition "$+$" est-elle une loi interne telle que $\forall a,b \in\mathbb{N}$ nous ayons:
	
	Nous pouvons d\'emontrer que c'est bien le cas en sachant que $1$ appartient à $\mathbb{N}$ tel que:
	
	Donc  $c\in\mathbb{N}$ et l'addition est bien une loi interne (nous disons \'egalement que l'ensemble $\mathbb{N}$ est "\NewTerm{stable}\index{stable}" par rapport à l'addition) et en même temps associative puisque $1$ peut être additionn\'e à lui-même par d\'efinition dans n'importe quel ordre sans que le r\'esultat en soit alt\'er\'e. Si vous vous rappelez que la multiplication est une loi qui se construit sur l'addition (\SeeChapter{voir la section Op\'erateurs page \pageref{multiplication}}), alors la loi de multiplication $\times$ est aussi une loi interne et associative !
	
	Nous admettrons à partir d'ici qu'il est trivial que la loi d'addition $+$ est \'egalement commutative et que le z\'ero "$0$" en est l'\'el\'ement neutre ($n$) à droite. Ainsi, la loi de multiplication est elle aussi commutative et il est trivial que "$1$" en est l'\'el\'ement neutre ($n$).

	Par ailleurs, pour parler d\'ejà de quelque chose qui n'est pas directement en relation avec le monoïde... mais qui nous sera utile un peu plus loin, existe-t-il en restant dans la lign\'ee de l'exemple pr\'ec\'edent pour la loi d'addition $+$ un sym\'etrique $\exists c$ tel que $\forall a,b\in\mathbb{N}$ nous ayons:
	
	avec $c\in\mathbb{N}$?
	
	Il est assez trivial que pour que cette \'egalit\'e soit satisfaite nous ayons:
	
	soit:
	
	or les nombres n\'egatifs n'existent pas dans $\mathbb{N}$. Ce qui nous amène aussi à la conclusion que la loi d'addition $+$ n'a pas de sym\'etrique dans $\mathbb{N}$ et que la loi de soustraction $-$ n'existe pas dans $\mathbb{N}$ (la soustraction \'etant rigoureusement l'addition d'un nombre n\'egatif!).
	
	De même, car cela va aussi nous être utile un peu plus loin, existe-t-il  pour la loi de multiplication $\times$ un sym\'etrique $a'$ tel que $\forall a\in\mathbb{N}$ nous ayons:
	
	avec $a'\in\mathbb{N}$?
	
	D'abord il est \'evident que:
	
	Mais except\'e pour $a=1$, le quotient $1/a$ n'existe pas dans $\mathbb{N}$. Donc nous devons conclure qu'il n'existe pas pour tout \'el\'ement de $\mathbb{N}$ de sym\'etriques pour la loi de multiplication et ainsi que la loi de division n'existe pas dans $\mathbb{N}$ et que la loi de multiplication ne forme pas un monoïde dans cet ensemble.

	Synthèse:
	\begin{table}[H]
		\centering
		\begin{tabular}{|
		>{\columncolor[HTML]{9B9B9B}}l |c|c|c|c|}
		\hline
		\multicolumn{1}{|c|}{\cellcolor[HTML]{9B9B9B}$\pmb{\mathbb{N}}$} & \cellcolor[HTML]{9B9B9B}$\pmb{(+)}$ & \cellcolor[HTML]{9B9B9B}$\pmb{(-)}$ & \cellcolor[HTML]{9B9B9B}$\pmb{\times}$ & \cellcolor[HTML]{9B9B9B}$\pmb{/}$ \\ \hline
		\textbf{Loi interne} & oui &  & oui &  \\ \cline{1-2} \cline{4-4}
		\textbf{Commutative} & oui &  & oui &  \\ \cline{1-2} \cline{4-4}
		\textbf{\'El\'ement neutre} & oui ($0$) &  & oui ($1$) &  \\ \cline{1-2} \cline{4-4}
		\textbf{\'El\'ement absorbant} & non &  & oui ($0$) &  \\ \cline{1-2} \cline{4-4}
		\textbf{Sym\'etrique} & non & \multirow{-5}{*}{non} & non & \multirow{-5}{*}{non} \\ \hline
		\end{tabular}
		\caption{Lois et leurs propri\'et\'es dans l'ensemble des entiers naturels $\mathbb{N}$}
	\end{table}
	Nous avons par exemple les propri\'et\'es suivantes relativement à l'ensemble des entiers naturels $\mathbb{N}$ et au concept de monoïde:
	\begin{enumerate}
		\item[P1.] $(\mathbb{N},\le,\ge)$ est totalement ordonn\'e (attention cette notation est un peu abusive! il suffit qu'il y ait juste une des deux relations d'ordre $\mathcal{R}$ pour que l'ensemble soit totalement ordonn\'e).
		
		\item[P2.] $(\mathbb{N},+)$ et $(\mathbb{N},\times)$ sont des monoïdes ab\'eliens.
		
		\item[P3.] L'\'el\'ement z\'ero "$0$" est l'\'el\'ement absorbant pour le monoïde $(\mathbb{N},\times)$.
		
		\item[P4.] Les lois de soustraction $-$ et division $/$ n'existent pas dans l'ensemble $\mathbb{N}$.
		
		\item[P5.] $\mathbb{N}$ est un monoïde ab\'elien totalement ordonn\'e par rapport aux lois d'addition et de multiplication (attention la notation suivante est abusive car le monoïde n'est compos\'e que d'une seule loi interne et d'une relation d'ordre $\mathcal{R}$ ce qui donnerait au total $4$ monoïdes):
		
	\end{enumerate}
	\begin{tcolorbox}[title=Remarques,colframe=black,arc=10pt]
	\textbf{R1.} Il est rare d'utiliser les monoïdes dansla pratique du physicien et de l'ing\'enieur. Effectivement, souvent, lorsque nous nous trouvons face à une structure trop pauvre pour pouvoir vraiment discuter, nous la prolongeons vers quelque chose de plus riche, comme un groupe, ou un anneau (voir plus loin) tel que l'ensemble des entiers relatifs $\mathbb{Q}$ ou les nombres r\'eels $\mathbb{R}$ (au moins...).\\

	\textbf{R2.} Dire qu'une structure alg\'ebrique est totalement ordonn\'ee par rapport à certaines lois signifie que soit $\star$ une loi, et $\mathcal{R}$ une relation d'ordre et $a$, $b$, $c$, $d$ quatre \'el\'ements de la structure int\'eress\'ee, alors si $a\;\mathcal{R}\;b$ et $c\;\mathcal{R}\;d$ implique $(a\star c)\mathcal{R}(b\star d)$. Nous notons alors cette structure $(S,\star, \mathcal{R})$ ou simplement $(S, \mathcal{R})$ et en indiquant la (ou les) loi concern\'ee.
	\end{tcolorbox}	
	
	\subsubsection{Groupes}\label{groups}
	\textbf{D\'efinition (\#\mydef):} Nous d\'esignons un ensemble par le terme "\NewTerm{groupe}\index{groupe}", si les composants le constituant satisfont aux trois conditions de ce que nous nommons la "\NewTerm{loi interne de groupe}\label{loi interne de groupe}", d\'efinie ci-dessous:
	
	Dans ce cas, la loi de compositions interne $\star$ sera souvent (mais pas exclusivement!) not\'ee "$+$" et appel\'ee "\NewTerm{l'addition}\index{l'addition}, le neutre $e$ not\'e "$0$" et le sym\'etrique de $x$ not\'e "$-x$".

	Insistons sur le fait que la structure de groupe est probablement une des plus importantes dans la pratique de l'ing\'enieur et de la physique moderne en g\'en\'eral. Raison pour laquelle il convient d'y porter une attention toute particulière (\SeeChapter{voir la section d'Algèbre Ensembliste page \pageref{groups}})!

	Si de plus, la loi interne $\star$ est \'egalement commutative, nous disons alors que le groupe est un "\NewTerm{groupe ab\'elien}\index{groupe ab\'elien}\label{abelian group}" ou simplement "\NewTerm{groupe commutatif}\index{groupe commutatif}".

	S'il existe dans $G$ au moins un \'el\'ement $a$ tel que tout \'el\'ement de $G$ est une puissance de $a$ ou du sym\'etrique $a'$ de $a$, nous disons que $(G,\star)$ est un "\NewTerm{groupe cyclique de g\'en\'erateur a}\index{groupe cyclique} s'il est fini, sinon nous disons qu'il est "\NewTerm{monogène}" (nous reviendrons sur les groupes cycliques dans la section d'Algèbre Ensembliste page \pageref{cyclic groups}).
	
	Plus g\'en\'eralement un groupe $(G,\star)$ d'\'el\'ement neutre $e$, non r\'eduit uniquement à $\{e\}$ sera monogène, s'il existe un \'el\'ement $a$ de $G$ distinct de $e$ tel que $G=\left\lbrace e,a^1,a^2,a^3,\ldots,a^n,\ldots\right\rbrace$. Un tel groupe sera cyclique, s'il existe un entier $n$ non nul pour lequel $a^n=e$. Le plus petit entier non nul v\'erifiant cette \'egalit\'e est alors nomm\'e "\NewTerm{l'ordre du groupe}\index{ordre du groupe}".
	
	Montrons tout de suite que l'ensemble des entiers relatifs $\mathbb{Z}$ est un groupe ab\'elien totalement ordonn\'e (comme nous l'avons vu dans la section des Op\'erateurs  page \pageref{comparators}) par rapport aux lois d'addition $+$ et de multiplication $\times$.
	
	D'abord pour raccourcir les d\'eveloppements, il est utile de rappeler que l'ensemble $\mathbb{Z}$ est un "prolongement" de $\mathbb{N}$ par le fait que nous y avons ajout\'e tous les nombres sym\'etriques de signe n\'egatif ($\mathbb{N}\subset \mathbb{Z}$).
	
	Ainsi, en abusant toujours des notations (car normalement un groupe n'a qu'une seule loi $\star$ et une seule relation d'ordre $\mathcal{R}$ suffit à l'ordonner):
	
	forme un groupe ab\'elien totalement ordonn\'e (4 groupes au fait!) et:
	
	un monoïde ab\'elien (deux monoïdes au fait!) totalement ordonn\'e.

	Remarquons aussi que la loi de division n'existe pas pour tout \'el\'ement de l'ensemble $\mathbb{Z}$! Donc en toute g\'en\'eralit\'e nous disons qu'elle n'y existe pas!

	Synthèse:
	\begin{table}[H]
		\centering
		\begin{tabular}{|
		>{\columncolor[HTML]{9B9B9B}}l |c|c|c|c|}
		\hline
		$\pmb{\mathbb{Z}}$ & \multicolumn{1}{c|}{\cellcolor[HTML]{9B9B9B}$\pmb{(+)}$} & \multicolumn{1}{c|}{\cellcolor[HTML]{9B9B9B}$\pmb{(-)}$} & \multicolumn{1}{c|}{\cellcolor[HTML]{9B9B9B}$\pmb{(\times)}$} & \multicolumn{1}{c|}{\cellcolor[HTML]{9B9B9B}$\pmb{(/)}$} \\ \hline
		\textbf{Loi interne} & oui & oui & oui &  \\ \cline{1-4}
		\textbf{Associativit\'e} & oui & non & oui &  \\ \cline{1-4}
		\textbf{Commutativit\'e} & oui & non & oui &  \\ \cline{1-4}
		\textbf{\'El\'ement neutre} & oui ($0$) & \begin{tabular}[c]{@{}c@{}}non\\ {\footnotesize ($0$ pas neutre à gauche)}\end{tabular} & oui ($1$) &  \\ \cline{1-4}
		\textbf{\'El\'ement absorbant} & non & non & oui ($0$) &  \\ \cline{1-4}
		\textbf{Sym\'etrique} & \begin{tabular}[c]{@{}c@{}}oui\\ {\footnotesize (signe oppos\'e)}\end{tabular} & oui & non & \multirow{-6}{*}{non} \\ \hline
		\end{tabular}
		\caption{Lois et leurs propri\'et\'es dans l'ensemble des entiers relatifs $\mathbb{Z}$}
	\end{table}
	Nous avons donc les propri\'et\'es suivantes:
	\begin{enumerate}
		\item[P1.] $(\mathbb{Z},\le,\ge)$ est totalement ordonn\'e (attention à nouveau cette notation est un peu abusive! il suffit qu'il y ait juste une des deux relations d'ordre $\mathcal{R}$ pour que l'ensemble soit totalement ordonn\'e).

		\item[P2.] $(\mathbb{Z},+)$ est un groupe commutatif dont z\'ero "$0$" est l'\'el\'ement neutre.

		\item[P3.] La loi de division n'existe pas dans l'ensemble $\mathbb{Z}$.

		\item[P4.] L'ensemble $\mathbb{Z}$ est un groupe ab\'elien totalement ordonn\'e par rapport à la loi d'addition (attention la notation suivante est encore une fois abusive car le groupe est compos\'e que d'une relation d'ordre $\mathcal{R}$ ce qui donnerait au total deux groupes):
		
		L'ensemble $\mathbb{Z}$ n'est pas un groupe commutatif totalement ordonn\'e par rapport à la loi de multiplication:
		
	\end{enumerate}
	Nous voyons de suite alors que $\mathbb{Z}$ a des propri\'et\'es trop restreintes, c'est la raison pour laquelle il est int\'eressant de le prolonger par l'ensemble des rationnels $\mathbb{Q}$  d\'efini de manière très simpliste... par (\SeeChapter{voir la section Nombres page \pageref{rational numbers}}):
	
	Ce qui signifie pour rappel que l'ensemble des rationnels $\mathbb{Q}$ est d\'efini par l'ensemble des quotients $p$ et $q$ appartenant chacun à  $\mathbb{Z}$ dont nous excluons à $q$ de prendre la valeur nulle (la notation $/q$ signifiant pour rappel l'exclusion).
	
	Et nous avons \'evidemment:
	
	Il est dès lors \'evident (sans d\'emonstration et toujours en utilisant la notation abusive d\'ejà comment\'ee maintes fois plus haut...) que  $(\mathbb{Q},\le,\ge)$ est aussi totalement ordonn\'e et aussi que $\mathbb{Q}$ est un groupe ab\'elien totalement ordonn\'e par rapport à la loi d'addition $+$ seulement:
	
	Ce qui devient int\'eressant avec $\mathbb{Q}$, c'est que la loi de multiplication $\times$ devient une loi interne et forme un groupe ab\'elien commutatif dit "groupe multiplicatif" par rapport à $\mathbb{Q}^{*}$.
	\begin{dem}
	D\'emontrons donc que le sym\'etrique existe pour la loi de multiplication $\times$ tel que:
	
	Puisque dans $\mathbb{Q}^{*}$ tout nombre peut se mettre sous la forme:
	
	avec  $p\in\mathbb{Z}$, $q\in\mathbb{Z}^{*}$.
	
	Alors puisque:
	
	Il existe donc un sym\'etrique à tout rationnel dans $\mathbb{Q}^{*}$ pour la loi de multiplication.
	\begin{flushright}
		$\blacksquare$  Q.E.D.
	\end{flushright}
	\end{dem}
	Par d\'efinition, ou par construction, la division existe dans  $\mathbb{Q}^{*}$ et est une loi interne. Mais est-elle associative telle que pour $\forall (p,q,r)\in\mathbb{Q}^{*}$ nous ayons:
	
	Au fait, la d\'emonstration est assez triviale si nous nous rappelons que la division se d\'efinit à partir de la loi de multiplication par l'inverse et que cette dernière loi est (elle!) associative. Ainsi, il vient:
	
	Donc la loi de division n'est pas associative dans $\mathbb{Q}^{*}$.
	
	Nous pouvons aussi nous demander si la loi de division ($/$) est cependant commutative tel que la relation:
	
	soit valable pour $\forall (a,b)\in\mathbb{Q}^{*}$.
	
	Nous voyons très bien que cela n'est pas le cas puisque nous pouvons \'ecrire cette dernière relation sous la forme:
	
	Pour r\'esumer:
	\begin{table}[H]
		\centering
		\begin{tabular}{|
		>{\columncolor[HTML]{9B9B9B}}l |c|c|c|c|}
		\hline
		$\pmb{\mathbb{Q}}$ & \multicolumn{1}{c|}{\cellcolor[HTML]{9B9B9B}$\pmb{(+)}$} & \multicolumn{1}{c|}{\cellcolor[HTML]{9B9B9B}$\pmb{(-)}$} & \multicolumn{1}{c|}{\cellcolor[HTML]{9B9B9B}$\pmb{(\times)}$} & \multicolumn{1}{c|}{\cellcolor[HTML]{9B9B9B}$\pmb{(/)}$} \\ \hline
		\textbf{Loi interne} & oui & oui & oui & oui \\ \hline
		\textbf{Associativit\'e} & oui & non & oui & non \\ \hline
		\textbf{Commutativit\'e} & oui & non & oui & non \\ \hline
		\textbf{\'el\'ement neutre} & oui ($0$) & \begin{tabular}[c]{@{}c@{}}non\\ {\footnotesize ($0$ pas neutre à gauche)}\end{tabular} & oui ($1$) & \begin{tabular}[c]{@{}c@{}}oui\\ {\footnotesize ($1$ neutre à droite)}\end{tabular} \\ \hline
		\textbf{\'el\'ement absorbant} & non & non & oui ($0$) & \begin{tabular}[c]{@{}c@{}}oui\\ {\footnotesize ($0$ au num\'erateur)}\end{tabular} \\ \hline
		\textbf{Sym\'etrique} & \begin{tabular}[c]{@{}c@{}}oui\\ {\footnotesize (signe oppos\'e)}\end{tabular} & \begin{tabular}[c]{@{}c@{}}oui\\ {\footnotesize (signe oppos\'e)}\end{tabular} & \begin{tabular}[c]{@{}c@{}}non\\ {\footnotesize (except\'e dans $\mathbb{Q}^{*}$)}\end{tabular} & non \\ \hline
		\end{tabular}
		\caption{Lois et leurs propri\'et\'es dans l'ensemble des rationnels $\mathbb{Q}$}
	\end{table}
	Nous avons donc les propri\'et\'es suivantes:
	\begin{enumerate}
		\item[P1.] $(\mathbb{Q},\le,\ge)$ est totalement ordonn\'e
	
		\item[P2.] $(\mathbb{Q},+),(\mathbb{Q}^{*},\times)$ sont ind\'ependamment des groupes ab\'eliens totalement ordonn\'es
	
		\item[P3.] Z\'ero  "$0$" est l'\'el\'ement absorbant par rapport au  groupe $(\mathbb{Q}^{*},\times)$
	
		\item[P4.] L'ensemble $\mathbb{Q}$ est un groupe ab\'elien totalement ordonn\'e par rapport aux lois d'addition et de multiplication que nous notons:
		
	\end{enumerate}
	Les mêmes propri\'et\'es sont applicables à $\mathbb{R}$ et à $\mathbb{C}$ mais à la diff\'erence que ce dernier n'est pas ordonnable.
	
	Cependant, il peut être compr\'ehensible que pour $\mathbb{C}$ vous soyez sceptiques. D\'eveloppons donc tout cela:
	
	Nous devons nous assurer que la somme "$+$", la diff\'erence "$-$", le produit "$\times$" et le quotient "$/$" de deux nombres de la forme  $x+\mathrm{i}y$ donne quelque chose d'encore de cette forme (c'est-à-dire une application du type $\mathbb{C}\mapsto \mathbb{C}$).

	Additionnons les nombres $a+\mathrm{i}b$ et  $c+\mathrm{i}d$ où $a$, $b$, $c$ et $d$ sont des r\'eels:
	
	Donc l'addition est bien une loi interne commutative et associative pour laquelle il existe un \'el\'ement neutre et sym\'etrique dans l'ensemble des complexes $\mathbb{C}$.
	
	Soustrayons les nombres $a+\mathrm{i}b$ et $c+\mathrm{i}d$ où  $a$, $b$, $c$ et $d$ sont ici encore, des r\'eels:
	
	Donc la soustraction est une op\'eration interne; elle n'est ni commutative, ni associative elle n'a pas d'\'el\'ement neutre à gauche et pas de sym\'etrique.
	
	Multiplions maintenant les nombres $a+\mathrm{i}b$ et $c+\mathrm{i}b$ où $a$, $b$, $c$ et $d$ là toujours, sont des r\'eels. Pour parvenir à nos fins, nous emploierons la distributivit\'e de la multiplication par rapport à l'addition:
	
	Donc la loi de multiplication est bien une op\'eration interne commutative, associative et distributive (!) pour laquelle il existe un \'el\'ement neutre et sym\'etrique dans $\mathbb{C}^{*}$ (voir ci-après).
	
	Une division est avant tout une multiplication par l'inverse. Prouver qu'il existe un inverse c'est prouver qu'il existe un sym\'etrique pour la multiplication. Inversons donc le nombre $x+\mathrm{i}y$ où  $x$ et $y$ sont des r\'eels (diff\'erents de z\'ero):
	
	Donc l'inverse d'un nombre complexe est bien une op\'eration interne non associative et non commutative pour laquelle il existe un \'el\'ement neutre, et elle est sym\'etrique. Il en est de même pour la division, qui correspond au produit par l'inverse d'un nombre complexe.
	
	
	Voyons un exemple de groupe cyclique (nous approfondirons le sujet dans la section d'Algèbre Ensemble à la page \pageref{cyclic groups}): Dans  $\mathbb{C}$, consid\'erons $G = \{1, \mathrm{i}, -1, -\mathrm{i}\}$ muni de la multiplication usuelle des nombres complexes. Alors $(G,\times)$ est \'evidemment un groupe ab\'elien. Un tel groupe est aussi monogène car engendr\'e par les puissances d'un de ses \'el\'ements: $\mathrm{i}$ (ou bien $-\mathrm{i}$). Ce groupe monogène \'etant fini, il s'agit alors d'un "cyclic group\index{cyclic group}".
	
	\subsubsection{Anneaux}\label{ring}
	L'anneau est le coeur de l'algèbre commutative qui est la structure alg\'ebrique correspondant aux concepts coll\'egiens d'addition, de soustraction, et de multiplication.
	
	\textbf{D\'efinition (\#\mydef):}  Un groupe commutatif (ou "groupe ab\'elien") $A$ est un "\NewTerm{anneau}\index{anneau}" s'il est muni d'une seconde loi de composition interne v\'erifiant les propri\'et\'es suivante:
	
	Comme nous le savons d\'ejà, l'\'el\'ement neutre de la première loi de composition interne "$+$" est not\'e "$0$" et appel\'e le "z\'ero" de l'anneau. La deuxième loi interne est souvent not\'ee par un point à mi-hauteur $\cdot$ (ou par une crois $\times$ dans les petites classes) et appel\'ee la "multiplication".
	
	\begin{tcolorbox}[title=Remarques,colframe=black,arc=10pt]
	\textbf{R1.} Si de plus, la deuxième loi interne de composition "$\times$" est \'egalement commutative, l'anneau est dit "\NewTerm{anneau commutatif}\index{anneau commutatif}\label{communtative ring}". Nous rencontrons aussi des anneaux non-commutatifs dans lesquels la relation de commutativit\'e n'est pas impos\'ee ou ne s'impose pas et alors nous devons parfois l'imposer, il faut alors renforcer la propri\'et\'e de l'\'el\'ement neutre de cette deuxième loi en imposant à "$1$" d'être un \'el\'ement neutre à la fois à droite et à gauche tel que: $1a=a1=a$ (un exemple d'anneau non-commutatif est fourni par l'ensemble des matrices carr\'ees $n\times n$ à coefficients dans un anneau $A$, par exemple $M_n(\mathbb{R})$ comme nous le verrons dans la section d'Algèbre Lin\'eaire page \pageref{non-commutativity matrices}).\\

	\textbf{R2.} Si de plus, il existe dans $A$ un \'el\'ement neutre pour la deuxième loi de composition interne "$\times$", et que cet \'el\'ement neutre est l'unit\'e "$1$" nous disons alors que l'anneau est un "\NewTerm{anneau unitaire}\index{anneau unitaire}" et $1$ est appel\'e "\NewTerm{unit\'e de l'anneau}". Si l'anneau est commutatif et possède un \'el\'ement neutre pour la deuxième loi de composition interne alors nous parlons "\NewTerm{d'anneau commutatif unitaire}\index{anneau commutatif unitaire}".\\
	
	\textbf{R3.}  Si $a\times b=0\Rightarrow (a=0\text{ ou } b=0)$, quels que soient les \'el\'ements $a$, $b$ de $A$, l'anneau est dit "\NewTerm{anneau intègre}\index{anneau intègre}" ou "\NewTerm{anneau sans diviseurs de z\'ero}" (dans le cas contraire il est bien \'evidemment un "\NewTerm{anneau non-intègre}").\\
	
	\textbf{R4.} Un "\NewTerm{anneau factoriel}\index{anneau factoriel}" est un anneau commutatif unitaire et intègre dans lequel le th\'eorème fondamental de l'arithm\'etique est v\'erifi\'e (\SeeChapter{voir Sectoin Th\'eorie des Nombres page \pageref{fundamental theorem of arithmetic}}).
	\end{tcolorbox}	

	\textbf{D\'efinitions (\#\mydef):}
	\begin{enumerate}
		\item[D1.] Un \'el\'ement  $a$ d'un anneau $A$ est un "\'el\'ement unit\'e" s'il existe  $b\in A$ tel que $ab=ba=1$. Si un tel $b$ existe il est unique (nous en avons vu un exemple lors de notre \'etude des classes de congruence dans la section de Th\'eorie des Nombres page \pageref{congruence}).

		\item[D2.] Un \'el\'ement $a$ d'un anneau $A$ s'appelle un "\NewTerm{diviseur nul à gauche} \index{diviseur nul à gauche}" s'il existe $x\neq 0$ tel que $ ax = 0 $. De même, un \'el\'ement $a$ d'un anneau est nomm\'e "\NewTerm{diviseur nul à droite}\index{diviseur nul à droite}" s'il existe $ y \neq 0 $ tel que $ ya = 0 $. Un \'el\'ement $a$ qui est à la fois un diviseur gauche ET un diviseur \'egal à z\'ero s'appelle un "\NewTerm{diviseur nul bilat\'eral}\footnote{Si l'anneau est commutatif, les diviseurs nuls à gauche et à droite sont \'evidemment les mêmes.}\index{diviseur nul bilat\'eral}".
	\end{enumerate}
	\begin{tcolorbox}[colframe=black,colback=white,sharp corners]
	\textbf{{\Large \ding{45}}Exemples:}\\\\
	E1. Le seul diviseur nul de l'anneau des entiers relatifs $\mathbb{Z}$ est $0$.\\
	
	E2. Dans l'anneau des matrices carr\'ees $2\times 2$ (sur tout anneau non-nul)nous pouvons trouver un $a$ et un $x$ non-nul comme suit:
	
	Dans ce dernier exemple, il est difficile de choisir laquelle des matrices le "diviseur null", c’est pourquoi nous pouvons trouver dans certains manuels scolaires que $a$ ET $x$ sont consid\'er\'es tous deux comme des "diviseurs nuls"...
	\end{tcolorbox} 
	\begin{tcolorbox}[title=Remarques,colframe=black,arc=10pt]
	\textbf{R1.}  Il devrait être clair à ce niveau de lecture qu'un anneau est intègre si et seulement si il ne possède aucun diviseur de z\'ero.\\

	\textbf{R2.} Les notions d'unit\'e et de diviseurs de z\'ero sont incompatibles mais un \'el\'ement d'un anneau peut être ni l'un ni l'autre. C'est le cas, par exemple, de tous les entiers  diff\'erents de $\{0,-1,1\}$ dans $\mathbb{Z}$. Ce ne sont ni des unit\'es, ni des diviseurs de z\'ero! Nous les appelons des "\NewTerm{\'el\'ements r\'eguliers}\index{\'el\'ements r\'eguliers}".
	\end{tcolorbox}
	Nous verrons un exemple important d'anneaux dans le cadre de notre \'etude des polynômes (\SeeChapter{voir section de Calcul Alg\'ebrique page \pageref{polynomial ring}}) mais nous en avons d\'ejà vu de très importants lors de notre \'etude des classes de congruence dans la section de Th\'eorie des Nombres page \pageref{congruence}.
	
	Voyons quelques exemples d'anneaux! Lors de notre \'etude des groupes plus haut, nous avons trouv\'e que les structures:
	
	sont tous les quatre des groupes ab\'eliens et les trois premiers sont en plus totalement ordonn\'es.
	
	La loi de division n'\'etant en aucun cas associative, nous pouvons nous restreindre à \'etudier pour chacun des groupes pr\'ecit\'es, le couple de lois: "$+$" et "$\times$".

	Ainsi, il vient très vite que:
	
	constituent des anneaux commutatifs unitaires et intègres.
	
	\begin{tcolorbox}[title=Remarque,colframe=black,arc=10pt]
	Nous consid\'ererons comme \'evident, à ce niveau du discours, que le lecteur aura remarqu\'e que est $\mathbb{Z}$ un "sous-anneau" de $\mathbb{Q}$ dans le sens où les op\'erations d\'efinies sont internes à chacun des ensembles et que les \'el\'ements neutres et identit\'e sont identiques et qu'il existe pour chaque \'el\'ement de ces ensembles un oppos\'e qui est dans le même ensemble. Nous allons approfondir le concept de sous-anneau un peu plus loin.
	\end{tcolorbox}	
	Soit $A$ un anneau. Nous avons les propri\'et\'es suivantes:
	\begin{enumerate}
		\item[P1.] $a+b=a+c\Rightarrow (b=c)\qquad \forall a,b,c\in A$
		\begin{dem}
		Ceci d\'ecoule de la d\'efinition D4 vue au d\'ebut de la partie concernant les structures alg\'ebriques (chaque \'el\'ement a un oppos\'e / sym\'etrique). En effet, nous pouvons ajouter à:
		
		l'\'el\'ement $-a$. Nous obtenons alors:
		
		par l'existence de l'oppos\'e cela donne:
		
		d'où:
		
		\begin{flushright}
			$\blacksquare$  Q.E.D.
		\end{flushright}
		\end{dem}

		\item[P2.] $0\cdot a=0\qquad \forall a\in A$
		\begin{dem}
		Cette propri\'et\'e d\'ecoule des d\'efinitions D3 (existence de l'\'el\'ement neutre), D4 (existence de l'oppos\'e/sym\'etrique), D5 (distributivit\'e par rapport à l'autre loi) ainsi que de la propri\'et\'e P1 ci-dessus. En effet, nous avons:
		
		Nous avons donc:
		
	 	La propri\'et\'e P1 ci-dessus permet de conclure que:
		
		(nous pourrions discuter de la pertinence de ce genre de d\'emonstration...).
		\begin{flushright}
		$\blacksquare$  Q.E.D.
		\end{flushright}
		\end{dem}

		\item[P3.] $(-1)\cdot a=-a\qquad \forall a\in A$
		\begin{dem}
		Cette propri\'et\'e se d\'emontre à l'aide de P2. Nous avons:
		
		en ajoutant $-a$ à cette dernière \'egalit\'e, nous avons:
		
			\begin{flushright}
			$\blacksquare$  Q.E.D.
		\end{flushright}
		\end{dem}
	\end{enumerate}
	
	\paragraph{Sous-anneau}\mbox{}\\\\
	\textbf{D\'efinitions (\#\mydef):} Soit $A$ un anneau et  $S\subset A$ un sous-ensemble de $A$. Nous disons que $S$ est un "\NewTerm{sous-anneau}\index{sous-anneau}" de $A$ si:
	\begin{enumerate}
		\item[P1.] $n\in S$ (l'\'el\'ement neutre de $A$ est aussi celui de $S$)

		\item[P2.] $a\in S \Rightarrow -a\in S$

		\item[P3.] $(a,b)\in S\Rightarrow a+b\in S$

		\item[P4.] $(a,b)\in S\Rightarrow a\cdot b\in S$
	\end{enumerate}
	\begin{tcolorbox}[colframe=black,colback=white,sharp corners]
	\textbf{{\Large \ding{45}}Exemple:}\\\\
	L'anneau $\mathbb{Z}$ est un sous-anneau de $\mathbb{Q}$ et $\mathbb{R}$
	\end{tcolorbox}
	
	\subsubsection{Corps}
	\textbf{D\'efinition (\#\mydef):} Nous d\'esignons un ensemble de nombres par le terme "\NewTerm{corps}\index{corps (ensemble)}\label{field (set)}" si:
	
	Donc un corps est un anneau non nul dans lequel tout \'el\'ement non nul est inversible ou en d'autres termes: un anneau dont tous les \'el\'ements non nuls sont des unit\'es est un corps.
	\begin{tcolorbox}[title=Remarques,colframe=black,arc=10pt]
	\textbf{R1.} Si la loi interne $\times$ est \'egalement commutative, le corps est dit "\NewTerm{corps commutatif}\index{corps commutatif}".\\

	\textbf{R2.} Les quaternions (\SeeChapter{voir section Nombres page \pageref{quaternions}}) forment par exemple un corps non commutatif pour l'addition et la multiplication.
	\end{tcolorbox}
	Voyons des exemples de corps parmi les anneaux unitaires suivant:
	
	Il nous faut d'abord d\'eterminer lesquels ne constituent pas des groupes par rapport à la loi interne de multiplication "$\times$".
	
	Comme nous l'avons d\'ejà vu dans notre \'etude des groupes pr\'ec\'edemment, il est \'evident qu'il nous faut \'eliminer $(\mathbb{Z},+,\times)$ à cause de l'existence des inverses qui n'est pas assur\'ee dans cet ensemble.

	Ainsi, les corps fondamentaux de l'arithm\'etique sont:
	
	et puisque la loi de multiplication "$\times$" est commutative dans ces ensembles, nous pouvons affirmer que ces corps sont \'egalement des corps commutatifs.

	Nous avons souvent dans les petites classes le sch\'ema suivant pour le corps le plus important:
	
	Ainsi, nous appellerons "corps" un système $C$ de nombres r\'eels ou complexes $a$ tels que la somme, la diff\'erence, le produit et le quotient de deux quelconques de ces nombres $a$ appartiennent au même système $C$.
	
	Nous \'enonçons \'egalement cette propri\'et\'e de la manière suivante: les nombres d'un corps se reproduisent par les op\'erations rationnelles (addition, soustraction, multiplication, division). Ainsi, il est \'evident que le nombre z\'ero ne pourra jamais former le d\'enominateur d'un quotient et l'ensemble des entiers ne peut former un corps car la division dans l'ensemble des nombres entiers ne donne pas n\'ecessairement un r\'esultat dans ce même ensemble.
	
	\pagebreak
	\subsubsection{Espaces vectoriels}\label{vector space}
	Lorsque nous d\'efinissons un "\NewTerm{vector}" (\SeeChapter{voir section de Calcul Vectorel page \pageref{vector}}), nous faisons habituellement r\'ef\'erence à un "espace euclidien" (voir aussi la section de Calcul Vectoriel) de $n$ dimensions de $\mathbb{R}^n$. Cependant, la notion d'espace vectoriel est beaucoup beaucoup plus vaste que ce dernier qui ne repr\'esente qu'un cas particulier!
	
	\textbf{D\'efinition (\#\mydef):} Un "\NewTerm{espace vectoriel}\index{vector space}" (EV) ou "\NewTerm{$K$-espace vectoriel}" (abr\'eg\'e parfois $K$-ev) sur le corps $K$ (nous prendrons fr\'equemment pour ce corps $\mathbb{R}$ ou $\mathbb{C}$) est un ensemble $(E,+,\cdot)$ poss\'edant les propri\'et\'es suivantes:
	
	Nous avons donc deux lois de composition (en prenant les notations traditionnelles des vecteurs qui sera peut-être plus parlante et utile pour la suite...):
	\begin{enumerate}
		\item Une loi de composition interne: l'addition not\'ee "$+$" qui v\'erifie:
		\begin{enumerate}[label*=\arabic*.]
			\item Associativit\'e:
			
			
			\item Commutativit\'e: 
			
			
			\item \'el\'ement neutre:
			
			
			\item \'el\'ement oppos\'e 
			
		\end{enumerate}
		\item Une loi de composition externe: la multiplication par un scalaire, not\'ee "$\cdot$" (pour \'eviter la confusion avec le produit vectoriel que nous introduirons dans la section de Calcul Vectoriel à la page \pageref{cross product}), qui v\'erifie:
		\begin{enumerate}[label*=\arabic*.]
			\item Associativit\'e:
			
			
			\item Distributivit\'e à droite par rapport au corps $K$:  
			
			
			\item Distributivit\'e à gauche par rapport à $E$:
				
	
			\item  \'El\'ement neutre (de $K$ sur $E$):
			
		\end{enumerate}
	\end{enumerate}
	D'où $10$ propri\'et\'es au total (oui... le fait d'avoir une loi interne ou externe est une propri\'et\'e en soit!).
	
	Nous disons alors que l'espace vectoriel a une "\NewTerm{structure alg\'ebrique vectorielle}" et que ces \'el\'ements sont des "\NewTerm{vectors}", les \'el\'ements de $K$ \'etant des "scalaires" (\SeeChapter{voir section Nombres page \pageref{scalar}}).
	\begin{tcolorbox}[title=Remarques,colframe=black,arc=10pt]
	\textbf{R1.} Les op\'erations respectives s'utilisent fr\'equemment comme l'addition et la multiplication que nous connaissons d\'ejà très bien sur equation, ce qui est bien commode pour nos habitudes...\\
	
	\textbf{R2.} Dor\'enavant, pour distinguer les \'el\'ements du corps $K$ et de l'ensemble $E$, nous noterons ceux de $K$ par des lettres grecques et ceux de $E$ par des lettres latines majuscules.
	\end{tcolorbox}
	Il est inutile de d\'emontrer que ces propri\'et\'es sont respect\'ees pour $\mathbb{R}^n$ et, par cons\'equent pour $\mathbb{R}^2$. Nous pouvons cependant nous poser la question à propos de certains sous-ensembles de $\mathbb{R}^n$.
	\begin{tcolorbox}[colframe=black,colback=white,sharp corners]
	\textbf{{\Large \ding{45}}Exemples:}\\\\
	E1.  Consid\'erons la r\'egion rectangulaire de $\mathbb{R}^3$ illustr\'ee dans la figure ($a$) et en perspective dans la figure ($c$) ci-dessous:
	\begin{figure}[H]
		\begin{center}
			\includegraphics[scale=0.9]{img/arithmetics/vector_space_concept.jpg}
		\end{center}	
		\caption{Exemple du concept d'espace vectoriel}
	\end{figure}
	Ce sous-ensemble de $\mathbb{R}^2$ n'est pas un espace vectoriel car, entre autres, la propri\'et\'e d'op\'eration interne du groupe ab\'elien n'est pas satisfaite. En effet, si nous prenons deux vecteurs $\vec{v}$,$\vec{w}$ à l'int\'erieur du rectangle et que nous les additionnons $\vec{v}+\vec{w}$, il se peut que le r\'esultat sorte du rectangle. Par contre, il est facile de voir que la droite $\mathbb{R}^1$ (infinie) illustr\'ee dans la figure ($b$) respecte toutes les propri\'et\'es \'enum\'er\'ees pr\'ec\'edemment et, par cons\'equent, d\'efini un espace vectoriel. Notons bien, cependant, que cette droite se doit de passer par l'origine, sinon la propri\'et\'e d'\'el\'ement neutre du groupe ab\'elien ne serait pas respect\'ee (l'\'el\'ement neutre n'existant plus).\\
	
	E2. Un autre exemple d'un espace vectoriel est "\NewTerm{l'espace vectoriel polynomial}\index{espace vectoriel polynomial}" des polynômes de degr\'e deux ou moins not\'e  $\mathcal{P}^2$  (\SeeChapter{voir section de Calcul Alg\'ebrique page \pageref{polynomial vector}}). Par exemple, deux \'el\'ements choisis al\'eatoirement de cet espace sont:
	
	Cet ensemble respecte les propri\'et\'es d'un espace vectoriel. En effet, si nous additionnons deux polynômes de degr\'e deux ou moins, nous obtenons un autre polynôme de degr\'e deux ou moins. Nous pouvons aussi multiplier un polynôme par un scalaire sans changer l'ordre (ou degr\'e) de celui-ci, etc. Nous pouvons donc repr\'esenter un polynôme par des vecteurs dont les termes sont les coefficients du polynôme.
	\end{tcolorbox}
	Mentionnons que nous pouvons aussi former des espaces vectoriels avec des ensembles de fonctions plus g\'en\'erales que des polynômes. Il importe seulement de respecter les $10$ propri\'et\'es fondamentales d'un espace vectoriel !
	
	Ainsi d\'efini, un espace vectoriel $E$ sur $K$ est une action de $(K,\times)$ sur $(E,+)$ qui est compatible avec la loi de groupe (par extension un "automorphisme" - voir la d\'efinition plus loin - sur $(E,+)$).
	
	\textbf{D\'efinition (\#\mydef):} Soit $E$ un espace vectoriel, nous appelons "\NewTerm{sous-espace vectoriel}\index{sous-espace vectoriel}" (SEV) $F$ de $E$ un sous-ensemble de $E$ si et seulement si (notation des matheux...):
	
	ou en utilisant une autre notation (celle utilis\'ee plutôt par les physiciens):
	
	
	\subsubsection{$C$-algèbre $A$}
	\textbf{D\'efinition (\#\mydef):}  Une "\NewTerm{$C$-algèbre $A$}\index{$C$-algèbre $A$}" où $C$ est un corps commutatif (appel\'ee aussi souvent "$K$-alègbre $A$" pour "Körper" en allemand)), est un ensemble $A$ muni de deux lois de composition internes $+$ (addition) et $\times$ (produit en croix) et d'une loi externe "$\cdot$" (multiplication) à domaine d'op\'erateurs $C$ (produit par un scalaire) si et seulement si:
	
	\begin{tcolorbox}[colframe=black,colback=white,sharp corners]
	\textbf{{\Large \ding{45}}Exemples:}\\\\
	E1. Pour reprendre un exemple dans la lign\'ee de celui sur les exemples vectoriels, l'espace euclidien $\mathbb{R}^3$ muni de l'addition "$+$", de la multiplication "$\cdot$" et du produit vectoriel "$\times$" est une $\mathbb{R}$-algèbre non associative et non commutative not\'ee $(\mathbb{R}^3,\mathbb{R},+,\cdot,\times)$.\\
	
	E2.  L'ensemble $\mathbb{C}$  est une $\mathbb{R}$-algèbre (un nombre complexe pouvant être vu comme un vecteur à deux composantes selon ce que nous avons vu dans la section des Nombres page \pageref{gauss plane}).
	\end{tcolorbox}
	
	
	\pagebreak
	\subsubsection{R\'esum\'e}
	Ok ... jusqu'ici il y a eu beaucoup de d\'efinitions et de concepts. Pour r\'esumer les structures alg\'ebriques les plus importantes observ\'ees jusqu'à pr\'esent, nous avons obtenu l'autorisation de Kevin Binz de reproduire ses très belles figures visuelles qui pourraient aider le cerveau du lecteur à avoir une vue d'ensemble des concepts les plus importants vus jusqu'à pr\'esent.

	Commençons donc par introduire les diff\'erentes structures d’ensembles du point de vue où nous les construisons en ajoutant à chaque fois une nouvelle propri\'et\'e:
	\begin{table}[H]
		\centering
		\begin{tabular}{llll|l|}
		\hline
		\rowcolor[HTML]{9B9B9B} 
		\multicolumn{1}{|l|}{\cellcolor[HTML]{9B9B9B}\textbf{Magma}} & \multicolumn{1}{l|}{\cellcolor[HTML]{9B9B9B}\textbf{Semi-groupe}} & \multicolumn{1}{l|}{\cellcolor[HTML]{9B9B9B}\textbf{Monoid}} & \textbf{Groupe} & \textbf{Groupe Ab\'elien} \\ \hline
		\multicolumn{1}{|l|}{Clos} & \multicolumn{1}{l|}{Clos} & \multicolumn{1}{l|}{Clos} & Clos & Clos \\ \hline
		\multicolumn{1}{l|}{} & \multicolumn{1}{l|}{Associativit\'e} & \multicolumn{1}{l|}{Associativit\'e} & Associativit\'e & Associativit\'e \\ \cline{2-5} 
		 & \multicolumn{1}{l|}{} & \multicolumn{1}{l|}{\'el\'ement identit\'e} & \'el\'ement identit\'e & \'el\'ement identit\'e \\ \cline{3-5} 
		 &  & \multicolumn{1}{l|}{} & \'el\'ement inverse & \'el\'ement inverse \\ \cline{4-5} 
		 &  &  &  & Commutativit\'e \\ \cline{5-5} 
		\end{tabular}
		\caption[R\'esum\'e des structures alg\'ebriques les plus courantes]{R\'esum\'e des structures alg\'ebriques les plus courantes\\ (source: \url{https://kevinbinz.com/tag/identity-element/}, auteur: Kevin Binz)}
	\end{table}
	Bien sûr, il n'y a pas de raison particulièrement forte pour "couper" les axiomes dans cet ordre. Des options plus \'esot\'eriques sont disponibles (ci-dessous, les axiomes rouges sont supprim\'es, les verts sont r\'eintroduits), introduisant en même temps certaines structures ensemblistes que nous n'avons pr\'esent\'ees pr\'ec\'edemment:
	\begin{figure}[H]
		\centering
		\includegraphics[scale=0.84]{img/arithmetics/abelian_other_group_types.pdf}	
		\caption[R\'esum\'e des structures alg\'ebriques les plus courantes]{R\'esum\'e des structures alg\'ebriques les plus courantes\\(source: \url{https://kevinbinz.com/tag/identity-element/}, auteur: Kevin Binz)}
	\end{figure}
	
	\begin{figure}[H]
		\begin{center}
			\includegraphics[scale=0.9]{img/arithmetics/group_addition_summary.pdf}
		\end{center}	
		\caption[Th\'eorie des groupes r\'esum\'e addition]{Th\'eorie des groupes r\'esum\'e addition\\(source: \url{https://kevinbinz.com/tag/identity-element/}, auteur: Kevin Binz)}
	\end{figure}
	Si nous consid\'erons ce qui pr\'ecède comme une fonction, alors trois entr\'ees nous concernent: la cible, l'op\'erateur et l'\'el\'ement d'identit\'e. Ainsi, nous pouvons condenser ce qui pr\'ecède en $ (S, +, 0)$.
	
	\begin{figure}[H]
		\begin{center}
			\includegraphics[scale=0.9]{img/arithmetics/group_multiplication_summary.pdf}
		\end{center}	
		\caption[Th\'eorie des groupes r\'esum\'e multiplication]{Th\'eorie des groupes r\'esum\'e multiplication\\(source: \url{https://kevinbinz.com/tag/identity-element/}, auteur: Kevin Binz)}
	\end{figure}
	Si nous consid\'erons ce qui pr\'ecède comme une fonction, alors trois entr\'ees nous concernent: la cible, l'op\'erateur et l'\'el\'ement d'identit\'e. Ainsi, nous pouvons condenser ce qui pr\'ecède en $ (S, \times, 0)$.\label{multiplication binary operator}.
	
	Les deux tableaux ci-dessus vous ont-ils \'et\'e douloureusement similaires? Oui c'est possible!

	Une leçon qu'on vous enseigne en informatique est la suivante: si vous remarquez que vous copiez-collez du code, vous devriez essayer de consolider votre logiciel en une seule fonction.

	De manière analogue, nous pouvons g\'en\'eraliser l'addition et la multiplication, comme ceci:
	\begin{figure}[H]
		\centering
		\includegraphics[scale=0.4]{img/arithmetics/group_addition_multiplication_merge.pdf}	
		\caption[Fusion de l'addition et de la multiplication]{Fusion de l'addition et de la multiplication\\(source: \url{https://kevinbinz.com/tag/identity-element/}, auteur: Kevin Binz)}
	\end{figure}
	Ici, nous g\'en\'eralisons nos trois entr\'ees d\'efinies ci-dessus:
	\begin{itemize}
		\item $S$ devient une instance sp\'ecifique d'un ensemble d'entr\'ee.
		\item $+$ et $\times$ deviennent des instances sp\'ecifiques d’une classe g\'en\'erale d’op\'erateurs.
		\item $0$ et $1$ deviennent des instances sp\'ecifiques d'une classe g\'en\'erale d'\'el\'ements d'identit\'e.
	\end{itemize}
	Nous avons vu pr\'ec\'edemment que cet ensemble particulier de cinq axiomes est un "groupe ab\'elien" (\'egalement appel\'e "groupe commutatif") qui peut être r\'esum\'e comme suit:
	\begin{figure}[H]
		\begin{center}
			\includegraphics[scale=1]{img/arithmetics/abelian_group_summary_structure.pdf}
		\end{center}	
		\caption[Structure de groupe ab\'elien]{Structure de groupe ab\'elien (source: \url{https://kevinbinz.com/tag/identity-element/}, auteur: Kevin Binz)}
	\end{figure}
	Fusionnons ensemble les groupes d'addition et de multiplication dans un corps:
	\begin{figure}[H]
		\begin{center}
			\includegraphics[scale=0.85]{img/arithmetics/field.pdf}
		\end{center}	
		\caption[Structure d'un corps]{Structure d'un corps (source: \url{https://kevinbinz.com/tag/identity-element/}, auteur: Kevin Binz)}
	\end{figure}
	
	\pagebreak
	\subsection{Morphismes}
	Dans de nombreux domaines math\'ematiques, le morphisme fait r\'ef\'erence à une application pr\'eservant les règles d'une structure math\'ematique à une autre. La notion de morphisme est r\'ecurrente dans une grande partie des math\'ematiques contemporaines. Dans la th\'eorie des ensembles, les morphismes sont des fonctions; en algèbre lin\'eaire, ce sont des transformations lin\'eaires; en th\'eorie des groupes, ce sont des homomorphismes de groupe; dans la topologie, ce sont les fonctions continues, etc.
	
	Le concept d'homomorphismes (du grec homoios = semblable et morphê = forme) a \'et\'e d\'efini par les math\'ematiciens car permettant de mettre en \'evidence des propri\'et\'es remarquables des fonctions en particulier avec leurs structures, leur noyau, et de ce que nous appelons les "id\'eaux" (voir plus loin). Ils nous permettront ainsi d'identifier une structure alg\'ebrique d'une autre.
	
	\textbf{D\'efinitions (\#\mydef):} 
	\begin{enumerate}
		\item[D1.] Si $(A,\star)$ et $(B,\circ)$ sont deux magmas (peu importe la notation utilis\'ee pour les lois internes), une application $f$ de $A$ dans $B$ est nomm\'e un "\NewTerm{homomorphisme de magma}\index{homomorphisme de magma}" ou "\NewTerm{morphisme de magma}\index{morphisme de magma}" (par abus de langage nous \'ecrivons parfois juste "homomorphisme") si:
		
		en d'autres termes, si l'image d'un compos\'e dans $A$ est le compos\'e des images dans $B$.
		
		\item[D2.]  Si $(A,\star)$ et $(B,\circ)$ sont deux monoïdes, une application $f$ de $A$ dans $B$ est un "\NewTerm{homomorphisme de monoïde}\index{homomorphisme de monoïde}\label{homomorphism of monoid}" si:
		
		où $1_A$ et $1_B$ sont les \'el\'ements neutres respectifs des monoïdes $A$, $B$.
		
		\item[D3.] Si $A$, $B$  sont deux anneaux, un "\NewTerm{homomorphisme d'anneaux}\index{homomorphisme d'anneaux}" (très important pour la section de Cryptographie page \pageref{cryptography}!) de $A$ dans $B$ est une application $f:A\mapsto B$ telle que nous ayons pour tout $a,a'\in A$:
		
		où $1_A$, $1_B$ sont les \'el\'ements neutres des anneaux $A$, $B$ par rapport à la multiplication "$\cdot$".
		
		Soit $f:A\mapsto B$ un homomorphisme d'anneaux. Alors:
		\begin{enumerate}
			\item[P1.] $f(0)=0$
			\begin{dem}
			Par:
			
			nous avons:
			
			En ajoutant $-f(a)$ des deux côt\'es de l'\'egalit\'e nous avons:
			
			\begin{flushright}
				$\blacksquare$  Q.E.D.
			\end{flushright}
			\end{dem}
	
			\item[P2.] $f(-a)=-f(a)$
			\begin{dem}
			Cette propri\'et\'e d\'ecoule aussi de:
			
			et de la propri\'et\'e P1. Effectivement, nous avons:
			
			En ajoutant $-f(a)$ des deux côt\'es de l'\'egalit\'e, nous obtenons:
			
			\begin{flushright}
				$\blacksquare$  Q.E.D.
			\end{flushright}
			\end{dem}
			
			\item[P3.] Si $a$ est une unit\'e de $A$, alors $f(a)$ est une unit\'e de $B$ et $f(a)^{-1}=f(a^{-1})$.
			\begin{dem}
			Soient $a,b\in A$ tels que:
			 
			Alors par $f(a\cdot b)=f(a)\cdot f(b)$ et $f(1_A)=1_B$, nous avons:
			
			et aussi:
			
			ce qui montre que $f(b)$ est l'inverse de $f(a)$ si $b$ est l'inverse de $a$.
			\begin{flushright}
				$\blacksquare$  Q.E.D.
			\end{flushright}
			\end{dem}
		\end{enumerate}
		Introduisons maintenant un th\'eroème relativement puissant:
		\begin{theorem}
		Montrons maintenant qu'un homomorphisme d'anneaux $f:A\mapsto B$ est injectif si et seulement si l'\'el\'ement $0$ est la seule pr\'e-image de $0$ (et donc r\'eciproquement), ce qui se note techniquement:
		
		c'est-à-dire que le noyau est "trivial".
		\end{theorem}
		\begin{dem}
		La condition est clairement n\'ecessaire. Montrons qu'elle est suffisante:

		Nous supposons donc que $\ker(f)=0$. Soit $a,a'\in A$ tel que $f(a)=f(a')$. Alors comme nous avons un homomorphisme d'anneaux nous pouvons \'ecrire:
		
		Ce qui implique que $a-a'=0$ donc que $a=a'$.
	
		Ce qui montre que $f$ est injectif si c'est un homomorphisme et que $\ker (f)=0$ en est effectivement une condition suffisante.
		\begin{flushright}
			$\blacksquare$  Q.E.D.
		\end{flushright}
		\end{dem}
	
		\item[D4.] Soient $(A,+)$ et $(B,\star)$ , deux groupes et $f$ une application $f:A\mapsto B$. Nous disons que $f$ est un "\NewTerm{homomorphisme de groupe}\index{homomorphisme de groupe}\label{homomorphism of group}" si (nous pourrions tout aussi bien mettre "$\star$" au lieu de "$+$" dans le premier groupe et "$+$" au lieu de "$\star$" dans le deuxième groupe, la d\'efinition resterait la même en remplaçant simplement les op\'erateurs respectifs!):
		
		où $1_A$, $1_B$  sont les \'el\'ements neutres respectifs des groupes $A$, $B$. Nous remarquons que la seule diff\'erence entre un homomorphisme d'anneau et un homomorphisme de groupe est que ce dernier à deux lois au lieu d'une et que nous y rajoutons le concept d'inverse.

		Ceci dit, la troisième proposition ci-dessus est en fait une cons\'equence de la d\'efinition compos\'ee uniquement des deux premières lignes. Effectivement, consid\'erons un homomorphisme $f$ entre les groupes $(A,+)$ et $(B,\star)$ avec $1_A$ et $1_B$ respectivement les \'el\'ements neutres de $A$ et $B$, nous avons alors:
		
		Dès lors:
		
		Enfin:
		
		\begin{tcolorbox}[colframe=black,colback=white,sharp corners]
		\textbf{{\Large \ding{45}}Exemple:}\\\\
		La fonction exponentielle $e$ est un mrophisme du groupe $(\mathbb{R}^+, +)$ sur le groupe $(\mathbb{R}^+, \cdot)$.
		\end{tcolorbox}
	
		\item[D5.] Soit $f$ une application $f:A\mapsto B$ d'un corps vers un autre. Nous disons que $f$ est un "\NewTerm{homorophisme de corps}\index{homorophisme de corps}" si $f$ est un homomorphisme d'anneaux...
		
		Effectivement, le fait que l'homomorphisme de corps soit le même que celui d'un anneau tient juste au fait que la diff\'erence entre les deux structures est que les \'el\'ements du corps sont tous inversibles (aucune loi ou propri\'et\'e de loi ne diffère entre les deux selon leur d\'efinition).
		
		Montrons maintenant que tout homomorphisme de corps est injectif ("homomorphisme injectif") en se rappelant que plus haut nous avons d\'emontr\'e que tout homomorphisme d'anneaux l'\'etait!
		\begin{dem}
		Si $a$ est diff\'erent de $0$ et $b=a^{-1}$ (nous utilisons ici la propri\'et\'e que les \'el\'ements d'un corps sont inversibles!) alors:
		
		Donc lorsque $a$ est diff\'erent de z\'ero, $f(a)$ est diff\'erent de $0$ ce qui prouve que $\ker (f)=\{0\}$ et donc que $f$ est injective.
		\begin{flushright}
			$\blacksquare$  Q.E.D.
		\end{flushright}
		\end{dem}
	
		\item[D6.] Soient $A$ et $B$ deux $K$-espace vectoriels et $f:A\mapsto B$ une application de $A$ dans $B$. Nous disons que $f$ est une "\NewTerm{application lin\'eaire}\index{application lin\'eaire}" ou "\NewTerm{homomorphisme d'espaces vectoriels}\index{homomorphisme d'espaces vectoriels}" (il est sous-entendu que c'est relativement aux lois indiqu\'ees et pour l'application choisie) si:
		
		et nous notons $L(A, B)$ l'ensemble des applications lin\'eaires.
		\begin{tcolorbox}[title=Remarques,colframe=black,arc=10pt]
		\textbf{R1.} Nous avions d\'ejà d\'efini plus haut le concept d'application lin\'eaire mais n'avions pas pr\'ecis\'e que les deux ensembles $A$ et $B$ \'etaient des $K$-espace vectoriels.\\
	
		\textbf{R2.} L'application lin\'eaire est appel\'ee "\NewTerm{forme lin\'eaire}\index{forme lin\'eaire}" si et seulement si $B=K$.
		\end{tcolorbox}
	
		\item[D7.] Si l'homomorphisme est bijectif nous dirons alors que $f$ est un "\NewTerm{isomorphisme}\index{isomorphisme}\label{isomorphism}". S'il existe un isomorphisme entre $A$ et $B$, nous disons que $A$ et $B$ sont "isomorphes" et nous noterons cela $A\simeq B$.
		\begin{tcolorbox}[title=Remarque,colframe=black,arc=10pt]
		L'isomorphisme permet au fait d'identifier deux ensembles munis d'une structure alg\'ebrique identique (que ce soit groupe, anneau, etc.) mais dont les \'el\'ements sont nomm\'es d'une façon diff\'erente.
		\end{tcolorbox}
		
		\item[D8.]  Si l'homomorphisme $f$ est une application uniquement interne, nous dirons alors que $f$ est un "\NewTerm{endomorphisme}\index{endomorphisme}\label{endomorphism}" (en d'autres termes, nous avons un endomorphisme si dans la d\'efinition de l'homomorphisme nous avons $A = B$).
		
		\begin{tcolorbox}[title=Remarque,colframe=black,arc=10pt]
		Si nous avons un endomorphisme $f$ de $E$, $f$ est donc restreint à $\Im(f)$. Donc le terme "endomorphisme" veut juste dire que l'application $f$ arrive dans $E$ et pas qu'elle touche tous les \'el\'ements de $E$ lui-même. Nous avons $f(E)\subset E$ et pas forc\'ement $f(E)=E$ car dans ce dernier cas nous disons que $f$ est surjective comme nous l'avons d\'ejà vu.
		\end{tcolorbox}
		
		\item[D9.] Si l'endomorphisme $f$ est en plus bijectif (donc en d'autres termes si l'homomorphisme est un endomorphisme \underline{et} un isomorphisme), nous dirons alors que $f$ est un "\NewTerm{automorphisme}\index{automorphisme}\label{automorphism}".
	\end{enumerate}
	\begin{figure}[H]
		\centering
	    \begin{tikzpicture}
		\def\homomorphism{(0:0cm) circle (5.0cm)}
	  	\def\isomorphism{(180:2.5cm) ellipse (2.0cm and 3.0cm)}
	  	\def\endomorphism{(90:2.0cm) ellipse (3.0cm and 1.5cm)}
	      \begin{scope}[fill opacity=0.1]
	        \fill[magenta] \homomorphism;
	      \end{scope}
	
	      \begin{scope}[fill opacity=0.5]
	        \fill[cyan] \endomorphism;
	      \end{scope}
	
	      \begin{scope}[fill opacity=0.5]
	        \fill[orange] \isomorphism;
	      \end{scope}
	
	      \draw \homomorphism;
	      \draw \isomorphism;
	      \draw \endomorphism;
	
	      {
	        \scalefont{2.0}
	        \node[text=black] at ( 1.8,-2) {Homomorphisme};
	      }
	
	      {
	        \scalefont{1.6}
	        \node[text=black] at (-2.5, 0) {Isomorphisme};
	      }
	
	      {
	        \scalefont{1.4}
	        \node[text=black] at (   1, 2) {Endomorphisme};
	      }
	      \node[align=left] at (  -2, 2) {Auto-\\morphisme};
	    \end{tikzpicture}
		\caption[Diagramme de Venn de diff\'erents types d'homomorphismes]{Diagramme de Venn de diff\'erents types d'homomorphismes (source: Wikipédia, auteur: Martin Thoma}
	\end{figure}
	
	
	\subsubsection{Id\'eal}
	\textbf{D\'efinition (\#\mydef):}  Soit $A$ un anneau commutatif (comme $(\mathbb{R},+,\cdot)$ par exemple). Un sous-ensemble $S\subset A$ est un "\NewTerm{id\'eal}\index{id\'eal}" si:
	\begin{enumerate}
		\item[P1.] Pour tout $a,a'\in S$:
		
	
		\item[P2.] Pour tout $a\in S$ et tout $r\in A$:
		
	\end{enumerate}
	En d'autres termes, un id\'eal est un sous-ensemble ferm\'e pour l'addition et stable pour la multiplication par un \'el\'ement quelconque de $A$.
	\begin{tcolorbox}[colframe=black,colback=white,sharp corners]
	\textbf{{\Large \ding{45}}Exemple:}\\\\
	L'ensemble des nombres pairs $\mathbb{Z}_{2k}$ est par un exemple d'id\'eal de l'ensemble des nombres relatifs $\mathbb{Z}$.
	\end{tcolorbox}
	
	\begin{tcolorbox}[title=Remarque,colframe=black,arc=10pt]
	Les id\'eaux $S=\{0\}$ et $S=A$ sont appel\'es les  "\NewTerm{id\'eaux triviaux}".
	\end{tcolorbox}
	Pour savoir si un id\'eal est \'egal à tout l'anneau, il est utile d'utiliser la propri\'et\'e suivante qui sp\'ecifie que si $A$ est un anneau et $I$ un id\'eal de $A$, alors si $1\in A$ nous avons $I=A$.
	
	\begin{dem}
	Ceci r\'esulte de la propri\'et\'e P2 de la d\'efinition d'un id\'eal:
	
	Pour tout $r\in A$, nous avons $r=r\cdot 1\in 1$ car $1\in I$.
	\begin{flushright}
		$\blacksquare$  Q.E.D.
	\end{flushright}
	\end{dem}
	Un premier exemple d'id\'eal est donn\'e par le noyau d'un homomorphisme d'anneaux. Effectivement, d\'emontrons que le noyau d'un homomorphisme $F:R\mapsto S$ est un id\'eal de $R$.
	\begin{dem}
	$R$ $a,a'\in \ker (f)$. Alors:
	
	ce qui montre que $a+a'\in \ker(f)$. Soit $r\in R$, alors:
	
	ce qui montre que $r\cdot a\in \ker(f)$.
	\begin{flushright}
		$\blacksquare$  Q.E.D.
	\end{flushright}
	\end{dem}
	Proposition: Soit $A$ un anneau et soit $a\in A$. Le sous-ensemble:
	
	not\'e $(a)$ ou $aA$, est un id\'eal (nous allons voir un exemple concret après la prochaine d\'efinition).
	
	\textbf{D\'efinitions (\#\mydef):}
	\begin{enumerate}
		\item[D1.] Un id\'eal $I\neq A$ d'un anneau $A$ est dit "\NewTerm{"id\'eal principal}" s'il existe $a\in A$ tel que $I=(a)$.
	
		\item[D2.] Un anneau dont tous les id\'eaux sont principaux est dit "\NewTerm{anneau principal}".
	\end{enumerate}
	\begin{tcolorbox}[colframe=black,colback=white,sharp corners]
	\textbf{{\Large \ding{45}}Exemple:}\\\\
	Montrons maintenant que l’anneau $\mathbb{Z}$ est principal (car tous ses id\'eaux sont principaux).\\
	
	Soit $I$ un id\'eal de $\mathbb{Z}$  (il est facile d'en choisir un: par exemple, tous les multiples de $2$ ou $3$, etc.). Soit $r\in I$ le plus petit entier positif non nul de $I$. Nous montrerons que $I=r\mathbb{Z}=(r)$.\\
	
	Soit $a$, un \'el\'ement quelconque de $I$. La division euclidienne nous permet d'\'ecrire:
	
	avec $0\ge r' <r$ (comme nous l'avons d\'ejà prouv\'e).\\
	
	Mais comme $r'= a-qr$ et que $a, r \in I$, selon la d\'efinition d’un id\'eal, nous avons $ r' \in I$ (la somme ou la diff\'erence des \'el\'ements d'un id\'eal appartenant à l'id\'eal). Par le choix de $r$ ($r'$ \'etant inf\'erieur à $r$), cela signifie que $r' = 0$ et donc que $a = rq$.\\
	
	Ainsi, chaque \'el\'ement de $I$ est un multiple $r$ de $q$:
	
	Donc pour l'ensemble des nombres pairs (not\'es comme nous le savons par $2\mathbb{Z}$ ou $\mathbb{Z}_{2k}$) nous avons:
	
	\end{tcolorbox}
	L'exemple ci-dessus utilise uniquement la division euclidienne sur $\mathbb{Z}$. On peut alors g\'en\'eraliser ce r\'esultat aux anneaux qui possèdent une division euclidienne. Ainsi, par exemple, l'anneau $K[X]$ des polynômes (\SeeChapter{voir la section de Calcul Alg\'ebrique page \pageref{polynomial ring}}) avec des coefficients dans un corps $K$ est un anneau principal car il a une division euclidienne.

	\begin{theorem}
	L'anneau $K[X]$ des polynômes à coefficients dans un corps $K$ est un anneau principal (tous ses id\'eaux sont principaux).
	\end{theorem}
	\begin{dem}
	Soit $I$ un id\'eal de $K[X]$. Notons $d$ le plus petit degr\'e que puisse avoir un polynôme non nul de $I$. Si $d=0$ alors $1\in I$ et donc $I=1\cdot K[X]=K[X]$. Autrement, soit $a(X)$ un polynôme de degr\'e $d$. Si $u(X)\in I$ alors on peut diviser $u(X)$ par $a(X)$. Il existe $q(X),r(X)\in K[X]$ tels que $\deg(r)<\deg(a)=d$ et:
	
	Donc $r(X)\in I$ ce qui entraîne $r=0$ (autrement contradiction avec la minimalit\'e de $d$). Par suite:
	
	Nous venons de montrer que:
	
	\begin{flushright}
		$\blacksquare$  Q.E.D.
	\end{flushright}
	\end{dem}
	Pour en revenir à $\mathbb{Z}$... nous venons alors de prouver que les seuls id\'eaux sont ceux de la forme $r\mathbb{Z}$. De plus si nous avons $d$ et  $r$ qui sont des entiers $>1$. Alors $r\mathbb{Z}\subset d\mathbb{Z}$ si et seulement si $d | r$.
	\begin{dem}
	Si $d | r$ alors il existe $n$ avec $r=d\cdot n$. Soit $m\cdot a$ un \'el\'ement de $r\mathbb{Z}$. Alors:
	
	ce qui montre que $r\mathbb{Z}\subset d\mathbb{Z}$.

	R\'eciproquement, si $r\in d\mathbb{Z}$ ceci implique que $r$ est de la forme $d\cdot n$ et ceci prouve que $d$ divise $r$.
	\begin{flushright}
		$\blacksquare$  Q.E.D.
	\end{flushright}
	\end{dem}
	\begin{theorem}
	D\'emontrons aussi qu'un anneau $R$ est un corps si et seulement s'il ne possède que les id\'eaux triviaux:
	
	\end{theorem}
	\begin{dem}
	Montrons que la condition est n\'ecessaire: Soit $I$ un id\'eal non nul de $R$ (c'est-à-dire diff\'erent de $\{0\}$) et $r\in I$ un \'el\'ement non nul. Par hypothèse (qu'il s'agit d'un corps), il est inversible, c'est-à-dire qu'il existe $t\in R$ tel que:
	
	Ceci implique que $1\in I$ et donc, par un r\'esultat obtenu plus haut $I=R$.
	
	R\'eciproquement, supposons que tout id\'eal $I\neq R$ soit l'id\'eal nul (c'est-à-dire $\{0\}$). Alors si $r\in R$ est un \'el\'ement non nul de $R$, l'id\'eal principal $(r)$ doit être \'egal à $R$. Mais ceci implique que $1\in (r)$ et donc qu'il existe $x\in R$ avec $r\cdot x=1$ ce qui montre que $r$ est inversible. L'anneau $R$ est donc un corps.
	\begin{flushright}
		$\blacksquare$  Q.E.D.
	\end{flushright}
	\end{dem}
	Cette caract\'erisation va nous permettre de d\'emontrer relativement facilement que:
	\begin{theorem}
	Tout homomorphisme partant d'un corps est injectif. Soit que si $f:R\mapsto S$ est un homomorphisme où $R$ est un corps, alors $f$ est injectif.
	\end{theorem}
	\begin{dem}
	Nous mettons ensemble ce qui a \'et\'e vu jusque-là:
	\begin{itemize}
		\item Nous avons d\'emontr\'e plus haut que le noyau $\ker(f)$ d'un homomorphisme d'anneau est un id\'eal.
	
		\item Mais nous avons \'egalement d\'emontr\'e plus haut qu'un ensemble \'etait un corps si $\{0\}$ et $R$ sont les seuls triviaux id\'eaux.
	\end{itemize}
	Dès lors pour les deux points, nous avons soit $\ker(f)=0$ ou $\ker(f)=R$ pour le corps (\'etant donn\'e qu'il inclut le concept d'anneau!).
	
	Mais vu que $f(1)=1\neq 0$ (de par la d\'efinition d'un homomorphisme) il s'ensuit qu'il reste $\ker(f)=\{0\}$ (puisque nous avons d\'emontr\'e que si $A$ est un anneau et $I$ un id\'eal de $A$ alors si $1\in I$ alors $I=R$). Ceci implique par un th\'eorème pr\'ec\'edent (où nous avons d\'emontr\'e que si $\ker(f)=\{0\}$ l'homomorphisme est injectif) que... $f$ est injective.
	\begin{flushright}
		$\blacksquare$  Q.E.D.
	\end{flushright}
	\end{dem}
	\'etudions maintenant les homomorphismes dont l'anneau de d\'epart est $\mathbb{Z}$. Soit $A$ un anneau et $f:\mathbb{Z}\mapsto A$  un homomorphisme. Par d\'efinition d'un homomorphisme et par ses propri\'et\'es, il faut que $f(0)=0$ et $f(1)=1$ (parmi d'autres). Mais il faut encore que:
	
	pour tout $k\in\mathbb{Z}$. Ainsi $f$ est complètement d\'etermin\'e par la donn\'ee de $f(1)$ et est donc unique. R\'eciproquement, nous montrons que l'application $f:\mathbb{Z}\mapsto A$ d\'efinie par:
	
	est un homomorphisme d'anneaux. En r\'esum\'e, il existe un et un seul homomorphisme de $\mathbb{Z}$ dans un anneau quelconque $A$.
	
	\textbf{D\'efinition (\#\mydef):} Soient $A$ un anneau et $f:\mathbb{Z}\mapsto A$ l'unique homomorphisme d\'efini pr\'ec\'edemment. Si $f$ est injectif, nous dirons que $A$ est de "\NewTerm{caract\'eristique nulle}\index{caract\'eristique nulle}" et nous notons cela:
	
	Sinon, $\ker(f)$ est un id\'eal non trivial de $\mathbb{Z}$ et comme $\mathbb{Z}$ est dès lors principal (comme nous l'avons d\'emontr\'e plus haut) il est de la forme $k\mathbb{Z}$ avec $k>0$. L'entier $k$ est appel\'e la "\NewTerm{caract\'eristique de $A$}\index{caract\'eristique}" et nous avons alors:
	
	\begin{tcolorbox}[title=Remarque,colframe=black,arc=10pt]
	Moins formellement, la caract\'eristique d'un anneau $A$ est le plus petit entier positif $k$ tel que $k\cdot 1_A=0$. S'il n'y en a pas, alors la caract\'eristique est nulle.
	\end{tcolorbox}	
	La d\'efinition ci-dessus n'est peut-être pas très claire (du moins pour moi cela ne l'est pas!). Voyons donc une autre approche pour introduire la caract\'eristique beaucoup plus d\'etaill\'ee ...:
	
	Soit $A$ un anneau et n'importe lequel de ses \'el\'ements $a$, d\'esignons les entiers dans ce qui va suivre en gras. Ainsi, $\pmb{1}$ est le nombre entier un, par exemple, $\pmb{0}$ est le nombre entier z\'ero, alors que $0_A$ et $1_A$ sont les \'el\'ements neutres de $A$.

	Comme $A$ sous l'addition est un groupe ab\'elien, on peut d\'efinir comme d'habitude:
	
	Par d\'efinition d’un homomorphisme, nous avons si $ f:\mathbb{Z}\mapsto A$ que:
	
	alors pour rappel:
	\begin{itemize}
		\item $f(\pmb{m}+\pmb{n})=f(\pmb{m})+f(\pmb{n})$
		\item $f(\pmb{mn})=f(\pmb{m})f(\pmb{n})$
		\item $f(\pmb{1})=1_A$
	\end{itemize}
	donc si $f$ est un homomorphisme d'anneaux. Comme tous les homomorphismes d'anneau, $f:\mathbb{Z}\mapsto A$ a un noyau qui est un id\'eal, comme nous venons de le prouver, donc nous avons $\ker(f)=\pmb{k}\mathbb{Z}$ pour un unique $\pmb{k}\ge\pmb{0}$.
	
	Si $k=0$ alors, comme nous l'avons prouv\'e beaucoup plus haut, $f$ est injectif. Sinon, nous avons:
	
	par d\'efinition du noyau, ce qui signifie que:
	
	pour chaque $a\in A$. Cet entier $\pmb{k}$ est la "caract\'eristique" de $S$.
	
	C'est-à-dire que $\text{char}(A)$ est le plus petit nombre positif $\pmb{k}$ tel que:
	
	si un tel nombre $\pmb{k} \in \mathbb{N}$ existe, et $0$ sinon.
	
	\begin{tcolorbox}[colframe=black,colback=white,sharp corners]
	\textbf{{\Large \ding{45}}Exemples:}\\\\
	E1. Le seul anneau qui a $\text{char}(A)=1$ tel que:
	
	est l'anneau trivial $A=\{0_A\}$.\\
	
	E2. L'anneau $\mathbb{Z}$ a la caract\'eristique z\'ero car l'homomorphisme unique $f:\mathbb{Z} \mapsto\mathbb{Z}$ est l'identit\'e. C'est donc injectif.\\
	
	E3. Les injections $\mathbb{Z}\mapsto \mathbb{Q}$ et $\mathbb{Z}\mapsto \mathbb{R}$ montrent que $\mathbb{Q}$ et $\mathbb{R}$ (et $\mathbb{C}$ \'egalement) sont des corps de caract\'eristique nulle.
	\end{tcolorbox}
	\begin{theorem}
	Nous nous proposons maintenant de d\'emontrer que la caract\'eristique d'un anneau intègre (et en particulier d'un corps) est \'egale $0$ ou à un premier $p$.
	\end{theorem}
	\begin{dem}
	Nous montrons la contrapos\'ee. Soit $A$ un anneau de caract\'eristique $m\neq 0$ avec $m$ non premier.
	
	Il existe alors des entiers naturels $n,r\in\mathbb{N}$ contraints par $n,r<m$ tels que $m=n\cdot r$.

	Soit $f:\mathbb{Z}\mapsto A$ l'unique homomorphisme (d\'efini pr\'ec\'edemment). Par d\'efinition (de l'id\'eal) de $m$, nous avons $f(m)=0_A$ mais (!) $f(r)\neq 0 \neq f(n)$. Mais alors (et c'est là le truc de la preuve!):
	
	ce qui montre que $A$ n'est pas intègre.
	\begin{flushright}
		$\blacksquare$  Q.E.D.
	\end{flushright}
	\end{dem}
	\begin{tcolorbox}[title=Remarque,colframe=black,arc=10pt]
	La r\'eciproque du th\'eorème n'est pas vraie comme le montre l'exemple de l'anneau $\mathbb{R}\times\mathbb{R}$ où l'addition et la multiplication se font composante par composante. C'est un anneau de caract\'eristique nulle mais avec des diviseurs de z\'ero:
	
	\end{tcolorbox}	
	
	
	\begin{flushright}
	\begin{tabular}{l c}
	\circled{95} & \pbox{20cm}{\score{4}{5} \\ {\tiny 16 votes, 62.5\%}} 
	\end{tabular} 
	\end{flushright}

	%to make section start on odd page
	\newpage
	\thispagestyle{empty}
	\mbox{}
	\section{Probabilit\'es}\label{probabilities}
	\lettrine[lines=4]{\color{BrickRed}L}e calcul des probabilit\'es s'occupe des ph\'enomènes al\'eatoires (dits plus esth\'etiquement: "\NewTerm{processus stochastiques}\index{processus stochastiques}" lorsqu'ils sont d\'ependants du temps), c'est-à-dire de ph\'enomènes qui ne mènent pas toujours à la même issue et qui peuvent être \'etudi\'es grâce aux nombres et à leurs cons\'equences et apparitions. N\'eanmoins, même si ces ph\'enomènes ont des issues vari\'ees, d\'ependant du hasard, nous observons cependant une certaine r\'egularit\'e statistique.

	La probabilit\'e est quantifi\'ee comme un nombre compris entre $0$ et $1$ (où $0$ indique l'impossibilit\'e et $1$ indique la certitude). Plus la probabilit\'e d'un \'ev\'enement est \'elev\'ee, plus nous sommes certains que l'\'ev\'enement se produira.
	
	Les probabilités sont très importantes dans la pratique, en particulier à un niveau très élevé de gestion d'entreprise ou politique. En effet, dans de nombreux cas, il est plus pratique d'utiliser une règle simple mais incertaine plutôt qu'une règle complexe mais certaine, même si la vraie règle est déterministe et notre système de modélisation a la fidélité de s'adapter à une règle complexe. Par exemple, la règle simple "les ventes augmenteront" est peu coûteuse à développer et est largement utile, tandis qu'une règle de la forme ", les ventes augmenteront si l'indice d'inflation augmente également, et qu'il n'y a pas d'épidémie, pas de tremblements de terre, pas de conflits, pas d'astéroïdes, pas d'inondations majeures, pas de changements politiques, pas de grèves ..." coûte cher à développer, à entretenir et à communiquer, et après tout cet effort le modèle sera encore très fragile et sujet à l'échec.
	
	Les concepts li\'es aux probabilit\'es ont reçu une formalisation math\'ematique axiomatique en th\'eorie des probabilit\'es (voir plus bas), largement utilis\'ee dans des domaines d’\'etudes tels que les math\'ematiques, la statistique, la finance, les jeux de hasard, les sciences (en particulier la physique), l’intelligence artificielle / l'apprentissage machine, l'informatique, la th\'eorie des jeux et la philosophie pour, par exemple, tirer des conclusions sur la fr\'equence attendue des \'ev\'enements. La th\'eorie des probabilit\'es est \'egalement utilis\'ee pour d\'ecrire les m\'ecanismes sous-jacents et les r\'egularit\'es de systèmes complexes.

	\textbf{D\'efinitions (\#\mydef):} Il existe plusieurs manières de d\'efinir une probabilit\'e. Principalement, nous parlons de:
	\begin{itemize}
		\item[D1.] "\NewTerm{Probabilit\'e exp\'erimentale ou inductive}\index{probabilit\'e inductive}" qui est la probabilit\'e d\'eduite de toute la population concern\'ee.
	
		\item[D2.] "\NewTerm{Probabilit\'e th\'eorique ou d\'eductive}\index{probabilit\'e d\'eductive}" qui est la probabilit\'e connue grâce à l'\'etude du ph\'enomène sous-jacent sans exp\'erimentation. Il s'agit donc d'une connaissance "a priori" par opposition à la d\'efinition pr\'ec\'edente qui faisait plutôt r\'ef\'erence à une notion de probabilit\'e "à posteriori".
	\end{itemize}
	Comme il n'est pas toujours possible de d\'eterminer des probabilit\'es a priori, nous sommes souvent amen\'es à r\'ealiser des exp\'eriences. Il faut donc pouvoir passer de la première à la deuxième solution. Ce passage est suppos\'e possible en termes de limite (avec une population dont la taille tend vers la taille de la population r\'eelle).
	
	La mod\'elisation formelle par le calcul des probabilit\'es a \'et\'e invent\'ee par A.N. Kolmogorov dans un livre paru en 1933. Cette mod\'elisation est faite à partir de l'espace de probabilit\'es ($U$, $A$, $P$) que nous d\'efinirons plus loin et que nous pouvons relier à la th\'eorie de la Mesure (voir section du même nom page \pageref{measure theory}). Cependant, les probabilit\'es ont \'et\'e \'etudi\'ees sur le point de vue scientifique par Pierre deFermat et Blaise Pascal au milieu du 17ème siècle.

	\begin{tcolorbox}[title=Remarque,colframe=black,arc=10pt]
	Si vous avez un professeur ou un formateur qui ose vous enseigner les statistiques et probabilit\'es avec des exemples bas\'es sur des jeux de hasard (cartes, d\'es, allumettes, pile ou face, etc.) d\'ebarrassez-vous en ou d\'enoncez-le à qui de droit car cela signifierait qu'il n'a aucune exp\'erience pratique du domaine et qu'il va vous enseigner n'importe quoi et n'importe comment (normalement les exemples devraient être bas\'es sur l'industrie, l'\'economie ou la R\&D, bref dans des domaines utilis\'es tous les jours par les entreprises mais surtout pas sur des jeux de hasard...!).
	\end{tcolorbox}	
	
	\subsection{Univers des \'ev\'enements}

	\textbf{D\'efinitions (\#\mydef):}

	\begin{itemize}
		\item[D1.]  "\NewTerm{L'univers des \'ev\'enements}\index{univers des \'ev\'enements}", ou "\NewTerm{univers des observables}\index{univers des observables}", $U$ est l'ensemble de toutes les issues (r\'esultats) possibles, appel\'ees "\'ev\'enements \'el\'ementaires", qui se pr\'esentent au cours d'une \'epreuve al\'eatoire d\'etermin\'ee. L'univers peut être fini (d\'enombrable) si les \'ev\'enements \'el\'ementaires sont en nombre fini ou continu (non d\'enombrable) s'ils sont infinis.
	
		\item[D2.] Un "\NewTerm{\'ev\'enement}\index{\'ev\'enement}" quelconque $A$ est un ensemble d'\'ev\'enements \'el\'ementaires et constitue une partie de l'univers des possibles $U$. Il est possible qu'un \'ev\'enement ne soit constitu\'e que d'un seul \'ev\'enement \'el\'ementaire.
	\end{itemize}

	\begin{tcolorbox}[colframe=black,colback=white,sharp corners]
	\textbf{{\Large \ding{45}}Exemples:}\\\\
	E1. Consid\'erons l'univers de tous les groupes sanguins possible, alors l'\'ev\'enement $A$ "l'individu est de rh\'esus positif" est repr\'esent\'e par:
	
	alors que l'\'ev\'enement $B$ "l'individu est donneur universel" est repr\'esent\'e par:
	
	qui constitue donc un \'ev\'enement \'el\'ementaire.\\
	
	E2. Dans $(\mathbb{N},+)$ chaque \'ev\'enement est r\'egulier et dans $(\mathbb{N},\times)$ tout \'ev\'enement non-nul est r\'egulier.
	\end{tcolorbox}
	
\begin{itemize}
	\item[D3.] Soit $U$ un univers et $A$ un \'ev\'enement, nous disons que l'\'ev\'enement $A$ "à lieu" (ou "se r\'ealise") si lors du d\'eroulement de l'\'epreuve se pr\'esente l'issue $i\:\left( i \in U \right)$ et que $i \in A$. Dans le cas contraire, nous disons que $A$ "n'a pas lieu" (ou "ne s'est pas r\'ealis\'e").

	\item[D4.] Le sous-ensemble vide $\varnothing$ de $U$ s'appelle "\NewTerm{\'ev\'enement impossible}\index{\'ev\'enement impossible}". En effet, si lors de l'\'epreuve l'issue $i$ se pr\'esente, nous avons toujours $i \in \varnothing$ et l'\'ev\'enement $\varnothing$ n'a donc jamais lieu.\\
	
	Si $U$ est fini, ou infini d\'enombrable, tout sous-ensemble de $U$ est un \'ev\'enement, ce n'est plus vrai si $U$ est non d\'enombrable (nous verrons dans le chapitre de Statistiques pourquoi).

	\item[D5.] L'ensemble $U$ s'appelle aussi  "\NewTerm{\'ev\'enement certain}\index{\'ev\'enement certain}". En effet, si lors de l'\'epreuve l'issue $i$ se pr\'esente, nous avons toujours $i\in U$ (car $U$ est l'univers des \'ev\'enements). L'\'ev\'enement $U$ a donc toujours lieu!

	\item[D6.] Soit $A$ et $B$ deux sous-ensembles de $U$. Nous savons que les \'ev\'enements   $A \cup B$ et $A \cap B$ sont tous deux des sous-ensembles de $U$ donc des \'ev\'enements qui sont respectivement des "\NewTerm{\'ev\'enements conjoints}\index{\'ev\'enements conjoints}" et des "\NewTerm{\'ev\'enements disjoints}\index{\'ev\'enements disjoints}\label{disjoint events}".
	
	\item[D7.] Si deux \'ev\'enements $A$ et $B$ sont tels que:
	
	les deux \'ev\'enements ne peuvent pas être r\'ealisables pendant la même \'epreuve, nous disons alors qu'ils sont des "\NewTerm{\'ev\'enements incompatibles}\index{\'ev\'enements incompatibles}".

	\item[D8.] Si deux \'ev\'enements $A$ et $B$ sont tels que:
	
	les deux \'ev\'enements peuvent être r\'ealisables dans la même \'epreuve (possibilit\'e de voir un chat noir au moment où on passe sous une \'echelle par exemple...), nous disons inversement qu'ils sont des "\NewTerm{\'ev\'enements ind\'ependants}\index{\'ev\'enements ind\'ependants}".
	
	\item[D9.] Le "\NewTerm{hasard}\index{hasard}" est le manque de structure ou de pr\'evisibilit\'e dans les \'ev\'enements. Une s\'equence al\'eatoire d'\'ev\'enements, de symboles ou d'\'etapes n'a pas d'ordre et ne suit pas un modèle ou une combinaison intelligible. Les \'ev\'enements individuels al\'eatoires sont par d\'efinition impr\'evisibles, mais dans de nombreux cas, la fr\'equence de r\'esultats diff\'erents sur un grand nombre d'\'ev\'enements (ou "essais") est pr\'evisible.
	
	\begin{tcolorbox}[title=Remarque,colframe=black,arc=10pt]
	Si nous choisissons al\'eatoirement (uniform\'ement) un nombre r\'eel dans l’intervalle $[0,1]$, il existe une probabilit\'e nulle que nous choisissions ce nombre. Cela ne signifie pas que nous n'avons pas choisi de num\'ero du tout!\\

	De même avec les rationnels, bien qu'infinis, et denses et tout cela, ils sont très très rares en termes de mesure et de probabilit\'e. Il est parfaitement possible que si nous lançons d'innombrables fl\'echettes sur la ligne r\'eelle, nous atteindrons exactement tous les rationnels et chaque rationnel exactement une fois. Ce sc\'enario est hautement improbable, car les nombres rationnels correspondent à une mesure nulle.\\

	Le domaines des probabilit\'es standard traite avec \textit{quelles sont les chances que cela se produise}? \underline{a priori}, pas a posteriori (cela est le rôle des probabilit\'es bay\'esiennes)!! Nous sommes donc int\'eress\'es par la mesure d’une certaine structure d'un ensemble, dans les aspects modernes de probabilit\'e et de mesure, les nombres rationnels ont une taille \'egale à z\'ero, ce qui signifie une probabilit\'e nulle. De manière plus formelle, comme nous le verrons dans la section de Statistiques à la page \pageref{probability density function}, les probabilit\'es sont obtenues en int\'egrant une fonction de densit\'e de probabilit\'e $f(x)$ sur un intervalle. La fonction est non n\'egative et a la propri\'et\'e:
	
	La probabilit\'e de s\'electionner un r\'eel dans un intervalle $[a,b]\subset \mathbb{R}$ est alors donn\'ee par:
	
	Le lecteur peut voir maintenant que, \'etant donn\'e un nombre r\'eel $x_0\in\mathbb{R}$, nous avons:
		
	\end{tcolorbox}	
\end{itemize}

	\pagebreak
	\subsubsection{Paradoxe du singe savant (loi de Borel)}
	Le "pradoxe du singe savant" stipule qu'un singe frappant des touches de manière al\'eatoire sur un clavier de machine à \'ecrire pendant une dur\'ee infinie tapera presque sûrement un texte donn\'e, tel que l'oeuvre complète de William Shakespeare. En fait, le singe taperait sûrement tous les textes finis possibles un nombre infini de fois. Cependant, la probabilit\'e qu'un Univers rempli de singes dactylographiant une oeuvre complète telle que le \textit{Hamlet} de Shakespeare est si infime que la chance qu'elle se produire au cours d'une p\'eriode de plusieurs centaines de milliers d'ordres de plus que l’âge de l'Univers est extrêmement faible ( mais techniquement pas z\'ero!).
	
	Dans ce contexte, "presque sûrement" est un terme math\'ematique avec une signification pr\'ecise, et le "singe" n'est pas un singe r\'eel, mais une m\'etaphore d'un dispositif abstrait produisant une s\'equence al\'eatoire infinie de lettres et de symboles.
	
	\begin{theorem}
	Si nous avons un nombre infini de singes frappant des touches au hasard sur des claviers de machine à \'ecrire, alors, avec une probabilit\'e de $1$, l'un d'entre eux dactylographiera les oeuvres complètes de William Shakespeare (finirait par \'ecrire tout livre qui ait jamais exist\'e, avec suffisamment de temps).
	\end{theorem}

	\begin{dem}
	Soit $A_n$ l'\'ev\'enement où $n$-ème singe tape l'oeuvre complète de Shakespeare. Ensuite, s'il y a $m$ caractères sur le clavier et $N$ caractères dans les oeuvres complètes de Shakespeare, la probabilit\'e d'obtenir les oeuvres complètes de Shakespeare devrait être de:
	
	pour chaque $n$. De plus, les $A_n$ sont mutuellement ind\'ependants (\'ev\'enements disjoints). Par cons\'equent:
	
	Un nombre infini d'\'ev\'enements $A_n$ se produisent, c'est-à-dire qu'un nombre infini de singes dactylographient les oeuvres complètes de Shakespeare.
	\begin{flushright}
		$\blacksquare$  Q.E.D.
	\end{flushright}
	\end{dem}
	Par cons\'equent, si nous acceptons l'infini dans un argument, nous pouvons alors accepter \'egalement que "l'infini \'etant donn\'e, tout peut arriver".
	
	\begin{tcolorbox}[title=Remarque,colframe=black,arc=10pt]
	Ce résultat est un cas particulier d'un résultat plus général nommé "\NewTerm{second lemme de Borel–Cantelli}\index{second de lemme Borel–Cantelli}". Ce lemme énonce que si les événements $E_n$ sont indépendants et que la somme des probabilités des $E_n$ diverge à l'infini, alors la probabilité qu'une infinité d'entre eux se produisent est de $1$. C'est-à-dire si $\sum_{n=1}^{+\infty} P\left(E_{n}\right)=+\infty$ et les événements $\left(E_{n}\right)_{n =1}^{+\infty}$ sont indépendants, alors $P\left(\limsup _{n \rightarrow +\infty} E_{n}\right)=1$. Formellement:
	
	\end{tcolorbox}
	
	\pagebreak
	 \subsubsection{Est-ce que les probabilit\'es r\'efutent les structures complexes?}
	 \begin{fquote} [P. S. de Laplace] Les questions les plus importantes de la vie ne sont en effet, pour la plupart, en réalité que des problèmes de probabilité.\end{fquote}
	 Les créationnistes traditionnels et les auteurs de livres sur Design Intelligent invoquent régulièrement des arguments de probabilité dans les critiques de l'évolution biologique. Ils soutiennent que certaines caractéristiques de la biologie sont si incroyablement improbables qu'elles n'auraient jamais pu être produites par un processus "naturel" purement aléatoire, même en supposant les milliards d'années d'histoire confirmées par les géologues et les astronomes. Ils assimilent souvent l'hypothèse d'évolution à la suggestion absurde que des singes tapant au hasard sur une machine à écrire pourraient composer une sélection des travaux de Shakespeare, ou qu'une explosion dans un parc d'équipements aérospatiaux pourrait produire un avion de ligne 747 fonctionnel (nous ne citerons pas les gens qui ont écrit cela parce que la qualité de leur contenu ne mérite pas d'être cité ici et cela pourrait leur faire de la publicité!). Des écrits plus récents de ce genre, dans le but de promouvoir une vision du monde de la "conception intelligente", soutient que la biologie fonctionnelle fonctionne sur un sous-ensemble extrêmement petit de l'espace de toutes les séquences d'ADN possibles, et que toute modification du "programme informatique" de la biologie sont, comme les changements apportés aux programmes informatiques humains, presque certains de rendre l'organisme non fonctionnel (encore une fois, nous ne citerons pas les écrits des personnes qui ont écrit cela pour les mêmes raisons qu'auparavant!).
	
	Bien que le public en général ne l'apprécie généralement pas, il est bien connu dans le monde de la recherche scientifique que les arguments fondés sur les probabilités et les statistiques sont chargés de biais et d'erreurs potentielles, et même les chercheurs "experts" peuvent se tromper avec un raisonnement invalide. Pour ces raisons, des cours rigoureux de probabilités et de statistiques sont désormais exigés des étudiants dans pratiquement tous les domaines scientifiques (gardez à l'esprit que l'ingénierie n'est pas une science, c'est pourquoi en fait la grande majorité des ingénieurs ont une très faible compréhension des probabilités et des statistiques) , ainsi que dans de nombreuses autres disciplines. Les avocats doivent être au moins modérément familiarisés avec les arguments de probabilité et de statistiques étant donné que ces derniers peuvent mal compris dans les arguments rhétoriques. Dans le monde de la finance, le surajustement statistique et d'autres erreurs de probabilité et statistiques sont désormais considérés comme l'une des principales raisons du fait que de nombreuses stratégies et fonds d'investissement qui ont fière allure sur papier échouent souvent lamentablement dans l'utilisation dans le monde réel.
	
	Pour illustrer les difficultés avec les arguments de probabilité, les professeurs de mathématiques demandent souvent à leur classe (disons qu'elle compte $23$ élèves) s'ils pensent qu'il est probable que deux personnes ou plus dans la classe aient exactement la même date d'anniversaire. La plupart des étudiants disent que c'est très peu probable, pensant que les chances que deux personnes aient le même anniversaire particulier sont de $1 /365 $, et donc $23$ fois cette quantité n'est que de $23/365 $. Mais cet argument est fallacieux, en effet, si l'on numérote les $23$ personnes de $1$ à $23$, l'événement selon lequel toutes $23$ personnes ont des anniversaires différents est le même que l'événement où la personne $2$ n'a pas la même date d'anniversaire que la personne $ 1 $, et que personne $ 3 $ n'a pas la même date d'anniversaire que la personne $ 1 $ ou la personne $ 2 $, et ainsi de suite, et finalement que le personne $ 23 $ n'a pas le même anniversaire que les personnes de $ 1 $ à $ 22 $.
	
	Appelons ces événements respectivement "Événement $ 2 $", "Événement $ 3 $", etc. On peut également ajouter un "Événement 1", correspondant à l'événement de la personne $ 1 $ ayant une date d'anniversaire, qui se produit avec la probabilité $ 1 $. Cette conjonction d'événements peut être calculée en utilisant une probabilité conditionnelle: la probabilité de l'Événement $2$ est de $364/365$, car la personne $2$ peut avoir une date d'anniversaire autre que la date d'anniversaire de la personne $1$. De même, la probabilité de l'Événement $ 3 $ étant donné que l'Événement $ 2 $ survenu est de $  363 / 365 $, car la personne $ 3 $ peut avoir l'une des dates d'anniversaires non déjà prises par les personnes $ 1 $ et $ 2 $. Cela continue jusqu'à ce que finalement la probabilité de l'Événement $ 23 $ étant donné que tous les événements précédents se soient produits est $ 343/365 $. Enfin, le principe de probabilité conditionnelle implique que la probabilité $ P (A ') $ que deux personnes dans la pièce n'aient pas le même anniversaire est égale au produit de ces probabilités individuelles:
	
	Par conséquent, la probabilité qu'au moins deux personnes de la classe aient la même date d'anniversaire est donnée par (car les deux événements s'excluent mutuellement):
	
	Un argument typique de conception intelligence des créationnistes (argument de niveau école secondaire ...) se présente ainsi: la molécule d'alpha-globine humaine (abrégé HBA1), un composant de l'hémoglobine qui remplit une fonction clé de transfert d'oxygène, est une chaîne protéique basée sur une séquence de $ 141 $ acides aminés \footnote{La molécule d'hémoglobine est composée de quatre chaînes polypeptidiques: deux chaînes alpha de 141 résidus d'acides aminés chacune et deux chaînes bêta de 146 résidus d'acides aminés chacune}. Il existe $20$ différents acides aminés communs dans les systèmes vivants:
	\begin{figure}[H]
		\centering
		\includegraphics[width=0.70\textwidth]{img/arithmetics/amino_acids.jpg}
		\caption[Acides aminés courants]{Acides aminés courants (auteur: ?)}
	\end{figure} 
	Avec le séquençage de l'alpha-globine qui est:
	
	\texttt{Val-Leu-Ser-Pro-Ala-Asp-Lys-Thr-Asn-Val-Lys-Ala-Ala-Trp-Gly-Lys-Val-Gly\\
	-Ala-His-Ala-Gly-Glu-Tyr-Gly-Ala-Glu-Ala-Leu-Glu-Arg-Met-Phe-Ser-Phe-Pro\\
	-Thr-Thr-Lys-Thr-Tyr-Phe-Pro-His-Phe-Leu-Ser-His-Gly-Ser-Ala-GIn-Val-Lys\\
	-Gly-His-Gly-Lys-Lys-Lys-Val-Ala-Asp-Ala-Leu-Thr-Ala-Val-His-Val-Hal-Hal\\
	-Hsp-Met-Pro-Asn-Ala-Leu-Ser-Ala-Leu-Ser-Asp-Leu-His-Ala-Heu-Arg-Val-Asp\\
	-Pro-Val-Asn-Phe-Lys-Leu-Leu-Ser-His-Cys-Leu-Leu-Leu-Ala-His-Leu-Pro-Ala\\
	-Glu-Phe-Thr-Pro-Ala-Val-His-Ala-Ser-Leu-Asp-Lys-Phe-Leu-Ala-Ser-Val-Ser\\
	-Thr-Val-Leu-Thr-Ser-Lys-Tyr-Arg}
	
	Ou pour une version abrégée des quatre chaînes polypeptidiques:
	\begin{itemize}
		\item Chaîne A:\\
		\texttt{VLSPADKTNV KAAUGKVGAH AGEYGAEALE RMFLSFPTTK TYFPHFDLSH GSAOVKGHGK KVADALTHAV AHVDDMPMAL SALSDLHAHK LRVDPVNFKL. LSHCLLVTLA AHLPAEFTPA VHASLDKFLA SVSTVLTSKY R}
		
		\item Chaîne B:\\
		\texttt{VHLTPEEKSA VTALUGKVNV DEVGGEALGR LLVVYPUTOR FFESFGDLST pDAVHGWRV KAHGKKWLGA FSDGLAHLDN LKGTFATLSE LHCDKLHWDP ENFRLLGNVL VCVLAHHFGK EFTPPVOAAY OKVVAGVANA LAHKYH}
		
		\item Chaîne C:\\
		\texttt{WLSPADKTNV KAAUGKVGAH AGEYGAEALE RMFLSFPTTK TYFPHFDLSH GSAOVKGHGK KVADALTNAV AHVDDHPNAL SALSDLHAHK LRVDPVNFKL LSHCLLVTLA AHLPAEFTPA VHASLDKFLA SVSTVLTSKY R}
		
		\item Chaîne D:\\
		\texttt{WHLTPEEKSA VTALUGKVNV DEVGEEALGR LLWYPUTOR FFESFGDLST PDAVMGNPKV KAHGKKVLGA FSDGLAHLDN LKGTFATLSE LHCDKLHVDP ENFRLLGNVL VCVLAHHFGK EFTPPVQ}
	\end{itemize}
	
	Donc, le nombre de chaînes potentielles de longueur $ 141 $ est\footnote{Ce calcul est de toute façon relataviment faux car les permutations symétriques (résultats miroir) sont comptés - mais cela ne changera cependant pas l'ordre de grandeur du résultat - et des preuves expérimentales ont montré qu'il existe des séquences d'acides aminés interdites et/ou instables!} (voir plus bas page \pageref{simple arrangements with repetitions} pour la preuve):
	
	Les créationnistes et aussi de nombreux autres théistes soutiennent alors que ce chiffre est si énorme que même après des milliards d'années d'essais moléculaires aléatoires, aucune molécule de protéine alpha-globine humaine n'apparaîtrait jamais "au hasard", et donc l'hypothèse que l'alpha-globine humaine est née par un processus évolutif est réfuté de manière décisive.
	
	Une erreur majeure dans l'argument alpha-globine mentionné ci-dessus, commun à beaucoup d'autres de ce genre, est qu'il ignore le fait qu'une grande classe de molécules d'alpha-globine peut remplir la fonction essentielle de transfert d'oxygène, de sorte que le calcul de la probabilité d'une seule instance est trompeuse a priori. En effet, la plupart des $141$ acides aminés dans l'alpha-globine peuvent être modifiés sans altérer la fonction clé de transfert d'oxygène, comme on peut le constater en notant la grande variété de molécules d'alpha-globine à travers le règne animal. Quand on révise le calcul ci-dessus, sur la base de seulement $25$ emplacements essentiels à la fonction de transport d'oxygène (ce qui est une surestimation généreuse), on obtient $10^{33}$ chaînes fondamentalement différentes, un chiffre énorme mais incomparablement plus petit que $10^{183}$.
	
	Un calcul comme celui-ci peut être affiné davantage, en tenant compte d'autres caractéristiques de l'alpha-globine et de sa biochimie associée. Certains de ces calculs produisent des valeurs de probabilité encore plus extrêmes que les précédentes. Mais un de ces calculs est-il vraiment important? Le problème principal est que tous ces calculs, qu'ils soient effectués avec précision ou non, souffrent de l'erreur fatale de présumer qu'une structure telle que l'alpha-globine humaine est née d'un seul événement d'essai aléatoire tout-en-un. Mais générer une molécule "au hasard" en un seul coup n'est décidément pas l'hypothèse scientifique en question - c'est une théorie créationniste, pas une théorie scientifique. Au lieu de cela, les preuves disponibles de centaines d'études publiées sur le sujet ont montré au-delà de tout doute raisonnable que l'alpha-globine était le produit final d'une longue séquence d'étapes intermédiaires, dont chacune était biologiquement utile dans un contexte antérieur.
	
	En bref, l'argument des créationniste de la conception intelligente affirmant que les scientifiques affirment une création tout à la fois "aléatoire" de diverses biomolécules, puis affirmant que cela est probablement impossible, est une erreur classique de "l'homme de paille". Les scientifiques ne l'observent pas ainsi, donc cette argumentation est complètement invalide. En d'autres termes, peu importe à quel point les mathématiques utilisées dans l'analyse sont bonnes ou mauvaises, si le modèle sous-jacent est une description fondamentalement invalide du phénomène en question. Tout calcul de probabilité simpliste de l'évolution qui ne prend pas en compte le processus pas à pas par lequel la structure est apparue est presque certainement fallacieux et peut facilement induire en erreur!
	
	Certaines des difficultés des arguments de probabilité des créationnistes peuvent être illustrées en considérant les flocons de neige (même si nous devons garder à l'esprit que \textit{comparaison n'est pas raison}!). Le livre de Bentley et Humphrey \textit{Snow Crystals} comprend plus de 2'000 photos haute résolution en noir et blanc de vrais flocons de neige, chacune avec des motifs complexes mais très réguliers qui sont presque parfaitement symétriques à six axes (voir \cite{libbrecht2007formation} pour plus d'informations).
	
	En utilisant un calcul basé sur une symétrie à six axes, on peut calculer la probabilité qu'une de ces structures puisse se former "au hasard" comme à peu près une  chance sur $10^{2500} $. Ce chiffre de probabilité est encore plus extrême que certains qui sont apparus dans la littérature du créationnisme intelligent. Est-ce donc la preuve que chaque flocon de neige individuel a été conçu par une entité intelligente surnaturelle? On en doutera...
	
	L'illusion ici, encore une fois, suppose un assemblage aléatoire de molécules tout à la fois. Au lieu de cela, les flocons de neige, comme les organismes biologiques, sont formés comme le produit d'une longue série d'étapes agissant selon des lois physiques bien connues, et les résultats de ces processus dépendent très sensiblement des conditions de départ et de nombreux paramètres environnementaux. C'est donc une folie de présumer que l'on peut correctement calculer les probabilités d'un résultat donné au moyen de calculs de probabilité superficiels qui ignorent les processus par lesquels ils se sont formés.
	\begin{figure}[H]
		\centering
		\includegraphics[width=0.8\textwidth]{img/arithmetics/snowflakes.jpg}
	\end{figure} 
	Il est temps de mettre des chiffres sur les relations de la Mécanique Statistique pour montrer à quel point les mécanismes à l'intérieur de nos cellules sont impressionantes.

	Pour commencer, nous vivons à des températures d'environ $300$ [K]. Si vous avez étudié la Mécanique Statistique, vous savez que l'énergie cinétique d'une molécule est (\SeeChapter{voir section Mécanique des milieux continus page \pageref{virial theorem}}) donnée par:
	
	Dès lors:
	
	Ou en termes de masse molaire:
	
	Par conséquent, pour l'eau avec $18.01528\cdot 10^{-3}\;[\text{kg}\cdot\text{mol}^{-1}]$ cela donne à température ambiante:
	
	Ce n'est pas un trop mauvais résultat car de nombreuses sources donnent $590\;[\text{m}\cdot\text{s}^{-1}]$. Nous prendrons cette dernière valeur.
	
	La relation de vitesse thermique fonctionne même pour quelque chose d'aussi grand que l'ARN polymérase II ($\sim 830\cdot 10^{-24}$ [kg]). 
	
	Pour rendre les choses vraiment faciles, nous pouvons travailler avec un complexe moléculaire de masse encore plus grand (c'est-à-dire plus complexe). Quelque chose comme: $1\cdot 10^{-21}$ [kg] [kg]. Une telle masse aurait une vitesse moyenne thermale de:
	
	Prenons une borne inférieure de $2\;[\text{m}\cdot\text{s}^{-1}]$ selon certaines sources non revues par les pairs...
	
	Les cellules sont petites. Les 3 polymérases transcrites d'ADN en ARN ont des masses de l'ordre de $1\cdot 10^{-21}$ [kg]. Alors, combien de temps cela devrait-il leur prendre pour traverser un assez gros noyau de 10 microns ($10^{-5}$ [m]) de diamètre? Si elles vont à $2\;[\text{m}\cdot\text{s}^{-1}]$, elle le parcourra donc statistiquement en moyenne $250,000$ fois en une seconde ou une fois tous les $4\;[\mu\text{s}]$.

	Nous avons clairement omis quelque chose - rien dans la cellule ne se déplace en ligne droite. Il y a beaucoup de monde, de sorte que même si les choses bougent très rapidement, leur trajectoire n'est pas droite évidemment!
	
	Nous avons prouvé dans la section Mécanique des Milieux Continus (page \pageref{mean free path}) que la distance moyenne parcourue par une molécule (supposée sphérique) entre deux collisions est donnée par:
	
	Et nous avons prouvé au même endroit que le nombre de collisions par seconde était donné par:
	
	Pour calculer la quantité d'eau pouvant entrer dans un noyau cellulaire, nous devons savoir quelle est sa taille. Une source dit que l'eau peut être considérée comme une sphère écrasée d'un rayon maximum de  $1.41$ Angstroms ($10^{-10}$ [m]).

	Alors, quel est le volume d'une molécule d'eau? Cela est donné par:
	
	La densité des molécules est donc donnée par:
		
	Dés lors:
	
	et (toujours pour l'eau seulement et dans un mètre cube!):
	
	c'est comme rencontrer une autre molécule d'eau chaque (environ) $10^{-12}$ seconde ($1$ pico-seconde).
	
	Remarquez qu'un milliard d'années a:
	
	Nous avons donc par milliard d'années pour un mètre cube d'eau égale un nombre de collisions égal à:
	
	Et comme la quantité d'eau sur Terre est estimée approximativement à (en supposant que toute cette eau était impliquée de manière égale sur toute la planète aux mêmes processus et conditions chimiques ...):
	
	Par conséquent, sur toute la Terre, nous avons:
	
	Donc, par milliard d'années, nous avons $1,000$ fois plus de collisions que le nombre de $10^{33}$ de combinaisons de différentes chaînes de l'alpha-globine. Même si nous considérons un facteur $100$ d'erreur dans le nombre de collisions par seconde (il n'est pas impossible que ce soit la magnitude d'erreur dans nos estimations précédentes!), Nous n'aurions besoin que de $1$ milliard d'années pour essayer les 10 $1^{33}$ combinaison juste par des collisions aléatoires.
	
	\begin{tcolorbox}[colback=red!5,borderline={1mm}{2mm}{red!5},arc=0mm,boxrule=0pt]
	\bcbombe Gardez à l'esprit que c'est le résultat de ne considérer qu'une seule planète Terre comme la nôtre. Quel que soit le résultat, nous aurions dû le multiplier par l'approximation de la limite inférieure des planètes semblables à la Terre dans notre Galaxie (estimée à au moins $1$ milliard, soit $10^9$) et le nombre de galaxies dans l'Univers observable (estimé au moins $1'000$ milliards, soit $10^{12} $). Et ceci n'est que pour l'Univers \underline{observable} ... (en gardant à l'esprit que l'Univers est à notre connaissance très probablement plat et donc infini avec $\Omega_0=1$ ou $k=0 $ comme on le verra dans le section Cosmogonie page \pageref{critical density}).
	\end{tcolorbox}
	
	
	\begin{tcolorbox}[title=Remarque,colframe=black,arc=10pt]
	Jusqu'à présent, nous voyons que le principal facteur du nombre de collisions n'est pas le rayon des molécules impliquées, ni leur vitesse ou leur masse mais leur quantité (c'est-à-dire la densité). Évidemment, le temps est important mais il y a pour l'instant trop d'incertitudes sur la plage de valeurs de ce dernier.
	\end{tcolorbox}
	
	Mais soyons plus réalistes! Nous devons tenir compte du fait que:
	\begin{enumerate}
		\item L'eau n'est pas un acide aminé (le plus grand acide aminé a un poids de $204.2\;[\text{g}\cdot\text{mole}^{-1}]$) et il y a des collisions élastiques entre les molécules d'eau et ces molécules organiques.
		
		\item La densité des molécules d'eau n'est pas égale à celle de la densité des acides aminés par litre d'eau. Malheureusement, c'est une valeur à laquelle il semble que nous n'avons pas accès et qui est hautement spéculative\footnote{Elle est liée au problème des composés proportionnels de l'expérience Miller-Urey.}.
	\end{enumerate}
	Considérons la relation pour une collision élastique en une dimension (\SeeChapter{voir section Mécanique Classique page \pageref{elastic collision one dimensions}}):
	
	Et considérons le pire des cas où $v_{2i}=0$. Alors:
	
	Ainsi, la vitesse des acides aminés ne ferait que de diminuer notre nombre de collisions d'un facteur maximum de $10$. Pas assez pour faire de la «vie» quelque chose de rare à l'échelle d'un milliard d'années (pour la rendre rare, il faudrait au moins une diminution de l'ordre de de l'ordre d'un facteur $10,000$).
	
	Faisons maintenant un peu d'ingénierie inverse. Pour rendre la vie émergente rare à l'échelle du milliard d'années, comme déjà mentionné et grâce aux calculs précédents, nous devrions avoir au moins une diminution de d'un facteur $10'000$. Grâce à la collision élastique, nous pouvons ramener cette valeur à $1'000$.
	
	Nous devrions donc trouver quelque part, quelque chose dans nos calculs, pour rendre les collisions à $1'000$ fois plus rares. L'idée serait de prendre cela dans le paramètre $n$ (densité de molécules) dans les relations ci-dessus. Si nous supposons que le rapport des acides aminés aux molécules d'eau est de $1$ pour $1'000$, cela commence à rendre les choses difficiles à l'échelle du milliard d'années! Cela signifie que dans $1$ mole d'eau (soit 18 $ [g] $ ou $18$ [ml]), nous aurions $0.018$ [g] d'acides aminés.
	
	Donc, comme les planètes extra-solaires étaient quelque chose d'improbable il y a un siècle, il est très probable que l'avenir prouve que la vie est quelque chose de très commun sur des planètes qui ont de l'eau ou des atmosphères avec les bons composants et la bonne température sur une période de plusieurs milliards d'années...
	
	\begin{tcolorbox}[title=Remarque,colframe=black,arc=10pt]
	Certaines personnes illettrées scientifiquement affirment parfois que Roger Penrose a calculé que la probabilité pour que notre Univers (en supposant que ce dernier est dans Trou Noir ...) soit dans l'état d'entropie qu'il est maintenant est de $1$ chance sur $10^{123}$. Il serait donc alors impossible que notre Univers ne soit pas créé par une divinité (en passant sous silence la question de savoir qui a créé cette divinité complexe ...). Mais ce n'est pas pertinent! Même si ce qu'il a dit R. Penreose serait $100\%$ précis ou faux d'un facteur googol, ce n'est absolument pas pertinent!! En effet, si nous exécutons un programme informatique qui émule une pièce de monnaie lancée un milliard de fois et que nous écrivons la séquence des piles et des faces dans un fichier, alors ce que nous venons de produire est une séquence plus que mille milliards de milliards de fois plus rare que le chiffre obtenu par R. Penrose!! En fait, nous pourrions exécuter ce même programme informatique jusqu'à la fin de l'existence humaine et il ne produirait probablement jamais la même chaîne de résultats. Et pourtant, cela s'est produit quand même...
	\end{tcolorbox}
	Si le lecteur a le temps, il peut calculer la probabilité de sa propre existence. Ceci peut être calculé naïvement comme la probabilité qu'un spermatozoïde et un ovule se soient unis multipliée par la probabilité que sa mère et son père se soient rencontrés, que ses grands-parents se soient rencontrés, et ainsi de suite à travers les générations. On parie que le nombre obtenu sera petit... Le fait est que des événements à faible probabilité se produisent chaque jour dans notre Univers. Une fois qu'ils se produisent, leurs probabilités sont de $100\%$.
	
	\begin{tcolorbox}[title=Remarque,colframe=black,arc=10pt]
	Le lecteur notera également l'art de la cueillette des cerises chez les croyants et les créationnistes. Une divinité aurait selon eux créé quelque chose de "parfait" comme l'ADN (ou les arbres peu importe!). Cependant elle l'aurait fait avec près de $4$ à $6$ mille maladies génétiques (erreurs de codage), avec une source de lumière (Soleil) qui donne le cancer, sur une planète (Terre) où la majorité de l'eau est imbuvable près d'une étoile (Soleil) qui explosera et qui sur une orbite qui fera qui la rendra invivable dans quelques milliards d'années, avec un écosystème de ressources limitées avec des maladies mortelles endémiques, dans un univers majoritairement inhabitable ... Oui cela fait sens hummm...!
	\end{tcolorbox}
	
	\begin{figure}[H]
		\centering
		\includegraphics[width=0.8\textwidth]{img/arithmetics/nature_timespiral.jpg}
		\caption[Spirale de la Nature]{Spirale de la Nature (auteur: Pablo Carlos Budassi)}
	\end{figure} 
	\begin{fquote}[Richard Leakey]J'ai votre lettre et la meilleure chose que je puisse faire est de me référer à mes travaux publiés, à la fois scientifiques et populaires. Le mouvement créationniste est dirigé par une bande de dirigeants malhonnêtes et les citation erronées sont la marque de fabrique de leur travail. Y répondre est une perte de temps et une lettre ne serait pas suffisante pour apaiser vos questions. Il y a des choses qu'il vaut mieux ignorer et la stupidité de ces prétendus fanatiques religieux continue de m'étonner.
 	\end{fquote}

	\pagebreak
	\subsection{Axiomatique de Kolmogorov}\label{kolmogorov axioms}
	La probabilit\'e d'un \'ev\'enement sera en quelque sorte le r\'epondant de la notion de fr\'equence d'un ph\'enomène al\'eatoire, en d'autres termes, à chaque \'ev\'enement nous allons attacher un nombre r\'eel, appartenant à l'intervalle $[0,1]$, qui mesurera sa probabilit\'e (chance) de r\'ealisation. Les propri\'et\'es des fr\'equences que nous pouvons mettre en \'evidence lors d'\'epreuves diverses nous permettent de fixer les propri\'et\'es des probabilit\'es.
	
	Soit $U$ un univers. Nous disons que nous d\'efinissons une probabilit\'e sur les \'ev\'enements de $U$ si à tout \'ev\'enement $A$ de $U$ nous associons un nombre ou une mesure $P(A)$, appel\'ee "\NewTerm{probabilit\'e a priori de l'\'ev\'enement $A$}\index{probabilit\'e a priori}" ou "\NewTerm{probabilit\'e marginale de $A$}\index{probabilit\'e marginale}". Voici les "\NewTerm{axiomes de Kolmogorov}\index{axiomes de Kolmogorov}":
	\begin{enumerate}
		\item[A1.] Pour tout \'ev\'enement $A$:
		
		Ainsi, la probabilit\'e de tout \'ev\'enement est un nombre r\'eel compris entre $0$ et $1$ inclus (c'est du bon sens humain...).

		\item[A2.] La probabilit\'e de l'\'ev\'enement certain ou de l'ensemble (somme) des \'ev\'enements possibles est \'egale à $1$:
		

		\item[A3.] Si $A \cap B = \varnothing $ sont deux \'ev\'enements incompatibles (disjoints), alors:
		
		la probabilit\'e de la r\'eunion ("ou") de deux \'ev\'enements incompatibles (ou mutuellement exclusifs) est donc \'egale à la somme de leurs probabilit\'es (loi d'addition). Nous parlons alors de "\NewTerm{probabilit\'e disjointe}\index{probabilit\'e disjointe}\label{disjoint probability}".
	
		Nous comprenons mieux que le troisième axiome exige que $A \cap B = \varnothing$ sinon quoi la somme de toutes les probabilit\'es pourrait être sup\'erieur à l'unit\'e (imaginez à nouveau le diagramme sagittal des deux \'ev\'enement dans votre tête!).
	\end{enumerate}
	
	\begin{tcolorbox}[title=Remarques,colframe=black,arc=10pt]
	\textbf{R1.} Les probabilit\'es sont souvent communiqu\'ees sous forme de pourcentages. Faites donc très attention aux biais typiques des m\'edias de masse (et pas seulement!) lorsque que ces derniers communiquent des informations en pourcentages. Nous vous avons d\'ejà pr\'evenu, avec des exemples, de ces dangers à la page \pageref{pourcentage}.\\
	
	\textbf{R2.} Vous devez \'egalement faire très attention aux probabilit\'es de correspondance! L'affaire la plus connue s'appelle le "\NewTerm{sophisme du procureur}\index{sophisme du procureur}\label{prosecutor fallacy}". Le piège sous-jacent peut être compris en comparant un \'echantillon d'ADN d'une scène de crime à une base de donn\'ees de $20'000$ individus. Une correspondance est trouv\'ee, un individus est alors accus\'e et lors de son procès, il est affirm\'e que la probabilit\'e que deux profils ADN correspondent par hasard est seulement de $1$ sur $10'000$ (argument du procureur!). Cela ne signifie pas que la probabilit\'e que le suspect soit innocent est de $1$ sur $10'000$ (ce que les jur\'es peuvent mal comprendre ...)! \'etant donn\'e que $20'000$ individus ont \'et\'e test\'es, il y avait $20'000$ opportunit\'es de trouver une correspondance par hasard (pour voir le calcul d\'etaill\'e sur la manière d'obtenir la probabilit\'e d'obtenir au moins une correspondance sur la base de l'approche fr\'equentiste, allez voir la page \pageref{prosecutor fallacy frequentist example} ainsi que la page \pageref{prosecutor fallacy bayesian example} pour l'approche bay\'esienne).
	\end{tcolorbox}	
	Soit $ A $ un événement, $ P $ la mesure de probabilité. $ A $ a une probabilité nulle si $ P (A) = 0 $. $ A $ est impossible si $ A = \varnothing $.

	L'impossibilité implique une probabilité nulle, mais l'inverse est faux! Considérez la vraie ligne $\mathbb{R} $; si vous sélectionnez au hasard un nombre $ x $, la probabilité que $ x = 0 $ soit $ 0 $, mais ce n'est pas impossible. En fait, la probabilité que $ x $ appartient à un ensemble dénombrable, par exemple $\mathbb {Q} $, est également $ 0 $.

	Ce que nous voulons dire, c'est que $ P (A) = 0 $ n'implique pas $ A = \varnothing $, c'est-à-dire que la mesure de probabilité $ = 0 $ ne vous aide pas à déterminer si un ensemble est vide ou non!
	\begin{tcolorbox}[colframe=black,colback=white,sharp corners]
	\textbf{{\Large \ding{45}}Exemple:}\\\\
	Supposons qu'une machine à \'ecrire a $50$ touches, et que le mot à taper est \textit{banane}. Si les touches sont enfonc\'ees de manière al\'eatoire et ind\'ependante, cela signifie que chaque touche a la même chance d'être enfonc\'ee. Ensuite, la probabilit\'e que la première lettre tap\'ee soit \textit{b} est de $1/50$, et la probabilit\'e que la deuxième lettre tap\'ee soit \textit{a} est \'egalement de $1/50$, et ainsi de suite. Par cons\'equent, la probabilit\'e que les six premières lettres donne le mot \textit{banane} soit:
	
	moins d’un sur 15 milliards, mais pas z\'ero, d’où un r\'esultat possible!\\
	
	De ce qui pr\'ecède, le risque de ne pas taper le mot \textit{banane} dans un bloc donn\'e de $6$ lettres est de $1-(1/50)^6$. \'etant donn\'e que chaque bloc est saisi ind\'ependamment, la probabilit\'e $X_n$ de ne pas saisir le mot \textit{banane} dans l'un des premiers blocs $n$ de $6$ lettres est alors de:
	
	Lorsque $n$ augmente, $X_n$ devient plus petit. Pour un $n$ valant un million, $X_n$ vaut environ $0.9999$, mais pour $n$ valant dix milliards de dollars, $X_n$ correspond à environ $0.53 $ et pour $n$ valant cent milliards, il vaut environ $0.0017$. Lorsque $n$ approche l'infini, la probabilit\'e $X_n$ approche donc z\'ero; c'est-à-dire qu'en rendant $n$ assez grand, $X_n$ peut être aussi petit que souhait\'e, et la probabilit\'e de taper \textit{banane} est proche de $100\%$.
	\end{tcolorbox}
	Nous trouverons un exemple de ce type de probabilit\'e disjointe dans la section d'Ing\'enierie Industrielle lors de l'\'etude de l'AMDEC (Analyse des modes de d\'efaillance, de leurs effets et de leur criticit\'e) pour les systèmes d’analyse de pannes à structure complexe.

	Autrement dit sous forme plus g\'en\'erale si $\left( A_{i} \right)_{i \in \mathbb{N}}$ est une suite d'\'ev\'enements disjoints deux à deux ($A_{i}$ et $A_{j}$ ne peuvent pas se produire en même temps si equation) alors:
	
	Nous parlons alors de "\NewTerm{$\sigma$-additivit\'e}\index{$\sigma$-additivit\'e}" car si nous regardons de plus près les trois axiomes ci-dessus la mesure $P$ forme une $\sigma$-algèbre (\SeeChapter{voir section Th\'eorie de la Mesure page \pageref{sigma algebra}}).

	À l'oppos\'e, si les \'ev\'enements ne sont pas incompatibles (ils peuvent se superposer ou autrement dit: ils ont une probabilit\'e jointe), nous avons alors comme probabilit\'e qu'au plus un des deux ait lieu:
	
	Ceci signifie que la probabilit\'e pour que l'un au plus des \'ev\'enements $A$ ou $B$ se r\'ealise est \'egale à la somme des probabilit\'es pour que se r\'ealise A ou pour que se r\'ealise $B$, moins la probabilit\'e pour que $A$ et $B$ se r\'ealisent simultan\'ement (nous d\'emontrerons plus loin que cela est simplement \'equivalent à la probabilit\'e que les deux n'aient pas lieu en même temps!).
	
	Un cas typique d'utilisation de la dernière relation est l'actuariat. Effectivement nous connaissons les probabilit\'es de survie de deux individus pendant une p\'eriode de temps impos\'ee et parfois nous souhaiterions calculer qu'elle est la probabilit\'e qu'au moins un des deux survive survive pendant la p\'eriode donn\'ee. Dès lors nous utilisons la relation ci-dessus (voir section de Dynamique des Populations page \pageref{population dynamics}).

	\begin{tcolorbox}[colframe=black,colback=white,sharp corners]
	\textbf{{\Large \ding{45}}Exemples:}\\\\
	E1. Consid\'erons que la probabilit\'e dans une r\'egion donn\'ee d'avoir sur $50$ ans un tremblement de terre majeur est de $5\%$ et que d'avoir sur la même p\'eriode une inondation majeure est $10\%$ et que ces deux \'ev\'enements ne sont incompatibles... (c'est-à-dire que pendant les 50 ans, soit il y a le tremblement de terre soit l'inondation mais pas les deux). Nous souhaiterions savoir qu'elle est la probabilit\'e qu'une centrale nucl\'eaire rencontre tout au plus un des deux \'ev\'enements pendant cette même p\'eriode. Nous avons alors la probabilit\'e qui se calcule à partir de la relation pr\'ec\'edente et qui donne alors $14.5\%$...\\
	
	E2. Si nous lançons une poign\'ee de deux d\'es, la probabilit\'e d’obtenir un double est de:
	
	Pourquoi ? Parce que tous les r\'esultats possibles au lancement donnent $6^2=36$ et que le nombre de cas favorables est le double $1$, le double $2$, ... le double $6$ est de $6$ et donc $P =\frac{6}{6^2}=\frac{6}{36}=\frac{1}{6}$.
	\end{tcolorbox}
	Et donc s'ils \'etaient incompatibles nous aurions $A \cap B = \varnothing$ et nous retrouverions alors bien la probabilit\'e disjointe:
	
	\begin{tcolorbox}[title=Remarque,colframe=black,arc=10pt]
	 Indiquons que si la somme venait à faire plus de $100\%$ c'est que de par l'axiome des probabilit\'es les deux \'ev\'enements ne sont en r\'ealit\'e pas incompatibles!!! Ainsi, en reprenant l'exemple d'avant si nous avons 6$60\%$ de probabilit\'e pour le tremblement de terre et $70\%$ de probabilit\'e pour l'indondation alors cela veut dire qu'il y a $(60\%+70\%)-100\%=30\%$ de probabilit\'es que toutefois les deux aient lieux "en même temps" pendant la p\'eriode de $50$ ans (et il y a donc une faible probabilit\'e pour qu'ils aient lieu "exactement" au même moment).
	\end{tcolorbox}
	Une cons\'equence imm\'ediate des axiomes (A2) et (A3) est la relation entre les probabilit\'es d'un \'ev\'enement $A$ et son compl\'ementaire, not\'e $\bar{A}$ (ou plus rarement conform\'ement à la notation utilis\'ee dans le chapitre de Th\'eorie De La D\'emonstration le compl\'ementaire peut être not\'e $\neg A$):
	
	Soit $U$ un univers comportant un nombre fini $n$ d'issues possibles:
	
	
	où les \'ev\'enements:
	
	sont appel\'es "\NewTerm{\'ev\'enements \'el\'ementaires}\index{\'ev\'enements \'el\'ementaires}". Lorsque ces \'ev\'enements ont même probabilit\'e, nous disons qu'ils sont "\NewTerm{\'equiprobables}\index{\'equiprobables}". Dans ce cas, il est très facile de calculer leur probabilit\'e. En effet, ces \'ev\'enements \'etant par d\'efinition incompatibles entre eux à ce niveau de notre discours, nous avons en vertu de l'axiome 3 des probabilit\'es:
	
	mais puisque:
	
	et que les probabilit\'es du membre de droite sont par hypothèse \'equiprobables, nous avons:
	

	\textbf{D\'efinition (\#\mydef):}
	Si $A$ et $B$ ne sont pas incompatibles mais qu'ils sont ind\'ependants, nous savons que par leur compatibilit\'e $A \cap B=\varnothing$, alors (très important en statistiques!):
	
	la probabilit\'e de l'intersection (op\'erateur "et") de deux \'ev\'enements ind\'ependants est \'egale au produit de leurs probabilit\'es (loi de multiplication). Nous parlons alors de "\NewTerm{probabilit\'e conjointe}\index{probabilit\'e conjointe}\label{joint probability}" (c'est le cas le plus fr\'equent) ou simplement de "\NewTerm{probabilit\'e jointe}\index{probabilit\'e jointe}". Si les deux probabilit\'es sont d\'efinies par des lois de distributions, nours parlons alors bien \'evidemment de "\NewTerm{distribution conjointe}\index{distribution conjointe}\label{joint distribution}".

	\begin{tcolorbox}[colframe=black,colback=white,sharp corners]
	\textbf{{\Large \ding{45}}Exemple:}\\\\
	Consid\'erons que la probabilit\'e dans une r\'egion donn\'ee d'avoir sur $50$ ans un tremblement de terre majeur est de $5\%$ et que d'avoir sur la même p\'eriode une inondation majeure est $10\%$. De plus supposons que ces deux \'ev\'enements ne soient pas incompatibles (en d'autres termes ils sont compatibles). Nous allons nous int\'eresser à leur ind\'ependance. Ainsi, nous souhaiterions savoir qu'elle est la probabilit\'e qu'une centrale nucl\'eaire rencontre les deux \'ev\'enements en même temps, à quel que moment que ce soit, pendant cette même p\'eriode. Nous avons alors la probabilit\'e qui se calcule à partir de la relation pr\'ec\'edente et qui donne alors $0.05\%$...
	\end{tcolorbox}
	Autrement dit sous forme plus g\'en\'erale, les \'ev\'enements $A_1,A_2,...,A_n$ sont ind\'ependants si la probabilit\'e de l'intersection est le produit des probabilit\'es:
	

	\begin{tcolorbox}[title=Remarque,colframe=black,arc=10pt]
	Attention donc à ne pas confondre "ind\'ependants" et "incompatibles"!
	\end{tcolorbox}

	Donc pour r\'esumer jusqu'ici nous avons donc:

	\begin{table}[H]
	\begin{center}
		\definecolor{gris}{gray}{0.85}
			\begin{tabular}{|p{7.5cm}|p{7.5cm}|}
				\hline
				\multicolumn{1}{c}{\cellcolor{black!30}\textbf{Type}} & 
  \multicolumn{1}{c}{\cellcolor{black!30}\textbf{Expression}}\\ \hline
				2 \'ev\'enements incompatibles (disjoints) & $P(A \cup B)=P(A)+P(B)$ \\ \hline
				2 \'ev\'enements non-incompatibles (joints) & $P(A \cup B)=P(A)+P(B)-P(A \cap B)$ \\ \hline
				2 \'ev\'enements non-incompatibles mais ind\'ependents & $P(A \cap B)=P(A) \cdot P(B)$\\ \hline
		\end{tabular}
	\end{center}
	\caption{Cas classiques de probabilit\'es}
	\end{table}	
	 Grâce à la d\'efinition pr\'ec\'edente, nous pouvons d\'emontrer que la probabilit\'e pour que soit $A$ ou soit $B$ ait lieu (donc au moins un des deux mais pas les deux en même temps), est simplement \'egale à... la probabilit\'e que les deux n'aient pas lieu en même temps:
	
	Nous pouvons aussi à l'aide de cette dernière d\'efinition d\'eterminer la probabilit\'e qu'un seul des deux \'ev\'enements ait lieu:
	

	\begin{tcolorbox}[colframe=black,colback=white,sharp corners]
	\textbf{{\Large \ding{45}}Exemple:}\\\\
	Consid\'erons que la probabilit\'e dans une r\'egion donn\'ee d'avoir sur $50$ ans un tremblement de terre majeur est de $5\%$ et que d'avoir sur la même p\'eriode une inondation majeure est $10\%$. Nous souhaiterions savoir qu'elle est la probabilit\'e qu'une centrale nucl\'eaire rencontre exactement un de ces deux \'ev\'enements pendant la même p\'eriode en consid\'erant qu'ils ne peuvent avoir lieu en même temps. Nous avons alors la probabilit\'e qui se calcule à partir de la relation pr\'ec\'edente et qui donne alors $14\%$...
	\end{tcolorbox}
	Il y a deux domaines courants dans l'industrie dans lequel sont appliqu\'ees fr\'equemment les quatre relations suivantes (en anglais):
	
	Il s'agit de "\NewTerm{l'analyse par arbres d'erreurs}\index{l'analyse par arbres d'erreurs}" ou "\NewTerm{analyse par arbres probabilistes}\index{analyse par arbres probabilistes}" qui est utilis\'ee pour analyser les raisons possibles de d\'efaillance d'un système quel qu'il soit (industriel, administratif ou autre) et la mod\'elication de processus dite "\NewTerm{event driven process chain}" (EPC).
	
	\begin{tcolorbox}[title=Remarque,colframe=black,arc=10pt]
	Les Data scientists utilisent \'egalement certains des r\'esultats ci-dessus, en particulier lorsqu'ils travaillent avec des proportions venant de sondages ou de problèmes de classification (\SeeChapter{voir section de M\'ethodes Num\'eriques page \pageref{lift association rule}}).
	\end{tcolorbox}

	Pour clore cette partie du chapitre consid\'erons la figure suivante qui montre les diagrammes de Venn (\SeeChapter{voir section Th\'eorie des Ensembles page \pageref{Venn diagrams}})  pour les $16$ \'ev\'enements (y compris l'\'ev\'enement impossible) qui peuvent être d\'ecrits en termes de deux \'ev\'enements donn\'es $A$ et $B$. Dans chaque cas, l'\'ev\'enement est repr\'esent\'e par la zone rouge:

	\begin{figure}[H]
		\begin{center}
			\includegraphics{img/arithmetics/venn.eps}
		\end{center}	
		\caption{Diagrammes de Venn possibles pour deux \'ev\'enements}
	\end{figure}
	Consid\'erons la situation où $A$ repr\'esente un tremblement de terre et $B$ repr\'esente une inondation majeure et $U$ l'univers de tous les \'ev\'enements dramatiques pour une centrale nucl\'eaire. Nous consid\'erons que les deux \'ev\'enements sont ind\'ependants. Ensuite, chacune des $16$ combinaisons d'\'ev\'enements peuvent être d\'ecrites comme suit, soit math\'ematiquement ou verbalement:
\begin{enumerate}
	\item Un tremblement de terre peut se produire ou une inondation ou rien ou l'ensemble à la fois ou tout autre \'ev\'enement (bref n'importe quel \'ev\'enement peut se produire).
	
	
	\item $A \cup B$: Tout \'ev\'enement incluant un tremblement de terre, une inondation ou les deux en même temps peut se produire.
	
	
	\item $A \cup B^c$: Tout \'ev\'enement incluant un tremblement de terre avec ou sans une inondation peut se produire à l'exception des \'ev\'enements incluant une inondation sans tremblement de terre.
	
	
	\item $A^c \cup B$: Tout \'ev\'enement incluant une inondation avec ou sans tremblement de terre peut se produire à l'exception des \'ev\'enements incluant un tremblement de terre sans inondation.
	
	
	\item $A^c \cup B^c$: Tout \'ev\'enement peut se produire sauf ceux incluant un tremblement de terre accompagn\'e d'une inondation.
	
	
	\item $A$: Tout \'ev\'enement avec un tremblement de terre peut se produire (cela inclut donc les \'ev\'enements associant un tremblement de terre et une inondation).
	
	
	\item $B$: Tout \'ev\'enement avec une inondation peut se produire (cela inclut donc les \'ev\'enements associant une inondation et un tremblement de terre).
	
	
	\item $(A \cap B) \cup (A^c \cap B^c)$: Tout \'ev\'enement peut se produire sauf ceux incluant un tremblement de terre sans inondation ou ceux incluant une inondation sans tremblement de terre.
	
	
	\item $(A \cap B^c) \cup (A^c \cup B)$: Tout \'ev\'enement incluant un tremblement de terre sans inondation ou une inondation sans tremblement de terre peut avoir lieu.
	
	
	\item $B^c$: Tout \'ev\'enement except\'e ceux associ\'es à une inondation peuvent avoir lieu.
	
	
	\item $A^c$: Tout \'ev\'enement except\'e ceux associ\'es à un tremblement de terre peuvent avoir lieu.
		
		
	\item $A \cap B$: Tout \'ev\'enement associant un tremblement de terre et une inondation peut avoir lieu.
	
	
	\item $A \cap B^c$: Tout \'ev\'enement avec un tremblement de terre sans inondation peut avoir lieu.
	
	
	\item $A^c \cap B$: Tout \'ev\'enement avec une inondation sans tremblement de terre peut avoir lieu.
	
	
	\item $A^c \cap B^c$: Tout \'ev\'enement peut avoir lieu except\'e ceux incluant un tremblement de terre et/ou une inondation.
	
	\item $A \cap A^c$ or $B \cap B^c$:  \'ev\'enement impossible.
			
\end{enumerate}

\subsection{Probabilit\'es conditionnelles}\label{bayesian inference}

	Que pouvons-nous d\'eduire sur la probabilit\'e d'un \'evènement $B$ sachant qu'un \'evènement $A$ est r\'ealis\'e sachant qu'il existe une lien entre $A$ et $B$? En d'autres termes, s'il existe bien un lien entre $A$ et $B$, la r\'ealisation de $A$ va modifier notre connaissance sur $B$ et nous voulons savoir s'il est possible de d\'efinir la probabilit\'e d'un \'ev\'enement conditionnellement (relativement) à un autre \'ev\'enement.

	Ce type de probabilit\'e est appel\'ee  "\NewTerm{probabilit\'e conditionnelle}\index{probabilit\'e conditionnelle}" ou "\NewTerm{probabilit\'e à posteriori}\index{probabilit\'e à posteriori}" de $B$ sachant $A$, et se note dans le cadre de l'\'etude des probabilit\'es conditionnelles:
	
	et souvent dans la pratique pour \'eviter la confusion avec une possible division au niveau de 	la notation:
	
	et nous trouvons parfois chez les am\'ericains la notation (malheureuse...):
	
	ou encore:
	
	Nous avons aussi le cas:
	
	qui est appel\'e "\NewTerm{fonction de vraisemblance de $A$}\index{fonction de vraisemblance}" ou encore "\NewTerm{probabilit\'e a priori de $A$ sachant $B$}\index{probabilit\'e a priori}".

	Historiquement, le premier math\'ematicien à avoir utilis\'e correctement la notion de probabilit\'e conditionnelle semble être Thomas Bayes (1702-1761). Aussi parlons-nous souvent de "\NewTerm{Bayes}\index{probababilit\'es bay\'esiennes}" dès que des probabilit\'es conditionnelles sont en jeu: "\NewTerm{formule de Bayes}\index{formule de Bayes}", "\NewTerm{statistique bay\'esienne}\index{statistique bay\'esienne}", etc.

	La notion de probabilit\'e conditionnelle que nous allons introduire est beaucoup moins simple qu'elle ne paraît a priori et les problèmes de conditionnement sont une source in\'epuisable d'erreurs en tout genre (il existe de fameux paradoxes sur le sujet).

	Commençons d'abord par un exemple simpliste: Supposons que nous ayons deux d\'es. Imaginons maintenant que nous ayons lanc\'e seulement le premier d\'e. Nous voulons savoir quelle est la probabilit\'e qu'en lançant le second d\'e, la somme des deux chiffres vaille une certaine valeur minimale. Ainsi, la probabilit\'e d'obtenir cette valeur minimale fix\'ee sachant la valeur du premier d\'e est totalement diff\'erente de la probabilit\'e d'obtenir cette même valeur minimale en lançant les deux d\'es en même temps. Comment calculer cette nouvelle probabilit\'e?

	Formalisons la d\'emarche! Après le lancer du premier d\'e, nous avons:
	
	Soit l'hypothèse que $B \subset A$, nous pressentons que $P(B/A)$ doit être proportionnel à $P(B)$, la constante de proportionnalit\'e \'etant d\'etermin\'ee par la normalisation:
	
	Soit maintenant $B \subset A^c$ ($B$ est inclus dans le compl\'ementaire de $A$ donc les \'ev\'enements sont incompatibles). Il est relativement intuitif.... que sous hypothèse pr\'ec\'edente d'incompatibilit\'e nous ayons la probabilit\'e conditionnelle:
	
	Ceci nous mène aux d\'efinitions suivantes des probabilit\'es à posteriori et respectivement à priori:
	
	Ainsi, le fait de savoir que $A$ est r\'ealis\'e r\'eduit l'ensemble des r\'esultats possibles de $U$ de $B$. A partir de là, seules les \'eventualit\'es de ont une importance. La probabilit\'e de $A$ sachant $B$ inversement (par sym\'etrie) doit donc être proportionnelle à $P(A \cap B)$!

	Le coefficient de proportionnalit\'e qui est le d\'enominateur permet d'assurer l'\'ev\'enement certain. Effectivement, si les deux \'ev\'enements $A$ et $B$ sont ind\'ependants (pensez à l'histoire du chat noir et de l'\'echelle par exemple), nous avons donc:
	
	et nous voyons alors  $P(B/A)$ qui vaut $P(B)$ et donc l'\'ev\'enement $A$ n'apporte aucune information compl\'ementaire sur $B$ et r\'eciproquement!! Donc en d'autres termes, si $A$ et $B$ sont ind\'ependants nous avons:
	
	Une autre façon assez intuitive pour voir les choses est de se repr\'esenter la mesure de probabilit\'e $P$ comme une mesure d'aires de sous-ensembles de $\mathbb{R}^2$.

	Effectivement, si $A$ et $B$ sont deux sous-ensembles d'aires respectives $P(A)$ et $P(B)$ alors à la question de savoir qu'elle est la probabilit\'e qu'un point du plan appartienne à $B$ sachant qu'il appartient à $A$ il est assez \'evident de r\'epondre que cette probabilit\'e est donn\'ee par: 
	
	Indiquons aussi que la d\'efinition des probabilit\'es conditionnelles s'utilise souvent sous la forme suivante:
	
	appel\'ee "\NewTerm{formule des probabilit\'es compos\'ees}\index{formule des probabilit\'es compos\'ees}\label{compound probabilities}" or "\NewTerm{règle produit}\index{règle produit}". Cela est not\'e d'un certain nombre de manières diff\'erentes dans la pratique: 
	
	 Ainsi, la probabilit\'e à posteriori de $B$ sachant $A$ peut donc aussi s'\'ecrire sous la forme:
	
	La manière dont cette formule met à jour l'hypothèse de probabilit\'e, $B$, à la lumière d'un ensemble de donn\'ees, $A$, est appel\'ee "\NewTerm{interpr\'etation diachronique}\index{interpr\'etation diachronique}". "Diachronique" signifie que quelque chose se passe dans le temps, dans ce cas la probabilit\'e de l'hypothèse change, au fil du temps, à mesure que nous voyons de nouvelles donn\'ees.

	Dans cette interpr\'etation, les diff\'erents termes, ont diff\'erents noms:
	\begin{itemize}
		\item $P(B)$ est la probabilit\'e de l'hypothèse avant de voir les donn\'ees et nomm\'ee comme nous le savons d\'ejà la "\NewTerm{probabilit\'e a priori}\index{probabilit\'e a priori}\footnote{D'où vient cette probabilit\'e a priori est un peu la question à un million dollars ... En principe, le statisticien bay\'esien est cens\'e choisir un a priori qui repr\'esente son information de base. Ce sera pour le moins difficile dans les cas multi-dimensionnels. De plus, les critiques diront que les avis ant\'erieurs ne doivent pas être inclus dans une exp\'erience scientifique car justement ce n'est pas très scientifique...}".

		\item $P(B/A) $ est ce que nous voulons calculer, la probabilit\'e de l'hypothèse après avoir vu les donn\'ees, nomm\'ees comme nous le savons d\'ejà, la "\NewTerm{probabilit\'e a posteriori}\index{probabilit\'e a posteriori}.

		\item $P(A/B)$ est la probabilit\'e des donn\'ees sous l'hypothèse, appel\'ee la "\NewTerm{vraisemblance}\index{vraisemblance}".

		\item $P(A)$ est la probabilit\'e que des donn\'ees sous l'hypothèse et est nomm\'ee "\NewTerm{constante de normalisation}".
	\end{itemize}
	\begin{tcolorbox}[colframe=black,colback=white,sharp corners]
	\textbf{{\Large \ding{45}}Exemples:}\\\\
	E1. Supposons une maladie comme la m\'eningite. La probabilit\'e de l'avoir sera not\'ee $P(M)=0.001$ (chiffre arbitraire pour l'exemple) et un signe de cette maladie comme le mal de tête sera not\'e $P(S)=0.1$. Supposons connue la probabilit\'e à posteriori d'avoir mal à la tête si nous avons une m\'eningite:
	
	La formule des probabilit\'es compos\'ees donne alors la probabilit\'e a priori d'avoir une m\'eningite si nous avons mal à la tête!:
	
	E2. Dans un jeu t\'el\'evis\'e (cet exemple s'appelle le "paradoxe de Montey Hall"), un candidat doit s\'electionner une porte parmi trois. Derrière deux portes se trouve une chèvre (un prix ind\'esirable). Derrière une des portes il y a une voiture. Au d\'ebut, le candidat tente de deviner la porte avec la voiture. Ensuite, le mod\'erateur Monty Hall ouvre l'une des portes avec une chèvre qui n'est pas la porte choisie par le candidat. Il reste donc deux portes, l’une d’elles \'etant celle choisie par le candidat. Monty demande au candidat s'il souhaite changer d'avis. Il est libre de choisir l'autre porte.\\

	En g\'en\'eral, vaut-il mieux changer d'avis, s'en tenir à la première s\'election ou est-ce important?\\
	
	Hypothèse (un des trois cas): le candidat choisit la porte \#1.
	\begin{center}
	  \begin{tabular}{cc}
	    $D_1$ & Voiture derrière la porte \#1 \\
	    $D_2$ & Voiture derrière la porte \#2 \\
	    $D_3$ & Voiture derrière la porte \#3
	  \end{tabular}
	\end{center}
	Nous d\'efinissons les probabilit\'es correspondantes:
	
	\end{tcolorbox}
	
	\begin{tcolorbox}[colframe=black,colback=white,sharp corners]
	Nous devons d'une certaine manière mod\'eliser le comportement de l'animateur (Monty). Nous introduisons donc une hypothèse $H$ selon laquelle il ouvre la porte \#3. La probabilit\'e qu'il s\'electionne la porte \#3 en faveur de la porte \#2 est de  $\frac{1}{2}$, car \#1 est d\'ejà pris par le candidat:
	
	De plus:
	
	
	Donc, la probabilit\'e qu’il ouvre la porte \#3 sous la condition que la porte \#1  contienne la voiture est $\frac{1}{2}$. La probabilit\'e est de $1$ si la voiture est derrière la porte \#2. Et si la voiture est derrière la porte \#3, le mod\'erateur ne montrera jamais la porte \#3.\\
	
	Nous pouvons rep\'erer une certaine asym\'etrie ici. Ce point est \'evidemment responsable de la raison pour laquelle une intuition telle que \textit{cela n'a pas d'importance} peut être fausse!\\
	
	Nous allons donc utiliser le th\'eorème de Bayes:
	
	La question importante est la suivante: dans l’hypothèse où nous avons s\'electionn\'e la porte \#1 (notre première hypothèse) et que Monty ouvre la porte \#3 (notre hypothèse ou notre esp\'erance), quelle est la probabilit\'e que la voiture soit derrière la porte \#1 par rapport à porte \#2?
	
	Donc, apparemment, la probabilit\'e que la porte \#2 contienne la voiture est plus \'elev\'ee que pour la porte \#1. Ou en d'autres termes: dans ce cas, le candidat devrait pr\'ef\'erer changer d'avis!\\
	
	Nous devons maintenant \'evaluer cela pour d’autres hypothèses (le mod\'erateur ouvre la porte \#2) et d'autres s\'elections du candidat (porte \#2 et porte \#3). L'approche est la même et n'est pas r\'ealis\'ee ici. Mais la conclusion reste de toute façon la même dans tous les cas: être ouvert au changement d'esprit vous fera gagner la voiture de façon probabiliste!
	\end{tcolorbox}
	
	Pour en revenir à la th\'eorie, notons que nous avons aussi:
	
	Nous pouvons donc connaître la probabilit\'e de l'\'ev\'enement $A$ connaissant les probabilit\'es $P(B_i)$ \'el\'ementaires de ses causes et les probabilit\'es conditionnelles de $A$ pour chaque $B_i$:
	
	qui est appel\'ee la "\NewTerm{formule des probabilit\'es totales}\index{formule des probabilit\'es totales}" ou "\NewTerm{th\'eorème des probabilit\'es totales}\index{th\'eorème des probabilit\'es totales}" (relation qui est \'egalement très utile lorsque nous ne connaissons pas la valeur de $ P(A)$ ou ne pouvons pas l’estimer directement et facilement). Mais aussi, pour tout $j$, nous avons le corollaire suivant en utilisant les r\'esultats pr\'ec\'edents qui nous donne suite à un \'ev\'enement $A$, la probabilit\'e que ce soit la cause $B_i$ qui l'ai produit:
	
	qui est la forme g\'en\'erale de la "\NewTerm{formule de Bayes}\index{formule de Bayes}\label{bayes formula}" ou "\NewTerm{th\'eorème de Bayes}\index{th\'eorème de Bayes}" que nous utiliserons un tout petit peu en M\'ecanique Statistique et dans le cadre de l'\'etude de la th\'eorie des files d'attentes (\SeeChapter{voir section de Gestion Quantitative page \pageref{queueing theory}}). Il faut savoir que les implications de ce th\'eorème sont cependant consid\'erables dans le quotidien, dans la m\'edecine, dans l'industrie et dans le domaine du Data Mining informatique.

	Nous retrouvons souvent dans la litt\'erature de nombreux exemples d'applications de la relation pr\'ec\'edente avec uniquement deux issues possibles $B$ relativement à l'\'ev\'enement $A$. Dès lors nous avons la formule de Bayes \'ecrite sous la forme suivante pour chacune des issues:
	
	et remarquons que dans ce cas particulier (des issues binaires):
	
	ce qui est un r\'esultat intuitif.

	Pour les \'ev\'enements binaires, nous avons aussi (en revenant au th\'eorème des probabilit\'es totales vu plus haut):
	
	Ou dans le cas du "rapport de cotes":
	

	\begin{tcolorbox}[colframe=black,colback=white,sharp corners]
	\textbf{{\Large \ding{45}}Exemples:}\\\\
	E1. Une maladie affecte $10$ personnes sur $10'000$ (soit $0.1\% = 0.001$). Un test a \'et\'e d\'evelopp\'e qui a $5\%$ de faux positifs (personnes non atteint pour lequel le test dit qu'ils sont atteintes) mais qui d\'etecte toujours cette maladie si une personne est atteinte. Quelle est la probabilit\'e qu'une personne al\'eatoire pour laquelle le test donne un r\'esultat positif a vraiment cette maladie?\\
	
	Il y a donc sur $10'000$ personnes, $500$ qui seront des faux positifs et nous savons a posteriori que $10$ personnes ont r\'eellement la maladie. Alors la probabilit\'e que quelqu'un qui a un r\'esultat de test positif soit vraiment malade est:
	
	Ce r\'esultat est souvent contre-intuitif et même scandaleux. Il met aussi en \'evidence pourquoi les tests de diagnostiques doivent être extrêmement fiables!\\

	E2. Deux machines $M_1$ et $M_2$ produisent respectivement $100$ et $200$ pièces. $M_1$ produit $5\%$ de pièces d\'efectueuses et $M_2$ en produit $6\%$ (probabilit\'es a posteriori). Quelle est la probabilit\'e a priori pour qu'un objet d\'efectueux ait \'et\'e fabriqu\'e par la machine $M_1$?\\

	Nous avons alors:
	
	E3. D'un lot de $10$ pièces dont le $30\%$ est d\'efectueux, nous pr\'elevons sans remise un \'echantillon de taille $3$ sans remise. Quelle est la probabilit\'e que la seconde pièce soit bonne (quelle que soit la première)?\\

	Nous avons:
	
	\end{tcolorbox}
	\begin{tcolorbox}[colframe=black,colback=white,sharp corners]
	E4. Terminons avec un exemple important dans les entreprises où les employ\'es doivent plusieurs fois dans leur carrière passer des examens sous forme de questionnaire à choix multiples (Q.C.M.). Si un employ\'e r\'epond à une question de deux choses l'une: soit il connaît la r\'eponse, soit il la devine. Soit $p$ la probabilit\'e que l'employ\'e connaisse la r\'eponse et donc $1-p$ celle qu'il la devine. Nous admettons que l'employ\'e qui devine r\'epondra correctement avec une probabilit\'e $1/m$ où $m$ est le nombre de r\'eponses propos\'ees. Quelle est alors la probabilit\'e a priori qu'un employ\'e connaisse (r\'eellement) la r\'eponse à une question à $5$ choix s'il y a r\'epondu correctement?\\

	Soient $B$ et $A$ respectivement les \'ev\'enements "l'employ\'e connaît la r\'eponse" et "l'employ\'e r\'epond correctement à la question". Alors la probabilit\'e à priori qu'un employ\'e connaisse (r\'eellement) la r\'eponse à une question qu'il a r\'epondu correctement est:
	
	E5. Supposons qu'un m\'edecin pense que la cote qu'un patient soit atteint d'une certaine maladie est de $2/1$ ($2$ contre $1$). Il commande deux tests ind\'ependants. Soit $H$ l'\'ev\'enement où le patient a la maladie et $E_1$ et $E_2$ le cas où les tests sont positifs. Supposons que le premier test ait une probabilit\'e de $0.1$ d'un faux positif et une probabilit\'e de $0.05$ d'un faux n\'egatif. Le deuxième test a des probabilit\'es de faux positif et de faux n\'egatif de $0.05$ et $0.08$ respectivement. Si les deux tests sont positifs, quelle est la probabilit\'e post\'erieure que le patient soit atteint de la maladie?\\
	
	En supposant que $E_1$ et $E_2$ soient conditionnellement ind\'ependants de $H$ et de $\bar{H}$, nous travaillons d'abord en termes de cotes, puis convertissons en probabilit\'e:
	
	Les donn\'ees num\'eriques sont:
	
	On substitue les valeurs pour obtenir:
	
	De manière similaire, nous obtenons:
	
	\end{tcolorbox}
	
	\begin{tcolorbox}[colframe=black,colback=white,sharp corners]
	E6.\label{prosecutor fallacy bayesian example} Revenons encore sur le paradoxe du procureur dans le dossier de l'ADN (voir page pr\'ec\'edente ci-dessus \pageref{prosecutor fallacy}). Consid\'erons que le profil ADN d’un suspect donne lieu à un trait rare, $X$, tel que:
	
	qui \'equivaut à:
	
	Ceci n'a aucun int\'erêt! Nous voulons ($G$ est pour "potentiel coupable"):
	
	\end{tcolorbox}
	
	\begin{tcolorbox}[colback=red!5,borderline={1mm}{2mm}{red!5},arc=0mm,boxrule=0pt]
	\bcbombe Attention! Soyez prudent lors de la lecture de livres contenant des calculs de probabilités surtout lorsque les auteurs sont des croyants et encore plus lorsque les probabilités impliquent le théorème de Bayes. Le physicien croyant Stephen Unwin, tel que présenté dans son livre de 2003, calcule la probabilité que son Dieu existe est de $67\%$. Il commence son calcul en supposant que la probabilité a priori de l'existence de son Dieu, avant que toute évidence ne soit considérée, soit $50\%$ ... Comment le sait-il alors qu'il y a plusieurs milliers de dieux très différents recensés sur Terre? Comparons le calcul d'Unwin avec celui non publié du physicien non-croyant Larry Ford. Ford a utilisé toutes les mêmes équations que Unwin, ne différant que par les nombres insérés dans le calcul. Il a fait valoir que la probabilité antérieure du Dieu d'Unwin doit être comparée à celle d'autres entités non observées telles que Big Foot ou le monstre du Loch Ness, qu'il a pris pour être un sur un million ($10^{-6}$). Avec son ensemble de nombres, la conclusion de Ford est que la probabilité que le Dieu d'Unwin existe est de un sur cent mille milliards ($10^{-17}$).
	\end{tcolorbox}
	
	L'analyse bay\'esienne fournit \'egalement un outil puissant pour formaliser le raisonnement dans l'incertitude et les exemples que nous avons montr\'es ci-dessus illustrent à quel point cet outil peut être difficile à utiliser!
	
	Enfin, pour une utilisation future dans la section d'Informatique Th\'eorique, pour la technique de Data Mining appel\'ee "classification bay\'esienne naïve binomiale", nous aurons besoin d’un dernier r\'esultat!
	
	Rappelons-nous la règle de produit vue plus haut:
	
	Nous pouvons \'etendre cela à trois variables. En effet, consid\'erons:
	
	et posons $ D = B \cap C $ alors:
	
	Mais comme règle de produit nous donne aussi:
	
	Alors:
	
	aussi parfois not\'e:
	
	et en g\'en\'eral pour $n$ variables:
	
	En g\'en\'eral, on parle de "\NewTerm{règle en chaîne}\index{règle en chaîne}\label{chain rule}".
	
	Cette relation est particulièrement importante pour les r\'eseaux bay\'esiens. Elle fournit un moyen de calculer de manière complète distribution de probabilit\'e conjointe; Dans les r\'eseaux bay\'esiens, de nombreuses variables $A_i$ seront ind\'ependantes de manière conditionnelle, ce qui signifie que la formule peut être simplifi\'ee comme indiqu\'e ici.

\subsubsection{Esp\'erance et Variance conditionnelle}\label{conditional expectation}

	Maintenant, passons à la version continue de la probabilit\'e conditionnelle en abordant le sujet directement avec un exemple particulier (la th\'eorie avec le cas g\'en\'eral \'etant indigeste) infiniment important dans le domaine de statistiques sociales et de la finance quantitative. Cependant, ce choix (de l'\'etude d'un cas particulier) implique que le lecteur ait lu au pr\'ealable le chapitre de Statistiques pour y \'etudier les fonctions de distributions continues et plus particulièrement celle de la loi de Pareto.

	Donc voilà le sc\'enario: Souvent, en sciences sociales ou en \'economie, nous trouvons dans la litt\'erature sp\'ecialis\'ee traitant des lois de Pareto des affirmations du type suivant (mais quasiment jamais avec une d\'emonstration d\'etaill\'ee): quel que soit votre revenu, le revenu moyen de ceux qui ont un revenu sup\'erieur au vôtre est dans un rapport constant, sup\'erieur à 1, à votre revenu si celui-ci suit une variable al\'eatoire de type Pareto. Nous disons alors que la loi est isomorphe à toute partie tronqu\'ee elle-même.

	Voyons de quoi il s'agit exactement!
	
	Soit $X$ une variable al\'eatoire \'egale au revenu et suivant une loi de Pareto de densit\'e (\SeeChapter{voir section de Statistiques page \pageref{pareto distribution}}):
	
	avec $k>1,x_m>0,x \geq x_m$ et qui a pour fonction de r\'epartition (voir aussi la section de Statistique pour la d\'emonstration d\'etaill\'ee):
	
	La phrase commence par "\textit{quel que soit votre revenu...}", choisissons donc un revenu quelconque $x_0 \geq x_m$.
	
	À pr\'esent nous devons calculer "\textit{le revenu moyen de ceux qui ont un revenu sup\'erieur à } $x_0$". Il s'agit donc de calculer l'esp\'erance (le revenu moyen) d'une nouvelle variable al\'eatoire $Y$ qui est \'egale à $X$ mais restreinte à la population des personnes ayant un revenu sup\'erieur à $x_0$:
	
	La fonction de r\'epartition de $Y$ est donn\'ee par:
	
	Cette expression est naturellement nulle si . $x \leq x_0$.
	
	Bon, jusqu'à maintenant nous n'avons fait que du vocabulaire. D'abord rappelons la relation de probabilit\'e conditionnelle suivante vue plus haut:
	
	pour $x \geq x_0 $ nous avons la loi conditionnelle à priori:
	
	Avant d'aller plus loin, il faut être conscient que le num\'erateur et d\'enominateur sont ind\'ependants mais que l'ensemble doit être toutefois consid\'er\'e comme la r\'ealisation d'une seule et unique variable al\'eatoire que nous noterons $Y$. Par ailleurs, seulement le num\'erateur est d\'ependant d'une variable. Le d\'enominateur peut lui être consid\'er\'e comme une constante de normalisation.

	Nous voyons donc que la densit\'e de $Y$ est donn\'ee par la fonction:
	
	À pr\'esent nous pouvons calculer l'esp\'erance de $Y$ (assumant $X\geq x_0$):
	
	Sachant que (\SeeChapter{voir section Statistiques page \pageref{pareto tail distribution}}):
	
	Nous avons finalement
	
	$\text{E}(Y)$ repr\'esente donc le revenu moyen de ceux qui ont un revenu sup\'erieur à $x_0$ et comme on peut le constater de l'\'egalit\'e ci-dessus il est bien dans un rapport constant, sup\'erieur à $1$, à votre revenu $x_0$.

	Nous pouvons v\'erifier ce r\'esultat en faisant une simulation de Monte-Carlo dans un tableur (c'est int\'eressant de le mentionner pour g\'en\'eraliser à des cas non calculable à la main). Il suffit effectivement d'y simuler l'inverse de la fonction de r\'epartition:
	
	soit dans MS Excel 11.8346 (version anglaise):

	\begin{center}
 	\texttt{=(\$B\$7\^{}\$B\$6}/(1-ALEA.ENTRE.BORNES(1,10000)/10000))\^{}(1/\$B\$6)
	\end{center}
	
	et ensuite de prendre la moyenne des valeurs obtenues sup\'erieurs ou \'egales à un $X$ donn\'e (ce qui correspondra à $x_0$) et v\'erifier que nous obtenons bien le r\'esultat d\'emontr\'e pr\'ec\'edemment!

	\'evidemment, nous pourrions aussi calculer la variance conditionnelle (in extenso l'\'ecart-type conditionnel). Cela viendra peut-être un jour...

	Par cons\'equent, g\'en\'eralement pour une variable discrète, "\NewTerm{l'esp\'erance conditionnelle}\index{esp\'erance conditionnelle}" est g\'en\'eralement d\'efini par:
	
	Mais ce n'est que le cas où nous prenons et l'\'egalit\'e avec la variable ant\'erieure. En g\'en\'eral, dans toute relation d'ordre, une meilleure d\'efinition serait:
	
	\'evidemment, nous pourrions aussi calculer la variance conditionnelle (ie l'\'ecart-type conditionnel) comme nous le verrons un peu plus bas!
	
	\begin{theorem}
	Voyons maintenant prouver une propri\'et\'e qui nous sera utile en particulier lors de notre \'etude des s\'eries chronologiques financières. Il est bas\'e sur le fait que parfois nous devons calculer des relations telles que la suivante:
	
	qui sont appel\'ees "\NewTerm{moyenne conditionnelle it\'er\'ee}\index{moyenne conditionnelle it\'er\'ee}\label{iterated conditional mean}" ou "\NewTerm{loi de l'esp\'erance totale}\index{loi de l'esp\'erance totale}" aussi parfois not\'e $\text{E}\left(\text{E}(X|Y=y)\right)$ ou $\text{E}\left(\text{E}(X|\pi)\right)$.
	\end{theorem}
	\begin{dem}
	
	\begin{flushright}
		$\blacksquare$  Q.E.D.
	\end{flushright}
	\end{dem}
	Faisons de même pour la variance! Rappelons d’abord la d\'efinition de la variance (dite aussi de la "variance inconditionnelle"). Si $Y$ est une variable al\'eatoire, et qu'on note:
	
	Alors, la variance inconditionnelle de $Y$ est bien \'evidemment d\'efinie comme suit:
	
	Consid\'erons maintenant deux variables al\'eatoires $X, Y$ et nous voulons calculer la variance conditionnelle de $Y$ donn\'ee  $X$ que nous noterons $\text{V}(Y|X)$. De même que la variance inconditionnelle, nous pouvons mettre:
	
	La variance conditionnelle sera d\'efinie comme suit:
	
	Gardez à l'esprit que la variance inconditionnelle est un scalaire, alors que la variance conditionnelle reste elle-même une variable al\'eatoire! Donc, pour obtenir un scalaire pour la variance conditionnelle, nous devons estimer cette dernière pour une valeur donn\'ee de $x$, ce que nous noterons par $\text{V}(Y|X=x)$.
	
	Dès lors:
	
	Comme les termes $\text{E}(Y|X=x)$ sont des scalaires, nous pouvons les factoriser:
	
	La valeur conditionnelle attendue d'une constante est toujours \'egale à cette constante, c'est-à-dire: $\text{E}(1|X=x)=1$. Donc:
	
	Dès lors:
	
	Nous reconnaissons dans la dernière \'egalit\'e ci-dessus, la relation de Huygens de la moyenne inconditionnelle. Par cons\'equent, la variance conditionnelle est d\'efinie comme suit:
	
	Ceci sera \'ecrit explicitement pour des variables continues:
	
	\begin{theorem}
	D\'emontrons maintenant la loi de la "\NewTerm{variance conditionelle it\'er\'ee}\index{variance conditionelle it\'er\'ee}\label{iterated condititional variance}", \'egalement appel\'ee "\NewTerm{loi de la variance totale}\index{loi de la variance totale}":
	
	\end{theorem}
	\begin{dem}
	Nous commençons par la relation d'Huygens:
	
	La loi de la moyenne conditionnelle it\'er\'ee nous donne:
	
	En l'injectant dans la relation pr\'ec\'edente, nous obtenons:
	
	En utilisant le r\'esultat obtenu plus haut:
	
	et en injectant cela dans la relation pr\'ec\'edente, nous obtenons:
	
	Maintenant, nous utilisons à nouveau la loi des moyennes conditionnelles it\'er\'ees que nous mettons au carr\'e:
	
	Et on l'injecte dans la relation pr\'ec\'edente pour obtenir:
	
	Nous reconnaissons dans cette dernière \'egalit\'e la relation de Huygens. Effectivement:
	
	Donc finalement:
	
	\begin{flushright}
		$\blacksquare$  Q.E.D.
	\end{flushright}
	\end{dem}

	\pagebreak
	\subsubsection{R\'eseaux bay\'esiens}
	Les "\NewTerm{R\'eseaux bay\'esiens}\index{R\'eseaux bay\'esiens}", aussi nomm\'es "\NewTerm{r\'eseaux bay\'esiens naifs}\index{r\'eseaux bay\'esiens naifs}" abr\'eg\'es RBN, sont simplement une repr\'esentation graphique d'un problème de probabilit\'es conditionnelles qui permet de mieux visualiser l'interaction entre les diff\'erentes variables lorsque que celles-ci commencent à être en grande nombre.

	C'est une technique de plus en plus utilis\'ee dans le d\'ecisionnel assist\'e par logiciel (Data Mining), l'intelligence artificielle (AI) et \'egalement dans l'analyse et la gestion du risque (norme ISO 31010).

	Les r\'eseaux bay\'esiens sont par d\'efinition des graphes orient\'es acycliques (\SeeChapter{voir section Th\'eorie des Graphes page \pageref{acyclic graph}}), afin qu'un \'ev\'enement ne puisse pas (même indirectement) influencer sa propre probabilit\'e, avec description quantitative des d\'ependances entre \'ev\'enements.
	
	Ces graphes servent à la fois de modèles de repr\'esentation des connaissances et de machines à calculer des probabilit\'es conditionnelles. Ils sont surtout utilis\'es pour le diagnostic (m\'edical et industriel), l'analyse de risques (diagnostics de pannes, anomalies ou accidents), la d\'etection des spams (filtre bay\'esien), l'analyse de texte de voix et d'images, l'analyse d'opinions, la d\'etection de fraudeurs ou de mauvais payeurs ainsi que dans le data mining (EGC: Extraction et Gestion de la Connaissance) en g\'en\'eral.

	\begin{tcolorbox}[title=Remarque,colframe=black,arc=10pt]
	De nombreux systèmes et logiciels permettent de construire et d'analyser des r\'eseaux bay\'esiens sur la base de dessins ou de d'informations existantes dans des bases de donn\'ees. Solutions payantes: SQL Server, Oracle, Hugin. Solutions gratuits (à ce jour): Bayesia, Tanagra, Microsoft Belief Network MSBNX 1.4.2, RapidMiner. Personnellement je pr\'efère la simplicit\'e du petit logiciel MSBNX de Microsoft. Pour information, en 10 ans d'exp\'erience professionnelle en tant que consultant je n'ai rencontr\'e à ce jour qu'une seule entreprise parmi plus de 800 multinationales dans mon portefeuille qui utilisait les r\'eseaux de bay\'esiens... (dans le domaine des transports).
	\end{tcolorbox}
	
	Utiliser un r\'eseau bay\'esien s'appelle faire de "\NewTerm{l'inf\'erence bay\'esienne}\index{inf\'erence bay\'esienne}" ". En fonction des informations observ\'ees, nous calculons la probabilit\'e des donn\'ees possibles connues mais non observ\'ees.

	Pour un domaine donn\'e (par exemple m\'edical), nous d\'ecrivons les relations causales entre variables d'int\'erêt par un graphe (plus besoin de pr\'eciser qu'il est acyclique). Dans ce graphe, les relations de cause à effet entre les variables ne sont pas d\'eterministes, mais probabilis\'ees. Ainsi, l'observation d'une cause ou de plusieurs causes n'entraîne pas syst\'ematiquement l'effet ou les effets qui en d\'ependent, mais modifie seulement la probabilit\'e de les observer.

	L'int\'erêt particulier des r\'eseaux bay\'esiens est de tenir compte simultan\'ement de connaissances a priori d'experts (dans le graphe) et de l'exp\'erience contenue dans les donn\'ees.

	Exemple de $5$ variables avec relations (graphe orient\'e acyclique) et num\'erotation des \'etats/variables (en anglais: "states"):
	\begin{figure}[H]
		\centering
		\includegraphics{img/arithmetics/bayes_network_example.jpg}
		\caption{Exemple de r\'eseau bay\'esien (acyclique orient\'e) à 5 \'etats}
	\end{figure}
	
	\'evidemment, la construction du graphe causal se fonde principalement sur le retour d'exp\'eriences (REX) et r\'esulte parfois de normes ou de rapports de comit\'es d'experts. Dans l'informatique, le graphe causal \'evolue automatiquement en fonction des bases de donn\'ees (pensez à la librairie Amazon qui cible les publicit\'es en fonction de vos achats pass\'es en temps r\'eel ou au service Genius de Apple). Cependant nous pourrons rarement penser à toutes les possibilit\'es et il y aura aussi parfois des \'etats cach\'es entre deux \'etats qui auront \'et\'e oubli\'es mais qui auraient permis de mieux mod\'eliser la situation.
	
	Avant de procéder à un exemple détaillé, rappelons d'abord que:
	
	et maintenant imaginons dans la figure ci-dessus qu'à l'aide d'une base de donn\'ees d'une entreprise, nous sachions que sur $100'000$ jours hommes, nous avons eu dans cette entreprise $1'000$ accidents du travail (soit $1\%$ du total) et $100$ pannes machines (soit $0.01\%$ du total). Nous repr\'esentons cela alors sous la forme traditionnelle suivante:
	\begin{figure}[H]
		\centering
		\includegraphics{img/arithmetics/bayes_network_departure.jpg}
		\caption{R\'eseau bay\'esien acyclique orient\'e avec probabilit\'es de d\'epart}
	\end{figure}
	où nous avons le sous-ensemble $S2, S4, S5$ qui constitue ce que les sp\'ecialistes appellent une "\NewTerm{connexion s\'erie ou lin\'eaire}\index{connexion s\'erie ou lin\'eaire}", le triplet $S3, S2, S4$ constitue une "\NewTerm{relation divergente}\index{relation divergente}" (si les flèches pour ce triplet \'etaient invers\'ees, nous aurions une "\NewTerm{relation convergente}\index{relation convergente}").
	
	Avant d'aller plus loin avec notre exemple faisons quelques constats par rapport à ces trois types de relations:
	
	Pour toute clart\'e, distinguons d'abord "\NewTerm{l'ind\'ependance conditionnelle}\index{l'ind\'ependance conditionnelle}" de la "\NewTerm{d\'ependance conditionnelle}\index{d\'ependance conditionnelle}".
	
	Nous disons que des \'ev\'enements $A$ et $C$ sont  "\NewTerm{ind\'ependants conditionnellement}\index{ind\'ependants conditionnellement}" si \'etant donn\'e un \'ev\'enement $B$  l'\'egalit\'e suivante est v\'erifi\'ee:
	
	Donc le qualificatif "conditionnellement" implique la pr\'esence de $B$ et le fait que $C$ n'influence pas la probabilit\'e de l'\'ev\'enement $A$.
	
	Concernant la  "\NewTerm{d\'ependance conditionnelle}", nous pouvons cette fois distinguer trois types de relations.
	\begin{enumerate}
		\item La d\'ependance conditionnelle du type suivant est appel\'ee "\NewTerm{connexion s\'erie ou lin\'eaire}\index{connexion s\'erie ou lin\'eaire}" (d\'ejà mentionn\'ee plus haut):
	\begin{figure}[H]
		\centering
		\includegraphics[scale=0.8]{img/arithmetics/linearconnection.eps}
		\caption{R\'eseau bay\'esien en connexion s\'erie/lin\'eaire}
	\end{figure}
	où $A$, $B$ et $C$ sont d\'ependants (dans cet exemple particulier il y a $3$ noeuds d\'ependants $A$, $B$ et $C$ mais d'une manière g\'en\'erale cette d\'ependance concernerait tous les noeuds s'il y en avait plus de $3$).

	En outre $A$ et $C$ sont d\'ependants mais conditionnellement à $B$. Mais si la variable $B$ est connue, $A$ n'apporte plus aucune information utile sur $C$ (le cheminement de l'incertitude est en quelque sorte rompu) et dès lors $A$ et $C$ deviennent ind\'ependants conditionnellement. Nous avons la probabilit\'e conditionnelle qui se simplifierait donc sous la forme suivante:
	
	
	\item La d\'ependance conditionnelle du type suivant est appel\'ee "\NewTerm{connexion divergente}\index{connexion divergente}" (aussi d\'ejà mentionn\'ee plus haut):
	\begin{figure}[H]
		\centering
		\includegraphics{img/arithmetics/divergentlink.eps}
		\caption{R\'eseau bay\'esien divergent}
	\end{figure}
	où l'ensemble des noeuds sont d\'ependants.
	
	En outre $B$ et $C$ sont d\'ependants conditionnellement à $A$. Mais si $A$ est connue, $B$ n'apporte plus aucune information sur $C$ (à nouveau le cheminement de l'incertitude est en quelque sorte rompu) et dès lors $B$ et $C$ deviennent ind\'ependants. Nous avons donc par exemple si $A$ est connue:
	
	
	\item La d\'ependance conditionnelle du type suivant est appel\'ee "\NewTerm{connexion convergente}\index{connexion convergente}" ou "\NewTerm{$V$-Structure}\index{$V$-structure}" (aussi d\'ejà mentionn\'ee plus haut):
	\begin{figure}[H]
		\centering
		\includegraphics{img/arithmetics/vstructure.eps}
		\caption{Réseau bayésien convergent}
	\end{figure}
	où cette fois les parents sont ind\'ependants.
	
	Donc $B$ et $C$ sont ind\'ependants mais deviennent d\'ependants conditionnellement à $A$. Si $A$ est connue, nous avons alors:
	
	La d\'ependance entre les parents passe donc par l'observation de leur enfant commun.
	\end{enumerate}
	
	Maintenant, pour faire un exemple concret, imaginons que notre base de donn\'ees nous donne (grâce aux responsables qualit\'e qui ont toujours su saisir les anomalies qualit\'e) que lorsqu'une panne machine a eu lieu, $99$ fois sur $100$ ($99\%$) il y a eu un arrêt total de la production (donc in extenso $1$ fois sur $100$: $1\%$ il n'y pas eu d'arrêt de la production) et que sur tous les arrêts de la production $1\%$ n'\'etait pas dû à une panne machine. Ce que nous repr\'esentons traditionnellement sous la forme suivante:
	\begin{figure}[H]
		\centering
		\includegraphics[width=1.0\textwidth]{img/arithmetics/first_level_bayesian_network.jpg}
		\caption{R\'eseau bay\'esien de 1er niveau}
	\end{figure}
	Donc la "\NewTerm{probabilit\'e implicite}\index{probabilit\'e implicite}" (constante de normalisation) qu'il y ait un arrêt de la production est donn\'ee par:
	
	Ce chiffre repr\'esente donc la proportion implicite d'arrêts de production parmi les $100'000$ jours hommes (nous pouvons donc donner une proportion de lignes de la base donn\'ees repr\'esentant un arrêt production quelle que soit la cause et ce sans même avoir les d\'etails de la base de donn\'ees).
		
	Il en d\'ecoule imm\'ediatement alors la probabilit\'e implicite (constante de normalisation) qu'il n'y ait pas d'arrêt de la production:
	
	Ce qui est conforme à ce que nous donne le logiciel gratuit MSBNX 1.4.2:
	\begin{figure}[H]
		\centering
		\includegraphics[scale=0.75]{img/arithmetics/msbnx_beginning.eps}
		\caption{D\'ebut du r\'eseau bay\'esien dans MSBNX 1.4.2}
	\end{figure}
	Maintenant supposons que nous avons observ\'e un arrêt de la production. Quelle est la probabilit\'e à posteriori qu'il soit dû à une panne machine? Nous avons alors:
	
	Ce que nous pouvons aussi v\'erifier avec le logiciel MSBNX 1.4.2:
	\begin{figure}[H]
		\centering
		\includegraphics[scale=0.75]{img/arithmetics/msbnx_machine_failure_aposteriori_probability.eps}
		\caption{Probabilit\'e a posteriori d'un arrêt dû à une panne machine dans MSBNX 1.4.2}
	\end{figure}
	Maintenant, imaginons que notre base de donn\'ees nous donne (toujours grâce aux responsables qualit\'e qui ont veill\'e à saisir les anomalies qualit\'e) que $99$ fois sur $100$ ($99\%$) lorsqu'il y a eu un arrêt de la production, il y a eu une \'evacuation. En revanche $5\%$ des \'evacuations ont \'et\'e identifi\'ees comme n'ayant rien à voir avec un arrêt de la production (donc $95\%$ des \'evacuations sont dues à des exercices d'incendie OU à d'autres \'ev\'enements):
	\begin{figure}[H]
		\centering
		\includegraphics[width=1.0\textwidth]{img/arithmetics/second_level_bayesian_network.jpg}
		\caption{R\'eseau bay\'esien de deuxième niveau}
	\end{figure}
	Maintenant, pour calculer la probabilit\'e implicite (constante de normalisation) des \'evacuations a posteriori par rapport aux pannes machines, nous avons vu que lorsque nous avions une d\'ependance conditionnelle s\'erie, la probabilit\'e conditionnelle ne d\'ependait que du parent direct. Ainsi, il vient:
	
	Ce qui peut se v\'erifier avec le logiciel MSBNX 1.4.2:
	\begin{figure}[H]
	\centering
	\includegraphics[scale=0.75]{img/arithmetics/msbnx_implicit_probability_evacuation.eps}
	\caption{Probabilit\'e implicite d'une \'evacuation dans MSBNX 1.4.2}
	\end{figure}
	Donc la probabilit\'e implicite (constante de normalisation) de l'\'evacuation ne d\'epend effectivement pas des pannes de machines.

	Maintenant supposons que nous avons observ\'e une \'evacuation. Nous voulons savoir quelle est la probabilit\'e a posteriori qu'elle soit due à une panne machine! Nous avons alors:
	
	Ce que nous pouvons aussi v\'erifier avec le logiciel MSBNX 1.4.2:
	\begin{figure}[H]
		\centering
		\includegraphics[scale=0.75]{img/arithmetics/msbnx_a_posteriori_probability_evacuation.eps}
		\caption{Probabilit\'e a posteriori d'une \'evacuation due à une panne machine dans MSBNX 1.4.2}
	\end{figure}
	
	Maintenant nous \'etudions le cas avec l'alarme et là aussi une base de donn\'ees nous permet de construire un tableau avec les diff\'erentes probabilit\'es:

	\begin{figure}[H]
		\centering
		\includegraphics[width=1.0\textwidth]{img/arithmetics/second_level_bayesian_network_second_branch.jpg}
		\caption{R\'eseau bay\'esien de deuxième niveau avec seconde branche}
	\end{figure}

	Maintenant, pour calculer la probabilit\'e implicite (constante de normalisation) qu'il y ait une alarme, il va falloir consid\'erer les quatre situations possibles. Nous avons alors en utilisant encore une fois le th\'eorème des probabilit\'es totales:
	
	Ce qui un peu plus rigoureusement devrait s'\'ecrire:
	
	L'application num\'erique donne donc pour la probabilit\'e implicite (constante de normalisation) d'une alarme:
	
	Ce qui se construit et se v\'erifie de la manière suivante avec MSBNX 1.4.2:
	\begin{figure}[H]
		\centering
		\includegraphics[scale=0.75]{img/arithmetics/msbnx_implicit_probability.eps}
		\caption{Probabilit\'e implicite d'une alarme dans MSBNX 1.4.2}
	\end{figure}

	Concernant les notations, il peut être utile au lecteur de savoir qu'il peut parfois trouver dans la litt\'erature:
	
	ou encore:
	
	
	\begin{tcolorbox}[title=Remarque,colframe=black,arc=10pt]
	Dans l'exemple particulier \'etudi\'e ici les \'ev\'enements ont tous deux \'etats. Mais dans la pratique cela peut aller à 3, 4 et plus. Dès lors les tableaux de croisement de probabilit\'es deviennent vite \'enormes.
	\end{tcolorbox}
	
	Comme pour les cas pr\'ec\'edents, supposons que nous savons qu'il y a eu un accident de travail. Nous souhaitons alors calculer la probabilit\'e implicite d'une alarme. Nous avons alors (observez que la probabilit\'e ne d\'epend effectivement alors plus que de l'\'etat $S2$ puisque l'\'etat $S1$ est entièrement connu!):
	
	Ce que nous pouvons aussi v\'erifier avec le logiciel MSBNX 1.4.2:
	\begin{figure}[H]
		\centering
		\includegraphics[scale=0.75]{img/arithmetics/msbnx_implicit_probability_alarm.eps}
		\caption{ Probabilit\'e implicite d'une alarme dans MSBNX 1.4.2}
	\end{figure}
	Pour terminer cet exemple, nous souhaiterions calculer les probabilit\'es a posteriori $P(S2/S3)$ et $P(S1/S3)$. Pour cela, nous devons d'abord calculer les probabilit\'es implicites $P(S3/S2)$ et $P(S3/S1)$ (cette dernière venant d'être calcul\'ee).

	Nous avons pour la valeur manquante (ce qui se v\'erifie aussi facilement qu'avant avec le logiciel MSNBX 1.4.2):
	
	Nous avons donc pour résumer:
	
	Nous avons maintenant tout ce qu'il faut pour calculer les probabilités a posteriori de $P(S2/S3)$ et $P(S1/S3)$:
	
	Donc la probabilit\'e a posteriori qu'il y ait une panne machine lorsque nous savons qu'il y a une alarme est de $0.979\%$ (donc in extenso $0.021\%$ que le d\'eclenchement de l'alarme ne soit pas dû a priori à une panne machine). Respectivement il y a, a posteriori, $0.998\%$ de probabilit\'e y ait un accident de travail lorsque nous savons qu'il y a une alarme (et donc $0.002\%$ que cela ne soit pas dû a posteriori à un accident de travail).

	Du point de vue critique, lorsqu'il y a donc une alarme finalement nous ne pouvons pas dire grand chose... Cela est dû, dans le cas pr\'esent, au fait que les \'ev\'enements d'int\'erêt notable ont tous deux de faibles probabilit\'es d'avoir lieu (accident de travail et panne machine) et que les employ\'es r\'eagissent plutôt bien au niveau du d\'eclenchement de l'alarme (sinon si les probabilit\'es a priori \'etaient grandes cela signifierait que le comportement des employ\'es n'est pas bon puisque nous pouvons deviner - avec exasp\'eration - à l'avance quel problème a lieu avec une certaine confiance).

	\begin{tcolorbox}[title=Remarque,colframe=black,arc=10pt]
	Nous n'avons pas trouv\'e comment v\'erifier ces derniers calculs avec MSNBX 1.4.2. Si quelqu'un trouve comment le faire, ce serait super de nous communiquer le d\'etail de la d\'emarche.
	\end{tcolorbox}	
	
	Pour clore, le lecteur aura remarqu\'e que les calculs peuvent vite devenir ennuyeux dès que le graphe devient complexe d'où l'usage de logiciels informatiques. De plus, dans le domaine bancaire qui utilise par exemple les r\'eseaux bay\'esiens pour les risques de cr\'edit, la probabilit\'e a priori peut être plus complexe. Par exemple nous pourrions vouloir connaître la probabilit\'e a priori qu'il y ait une panne machine sachant que nous avons une alarme et un accident de travail:
	
	Pour les personnes souhaitant approfondir le sujet, avec des cas pratiques sur le logiciel R et l'utilisation des réseaux bayésians gaussian, nous conseillons vivement la lecture de \cite{scutari2014bayesian}!

\subsection{Martingales}

Une martingale en probabilit\'es (il en existe une autre dans les processus stochastiques) est une technique permettant d'augmenter les chances de gain aux jeux de hasard tout en respectant les règles de jeu. Le principe d\'epend complètement du type de jeu qui en est la cible, mais le terme est accompagn\'e d'une aura de mystère qui voudrait que certains joueurs connaissent des techniques secrètes mais efficaces pour tricher avec le hasard. Par exemple, de nombreux joueurs (ou candidats au jeu) cherchent LA martingale qui permettra de battre la banque dans les jeux les plus courants dans les casinos (des institutions dont la rentabilit\'e repose presque entièrement sur la diff\'erence - même faible - qui existe entre les chances de gagner et celles de perdre).

De nombreuses martingales ne sont que le rêve de leur auteur, certaines sont en fait inapplicables, quelques-unes permettent effectivement de tricher un peu. Les jeux d'argent sont en g\'en\'eral in\'equitables: quel que soit le coup jou\'e, la probabilit\'e de gain du casino (ou de l'\'etat dans le cas d'une loterie) est plus importante que celle du joueur. Dans ce type de jeu, il n'est pas possible d'inverser les chances, seulement de minimiser la probabilit\'e de ruine du joueur.

L'exemple le plus courant est la martingale de la roulette. Elle consiste à jouer une chance simple à la roulette (noir ou rouge, paire ou impaire) de façon à gagner, par exemple, une unit\'e dans une s\'erie de coups en doublant sa mise si l'on perd, et cela jusqu'à ce que l'on gagne. Exemple: le joueur mise 1 unit\'e sur le rouge, si le rouge sort, il arrête de jouer et il a gagn\'e 1 unit\'e (2 unit\'es de gain moins l'unit\'e de mise), si le noir sort, il double sa mise en pariant 2 unit\'es sur le rouge et ainsi de suite jusqu'à ce qu'il gagne.

\begin{figure}[H]
\centering
\includegraphics{img/arithmetics/casino_roulette.eps}
\caption[]{Roulette de Casino}
\end{figure}

Ayant une chance sur deux de gagner, il peut penser qu'il va finir par gagner ; quand il gagne, il est forc\'ement rembours\'e de tout ce qu'il a jou\'e, plus une fois sa mise de d\'epart.

Cette martingale semble être sûre en pratique. À noter que sur le plan th\'eorique, pour être sûr de gagner, il faudrait avoir la possibilit\'e de jouer au cas où un nombre de fois illimit\'e... Ce qui pr\'esente des inconv\'enients majeurs:

Cette martingale est en fait limit\'ee par les mises que le joueur peut faire car il faut doubler la mise à chaque coup tant que l'on perd: $2$ fois la mise de d\'epart, puis $4, 8, 16$... s'il perd $10$ fois de suite, il doit pouvoir avancer $1024$ fois sa mise initiale pour la $11$ème partie ! Il faut donc beaucoup d'argent pour gagner peu.

Les roulettes comportent de plus un "$0$" qui n'est ni rouge ni noir. Le risque de perdre lors de chaque coup est ainsi plus grand que $1/2$...

De plus, pour paralyser cette strat\'egie, les casinos proposent des tables de jeu par tranche de mise: de $1$ à $100$.-, de $2$ à $200$.-, de $5$ à $500$.-, ... Impossible donc d'utiliser cette m\'ethode sur un grand nombre de coups, ce qui augmente le risque de tout perdre.

Le black jack est un jeu qui possède des strat\'egies gagnantes: plusieurs techniques de jeu, qui n\'ecessitent g\'en\'eralement de m\'emoriser les cartes, permettent de renverser les chances en faveur du joueur. Le math\'ematicien Edward Thorp a ainsi publi\'e en 1962 un livre qui fut à l'\'epoque un v\'eritable best-seller. Mais toutes ces m\'ethodes demandent de longues semaines d'entraînement et sont facilement d\'ecelables par le croupier (les brusques changements de montant des mises sont caract\'eristiques). Le casino a alors tout loisir d'\'ecarter de son \'etablissement les joueurs en question.

Notons qu'il y a assez de m\'ethodes avanc\'ees. L'une d'elle est bas\'ee sur les combinaisons les moings jou\'ees. Dans le jeux où les gains d\'ependent sur le nombre de joueurs gagnants (jeux du type Lotto...), jouer les num\'eros les moins utilis\'es maximise alors potentiellement les gains. C'est pourquoi certaines personnes vendent des combinaisons rare qui statistiquement sont très très rarement utilis\'ees par les autres joueurs.

Bas\'e sur ce raisonnement, nous pouvons cependant conculre qu'un joueur qui aurait pu d\'eterminer ces combinaisons sp\'ecialement rares, pour maximiser son gain, sera probablement pas le seul joueur à proc\'eder ainsi (au fait il y a toujours cependant le premier à avoir trouv\'e une martingale qui ensuite se faire connaître par la masse...!). Cela signifie sous cette hypothèse que les combinaisons les moins jou\'ees deviennent peut-être les plosu journ\'ees, le meilleur \'etant peut-être de jouer un m\'elange de combinaisons normales et rares. Une autre conclusion à tout cela est peut-être de jouer avec des combinaisons al\'eatoires qui sont probablement celles ayant le moins de chances d'être choisies par des joueurs humains qui pour la majorit\'e aiment bien prendre des pattern (motifs) de nombre harmonieux ou sentimentaux.

\subsection{Analyse Combinatoire}\label{combinatorial analysis}

"\NewTerm{L'analyse combinatoire}\index{analyse combinatoire}" (techniques de d\'enombrement) est le domaine de la math\'ematique qui s'occupe de l'\'etude de l'ensemble des issues, \'ev\'enements ou faits (distinguables ou non tous distinguables) avec leurs arrangements (combinaisons) ordonn\'es ou non selon certaines contraintes donn\'ees.

\textbf{D\'efinitions (\#\mydef):}
\begin{enumerate}
	\item[D1.] Une suite d'objets (\'ev\'enements, issues, objets,...) est dite "\NewTerm{ordonn\'ee}\index{s\'equence ordonn\'ee}" si chaque suite compos\'ee d'un ordre particulier des objets est comptabilis\'ee comme une configuration particulière.
	
	\item[D2.] Une suite est donc "\NewTerm{non ordonn\'ee}\index{s\'equence non ordonn\'ee}" si et seulement si nous int\'eresse la fr\'equence d'apparition des objets ind\'ependamment de leur ordre.
	
	\item[D3.] Des objets (d'une suite) sont dits "\NewTerm{distincts}\index{objects distincts}" si leurs caract\'eristiques ne permettent pas de les confondre avec des autres objets.
\end{enumerate}

	\begin{tcolorbox}[title=Remarque,colframe=black,arc=10pt]
	Nous avons choisi de mettre l'analyse combinatoire dans ce chapitre car lorsque nous calculons des probabilit\'es, nous avons \'egalement assez souvent besoin de savoir quelle est la probabilit\'e de tomber sur une combinaison ou un arrangement d'\'ev\'enements donn\'es sous certaines contraintes.
	\end{tcolorbox}

	Souvent les \'etudiants ont de la peine à se rappeler de la diff\'erence entre une permutation, un arrangement et une combinaison. Voici donc un petit r\'esum\'e de ce que nous allons voir:
	\begin{itemize}
	 	\item \NewTerm{Permutation}\index{permutation}: On prend tous les \'el\'ements.

	 	\item \NewTerm{Arrangement}\index{arrangement}: On choisit des \'el\'ements parmi ceux de l'ensemble de d\'epart et l'ordre intervient.

	 	\item \NewTerm{Combinaison}\index{combinaison}: Idem que pour l'arrangement mais l'ordre n'intervient pas.
	\end{itemize}

	Il ne faut pas oublier que pour le r\'esultat de chacun, l'inverse donnera la probabilit\'e de tomber respectivement sur une Permutation/Arrangement/Combinaison donn\'ee!!

	Nous allons pr\'esenter et d\'emontrer ci-dessous les 6 cas les plus r\'epandus à partir desquels nous pouvons trouver (habituellement) tous les autres:

\subsubsection{Arrangements simples avec r\'ep\'etitions}\label{simple arrangements with repetitions}

	\textbf{D\'efinition (\#\mydef):} Un "\NewTerm{arrangement simple avec r\'ep\'etitions}\index{arrangement simple avec r\'ep\'etitions}" ou de manière \'equivalent un "\NewTerm{\'echantillonnage simple avec remplacement}\index{\'echantillonnage simple avec remplacement}" (ESAR) est une suite ordonn\'ee de longueur $m$ de $n$ objets distincts non n\'ecessairement tous diff\'erents dans la suite (soit avec r\'ep\'etitions possibles!).

	Soient $A$ et $B$ deux ensembles finis de cardinaux respectifs $m, n$ tels que trivialement il y ait $m$ façons de choisir un objet dans $A$ (de type $a$) et $n$ façons de choisir un objet dans $B$ (de type $b$).

	Nous avons vu dans la section de Th\'eorie Des Ensembles que si $A$ et $B$ sont disjoints, que:
	
	Nous en d\'eduisons donc les propri\'et\'es suivantes:
	\begin{enumerate}
		\item[P1.]  Si un objet ne peut être à la fois de type $a$ et de type $b$ et s'il y a $m$ façons de choisir un objet de type $a$ et $n$ façons de choisir un objet de type $b$, alors l'union des objets donne $m+n$ s\'elections (c'est typiquement le r\'esultat des requêtes d'UNION en SQL, sans filtres, dans les SGBDR des entreprises).
		
		\item[P2.] Si nous pouvons choisir un objet de type $a$ de $m$ façons puis un objet de type $b$ de $n$ façons, alors il y a selon le produit cart\'esien de deux ensembles (\SeeChapter{voir section Th\'eorie des Ensembles page \pageref{cartesian product}}):
		
		de manières choisir un seul et unique objet de type $a$ puis un objet de type $b$ (c'est g\'en\'eralement le r\'esultat de requêtes SELECT en SQL, sans filtres, avec plusieurs tables non li\'ees dans les SGBDR d'entreprise).
	\end{enumerate}
	Avec les mêmes notations pour $m$ et $n$, nous pouvons donc choisir pour chaque \'el\'ement de $A$, son unique image parmi les $n$ \'el\'ements de $B$. Il y a donc $n$ façons de choisir l'image du premier \'el\'ement de $A$, puis aussi $n$ façons de choisir l'image du deuxième \'el\'ement de $A$, ..., puis $n$ façons de choisir l'image du $m$-ème \'el\'ement de $A$. Le nombre d'applications totales cons\'ecutives possibles de $A$ dans $B$ est donc \'egal aux $m$ produits de $n$ ($m$ fois le produit cart\'esien du cardinal de l'ensemble $B$ avec lui-même donc!). Ce qu'il est d'usage d'\'ecrire (nous avons mis les diff\'erentes \'ecritures que l'on peut trouver dans les livres scolaires):
	
	où $B^A$ est l'ensemble des applications de $A$ dans $B$. La progression du nombre de possibilit\'es est donc g\'eom\'etrique (et non "exponentielle" comme il est souvent dit à tort!).

	Ce r\'esultat math\'ematique est assimilable au r\'esultat ordonn\'e (un arrangement equation dont l'ordre des \'el\'ements de la suite est pris en compte) de $m$ tirages dans un sac contenant $n$ boules diff\'erentes avec remise après chaque tirage. Il est d'usage en France d'appeler cela une "\NewTerm{$p$-liste}\index{$p$-liste}".

	\begin{tcolorbox}[colframe=black,colback=white,sharp corners]
	\textbf{{\Large \ding{45}}Exemples:}\\\\
	E1. Combien de "mots" (ordonn\'es) de $7$ lettres pouvons-nous former à partir d'un alphabet de $24$ lettres distinctes (très utile pour connaître le nombre d'essais pour trouver un mot de passe par exemple)? La solution est:
	
	E2.  Combien de groupes d'individus aurons-nous lors d'une votation sur $5$ sujets et où chacun peut être soit accept\'e, soit rejet\'e? La solution (très utilis\'ee dans les entreprises en Suisse) est:
	
	\end{tcolorbox}
	Une g\'en\'eralisation simple de ce dernier r\'esultat peut consister dans l'\'enonc\'e du problème suivant:

	Si nous disposons de m objets $k_1,k_2,...,k_m$ tels que $k_i$ peut prendre $n_i$ \'etats diff\'erents alors le nombre de combinaisons possibles est:
	
	Et si nous avons $n_1=n_2=...=n_m$ alors nous retombons sur:
	
	
	\subsubsection{Permutations simples sans r\'ep\'etitions}
	\textbf{D\'efinition (\#\mydef):} Une "\NewTerm{permutation simple sans r\'ep\'etitions}\index{permutation simple sans r\'ep\'etitions}\label{simple permutation without repetitions}" (appel\'ee anciennement "substitution") de $n$ objets distincts est une suite ordonn\'ee (diff\'erente) de ces $n$ objets par d\'efinition tous diff\'erents dans la suite (sans r\'ep\'etition).

	\begin{tcolorbox}[title=Remarque,colframe=black,arc=10pt]
	Attention à ne pas confondre le concept de permutation (de $n$ \'el\'ements entre eux) et d'arrangement (de $n$ \'el\'ements parmi $m$)!
	\end{tcolorbox}	
	
	Le nombre de permutations de $n$ \'el\'ements peut être calcul\'e par r\'ecurrence: il y a $n$ places pour un premier \'el\'ement, $n-1$ pour un deuxième \'el\'ement, ..., et il ne restera qu'une place pour le dernier \'el\'ement restant.

	Il est dès lors trivial que nous aurons un nombre de permutations donn\'e par: 
	
	Rappelons que le produit:
	
	est appel\'e "\NewTerm{factorielle de $n$}\index{factorielle}" et nous la notons $n!$ pour $n \in \mathbb{N}$.

	Il y a donc pour $n$ \'el\'ements distinguables\label{distinguishable elements}:
	
	permutations possibles. Ce type de calcul peut être par exemple utile en gestion de projets (calcul du nombre de manière diff\'erentes de recevoir dans une chaîne de production $n$ pièces toutes diff\'erentes command\'ees chez des fournisseurs externes).

	\begin{tcolorbox}[colframe=black,colback=white,sharp corners]
	\textbf{{\Large \ding{45}}Exemple:}\\\\
	Combien de "mots" (ordonn\'es) de $7$ lettres distinctes sans r\'ep\'etition pouvons-nous former?
	
	Ce r\'esultat nous amène à l'assimiler au r\'esultat ordonn\'e (un arrangement $A_n$ dont l'ordre des \'el\'ements de la suite est pris en compte) du tirage de toutes les boules diff\'erentes d'un sac contenant $n$ boules distinguables sans remise.
	\end{tcolorbox}
	Les \'enormes valeurs que peut donner la factorielle peuvent être assez surprenantes ou même parfois contre-intuitives (consid\'erons par exemple le nombre de façons dont vous pouvez cr\'eer des groupes de disons simplement ... $7$ amis!). À propos, on peut aussi se demander si l'Univers est calculable ou non, car de tels calculs conduisent à des nombres \'enormes!

	\subsubsection{Permutations simples avec r\'ep\'etitions}

	\textbf{D\'efinitions (\#\mydef):} Lorsque nous consid\'erons le nombre de permutations ordonn\'ees (diff\'erentes) d'une suite de $n$ objets distincts tous n\'ecessairement non diff\'erents dans une quantit\'e donn\'ee dans la suite nous parlons de "\NewTerm{permutation simple avec r\'ep\'etitions}\index{permutation simple avec r\'ep\'etitions}".

	\begin{tcolorbox}[title=Remarque,colframe=black,arc=10pt]
	Il ne faut pas confondre cette dernière d\'efinition avec "l'arrangement avec r\'ep\'etition" vu plus haut!
	\end{tcolorbox}	
	
	Lorsque certains \'el\'ements ne sont pas tous distinguables dans une suite d'objets (ils sont r\'ep\'etitifs dans la suite), alors le nombre de permutations que nous pouvons constituer se r\'eduit alors assez trivialement à un nombre plus petit que si tous les \'el\'ements \'etaient tous distinguables.

	Soit $n_i$ le nombre d'objets du type $i$, avec:	
	
	alors, nous notons:
	
	avec $i=1,2,\ldots,k$ le nombre de permutations possibles (pour l'instant inconnu) avec r\'ep\'etition (un ou plusieurs \'el\'ements r\'ep\'etitifs dans une suite d'\'el\'ements sont non distinguables par permutation).

	Si chacune des $n_i$ places occup\'ees par des \'el\'ements identiques \'etait occup\'ee par des \'el\'ements diff\'erents, le nombre de permutations serait alors à multiplier par chacun des $n_i!$ (cas pr\'ec\'edent).
	
	Il vient alors que nous retombons sur la factorielle telle que:
	
	dont nous d\'eduisons imm\'ediatement:
	
	Si les $n$ objets sont tous diff\'erents dans la suite, nous avons alors:
	
	Si les $n$ objets sont tous diff\'erents dans la suite, nous avons alors:
	
	Il conviendra donc de se rappeler que les permutations avec r\'ep\'etition sont en plus petit nombre que celles sans r\'ep\'etition (\'evident puisque nous ne prenons pas en compte les permutations des \'el\'ements identiques entre eux!).
	
	\begin{tcolorbox}[colframe=black,colback=white,sharp corners]
	\textbf{{\Large \ding{45}}Exemple:}\\\\
	Combien de "mots" (ordonn\'es) pouvons-nous former avec les lettres du mot "Mississippi":
	
	\end{tcolorbox}
	Ce r\'esultat nous amène à l'assimiler au r\'esultat ordonn\'e (une permutation $\bar{A}_n$ dont l'ordre des \'el\'ements de la suite n'est pas pris en compte) du tirage de $n$ boules non toutes distinguables d'un sac contenant $k \geq n$  boules avec remise limit\'ee pour chaque boule.

	\subsubsection{Arrangements simples sans r\'ep\'etitions}\label{simple arrangements without repetitions}

	\textbf{D\'efinition (\#\mydef):} Un "\NewTerm{arrangement simple sans r\'ep\'etitions}\index{arrangement simple sans r\'ep\'etitions}", ou de manière \'equivalente un "\NewTerm{\'echantillonnage simple sans remise}\index{\'echantillonnage simple sans remise}" (ESSR), est une suite ordonn\'ee de $p$ objets tous distincts pris parmi $n$ objets distincts avec $n \geq p$.

	Nous nous proposons donc maintenant de d\'enombrer les arrangements possibles sans r\'ep\'etition de $p$ objets parmi $n$. Nous noterons $A_n^p$ le nombre de ces arrangements.
	
	Il est ais\'e de calculer $A_n^1=n$ et de v\'erifier que $A_n^2=n(n-1)$. Effectivement, il existe $n$ façons de choisir le premier objet et $(n-1)$ façons de choisir le deuxième lorsque nous avons d\'ejà le premier.
	
	Pour d\'eterminer $A_n^p$, nous raisonnons alors par r\'ecurrence. Nous supposons $A_n^{p-1}$ connu et nous en d\'eduisons:
	
	Dès lors:
	
	Alors:
	
	d'où:
	
	Ce r\'esultat nous amène à l'assimiler au r\'esultat ordonn\'e (un arrangement $A_n^p$ dont l'ordre des \'el\'ements de la suite est pris en compte) du tirage de $p$ boules distinctes d'un sac contenant $n$ boules diff\'erentes sans remise.


	\begin{tcolorbox}[colframe=black,colback=white,sharp corners]
	\textbf{{\Large \ding{45}}Exemple:}\\\\
	Soit les $24$ lettres de l'alphabet, combien de "mots" (ordonn\'es) de $7$ lettres distinctes pouvons-nous former?
	
	\end{tcolorbox}
	Le lecteur aura peut-être remarqu\'e que si nous prenons $p=n$ nous nous retrouvons avec:
	
	donc nous pouvons dire qu'une permutation simple de $n$ \'el\'ements est comme un arrangement simple sans r\'ep\'etition avec $n=p$.

	\subsubsection{Combinaisons simples sans r\'ep\'etitions}\label{simple combinations without repetitions}

	\textbf{D\'efinition (\#\mydef):} Une "\NewTerm{combinaison simple sans r\'ep\'etitions}\index{combinaison simple sans r\'ep\'etitions}" ou "\NewTerm{fonction de choix}\index{fonction de choix}\label{choice function}" est une suite non-ordonn\'ee (dont l'ordre ne nous int\'eresse pas!) de $p$ \'el\'ements tous diff\'erents (pas n\'ecessairement dans le sens visuel du terme!) choisis parmi $n$ objets distincts et est par d\'efinition not\'ee dans ce livre $C_p^n$ et appel\'ee la "\NewTerm{binomiale}\index{binomiale}" ou "\NewTerm{coefficient binomial}\index{coefficient binomial}\label{binomial coefficient}".

	Si nous permutons les \'el\'ements de chaque arrangement simple de $p$ \'el\'ements parmi $n$, nous obtenons toutes les permutations simples et nous savons qu'il y en a p! d'où en utilisant la convention d'\'ecriture du pr\'esent site internet (contraire à celle pr\'econis\'ee par la norme ISO 31-11 et ISO 80000-2:2009!):
	
	C'est une relation très souvent utilis\'ee dans les jeux de hasard mais \'egalement dans l'industrie via la loi hyperg\'eom\'etrique (\SeeChapter{voir section G\'enie industriel page \pageref{quality control hypergeometric}}).
	
	L'astuce suivante est un moyen simple de retenir cette fonction: consid\'erons que nous devons s\'electionner $p$ parmi $n$ ind\'ependamment de l'ordre, quel est le nombre de possibilit\'es?
	
	Nous savons que nous avons $6 \dot 5 \cdot 4 = 120 $ possibilit\'es pour les s\'electionner en tenant compte de la commande! Le calcul que nous venons de faire est \'evidemment \'egal à $ n! / P! = 6! / 3! = 6 \cdot 5 \cdot 4 $. Mais comme l'ordre ne doit pas être prise en compte, nous devons diviser $120$ par le nombre de façons dont nous pouvons organiser les $3$ personnes du groupe. Nous divisons donc $120$ par $3!$ ou plus g\'en\'eralement et logiquement par $(n-p)!$. D'où la relation ci-dessus!
	\begin{tcolorbox}[title=Remarques,colframe=black,arc=10pt]
	\textbf{R1.} Nous avons n\'ecessairement par construction $C_p^n \leq A_n^p$.\\

	\textbf{R2.} En fonction des auteurs nous inversons l'indice ou le suffixe de $C$ il faut donc être prudent!
	\end{tcolorbox}
	Ce r\'esultat nous amène à l'assimiler au r\'esultat non ordonn\'e (un arrangement equation dont l'ordre des \'el\'ements de la suite n'est pas pris en compte) du tirage de $p$ boules d'un sac contenant $n$ boules diff\'erentes sans remise.
	\begin{tcolorbox}[colframe=black,colback=white,sharp corners]
	\textbf{{\Large \ding{45}}Exemples:}\\\\
	E1. Soit un alphabet de $24$ lettres, combien avons-nous de choix de prendre $7$ lettres parmi les $24$ sans prendre en compte l'ordre dans lequel sont tri\'ees les lettres:
		
		La même valeur peut être obtenue avec la fonction  $\texttt{COMBIN( )}$ de Microsoft Excel 11.8346 (version française).\\
	
	E2. Dans un plan d'exp\'erience (\SeeChapter{voir section de G\'enie Industriel page \pageref{doe}}), nous avons $2$ facteurs de $L=3$ niveaux chacun et nous avons donc besoin de $N =9$ essais pour d\'eterminer complètement toutes les interactions. Si nous consid\'erons que nous pouvons prendre un sous-ensemble de $S =3$ essais, combien de combinaisons de $3$ parmi les $9$ pouvons-nous choisir si les r\'ep\'etitions sont interdites?
	
	Nous comprenons donc pourquoi dans les plans d'exp\'erience, il est important de trouver une astuce pour choisir le meilleur sous-ensemble (plans D-optimum par exemple!).
	\end{tcolorbox}
	Il existe, relativement à la binomiale, une autre relation très souvent utilis\'ee dans de nombreux cas d'\'etudes ou \'egalement de manière plus globale en physique ou analyse fonctionnelle. Il s'agit de la "\NewTerm{formule de Pascal}\index{formule de Pascal}\label{pascal formula}":
	\begin{dem}
	
	Or $p!=p(p-1)!$, donc:
	
	et de même $(n-p)(n-p-1)!=(n-p)!$:
	
	 Ainsi:
	
	\begin{flushright}
		$\blacksquare$  Q.E.D.
	\end{flushright}
	\end{dem}
	
	\subsubsection{Combinaisons simples avec r\'ep\'etitions}
	
	\textbf{D\'efinition (\#\mydef):}  Une "\NewTerm{combinaison simple avec r\'ep\'etitions}\index{combinaison simple avec r\'ep\'etitions}" de $p$ \'el\'ements parmi $n$ est une collection de $p$ \'el\'ements non ordonn\'ee, et non n\'ecessairement distincts.

	Les combinaisons simples avec r\'ep\'etitions ont une grande importance pour le test statistique de Wald–Wolfowitz utilis\'ee en \'economie et biologique que nous \'etudierons dans le chapitre de Statistiques.

	Introduisons ce type de combinaison directement avec un exemple et une approche ing\'enieuse que l'on doit (du moins c'est ce qui ce dit...) au physicien prix Nobel de physique 1938: Enrico Fermi.
	
	Consid\'erons $\left\lbrace a, b, c, d, e, f\right\rbrace $ un ensemble ayant un nombre $n$ d'\'el\'ements \'egal à $6$ et dont nous tirons un nombre $p$ \'egal à $8$. Nous souhaiterions calculer le nombre de combinaisons avec r\'ep\'etitions des \'el\'ements d'un ensemble de d\'epart de cardinal $6$ dans une ensemble d'arriv\'ee de cardinal $8$.

	Envisageons, par exemple, les trois combinaisons suivantes:
	
	où comme l'ordre des \'el\'ements n'intervient pas, nous avons regroup\'e les \'el\'ements afin de faciliter la lecture. Repr\'esentons maintenant tous les \'el\'ements ci-dessus par un même symbole: "$0$" et s\'eparons les groupes constitu\'es d'un même \'el\'ement par des barres (c'est là l'astuce d'Enrico Fermi). Ainsi, lorsqu'un ou plusieurs \'el\'ements ne figurent pas dans une combinaison, nous noterons tout de même les barres de s\'eparation (correspondant au nombre d'\'el\'ements absents + la s\'eparation du groupe). Les trois combinaisons ci-dessus s'\'ecrivent alors:
	
	Nous voyons ci-dessus que dans chaque cas, il a y huit "$0$" (logique...) mais surtout qu'il y a toujours cinq "$\mid$". Le nombre de combinaisons avec r\'ep\'etitions des 6 \'el\'ements de l'ensemble de d\'epart à celui d'arriv\'ee de 8 \'el\'ements est donc \'egal au nombre de permutations avec repetitions de $8+5=13$ \'el\'ements, donc:
	
	Nous remarquons que dans le cas g\'en\'eral le nombre de combinaisons avec r\'ep\'etitions sans prise en compte de l'ordre s'\'ecrit alors:
	
	Ce qu'il est de tradition de noter:
	
	Nous remarquons par ailleurs que:
	
	Soit au final:
	
	Ce que suivant le context, nous notons aussi:
	
	\begin{tcolorbox}[colframe=black,colback=white,sharp corners]
	\textbf{{\Large \ding{45}}Exemple:}\\\\
	Combien de combinaisons de r\'esultats diff\'erents pouvons-nous faire en lançant trois d\'es standard ($ 6 $ face) si l'ordre des d\'es n'a pas d'importance??\\
	
	La r\'eponse est:
	
	Mais si l'ordre importe, alors c'est simplement $6\cdot 6\cdot 6=216$.
	\end{tcolorbox}
	Pour r\'esumer:
	\begin{table}[H]
		\begin{center}
			\definecolor{gris}{gray}{0.85}
				\begin{tabular}{|p{8cm}|p{6cm}|}
					\hline
					\multicolumn{1}{c}{\cellcolor{black!30}\textbf{Type}} & 
	  \multicolumn{1}{c}{\cellcolor{black!30}\textbf{Expression}} \\ \hline
					Arrangement simple avec r\'ep\'etitions (not\'e $^RV_n^m$ selon ISO 80000-2:2009) & \centering\arraybackslash\ $\bar{A}_n^m=n^m$ \\ \hline
					Arrangement simple sans r\'ep\'etitions (not\'e $V_n^m$ selon ISO 80000-2:2009) & \centering\arraybackslash\ $A_m^n=\dfrac{n!}{(n-m)!}$  \\ \hline
					Permutation simple sans r\'ep\'etitions & \centering\arraybackslash\ $A_n=n!$  \\ \hline
					Permutation simple avec r\'ep\'etitions & \centering\arraybackslash\ $\bar{A}_n(n_1,n_2,...,n_k)=\dfrac{n!}{n_1!n_2!...n_k!}$  \\ \hline
					Combinaison simple sans r\'ep\'etitions: cas de l'arrangement simple sans r\'ep\'etitions où l'ordre n'est pas pris en compte & 		\centering\arraybackslash\ $C_m^n=\begin{pmatrix}n\\p\end{pmatrix}=\dfrac{A_n^m}{m!}=\dfrac{n!}{m!(n-m)!}$  \\ \hline
					Combinaison simple avec r\'ep\'etitions: cas de permutation simple avec r\'ep\'etitions où l'ordre n'est pas pris en compte (not\'e $^RC_n^p$ selon ISO 80000-2:2009) & \centering\arraybackslash\ $\Gamma_p^n=C_p^{n+p-1}=\dfrac{(n+p-1)!}{(n-1)!p!}$  \\ \hline
			\end{tabular}
		\end{center}
		\caption{R\'esum\'e des cas possibles}
	\end{table}
	Quatre des relations ci-dessus peuvent \'egalement être pr\'esent\'ees comme suit:
	\begin{table}[H]
		\centering
		\begin{tabular}{|
		>{\columncolor[HTML]{9B9B9B}}l |c|c|}
		\hline
		 & \cellcolor[HTML]{9B9B9B}\textbf{L'ordre importe} & \cellcolor[HTML]{9B9B9B}\textbf{L'ordre n'importe pas} \\ \hline
		\textbf{Avec remplacement} & $\bar{A}_n^m=n^m$ & $\Gamma_p^n=C_p^{n+p-1}=\dfrac{(n+p-1)!}{(n-1)!p!}$  \\ \hline
		\textbf{Sans remplacement} &  $A_m^n=\dfrac{n!}{(n-m)!}$ & $C_m^n=\begin{pmatrix}n\\p\end{pmatrix}=\dfrac{A_n^m}{m!}=\dfrac{n!}{m!(n-m)!}$ \\ \hline
		\end{tabular}
	\end{table}
	
	\begin{tcolorbox}[title=Remarque,colframe=black,arc=10pt]
	Si nous consid\'erons trois \'el\'ements $ A $, $ B $ et $ C $, le nombre total cumul\'e de combinaisons où l'ordre est important est de façon relativement \'evidente donn\'e par (ceci est directement tir\'e du th\'eorème binomial prouv\'e à la page \pageref{binomial theorem}, où nous posons $a=1$ et $b=1$):
	
	Mais si l'ordre n'a pas d'importance, nous avons:
	
	\end{tcolorbox}	
	
	Nous avons \'egalement que le nombre de façons dont $ m \cdot n $ diff\'erents peuvent être divis\'es en groupes \'egaux de taille $m$ contenant chacun des objets $n$ et où l'ordre des groupes est important qui est donn\'e par:
	
	et le nombre de façons dont $ m \cdot n $ diff\'erents \'el\'ements  peuvent être divis\'es en groupes \'egaux de taille $m$, chacun contenant $n$ objets et où l'ordre des groupes n'a pas d'importance qui est donn\'e par:
	
	\begin{figure}[H]
		\centering
		\includegraphics[scale=0.65]{img/arithmetics/meilleur_en_theorie_quen_pratique.jpg}
	\end{figure}
	Pour clore cette \'etude sur les bases de l'analyse combinatoire, il y aurait un autre cas important mais qui semble encore non r\'esolu en ce d\'ebut de XXIe siècle et qui est li\'e à la chimie (certains experts parlent de "\NewTerm{chimie combinatoire}"). En effet, \'etant donn\'e un certain nombre d’\'el\'ements chimiques purs ayant leur valence respective dans certaines conditions de thermodynamique, combien de mol\'ecules impliquant l’une d’entre elles peuvent être cr\'e\'ees?
	
	Par exemple, avec l'ad\'enine (l'un des quatre composants fondamentaux de l'ADN) donn\'ee par $\mathrm{C}_5\mathrm{N}_5\mathrm{H}_5$ et visible dans la figure ci-dessous:
	\begin{figure}[H]
		\centering
		\includegraphics[scale=0.75]{img/arithmetics/acgt_dna.jpg}
	\end{figure}
	Nous pourrions alors nous demander quel est le nombre total de combinaisons impliquant$5\mathrm{C}$, $5\mathrm{N}$ et $5\mathrm{H}$?

	Une limite sup\'erieure à cette question serait donn\'ee (sur la base de leur valence respective par):
	
	Mais si nous dessinons toutes les combinaisons physiques possibles dans des conditions de temp\'erature et de pression standard, il semble qu'une limite sup\'erieure se situe plutôt entre $200$ et $2'000$ mol\'ecules diff\'erentes.
	
	Il s’agit donc toujours d’un «problème ouvert» important dans le domaine de la chimie avec d’\'enormes possibilit\'es d’application (certaines industries utilisent en r\'ealit\'e des ordinateurs pour trouver des combinaisons possibles d’\'el\'ements chimiques purs pouvant avoir des applications en m\'edecine!).
	
	

\pagebreak
\subsection{Chaînes de Markov}\label{markov chains}

Les chaînes de Markov sont des outils statistiques et probabilistes simples et puissants mais dont la forme de pr\'esentation math\'ematique prête parfois à l'horreur.... Nous allons tenter ici de simplifier un maximum les notations pour introduire cet outil formidable très utilis\'e au sein des entreprises pour g\'erer la logistique, les files d'attentes aux centrales d'appel ou aux caisses de magasins jusqu'à la th\'eorie de la d\'efaillance pour la maintenance pr\'eventive, en physique statistique ou en g\'enie biologique (et la liste est encore longue et pour plus de d\'etails le lecteur pourra se reporter aux chapitres concern\'es disponibles dans ce livre...).

\textbf{D\'efinitions (\#\mydef):}
	\begin{enumerate}
		\item[D1.] Nous noterons $\left\lbrace X(t) \right\rbrace_{t \in T} $ un processus probabiliste fonction du temps dont la valeur à chaque instant d\'epend de l'issue d'une exp\'erience al\'eatoire. Ainsi, à chaque instant $t$, $X(t)$ est donc une variable al\'eatoire que nous d\'esignons par "\NewTerm{processus stochastique}\index{processus stochastique}" (pour plus de d\'etails dans le cadre de la finance, voir le chapitre d'\'economie).

		\item[D2.] Si nous consid\'erons un temps discret, nous notons alors $\left\lbrace X_n \right\rbrace_{n \in \mathbb{N}} $ un "\NewTerm{processus stochastique à temps discret}\index{processus stochastique à temps discret}".

		\item[D3.] Si nous supposons en outre que les variables al\'eatoires $X_n$ ne peuvent prendre qu'un ensemble discret de valeurs nous parlons alors de "\NewTerm{processus à temps discret et à espace discret}\index{processus à temps discret et à espace discret}".
	\end{enumerate}
	\begin{tcolorbox}[title=Remarque,colframe=black,arc=10pt]
	Il est tout à fait possible comme dans la th\'eorie des files d'attentes (\SeeChapter{voir section Techniques de Gestion page \pageref{queueing theory}}) d'avoir un processus à temps continu et à espace d'\'etats discrets.
	\end{tcolorbox}
	\textbf{D\'efinition (\#\mydef):} $\left\lbrace X_n \right\rbrace_{n \in \mathbb{N}}$ est une "\NewTerm{chaîne de Markov}\index{chaîne de Markov}" si et seulement si:
	
	en d'autres termes (c'est très simple!) la probabilit\'e pour que la chaîne soit dans un certain \'etat à la $n$-ème \'etape du processus ne d\'epend que de l'\'etat du processus à l'\'etape $n$-1 et pas des \'etapes pr\'ec\'edentes!

	\begin{tcolorbox}[title=Remarque,colframe=black,arc=10pt]
	Donc en probabilit\'es un processus stochastique v\'erifie la propri\'et\'e markovienne ci-dessus si et seulement si la distribution conditionnelle de probabilit\'e des \'etats futurs, \'etant donn\'e l'instant pr\'esent, ne d\'epend que de ce même \'etat pr\'esent et pas des \'etats pass\'es. Un processus qui possède cette propri\'et\'e est aussi appel\'e "\NewTerm{processus de Markov}\index{processus de Markov}".
	\end{tcolorbox}
	\textbf{D\'efinition (\#\mydef):} Une "\NewTerm{chaîne de Markov homogène}\index{chaîne de Markov homogène}" est une chaîne telle que la probabilit\'e qu'elle a pour passer dans un certain \'etat à la $n$-ième \'etape soit ind\'ependante du temps. En d'autres termes, la loi de probabilit\'e caract\'erisant la prochaine \'etape ne d\'epend pas du temps (de l'\'etape pr\'ec\'edente), et en tout temps la loi de probabilit\'e de la chaîne est toujours la même pour caract\'eriser la transition à l'\'etape en cours.

	Nous pouvons alors d\'efinir (r\'eduire) la loi de  "\NewTerm{probabilit\'e de transition}\index{probabilit\'e de transition}" d'un \'etat $i$ vers un \'etat $j$ par:
	
	Il est alors naturel de d\'efinir la  "\NewTerm{matrice de transition}\index{matrice de transition}\label{transition matrix}" ou "\NewTerm{matrice stochastique}\index{matrice stochastique}":
	
	comme la matrice qui contient donc tous les probabilit\'es possibles de transitions des \'etats d'un graphe d'\'etats orient\'e.
	
	Les chaînes de Markov peuvent être repr\'esent\'ees graphiquement sous la forme d'un graphe orient\'e $G$ (\SeeChapter{voir section Th\'eorie des Graphes page \pageref{oriented graph}}) appel\'e parfois \NewTerm{automate}\index{automate}" ayant pour sommet les points (\'etats) $i$ et pour arêtes les couples orient\'es $(i, j)$. Nous associons alors à chaque composante un arc orient\'e et sa probabilit\'e de transition.

	\begin{tcolorbox}[colframe=black,colback=white,sharp corners]
	\textbf{{\Large \ding{45}}Exemple:}\\\\
	\begin{figure}[H]
	\centering
	\includegraphics[scale=0.9]{img/arithmetics/markov_chain.eps}
	\caption{Exemple g\'en\'erique d'une chaîne de Markov}
	\end{figure}
	Ainsi, dans l'exemple du graphe orient\'e ci-dessus, les seules transitions permises par les $4$ \'etats (matrice $4 \times 4$) ci-dessus sont celles indiqu\'ees par les flèches. Ce qui fait que la matrice de transition se simplifie alors en:
	
	\end{tcolorbox}
	où le lecteur remarquera que nous avons la propri\'et\'e triviale (par construction!) que la somme des termes (probabilit\'es) d'une ligne de la matrice P est toujours unitaire (et donc que la somme des termes d'une colonne de la transpos\'ee de la matrice P est toujours unitaire aussi):
	
	et que la matrice est positive (ce qui signifie que tous ces termes sont positifs ou nuls).

	\begin{tcolorbox}[title=Remarque,colframe=black,arc=10pt]
	Se rappeler que la somme des probabilit\'es des colonnes obtenues est toujours \'egale à $1$ pour la transpos\'ee de la matrice stochastique!!
	\end{tcolorbox}	
	L'analyse du r\'egime transitoire (ou: promenade al\'eatoire) d'une chaîne de Markov consiste à d\'eterminer (ou à imposer à!) la matrice-colonne (vecteur) $p(n)$ d'être dans un \'etat donn\'e à la $n$-ième \'etape de la promenade:
	
	avec la somme des composantes qui vaut \'evidemment toujours $1$ (car la somme des probabilit\'es de se trouver dans un quelconque des sommets du graphe à un moment/\'etape donn\'e(e) doit être \'egale à $100\%$).
	
	Nous appelons fr\'equemment cette matrice-colonne "\NewTerm{vecteur stochastique}\index{vecteur stochastique}" ou "\NewTerm{mesure de probabilit\'e sur le sommet $i$}\index{mesure de probabilit\'e}".

\begin{theorem}
D\'emontrons que la probabilit\'e de ce vecteur stochastique est effectivement toujours unitaire.
\end{theorem}

\begin{dem}
	Si $p(n)$ est un vecteur stochastique, alors son image:
	
	l'est aussi. Effectivement, $p_i(n+1) \geq 0$ car:
	
	est une somme de termes positifs ou nuls. De plus, nous trouvons:
	
		\begin{flushright}
			$\blacksquare$  Q.E.D.
		\end{flushright}
\end{dem}
	Ce vecteur de probabilit\'es, dont les composantes sont positives ou nulles, d\'epend (c'est assez intuitif) de la matrice de transition $P$ et du vecteur de probabilit\'es initiales $p(0)$.

	Bien que cela soit d\'emontrable (th\'eorème de Perron-Frobenius) le lecteur pourra v\'erifier par un cas pratique (informatis\'e ou non!) que si nous choisissons un vecteur d'\'etat $p(n)$ quelconque alors il existe pour toute matrice stochastique $P$ un vecteur unique de probabilit\'e not\'e traditionnellement $\pi$ tel que:
	
	Une telle mesure de probabilit\'e $\pi$ v\'erifiant la relation pr\'ec\'edente est appel\'ee une "\NewTerm{mesure invariante}\index{mesure invariante}" ou "\NewTerm{mesure stationnaire}\index{mesure stationnaire}\label{stationary measure}" ou encore "\NewTerm{mesure d'\'equilibre}\index{mesure d'\'equilibre}" qui repr\'esente l'\'etat d'\'equilibre du système. En termes d'algèbre lin\'eaire (voir section du même nom page \pageref{linear algebra}), pour la valeur propre $1$, $\pi$ est un vecteur propre de $P$ (cf. chapitre d'Algèbre Lin\'eaire).
	
	Nous en verrons un exemple trivial dans le chapitre de Th\'eorie des Graphes (page \pageref{adjacency matrix}) qui sera red\'evelopp\'e sous forme d\'etaill\'ee et complète ainsi que dans le chapitre de Th\'eorie Des Jeux Et De La D\'ecision dans le cadre de la pharmaco-\'economie (page \pageref{markov decision process}) et de l'ing\'enierie logicielle quand nous \'etudierons les fondamentaux de l'algorithme PageRank de Google (page \pageref{google pagerank algorithm}). Mais signalons \'egalement que les chaînes de Markov sont \'egalement utilis\'ees le domaine de la casse de mots de passe informatiques et en m\'et\'eorologie:
	\begin{figure}[H]
		\centering
		\includegraphics{img/arithmetics/markov_chain_meteo.jpg}
		\caption{ Exemple concret très simpliste d'une chaîne de Markov}
	\end{figure}
	ou dans le domaine m\'edical, financier (MCMC), des transports, du marketing, etc.
	
	Signalons \'egalement une \'egalit\'e trivial à laquelle les math\'ematicien auraient donn\'e le nom "\NewTerm{d'\'equation de Chapman-Kolmogorov}\index{\'equation de Chapman-Kolmogorov}". Il s'agit simple de l'\'egalit\'e suivante obtenue par r\'ecurrence (qui permet de gagner beaucoup de temps en termes de calculs appliqu\'es):
	

	Dans le domaine du language, à partir de l'analyse fr\'equentielle de s\'equence de mots, les ordinateurs arrivent à construire aussi des chaînes de Markov et donc à proposer une s\'emantique plus correcte lors de corrections grammaticales informatis\'ees ou de transcription de \'ecrite de pr\'esentations orales.

	\textbf{D\'efinitions (\#\mydef):}
	\begin{enumerate}
		\item[D1.]  Une chaîne de Markov est dite "\NewTerm{chaîne de Markov irr\'eductible}\index{chaîne de Markov irr\'eductible}" si tous les \'etats sont li\'es aux autres (c'est le cas de la chaîne dans la figure ci-dessus).

		\item[D2.] Une chaîne de Markov est dite "\NewTerm{chaîne de Markov absorbante}\index{chaîne de Markov absorbante}" si un quelconque des \'etats de la chaîne absorbe les transitions (donc rien n'en sort pour dire simplement les choses!).
	\end{enumerate}
	
	\begin{flushright}
	\begin{tabular}{l c}
	\circled{90} & \pbox{20cm}{\score{4}{5} \\ {\tiny 27 votes, 51.11\%}} 
	\end{tabular} 
	\end{flushright}
	
		%to make section start on odd page
	\newpage
	\thispagestyle{empty}
	\mbox{}
	\section{Statistiques}\label{statistics}
	\lettrine[lines=4]{\color{BrickRed}L}a statistique est une science qui a pour objet le groupement m\'ethodique de faits ou \'ev\'enements r\'ep\'etitifs qui se prêtent à une \'evaluation num\'erique ou qualitative dans le temps suivant une loi donn\'ee. Dans l'industrie et dans l'\'economie en g\'en\'eral, la statistique est une science qui permet dans un environnement incertain de faire des inf\'erences valides. 
	
	Il faut savoir que parmi tous les domaines de la math\'ematique, celui qui est utilis\'e à la plus large \'echelle dans les entreprises et centres de recherches est bien la statistique et particulièrement depuis que des logiciels en facilitent grandement les calculs! Raison pour laquelle cette section est une des plus grosse du livre alors que seuls les concepts \'el\'ementaires y sont pr\'esent\'es!

	Signalons aussi que les statistiques ont très mauvaise r\'eputation à l'universit\'e car les notations y sont souvent confuses et varient grandement d'un professeur à l'autre, d'un livre à l'autre, d'un praticien à l'autre. En toute rigueur, il faudrait se conformer au vocabulaire et notations de la norme ISO 3534-1:2006 et comme malheureusement cette section a \'et\'e \'ecrite avant la publication de cette norme... un certain temps d'adaptation sera n\'ecessaire avec qu'il y ait conformit\'e.

	Il est peut être inutile de pr\'eciser que la statistique est beaucoup utilis\'ee en ing\'enierie, physique th\'eorique, physique fondamentale, \'econom\'etrie, gestion de projets ainsi que dans l'industrie des processus, dans les domaines des assurances vies et non vies, dans l'actuariat ou dans la simple analyse de banque de donn\'ees (avec Microsoft Excel très souvent... malheureusement....) et la liste est encore longue. Par ailleurs, nous rencontrerons les outils pr\'esent\'es ici assez souvent dans les chapitres de M\'ecanique des Fluides, de Thermodynamique, des Techniques de Gestion, du G\'enie Industriel et d'\'economie (en particulier dans ces deux dernières). Le lecteur pourra donc s'y reporter pour avoir des applications pratiques concrètes de quelques-uns des \'el\'ements th\'eoriques les plus importants qui seront vus ici.

	Signalons \'egalement que outre les quelques exemples simples donn\'es sur ces pages, de nombreux autres exemples applicatifs sont donn\'es sur le serveur d'exercices du site dans les cat\'egories Probabilit\'es et Statistiques, G\'enie Industriel, \'econom\'etrie et Techniques de Gestion.

	\textbf{D\'efinition (\#\mydef):}  Le but principal de la statistique est de d\'eterminer les caract\'eristiques d'une population donn\'ee à partir de l'\'etude d'une partie de cette population, appel\'ee "\NewTerm{\'echantillon}\index{\'echantillon}" ou "\NewTerm{\'echantillon repr\'esentatif}\index{\'echantillon repr\'esentatif}". La d\'etermination de ces caract\'eristiques doit permettre aux statistiques d'être un outil d'aide à la d\'ecision!

	\begin{tcolorbox}[title=Remarque,colframe=black,arc=10pt]
	Le traitement des donn\'ees concerne la "\NewTerm{statistique descriptive}\index{statistique descriptive}". L'interpr\'etation des donn\'ees à partir des estimateurs s'appelle  "\NewTerm{l'inf\'erence statistique}" (ou "\NewTerm{statistique inf\'erentielle}\index{statistique inf\'erentielle}"), et l'analyse de donn\'ees en masse la "\NewTerm{statistique fr\'equentielle}\index{statistique fr\'equentielle}" en opposition à l'inf\'erence bay\'esienne (\SeeChapter{voir section de Probabilit\'es page \pageref{bayesian inference}}).
	\end{tcolorbox}	

	Lorsque nous observons un \'ev\'enement prenant en compte certains facteurs, il peut arriver qu'une deuxième observation ait lieu dans des conditions qui semblent identiques. En r\'ep\'etant ces mesures plusieurs fois sur diff\'erents objets suppos\'es similaires, nous pouvons constater que les r\'esultats observables sont distribu\'es statistiquement autour d'une valeur moyenne qui est, finalement le r\'esultat possible le plus probable. Dans la pratique, nous n'effectuons cependant parfois qu'une seule mesure et il s'agit alors de d\'eterminer la valeur de l'erreur que nous commettons en adoptant celle-ci comme moyenne mesur\'ee. Cette d\'etermination n\'ecessite de connaître le type de distribution statistique auquel nous avons affaire et c'est ce que nous allons nous attarder (entre autres) à \'etudier ici (les bases du moins!). Il existe cependant plusieurs approches m\'ethodologiques courantes (les moins courantes n'\'etant pas cit\'ees pour l'instant) face au hasard:
	\begin{enumerate}
		\item Une toute première consiste à ignorer purement et simplement les \'el\'ements al\'eatoires, pour la bonne raison que l'on ne sait pas comment les int\'egrer. Nous utilisons alors la "m\'ethode des sc\'enarios" appel\'ee aussi "simulation d\'eterministe". C'est typiquement un outil utilis\'e par les financiers ou gestionnaires non diplôm\'es travaillant avec des outils comme Microsoft Excel (qui inclut un outil de gestion de sc\'enarios) ou Microsoft Project (qui inclut un outil de sc\'enarios d\'eterministes du type optimiste, pessimiste et attendu).
		
		\item Une seconde approche envisageable, quand nous ne savons pas associer des probabilit\'es pr\'ecises aux futurs \'ev\'enements al\'eatoires, est la th\'eorie des jeux (\SeeChapter{voir section Th\'eorie des Jeux et de la D\'ecision page \pageref{game and decision theory}}) où l'on utilise des critères de s\'election semi-empiriques comme le critère du maximax, du minimax, de Laplace, de Savage, etc.
		
		\item Enfin, quand nous pouvons lier des probabilit\'es aux \'ev\'enements al\'eatoires, soit que ces probabilit\'es d\'ecoulent de calculs ou de mesures, soit qu'elles reposent sur une exp\'erience acquise auprès de situations ant\'erieures de même nature que la situation actuelle, nous pouvons faire appel aux statistiques descriptives et inf\'erentielles (contenu du pr\'esent chapitre) pour tirer des informations exploitables et pertinentes de cette masse de donn\'ees acquises.
		
		\item Une dernière approche quand nous avons connaissance de probabilit\'es relatives aux issues intervenantes faisant suite à des choix strat\'egiques est l'utilisation de la th\'eorie de la d\'ecision (\SeeChapter{voir section de Th\'eorie des Jeux et de la D\'ecision pages \pageref{extensive representation of a decision} ou \pageref{graphical strategy with probabilities}, \pageref{hurwitz criteria}, \pageref{dove hawk game probabilities} et \pageref{markov decision process}}).
	\end{enumerate}
	\begin{tcolorbox}[title=Remarques,colframe=black,arc=10pt]
	\textbf{R1.}  Sans la statistique math\'ematique, un calcul sur des donn\'ees (par exemple une moyenne), n'est qu'un "\NewTerm{indicateur ponctuel}\index{indicateur ponctuel}". C'est la statistique math\'ematique qui lui donne le statut d'estimateur dont on maîtrise le biais, l'incertitude et autres caract\'eristiques statistiques. Nous cherchons en g\'en\'eral à ce que l'estimateur soit sans biais, convergeant et efficace (nous verrons lors de notre \'etude des estimateurs plus loin de quoi il s'agit exactement).\\

	\textbf{R2.} Lorsque nous communiquons une statistique il devrait être obligatoire de pr\'eciser l'intervalle de confiance, la $p$-valeur ainsi que la taille de l'\'echantillon \'etudi\'e (statistiques absolues) et ses caract\'eristiques d\'etaill\'ees et mettre à disposition les donn\'ees sources ainsi que le protocole de mesure sinon quoi elle n'a quasiment aucune valeur scientifique (nous verrons toute ces notions en d\'etails plus loin). Une erreur courante est de communiquer en valeur relative. Par exemple sur un groupe test $1'000$ femmes, $5$ femmes mourront d'un cancer du sein sans d\'epistage, alors qu'avec d\'epistage $4$ femmes. Un peu rapidement on dira (typiquement les m\'edecins....) que le d\'epistage sauve donc $20\%$ des femmes (valeur relative). Ce qui est faux puisqu'en absolu l'avantage du d\'epistage est non significatif!\\
	
	\textbf{R3.} Si vous avez un professeur ou un formateur qui ose vous enseigner les statistiques et probabilit\'es uniquement avec des exemples bas\'es sur des jeux de hasard (cartes, d\'es, allumettes, pile ou face, etc.) d\'ebarrassez-vous en ou d\'enoncez-le. Normalement les exemples devraient être bas\'es sur l'industrie, l'\'economie ou la R\&D, bref dans des domaines utilis\'es tous les jours par les entreprises!
	\end{tcolorbox}	
	Introduisons avant de continuer quelques d\'efinitions qui vont nous être utiles pour la suite sur le concept d'\'echantillons et de moyennes (nous reviendrons plus tard sur ces d\'efinitions avec une approche plus formelle et technique!):
	
	\subsection{\'Echantillons}

	Lors de l'\'etude statistique d'ensembles d'informations, la façon de s\'electionner l'\'echantillon est aussi importante que la manière de l'analyser. Il faut que l'\'echantillon soit repr\'esentatif de la population (nous ne faisons pas n\'ecessairement r\'ef\'erence à des populations humaines!). Pour cela, l'\'echantillonnage al\'eatoire est le meilleur moyen d'y parvenir.

	\textbf{D\'efinitions (\#\mydef):}
	\begin{enumerate}
		\item[D1.] Le statisticien part toujours de l'observation d'un ensemble fini d'\'el\'ements, que nous qualifions de "\NewTerm{population}\index{population}". Les \'el\'ements observ\'es, en nombre $n$, sont tous de même nature, mais cette nature peut être fort diff\'erente d'une population à l'autre.
		
		\item[D2.] Nous sommes en pr\'esence d'un "\NewTerm{caractère quantitatif}\index{caractère quantitatif}" lorsque chaque \'el\'ement observ\'e fait explicitement l'objet d'une même mesure. À un caractère quantitatif donn\'e, nous associons une "\NewTerm{variable quantitative}\index{variable quantitative}" continue ou discrète qui synth\'etise toutes les valeurs possibles que la mesure consid\'er\'ee est susceptible de prendre (ce type d'information \'etant repr\'esent\'e par des distributions du type distribution de Gauss-Laplace, distribution bêta, distribution de Poisson, etc.).
		
		\begin{tcolorbox}[title=Remarque,colframe=black,arc=10pt]
		Nous reviendrons sur le concept de "variable" et de "distribution" un peu plus loin...
		\end{tcolorbox}	
		
		\item[D3.] Nous sommes en pr\'esence d'un "\NewTerm{caractère qualitatif}\index{caractère qualitatif}" lorsque chaque \'el\'ement observ\'e fait explicitement l'objet d'un rattachement unique à une "\NewTerm{modality}\index{modality}" choisie dans un ensemble de modalit\'es exclusives (de type: homme | femme) permettant de classer tous les \'el\'ements de l'ensemble \'etudi\'e selon un certain point de vue (ce type d'information \'etant repr\'esent\'e par des diagrammes à barre, fromages, diagrammes à bulles, etc.). L'ensemble des modalit\'es d'un caractère peut être \'etabli a priori avant l'enquête (une liste, une nomenclature, un code) ou après enquête. Une population \'etudi\'ee peut être repr\'esent\'ee par un caractère mixte, ou ensemble de modalit\'es tel que genre, tranche salariale, tranche d'âge, nombre d'enfants, situation matrimoniale par exemple pour un individu.
		
		\item[D4.] Un "\NewTerm{\'echantillon al\'eatoire}\index{\'echantillon al\'eatoire}" est un \'echantillon tir\'e au hasard dans lequel tous les individus d'une population ont la même chance, ou "\NewTerm{\'equiprobabilit\'e}\index{\'equiprobabilit\'e}" (et nous insistons sur le fait que cette probabilit\'e doit être \'egale), de se retrouver dans l'\'echantillon.
		
		\item[D5.]  Dans le cas contraire d'un \'echantillon dont les \'el\'ements n'ont pas \'et\'e pris au hasard, nous parlons alors "\NewTerm{d'\'echantillon biais\'e}\index{d'\'echantillon biais\'e}" (dans le cas inverse nous parlons "\NewTerm{d'\'echantillon non-biais\'e}\index{d'\'echantillon non-biais\'e}").
		
		\begin{tcolorbox}[title=Remarque,colframe=black,arc=10pt]
		Un petit \'echantillon repr\'esentatif est, de loin, pr\'ef\'erable à un grand \'echantillon biais\'e. Mais lorsque la taille des \'echantillons utilis\'es est petite, le hasard peut donner un r\'esultat moins bon que celui qui est biais\'e...
		\end{tcolorbox}	
	\end{enumerate}

	\subsection{Moyennes}\label{averages}
	La notion de "\NewTerm{moyenne}\index{moyenne}" ou "\NewTerm{tendance centrale}\index{tendance centrale}" (les financiers appellent cela aussi une "mesure de localisation"...) est avec la notion de "variable" à la base des statistiques.

	Cette notion nous semble très familière et nous en parlons beaucoup sans nous poser trop de questions. Pourtant il existe divers qualificatifs (nous insistons sur le fait que ce ne sont que des qualificatifs!) pour distinguer la forme de la r\'esolution d'un problème consistant à calculer la moyenne. 

	Ainsi, il faut être très très prudent quant aux calculs de moyennes car il y a une fâcheuse tendance dans les entreprises à se pr\'ecipiter et à utiliser syst\'ematiquement la moyenne arithm\'etique sans r\'efl\'echir, ce qui peut amener à de graves erreurs! Un exemple sympathique (pour faire un analogie) est qu'un nombre consid\'erable de l\'egislations exigent seulement des seuils moyens de pollution par ann\'ee alors que par exemple, fumer 1 cigarette par jour pendant 365 jours n'a pas le même impact que fumer 365 cigarettes en une journ\'ee sur une ann\'ee alors que les deux ont la même moyenne pris sur un an... C'est une preuve flagrante d'incomp\'etence statistique du l\'egislateur.

	Voici un petit \'echantillon d'erreurs courantes: 
	\begin{itemize}
		\item Consid\'erer que la moyenne arithm\'etique est la valeur qui coupe la population en deux parties \'egales (alors que c'est la m\'ediane qui fait cela).
	
		\item Consid\'erer que la moyenne de ratios du type objectifs/r\'ealis\'es est \'egale au ratio des moyennes des objectifs et des moyenn\'ees des r\'ealisations (alors que ce n'est pas la même chose!).
	
		\item Consid\'erer que la moyenne des salaires de diff\'erentes filliales est \'egale à la moyenne g\'en\'erale des salaires (alors que ceci n'est vrai que si et seulement si il y a le même nombre d'employ\'es dans chaque filliale).
	
		\item Consid\'erer que la moyenne de la moyenne des lignes d'un tableau est toujours \'egal à la moyenne des moyennes des colonnes (alors que ceci n'est vrai que si et seulement si le contenu des cellules est non vide).
	
		\item Calculer la moyenne arithm\'etique de progression de chiffres d'affaires donn\'ees en \% (alors qu'il faut utiliser la moyenne g\'eom\'etrique).
	
		\item etc.
	\end{itemize}

Nous verrons ci-dessous diff\'erentes moyennes avec des exemples relatifs à l'arithm\'etique, au d\'enombrement, à la physique, à l'\'econom\'etrie, à la g\'eom\'etrie et à la sociologie. Le lecteur trouvera d'autres exemples pratiques en parcourant l'ensemble de ce livre.

\pagebreak
\textbf{D\'efinitions (\#\mydef):} Soient $x_i$ des nombres r\'eels, nous avons alors:
\begin{enumerate}
	\item[D1.] La "\NewTerm{moyenne arithm\'etique}\index{moyenne arithm\'etique}\label{arithmetic average}" ou "\NewTerm{moyenne empirique}\index{sample average}" (la plus commun\'ement connue) est d\'efinie par le quotient de la somme des
	
	et très souvent not\'ee $\bar{x}$ ou encore $\widehat{\mu}$ est pour toute loi statistique discrète ou continue un estimateur sans biais de l'esp\'erance.

	La moyenne arithm\'etique repr\'esente donc une mesure statistique (non robuste car trop sensible aux valeurs extrêmes contrairement à la m\'ediane) exprimant la grandeur qu'aurait chacun des membres d'un ensemble de mesures si la somme doit être identique au produit de la moyenne arithm\'etique par le nombre de membres.

	Si plusieurs valeurs occurrent plus d'une fois dans les mesures, la moyenne arithm\'etique sera alors souvent not\'ee formellement:
	
	et appel\'ee "\NewTerm{moyenne pond\'er\'ee (par les effectifs)}\index{moyenne pond\'er\'ee}\label{weighted mean}". Enfin, indiquons que dans le cadre de cette d\'emarche, la moyenne pond\'er\'ee par les effectifs prendra le nom "\NewTerm{d'esp\'erance math\'ematique}\index{d'esp\'erance math\'ematique}" ou simplement "\NewTerm{esp\'erance}\index{esp\'erance}" dans le domaine d'\'etude des probabilit\'es.
	
	Nous pouvons tout aussi bien utiliser les fr\'equences d'apparition des valeurs observ\'ees (dites "\NewTerm{fr\'equence des classes}\index{fr\'equence des classes}"):
	
	De telle manière que nous obtenons une d\'efinition \'equivalent appel\'ee "\NewTerm{moyenne pond\'er\'ee par les fr\'equences de classe}\index{moyenne pond\'er\'ee par les fr\'equences de classe:}\label{weighted average by the classes frequencies}":
	
	Avant de continuer, indiquons que dans le domaine de la statistique il est souvent utile et n\'ecessaire de regrouper les mesures/donn\'ees dans des intervalles de classe de largeur donn\'ee (voir les exemples plus loin). Il faut souvent faire plusieurs essais pour cela même s'il existe des formules semi-empiriques pour choisir le nombre de classes lorsque nous avons $n$ valeurs à disposition. Une de ces règles semi-empiriques (nous parlons alros de technique de "discr\'etisation des variables") utilis\'ee par de nombreux praticiens consiste à retenir le plus petit nombre entier de classes $k$ tel que:
	
	la largeur de l'intervalle de classe \'etant alors obtenue en divisant l'\'etendue (diff\'erence entre la valeur maximale mesur\'ee et la minimale) par $k$. Soit:
	
	Par convention et en toute rigueur... (donc rarement respect\'e dans les notations), un intervalle de classe est ferm\'e à gauche et ouvert à droite (\SeeChapter{voir section Nombres page \pageref{domain of definition}}):
	
	Cette règle empirique se nomme la "\NewTerm{règle de Sturges}\index{règle de Sturges}" et est bas\'ees sur le raisonnement suivant:
	
	Nous admettons que les valeurs du coefficient binomial $C_k^i$ donnent le nombre d'individus d'un histogramme id\'eal (nous laissons le lecteur v\'erifier cela simplement avec un tableau comme Microsoft Excel 11.8346 et la fonction $\texttt{COMBIN(k,i)}$ qui y est disponible dans la version française) de $k$ intervalles pour le $i$-ème intervalle. Au fur et à mesure que $k$ devient grand l'histogramme ressemble de plus en plus à une courbe continue appel\'ee "courbe Normale" que nous verrons plus loin.

	Dès lors, en nous basant sur le th\'eorème binomial (\SeeChapter{voir section Calcul Alg\'ebrique page \pageref{binomial theorem}}), nous avons:
	
	Ensuite, pour chaque intervalle $i$ le praticien prendra par tradition la moyenne entre les deux bornes pour le calcul et la multipliera par la fr\'equence $f_i$ de classe correspondante. Dès lors, le regroupement en fr\'equence de classes fait que:
	\begin{enumerate}
		\item La moyenne pond\'er\'ee par les effectifs diffère de la moyenne arithm\'etique.
		
		\item Vue l'approximation effectu\'ee elle sera un moins bon indicateur que la moyenne arithm\'etique...
		
		\item Elle est très sensible au choix du nombre de classes donc m\'ediocre à ce niveau-là.
	\end{enumerate}
	Il existe de nombreuses autres règles empiriques de discr\'etisation des variables al\'eatoires. Le logiciel XLStat en propose par exemple pas moins de 10 (amplitude constante, algorithme de Fisher, $K$-means, 20/80, etc.).

	Plus loin, nous verrons deux propri\'et\'es extrêmement importantes de la moyenne arithm\'etique et de l'esp\'erance math\'ematique qu'il vous faudra absolument comprendre (moyenne pond\'er\'ee des \'ecarts à la moyenne et la moyenne des \'ecarts à la moyenne).

	\begin{tcolorbox}[title=Remarque,colframe=black,arc=10pt]
	Le "\NewTerm{mode}\index{mode}", not\'e \textit{Mod} ou simplement $M_0$,, est par d\'efinition la valeur qui apparaît le plus grand nombre de fois dans une s\'erie de valeurs. Dans Microsoft Excel 11.8346 (version française), soulignons que la fonction $\texttt{MODE( )}$ renvoie la première valeur dans l'ordre des valeurs ayant le plus grand nombre d'occurrences en supposant donc une distribution unimodale. Attention! La valeur modale peut être suivant les cas plus grande ou plus petite que la moyenne. Il n'y a donc pas de règle g\'en\'erale comme quoi elle sera toujours plus petite que la moyenne contrairement à ce qui est enseign\'e dans certains livres de gestion de projets.
	\end{tcolorbox}	
	
		\item[D2.] La "\NewTerm{m\'ediane}\index{m\'ediane}" ou "\NewTerm{moyenne milieu}\index{moyenne milieu}", not\'ee $M_e$ (ou plus simplement $M$ quand aucune confusion avec autre chose n'est possible...), est la valeur qui coupe une population en deux parties \'egales. Dans le cas d'une distribution statistique continue $f(x)$ d'une variable al\'eatoire $X$, il s'agit de la valeur qui repr\'esente $50\%$ de probabilit\'es cumul\'ees d'avoir lieu tel que (nous d\'etaillerons le concept de distribution statistique plus loin très en d\'etails):
		
		Dans le cas d'une s\'erie de valeurs ordonn\'ees $x_1,x_2,...,x_i,...x_n$, la m\'ediane est donc de par sa d\'efinition la valeur de la variable telle que l'on ait autant d'\'el\'ements qui ont une valeur qui lui est sup\'erieure ou \'egale, que d'\'el\'ements qui ont une valeur qui lui est inf\'erieure ou \'egale.
	\begin{tcolorbox}[title=Remarques,colframe=black,arc=10pt]
	\textbf{R1.} La m\'ediane est principalement utilis\'ee pour les distributions asym\'etriques, car elle les repr\'esente mieux que la moyenne arithm\'etique.\\
	
	\textbf{R2.} La m\'ediane n'est dans la pratique souvent pas une valeur unique (du moins dans le cas où $n$  est pair). Effectivement, entre les valeurs correspondantes aux ranges $\dfrac{n}{2}$ et $\dfrac{n}{2}+1$ il y a une infinit\'e de valeurs à choix qui coupent la population en deux.
	\end{tcolorbox}	
	Plus rigoureusement:
	\begin{itemize}
		\item Si le nombre de termes est impair, de la forme $2n + 1$, la m\'ediane de la s\'erie est le terme de rang $n + 1$ (que les termes soient tous distincts ou non!).
		
		\item Si le nombre de termes est pair, de la forme $2n$, la m\'ediane de la s\'erie est la demi-somme (moyenne arithm\'etique) des valeurs des termes de rang $n$ et $n + 1$ (que les termes soient tous distincts ou non!).
	\end{itemize}
	Dans tous les cas, de par cette d\'efinition, il d\'ecoule qu'il y a au moins $50\%$ des termes de la s\'erie inf\'erieurs ou \'egaux à la m\'ediane, et au moins $50\%$ des termes de la s\'erie sup\'erieurs ou \'egaux à la m\'ediane.

	Consid\'erons par exemple la table de salaires ci-dessous:

	\begin{table}[H]
		\centering
		\definecolor{gris}{gray}{0.85}
			\begin{tabular}{|c|c|c|c|}
				\hline
				\multicolumn{1}{c}{\cellcolor{black!30}\textbf{N\degre Employ\'e}} & 
\multicolumn{1}{c}{\cellcolor{black!30}\textbf{Salaire}} & \multicolumn{1}{c}{\cellcolor{black!30}\textbf{Cumul employ\'es}} & \multicolumn{1}{c}{\cellcolor{black!30}\textbf{\%Cumul employ\'es}}\\ \hline
		1 & 1,200 & 1 & 6\%\\ \hline
		2 & 1,220 & 2 & 12\%\\ \hline
		3 & 1,250 & 3 & 18\%\\ \hline
		4 & 1,300 & 4 & 24\%\\ \hline
		5 & 1,350 & 5 & 29\%\\ \hline
		6 & 1,450 & 6 & 35\%\\ \hline
		7 & 1,450 & 7 & 41\%\\ \hline
		8 & 1,560 & 8 & 47\%\\ \hline
\multicolumn{1}{|c|}{\cellcolor{green!30}9} & 
\multicolumn{1}{|c|}{\cellcolor{green!30}1,600} & \multicolumn{1}{|c|}{\cellcolor{green!30}9} & \multicolumn{1}{|c|}{\cellcolor{green!30}53\%}\\ \hline	
		10 & 1,800 & 10 & 59\%\\ \hline
		11 & 1,900 & 11 & 65\%\\ \hline
		12 & 2,150 & 12 & 71\%\\ \hline
		13 & 2,310 & 13 & 76\%\\ \hline
		14 & 2,610 & 14 & 82\%\\ \hline
		15 & 3,000 & 15 & 88\%\\ \hline
		16 & 3,400 & 16 & 94\%\\ \hline
		17 & 4,800 & 17 & 100\%\\ \hline
		\end{tabular}
		\caption{Identification de la m\'ediane}
	\end{table}	
	
	Il y a dans le tableau un nombre impair $2n + 1$ de valeurs. Donc la m\'ediane de la s\'erie est le terme de rang $n + 1$. Soit $1'600.-$  (r\'esultat que vous donnera n'importe quel tableur informatique). La moyenne arithm\'etique quant à elle vaut $2'020.-$.
	
	\begin{figure}[H]
		\centering
		\includegraphics[width=0.8\textwidth]{img/arithmetics/rts_moyenne_vs_mediane.jpg}
		\caption[]{Article daté de la RTS s'excusant d'avoir confondu moyenne et médiane (...)}
	\end{figure}

	En relation directe avec la m\'ediane il est important de d\'efinir le concept suivant afin de comprendre le m\'ecanisme sous-jacent:

	\item[D3.] Soit donn\'ee une s\'erie statistique $x_1,x_2,...,x_i,...,x_n$, nous appelons "\NewTerm{dispersion des \'ecarts absolus}\index{dispersion des \'ecarts absolus}" autour de $x$ le nombre $\varepsilon '(x)$ d\'efini par:
	
	$\varepsilon '(x)$ est minimum pour une valeur de $x$ la plus proche d'une valeur $x_i$ donn\'ee au sens de l'\'ecart absolu. La m\'ediane est la valeur qui r\'ealise ce minimum (extr\'emum)! L'id\'ee va alors consister à \'etudier les variations de la fonction pour trouver le rang de cet extr\'emum.

	En effet, nous pouvons \'ecrire:
	
	Donc par d\'efinition de la valeur $x$:
	
	Ce qui nous permet donc de faire sauter les valeurs absolues est simplement le choix de l'indice $r$ qui est pris de telle manière que la s\'erie de valeurs peut en pratique toujours être coup\'ee en deux parties: tout ce qui est inf\'erieur à un \'el\'ement de la s\'erie index\'e par $r$ et tout ce qui lui est sup\'erieur (la m\'ediane donc par anticipation...).

	$\varepsilon '(x)$ est donc une fonction affine (assimilable à l'\'equation d'une droite pour $r$ et $n$ fix\'es) par morceaux (discrète) où l'on peut assimiler le facteur: 
	
	à la pente et: 
	
	à l'ordonn\'ee à l'origine.
	
	La fonction est donc d\'ecroissante (pente n\'egative) tant que r est inf\'erieur à $n/2$ et croissante quand $r$ est sup\'erieur à $n/2$ (elle passe donc par un extremum!). Plus pr\'ecis\'ement, nous distinguons deux cas qui nous int\'eressent particulièrement puisque $n$ est un entier:

	\begin{itemize}
		\item Si $n$ est pair, nous pouvons poser $n=2n'$, alors la pente peut s'\'ecrire $2(r-n')$ et elle est nulle si $r=n$ et dès lors puisque ce r\'esultat n'est valable par construction que pour $\forall x \in \left[ x_r,x_{r+1}\right] $ alors $\varepsilon '(x)$ est constante sur $\left[ x_{n'},x_{n'+1}\right]$ et nous avons un extr\'emum obligatoirement au milieu de cet intervalle (moyenne arithm\'etique des deux termes).
		
		\item Si $n$ est impair, nous pouvons poser $n=2n'+1$ (nous coupons la s\'erie en deux parties \'egales), alors la pente peut s'\'ecrire $(2r-2n'-1)$ et elle est donc nulle si $r=n'+\dfrac{1}{2}$ et dès lors puisque ce r\'esultat n'est valable que pour $\forall x \in \left[ x_r,x_{r+1}\right] $ alors il est imm\'ediat que la valeur du milieu sera la m\'ediane $x_{n'+1}$.
	\end{itemize}
	Nous retrouvons donc bien la m\'ediane dans les deux cas. Nous verrons aussi plus loin comment la m\'ediane est d\'efinie pour une variable al\'eatoire continue (l'id\'ee sous-jacent \'etant exactement la même).

	Il existe un autre cas pratique où le statisticien n'a à sa disposition que des valeurs regroup\'ees sous forme d'intervalles de classes statistiques. La proc\'edure pour d\'eterminer la m\'ediane est alors diff\'erente:
	
	Lorsque nous avons à notre disposition uniquement une variable class\'ee, l'abscisse du point de la m\'ediane se situe en g\'en\'eral à l'int\'erieur d'une classe. Pour obtenir alors une valeur plus pr\'ecise de la m\'ediane, nous proc\'edons à une interpolation lin\'eaire. C'est ce que nous appelons la "\NewTerm{m\'ethode d'interpolation lin\'eaire de la m\'ediane}\index{m\'ethode d'interpolation lin\'eaire de la m\'ediane}".
	
	La valeur de la m\'ediane peut être lue sur un graphique ou calcul\'ee analytiquement. Effectivement, consid\'erons le graphique repr\'esentant la probabilit\'e cumul\'ee $F(x)$ en intervalles de classe comme ci-dessous où les bornes des intervalles ont \'et\'e reli\'ees par des droites:
	\begin{figure}[H]
		\centering
		\includegraphics{img/arithmetics/median_linear_interpolation.eps}
		\caption{Repr\'esentation graphique de l'estimation par interpolation lin\'eaire de la m\'ediane}
	\end{figure}
	La valeur de la m\'ediane $M_e$ se trouve \'evidemment au croisement entre la probabilit\'e cumul\'ee de $50\%$ ($0.5$) et l'abscisse. Ainsi, en appliquant les notions \'el\'ementaires d'analyse fonctionnelle, il vient (en observant bien \'evidemment que la pente dans l'intervalle contenant la m\'ediane est \'egale dans les demi-intervalle de gauche et de celui à droite adjacents à la m\'ediane):
	
	Ce que nous \'ecrivons fr\'equemment:
	
	D'où la valeur de la m\'ediane:
		
	Prenons le tableau suivant que nous retrouverons bien plus tard dans le pr\'esent chapitre:

	\begin{table}[H]
		\centering
		\renewcommand{\arraystretch}{1.2}
		\small
		\begin{tabular}{cccc}\hline
		Montant &  Nombre & Nombre cumul\'e & Fr\'equence relative  \\[-3pt]
des tickets & de tickets & de tickets & de tickets \\ \hline % ne pas enlever les espaces vides entre les lignes!!!
		[0,50[ & 668 & 668 & 0.068 \\

		[50,100[ & 919 & 1,587 & 0.1587 \\

		[100,150[ & 1,498 & 3,085 & 0.3085 \\

		[150,200[ & 1,915 & 5,000 & 0.5000 \\

		[200,250[ & 1,915 & 6,915 & 0.6915\\

		[250,300[ & 1,498 & 8,413 & 0.8413\\

		[300,350[ & 919 & 9,332 & 0.9332 \\

		[350,400[ & 440 & 9,772 & 0.9772 \\

		[400 et + & 228 & 10,000 & 1 \\ \hline
		\end{tabular}
		\caption{Identification de la classe m\'ediane et du mode}
	\end{table}
	Nous voyons que la "\NewTerm{classe m\'ediane}\index{classe m\'ediane}" est dans l'intervalle $[150,200]$ car la valeur cumul\'ee de $0.5$ s'y trouve (colonne toute à droite du tableau) mais la m\'ediane a elle, en utilisant la relation \'etablie pr\'ec\'edemment, pr\'ecis\'ement une valeur de (c'est trivial dans l'exemple particulier du tableau ci-dessus mais faisons quand même le calcul...):
	
	et nous pouvons faire de même avec n'importe quel autre centile bien \'evidemment!

	\textbf{D\'efinition (\#\mydef):} Nous pouvons aussi donner chemin faisant, une d\'efinition pour d\'eterminer la "\NewTerm{valeur modale}\index{valeur modale}" $M_0$ (la valeur qui survient le plus fr\'equemment) si nous seulement en possession des fr\'equences d'intervalles de classe. Pour voir cela, nous commençons avec le diagramme ci-dessous nomm\'e "\NewTerm{distribution group\'ee} \index{distribution group\'ee}" en barres de fr\'equences:
	\begin{figure}[H]
		\centering
		\includegraphics{img/arithmetics/modal_value_class_interval.eps}
		\caption{Repr\'esentation graphique de l'estimation par classess d'intervalles de la valeur modale}
	\end{figure}
	En utilisant les relations de Thalès (\SeeChapter{voir section de de G\'eom\'etrie Euclidienne page \pageref{thales theorem}}), nous avons imm\'ediatement, en notant $M_0$ la valeur modale:
	
	Comme dans une proportion, nous ne changeons pas la valeur du rapport en additionnant les num\'erateurs et en additionnant les d\'enominateurs, il vient:
	
	Nous avons alors:
	
	Avec l'exemple pr\'ec\'edent cela donne alors:
	
	La question qui se pose ensuite est celle de la pertinence du choix de la moyenne, du mode ou de la m\'ediane en termes de communication...

	Un bon exemple reste celui du march\'e du travail où de façon g\'en\'erale, alors que le salaire moyen et le salaire m\'edian sont relativement diff\'erents, les institutions de statistiques \'etatiques calculent la m\'ediane que beaucoup de m\'edias traditionnels assimilent alors explicitement au concept de "moyenne arithm\'etique" dans leurs communiqu\'es...

	\begin{tcolorbox}[title=Remarque,colframe=black,arc=10pt]
	Pour \'eviter d'obtenir une moyenne arithm\'etique ayant peu de sens, nous calculons souvent une "\NewTerm{moyenne \'elagu\'ee}\index{moyenne \'elagu\'ee}", c'est à dire une moyenne arithm\'etique calcul\'ee après avoir enlev\'e des valeurs aberrantes à la s\'erie.
	\end{tcolorbox}
	Les  "\NewTerm{quantiles}\index{quantiles}" g\'en\'eralisent la notion de m\'ediane en coupant la distribution en des ensembles donn\'es de parties \'egales (de même cardinal pourrions-nous dire...) ou autrement dit en intervalles r\'eguliers. Nous d\'efinissons ainsi les "\NewTerm{quartiles}\index{quartiles}", les "\NewTerm{d\'eciles}\index{d\'eciles}" et les "\NewTerm{centiles}\index{centiles}" (ou "percentiles" en franglais...) sur la population, ordonn\'ee dans l'ordre croissant, que nous divisons en $4$, $10$ ou $100$ parties de même effectif. Nous parlerons ainsi du centile $90$ pour indiquer la valeur s\'eparant les premiers $90\%$ de la population des $10\%$ restants.
	
	Soit $F(x)$ la fonction de distribution de probabilit\'e cumulative, le quantile est d\'efini formellement comme:
	
	
	\textbf{D\'efinition (\#\mydef):} Le "\NewTerm{$P$-ème centile}" ($0<P\leq 100$) de la liste des $N$ valeurs ordonn\'ees (tri\'ees du plus petit au plus grand) est la plus petite valeur de la liste telle que pas plus de $P$ pourcent des donn\'ees soit strictement inf\'erieur à la valeur et au moins $P$ pourcent des donn\'ees est inf\'erieur ou \'egal à cette valeur. Ceci est obtenu en calculant d'abord le rang ordinal, puis en prenant la valeur de la liste ordonn\'ee qui correspond à ce rang. Le rang ordinal $n$ est calcul\'e à l'aide de la relation (en utilisant la notation arrondie à l'entier inf\'erieur le plus proche):
	

	Pr\'ecisons que dans la version francophone de Microsoft Excel 11.8346 les fonctions \texttt{QUARTILE( )} , \texttt{CENTILE( )}, \texttt{MEDIANE( )}, \texttt{RANG.POURCENTAGE( )} sont disponibles et sp\'ecifions qu'il existe plusieurs variantes de calcul de ces centiles d'où une variation possible entre les r\'esultats sur diff\'erents logiciels\footnote{Le logiciel statistique R Statistical, par exemple, inclut les neuf m\'ethodes à travers un paramètre de sa fonction \texttt{quantile()}.} (au moins neuf m\'ethodes diff\'erentes d\'etaill\'ees dans l'excellent article \cite{HF96}!).
	\begin{tcolorbox}[title=Remarque,colframe=black,arc=10pt]
	À propos de Microsoft Excel ... et de son outil de mise en forme conditionnelle ... l'utilisateur du tableur ne doit pas confondre l'option "centile" bas\'ee sur les num\'eros de ligne des valeurs tri\'ees d'int\'erêt de celle de "pourcentage" d\'efinie pour un vecteur $\vec{X}$ de valeurs par:
	
	\end{tcolorbox}
	Ce concept est très important dans le cadre des intervalles de confiance que nous verrons beaucoup plus loin dans ce chapitre et très utile dans le domaine de la qualit\'e avec l'utilisation des boîtes à moustaches (traduction de "\NewTerm{Box \& Whiskers plots}\index{Box \& Whiskers plots}" ou "\NewTerm{box plots}\index{box plots}") permettant de comparer ("discriminer" comme disent les sp\'ecialistes) rapidement deux populations de donn\'ees ou plus et surtout d'\'eliminer les valeurs aberrantes (prendre comme r\'ef\'erence la m\'ediane sera justement plus judicieux!):
	\begin{figure}[H]
		\centering
		\includegraphics[scale=0.8]{img/arithmetics/box_plot_template.jpg}
		\caption{Box \& Whiskers Plot id\'eal}
	\end{figure}
	Une autre repr\'esentation mentale très importante des boîtes à moustache est la suivante (elle permet donc de se donner une id\'ee de l'asym\'etrie de la distribution comme le permet de faire le logiciel R):
	\begin{figure}[H]
		\centering
		\includegraphics{img/arithmetics/median_mode_quartiles_symetric.eps}
	\end{figure}
	\begin{figure}[H]
		\centering
		\includegraphics[scale=0.75]{img/arithmetics/median_mode_quartiles_asymetric.eps}
		\caption{ Repr\'esentation graphique du mode, de la m\'ediane et des quartiles par rapport à une distribution}
	\end{figure}
	Les notions de m\'ediane, valeurs ab\'errantes et intervalles de confiance que nous venons de d\'emontrer et/ou de citer sont à ce point importantes qu'il existe des normes internationales pour les utiliser correctement. Citons d'abord la norme ISO 16269-7:2001 \textit{M\'ediane - Estimation et intervalles de confiance} et aussi la norme ISO 16269-4:2010 \textit{D\'etection et traitement des valeurs aberrantes}.

		Une belle d\'emonstration sur la manière dont les gouvernements (au fait c'est les politiques!) peuvent faire mentir les statistiques (esp\'erons que la plupart d’entre eux fournissent maintenant gratuitement les donn\'ees à la population) est par exemple la R\'eserve F\'ed\'erale Am\'ericaine (FED) qui a soulign\'e le fait que le revenu moyen avait augment\'e depuis 2010. Mais quand vous avez accès aux donn\'ees brutes:
		\begin{center}
			\url{https://www.federalreserve.gov/econres/scfindex.htm}
		\end{center}
		et que vous tracez la m\'ediane par rapport à la moyenne avec, par exemple, Minitab 17, vous obtenez:
		\begin{figure}[H]
			\centering
			\includegraphics[scale=0.8]{img/arithmetics/median_vs_average.jpg}
		\end{figure}
		et après une analyse, vous pouvez constater que l’augmentation de la moyenne est due en grande partie à l’augmentation des revenus les plus \'elev\'es ($10\%$). La baisse de la m\'ediane au même moment suggère que les Am\'ericains typiques ne se portent pas aussi bien en r\'ealit\'e depuis 2010 ... Q.E.D!
		
		\begin{tcolorbox}[title=Remarque,colframe=black,arc=10pt]
		Soyez prudent avec les conclusions \'evidentes lors de l'analyse des tendances! Comme nous le verrons au cours de notre \'etude du paradoxe de Simpson (voir plus loin, page \pageref{Simpson paradox}), lorsqu'un ensemble de donn\'ees affiche une tendance, une fois qu'il est divis\'e (ie stratifi\'e) en sous-groupes, la tendance oppos\'ee peut être visible pour chaque de ces sous-groupes!
		\end{tcolorbox}
		
		\item[D3.] Par analogie avec la m\'ediane, nous d\'efinissons la "\NewTerm{m\'ediale}\index{m\'ediale}" comme \'etant la valeur (dans l'ordre croissant des valeurs) qui partage la somme (cumuls) des valeurs en deux masses \'egales (donc la somme totale divis\'ee par deux).
		
		Dans le cas de salaires, alors que le m\'ediane donne le $50\%$ des salaires se trouvant en-dessous et en-dessus, la m\'ediale donne combien de salari\'es se partagent (et donc le salaire partageant) la première moiti\'e et combien de salari\'es se partagent la seconde moiti\'e de l'ensemble des coûts salariaux.

		Par exemple pour revenir à notre tableau sur les salaires:

		\begin{table}[H]
			\centering
			\definecolor{gris}{gray}{0.85}
				\begin{tabular}{|c|c|c|c|}
					\hline
					\multicolumn{1}{c}{\cellcolor{black!30}\textbf{ N\degre Employ\'e}} & 
	\multicolumn{1}{c}{\cellcolor{black!30}\textbf{Salaire}} & \multicolumn{1}{c}{\cellcolor{black!30}\textbf{Cumul salaire}} & \multicolumn{1}{c}{\cellcolor{black!30}\textbf{\%Cumul salaire}}\\ \hline
			1 & 1,200 & 1,200 & 3.5\%\\ \hline
			2 & 1,220 & 2,420 & 7\%\\ \hline
			3 & 1,250 & 3,670 & 10.7\%\\ \hline
			4 & 1,300 & 4,970 & 14.5\%\\ \hline
			5 & 1,350 & 6,320 & 18.4\%\\ \hline
			6 & 1,450 & 7,770 & 22.6\%\\ \hline
			7 & 1,450 & 9,220 & 26.8\%\\ \hline
			8 & 1,560 & 10,780 & 31.4\%\\ \hline
			9 & 1,600 & 12,380 & 36.1\%\\ \hline
			10 & 1,800 & 14,180 & 41.3\%\\ \hline
	\multicolumn{1}{|c|}{\cellcolor{green!30}11} & 
	\multicolumn{1}{|c|}{\cellcolor{green!30}1,900} & \multicolumn{1}{|c|}{\cellcolor{green!30}16,080} & \multicolumn{1}{|c|}{\cellcolor{green!30}46.8\%}\\ \hline
	\multicolumn{1}{|c|}{\cellcolor{green!30}12} & 
	\multicolumn{1}{|c|}{\cellcolor{green!30}2,150} & \multicolumn{1}{|c|}{\cellcolor{green!30}18,230} & \multicolumn{1}{|c|}{\cellcolor{green!30}53.1\%}\\ \hline	
			13 & 2,310 & 20,540 & 59.8\%\\ \hline
			14 & 2,610 & 23,140 & 67.4\%\\ \hline
			15 & 3,000 & 26,140 & 76.1\%\\ \hline
			16 & 3,400 & 29,540 & 86\%\\ \hline
			17 & 4,800 & 34,340 & 100\%\\ \hline
			\end{tabular}
			\caption{ Identification de la m\'ediale}
		\end{table}	
		La somme de tous les salaires fait donc $34'340$ et la m\'ediale est alors $17'170$ (entre l'employ\'e num\'ero $11$ et $12$) alors que la m\'ediane \'etait de $1'600$. Nous voyons alors que la m\'ediale correspond au $50\%$ du cumul. Ce qui est un indicateur très utile dans le cadre des analyses de Pareto ou de Lorenz par exemple (\SeeChapter{voir section de Gestion Quantitative page \pageref{pareto analysis}}).
	
		\item[D4.] La "\NewTerm{moyenne quadratique}\index{moyenne quadratique}" parfois simplement not\'ee $Q$ qui est d\'efinie par:
		
		avec $m = 2$.
		\begin{tcolorbox}[title=Remarque,colframe=black,arc=10pt]
		 C'est une des moyennes les plus connues en statistiques car l'\'ecart-type est une moyenne quadratique (voir plus loin)!
		\end{tcolorbox}
		\begin{tcolorbox}[colframe=black,colback=white,sharp corners]
		\textbf{{\Large \ding{45}}Exemple:}\\\\
		Consid\'erez un carr\'e de côt\'e $a$ et un autre carr\'e de côt\'e $b$. La surface moyenne de deux carr\'es est \'egale à un carr\'e de côt\'e:
		
		\end{tcolorbox}
		Dans Microsoft Excel 11.8346, vous pouvez combiner les fonctions \texttt{SOMMECARRES( )} et \texttt{NB( )} et calculer rapidement la moyenne quadratique comme suit:
		\begin{center}
			\texttt{=(SOMMECARRES(...)/NB(...))\string^(1/NB(...))}
		\end{center}
	
		\item[D5.] La "\NewTerm{moyenne harmonique}\index{moyenne harmonique}\label{harmonic mean}" parfois simplement not\'ee $H$ est d\'efinie par:
		
		Elle est peu connue en dehors de la finance mais d\'ecoule souvent de raisonnements simples et pertinents (typiquement la r\'esistance \'equivalente d'un circuit \'electrique ayant plusieurs r\'esistances en parallèles). Il existe une fonction \texttt{HARMEAN( )} dans Microsoft Excel 11.8346 (version française) pour la calculer.
		\begin{tcolorbox}[colframe=black,colback=white,sharp corners]
		\textbf{{\Large \ding{45}}Exemple:}\\\\
		Pour introduire cette moyenne consid\'erons le cas scolaire d'une distance $d$ parcourue dans un sens à la vitesse $v_1$ et dans l'autre (ou pas) à la vitesse $v_2$. La vitesse moyenne arithm\'etique s'obtiendra en divisant la distance totale $2d$ par le temps mis à la parcourir:
		
		Si nous calculons le temps mis lorsqu'on parcourt $d$ avec une vitesse $v_i$ c'est tout simplement le quotient:
		
		Le temps total vaut donc:
		
		Si la distance n'est pas la même pour les deux vitesses, chaque vitesse reste de toute façon la même, c'est pourquoi $d$ disparaît!
		\end{tcolorbox}
		En d'autres termes: nous utilisons la moyenne harmonique quand nous sont donn\'es des ratios! Voyons d'autres exemples importants pour illustrater cela:
		
		\begin{tcolorbox}[colframe=black,colback=white,sharp corners]
		\textbf{{\Large \ding{45}}Exemples:}\\\\
		E1. Trois investissements ont un ratio Prix/Retour de respectivement $104\%$, $106\%$ et $109\%$ (donc on a perdu de l'argent dans tous les trois investissements dans ce cas particulier!). Sachant que le prix des trois investissements \'etait initialement le même, nous utilisons la moyenne harmonique:
		
		Alors qu'une moyenne arithm\'etique donnerait environ $106.33\%$. Ce qui fait une diff\'erence importante quand nous g\'erons des millions en num\'eraires!\\
		
		Remarquons au passage l'\'ecriture suivante qui montre que la moyenne harmonique est un cas particulier de la "\NewTerm{moyenne arithm\'etique pond\'er\'ee}\index{moyenne arithm\'etique pond\'er\'ee}":
		
		Si tous les poids sont tels sont \'equipond\'er\'es alors nous retrouvons la moyenne arithm\'etique standard!\\
		
		E2. Consid\'erons qu'un investisseur nous mette chaque mois à disposition $300.-$ pour acheter un actif donn\'e. Le premier mois cet actif vaut $9.-$ donc nous pouvons en acheter $33.333$ unit\'es, le deuxième mois l'actif vaut $11.-$ donc nous pouvons en acheter $27.27227$ unit\'es, enfin le dernier et le troisième mois l'actif vaut $4.-$ nous pouvons donc en acheter $75$ unit\'es. La question est alors de savoir quel est le prix moyen de cet actif dans notre portefeuille. Nous avons alors:
		
		\end{tcolorbox}
	
		\item[D6.] La "\NewTerm{moyenne g\'eom\'etrique}\index{moyenne g\'eom\'etrique}\label{geometric mean}" parfois not\'ee simplement $G$ est d\'efinie par:
		
		Cette moyenne est souvent oubli\'ee mais n\'eanmoins très connue dans le domaine de l'\'econom\'etrie (surtout quand nous \'etudierons le rendement g\'eom\'etrique moyen) et de la finance d'entreprise (\SeeChapter{voir section de Techniques de Gestion page \pageref{geometric mean for finance}}) raison pour laquelle il existe une fonction \texttt{MOYENNE.GEOMETRIQUE( )} dans Microsoft Excel 11.8346 (version française) pour la calculer.
		
		Une moyenne g\'eom\'etrique est souvent utilis\'ee lors de la comparaison de diff\'erents \'el\'ements - trouver un seul "indicateur" pour ces \'el\'ements - lorsque chaque \'el\'ement possède plusieurs propri\'et\'es ayant des plages num\'eriques diff\'erentes. Par exemple, la moyenne g\'eom\'etrique peut donner une "moyenne" significative pour comparer deux entreprises \'evalu\'ees chacune entre $0$ et $5$ pour leur durabilit\'e environnementale et entre $0$ et $100$ pour leur viabilit\'e financière. Si une moyenne arithm\'etique \'etait utilis\'ee à la place d'une moyenne g\'eom\'etrique, la viabilit\'e financière aurait plus de poids car son intervalle num\'erique est plus large, de sorte qu'un petit pourcentage de variation de la notation financière (allant de $80$ à $90$) cr\'ee une diff\'erence beaucoup plus grande dans la moyenne arithm\'etique que la variation d'un grand pourcentage dans leur durabilit\'e de l'environnement (passant par exemple de $2$ à $5$). L'utilisation d'une moyenne g\'eom\'etrique "normalise" les plages dont la moyenne est calcul\'ee, de sorte qu'aucune plage ne domine la pond\'eration, et qu'un pourcentage donn\'e de variation de l'une des propri\'et\'es a le même effet sur la moyenne g\'eom\'etrique. Ainsi, un changement de $20\%$ dans la durabilit\'e environnementale de $4$ à $4.8$ a le même effet sur la moyenne g\'eom\'etrique qu'un changement de $20\%$ de $60$ à $72$ dans viabilit\'e financière.
		
		\begin{tcolorbox}[title=Remarque,colframe=black,arc=10pt]
		En 2010, la moyenne g\'eom\'etrique a \'et\'e introduite pour calculer l'Indice de D\'eveloppement Humain par le Programme des Nations Unies pour le d\'eveloppement. Les mauvaises performances dans toutes les facteurs sont directement refl\'et\'es dans la moyenne g\'eom\'etrique. C'est-à-dire qu'une faible r\'ealisation dans une dimension n'est plus compens\'ee lin\'eairement par une performance \'elev\'ee dans une autre dimension. La moyenne g\'eom\'etrique r\'eduit le niveau de substituabilit\'e entre les dimensions tout en garantissant qu'une baisse de $1\%$ de l'indice d'esp\'erance de vie, par exemple, a le même impact sur l'IDH qu'une baisse de $1\%$ de l'\'education ou de l'indice de revenu. Ainsi, comme base de comparaison des r\'ealisations, cette m\'ethode est \'egalement plus respectueuse des diff\'erences intrinsèques entre les facteurs qu'une simple moyenne arithm\'etique.
		\end{tcolorbox}
		Comme pour avec les valeurs nulles il est impossible de calculer la moyenne g\'eom\'etrique de nombres n\'egatifs! Cependant, il existe plusieurs solutions de contournement pour ce problème, qui exigent toutes que les valeurs n\'egatives soient converties ou transform\'ees en une valeur \'equivalente positive. Le plus souvent, ce problème se pose lorsque l'on d\'esire calculer la moyenne g\'eom\'etrique d'un changement en pourcents dans une population ou un retour financier, qui peut comprendre bien \'evidemment des nombres n\'egatifs.

		Par exemple, pour calculer la moyenne g\'eom\'etrique des valeurs de $+12\%, -8\%, 0\%$ et $2\%$, nous calculerons la moyenne g\'eom\'etrique de leurs multiplicateurs \'equivalents d\'ecimaux qui sont $1.12, 0.92, 1$ et $1.02$ pour obtenir une moyenne g\'eom\'etrique de $1.0125$. Soustrayant $1$ de cette valeur donne la moyenne g\'eom\'etrique de $1.25\%$ (ou dans les milieux financiers on parlera de "\NewTerm{taux de croissance annuel compos\'e TCAC}\index{taux de croissance annuel compos\'e}").
		\begin{tcolorbox}[colframe=black,colback=white,sharp corners]
		\textbf{{\Large \ding{45}}Exemple:}\\\\
		Supposons qu'une banque offre une possibilit\'e de placement et pr\'evoit pour la première ann\'ee un int\'erêt (c'est absurde mais c'est un exemple) avec un taux  $(X-Y)\%$, mais pour la deuxième ann\'ee un int\'erêt avec un taux $(X+Y)\%$. Au même moment une autre banque offre un int\'erêt à taux constant pour deux ans: $X\%$. C'est pareil, dirons-nous un peu rapidement. En fait les deux placements n'ont pas la même rentabilit\'e.\\
	
		Dans la première banque, un capital $C_0$ donnera au bout de la première ann\'ee un int\'erêt:
		
		et la seconde ann\'ee: 
		
		Dans l'autre banque nous aurons au bout d'un an: 
		
		Dans l'autre banque nous aurons au bout d'un an: 
		
		et ainsi de suite...\\
		
		Comme vous pouvez le voir le placement ne sera pas identique si $Y\neq 0$ !$X\%$ n'est donc pas la moyenne arithm\'etique de $(X-Y)\%$ et $(X+Y)\%$.

		Posons maintenant:
		
		Quelle est en fait la valeur moyenne $r$?\\
	
		Au bout de deux ans le capital est multipli\'e par $r_1 \cdot r_2$. Si la moyenne vaut $r$ il sera alors multipli\'e par $r^2$. Nous avons donc la relation:
		
		C'est un exemple d'application où nous retrouvons donc la moyenne g\'eom\'etrique. L'oubli de l'utilisation de la moyenne g\'eom\'etrique est une erreur fr\'equente dans les entreprises lorsque certains employ\'es calculent le taux moyen d'augmentation d'une valeur de r\'ef\'erence.
		\end{tcolorbox}
		
		\item[D7.] La  "\NewTerm{moyenne mobile}\index{moyenne mobile}\label{moving average}", appel\'ee aussi "\NewTerm{moyenne glissante}\index{moyenne glissante}" d'ordre $k$ est d\'efinie sur une s\'equence $x_1,\ldots,x_i,\ldots,x_n$ de r\'ealisations par:
		
		La moyenne mobile est particulièrement utilis\'ee en \'economie, où elle permet de repr\'esenter une courbe de tendance d'une s\'erie de valeurs, dont le nombre de points est \'egal au nombre total de points de la s\'erie de valeurs moins le nombre que vous sp\'ecifiez pour la p\'eriode (pour un exemple d\'etaill\'e voir la section d'\'economie page page \pageref{simple moving average}).
		
		\begin{tcolorbox}[title=Remarque,colframe=black,arc=10pt]
		Sur certaines pages Internet et certains logiciels statistiques, la "\NewTerm{moyenne glissante}" d'ordre $k$ (devant être un nombre positif impair sup\'erieur à $2$!) Est d\'efini pour une s\'equence $x_1, \ldots, x_i, \ldots, x_n$ (avec $n \geq k$) par:
		
		\end{tcolorbox}	
		
		Une moyenne mobile en finance est calcul\'ee à partir des moyennes des cours d'une valeur, sur une p\'eriode donn\'ee: chaque point d'une moyenne mobile sur $100$ s\'eances est la moyenne des $100$ derniers cours de la valeur consid\'er\'ee. Cette courbe, affich\'ee simultan\'ement avec la courbe d'\'evolution des cours de la valeur, permet de lisser les variations journalières de la valeur, et de d\'egager des tendances.

		Les moyennes mobiles peuvent être calcul\'ees sur diff\'erentes p\'eriodes, ce qui permet de d\'egager des tendances à court terme MMC ($20$ s\'eances selon les habitudes de la branche), moyen terme ($50$-$100$ s\'eances) ou long terme MML (plus de $100$ s\'eances).
	
		\begin{figure}[H]
			\centering
			\includegraphics[scale=0.75]{img/analysis/time_serie.eps}
			\caption{Repr\'esentation graphique des quelques moyennes mobiles}
		\end{figure}
	
		Les croisements des moyennes mobiles par la courbe des cours (d\'ecoup\'ee avec une certaine granularit\'e) de la valeur g\'enèrent des signaux d'achat ou de vente (selon les professionnels) suivant le cas:
		\begin{itemize}
			\item Signal d'achat: lorsque la courbe des cours franchit la MM vers le haut.
			\item Signal de vente: lorsque la courbe des cours franchit la MM vers le bas.
		\end{itemize}
		Outre la moyenne mobile, pr\'ecisons qu'il existe une quantit\'e d'autres indicateurs artificiels souvent utilis\'es en finance comme par exemple le "\NewTerm{upside/downside ratio}\index{upside/downside ratio}".
		
		L'id\'ee est la suivante: Si vous avez un produit financier (\SeeChapter{voir section d' \'economie  page \pageref{marketable assets}}) actuellement de prix $P_c$ (prix courant) pour lequel vous avez un objectif de gain haut à un prix haut correspondant que nous noterons $P_h$ (high price) et inversement le potentiel de perte que vous estimez à un prix $P_l$ (low price).

		Alors le ratio:
		
		donne le Upside/Downside Ratio.
	
		\begin{tcolorbox}[colframe=black,colback=white,sharp corners]
		\textbf{{\Large \ding{45}}Exemples:}\\\\
		E1. Par exemple, un produit financier de $10.-$ avec un prix bas de $5.-$ et un prix haut de $15.-$ a donc un ratio $\text{UD}_R=1$ et donc un facteur sp\'eculatif identique pour permettre le gain ou une perte de $5.-$.\\
		
		E2. Un produit financier de $10.-$ avec un prix bas de $5.-$ et un prix haut de $20.-$ a donc un $\text{UD}_R=2$ donc deux fois le potentiel sp\'eculatif de gain par rapport à celui de perte.
		\end{tcolorbox}
		\begin{tcolorbox}[title=Remarque,colframe=black,arc=10pt]
		Certaines associations boursières recommandent de refuser les $\text{UD}_R$ inf\'erieurs à $3$. Les investisseurs ont tendance à rejeter les $\text{UD}_R$ trop \'elev\'es pouvant être un signe de gonflage artificiel.
		\end{tcolorbox}	
		
		\item[D8.] La "\NewTerm{moyenne pond\'er\'ee}\index{moyenne pond\'er\'ee}" (dont nous avons d\'ejà fait mention plus haut d'un cas particulier\footnote{La moyenne arithm\'etique et la moyenne mobile sont juste un cas particulier de la moyenne pond\'er\'ee où $w_i=1$}) est d\'efinie par:
		
		et est utilis\'ee par exemple en g\'eom\'etrie pour localiser le barycentre d'un polygone, en physique pour d\'eterminer le centre de gravit\'e ou en statistiques pour calculer une esp\'erance (le d\'enominateur \'etant toujours \'egal à l'unit\'e en probabilit\'es) et en gestion de projets pour estimer les dur\'ees des tâches.
		
		Dans le cas g\'en\'eral le poids $w_i$ repr\'esente l'influence pond\'er\'ee ou arbitraire/empirique de l'\'el\'ement $x_i$ par rapport aux autres.
		
		\item[D9.] La "\NewTerm{moyenne fonctionnelle}\index{moyenne fonctionnelle}" ou "\NewTerm{moyenne int\'egrale}\index{moyenne int\'egrale}\label{integral average}" est d\'efinie par:
		
		où $\mu_f$ d\'epend d'une fonction $f$ d'une variable r\'eelle int\'egrable (\SeeChapter{voir section de Calcul Int\'egral et Diff\'erentiel page \pageref{integral calculus}}) sur un intervalle $[a,b]$. Elle est très souvent utilis\'ee en th\'eorie du signal (\'electronique, \'electrotechnique).
	\end{enumerate}
	
	\pagebreak
	\subsubsection{Lissage de Laplace}
	Pour en revenir à nos fr\'equences de classes vues bien plus haut et avant de continuer avec l'\'etude de quelques propri\'et\'es math\'ematiques des moyennes... il faut savoir que lorsque nous travaillons avec des lois discrètes de probabilit\'es il arrive très (très) fr\'equemment que nous rencontrions un problème typique dont la source est la taille de la population.

	Consid\'erons comme exemple le cas où nous avons $12$ documents et que nous souhaiterions estimer la probabilit\'e d'occurrence du mot "Viagra". Nous avons sur un \'echantillon les valeurs suivantes:

	\begin{table}[H]
		\centering
		\definecolor{gris}{gray}{0.85}
		\begin{tabular}{|c|c|}
		\hline
\multicolumn{1}{c}{\cellcolor{black!30}\textbf{ID Document}} & \multicolumn{1}{c}{\cellcolor{black!30}\textbf{Occurrences du mot}} \\ \hline
		1 & 1 \\ \hline
		2 & 0 \\ \hline
		3 & 2 \\ \hline
		4 & 0 \\ \hline
		5 & 4 \\ \hline
		6 & 6 \\ \hline
		7 & 3 \\ \hline
		8 & 0 \\ \hline
		9 & 6 \\ \hline
		10 & 2 \\ \hline	
		11 & 0 \\ \hline
		12 & 1 \\ \hline
		\end{tabular}
		\caption[]{Fr\'equences de classe du mot}
	\end{table}

	Tableau que nous pouvons repr\'esenter d'une autre manière:

	\begin{table}[H]
		\centering
		\definecolor{gris}{gray}{0.85}
		\begin{tabular}{|c|c|c|}
		\hline
\multicolumn{1}{c}{\cellcolor{black!30}\textbf{Occurrences du mot}} & \multicolumn{1}{c}{\cellcolor{black!30}\textbf{Documents}} & \multicolumn{1}{c}{\cellcolor{black!30}\textbf{Probabilit\'e}} \\ \hline
		0 & 4 & 0.33\\ \hline
		1 & 2 & 0.17\\ \hline
		2 & 2 & 0.17\\ \hline
		3 & 1 & 0.083 \\ \hline
		4 & 1 & 0.083\\ \hline
		5 & 0 & 0\\ \hline
		6 & 2 & 0.17\\ \hline
		\end{tabular}
		\caption[]{Fr\'equences de classe respective des documents}
	\end{table}
	Et ici nous avons un ph\'enomène courant. Il n'y a aucun document avec 5 occurrences du mot qui nous int\'eresse. L'id\'ee (très courante dans le domaine du Data Mining) est alors d'ajouter artificiellement et empiriquement un comptage en utilisant une technique appel\'ee "\NewTerm{lissage de Laplace}\index{lissage de Laplace}" qui consiste à additionner k unit\'es à chaque occurrence. Dès lors le tableau devient:
	\begin{table}[H]
		\centering
		\begin{tabular}{|c|c|c|}
				\hline
\multicolumn{1}{c}{\cellcolor{black!30}\textbf{Occurrences du mot}} & \multicolumn{1}{c}{\cellcolor{black!30}\textbf{Documents}} & \multicolumn{1}{c}{\cellcolor{black!30}\textbf{Probabilit\'e}} \\ \hline
		0 & 5 & 0.26\\ \hline
		1 & 3 & 0.16\\ \hline
		2 & 3 & 0.16\\ \hline
		3 & 2 & 0.11 \\ \hline
		4 & 2 & 0.11\\ \hline
		5 & 1 & 0.05\\ \hline
		6 & 3 & 0.16\\ \hline
		\end{tabular}	
		\caption[]{Fr\'equences de classes des documents avec lissage}
	\end{table}
	\'evidemment ce type de technique est sujet à d\'ebat et sort du cadre scientifique... Nous avons même h\'esit\'e à pr\'esenter cette technique dans le chapitre de M\'ethodes Num\'eriques (avec le reste de toutes les techniques num\'eriques empiriques)...

	\subsubsection{Propri\'et\'es des moyennes}\label{means and averages properties}
	
	Voyons maintenant quelques propri\'et\'es pertinentes qui relient quelques-unes de ces moyennes ou qui sont propres à une moyenne donn\'ee.

	\label{properties of the mean}Les premières propri\'et\'es sont importantes donc prenez garde à bien les comprendre:
	\begin{enumerate}
		\item[P1.] Le calcul des moyennes arithm\'etique, quadratique et harmonique peut être g\'en\'eralis\'e à l'aide de la relation suivante:
		
		où nous retrouvons:
		\begin{enumerate}
			\item Pour $m=1$, la moyenne arithm\'etique
			\item Pour $m=2$, la moyenne quadratique
			\item Pour $m=-1$, la moyenne harmonique
		\end{enumerate}
		
		\item[P2.] La moyenne arithm\'etique a une propri\'et\'e de lin\'earit\'e, c'est-à-dire que (sans d\'emonstration car simple à v\'erifier):
		
		C'est la version statistique de la propri\'et\'e de l'esp\'erance en probabilit\'e que nous verrons plus loin (l'esp\'erance est lin\'eaire que les variables a\'eatoires soient d\'ependantes ou non!).
		
		\item[P3.] La somme pond\'er\'ee des \'ecarts à la moyenne arithm\'etique est nulle.
		\begin{dem}
		D'abord, par d\'efinition, nous savons que:
		
		Nous avons alors:
		\thickmuskip=0mu
		\medmuskip=0mu
		
		\thickmuskip=3mu
		\medmuskip=3mu
		Ainsi, cet outil ne peut être utilis\'e comme mesure de dispersion!

		Par extension la moyenne arithm\'etique des \'ecarts pond\'er\'es à la moyenne par les effectifs est nulle aussi:
		
		\begin{flushright}
			$\blacksquare$  Q.E.D.
		\end{flushright}
		\end{dem}
		Ce r\'esultat est relativement important car il permettra plus loin de mieux saisir le concept d'\'ecart-type et de variance.
				
		\item[P4.] Soit à d\'emontrer:
		
		\begin{tcolorbox}[title=Remarque,colframe=black,arc=10pt]
		Les comparaisons entre les moyennes ci-dessus et la m\'ediane ou les moyennes pond\'er\'ees ou mobiles n'ont pas de sens, c'est pourquoi nous ne les comparerons pas!
		\end{tcolorbox}
		\begin{dem}
		Tout d'abord, nous prenons deux nombres r\'eels non nuls $x_1$ et $x_2$ tels que $x_2>x_1>0$ et nous \'ecrivons:
		\begin{enumerate}
			\item La moyenne arithm\'etique:
			
			
			\item La moyenne g\'eom\'etrique:
			
			
			\item La moyenne harmonique:
			
			
			\item La moyenne quadratique:
						
		\end{enumerate}
		Prouvons d\'ejà que $\mu_g>\mu_h$ par l'absurde en posant $\mu_g-\mu_h<0$:
		
		Par commodit\'e posons:
		
		Et nous savons que $y>1$. Or:
		
		Et nous cherchons alors à montrer que:
		
		n'est pas possible. Mais ceci d\'ecoule des \'equivalences suivantes:
		 
		Il y a donc contradiction ce qui v\'erifie notre hypothèse initiale:
		
		Regardons maintenant si $\mu_g>\mu_a$.
			
		Sous l'hypothèse $x_2>x_1>0$. Nous cherchons donc maintenant à montrer que:
		
		Or nous avons les \'equivalences suivantes:
		
		et la dernière expression est \'evidemment correcte.
		
		Or le carr\'e d'un nombre (r\'eel) est toujours positif ce qui v\'erifie notre hypothèse initiale:
		
		Nous allons d\'emontrer maintenant $\mu_q > \mu_a$ par l'absurde en posant $\mu_q-\mu_a<0$:
		
		Or le carr\'e d'un nombre (r\'eel) est toujours positif ce qui v\'erifie notre hypothèse initiale:
		
		Nous avons donc bien:
		
		\begin{flushright}
			$\blacksquare$  Q.E.D.
		\end{flushright}
		\end{dem}
		Il est important de remarquer ici (car c'est une erreur fr\'equente dans les entreprise et administrations) que la moyenne g\'eom\'etrique est inf\'erieure à la moyenne arithm\'etique. Ce qui suivant les cas pratiques peut être une erreur de confusion favorable ou d\'efavorable à la personne qui utilise la moyenne arithm\'etique en lieu et place de la moyenne g\'eom\'etrique.

		Ces in\'egalit\'es d\'emontr\'ees, nous pouvons alors passer à une figure que nous attribuons à Archimède pour placer trois de ces moyennes. L'int\'erêt de cet exemple est de montrer qu'il existe des relations remarquables parfois entre la statistique et la g\'eom\'etrie (fruit du hasard ???).
			
		\begin{figure}[H]
			\centering
			\includegraphics[scale=0.75]{img/arithmetics/averages.eps}
			\caption{Point de d\'epart pour la repr\'esentation g\'eom\'etrique desd diff\'erentes moyennes}
		\end{figure}
		Nous allons d'abord poser $a=\overline{AB},b=\overline{BC}$ et $\text{O}$ est le milieu de $\overline{AC}$ . Ainsi, le cercle dessin\'e $\Omega$ est de centre $\text{O}$ et de rayon $\overline{\text{O}A}$. $D$ est l'intersection de la perpendiculaire à $\overline{AC}$ passant par $B$ et du cercle $\Omega$ (nous choisissons l'intersection que nous voulons). $H$ est quant à lui le projet\'e orthogonal de $B$ sur $\overline{OD}$.
		
		Archimède affirme que $\overline{\text{O}A}$ est la moyenne arithm\'etique de $a$ et $b$ et que $\overline{BD}$ est la moyenne g\'eom\'etrique de $a$ et $b$, et $\overline{DH}$ est la moyenne harmonique de $a$ et $b$.
		
		Nous d\'emontrons donc que (trivial):
		
		Donc $\overline{\text{O}A}$ est bien la moyenne arithm\'etique $\mu_a$ de $a$ et $b$.
		
		Ensuite nous avons dans le triangle rectangle $ADB$:
		
		Puis dans le triangle rectangle $BDC$ nous avons:
		
		Nous additionnons alors ces deux \'egalit\'es, et nous trouvons:
		
		Nous savons que $D$ est sur un cercle de diamètre $\overline{AC}$, donc $ADC$ est rectangle en $D$. Alors:
		
		Puis nous remplaçons $\overline{BA}$ et $\overline{BC}$ par $a$ et $b$:
			
		Donc finalement:
		
		Et donc, $\overline{DB}$ est bien la moyenne g\'eom\'etrique $\mu_g$ de $a$ et $b$.
		
		Nous reste à prouver alors que $\overline{DH}$ est la moyenne harmonique de $a$ et $b$. Nous avons dans un premier temps en utilisant la projection orthogonale (\SeeChapter{voir section de Calcul Vectoriel page \pageref{orthogonal projection vector}}):
		
		Puis nous avons \'egalement (projection orthogonale aussi):
		
		Nous avons donc:
		
		et comme ${DB}=\sqrt{ab}$, nous avons donc:
			
		$\overline{DH}$ est donc bien la moyenne harmonique de $a$ et $b$. Archimède ne s'\'etait pas tromp\'e!
	\end{enumerate} 
	
	\paragraph{In\'egalit\'e de Jensen}\mbox{}\\\\
	"\NewTerm{L'in\'egalit\'e de Jensen}\index{in\'egalit\'e de Jensen}\label{jensen inequality}" est une relation (ou propri\'et\'e) particulièrement importante dans le secteur de la finance et de l'assurance, car elle montre pourquoi le vendeur d'options cacule l'esp\'erance du pay-off plutôt que le pay-off de l'esp\'erance ou pourquoi, au bout du compte, l’assur\'e paiera toujours un montant sup\'erieur ou \'egal au diff\'erentiel de l'esp\'erance des coûts d’accidents, ainsi que dans l’analyse chronologique pourquoi les processus GARCH sont naturellement leptokurtic et aussi dans la preuve de la divergence de Kullback-Leibler pourquoi cette dernière est strictement positive ou nulle (\SeeChapter{voir section M\'ecanique Statistiques page \pageref{kullback-leibler divergence}}).

	\begin{theorem}
	L'in\'egalit\'e de Jensen nous permettra de prouver que pour une fonction convexe (\SeeChapter{voir section Analyse Fonctionelle page \pageref{convex function}}):
	
	avec $p_i>0$, $\sum_i p_i=1$. Soit en utilisant des notations sp\'ecifiques au domaine de la statistique:
	
	et à la valorisation des options (\SeeChapter{voir section d'\'economie page \pageref{jensen inequality finance}}):
	
	\end{theorem}
	Avant de poursuivre, rappelons qu'une fonction $f$ est concave si $ -f $ est convexe et inversement. C'est alors imm\'ediat que:
	
	Soit en utilisant des notations sp\'ecifiques au domaine de la statistique:
	
	Passons à maintenant à la d\'emonstration, puis ensuite nous ferons un petit exemple pratique simplifi\'e avec des d\'eriv\'es en finance!
	\begin{dem}
	Voyons si:
	
	est vrai en proc\'edant par induction. D'abord pour $ m = 2 $, nous retombons sur la d\'efinition classique d'une fonction convexe:
	
	Ou autrement \'ecrit puisque $p_1+p_2=1$:
	
	Maintenant, supposons la relation pr\'ec\'edente vraie pour $ m = n $, et montrons que c'est \'egalement vrai pour $ m = n + 1 $:
	
	Puisque $ f $ est convexe, on peut \'ecrire:
	
	Maintenant posons:
	
	pour $i=1\ldots n$. Nous avons $q_i>0$ pour tout $i=1\ldots n$ et:
	
	parce que pour rappel:
	
	et par cons\'equent:
	
	Par hypothèse de r\'ecurrence nous avons alors:
	
	Et donc:
	
	en simplifiant l'expression à droite on obtient bien:
	
	\begin{flushright}
		$\blacksquare$  Q.E.D.
	\end{flushright}
	\end{dem}
	\begin{tcolorbox}[colframe=black,colback=white,sharp corners]
	\textbf{{\Large \ding{45}}Exemple:}\\\\
	Nous savons que les pay-off d'un Call (option d'achat) et d'un Put (option de vente) sont respectivement à l'\'ech\'eance (du point de vue de l'acheteur et donc par sym\'etrie du point de vue du vendeur \'egalement en valeur absolue):
	
	Consid\'erons un prix à l'\'ech\'eance de $102$ .- et que les strikes possibles à l'\'ech\'eance sont de $\{100,110,150 \} $, nous avons alors:
	
	et:
	
	et nous avons donc bien pour un Call du point de vue d'un acheteur ou d'un vendeur en valeurs absolues:
	
	\end{tcolorbox}

	\pagebreak
	\subsection{Types de Variables}
	Lorsque nous avons parl\'e des \'echantillons au d\'ebut de ce chapitre, nous avons fait mention de deux types d'informations: les variables quantitatives et qualitatives. Nous n'avons cependant pas pr\'ecis\'e qu'il existait trois types de variables quantitatives très importantes qu'il convient absolument de diff\'erencier:.

	\textbf{D\'efinitions (\#\mydef):}
	\begin{enumerate}
		\item[D1.] Les "\NewTerm{variables discrètes}\index{variables discrètes}" (de comptage) appartenant à $\mathbb{Z}$: Elles sont analys\'ees avec des lois statistiques bas\'ees le plus souvent sur un domaine de d\'efinition d\'enombrable strictement positif (distribution de Poisson ou Hypergeoemetric) sont un cas typique dans l’industrie). Elles sont presque toujours repr\'esent\'es graphiquement par des histogrammes.

		\item[D2.] Les "\NewTerm{variables continues}\index{variable continue}" (par mesure) appartenant à $\mathbb{R}$: Elles sont analys\'es avec des lois statistiques bas\'ees sur un domaine  de d\'efinition non-d\'enombrable strictement positif ou peuvent prendre toute valeur n\'egative ou positive (g\'en\'eralement comme la distribution Normale dans l'industrie). Elles sont presque toujours repr\'esent\'es graphiquement par des histogrammes avec des intervalles de classe.

		\item[D3.] Les "\NewTerm{variables d'attribut}\index{variable d'attribut}" (par classification): ce ne sont pas des donn\'ees num\'eriques en tant que telles (uniquement lorsqu'elles sont cod\'ees à l'aide de chiffres!), Mais un type de donn\'ees qualitatif \{Oui, Non\}, \{Transmis, \'echec\} , \{À l'heure, en retard\}, \{rouge, vert bleu, noir\}, etc. L'attribut de type de donn\'ees binaire suivent typiquement une distribution de Bernoulli ou Binomiale tandis que les variables qualitatives d'ordre sup\'erieur n'ont pas de moyenne ou d'\'ecart type (en fait, essayez de calculer la moyenne et la norme \'ecart entre les variables qualitatives Rouge, Vert et Rose ...).
	
		Dans la famille des variables d'attributs, nous distinguons principalement deux sous-types de variables:
		\begin{enumerate}
			\item Une "\NewTerm{variable cat\'egorielle}\index{variable cat\'egorielle}" (parfois appel\'ee "\NewTerm{variable nominale}\index{variable nominale}") est une variable qui comporte deux cat\'egories ou plus, mais où il n'y a pas d'ordre intrinsèque pour les cat\'egories. Par exemple, le genre est une variable cat\'eroielle ayant deux cat\'egories (homme et femme) et il n’existe aucun ordre intrinsèque pour les cat\'egories. La couleur des cheveux est \'egalement une variable cat\'egorielle comportant un certain nombre de cat\'egories (blonde, brune, brune, rouge, etc.) et là encore, il n’existe aucun moyen convenu de les classer du plus haut au plus bas. Une variable purement cat\'egorielle est une variable qui vous permet simplement d’affecter des cat\'egories mais vous ne pouvez pas clairement les classer. Si la variable a un ordre clair, alors cette variable serait une variable ordinale, comme d\'ecrit ci-dessous.
			
			\item Une "\NewTerm{variable ordinale}\index{variable ordinale}" est similaire à une variable cat\'egorielle. La diff\'erence entre les deux est qu’il existe un ordre clair des variables. Par exemple, supposons que vous ayez une variable, nomm\'ee \textit{statut \'economique}, avec trois cat\'egories: \{bas, haut, moyen\}. En plus de pouvoir classer les personnes dans ces trois cat\'egories, nous pouvez les classer comme suit: bas (1) $<$ moyen (2) $<$ haut (3). Consid\'erons maintenant une variable telle que \textit{niveau d'\'education} (avec des valeurs telles que: \{diplôm\'e universitaire, diplôm\'e du secondaire, diplôm\'e du primaire,  un peu d'universit\'e\}). Ceux-ci peuvent \'egalement être command\'es comme: diplôm\'e du primaire (1) $<$ diplôm\'e du secondaire (2) $<$ un peu d'universit\'e (3) $<$ diplôm\'e universitaire (4).
		\end{enumerate}
	\end{enumerate}
	\begin{figure}[H]
		\centering
		\includegraphics[scale=0.75]{img/arithmetics/type_of_variables.jpg}	
		\caption{Types de variables en statistiques}
	\end{figure}
	Comprendre les diff\'erents types de donn\'ees est une discipline importante pour les ing\'enieurs car cela a des implications importantes pour le type d'outils et de techniques d'analyse qui seront utilis\'es.

	Une question commune concernant la collecte de donn\'ees est la suivante: quelle est la quantit\'e à collecter? En fait, cela d\'epend du niveau de pr\'ecision souhait\'e. Nous verrons beaucoup plus loin dans cette section (avec d\'emonstration math\'ematique à l'appui!) comment d\'eterminer math\'ematiquement la quantit\'e de donn\'ees à collecter.

	Maintenant que nous connaissons relativement bien le concept de moyenne et de variable, nous pouvons discuter de calculs plus formels qui feront sens!

	\subsubsection{Variables discrètes et Moments}
	
	Soit $X$ une variable ind\'ependante (un individu d'un \'echantillon dont la propri\'et\'e est ind\'ependante des autres individus) qui peut prendre les valeurs al\'eatoires discrètes  (r\'ealisations du vecteur $(X_1,X_2,...,X_n)$) avec les probabilit\'es respectives $(p_1,p_2,...,p_n)$ où, de par l'axiomatique des probabilit\'es (\SeeChapter{voir section Probabilit\'es page \pageref{kolmogorov axioms}}):
	
	Soit $X$ une variable al\'eatoire num\'erique (quantitative) (v.a.). Dans la pratique, la plupart du temps, elle sera entièrement d\'ecrite par la valeur de sa probabilit\'e associ\'ee (pour les variables discrètes) pour sa r\'ealisation ou par:
	
	\textbf{D\'efinitions (\#\mydef):}
	\begin{enumerate}
		\item[D1.] La probabilit\'e correspondante à sa valeur, nomm\'ee "\NewTerm{fonction de densit\'e de probabilit\'e}\footnote{Pour les variables discrètes certains utilisents à la place la terminologie suivante:" \NewTerm{fonction de masse de probabilit\'e} "}\index{fonction de densit\'e de probabilit\'e}" (PMF) et not\'e par:
		
		aussi parfois (et plus correctement) not\'e:
		
		en gardant à l'esprit que $\sum f_X(x)$ doit être \'egal à $1$.
		
		\item[D2.] La probabilit\'e cumulative (pour les variables discrètes ET continues) d'être g\'en\'eralement inf\'erieure ou \'egale à $X$ pour toutes les r\'ealisations $x$. Cette "\NewTerm{fonction de distribution cumulative}\index{fonction de distribution cumulative}" (F.D.C.), aussi souvent appel\'ee "\NewTerm{fonction de r\'epartition}\index{fonction de r\'epartition}", est not\'ee par:
		
		aussi parfois (et plus correctement) not\'e (pour rappel $D$ est le domaine de d\'efinition tel que d\'efini à la page \pageref{domain of definition} dans la section d'Analyse Fonctionnelle):
		
		avec:
		
		où $F(x)$ s'appelle la "fonction de r\'epartition" de la variable $X$. C'est la proportion th\'eorique de la population consid\'er\'ee dont la valeur est inf\'erieure ou \'egale à $x$. Il s'ensuit:
		
		Plus g\'en\'eralement, pour toute paire de nombres $a$ et $b$ avec $a<b$, nous avons:
		
		\begin{figure}[H]
			\centering
			\includegraphics{img/arithmetics/pdf_cdf.jpg}
		\end{figure}
		Cette d\'efinition s'applique bien \'evidemment aussi aux variables continues.
	
		\item[D3.] La "\NewTerm{fonction de r\'epartition empirique}\index{fonction de r\'epartition empirique}" (F.R.E.) est quant à elle d\'efinie naturellement par (nous avons indiqu\'e les diff\'erentes notations courantes dans la litt\'erature):
		
		avec $\hat{F}_n(x)\in [0,1]$ associ\'e à l'\'echantillon de variables al\'eatoires ind\'ependantes et identiquement distribu\'ees (ce que l'on nomme aussi un "vecteur al\'eatoire" not\'e $\vec{X}=(x_1,x_2,...,x_n)$).
		
		Il s'agit simplement du cumul normalis\'e à l'unit\'e des fr\'equences d'apparition en-dessous d'un certaine valeur fix\'ee (d\'emarche que la majorit\'e des êtres humains font naturellement en cherchant la fonction de r\'epartition).
		
		Donc si nous reprenons l'exemple de salaires, vu plus haut, nous avons alors par exemple pour $x$ fix\'e à $1'800$:
		\begin{table}[H]
		\centering
		\definecolor{gris}{gray}{0.85}
				\begin{tabular}{|c|c|}
					\hline
	\multicolumn{1}{c}{\cellcolor{black!30}\textbf{Salaires ordonn\'ees $x_i \leq x$}} & \multicolumn{1}{c}{\cellcolor{black!30}\textbf{Fr\'equences $1_{x_i \leq x}$}} \\ \hline
			1,200 & 1 \\ \hline
			1,220 & 1 \\ \hline
			1,250 & 1 \\ \hline
			1,300 & 1 \\ \hline
			1,350 & 1 \\ \hline
			1,450 & 1 \\ \hline
			1,450 & 1 \\ \hline
			1,560 & 1 \\ \hline
			1,600 & 1 \\ \hline
			1,800 & 1 \\ \hline	
			1,900 & 0 \\ \hline
			2,150 & 0 \\ \hline
			2,310 & 0 \\ \hline
			2,600 & 0 \\ \hline
			3,000 & 0 \\ \hline
			3,400 & 0 \\ \hline
			4,800 & 0 \\ \hline
		\end{tabular}
		\caption[]{Exemple de la fonction de r\'epartition empirique}
		\end{table}
		et donc:
		
		La fonction de r\'epartition est clairement une fonction monotone croissante (ou plus pr\'ecis\'ement "non d\'ecroissante") dont les valeurs vont de $0$ à $1$.
		
		Voyons une propri\'et\'e que nous utiliserons plusieurs fois et qui est importante dans la pratique (propri\'et\'et\'e valable aussi bien pour les variables al\'eatoires discrètes que continues):

		Soit $X$ une variable al\'eatoire de fonction de r\'epartition $F_{X}$ et de fonction de densit\'e $f_{X}$ (rappel: $f_{X}=F_{X}^{\prime}$) et $c^{te} \in \mathbb{R}^{*}$ une constante tel que $y=c^{te}X$. Nous voulons d\'eterminer la fonction de r\'epartition et la fonction de densit\'e de la va. $Y$ à partir de $F_{X}$ et $f_{X}$ respectivement.
		
		Par d\'efinition pour $y \in \mathbb{R}$ :
		
		Ainsi:
		
		En ce qui concerne la fonction de densit\'e nous avons:
		
		Ainsi:
		
	\end{enumerate}
	
	\paragraph{Esp\'erance et Variance de variables al\'eatoires discrètes}\mbox{}\\\\
	\textbf{D\'efinition (\#\mydef):} Nous d\'efinissons "\NewTerm{l'esp\'erance math\'ematique}\index{l'esp\'erance math\'ematique}", appel\'ee aussi "\NewTerm{moment d'ordre $1$}\index{moment d'ordre $1$}", de la variable al\'eatoire $X$ par la relation (avec diff\'erentes notations courantes)\label{expected mean discrete variable}:
	
	appel\'ee aussi parfois "\NewTerm{règle des parties}\index{règle des parties}".
	
	En d'autres termes, nous savons qu'à chaque \'ev\'enement de l'espace des \'echantillons est associ\'e une probabilit\'e à laquelle nous associons \'egalement une valeur (donn\'ee par la variable al\'eatoire). La question \'etant alors de savoir quelle valeur, à long terme, nous pouvons obtenir? La valeur esp\'er\'ee, (l'esp\'erance math\'ematique donc...) est alors la moyenne pond\'er\'ee, par la probabilit\'e, de toutes les valeurs des \'ev\'enements de l'espace des \'echantillons.

	Si la probabilit\'e est donn\'ee par une fonction de distribution discrète  $f(x_i)$ (voir les d\'efinitions des fonctions de distribution plus bas) de la variable al\'eatoire, nous avons: 
	

	\begin{tcolorbox}[title=Remarques,colframe=black,arc=10pt]
	\textbf{R1.} L'esp\'erance $\mu_X$ peut être not\'ee $\mu$ s'il n'y pas de confusion possible.\\

	\textbf{R2.} Si nous consid\'erons chaque r\'ealisation de la variables al\'eatoire $(x_1,x_2,...,x_n)$ comme les composantes d'un vecteur $\vec{X}$ et chaque probabilit\'e (ou pond\'eration)  $(p_1,p_2,...,p_n)$  comme les composantes d'un vecteur $\vec{p}$  alors nous pouvons \'ecrire l'esp\'erance de manière technique sous la forme d'un produit scalaire souvent not\'e:
	
	\end{tcolorbox}	
	Voici les propri\'et\'es math\'ematiques les plus importantes de l'esp\'erance pour toute variable al\'eatoire (quelle que soit sa loi de distribution!) ou pour toute s\'erie de variables al\'eatoires et que nous utiliserons souvent tout au long de cette section (et de nombreuses autres sections faisant usage des statistiques):
	\begin{enumerate}
		\item[P1.] Multiplication par une constante:
		
			
		\item[P2.] Somme de deux variables al\'eatoires (ind\'ependantes ou non!):
		
		où nous avons utilis\'e dans la 4ème ligne, la propri\'et\'e vue dans le chapitre de Probabilit\'es (page \pageref{disjoint probability}):
		
		Nous en d\'eduisons que pour $n$ variables al\'eatoires equation, non n\'ecessairement d\'efinies sur une même loi de distribution:
		
			
		\item[P3.] L'esp\'erance d'une constant $a$ est \'egale à la constante elle-même:
		
	
		\item[P4.] Esp\'erance du produit de deux variables al\'eatoires:
		
		Et si les deux variables al\'eatoires sont ind\'ependantes, alors la probabilit\'e conjointe est \'egale au produite des probabilit\'es (\SeeChapter{voir section de Probabilit\'es page \pageref{joint probability}}). Il vient alors:
		
	\end{enumerate}
	Donc l'esp\'erance du produit de variables al\'eatoires ind\'ependantes est toujours \'egal au produit des esp\'erances.

	\'evidemment, nous supposerons comme \'evident que ces quatre propri\'et\'es s'\'etendent au cas continu!

	\textbf{D\'efinition (\#\mydef):} Après avoir traduit la tendance par l'esp\'erance il est int\'eressant de traduire la dispersion ou "\NewTerm{d\'eviation standard}\index{d\'eviation standard}" autour de l'esp\'erance par une valeur appel\'ee "\NewTerm{variance de $X$}\index{variance}" ou encore "\NewTerm{moment centr\'e du deuxième ordre}\index{moment centr\'e du deuxième ordre}", not\'ee $\text{V}(X)$ ou $\sigma_X^2$ (lire "sigma-deux") et donn\'ee sous sa forme discrète par:
	
	La variance n'est cependant pas comparable directement à la moyenne, car l'unit\'e de la variance est le carr\'e de l'unit\'e de la variable al\'eatoire, ce qui d\'ecoule directement de sa d\'efinition. Pour que l'indicateur de dispersion puisse être compar\'e aux paramètres de tendance centrale (moyenne, m\'ediane et... mode), il suffit alors de prendre la racine carr\'ee de la variance.

	Par commodit\'e, nous d\'efinissons ainsi "\NewTerm{l'\'ecart-type}" de $X$ par
	
	L'\'ecart-type est donc la moyenne quadratique des \'ecarts (ou "\'ecart moyen quadratique") entre les observations et leur moyenne.
	\begin{tcolorbox}[title=Remarques,colframe=black,arc=10pt]
	\textbf{R1.} L'\'ecart-type $\sigma_X$ de la variable al\'eatoire $X$ peut être not\'e $\sigma$ s'il n'y pas de confusion possible.\\

	\textbf{R2.} L'\'ecart-type et la variance sont, dans la litt\'erature, souvent appel\'es "\NewTerm{paramètres de dispersion}\index{paramètres de dispersion}" à l'oppos\'e de la moyenne, mode et m\'ediane qui sont appel\'es des "\NewTerm{paramètres de position}\index{paramètres de position}".
	\end{tcolorbox}	

	\textbf{D\'efinition (\#\mydef):} Le ratio (exprim\'e en \%):
	
 	souvent utilis\'e dans les entreprises comme comparaison de la moyenne et de l'\'ecart-type est appel\'e le "\NewTerm{coefficient de variation}\index{coefficient de variation}\label{coefficient of variation}" (C.V.) car il n'a pas d'unit\'es (ce qui est son avantage!) et parce que plusieurs m\'ethodes industrielles statistiques considèrent qu'un bon C.V. doit id\'ealement être juste de l'ordre de quelques \% seulement!

	Plus g\'en\'eralement, pour tout estimateur statistique $\hat{\theta}$ (somme, moyenne, m\'ediane, etc.), nous pouvons construire un coefficient de variation tel que:
	

	Ainsi, dans la pratique nous consid\'erons que:

	\begin{table}[H]
	\begin{center}
		\definecolor{gris}{gray}{0.85}
			\begin{tabular}{|p{3.5cm}|p{3cm}|}
				\hline
				\multicolumn{1}{c}{\cellcolor{black!30}\textbf{Coefficient de variation}} & 
  \multicolumn{1}{c}{\cellcolor{black!30}\textbf{Qualit\'e}} \\ \hline
				\centering\arraybackslash\ 20\% & \centering\arraybackslash\ Poor \\ \hline
				\centering\arraybackslash\ 10\% & \centering\arraybackslash\ Acceptable  \\ \hline
				\centering\arraybackslash\ 5\% & \centering\arraybackslash\ Controlled  \\ \hline
				\centering\arraybackslash\ 2.5\% & \centering\arraybackslash\ Excellent  \\ \hline
				\centering\arraybackslash\ 1.25\% & \centering\arraybackslash\ World Class  \\ \hline
				\centering\arraybackslash\ 0.0625\% & \centering\arraybackslash\ Rarely achieved \\ \hline
		\end{tabular}
	\end{center}
	\caption{Jugements qualitatifs du C.V. commun\'ement accept\'es}
	\end{table}	
	Pourquoi trouvons-nous un carr\'e (r\'eciproquement une racine) dans la d\'efinition de la variance? La raison intuitive est simple (la rigoureuse l'est nettement moins...). Souvenez-vous que nous avons d\'emontr\'e plus haut que la somme des \'ecarts à la moyenne pond\'er\'es par les effectifs, est toujours nulle:
	
	Or, si nous assimilons les effectifs par la probabilit\'e en normalisant ceux-ci par rapport à $n$, nous tombons sur une relation qui est la même que la variance à la diff\'erence que le terme entre parenthèse n'est pas au carr\'e. Et nous voyons alors imm\'ediatement le problème... la mesure de dispersion serait toujours nulle d'où la n\'ecessit\'e de porter cela au carr\'e.

	Nous pourrions imaginer cependant d'utiliser la valeur absolue des \'ecarts à la moyenne, mais pour un certain nombre de raisons que nous verrons plus loin lors de notre \'etude des estimateurs, le choix de porter au carr\'e s'impose assez naturellement.

	Signalons cependant quand même l'utilisation courante dans l'industrie deux autres indicateurs fr\'equents de la dispersion:
	\begin{enumerate}
		\item  "\NewTerm{L'\'ecart absolu moyen}\index{ecart absolu moyen}" (moyenne des valeurs absolues des \'ecarts à la moyenne):
		
		qui est un indicateur \'el\'ementaire très utilis\'e lorsque nous ne souhaitons pas faire de l'inf\'erence statistique sur une s\'erie de mesures. Cet \'ecart peut être facilement calcul\'e dans la version française Microsoft Excel 11.8346 à l'aide de la fonction \texttt{ECART.MOYEN( )}.
			
		\item La "\NewTerm{d\'eviation absolue de la m\'ediane}\index{d\'eviation absolue de la m\'ediane}" not\'ee MAD (m\'ediane des valeurs absolues des \'ecarts à la m\'ediane):
		
		qui est consid\'er\'ee comme un indicateur plus robuste de la dispersion que ceux donn\'es par l'\'ecart absolu moyen ou l'\'ecart-type (malheureusement cet indicateur n'est pas int\'egr\'e à ma connaissances nativement dans les tableurs). Ce dernier ne doit pas être confondu avec "\NewTerm{moyenne absolue des d\'eviations à la m\'ediane}\index{moyenne absolue des d\'eviations à la m\'ediane}" d\'efini par:
		
		\begin{tcolorbox}[title=Remarque,colframe=black,arc=10pt]
		Afin d'utiliser le MAD comme estimateur consistent de l'\'ecart-type $\sigma$ (voir plus bas pour la d\'efinition de l'\'ecart-type) comme nous en aurons besoin lors de notre \'etude des Cartes de Contrôle (\SeeChapter{voir section du même nom page \pageref{quality control charts}}), nous prenons en g\'en\'eral:
		
		où $k$ est une constante de facture d'\'echelle, qui d\'epend de la distrubution.\\
		
		Pour des donn\'ees distribu\'ees selon un loi Normale, comme nous le d\'emontrerons pendant notre \'etude de la loi demi-Normale (voir page \pageref{half normal distribution}), nous prenons:
		 
		Dès lors:
		
		\end{tcolorbox}	
	\end{enumerate}

	\begin{tcolorbox}[colframe=black,colback=white,sharp corners]
	\textbf{{\Large \ding{45}}Exemple:}\\\\
	Consid\'erons les mesures d'une variable al\'eatoire $X$:
	
	dont la m\'ediane vaut:
	
	Les d\'eviations absolues par rapport à la m\'ediane sont alors:
	
	Mis dans l'ordre croissant, nous avons alors:
	
	où nous identifions facilement la d\'eviation absolue de la m\'ediane qui vaut:
	
	\end{tcolorbox}
	
	Dans le cas où nous avons à disposition une s\'erie de mesures, nous pouvons estimer la valeur exp\'erimentale de la moyenne (l'esp\'erance) et de la variance par les estimateurs suivants (il s'agit simplement au fait de l'esp\'erance et l'\'ecart-type d'un \'echantillon dont les \'ev\'enements sont tous \'equiprobables) dont la notation est particulière:
		
		\begin{dem}
		D'abord pour l'esp\'erance:
		
		Et pour la variance:
		
		\begin{flushright}
			$\blacksquare$  Q.E.D.
		\end{flushright}
		\end{dem}
		\begin{theorem}
		Et d\'emontrons un petite propri\'et\'e bien sympathique comme quoi la moyenne arithm\'etique est un optimum de la somme des carr\'es des \'ecarts. Effectivement, nous avons:
		\end{theorem}
		\begin{dem}
		
		et si nous cherchons  $\alpha$ tel que la d\'eriv\'ee de l'expression ci-dessus est nulle:
		
		alors $\alpha$  est un optimum. Nous avons alors:
		
		soit après r\'earrangement et simplification \'el\'ementaire:
					
		\begin{flushright}
			$\blacksquare$  Q.E.D.
		\end{flushright}		
		\end{dem}
		Il s'agit donc bien de la moyenne arithm\'etique. Maintenant pour savoir s'il s'agit d'un extrema de type maximum ou minimum il suffit de faire la d\'eriv\'ee seconde (\SeeChapter{voir section de Calcul Diff\'erentiel et Int\'egral page \pageref{second derivative}}) et de voir que cela donne une constante positive (donc la d\'eriv\'ee première augmente quand $\alpha$ augmente). Il s'agit alors d'un bien d'un extrema de type minimum!!!
		
	\begin{tcolorbox}[title=Remarque,colframe=black,arc=10pt]
	Le terme de la somme se trouvant dans l'expression de la variance (\'ecart-type) est appel\'ee  "\NewTerm{somme des carr\'es des \'ecarts à la moyenne}\index{somme des carr\'es des \'ecarts à la moyenne}" ou "\NewTerm{somme des carr\'es des erreurs à la moyenne}\index{somme des carr\'es des erreurs à la moyenne}". Nous l'appelons aussi la "\NewTerm{somme des carr\'es totale}\index{somme des carr\'es totale}", ou encore la "\NewTerm{variation totale}\index{variation totale}" dans le cadre de l'\'etude de l'ANOVA (voir plus loin).
	\end{tcolorbox}	
	Avant de poursuivre, rappelons le concept de moyenne g\'eom\'etrique vu plus haut (très utilis\'ee pour les rendements en finance ou les analyses de croissances en \% de chiffres d'affaires ou ventes):
	
	C'est bien joli mais les financiers ont besoin de calculer aussi l'\'ecart-type d'une telle moyenne. L'id\'ee est alors d'en prendre le logarithme pour la ramener à une simple moyenne arithm\'etique (il s'agit toujours bien \'evidemment d'estimateurs!):
	
	Dès lors, puisqu'en prenant le logarithme des valeurs nous nous ramenons à la moyenne arithm\'etique du logarithme des valeurs, alors le logarithme de l'\'ecart-type g\'eom\'etrique (avec un raisonnement à la physicienne...) sera:
	
	Il suffit alors de prendre l'exponentielle de l'\'ecart-type des logarithmes des valeurs pour avoir "\NewTerm{l'\'ecart-type g\'eom\'etrique}\index{l'\'ecart-type g\'eom\'etrique}":
	
	La variance peut \'egalement s'\'ecrire sous la forme très importante de la "\NewTerm{relation de Huyghens}\index{relation de Huyghens}\label{huygens relation}" appel\'ee \'egalement "\NewTerm{th\'eorème de König-Huyghens}\index{th\'eorème de König-Huyghens}" ou "\NewTerm{th\'eorème de translation de Steiner}\index{th\'eorème de translation de Steiner}" que nous r\'eutiliserons plusieurs fois par la suite. Voyons de quoi il s'agit:
	
	\begin{tcolorbox}[title=Remarque,colframe=black,arc=10pt]
	Le lecteur trouvera peut-être int\'eressant de savoir que dans de nombreuses applications commerciales de statistiques ("apprentissage automatique"), les statistiques doivent être mises à jour "à la vol\'ee" (domaine nomm\'e "\NewTerm{statistiques de flux de donn\'ees}") chaque fois que de nouvelles donn\'ees apparaissent. Il faut donc trouver un algorithme efficace pour mettre à jour les calculs sans tout recalculer à partir de z\'ero! L'exemple typique le plus simple est la variance! Mettre à jour le calcul de la variance de l'\'echantillon semble \'evidemment très fastidieux si nous ajoutons simplement de nouvelles donn\'ees selon:
	
	Mais en utilisant la relation de Huygens, nous constatons que cette variance d'\'echantillon devient beaucoup moins fastidieuse à calculer car il ne s'agit que d'un problème de mise à jour de deux sommes:
	
	\end{tcolorbox}
	Faisons maintenant un petit crochet relativement à un sc\'enario fr\'equent g\'en\'erateur d'erreurs dans les entreprises lorsque plusieurs s\'eries statistiques sont manipul\'ees (cas très fr\'equent dans l'industrie ainsi que dans les assurances ou la finance).

	Consid\'erons deux s\'eries statistiques portant sur le même caractère:
	\begin{itemize}
		\item $(x_1,n_1),(x_2,n_2),...,(x_p,n_p)$ \'echantillon d'effectif total $n$, moyenne arithm\'etique $\bar{x}$, \'ecart-type $\sigma_x$.
		
		\item $(y_1,m_1),(y_2,m_2),...,(y_p,m_q)$ \'echantillon d'effectif total $m$, moyenne arithm\'etique $\bar{y}$, \'ecart-type $\sigma_y$.
	\end{itemize}
	Nous avons alors:
	
	avec $n+m=N$.
	
	Donc la moyenne des moyennes n'est pas \'egale à la moyenne globale (première erreur fr\'equente dans les entreprises) except\'ee si les deux s\'eries statistiques ont le même nombre d'effectifs ($n=m=\sfrac{1}{2}N$)!!!

	Concernant l'\'ecart-type, rappelons d'abord que nous avons:
	
	Pour la suite, rappelons que nous avons d\'emontr\'e pr\'ec\'edemment la relation de Huygens:
	
	Il vient alors:
	
	Si nous faisons les mêmes calculs avec l'estimateur de la variance nous obtenons en en refaisant exactement les mêmes développements:
	
	Donc nous voyons que l'\'ecart-type global n'est pas \'egal à la somme des \'ecarts-types (deuxième erreur courante dans les entreprises)!
	
	Consid\'erons maintenant $X$ une variable al\'eatoire d'esp\'erance $\mu$ (valeur constante et d\'etermin\'ee) et de variance $\sigma^2$  (valeur constante et d\'etermin\'ee), nous d\'efinissons la "\NewTerm{variable centr\'ee r\'eduite}\index{variable centr\'ee r\'eduite}\label{reduced centered variable}" par la relation:
	 
	\begin{tcolorbox}[title=Remarque,colframe=black,arc=10pt]
	Lorsque la moyenne r\'eelle est inconnue et identique pour la variance, nous notons la relation ci-dessus comme suit (diff\'erentes notations sont repr\'esent\'ees):
	
	et cela s'appelle une "\NewTerm{variable studentis\'ee}\index{variable \'etudi\'ee}".
	\end{tcolorbox}
	\begin{theorem}
	Nous d\'emontrons de façon très simple en utilisant la propri\'et\'e de lin\'earit\'e de l'esp\'erance et la propri\'et\'e de multiplication par un scalaire de la variance (voir de suite après) que:
	
	\end{theorem}
	\begin{dem}
	Pour la preuve, nous utilisons simplement les d\'efinitions de l'esp\'erance et de la variance attendues (en utilisant le th\'eorème de Huygens pour cette dernière). Commençons donc par l'esp\'erance:
	
	Et maintenant, avec la variance utilisant le th\'eorème de Huygens:
	
	\begin{flushright}
		$\blacksquare$  Q.E.D.
	\end{flushright}
	\end{dem}
	Ainsi, toute r\'epartition statistique d\'efinie par une moyenne et un \'ecart-type peut être transform\'ee en une autre distribution statistique souvent plus simple à analyser. Ainsi en faisant cette transformation, nous obtenons une variable al\'eatoire dont les paramètres de la loi de distribution ne sont plus utiles à connaître. Quand nous faisons cela avec d'autres lois, et dans le cas g\'en\'eral, nous parlons alors de "\NewTerm{variables pivotables}\index{variables pivotales}".

	Voici quelques propri\'et\'es importantes de la variance\label{properties of the variance}:
	\begin{enumerate}
		\item[P1.] Multiplication par une constante:
		
		
		\item[P2.] Somme de deux variables al\'eatoires (combinaison lin\'eaire):
		
		Où nous rencontrons pour la première fois le concept de "\NewTerm {covariance}\index{covariance}\label{covariance}" not\'e par $\text{cov}( $. Nous verrons plus loin à la page \pageref{variance sum multiple random variables} le cas g\'en\'eral à $n$ variables al\'eatoires!
		
		\item[P3.] Transformation affine:
		
		sans trop de surprises ...
		
		\item[P4.] Produit de deux variables al\'eatoires (en utilisant la relation de Huyghens):
		
		Et si les deux variables al\'eatoires sont ind\'ependantes, il vient:
		
		Ce que l'on peut r\'e\'ecrire en utilisant encore une fois la relation de Huyghens:
		
	\end{enumerate}
	\'evidemment, nous supposerons comme \'evident que ces quatre propri\'et\'es s'\'etendent au cas continu!
	
	\subparagraph{Semivariance}\mbox{}\\\\
	La "\NewTerm{semi-variance}\index{semivariance}" est un indicateur de dispersion qui aurait \'et\'e propos\'e par Harry Markowitz au lieu de la variance pour l'optimisation de portefeuilles notamment.
	
	Les analyses en laboratoire ont a posteriori montr\'e que les portefeuilles optimis\'es pour la minimisation de la semi-variance (\SeeChapter{voir section \'economie \pageref{portfolio efficient diversification models}}) ont de meilleures performances que ceux bas\'es sur la variance. En effet, la variance considère les extrêmes de rendements positifs et n\'egatifs comme ayant une \'egale non-d\'esirabilit\'e, alors que, comme nous allons de suite le voir, la semi-variance ne considère que la partie n\'egative. En revanche, si la distribution des rendements tend à une sym\'etrie parfaite, les deux m\'ethodes tendent (\'evidemment) à donner exactement le même r\'esultat.
	
	\textbf{D\'efinition (\#\mydef):} La "\NewTerm{semi-variance}", aussi appel\'ee "\NewTerm{d\'eviation standard partielle inf\'erieure}\index{d\'eviation standard partielle inf\'erieure}" (LSPD), est une mesure de la dispersion uniquement des r\'ealisations d'une variable al\'eatoire $X$ qui sont inf\'erieures à la moyenne attendue de cette même variable al\'eatoire selon:
	
	ou not\'e dans le cas commun biais\'e et discret, comme l'\'ecart-type:
	
	\begin{tcolorbox}[title=Remarque,colframe=black,arc=10pt]
	À propos, le piège (de notation) suivant doit être \'evit\'e:
	
	\end{tcolorbox}
	À travers cette d\'efinition, nous voyons que la semi-variance est similaire à la variance. Cependant, elle ne considère que les observations en dessous de la moyenne. Outil utile pour l’analyse de portefeuille ou d’actif, la m\'ethode de la semivariance permet de mesurer le "\NewTerm{risque de perte à la baisse}\index{risque de perte à la baisse}". Alors que l'\'ecart-type et la variance fournissent des mesures de la volatilit\'e, la semi-variance ne prend en compte que les fluctuations n\'egatives d'un actif. En neutralisant toutes les valeurs sup\'erieures à la moyenne, ou le rendement cible de l'investisseur, la semivariance estime la perte moyenne qu'un portefeuille pourrait subir!

	Pour les investisseurs peu enclins à prendre des risques, r\'esoudre le problème d’une r\'epartition optimale du portefeuille en minimisant la semi-variance limiterait la probabilit\'e d’une perte importante.
	\begin{tcolorbox}[colframe=black,colback=white,sharp corners]
	\textbf{{\Large \ding{45}}Exemple:}\\\\
	Consid\'erons le petit tableau suivant:
	\begin{table}[H]
		\centering
		\begin{tabular}{|l|c|c|c|}
		\hline
		\rowcolor[HTML]{C0C0C0} 
		\multicolumn{1}{|c|}{\cellcolor[HTML]{C0C0C0}\textbf{Date}} & \textbf{Prix} & \textbf{Rendement ($\pmb{X}$)} & \textbf{$\pmb{\left(\min(X-\text{E}(X),0)\right)^2}$} \\ \hline
		Mars & $20$ &  & \multicolumn{1}{l|}{} \\ \hline
		Avril & $19$ & $-5\%$ & $(-4.6\%)^2$ \\ \hline
		Mai & $17.10$ & $-10\%$ & $(-9.6\%)^2$ \\ \hline
		Juin & $18.81$ & $+10\%$ & $0^2$ \\ \hline
		Juillet & $19.25$ & $+5\%$ & $0^2$ \\ \hline
		Ao'ut & $19.36$ & $-1.99\%$ & $(-1.59\%)^2$ \\ \hhline{|=|=|=|=|}
		\multicolumn{2}{|l|}{\textbf{Moyenne arithm\'etique:}} & $\text{E}(X)\cong 0.4\%$ &  \\ \hline
		\end{tabular}
	\end{table}
	La semi-variance est alors \'egale à:
	
	\end{tcolorbox}

	\paragraph{Covariance Discrète}\mbox{}\\\\ 
	Nous venons de voir dans l'une des dernières relations plus haut le concept de "\NewTerm{covariance}" dont nous d\'eterminerons une expression plus commode un peu plus bas mais donc d\'efinie par:
	
	Introduisons maintenant une forme plus g\'en\'erale et extrêmement importante de la covariance dans de nombreux domaines\label{variance sum multiple random variables}:
	
	\begin{tcolorbox}[title=Remarque,colframe=black,arc=10pt]
	Certains lecteurs ont peut-être remarqu\'e qu'avec $1$ variable, nous avons $0$ termes de covariance, avec $2$ variables, nous avons $1$ terme de covariance, avec $3$ variables nous $2 + 2 + 2 = 6$ termes de covariance, et en continuant ainsi nous voyons que nous avons dans le cas g\'en\'eral $n(n-1)$ termes de covariance pour $n$ variables.
	\end{tcolorbox}	
	Maintenant, nous changeons la notation pour simplifier encore plus:
	
	Donc dans le cas g\'en\'eral:
	
	Ou en utilisant l'\'ecart type:
	
	En utilisant les propri\'et\'es de l'esp\'erance (particulièrement $\text{E}(X)=c^{te}$ et $\text{E}(c^{te})=c^{te}$) nous pouvons \'ecrire la covariance d'une manière beaucoup plus simple à des fins de calculs:
	
	et donc nous obtenons la relation très utilis\'ee en statistiques et finance dans la pratique appel\'ee "\NewTerm{formule de la covariance}"...:
	
	qui est plus connu quand \'ecrit sous la forme suivante:
	
	
	Indiquons \'egalement que si $X=Y$, ce qui \'equivaut donc à une covariance univari\'ee, nous retrouvons la relation de Huyghens:
	

	\begin{tcolorbox}[title=Remarque,colframe=black,arc=10pt]
	Les statistiques peuvent être d\'ecoup\'ees selon le nombre de variables al\'eatoires que nous \'etudions. Ainsi, lorsqu'une seule variable al\'eatoire est \'etudi\'ee, nous parlons de  "\NewTerm{statistique univari\'ee}\index{statistique univari\'ee}", pour deux variables al\'eatoires de "\NewTerm{statistique bivari\'ee}\index{statistique bivari\'ee}" et en g\'en\'eral, de "\NewTerm{statistique multivari\'ee}\index{statistique multivari\'ee}".
	\end{tcolorbox}	
	
	Si et seulement si les variables sont \'equiprobables, nous retrouvons la covariance dans la litt\'erature sous la forme suivante, appel\'ee parfois "\NewTerm{covariance de Pearson}\index{covariance de Pearson}", qui d\'ecoule de calculs que nous avons d\'ejà fait ant\'erieurement avec l'esp\'erance:
	
	La covariance est un indicateur de la variation simultan\'ee de $X$ et $Y$. En effet, si en g\'en\'eral $X$ et $Y$ croissent simultan\'ement, les produits $(y_i-\mu_Y)(x_i-\mu_X)$ seront positifs (corr\'el\'es positivement), tandis que si $Y$ d\'ecroît lorsque $X$ croît, ces même produits seront n\'egatifs (corr\'el\'es n\'egativement).
	
	Signalons que si nous distribuons les termes de la dernière relation, nous avons:
	
	et nous avons d\'ejà d\'emontr\'e que la somme des \'ecarts à la moyenne est nulle. Dès lors nous obtenons une autre forme courante de la covariance (très utile dans les tableurs!):
	
	et par sym\'etrie:
	
	Donc au final, dans le cas \'equiprobable, nous avons finalement les trois relations \'equivalentes importantes utilis\'ees dans diff\'erents chapitres du pr\'esent site:
	
	Dans la section de M\'ethodes Num\'eriques pour notre \'etude de la r\'egression lin\'eaire (page \pageref{least squares method}) et de l'analyse factorielle nous aurons besoin de l'expression explicite de la propri\'et\'e de bilin\'earit\'e de la variance. Pour voir en quoi cela consiste exactement, consid\'erons trois variables al\'eatoires $X$, $Y$ et $Z$ et $a$ et $b$ deux constantes. Alors en utilisant la troisième relation donn\'ee pr\'ec\'edemment, nous avons:
	
	Cette dernière relation est elle aussi importante et sera utilis\'ee dans plusieurs chapitres du site (\'economie page \pageref{economy}, M\'ethodes Num\'eriques page \pageref{numerical methods}). Elle nous permet aussi d'obtenir directement des covariances entre des sommes de variables al\'eatoires.
	\begin{tcolorbox}[colframe=black,colback=white,sharp corners]
	\textbf{{\Large \ding{45}}Exemple:}\\\\
	Si $X$, $Y$, $Z$, $T$ sont quatre variables al\'eatoires d\'efinies sur la même population, nous voulons calculer la covariance suivante:
	
	Nous allons donc d\'evelopper en deux fois (raison pour laquelle nous appelons cela la "bilin\'earit\'e"). D'abord par rapport au second argument (arbitrairement!):
	
	et ensuite par rapport au premier:
	
	Donc au final:
	
	\end{tcolorbox}
	Il existe une m\'ethode simple et \'el\'egante pour vous rappeler comment calculer la covariance de deux variables al\'eatoires. En effet, rappelons-nous que nous avons prouv\'e un peu plus tôt que:
	 
	Il vient en soustrayant ces deux relations et en simplifiant / r\'eordonnant:
	 
	et c'est tout! Un moyen simple de se rappeler le calcul de la covariance.
	
	Consid\'erons maintenant un ensemble de vecteurs al\'eatoires $\vec{X}_i:=X_i$ de composants $(x_1,x_2,...,x_n)_i$. Le calcul de la covariance des composants par paires donne ce qu'on appelle la "\NewTerm{matrice de covariance}\index{matrice de covariance}" (un outil largement utilis\'e dans les m\'ethodes financières, de management et m\'ethodes statistiques num\'eriques!).

	En effet, nous d\'efinissons la composante $ (m, n) $ de la matrice de covariance par:
	
	On peut donc \'ecrire une matrice sym\'etrique (g\'en\'eralement ce doit être une matrice carr\'ee ...) sous la forme:
	
	où  $\Sigma$ est la lettre habituelle d\'esignant la "\NewTerm{matrice de covariance}\index{matrice de covariance}"
	
	Par sym\'etrie et parce que c'est une matrice carr\'ee de $n$ par $n$ matrice, seul le nombre $\dfrac{n(n+1)}{2}$ de composants est utile pour d\'eterminer toute la matrice (informations triviales mais importantes pour quand nous \'etudierons la mod\'elisation d’\'equations structurelles dans la section de M\'ethodes Num\'eriques).
	
	Cette matrice a la propri\'et\'e remarquable que si nous prenons l’ensemble des vecteurs al\'eatoires et calculons la matrice de covariance, la diagonale nous donnera \'evidemment les variances de chaque paire de vecteurs (voir des exemples dans les chapitres d'\'economie, M\'ethodes num\'eriques ou G\'enie industriel) parce que nous avons pour rappel:
	
	D'où le nom de "\NewTerm{matrice de variances-covariances}\index{matrice de variances-covariances}\label{matrice de variances-covariances}":
	
	Et ceci est un peu abusivement parfois \'ecrit de la manière suivante:
	
	Cette matrice a l'avantage de montrer rapidement quelles paires de variables al\'eatoires ont une covariance n\'egative et... pour quelle variable al\'eatoire la variance de la somme est inf\'erieure à la somme des variances!
	
	\begin{tcolorbox}[title=Remarque,colframe=black,arc=10pt]
	Comme nous l’avons d\'ejà mentionn\'e, cette matrice est très importante et nous la reverrons souvent dans la section d'\'economie lors de notre \'etude de la th\'eorie moderne des portefeuilles (page \pageref{portfolio efficient diversification models}), ainsi que pour les techniques d’exploration de donn\'ees dans la section de M\'ethodes Num\'eriques (page \pageref{data mining}) et \'egalement pour l'analyse en composants principaux (page \pageref{principal component analysis}) et \'egalement dans la section de G\'enie Industriel lors de notre \'etude des cartes de contrôle bivari\'ees (page \pageref{quality control charts}).
	\end{tcolorbox}	
	
	Rappelons maintenant que nous avons un axiome de probabilit\'e (\SeeChapter{voir section de Probabilit\'es page \pageref{kolmogorov axioms}}) qui nous amène au fait que deux \'ev\'enements $A$ et $B$ sont ind\'ependants si et seulement si:
	
	De même, par extension, on d\'efinit l'ind\'ependance des variables al\'eatoires discrètes.
	
	\textbf{D\'efinition (\#\mydef):} Soient $X$ et $Y$ deux variables al\'eatoires discrètes. Nous disons que $X$ et $Y$ sont ind\'ependants si et seulement si:
	
	Plus g\'en\'eralement, les variables discrètes $X_1, X_2, ..., X_n$ sont ind\'ependantes (en bloc) si:
	
	\begin{theorem}
	L'ind\'ependance de deux variables al\'eatoires implique que leur covariance est \'egale à z\'ero (l'inverse est faux!).
	\end{theorem}
	\begin{dem}
	Nous allons le prouver dans le cas où les variables al\'eatoires ne prennent qu'un nombre fini de valeurs $\left\lbrace x_i \right\rbrace_I$ et $\left\lbrace y_j \right\rbrace_J$, respectivement, avec $I$ et $J$ \'etant des ensembles finis.
	
	Pour le prouver, rappelons-nous que:
	
	et donc:
	
	\begin{tcolorbox}[title=Remarque,colframe=black,arc=10pt]
	Dès lors, plus La covariance est si petite (proche de z\'ero) plus les s\'eries sont ind\'ependantes. Inversement, plus la covariance est grande (en valeur absolue), plus les s\'eries sont d\'ependantes.
	\end{tcolorbox}	
	\'etant donn\'e que:
	
	et le fait que si  $X$ et $Y$ sont ind\'ependants, nous avons $c_{X,Y}=0$. Alors:
	
	Plus g\'en\'eralement, si $X_1,...,X_n$ sont ind\'ependants (en bloc), alors pour toute loi de distribution statistique discrète ou continue (!) nous avons en utilisant les deux notations les plus courantes:
	
	Ou en utilisant l'\'ecart type:
	
	\begin{flushright}
		$\blacksquare$  Q.E.D.
	\end{flushright}
	\end{dem}
	
	Nous aurons besoin d’une autre relation importante pour notre \'etude du Data Mining (fouille de donn\'ees) / Machine Learning (apprentissage machine) et en particulier pour comparer diff\'erents modèles impliquant la matrice de covariance.
	
	Soit $X=(X_i)$ un vecteur al\'eatoire $n\times 1$ et soit $A$ une matrice sym\'etrique $n\times n$. Si $\text{E}(X)=\mu$ et $\text{V}(X)=\Sigma=\sigma_{ij}$ alors  $\text{E}[X^TAX]=\text{tr}(A\sigma)+\mu^TA\mu$ (en gardant à l'esprit que $X^TAX$ est une matrice $1\times 1$, ou en d'autres termes: un scalaire!).
	
	Pour prouver cela, notons premièrement que nous avons:
	
	Alors, quelle est son esp\'erance? Clairement, nous avons d’abord en utilisant la lin\'earit\'e proprement dite de l'esp\'erance:
	
	En appliquant la relation de covariance prouv\'ee un peu plus haut:
	
	nous obtenons alors:
	
	Comme $\Sigma$ est sym\'etrique:
		
	La relation suivante sera importante pour nous lors de notre \'etude du critère d'information d'Akaike (AIC):
	
	Notez que la notation ci-dessus est courante dans le domaine de la statistique mais que nous devrions \'ecrire plus rigoureusement\label{quadratic relation for akaike information criterion}:
	
	Nous avons \'egalement besoin d’une autre propri\'et\'e de la matrice de covariance pour notre \'etude du critère d’information d’Akaike. Consid\'erons  $\vec{x}\in \mathbb{R}^{p}$ un vecteur (colonne) al\'eatoire avec la matrice de covariance $\sigma \in \mathbb{R}^{p\times p}$. Soit $A\in \mathbb{R}^{p\times p}$ une matrice non al\'eatoire. Alors:
	
	Nous obtenons le même r\'esultat si nous remplaçons le vecteur $\vec{x}$ par une matrice $X$  telle que:
	
	
	\subparagraph{Matrice de quasi-corr\'elation}\mbox{}\\\\
	Dans la pratique, certains analystes financiers estiment qualitativement (a priori!) la matrice de corr\'elations en fonction de leur exp\'erience, nous parlons alors de "\NewTerm{matrice de quasi-corr\'elation}\index{matrice de quasi-corr\'elation}", et  rappelons que la matrice de variance-covariance et la matrice de corr\'elations est la même chose mais avec deux perspectives diff\'erentes, cependant il n'est pas possible de faire une estimation de cette dernière compl\'etement n'importe comment (sans règles)!

	En effet, consid\'erons $n$ variables al\'eatoires r\'eelles $X_1,\ldots,X_n$ de moyennes respectives $\mu_1,\ldots,\mu_n$. Notons exceptionnellement $\Sigma$ la matrice sym\'etrique de variances-covariances d\'efinie par:
	
	Montrons que $\Sigma$ est positive semi-d\'efinie (\SeeChapter{voir section d'Algèbre Lin\'eaire page \pageref{positive semidefinite matrix}})\label{positive semi-definitiveness of covariance matrix}:
	\begin{dem}
	\'etant donn\'e $\vec{\alpha}=(\alpha_1,\ldots,\alpha_n)^T$ un vecteur $\mathbb{R}^n$. Nous avons:
	
	Dès lors:
	
	\'etant donn\'e que la variance est par construction positive. Ce qui signifie que $\Sigma$ est semi-positive. C'est une propri\'et\'e très importante pour de nombreuses applications statistiques (distance de Mahalonobis, d\'ecomposition de Cholesky, etc.).
	\begin{flushright}
		$\blacksquare$  Q.E.D.
	\end{flushright}
	\end{dem}
	Maintenant, par la g\'en\'eralit\'e de  $\vec{\alpha}$, admettons(rien ne nous empêche de le faire!) que ce dernier est un vecteur propre (\SeeChapter{voir section d'Algèbre Lin\'eaire page \pageref{eigenvector}}). Dès lors, puisque:
	
	et par d\'efinition de leurs valeurs propres et vecteurs propres:
	
	Dès lors nous avons:
	
	Et concentrons-nous sur le terme de droite:
	
	Comme la valeur propre est un scalaire, nous pouvons r\'eorganiser cela par exemple sous la forme suivante (commutativit\'e):
	
	et comme $\vec{\alpha}^T\vec{\alpha}$ est n\'ecessairement positif alors les $\lambda_i$ ne peuvent être que positifs ou nuls.
	
	Nous avons donc pu v\'erifier que notre matrice de variances-covariances (ou corr\'elation) est bien ce qu'elle est suppos\'ee être !!!
	
	De plus, comme nous l’avons d\'emontr\'e dans notre \'etude du th\'eorème spectral dans la section Algèbre lin\'eaire page \pageref{spectral theorem}, nous avons pour une matrice telle que $\Sigma$ (sym\'etrique, r\'eelle, d\'efinie semi-positive):
	
	\label{positive semi-definite matrix not always invertible}Une autre manière de contrôler  que $ \Sigma$ est positive semi-d\'efinie, c'est que le d\'eterminant de la matrice de variance-covariance est positif ou nul. Et gardez à l'esprit que si le d\'eterminant est z\'ero, la matrice n'est pas inversible! La matrice de variance-covariance n'est donc pas toujours inversible!
	\begin{tcolorbox}[title=Remarque,colframe=black,arc=10pt]
	Cependant, la r\'eciproque est fausse (\'evidemment!), nous pouvons avoir un d\'eterminant positif sans que la matrice soit positive. Tandis que pour les valeurs propres, il y a \'equivalence: la matrice positive est \'equivalente à ce que toutes les valeurs propres sont positives.
	\end{tcolorbox}
	
	\pagebreak
	\subparagraph{Quartet d'Anscombe}\mbox{}\\\\
	Le quartet d'Anscombe comprend quatre jeux de donn\'ees qui possèdent des propri\'et\'es statistiques \'el\'ementaires presque identiques, mais qui paraissent très diff\'erentes lorsqu'ils sont repr\'esent\'es par un graphique ou analys\'es à l'aide de statistiques de niveau premier cycle universitaire plutôt que de statistiques du niveau du secondaire. Chaque jeu de donn\'ees comprend onze couples de point $(x, y)$. Ils ont \'et\'e construits en 1973 par le statisticien Francis Anscombe pour d\'emontrer à la fois l’importance de la repr\'esentation graphique des donn\'ees avant de les analyser et l’effet des valeurs aberrantes sur les propri\'et\'es statistiques. Ce quartet est \'egalement utilis\'e pour tester si un outil analytique peut être accept\'e comme "conforme aux statistiques" (car les six statistiques utilis\'ees correspondantes devraient être le minimum fourni par tout outil analytique de niveau secondaire!).
	
	Les jeux de donn\'ees sont les suivants. Les valeurs de $ x $ sont les mêmes pour les trois premiers ensembles de donn\'ees:
	\begin{table}[H]
		\begin{center}
		\begin{tabular}{|c|c|c|c|c|c|c|c|}
			\hline
			\multicolumn{2}{|c|}{\cellcolor{black!30}I} & \multicolumn{2}{|c|}{\cellcolor{black!30}II} & \multicolumn{2}{|c|}{\cellcolor{black!30}III} & \multicolumn{2}{|c|}{\cellcolor{black!30}IV} \\ \hline
			\cellcolor{black!30} $x$ & \cellcolor{black!30} $y$ & \cellcolor{black!30} $x$ & \cellcolor{black!30} $y$ & \cellcolor{black!30} $x$ & \cellcolor{black!30} $y$ & \cellcolor{black!30} $x$ & \cellcolor{black!30} $y$ \\ \hline
			$10.0$ & $8.04$ & $10.0$ & $9.14$ & $10.0$ & $7.46$ & $8.0$ & $6.58$\\
			$8.0$ & $6.95$ & $8.0$ & $8.14$ & $8.0$ & $6.77$ & $8.0$ & $5.76$\\
			$13.0$ & $7.58$ & $13.0$ & $8.74$ & $13.0$ & $12.74$ & $8.0$	& $7.71$\\
			$9.0$ & $8.81$ & $9.0$ & $8.77$ & $9.0$ & $7.11$ & $8.0$ & $8.84$\\
			$11.0$ & $8.33$ & $11.0$ & $9.26$ & $11.0$ & $7.81$ & $8.0$ & $8.47$\\
			$14.0$ & $9.96$ & $14.0$ & $8.10$ & $14.0$ & $8.84$ & $8.0$ & $7.04$\\
			$6.0$ & $7.24$ & $6.0$ & $6.13$ & $6.0$ & $6.08$ & $8.0$ & $5.25$\\
			$4.0$ & $4.26$ & $4.0$ & $3.10$ & $4.0$ & $5.39$ & $19.0$ & $12.50$\\
			$12.0$ & $10.84$ & $12.0$ & $9.13$ & $12.0$ & $8.15$ & $8.0$ & $5.56$\\
			$7.0$ & $4.82$ & $7.0$ & $7.26$ & $7.0$ & $6.42$ & $8.0$ & $7.91$\\
			$5.0$ & $5.68$ & $5.0$ & $4.74$ & $5.0$ & $5.73$ & $8.0$ & $6.89$\\ \hline
		\end{tabular}
		\caption{Quartet d'Anscombe}
	\end{center}
	\end{table}
	Le quartet est encore souvent utilis\'e pour illustrer l’importance de regarder graphiquement un ensemble de donn\'ees avant de commencer à les analyser selon un type particulier de relation et l’insuffisance des propri\'et\'es statistiques de base pour d\'ecrire des ensembles de donn\'ees r\'eelles.
	
	Avec Microsoft Excel 14.0.7166 nous obtenons:
	\begin{figure}[H]
		\centering
		\includegraphics[scale=0.85]{img/arithmetics/anscombe_quartet.jpg}
		\caption{Résumé du quartet statistique de Anscombe}
	\end{figure}
	Comme nous pouvons le constater avec les indicateurs statistiques \'el\'ementaires, il est presque impossible de deviner une diff\'erence entre les quatre ensembles de donn\'ees. Mais si nous utilisons le coefficient d'asym\'etrie (skewness) ou le coefficient d'asym\'etrie (voir page\pageref{skewness and kurtosis}), cela change tout!
	
	En regardant les graphiques correspondants, nous arrivons à la même conclusion:
	\begin{figure}[H]
		\begin{center}
			\includegraphics[scale=0.8]{img/arithmetics/anscombe_quartet_chart.jpg}
		\end{center}	
		\caption{R\'esum\'e graphique du quarter d'Anscome}
	\end{figure}

	
	\paragraph{Esp\'erance et variance de la moyenne (erreur standard)}\mbox{}\\\\
	Souvent en statistique, il est utile de d\'eterminer l’\'ecart-type de la moyenne de l’\'echantillon et d’obtenir des r\'esultats analytiques importants dans les domaines de la gestion de projets et de production industrielle. Voyons de quoi il s'agit!
	
	Soit la moyenne d'une s\'erie de termes d\'etermin\'es chacun par la mesure de plusieurs valeurs (il s'agit au fait de son estimateur dans un cas particulier comme nous le verrons beaucoup plus loin): 
	
	alors en utilisant les propri\'et\'es de l'esp\'erance:
	
	et si toutes les variables al\'eatoires sont identiquement distribu\'ees et ind\'ependantes nous avons alors:
	
	\begin{tcolorbox}[title=Remarque,colframe=black,arc=10pt]
	Nous d\'emontrerons bien plus loin que si toutes les variables al\'eatoires sont identiquement distribu\'ees et ind\'ependantes et de variance finie, alors l'esp\'erance suit asymptotiquement ce que nous appelerons une "loi Normale".
	\end{tcolorbox}
	Pour la variance, le même raisonnement s'applique:
	
	et si les variables al\'eatoires sont toutes identiquement distribu\'ees et ind\'ependantes (nous \'etudierons plus loin le cas très important et courant dans la pratique où cette dernière condition n'est pas satisfaite):
	
	d'où l'\'ecart-type de la moyenne appel\'e aussi "\NewTerm{erreur-type}\index{erreur-type}",  "\NewTerm{erreur-standard}\index{erreur-standard}\label{erreur-standard}" ou encore "\NewTerm{variation non syst\'ematique}\index{variation non syst\'ematique}":
	
	et il s'agit rigoureusement de l'\'ecart-type de l'estimateur de la moyenne (c'est peut-être plus clair ainsi)!
	
	La forme la plus intuitive permettant d’exprimer l’erreur type en termes de pourcentage pour les personnes en emploi et non-expertes en analyse, gestionnaires de projets et directeurs g\'en\'eraux est nomm\'ee "\NewTerm{Erreur Type Relative}\index{erreur type relative}" (ETR), qui est l’expression de l'erreur standard en pourcentage, c'est-à-dire:
	
	Ce dernier indicateur est très utile lorsque nous devons traiter de nombreuses variables avec des unit\'es diff\'erentes!
	
	La relation de $\sigma_{\bar{X}}$ se trouve dans de nombreux logiciels dont les graphiques Microsoft Excel (mais il n'y a pas de fonction int\'egr\'ee dans Excel), \'ecrite soit avec l'\'ecart-type (comme ci-dessus), soit avec la notation de la variance (suffit de mettre au carr\'e...).

	Signalons que la dernière relation peut être utilis\'ee même si la moyenne des n variables al\'eatoires n'est pas identique! La condition principale \'etant juste que les \'ecarts-types soient tous \'egaux et c'est le cas dans la pratique de l'industrie (production).

	Nous avons donc:
	
	où $S_n$ d\'esigne la somme des $n$ variables al\'eatoires et $M_n$ leur moyenne estim\'ee.
	
	La variable centr\'ee r\'eduite que nous avions introduite plus haut:
	
	peut alors s'\'ecrire de plusieurs manières très utiles:
	
	Par ailleurs, en supposant que le lecteur sache d\'ejà ce qu'est une loi Normale $\mathcal{N}(\mu,\sigma)$, nous d\'emontrerons plus loin en d\'etails car c'est extrêmement important (!) que la loi de probabilit\'e de la variable al\'eatoire $\bar{X}$, moyenne de  $n$ variables al\'eatoires identiquement distribu\'ees et lin\'eairement ind\'ependantes, est alors la loi:
	
	
	
	\pagebreak
	\paragraph{Coefficient of Correlation}\label{coefficient of correlation}\mbox{}\\\\
	Maintenant, consid\'erons $X$ et $Y$ deux variables al\'eatoires ayant pour covariance:
	
	\begin{theorem}
	Alors nous avons:
	
	Nous allons d\'emontrer cette relation imm\'ediatement car l'utilisation de la covariance seule pour l'analyse des donn\'ees n'est pas g\'eniale car elle n'est pas à proprement parler born\'ee et simple d'usage (au niveau de l'interpr\'etation). Nous allons donc construire un indicateur plus facile d'usage en entreprise.
	\end{theorem}
	\begin{dem}
		Choisissons une constante $a$  quelconque et calculons la variance de:
		
		Nous pouvons alors imm\'ediatement \'ecrire à l'aide des propri\'et\'es de la variance et de l'esp\'erance:
		
		La quantit\'e de droite est positive ou nulle en tout $a$ par construction de la variance (de gauche). Donc le discriminant de l'expression, vue comme un trinôme en $a$ est du type:
		
		Donc pour que $P(a)$ soit positif pour tout $a$ nous avons comme seule possibilit\'e que:
		
		Soit après simplification:
		
		\begin{flushright}
			$\blacksquare$  Q.E.D.
		\end{flushright}
	\end{dem}
	Ce qui nous donnea aussi:
	
	Finalement nous obtenons une forme d'in\'egalit\'e statistique (c’est une forme particulière de l’in\'egalit\'e de Cauchy-Schwartz que nous verrons plus tard dans la section de Calcul Vectoriel page \pageref{cauchy-schwarz inequality}):
	
	Si les variances de $X$ et $Y$ sont non nulles, la corr\'elation entre $X$ et $Y$ est d\'efinie par le \NewTerm{coefficient de corr\'elation lin\'eaire}\index{coefficient de corr\'elation lin\'eaire}\label{linear correlation coefficient}" (il s'agit donc de la covariance standardis\'ee afin que son amplitude ne soit pas d\'ependante de l'unit\'e de mesure choisie) et not\'e:
	
	Ce qui peut aussi s'\'ecrire sous forme d\'evelopp\'ee (en utilisant la relation de Huyghens):
	
	ou encore plus condens\'ee:
	
	\begin{tcolorbox}[title=Remarque,colframe=black,arc=10pt]
	Signalons que normalement, la lettre $R$ est r\'eserv\'ee pour dire qu'il s'agit d'un estimateur du coefficient de corr\'elation alors que la d\'efinition ci-dessus n'est pas un estimateur et qu'en toute rigueur, nous devrions alors noter $\rho_{X,Y}$ selon les traditions d'usage.
	\end{tcolorbox}
	
	Quels que soient l'unit\'e et les ordres de grandeur, le coefficient de corr\'elation est donc un nombre sans unit\'es (donc sa valeur ne d\'epend pas de l'unit\'e de mesure choisie, ce qui n'est de loin pas le cas de tous les indicateurs statistiques!), compris entre $-1$ et $+1$. Il traduit la plus ou moins grande d\'ependance lin\'eaire de $X$ et $Y$ et ou, g\'eom\'etriquement, le plus ou moins grand aplatissement. Nous pouvons donc dire qu'un coefficient de corr\'elation nul ou proche de$ $0 signifie qu'il n'y a pas de relation lin\'eaire entre les caractères. Mais il n'entraîne aucune notion d'ind\'ependance plus g\'en\'erale!
	
	Dès lors, quand la corr\'elation est nulle:
	
	Alors:
	
	Donc:
	
	Et nous savons que cette relation est valable lorsque les variables al\'eatoires sont ind\'ependantes. Mais penser que dans le cas g\'en\'eral, "\textit{non corr\'el\'e}" signifie "\textit{ind\'ependance}" est totalement! Seule l'ind\'ependance implique la non-corr\'elation! Le contraire n'est
	\begin{tcolorbox}[colframe=black,colback=white,sharp corners]
	\textbf{{\Large \ding{45}}Exemple:}\\\\
	Prenons comme exemple de variables non corr\'el\'ees qui ne sont pas ind\'ependantes, l'exemple suivant (exemple utile en finance!) donn\'e par:
	
	$Y$ d\'epend clairement de $X$. Comme nous le d\'emontrerons plus tard, lors de notre \'etude de la distribution Normale centr\'ee r\'eduite, nous avons:
	
	Et comme:
	
	Nous avons alors bien des variables d\'ependantes qui ne sont pas corr\'el\'ees!
	\end{tcolorbox}
	Le seul cas g\'en\'eral où l'absence de corr\'elation implique l'ind\'ependance est le cas lorsque la distribution conjointe de $X$ et $Y$ est gaussienne!
	
	La figure ci-dessous illustre quelques cas courants et la dernière rang\'ee met en \'evidence des situations avec des variables d\'ependantes mais non corr\'el\'ees\label{correlations figure}:
	\begin{figure}[H]
		\centering
		\includegraphics[width=1.0\textwidth]{img/arithmetics/correlation_coefficients.jpg}	
		\caption[Quelques corr\'elations de Pearson]{Plusieurs ensembles de points $(x, y)$, avec la corr\'elation de Pearson correspondante\\ (source: Wikipédia, auteur: Denis Boigelot)}
	\end{figure}
	Quand le coefficient de corr\'elation est proche de $+1$ ou $-1$, les caractères sont dits fortement corr\'el\'es. Il faut prendre garde à la confusion fr\'equente entre corr\'elation et causalit\'e. Ainsi, que deux ph\'enomènes soient corr\'el\'es n'implique en aucune façon que l'un soit cause de l'autre (cette erreur est \'egalement connue sous le nom de "\NewTerm{corr\'elation parasite}" ou "\NewTerm{cum hoc ergo propter hoc}", expression latine pour "avec ceci, donc à cause de ceci" et "fausse cause")!!!!
	
	En effet, pour deux \'ev\'enements corr\'el\'es, $A$ et $B$, les diff\'erentes relations possibles sont les suivantes:
	\begin{itemize}
		\item $A$ cause $B$ (causalit\'e directe);
		\item $B$ cause $A$ (causalit\'e inverse);
		\item $A$ et $B$ sont les cons\'equences d'une cause commune, mais n'ont pas de causlit\'e directe;
		\item $A$ cause $B$ et $B$ cause $A$ (causalit\'e directe et bidirectionnelle);
		\item $A$ cause $C$ qui cause $B$ (causalit\'e indirecte);
		\item Il n'y pas de causalit\'e entre $A$ et $B$;
		\item La corr\'elation est une coincidence.
	\end{itemize}
	\begin{tcolorbox}[title=Remarque,colframe=black,arc=10pt]
	Une situation dans laquelle une mesure d'association ou de relation entre l'exposition et le r\'esultat est fauss\'ee par la pr\'esence d'une autre variable est appel\'ee "\NewTerm{confusion}\index {confusion}". Une variable \'etrangère qui est la cause totale ou partielle de l'effet observ\'e est appel\'ee "\NewTerm{variable de confusion}\index{variable de confusion}".
	\end{tcolorbox}
	Nous savons donc que la corr\'elation n'implique pas n\'ecessairement une causalit\'e et que, dans la pratique, nous ne pouvons pas toujours avoir des \'evidences scientifiques (c.-à-d. statistiques) que la corr\'elation implique une causalit\'e! Cependant, il existe un consensus dans le domaine scientifique selon lequel une corr\'elation implique une causalit\'e lorsque les conditions suivants, nomm\'ees "\NewTerm{critères de Braford Hill}\index{critères de Braford Hill}", sont satisfaits:
	\begin{table}[H]
		\centering
		\begin{tabular}{|l|p{10cm}|}
		\hline
		\rowcolor[HTML]{9B9B9B} 
		\multicolumn{1}{|c|}{\cellcolor[HTML]{9B9B9B}\textbf{Critère}} & \multicolumn{1}{c|}{\cellcolor[HTML]{9B9B9B}\textbf{Signification}} \\ \hline
		\textbf{Force de l'association} & Une association forte est plus susceptible d'avoir une composante causale qu'une association modeste. La force de l'association est d\'etermin\'ee par les types d'\'etudes existantes. Les \'etudes au plus haut niveau de la pyramide des preuves repr\'esenteraient les associations les plus fortes (à savoir, les revues syst\'ematiques avec m\'eta-analyses et ECR). Les r\'esultats de ces \'etudes doivent d\'emontrer un rapport de cotes (OR) ou un risque relatif (RR) d'au moins $2$ ou plus pour être significatifs. Tout ce qui se situe entre $1$ et $2$ est faible, ce qui se situe au-delà de $2$ est mod\'er\'e et au-delà de $4$ consid\'er\'e comme fort.\\ \hline
		\textbf{Consistence} & Une relation est observ\'ee à plusieurs reprises dans toutes les \'etudes disponibles. \\ \hline
		\textbf{Sp\'ecificit\'e} & Un facteur influence sp\'ecifiquement un r\'esultat ou une population particulière. Plus l'association entre un facteur et un effet est sp\'ecifique, plus la probabilit\'e qu'il soit causal est grande. \\ \hline
		\textbf{Temporalit\'e} & La cause doit pr\'ec\'eder le r\'esultat suppos\'e (par exemple, le tabagisme avant l’apparition du cancer du poumon) lorsque le r\'esultat est mesur\'e dans le temps (\'etude longitudinale). \\ \hline
		\textbf{\begin{tabular}[c]{@{}l@{}}Gradient Biologique\\ (r\'eponse à la dose)\end{tabular}} & Le r\'esultat augmente de manière monotone avec l’augmentation de la dose d’exposition ou selon une fonction pr\'edite par une th\'eorie de substantive (c’est-à-dire que plus on fume de cigarettes, plus le cancer est probable).\\ \hline
		\textbf{Plausibilit\'e} & L’association observ\'ee peut être expliqu\'ee de manière plausible par une question de fond, c’est-à-dire biologiquement possible. \\ \hline
		\textbf{Coh\'erence} & Une conclusion de causalit\'e ne doit pas fondamentalement contredire les connaissances de fond actuelles (les \'etudes ne doivent pas se contredire). \\ \hline
		\textbf{Exp\'erimentation} & La causalit\'e est plus probable si les preuves sont bas\'ees sur des exp\'eriences randomis\'ees ou sur une revue syst\'ematique avec une m\'eta-analyse d'exp\'eriences randomis\'ees. Cependant, si les ECR ne sont pas \'ethiquement possibles, les \'etudes de cohorte prospectives peuvent fournir le niveau de preuve le plus \'elev\'e. \\ \hline
		\textbf{Analogie} & Pour des expositions et des r\'esultats analogues, un effet a d\'ejà \'et\'e d\'emontr\'e, par exemple des effets d\'emontr\'es pour la première fois sur des animaux ou un effet survenant pr\'ec\'edemment sur des êtres humains, tels que les effets de la thalidomide sur le foetus pendant la grossesse.\\ \hline
		\end{tabular}
		\caption{Critères de corr\'elation-cause de Bradford Hill}
	\end{table}
	Revenons à l'aspect math\'ematique de la corr\'elation:
	\begin{itemize}
		\item Si $R_{X, Y}=-1 $ nous avons affaire à une "\NewTerm{corr\'elation n\'egative pure}\index{corr\'elation n\'egative pure}" (dans le cas d'une relation lin\'eaire, tous les points de mesure sont situ\'es sur une ligne droite avec une pente n\'egative).
		
		\item Si $-1 <R_{X, Y}<+1$ nous avons affaire à une corr\'elation positive ou n\'egative nomm\'ee "\NewTerm{corr\'elation imparfaite}\index{corr\'elation imparfaite}" (dans le cas d'une relation lin\'eaire, tous les points de mesure sont situ\'es sur une droite de pente n\'egative ou respectivement positive).
		
		 \item Si $R_{X, Y} = 0$, la corr\'elation est \'egale à z\'ero ... (dans le cas d'une relation lin\'eaire, tous les points de mesure sont situ\'es sur une droite dont la pente est nulle).
		 
		 \item Si $R_{X, Y} = +1$ nous avons affaire à une "\NewTerm{corr\'elation positive pure}\index{corr\'elation positive pure}" (dans le cas d'une relation lin\'eaire, tous les points de mesure sont situ\'es sur une ligne droite avec une pente positive).
	\end{itemize}
	
	L'analyse du coefficient de corr\'elation poursuit donc l'objectif de d\'eterminer le degr\'e d'association entre les diff\'erentes variables: celui-ci est souvent exprim\'e par le coefficient de d\'etermination, qui est le carr\'e du coefficient de corr\'elation. Le coefficient de d\'etermination mesure donc la contribution d'une des variables à l'explication de la seconde. Cependant, demander \textit{quelle magnitude doit avoir $R$ (ou $R^2$)?} n'a pas de sens. Un faible $R$ n'annule pas un pr\'edicteur important ni ne modifie la signification de son coefficient. Le $R$ est simplement ce qu'il est, quelle que soit sa valeur, et il n'est pas n\'ecessaire que ce soit une valeur particulière pour permettre une interpr\'etation valide.

	En utilisant les expressions de la moyenne et de l'\'ecart-type de variables \'equiprobables telles que d\'emontr\'ees plus haut (donc cela restreint l'application de ce coefficient à des variables al\'eatoires dont la distribution jointe est Normale!!), nous passons de:
	
	à l'estimateur du coefficient de corr\'elation:
	
	où nous voyons que la covariance devient alors la moyenne des produits moins le produit des moyennes.
	
	Soit après simplification:
	
	Le coefficient de corr\'elation peut être calcul\'e dans la version française de Microsoft Excel 11.8346 avec entre autres la fonction int\'egr\'ee \texttt{COEFFICIENT.CORRELATION( )}.
	
	Nous verrons dans la section M\'ethodes Num\'eriques une expression plus g\'en\'erale du coefficient de corr\'elation (page \pageref{correlation coefficient numerical methods}) et dans la section M\'ecanique Statistique un indicateur alternatif (page \pageref{mean mutual information}).
	
	\begin{tcolorbox}[title=Remarques,colframe=black,arc=10pt]
	\textbf{R1.} Dans la litt\'erature le coefficient de corr\'elation est souvent appel\'e "\NewTerm{coefficient d'\'echantillonnage de Pearson}\index{coefficient d'\'echantillonnage de Pearson}" (dans le cas \'equiprobable) et lorsque nous le portons au carr\'e, nous parlons alors de "\NewTerm{coefficient de d\'etermination}\index{coefficient de d\'etermination}".\\

	\textbf{R2.}  Souvent le carr\'e de ce coefficient est un peu abusivement interpr\'et\'e comme le \% de variation expliqu\'e de la variable \'etudi\'ee $Y$ par la variable explicative $X$.
	\end{tcolorbox}
	
	 Enfin, à noter que nous avons donc la relation suivante qui est \'enorm\'ement utilis\'ee dans la pratique (voir la section d'\'economie page \pageref{portfolio efficient diversification models} pour des exemples fameux!):
	
	ou sa version avec l'\'ecart-type qui est encore plus fameux:
	
	Il s'agit d'une relation que l'on retrouve souvent en finance dans le cadre du calcul de la VaR (Value at Riks) selon la m\'ethodologie RiskMetrics propos\'ee par J.P. Morgan (\SeeChapter{voir la section d'\'economie page \pageref{parametric VaR}}).
	
	Voyons un petit exemple d'application de la corr\'elation mais cela n'a rien à voir avec la VaR (du moins pour le moment ...).
	
	\begin{tcolorbox}[colframe=black,colback=white,sharp corners]
	\textbf{{\Large \ding{45}}Exemple:}\\\\
	Une compagnie a\'erienne a à sa disposition 120 sièges qu'elle r\'eserve pour des passagers en correspondance venant de deux autres vols arriv\'es un peu plus tôt dans la journ\'ee et en partance pour Francfort. Le premier vol arrive de Manille et le nombre de passagers à son bord suit une loi Normale de moyenne $50$ et de variance $169$. Le second vol arrive de Taipei et le nombre de passagers à son bord suit une loi Normale de moyenne $45$ et de variance $196$.\\

	Le coefficient de corr\'elation lin\'eaire entre le nombre de passagers des deux vols est mesur\'e comme \'etant:
	
	La loi que suit le nombre de passagers pour Francfort si nous supposons que la loi du couple suit elle aussi une loi Normale (selon \'enonc\'e!) est:
	
	avec:
	
	C'est donc mal parti au niveau satisfaction de la clientèle au long terme...
	\end{tcolorbox}

	Nous avons prouv\'e pr\'ec\'edemment que pour une somme de variables al\'eatoires ind\'ependantes et identiquement distribu\'ees, l'\'ecart-type de leur moyenne \'etait donn\'ee par l'erreur-type:
		
	Et que se passe-t-il si elles ne sont pas ind\'ependantes mais toujours identiquement distribu\'ees de la variance $\sigma^2$ (cas important dans le Machine Learning mais pas seulement!)? Pour r\'epondre, rappelons tout d'abord que:
	
	Dès lors:
	
	En utilisant la d\'efinition de la corr\'elation vue pr\'ec\'edemment:
	
	Nous obtenons (en gardant à l'esprit que tous les variances sont suppos\'ees être les mêmes, de sorte que $\sigma_{X_i}\sigma_{Y_i}=\sigma$):
	
	Supposons maintenant que toutes les corr\'elations par paires sont \'egales (telles que $ R_{X_i, Y_i} = R$), alors nous obtenons:
	
	Cette dernière relation est souvent d\'esign\'ee par les scientifiques des donn\'ees (data scientists)\label{variance mean correlated variables} comme suit:
		
	
	Enfin, rappelons que la matrice de variance-covariance est donn\'ee par:
	
	En divisant chaque terme par les termes correspondants $\sqrt{\text{V}_i}\sqrt{\text{V}_j}$, nous donne la "\NewTerm{matrice de corr\'elation}\index{matrice de corr\'elation}\label{correlation matrix}":
	
	Notez que si les variances sont toutes \'egales à l’unit\'e ($\text{V}_i=1$), la matrice de covariance est \'egale à la matrice de corr\'elation, c’est-à-dire $R=\Sigma$ (il s’agit d’une propri\'et\'e utile dont nous aurons usage beaucoup plus tard au cours de notre \'etude du facteur d’inflation de la variance pour la d\'etection multicolin\'earit\'e)!
	
	Voyons ce que nous venons de dire d'une autre manière! Comme nous le savons, l'\'equation de normalisation d'une variable s'\'ecrit:
	
	Nous savons \'egalement que la matrice originale de covariance est d\'efinie comme suit (à l'aide d'un estimateur sans biais):
	
	Et après normalisation des deux variables:
	
	nous obtenons:
	
	Alors finalement:
	
	La matrice de variance-covariance devient donc bien la matrice de corr\'elation avec des variables standardis\'ees!
	
	\subparagraph{Interpr\'etation g\'eom\'etrique}\mbox{}\\\\
	Une mesure de la relation entre deux variables al\'eatoires est la covariance telle que nous la connaissons, donn\'ee par:
	
	La covariance normalis\'ee à l'unit\'e est le coefficient de corr\'elation comme nous venons de le voir:
	
	Dès lors:
	
	L'expression r\'esultante:
	
	est le rapport de deux \'el\'ements. Le num\'erateur est le produit scalaire de deux vecteurs, tandis que le d\'enominateur est le produit de leurs longueurs:
	
	Cette expression montre l'identit\'e formelle entre le coefficient de corr\'elation et le cosinus de l'angle entre deux vecteurs al\'eatoires!
	
	\subparagraph{Corr\'elation partielle}\mbox{}\\\\
	La corr\'elation partielle mesure le degr\'e d'association entre deux variables al\'eatoires, en supprimant l'effet d'un ensemble de variables al\'eatoires de contrôle. Si nous voulons savoir s’il existe ou non une relation num\'erique entre deux variables d’int\'erêt, utiliser leur coefficient de corr\'elation donnera des r\'esultats erron\'es s’il existe une autre variable, "\NewTerm{ variable deconfusion}\index{ variable de confusion}", qui est num\'eriquement li\'ee aux deux variables d'int\'erêt. Cette information trompeuse peut être \'evit\'ee en contrôlant (c'est-à-dire en annulant) la variable de confusion, en calculant le coefficient de corr\'elation partiel.
	
	Lorsque dans le cas de trois variables nous voulons contrôler (annuler) une des variables et que celle-ci est l’unique qualitative, la strat\'egie est alors simple car elle consiste à calculer la corr\'elation entre les deux variables restantes pour chaque niveau de la variable qualitative. Mais les choses commencent à être plus compliqu\'ees lorsque la variable à contrôler (annuler) est continue. C'est pourquoi nous devons construire une approche un peu plus sophistiqu\'ee!
	
	Le "\NewTerm{coefficient de corr\'elation partiel}\index {coefficient de corr\'elation partiel}", not\'e $R_{AB.C} $, permet de connaître la valeur de la corr\'elation entre deux variables $A$ et $B$, si la variable $C$ \'etait rest\'ee constante pour la s\'erie d'observations consid\'er\'ee.
	
	En d’autres termes, le coefficient de corr\'elation partiel $R_{AB.C}$ est le coefficient de corr\'elation total entre les variables $A$ et $B$ lorsque leur meilleure explication lin\'eaire a \'et\'e supprim\'ee en termes de $C$. Il est donn\'e par la relation (dans le cas particulier \'evident de trois variables!):
	
	La preuve la plus rapide de cette relation est de s’appuyer sur l’interpr\'etation g\'eom\'etrique de la corr\'elation (cosinus).
	
	Mais notez auparavant que, tout comme le coefficient de corr\'elation, le coefficient de corr\'elation partiel prend une valeur comprise entre $-1$ et $+1$. La valeur $-1$ traduit une corr\'elation n\'egative parfaite en contrôlant certaines variables (c'est-à-dire une relation lin\'eaire exacte dans laquelle les valeurs les plus \'elev\'ees d'une variable sont associ\'ees à des valeurs plus faibles de l'autre); la valeur $+1$ traduit une relation lin\'eaire positive parfaite, tandis que la valeur $0$ indique qu'il n'y a pas de relation lin\'eaire.
	
	Notez \'egalement que si $R_ {AC}$ et $R_ {BC}$ sont tous deux \'egaux à z\'ero (c'est-à-dire qu'il n'y a pas de variable de confusion), alors nous avons l'\'egalit\'e:
	
	Il est donc facile de comprendre qu'en g\'en\'eral, lorsqu'il existe quelque part une corr\'elation entre certaines variables, alors $R_{AB.C}>R_ {AB}$. Pour les autres valeurs de configuration, le lecteur doit être très prudent sur l'interpr\'etation car le r\'esultat d\'epend \'evidemment de la valeur de $R_{AB}$!
	
	Ok maintenant passons à la preuve dans le cas particulier de trois variables!
	
	La s\'erie d'observations $A$, $B$ et $C$, une fois centr\'ee r\'eduite (donc de rayon $1$), sont des vecteurs centr\'es $\vec{A}$, $\vec{B}$, $\vec{C} $ de longueur d'unit\'e sur la même sphère:
	\begin{figure}[H]
		\centering
		\includegraphics{img/arithmetics/partial_autocorrelation.jpg}
	\end{figure}
	Leurs extr\'emit\'es (sommets) d\'eterminent un triangle sph\'erique $ABC$, dont les côt\'es $a$, $b$ et $c$ sont les arcs de grands cercles $\overarc{BC}$, $\overarc{AC}$ et $\overarc{AB}$. Les coefficients de corr\'elation entre ces vecteurs sont (rappelons-nous que pour un cercle / sphère unitaire, nous avons $\cos(\alpha)=\cos(a) $ comme indiqu\'e dans la section de Trigonom\'etrie):
	  
	Nous avons ensuite par la formule cosinus des triangles sph\'eriques (\SeeChapter{voir la section Trigonom\'etrie page \pageref{cosine formula}}) la valeur suivante pour l'angle d'ouverture de $C$:
	 
	Tout comme $c$ est l’angle entre les points $A$ et $B$ (et sur une sphère unitaire, la longueur arclength $\overarc{AB}$ \'egalement), vue du centre de la sphère, $\hat{C}$ est l'angle sph\'erique entre les points $A$ et $B$, vus du point $C$ à la surface de la sphère, et:
		
	est la "corr\'elation partielle" entre $A$ et $B$ lorsque $C$ est fixe.
	
	\begin{tcolorbox}[title=Remarques,colframe=black,arc=10pt]
	\textbf{R1.} Si nous s\'eparons une variable d'une corr\'elation, cette corr\'elation partielle est appel\'ee "\NewTerm{corr\'elation partielle de premier ordre}". Si nous s\'eparons partiellement les $2$ variables  à partir de cette corr\'elation (par exemple, $R_{12.34}$), nous obtenons une "\NewTerm{corr\'elation partielle de second ordre}", etc. Il est habituel de faire r\'ef\'erence aux "\NewTerm{corr\'elations non s\'epar\'ees}" en tant que "\NewTerm{corr\'elations d'ordre z\'ero}".\\
	
	\textbf{R2.} N'oubliez pas qu'en ce qui concerne la corr\'elation, la corr\'elation partielle lorsqu'elle est \'elev\'ee n'implique pas n\'ecessairement une causalit\'e!
	\end{tcolorbox} 
	
	\pagebreak
	\subparagraph{Coefficient de corr\'elation intraclasse}\mbox{}\\\\
	Le "\NewTerm{coefficient de corr\'elation intraclasse}\index {coefficient de corr\'elation intraclasse}", not\'e ICC, est une m\'etrique qui peut être utilis\'ee lorsque des mesures quantitatives sont effectu\'ees sur des mesures organis\'ees en groupes. Il d\'ecrit en quoi les mesures du même groupe sont similaires (c’est donc toujours l’un des nombreux indicateurs de la famille des \'etudes de r\'ep\'etabilit\'e et de reproductibilit\'e comme le kappa d’approbation de Cohen que nous pr\'esenterons plus tard). Cette statistique est consid\'er\'ee comme un type de corr\'elation et, contrairement à la plupart des autres mesures de corr\'elation, elle fonctionne donc sur des donn\'ees structur\'ees en groupes.
	
	Une application importante est l’\'evaluation de la coh\'erence ou de la reproductibilit\'e de mesures quantitatives effectu\'ees par diff\'erents observateurs mesurant la même quantit\'e (\'etudes R\&R). Par exemple, si on demande à plusieurs m\'edecins de noter les r\'esultats d'une analyse pour d\'etecter des signes de progression du cancer, on peut se demander en quoi les scores sont coh\'erents d'un m\'edecin à l'autre (c'est-à-dire une ANOVA avec facteurs al\'eatoires).
	
	Un aspect important de cette statistique (et aussi d'une faiblesse) est qu'elle quantifie à la fois la variabilit\'e inter-observateur et la variabilit\'e intra-observateur. La variabilit\'e inter-observateur fait r\'ef\'erence aux diff\'erences syst\'ematiques observ\'ees entre observateurs - par exemple, un m\'edecin peut toujours marquer les patients avec un niveau de risque sup\'erieur à celui des autres m\'edecins. La variabilit\'e intra-observateur fait r\'ef\'erence aux \'ecarts de mesure d'un observateur unique.
	\begin{tcolorbox}[title=Remarque,colframe=black,arc=10pt]
	\'etant donn\'e que le coefficient de corr\'elation intraclasse donne une composition de la variabilit\'e inter et intra-observateur, ses r\'esultats sont parfois jug\'es difficiles à interpr\'eter lorsque les observateurs ne sont pas \'echangeables (al\'eatoires). Des mesures alternatives telles que la statistique kappa d'approbation de Cohen (voir plus bas), le kappa de Fleiss et le coefficient de concordance de corr\'elation ont \'et\'e propos\'ees en tant que mesures d'accord plus appropri\'ees entre les observateurs non \'echangeables.
	\end{tcolorbox}	
	Il existe plusieurs types de corr\'elations intra-classe. Ils concernent diff\'erentes façons de conceptualiser les variables, souvent d\'esign\'ees de manière g\'en\'erique comme "sujets" et "juges". Chacune de ces variables peut être une variable al\'eatoire (leurs niveaux ont \'et\'e choisis au hasard parmi une population de niveaux possibles) ou une variable fixe, dont les diff\'erents niveaux ont \'et\'e sp\'ecifiquement s\'electionn\'es pour ce type de plan exp\'erimental, et les mêmes niveaux seraient s\'electionn\'es à nouveau dans une configuration ult\'erieure pour r\'eplication. De plus, nous pourrions utiliser des notations individuelles, ou nous pourrions prendre la moyenne d’un ensemble de notations attribu\'ees par diff\'erents noteurs. Cela modifierait le coefficient de corr\'elation intraclasse en fournissant une mesure plus fiable.
	
	Voici un exemple figuratif d'un ICC (pour une application num\'erique, voir les livres compagnons sur Minitab ou R):
	\begin{figure}[H]
		\centering
		\includegraphics[scale=0.9]{img/arithmetics/intra_class_correlation_coefficient.jpg}	
		\caption[Illustration du coefficient de corr\'elation intra-classe]{Illustration du coefficient de corr\'elation intra-classe (source: Wikipédia)}
	\end{figure}
	Le coefficient de corr\'elation intra-classe qui va nous int\'eresser ici, not\'e ICC1, le plus souvent utilis\'e dans la pratique est de consid\'erer un facteur fixe (non al\'eatoire) et l’autre al\'eatoire (il existe donc d’autres variantes qui doivent être appliqu\'ees avec prudence!).
	
	En effet, imaginons une \'etude concernant la productivit\'e des employ\'es d’une grande entreprise. Le comit\'e de direction veut avoir une id\'ee de la production journalière et de la façon dont cela d\'epend des machines de production utilis\'ees par les employ\'es. Si nous prenons pour l’\'etude un \'echantillon al\'eatoire de $n$ employ\'es parmi $N$, avec toutes les machines en quantit\'e $k$ alors nous avons un modèle avec un facteur fixe (machines) et l’autre qui est al\'eatoire (les employ\'es).

	L'id\'ee empirique est alors d'\'ecrire le modèle sous la forme très sp\'ecifique suivante (pour les autres modèles plus g\'en\'eraux, il faudra attendre que nous \'ecrivions les d\'eveloppements!):
	
	où $i$ est la mesure du facteur fixe (souvent sous forme de lignes dans les tables d'analyse ICC) et $j$ est le facteur al\'eatoire repr\'esent\'e par les juges (souvent sous forme de colonnes dans les tables d'analyse ICC) et donc $\varepsilon_{ij}$ d\'ecrit la variabilit\'e de la mesure des \'el\'ements dans un niveau donn\'e de facteur al\'eatoire et l'erreur des juges est indissociable de l'\'el\'ement mesur\'e. La relation pr\'ec\'edente est aussi parfois not\'ee:
	
	où \'evidemment $\mu_j=\mu+\tau_j$.

	Comme d'habitude (...) on suppose que:
	
	On suppose donc que les variances des $\alpha_j$ sont identiques (ce qui est n\'eanmoins une hypothèse assez forte en pratique!).

	Nous avons donc de par la lin\'earit\'e de la moyenne:
	
	La d\'ecomposition de la variance est imm\'ediate si les erreurs sont ind\'ependantes des facteurs al\'eatoires:
	
	Nous avons donc:
	
	Pour deux \'el\'ements (observations) pris sous le même niveau de facteur $i$, nous avons:
	
	Mais comme les variances de tous les $\alpha_j$ sont suppos\'ees être identiques et que les erreurs sont ind\'ependantes des effets fixes, ces derniers se r\'eduisent à:
	
	Et leur corr\'elation est alors:
	
	Ceci est souvent not\'e dans les manuels de r\'ef\'erence comme:
	
	Où $S_w^2$ est la variance group\'ee (voir page \pageref{pooled variance}) au sein des sujets, et $S^2_b$ est la variance des mesures entre les sujets. $S^2_b + S^2_w$ est la variance totale.
	
	C’est donc le rapport entre la variance des facteurs al\'eatoires et la variance totale (le d\'enominateur incluant la variance de l’erreur) qui est souvent d\'esign\'e ICC1 ou ICC(1,1) pour indiquer qu’il s’agit du coefficient de corr\'elation Intraclass pour $1$ facteur al\'eatoire et $1$ facteur fixe.
	
	Donc, si $\sigma_\varepsilon\rightarrow 0$, la corr\'elation intraclasse est unitaire et c’est alors un indicateur de bonne reproductibilit\'e!
	
	Rappelons maintenant que de notre \'etude des ANOVA à un facteur fixe (page \pageref{anova one way fixed factor}) que nous avons d\'emontr\'e la d\'ecomposition suivante en somme des carr\'es:
	
	Si $n_i=n$ (exp\'erience \'equilibr\'ee), alors nous avons \'evidemment:
	
	Ceci est souvent not\'e dans le contexte de l'utilisation du coefficient de corr\'elation intra-classe:
	
	où pour rappel $N = kn$. Nous supposons donc ici que les juges sont index\'es par $i$, les observations par $j$ et que nous avons $n$ juges et $k$ observations!
	
	Et nous avons aussi r\'egulièrement les notations suivantes dans certains manuels de r\'ef\'erence:
	
	ou:
	
	où SSW signifie "Sum of Squares Within" ("somme des carr\'es dans" en français) et SSB  "Sum of Squares Between" ("somme des carr\'es entre" en français) ou (...)
	
	où MSE signifie "Mean Square for Error" ("moyenne des carr\'es des des erreurs" en français) et MSk pour "Mean Square for traitements" ("moyenne des carr\'ees pour les traitements en français). Donc:
	
	Dès lors:
	
	\begin{tcolorbox}[title=Remarque,colframe=black,arc=10pt]
	Comme le lecteur l’aura peut-être remarqu\'e, le num\'erateur peut être n\'egatif et donc aussi $\sigma^2_\alpha$... ... Si nous obtenons des composantes de variance n\'egatives, voici des moyens possibles de traiter ce genre de situation:
	\begin{itemize}
		\item Accepter l’estimation comme \'evidence vraie d’une valeur nulle et utiliser z\'ero comme estimation, en sachant que l’estimateur ne sera plus non-biais\'e!
	
		\item Conserver l'estimation n\'egative en sachant que les calculs ult\'erieurs utilisant ce r\'esultat risquent de ne pas avoir beaucoup de sens...
	
		\item Interpr\'eter que l'estimation de la composante n\'egative indique un modèle statistique incorrect.
	
		\item Recueillir plus de donn\'ees et les analyser s\'epar\'ement ou en conjonction avec les donn\'ees existantes, en esp\'erant que davantage d'informations g\'en\'ereront des estimations positives.
	\end{itemize}
	Par exemple, lorsqu'une estimation de composante de variance est inf\'erieure à z\'ero, le logiciel Minitab affiche l'estimation n\'egative, mais met l'estimation à z\'ero afin de calculer le pourcentage de la variabilit\'e totale.
	\end{tcolorbox}
	C'est la raison pour laquelle nous trouvons souvent le coefficient de corr\'elation intraclasse sous la forme commune suivante dans les manuels de r\'ef\'erence car il permet une approche parfois plus intuitive du concept:
	
	Cela est \'egalement souvent not\'e dans certains manuels de r\'ef\'erence sous la forme:
	
	Domenic V. Cicchetti (1994) propose les directives suivantes souvent cit\'ees pour l’interpr\'etation des mesures d’accord inter-\'evaluateur kappa ou ICC:
	\begin{table}[H]
		\centering
		\begin{tabular}{|c|c|}
		\hline
		\rowcolor[HTML]{C0C0C0} 
		\textbf{Valeur ICC} & \textbf{Critère} \\ \hline
		$0.4<$ & Pauvre \\ \hline
		$[0.4,0.59]$ & Acceptable \\ \hline
		$[0.60,0.74]$ & Bon \\ \hline
		$>0.75$ & Excellent \\ \hline
		\end{tabular}
		\caption{Interpr\'etation pour accord inter-\'evaluateur kappa ou ICC}
	\end{table}
	
	\pagebreak
	\subsubsection{Variables continues et Moments}
	
	\textbf{D\'efinitions (\#\mydef):}
	\begin{enumerate}
		\item[D1.] Nous disons que $X$ est une variable continue si sa "\NewTerm{fonction de r\'epartition}\index{fonction de r\'epartition}" est continue (d\'ejà d\'efini plus haut). La fonction de r\'epartition de $X$ \'etant pour $x \in \mathbb{R}$ ou un sous-ensemble tronqu\'e de $\mathbb{R}$ par:
		
		soit la probabilit\'e cumul\'ee que la variable al\'eatoire $X$ soit plus petite ou \'egale à la valeur $x$ fix\'ee. Nous avons aussi bien \'evidemment:
		
		avec:
		
		
		\item[D2.] Nous notons par:
		
		la "\NewTerm{fonction de survie}\index{fonction de survie}" ("survival function" en anglais) ou "\NewTerm{fonction de queue}\index{fonction de queue}" ("tail distribution function" en anglais).
		
		\item[D3.] Si de plus la fonction de r\'epartition $F$ de $X$ est continûment d\'erivable de d\'eriv\'ee $f$ (ou parfois $\rho$) appel\'ee "\NewTerm{fonction de densit\'e de probabilit\'e}\footnote{Gardez à l'esprit que l'expression "\NewTerm{fonction de masse de de probabilit\'e}\index{fonction de masse de de probabilit\'e}" est r\'eserv\'ee aux variables discrètes.}\index{fonction de densit\'e de probabilit\'e}" ou encore simplement "\NewTerm{fonction de distribution}\index{fonction de distribution}\label{distribution function}" alors nous disons que $X$ est absolument continue et dans ce cas nous avons:
		
		avec la condition de normalisation:
		
		Toute fonction de distribution de probabilit\'e doit satisfaire l'int\'egrale de normalisation dans son domaine de d\'efinition!
	\end{enumerate}
	
	\label{random variable transformation}Considérons une transformation type de variables aléatoires assez courante pour généraliser certains cas particuliers à des cas plus généraux (typiquement de la loi Normale centrée réduit à la loi Normale générale ou de la loi de Cauchy standard à la loi de Cauchy généralisée) ! Considérons pour cela $X$ une variable aléatoire continue et la transformation $Y=a+b X$ où $a$ est nommé le "paramètre de localisation" et avec $b>0$ nommé le "paramètre d'échelle" et notons $F$ la fonction de distribution de $X$ et $G$ la fonction de distribution de $Y$. Alors:
	
	Maintenant, nous différencions et nous obtenons le résultat important et trivial :
	
	où $g$ est la fonction de densité pour $Y$ et $f$ est la fonction de densité pour $X$.
	
	On peut aussi considérer le cas où $b<0$. Dès lors:
	
	En différenciant:
	
	Le lecteur doit essayer si possible de garder à l'esprit ces relations.
	
	\begin{tcolorbox}[title=Remarque,colframe=black,arc=10pt]
	Il est int\'eressant de remarquer que la d\'efinition amène à ce que la probabilit\'e qu'une variable al\'eatoire totalement continue prenne une valeur donn\'ee est nulle! Donc ce n'est pas parce qu'un \'ev\'enement a une probabilit\'e nulle qu'il ne peut arriver!!!
	\end{tcolorbox}	
	La moyenne ayant \'et\'e d\'efinie par la somme pond\'er\'ee par les probabilit\'es pour une variable discrète, elle devient une int\'egrale pour une variable continue\index{esp\'erance variable continue}\label{expected mean continuous variable}:
	
	et donc la variance est \'ecrite comme:
	
	Ensuite, nous avons \'egalement la m\'ediane qui est logiquement red\'efinie dans le cas d’une variable al\'eatoire continue par:
	
	et qui coïncide rarement avec la moyenne!
	
	Et la valeur modale est donn\'ee par la valeur de $x$ où:
	
	
	Les statisticiens utilisent souvent les notations suivantes pour l'esp\'ereance d'une variable continue:
	
	et pour la variance:
	
	C'est la même notation que pour les moments de variables discrètes.

	Par la suite, nous allons calculer ces diff\'erents indicateurs de moments avec des preuves d\'etaill\'ees uniquement pour les cas les plus utilis\'es.
	
	Mais avant ... parlons de quelques autres moments bien connus et très discut\'es/d\'ebattus ...!
	
	\pagebreak
	\label{skewness and kurtosis}
	\paragraph{Coefficients d'asym\'etrie et d'aplattissement}\mbox{}\\
	\begin{tcolorbox}[colback=red!5,borderline={1mm}{2mm}{red!5},arc=0mm,boxrule=0pt]
	\bcbombe Attention!!! J'ai h\'esit\'e pendant de nombreuses ann\'ees (plus de 10 ans!) à \'ecrire ce texte à propos du coefficient d'asym\'etrie et du coefficient d'aplattissement dans ce livre, car ils sont la plupart du temps des concepts trompeurs et qu'il est difficile (voire impossible ...) de les introduire avec toute la rigueur math\'ematique n\'ecessaire. J'ai chang\'e d'avis quand j'ai vu que mes \'etudiants s'attendaient à quelque chose de plus complet et de plus robuste que le contenu Wikip\'edia. Le lecteur doit bien garder à l'esprit que ces deux indicateurs doivent être utilis\'es avec prudence et constituent de bons outils si nous savons que nous travaillons avec des donn\'ees Normalement distribu\'ees.
	\end{tcolorbox}
	Comme d\'ejà mentionn\'e dans l'avertissement pr\'ec\'edent, le coefficient d'asym\'etrie et le  coefficient d'aplattissement sont des notions quelque peu vagues. Le lecteur ne devrait donc pas s'attendre ici  un preuve de leur origine ... De plus, le concept de "sym\'etrie" est math\'ematiquement pr\'ecis, mais l'asym\'etrie au contraire est un concept \'etonnamment flou... et l'aplatissement l'est peut-être encore plus!
	
	Les d\'efinitions correspondantes aux id\'ees sous-jacentes sont les suivantes (les d\'efinitions sont donn\'ees pour une variable discrète mais peuvent facilement être \'etendues à une variable continue en convertissant les sommes en int\'egrales!):
	
	\textbf{D\'efinitions (\#\mydef):}
	\begin{enumerate}
		\item[D1.] Si le "\NewTerm{coefficient d'asym\'etrie}\index{coefficient d'asym\'etrie}" ("skewness\index{skewness}" en anglais), not\'e $\gamma$ en g\'en\'eral, et associ\'e la plupart du temps au moment d'ordre $3$, et \'egalement appel\'e "\NewTerm{coefficient d'asym\'etrie de Fisher-Pearson}\index{coefficient d'asym\'etrie de Fisher-Pearson}":
		
		est nul, la distribution est sym\'etrique (cela ne signifie pas que la sym\'etrie a lieu sur un pic de la distribution car dans le cas bimodal, l'axe de sym\'etrie peut se situer entre les deux valeurs modales sym\'etriques). Si l'asym\'etrie est positive (modale droite), la distribution (valeur modale / m\'ediane) est maigre à droite (ou il existe des valeurs extrêmes à droite). Si l'asym\'etrie est n\'egative (modale gauche), la distribution (la valeur modale / m\'ediane) est inclin\'ee à gauche (ou des valeurs extrêmes se trouvent à gauche).
		
		\begin{tcolorbox}[title=Remarque,colframe=black,arc=10pt]
		Une question commune est de savoir pourquoi l'asym\'etrie est d\'efinie comme le troisième moment central et non comme le cinquième ou un autre nombre? Quelle est la logique derrière tout ça? En fait rien de sp\'ecial! Nous pourrions choisir n'importe quelle puissance \'etrange, mais le fait est que le moment d'ordre $ 1 $ est l'esp\'erance (d\'ejà prise ...) et que le moment d'ordre qui suite le plus facile à g\'erer pour des calculs ult\'erieurs est la puissance de $3$ (nous laissons le soin au lecteur d'imaginer l'int\'egration de la fonction de distribution avec des variables d'ordre telles que $5$ ou plus ...).
		\end{tcolorbox}
		
		Il est possible qu'une distribution soit sym\'etrique sans avoir un troisième moment qui soit \'egal à z\'ero nul. Un contre-exemple simple est toute distribution de student $T$ avec $3$ ou moins de degr\'es de libert\'e (voir la preuve lors de notre \'etude de la distribution correspondante). Dans les \'echantillons, cependant, la sym\'etrie implique un troisième moment nul - mais les \'echantillons ne sont presque jamais parfaitement sym\'etriques et ne sont donc pas d'une grande utilit\'e. Il est \'egalement possible de construire des distributions asym\'etriques qui ont un troisième moment non nul (voir notre livre compagnon R!) !!! Donc, la sym\'etrie n'implique pas n\'ecessairement un troisième moment \'egal à z\'ero et un troisième moment \'egal à z\'ero n'implique pas n\'ecessairement une sym\'etrie!
	
		\item[D2.] Le "\NewTerm{coefficient d'asym\'etrie}\index{coefficient d'asym\'etrie}" (ou "kurtosis\index{kurtosis}" en anglais), not\'e $\kappa$ en g\'en\'eral (voir les maths juste après les d\'efinitions), est associ\'e le plus souvent au moment d'ordre $4$ (donc toujours positif ou \'egal à z\'ero) suivant:
		
		Si nous le normalisons pour avoir "\NewTerm{l'excès du coefficient d'asym\'etrie}\index{excès du coefficient d'asym\'etrie}" d\'efini par $\kappa_1-3$, alors il y a trois sc\'enarios à consid\'erer:
		\begin{itemize}
			\item L'excès du coefficient d'asym\'etrie est nul, c'est-à-dire "\NewTerm {mesokurtique}\index{mesokurtique}", l'aplatissement est alors similaire à celui d'une distribution Normale \footnote{Une distribution Normale a rigoureusement un coefficient d'asym\'etrie nul. Mais le contraire n'est pas vrai! Les distributions qui ne semblent pas du tout Normales peuvent avoir un coefficient d'asym\'etrie \'egal à z\'ero!}.
			
			\item  L'excès du coefficient d'asym\'etrie en excès est sup\'erieure à z\'ero, c'est-à-dire "\NewTerm{leptokurtique}\index{leptokurtique}", la distribution que l'on \'etudie est alors sup\'erieure à celle d'une distribution Normale avec une moyenne \'egale.
			
			\item L'excès du coefficient d'asym\'etrie en excès est inf\'erieur à z\'ero, c'est-à-dire "\NewTerm{platikurtique}\index{platikurtique}", la distribution que l'on \'etudie est donc inf\'erieure à celle d'une distribution Normale avec une moyenne \'egale.
		\end{itemize}
		Rigoureusement, le coefficient d'asym\'etrie mesure à quel point la distribution sous-jacente est sym\'etrique et bimodale (alg\'ebriquement, une distribution parfaitement sym\'etrique et bimodale aura un kurtosis de $1$, ce qui est la plus petite valeur possible d'un kurtosis). La plupart des textes \'el\'ementaires d\'ecrivent le le coefficient d'asym\'etrie comme une mesure du "pic" d'une distribution. Ce terme est trompeur, et un terme bien meilleur pour d\'ecrire le kurtosis est la mesure "bimodalit\'e", où plus le coefficient d'asym\'etrie est bas, plus la bimodalit\'e est grande.
	\end{enumerate}
	En voici une illustration:
	\begin{figure}[H]
		\centering
		\includegraphics[scale=0.9]{img/arithmetics/moments.pdf}
		\caption[Repr\'esentation de plages de valeurs du Kurtosis et Skewness de quelques distributions]{Repr\'esentation de plages de valeurs du Kurtosis et Skewness de quelques distributions (auteur: Simon H. Hess)}
	\end{figure}
	\begin{tcolorbox}[title=Remarque,colframe=black,arc=10pt]
	Si vous effectuez une recherche d'image pour le "kurtosis" sur Google, plusieurs images montrent que le kurtosis (coefficient d'asym\'etrie) n'est qu'une distribution gaussienne avec un moment de second ordre plus grand ou plus petit (c'est-à-dire que la gaussienne est juste plus grande ou plus petite). De telles images sont des repr\'esentations trompeuses des repr\'esentations du kurtosis.
	\end{tcolorbox}
	Il y a eu un assez grand nombre de tentatives visant à donner des mesures ad\'equates à ces notions d'asym\'etrie et d'aplatissement, mais ces mesures souvent utiles peuvent parfois être \'etonnamment contre-intuitives. Par exemple, l'asym\'etrie bas\'ee sur le moment peut être nulle lorsque la distribution est asym\'etrique (contredisant une affirmation que l'on peut trouver \'etonnamment souvent lors de la lecture de textes \'el\'ementaires traitant de l'asym\'etrie...).
	
	Dans les faits, les moments d'une distribution continue, et leurs fonctions comme le kurtosis, nous en disent parfois très peu sur le graphique de sa fonction de densit\'e !!!! Il devrait être parfaitement clair que le kurtosis et l'asym\'etrie ne constituent pas une mesure facilement interpr\'etable ou intuitive celui de la sym\'etrie, de l'unimodalit\'e, de la bimodalit\'e, de la convexit\'e ou de toute autre caract\'erisation g\'eom\'etrique familière d'une courbe. Par cons\'equent, les fonctions des moments (et du kurtosis en tant que cas particulier) ne d\'ecrivent pas les propri\'et\'es g\'eom\'etriques du graphe de la fonction de densit\'e de probabilit\'e. Cela a un sens intuitif: comme une function de densit\'e de probabilit\'e repr\'esente une probabilit\'e au moyen d’une surface, nous pouvons presque librement d\'eplacer la densit\'e de probabilit\'e d’un endroit à un autre, en modifiant radicalement l’apparence de la fonction de densit\'e de probabilit\'e, tout en fixant un nombre fini de moments pr\'ed\'efinis!
	
	Voici les coefficients d'asym\'etrie utilis\'es par les logiciels R (ce dernier dispose de $\gamma_1$, $\gamma_2$, $\gamma_3$), MATLAB™ (ce dernier dispose par d\'efaut de $\gamma_1$, $\gamma_2$), Mathematica (ce dernier dispose par d\'efaut de  $\gamma_1$), et MINITAB/IBM SPSS ainsi que Microsoft Excel (ces derniers disposent par d\'efaut de $\gamma_3$):
	
	
	et les coefficients de Kurtosis utilis\'es par les logiciels (ce dernier dispose de $\kappa_3$, $\kappa_4$, $\kappa_5$), MATLAB™ (ce dernier dispose de $\kappa_1$, $\kappa_2$), Mathematica (ce dernier dispose de $\kappa_1$), et MINITAB/IBM SPSS ainis que Microsoft Excel (ces derniers disposent par d\'efaut de $\kappa_3$):
	
	Parmi les relations ci-dessus, seuls $\gamma_1$, $\gamma_2$, $\kappa_1$ et $\kappa_3$ sont r\'eellement conformes à la norme ISO 3534-1.
	\begin{tcolorbox}[title=Remarque,colframe=black,arc=10pt]
	Il existe au moins $11$ m\'ethodes pour calculer le coefficient d'asym\'etrie. Pour plus de d\'etails, voir l'\'etude de Tabor, J. (2010), \textit{Investigating the Investigative Task: Testing for Skewness - An Investigation of Different Test Statistics and their Power to Detect Skewness}, Journal of Statistics Education, 18, 1-13. www.amstat.org/publications/jse/v18n2/tabor.pdf
	\end{tcolorbox}
	Le coefficient d'asym\'etrie (skewness) $\gamma_1$ peut être exprim\'e en termes du moment non central $\text{E}[X^3]$ en d\'eveloppant sa d\'efinition g\'en\'erale (cette forme est très utile en pratique pour calculer l'asym\'etrie explicite d'une distribution donn\'ee!):
	
	Pour un \'echantillon de $n$ valeurs, un estimateur naturel de l’asym\'etrie de population $\gamma_1$ par la m\'ethode des moments est \'evidemment tel que d\'ejà d\'efini ci-dessus:
	
	Pareil pour le coefficient d'aplatissement (Kurtosis):
	
	Voici une figure donnant une id\'ee de la plage de valeurs du coefficient d'asym\'etrie au carr\'e et de kurtosis pour certaines distributions:
	\begin{figure}[H]
		\centering
		\includegraphics[scale=1]{img/arithmetics/skewness_kurtosis.jpg}
		\caption[Familles bêta et gamma de distributions sur le plan de caract\'erisation de forme]{Familles bêta et gamma de distributions sur le plan de caract\'erisation de forme (source: ?)}
	\end{figure}
	Plus loin, au cours de l’\'etude de certaines distributions de probabilit\'es importantes, nous allons d\'emontrer en d\'etail les valeurs du coefficient d’asym\'etrie (skewness) et d'aplatissement (kurtosis) pour: la distribution uniforme, triangulaire, Normale, de Student et bêta.
	
	Mais avant cela, prouvons un th\'eorème int\'eressant!
	\begin{theorem}
	\'etant donn\'e une variable al\'eatoire $X$ avec une fonction de densit\'e de probabilit\'e sym\'etrique $f(x)$, nous avons $\text{E}(X)=a\text{E}(X)=a$ où $a$ est le point de sym\'etrie en supposant que $\text{E}(X)$ existe.
	\end{theorem}
	
	\begin{dem}
	Premièrement, il est \'evident que la sym\'etrie autour de $a$ signifie selon les hypothèses de l’\'enonc\'e du th\'eorème que:
	
	pour tout $z$.
	
	Supposons que l'esp\'erance $\text{E}(X)$  existe. Alors nous savons que:
	
	R\'e\'ecrivons ceci comme:
	
	qui est \'egal à:
	
	La première int\'egrale est \'evidemment \'egale à $ a $ car $ f (x) $ est une fonction de densit\'e. 
	
	Dans la deuxième int\'egrale, nous faisons le changement de variable $ y = x-a $. Alors cette seconde int\'egrale devient:
	
	En cassant à $ y = 0 $. Nous avons:
	
	Pour la première int\'egrale, effectuons le changement de variable $ z = -y $. On a:
	
	En utilisant $ f (a-z) = f (a + z) $ et quelques inversions avec les signes moins, nous aboutissons à:
	
	Dès lors:
	
	devient:
	
	Et ceci est \'egal à z\'ero! Ensuite nous avons:
	
	\begin{flushright}
		$\blacksquare$  Q.E.D.
	\end{flushright}
	\end{dem}
	
	\begin{tcolorbox}[title=Remarque,colframe=black,arc=10pt]
	L'asym\'etrie et le kurtosis sont les moments fondamentaux pour d\'evelopper le fameux test de normalit\'e de Jarque-Bera ou \'egalement la Value at Risk de Cornish-Fisher (nous verrons tout cela en d\'etails plus loin).
	\end{tcolorbox}	
	
	\subsection{Postulat fondamental de la statistique}
	
	Un des buts ultime de la statistique est de remonter de l'\'echantillon à la fonction de r\'epartition analytique qui lui aurait donn\'e naissance. Ce but sera pr\'esent\'e dans le cadre de ce livre comme un postulat (bien que ce postulat soit relativement difficile à appliquer dans la pratique).

	Postulat: À toute fonction de r\'epartition empirique $\hat{F}_n(x)$ de $n$-r\'ealisation d'une variable al\'eatoire $x$ nous pouvons associer une fonction de r\'epartition th\'eorique $F(x)$ vers laquelle elle converge quand la taille de l'\'echantillon est suffisamment grande.

	Si:
	
	est la variable al\'eatoire d\'efinie comme la plus grande diff\'erence (en valeur absolue) entre $\hat{F}_n(x)$ et $F(x)$ (observ\'ee pour toutes les valeurs de $x$ pour un \'echantillon donn\'e), alors $X_n$ converge vers $0$ presque sûrement.
	
	\begin{tcolorbox}[title=Remarque,colframe=black,arc=10pt]
	Les math\'ematiciens de la statistique d\'emontrent ce postulat de manière rigoureuse sous la forme d'un th\'eorème appel\'e le "\NewTerm{th\'eorème fondamental de la statistique}\index{th\'eorème fondamental de la statistique}" ou "\NewTerm{th\'eorème de Glivenko-Cantelli}\index{Gth\'eorème de Glivenko-Cantelli}" en ce qui concerne les fonctions continues. Personnellement, quitte à choquer les connaisseurs, je considère que cette d\'emonstration n'en est pas une car elle est très \'eloign\'ee de ce que montre l'exp\'erience (oui c'est mon côt\'e physicien qui ressort...) et ce r\'esultat th\'eorique amène un grand nombre de praticiens à faire souvent tout leur possible (exclusion de donn\'ees, transformations et autres abominations) pour trouver une loi connue à laquelle ils peuvent ajuster leurs donn\'ees mesur\'ees.
	\end{tcolorbox}	
	
	\subsection{Indice de diversit\'e}
	
	Il arrive dans le domaine de la biologie ou de l'entreprise que l'on demande à un statisticien ou analyste de mesurer la diversit\'e d'un certain nombre d'\'el\'ements pr\'ed\'efinis. Par exemple, imaginons une multinationale ayant une gamme de produits bien d\'efinie et dont certains magasins (clients) dans le monde peuvent choisir un sous-ensemble de cette gamme pour leur commerce. La question \'etant alors de faire un ranking des magasins qui vendent la plus grande diversit\'e de produits de la marque et ce en prenant en compte aussi les quantit\'es.

	Par exemple, nous avons une liste de $4$ produits au total dans notre catalogue. Le hasard faisant, trois de nos clients vendent nos $4$ produits mais nous souhaiterions savoir lequel en vend la plus grande diversit\'e et ce en prenant en compte les quantit\'es.

	Nous avons les donn\'ees de ventes par produit suivantes pour le client $1$:
	\begin{table}[H]
		\begin{center}
		\begin{tabular}{|c|c|}
			\hline
			\multicolumn{2}{|c|}{\cellcolor{black!30}Client 1} \\ \hline
			Produit 1 & 5 \\ \hline
			Produit 2 & 5 \\ \hline
			Produit 3 & 5 \\ \hline
			Produit 4 & 5 \\ \hline
		\end{tabular}
	\end{center}
	\end{table}
	
	Et pour le client $2$:
	\begin{table}[H]
		\begin{center}
		\begin{tabular}{|c|c|}
			\hline
			\multicolumn{2}{|c|}{\cellcolor{black!30}Client 2} \\ \hline
			Produit 1 & 1 \\ \hline
			Produit 2 & 1 \\ \hline
			Produit 3 & 1 \\ \hline
			Produit 4 & 17 \\ \hline
		\end{tabular}
	\end{center}
	\end{table}
	Et pour le client $3$:
	\begin{table}[H]
		\begin{center}
		\begin{tabular}{|c|c|}
			\hline
			\multicolumn{2}{|c|}{\cellcolor{black!30}Client 3} \\ \hline
			Product 1 & 2 \\ \hline
			Product 2 & 2 \\ \hline
			Product 3 & 2 \\ \hline
			Product 4 & 34 \\ \hline
		\end{tabular}
	\end{center}
	\end{table}
	Une mesure de l'information (diversit\'e des \'etats) qui peut être bien adapt\'ee à cet objectif est la formule de Shannon introduite dans le chapitre de M\'ecanique Statistique dont l'esp\'erance est:
	
	Arbitrairement, nous prendrons $\lambda=1$ et la base $10$ pour le logarithme (ainsi, si nous avons $10$ variables \'equiprobables, l'entropie sera unitaire par exemple...).
	
	Dès lors il vient:
	
	Nous allons r\'ecrire cela de manière plus ad\'equate pour l'application en entreprise. Ainsi, si $n$ est le nombre de produits et equation est la proportion (ou "fr\'equence relative") de ventes du produit $i$ parmi la totalit\'e des ventes $N$ nous avons alors:
	
	Il vient alors:
	
	Nous avons alors pour le client $1$ (nous restons en base $10$ pour le logarithme):
	
	qui est la valeur maximale possible (chaque \'etat est \'equiprobable). Et pour le client $2$ nous avons:
	
	et pour le client 3:
	
	Ainsi, le client ayant la plus grande diversit\'e est le premier. Nous voyons aussi une propri\'et\'e int\'eressante de la formule de Shannon à l'aide des clients 2 et 3 c'est que la quantit\'e n'influe pas sur la diversit\'e (puisque la seule diff\'erence entre les deux clients est la quantit\'e qui est multipli\'ee d'un facteur 2 et non la diversit\'e)!
	
	Notez que si nous mettons tous les produits dans un seul client (ensemble), nous obtenons:
	
	Donc on devine que (sans d\'emonstration math\'ematique g\'en\'erale):
	
	D’où le fait qu’il vaut mieux avoir plusieurs produits au même endroit que à plusieurs endroit (r\'esultat utilis\'e par exemple dans le domaine de la gestion de documents où le but est d’\'eviter que le même type de documents soit enregistr\'e dans diff\'erents lecteurs partag\'es ou bibliothèques).
	
	\pagebreak
	\subsection{Fonctions de distributions (lois de probabilit\'es)}\label{statistical distributions}
	Lorsque nous observons des ph\'enomènes probabilistes, et que nous prenons note des valeurs prises par ces derniers et que nous les reportons graphiquement, nous observons toujours que les diff\'erentes mesures obtenues suivent une caract\'eristique courbe typique qui est parfois ajustable th\'eoriquement avec un bon niveau de qualit\'e.

	Dans le domaine des probabilit\'es et statistiques, nous appelons ces caract\'eristiques des "\NewTerm{fonctions de distribution}" car elles indiquent la fr\'equence avec laquelle la variable al\'eatoire apparaît avec certaines valeurs.

	\begin{tcolorbox}[title=Remarque,colframe=black,arc=10pt]
	Nous utilisons aussi simplement le terme "fonction" ou encore "loi" pour d\'esigner ces caract\'eristiques.
	\end{tcolorbox}	
	
	Ces fonctions sont en pratique born\'ees par ce que nous appelons "\NewTerm{l'\'etendue de la distribution}\index{l'\'etendue de la distributionn}", ou "\NewTerm{dispersion de la distribution}\index{dispersion de la distribution}", qui correspond à la diff\'erence entre la donn\'ee maximale (à droite) et la donn\'ee minimale (à gauche) des valeurs observ\'ees:
	
	not\'ee souvent aussi $R$ (pour "range" en anglais) dans l'ing\'enierie de la qualit\'e (\SeeChapter{voir section d'Ing\'enierie Industrielle page \pageref{industrial engineering}}). Dans la th\'eorie elles sont non n\'ecessairement born\'ees et nous parlons alors (\SeeChapter{voir section d'Analyse Fonctionnelle page \pageref{natural domain of definition}}) de "\NewTerm{domaine de d\'efinition}" ou plus simplement du "\NewTerm{support}\index{support}" de la fonction.

	Si les valeurs observ\'ees se distribuent d'une certaine manière c'est qu'elles ont alors une probabilit\'e (ou probabilit\'e cumul\'ee dans le cadres des fonctions continues) d'avoir une certaine valeur de la fonction de distribution. 

	Dans la pratique industrielle (\SeeChapter{voir section d'Ing\'enierie Industrielle page \pageref{range six sigma}}), l'\'etendue des valeurs statistiques est importante (de même que l'\'ecart-type) parce qu'elle donne une indication sur la variation d'un processus (variabilit\'e).
	
	Si $ L $ d\'enote toute fonction de distribution univari\'ee possible, la plage de la fonction est simplement not\'ee $ L $ si son domaine de d\'efinition est $\mathbb{R}$, sinon, si elle est born\'ee, vous verrez g\'en\'eralement quelque chose comme $L_{]a,b]}$.

	\textbf{D\'efinitions (\#\mydef):}
	\begin{enumerate}
		\item[D1.] La relation math\'ematique qui donne la probabilit\'e qu'a une variable al\'eatoire d'avoir une valeur pr\'ecise de la fonction de distribution est appel\'ee "\NewTerm{fonction de densit\'e}\index{fonction de densit\'e}" (or "\NewTerm{fonction de densit\'e de probabilit\'e}\index{fonction de densit\'e de probabilit\'e}\label{probability density function}"), "\NewTerm{fonction de masse}\index{fonction de masse}" or "\NewTerm{fonction marginale}\index{marginal function}".
		
		\item[D2.] La relation math\'ematique qui donne la probabilit\'e cumul\'ee qu'a une variable al\'eatoire d'être inf\'erieure ou \'egale à une certaine valeur est nomm\'ee la "\NewTerm{fonction de r\'epartition}\index{fonction de r\'epartition}" ou "\NewTerm{fonction cumul\'ee}\index{fonction cumul\'ee}".
		
		\item[D3.] Des variables al\'eatoires sont dites "\NewTerm{ind\'ependantes et identiquement distribu\'ees}\index{ind\'ependantes et identiquement distribu\'ees}" (i.i.d.) si elles suivent toutes la même fonction de distribution et qu'elles sont ind\'ependantes.
	\end{enumerate}
	De telles fonctions \'etant très nombreuses dans la nature, nous proposons au lecteur ci-après une \'etude d\'etaill\'ee des plus connues seulement.

	Indiquons avant d'aller plus loin que si nous notons $X$ une variable al\'eatoire continue ou discrète, il y a plusieurs usages de notation dans la litt\'erature scientifique pour indiquer qu'elle suit une loi de probabilit\'e donn\'ee $L$. Voici les plus courantes:
	
	Dans la pr\'esente section et tout le reste du livre en g\'en\'eral, nous utiliserons la dernière notation.

	Voici la liste des fonctions de distribution que nous allons voir ici ainsi que les fonctions de distributions utilis\'ees couramment dans l'industrie et se trouvant dans d'autres chapitres et celles qui dont la d\'emonstration doit encore être r\'edig\'e:

	\begin{itemize}[noitemsep,nolistsep]
		\item Distribution Uniforme Discrète $\mathcal{U}(a,b)$ (voir plus bas)
		\item Distribution de Bernoulli $\text{B}(1,p)$ (voir plus bas)
		\item Distribution G\'eom\'etrique $\mathcal{G}(N)$ (voir plus bas)
		\item Distribution Binomiale $\mathcal{B}(N,k)$ (voir plus bas)
		\item Distribution Binomiale N\'egative $\text{NB}(N,k,p)$ (voir plus bas)
		\item Distribution Hyperg\'eom\'etrique$\mathcal{H}(n,p,m,k)$ (voir plus bas)
		\item Distribution Multinomiale $\mathcal{H}(\vec{k},\vec{p},m)$ (voir plus bas)
		\item Distribution de Poisson  $\mathcal{P}(\mu,k)$ (voir plus bas)
		\item Distribution de Gauss-Laplace/Loi Normale $\mathcal{N}(\mu,\sigma)$ (voir plus bas)
		\item Distribution Normale repli\'ee (voir plus bas)
		\item Distribution demi-Normale (voir plus bas)
		\item Distribution Log-Normale  $\mathcal{LN}(\mu,\sigma)$ (voir plus bas)
		\item Distribution Uniforme continue (voir plus bas)
		\item Distribution Triangulaire (voir plus bas)
		\item Distribution de Pareto (voir plus bas)
		\item Distribution Exponentielle (voir plus bas)
		\item Distribution de Weibull (voir section de G\'enie Industriel page \pageref{weibull distribution}) 
		\item Distribution Exponentielle G\'en\'eralis\'ee (à r\'ediger...)
		\item Distributions d'Erlang/Erlang-B/Erlang-C (voir section de Gestion Quantitative page \pageref{erlang distribution})
		\item Distribution de Cauchy (voir plus bas)
		\item Distribution Bêta (voir plus bas et de Gestion Quantitative \pageref{beta distribution application})
		\item Distribution Gamma (voir plus bas)
		\item Distribution du Chi-2 (voir plus bas)
		\item Distribution de Student (voir plus bas)
		\item Distribution de Fisher-Snedeco (voir plus bas)
		\item Distribution de Benford (voir plus bas)
		\item Distribution Logistique (voir section M\'ethodes num\'eriques page \pageref{logistic distribution})	
		\item Distribution de Laplace $\mathcal{L}(\mu,b)$ (voir plus bas)
		\item Distribution Normale carr\'ee (voir plus bas)
		\item Generalized extreme value distribution (à r\'ediger)
	\end{itemize}

	\begin{tcolorbox}[title=Remarque,colframe=black,arc=10pt]
	Le lecteur trouvera les d\'eveloppements math\'ematiques de la fonction de distribution de Weibull dans la section G\'enie industriel (page \pageref{weibull distribution}), et la fonction de distribution logistique dans la section M\'ethodes num\'eriques (page \pageref{logistic distribution}).
	\end{tcolorbox}	
	
	\subsubsection{Distribution Uniforme Discrète}
	Si nous admettons qu'il est possible d'associer une probabilit\'e à un \'ev\'enement, nous pouvons concevoir des situations où nous pouvons supposer a priori que tous les \'ev\'enements \'el\'ementaires sont \'equiprobables (c'est-à-dire qu'ils ont même probabilit\'e). Nous utilisons alors le rapport entre le nombre de cas favorables et le nombre de cas possibles pour calculer la probabilit\'e de tous les \'ev\'enements de l'Univers des \'ev\'enements $U$. Plus g\'en\'eralement si $U$ est un ensemble fini d'\'ev\'enements \'equiprobables et A une partie de $U$ nous avons sous forme ensembliste (\SeeChapter{voir section Th\'eorie des Ensembles page \pageref{cardinal}}):
	 
	 Plus commun\'ement, soit $e$ un \'ev\'enement pouvant avoir $N$ issues \'equiprobables possibles. Alors la probabilit\'e d'observer l'issue donn\'ee de l'\'ev\'enement suit une "\NewTerm{distribution uniforme discrète}\index{distribution uniforme discrète}" (ou "\NewTerm{loi uniforme discrète}\index{loi uniforme discrète}") donn\'ee par la relation:
	 
	 Ayant pour esp\'erance (ou moyenne):
	  
	 Si nous nous mettons dans le cas particulier où avec . Nous avons alors (\SeeChapter{voir section S\'equences et S\'eries page \pageref{gauss series}}):
	 
	 Si la variable al\'eatoire $e$ prend toutes les valeurs comprises entre $[a, b]$ (un autre cas sp\'ecial) telle que la distribution sera maintenant not\'ee $\mathcal {U}(a, b)$, alors il devrait être \'evident que nous avons pour l'esp\'erance:
	 
	 Et pour la variance, nous avons (toujours en utilisant les r\'esultats de la section de S\'equences et S\'eries page \pageref{sum of squares integers}):
	 
	 Si la variable al\'eatoire $e$ prend toutes les valeurs comprises entre $[a, b]$ (un autre cas sp\'ecial) telle que la distribution sera maintenant not\'ee $\mathcal{U}(a, b)$, alors il devrait être \'evident que nous avons pour la variance:
	 
	 Par sym\'etrie de la distribution, si toutes les valeurs du domaine de la d\'efinition $[a, b]$ sont prises par la variable al\'eatoire alors nous avons pour la m\'ediane:
	 
	 Voici un exemple de trac\'e de la fonction de distribution de masse et de la fonction de distribution cumulative, respectivement, pour la loi uniforme discrète avec paramètres $\left \lbrace 1,5,8,11,12 \right\rbrace $ (nous voyons que chaque valeur a la même probabilit\'e):
	\begin{figure}[H]
		\begin{center}
			\includegraphics[scale=0.75]{img/arithmetics/law_uniform.jpg}
		\end{center}	
		\caption{Loi Uniforme $\mathcal{U}$ (fonction de densit\'e et de distribution cumulative)}
	\end{figure}
	
	Comme on peut le voir dans le diagramme ci-dessus, la fonction de distribution cumulative peut être \'ecrite:
	
 
	 \begin{tcolorbox}[title=Remarque,colframe=black,arc=10pt]
	Bien \'evidemment, la distribution uniforme discrète n'a pas de valeur modale sp\'ecifique $M_0$!
	\end{tcolorbox}	
	

	\subsubsection{Distribution de Bernoulli}\label{bernoulli distribution}
	Si nous avons affaire à une observation binaire alors la probabilit\'e d'un \'ev\'enement reste constante d'une observation à l'autre s'il n'y a pas d'effet m\'emoire (autrement dit: une somme de variables de Bernoulli, deux à deux ind\'ependantes).
	
	Nous appelons ce genre d'observations où la variable al\'eatoire a valeurs $0$ (faux) ou $1$ (vrai), avec probabilit\'e equation respectivement $p$, des \NewTerm{essais de Bernoulli}\index{essais de Bernoulli}" avec  "\NewTerm{\'ev\'enements contraires à probabilit\'es contraires}".
	
	Ainsi, une variable al\'eatoire $X$ suit une "\NewTerm{distribution de Bernoulli}\index{distribution de Bernoulli}" (ou "\NewTerm{loi de Bernoulli}\index{loi de Bernoulli}") si elle ne peut prendre que les valeurs $0$ ou $1$, associ\'ees aux probabilit\'es $p$ et $q$ de sorte que $q+p=1$ et:
	
	L'exemple classique d'un tel processus est le jeu de pile de face ou de tirage avec remise ou pouvant être consid\'er\'e tel quel (ce dernier cas \'etant très important dans la pratique industrielle). Il est certainement inutile pour le lecteur de v\'erifier formellement que la probabilit\'e cumul\'ee est unitaire...
	
	\begin{tcolorbox}[title=Remarque,colframe=black,arc=10pt]
	L'introduction ci-dessus n'est peut-être pas pertinente pour le monde des affaires (le business), mais nous verrons dans la section Techniques de Gestion (page \pageref{queueing theory}) que la fonction de Bernoulli apparaît naturellement au d\'ebut de notre \'etude de la th\'eorie des files d'attente.
	\end{tcolorbox}	
	
	Remarquons que par extension, si nous consid\'erons $N$ \'ev\'enements où nous obtenons dans un ordre particulier $k$ fois une des issues possible (r\'eussite) et $N-k$ l'autre (\'echec), alors la probabilit\'e d'obtenir une telle s\'erie (de $k$ r\'eussites et $N-k$ \'echecs ordonn\'es dans un ordre particulier) sera donn\'ee par:
	
	avec $N \in \mathbb{N}^{*}$ conform\'ement à ce que nous avions obtenu en combinatoire dans le chapitre de Probabilit\'es!

	 Voici un exemple de trac\'e de la fonction de r\'epartition pour$q=0.3$:

	\begin{figure}[H]
		\centering
		\includegraphics[scale=0.8]{img/arithmetics/law_bernoulli.jpg}	
		\caption{Loi de Bernoulli $\text{B}$ (fonction de distribution cumulative)}
	\end{figure}
	La distribution de Bernoulli a donc pour esp\'erance (moyenne) en choisissant $p$ comme probabilit\'e de l'\'ev\'enement d'int\'erêt:
	
	et pour variance (nous utilisons la relation de Huygens d\'emontr\'ee plus haut):
	
	La valeur modale $M_0$ de la loi de Bernoulli d\'epend des valeurs de $p$ ou $q$. Donc nous avons (cela pourrait être \'evident pour le lecteur):
	
	
	\begin{tcolorbox}[title=Remarque,colframe=black,arc=10pt]
	Bien sûr, la distribution de Bernoulli n’a pas de valeur m\'ediane sp\'ecifique $ M_e $!
	\end{tcolorbox}

	
	\subsubsection{Distribution G\'eom\'etrique}\label{geometric distribution}
	La loi g\'eom\'etrique $\mathcal{G}(N)$ ou "\NewTerm{loi de Pascal}\index{loi de Pascal}" consiste dans une \'epreuve de type Bernoulli, dont la probabilit\'e de succès est $p$ et celle d'\'echec $q=1-p$ sont constantes, que nous renouvelons de manière ind\'ependante jusqu'au premier succès.
	
	Rappelez-vous que lors de notre pr\'esentation de la loi Bernoulli, nous avons d\'eduit une extension à $N$ telle que:
	
	Par cons\'equent, la probabilit\'e d'obtenir le premier succès $k = 1$ après $N$ essais, est donn\'ee par:
	
	avec $N \in \mathbb{N}^{*}$.
	
	Comme le lecture peut le constater, plus $N$ est grand, plus petite est la probabilit\'e $\mathcal{G}(N) $. Cela peut sembler non logique mais en fait cela l'est! En effet, dans la phrase "\textit{la probabilit\'e d’obtenir le premier succès après $N$ essais}", il ne faut pas oublier qu'il y est \'ecrite \underline{après} et non pas \underline{pendant}!
	
	Donc bien sûr ... la probabilit\'e d'avoir $N-1$ \'echecs suivis de $1$ succès sera toujours plus petite lorsque $N$ augmentera (regardez la figure un peu plus bas pour pour $ p = 0.5 $, cela peut aider comprendre).
	
	Cette loi a pour esp\'erance:
	
	Cependant, la dernière relation peut aussi s'\'ecrire:
	
	En effet, nous avons prouv\'e dans la section S\'equences et S\'eries (page \pageref{geometric series}) lors de notre \'etude des s\'eries g\'eom\'etriques que:
	
	geometric series $n\rightarrow +\infty$  nous obtenons:
	
	car $0\leq q < 1$. 
	
	Ensuite, il suffit de d\'eriver les deux membres de l'\'egalit\'e par rapport à $q$ et nous obtenons:
	
	Ceci fait, continuons ...
	
	Nous avons donc le nombre moyen d'essais $X$ qu'il faut faire pour arriver au premier succès (ou autrement dit: le rang esp\'er\'e (nombre d'essais esp\'er\'e) pour voir le premier succès):
	
	
	Calculons maintenant la variance en rappelant comme à chaque fois que (relation de Huyghens):
	
	Commençons donc par calculer $\text{E}(X^2)$:
	
	Le dernier terme de cette expression est l'\'equivalent de l'esp\'erance calcul\'ee pr\'ec\'edemment. Soit:
	
	Il reste à calculer:
	
	Nous avons:
	
	Or en d\'erivant l'\'egalit\'e:
	
	Nous obtenons:
	
	Dès lors:
	
	Donc:
	
	Enfin, s’agissant de classer la variance attendue du premier succès (c'est-à-dire la valeur attendue de la variance avant les premiers essais r\'eussis):
	
	La valeur modale est facile à obtenir car nous devons trouver la valeur de $N$ qui maximise la d\'efinition de la loi g\'eom\'etrique:
	
	et nous esp\'erons qu’il est imm\'ediat pour le lecteur que ceci est satisfait lorsque $ N = 1 $:
	
	D\'eterminons maintenant la m\'ediane $M_e$. Pour cela, nous savons par d\'efinition que nous devons avoir:
	
	Mais on peut r\'e\'ecrire:
	
	Dès lors (en base $10$):
	
	Enfin, en nous basant sur notre d\'efinition de la m\'ediane, nous obtenons:
	
	Maintenant, d\'eterminons la fonction cumulative de la loi g\'eom\'etrique. Nous partons de:
	
	Ensuite, nous avons par d\'efinition de la probabilit\'e cumulative, la probabilit\'e de $N$ r\'eussites lors des premiers essais donn\'ee par:
	
	avec $N$ \'etant \'evidemment un entier de valeurs $ 0,1,2,\ldots$
	
	Nous \'ecrivons
	
	Nous avons alors pour le function de probabilit\'e cumul\'ee:
	
	\begin{tcolorbox}[colframe=black,colback=white,sharp corners]
	\textbf{{\Large \ding{45}}Exemple:}\\\\
	Vous essayez, tard dans la nuit et dans le noir, d'ouvrir une serrure avec un trousseau comportant cinq cl\'es, sans faire attention, car vous êtes un peu fatigu\'e (ou un peu \'em\'ech\'e ...), vous allez essayer chaque cl\'e. Sachant qu'une seule cl\'e fonctionnera, quelle est la probabilit\'e d'utiliser la bonne cl\'e au $N$-essai?\\
	
	La solution est:
	
	\end{tcolorbox}
	
	Trac\'e de la fonction de densit\'e et de la fonction de distribution cumulative pour la loi g\'eom\'etrique avec paramètre $p=0.5$:
	\begin{figure}[H]
		\begin{center}
			\includegraphics{img/arithmetics/law_geometric.jpg}
		\end{center}	
		\caption{Loi g\'eom\'etrique $\mathcal{G}$ (fonction de densit\'e et de distribution cumulative)}
	\end{figure}
	
	\subsubsection{Distribution Binomiale}\label{binomial distribution}
	Revenons maintenant à notre \'epreuve de Bernoulli. Plus g\'en\'eralement, tout $N$-uplet particulier form\'e de $k$ succès et de $N-k$ \'echecs aura pour probabilit\'e (dans le cadre d'un tirage avec remise ou sans remise si la population est grande en première approximation...):	
	
	d'être tir\'e (ou d'apparaître) quel que soit l'ordre d'apparition des \'echecs et r\'eussites (le lecteur aura peut-être remarqu\'e qu'il s'agit d'une g\'en\'eralisation de la loi g\'eom\'etrique, il suffit de poser $k = 1$ pour retrouver la loi g\'eom\'etrique).
	
	Mais, nous savons que la combinatoire permet de d\'eterminer le nombre de $N$-uplets de ce type (le nombre de manières d'ordonner les apparitions d'\'echecs et de r\'eussites). Le nombre d'arrangements possibles \'etant, nous l'avons d\'emontr\'e (\SeeChapter{voir section de Probabilit\'es page \pageref{choice function}}), donn\'e par le coefficient binomial (notation - pour rappel - non conforme dans ce livre à la norme ISO 31-11):
	
	Donc comme la probabilit\'e d'obtenir une s\'erie de $k$ succès et $N-k$ \'echecs particuliers est toujours identique (quel que soit l'ordre) alors il suffit de multiplier la probabilit\'e d'une s\'erie particulière par la combinatoire (ceci \'etant \'equivalent à faire une somme):
	
	pour avoir la probabilit\'e totale d'obtenir une quelconque de ces s\'eries possibles (puisque chacune est possible).
	
	\begin{tcolorbox}[title=Remarque,colframe=black,arc=10pt]
	Cela \'equivaut à l'\'etude d'un tirage avec remise (\SeeChapter{voir section de Probabilit\'es page \pageref{choice function}}) simple avec contrainte sur l'ordre ou à l'\'etude d'une s\'erie de succès ou d'\'echecs. Nous utiliserons cette relation dans le cadre de la th\'eorie des files d'attentes ou en fiabilit\'e (\SeeChapter{voir section de G\'enie Industriel page \pageref{industrial engineering}}). Il faut noter que dans le cas de grandes populations, même si le tirage n'est pas avec remise il peut être consid\'er\'e comme tel...
	\end{tcolorbox}
	
	\'ecrite autrement ceci donne la "\NewTerm{distribution Binomiale}\index{distribution Binomiale}" (ou "\NewTerm{loi Binomiale}\index{loi Binomiale}") connue aussi sous la forme de la fonction de distribution suivante:
	
	et parfois not\'ee par $\beta(n,p)$ avec $n$ minuscule ou $N$ majuscule (ce n'est pas vraiment imporant) et peut être calcul\'ee dans la version française de Microsoft Excel 11.8346 à l'aide de la fonction \texttt{LOI.BINOMIALE( )}.
	
	Nous disons parfois que la loi Binomiale est non exhaustive car la taille de la population initiale n'est pas apparente dans l'expression de la loi.
	
	\begin{tcolorbox}[title=Remarque,colframe=black,arc=10pt]
	La distribution binomiale est nomm\'ee "\NewTerm{distribution binomiale sym\'etrique}\index{distribution binomiale sym\'etrique}" lorsque $p=0.5$.
	\end{tcolorbox} 
	
	\begin{tcolorbox}[colframe=black,colback=white,sharp corners]
	\textbf{{\Large \ding{45}}Exemples:}\\\\
	E1. Nous souhaitons tester l'alternateur d'un groupe \'electrogène. La probabilit\'e de d\'efaillance à la sollicitation de ce mat\'eriel est estim\'ee à $1$ d\'efaillance pour $1,000$ d\'emarrages.\\

	Nous d\'ecidons d'effectuer un test de $100$ d\'emarrages. La probabilit\'e d'observer $1$ panne au cours de ce test est de:
	
	E2.\label{prosecutor fallacy frequentist example} Peut-être vous souvenez-vous de l'erreur du procureur que nous avons introduite à la page \pageref{prosecutor fallacy}. Revenons à l'exemple li\'e à l'\'echantillon d'ADN sur le lieu du crime. Maintenant que nous avons introduit la distribution binomiale, nous pouvons calculer la probabilit\'e d'obtenir au moins une correspondance parmi les enregistrements:
	
	Cette preuve est donc à elle seule un r\'esultat de hackage de donn\'ees peu convaincant. Si le coupable se trouvait dans la base de donn\'ees, lui et d'autres individus seraient aussi d\'etect\'es comme coupables probables ; dans un cas comme dans l'autre, il serait fallacieux d'ignorer le nombre d'enregistrements recherch\'es lors de l'\'evaluation des preuves. Les "correspondances à froid" comme celui-ci sur les banques de donn\'ees ADN doivent maintenant être pr\'esent\'ees avec soin comme preuves dans les procès.
	\end{tcolorbox}
	Nous avons bien \'evidemment pour la fonction de r\'epartition (très utile dans la pratique comme le contrôle de lots de fournisseurs ou la fiabilit\'e comme nous le verrons dans la section de G\'enie Industriel page \pageref{sampling plans}!):
	
	Effectivement, nous avons d\'emontr\'e dans la section de Calcul Alg\'ebrique (page \pageref{binomial theorem}) le "\NewTerm{th\'eorème binomial}\index{th\'eorème binomial}":
	
	Donc:
	
	Il vaut mieux utiliser Microsoft Excel 11.8346 (ou tout autre logiciel largement r\'epandu) pour ne pas s'embêter à calculer ce genre de relations en utilisant la fonction \texttt{CRITERE.LOI.BINOMIALE()} dans la version française.
	
	L'esp\'erance de $\mathcal{B}(N,k)$ est alors donn\'ee par:
	
	
	Mais comme nous avons:
	
	Nous obtenons finalement
	
	donne le nombre moyen de fois que l'on obtiendra l'issue souhait\'ee de probabilit\'e $p$ après $N$ essais.

	L'esp\'erance de la loi binomiale est aussi parfois not\'ee dans la litt\'erature sp\'ecialis\'ee sous la forme suivante si $r$ est le nombre potentiel d'issues attendues possibles dans une population de taille $n$:
	
	Avant de calculer la variance, introduisons la relation suivante:
	
	En effet, en utilisant les d\'eveloppements pr\'ec\'edents:
	
	On reconnaît dans la dernière \'egalit\'e la fonction de distribution cumulative \'egale à $1$. Par cons\'equent:
	
	Commençons maintenant le (long) calcul de la variance de la loi binomiale dans lequel nous allons utiliser les r\'esultats pr\'ec\'edents:
	
	Finalement:
	
	L’\'ecart type de la distribution binomiale est parfois not\'e dans la litt\'erature sp\'ecialis\'ee de la manière suivante si $ r $ est le nombre potentiel de r\'esultats attendus dans une population de taille $ n $ et que $ s $ est le nombre non attendu:
	
	Voici un exemple de trac\'e de la fonction de distribution et respectivement de r\'epartition de la loi binomiale $\mathcal{B}(10,0.5)$:
	\begin{figure}[H]
		\begin{center}
			\includegraphics{img/arithmetics/law_binomial.jpg}
		\end{center}	
		\caption{Loi binomiale $\mathcal{B}$ (fonction de densit\'e et de distribution cumulative)}
	\end{figure}
	
	\label{normalized variance and mean of binomial distribution}Il pourrait être utile de noter que certains employ\'es dans les entreprises (en particulier les m\'edecins sp\'ecialis\'es dans l'analyse de la survie - les \'etudes cliniques - comme nous le verrons plus tard) normalisent le calcul de la moyenne et de l'\'ecart type par rapport à l'unit\'e de $N$. Nous avons alors:
	
	
	\begin{tcolorbox}[colframe=black,colback=white,sharp corners]
	\textbf{{\Large \ding{45}}Exemple:}\\\\
	Sur un \'echantillon de $100$ travailleurs, $25\%$ sont en retard au moins une fois par semaine. L'esp\'erance du nombre de retard est alors:
	
	Rapport\'e à l'unit\'e de $N$, cela nous donne:
	
	\end{tcolorbox}
	Calculons maintenant la valeur modale. Parce que la fonction est discrète, nous ne pouvons pas utiliser la d\'eriv\'ee bien \'evidemment. Donc, nous allons utiliser une astuce. Nous calculons le ratio:
		
	et nous v\'erifions que ce rapport est $>1$ pour chaque $k<k^{*}$ et $\leq 1$ pour chaque $k\geq k^{*}$, pour un entier entier $k^{*}$ c'est la valeur de $k$ correspondant à la valeur modale.
	 
	 Soit $a_k=P(X=k)$. Nous avons:
	 
	 Nous calculons le ratio $\dfrac{a_{k+1}}{a_k}$. Notez que:
	 
	 L’important maintenant est d’analyser:
	  
	 en fonction de la valeur de $k$. Tout d'abord, nous pouvons voir que ce rapport est \'egal à $1$ et nous devons donc deux valeurs modales si:
	 
	 C'est-à-dire si $ k = np + p-1 = p (n + 1) -1 $. Cela peut être consid\'er\'e comme le point d’int\'erêt limite. Mais n'oubliez pas que nous recherchons le $k$ tel que le rapport soit inf\'erieur à $1$. Nous essayons donc deux valeurs:
	 
	 En injectant cela dans notre ratio, nous voyons que:
	 
	 Est la valeur que nous recherchions! Enfin, il existe deux situations possibles pour les modes. Une valeur modale unique et une double valeur modale.
	
	Comme nous le savons, la valeur médiane est la valeur de $ X $ telle que nous avons:
	
	Mais nous n'avons pas encore trouvé de preuve facile pour déterminer $ M_e $ dans le cas général de la loi binomiale.
	
	Pour clore concernant avec la loi binomiale, nous allons développer un résultat qui nous sera indispensable pour construire le test de données appariées de McNemar d'un tableau (carré) de contingence (et comme il est carré il est in extenso dichotomique) que nous étudierons dans le section de Méthodes Numériques (page \pageref{mcnemar test}).

	Nous avons besoin pour ce test de calculer la covariance de deux variables aléatoires binomiales appariées (raison pour laquelle la covariance est non  nulle):
	
	Comme elles sont appariées, cela signifie que:
	
	et donc:
	
	Maintenant, vient la difficulté qui est de calculer $\text{E}(n_in_j)$. Pour calculer ce terme il n'existe pas à notre connaissance d'autres méthodes que de chercher la loi du couple (parfois on peut contourner cela). Dans le cas présent il s'agit d'une loi multinomiale (plus précisément: trinomiale) qu'il est d'usage d'écrire sous la forme:
	
	que nous noterons temporairement pour la suite sous la forme suivante afin de condenser l'écriture:
	
	Nous avons donc une loi trinomiale car  nous cherchons le nombre de fois d'avoir l'événement $k$, l'événement $l$ et ni l'un ni l'autre (donc le reste du temps).  
	\begin{figure}[H]
		\begin{center}
			\includegraphics[scale=0.5]{img/arithmetics/trinomial_distribution.jpg}
		\end{center}	
		\caption{Distribution trinomoial avec $p=1/5,q=2/5$ et $n=5$}
	\end{figure}
	Nous avons alors:
	
	Si $k \geq 1$ et $l \geq 1$, nous avons:
	
	Maintenant utilisons cette relation dans l'espérance conjointe:
	
	Considérons le cas où $n$ vaut $2$. Nous avons:
	
	où la somme est réduite à un seul terme car si nous prenons par exemple $ k = 2, l = 1 $, nous obtenons une factorielle négative au dénominateur.
	
	et pour $n$ valant $3$, le résultat sera aussi $1$, et ainsi de suite (nous supposerons afin de simplifier... que quelques exemples numériques suffiront au lecteur pour le convaincre de la généralité de cette propriété parce que c'est vraiment très laborieux à écrire en \LaTeX). 
	
	Nous avons alors:
	
	Donc au final\label{covariance trinomial distribution}:
	
	Et c’est le principal résultat dont nous aurons besoin pour l’étude du test de McNemar.
	
	Soit $X$ et $Y$ des variables aléatoires binomiales indépendantes avec les paramètres $(n, p)$ et $(m, p)$, respectivement. Soit $Z = X + Y$. Alors la distribution de $Z$ est donnée par:
	
	La distribution binomiale est donc bien stable par addition:
	
	En faisant le même développement avec des probabilités différentes et sans oublier que $p_i + p_j = 1$, nous obtenons:
	

	\subsubsection{Distribution Négative Binomiale}\label{distribution négative binomiale}
	
	La loi binomiale négative s'applique dans la même situation que la loi binomiale mais elle donne la probabilité d'avoir $E$ échecs avant la $R$-ème réussite quand la probabilité de succès est $p$ (ou inversement la probabilité d'avoir $R$ réussites avant le $E$-ème échec quand la probabilité d'échec est $p$).

	Introduisons cette distribution par l'exemple. Considérons pour cela les probabilités suivantes:
	
	Imaginons que nous ayons fait $10$ essais et que nous voulions nous arrêter à la troisième réussite et que le $10$-ème essai est la troisième réussite! Nous allons noter cela:
	
	Mettons en évidence les réussites (R) et échecs (E):
	
	Nous avons donc $7$ échecs et $3$ réussites. Dans le cadre d'une expérience où les tirages sont indépendants (ou pouvant être considérés comme tel...), la probabilité que nous avions d'obtenir ce résultat particulier est alors:
	
	Mais l'ordre des succès et échecs dans la partie entre crochets n'a aucune importance. Donc comme nous avons $2$ succès parmi $9$ dans les crochets il vient que la probabilité d'obtenir le même résultat indépendamment de l'ordre est alors en utilisant la combinatoire:
	
	Ce qui correspond donc à la probabilité d'avoir $7$ échecs avant la $3$ème réussite (ou autrement vu: $3$ réussites après $10$ essais). Ce qui s'écrit avec Microsoft  Excel 14.0.6123 ou ultérieur en français ($7+3=10$ essais, $7$ échecs dont $3$ réussites):
	\begin{center}
		\texttt{=LOI.BINOMIALE.NEG.N(7,3,0.2,0)=0.0604}
	\end{center}
	Généralisons l'écriture antéprécédente notant $k$ le nombre d'échecs, $N$ le nombre total d'essais et $p$ la probabilité d'une réussite:
	
	Il y a cependant plusieurs écritures possibles car la relation précédente n'est pas très intuitive à mettre en pratique comme l'aura peut-être remarqué le lecteur. Ainsi, si nous notons $k$ comme étant le nombre de succès et non le nombre d'échecs, nous avons alors (écriture la plus courante selon moi parmi d'autres équivalentes) la probabilité suivante d'avoir un $N-k$ réussites avant d'avoir un nombre $k$ d'échecs avec un probabilité d'échec $p$ (ou d'échecs avant d'avoir $k$ réussites... c'est symétrique!):
	
	donc la comparaison avec la formulation de la loi binomiale démontrée plus haut est alors peut-être plus évidente!
	
	Il est cependant plus courant de noter la relation précédente en faisant disparaître $N$ car pour l'instant l'écriture n'est toujours pas très claire. Pour cela, nous notons $R$ le nombre de réussites, $E$ le nombre d'échecs, $p$ la probabilité d'une réussite et il vient alors la probabilité d'avoir $R$ réussites après $E$ échecs (c'est beaucoup plus clair...):
	
	Nous trouvons aussi parfois cette dernière relations sous la forme suivante en utilisant explicitement le coefficient binomial:
	
	La probabilité cumulée que nous ayons au moins R réussites avant le E-ème échec vient immédiatement:
	
	
	\begin{tcolorbox}[title=Remarque,colframe=black,arc=10pt]
	Le nom de cette loi provient du fait que certains statisticiens utilisent une définition d'un coefficient combinatoire avec valeur négative pour l'expression de la fonction. Comme c'est une forme plutôt rare, nous ne souhaitons pas la démontrer. Il faut savoir aussi que cette loi est aussi connue sous le nom de "\NewTerm{loi de Pascal}\index{loi de Pascal}" (au même titre que la loi géométrique...) en l'honneur de Blaise Pascal et de "\NewTerm{loi de Pólya}\index{loi de Pólya}", en l'honneur de George Pólya. 
	\end{tcolorbox}
	
	\begin{tcolorbox}[colframe=black,colback=white,sharp corners]
\textbf{{\Large \ding{45}}Exemples:}\\\\
	E1. Un contrôle de qualité long terme nous a permis de calculer l'estimateur de proportion $p$ des pièces non-conformes comme valant $2\%$ à la sortie d'une ligne de production. Nous souhaiterions savoir la probabilité cumulée d'avoir $200$ pièces bonnes avant que la 3ème pièce défectueuse apparaisse. Avec Microsoft Excel 14.0.6123 ou ultérieur en français il vient en utilisant la loi binomiale négative:
	\begin{center}
		\texttt{=LOI.BINOMIALE.NEG.N(200,3,0.02,1)=77.35\%}
	\end{center}
	E2. Pour comparer avec la loi binomiale, demandons-nous quelle est la probabilité cumulée de tirer $198$ pièces non-défectueuses parmi $201$ avec Microsoft Excel 14.0.6123 ou ultérieur en français:
	\begin{center}
		\texttt{=LOI.BINOMIALE.N(198,201,0.98,1)=76.77\%}
	\end{center}
	Nous voyons donc que la différence est faible. Au fait la différence entre les deux lois est dans la pratique quasiment toujours tellement faible que nous n'utilisons alors que la loi binomiale (mais il faut quand même être prudent!).
	\end{tcolorbox}
	Comme à l'habitude, déterminons maintenant la variance et l'espérance de cette loi. Commençons par l'espérance d'avoir $R$ réussites lors de l'apparition du $E$-ème échec sachant que la probabilité d'avoir un échec est $p$. Pour cela nous allons utiliser une astuce très simple et géniale (tout l'art était d'y penser...). Si nous reprenons notre exemple de départ:
	
	et que nous le réécrivons sous la forme suivante:
	
	Nous remarquons alors que la troisième réussite $R$ de la première écriture peut être décomposée en la somme de trois variables aléatoires géométriques telle que:
	
	Avec dans le cas du présent exemple particulier $n=3$ correspondant au fait à $E=3$. Donc en toute généralité la somme de $n$ variables aléatoires géométriques donne toujours une loi binomiale négative si la probabilité $p$ est égale pour chaque variable géométrique! Bref... comme nous avons démontré l'expression de l'espérance et la variance de la loi Géométrique comme étant (donnant donc l'espérance du rang du premier échec):
	
	puisque les variables aléatoires sont de même paramètres et indépendantes il vient alors pour la loi binomiale négative l'esprance du rang d'avoir le $E$-ième échec:
	
	Et alors pour la variance de la loi binomiale négative:
	
	Donc l'espérance et la variance du rang (correspondant donc bien évidemment au nombre d'essais $N$ ou autrement vu: à l'espérance du nombre de réussites en faisant la simple soustraction $X - E$) d'avoir le $E$-ième échec est donc pour résumer:
	
	Ainsi, en posant $E = 1$, nous retombons sur l'espérance et la variance de la loi géométrique.
	
	Maintenant, notons $Y$ la variable aléatoire représentant le nombre d'essais \underline{avant} d'avoir la $R$-ième réussite. Nous avons alors les expressions suivantes de la variance et de l'espérance qui sont très courantes dans la littérature (il s'agit des expressions de l'espérance et de la variance telles que nous pouvons les trouver pour la loi binomiale négative sur Wikipédia par exemple)
	
	\begin{tcolorbox}[colframe=black,colback=white,sharp corners]
	\textbf{{\Large \ding{45}}Exemple:}\\\\
	Quel est le nombre de tirages espérés auquel nous pouvons nous attendre lorsque nous tomberons sur la troisième pièce non-conforme, sachant que la probabilité d'une pièce non-conforme est de $2\%$?
	
	et pour la variance:
	
	\end{tcolorbox}
	Ci-dessous le lecteur trouvera comme à l'habitude un exemple de tracé de la fonction de distribution et répartition pour la fonction binomiale négative de paramètres $\text{NB}(N,k,p)=P(N,3,0.6)$ basé sur l'exemple du début mais avec comme seule différence d'avoir pris comme probabilité de réussite de $60\%$ au lieu de $20\%$.
	
	Ainsi, il y a $21.6\%$ de probabilité d'avoir la 3ème réussite au 3ème essai successif (donc 0 essai de plus que le nombre de réussites), $25.92\%$ de probabilité d'avoir la 3ème réussite au 4ème essai successif (donc 1 essai de plus que le nombre de réussites), $20.7\%$ de probabilité d'avoir la 3ème réussite au 5ème essai successif (donc 2 essais de plus que le nombre de réussites) et ainsi de suite...:	
	\begin{figure}[H]
		\begin{center}
			\includegraphics{img/arithmetics/law_binomial_negative.jpg}
		\end{center}	
		\caption{Loi binomiale négative $\texttt{NB}$ (fonction de densité et de distribution cumulative)}
	\end{figure}
	
	Les distributions ci-dessus sont tronquées à $9$ (correspondant donc à $12$ essais) mais continue théoriquement à l'infini. Ce qui différencie particulièrement les lois binomiale et géométrique de la loi binomiale négative sont les queues de la distribution.

	La distribution binomiale négative a une place importante dans une technique de régression spéciale que nous verrons dans la section de Méthodes Numériques \pageref{regression techniques}.

	\subsubsection{Distribution Hypergéométrique}\label{hypergeometric distribution}
	
	Nous considérons pour approcher cette fonction un exemple simple (mais guère intéressant dans la pratique) qui est celui d'une urne contenant $n$ boules dont $m$ sont noires et les autres $m'$ blanches (pour plusieurs exemples concrets utilisés dans l'industrie se reporter à la section de Génie Industriel page \pageref{quality control}). Nous tirons successivement, et sans les remettre dans l'urne, p boules. La question est de trouver la probabilité que parmi ces $p$ boules, il y en ait $k$ qui soient noires (dans cet énoncé l'ordre du tirage ne nous intéresse donc pas!).

	Nous parlons souvent de "tirage exhaustif" avec la loi hypergéométrique car contrairement à la loi binomiale, la taille du lot qui sert de base au tirage va apparaître dans la loi.
	
	\begin{tcolorbox}[title=Remarque,colframe=black,arc=10pt]
	Cela équivaut donc à l'étude non ordonnée d'un tirage sans remise (\SeeChapter{voir section de Probabilités page \pageref{choice function}}) avec contrainte sur les occurrences appelé parfois "tirage simultané". Nous utiliserons souvent la distribution hypergéométrique dans le domaine de la qualité ou de la fiabilité où les boules noires sont associées à des éléments avec défauts et les blanches à des éléments sans défauts.
	\end{tcolorbox}
	
	Les $p$ boules peuvent être choisies parmi les $n$ boules de $C_p^n$ façons (représentant donc le nombre de tirages différents possibles) avec pour rappel (\SeeChapter{voir section de Probabilités page \pageref{choice function}}):
	
	Les $k$ boules noires peuvent être choisies parmi les $m$ noires de $C_k^m$ façons. Les $p-k$ boules blanches peuvent être elles choisies de $C_{p-k}^{n-m}$ façons. Il y a donc $C_k^mC_{p-k}^{n-m}$ tirages qui donnent $k$ boules noires et $p-k$ boules blanches.
	
	La probabilité recherchée vaut donc (nous en verrons une autre formulation possible dans la section de Génie Industriel page \pageref{quality control}):
	
	\pageref{quality control}
	et est dite suivre une "\NewTerm{distribution Hypergéométrique}\index{distribution Hyperg\'eom\'etrique}" (ou "\NewTerm{loi Hypergéométrique}\index{loi Hyperg\'eom\'etrique}") et peut être obtenue heureusement de manière directe dans Microsoft Excel 11.8346 avec la fonction \texttt{LOI.HYPERGEOMETRIQUE()}.
	
	\begin{tcolorbox}[colframe=black,colback=white,sharp corners]
\textbf{{\Large \ding{45}}Exemples:}\\\\
	E1. Nous souhaitons mettre en production un petit développement informatique de $10'000$ lignes de code ($n$). Le retour d'expérience montre que la probabilité de défaillance est de $1$ bug pour $1'000$ lignes de code (soit $0.1\%$ de $10'000$ lignes) ce qui correspond à valeur de $m$.\\
	
		Nous testons environ $50\%$ des fonctionnalités du logiciel au hasard avant l'envoi au client (soit l'équivalent de $5'000$ lignes de code correspondant à $p$). La probabilité d'observer $5$ bugs ($k$) est avec Microsoft Excel 11.8346:
	\begin{center}
		\texttt{LOI.HYPERGEOMETRIQUE(k,p,m,n)=}\\
		\texttt{LOI.HYPERGEOMETRIQUE(5,5000,1\%*10000,10000)=24.62\%}
	\end{center}

	E2. Dans une petite production unique d'un lot de $1'000$ pièces nous savons que $30\%$ en moyenne sont mauvaises à cause de la complexité des pièces par retour d'expérience d'une fabrication précédente similaire. Nous savons qu'un client va tirer $20$ pièces au hasard pour décider d'accepter ou de rejeter le lot. Il ne rejettera pas le lot s'il trouve zéro pièce défectueuse parmi ces $20$. Quelle est la probabilité d'en avoir exactement $0$ de défectueuse?
	\begin{center}
		\texttt{=LOI.HYPERGEOMETRIQUE(0,20,300,1000)=0.073\%}
	\end{center}
	et comme on exige un tirage nul, le calcul de la loi hypergéométrique se simplifie en:
	
	\end{tcolorbox}
	Il n'est pas interdit de faire le calcul direct de l'espérance et de la variance de la distribution hypergéométrique mais le lecteur pourra sans trop de peine imaginer que ce calcul va être... relativement indigeste. Alors nous pouvons utiliser une méthode indirecte qui de plus est intéressante!

	D'abord le lecteur aura peut-être, même certainement, remarqué qu'au fait l'expérience de la loi hypergéométrique est une série d'essais de Bernoulli (sans remise bien entendu!).

	Alors, nous allons tricher en utilisant dans un premier temps la propriété de linéarité de l'espérance. Définissons pour cela une nouvelle variable correspondant implicitement au fait à l'expérience de la distribution hypergéométrique ($k$ essais de Bernoulli à la suite!):
	
	où $X_i$ représente la réussite d'obtenir au $i$-ème tirage une boule noire (soit $0$ ou $1$). Or, nous savons que pour tout $i$ la variable aléatoire $X_i$ suit une distribution de Bernoulli pour laquelle nous avons démontré lors de notre étude de la loi de Bernoulli que $\text{E}(X_i)=p$.

	Dès lors, de par la propriété de linéarité de l'espérance nous avons (attention ici $p$ n'est plus le nombre de boules mais la probabilité associée à une issue attendue!):
	
	Dans l'essai de Bernoulli, $p$ est donc la probabilité d'obtenir l'élément recherché (pour rappel...). Dans la loi hypergéométrique ce qui nous intéresse est la probabilité d'avoir une boule noire (qui sont en quantité $m$, avec donc $m'$ boules blanches) par rapport à la quantité totale de boules n. Et le rapport nous donne évidemment cette probabilité. Ainsi, nous avons:
	
	où $k$ est le nombre de tirages (attention à ne pas confondre avec la notation de l'énoncé initial où il était noté par la variable $p$!). Cette espérance donne donc le nombre moyen de boules noires lors d'un tirage de $k$ boules parmi $n$, dont $m$ sont connues comme étant noires. Le lecteur aura remarqué que l'espérance de la loi hypergéométrique est donc la même que la loi binomiale!
	
	Pour déterminer la variance, nous allons utiliser la variance de la loi de Bernoulli et la relation suivante démontrée lors de l'introduction de l'espérance et de la covariance au début de ce chapitre:
	
	Dons en rappelant que nous avons $X=\displaystyle\sum_{i=1}^k X_i$, il vient:
	
	Or, pour la loi de Bernoulli, nous avons:
	
	Alors nous avons déjà:
	
	Ensuite, nous avons facilement:
	
	Le calcul de $\text{E}(X_iX_j)$ nécessite lui une bonne compréhension des probabilités (ce sera un bon rappel!).
	
	L'espérance $\text{E}(X_iX_j)$ est donnée (implicitement) par la somme pondérée des probabilités que deux événements aient lieu en même temps comme nous le savons. Or, nos événements sont binaires: soit c'est une boule noire ($1$) soit c'est une boule blanche ($0$). Donc tous les termes de la somme n'ayant pas deux boules noires consécutivement seront nuls!
	
	Le problème est alors de calculer la probabilité d'avoir deux boules noires consécutives et celle-ci s'écrit donc:
	
	Donc nous avons finalement:
	
	Soit:
	
	Finalement (en utilisant le résultat de la série de Gauss vu dans la section des Séquences et Séries page \pageref{gauss series}):
	
	où nous avons utilisé le fait que:
	
	est composé de:
	
	termes puisqu'il correspond au nombre de façons qu'il y a de choisir le couple $(i, j)$ avec $i<j$.

	Donc finalement:
	
	Nous pouvons écrire:
	
	Dans la littérature spécialisée, nous retrouvons souvent la variance écrite sous la forme suivante en notant comme pour lors de notre étude de la loi Normale l'événement attendu $r$ et l'événement non-attendu $s$:
	
	avec donc $l = n - k$. Cette dernière forme d'écriture nous sera très utile dans la section de Méthodes Numériques lors de notre étude du test de Haenzel-Mantel (page \pageref{cochran mantel test}).
	
	De plus nous voyons que dans:
	
	il y a l'écart-type de la distribution binomiale, à la différence d'un facteur qui est noté:
	
	que l'on retrouve assez souvent en statistiques et qui est appelé "\NewTerm{facteur de correction de population}\index{facteur de correction de population}". Certains praticiens on tendance à faire l'approximation suivante quand $n$ est assez grand:
	
	Voici un exemple de tracé de la fonction de distribution et répartition pour la distribution Hypergéométrique de paramètres $(n,p,m,k)=(10,6,5,k)$:
	\begin{figure}[H]
	\begin{center}
			\includegraphics{img/arithmetics/law_hypergeometric.jpg}
		\end{center}	
		\caption{Loi hypergéométrique $\mathcal{H}$ (fonction de densité et de distribution cumulative)}
	\end{figure}
	Démontrons que la loi Hypergéométrique tend vers une loi binomiale puisqu'il en est fait usage de nombreuses fois dans différents chapitres du site (et particulièrement la section de Génie Industriel \pageref{quality control}).

	Pour cela, décomposons:
	
	Il vient alors:
	
	Pour le deuxième terme:
	
	Pour $m\rightarrow +\infty$ (...) tous les termes sont alors de l'ordre de $m$. Nous avons alors:
	
	Pour le troisième terme un développement identique en tous points au précédent permet d'obtenir (bien évidemment nous avons aussi besoin que $n \rightarrow +\infty$ (...)):
	
	Et bien sûr ... on peut donc débattre de $ n-m $ quand les deux termes tendent vers l'infini...
	
	Idem pour le quatrième terme:
	
	En conclusions nous avons:
	
	Changeons d'écriture en posant $p$ (le nombre d'individus tirés) comme étant $N$. Il vient alors:
	
	Faisons un autre changement d'écriture en notant $b$ les boules noires (black) et $w$ les boules blanches (white). Il vient alors:
	
	Enfin, notons $p$ la proportion de boules noires et $q$ celle de boules blanches dans le lot $n$. Il vient alors:
	
	Nous retrouvons donc bien la loi binomiale!! En pratique, il est courant d'approximer la loi hypergéométrique par une loi binomiale dès que le rapport nombre d'individus tirés sur le nombre total d'individus est inférieur à $10\%$ (c'est-à-dire lorsque l'échantillon est $10$ fois plus petit que la population). Il s'ensuit que la loi hypergéométrique tend aussi (comme nous le démontrerons plus loin) vers une loi Normale lorsque la population tend vers l'infini et que l'échantillon est petit.

	Dans la pratique, des simulations de Monte-Carlo avec des tests d'ajustements (voir plus loin dans ce chapitre), ont montré qu'une loi hypergéométrique pouvait être approximée par une loi Normale (cas très important dans les tests statistiques de contingence que nous étudierons dans le chapitre de Méthodes Numériques) si les trois conditions suivantes étaient remplies en même temps:
	
	Soit sous forme graphique très approximative...:
	\begin{figure}[H]
		\centering
		\includegraphics{img/arithmetics/hypergeometrice_normal_approximation.jpg}	
		\caption{Conditions d'application de l'approximation par une loi Normale}
	\end{figure}
	Dès lors:
	
	
	\subsubsection{Distribution Multinomiale}\label{multinomial distribution}
	La loi multinomiale (appelée ainsi car elle fait intervenir plusieurs fois le coefficient binomial) est une loi applicable à n événements distinguables, chacun ayant une probabilité donnée, qui surviennent une ou plusieurs fois et ce de façon non nécessairement ordonné. Il s'agit d'un cas fréquent dans les études marketing et qui nous sera utile pour construire le test statistique de McNemar beaucoup plus loin (voir plus bas page \pageref{mcnemar test}). Nous retrouvons également cette loi en finance quantitative (voir plus bas page \pageref{multinomial logistic regression}).
	
	La loi multinomiale est une généralisation de la loi binomiale applicable par exemple aux $n$ essais d'un dé à six faces mais à la différence que les probabilités ne sont pas égales!
	
	Plus techniquement, considérons l'espace des événements $\Omega=\left\lbrace 1,2,...,m \right\rbrace$ muni d'une probabilité $P(\left\lbrace i\right\rbrace)=p_i,i=1,2,...,n$. Nous tirons $n$ fois de suite avec remise un élément de $\Omega$ avec la probabilité $p_i,i=1,2,...,n$. Nous allons chercher quelle est la probabilité d'obtenir de manière non nécessairement ordonnée l'événement $1$, $k_1$ fois, l'événement $2$, $k_2$ fois et ce sur une suite d'un tirage de $n$ éléments.
	
	\begin{tcolorbox}[title=Remarque,colframe=black,arc=10pt]
	Cela équivaut à l'étude d'un tirage avec remise ET contraintes sur les occurrences. Donc sans contraintes nous verrons par l'exemple que nous retombons sur un tirage avec remise simple.
	\end{tcolorbox}	
	Nous avons vu dans la section de Probabilités (\SeeChapter{voir page \pageref{choice function}}), que si nous prenons un ensemble d'événements ayant plusieurs issues, alors les différentes combinaisons de suites que nous pouvons obtenir en prenant $p$ éléments choisis parmi $n$ est:
	
	Il y a donc:
	
	façons différentes d'obtenir $k_1$ fois un certain événement. Soit une probabilité associée de:
	
	Maintenant, intervient la particularité de la loi multinomiale!: il n'y a pas d'échecs contrairement à la loi binomiale. Chaque "pseudo-échec" peut être considéré comme un sous tirage de $k_2$ éléments parmi les $n-k_1$ éléments restants.
	
	Ainsi le terme:
	
	s'écrira sur l'ensemble de l'expérience si nous considérons un cas particulier limité à deux types d'événements:
	
	avec donc:
	
	qui donne le nombre de façons différentes d'obtenir $k_2$ fois un second événement  puisque dans l'ensemble de la suite de $n$ éléments déjà $k_1$ ont été tirés ce qui fait qu'il n'en reste plus que $n-k_1$ sur lesquels nous pouvons obtenir les $k_2$ voulus.
	
	Ces relations nous montrent donc qu'il s'agit d'une situation où chaque probabilité d'événement est considérée comme une sous loi binomiale (d'où son nom aussi...).

	Alors nous avons dans le cas particulier de deux séries d'uplets:
	
	et comme:
	
	il vient:
	
	et nous voyons que la construction de cette loi impose donc que:
	
	Ainsi, par récurrence nous avons la probabilité $\mathcal{M}$ recherchée appelée "\NewTerm{distribution Multinomiale}\index{distribution Multinomiale}" (ou "\NewTerm{"loi Multinomiale}\index{"loi Multinomiale}") et donnée par:
	
	En d'autres termes, s'il y a $m$ variables aléatoires, i.e. $X_i$, $i\in[1,n]$, où $X_i$ représente le nombre d'occurrences de l'élément $i$ dans un choix de $n$ éléments, avec l'entrée $i$ dans le vecteur de probabilités $\vec{p}$, $p_i$ donnant la probabilité de tirer l'élément  d'intérêt $i$. La probabilité d'obtenir le résultat global attendu $m$ fois en sélectionnant $k_i$ fois des éléments de $k_1\ldots k_m$ est alors donné par la relation $\mathcal{M}(\vec{k},\vec{p},m)$ donnée plus haut (voir l'exemple plus bas si cette explication n'est pas claire!). 
	
	\begin{tcolorbox}[title=Remarque,colframe=black,arc=10pt]
	Notez que si $m=2$ alors la relation précédente devient:
	
	Mais comme $k_1+k_2=n$, alors:
	
	Et comme $p_1+p_2=1$ nous avons finalement:
	
	On reconnaît ici la distribution binomiale!
	\end{tcolorbox}
	
	Dans un tableur Microsoft Excel 11.8346, le terme:
	
	est appelé le "\NewTerm{coefficient multinomial}\index{coefficient multinomial}" est disponible sous le nom de la fonction \texttt{MULTINOMIALE( )} dans la version française. Dans la littérature nous trouvons également ce terme parfois sous les formes respectives suivantes:
	
	Ce qui donne alors pour l'expression de la distribution multinomiale:
	
	\begin{theorem}	
	Démontrons que la loi multinomiale est bien une loi de probabilité (car nous pourrions en douter...). Si c'est bien le cas, la somme des probabilités doit être comme nous le savons, égale à l'unité.
	\end{theorem}	
	\begin{dem}
		Rappelons que dans la section de Calcul Algébrique nous avons démontré que le théorème binomial (\SeeChapter{voir section de Calcul Algébrique page \pageref{binomial theorem}}):
		
		Faisons maintenant un petit peu de notation:
		
		et cette fois-ci un changement de variables:
		
		Cette dernière relation (qui est un cas à deux termes du "\NewTerm{théorème multinomial}\index{théorème multinomial}") va nous être utile pour démontrer que la loi multinomiale est bien une loi de probabilité. Nous prenons donc le cas particulier avec deux groupes de tirage:
		
		ce qui s'écrit aussi de par la construction de la loi multinomiale:
		
		et donc la somme doit être égale à l'unité telle que:
		
		Pour vérifier cela nous utilisons le théorème multinomial montré précédemment
		
		Or, comme par construction de la loi multinomiale la somme des probabilités est unitaire, nous avons bien:
		
		\begin{flushright}
			$\blacksquare$  Q.E.D.
		\end{flushright}
	\end{dem}
	\begin{tcolorbox}[colframe=black,colback=white,sharp corners]
	\textbf{{\Large \ding{45}}Exemples:}\\\\
	E1. Nous lançons un dé non-pipé $12$ fois. Quelle est la probabilité que les $6$ faces apparaissent le même nombre de fois (mais pas nécessairement consécutivement!) soit deux fois pour chaque:
	
	où nous voyons bien que $m$ correspond au nombre de groupes de réussites.\\
	
	E2. Nous lançons un dé non-pipé 12 fois. Quelle est la probabilité qu'une seule et unique face apparaisse 12 fois (donc que le "$1$" apparaisse $12$ fois de suite, ou le "$2$", ou le "$3$", etc.):
	
	Nous retrouvons donc avec ce dernier exemple un résultat connu de la binomiale.
	\end{tcolorbox}
	Rappelons que nous avons prouvé un peu plus haut (nous modifions juste un peu la notation):
	
	Et nous avons alors trivialement:
	
	L'espérance de la distribution multinomiale est en fait une moyenne de vecteur. Mais comme la moyenne d'un vecteur est égale à la moyenne de ses composantes, nous voulons calculer $\text{E}(x_i)$. Pour cela, sans perte de généralité, calculons $\text{E}(x_1)$. Pour ce faire, rappelons que par définition de l'espérance:
	
	Mais pour $x_1 = 0$, les termes correspondants sont égaux à zéro. On peut alors directement écrire:
	
	Maintenant comme la somme agit déjà sur le terme $P(X_1=x_1)$, pour ce dernier nous n'avons qu'à additionner sur tous les termes sauf $x_1$ tel que:
	
	Maintenant, nous effectuons le changement de variables suivant $x_1^\prime=x_1-1$. Alors:
	
	Mais la deuxième somme nous est connue! En effet, c'est la probabilité cumulée sur tous les $n-1$ (au lieu du $n$ classique). Et celui-là est égal à $1$. Nous avons donc:
	
	Dès lors:
	
	Voyons maintenant la variance (c'est aussi un vecteur de variance en fait!). L'idée est tout à fait la même que pour l'espérance. Nous allons d'abord utiliser la relation d'Huygens:
	
	Donc, seul le premier terme $\text{E}(x_i^2)$ nous est inconnu. Sans perte de généralité, et encore en prenant la définition de l'espérance, considérons:
	
	Faisons maintenant un changement de variable:
	
	Dès lors:
	
	Avec finalement:
	

	\pagebreak
	\subsubsection{Distribution de Poisson}\label{poisson distribution}
	
	Pour certains événements forts rares, la probabilité $p$ est très faible et tend vers zéro. Toutefois la valeur moyenne $np$ tend vers une valeur fixe lorsque $n$ tend vers l'infini.

	Nous partirons donc d'une distribution binomiale de moyenne equation que nous supposerons finie lorsque n tend vers l'infini.

	La probabilité de $k$ réussites lors de $n$ épreuves vaut (loi Binomiale):
	
	En posant $p=\dfrac{m}{n}$ (où $m$ est temporairement la nouvelle notation pour la moyenne selon $\mu=np$), cette expression peut s'écrire:
	
	En regroupant les termes, nous pouvons mettre la valeur de $\mathcal{B}(n,k)$ sous la forme:
	
	Nous reconnaissons que, lorsque $n$ tend vers l'infini, le deuxième facteur du produit a pour limite $e^{-\mu}$  (\SeeChapter{voir section d'Analyse Fonctionnelle page \pageref{Euler number}}). Quant au troisième facteur, puisque nous nous intéressons aux petites valeurs de $k$ (la probabilité de réussite est très faible), sa limite pour $n$ tendant vers l'infini vaut $1$.
	
	Cette technique de passage à la limite est parfois appelée dans ce contexte: "\NewTerm{théorème limite de Poisson}\index{théorème limite de Poisson}".
	
	Nous obtenons ainsi la  "\NewTerm{distribution de Poisson}\index{distribution de Poisson}" (ou "\NewTerm{loi de Poisson}\index{loi de Poisson}"), appelée également parfois "\NewTerm{loi des événements rares}\index{loi des événements rares}", donnée donc par:
	
	qui peut être obtenue dans Microsoft Excel 11.8346 avec la fonction \texttt{LOI.POISSON( )} et qui dans la pratique et la littérature spécialisée est souvent notrée par la lettre $u$.
	
	Il s'agit bien d'une loi de probabilité puisqu'en utilisant les séries de Taylor (\SeeChapter{voir section de Séquences et Séries page \pageref{usual maclaurin developments}}), nous montrons que la somme des probabilités cumulées est bien:
	
	\begin{tcolorbox}[title=Remarque,colframe=black,arc=10pt]
	Nous retrouverons fréquemment cette loi dans différentes sections du livre comme par exemple lors de l'étude de la maintenance préventive dans la section de Génie Industriel page \pageref{preventive maintenance} ou encore dans la section des Techniques De Gestion lors de l'étude des théories des files d'attentes page \pageref{queueing theory} (le lecteur peut s'y reporter pour des exemples intéressants et pragmatiques) et enfin dans le domaine de l'assurance vie et non vie.
	\end{tcolorbox}	
	Voici un exemple de tracé de la fonction de distribution et répartition pour la distribution de Poisson de paramètre equation: $\mu=3$:
	\begin{figure}[H]
	\begin{center}
		\includegraphics{img/arithmetics/law_poisson.jpg}
		\end{center}	
		\caption{Loi de Poisson $\mathcal{P}$ (fonction de densité et de distribution cumulative)}
	\end{figure}
	Cette distribution est importante car elle décrit beaucoup de processus dont la probabilité est petite et constante. Elle est souvent utilisée dans la théorie des files d'attente (temps d'attente), test d'acceptabilité et fiabilité et contrôles statistiques de qualité. Entre autres, elle s'applique aux processus tels que l'émission des quanta de lumière par des atomes excités, le nombre de globules rouges observés au microscope, le nombre d'appels arrivant à une centrale téléphonique. La distribution de Poisson est valable pour de nombreuses observations faites en physique nucléaire ou corpusculaire.

	L'espérance (moyenne) de la distribution de Poisson est (nous utilisons la série de Taylor de l'exponentielle):
	
	et donne le nombre moyen de fois que l'on obtiendra l'issue souhaitée.
	
	Ce résultat peut paraître déroutant.... la moyenne s'exprime par la moyenne??? Oui il ne faut simplement pas oublier que celle-ci est donnée au début par:
	
	\begin{tcolorbox}[title=Remarque,colframe=black,arc=10pt]
	 Pour plus de détails le lecteur peut aussi se reporter à la partie concernant les "estimateurs" plus loin dans la présente section page \pageref{likelihood estimators}. 
	\end{tcolorbox}	
	La variance de la fonction de distribution de Poisson est quant à elle donnée par (en utilisant à nouveau les séries de Taylor):
	
	toujours avec:
	
	Le fait important que pour la loi de Poisson nous ayons la variance qui soit égale à l'espérance est appelé "\NewTerm{propriété d'équidispersion de la de Poisson}\index{\'equidispersion de la de Poisson}". Il s'agit d'une propriété souvent utilisée dans la pratique comme indicateur pour identifier si des données (à support discret) sont distribuées selon une loi de Poisson.
	
	Les lois théoriques de distributions statistiques sont établies en supposant la réalisation d'un nombre infini de mesures. Il est évident que nous ne pouvons en effectuer qu'un nombre fini $N$. D'où la nécessité d'établir des correspondances entre les valeurs utiles théoriques et expérimentales. Pour ces dernières nous n'obtenons évidemment qu'une approximation dont  la validité est toutefois souvent admise comme suffisante.
	
	Maintenant démontrons une propriété importante de la loi Poisson dans le domaine de l'ingénierie que nous appelons la "stabilité par l'addition". L'idée est la suivante:

	Soit $X$ et $Y$ deux variables aléatoires indépendantes $X$ et $Y$ de loi de Poisson de paramètre respectif $\lambda$ et $\mu$. Nous voulons vérifier que leur somme est aussi une loi de Poisson:
	
	Voyons cela:
	
	car les événements sont indépendants. Nous avons alors:
	
	Or, en appliquant le théorème binomial (\SeeChapter{voir section de Calcul Algébrique page \pageref{binomial theorem}}):
	
	Donc au final:
	
	et donc la loi de Poisson est stable par l'addition. Donc toute loi de Poisson dont le paramètre est connu est in extenso indéfiniment divisable en une quantité finie ou infinie de lois de Poisson indépendante qui se somment.
	
	\newpage
	Pour les personnes qui n'ont peut-être pas accès à un tableur ou à un logiciel statistique, voici quelques tables utiles:
	
	\begin{center}
	FONCTION DE DISTRIBUTION CUMULATIVE DE POISSON
	\end{center}

	\begin{center}
	\begin{tabular}{rr@{\ \,}r@{\ \,}r@{\ \,}r@{\ \,}r@{\ \,}r@{\ \,}r@{\ \,}r@{\ \,}r@{\ \,}r@{\ \,}r}
	$k/\mu$&
	\multicolumn{1}{c}{0.1}&\multicolumn{1}{c}{0.2}&
	\multicolumn{1}{c}{0.3}&\multicolumn{1}{c}{0.4}&
	\multicolumn{1}{c}{0.5}&\multicolumn{1}{c}{0.6}&
	\multicolumn{1}{c}{0.7}&\multicolumn{1}{c}{0.8}&
	\multicolumn{1}{c}{0.9}&\multicolumn{1}{c}{1.0}\\
	\ \\
	0&0.905&0.819&0.741&0.670&0.607&0.549&0.497&0.449&0.407&0.368\\
	1&0.995&0.982&0.963&0.938&0.910&0.878&0.844&0.809&0.772&0.736\\
	2&1.000&0.999&0.996&0.992&0.986&0.977&0.966&0.953&0.937&0.920\\
	3&1.000&1.000&1.000&0.999&0.998&0.997&0.994&0.991&0.987&0.981\\
	4&1.000&1.000&1.000&1.000&1.000&1.000&0.999&0.999&0.998&0.996\\
	5&1.000&1.000&1.000&1.000&1.000&1.000&1.000&1.000&1.000&0.999\\
	6&1.000&1.000&1.000&1.000&1.000&1.000&1.000&1.000&1.000&1.000\\
	\ \\
	$k/\mu$&
	\multicolumn{1}{c}{1.1}&\multicolumn{1}{c}{1.2}&
	\multicolumn{1}{c}{1.3}&\multicolumn{1}{c}{1.4}&
	\multicolumn{1}{c}{1.5}&\multicolumn{1}{c}{1.6}&
	\multicolumn{1}{c}{1.7}&\multicolumn{1}{c}{1.8}&
	\multicolumn{1}{c}{1.9}&\multicolumn{1}{c}{2.0}\\
	\ \\
	0&0.333&0.301&0.273&0.247&0.223&0.202&0.183&0.165&0.150&0.135\\
	1&0.699&0.663&0.627&0.592&0.558&0.525&0.493&0.463&0.434&0.406\\
	2&0.900&0.879&0.857&0.833&0.809&0.783&0.757&0.731&0.704&0.677\\
	3&0.974&0.966&0.957&0.946&0.934&0.921&0.907&0.891&0.875&0.857\\
	4&0.995&0.992&0.989&0.986&0.981&0.976&0.970&0.964&0.956&0.947\\
	5&0.999&0.998&0.998&0.997&0.996&0.994&0.992&0.990&0.987&0.983\\
	6&1.000&1.000&1.000&0.999&0.999&0.999&0.998&0.997&0.997&0.995\\
	7&1.000&1.000&1.000&1.000&1.000&1.000&1.000&0.999&0.999&0.999\\
	8&1.000&1.000&1.000&1.000&1.000&1.000&1.000&1.000&1.000&1.000\\
	\ \\
	$k/\mu$&
	\multicolumn{1}{c}{2.2}&\multicolumn{1}{c}{2.4}&
	\multicolumn{1}{c}{2.6}&\multicolumn{1}{c}{2.8}&
	\multicolumn{1}{c}{3.0}&\multicolumn{1}{c}{3.2}&
	\multicolumn{1}{c}{3.4}&\multicolumn{1}{c}{3.6}&
	\multicolumn{1}{c}{3.8}&\multicolumn{1}{c}{4.0}\\
	\ \\
	0&0.111&0.091&0.074&0.061&0.050&0.041&0.033&0.027&0.022&0.018\\
	1&0.355&0.308&0.267&0.231&0.199&0.171&0.147&0.126&0.107&0.092\\
	2&0.623&0.570&0.518&0.469&0.423&0.380&0.340&0.303&0.269&0.238\\
	3&0.819&0.779&0.736&0.692&0.647&0.603&0.558&0.515&0.473&0.433\\
	4&0.928&0.904&0.877&0.848&0.815&0.781&0.744&0.706&0.668&0.629\\
	5&0.975&0.964&0.951&0.935&0.916&0.895&0.871&0.844&0.816&0.785\\
	6&0.993&0.988&0.983&0.976&0.966&0.955&0.942&0.927&0.909&0.889\\
	7&0.998&0.997&0.995&0.992&0.988&0.983&0.977&0.969&0.960&0.949\\
	8&1.000&0.999&0.999&0.998&0.996&0.994&0.992&0.988&0.984&0.979\\
	9&1.000&1.000&1.000&0.999&0.999&0.998&0.997&0.996&0.994&0.992\\
	10&1.000&1.000&1.000&1.000&1.000&1.000&0.999&0.999&0.998&0.997\\
	11&1.000&1.000&1.000&1.000&1.000&1.000&1.000&1.000&0.999&0.999\\
	12&1.000&1.000&1.000&1.000&1.000&1.000&1.000&1.000&1.000&1.000\\
	\end{tabular}
	\end{center}
	
	\newpage
	
	\begin{center}
	\begin{tabular}{rr@{\ \,}r@{\ \,}r@{\ \,}r@{\ \,}r@{\ \,}r@{\ \,}r@{\ \,}r@{\ \,}r@{\ \,}r@{\ \,}r}
	$k/\mu$&
	\multicolumn{1}{c}{4.2}&\multicolumn{1}{c}{4.4}&
	\multicolumn{1}{c}{4.6}&\multicolumn{1}{c}{4.8}&
	\multicolumn{1}{c}{5.0}&\multicolumn{1}{c}{5.2}&
	\multicolumn{1}{c}{5.4}&\multicolumn{1}{c}{5.6}&
	\multicolumn{1}{c}{5.8}&\multicolumn{1}{c}{6.0}\\
	\ \\
	0&0.015&0.012&0.010&0.008&0.007&0.006&0.005&0.004&0.003&0.002\\
	1&0.078&0.066&0.056&0.048&0.040&0.034&0.029&0.024&0.021&0.017\\
	2&0.210&0.185&0.163&0.143&0.125&0.109&0.095&0.082&0.072&0.062\\
	3&0.395&0.359&0.326&0.294&0.265&0.238&0.213&0.191&0.170&0.151\\
	4&0.590&0.551&0.513&0.476&0.440&0.406&0.373&0.342&0.313&0.285\\
	5&0.753&0.720&0.686&0.651&0.616&0.581&0.546&0.512&0.478&0.446\\
	6&0.867&0.844&0.818&0.791&0.762&0.732&0.702&0.670&0.638&0.606\\
	7&0.936&0.921&0.905&0.887&0.867&0.845&0.822&0.797&0.771&0.744\\
	8&0.972&0.964&0.955&0.944&0.932&0.918&0.903&0.886&0.867&0.847\\
	9&0.989&0.985&0.980&0.975&0.968&0.960&0.951&0.941&0.929&0.916\\
	10&0.996&0.994&0.992&0.990&0.986&0.982&0.977&0.972&0.965&0.957\\
	11&0.999&0.998&0.997&0.996&0.995&0.993&0.990&0.988&0.984&0.980\\
	12&1.000&0.999&0.999&0.999&0.998&0.997&0.996&0.995&0.993&0.991\\
	13&1.000&1.000&1.000&1.000&0.999&0.999&0.999&0.998&0.997&0.996\\
	14&1.000&1.000&1.000&1.000&1.000&1.000&1.000&0.999&0.999&0.999\\
	15&1.000&1.000&1.000&1.000&1.000&1.000&1.000&1.000&1.000&0.999\\
	16&1.000&1.000&1.000&1.000&1.000&1.000&1.000&1.000&1.000&1.000\\
	\ \\
	$k/\mu$&
	\multicolumn{1}{c}{6.5}&\multicolumn{1}{c}{7.0}&
	\multicolumn{1}{c}{7.5}&\multicolumn{1}{c}{8.0}&
	\multicolumn{1}{c}{8.5}&\multicolumn{1}{c}{9.0}&
	\multicolumn{1}{c}{9.5}&\multicolumn{1}{c}{10.0}&
	\multicolumn{1}{c}{10.5}&\multicolumn{1}{c}{11.0}\\
	\ \\
	0&0.002&0.001&0.001&0.000&0.000&0.000&0.000&0.000&0.000&0.000\\
	1&0.011&0.007&0.005&0.003&0.002&0.001&0.001&0.000&0.000&0.000\\
	2&0.043&0.030&0.020&0.014&0.009&0.006&0.004&0.003&0.002&0.001\\
	3&0.112&0.082&0.059&0.042&0.030&0.021&0.015&0.010&0.007&0.005\\
	4&0.224&0.173&0.132&0.100&0.074&0.055&0.040&0.029&0.021&0.015\\
	5&0.369&0.301&0.241&0.191&0.150&0.116&0.089&0.067&0.050&0.038\\
	6&0.527&0.450&0.378&0.313&0.256&0.207&0.165&0.130&0.102&0.079\\
	7&0.673&0.599&0.525&0.453&0.386&0.324&0.269&0.220&0.179&0.143\\
	8&0.792&0.729&0.662&0.593&0.523&0.456&0.392&0.333&0.279&0.232\\
	9&0.877&0.830&0.776&0.717&0.653&0.587&0.522&0.458&0.397&0.341\\
	10&0.933&0.901&0.862&0.816&0.763&0.706&0.645&0.583&0.521&0.460\\
	11&0.966&0.947&0.921&0.888&0.849&0.803&0.752&0.697&0.639&0.579\\
	12&0.984&0.973&0.957&0.936&0.909&0.876&0.836&0.792&0.742&0.689\\
	13&0.993&0.987&0.978&0.966&0.949&0.926&0.898&0.864&0.825&0.781\\
	14&0.997&0.994&0.990&0.983&0.973&0.959&0.940&0.917&0.888&0.854\\
	15&0.999&0.998&0.995&0.992&0.986&0.978&0.967&0.951&0.932&0.907\\
	16&1.000&0.999&0.998&0.996&0.993&0.989&0.982&0.973&0.960&0.944\\
	17&1.000&1.000&0.999&0.998&0.997&0.995&0.991&0.986&0.978&0.968\\
	18&1.000&1.000&1.000&0.999&0.999&0.998&0.996&0.993&0.988&0.982\\
	19&1.000&1.000&1.000&1.000&0.999&0.999&0.998&0.997&0.994&0.991\\
	20&1.000&1.000&1.000&1.000&1.000&1.000&0.999&0.998&0.997&0.995\\
	21&1.000&1.000&1.000&1.000&1.000&1.000&1.000&0.999&0.999&0.998\\
	22&1.000&1.000&1.000&1.000&1.000&1.000&1.000&1.000&0.999&0.999\\
	23&1.000&1.000&1.000&1.000&1.000&1.000&1.000&1.000&1.000&1.000
	\end{tabular}
	\end{center}

	\subsubsection{Distribution de Gauss-Laplace/Loi Normale}\label{gauss distribution}
	Cette caractéristique est la plus importante fonction de distribution en statistiques suite au résultat d'un théorème connu appelé "théorème central limite" qui comme nous le verrons, permet de démontrer (entre autres) que la somme de toute suite de variables aléatoires indépendantes de même loi ayant une espérance et un écart-type fini converge vers une distribution de Gauss-Laplace (loi Normale).

	Il est donc très important de focaliser particulièrement son attention sur les développements qui vont être faits ici!

	Partons d'une distribution binomiale et faisons tendre le nombre $n$ d'épreuves vers l'infini. Si $p$ est fixé au départ, la moyenne $\mu=np$ tend également vers l'infini, de plus l'écart-type $\sigma=npq$ tend également vers l'infini

	\begin{tcolorbox}[title=Remarque,colframe=black,arc=10pt]
	Le cas où $p$  varie et tend vers $0$ tout en laissant fixe la moyenne equation ayant déjà été étudié lors de la présentation de la distribution de Poisson.
	\end{tcolorbox}
	Si nous voulons calculer la limite de la distribution Binomiale, il s'agira donc de faire un changement d'origine qui stabilise la moyenne, en $0$ par exemple, et un changement d'unité qui stabilise l'écart-type, à $1$ par exemple.
	
	Voyons tout d'abord comment varie equation en fonction de k (nombre de réussites) et calculons la différence:
	
	Notons maintenant par $P_n(k)$ la probabilité binomiale de $k$  succès  et voyons d'abord comment $P_n(k)$ varie avec $ k $ et calculons la différence:
	
	Nous en concluons que $P_n(k)$  est une fonction croissante de $k$, tant que $np-k-q$ est positif (pour $n, p$ et $q$ fixés). Pour le voir il suffit de prendre quelques valeurs (du membre de droite de l'égalité) ou d'observer la distribution graphique de la distribution Binomiale en se souvenant bien que:
	
	Comme $q<1$ il est par conséquent évident que la valeur de $k$ voisine de l'espérance de la loi Binomiale $\mu=np$ constitue le maxima de $P_n(k)$.
	
	D'autre part, la différence $P_n(k+1)-P_n(k)$ est le taux d'accroissement de la fonction $P_n(k)$. Nous pouvons alors écrire:
	
	comme étant la pente de la fonction.

	Définissons maintenant une nouvelle variable aléatoire telle que sa moyenne soit nulle (variations négligeables) et son écart-type unitaire (une variable centrée-réduite en d'autres termes). Nous avons alors:
	
	Nous avons alors aussi avec cette nouvelle variable aléatoire:
	
	Notons $F(x)$ l'expression de $P_n(k)$ calculée en fonction de la nouvelle variable aléatoire de moyenne nulle et d'écart-type unitaire dont nous recherchons l'expression quand $n$ tend vers l'infini.
	
	Reprenons:
	
	Afin de simplifier l'étude de cette relation quand $n$ tend vers l'infini et $k$ vers l'espérance $\mu=np$, multiplions des deux côtés par $npq/\sqrt{npq}$:
	
	Réécrivons le terme de droite de l'égalité. Il vient alors:
	
	Et maintenant réécrivons le terme de gauche de la relation antéprécédente. Il vient:
	
	Après un passage à la limite pour n tendant vers l'infini nous avons dans un premier temps pour le dénominateur du deuxième terme de la relation antéprécédente:
	
	la simplification suivante:
	
	Donc:
	
	et dans un second temps, tenant compte du fait que les valeurs de $k$ considérées se trouvent alors au voisinage de l'espérance $np$, nous obtenons:
	
	et:
	
	Donc:
	
	et comme:
	
	où $F(x)$ représentera (maladroitement) pour les quelques lignes qui vont suivre, la fonction de densité lorsque $n$ tend vers l'infini.
	
	Nous avons finalement:
	
	Cette relation peut encore s'écrire en réarrangeant les termes: 
	
	et en intégrant les deux membres de cette égalité nous obtenons (\SeeChapter{voir section de Calcul Différentiel et Intégral page \pageref{integral calculus}}):
	
	La fonction suivante est une des solutions de la relation précédente: 
	
	Effectivement:
	
	La constante est déterminée par la condition que:
	
	qui représente la somme de toutes les probabilités, qui doit valoir $1$. Nous pouvons montrer pour cela que nous devons avoir:
	
	\begin{dem}
		Nous avons:
		
		Donc concentrons-nous sur le dernier terme de l'égalité appelé "\NewTerm{intégrale de Gauss}\index{intégrale de Gauss}\label{Gauss integral}" (en fait, le cas ci-dessous est un cas particulier de cette intégrale). Ainsi:
		
		puisque $e^{-x^2}$ est une fonction paire (\SeeChapter{voir section d'Analyse Fonctionnelle page \pageref{even function}}). Écrivons maintenant le carré de l'intégrale de la manière suivante:
		
		et faisons un changement de variable en passant en coordonnées polaires, dès lors nous faisons aussi usage du Jacobien dans ces mêmes coordonnées (\SeeChapter{voir section de Calcul Différentiel et Intégral page \pageref{jacobian}}):
		
		Dès lors:
		
		Par extension pour $e^{-\frac{x^2}{2}}$ nous avons:
		
		Par conséquent, le lecteur doit garder à l'esprit pour des développements ultérieurs que:
		
		et que:
		
		\begin{flushright}
			$\blacksquare$  Q.E.D.
		\end{flushright}
	\end{dem}
	Nous obtenons donc la "\NewTerm{loi Normale centrée réduite}\index{loi Normale centrée réduite}" notée sous forme de fonction de densité de probabilité (la notation avec le $F$ majuscule peut malheureusement porter à confusion dans le cadre du présent développement avec le fonction de répartition... désolé...):
	
	qui peut être calculée dans la version française Microsoft Excel 11.8346 avec la fonction \texttt{LOI.NORMALE.STANDARD( )}. Notez également que dans certains manuels de référence, cette dernière relation est désignée par la lettre $\phi(x)$.
	
	Pour information, une variable suivant une loi Normale centrée réduite est très souvent par tradition notée $Z$ (pour "Zentriert" en allemand) et ses réalisations par la lettre $z$ minuscule.

	En revenant aux variables non normées (ie non-standardisées) en se rappelant que \label{normal centered reduced variable}:
	
	nous obtenons donc la  "\NewTerm{distribution Gauss-Laplace}\index{distribution Gauss-Laplace}" (ou "\NewTerm{loi de Gauss-Laplace}\index{loi de Gauss-Laplace}") ou également appelée "\NewTerm{distribution Normale}\index{distribution Normale}"\index{loi Normale}" donnée sous forme de densité de probabilité dans ce livre par:
	 
	\begin{tcolorbox}[title=Remarque,colframe=black,arc=10pt]
	Donc, en général, le lecteur doit garder à l'esprit que lorsqu'une variable aléatoire $x$ suit un loi Normale $\mathcal{N}(\mu,\sigma)$, dès lors la normaliser en soustrayant sa moyenne (c'est-à-dire une "\NewTerm{normalisation par la moyenne}\index{normalisation moyenne}") et la diviser par son propre écart-type, fait-lui suivre une distribution Normale centrée réduité $Z$ telle que sa réalisation soit notée:
	
	\end{tcolorbox}
	
	La probabilité cumulée (fonction de répartition) de valoir une certaine valeur $k$ étant bien évidemment donnée par:
	
	Voici un exemple de tracé de la fonction de distribution et répartition pour la distribution Normale de paramètres $(\mu,\sigma)=(0,1)$ qui est dès lors la loi Normale centrée réduite:
	\begin{figure}[H]
	\begin{center}
		\includegraphics{img/arithmetics/law_normal.jpg}
		\end{center}	
		\caption{Loi Normale $\mathcal{N}$ (fonction de densité et de distribution cumulative)}
	\end{figure}
	Cette loi régit sous des conditions très générales, et souvent rencontrées, beaucoup de phénomènes aléatoires. Elle est par ailleurs symétrique par rapport à la moyenne $\mu$ (c'est important de s'en souvenir).
	
	Montrons maintenant que $\mu$ représente bien l'espérance mathématique (ou la moyenne) de x (c'est un peu bête mais on peut quand même vérifier...):
	
	Posons:
	
	Nous avons dès lors:
	
	Calculons la première intégrale:
	
	Donc il vient au final:
	
	\begin{tcolorbox}[title=Remarques,colframe=black,arc=10pt]
	\textbf{R1.} Le lecteur pourrait trouver cela déroutant dans un premier temps que le paramètre d'une distribution soit un des résultats que nous cherchons de la distribution (comme cela était le cas pour la loi de Poisson). Ce qui dérange est la mise en pratique d'une telle chose. Au fait, tout s'éclairera lorsque nous étudierons plus loin dans ce chapitre les concepts "d'estimateurs de vraisemblance".\\

	\textbf{R2.} Indiquons que dans la pratique (finance, qualité, assurance, etc.) il est fréquent de devoir calculer l'espérance uniquement pour des valeurs positives de la variable aléatoire qui est définie alors naturellement comme étant "l'espérance positive" et donnée par:
	
	Nous en verrons un exemple pratique dans le chapitre d'Économie lors de notre étude du modèle théorique de la spéculation de Louis Bachelier (page \pageref{theory of speculation}).
	\end{tcolorbox}
	Montrons aussi (...) que $\sigma$ représente bien l'écart-type de $X$ (en d'autres termes de montrer que $\text{V}(X)=\sigma^2$) et pour cela rappelons que nous avions démontré que (relation de Huygens):
	
	Nous savons déjà qu'au niveau des notations:
	
	commençons alors par calculer $\text{E}(X^2)$:
	
	Posons $y=(x-\mu)/\sqrt{2}\sigma$ qui conduit dès lors à:
	
	Or, nous savons (déjà démontré plus haut):
	
	Il reste donc à calculer la première intégrale. Pour cela, procédons par une intégration par parties (\SeeChapter{voir section de Calcul Différentil et Intégral page \pageref{integration by parts}}):
	
	D'où:
	
	Nous obtenons alors:
	
	Et finalement:
	
	Une signification supplémentaire de l'écart-type dans la loi de Gauss-Laplace est une mesure de la largeur de la distribution telle que (cela ne peut se vérifier qu'à l'aide d'intégration à l'aide de méthodes numériques) que toute moyenne et pour tout écart-type non nul nous avons (merci à John Cannin pour la figure \LaTeX):
	\begin{figure}[H]
		\centering
		\pgfplotsset{compat=1.7}
		\pgfmathdeclarefunction{gauss}{2}{\pgfmathparse{1/(#2*sqrt(2*pi))*exp(-((x-#1)^2)/(2*#2^2))}%
		}
		\begin{tikzpicture}
		\begin{axis}[no markers, domain=0:10, samples=100,
		axis lines*=left, xlabel=Standard deviations, ylabel=Frequency,,
		height=6cm, width=10cm,
		xtick={-3, -2, -1, 0, 1, 2, 3}, ytick=\empty,
		enlargelimits=false, clip=false, axis on top,
		grid = major]
		\addplot [fill=cyan!20, draw=none, domain=-3:3] {gauss(0,1)} \closedcycle;
		\addplot [fill=orange!20, draw=none, domain=-3:-2] {gauss(0,1)} \closedcycle;
		\addplot [fill=orange!20, draw=none, domain=2:3] {gauss(0,1)} \closedcycle;
		\addplot [fill=blue!20, draw=none, domain=-2:-1] {gauss(0,1)} \closedcycle;
		\addplot [fill=blue!20, draw=none, domain=1:2] {gauss(0,1)} \closedcycle;
		\addplot[] coordinates {(-1,0.4) (1,0.4)};
		\addplot[] coordinates {(-2,0.3) (2,0.3)};
		\addplot[] coordinates {(-3,0.2) (3,0.2)};
		\node[coordinate, pin={68.2\%}] at (axis cs: 0, 0.4){};
		\node[coordinate, pin={95\%}] at (axis cs: 0, 0.3){};
		\node[coordinate, pin={99.7\%}] at (axis cs: 0, 0.2){};
		\node[coordinate, pin={34.1\%}] at (axis cs: -0.5, 0){};
		\node[coordinate, pin={34.1\%}] at (axis cs: 0.5, 0){};
		\node[coordinate, pin={13.6\%}] at (axis cs: 1.5, 0){};
		\node[coordinate, pin={13.6\%}] at (axis cs: -1.5, 0){};
		\node[coordinate, pin={2.1\%}] at (axis cs: 2.5, 0){};
		\node[coordinate, pin={2.1\%}] at (axis cs: -2.5, 0){};
		\end{axis}
		\end{tikzpicture}
		\caption{Intervalles Sigma pour la distribution Normale}
	\end{figure}
	La largeur de l'intervalle a une très grande importance dans l'interprétation des incertitudes d'une mesure. La présentation d'un résultat comme $\bar{N}\pm \sigma$ signifie que la valeur moyenne a environ $68.3\%$ de chance (probabilité) de se trouver entre les limites de $\bar{N}- \sigma$ et $\bar{N}+ \sigma$, ou qu'elle a environ  $95.4\%$ de se trouver entre $\bar{N}- 2\sigma$ et $\bar{N}+2\sigma$ etc.
	\begin{tcolorbox}[title=Remarque,colframe=black,arc=10pt]
	Ce concept est beaucoup utilisé en gestion de la qualité en entreprise particulièrement avec le concept industriel anglo-saxon Six Sigma (\SeeChapter{voir section de Génie Industriel page \pageref{six sigma}}) qui impose une maîtrise de $6\sigma$ autour de chaque côté (!) de la moyenne des côtés des pièces fabriquées (ou tout autre sujet dont on mesure la déviation).
	\begin{table}[H]
		\centering
		\begin{tabular}{|c|c|c|}
		\hline
		\rowcolor[HTML]{C0C0C0} 
		\multicolumn{1}{|c|}{\cellcolor[HTML]{C0C0C0}\textbf{\begin{tabular}[c]{@{}c@{}}Niveau de Qualité\\ Sigma\end{tabular}}} & \multicolumn{1}{c}{\cellcolor[HTML]{C0C0C0}\textbf{\begin{tabular}[c]{@{}c@{}}Taux de non-défection \\ assuré en \%\end{tabular}}} & \multicolumn{1}{c}{\cellcolor[HTML]{C0C0C0}\textbf{\begin{tabular}[c]{@{}c@{}}Taux de défection en \\ parties par million\end{tabular}}} \\ \hline
		$1\sigma$ & $68.26894$ & $317'311$ \\ \hline
		$2\sigma$ & $95.4499$ & $45'500$ \\ \hline
		$3\sigma$ & $99.73002$ & $2'700$ \\ \hline
		$4\sigma$ & $99.99366$ & $63.4$ \\ \hline
		$5\sigma$ & $99.999943$ & $0.57$ \\ \hline
		$6\sigma$ & $99.9999998$ & $0.002$ \\ \hline
		\end{tabular}
		\caption{Niveau de qualité Sigma avec taux de défection/non-défection}
	\end{table}
	La deuxième colonne du tableau peut facilement être obtenue avec Maple 4.00b (ou aussi avec le tableur de Microsoft). Par exemple pour la première ligne:\\

	\texttt{>S:=evalf(int(1/sqrt(2*Pi)*exp(-x\string^ 2/2),x=-1..1));}\\

	et la première ligne de la troisième colonne par:\\

	\texttt{>(1-S)*1E6;}\\

	Si la loi Normale était décentrée, il suffirait alors d'écrire pour la deuxième colonne:\\

	\texttt{>S:=evalf(int(1/sqrt(2*Pi)*exp(-(x-mu)\string^ 2/2),x=-1..1));}\\

	et ainsi de suite pour tout écart-type et toute moyenne on retombera sur les mêmes intervalles!!!
	\end{tcolorbox}
	Calculons maintenant le coefficient d'asymétrie de la distribution Normale centrée réduite, qui est défini pour rappel par:
		
	Et donc pour la distribution Normale centrée réduite $\mathcal{N}(0,1)$:
	
	Soit maintenant $Y=X-a$ une variable aléatoire. Notez maintenant qu'en raison de la symétrie de la loi Normale, $ Y $ et $ -Y $ ont la même distribution. Cela implique:
	
	Cela implique que la seule solution possible est $\text{E}[Y^3]=0$. D'où pour la loi Normale:
	
	Dérivons maintenant les kurtosis (coefficient d'aplatissement) $\kappa_1$ et $\kappa_4$ pour la distribution Normale centrée réduite:
	
	À cet effet, nous utilisons la primitive habituelle suivante (\SeeChapter{voir section de Calcul Différentiel et Intégral \pageref{usual primitives}}):
	
	Par conséquent, en utilisant également l'intégrale de Gauss:
	
	Par conséquent, le kurtosis est:
	
	D'où le kurtosis normalisé:
	
	La loi de Gauss-Laplace n'est par ailleurs pas qu'un outil d'analyse de données mais également de génération de données. Effectivement, cette loi est une des plus importantes dans le monde des multinationales qui recourent aux outils statistiques pour la gestion du risque, la gestion de projets et la simulation lorsqu'un grand nombre de variables aléatoires sont en jeu. Les meilleurs exemples d'applications en étant les logiciels CrystalBall ou @Risk de Palisade (ce dernier étant mon préféré...).

	Dans ce cadre d'application (gestion de projets), il est par ailleurs très souvent fait usage de la somme (durée des tâches) ou le produit de variables aléatoires (facteur d'incertitude du client) suivant des lois de Gauss-Laplace. Voyons comment cela se calcule:

	\paragraph{Somme de deux variables aléatoires normales}\label{sum of two random normal variables}\mbox{}\\\\
	Soient $X, Y$ deux variables aléatoires indépendantes. Supposons que $X$ suit la loi $\mathcal{N}(\mu_1,\sigma_1)$ et que $Y$ suit la loi $\mathcal{N}(\mu_2,\sigma_2)$. Alors, la variable aléatoire $Z=X+Y$ aura une densité égale au produit de convolution de $f_x$ et $f_y$ (\SeeChapter{voir section d'Analyse Fonctionnelle page \pageref{convolution}}). C'est-à-dire:
	
	ce qui équivaut à faire le produit conjoint (\SeeChapter{voir section Probabilités page \pageref{joint probability}}) des probabilités d'apparition des deux variables continues (se rappeler le même genre de calcul sous forme discrète!).
	
	Pour simplifier l'expression, faisons le changement de variable  $t=x-\mu_1$ et posons: 
	
	Comme:
	
	nous obtenons après une astuce de simplification difficile à obtenir:
	
	Nous posons:
	
	Alors:
	
	Sachant que (déjà démontré plus haut!):
	
	et:
	
	notre expression devient:
	
	Nous reconnaissons l'expression de la loi de Gauss-Laplace de moyenne $\mu_1+\mu_2$ et d'écart type  $\sigma=\sqrt{\sigma_1^2+\sigma_2^2}$.
	
	Par conséquent, $ X + Y $ suit la distribution telle qu'écrite par les physiciens (les deux arguments ont les mêmes unités):
	
	et telle qu'écrite par la majorité des mathématiciens et statisticiens:
	
	Le fait que la somme de deux variables aléatoires Normales donne toujours une loi Normale est ce que nous nommons en statistiques la "\NewTerm{stabilité par la somme}\index{stabilité par la somme (statistique)}\label{stability of the sum in statistics}" de la loi de Gauss-Laplace. Nous retrouverons ce type de propriétés pour d'autres lois que nous étudierons plus loin.

	Donc au même titre que des variables aléatoires de Poisson, toute variable aléatoire suivant une loi Normale dont les paramètres sont connus est in extenso indéfiniment divisable en une quantité finie ou infinie de de variables aléatoires Normalement indépendantes qui se somment telles que:
	
	\begin{tcolorbox}[title=Remarque,colframe=black,arc=10pt]
	Les familles de lois stables par addition constituent un domaine important d'étude en physique, finance et statistiques appelé "\NewTerm{distributions de Lévy alpha-stables}\index{distributions de Lévy alpha-stables}". Si le temps me le permet, je présenterai les détails de ce domaine d'étude extrêmement important dans le présent chapitre.
	\end{tcolorbox}
	
	\paragraph{Produit de deux variables aléatoires normales}\mbox{}\\\\
	Soient $X, Y$ deux variables aléatoires indépendantes réelles. Nous désignerons par $f_X$ et $f_Y$ les densités correspondantes et nous cherchons à déterminer la densité de la variable $Z=XY$ (cas très important et particulièrement en ingénierie).
	
	Soit $ F $ la fonction de densité du couple $ (X, Y) $. Puisque $ X, Y $ sont indépendants (\SeeChapter{voir section Probabilités page \pageref{joint probability}}):
	
	La fonction de répartition de $Z$ est:
	
	où $D=\left\lbrace(x,y) \vert xy<z\right\rbrace$. 
	
	$D$ peut se réécrire comme union disjointe (nous faisons cette opération pour anticiper lors du futur changement de variables une division par zéro):
	
	avec:
	
	Nous avons:
	
	La dernière intégrale vaut zéro car $D_3$ est de mesure (épaisseur) nulle pour l'intégrale selon $x$.
	
	Nous effectuons ensuite le changement de variable suivant:
	
	Le jacobien de la transformation (\SeeChapter{voir section de Calcul Différentiel et Intégral page \pageref{jacobian}}) est:
	
	Donc:
	
	Notons $f_Z$ la densité de la variable $Z$. Par définition:
	
	D'un autre côté:
	
	comme nous venons de le voir. Par conséquent:
	
	Ce qui est un peu triste c'est que dans le cas d'une loi de Gauss-Laplace (loi Normale), cette intégrale ne peut être calculée simplement que numériquement... il faut alors faire appel à des méthodes d'intégration du type Monte-Carlo (\SeeChapter{voir section de Méthodes Numériques page \pageref{monte carlo integration}}).

	HD'après quelques recherches faites sur Internet cependant, mais sans certitude, cette intégrale pourrait être calculée et donnerait une nouvelle loi appelée "\NewTerm{loi de Bessel}\index{loi de Bessel}".
	
	\paragraph{Distribution normale bivariée}\label{bivariate normal distribution}\mbox{}\\\\
	Si deux variables aléatoires Normalement distribuées sont indépendantes, nous savons que la probabilité jointe est égale au produit des probabilités. Nous avons alors:
	
	Vient maintenant une approche que nous retrouverons souvent dans les développements à suivre: pour généraliser des modèles en algèbre simple, il faut penser matriciel! Dès lors on se retrouve avec deux vecteurs faisant intervenir un produit scalaire:
	
	Mais nous pouvons faire encore mieux car pour l'instant il n'y a aucune plus valeur à cette écriture! Effectivement l'idée subtile vient à faire intervenir le déterminant d'une matrice (\SeeChapter{voir section d'Algèbre Linéaire page \pageref{determinant}}) et l'inverse de cette même matrice dans la relation précédente:
	
	Nous retrouvons donc un cas particulier de la matrice des variances-covariances. Dans le domaine de la loi Normale bivariée est il est d'usage d'écrire cette dernière relation sous la forme suivante dans le cas bivarié:
	
	Si nous faisons un plot de cette fonction de densité nous obtenons:
	\begin{figure}[H]
		\centering
		\includegraphics{img/arithmetics/law_bivariate_normal_perspective.jpg}
		\caption{Plot de la fonction de densité Normale bivariée avec MATLAB™}
	\end{figure}
	ou un autre (pas avec les mêmes valeurs) avec des projections planes verticales correspondantes:
	\begin{figure}[H]
		\centering
		\includegraphics[scale=0.7]{img/arithmetics/normal_bivariate_projection.jpg}
		\caption{Plot de la fonction de densité Normale bivariée pgfplots}
	\end{figure}
	Maintenant considérons un cas important en ingénierie en revenant à l'écriture suivante:
	
	et en nous intéressant aux iso-lignes tels que pour tout couple de valeurs des deux variables aléatoires, nous ayons:
	
	En faisant quelques manipulations algébriques très élémentaires, nous obtenons:
	
	Soit:
	
	et il vient:
	
	Nous reconnaissons ici l'équation analytique d'une ellipse (\SeeChapter{voir section de Géométrie Analytique page \pageref{analytical expression ellipse}})! Il est alors aisé de déterminer le petit ou grand axe de l'ellipse (ce qui est très utilisé dans le cartes de contrôle bivariées dans le domaine du Génie Industriel). Mais il ne faut pas oublier que cette équation n'est valable que dans le cas particulier ou la corrélation est nulle!
	
	Un tracé des iso-lignes avec $\mu=\begin{pmatrix}3\\2\end{pmatrix},\Sigma=\begin{bmatrix}25 & 0\\0 & 9\end{bmatrix}$ nous donne:
	\begin{figure}[H]
		\centering
		\includegraphics{img/arithmetics/law_bivariate_normal_isolines.jpg}
		\caption{Plot des iso-lignes de la fonction de densité Normale bivariée (cas non corrélé)}
	\end{figure}
	Mais maintenant rappelons que lorsque nous avions obtenu:
	
	la matrice des variances-covariances était nulle partout sauf sur la diagonale, ce qui impliquait in extenso l'indépendance des deux variables aléatoires. Nous pouvons évidemment deviner que la généralisation consiste à dire que la matrice des variances-covariances n'est pas non-nulle que dans la diagonale et alors les deux variables aléatoires sont corrélées. Dès lors, les iso-lignes deviennent par exemple avec les valeurs $\mu=\begin{pmatrix}3\\2\end{pmatrix},\Sigma=\begin{bmatrix}10 & 5\\5 & 5\end{bmatrix}$:
	\begin{figure}[H]
		\centering
		\includegraphics{img/arithmetics/law_bivariate_normal_isolines_correlation.jpg}
		\caption{Tracé des iso-lignes de la fonction normale bivariée (cas corrélé)}
	\end{figure}
	Donc la corrélation fait pivoter l'axe des ellipses! Remarquons que nous avons dès lors:
	
	et donc in extenso:
	
	Rappelons que nous avons vu lors de notre étude du coefficient de corrélation que (bon normalement.... la notation $R$ pour la corrélation est prise que si les variances sont estimées mais comme c'est la notation la plus courante dans la pratique nous la garderons):
	
	Dès lors:
	
	et l'exposant de l'exponentielle de la Normale bivariée prend alors une forme que nous retrouvons très souvent dans la littérature spécialisée:
	
	Notez que si les variables aléatoires sont centrées réduites, alors nous avons:
	
	et dès lors, l'exposant de l'exponentielle de la loi Normale bivariée est alors:
	
	Ainsi, la fonction de densité de la loi Normale centrée réduite bivariée s'écrit:
	
	Ainsi, nous pouvons voir qu'une distribution Normale bivariée centrée réduite peut être construite par la multiplication de deux lois normales centrées réduites et par la multiplication d'un terme dépendant principalement du paramètre de corrélation. Ce dernier terme contient la nature de la dépendance des deux variables aléatoires et permet de coupler les fonctions marginales (les deux distributions de probabilités Normales centrées réduites marginales) afin d'obtenir la fonction (distribution) jointe Normale bivarieé.

	Si jamais (cela peut-être très utile dans la pratique), voici le code Maple 4.00b pour tracer une fonction bivariée Normale (en reprenant le dernier exemple) même si c'est aussi relativement simple à faire avec un tableur comme Microsoft Excel:

	\texttt{>f:=(x,y,rho,mu1,mu2,sigma1,sigma2)->(1/(2*Pi*sqrt(sigma1*sigma2*(1-rho\string^2))))}\\
	\texttt{*exp((-1/(2*(1-rho\string^2)))*(((x-mu1)/sqrt(sigma1))\string^2+((y-mu2)/sqrt(sigma2))\string^2}\\
	\texttt{-2*rho*((x-mu1)/sqrt(sigma1))*((y-mu2)/sqrt(sigma2))));}\\

	\texttt{>plot3d(f(x,y,5/sqrt(10*5),3,2,10,5),x=-4..10,y=-4..9,grid=[40,40]);}

	et pour le tracé avec les iso-lignes:

	\texttt{>with(plots):}\\
	\texttt{>contourplot(f(x,y,5/sqrt(10*5),3,2,10,5),x=-4..10,y=-4..9,grid=[40,40]);}

	et nous pouvons contrôler qu'il s'agit bien d'une fonction de densité de probabilité en écrivant:

	\texttt{>int(int(f(x,y,5/sqrt(10*5),3,2,10,5),x=-infinity...+infinity)}\\
	\texttt{,y=-infinity...+infinity);}

	ou calculer la probabilité cumulée entre deux intervalles:

	\texttt{>evalf(int(int(f(x,y,5/sqrt(10*5),3,2,10,5),x=-3...+4),y=-5...+2));}
	
	Notez que ce n'est qu'en 2014 que la "\NewTerm{conjecture de corrélation gaussienne}\index{conjecture de corrélation gaussienne}" (également nommée "\NewTerm{inégalité de corrélation gaussienne}\index {inégalité de corrélation gaussienne}") a été prouvée par Thomas Royen, professeur de statistique allemand à la retraite. Cette conjecture peut être résumée par la figure suivante:
	\begin{figure}[H]
		\centering
		\includegraphics{img/arithmetics/gaussian_conjecture.jpg}
	\end{figure}
	Nous ne fournirons pas la preuve ici car je n'en ai jamais eu besoin pour mon entreprise ou aussi celle de mes clients mais c'est assez drôle que de savoir qu'une inégalité qui semble aussi évidente ait eu besoin de tant d'années avant que quelqu'un en trouve la démonstration.
	
	\begin{tcolorbox}[title=Remarque,colframe=black,arc=10pt]
	Le 17 juillet 2014, quelques années après sa retraite, Thomas Royen a eu un éclair de de lucidité: comment utiliser la transformée de Laplace de la distribution gamma multivariée pour obtenir une preuve relativement simple de l'inégalité de corrélation gaussienne, une conjecture à l'intersection de la géométrie, de la théorie des probabilités et des statistiques, formulé après les travaux de Dunnett et Sobel (1955) et du statisticien américain Olive Jean Dunn (1958) qui n'avaient pas été résolue depuis lors. Il a envoyé une copie de sa preuve à Donald Richards, un mathématicien américain connu, qui a travaillé sur une preuve de cette conjecture pendant 30 ans. Richards a immédiatement vu la validité de la preuve de Royen et l'a aidé par la suite à transformer les relation mathématiques en \LaTeX{} (d'où l'importance de s'appuyer sur \LaTeX{} lorsque nous sommes ingénieurs ou scientifiques!). Lorsque Royen a contacté d'autres mathématiciens réputés, ces derniers n'ont pas pris la peine d'analyser sa démonstration, car Royen était relativement inconnu, et ces mathématiciens ont donc estimé que la probabilité que la démonstration que Royen soit fausse était très élevée (en particulier en envoyant un fichier Microsoft Word pour lecture vos chances d'être lu sont assez faibles...).
	\end{tcolorbox}	
	
	\paragraph{Distribution Normale centrée réduite}\label{normal reduced centered distribution}\mbox{}\\\\
	La distribution de Gauss-Laplace n'est pas tabulée puisqu'il faudrait autant de tables numériques que de valeurs possibles pour la moyenne $\mu$ et l'écart-type $\sigma$ (qui sont donc des paramètres de la distribution comme nous l'avons vu).

	C'est pourquoi, en opérant un changement de variable, la loi Normale devient la "\NewTerm{loi Normale centrée réduite}\index{loi Normale centrée réduite}" où:
	\begin{enumerate}
		\item "Centrée" signifie soustraire la moyenne $\mu$ (la fonction de distribution a alors pour axe de symétrie l'axe des ordonnées).
		
		\item  "Réduite" signifie, diviser par l'écart-type $\sigma$ (la fonction de distribution a alors une variance unitaire).
	\end{enumerate}
	Par ce changement de variable, la variable $k$ est remplacée par la variable aléatoire centrée réduite:
	
	Si la variable $k$ a pour moyenne $\mu$ et pour écart-type $\sigma$ alors la variable $k^{*}$ a pour moyenne $0$ et pour écart-type $1$ (cette dernière étant le plus souvent notée $Z$).
	
	Donc la relation:
	
	s'écrit alors (trivialement) plus simplement:
	
	qui n'est d'autre que l'expression de la loi Normale centrée réduite souvent notée $\mathcal{N}(0,1)$ que nous retrouverons très fréquemment dans les chapitres relatifs à la physique, la finance, la gestion et l'ingénierie!
	
	\begin{tcolorbox}[title=Remarque,colframe=black,arc=10pt]
	Calculer l'intégrale de la relation précédente entre n'importe quelles bornes n'est pas possible formellement parlant de manière exacte. Une idée possible et simple consiste alors à exprimer l'exponentielle en série de Taylor et de faire ensuite l'intégration terme par terme de la série (en s'assurant de prendre suffisamment de termes pour la convergence!).
	\end{tcolorbox}	
	
	\paragraph{Droite de Henry}\label{droite de Henry}\mbox{}\\\\
	Souvent, dans les entreprises c'est la loi de Gauss-Laplace (Normale) qui est analysée mais des logiciels courants et facilement accessibles comme Microsoft Excel sont incapables de vérifier que les données mesurées suivent une loi Normale lorsque nous faisons de l'analyse fréquentielle (aucun outil intégré par défaut ne permet de le faire) et que nous n'avons pas les données d'origines non groupées.

	L'astuce consiste alors à utiliser la variable centrée réduite qui se construit comme nous l'avons démontré plus haut avec la relation suivante:
	
	L'idée de la droite d'Henry est alors d'utiliser la relation linéaire entre k et k* donnée par l'équation de la droite:
	
	et qui peut être tracée pour déterminer la moyenne et l'écart-type de la loi Normale!
	
	\begin{tcolorbox}[colframe=black,colback=white,sharp corners]
\textbf{{\Large \ding{45}}Exemple:}\\\\
	Supposons que nous ayons l'analyse fréquentielle suivante de $10'000$ tickets de caisse dans un supermarché:
	\begin{table}[H]
		\centering
		\renewcommand{\arraystretch}{1.2}
		\small
		\begin{tabular}{cccc}\hline
		Montant des &  Nombre & Nombre cumulé & Fréquences relatives  \\[-3pt]
tickets & de tickets & de tickets & cumulées\\ \hline % ne pas enlever les espaces vides entre les lignes!!!
		[0,50[ & 668 & 668 & 0.068 \\

		[50,100[ & 919 & 1,587 & 0.1587 \\

		[100,150[ & 1,498 & 3,085 & 0.3085 \\

		[150,200[ & 1,915 & 5,000 & 0.5000 \\

		[200,250[ & 1,915 & 6,915 & 0.6915\\

		[250,300[ & 1,498 & 8,413 & 0.8413\\

		[300,350[ & 919 & 9,332 & 0.9332 \\

		[350,400[ & 440 & 9,772 & 0.9772 \\

		[400 et + & 228 & 10,000 & 1 \\ \hline
		\end{tabular}
		\caption[]{Intervalles de classe pour la détermination de la droite de Henry}
	\end{table}
	Si nous traçons maintenant cela sous Microsoft Excel 11.8346 nous obtenons:
	\begin{figure}[H]
		\centering
		\fbox{\includegraphics{img/arithmetics/distribution_example_henry_law.jpg}}
		\caption[]{Distribution des ventes de tickets}
	\end{figure}
	Ce qui ressemble terriblement à une loi Normale d'où l'autorisation, sans trop de risques, d'utiliser dans cet exemple la technique de la droite d'Henry.\\

	Mais que faire maintenant? Eh bien connaissant les fréquences cumulées, il ne nous reste plus qu'à calculer pour chacune d'entre elles $k^*$ à l'aide de tables numériques ou avec la fonction \texttt{NORMSINV( )} de la version anglaise de Microsoft Excel 11.8346 (car rappelons que l'intégration formelle de la distribution gaussienne n'est pas des plus faciles...).
	\end{tcolorbox}
	
	\pagebreak
	\begin{tcolorbox}[colframe=black,colback=white,sharp corners]
	Ceci nous donnera les valeurs de la loi Normale centrée réduite $\mathcal{N}(0,1)$ de ces mêmes fréquences respectives cumulées (fonction de répartition). Ainsi nous obtenons (nous laissons le soin au lecteur de chercher sa table numérique ou d'ouvrir son logiciel préféré...):
	\begin{table}[H]
		\centering
		\renewcommand{\arraystretch}{1.2}
		\small
		\begin{tabular}{cccc}\hline
		Borne supérieure &  Fréquences relatives & Correspondance pour $k^*$  \\[-3pt]
de l'intervalle & cumulées & ou $\mathcal{N}(0,1)$ \\ \hline % ne pas enlever les espaces vides entre les lignes!!!
		50 & 0.068 & -1.5 \\

		100 & 0.1587 & -1 \\

		150 & 0.3085 & -0.5 \\

		200 & 0.5000 & 0 \\

		250 & 0.6915 & 0.5 \\

		300 & 0.8413 & 1\\

		350 & 0.9332 & 1.5 \\

		400 & 0.9772 & 2 \\

		- & 1 & - \\ \hline
		\end{tabular}
		\caption[]{Fréquences relatives cumulées pour la droite de Henry}
	\end{table}
	Signalons que dans le type de tableau ci-dessus, dans Microsoft Excel, les valeurs de fréquences cumulées nulles et unitaires (extrêmes) posent des problèmes. Il faut alors jouer un petit peu...\\

	Comme nous l'avons spécifié plus haut, nous avons sous forme discrète:
	
	Donc graphiquement sous Microsoft Excel 11.8346 nous obtenons grâce à notre tableau le graphique suivant (évidemment en toute rigueur on fera une régression linéaire dans les règles de l'art comme vu dans le chapitre de Méthodes Numériques avec intervalles de confiance, de prédiction et tout le toutim...):
	\begin{figure}[H]
		\centering
		\fbox{\includegraphics{img/arithmetics/linearized_distribution_for_henry.jpg}}
		\caption[]{Forme linéarisée de la distribution}
	\end{figure}
	\end{tcolorbox}
	
	\pagebreak
	\begin{tcolorbox}[colframe=black,colback=white,sharp corners]
	Donc à l'aide de la régression donnée par Microsoft Excel 11.8346 (ou calculée par vos soins selon les techniques de régressions linéaires vues dans la section de Méthodes Numériques \pageref{regression techniques}). Il vient:
	
	dont nous déduisons immédiatement:
	
	Il s'agit donc d'une technique particulière pour une distribution particulière! Des techniques similaires plus ou moins simples (ou compliquées suivant les cas...) existent pour d'autres distributions.\\

	Voyons une autre manière approximative d'aborder le problème. Reprenons pour cet exemple notre tableau:
	\begin{table}[H]
		\centering
		\renewcommand{\arraystretch}{1.2}
		\small
		\begin{tabular}{cccc}\hline
		Prix &  Borne droite & Centre & Fréquences relatives  \\[-3pt]
des tickets & de l'intervalle &  & en \% \\ \hline % ne pas enlever les espaces vides entre les lignes!!!
		[0,50[ & 50 & 25 & 6.8 \\

		[50,100[ & 100 & 75 & 15.87 \\

		[100,150[ & 150 & 125 & 30.85 \\

		[150,200[ & 200 & 175 & 50.00 \\

		[200,250[ & 250 & 225 & 69.15\\

		[250,300[ & 300 & 275 & 84.13\\

		[300,350[ & 350 & 325 & 93.32 \\

		[350,400[ & 400 & 375 & 97.72 \\

		[400 et + & - & - & 100 \\ \hline
		\end{tabular}
	\end{table}
	La moyenne sera maintenant calculée à l'aide de la valeur centrale des intervalles et des effectifs selon la relation vue au début de cette section (page \pageref{arithmetic average}):
	
	
	\end{tcolorbox}
	
	\pagebreak
	\begin{tcolorbox}[colframe=black,colback=white,sharp corners]
	\begin{table}[H]
		\centering
		\renewcommand{\arraystretch}{1.2}
		\small
		\begin{tabular}{cccc}\hline
		Prix &  Centre & Fréquences relatives & Calculs  \\[-3pt]
des tickets &  & cumulées & \% \\ \hline % ne pas enlever les espaces vides entre les lignes!!!
		[0,50[ & 25 & 668 & 16,700 \\

		[50,100[ & 75 & 919 & 68,925 \\

		[100,150[ & 125 & 1,498 & 187,250 \\

		[150,200[ & 175 & 1,915 & 335,125 \\

		[200,250[ & 225 & 1,915 & 430,875\\

		[250,300[ & 275 & 1,498 & 411,950\\

		[300,350[ & 325 & 919 & 411,950 \\

		[350,400[ & 375 & 440 & 165,000 \\

		[400 et + & - & - & - \\ \hline
		 & Somme: & 9,772 & 1,914,500 \\ \hline
		 & & Moyenne: & $\dfrac{1,914,500}{9,772}=195.92$ \\ \hline
		\end{tabular}
	\end{table}
	La moyenne que nous venons de calculer est donc assez proche de la moyenne obtenue précédemment avec la droite de Henry. L'écart-type sera maintenant calculé à l'aide de la valeur centrale des intervalles et des effectifs selon la relation vue aussi au début de cette section:
	
	\begin{table}[H]
		\centering
		\renewcommand{\arraystretch}{1.2}
		\small
		\begin{tabular}{cccc}\hline
		Prix &  Centre & Fréquences relatives & Calculs  \\[-3pt]
des tickets &  & cumulées & \% \\ \hline % ne pas enlever les espaces vides entre les lignes!!!
		[0,50[ & 25 & 668 & 16,700 \\

		[50,100[ & 75 & 919 & 68,925 \\

		[100,150[ & 125 & 1,498 & 187,250 \\

		[150,200[ & 175 & 1,915 & 335,125 \\

		[200,250[ & 225 & 1,915 & 430,875\\

		[250,300[ & 275 & 1,498 & 411,950\\

		[300,350[ & 325 & 919 & 411,950 \\

		[350,400[ & 375 & 440 & 165,000 \\

		[400 et + & - & 228 & - \\ \hline
		 &  & Variance: & 8364.16 \\ \hline
		 & & Écart-Type: & 91.45 \\ \hline
		\end{tabular}
	\end{table}
	L'écart-type que nous venons de calculer est donc assez proche de l'écart-type obtenu avec la méthode de la droite de Henry.
	\end{tcolorbox}
	
	\paragraph{Diagramme quantile-quantile (Q-Q plot)}\mbox{}\\\\
	Une autre manière de juger qualitativement de l'ajustement de données expérimentales avec une loi théorique (quelle qu'elle soit!!!) est l'utilisation d'un "\NewTerm{diagramme quantile-quantile}\index{diagramme quantile-quantile}" ou également appelé "\NewTerm{q-q plot}\index{q-q plot}".

	L'idée est assez simple, il s'agit de comparer les données expérimentales aux données théoriques supposées suivre une loi donnée. Ainsi, dans le cas de notre exemple nous avons en prenant les valeurs de la moyenne ($\sim 200$) et l'écart-type ($\sim 100$) obtenus avec la droite de Henry comme paramètres théorique de la loi Normale, nous obtenons alors:
		\begin{table}[H]
		\centering
		\renewcommand{\arraystretch}{1.2}
		\small
		\begin{tabular}{cccc}\hline
		Prix &  Borne de droite & Fréquences relatives  & Borne de droite   \\[-3pt]
des tickets & expérimentale (imposée) & cumulées en \% & théorique (calculée)  \\ \hline % ne pas enlever les espaces vides entre les lignes!!!
		[0,50[ & 50 & 6.80\% & 50.91 \\

		[50,100[ & 100 & 15.87\% & 100.02 \\

		[100,150[ & 150 & 30.85\% & 149.99 \\

		[150,200[ & 200 & 50.00\% & 200 \\

		[200,250[ & 250 & 69.15\% & 250.01\\

		[250,300[ & 300 & 84.13\% & 299.98\\

		[300,350[ & 350 & 93.32\% & 350.00 \\

		[350,400[ & 400 & 97.72\% & 399.90 \\

		[400 et + & - & 100\% & - \\ \hline
		\end{tabular}
	\end{table}
	Représenté graphiquement, cela nous donne donc le fameux diagramme quantile-quantile:
	\begin{figure}[H]
		\centering
		\fbox{\includegraphics{img/arithmetics/q_q_plot.jpg}}
		\caption{Diagramme quantile-quantile de la distribution}
	\end{figure}
	Et bien évidemment on peut comparer les quantiles observés à toute loi théorique supposée. Plus les points seront alignés sur la droite de pente unitaire et d'ordonnée à l'origine nulle, meilleur sera l'ajustement! C'est très visuel, très simple et beaucoup utilisé par les non spécialistes en statistiques dans les entreprises.
	
	\pagebreak
	Pour les personnes qui n'ont peut-être pas accès à un tableur ou à un logiciel statistique, voici un tableau qui peut être utile relativement à la loi Normale:
	\begin{center}
		\begin{tabular}{rr@{\ }r@{\ }r@{\ }r@{\ }r@{\ }r@{\ }r@{\ }r@{\ }r@{\ }r@{\ }r}
		\multicolumn{11}{c}{DISTRIBUTION CUMULATIVE DE LA LOI NORMALE}\\
		\ \\
		$x$&0.00&0.01&0.02&0.03&0.04&0.05&0.06&0.07&0.08&0.09\\
		\ \\
		0.0&0.5000&0.5040&0.5080&0.5120&0.5160&0.5199&0.5239&0.5279&0.5319&0.5359\\
		0.1&0.5398&0.5438&0.5478&0.5517&0.5557&0.5596&0.5636&0.5675&0.5714&0.5753\\
		0.2&0.5793&0.5832&0.5871&0.5910&0.5948&0.5987&0.6026&0.6064&0.6103&0.6141\\
		0.3&0.6179&0.6217&0.6255&0.6293&0.6331&0.6368&0.6406&0.6443&0.6480&0.6517\\
		0.4&0.6554&0.6591&0.6628&0.6664&0.6700&0.6736&0.6772&0.6808&0.6844&0.6879\\
		0.5&0.6915&0.6950&0.6985&0.7019&0.7054&0.7088&0.7123&0.7157&0.7190&0.7224\\
		0.6&0.7257&0.7291&0.7324&0.7357&0.7389&0.7422&0.7454&0.7486&0.7517&0.7549\\
		0.7&0.7580&0.7611&0.7642&0.7673&0.7703&0.7734&0.7764&0.7794&0.7823&0.7852\\
		0.8&0.7881&0.7910&0.7939&0.7967&0.7995&0.8023&0.8051&0.8078&0.8106&0.8133\\
		0.9&0.8159&0.8186&0.8212&0.8238&0.8264&0.8289&0.8315&0.8340&0.8365&0.8389\\
		1.0&0.8413&0.8438&0.8461&0.8485&0.8508&0.8531&0.8554&0.8577&0.8599&0.8621\\
		1.1&0.8643&0.8665&0.8686&0.8708&0.8729&0.8749&0.8770&0.8790&0.8810&0.8830\\
		1.2&0.8849&0.8869&0.8888&0.8907&0.8925&0.8944&0.8962&0.8980&0.8997&0.9015\\
		1.3&0.9032&0.9049&0.9066&0.9082&0.9099&0.9115&0.9131&0.9147&0.9162&0.9177\\
		1.4&0.9192&0.9207&0.9222&0.9236&0.9251&0.9265&0.9279&0.9292&0.9306&0.9319\\
		1.5&0.9332&0.9345&0.9357&0.9370&0.9382&0.9394&0.9406&0.9418&0.9429&0.9441\\
		1.6&0.9452&0.9463&0.9474&0.9484&0.9495&0.9505&0.9515&0.9525&0.9535&0.9545\\
		1.7&0.9554&0.9564&0.9573&0.9582&0.9591&0.9599&0.9608&0.9616&0.9625&0.9633\\
		1.8&0.9641&0.9649&0.9656&0.9664&0.9671&0.9678&0.9686&0.9693&0.9699&0.9706\\
		1.9&0.9713&0.9719&0.9726&0.9732&0.9738&0.9744&0.9750&0.9756&0.9761&0.9767\\
		2.0&0.9772&0.9778&0.9783&0.9788&0.9793&0.9798&0.9803&0.9808&0.9812&0.9817\\
		2.1&0.9821&0.9826&0.9830&0.9834&0.9838&0.9842&0.9846&0.9850&0.9854&0.9857\\
		2.2&0.9861&0.9864&0.9868&0.9871&0.9875&0.9878&0.9881&0.9884&0.9887&0.9890\\
		2.3&0.9893&0.9896&0.9898&0.9901&0.9904&0.9906&0.9909&0.9911&0.9913&0.9916\\
		2.4&0.9918&0.9920&0.9922&0.9925&0.9927&0.9929&0.9931&0.9932&0.9934&0.9936\\
		2.5&0.9938&0.9940&0.9941&0.9943&0.9945&0.9946&0.9948&0.9949&0.9951&0.9952\\
		2.6&0.9953&0.9955&0.9956&0.9957&0.9959&0.9960&0.9961&0.9962&0.9963&0.9964\\
		2.7&0.9965&0.9966&0.9967&0.9968&0.9969&0.9970&0.9971&0.9972&0.9973&0.9974\\
		2.8&0.9974&0.9975&0.9976&0.9977&0.9977&0.9978&0.9979&0.9979&0.9980&0.9981\\
		2.9&0.9981&0.9982&0.9982&0.9983&0.9984&0.9984&0.9985&0.9985&0.9986&0.9986\\
		3.0&0.9987&0.9987&0.9987&0.9988&0.9988&0.9989&0.9989&0.9989&0.9990&0.9990\\
		3.1&0.9990&0.9991&0.9991&0.9991&0.9992&0.9992&0.9992&0.9992&0.9993&0.9993\\
		3.2&0.9993&0.9993&0.9994&0.9994&0.9994&0.9994&0.9994&0.9995&0.9995&0.9995\\
		3.3&0.9995&0.9995&0.9995&0.9996&0.9996&0.9996&0.9996&0.9996&0.9996&0.9997\\
		3.4&0.9997&0.9997&0.9997&0.9997&0.9997&0.9997&0.9997&0.9997&0.9997&0.9998\\
		3.5&0.9998&0.9998&0.9998&0.9998&0.9998&0.9998&0.9998&0.9998&0.9998&0.9998\\
		3.6&0.9998&0.9998&0.9999&0.9999&0.9999&0.9999&0.9999&0.9999&0.9999&0.9999\\
		3.7&0.9999&0.9999&0.9999&0.9999&0.9999&0.9999&0.9999&0.9999&0.9999&0.9999\\
		3.8&0.9999&0.9999&0.9999&0.9999&0.9999&0.9999&0.9999&0.9999&0.9999&0.9999\\
		3.9&1.0000&1.0000&1.0000&1.0000&1.0000&1.0000&1.0000&1.0000&1.0000&1.0000\\
		\end{tabular}
	\end{center}

	\pagebreak
	\subsubsection{Distribution Log-Normale}\label{log normal distribution}
	Nous disons qu'une variable aléatoire positive $X $suit une "\NewTerm{distribution log-normale}\index{distribution log-normale}" (ou "\NewTerm{loi log-normale}\index{loi log-normale}") si en posant:
	
	nous voyons que $y$ suit une fonction de probabilité de type loi Normale de moyenne $\mu$ et de variance $\sigma^2$ (moments de la loi Normale).
	
	In extenso, de par les propriétés des logarithmes, une variable peut être modélisée par une loi log-normale si elle est le résultat de la multiplication d'un grand nombre de petits facteurs indépendants (propriété du produit en somme des logarithmes et de la stabilité de la loi Normale par l'addition).
	
	La fonction de densité de $X$ pour $x \geq 0$ est alors (cela sera justifié plus bas!):
	
	qui peut être calculée dans la version française de Microsoft Excel 11.8346 avec la fonction \texttt{LOI.LOGNORMALE( )} ou pour la réciproque par \texttt{LOI.LOGNORNALE.INVERSE( )}.

	Ce type de scénario se retrouve fréquemment en physique, dans les techniques de maintenance ou encore en finance des marchés dans le modèle de pricing des options (voir ces chapitres respectifs du site pour des exemples concrets). Il y a par ailleurs une remarque importante relativement à la loi log-normale dans le traitement plus loin du théorème central limite!

	Montrons que la fonction de probabilité cumulée correspond bien à une loi Normale si nous faisons le changement de variable mentionné précédemment:
	
	en posant:
	
	et (par définition):
	
	nous avons bien:
	
	Nous retombons donc bien sur une loi Normale!

	L'espérance (moyenne) de $X$ est donnée alors par (le logarithme népérien n'étant pas défini pour $x<0$ nous bornons l'intégrale à partir de zéro):
	
	où nous avons effectué le changement de variable:
	
	L'expression:
	
	étant par ailleurs égale à:
	
	la dernière intégrale devient donc:
	
	et où nous avons utilisé la propriété qui a émergée lors de notre étude de la loi Normale, c'est-à-dire que toute intégrale de la forme:	
	
	a donc toujours la même valeur!
	
	Pour le calcul de la variance, rappelons que pour une variable aléatoire $X$, nous avons la relation de Huyghens:
	
	
	Calculons  $\text{E}(X^2)$ en procédant de manière similaire aux développements précédents:
	
	où nous avons encore une fois le changement de variable:
	
	et où nous avons transformé l'expression:
	
	sous la forme:
	
	Donc:
	
	Voici un exemple de tracé de la fonction de distribution et répartition pour la fonction Log-Normale de paramètres  $(\mu,\sigma)=(0,1)$:
	\begin{figure}[H]
		\centering
		\includegraphics{img/arithmetics/law_log_normal.jpg}
		\caption{Loi Log-Normale $\texttt{NB}$ (fonction de densité et de distribution cumulative)}
	\end{figure}
	
	\subsubsection{Distribution Uniforme Continue}\index{distribution uniforme continue}
	Soient $a<b$. Nous définissons la fonction de distribution de la "\NewTerm{fonction uniforme}\index{fonction uniforme}" (ou "\NewTerm{loi uniforme}\index{loi uniforme}") par la relation:
	
	où $1_{[a,b]}$ signifie qu'en dehors du domaine de définition $[a, b]$ la fonction de distribution est nulle. Nous retrouverons ce type de notation pour certaines autres fonctions de distribution plus loin.
	
	Nous avons donc pour fonction de répartition:
	
	Il s'agit bien d'une fonction de distribution car elle vérifie (intégrale simple):
	
	La distribution uniforme a par ailleurs pour espérance (moyenne):
	
	et pour variance en utilisant le théorème de Huygens: 
	
	Voici un exemple de tracé de la fonction de distribution et respectivement de répartition pour la loi Uniforme continue de paramètres $(a,b)=(0,1)$:
	\begin{figure}[H]
		\centering
		\includegraphics{img/arithmetics/law_uniform_continuous.jpg}
		\caption{Loi uniforme continue (fonction de densité et de distribution cumulative)}
	\end{figure}
	\begin{tcolorbox}[title=Remarque,colframe=black,arc=10pt]
	Cette distribution est souvent utilisée en simulation dans les entreprises pour signaler que la variable aléatoire a des probabilités égales d'avoir une valeur comprise dans un certain intervalle (typiquement dans les rendements de portefeuilles ou encore dans l'estimation des durées des projets). Le meilleur exemple d'application étant à nouveau le logiciel CrystalBall ou @Risk qui s'intègrent dans Microsoft Project.
	\end{tcolorbox}

	Voyons un résultat intéressant de la loi Uniforme continue (et qui s'applique à la discrète aussi en fait...).

	Souvent j'entends des gestionnaires (qui se jugent de haut niveau) dire que comme une mesure a une probabilité égale d'avoir lieu dans un intervalle fermé donné, alors la somme de deux variables aléatoires indépendantes du même type aussi!

	Or nous allons démontrer ici que ce n'est pas le cas (si quelqu'un a une démonstration plus élégante je suis preneur)!
	\begin{dem}
	Considérons deux variables aléatoires indépendantes $X$ et $Y$ qui suivent une loi uniforme dans un intervalle fermé $[0, a]$. Nous cherchons donc la densité de leur somme qui sera notée:
	
	Nous avons alors:
	
	avec la variable:
	
	Pour calculer la loi de la somme, rappelons que nous savons qu'en termes discrets cela équivaut à faire le produit conjoint des probabilités (\SeeChapter{voir section Probabilités page \pageref{joint probability}}) d'apparition des deux variables continues (se rappeler le même genre de calcul sous forme discrète!).

	C'est-à-dire:
	
	Comme $f_Y(y)=1$  si $0\leq y \leq a$ et $0$ sinon alors le produit de convolution précédent se réduit à:
	
	L'intégrant vaut par définition $0$ sauf lorsque par construction $0\leq z-y \leq a$ où il vaut alors $1$.
	
	Intéressons-nous alors aux bornes de l'intégrale dans ce dernier cas qui est bien évidemment le seul qui est intéressant....

	Faisons d'abord un changement de variables en posant:
	
	d'où:
	
	L'intégrale s'écrit alors dans cet intervalle après ce changement de variable:
	
	En se rappelant comme vu au début que $0\leq z \leq 2a$, alors nous avons immédiatement si $z<0$ et $z >2a$  que l'intégrale est nulle.
	
	Nous allons considérer deux cas pour cet intervalle car la convolution de ces deux fonctions rectangulaires peut se distinguer selon la situation où dans un premier temps elles se croisent (s'emboîtent), c'est-à-dire où $0\leq z \leq a$, et ensuite s'éloignent l'une de l'autre, c'est-à-dire $a< z \leq 2a$.
	
	\begin{itemize}
		\item Dans le premier cas (emboîtement) où $0\leq z \leq a$:
		
		où nous avons changé la borne inférieure à $0$ car de toute façon $f_X(u)$ est nulle pour toute valeur négative (et lorsque $0\leq z \leq a$, $z-a$ est justement négatif ou nul!).
		
		\item Dans le deuxième cas (déboîtement) où $a< z \leq 2a$:
		
		où nous avons changé la borne supérieure à $a$ car de toute façon $f_X(u)$ est nulle pour toute valeur supérieure (et lorsque $a \leq  z \leq 2a$, $z$ est justement plus grand que $a$).
		
		Donc au final, nous avons:
		
	\end{itemize}
	\begin{flushright}
		$\blacksquare$  Q.E.D.
	\end{flushright}
	\end{dem}
	Il s'agit d'un cas particulier, volontairement simplifié, de la loi triangulaire que nous allons voir de suite.
	
	Ce résultat (qui peut sembler contre intuitif) se vérifie en quelques secondes avec un tableur comme Microsoft Excel 11.8346 en utilisant la fonction \texttt{ALEA.ENTRE.BORNES( )} et la fonction \texttt{FREQUENCE( )} dans la version française.
	
	\pagebreak
	\subsubsection{Distribution Triangulaire}\label{triangular distribution}
	
	Soient $a<c<b$. Nous définissons la "\NewTerm{distribution triangulaire}\index{distribution triangulaire}" (ou "\NewTerm{loi triangulaire}\index{loi triangulaire}") par construction basée sur les deux fonctions de distribution suivantes:
	
	où $a$ est souvent assimilée à la valeur optimiste, $c$ la valeur attendue (le mode) et $b$ la valeur pessimiste.

	C'est aussi la seule manière d'écrire cette fonction de distribution si le lecteur garde à l'esprit que le triangle de base $c-a$ doit avoir une hauteur $h$ valant $2/(c-a)$ telle que sa surface totale soit égale à l'unité (nous allons de suite le démontrer).

	Voici un exemple de tracé de la fonction de distribution et répartition pour la fonction triangulaire de paramètres $(a, c, b) = (0, 3, 5)$:
	\begin{figure}[H]
		\centering
		\includegraphics{img/arithmetics/law_triangular.jpg}
		\caption{Loi triangulaire (fonction de densité et de distribution cumulative)}
	\end{figure}
	La pente de la première droite (croissante de gauche) est donc bien évidemment:
	
	et la pente de la deuxième droite (décroissante à droite):
	
	Cette fonction est une fonction de distribution si elle vérifie:
	
	Il s'agit dans ce cas, simplement de l'aire du triangle qui rappelons-le est simplement la base multipliée par la hauteur le tout divisé par $2$ (\SeeChapter{voir section Formes Géométriques page \pageref{unspecified triangle}}):
	
	\begin{tcolorbox}[title=Remarque,colframe=black,arc=10pt]
	Cette distribution est beaucoup utilisée en gestion de projet dans le cadre de l'estimation des durées des tâches ou encore en simulations industrielles. La valeur $a$ correspondant à la valeur optimiste, la valeur $c$ à la valeur attendue (mode) et la valeur $b$ à la valeur pessimiste. Le meilleur exemple d'application étant à nouveau les logiciels CrystalBall ou @Risk qui s'intègrent dans Microsoft Project.
	\end{tcolorbox}
	La distribution triangulaire a par ailleurs une espérance (moyenne):
	
	
	et pour variance:
	
	Nous pouvons remplacer $\mu$ par l'expression obtenue précédemment et simplifier (c'est de l'algèbre élémentaire pénible...):
	
	Nous pouvons montrer que la somme de deux variables aléatoires indépendantes chacune de loi uniforme sur $[a, b]$ (donc indépendantes et identiquement distribuées) suit une loi triangulaire sur $[2a, 2b]$ mais si elles n'ont pas les mêmes bornes, alors leur somme donne un truc qui n'a pas de nom à ma connaissance...
	
	\subsubsection{Distribution de Pareto}\label{pareto distribution}
	La "\NewTerm{distribution de Pareto}\index{distribution de Pareto}" (ou "\NewTerm{loi de Pareto}\index{loi de Pareto}"), appelée aussi "\NewTerm{loi de puissance}\index{loi de puissance (statistique)}" ou encore "\NewTerm{loi scalante}\index{loi scalante (statistique)}" est la formalisation du principe des $80-20$. Cet outil d'aide à la décision détermine les facteurs (environ $20\%$) cruciaux qui influencent la plus grande partie ($80\%$) de l'objectif.
	
	\begin{tcolorbox}[title=Remarque,colframe=black,arc=10pt]
	Cette loi est un outil fondamental et basique en gestion de la qualité (voir section de Génie Industriel page \pageref{abc method} et Techniques de Gestion \pageref{pareto analysis}). Elle est aussi utilisée en réassurance. La théorie des files d'attente s'est intéressée à cette distribution, lorsque des recherches des années 1990 ont montré que cette loi régissait aussi nombre de grandeurs observées dans le trafic Internet (et plus généralement sur tous les réseaux de données à grande vitesse).
	\end{tcolorbox}
	
	Une variable aléatoire est dite par définition suivre une loi de Pareto si sa fonction de répartition est donnée par:
	
	avec $x$ qui doit être supérieur ou égal à $x_m$.
	
	La fonction de densité (fonction de distribution) de Pareto est alors:
	
	avec  $k\in \mathbb{R}_+$ et $x \geq x_m \geq 0$ (donc $x>0$).
	
	La distribution de Pareto est donc définie par deux paramètres, $x_m$ et $k$ (nommé "\NewTerm{index de Pareto}\index{index de Pareto}"). C'est une loi dite aussi à "\NewTerm{invariance d'échelle}\index{invariance d'échelle}" ou "\NewTerm{loi fractale}\index{loi fractale}", terme définissant la propriété suivante:
	
	La loi de Pareto est par ailleurs bien une fonction de distribution puisque étant connue sa fonction de répartition:
	
	L'espérance (moyenne) est donnée par:
	
	si $k>1$. Si $k\leq 1$, l'espérance n'existe pas.
	
	Pour calculer la variance, en utilisant la relation de Huygens:
	
	nous avons:
	
	si $k>2$. Si $k\leq 2$, $\text{E}(X^2)$ n'existe pas! 
	
	Donc si $k>2$:
	
	Si  $k\leq 2$, la variance n'existe pas!
	
	Voici un exemple de tracé de la fonction de distribution et répartition pour la fonction de Pareto de paramètres $(x,x_m,k)=(x,1,2)$:
	\begin{figure}[H]
		\centering
		\includegraphics{img/arithmetics/law_pareto.jpg}
		\caption{Loi de Pareto (fonction de densité et de distribution cumulative)}
	\end{figure}
	\begin{tcolorbox}[title=Remarque,colframe=black,arc=10pt]
	Il faut noter que lorsque $k\rightarrow +\infty$ la distribution s'approche de $\delta (x-x_m)$ où $\delta$ est la fonction Delta de Dirac.
	\end{tcolorbox}
	Il existe une autre manière importante de déduire la famille des lois de Pareto qui permet de comprendre bien des choses concernant d'autres lois et qui est souvent présentée de la façon suivante:
	
	Notons $x_0$ le seuil au-delà duquel nous calculons l'espérance de la quantité examinée, et $\text{E}(Y)$ l'espérance au-delà de ce seuil equation tel qu'il soit proportionnel (linéairement dépendant) au seuil choisi:
	
	Cette relation fonctionnelle exprime l'idée que la moyenne conditionnelle au-delà du seuil $x_0$ est un multiple de ce seuil à une constante près, c'est-à-dire une fonction linéaire de ce seuil.

	Ainsi, en gestion de projets par exemple, nous pourrions dire qu'une fois une certain seuil de durée dépassé, la durée espérée est un multiple de ce même seuil à une constante près.

	Si une relation linéaire de ce type existe et est bien vérifiée, nous parlons alors de distribution de probabilité sous la forme d'une loi de Pareto généralisée.

	Considérons l'espérance mathématique de la fonction conditionnelle bayésienne donnée par (\SeeChapter{voir section Probabilités page \pageref{conditional expectation}}):
	
	où la notation à gauche est un peu raccourcie mais le lecteur comprendra implicitement qu'il s'agit d'un espérance conditionnelle.
	
	Si nous notons $F(y)$ la fonction de répartition de $f(y)$, nous avons alors par définition:
	
	Dès lors:
	
	et si nous définissons:
	
	que nous pouvons assimiler à la "queue de la distribution".
	
	Il vient:
	
	et donc nous cherchons le cas très particulier où:
	
	c'est-à-dire:
	
	En dérivant par rapport à $x$, nous trouvons:
	
	La dérivée de l'intégrale définie ci-dessus sera la dérivée d'une constante (valorisation de l'intégrale en $+\infty$) moins la dérivée de l'intégrale de l'expression analytique en $x_0$. Nous avons donc:
	
	Soit:
	
	et comme:
	
	il vient:
	
	Après simplification et réarrangement nous obtenons:
	
	qui est donc une équation différentielle en $\bar{F}(x)$. Sa résolution fournit toutes les formes de lois de Pareto recherchées, selon les valeurs que prennent les paramètres $a$ et $b$.
	
	Pour résoudre cette équation différentielle, considérons le cas particulier où $a>1,b=0$. Nous avons alors:
	
	En posant:
	
	Nous avons alors:
	
	et donc:
	
	Il vient:
	
	et dès lors:
	
	et donc\label{pareto tail distribution}:
	
	Ensuite, il vient pour la fonction de distribution cumulative:
	
	Si nous cherchons la fonction de distribution, nous dérivons par $x$ pour obtenir:
	
	Il s'agit de la loi de Pareto que nous avons utilisée depuis le début et nommée \NewTerm{"distribution de Pareto de type I}\index{"distribution de Pareto de type I}" (nous ne montrerons pas dans ce livre celles de type II).
	
	Une chose intéressante à observer au passage est le cas de la résolution de l'équation différentielle:
	
	lorsque $a=1,b>0$. L'équation différentielle se réduit alors à:
	
	Soit:
	
	Après intégration:
	
	et donc:
	
	Si nous faisons un petit changement de notation:
	
	et que nous écrivons la fonction de répartition:
	
	et en dérivant nous obtenons la fonction de distribution de la loi exponentielle:
	
	Donc la loi exponentielle a une espérance conditionnelle seuil qui est égale à:
	
	Donc l'espérance conditionnelle seuil est égale à elle-même augmenté de l'écart-type de la distribution.

	\pagebreak
	\subsubsection{Distribution exponentielle}\label{exponential distribution}
	Nous définissons la "\NewTerm{distribution exponentielle}\index{distribution exponentielle}" (ou "\NewTerm{loi exponentielle}\index{loi exponentielle}") par la relation de fonction de distribution suivante:
	
	avec  $\lambda > 0$ qui comme nous allons de suite le montrer n'est au fait que l'inverse de la moyenne et où $x$ est une variable aléatoire sans mémoire. Cette loi est parfois notée $\mathcal{E}(\lambda)$.
	
	Au fait la loi exponentielle découle naturellement de développements très simples (voir celui dans le chapitre de Physique Nucléaire par exemple) sous des hypothèses qui imposent une constance dans le vieillissement d'un phénomène. Dans le chapitre des Techniques de Gestion, nous avons aussi démontré en détails dans la partie concernant la théorie des files d'attentes (page \pageref{without memeory process}), que cette loi était sans mémoire. C'est-à-dire que la probabilité cumulée qu'un phénomène se produise entre les temps $t$ et $t + s$, s'il ne s'est pas produit avant, est la même que la probabilité cumulée qu'il se produise entre les temps $0$ et $s$.
	
	\begin{tcolorbox}[title=Remarques,colframe=black,arc=10pt]
	\textbf{R1.} Cette distribution se retrouve fréquemment en physique nucléaire (voir chapitre du même nom) ou encore en physique quantique (voir aussi chapitre du même nom) ainsi qu'en fiabilité (\SeeChapter{voir section Techniques de Gestion page \pageref{industrial engineering}}) ou dans la théorie des files d'attentes (\SeeChapter{voir section Techniques de Gestion page \pageref{queueing theory}}).\\

	\textbf{R2.} Nous pouvons obtenir cette loi dans la version française de Microsoft Excel 11.8346 avec la fonction \texttt{LOI.EXPONENTIELLE( )}.
	\end{tcolorbox}	
	
	Il s'agit par ailleurs bien d'une fonction de distribution car elle vérifie:
	
	La distribution exponentielle a pour espérance (moyenne) en utilisant l'intégration par parties:
	
	et pour variance en utilisant à nouveau la relation de Huygens:
	
	il ne nous reste plus qu'à calculer:
	
	Un changement de variable $y=\lambda$ conduit à:
	
	Une double intégration par parties donne:
	
	D'où:
	
	Il vient dès lors:
	
	Donc l'écart-type (racine carrée de la variance pour rappel) et l'espérance ont exactement la même expression!

	Voici un exemple de tracé de la fonction de distribution et répartition pour la fonction exponentielle de paramètre $\lambda=1$:
	\begin{figure}[H]
		\centering
		\includegraphics{img/arithmetics/law_exponential.jpg}
		\caption{Loi exponentielle (fonction de densité et de distribution cumulative)}
	\end{figure}
	Déterminons maintenant la fonction de répartition de la loi exponentielle:
	
	\begin{tcolorbox}[title=Remarque,colframe=black,arc=10pt]
	Nous verrons plus loin que la fonction de distribution exponentielle n'est qu'un cas particulier d'une distribution plus générale qui est la distribution du Khi-deux, cette dernière aussi n'étant qu'un cas particulier d'une distribution encore plus générale qui est la distribution Gamma. Il s'agit d'une propriété très importante utilisée dans le "test de Poisson" pour les événements rares (voir plus loin aussi).
	\end{tcolorbox}
	
	\subsubsection{Distribution de Cauchy}
	Soient $X, Y$ deux variables aléatoires indépendantes suivant des lois Normales centrées réduites (variance unité et espérance nulle). La fonction de densité est donc donnée pour chacune des variables par:
	
	La variable aléatoire:
	
	(la valeur absolue interviendra dans une intégrale lors d'un changement variable) suit une allure caractéristique appelée "\NewTerm{distribution de Cauchy}\index{distribution de Cauchy}" (ou "loi de Cauchy") ou encore "\NewTerm{loi de Lorentz}\index{loi de Lorentz}".
	
	Déterminons sa fonction de densité $f$. Pour cela, rappelons que f est déterminée par la relation (générale):
	
	Donc (application du calcul intégral élémentaire):
	
	dans le cas où $f$ est continue.
	
	Etant donné que $X$ et $Y$ sont indépendants, la fonction de densité du vecteur aléatoire est donnée par un des axiomes des probabilités (\SeeChapter{voir section Probabilités page \pageref{joint probability}}):
	
	Donc:
	
	où $D=\left\lbrace(x,y)\vert x<t\vert y\vert\right\rbrace$. Cette dernière intégrale devient:
	
	Faisons le changement de variable $x=u\vert y \vert$ dans l'intégrale intérieure. Nous obtenons:
	
	Donc:
	
	C'est maintenant que la valeur absolue va nous être utile pour écrire:
	
	Pour la première intégrale nous avons:
	
	Il ne reste donc plus que la seconde intégrale et en faisant le changement de variable $v=y^2$, nous obtenons:
	
	Ce que nous noterons par la suite (afin de respecter les notations adoptées jusqu'à présent):
	
	et qui n'est d'autre que la "distribution de Cauchy"!
	
	Il s'agit par ailleurs bien d'une fonction de distribution car elle vérifie (\SeeChapter{voir section Calcul Différentiel et Intégral page \pageref{usual primitives}}):
	
	Il est évident que l'on obtient donc pour la fonction de distribution cumulée:
	
	Voici un exemple de tracé de la fonction de distribution de Cauchy:
	\begin{figure}[H]
		\centering
		\includegraphics{img/arithmetics/law_cauchy.jpg}
		\caption{Loi de Cauchy (fonction de densité)}
	\end{figure}
	La distribution de Cauchy a pour espérance (moyenne):
	
	
	\begin{tcolorbox}[colback=red!5,borderline={1mm}{2mm}{red!5},arc=0mm,boxrule=0pt]
	\bcbombe Attention!!! Les calculs précédents ne donnent pas zéro en réalité fait car la soustraction d'infinis n'est non pas nulle mais indéterminée! La loi de Cauchy n'admet donc pas d'espérance rigoureusement parlant!
	\end{tcolorbox}
	
	Ainsi, même si nous pouvons bricoler une variance:
	
	celle-ci est absurde et n'existe rigoureusement parlant pas puisque l'espérance n'existe pas...!
	
	La distribution de Cauchy est beaucoup utilisée en ingénierie financière car elle est dite à "queue lourde" et donc un très bon candidat pour être plus précis dans la prédiction de valeurs extrêmes à l'opposé de la distribution Normale qui a les queues décroissant radidement (exponentiellement). De plus, la distribution de Cauchy est une loi à queue lourde avec un support sur $\mathbb{R}$ alors que la distribution de Pareto (également à queue lourde) est définie uniquement sur $\mathbb{R}^+$.
	
	La distribution de Cauchy est l'une des fonctions de distribution les plus célèbres que ... nous ne pouvons pas trouver dans les logiciels de type tableurs comme  Microsoft Excel. Pour pouvoir obtenir la forme fermée du la distribution cumulée inverse de Cauchy, nous partons de fonction de distribution cumullée prouvée précédemment:
	
	et donc si on pose:
	
	Nous obtenons immédiatement le fonction de distribution cumulée inverse:
	
	Cela est utile en finance comme nous le savons (\SeeChapter{voir section d'Informatique Théorique page \pageref{inverse transform sampling}}) pour simuler une variable de Cauchy lorsque nous utilisons l'échantillonnage à transformée inverse:
	
	En utilisant la transformation de variable aléatoire (voir page \pageref{random variable transformation}) nous déduisons immédiatement la "\NewTerm{distribution générale de Cauchy}\index{distribution g\'en\'erale de Cauchy}" ou simplement nommée la "\NewTerm{distribution de Cauchy}\index{distribution de Cauchy}" et donnée par :
	
	
	\subsubsection{Distribution Bêta}\label{beta distribution}
	Rappelons d'abord que la fonction Gamma d'Euler est définie par la relation (\SeeChapter{voir section de Calcul Intégral et Différentiel page \pageref{gamma euler function}}):
	
	Nous avons démontré (\SeeChapter{voir section de Calcul Différentiel et Intégral page \pageref{gamma euler function}}) qu'une propriété non triviale de cette fonction est que:
	
	Posons maintenant:
	
	où:
	
	En faisant le changement de variables:
	
	nous obtenons:
	
	Pour l'intégrale interne nous utilisons maintenant la substitution $v=ut, 0\leq t\leq 1$ et nous trouvons alors:
	
	La fonction $B$ qui apparaît dans l'expression ci-dessus est appelée "\NewTerm{fonction bêta}\index{fonction bêta}\label{beta function}" et nous avons donc:
	
	Maintenant que nous avons défini ce qu'était la fonction bêta, considérons les deux paramètres $a>0,b>0$ et considérons la relation particulière ci-dessous comme étant la "\NewTerm{distribution bêta}\index{distribution bêta}" ou "\NewTerm{loi bêta}\index{loi bêta}" (il existe plusieurs formulations de la loi bêta dont une très importante qui est étudiée en détails dans le chapitre de Techniques de Gestion à la page \pageref{probabilitic pert}):
	
	où:
	
	\begin{tcolorbox}[title=Remarque,colframe=black,arc=10pt]
	L'intégrale réécrite comme suit:
	
	est appelée "\NewTerm{intégrale de Chebyshev}\index{intégrale de Chebyshev}".\\
	
	L'intégrale généralisée comme suit:
	
	est appelée "\NewTerm{fonction bêta incomplète}\index{fonction bêta incomplète}\label{incomplete beta function}". Bien qu'elle soit surtout connue pour ses applications en statistique, elle est également largement utilisée dans de nombreux autres domaines tels que la science actuarielle, l'économie, la finance, l'analyse de survie, les tests de survie et les télécommunications.
	\end{tcolorbox}	
	
	Nous vérifions d'abord que $P_{a,b}(x)$ est bien une fonction de distribution (sans trop aller dans les détails...):
	
	Maintenant, calculons son espérance (moyenne):
	
	en utilisant la relation:
	
	et sa variance:
	
	En sachant que
	
	 nous trouvons:
	
	et donc:
	
	Exemples de tracés de la fonction de distribution (densité) bêta pour $(a,b)=(0.1,0.5)$ en rouge, $(a,b)=(0.3,0.5)$ en vert, $(a,b)=(0.5,0.5)$ en noir, $(a,b)=(0.8,0.8)$ en bleu, $(a,b)=(1,1)$ en magenta, $(a,b)=(1,1.5)$ en cyan, $(a,b)=(1,2)$ en gris, $(a,b)=(1.5,2)$ en turquoise, $(a,b)=(2,2)$ en jaune,$(a,b)=(3,3)$ en couleur or:
	\begin{figure}[H]
		\centering
		\includegraphics{img/arithmetics/law_beta_samples.jpg}
		\caption{Loi bêta (fonctions de densité)}
	\end{figure}
	et tracé de la fonction de distribution et répartition de la loi bêta de paramètres $(a,b)=(2,3)$:
	\begin{figure}[H]
		\centering
		\includegraphics{img/arithmetics/law_beta.jpg}
		\caption{Loi bêta (fonction de densité et distribution cumulative)}
	\end{figure}
	Le fait que la loi bêta soit une des rares fonctions de distribution dont le support soit compris entre $]0,1[$ explique son usage courant dans les statistiques bayésiennes en tant que loi a priori de la distribution d'une proportion!
	\begin{tcolorbox}[title=Remarque,colframe=black,arc=10pt]
	Remarquez le cas spécial important  lorsque $ a, b = 1 $, la distribution bêta, se réduit à une distribution continue uniforme! En effet:
	
	\end{tcolorbox}
	En général, si vous voulez une distribution bêta a priori avec les paramètres $a$ et $b$ mais que vous ne connaissez que $\mu$ et $\sigma^2$, ils peuvent être trouvés sur via les deux relations:
	
	Nous pouvons résoudre le premier comme suit pour $b$:
	
	La substitution de ce dernier dans le $\sigma^2$ ci-dessus donne:
	
	Réorganiser ce dernier donne:
	
	dès lors:
	
	
	\paragraph{Fonction bêta integrale régularisée}\mbox{}\\\\
	Ce qui suit devrait être placé dans la section de d'Analyse Fonctionnelle, mais il nous a semblé plus pédagogique de le placer juste ici pour des raisons évidentes.
	
	Considérons la "\NewTerm{fonction bêta régularisée incomplète $I_x(a,b)$}\index{fonction bêta régularisée incomplète}\label{incomplete Beta function}" (ou "\NewTerm{fonction bêta régularisée}" pour faire court) définie en termes de fonction bêta incomplète et de fonction bêta complète comme (pour $\alpha>0$):
	
	où pour rappel $a>0$ et $b>0$. Cette fonction nous sera utile dans certaines statistiques bayésiennes A/B (\SeeChapter{voir section Statistiques page \pageref{A/B testing for binary outcomes}}) mais aussi en Théorie des Cordes (\SeeChapter{voir section Théorie des Cordes page \pageref{string theory}})!
	
	Remarquez les valeurs canoniques qui sont immédiates:
	
	Nous voulons prouver ici la propriété récursive:
	
	\begin{tcolorbox}[title=Remarque,colframe=black,arc=10pt]
	Cela signifie également que:
	
	mais pour $a>1$ (et toujours $b>0$).
	\end{tcolorbox}
	\begin{dem}
	En utilisant la propriété suivant de la fonction bêta (voir page \pageref{beta function}):
	
	\begin{tcolorbox}[title=Remarque,colframe=black,arc=10pt]
	De la relation ci-dessus et en raison de la symétrie de la fonction bêta, les trois résultats suivants suivent presque immédiatement:
	
	\end{tcolorbox}
	Nous calculons la dérivée:
	
	Maintenant posons:
	
	Alors $f(a,b,0)=0$ et:
	
	Dès lors $f\equiv 0$ et la propriété de récursivité est prouvée.
	\begin{flushright}
		$\blacksquare$  Q.E.D.
	\end{flushright}
	\end{dem}
	Dons nous venons de prouver que:
	
	Et notez que nous avons le cas limite:
	
	Maintenant, itérons récursivement:
	
	Jusqu'au cas:
	
	
	En incluant le terme zéro dans la somme:
	
	Et nous nous arrêterons ici pour l'étude de cette fonction car c'est tout ce dont nous avons besoin dans l'état actuel de ce livre pour des applications pratiques!
	
	\subsubsection{Distribution Gamma}\label{gamma distribution}
	La fonction Gamma d'Euler étant connue, considérons deux paramètres $a>0,\lambda>0$ et définissons la "\NewTerm{distribution Gamma}\index{distribution Gamma}" (ou "\NewTerm{loi Gamma}\index{loi Gamma}") comme étant donnée par la relation (fonction de densité):
	
	En faisant le changement de variables $t=\lambda x$ nous obtenons:
	
	et pouvons alors écrire la relation sous une forme plus classique que nous trouvons fréquemment dans la littérature spécialisée:
	
	et c'est sous cette forme que nous retrouvons cette fonction de distribution dans la version française de Microsoft Excel 11.8346 sous le nom \texttt{LOI.GAMMA( }) et pour sa réciproque par \texttt{ LOI.GAMMA.INVERSE( )}. Indiquons que la relation précédente est aussi parfois notée sous la forme:
	
	Voyons maintenant une propriété simple de la loi Gamma qui nous sera en partie utile pour l'étude du test statistique de Welch. Rappelons d'abord que nous avons démontré plus haut que:
	
	Posons $Y=c^{te}X$,  nous avons alors immédiatement:
	
	Donc la multiplication par une constante d'une variable aléatoire qui suit une loi Gamma n'a que pour effet de diviser le paramètre $\lambda$ par cette même constante. Raisons pour laquelle $\lambda$ est appelé "\NewTerm{paramètre d'échelle}\index{paramètre d'échelle}".
	
	Si $a \in \mathbb{N}$, la loi Gamma au dénominateur devient (\SeeChapter{voir section Calcul Différentiel et Intégral page \pageref{gamma euler function}}) la factorielle $(a-1)!$. La distribution Gamma peut alors s'écrire:
	
	Cette forme particulière la fonction de distribution Gamma s'appelle alors la "fonction d'Erlang" que nous retrouvons naturellement dans la théorie des files d'attentes et qui est donc très importante dans la pratique!
	
	\begin{tcolorbox}[title=Remarque,colframe=black,arc=10pt]
	 Si $a=1$ alors $\Gamma(a)=1$ et $x^{a-1}=1$ et nous retombons sur la loi exponentielle.
	\end{tcolorbox}	
	Ensuite, nous vérifions avec un raisonnement similaire en tout point à celui de la fonction bêta que $P_{a,\lambda}(x)$ est bien une fonction de distribution:
	
	Exemples tracés de la fonction de distribution pour $(a,\lambda)=(0.5,1)$ en rouge, $(a,\lambda)=(1,1)$ en vert, $(a,\lambda)=(2,1)$ en noir, $(a,\lambda)=(4,2)$ en bleu, $(a,\lambda)=(16,8)$ en magenta:
	\begin{figure}[H]
		\centering
		\includegraphics{img/arithmetics/law_gamma_samples.jpg}
		\caption{Quelques fonctions de densité de la loi Gamma}
	\end{figure}
	et tracé de la fonction de distribution et répartition pour la fonction Gamma de paramètres $(a,\gamma)=(4,1)$:
	\begin{figure}[H]
		\centering
		\includegraphics{img/arithmetics/law_gamma.jpg}
		\caption{Loi Gamma (fonction de densité et distribution cumulative)}
	\end{figure}
	La distribution Gamma a par ailleurs pour espérance (moyenne):
	
	et pour variance:
	
	Démontrons une propriété de la distribution Gamma qui nous servira à établir plus tard dans ce chapitre, lors de notre étude de l'analyse de la variance et des intervalles de confiance sur des petits échantillons, une autre propriété extrêmement importante de la loi du Khi-deux.
	
	Comme nous le savons, la fonction de densité d'une variable aléatoire suivant une distribution Gamma de paramètres $a,\lambda>0$ est:
		
	avec (\SeeChapter{voir section de Calcul Différentiel et Intégral page \pageref{gamma euler function}}) la fonction Gamma d'Euler:
	
	Par ailleurs, quand une variable aléatoire suit une distribution Gamma nous la notons souvent sous la forme suivante :
	
	Soient $X, Y$ deux variables indépendantes. Montrons que si $X=\gamma(p,\lambda)$ et $Y=\gamma(q,\lambda)$, donc avec le même paramètre d'échelle, alors:
	
	Notons $f$ la fonction de densité du couple $X, Y,f(x)$ la fonction de densité de $X$ et $f_Y$ la fonction de densité de $Y$. Vu que $X$ et $Y$  sont indépendantes, nous avons:
	
	pour tout $x,y>0$.
	
	Soit $Z=X+Y$. La fonction de répartition de $Z$ est alors:
	
	où $D=\left\lbrace(x,y)\vert x+y\leq z \right\rbrace$.
	\begin{tcolorbox}[title=Remarque,colframe=black,arc=10pt]
	Nous appelons un tel calcul une "\NewTerm{convolution}\index{convolution}" et les statisticiens ont souvent à manipuler de telles entités ayant à travailler sur de nombreuses variables aléatoires qu'il faut sommer ou même multiplier.
	\end{tcolorbox}
	En simplifiant:
	
	Nous effectuons le changement de variable suivant: $x=x,y=s-x$. Le jacobien est alors (\SeeChapter{voir section Calcul Différentiel et Intégral page \pageref{jacobian}}):
	
	Donc avec la nouvelle borne d'intégration $s=x+y=x+(z-x)=z$ nous avons:
	
	Si nous notons par la lettre $g$ la fonction de densité de $Z$, nous avons:
	
	Par suite:
	
	$f_X$ et $f_Y$ étant nulles lorsque leur argument est négatif, nous pouvons changer les bornes d'intégration:
	
	Calculons $g$:
	
	Après le changement de variable $x=st$ nous obtenons:
	
	où $B$ est la fonction bêta que nous avons vue plus haut dans notre étude de la fonction de distribution bêta. Or nous avons aussi démontré la relation:
	
	Donc:
	
	Soit plus explicitement:
	
	Ce qui finalement nous donne:
	
	Ce qui montre que bien que si deux variables aléatoires suivent une distribution Gamma alors leur somme aussi telle que:
	
	Donc la distribution Gamma est stable par addition de même que le sont toutes les lois qui découlent de la loi Gamma et que nous allons aborder ci-après.
	
	\paragraph{Distribution Gamma Généralisée}\mbox{}\\\\
	La distribution gamma généralisée est une distribution de probabilité continue avec trois paramètres. Il s'agit d'une généralisation de la distribution gamma à deux paramètres. Étant donné que de nombreuses distributions couramment utilisées pour les modèles paramétriques dans l'analyse de survie (telles que la distribution exponentielle, la distribution de Weibull et la distribution Gamma, et log-normale) sont des cas particuliers du gamma généralisé, elle est parfois utilisé pour déterminer quel modèle paramétrique est approprié pour un ensemble de donné de données.

	Par conséquent, notons que si nous écrivons après plusieurs essais et erreurs, la fonction de densité suivante nommée "\NewTerm{loi Gamma généralisée}\index{loi Gamma généralisée}":
	
	où $x>0$, $\alpha>0$, $\eta>0$, $\kappa>0$.
 
 	Alors, pour $\kappa=1$ on retombe sur la fonction de densité de la distribution de Weibull (\SeeChapter{voir section Génie Industriel page \pageref{weibull distribution}}) qui avec nos propres notations de la section correspondante est donnée par:
	 
	Pour $\eta=1$, on retombe sur la fonction de densité gamma étudiée juste précédemment:
	
	Pour $\kappa=1$ et $\eta=1$, on retombe sur la distribution exponentielle également étudiée précédemment:
	
	et enfin pour $\eta\rightarrow 0$, $\kappa\rightarrow +\infty$ nous retombons sur une distribution log-normale après avoir développé les limites en utilisant les techniques de Stirling, Hospital et Taylor (\SeeChapter{voir section d'Informatique Théorique page \pageref{stirling}, Calcul Différentiel et Intégral page \pageref{Hospital rule}, Séquences et Séries page \pageref{taylor series}}):
	
	Comme toujours, sur demande, nous pouvons détailler les développements!
	
	\subsubsection{Distribution du Khi-deux (Pearson)}\label{chi-square distribution}
	La "\NewTerm{distribution du Khi-deux}\index{distribution du Khi-deux}" (appelée aussi "\NewTerm{"loi du Khi-deux}\index{"loi du Khi-deux}" ou encore "\NewTerm{loi de Pearson}\index{loi de Pearson}") a une place très importante dans la pratique industrielle pour certains tests d'hypothèses courants (voir plus beaucoup plus loin...) et n'est par définition qu'un cas particulier de la distribution Gamma dans le cas où $a=k/2$ et $\lambda=1/2$, avec $k$ entier positif:
	
	\begin{tcolorbox}[title=Remarque,colframe=black,arc=10pt]
	Nous prouverons également plus tard lors de notre étude de l'intervalle de confiance sur la variance à moyenne connue (voir page \pageref{ci on the variance with known mean}), la propriété très importante que si $X$ suit une distribution standard Normale $\mathcal{N}(0,1)$ alors $X^2=\chi^2_1$.
	\end{tcolorbox}
	Cette relation qui relie la distribution du Khi-deux à la distribution Gamma est importante dans la version française de Microsoft Excel 11.8346 car la fonction \texttt{LOI.KHIDEUX( )} donne le seuil de confiance et non la fonction de distribution. Il faut alors utiliser la fonction \texttt{LOI.GAMMA( )} avec les paramètres donnés ci-dessus (à part qu'il faut prendre l'inverse de $1/2$, soit $2$ comme paramètre) pour avoir la fonction de distribution et de répartition.
	
	Le lecteur qui voudra vérifier que la loi du Khi-2 est seulement un cas particulier de la loi Gamma, pourra écrire dans la version française de Microsoft Excel 14.0.6123:
	
	\begin{center}
		\texttt{=LOI.KHIDEUX.N(2*x,2*k,VRAI)}\\
		\texttt{=LOI.GAMMA.N(x,k,1,VRAI)}
	\end{center}
	Tous les calculs faits auparavant s'appliquent et nous avons alors immédiatement:
	
	Exemples de tracés de la fonction de distribution pour $k=1$ en rouge, $k=3$ en vert, equation en noir, $k=4$ en bleu:
	\begin{figure}[H]
		\centering
		\includegraphics{img/arithmetics/law_chi2_samples.jpg}
		\caption{Loi du $\chi^2$ (fonctions de distribution)}
	\end{figure}
	et tracé de la fonction de distribution et respectivement de répartition pour la loi du Khi-deux pour $k=2$ (pour plus de tracés sur cette distribution voir plus bas à la page \pageref{continuous distributions}):
	\begin{figure}[H]
		\centering
		\includegraphics{img/arithmetics/law_chi2.jpg}
		\caption{Loi du $\chi^2$ (fonction de densité et distribution cumulative) }
	\end{figure}
	Dans la littérature, il est de tradition de noter:
	
	pour indiquer que la distribution de la variable aléatoire $X$ est la loi du Khi-deux. Par ailleurs il est courant de nommer le paramètre $k$ "\NewTerm{degré de liberté}\index{degré de liberté (statistique)}" et de l'abréger "$\text{ddl}$".
	
	La fonction du Khi-deux découle donc de la loi Gamma et par ailleurs en prenant $k=2$ nous retrouvons aussi la loi exponentielle (voir plus haut) pour $\lambda=1/2$:
	
	Par ailleurs, puisque (\SeeChapter{voir section de Calcul Différentiel et Intégral page \pageref{gamma euler function}}):
	
	la loi du Khi-deux avec $k$ égal à l'unité peut s'écrire sous la forme:
	
	Enfin, terminons avec une propriété assez importante dans les tests statistiques que nous étudierons un peu plus loin et particulièrement dans les intervalles de confiance des événements rares et la fameuse méthode de Fisher pour les $p$-valeur de multiples tests d'hypothèses. Effectivement, le lecteur pourra vérifier dans un tableur comme Microsoft Excel 14.0.6123 (version française), que nous avons:
	\begin{tabbing}
		\= \\
		\>\texttt{=LOI.POISSONS.N(} $x \in \mathbb{N},\mu,$\texttt{VRAI)}\\
		\>\texttt{=1-LOI.KHIDEUX.N(} $2\mu,2(x+1),$\texttt{VRAI)}\\
		\>\texttt{=1-LOI.GAMMA.N(} $2\mu,x+1,$\texttt{VRAI)}\\
		\>\texttt{=1-LOI.EXPONENTIELLE.N(} $x,0.5,$\texttt{VRAI)}
	\end{tabbing}
	Il nous faut donc démontrer cette relation entre loi du khi-2 et loi de Poisson. Voyons cela en partant de la loi Gamma:
	
	Si nous posons $\lambda=1/2$ et $a=k/2$ nous avons alors la loi du khi-2 à $k$ degrés de libertés:
	
	Maintenant, rappelons que nous avons vu dans le chapitre de Suites Et Séries (page \pageref{usual maclaurin developments}), la série de Taylor (Maclaurin) avec reste intégral à l'ordre $n - 1$ autour de $0$ jusqu'à $\lambda$ suivante:
	
	Nous multiplions par $e^{-\lambda}$:
	
	Et donc:
	
	Or, concentrons-nous sur le terme:
	
	et faisons un premier changement de variable:
	
	et un second changement de variable (attention! le $k$ dans le changement de variable n'est pas le même que celui de la somme de la loi de Poisson...):
	
	Or, nous avons démontré dans le chapitre de Calcul Différentiel Et Intégral (page \pageref{gamma euler function}) que si $x$ est un entier strictement positif:
	
	Il vient alors:
	
	Nous avons finalement:
	
	où nous retrouvons donc bien la fonction de distribution du khi-2 sous l'intégrale! Donc au final:
	
	D'où la relation donnée plus haut pour les tableurs en se rappelant bien que nous avons posé:
	
	
	Enfin pour les personnes n'ayant pas accès à un tableur ou à un logiciel statistique:
	\begin{center}
		\begin{tabular}
		      {r@{\ }r@{\ }r@{\ }r@{\ }r@{\ }r@{\ }r@{\ }r@{\ }r@{\ }r@{\ }r@{\ }r}
		\multicolumn{12}{c}{DISTRIBUTION CUMULATIVE DE LA LOI DU KHI-DEUX}\\
		\ \\
		$k$&0.1\%&0.5\%&1.0\%&2.5\%&5.0\%&10.0\%&12.5\%&20.0\%&25.0\%&33.3\%&50.0\%\\
		\ \\
		 1&0.000&0.000&0.000&0.001&0.004&0.016&0.025&0.064&0.102&0.186&0.455\\
		 2&0.002&0.010&0.020&0.051&0.103&0.211&0.267&0.446&0.575&0.811&1.386\\
		 3&0.024&0.072&0.115&0.216&0.352&0.584&0.692&1.005&1.213&1.568&2.366\\
		 4&0.091&0.207&0.297&0.484&0.711&1.064&1.219&1.649&1.923&2.378&3.357\\
		 5&0.210&0.412&0.554&0.831&1.145&1.610&1.808&2.343&2.675&3.216&4.351\\
		 6&0.381&0.676&0.872&1.237&1.635&2.204&2.441&3.070&3.455&4.074&5.348\\
		 7&0.598&0.989&1.239&1.690&2.167&2.833&3.106&3.822&4.255&4.945&6.346\\
		 8&0.857&1.344&1.646&2.180&2.733&3.490&3.797&4.594&5.071&5.826&7.344\\
		 9&1.152&1.735&2.088&2.700&3.325&4.168&4.507&5.380&5.899&6.716&8.343\\
		10&1.479&2.156&2.558&3.247&3.940&4.865&5.234&6.179&6.737&7.612&9.342\\
		11&1.834&2.603&3.053&3.816&4.575&5.578&5.975&6.989&7.584&8.514&10.341\\
		12&2.214&3.074&3.571&4.404&5.226&6.304&6.729&7.807&8.438&9.420&11.340\\
		13&2.617&3.565&4.107&5.009&5.892&7.042&7.493&8.634&9.299&10.331&12.340\\
		14&3.041&4.075&4.660&5.629&6.571&7.790&8.266&9.467&10.165&11.245&13.339\\
		15&3.483&4.601&5.229&6.262&7.261&8.547&9.048&10.307&11.037&12.163&14.339\\
		16&3.942&5.142&5.812&6.908&7.962&9.312&9.837&11.152&11.912&13.083&15.338\\
		17&4.416&5.697&6.408&7.564&8.672&10.085&10.633&12.002&12.792&14.006&16.338\\
		18&4.905&6.265&7.015&8.231&9.390&10.865&11.435&12.857&13.675&14.931&17.338\\
		19&5.407&6.844&7.633&8.907&10.117&11.651&12.242&13.716&14.562&15.859&18.338\\
		20&5.921&7.434&8.260&9.591&10.851&12.443&13.055&14.578&15.452&16.788&19.337\\
		21&6.447&8.034&8.897&10.283&11.591&13.240&13.873&15.445&16.344&17.720&20.337\\
		22&6.983&8.643&9.542&10.982&12.338&14.041&14.695&16.314&17.240&18.653&21.337\\
		23&7.529&9.260&10.196&11.689&13.091&14.848&15.521&17.187&18.137&19.587&22.337\\
		24&8.085&9.886&10.856&12.401&13.848&15.659&16.351&18.062&19.037&20.523&23.337\\
		25&8.649&10.520&11.524&13.120&14.611&16.473&17.184&18.940&19.939&21.461&24.337\\
		26&9.222&11.160&12.198&13.844&15.379&17.292&18.021&19.820&20.843&22.399&25.336\\
		27&9.803&11.808&12.879&14.573&16.151&18.114&18.861&20.703&21.749&23.339&26.336\\
		28&10.391&12.461&13.565&15.308&16.928&18.939&19.704&21.588&22.657&24.280
		  &27.336\\
		29&10.986&13.121&14.256&16.047&17.708&19.768&20.550&22.475&23.567&25.222
		  &28.336\\
		30&11.588&13.787&14.953&16.791&18.493&20.599&21.399&23.364&24.478&26.165
		  &29.336\\
		35&14.688&17.192&18.509&20.569&22.465&24.797&25.678&27.836&29.054&30.894
		  &34.336\\
		40&17.916&20.707&22.164&24.433&26.509&29.051&30.008&32.345&33.660&35.643
		  &39.335\\
		45&21.251&24.311&25.901&28.366&30.612&33.350&34.379&36.884&38.291&40.407
		  &44.335\\
		50&24.674&27.991&29.707&32.357&34.764&37.689&38.785&41.449&42.942&45.184
		  &49.335\\
		55&28.173&31.735&33.570&36.398&38.958&42.060&43.220&46.036&47.610&49.972
		  &54.335\\
		60&31.738&35.534&37.485&40.482&43.188&46.459&47.680&50.641&52.294&54.770
		  &59.335
		\end{tabular}
		\end{center}
		
		\newpage
		
		\begin{center}
		\begin{tabular}
		      {r@{\ }r@{\ }r@{\ }r@{\ }r@{\ }r@{\ }r@{\ }r@{\ }r@{\ }r@{\ }r@{\ }r}
		\multicolumn{12}{c}{DISTRIBUTION CUMULATIVE DE LA LOI DU KHI-DEUX}\\
		\ \\
		$k$&60.0\%&66.7\%&75.0\%&80.0\%&87.5\%&90.0\%&95.0\%&97.5\%&99.0\%&99.5\%
		     &99.9\%\\
		\ \\
		1&0.708&0.936&1.323&1.642&2.354&2.706&3.841&5.024&6.635&7.879&10.828\\
		2&1.833&2.197&2.773&3.219&4.159&4.605&5.991&7.378&9.210&10.597&13.816\\
		3&2.946&3.405&4.108&4.642&5.739&6.251&7.815&9.348&11.345&12.838&16.266\\
		4&4.045&4.579&5.385&5.989&7.214&7.779&9.488&11.143&13.277&14.860&18.467\\
		5&5.132&5.730&6.626&7.289&8.625&9.236&11.070&12.833&15.086&16.750&20.515\\
		6&6.211&6.867&7.841&8.558&9.992&10.645&12.592&14.449&16.812&18.548&22.458\\
		7&7.283&7.992&9.037&9.803&11.326&12.017&14.067&16.013&18.475&20.278&24.322\\
		8&8.351&9.107&10.219&11.030&12.636&13.362&15.507&17.535&20.090&21.955&26.125\\
		9&9.414&10.215&11.389&12.242&13.926&14.684&16.919&19.023&21.666&23.589
		 &27.877\\
		10&10.473&11.317&12.549&13.442&15.198&15.987&18.307&20.483&23.209&25.188
		  &29.588\\
		11&11.530&12.414&13.701&14.631&16.457&17.275&19.675&21.920&24.725&26.757
		  &31.264\\
		12&12.584&13.506&14.845&15.812&17.703&18.549&21.026&23.337&26.217&28.300
		  &32.910\\
		13&13.636&14.595&15.984&16.985&18.939&19.812&22.362&24.736&27.688&29.819
		  &34.528\\
		14&14.685&15.680&17.117&18.151&20.166&21.064&23.685&26.119&29.141&31.319
		  &36.123\\
		15&15.733&16.761&18.245&19.311&21.384&22.307&24.996&27.488&30.578&32.801
		  &37.697\\
		16&16.780&17.840&19.369&20.465&22.595&23.542&26.296&28.845&32.000&34.267
		  &39.252\\
		17&17.824&18.917&20.489&21.615&23.799&24.769&27.587&30.191&33.409&35.718
		  &40.790\\
		18&18.868&19.991&21.605&22.760&24.997&25.989&28.869&31.526&34.805&37.156
		  &42.312\\
		19&19.910&21.063&22.718&23.900&26.189&27.204&30.144&32.852&36.191&38.582
		  &43.820\\
		20&20.951&22.133&23.828&25.038&27.376&28.412&31.410&34.170&37.566&39.997
		  &45.315\\
		21&21.991&23.201&24.935&26.171&28.559&29.615&32.671&35.479&38.932&41.401
		  &46.797\\
		22&23.031&24.268&26.039&27.301&29.737&30.813&33.924&36.781&40.289&42.796
		  &48.268\\
		23&24.069&25.333&27.141&28.429&30.911&32.007&35.172&38.076&41.638&44.181
		  &49.728\\
		24&25.106&26.397&28.241&29.553&32.081&33.196&36.415&39.364&42.980&45.559
		  &51.179\\
		25&26.143&27.459&29.339&30.675&33.247&34.382&37.652&40.646&44.314&46.928
		  &52.620\\
		26&27.179&28.520&30.435&31.795&34.410&35.563&38.885&41.923&45.642&48.290
		  &54.052\\
		27&28.214&29.580&31.528&32.912&35.570&36.741&40.113&43.195&46.963&49.645
		  &55.476\\
		28&29.249&30.639&32.620&34.027&36.727&37.916&41.337&44.461&48.278&50.993
		  &56.892\\
		29&30.283&31.697&33.711&35.139&37.881&39.087&42.557&45.722&49.588&52.336
		  &58.301\\
		30&31.316&32.754&34.800&36.250&39.033&40.256&43.773&46.979&50.892&53.672
		  &59.703\\
		35&36.475&38.024&40.223&41.778&44.753&46.059&49.802&53.203&57.342&60.275
		  &66.619\\
		40&41.622&43.275&45.616&47.269&50.424&51.805&55.758&59.342&63.691&66.766
		  &73.402\\
		45&46.761&48.510&50.985&52.729&56.052&57.505&61.656&65.410&69.957&73.166
		  &80.077\\
		50&51.892&53.733&56.334&58.164&61.647&63.167&67.505&71.420&76.154&79.490
		  &86.661\\
		55&57.016&58.945&61.665&63.577&67.211&68.796&73.311&77.380&82.292&85.749
		  &93.168\\
		60&62.135&64.147&66.981&68.972&72.751&74.397&79.082&83.298&88.379&91.952
		  &99.607
		\end{tabular}
	\end{center} 
	
	\paragraph{Loi du khi-deux non centrée}\label{noncentral chi-square distribution}\mbox{}\\\\
	La "\NewTerm{loi du khi-deux non centrée}\index{loi du khi-deux non centrée}" peut être trouvée dans deux situations majeures dans les statistiques industrielles (et donc dans les monde des affaires):
	\begin{itemize}
		\item Dans les estimations inférieures et supérieures de l'intervalle de tolérance pour une distribution normale univariée.
		
		\item Comme la distribution de l'hypothèse alternative ($H_1$) de nombreux tests NHST (Null Hypothesis Tests) et donc utilisée pour calculer la puissance des tests correspondants.
	\end{itemize}
	Déterminons maintenant la fonction de probabilité de densité de la distribution du khi-deux non centrée.
	
	Nous savons que la fonction de probabilité de densité de $X^2=\mathcal{N}(0,1)^2$ suit par définition une distribution $\chi^2$ à degré de liberté. La question est maintenant de savoir quelle distribution suit $\mathcal{N}(\lambda,1)^2$ ??

	D'abord, nous savons que nous pouvons réécrire ceci comme:
	
	Dès lors, $T$ a la distribution $\Phi(t)$ et la fonction de densité (voir plus tôt notre étude de la distribution Normale):
	
	La distribution de $X^2$ est évidemment:
	
	La densité est donc:
	
	Et nous connaissons (\SeeChapter{voir section Suites et Séries page \pageref{euler maclaurin expansion}}) la série de Macluarin suivante:
	
	Dès lors:
	
	Rappelons maintenant que si $K$ est une variable aléatoire avec une distribution de Poisson et espérance $\lambda^2/2 $, alors sa distribution est égale à:
	
	avec $k=0,1,2,\ldots$.
	
	Dès lors:
	
	Rappelons maintenant que la distribution du chi carré est donnée par:
	
	Alors:
	
	Ainsi:
	
	Dès lors:
	
	Rappelons maintenant que nous avons prouvé dans la section Calcul différentiel et intégral (voir page \pageref{gamma euler function}) que:
	
	Dès lors:
	
	Donc finalement la fonction de distribution du khi-deux non centrée avec un degré de liberté est donnée par:
	
	Ou explicitement:
	
	\begin{tcolorbox}[title=Remarque,colframe=black,arc=10pt]
	Il peut être prouvé (mais jusqu'à présent, nous n'avons besoin de ce résultat nulle part dans ce livre), que le cas général avec $ n $ degrés de liberté est donné par:
	
	\end{tcolorbox}
	C'est tout ce dont nous avons besoin pour notre étude ultérieure des intervalles de tolérance! Nous ne déduirons donc pas la moyenne et la variance de cette distribution car elle serait inutile dans l'état actuel de ce livre!
	
	\subsubsection{Distribution de Student}\label{student distribution}
	La "\NewTerm{distribution de Student}\index{distribution de Student}" ou "\NewTerm{loi de Student}\index{loi de Student}") de paramètre $k$ est définie par la relation:
	
	avec $k$ étant le degré de liberté de la loi du $\chi^2$ sous-jacente à la construction de la fonction de Student comme nous allons le voir.
	
	Indiquons qu'elle peut aussi être obtenue dans la version français de Microsoft Excel 11.8346 à l'aide des fonctions \texttt{LOI.STUDENT( )} et sa réciproque par \texttt{LOI.STUDENT.INVERSE( )}.

	Il s'agit bien d'une fonction de distribution car elle vérifie bien:
	
	Voyons la démonstration la plus simple pour justifier la provenance de la loi de Student et qui nous sera en même temps très utile dans l'inférence statistique et l'analyse de la variance plus loin.
	
	Pour cette démonstration, rappelons que:
	\begin{enumerate}
		\item Si $X$, $Y$ sont deux variables aléatoires indépendantes de densités respectives $f_X,f_Y$, la loi du couple $(X, Y)$ possède une densité $f$ vérifiant (axiome des probabilités!):
		
		
		\item La distribution $\mathcal{N}(0,1)$ est donnée par (voir plus haut):
		
		
		\item La distribution du $\chi_n^2$ est donnée par (voir précédemment):
		
		pour $y\geq 0$ et $n \geq 1$.
		
		\item La fonction Gamma d'Euler $\Gamma$ est définie pour tout $\alpha>0$ par (\SeeChapter{voir section de Calcul Différentiel et Intégral page \pageref{gamma euler function}}):
		
		et vérifie (\SeeChapter{voir section de Calcul Différentiel et Intégral page \pageref{gamma euler function}}):
		
		pour $\alpha\geq 2$.
	\end{enumerate}
	Ces rappels étant faits, considérons maintenant $X$ une variable aléatoire suivant la loi $\mathcal{N}(0,1)$ et $Y$ une variable aléatoire suivant la loi $\chi_n^2$.
	
	Nous supposons $X$ et $Y$ indépendantes et nous considérons la variable aléatoire (c'est à l'origine l'étude historique de la loi de Student dans le cadre de l'inférence statistique qui a amené à poser cette variable dont nous justifierons l'origine plus loin):	
	
	Nous allons démontrer que $T$ suit une loi de Student de paramètre $n$.
	\begin{dem}
	Notons $F$ et $f$ les fonctions de répartition et de densité de $T$ et $f_X,f_Y$ les fonctions de densité de $X, Y$ et $(X, Y)$ respectivement. Nous avons alors pour tout $t \in \mathbb{R}$:
	
	où: 
	
	la valeur imposée positive et non nulle de y étant due au fait qu'elle est sous une racine et en plus au dénominateur.
	
	Ainsi:
	
	où comme $X$ suit une loi Normale $\mathcal{N}(0,1)$:
	
	est la fonction de répartition de la loi Normale centrée réduite.
	
	Nous obtenons alors la fonction de densité de $T$ en dérivant $F$:
	
	car (la dérivée d'une fonction est égale à sa dérivée multipliée par sa dérivée intérieure):
	
	Donc:
	
	En faisant le changement de variable:
	
	nous obtenons:
	
	ce qui est bien la loi de Student de paramètre $n$.
	\begin{flushright}
		$\blacksquare$  Q.E.D.
	\end{flushright}
	\end{dem}
	Voyons maintenant quelle est l'espérance de la loi de Student:
		
	Nous avons:
	
	Mais $\text{E}\left(\dfrac{1}{\sqrt{Y}}\right)$ existe si et seulement si $n\geq 2$. En effet pour $n=1$:
	
	et:
	
	Tandis que pour $n\geq 2$ nous avons:
	
	Ainsi pour $n=1$, l'espérance n'existe pas.
	
	Donc pour $n\geq 2$:
	
	Voyons maintenant la valeur de la variance. Nous avons donc:
	
	Discutons de l'existence de $\text{E}(T^2)$. Nous avons trivialement:
	
	$X$ suit une loi Normale centrée réduite donc:
	
	Pour ce qui est de $\text{E}\left(\dfrac{1}{Y}\right)$ nous avons:
	
	où nous avons fait le changement de variable $u=y/2$.
	
	Mais l'intégrale définissant $\Gamma\left(\dfrac{n}{2}-1\right)$ converge seulement si $n\geq 3$.
	
	Donc $\text{E}(T^2)$ existe si et seulement si $n\geq 3$ et vaut alors selon les propriétés de la fonction Gamma d'Euler démontrées dans le chapitre de Calcul Différentiel Et Intégral:
	
	Ainsi pour $n\geq 3$:
	
	Il est par ailleurs important de remarquer que cette loi est symétrique par rapport à $0$!

	Exemple de tracé de la fonction de distribution et répartition pour la fonction de Student de paramètre $k=3$ (pour plus de plots sur cette distribution voir plus bas ci-dessous à la page \pageref{continuous distributions}):
	\begin{figure}[H]
		\centering
		\includegraphics{img/arithmetics/law_student.jpg}
		\caption{Loi $T$ de Student (fonction de densité et distribution cumulative)}
	\end{figure}
	
	\newpage
	Enfin pour les personnes n'ayant pas accès à un tableur ou à un logiciel statistique:
	\begin{center}
		\begin{tabular}
		      {r@{\quad}r@{\,}r@{\,}r@{\,}r@{\,}r@{\,}r@{\,}r@{\,}r@{\,}r@{\,}r@{\,}r}
		\multicolumn{12}{c}{DISTRIBUTION CUMULATIVE DE LA LOI $T$ DE STUDENT} \\
		\ \\
		$k$&60.0\%&66.7\%&75.0\%&80.0\%&87.5\%&90.0\%&95.0\%&97.5\%&99.0\%&99.5\%
		     &99.9\% \\
		\ \\
		 1&0.325&0.577&1.000&1.376&2.414&3.078&6.314&12.706&31.821&63.657&318.31 \\
		 2&0.289&0.500&0.816&1.061&1.604&1.886&2.920&4.303&6.965&9.925&22.327 \\
		 3&0.277&0.476&0.765&0.978&1.423&1.638&2.353&3.182&4.541&5.841&10.215 \\
		 4&0.271&0.464&0.741&0.941&1.344&1.533&2.132&2.776&3.747&4.604&7.173 \\
		 5&0.267&0.457&0.727&0.920&1.301&1.476&2.015&2.571&3.365&4.032&5.893 \\
		 6&0.265&0.453&0.718&0.906&1.273&1.440&1.943&2.447&3.143&3.707&5.208 \\
		 7&0.263&0.449&0.711&0.896&1.254&1.415&1.895&2.365&2.998&3.499&4.785 \\
		 8&0.262&0.447&0.706&0.889&1.240&1.397&1.860&2.306&2.896&3.355&4.501 \\
		 9&0.261&0.445&0.703&0.883&1.230&1.383&1.833&2.262&2.821&3.250&4.297 \\
		10&0.260&0.444&0.700&0.879&1.221&1.372&1.812&2.228&2.764&3.169&4.144 \\
		11&0.260&0.443&0.697&0.876&1.214&1.363&1.796&2.201&2.718&3.106&4.025 \\
		12&0.259&0.442&0.695&0.873&1.209&1.356&1.782&2.179&2.681&3.055&3.930 \\
		13&0.259&0.441&0.694&0.870&1.204&1.350&1.771&2.160&2.650&3.012&3.852 \\
		14&0.258&0.440&0.692&0.868&1.200&1.345&1.761&2.145&2.624&2.977&3.787 \\
		15&0.258&0.439&0.691&0.866&1.197&1.341&1.753&2.131&2.602&2.947&3.733 \\
		16&0.258&0.439&0.690&0.865&1.194&1.337&1.746&2.120&2.583&2.921&3.686 \\
		17&0.257&0.438&0.689&0.863&1.191&1.333&1.740&2.110&2.567&2.898&3.646 \\
		18&0.257&0.438&0.688&0.862&1.189&1.330&1.734&2.101&2.552&2.878&3.610 \\
		19&0.257&0.438&0.688&0.861&1.187&1.328&1.729&2.093&2.539&2.861&3.579 \\
		20&0.257&0.437&0.687&0.860&1.185&1.325&1.725&2.086&2.528&2.845&3.552 \\
		21&0.257&0.437&0.686&0.859&1.183&1.323&1.721&2.080&2.518&2.831&3.527 \\
		22&0.256&0.437&0.686&0.858&1.182&1.321&1.717&2.074&2.508&2.819&3.505 \\
		23&0.256&0.436&0.685&0.858&1.180&1.319&1.714&2.069&2.500&2.807&3.485 \\
		24&0.256&0.436&0.685&0.857&1.179&1.318&1.711&2.064&2.492&2.797&3.467 \\
		25&0.256&0.436&0.684&0.856&1.178&1.316&1.708&2.060&2.485&2.787&3.450 \\
		26&0.256&0.436&0.684&0.856&1.177&1.315&1.706&2.056&2.479&2.779&3.435 \\
		27&0.256&0.435&0.684&0.855&1.176&1.314&1.703&2.052&2.473&2.771&3.421 \\
		28&0.256&0.435&0.683&0.855&1.175&1.313&1.701&2.048&2.467&2.763&3.408 \\
		29&0.256&0.435&0.683&0.854&1.174&1.311&1.699&2.045&2.462&2.756&3.396 \\
		30&0.256&0.435&0.683&0.854&1.173&1.310&1.697&2.042&2.457&2.750&3.385 \\
		35&0.255&0.434&0.682&0.852&1.170&1.306&1.690&2.030&2.438&2.724&3.340 \\
		40&0.255&0.434&0.681&0.851&1.167&1.303&1.684&2.021&2.423&2.704&3.307 \\
		45&0.255&0.434&0.680&0.850&1.165&1.301&1.679&2.014&2.412&2.690&3.281 \\
		50&0.255&0.433&0.679&0.849&1.164&1.299&1.676&2.009&2.403&2.678&3.261 \\
		55&0.255&0.433&0.679&0.848&1.163&1.297&1.673&2.004&2.396&2.668&3.245 \\
		60&0.254&0.433&0.679&0.848&1.162&1.296&1.671&2.000&2.390&2.660&3.232 \\
		$\infty$
		  &0.253&0.431&0.674&0.842&1.150&1.282&1.645&1.960&2.326&2.576&3.090
		\end{tabular}
	\end{center}
	
	\subsubsection{Distribution de Fisher}
	La "\NewTerm{fonction de Fisher}\index{fonction de Fisher}" (ou "\NewTerm{loi de Fisher-Snedecor}\index{loi de Fisher-Snedecor}") de paramètres $k$ et $l$ est définie par la relation:
	
	si $x\geq 0$ (voir le graphique ci-dessous à la page \pageref{continuous distributions}). Les paramètres $k$ et $l$ sont des entiers positifs et correspondent aux degrés de liberté des deux lois du Khi-deux sous-jacentes. Cette distribution est souvent notée equation ou $F_{k,l}$ ou $F(k, l)$ et peut être obtenue dans la version française de Microsoft Excel 11.8346 par la fonction \texttt{LOI.F( )}.
	
	Il s'agit bien d'une fonction de distribution car elle vérifie également (reste à démontrer directement mais bon comme nous allons le voir elle est le produit de deux fonctions de distribution donc indirectement...):
	
	Voyons la démonstration la plus simple pour justifier la provenance de la loi de Fisher et qui nous sera en même temps très utile dans l'inférence statistique et l'analyse de la variance plus loin.

	Pour cette démonstration, rappelons que:
	\begin{enumerate}
		\item  La loi $\chi_n^2$ est donnée par (voir plus haut):
		
		pour $y\geq 0$ et $n\geq 1$.
		
		\item La fonction $\Gamma$ est définie pour tout $\alpha>0$ par (\SeeChapter{voir section de Calucl Différentiel et Intégral page \pageref{gamma euler function}}):
		
		Soient $X, Y$ deux variables aléatoires indépendantes suivant respectivement les lois $\chi_n^2$ et $\chi_m^2$.
	\end{enumerate}
	Nous considérons la variable aléatoire:
	
	Nous allons donc montrer que la loi de $T$ est la loi de Fisher-Snedecor de paramètres $n, m$.
	
	Notons pour cela $F$ et $f$ les fonctions cumulative et de densité de $T$ et $f_X,f_Y,f$, les fonctions de densité de $X, Y$ et $(X,Y)$ respectivement.  Nous avons pour tout $t\in \mathbb{R}$:
	
	où:
	
	où les valeurs positives imposées proviennent à l'origine d'une loi du Khi-deux pour $x$ et $y$.
	
	Ainsi:
	
	Nous obtenons la fonction de densité de $T$ en dérivant $F$. D'abord la dérivée intérieure:
	
	Ensuite en explicitant puisque:
	
	nous avons alors:
	
	En faisant le changement de variable:
	
	nous obtenons:
	
	
	\pagebreak
	Pour les personnes qui n'ont pas accès à un tableur ou à un logiciel statistique, voici quelques tableaux utiles:
	\begin{center}
		\begin{tabular}{rrr@{\,}r@{\,}r@{\,}r@{\,}r@{\,}r@{\,}r@{\,}r
		                   @{\,}r@{\,}r@{\,}r@{\,}r@{\,}r@{\,}r@{\,}r}
		&&\multicolumn{14}{c}{POURCENTAGE DE POINTS DE LA DISTRIBUTION $F$}\\
		\ \\
		$\nu_2\backslash\nu_1$ & & 
		\multicolumn{1}{c}{2} &\multicolumn{1}{c}{3} &
		\multicolumn{1}{c}{4} &\multicolumn{1}{c}{5} &
		\multicolumn{1}{c}{6} &\multicolumn{1}{c}{7} &
		\multicolumn{1}{c}{8} &\multicolumn{1}{c}{10}&
		\multicolumn{1}{c}{12}&\multicolumn{1}{c}{15}&
		\multicolumn{1}{c}{20}&\multicolumn{1}{c}{30}&
		\multicolumn{1}{c}{50}&\multicolumn{1}{c}{$\infty$}\\
		& $q$ \\
		1&0.900&49.5&53.6&55.8&57.2&58.2&59.1&59.7&60.5&61.0&61.5&62.0&62.6&63.0&63.3\\
		 &0.950&199.&216.&225.&230.&234.&237.&239.&242.&244.&246.&248.&250.&252.&254.\\
		 &0.975&800.&864.&900.&922.&937.&948.&957.&969.&977.&985.&993.\\
		 &0.990\\
		 &0.999\\
		2&0.900&9.00&9.16&9.24&9.29&9.33&9.35&9.37&9.39&9.41&9.43&9.44&9.46&9.47&9.49\\ 
		 &0.950&19.0&19.2&19.2&19.3&19.3&19.4&19.4&19.4&19.4&19.4&19.4&19.5&19.5&19.5\\
		 &0.975&39.0&39.2&39.2&39.3&39.3&39.4&39.4&39.4&39.4&39.4&39.4&39.5&39.5&39.5\\
		 &0.990&99.0&99.2&99.2&99.3&99.3&99.4&100.&100.&100.&100.&100.&100.&100.&99.5\\
		 &0.999&999.&999.\\
		3&0.900&5.46&5.39&5.34&5.31&5.28&5.27&5.25&5.23&5.22&5.20&5.18&5.17&5.15&5.13\\
		 &0.950&9.55&9.28&9.12&9.01&8.94&8.89&8.85&8.79&8.74&8.70&8.66&8.62&8.58&8.53\\
		 &0.975&16.0&15.4&15.1&14.9&14.7&14.6&14.5&14.4&14.3&14.3&14.2&14.1&14.0&13.9\\
		 &0.990&30.8&29.5&28.7&28.2&27.9&27.7&27.5&27.2&27.1&26.9&26.7&26.5&26.4&26.1\\
		 &0.999&149.&141.&137.&135.&133.&132.&131.&129.&128.&127.&126.&125.&125.&123.\\
		4&0.900&4.32&4.19&4.11&4.05&4.01&3.98&3.95&3.92&3.90&3.87&3.84&3.82&3.79&3.76\\
		 &0.950&6.94&6.59&6.39&6.26&6.16&6.09&6.04&5.96&5.91&5.86&5.80&5.75&5.70&5.63\\
		 &0.975&10.6&9.98&9.60&9.36&9.20&9.07&8.98&8.84&8.75&8.66&8.56&8.46&8.38&8.26\\
		 &0.990&18.0&16.7&16.0&15.5&15.2&15.0&14.8&14.5&14.4&14.2&14.0&13.8&13.7&13.5\\
		 &0.999&61.2&56.2&53.4&51.7&50.5&49.7&49.0&48.0&47.4&46.8&46.1&45.4&44.9&44.1\\
		5&0.900&3.78&3.62&3.52&3.45&3.40&3.37&3.34&3.30&3.27&3.24&3.21&3.17&3.15&3.10\\
		 &0.950&5.79&5.41&5.19&5.05&4.95&4.88&4.82&4.74&4.68&4.62&4.56&4.50&4.44&4.36\\
		 &0.975&8.43&7.76&7.39&7.15&6.98&6.85&6.76&6.62&6.52&6.43&6.33&6.23&6.14&6.02\\
		 &0.990&13.3&12.1&11.4&11.0&10.7&10.5&10.3&10.1&9.89&9.72&9.55&9.38&9.24&9.02\\
		 &0.999&37.1&33.2&31.1&29.8&28.8&28.2&27.6&26.9&26.4&25.9&25.4&24.9&24.4&23.8\\
		6&0.900&3.46&3.29&3.18&3.11&3.05&3.01&2.98&2.94&2.90&2.87&2.84&2.80&2.77&2.72\\
		 &0.950&5.14&4.76&4.53&4.39&4.28&4.21&4.15&4.06&4.00&3.94&3.87&3.81&3.75&3.67\\
		 &0.975&7.26&6.60&6.23&5.99&5.82&5.70&5.60&5.46&5.37&5.27&5.17&5.07&4.98&4.85\\
		 &0.990&10.9&9.78&9.15&8.75&8.47&8.26&8.10&7.87&7.72&7.56&7.40&7.23&7.09&6.88\\
		 &0.999&27.0&23.7&21.9&20.8&20.0&19.5&19.0&18.4&18.0&17.6&17.1&16.7&16.3&15.7\\
		7&0.900&3.26&3.07&2.96&2.88&2.83&2.78&2.75&2.70&2.67&2.63&2.59&2.56&2.52&2.47\\
		 &0.950&4.74&4.35&4.12&3.97&3.87&3.79&3.73&3.64&3.57&3.51&3.44&3.38&3.32&3.23\\
		 &0.975&6.54&5.89&5.52&5.29&5.12&4.99&4.90&4.76&4.67&4.57&4.47&4.36&4.28&4.14\\
		 &0.990&9.55&8.45&7.85&7.46&7.19&6.99&6.84&6.62&6.47&6.31&6.16&5.99&5.86&5.65\\
		 &0.999&21.7&18.8&17.2&16.2&15.5&15.0&14.6&14.1&13.7&13.3&12.9&12.5&12.2&11.7\\
		8&0.900&3.11&2.92&2.81&2.73&2.67&2.62&2.59&2.54&2.50&2.46&2.42&2.38&2.35&2.29\\
		 &0.950&4.46&4.07&3.84&3.69&3.58&3.50&3.44&3.35&3.28&3.22&3.15&3.08&3.02&2.93\\
		 &0.975&6.06&5.42&5.05&4.82&4.65&4.53&4.43&4.29&4.20&4.10&4.00&3.89&3.81&3.67\\
		 &0.990&8.65&7.59&7.01&6.63&6.37&6.18&6.03&5.81&5.67&5.52&5.36&5.20&5.07&4.86\\
		 &0.999&18.5&15.8&14.4&13.5&12.9&12.4&12.0&11.5&11.2&10.8&10.5&10.1&9.80&9.33
		\end{tabular}
		\end{center}
		
		\newpage
		
		\begin{center}
		\begin{tabular}{rrr@{\,}r@{\,}r@{\,}r@{\,}r@{\,}r@{\,}r@{\,}r
		                   @{\,}r@{\,}r@{\,}r@{\,}r@{\,}r@{\,}r@{\,}r}
		&&\multicolumn{14}{c}{POURCENTAGE DE POINTS DE LA DISTRIBUTION $F$}\\
		\ \\
		$\nu_2\backslash\nu_1$ & & 
		\multicolumn{1}{c}{2} &\multicolumn{1}{c}{3} &
		\multicolumn{1}{c}{4} &\multicolumn{1}{c}{5} &
		\multicolumn{1}{c}{6} &\multicolumn{1}{c}{7} &
		\multicolumn{1}{c}{8} &\multicolumn{1}{c}{10}&
		\multicolumn{1}{c}{12}&\multicolumn{1}{c}{15}&
		\multicolumn{1}{c}{20}&\multicolumn{1}{c}{30}&
		\multicolumn{1}{c}{50}&\multicolumn{1}{c}{$\infty$}\\
		& $q$ \\
		 9&0.900&3.01&2.81&2.69&2.61&2.55&2.51&2.47&2.42&2.38&2.34&2.30&2.25&2.22&2.16\\
		  &0.950&4.26&3.86&3.63&3.48&3.37&3.29&3.23&3.14&3.07&3.01&2.94&2.86&2.80&2.71\\
		  &0.975&5.71&5.08&4.72&4.48&4.32&4.20&4.10&3.96&3.87&3.77&3.67&3.56&3.47&3.33\\
		  &0.990&8.02&6.99&6.42&6.06&5.80&5.61&5.47&5.26&5.11&4.96&4.81&4.65&4.52&4.31\\
		  &0.999&16.4&13.9&12.6&11.7&11.1&10.7&10.4&9.89&9.57&9.24&8.90&8.55&8.26&7.81\\
		10&0.900&2.92&2.73&2.61&2.52&2.46&2.41&2.38&2.32&2.28&2.24&2.20&2.16&2.12&2.06\\
		  &0.950&4.10&3.71&3.48&3.33&3.22&3.14&3.07&2.98&2.91&2.84&2.77&2.70&2.64&2.54\\
		  &0.975&5.46&4.83&4.47&4.24&4.07&3.95&3.85&3.72&3.62&3.52&3.42&3.31&3.22&3.08\\
		  &0.990&7.56&6.55&5.99&5.64&5.39&5.20&5.06&4.85&4.71&4.56&4.41&4.25&4.11&3.91\\
		  &0.999&14.9&12.6&11.3&10.5&9.93&9.52&9.20&8.75&8.45&8.13&7.80&7.47&7.19&6.76\\
		11&0.900&2.86&2.66&2.54&2.45&2.39&2.34&2.30&2.25&2.21&2.17&2.12&2.08&2.04&1.97\\
		  &0.950&3.98&3.59&3.36&3.20&3.09&3.01&2.95&2.85&2.79&2.72&2.65&2.57&2.51&2.40\\
		  &0.975&5.26&4.63&4.28&4.04&3.88&3.76&3.66&3.53&3.43&3.33&3.23&3.12&3.03&2.88\\
		  &0.990&7.21&6.22&5.67&5.32&5.07&4.89&4.74&4.54&4.40&4.25&4.10&3.94&3.81&3.60\\
		  &0.999&13.8&11.6&10.3&9.58&9.05&8.66&8.35&7.92&7.63&7.32&7.01&6.68&6.42&6.00\\
		12&0.900&2.81&2.61&2.48&2.39&2.33&2.28&2.24&2.19&2.15&2.10&2.06&2.01&1.97&1.90\\
		  &0.950&3.89&3.49&3.26&3.11&3.00&2.91&2.85&2.75&2.69&2.62&2.54&2.47&2.40&2.30\\
		  &0.975&5.10&4.47&4.12&3.89&3.73&3.61&3.51&3.37&3.28&3.18&3.07&2.96&2.87&2.72\\
		  &0.990&6.93&5.95&5.41&5.06&4.82&4.64&4.50&4.30&4.16&4.01&3.86&3.70&3.57&3.36\\
		  &0.999&13.0&10.8&9.63&8.89&8.38&8.00&7.71&7.29&7.00&6.71&6.40&6.09&5.83&5.42\\
		13&0.900&2.76&2.56&2.43&2.35&2.28&2.23&2.20&2.14&2.10&2.05&2.01&1.96&1.92&1.85\\
		  &0.950&3.81&3.41&3.18&3.03&2.92&2.83&2.77&2.67&2.60&2.53&2.46&2.38&2.31&2.21\\
		  &0.975&4.97&4.35&4.00&3.77&3.60&3.48&3.39&3.25&3.15&3.05&2.95&2.84&2.74&2.60\\
		  &0.990&6.70&5.74&5.21&4.86&4.62&4.44&4.30&4.10&3.96&3.82&3.66&3.51&3.37&3.17\\
		  &0.999&12.3&10.2&9.07&8.35&7.86&7.49&7.21&6.80&6.52&6.23&5.93&5.63&5.37&4.97\\
		14&0.900&2.73&2.52&2.39&2.31&2.24&2.19&2.15&2.10&2.05&2.01&1.96&1.91&1.87&1.80\\
		  &0.950&3.74&3.34&3.11&2.96&2.85&2.76&2.70&2.60&2.53&2.46&2.39&2.31&2.24&2.13\\
		  &0.975&4.86&4.24&3.89&3.66&3.50&3.38&3.29&3.15&3.05&2.95&2.84&2.73&2.64&2.49\\
		  &0.990&6.51&5.56&5.04&4.69&4.46&4.28&4.14&3.94&3.80&3.66&3.51&3.35&3.22&3.00\\
		  &0.999&11.8&9.73&8.62&7.92&7.44&7.08&6.80&6.40&6.13&5.85&5.56&5.25&5.00&4.60\\
		15&0.900&2.70&2.49&2.36&2.27&2.21&2.16&2.12&2.06&2.02&1.97&1.92&1.87&1.83&1.76\\
		  &0.950&3.68&3.29&3.06&2.90&2.79&2.71&2.64&2.54&2.48&2.40&2.33&2.25&2.18&2.07\\
		  &0.975&4.77&4.15&3.80&3.58&3.41&3.29&3.20&3.06&2.96&2.86&2.76&2.64&2.55&2.40\\
		  &0.990&6.36&5.42&4.89&4.56&4.32&4.14&4.00&3.80&3.67&3.52&3.37&3.21&3.08&2.87\\
		  &0.999&11.3&9.34&8.25&7.57&7.09&6.74&6.47&6.08&5.81&5.53&5.25&4.95&4.70&4.31\\
		16&0.900&2.67&2.46&2.33&2.24&2.18&2.13&2.09&2.03&1.99&1.94&1.89&1.84&1.79&1.72\\
		  &0.950&3.63&3.24&3.01&2.85&2.74&2.66&2.59&2.49&2.42&2.35&2.28&2.19&2.12&2.01\\
		  &0.975&4.69&4.08&3.73&3.50&3.34&3.22&3.12&2.99&2.89&2.79&2.68&2.57&2.47&2.32\\
		  &0.990&6.23&5.29&4.77&4.44&4.20&4.03&3.89&3.69&3.55&3.41&3.26&3.10&2.97&2.75\\
		  &0.999&11.0&9.01&7.94&7.27&6.80&6.46&6.19&5.81&5.55&5.27&4.99&4.70&4.45&4.06\\
		17&0.900&2.64&2.44&2.31&2.22&2.15&2.10&2.06&2.00&1.96&1.91&1.86&1.81&1.76&1.69\\
		  &0.950&3.59&3.20&2.96&2.81&2.70&2.61&2.55&2.45&2.38&2.31&2.23&2.15&2.08&1.96\\
		  &0.975&4.62&4.01&3.66&3.44&3.28&3.16&3.06&2.92&2.82&2.72&2.62&2.50&2.41&2.25\\
		  &0.990&6.11&5.18&4.67&4.34&4.10&3.93&3.79&3.59&3.46&3.31&3.16&3.00&2.87&2.65\\
		  &0.999&10.7&8.73&7.68&7.02&6.56&6.22&5.96&5.58&5.32&5.05&4.77&4.48&4.24&3.85
		\end{tabular}
	\end{center}
	
	\subsubsection{Loi Normale repliée généralisée}
	La  "\NewTerm{distribution Normale repliée}\index{distribution normale repliée}" est la distribution de la valeur absolue d'une variable aléatoire avec une distribution Normale \footnote{La majorité du texte ci-dessous provient de \url{http://www.math.uah.edu/stat/}}. Comme nous l'avons mentionné précédemment, la distribution Normale est peut-être la plus importante en probabilités et est utilisée pour modéliser une incroyable variété de phénomènes aléatoires. Étant donné que l'on ne peut s'intéresser qu'à l'ampleur d'une variable Normalement distribuée, la Normale repliée apparaît de manière très naturelle, en particulier en Finance et en Génie Industriel pour la conception des Plans d'Expériences (\SeeChapter{voir section de Génie Industriel \pageref{doe}}) . Le nom vient du fait que la mesure de probabilité de la distribution Normale sur $]- \ infty, 0] $ est repliée sur $ [0, +\infty[$.
	
	\textbf{Définition (\#\mydef):} Supposons que $X$ ait une distribution Normale avec  moyenne $\mu \in \mathbb{R}$ et écart-type $\sigma \in [0, + \ infty[$ . Alors $ Y = | X | $ a la distribution Normale repliée avec les paramètres $\mu$ et $\sigma$.
	
	Supposons que $Z$ suit la distribution Normale centrée réduite. Rappelons que $ Z $ a alors une fonction de densité de probabilité notée $\phi$ et une fonction de distribution notée $\Phi$ données par:
	
	avec $z \in \mathbb{R}$.
	
	Si $\mu\in\mathbb{R}$ et $\sigma\in [0,+\infty[$, alors $X=\mu +\sigma Z$ a une distribution Normale avec une moyenne $\mu$ et écart-type $\sigma$, et il est donc évident à ce niveau que:
	 
	suit la distribution Normale repliée avec les paramètres $\mu$ et $\sigma$.
	
	Déterminons maintenant la fonction de probabilité cumulée d'une telle variable! Pour $y\in[0,+\infty[$:
	 
	Comme $\Phi(-Z)=1-\Phi(Z)$ nous avons:
	
	Nous ne pouvons pas calculer la fonction quantile $F^{-1}$ sous forme fermée, mais les valeurs de cette fonction peuvent être approximées.
	
	Il vient donc immédiatement que $ Y $ a une fonction de densité de probabilité $f$ donnée par:
	 
	Cela découle de la différenciation de la fonction de répartition cumulée par rapport à $y$ comme nous le savons!
	
	Maintenant, comme toujours dans ce livre, nous nous concentrerons uniquement sur ce dont nous avons besoin pour les applications des autres chapitres! Donc, comme nous n'avons pas besoin des moments du dossier Distribution normale, nous ne les calculerons pas. Le seul but du développement ci-dessus était de construire les outils pour pouvoir introduire un cas particulier du dossier Distribution normale.
	
	\paragraph{Distribution demi-Normale}\label{half normal distribution}\mbox{}\\\\
	En théorie des probabilités et statistiques, la "\NewTerm{distribution demi-Normale}\index{distribution demi-Normale}" est un cas particulier de la distribution normale repliée.

	Soit $X$ une variable aléatoire suivante une distribution Normale ordinaire, $\mathcal{N}(0, \sigma^{2})$, alors $ Y = | X | $ est dite suivre une distribution demi-normale. Ainsi, la distribution semi-normale est un pli à la moyenne d'une distribution Normale ordinaire avec une moyenne $\mu = 0$.
	
	Ainsi, soit $Y=|\sigma Z|=\sigma |Z|$ où $Z$ a une distribution Normale standard et $\sigma\in [0,+\infty[$. Il est clair que $\sigma$ est un paramètre d'échelle, contrairement au cas de la distribution Normale repliée généralisée. La distribution de $Y$ lorsque $\sigma=1$, $Y=|Z|$ suite une "\NewTerm{distribution demi-Normale standardisée}\index{distribution demi-Normale standardisée}".
	
	Comme la distribution demi-Normale n'est qu'un cas particulier de la distribution Normale repliée avec $\mu=0$, il vient immédiatement:
	 
	avec $y\in\mathbb{R}^+$ et cette distribution est parfois notée $\mathcal{HN}(0,\sigma^2)$.
	
	Or ce qui nous intéresse pour les autres chapitres de ce livre (notamment pour les Pland d'Expérience) sont les moments de ce dernier!
	
	Pour les calculer, rappelez-vous d'abord que:
	
	avec $z\in\mathbb{R}$. Il est donc immédiat que:
	
	Souvenez-vous également que nous avons déjà prouvé que:
	
	
	Nous devons d'abord déterminer les moments de la distribution Normale! Donc pour $n\in\mathbb{N}^+$:
	
	Maintenant, nous intégrons par parties (\SeeChapter{voir section de Calcul Différentiel et Intégral page \pageref{integration by parts}}), avec $u=z^n$ et $\mathrm{d}v=\phi'(z)\mathrm{d}z$ pour obtenir:
	
	Par conséquent, pour $n\in\mathbb{N}$, avec $n> 1$:
	
	Les moments de la distribution Normale standard sont maintenant faciles à calculer. Nous savons d'abord que:
	\begin{itemize}
		\item $\text{E}(Z)=0$
		\item $\text{E}(Z^2)=1$
	\end{itemize}
	Dès lors:
	\begin{itemize}
		\item $\text{E}(Z)=0$
		\item $\text{E}(Z^2)=1$
		\item $\text{E}(Z^3)=\text{E}(Z^{2+1})=2\cdot\text{E}(Z^{2-1})=2\cdot\text{E}(Z)=0$
		\item $\text{E}(Z^4)=\text{E}(Z^{3+1})=3\cdot\text{E}(Z^{3-1})=3\cdot\text{E}(Z^2)=3\cdot 1=1\cdot 3=3$
		\item $\text{E}(Z^5)=\text{E}(Z^{4+1})=4\cdot\text{E}(Z^{4-1})=4\cdot\text{E}(Z^3)=4\cdot 0=0$
		\item $\text{E}(Z^6)=\text{E}(Z^{5+1})=5\cdot\text{E}(Z^{5-1})=5\cdot\text{E}(Z^4)=5\cdot 3=1\cdot 3\cdot 5=15$
		\item $\text{E}(Z^7)=\text{E}(Z^{6+1})=6\cdot\text{E}(Z^{6-1})=6\cdot\text{E}(Z^5)=6\cdot 0=0$
		\item $\text{E}(Z^8)=\text{E}(Z^{7+1})=7\cdot\text{E}(Z^{7-1})=7\cdot\text{E}(Z^6)=7\cdot 15=1\cdot 3\cdot 5\cdot 7$
		\item $\ldots$
	\end{itemize}
	On voit donc que pour les puissances impaires, c'est-à-dire $Z^{2n + 1} $ avec $ n \in \mathbb{N}$ alors:
	
	et même pour les puissances paires:
	
	Il s'ensuit $X=\mathcal{N}(0,\sigma)$ (vous pouvez vérifier par vous-même avec $n=0$ et $n=1$):
	
	Les moments de la distribution demi-Normale peuvent désormais être calculés explicitement.
	
	Tout d'abord, il devrait être assez évident par construction que les moments d'ordre pair de $Y$ sont les mêmes que les moments d'ordre pair de $Z$ (dans les deux cas, les valeurs sont toutes positives et donc égales). Par conséquent:
	
	Pour les moments d'ordre impair, nous devons utiliser (voir ci-dessus):
	 
	avec $x\in\mathbb{R}^+$. Par conséquent, nous avons par définition:
	
	dès lors:
	
	Maintenant, nous faisons le changement de variable $u=y^2/(2\sigma^2)$, donc nous avons d'abord (c'est évident):
	
	anetd:
	
	Par conséquent, nous avons jusqu'à présent:
	
	Alors finalement:
	
	On reconnaît dans cette expression l'intégrale de la fonction Gamma Euler (\SeeChapter{voir section de Calcul Différentiel et Intégral page \pageref{gamma euler function}})! Il est donc immédiat que:
	
	Donc en résumé:
	
	Donc, finalement, nous obtenons le résultat dont nous avons besoin pour certaines propriétés du mouvement brownien en finance:
	
	Mais encore une propriété manque et maintenant pour nos besoins dans la section de génie industriel (page \pageref{doe}): la valeur de la médiane!
	
	Soit donc  $M_e$ la médiane de la distribution demi-normale. Alors, par définition, il s'ensuit que:
	
	Substituant $y/(\sqrt{2}\sigma)=u$ nous avons:
	
	On reconnaît ici la fonction d'Erreur (\SeeChapter{voir section Thermodynamique page \pageref{error function}}). Dès lors:
	
	Ainsi:
	
	Un tableur comme Microsoft Excel nous donne la fonction d'erreur complémentaire\index{error function}\index{complementary error function}:
	\begin{center}
	\texttt{=ERFC(0.5)=0.479500122}
	\end{center}
	Dès lors:
	
	La technique que nous verrons dans la section Génie Industriel (page \pageref{pareto margin error}) fait donc l'approximation:
	
	en effet ... c'est de l'ingénierie ...
	
	\subsubsection{Distribution de Laplace}
	Dans la théorie des probabilités et les statistiques, la "\NewTerm{distribution de Laplace}\index{distribution de Laplace}\label{Laplace distribution}" est une distribution de probabilité continue nommée aussi parfois nommée "\NewTerm{distribution exponentielle double}\index{distribution exponentielle double}", car elle peut être considérée comme deux distributions exponentielles (avec un paramètre d'emplacement supplémentaire) mélangées ensemble (bien que le terme soit également parfois utilisé pour désigner la distribution de Gumbel). Cette distribution a de nombreuses applications mais celle qui nous intéresse dans ce livre est l'application pour la régression LASSO (\SeeChapter{voir section de Méthodes Numériques page \pageref{LASSO regularization}}).
	
	La différence entre deux variables aléatoires exponentielles indépendantes distribuées de manière identique est régie par une distribution de Laplace.
	
	\begin{dem}
	Soit $X_1$ et $X_2$ deux variables aléatoirse exponentielles indépendantse dont la moyenne des populations est respectivement $\alpha_1$ et $\alpha_2$. Définissons $Y=X_1-X_2$. Le but est de trouver la distribution de $Y$ par la technique de la fonction de distribution cumulative. Nous savons d'abord que:
	
	et:
	
	Par indépendance, il s'ensuit que la fonction de densité de probabilité conjointe de $X_1$ et $X_2$ est:
	
	Nous voulons $F_Y(y)=P(Y\leq y)$ où $-\infty<y<+\infty$. En regardant le support de $f_{X_1,X_2}(x_1,x_2)$ donné ci-dessus, nous voyons que la fonction de distribution cumulative résultante pour $ Y $ sera (doit) être par morceaux, où les morceaux sont séparés à $y = 0$. Nous considérons donc d'abord le cas où $y\leq 0$. La double intégrale nous donnant la fonction de distribution cumulative $F_Y(y)$ de $y\leq 0$ est:
	
	Pour $y> 0$, la fonction de distribution cumulative de $Y$ est:
	
	La différenciation par rapport à $y$ donne:
	
	qui est la fonction de densité de probabilité d'une variable aléatoire de Laplace avec les paramètres $\alpha_1$ et $\alpha_2$.
	\begin{flushright}
		$\blacksquare$  Q.E.D.
	\end{flushright}
	\end{dem}
	
	Une variable aléatoire a une distribution $\mathcal{L}(\mu,b)$ si sa fonction de densité de probabilité est:
	
	Ici, $\mu$ est un paramètre de localication et $b > 0$, qui est parfois appelé "diversité", est un paramètre d'échelle. Si $\mu =0$ et $b=1$, la demi-ligne positive est exactement une distribution exponentielle mise à l'échelle par $1/2$.
	
	Notez que la distribution de Laplace est souvent notée:
	
	La fonction de densité de probabilité de la distribution de Laplace rappelle également la distribution Normale; cependant, alors que la distribution Normale est exprimée en termes de différence au carré de la moyenne $\mu$, la densité de Laplace est exprimée en termes de différence absolue par rapport à la moyenne. Par conséquent, la distribution de Laplace a des queues plus épaisses que la distribution Normale.
	
	La distribution de Laplace est facile à intégrer (si l'on distingue deux cas symétriques) grâce à l'utilisation de la fonction de valeur absolue. Sa fonction de distribution cumulée est la suivante:
	
	Dérivons maintenant la moyenne et la variance de la distribution de Laplace en adoptant la notation suivante:
	
	Soit $u=x-\mu$, nous avons alors $x=u+\mu$ et $\mathrm{d}x=\mathrm{d}u$. Par linéarité $\text{E}(X)=\text{E}(U+\mu)=\text{E}(U)+\mu$, dès lors:
	
	et:
	
	où:
	
	Dès lors:
	
	
	\pagebreak
	\subsubsection{Distribution de Benford}
	Cette distribution aurait été découverte une première fois en 1881 par Simon Newcomb, un astronome américain, après qu'il se fut aperçu de l'usure (et donc de l'utilisation) préférentielle des premières pages des tables de logarithmes (alors compilées dans des ouvrages). Frank Benford, qui aux alentours de 1938 remarqua à son tour cette usure inégale, crut être le premier à formuler cette loi qui porte indûment son nom aujourd'hui et arriva aux mêmes résultats après avoir répertorié des dizaines de milliers de données (longueurs de fleuves, cours de la bourse, etc.).

	Seule explication possible: nous avons plus souvent besoin d'extraire le logarithme de chiffres commençant par 1 que de chiffres commençant par 9, ce qui implique que les premiers sont "plus nombreux" que les seconds.

	Bien que cette idée lui paraisse tout à fait invraisemblable, Benford entreprend de vérifier son hypothèse. Rien de plus simple: il se procure des tables de valeurs numériques et calcule le pourcentage d'apparition du chiffre le plus à gauche (première décimale). Les résultats qu'il obtient confirment son intuition:
	\begin{table}[H]\centering\small
	\begin{tabular}{cccccccccc}\hline
	Chiffre initial &
	1 &  2 & 3 & 4 & 5 & 6 & 7 & 8 & 9 \\
	Probabilité d'apparition (\%)& 30.1 & 17.6 & 12.5 & 9.7 & 7.9
	& 6.7 & 5.8 & 5.1 & 4.6 \\ \hline
	\end{tabular}
	\caption{Probabilité d'apparition d'un chiffre selon la loi de Benford}
	\end{table}
	À partir de ces données, Benford trouve expérimentalement que la probabilité cumulée qu'un nombre commence par le chiffre $n$ (excepté $0$) est (nous allons le démontrer plus loin) donnée par la relation:
	
	appelée "\NewTerm{distribution de Benford}\index{distribution de Benford}" (or "\NewTerm{loi de Benford}\index{loi de Benford}").
	
	Voici un tracé de la fonction précédente:
	\begin{figure}[H]
		\centering
		\includegraphics{img/arithmetics/benford.jpg}
		\caption{Tracé de la fonction de Benford (fonction de répartition)}
	\end{figure}
	Il convient de préciser que cette loi ne s'applique qu'à des listes de valeurs "naturelles", c'est-à-dire à des chiffres ayant une signification physique. Elle ne fonctionne évidemment pas sur une liste de chiffres tirés au hasard.

	La loi de Benford a été testée sur toutes sortes de tables: longueur des fleuves du globe, superficie des pays, résultat des élections, liste des prix de l'épicerie du coin... Elle se vérifie presque à tous les coups.
	
	La distribution serait en plus indépendante de l'unité choisie. Si l'on prend par exemple la liste des prix d'un supermarché, elle fonctionne aussi bien avec les valeurs exprimées en Francs qu'avec les mêmes prix convertis en Euros.
	
	Cet étrange phénomène est resté peu étudié et inexpliqué jusqu'à une époque assez récente. Puis une démonstration générale en a été donnée en 1996, qui fait appel au théorème de la limite centrale.
	
	Aussi surprenant que cela puisse paraître, cette loi a trouvé une application: le fisc l'utiliserait aux Etats-Unis pour détecter les fausses déclarations. Le principe est basé sur la restriction vue plus haut: la loi de Benford ne s'applique que sur des valeurs ayant une signification physique.
	
	S'il existe une distribution de probabilité universelle $P(n)$ sur de tels nombres, ils doivent être invariants sous un changement d'échelle tel que:
	
	Si:
	
	Alors:
	
	et la normalisation de la distribution donne:
	
	Si nous dérivons $P(kn)=f(k)P(n)$ par rapport à $k$ nous obtenons:
	
	en posant $k = 1$ nous avons:
	
	Cette équation différentielle a pour solution:
	
	Cette fonction, n'est pas en premier lieu à proprement parler une fonction de distribution de probabilité (elle diverge) et deuxièmement, les lois de la physique et humaines imposent des limites.

	Nous devons donc comparer cette distribution par rapport à une référence arbitraire. Ainsi, si le nombre décimal étudié contient plusieurs puissances de $10$ ($10$ au total: $0,1,2,3,4,5,6,7,9$) la probabilité que le premier chiffre non nul (décimal) soit $D$ est alors donnée par:
	
	Les bornes de l'intégrale sont de $1$ à $10$ puisque la valeur nulle est interdite.

	L'intégrale du dénominateur donne:
	
	L'intégrale du numérateur donne:
	
	Ce qui nous donne finalement:
	
	De par les propriétés des logarithmes (\SeeChapter{voir section d'Analyse Fonctionelle page \pageref{logarithms}}) nous avons:
	
	Cependant, la loi de Benford ne s'applique pas uniquement aux données invariantes par changement d'échelle mais également à des nombres provenant de sources quelconques. Expliquer ce cas implique une investigation plus rigoureuse en utilisant le théorème de la limite centrale. Cette démonstration a été effectuée seulement en 1996 par T. Hill par une approche utilisant la distribution des distributions.

Pour résumer un partie importante de tout ce que nous avons vu jusqu'ici, l'illustration ci-dessous est très utile car elle résume les relations de 76 distributions univariées courantes (57 continues et 19 discrètes):
	\begin{figure}[H]
		\centering
		\includegraphics[width=\textwidth]{img/arithmetics/distributions.pdf}
		\caption[Relations entre distributions]{Relations entre distributions (source: AMS Lawrence M. Leemis et Jacquelyn T. McQueston)}
	\end{figure}
	et un résumé visuel de certaines distributions continues importantes:
	\begin{figure}[H]
		\centering
		\includegraphics[width=\textwidth]{img/arithmetics/continuous_distributions.jpg}
		\caption[Résumé visuel de quelques distributions statistiques continues importantes]{Résumé visuel de quelques distributions statistiques continues importantes (source: ?)}
	\end{figure}
	
	\pagebreak
	\subsection{Estimateurs de Vraisemblance}\label{likelihood estimators}
	Ce qui va suivre est d'une extrême importance en statistiques et est utilisé énormément en pratique. Il convient donc d'y accorder une attention toute particulière! Outre le fait que nous utiliserons cette technique dans la présent chapitre, nous la retrouverons dans le chapitre de Méthodes Numériques pour les techniques avancées de régressions linéaires généralisées ((y compris leur généralisation aux $M$-estimateurs ou même $W$-estimateurs!) ainsi que dans le chapitre de Génie Industriel dans le cadre de l'estimation des paramètres de fiabilité.

	Nous supposons que nous disposons d'observations $x_1,x_2,x_3,\ldots,x_n$ qui sont des réalisations de variables aléatoires non biaisées (dans le sens qu'elles sont choisies aléatoirement parmi un lot) indépendantes $X_1,X_2,X_3,\ldots,X_n$ de loi de probabilité inconnue mais identique.
	
	Supposons que nous procédons par tâtonnements pour estimer la loi de probabilité $P$ inconnue. Une manière de procéder est de se demander si les observations $x_1,x_2,x_3,...,x_n$ avaient une probabilité élevée ou non de sortir avec cette loi de probabilité arbitraire $P$? 
	
	Nous devons pour cela calculer la probabilité conjointe qu'avaient les observations $x_1,x_2,x_3,\ldots,x_n$ de sortir avec les probabilités $p_1,p_2,p_3,\ldots,p_n$. Cette probabilité conjointe vaut (\SeeChapter{voir section Probabilités page \pageref{joint probability}}):
	
	en notant $P$ la loi de probabilité supposée associée à $p_1,p_2,p_3,\ldots,p_n$. Il faut avouer qu'il serait alors particulièrement maladroit, au niveau de la notion intuitive de risque, de choisir une loi de probabilité (avec ses paramètres!) qui minimise cette quantité...
	
	Au contraire, nous allons chercher les probabilités $p_1,p_2,p_3,\ldots,p_n$ (ou les paramètres de la loi associée) qui maximisent $\prod_{i=1}^n P(X_i=x_i)$, c'est-à-dire qui rende les observations $x_1,x_2,x_3,\ldots,x_n$ le plus vraisemblable possible.

	Nous sommes donc amenés à chercher le (ou les) paramètre(s) $\theta$ qui maximise(nt) la quantité:
	 
	et où le paramètre $\theta$ est souvent dans les cas scolaires un moment d'ordre un (espérance) ou d'ordre deux (variance).

	Cette quantité $L$ porte le nom de "vraisemblance". C'est une fonction du ou des paramètres equation et des observations $\theta$.

	La ou les valeurs du paramètre $\theta$ qui maximisent la vraisemblance $L_n(\theta)$ sont appelées "\NewTerm{estimateurs du maximum de vraisemblance}\index{estimateurs du maximum de vraisemblance}\label{maximum likelihood estimators}" (estimateur MV/EMV).
	
	Dans le cas très particulier mais formateur de la loi Normale, un des paramètres $\theta$ sera donc la variance (voir un peu plus loin l'exemple concret) et il peut être considéré comme intuitif au physicien que pour maximiser la probabilité, l'écart-type doit être le plus petit possible (pour que le maximum d'évenements se trouve dans un même intervalle). Ainsi, lorsque nous calculons un EMV qui est le plus petit parmi plusieurs possibles, nous parlons alors d'estimateur UMV pour "\NewTerm{Uniform Minimum Variance Unbiased}\index{uniform minimum variance unbiased}" car leur propre variance doit être la plus petite possible. Cela se démontre (mais c'est peu élégant) en utilisant la définition de l'Information de Fisher et du théorème de Fréchet (ou de Rao-Cramer) qui fait usage de l'inégalité de Cauchy-Schwartz (\SeeChapter{voir section de Calcul Vectoriel page \pageref{cauchy-schwarz inequality}}) et de l'analogie entre espérance et produit scalaire... Cette démonstration ne sera pas présentée dans ce livre.

	Faisons quand même sept petits exemples (très classiques, utiles et importants dans l'industrie) avec dans l'ordre d'importance (donc pas forcément dans l'ordre de facilité...) la fonction de distribution de Gauss-Laplace (Normale), la fonction de distribution de Poisson, la distribution Binomiale, la distribution Géométrique, la distribution de Weibull, la distribution de Gamma et finalement la distribution Pareto.
	
	\begin{tcolorbox}[title=Remarques,colframe=black,arc=10pt]
	\textbf{R1.} Ces sept exemples sont importants car utilisés dans les SPC (maîtrise statistiques de processus) dans différentes multinationales à travers le monde (\SeeChapter{voir section de Génie Industriel page \pageref{industrial engineering}}).\\
	
	\textbf{R2.} Une estimation fréquentiste du maximum de vraisemblance est semblable au "mode" de la fonction de vraisemblance, donc dans un contexte bayésien avec un a priori uniforme, le MLE est comme un mode du postérieur.\\
	
	\textbf{R3.} Dans les statistiques bayésiennes (voir plus bas), nous avons le "\NewTerm{maximum à posteriori}\index{maximum à posteriori}" (MAP) qui trouve un maximum de la distribution postérieure pour une probabilité a priori uniforme. L'estimateur MAP coïncide avec l'estimateur du maximum de vraisemblance.
	\end{tcolorbox}
	
	\subsubsection{Estimateurs de la distribution Normale}\label{normal distribution mle}
	Soit $x_1,x_2,...,x_n$ un $n$-échantillon de variables aléatoires identiquement distribuées supposées suivre une loi de Gauss-Laplace (loi Normale) de paramètres $\mu$ et $\sigma^2$.
	
	Nous recherchons quelles sont les valeurs des estimateurs du maximum de vraisemblance equation qui maximisent la vraisemblance $L_n(\theta)$ de la loi Normale?

	\begin{tcolorbox}[title=Remarque,colframe=black,arc=10pt]
	Il va de soi que le vecteur d'estimateurs du maximum de vraisemblance est ici:
	
	\end{tcolorbox}
	Nous avons démontré plus haut que la densité d'une variable aléatoire gaussienne était donnée par: 
	
	La vraisemblance est alors donnée par:
	
	Maximiser une fonction ou maximiser son logarithme est équivalent donc la "\NewTerm{log-vraisemblance}\index{log-vraisemblance}\label{log-likelihood}" sera:
	
	Pour déterminer les deux estimateurs de la loi Normale, fixons d'abord l'écart-type. Pour cela, dérivons  $\ln(L(\mu,\sigma))$ par rapport à $\ln(L(\mu,\sigma))$ et regardons pour quelle valeur de la moyenne $\mu$ la fonction s'annule.
	
	\begin{tcolorbox}[title=Remarque,colframe=black,arc=10pt]
	La dérivéer partielle que nous cherchons à annuler est souvent appelée la "fonction score" (nous retrouverons souvent ce concept lors de notre étude des techniques d'optimisation et de l'apprentissage machine/machine learning dans le chapitre d'Informatique Théorique):
	
	\end{tcolorbox}
	
	Il nous reste après simplification le terme suivant qui est égal à zéro:
	
	Ainsi, l'estimateur du maximum de vraisemblance de la moyenne (espérance) de la loi Normale est donc après réarrangement:
	
	et nous voyons qu'il s'agit simplement de la moyenne arithmétique (ou appelée aussi "\NewTerm{moyenne empirique}\index{moyenne empirique}").
	
	Fixons maintenant la moyenne. L'annulation de la dérivée de $\ln(L(\mu,\sigma))$ en $\sigma$ conduit à:
	
	Ce qui nous permet d'écrire l'estimateur du maximum de vraisemblance pour l'écart-type (la variance lorsque la moyenne est connue selon la loi de distribution supposée elle aussi connue!):
	
	que certains appellent aussi "\NewTerm{écart-type de Pearson}\index{\'ecart-type de Pearson}"...
	
	Même si elle est un peu redondante, certaines personnes nous ont demandé de montrer la preuve de l'estimateur de la matrice de covariance (et donc de la matrice de corrélation).
	
	N'oubliez pas que nous avons prouvé précédemment que pour le cas bivarié, nous avions:
	
	En fait, la relation est la même pour le cas multivarié avec $T$!
	
	La log-vraisemblance est donc immédiate par analogie avec le cas univarié:
	
	Que nous pouvons aussi écrire (étant donné que $\Sigma$ est diagonale et $\Sigma^{-1}$ de même):
	
	Où par définition et en utilisant l'estimateur de la moyenne:
	
	Nous en déduisons ensuite:
	
	et on obtient enfin:
	
	Cependant, nous n'avons pas encore défini ce qu'était un bon estimateur ! Ce que nous entendons par là:
	\begin{itemize}
		\item Si l'espérance d'un estimateur est égale à elle-même, nous disons que cet estimateur est "sans biais\label{unbiased estimator}" et c'est bien évidemment ce que nous voulons!
	
		\item Si l'espérance d'un estimateur n'est pas égale à elle-même, nous disons alors que cet estimateur est "biaisé" et c'est forcément moins bien...
	\end{itemize}
	Dans l'exemple précédent, la moyenne est donc non biaisée (trivial car la moyenne de la moyenne arithmétique est égale à elle-même). Mais qu'en est-il de la variance (in extenso de l'écart-type)?
	
	Un petit calcul simple par linéarité de l'espérance (puisque les variables aléatoires sont identiquement distribuées) va nous donner la réponse dans le cas où la moyenne théorique est approchée comme dans la pratique (industrie) par l'estimateur de la moyenne (cas le plus fréquent).

	Nous avons donc le calcul de l'espérance de la "\NewTerm{variance empirique}\index{variance empirique}":
	
	Or, comme les variables sont équidistribuées:
	
	Et nous avons (relation de Huyghens):
	
	où la deuxième relation ne peut s'écrire que parce que nous utilisons l'estimateur du maximum de vraisemblance de la moyenne (moyenne empirique). Par conséquent, en combinant les deux relations ci-dessus avec l'anté-précédente, nous obtenons:
	
	et comme:
	
	Nous avons finalement:
	
	nous avons donc un biais de moins une fois l'erreur-standard:
	
	nous disons alors que cet estimateur à un biais négatif (il sous-estime la vraie valeur!).
	
	Nous noterons également que l'estimateur tend vers un estimateur sans biais (E.S.B.) lorsque le nombre d'individus tend vers l'infini $n\rightarrow +\infty$. Nous disons alors que nous avons un "\NewTerm{estimateur asymptotiquement non biais\'e}\index{estimateur asymptotiquement non biaisé}" ou "\NewTerm{estimateur asymptotiquement débiaisé}".
	
	Ensuite, le "\NewTerm{biais d'un estimateur}\index{biais d'un estimateur}" est défini, selon le cas particulier ci-dessus, en général comme:
	
	Il est important de prendre note que nous avons démontré que la variance empirique tend vers la variance théorique quand $n$ tend vers l'infini et ce... que les données suivent une loi Normale ou non!
	
	De par les propriétés de l'espérance, nous avons alors:
	
	Il vient alors:
	
	que certains appellent aussi "\NewTerm{standard deviation}"... (à ne pas confondre avec "l'erreur-standard" que nous verrons plus loin!).
	
	Un estimateur est dit "\NewTerm{estimatuer consistant}\index{estimatuer consistant}" s'il converge \underline{en probabilités} (car il peut être considéré lui-même comme une variable aléatoire ayant sa propre distribution), lorsque $n\rightarrow +\infty$, vers la vraie valeur du paramètre. Techniquement, cette condition sera écrite:
	
	Peut-être que l'exemple suivant peut aider à la compréhension. Supposons que les échantillons proviennent d'une distribution Normale. Ensuite, en utilisant le fait que $\frac{(n-1)S^2}{\sigma^2}$ (voir page \pageref{Cochran theorem}) est une variable aléatoire chi-carré avec $n-1$ degrés de la liberté, on obtient:
	
	où nous avons utilisé le fait (voir page \pageref{chi-square distribution}):
	
	Donc, écrit différemment (selon la notation ci-dessus):
	
	De ces faits, nous pouvons voir officieusement que la distribution de $\text{V}\left(\hat{\sigma}^2\right)$ devient de plus en plus concentrée sur $\sigma^2$ à mesure que la taille de l'échantillon augmente puisque la moyenne $\text{E}\left(\hat{\sigma}^2\right)$ converge vers $\sigma^2$ (ie $\hat{\sigma}$ est sans biais) et la variance $\text{V}\left(\hat{\sigma}^2\right)$ converge vers $0$ (ie $\hat{\sigma}$ est consistent)!

	Nous avons donc finalement pour résumer les deux résultats importants suivants
	\begin{enumerate}
	
		\item "\NewTerm{L'estimateur du maximum de vraisemblance biaisé}\index{estimateur du maximum de vraisemblance biais\'e}" ou appelé également "\NewTerm{écart-type empirique}\label{empirical standard deviation}\index{\'ecart-type empirique}" ou encore "\NewTerm{écart-type échantillonnal}\index{\'ecart-type échantillonnal}" ou encore "\NewTerm{écart-type de Pearson}\index{\'ecart-type de Pearson}" ... et donc donné par:
		
		lorsque $n\rightarrow +\infty$. Nous retrouvons cet écart-type suivant les contextes (par tradition) noté de cinq autres différentes façons qui sont:
		
		et même parfois (mais c'est très malheureux car cela génère alors souvent de la confusion avec l'estimateur non biaisé) $\sigma$ ou $S$.
		
		\item "\NewTerm{L'estimateur du maximum de vraisemblance non biaisé}\index{estimateur du maximum de vraisemblance non biais\'e}" ou appelé également "\NewTerm{écart-type standard}\index{\'ecart-type standard}" avec la "\NewTerm{correction de Bessel}\index{correction de Bessel}" (le -1 au dénominateur est la correction en question...):
		
		qui comme nous le voyons est un estimateur convergent (quand $n$ tend vers l'infini celui-ci tend vers l'estimateur du maximum de vraisemblance biaisé).

		Nous retrouvons cet écart-type suivant les contextes (par tradition) noté de trois autres différentes façons qui sont:
		
	\end{enumerate}
	Nous retrouverons ces deux dernières notations souvent dans les tables et dans de nombreux logiciels et que nous utiliserons plus bas dans les développements des intervalles de confiance et des tests d'hypothèses!

	Par exemple, dans la version française de de Microsoft Excel 11.8346 l'estimateur biaisé est donné par la fonction \texttt{ECARTYPEP( )} et le non biaisé par \texttt{ECARTTYPE( )}.

	Au total, cela nous fait donc trois estimateurs pour la même quantité!! Comme dans l'écrasante majorité des cas de l'industrie la moyenne théorique n'est pas connue, nous utilisons le plus souvent les deux dernières relations encadrées ci-dessus. Maintenant, c'est là que c'est le plus vicieux: lorsque nous calculons le biais des deux estimateurs, le premier est biaisé, le second ne l'est pas. Donc nous aurions tendance à n'utiliser que le second. Que nenni! Car nous pourrions aussi parler de la variance et de la précision d'un estimateur, qui sont aussi des critères importants pour juger de la qualité d'un estimateur par rapport à un autre. Si nous faisions le calcul de la variance des deux estimateurs, alors le premier, qui est biaisé, a une variance plus petite que le second qui est sans biais! Tout ça pour dire que le critère du biais n'est pas (et de loin) le seul à étudier pour juger de la qualité d'un estimateur.

	Enfin, il est important de se rappeler que le facteur $-1$ du dénominateur de l'estimateur du maximum de vraisemblance non biaisé provient du fait qu'il fallait corriger l'espérance de l'estimateur biaisé à la base minoré de une fois l'erreur-standard!!

	\subsubsection{Estimateur de la distribution de Poisson}\label{poisson distribution mle}
	En utilisant la même méthode que pour la loi Normale (Gauss-Laplace), nous allons donc rechercher les estimateurs du maximum de vraisemblance de la loi de Poisson qui rappelons-le, est donnée par:
	
	Dès lors, la vraisemblance est donnée par:
	
	Maximiser une fonction ou maximiser son logarithme est équivalent donc:
	
	Nous cherchons maintenant à la maximiser:
	
	et obtenons donc son unique estimateur du maximum de vraisemblance qui s
	
	Il est tout à fait normal de retrouver dans cet exemple didactique la moyenne empirique, car c'est le meilleur estimateur possible pour le paramètre de la loi de Poisson (qui représente aussi l'espérance d'une loi de Poisson).

	Sachant que l'écart-type de cette distribution particulière (voir plus haut lors de notre développement de la loi de Poisson) n'est que la racine carrée de la moyenne, nous avons alors pour l'écart-type du maximum de vraisemblance:
	
	\begin{tcolorbox}[title=Remarque,colframe=black,arc=10pt]
	Nous montrons de la même manière des résultats identiques pour la loi exponentielle très utilisée en maintenance préventive et fiabilité!
	\end{tcolorbox}
	
	\subsubsection{Estimateur de la distribution Binomiale}
	Supposons que nous ayons un modèle de Bernouilli dans lequel chaque observation a une probabilité constante et égale de succès notée $p$. Une variable de Bernoulli prend l'une des deux valeurs, conventionnellement $ 1 $ ou $ 0 $, qui indiquent un «succès» ou un «échec». La distribution de probabilité pour cette variable est donc:
	
	Si $ S $ représente le nombre de succès et $ F $ le nombre d'échecs, alors la vraisemblance est:
	
	et la log-vraisemblance serait alors:
	
	Nous prenons ensuite la dérivée de ceci par rapport à $p$:
	
	Nous définissons maintenant cela égal à zéro et résolvons pour $ p $ et nous obtenons immédiatement:
	
	
	\subsubsection{Estimateur de la distribution Géométrique}
	En utilisant la même méthode que pour la loi Normale (Gauss-Laplace) et la loi de Poisson, nous allons donc rechercher l'estimateur du maximum de vraisemblance de la loi Binomiale qui rappelons-le, est donnée par:
	
	Dès lors, la vraisemblance est donnée par:
	
	Il convient de se rappeler que le facteur qui suit le terme combinatoire exprime déjà les variables successives selon ce que nous avons vu lors de notre étude de la fonction de distribution de Bernoulli et de la fonction binomiale. D'où la disparition du produit dans la dernière égalité précédente.

	Maximiser une fonction ou maximiser son logarithme est équivalent donc:
	
	Nous cherchons maintenant à la maximiser:
	
	Le lecteur aura peut-être remarqué que le coefficient binomial a disparu. Dès lors, nous en déduisons immédiatement que l'estimateur de la loi binomiale sera le même que celui de la loi géométrique.

	Ce qui donne
	
	d'où nous tirons l'estimateur du maximum de vraisemblance qui sera donc la simple moyenne empirique:
	
	Ce résultat est assez intuitif si l'on considère l'exemple classique d'une pièce de monnaie qui a une chance sur deux de tomber sur une de ces faces. La probabilité $p$ étant le nombre de fois $k$ où une face donnée a été observée sur le nombre d'essais total (toutes faces confondues).
	
	\begin{tcolorbox}[title=Remarque,colframe=black,arc=10pt]
	Dans la pratique, il n'est pas aussi simple d'appliquer ces estimateurs! Il faut bien réfléchir lesquels sont les plus adaptés à une expérience donnée et idéalement calculer également l'erreur quadratique moyenne (erreur-standard) de chacun des estimateurs de la moyenne (comme nous l'avons déjà fait pour la moyenne empirique plus tôt). Bref c'est un long travail de réflexion.
	\end{tcolorbox}
	
	\subsubsection{Estimateurs de la distribution de Weibull}
	Nous avons vu dans la section de Génie Industriel une étude très détaillée de la loi de Weibull à trois paramètres avec son écart-type et son espérance car nous avions précisé qu'elle était assez utilisée dans le domaine de l'ingénierie de la fiabilité (\SeeChapter{voir section Génie Industriel page \pageref{weibull distribution}}).

	Malheureusement les trois paramètres de cette loi  nous sont en pratique inconnus. A l'aide des estimateurs nous pouvons cependant déterminer l'expression de deux des trois en supposant $\gamma$ comme étant nul. Cela nous donne donc la loi de Weibull dite "loi de Weibull à deux paramètres" suivante:
	
	avec pour rappel $\beta>0$ et $\eta>0$.
	
	Dès lors la vraisemblance est donnée par:
		
	Maximiser une fonction ou maximiser son logarithme est équivalent donc:
	
	Cherchons maintenant à maximiser cela en se rappelant que (\SeeChapter{voir section de Calcul Différentiel et Intégral page \pageref{usual derivatives}}):
	
	d'où:
	
	Et nous avons pour le deuxième paramètre:
		
	Alors:
	
	Enfin pour résumer avec les notations correctes (et dans l'ordre de résolution en pratique):
	
	La résolution de ces équations implique des calculs lourds et on ne peut a priori rien faire avec ça dans des tableurs conventionnels comme Microsoft Excel ou Open Office Calc sans programmation (du moins pour autant que l'on sache ...).

	Nous adoptons ensuite une approche différente en écrivant notre distribution de Weibull avec deux paramètres comme suit:
	
	avec pour rappel $\beta>0$ et $\theta>0$.

	Par conséquent, la vraisemblance est donnée par:
	
	Maximiser une fonction ou maximiser son logarithme est équivalent donc:
	
	Maintenant, nous cherchons à maximiser cela en nous rappelant que (\SeeChapter{voir section de Calcul Différentiel et Intégral page \pageref{usual derivatives}}):
	
	alors:
	
	Et nous avons pour le deuxième paramètre:
	
	Il est alors immédiat que:
	
	injecté dans l'équation:
	
	Nous obtenons:
	
	En simplifiant:
	
	La résolution des deux équations (dans l'ordre de haut en bas):
	
	peut être facilement calculé avec l'Outil Cible de Microsoft Excel ou Open Office Calc.	
	
	\subsubsection{Estimateurs de la distribution Gamma}
	Nous allons utiliser ici une technique appelée "\NewTerm{méthode des moments}\index{méthode des moments}" pour déterminer les estimateurs des paramètres de la loi Gamma.
	
	Supposons que $X_1, ..., X_n$ sont des variables aléatoires indépendantes et identiquement distribuées selon la loi Gamma avec pour densité:
	
	Nous cherchons à estimer $a,\lambda$. Pour cela, nous déterminons d'abord quelques moments théoriques. Le premier moment est l'espérance qui comme nous l'avons démontré vaut:
	
	et le second moment, l'espérance du carré de la variable aléatoire, est comme nous l'avons démontré implicitement lors de la démonstration de la variance de la loi Gamma:
	
	Nous exprimons ensuite la relation entre les paramètres et les moments théoriques:
	
	La résolution de ce système simple donne:
	
	Une fois ce système établi, la méthode des moments consiste à utiliser les moments empiriques, en l'occurrence pour notre exemple les deux premiers, $\hat{m}_1,\hat{m}_2$:
	
	que nous posons égaux aux moments théoriques vrais... Dès lors, il vient:
	
	
	\subsubsection{Estimateur de la loi de Pareto}
	Rappelons la densité de Pareto est donnée par:
	
	Choisissons une notation plus pratique et traditionnelle:
	
	La fonction de vraisemblance de la distribution de Pareto sur un échantillon $x=(x_1,\ldots, x_n)$ est donnée par:
	
	En prenant le log de la vraisemblance, on obtient:
	
	Étant donné qu'un $\beta$ plus élevé entraînera toujours une probabilité plus élevée -  $\ln(\beta) $ augmente de façon monotone - nous maximisons la probabilité en fixant $\hat{\beta} $ le plus haut possible (nous ne pouvons pas utiliser la différenciation pour maximiser la probabilité dans ce cas). Puisque $\beta \leq x_{i}$ pour tout $i$, nous maximisons la probabilité en définissant $\hat{\beta}=\underset{i}{\min}\; x_{i}$, le plus petit $x_{i}$ dans l'échantillon.
	
	Pour $\alpha$, nous définissons la dérivée partielle de $\ell$ par rapport à $\alpha$ égale à $0$:
	
	Dès lors:
	
	
	\subsubsection{Estimateurs de données censurées}
	Le "\NewTerm{vraisemblance de données censurées}\index{vraisemblance de données censurées}" est d'une importance cruciale dans le domaine de la fiabilité industrielle ainsi que dans les pharmacies et les études cliniques, sachant que pour des raisons de coût et de circonstances imprévues, il y a souvent une limite au temps d'observation ou aux événements qui font que les sujets d'étude quittent l'expérience. Sachant qu'il existe quatre types de situations (voir la figure ci-dessous si nécessaire):
	\begin{enumerate}
		\item "\NewTerm{Données non-censurées}\index{données non-censurées}" (nc), l'événement d'intérêt est observé entre le moment de début de la mesure et la fin de la mesure.
	
		\item "\NewTerm{Censurées à droite}\index{Censurées à droite}" (cd), l'événement d'intérêt ne s'est pas produit pendant le temps d'observation limité, c'est-à-dire entre le début et la fin de la mesure.
		
		\item "\NewTerm{Censurées à gauche}\index{Censurées à gauche}" (cg), l'objet d'étude a quitté l'expérience pour une raison quelconque observée entre le début de la mesure et la fin de la mesure, ou nous ne savons pas avec précision quand l'événement d'intérêt qui aurait dû être mesuré a commencé.
	
		\item "\NewTerm{Intervalle censurée}\index{intervalle censurée}" (ic), l'événement d'intérêt a eu lieu entre deux instants, mais nous ne savons pas exactement quand dans cet intervalle (avant, entre ou après les périodes de mesure).
	\end{enumerate}
	\begin{figure}[H]
		\centering
		\includegraphics{img/arithmetics/censored_data.jpg}
		\caption{Type de données censurées}
	\end{figure}
	
	Il est alors relativement intuitif que si tous les événements sont indépendants, la fonction de vraisemblance sera donnée par:
	
	où l'index $l$ est pour "gauche" et l'index $r$ est pour "droite".
	
	On a évidemment (pensez au cas discret avant de penser au cas continu):
	
	Dans le domaine de l'ingénierie de fiabilité qui nous intéresse ici, nous nous réduisons souvent aux cas suivants:
	
	mais avec la notation suivante (...):
	
	Dans de nombreuses expériences, le nombre d'éléments défectueux $f$ à observer sur $n$ est fixé à l'avance (on parle souvent de "\NewTerm{censure de type II}\index{censure de type II}") ou simplement connu par la suite. Par conséquent, nous avons:
	
	Explicitons cette probabilité avec la loi de Weibull. Rappelons qu'une forme possible d'écriture de la fonction de densité s'écrit pour cette dernière (\SeeChapter{voir section de Génie Industriel page \pageref{weibull distribution}}):
	
	et que nous avons prouvé que sa fonction de survie correspondante était donnée par:
	
	Dès lors:
	
	Maximiser la probabilité équivaut à maximiser le logarithme. Dès lors:
	
	Nous calculons la dérivée partielle par rapport à $\alpha$, $\beta$ (\SeeChapter{voir section de Calcul Différentiel et Intégral page \pageref{usual derivatives}}):
	
	La deuxième relation peut facilement être explicitée par rapport à $\beta$ de telle sorte que:
	
	Dans tous les cas, il semble qu'il sera nécessaire d'utiliser des techniques d'optimisation numérique pour déterminer tous les paramètres.
	
	\subsubsection{Matrice d'information de Fisher}
	Nous allons introduire maintenant une matrice qui nous sera utile beaucoup plus tard pour certains tests statistiques avancés mais aussi pour déterminer (entre autres!) des paramètres de loi statistique non triviaux et aussi un critère d'information important nommé le "critère d'information Akaike".
	
	Nous avons vu précédemment que la dérivée partielle de la log-vraisemblance (que nous désignerons par $\mathcal{L}$ pour les développements suivants) relativement à un paramètre donné $\theta_i$ d'une fonction de probabilité (dépendante d'un ensemble de paramètres $\vec{\theta}$) nous donnera un estimateur de ce paramètre si nous le posons comme égal à zéro.
	
	Aussi, respectivement, nous avons alors que la dérivée partielle évaluée à cet estimateur correspondant donne alors:
	
	\begin{tcolorbox}[title=Remarque,colframe=black,arc=10pt]
	L'expression:
		
	est connue sous le nom de "\NewTerm{fonction de score}\index{fonction de score}".
	\end{tcolorbox}
	
	Rappelons que lors de notre étude des séries bivariées de Taylor (\SeeChapter{voir section Séquences et Séries page \pageref{multivariate taylor series}}), nous avons prouvé que:
	
	où:
	
	Faisons le remplacement suivant:
	
	Nous avons alors:
	
	\begin{tcolorbox}[title=Remarque,colframe=black,arc=10pt]
	Dans le cas univarié ceci est aussi noté:
	 
	ou encore dans certains livre de référence:
	
	\end{tcolorbox}
	Sous forme vectorielle ceci est souvent noté de la manière suivante:
	
	Maintenant rappelons-nous que $f(\vec{v}_0)$ et $\vec{\nabla}f(\vec{v}_0)$ ne contiennent plus aucun terme en $x$ ou $y$ étant donné que la fonction est valuée. Dès lors, si nous prenons le gradient dans l'ensemble de l'expression précédente, nous obtenons:
	
	Si le lecteur ne comprend pas ce résultat, refaire le calcul dans le cas univarié peut aider. Effectivement, en partant de:
	
	Nous avons alors de manière très détaillée:
	
	Cette dernière relation peut être écrite:
	
	Maintenant revenons à l'égalité:
	
	Nous pouvons résoudre cette expression pour le point optimal $\vec{\nabla}f(\vec{v}) = \vec{0}$, ce qui donne:
	
	Dès lors:
	
	Il s'agit de la méthode de Newton-Raphson pour l'optimisation multidimensionnelle\index{optimisation multidimensionnelle de Newton-Raphson} pour information... 
	
	Concentrons-nous maintenant sur l'égalité suivante de la précédente:
	
	Et changeons la notation $f$ par $\mathcal{L}$:
	
	où $\vec{\nabla}\mathcal{L}(\vec{\theta})$ (le gradient de la vraisemblance au point $\vec{\theta}$) est nommé "\NewTerm{vecteur de score}\index{vecteur de score}".
	
	\begin{tcolorbox}[title=Remarque,colframe=black,arc=10pt]
	Dans le cas univarié, ceci ce réduira à:
	
	\end{tcolorbox}
	Maintenant ayons un peu de 'fun'... Prenons le produit scalaire (\SeeChapter{voir section Algèbre Linéaire page \pageref{outer product}}) of $\vec{\theta}$, ie:
	
	Ok jusqu'à présent rien d'intéressant. Mais prenons l'espérance:
	
	Ici, nous reconnaissons quelque chose! Les éléments de la diagonale ressemblent à une variance... La variance de $\theta_1$ par rapport à $\theta_{1,0}$. Il est courant de supposer que $\theta_1$ est l'estimateur et $\theta_{1,0}$ est la vraie valeur, puis nous écrivons (gardez à l'esprit que cela peut être généralisé à plus de $2$ paramètres!):
	
	Nous reconnaissons ici la matrice de variance-covariance\footnote{Tristement désignée dans certains manuels $\text{V}(\vec{\theta})$....} (\SeeChapter{voir plus haut page \pageref{variance covariance matrix}} ). Ok c'est bien mais alors quoi ??? Qu'est-ce que cela nous apporte? En fait, en utilisant les égalités précédentes, nous avons:
	
	En utilisant les propriétés de transposition comme on les a démontrées dans la section d'Algèbre Linéaire et la propriété d'association de multiplication matricielle, cette dernière relation peut être écrite:
	
	En utilisant une autre propriété de la transposition, cela peut aussi s'écrire:
	
	En supposant que la Hessienne est symétrique de telle sorte que $H^T=H$ (si le théorème de Schwarz \footnote{voir section Calcul Différentiel et Intégral page  \pageref{Schwarz theorem}} s'applique à la fonction considérée à l'intérieur de la Hessienne!), nous avons alors:
	
	En utilisant la propriété de linéarité de la moyenne:
	
	Concentrons-nous maintenant sur:
	
	Mais nous avons besoin d'un résultat important avant! Comme la probabilité $L(\vec{x}),\vec{\theta})$ est une densité, nous savons que:
	
	Maintenant, en différenciant les deux côtés par rapport à l'un des paramètres $\theta_i$, nous obtenons:
	
	En supposant (cette hypothèse doit être vérifiée dans tous les cas!) que nous avons:
	
	Alors:
	
	Mais ceci aussi égal à:
	
	En différenciant à nouveau par rapport à $\theta_i$ et en prenant la dérivée intérieure, cela nous donne:
	
	Dès lors:
	
	Écrit d'une manière plus courante par les statisticiens comme suit:
	
	Et en utilisant la notation définie précédemment:
	
	Si nous refaisons le même développement mais en prenant la dérivée partielle une deuxième fois relativement à $\theta_j$ au lieu de $\theta_i$, nous obtenons immédiatement:
	
	Cela peut être mis évidemment sous forme de matrice pour chaque composant $i,j$. Il est alors habituel de définir la moyenne de la matrice composée des composants $\frac{\partial \mathcal{L}(\vec{x},\vec{\theta})}{\partial \theta_i} \frac{\partial \mathcal{L}(\vec{x},\vec{\theta})}{\partial \theta_j}$ comme la "\NewTerm{matrice d'information de Fisher}\index{matrice d'information de Fisher}\label{Fisher information matrix}" noté $\mathcal{I}(\vec{\theta})$ et nous reconnaissons qu'une matrice composée des composants $\frac{\partial^2 \mathcal{L}(\vec{x},\vec{\theta})}{\partial \theta_i \partial \theta_j}$ est la "Hessienne" (\SeeChapter{voir section Séquences et Séries page \pageref{hessian matrix}}) notée pour rappel par $H(\vec{\theta})$, alors nous avons:
	
	Et et il s'agit bien de la relation que nous nous attendions à trouver!
	
	Revenons donc maintenant à:
	
	Que nous pouvons maintenant réécrire:
	
	Donc, finalement, nous obtenons le résultat final important que nous recherchions depuis le début nommé "\NewTerm{égalité de la matrice d'information}\index{\'egalit\'e de la matrice d'information}":
	
	C'est-à-dire que la matrice de covariance-variance (elle-même égale au produit extérieur de $\vec{\theta}$ avec elle-même!), est égale au négatif de l'inverse de la matrice Hessienne! Ou en d'autres termes:  matrice hessienne d'une fonction de vraisemblance est égale à la matrice d'information, ou la matrice de variance-covariance des fonctions de score!
	
	\begin{tcolorbox}[colframe=black,colback=white,sharp corners]
	\textbf{{\Large \ding{45}}Exemple:}\\\\
	Notre échantillon est constitué des premiers $n$ termes d'une séquence indépendante et identiquement distribuée $\{X_n\}$de variables aléatoires Normalement distribuées ayant une moyenne $\mu$ et une variance $\sigma^2$. La fonction de densité de probabilité d'un terme générique de la séquence est:
	
	La moyenne $\mu$ et la variance $\sigma^2$sont les deux paramètres à estimer.\\
	
	Nous pouvons accéder aux erreurs standards de $\vec{\theta}$ à partir des racines carrées des éléments diagonaux de la matrice $\Sigma$. Notez que ceux-ci ne sont corrects qu'asymptotiquement et sont difficiles à calculer dans des échantillons finis.\\
	
	Nous savons que la fonction de vraisemblance est:
	
	et donc (pour rappel):
	
	Les vraisemblances et leur solution sont pour rappel:
	
	\end{tcolorbox}

	\begin{tcolorbox}[colframe=black,colback=white,sharp corners]
	Calculons maintenant la Hessienne de $\hat{\theta}=(\hat{\mu},\hat{\sigma}^2$, c'est-à-dire la Hessienne de $\mathcal{L}(\vec{\theta})$. Nous devons pour cela calculer $4$ composantes de la matrice étant donné que la vraisemblance a $2$ paramètres ($2\times 2=4$) et les composantes hors diagonale sont égales en supposant que le théorème de Schwarz s'applique. Dès lors:
	
	Quand valué au point $\hat{\vec{\theta}}$:
	
	Dès lors:
	
	Nous avons aussi:
	
	Nous avons:
	
	Par conséquent, la matrice d'information de Fisher est donnée par:	
	\end{tcolorbox}	
	
	\begin{tcolorbox}[colframe=black,colback=white,sharp corners]
	
	L'inversion de matrice nous donne aussi (\SeeChapter{voir section Algèbre Linéaire page \pageref{determinant matrix inverse}}):
	
	Dès lors:
	
	\end{tcolorbox}	
	Rappelons maintenant que lors de notre démonstration du théorème de limite centrale, nous avons prouvé que (page \pageref{central limit theorem}):
	
	Dès lors:
	
	Maintenant, au lieu de l'estimateur $\hat{\mu}$, considérons l'estimation de la fonction de score (le $\sigma$ est évidemment le même que la relation précédente):
	
	Maintenant, nous considérons que la valeur de $\hat{\mathcal{S}}$ valué au point de la vraie valeur de $\theta$ (c'est-à-dire $\theta_0$) n'est pas nul car il s'agit d'un estimateur (mais il est cependant probable proche de zéro)! Mais $\mathcal{S}$ est cependant nul à $\theta_0$. Nous avons alors:
	
	Dès lors:
	
	Ou sous forme vectorielle:
	
	Avec explicitement, par construction de la matrice variance-covariance (pour le cas particulier de dimension $2\times 2$ pour simplifier l'exemple):
	
	Cependant, estimées au point $\vec{\theta}_0$, les vraies valeurs $\mathcal{S}_1$ et $\mathcal{S}_2$ sont égales à zéro. La matrice variance-covariance se réduit alors à:
	
	Ou en explicitant les fonctions de partition:
	
	Nous reconnaissons ici la matrice d'information Fisher! Par conséquent:
	
	Et donc finalement:
	
	
	\subsection{Facteur de correction sur population finie}
	Maintenant démontrons un autre résultat qui nous sera indispensables dans certains tests statistiques que nous verrons plus loin.
	
	Supposons que nous avons une population de $N$ individus que nous représentons par l'ensemble $\left\lbrace1,2,...,N\right\rbrace$ et une variable aléatoire $X$ qui est donc une application de $\left\lbrace1,2,...,N\right\rbrace$ dans $\mathbb{R}$. Nous posons $x_i=X(i)$. La moyenne de $X$ est alors donnée par:
	
	Rappelons que la variance de $X$ est par définition:
	
	Considérons à présent l'ensemble $E$ des échantillons equation de taille $n$ pris dans $\left\lbrace1,2,...,N\right\rbrace$ avec $0<n<N$. Chaque individu a une probabilité d'être tiré égale à:
		
	Nous nous intéressons à la variable aléatoire $\bar{X}$ définie sur $E$ et étant égale à la moyenne de l'échantillon. Plus précisément:
	
	Afin de calculer la variance $\text{V}(\bar{X})$, nous allons exprimer $\bar{X}$ comme somme de variables aléatoires. En effet si nous définissons les variables $X_k$ avec $k=1...N$ par:
	
	Nous avons naturellement (donc de la par la définition précédente):
	
	et donc il vient:
	
	Les variables aléatoires  $X_k$ ne sont pas indépendantes deux à deux, en effet comme nous allons le voir, leurs covariances ne sont pas nulles si $N$ est fini. Dans le cas contraire (covariance nulle), nous retrouvons un résultat déjà démontré plus haut:
	
	Il nous faut donc calculer les variances $\text{V}(X)$ et les covariances $\text{cov}(X_i,X_j)$.
	
	Pour ce faire nous allons utiliser la relation de Huygens et nous allons commencer par calculer l'espérance $\text{E}(X_k)$:
	
	Or $P(X_k=x_k)$ est la probabilité qu'un échantillon contienne $k$. Cette probabilité vaut bien évidemment $n/N$ et par suite:
	
	De la même façon nous obtenons:
	
	Nous pouvons donc calculer la variance $\text{V}(X_k)$:
	
	Pour calculer les covariances nous avons à présent besoin de calculer les espérances $\text{E}(X_iX_j)$:
	
	Or  $P(X_i=x_i,X_j=x_j)$ est la probabilité qu'un échantillon contienne $i$ et $j$. Cette probabilité vaut bien évidemment:
	
	et par suite:
	
	Nous pouvons à présent calculer les covariances:
	
	Nous sommes maintenant en mesure de calculer $\text{V}(\bar{X})$:
	
	En utilisant le théorème de Huygens, nous obtenons:
	
	En utilisant le résultat démontré juste plus haut et la relation précédente:
	
	Dès lors:
	
	Pour la double somme $\displaystyle \sum_{i\neq j}^N x_ix_j$, nous avons:
	
	Dès lors:
	
	Ainsi:
	
	Le fameux facteur:
	
	que nous avons déjà rencontré lors de notre étude la loi hypergéométrique est appelé "\NewTerm{facteur de correction sur population finie}\index{facteur de correction sur population finie}" et il a pour effet de réduire l'erreur-standard d'autant plus que $n$ est grand.
	
	Si l’échantillon est relativement important (plus de $5\%$ de la population), il est parfois d'usage d'utiliser ce facteur de correction.
	
	Dans la science des Sondages, nous pouvons conclure que pour "l'échantillon aléatoire sans remplacement (EASR)" et pour "l'échantillon aléatoire avec remplacement (EAAR)", la moyenne est presque toujours donnée par:
	
	mais pour l'écart type:
	
	
	\pagebreak
	\subsection{Intervalles de Confiance (inférence)}\label{confidence interval}
	Jusqu'à présent, nous avons toujours déterminé les estimateurs de vraisemblance ou estimateurs simples (variance, écart-type) à partir de distributions statistiques théoriques ou mesures sur une population entière de données. Mais maintenant que nous avons couvert les statistiques descriptives, les probabilités et les bases de l'échantillonnage, nous sommes prêts à faire de "\NewTerm{l'inférence fréquentiste}\index{inférence fréquentiste}" et de "\NewTerm{inférence bayésienne}\index{inférence bayésienne}" à partir de notre échantillon de données pour notre population d'intérêt.
	
	Dans l'inférence fréquentiste statistique, nous prenons ce que nous savons de l'échantillon, appliquons la théorie sous-jacente de l'échantillonnage (théorème central limite) pour faire des déclarations sur notre population d'intérêt. Nous faisons des estimations sur la population en utilisant les données de l'échantillon. Les estimations peuvent être des estimations ponctuelles ou des estimations d'intervalle. Dans l'inférence statistique bayésienne, nous prenons la connaissance vraisemblable que nous avons sur les données et les a priori sur certains paramètres pour inférer une distribution postérieure ayant de nouveaux "hyperparamètres".
	
	
	\subsubsection{Inférence Fréquentiste}\label{frequentist inference}
	\textbf{Définition (\#\mydef):} Un "\NewTerm{intervalle de confiance}\index{intervalle de confiance}" (I.C.) dans l'inférence fréquentiste est une paire de nombres (dans le cas univarié) qui définit (a priori) la plage de valeurs possibles avec une certaine probabilité cumulée de la distribution d'un estimateur donné à partir d'un échantillon d'une expérience (la plage de l'indicateur statistique étant généralement calculée à l'aide de paramètres réels mesurés). En d'autres termes, l'I.C. est une collection d'intervalles avec un pourcentage donné (probabilité) d'entre eux contenant le vrai paramètre!
	
	\begin{tcolorbox}[title=Remarque,colframe=black,arc=10pt]
	L'intervalle de confiance est souvent confondu avec "\NewTerm{l'intervalle de crédibilité bayésien}\index{intervalle de cr\'edibilité bay\'esien}" qui est un intervalle dans lequel une valeur de paramètre non observée tombe avec une probabilité particulière. Il s'agit d'un intervalle dans le domaine d'une distribution de probabilité postérieure ou d'une distribution prédictive (la généralisation aux problèmes multivariés est la "région crédible").
	\end{tcolorbox}
	
	Les auteurs d'ouvrages de référence et les partisans des intervalles de confiance comblent l'écart de manière transparente en affirmant que les intervalles de confiance ont trois propriétés souhaitables: premièrement, que le coefficient de confiance peut être lu comme une mesure de l'incertitude que l'on devrait avoir pour que l'intervalle contient le paramètre; deuxièmement, la largeur de l'I.C. est une mesure de l'incertitude d'estimation; et troisièmement, que l'intervalle contient les valeurs "probables" ou "raisonnables" du paramètre. Tout cela implique un raisonnement sur le paramètre à partir des données observées: c'est-à-dire qu'il s'agit d'inférences "post-données".
	
	Ces interprétations des intervalles de confiance ne sont pas correctes. Nous appelons l'erreur que ces auteurs commettent le "\NewTerm{l'Erreur de Confiance Fondamentale}" (ECF) car elle semble découler naturellement de la définition de l'intervalle de confiance (voir \cite{perezgonzalez2017fallacy}).
	
	Le raisonnement derrière l'erreur de confiance fondamentale semble plausible: sur un échantillon donné, nous pourrions obtenir l'un des intervalles de confiance possibles. Si $95\%$ des intervalles de confiance possibles contiennent la vraie valeur, sans aucune autre information, il semble raisonnable de dire que nous avons $95 \%$ de certitude que nous avons obtenu l'un des intervalles de confiance qui contiennent la vraie valeur. Cette interprétation est suggérée par le nom "d'intervalle de confiance" lui-même: le mot "confiant", en usage profane, est étroitement lié aux concepts de plausibilité et de croyance. Le nom "intervalle de confiance" - plutôt que, par exemple, l'expression "\NewTerm{procédure de couverture}" plus précise - encourage l'Erreur de Confiance Fondamentale.
	
	Donc, si les intervalles de confiance de deux paramètres ne se chevauchent, vous ne pouvez pas conclure que ces paramètres ne sont pas significativement différents l'un de l'autre! Et c'est parce que si vous construisez cent intervalles de confiance à $95\% $, alors environ $95$ de ces intervalles contiendront la valeur de la population (vous devez toujours garder à l'esprit que vous travaillez avec des objets statistiques !!!!). Les valeurs spécifiques d'un intervalle de confiance calculé ne doivent donc pas être rigoureusement interprétées directement!
	
	En bref, voici ce dont vous devez vous souvenir lorsque vous comparez deux intervalles de confiance:
	\begin{enumerate}
		\item Lorsque deux intervalles de confiance d'un paramètre donné ne se chevauchent pas, la différence entre les deux paramètres sera significative.
		
		\item Lorsque deux intervalles de confiance d'un paramètre donné se chevauchent, la différence entre les deux paramètres peut être significative ou non significative.
	\end{enumerate}
	Gardez bien à l'esprit par exemple lorsque le paramètre est la moyenne, que les erreurs-types de moyennes ne sont pas les mêmes que les erreurs-types de différences de moyennes!
	
	\begin{figure}[H]
		\centering
		\includegraphics[scale=0.7]{img/arithmetics/confidence_interval_weather_forecasting.jpg}
		\caption[Exemple d'intervalle de confiance dans les prévisions météorologiques]{Exemple d'intervalles de confiance qui ont intéressé des millions de personnes en 2017}
	\end{figure}
	Nous passons maintenant à la tâche qui consiste naturellement à nous demander quelle doit être la taille des échantillons de nos données mesurées pour avoir une certaine validité pour nos estimateurs ou même à quel intervalle de confiance correspond un certain écart-type ou un quantile donné dans une distribution Normale réduite centrée  (pour grands échantillons), dans une distribution du khi-deux, une distribution de Student ou une distribution de Fisher (nous verrons les deux derniers cas de petits échantillons plus loin lors de notre étude de l'analyse de variance, ie ANOVA) lorsque la moyenne ou la variance sont respectivement connus ou inconnus sur tout ou partie de la population donnée.
	
	Il est important de savoir que ces intervalles de confiance utilisent souvent le théorème de la limite centrale qui sera démontré plus loin (pour éviter toute frustration possible) et les développements que nous allons faire maintenant sont également utiles dans le domaine des tests d'hypothèses (a posteriori) qui ont un rôle majeur en statistique et donc indirectement dans tous les domaines scientifiques!!!

	Enfin, il pourrait être utile d'indiquer qu'un grand nombre d'organisations (privées ou institutionnelles) font de fausses statistiques car les hypothèses et conditions d'utilisation de ces intervalles de confiance (verbatim des tests d'hypothèses) ne sont pas rigoureusement vérifiées ou simplement omises ou pire, toute la base (les mesures) n'est pas collectée dans les règles de l'art (fiabilité des protocoles de collecte et de reproductibilité des données non validée par des pairs scientifiques).
	
	Le lecteur doit également savoir que nous avons mis de nombreuses autres techniques d'intervalle de confiance, avec leurs preuves détaillées, relatives par exemple aux techniques de régression dans la section de Méthodes Numériques théorique \pageref{regression techniques}.

	\begin{tcolorbox}[title=Remarque,colframe=black,arc=10pt]
	Le praticien doit faire très attention au calcul des intervalles de confiance et à l'utilisation des tests d'hypothèse dans la pratique. C'est pourquoi, pour éviter toute erreur d'utilisation ou erreur d'interprétation, il est important de se référer aux normes internationales suivantes, par exemple: ISO 2602:1980 \textit{(Interprétation statistique des résultats des tests - Estimation de la moyenne - Intervalle de confiance)}, ISO 2854:1976 (\textit{Interprétation statistique des données - Techniques d'estimation et tests relatifs aux moyennes et variances}), ISO 3301:1975 (\textit{Interprétation statistique des données - Comparaison de deux moyennes dans le cas d'observations appariées}), ISO 3494:1976 (\textit{Interprétation statistique des données - Efficacité des tests relatifs aux moyennes et aux variances}), ISO 5479:1997 (\textit{Interprétation statistique des données - Tests pour écarts à la distribution Normale}), ISO 10725:2000 + ISO 11648-1:2003 + ISO 11648-2:2001 (\textit{Plans d'échantillonnage et procédures d'acceptation pour le contrôle des matières en vrac}), ISO 11453:1996 (\textit{Interprétation statistique des données - Essais et intervalles de confiance relatives aux proportions}), ISO 16269-4:2010 (\textit{Interprétation statistique des données - Détection et traitement des valeurs aberrantes}), ISO 16269-6:2005 (\textit{Interprétation statistique des données - Détermination des intervalles de tolérance statistique}), ISO 16269-8: 2004 (\textit{Interprétation statistique des données - Détermination des intervalles de prédiction}), ISO / TR 18532:2009 (\textit{Lignes directrices pour l'application de la qualité statistique et des normes industrielles}).
	\end{tcolorbox}
	
	\paragraph{I.C. sur la Moyenne avec Variance théorique connue}\label{ci on the mean with know variance}\mbox{}\\\\
	Commençons par le cas le plus simple et le plus courant qui est la détermination du nombre d'individus pour avoir une certaine confiance dans la moyenne des mesures d'une variable aléatoire supposée suivre une distribution Normale.

	D'abord rappelons que nous avons démontré au début de ce chapitre que l'erreur-type (écart-type à la moyenne) était sous l'hypothèses de variables indépendantes et identiquement distribuées (i.i.d.):
	
	Maintenant, avant d'aller plus loin, considérons $X$ comme une variable aléatoire suivant une loi Normale de moyenne equation et d'écart-type $\sigma$. Nous souhaiterions que la variable aléatoire ait par exemple $95\%$ de probabilité cumulée de se trouver dans un intervalle symétrique borné donné. Ce qui s'exprime donc sous la forme suivante:
	
	\begin{tcolorbox}[title=Remarque,colframe=black,arc=10pt]
	Donc avec un intervalle de confiance de $95\%$ vous aurez raison $19$ fois sur $20$, ou n'importe quel autre niveau de confiance ou niveau de risque $\alpha$ ($1$-niveau de confiance, soit $5\%$) que vous vous serez fixé à l'avance. En moyenne, vos conclusions seront donc bonnes, mais nous ne pourrons jamais savoir si une décision particulière est bonne! Si le niveau de risque est très faible mais que l'événement a quand même lieu, les spécialistes parlent alors de "\NewTerm{grande déviation}\index{grande déviation}" ou de "\NewTerm{black swan}\index{black swan}" (cygne noir). La gestion des valeurs aberrantes est traitée dans la norme ISO 16269-4:2010 \textit{Détection et traitement des valeurs aberrantes} que tout ingénieur faisant des statistiques en entreprise se doit de respecter.
	\end{tcolorbox}
	En centrant et réduisant la variable aléatoire:
	
	Notons maintenant $Y$ la variable centrée réduite:
	
	Puisque la loi Normale centrée réduite est symétrique:
	
	D'où:
	
	A partir de là en lisant dans les tables numériques de la loi Normale centrée réduite (ou en utilisant un simple tableur), nous avons pour satisfaire cette égalité que:
	
	WCe qui s'obtient facilement avec la version anglaise de Microsoft Excel 11.8346 en utilisant la fonction:
	\begin{center}
	\texttt{=-NORMSINV((1-0.95)/2)}
	\end{center}
	Ce qui est noté de façon traditionnelle dans le cas général autre que $95\%$ par ($Z$ étant la variable aléatoire correspondant donc à la moitié du quantile du seuil fixé de la loi Normale centrée réduite):
	
	Or, considérons que la variable $X$ sur laquelle nous souhaitons faire de l'inférence statistique est justement la moyenne (et nous démontrerons plus loin que celle-ci suit une loi Normale centrée réduite). Dès lors:
	
	Nous en tirons la taille de l'échantillon:
	
	dont nous prenons évidemment (normalement...) la valeur entière supérieure...
	
	Cette dernière notation est plus souvent écrite sous la forme suivante mettant mieux en évidence la largeur de l'intervalle de confiance à un niveau $\alpha$ sous-jacent:
	
	Relation appelée "\NewTerm{effectif de l'échantillon pour estimation par loi Normale}\index{effectif de l'échantillon pour estimation par loi Normale}".

	Ainsi, nous pouvons maintenant savoir le nombre d'individus à avoir pour s'assurer un intervalle de précision $\delta$ (marge d'erreur) autour de la moyenne et pour qu'un pourcentage donné des mesures se trouvent dans cet intervalle et en supposant l'écart-type théorique $\sigma$  connu (ou imposé) d'avance (typiquement utilisé dans l'ingénierie de la qualité ou les instituts de sondages/enquêtes).
	
	Dans le cas des sondages/enequêtes où la population n'est pass assez grande pour considéréer que nous avons un échantillonnage avec remise, nous parlons devons alors introduire la facteur fcp que nous avons démontré plus haut (nous parlons alors dans PSSR pour "plan de sondage sans remise"). Il vient alors en prenant l'approximation de fcp:
	
	Autrement dit, nous pouvons calculer le nombre $n$ d'individus à mesurer pour s'assurer un intervalle de confiance donné (associé au quantile $Z$) de la moyenne mesurée en supposant l'écart-type théorique connu (ou imposé) et en souhaitant un précision de $\delta$ en valeur absolue sur la moyenne.

	Cependant... en réalité, la variable $Z$ provient du théorème central limite (voir plus bas) qui donne pour un échantillon de grande taille (approximativement):
	
	En réarrangeant nous obtenons:
	
	et comme $Z$ peut être négatif ou positif alors il est plus censé d'écrire cela sous la forme:
	
	Soit:
	
	que les ingénieurs notent parfois:
	
	avec LCL étant la lower confidence limit et UCL la upper confidence limit. C'est de la terminologie Six Sigma (\SeeChapter{voir section de Génie Industriel \pageref{six sigma}}).
	
	Et nous venons de voir plus avant que pour avoir un intervalle de confiance à $95\%$ nous devrions avoir $Z=1.96$. Et puisque la loi Normale est symétrique:
	
	Nous écrivons finalement le "\NewTerm{test $Z$ à un échantillon}\index{test $Z$ à un échantillon}\label{one sample z test}":
	
	où nous définissons pour tous les tests ayant le même type de structure, la "\NewTerm{marge d'erreur}\index{marge d'erreur}" par:
	
	Si les éléments échantillonnés ne sont pas indépendants alors il faut bien évidemment écrire:
	
	où pour rappel comme prouvé précédemment:
	
	est le facteur de correction de la population finie.
	
	Comme nous l'avons déjà mentionné, et nous le démontrerons un peu plus loin, la moyenne arithmétique centrée réduite d'une séries de variables aléatoires indépendantes et identiquement distribuées de variance fini suit asymptotiquement une loi Normale centrée réduite, alors l'intervalle de confiance ci-dessus a une portée très générale! Raison pour laquelle nous parlons parfois de "\NewTerm{d'intervalle de confiance asymptotique de la moyenne}".

	Ces intervalles ont évidemment pour origine que nous travaillons très souvent en statistiques sur des échantillons et non sur toute la population disponible. L'échantillonnage choisi influe donc sur l'estimateur ponctuel. Nous parlons alors de "\NewTerm{fluctuation d'échantillonnage}".
	
	Dans le cas particulier d'un I.C. (intervalle de confiance) à $95\%$, la dernière relation s'écrit:
	
	Parfois nous retrouvons l'inégalité antéprécédente sous la forme équivalente suivante:
	
	ou encore plus rarement sous la forme générale suivante (que l'on retrouve pour toutes les intervalles):
	
	où ME signifie "\NewTerm{margin of error}\index{margin error}".
	
	Nous sommes ainsi capables maintenant d'estimer des tailles de population nécessaires à obtenir un certain niveau de confiance $\alpha$ dans un résultat, soit d'estimer dans quel intervalle de confiance se trouve la moyenne théorique en connaissant la moyenne expérimentale (empirique) et l'estimateur du maximum de vraisemblance de l'écart-type. Nous pouvons bien évidemment dès lors aussi déterminer la probabilité avec laquelle la moyenne est en dehors d'un certain intervalle... (l'un comme l'autre étant beaucoup utilisés dans l'industrie).

	Enfin, signalons que du résultat précédent, nous déduisons immédiatement par la propriété de stabilité de la loi Normale (démontrée plus haut) le test suivant que nous retrouvons dans de très nombreux logiciels de statistiques:
	
	appelé ""\NewTerm{test $Z$ bilatéral sur la différence de deux moyennes}\index{test $Z$ bilatéral sur la différence de deux moyennes}, aussi parfois nommé "\NewTerm{test $Z$ à deux échantillons}\index{test $Z$ à deux échantillons}", avec l'intervalle de confiance correspondant:
	
	Et ce n'est pas parce que deux moyennes sont significativement différentes que leurs intervalles de tolérance ne se superposent pas! Comme le montre le graphique ci-dessous obtenu avec le logiciel Minitab 16 où le test-$Z$ de la différence est significative à  $95\%$:
	\begin{figure}[H]
		\centering
		\includegraphics[scale=1]{img/arithmetics/confidence_interval_line_plot_overlap.jpg}
		\caption{ Illustration de la superposition d'intervalle de confiance à $95\%$}
	\end{figure}
	alors que leur moyenne est significativement différente à un seuil de confiance de $95\%$.
	
	Notez également que de nombreuses personnes (en particulier les non scientifiques) ne comprennent pas pourquoi une "différence moyenne pratique" n'est pas la même chose qu'une "différence statistique moyenne". Mais la raison est évidente! Le premier n'implique pas la variance et le second le fait! D'où le fait que le second est plus rigoureux !!!
	
	\begin{tcolorbox}[title=Remarque,colframe=black,arc=10pt]
	La taille de la population mère pour les relations développées plus haut n'entre pas en ligne de compte dans le calcul des intervalles de confiance ni dans celui de la taille de l'échantillon, et pour cause, elle est considérée infinie. Il faut donc faire attention à ne pas avoir parfois des tailles d'échantillons qui sont plus grandes que la population mère réelle possible...
	\end{tcolorbox}
	
	\paragraph{I.C. sur la Variance avec Moyenne théorique connue}\label{ci on the variance with known mean}\mbox{}\\\\
	Commençons par démontrer une propriété fondamentale de la loi du Khi-deux:

	Si une variable aléatoire $X$ suit une loi Normale centrée réduite $X=\mathcal{N}(0,1)$ alors son carré suit une loi du Khi-deux de degré de liberté $1$:	
	
	Ce résultat est parfois appelé "\NewTerm{statistique de Wald}\index{statistique de Wald}" et tout test statistique l'utilisant directement (on devrait plutôt parler de "famille de tests") peut être rangé sous la dénomination de "\NewTerm{test de Wald}\index{test de Wald}" (pour un exemple concret voir plus bas le test de Cochran–Mantel–Haenszel à la page \pageref{cochran mantel test}).
	\begin{dem}
	Pour démontrer cette propriété, il suffit de calculer la densité de la variable aléatoire $X^2$ avec $X=\mathcal{N}(0,1)$. Or, si $X=\mathcal{N}(0,1)$ et si nous posons $Y=X^2$, alors pour tout $y \geq 0$ nous obtenons:
	
	Puisque la loi Normale centrée réduite est symétrique par rapport à 0 pour la variable aléatoire $X$, nous pouvons écrire:
	
	En notant $\Phi$ la fonction de répartition de la loi Normale centrée réduite (sa probabilité cumulée en d'autres termes pour rappel...), nous avons:
		
	et comme:
	
	alors:
	
	La fonction de répartition de la variable aléatoire (probabilité cumulée) $Y=X^2$ est donc donnée par:
	
	si $y$ est supérieur ou égal à zéro, nulle si $y$ inférieur à zéro. Nous noterons cette répartition $f_Y(y)$ pour la suite des calculs.
	
	Puisque la fonction de distribution est la dérivée de la fonction de répartition et que $X$ suit une loi Normale centrée réduite alors nous avons pour la variable aléatoire $X$:
	
	et il s'ensuit pour la loi de distribution de $Y$ (qui est donc le carré de $X$ pour rappel!):
	
	cette dernière expression correspond exactement à la relation que nous avions obtenue lors de notre étude de la loi du Khi-deux en imposant un degré de liberté unité.

	Le théorème est donc bien démontré, à savoir que si $X$ suit une loi Normale centrée réduite alors son carré suit une loi du Khi-deux à $1$ degré de liberté tel que:
	
	\begin{flushright}
		$\blacksquare$  Q.E.D.
	\end{flushright}	
	\end{dem}
	Ce type de relation est utilisé dans les processus industriels et leur contrôle (\SeeChapter{voir section de Génie Industriel page \pageref{quality control charts}}).
	
	Ouvrons maintenant une parenthèse qui est assez importante dans certains rapports  de régression linéaire donnés par certains logiciels de statistiques et en particulier dans le test de courbure de Fisher dans le domaine des Plans d'Expérience (\SeeChapter{voir section Génie Industriel page \pageref{doe}}). Rappelons que nous avons:
	
	Et nous venons de prouver ci-dessus que:
	
	Dès lors:
	
	Et comme nous l'avons également vu
	
	il s'ensuit que:
	
	ou plus couramment dans la pratique:
	
	Nous allons maintenant utiliser un résultat démontré lors de notre étude de la loi Gamma. Nous avons effectivement vu plus haut que la somme de deux variables aléatoires suivant une loi Gamma suit aussi une loi Gamma dont les paramètres s'additionnent:
	
	Comme la loi du Khi-deux n'est qu'un cas particulier de la loi Gamma, le même résultat s'applique.

	Pour être plus précis, cela revient à dire: Si $X_1,...,X_k$ sont des variables aléatoires indépendantes (!) et identiquement distribuées  $\mathcal{N}(0,1)$ alors par extension de la démonstration précédente où nous avons montré que:	
	
	et de la propriété d'addition de la loi Gamma, la somme de leurs carrés suit alors une loi du Khi-deux de degrés de liberté $k$ telle que:
	
	Ainsi, la loi du $\chi^2$ à $k$ degrés de liberté est la loi de probabilité de la somme des carrés de $k$ variables normales centrées réduites linéairement indépendantes entre elles. Il s'agit de la propriété de linéarité de la loi du Khi-deux (implicitement de la linéarité de la loi Gamma)!

	Maintenant voyons une autre propriété importante de la loi du Khi-deux: Si $X_1,...,X_n$ sont des variables aléatoires indépendantes et identiquement distribuées $\mathcal{N}(\mu,\sigma)$ (donc de même moyenne et même écart-type et suivant une loi Normale) et si nous notons l'estimateur du maximum de vraisemblance de la variance:
	
	alors, le rapport de la variable aléatoire $S_*^2$ sur l'écart-type supposé connu de l'ensemble de la population (dit "écart-type vrai" ou "écart-type théorique" pour bien différencier!) multiplié par le nombre d'individus  $n$ de la population suit une loi du Khi-deux de degré  $n$ telle que:
	
	Ce résultat est appelé "\NewTerm{théorème de Cochran}\index{théorème de Cochran}\label{Cochran theorem}" ou encore "\NewTerm{théorème de Fisher-Cochran}\index{théorème de Fisher-Cochran}" (dans le cas particulier d'échantillons gaussiens) et nous donne donc une distribution pour les écarts-types empiriques (dont la loi parente est une loi Normale).

	En utilisant la valeur de l'écart-type démontrée lors de notre étude da la loi du khi-deux nous avons donc:
	
	Mais $n$ et $\sigma$ sont imposés et sont donc considérés comme des constantes. Il vient alors:
	
	Et dès lors nous avons une expression de l'écart-type de l'écart-type empirique si nous connaissons l'écart-type de la population:
	
	Mais nous avons démontré lors de notre étude des estimateurs que:
	
	Dès lors il vient que:
	
	Il en découle donc la relation parfois importante dans la pratique de l'estimateur de l'écart-type de.... l'écart-type:
	
	Rappelons que la population parente est dite "infinie" si le tirage de l'échantillon est avec remise ou encore si la taille $N$ de la population parente est très supérieure à celle de $n$ de l'échantillon.
	
	\begin{tcolorbox}[title=Remarques,colframe=black,arc=10pt]
	\textbf{R1.} En laboratoire, les $X_1,...,X_n$ peuvent être vus comme une classe d'individus d'un même produit étudié identiquement par différentes équipes de recherche avec des instruments de même précision (écart-type de mesure identique).\\
	
	\textbf{R2.} $S_{*}^2$ est la "\NewTerm{variance interclasse}\index{variance interclasse}" également appelée "\NewTerm{variance expliquée}\index{variance expliquée}". Donc elle donne la variance d'une mesure ayant eu lieu dans les différents laboratoires.
	\end{tcolorbox}
	Ce qui est intéressant c'est qu'à partir du calcul de la loi du Khi-deux en connaissant $n$ et l'écart-type $\sigma^2$ il est possible d'estimer cette variance (écart-type) interclasse.

	Pour voir que cette dernière propriété est une généralisation élémentaire de la relation:
	
	il suffit de constater que la variable aléatoire $nS_*^2/\sigma^2$ est une somme de $n$ carrés de $\mathcal{N}(0,1)$ indépendants les uns des autres. Effectivement, rappelons qu'une variable aléatoire centrée réduite (voir notre étude de la loi Normale) est donnée par:
	
	Dès lors:
	
	Or, puisque les variables aléatoires $X_1,...,X_n$ sont indépendantes et identiquement distribuées selon une loi Normale, alors les variables aléatoires:
	
	sont aussi indépendantes et identiquement distribuées mais selon une loi Normale centrée réduite.
	
	Puisque:
	
	en réarrangeant nous obtenons:
	
	Donc sur la population de mesures, l'écart-type vrai suit la relation donnée ci-dessus. Il est donc possible de faire de l'inférence statistique sur l'écart-type lorsque la moyenne théorique est connue (...).
	
	Puisque la fonction du Khi-deux n'est pas symétrique, la seule possibilité pour faire l'inférence c'est de faire appel au calcul numérique et nous noterons alors l'intervalle de confiance à $95\%$ (par exemple...) de la manière suivante:
	
	Soit en notant $95\%=1-\alpha$:
	
	le dénominateur étant alors bien évidemment le quantile de la loi du khi-2. Cette relation est rarement utilisée dans la pratique car la moyenne théorique n'est pas connue. Indiquons, aussi, qu'afin d'éviter toute confusion, cette dernière relation est souvent notée sous la forme suivante:
	
	Voyons donc le cas le plus courant:
	
	\paragraph{I.C. sur la Variance avec Moyenne empirique connue}\label{ci on the variance with empirical mean}\mbox{}\\\\
	Cherchons maintenant à faire de l'inférence statistique lorsque la moyenne théorique de la population $\mu$ n'est pas connue. Pour cela, considérons maintenant la somme:
	
	où pour rappel $\bar{X}$ est la moyenne empirique (arithmétique) de l'échantillon:
	
	En continuant le développement nous avons:
	
	Or, nous avons démontré au début de ce chapitre que la somme des écarts à la moyenne était nulle. Donc:
	
	et reprenons l'estimateur sans biais de la loi Normale (nous changeons de notation pour respecter les traditions et bien différencier la moyenne empirique de la moyenne théorique):
	
	Dès lors:
	
	ou autrement écrit:
	
	Puisque le deuxième terme (au carré) suit une loi Normale centrée réduite aussi, alors si nous le supprimons nous obtenons de par la propriété démontrée plus haut de la loi du Khi-deux:
	
	Ces développements nous permettent cette fois-ci de faire aussi de l'inférence sur la variance d'une loi  $\mathcal{N}(0,1)$ lorsque les paramètres $\mu$ et $\sigma$ sont tous les deux inconnus pour l'ensemble de la population. C'est ce résultat qui nous donne, par exemple, l'intervalle de confiance:
	
	lorsque la moyenne théorique $\mu$ est donc inconnue. Et à aussi, pour éviter tout confusion, il est plutôt d'usage d'écrire:
	
	De la même manière que plus haut, nous pouvons calculer l'écart-type de l'écart-type et qui a une grande importance dans la pratique de la finance:
	
	
	
	\paragraph{I.C. sur la Moyenne avec Variance empirique connue}\mbox{}\\\\
	Nous avons démontré beaucoup plus haut que la loi de Student provenait de la relation suivante:
	
	si $Z$ et $U$ sont des variables aléatoires indépendantes et si $Z$ suit une loi Normale centrée réduite $\mathcal{N}(0,1)$ et $U$ une loi du Khi-deux $\chi^2(k)$ tel que:
	
	et rappelons que la fonction de densité (distribution) est symétrique!
	
	Voici une application très importante du résultat ci-dessus:
	
	Supposons que $X_1,...,X_n$ constituent un échantillon aléatoire de taille n issu de la loi $\mathcal{N}(\mu,\sigma)$. Alors nous pouvons déjà écrire que selon les développements faits plus haut:
	
	Et pour $U$ qui suit une loi $\chi^2(k)$, si nous posons equation alors selon les résultats obtenus plus haut:
	
	Nous avons alors après quelques simplifications triviales:
	
	Donc puisque:
	
	suit une loi de Student de paramètre $k$ alors nous obtenons le "\NewTerm{independent one-sample $t$-test}\index{independent one-sample $t$-test}" (en anglais) ou "\NewTerm{test-$T$ à $1$ échantillon}\index{test-$T$ à $1$ échantillon}":
	
	qui suit aussi une loi de Student de paramètre $n-1$ et qui est très utilisé dans les laboratoires pour les tests d'étalonnages\footnote{L'ingénieur ne doit pas oublier qu'un instrument calibré, ne veut pas dire qu'il mesure correctement! Cependant cela signifie qu'il a des valeurs d'incertitude connues sur ses mesures par rapport à une référence ou une spécification!}.

	Ce qui nous donne aussi après réarrangement:
	
	Ce qui nous permet de faire de l'inférence sur la moyenne $\mu$ d'une loi Normale d'écart-type théorique inconnu (sous-entendu qu'il n'y a pas assez de valeurs expérimentales) mais dont l'estimateur sans biais de l'écart-type est connu. C'est ce résultat qui nous donne l'intervalle de confiance\label{student confidence interval of the mean}:
	
	où nous retrouvons les mêmes indices que pour l'inférence statistique sur la moyenne (espérance) d'une variable aléatoire d'écart-type (théorique) connu puisque la loi de Student tend asymptotiquement pour de grandes valeurs de $n$ vers une loi Normale. Ainsi, l'intervalle précédent et l'intervalle suivant:
	
	donneront des valeurs très proches (à la troisième décimale) pour des grandeurs de $n$ aux alentours des $10'000$ (dans la pratique on considère qu'à partir de 100 c'est identique...).
	
	Nous déduisons immédiatement par la propriété de stabilité de la loi du Khi-deux (démontrée plus haut par le fait qu'elle découle de la loi Gamma) le test suivant que nous retrouvons dans de très nombreux logiciels de statistiques:
	
	appelé "\NewTerm{test-$T$ (de Student) bilatéral sur la différence de deux moyennes}\index{test-$T$ (de Student) bilatéral sur la différence de deux moyennes}" ou plus simplement "\NewTerm{test-$T$ à deux échantillons}\index{test-$T$ à deux échantillons}" (rigoureusement... sommer les degrés de liberté comme nous venons de le faire n'est valable que si les deux variances sont égales et nous démontrerons le cas général où les variances ne sont pas égales lors de la démonstration du test de Welch plus loin).

	Nous pouvons bien évidemment dès lors aussi déterminer la probabilité avec laquelle la moyenne est dedans ou en dehors d'un certain intervalle... (l'une comme l'autre étant beaucoup utilisées dans l'industrie).

	Le lecteur peut pour plaisir contrôler avec Microsoft Excel 11.8346 que pour un grand nombre de mesures $ n $, la distribution de  Student tend vers la distribution Normale centrée réduite en comparant les valeurs des deux fonctions ci-dessous:
	\begin{center}
	\texttt{=LOI.STUDENT.INVERSE.N(5\%/2,n-1)}\\
	\texttt{=LOI.NORMALE.STANDARD.INVERSE.N(5\%/2)}
	\end{center}
	
	\begin{tcolorbox}[title=Remarque,colframe=black,arc=10pt]
	Le résultat précédent fut obtenu par William S. Gosset aux alentours de 1910. Gosset qui avait étudié la mathématique et la chimie, travaillait comme statisticien pour la brasserie Guinness en Angleterre. À l'époque, on savait que si $X_1,...,X_n$ sont des variables aléatoires indépendantes et identiquement distribuées alors:
	
	Toutefois, dans les applications statistiques on s'intéressait bien évidemment plutôt à la quantité:
	
	On se contentait alors de supposer que cette quantité suivait à peu près une loi Normale centrée réduite ce qui n'était pas une mauvaise approximation comme le montre l'image ci-dessous ($\mathrm{d}f=n-1$):
	\begin{figure}[H]
		\centering
		\includegraphics{img/arithmetics/comparison_normal_student_distributions.jpg}
		\caption{Comparaison entre la fonction de distribution Normale et celle de Student}
	\end{figure}
	Suite à de nombreuses simulations, Gosset arriva à la conclusion que cette approximation était valide seulement lorsque n est suffisamment grand (donc cela lui donnait l'indication comme quoi il devait y avoir quelque part derrière le théorème central limite). Il décida de déterminer l'origine de la distribution et après avoir suivi un cours de statistique avec Karl Pearson il obtint son fameux résultat qu'il publia sous le pseudonyme de Student. Ainsi, on appelle loi de Student la loi de probabilité qui aurait dû être appelée la "loi ou fonction de Gosset".
	\end{tcolorbox}
	Signalons enfin que le test-$T$ de Student est aussi très utilisé pour identifier si des variations (progressions ou l'inverse) de la moyenne des chiffres de deux populations identiques sont statistiquement significatives. C'est-à-dire que si la taille de deux échantillons dépendants est identique alors nous pouvons créer le test suivant (nous avons indiqué tous les différents types d'écritures que l'on peut retrouver dans la littérature et dans les nombreux logiciels implémentant ce test):	
	
	avec:
	
	La relation antéprécédente est donc très utile pour comparer deux fois le même échantillon dans des situations différentes de mesure (ventes avant ou après rabais d'un article par exemple). La relation antéprécédente est appelée "\NewTerm{test-$T$ (de Student) de deux moyennes d'\'echantillons appari\'es (ou \'echantillons d\'ependants)}"  ou plus simplement "\NewTerm{test-$T$ de Student pour \'echantillons appari\'es}\index{test-$T$ de Student pour \'echantillons appari\'es}".

	\textbf{Définition (\#\mydef):} Nous parlons "\NewTerm{d'\'echantillons appari\'es}\index{\'echantillons appari\'es}" (par paires) si les échantillons de valeurs sont prises $2$ fois sur les mêmes individus (donc les valeurs des paires ne sont pas indépendantes, contrairement à deux échantillons pris indépendamment).
	
	\paragraph{Test Binomial exact}\mbox{}\\\\
	Il arrive fréquemment lors de mesures que l'on souhaite comparer si deux échantillons de petite taille pris au hasard (sans remise!) d'une population elle aussi petite... sont statistiquement significativement différents ou non alors que l'on attendait une égalité parfaite!

	Nous cherchons donc un test adapté aux cas suivants:

	\begin{itemize}
		\item Savoir si un échantillon d'une population préfère utiliser une technique de travail plutôt qu'une autre alors que l'on s'attend à ce que la population utilise autant l'une que l'autre

		\item  Savoir si un échantillon d'une population a une caractéristique prédominante parmi deux possibilités alors que l'on s'attend à ce que la population soit parfaitement équilibrée
	\end{itemize}

	Avant d'aller plus en détails, rappelons qu'il faut être extrêmement prudent quant à la manière d'obtenir les deux échantillons. Il faut que l'expérience soit non biaisée, cela signifie pour rappel, que le protocole de tirage ne doit en aucun cas avantager l'une au l'autre des caractéristiques de la population (si vous étudiez l'équilibre homme/femme dans une population en attirant dans le sondage des personnes grâce à un cadeau sous la forme de bijoux ou en appelant pandans les jours ouvrés vous aurez alors un échantillon biaisé... car vous aurez probablement naturellement plus de femmes que d'hommes...).
	
	Ceci étant dit, cette situation correspond donc à une loi binomiale pour laquelle nous avons démontré plus haut dans ce chapitre que la probabilité de $k$ réussites pour une population de taille $N$ dont la probabilité de réussite est $p$ (et la probabilité d'échec $q$ donc de $1 - p$) était donnée par la relation:
		
	Dans le cas qui nous intéresse, nous avons donc $p=q=0.5$:
	
	tout en se rappelant que la distribution ne sera pas pour autant symétrique et ce surtout si la taille $N$ de la population est petite.

	Si nous notons maintenant $x$ le nombre de réussites (considéré comme la taille du premier échantillon) et $y$ le nombre d'échecs (considéré comme la taille du deuxième échantillon), nous avons alors:
	
	Ceci étant fait, pour construire le test et de par l'asymétrie de la distribution, nous allons calculer la probabilité cumulée que $k$ soit plus petit que le $x$ obtenu par l'expérience et la sommer à la probabilité cumulée pour que $k$ soit plus grand que le $y$ obtenu par l'expérience (ce qui correspond à la probabilité cumulée des queues respectivement gauche et droite de la distribution). Cette somme correspond donc à la probabilité:
	
	et cette dernière relation est appelée "\NewTerm{test binomial exact (bilat\'ral)}\index{test binomial exact (bilat\'ral)}".

	Si la probabilité  $P$ obtenue pour la somme est au-dessus d'une certaine probabilité cumulée fixée à l'avance, nous dirons alors que la différence avec un échantillon tiré au hasard dans une population parfaitement équilibrée n'est pas statistiquement significative (en bilatéral...) et respectivement si elle est en-dessous, la différence sera donc statistiquement significative et nous rejetterons l'équilibre supposé

	Ainsi,si:
	
	la différence par rapport à une population équilibrée sera considérée comme non statistiquement significative. Souvent on prendra au maximum $\alpha$ comme valant $5\%$ (mais rarement en-dessous) ce qui correspond donc à un intervalle de confiance de $95\%$.

	Malheureusement d'un logiciel de statistiques à l'autre les paramètres demandés ou les résultats obtenus ne seront pas nécessairement les mêmes (les tableurs n'intègrent pas de fonction spécifique pour le test binomial, il faudra souvent construire un tableau ou programmer soi-même la fonction). Par exemple, certains logiciels calculent systématiquement et imposent (ce qui est assez logique dans un sens...):
	
	\begin{tcolorbox}[colframe=black,colback=white,sharp corners]
	\textbf{{\Large \ding{45}}Exemple:}\\\\
	D'une petite population ayant deux caractéristiques $x$ et $y$ particulières qui nous intéressaient et pour laquelle nous nous attendions à avoir un parfait équilibre tel que equation mais nous avons en réalité obtenu $x=5$ et $y=7$. Nous souhaiterions faire le calcul avec Microsoft Excel 11.8346 pour savoir si cette différence est statistiquement significative ou non à un niveau de $5\%$?\\

	Pour répondre à cette question, nous allons donc calculer la probabilité cumulée:
	
	ce qui nous donne:
	\begin{figure}[H]
		\centering
		\includegraphics[scale=0.56]{img/arithmetics/binomial_coefficient_calculated.jpg}
		\caption[]{Valeurs du calcul des coefficients binomiaux dans Microsoft Excel 11.8346}
	\end{figure}
	soit explicitement:
	\begin{figure}[H]
		\centering
		\includegraphics[scale=0.56]{img/arithmetics/binomial_coefficient_calculated_explicit.jpg}
		\caption[]{Formules du calcul des coefficients binomiaux dans Microsoft Excel 11.8346}
	\end{figure}
	donc la probabilité cumulée étant de $0.774$ (soit $77.4\%$) la différence par rapport à une population équilibrée sera considérée donc comme non statistiquement significative.
	\end{tcolorbox}
	\begin{tcolorbox}[title=Remarque,colframe=black,arc=10pt]
	Ce test est également utilisé par la majorité des logiciels de statistiques (comme Minitab) pour donner un intervalle de confiance de la conformité d'opinions par rapport à celle d'un expert. C'est ce que nous appelons une étude R\&R (reproductabilité \& répétabilité) par attributs (voir mon livre sur Minitab pour un exemple).
	\end{tcolorbox}	
	
	\paragraph{I.C. pour une Proportion}\mbox{}\\\\
	Indiquons que certains statisticiens utilisent le fait que la loi Normale découle de la loi de Poisson qui elle-même découle de la loi Binomiale (nous l'avons démontré lorsque $n$ tend vers l'infini et que $p$ et $q$ sont du même ordre) pour faire un intervalle de confiance dans le cadre de l'analyse de proportions (très utilisé dans l'analyse de la qualité dans les industries).

	Pour voir cela, notons $X_i$ la variable aléatoire définie par:
	
	où l'attribut $A$  peut être la propriété "défectueux" ou "non défectueux" par exemple pour une analyse de pièces. Nous noterons $k$ le nombre de réussites de l'attribut $A$.

	La variable aléatoire $X=X_1+X_2+...+X_n$ nous l'avons démontré au début de ce chapitre, suit une loi Binomiale de paramètres $n$ et $p$ avec les moments:
	
	Ceci étant, nous ne connaissons pas la valeur vraie de $p$. Nous allons donc utiliser l'estimateur de la loi Binomiale démontré plus haut:
	
	D'après les propriétés de l'espérance nous avons alors:
	
	Et nous avons d'après les propriétés de la variance, la relation suivante pour la variance de la moyenne empirique de la proportion:
	
	Ce qui nous amène alors à:
	
	Maintenant rappelons enfin que nous avons démontré que la loi Normale découlait de la loi Binomiale sous certaines conditions (les praticiens admettent que c'est applicable tant que $n>50$ et $np \geq 5$). Autrement dit, que la variable aléatoire $X$ suivant une loi Binomiale suit une loi Normale sous certaines conditions. Évidemment, si $X$ suit une loi Normale alors $X/n$ aussi (et donc $\hat{p}$...). Dès lors nous pouvons centrer et réduire $\hat{p}$ afin qu'il se comporte comme la variable aléatoire Normale centrée réduite notée $Z$:
	
	\begin{tcolorbox}[colframe=black,colback=white,sharp corners]
\textbf{{\Large \ding{45}}Exemples:}\\\\
	E1. Si $5\%$ de la production annuelle d'une entreprise est défectueuse, quelle est la probabilité qu'en prenant un échantillon de $75$ pièces de la ligne de production que seulement $2\%$ ou moins soit défectueux?\\
	
	Nous avons dès lors avec:
	
	La probabilité cumulée correspondante à cette valeur de la variable aléatoire est avec la version anglaise de Microsoft Excel 11.8346:

	\begin{center}
	\texttt{=NORMSDIST(-1.19)=11.66\%}
	\end{center}
	Mais remarquez que nous n'avons pas $np\geq 5$ qui est satisfait donc normalement il est exclu d'utiliser ce résultat.\\
	
	E2. Dans son rapport de 1998, la banque J.P. Morgan a expliqué que durant l'année 1998 ses pertes allèrent au-delà de la Value at Risk (\SeeChapter{voir section d'Économie page \pageref{value at risk}}) $20$ jours sur les $252$ jours ouvrés de l'année en se basant sur une VaR temporelle de $95\%$ (donc $5\%$ des journées ouvrées considérées comme à perte). Au seuil de $95\%$ est-ce de la malchance ou est-ce que le modèle de VaR utilisé était mauvais?
	
	Donc c'était juste de la malchance.
	\end{tcolorbox}
	Nous pouvons maintenant approximer l'intervalle de confiance pour la proportion en se basant sur la loi Binomiale et son comportement asymptotiquement Normal dans les conditions démontrées lors de notre introduction de la loi Normale tel que nous avons le "\NewTerm{test $Z$ à une proportion}\index{test $Z$ à une proportion}" ou "\NewTerm{test $p$ à une proportion}\index{test $p$ à une proportion}" :
	
	Avant de passer à un exemple, il est peut-être utile de préciser au lecteur que cette approximation par une loi Normale est très courante et que nous allons la rencontrer encore de nombreuse fois dans les démonstrations qui vont suivre. C'est tellement courant qu'on a même donné un nom à cette méthode...: la "\NewTerm{méthode de Wald}\index{méthode de Wald}" (bon en réalité il y a plusieurs méthodes de Wald mais c'est la plus connue que nous utiliserons à chaque fois).
	
	\begin{tcolorbox}[colframe=black,colback=white,sharp corners]
\textbf{{\Large \ding{45}}Exemple:}\\\\
	Prenons $\alpha=5\%$, nous avons alors:
	
	C'est-à-dire:
	
	Sur une production de $300$ éléments nous en avons trouvé $8$ qui étaient défectueux. Quel est donc l'intervalle de confiance?\\

	Nous vérifions d'abord avec:
	
	que:
	
	Donc il est acceptable d'utiliser l'intervalle de confiance par la loi Normale. Nous avons dès lors:
	
	C'est-à-dire:
	
	\end{tcolorbox}
	Pour clore ce sujet, nous pouvons évidemment nous intéresser aussi au nombre d'individus (taille d'échantillon) qu'il faut avoir pour satisfaire une certaine précision d'intervalle de confiance (imposé) en ayant un écart-type imposé.

	Nous avons donc selon les hypothèses susmentionnées et dans l'acceptation de l'approximation par une loi Normale que:
	
	Et en procédant de manière identique aux développements effectués plus haut avec la loi Normale, nous obtenons "\NewTerm{l'effectif de l'échantillon pour estimation par loi binomiale}":
	
	dont nous prenons évidemment normalement la valeur entière supérieure dans la pratique...

	Une question qui revient souvent dans la pratique concerne le fait de savoir s'il faut appliquer ce test en unilatéral ou bilatéral. Au fait il n'y a pas de réponse précise, tout dépend de ce que nous cherchons à mettre en évidence.
	
	\begin{tcolorbox}[title=Remarque,colframe=black,arc=10pt]
	La taille de la population mère pour les relations développées plus haut n'entre pas en ligne de compte dans le calcul des intervalles de confiance ni dans celui de la taille de l'échantillon, et pour cause, elle est considérée infinie. Il faut donc faire attention à ne pas avoir parfois des tailles d'échantillons qui sont plus grandes que la population mère réelle possible... Sinon, nous devez insérer le fpc (facteur de correction de population finie).
	\end{tcolorbox}
	Le lecteur peut trouver dans la plupart des manuels de marketing les valeurs tabulées (incluant le fpc dans la résultat ci-dessus) et qui conduisent à la construction du tableau suivant:
	\begin{table}[H]
		\centering
		\begin{tabular}{|r|ccc|ccc|}
		\hline & \multicolumn{3}{|c|} { Niveau de confiance $=95 \%$} & \multicolumn{3}{c|} { Niveau de confiance $=99 \%$} \\
		\hline & \multicolumn{3}{|c|} { Marge d'erreur } & \multicolumn{3}{|c|} { Marge d'erreur  } \\
		\hline Taille de la population & $5 \%$ & $2,5 \%$ & $1 \%$ & $5 \%$ & $2,5 \%$ & $1 \%$ \\
		\hline 100 & 80 & 94 & 99 & 87 & 96 & 99 \\
		500 & 217 & 377 & 475 & 285 & 421 & 485 \\
		1.000 & 278 & 606 & 906 & 399 & 727 & 943 \\
		10.000 & 370 & 1.332 & 4.899 & 622 & 2.098 & 6.239 \\
		100.000 & 383 & 1.513 & 8.762 & 659 & 2.585 & 14.227 \\
		500.000 & 384 & 1.532 & 9.423 & 663 & 2.640 & 16.055 \\
		1.000 .000 & 384 & 1.534 & 9.512 & 663 & 2.647 & 16.317 \\
		\hline
		\end{tabular}
		\caption{Taille de l'échantillon d'enquête à une question avec facteur fpc}
	\end{table}
	\begin{tcolorbox}[colframe=black,colback=white,sharp corners]
	\textbf{{\Large \ding{45}}Exemple:}\\\\
	Nous souhaiterions savoir le nombre d'individus (taille d'échantillon) à prendre d'un lot de production sachant que la proportion de défectueux est imposée à $30\%$ avec une erreur tolérée d'environ $5\%$ entre la proportion réelle et empirique et ce afin d'obtenir un intervalle de confiance à un niveau de $95\%$ du résultat:
	
	\end{tcolorbox}
	\begin{tcolorbox}[title=Remarque,colframe=black,arc=10pt]
	La dernière relation est très très souvent utilisée en théorie des sondages (analyses pour des votations avec réponses de type: Oui/Non) où parfois la taille de l'échantillon $n$ est imposée pour des raisons de coûts du sondage et dont nous cherchons à calculer l'incertitude $\delta$ et parfois l'inverse (l'incertitude est imposée et donc nous cherchons à connaître la taille de l'échantillon).
	\end{tcolorbox}
	
	\pagebreak
	\subparagraph{Test de l'égalité de deux Proportions}\mbox{}\\\\
	Toujours dans le même contexte que l'approximation précédente de la loi Binomiale par une loi Normale, l'industrie (en particulier la biostatistique) est friande de comparer deux proportions de deux populations différentes afin de savoir si elles sont statistiquement égales ou non (autrement dit: statistiquement significativement différentes ou pas).

	Dès lors rappelons que nous avons démontré la stabilité de la loi Normale si deux variables aléatoires étaient indépendantes et identiquement distribuées (selon une loi Normale donc!):
	
	Dans le cadre des hypothèses susmentionnées il en est alors de même approximativement pour la différence de deux proportions:
	
	Dès lors nous savons que cette nouvelle variable centrée réduite suit une loi Normale selon:
	
	et comme nous cherchons à savoir la probabilité cumulée que l'espérance théorique de la différence est nulle, cette dernière relation se réduit alors dans ce cas à:
	
	Évidemment nous pouvons aussi construire (comme toujours...) un intervalle de confiance à partir de cette relation.
	
	\begin{tcolorbox}[colback=red!5,borderline={1mm}{2mm}{red!5},arc=0mm,boxrule=0pt]
	\bcbombe Attention! Le test de la différence de deux proportions de deux échantillons différents n'est évidemment pas le même que le test de la différence de deux proportions dans un même échantillon (covariance oublige puisque les deux proportions ne sont dès lors plus indépendantes)! Dans le dernier cas nous utilisons le test de McNemar (voir plus bas page \pageref{mcnemar test}).
	\end{tcolorbox}
	Il semblerait cependant que cette dernière relation approximative serait d'après l'expérience plus correcte en prenant pour dénominateur:
	
	où $\hat{p}$ sera pris comme le mélange de deux populations. C'est-à-dire:
	
	soit (en changeant la notations des indices des proportions expérimentales):
	
	Ce test est appelé "\NewTerm{test $Z$ de l'égalité de deux proportions}\index{test $Z$ de l'égalité de deux proportions}" ou plus simplement "\NewTerm{test $p$ de deux proportions}\index{test $p$ de deux proportions}". En médecine, on appelle cela le "\NewTerm{test des différences de risque}\index{test des différences de risque}" (en sous-entendant que chaque proportion est une catégorie de population étudiée par rapport à un événement indésirable).
	\begin{tcolorbox}[colframe=black,colback=white,sharp corners]
	\textbf{{\Large \ding{45}}Exemple:}\\\\
	Dans le cadre d'un plan d'échantillonnage  (\SeeChapter{voir section Génie Industriel page \pageref{sampling plans}}) nous avons prélevé sur un premier lot de $50$ individus, $48$ en parfait états. Dans un second lot de $30$ individus, $26$ étaient en bon état.\\

	Nous avons donc:
	
	Nous souhaiterions donc savoir si la différence est statistiquement significative avec une certitude de $95\%$ ou simplement due au hasard. Nous utilisons alors:
	
	et:
	
	Ce qui correspond à une probabilité cumulée en utilisant la version anglaise de Microsoft Excel 11.8346 de:
	\begin{center}
		\texttt{=NORMSDIST(1.535)=93.77\%}
	\end{center}

	Donc la différence est due au hasard (ceci dit c'est presque in extremis...). Autrement dit, elle n'est pas statistiquement significative sous les contraintes énoncées.
	\end{tcolorbox}
	
	\pagebreak
	\paragraph{Test des signes}\mbox{}\\\\
	Nous mesurons quelque chose sur un échantillon puis, plus tard, nous mesurons la même chose sur ce même échantillon mais avec une autre méthode (donc il s'agit donc d'échantillons appariés!). Les deux classements ordonnées des mesures sont comparés et chaque observation est affectée d'un signe ("$+$" en cas d'élévation dans le classement, "$–$" en cas de descente). Celles qui restent au même niveau sont éliminées.
	
	Selon l'hypothèse à tester, il y a autant de "$+$" que de "$–$", c'est-à-dire que la médiane de la distribution n'a pas bougé (cette affirmation peut ne pas paraître évidente à la première lecture il faut donc bien prendre du temps parfois pour réfléchir là-dessus).
	
	L'idée étant que pour chaque couple de valeurs, il n'y a que deux signes possibles de variations, nous avons une chance sur deux ($50\%$ de probabilité) que la différence soit positive ou négative. Ce test est donc basé uniquement sur l'étude des signes des différences observées entre les paires d'individus, quelles que soient les valeurs de ces différences.

	Nous pouvons alors souhaiter contrôler deux hypothèses:
	\begin{itemize}
		\item L'inégalité des proportions de signes doit être statistiquement significative. Donc l'un deux signes doit être en petit nombre par rapport à l'autre, ce qui correspond à un test unilatéral gauche (la probabilité cumulée d'avoir ce petit nombre de signes doit être inférieur à un niveau $\alpha$ donné).
		
		\item La proportion des deux signes doit être faiblement déséquilibrée $(P(+)=P(-)=0.5)$. Il s'agit donc dans ce cas d'un test en bilatéral (c'est le cas le plus courant) avec un certain niveau $\alpha$ donné.
	\end{itemize}

	Pour pouvoir créer un tel test, nous allons considérons l'apparition des "$+$" et des "$-$" comme un système de tirage aléatoire binaire dont l'ordre des succès n'est pas pris en compte (il s'agit donc d'une loi binomiale ou hypergéométrique) et avec remise (ce qui élimine d'emblée la loi hypergéométrique qui n'est pas symétrique et pose des problèmes d'utilisation dans la pratique...). Pour considérer un tirage aléatoire avec remise (alors qu'on ne fait pas réellement de remise), il faut que la population N soit grande. Raison pour laquelle le test des signes considère que les valeurs appariées doivent être continues (ce qui permet in extenso d'approcher la loi hypergéométrique par la loi binomiale). Cependant certains logiciels de statistiques utilisent la loi hypergéométrique pour des soucis de précision.
	
	\begin{tcolorbox}[title=Remarque,colframe=black,arc=10pt]
	Il faut savoir que la majorité des logiciels de statistiques, font implicitement l'hypothèse lors de ce test que les données sont continues et utilisent la loi binomiale.
	\end{tcolorbox}
	\begin{tcolorbox}[colframe=black,colback=white,sharp corners]
	\textbf{{\Large \ding{45}}Exemple:}\\\\
	Considérons deux séries de mesures avec deux méthodes différentes. Nous souhaiterions tester l'hypothèse avec un niveau $\alpha$ de $5\%$ si la différence entre les deux méthodes est statistiquement significative (nous nous attendons donc à une équilibre des signes). Il s'agit donc d'un test des signes à deux échantillons (sachant qu'il est possible de faire la même chose en comparant les valeurs d'un seul et unique échantillon à sa médiane):
	
	Nous avons donc les différences:
	
	Avec les signes:
	
	Bon il est déjà clair que le résultat va être le rejet de l'hypothèse comme quoi il n'y pas de différence. Mais faisons quand même le calcul. Comme le test est en bilatéral à un niveau de $5\%$, la probabilité cumulée d'avoir obtenu au moins deux signes "+" ne doit pas être inférieure à $2.5\%$ et pas supérieure à $97.5\%$ si l'on veut accepter (ne pas rejeter) l'hypothèse comme quoi la différence n'est pas statistiquement significative.
	\end{tcolorbox}
	\begin{tcolorbox}[colframe=black,colback=white,sharp corners]
	Nous avons alors:
	
	Soit avec la version française de Microsoft Excel 14.0.6123:
	\begin{center}
		\texttt{=LOI.BINOMIALE(2,12,0.5,VRAI)=1.928\%}
	\end{center}	
	ou si nous ne faisons pas d'approximation en étant plus précis avec la loi hypergéométrique:	
	\begin{center}
		\texttt{=LOI.HYPERGEOMETRIQUE.N(2,24/2,12,24,VRAI)=0.17\%}
	\end{center}	
	ce qui n'est guère plus brillant...\\
	
	Donc la probabilité cumulée est inférieure à $2.5\%$ et n'est de loin pas supérieure à $97.5\%$, nous rejetons l'hypothèse comme quoi la différence n'est pas statistiquement significative.\\
	
	Nous pourrions accepter l'hypothèse si nous prenions pour  equation la valeur:	
	
	mais bon ce n'est pas le cas!
	\end{tcolorbox}	
	Enfin, pour terminer concernant ce test des signes (test de la médiane), indiquons que certains logiciels de statistiques proposent un intervalle de confiance de la médiane basé sur la méthode de calcul exposée précédemment (intervalle de confiance d'une loi binomiale). Cependant, nous pensons qu'il vaudrait mieux favoriser le bootstrapping comme nous l'avons vu dans la section de Méthodes Numériques (page \pageref{bootstrap}), nous nous abstiendrons donc de présenter cette technique ici. De plus il est peu utile de préciser que certains font un approximation en loi Normale (comme avec la majorité des tests mais nous nous en abstiendrons dans le cas présent).

	\paragraph{Test de la médiane de Mood}\mbox{}\\\\
	Nous allons ici introduire un test qui a de multiples noms: "\NewTerm{test de la médiane}\index{test de la médiane}", "\NewTerm{test de la médiane}\index{test de la médiane}", "\NewTerm{test de la médiane de Mood}\index{test de la médiane de Mood}" ou encore "\NewTerm{test de la médiane de Westenberg-Mood}\index{test de la médiane de Westenberg-Mood}" ou "\NewTerm{test de la médiane de Brown-Mood}\index{test de la médiane de Brown-Mood}"...

	Nous considérons deux échantillons indépendants $(X_1,...,X_{n_1})$ et $(Y_1,...,Y_{n_2})$. Nous supposons que $(X_1,...,X_{n1})$ est un échantillon indépendant et distribué selon une loi continue $F$ et $(Y_1,...,Y_{n_2})$ est un échantillon indépendant et identiquement distribué d'une loi continue $G$.

	Après regroupement des $n_1+n_2$ valeurs des deux échantillons, $k=n_1M_n$ (la notation n'est pas géniale car elle peut faire croire à une multiplication mais bon...) est le nombre d'observations $X_i$ du premier échantillon qui sont supérieures à la médiane des $N=n_1+n_2$ observations.

	Sous l'hypothèse nulle que les variables $X$ et $Y$ suivent la même loi continue (c'est-à-dire $G = F$), la variable $k=n_1M_n$ peut prendre les valeurs $0,1,...,n_1$ selon la loi hypergéométrique:
	
	Dès lors, nous pouvons calculer la probabilité cumulée en unilatéral d'avoir $k$. Le test de Mood est donc un test purement unilatéral.
	\begin{tcolorbox}[colframe=black,colback=white,sharp corners]
\textbf{{\Large \ding{45}}Exemple:}\\\\
	Considérons les deux échantillons:
	
	La médiane globale calculée avec Microsoft Excel 14.0.6123 est de 26.10. Nous avons au total:
	
	Il vient alors avec la version française de Microsoft Excel 14.0.6123:
	\begin{center}
		\texttt{=LOI.HYPERGEOMETRIQUE.N(8,26/2,13,26,VRAI)=94.24\%}
	\end{center}
	Donc à un seuil de $5\%$, nous ne rejettons pas l'hypothèse nulle (mais bon étant proche de la limite c'est un peu périlleux de conclure cela...). Si nous faisons le même calcul avec la loi Binomiale nous obtenons:
	\begin{center}
		\texttt{=LOI.BINOMIALE.N(8,26/2,0.5,VRAI)=86.65\%}
	\end{center}
	Mais bien évidemment ici l'approximation ne s'applique pas puisque l'approximation par une loi binomiale est acceptable dans la pratique que lorsque l'échantillon est environ 10 fois plus petit que la population.
	\end{tcolorbox}
	\begin{tcolorbox}[title=Remarque,colframe=black,arc=10pt]
	Il existe malheureusement plusieurs versions du test de Mood. Par exemple un logiciel comme Minitab compare à l'aide d'une table de contingence... le contingent de valeurs au-dessus ou en-dessous de la médiane et fait un simple test d'indépendance du Khi-deux (test de Pearson) comme vu dans la section de Méthodes Numériques. 
	\end{tcolorbox}
	
	\paragraph{Test de Poisson (1 échantillon)}\mbox{}\\\\
	Nous savons qu'un certain nombre d'événements rares suivent une loi de Poisson. Nous pouvons alors nous permettre comme pour toute autre loi, de calculer la probabilité cumulée dans un intervalle donné (bilatéral ou unilatéral).

	Donc, si nous avons une variable aléatoire discrète suivant une loi de Poisson:
	
	Nous avons alors en unilatéral droite à un certain niveau de confiance $\alpha$, la valeur de $n$ de $k$ la plus proche satisfaisant la condition:
	
	Donc pour trouver la valeur de $n$ (entier strictement positif ou nul) il faudrait inverser la somme, ce qui est peu... pratique (raison pour laquelle aucun tableur à ce jour ne propose de fonction pour la loi de Poisson inverse).

	Maintenant, rappelons que nous avons vu dans la section de Suites Et Séries, la série de Taylor (Maclaurin) avec reste intégral à l'ordre $n-1$ autour de $0$ jusqu'à $\lambda$ suivante:
	
	
	Résultat que nous avions également donné sous la forme de fonctions pour la version française de Microsoft Excel 14.0.6123 pour que le lecteur puisse vérifier cette équivalence:
	\begin{center}
		\texttt{=LOI.POISSON.N(}$x \in \mathbb{N},\mu,$\texttt{VRAI)}\\
		\texttt{=1-LOI.KHIDEUX.N(}$2\mu,2(x+1),$\texttt{VRAI)}
	\end{center}
	Il vient alors que dans les tableurs, nous pouvons utiliser la loi du Khi-deux inverse pour calculer l'inverse de la loi de Poisson avec cette fois cependant une petite nuance: le résultat ne donnera pas nécessairement un nombre entier.

	Si par exemple nous prenons (toujours avec la version française de Microsoft Excel 14.0.6123):
	\begin{center}
		\texttt{=1-LOI.KHIDEUX.N(2*20,2*(15+1),VRAI)=15.6513135\%}
	\end{center}
	La question est alors de trouver l'écriture pour l'inverse... Celle-ci est alors donnée par (on divise par $2$  pour tomber pile poile sur la moyenne qui est donc la valeur qui nous intéresse):	
	\begin{center}
		\texttt{=CHIINV(1-15.6513135\%,2*(15+1))/2=15.53194258}
	\end{center}
	Finalement, l'écriture de l'inverse est assez naturelle. Ainsi, le "\NewTerm{test de Poisson à $1$ échantillon}\index{test de Poisson à $1$ échantillon}" à un niveau $\alpha$ donné en unilatéral droite peut s'écrire:
	\begin{center}
		$k\leq $\texttt{=KHIDEUX.INVERSE(1-alpha,2*(nombre de mesures+1))/2}
	\end{center}
	Soit formellement:
	
	Attention cependant à une chose! Il semblerait que certains logiciels de statistiques approximent parfois un peu abusivement la loi de Poisson par une loi Normale. Dès lors, l'intervalle unilatéral se calcule à partir de:
	
	Mais avec la loi de Poisson, nous avons:
	
	Il vient alors:
	
	\begin{tcolorbox}[colframe=black,colback=white,sharp corners]
	\textbf{{\Large \ding{45}}Exemple:}\\\\
	Une société fabrique des télévisions en quantité constante et a mesuré le nombre d'appareils défectueux produits chaque trimestre pendant les dix dernières années (donc $4$ fois $10$ mesures). La direction décide que le nombre maximum acceptable d'unités défectueuses est de $20$ par trimestre et souhaite déterminer si l'usine satisfait à ces exigences (sous l'hypothèse que la distribution des défectueux suive une loi de Poisson) à un niveau de confiance de $5\%$.\\
	
	Les $40$ mesures nous donnent une moyenne de:
	
	Nous avons alors avec l'approximation grossière:
	
	Soit dans un tableur comme la version française de Microsoft Excel 14.0.6123:
	\begin{center}
		\texttt{=LOI.NORMALE.STANDARD.INVERSE.N(1-5\%)*RACINE(20/40)+17.825=18.988}
	\end{center}
	ou:
	
	Soit dans un tableur comme la version française de Microsoft Excel 11.8346:
	\begin{center}
		\texttt{=KHIDEUX.INVERSE(1-5\%,2*(20+1))/2=14.072 }
	\end{center}
	Dans les deux cas, nous sommes en-dessous de la moyenne imposée de $20$ (donc on rejette l'hypothèse nulle comme quoi le nombre de défauts est supérieur ou égal à $20$). Bien évidemment, il est possible pour chacune des méthodes de déterminer quelle devrait être la probabilité cumulée (niveau de confiance) qui nous amène à la limite des $20$ (donc la $p$-valeur en d'autres termes sur laquelle nous reviendrons plus loin). Avec la première méthode (approximation Normale), la $p$-valeur est de $0.104\%$.
	\end{tcolorbox}
	Évidemment, dans le cas bilatéral, nous avons:
	
	
	\begin{tcolorbox}[colframe=black,colback=white,sharp corners]
	\textbf{{\Large \ding{45}}Exemple:}\\\\
	Une compagnie d'aviation a eu $2$ deux crashs en $1,000,000$ de vols (événement très rare). Quelle est l'intervalle de confiance en bilatéral à $95\%$ sachant qu'au niveau mondial le nombre d'accident par millions est de $0.4$.\\

	Nous avons alors:
	
	Soit pour la borne supérieure avec un tableur comme la version française de Microsoft Excel 11.8346:
	\begin{center}
		\texttt{=LOI.KHIDEUX.INVERSE(1-5\%/2,2*(2+1))/2=7.224}
	\end{center}
	et pour la borne inférieure:
	\begin{center}
		\texttt{=LOI.KHIDEUX.INVERSE(1-5\%/2,2*(2+1))/2=0.618 }
	\end{center}
	Donc statistiquement, cette compagnie est moins sûre que l'ensemble des compagnies.
	\end{tcolorbox}
	
	\paragraph{Test de Poisson (2 échantillons)}\mbox{}\\\\
	Nous venons de voir que:
	
	Or, en suivant le même raisonnement que celui qui nous a amené à construire le test de comparaison des moyennes suivant:
	
	ou son équivalent avec la loi de Student quand l'écart-type vrai n'est pas connu et en utilisant le fait que nous avons démontré que la loi de Poisson est stable par l'addition (et donc aussi par la soustraction), que la loi de Gamma était aussi stable par l'addition (et donc aussi par la soustraction) et la loi du Khi-deux aussi puisque ce n'est qu'un cas particulier de la loi Gamma. Nous aurions peut-être tendance à écrire un peu une généralisation logique de ce que nous avons vu juste plus haut:
	
	Et au fait cela constitue un piège selon certains praticiens... Car la loi du Khi-deux a un support qui est défini comme étant strictement positif et l'intervalle de confiance peut naturellement avoir la borne de gauche qui est négative (... O\_o). Une solution consiste alors à utiliser le test de la différence de deux proportions que nous avons déjà étudié plus haut:
	
	À condition bien évidemment que les conditions permettant d'approcher le test par une loi Normale soient satisfaites (les proportions doivent être inférieures typiquement à $0.1$ et les $n$ supérieurs à $50$).

	Certains logiciels semblent avoir implémenté cette dernière méthode (avec laquelle je ne suis pas forcément d'accord).
	
	\begin{tcolorbox}[colframe=black,colback=white,sharp corners]
	\textbf{{\Large \ding{45}}Exemple:}\\\\
	Une compagnie d'aviation a eu $2$ deux crashs en $1'000'000$ vols (événement très rare). Une autre compagnie a eu $3$ crashs en $1'200'000$ vols. Quel est l'intervalle de confiance en bilatéral à $95\%$ en supposant que la différence est nulle.\\

	Les proportions sont alors respectivement:
	
	Notons:
	
	Nous avons alors:
	
	ce qui donne un intervalle de confiance pour la différence de proportion théorique attendue:
	
	et donc comme $-0.0000005$ est dans cet intervalle, nous acceptons l'hypothèse comme quoi la différence des proportions n'est pas statistiquement significative au seuil de $5\%$.

	Ou en prenant l'expression non approximée, nous avons (avec la même conclusion):
	
	\end{tcolorbox}
	
	Donc pour résumer un peu les convergences de lois dans tous ces différents tests et intervalles que nous avons vu jusqu'à maintenant, nous proposons au lecteur le schéma suivant qui clarifiera peut-être plus ou moins bien les choses:
	\begin{figure}[H]
		\centering
		\includegraphics{img/arithmetics/convergence_criteria_statistical_inference.jpg}
		\caption{Convergence des différentes lois usuelles en inférence statistique élémentaire}
	\end{figure}
	Et aussi ce tableau où toutes les relations ont été démontrées en détail plus haut et certains déjà utilisées (d'autres le seront plus loin):
	\begin{table}[H]
		\begin{center}
			\definecolor{gris}{gray}{0.85}
				\begin{tabular}{|p{4cm}|c|c|}
					\hline
	\multicolumn{1}{c}{\cellcolor{black!30}\textbf{Statistique d'échantillonnage}} & \multicolumn{1}{c}{\cellcolor{black!30}\textbf{Moyenne
de la statistique}} & \multicolumn{1}{c}{\cellcolor{black!30}\textbf{Écart-type
de la statistique}} \\ \hline
			\pbox{20cm}{Moyenne \\ (population infinie)} & $\mu$ & $\dfrac{\sigma}{\sqrt{n}}$\\ \hline
			\pbox{20cm}{Moyenne \\ (population finie)} & $\mu$ & $\dfrac{\sigma}{\sqrt{n}}\sqrt{\dfrac{N-n}{N-1}}$\\ \hline
			\pbox{20cm}{Proportion \\ (population finie)} & $p$ & $\sqrt{\dfrac{p(1-p)}{n}}$\\ \hline
			\pbox{20cm}{Proportion \\ (population infinie)} & $p$ & $\sqrt{\dfrac{p(1-p)}{n}}\sqrt{\dfrac{N-n}{N-1}}$ \\ \hline
			\pbox{20cm}{$\hat{\sigma}^2$ \\ (population infinie)} & $\dfrac{n-1}{n}\sigma^2$ & $\sqrt{\dfrac{2(n-1)}{n^2}}\sigma^2$\\ \hline
		\end{tabular}
		\end{center}
		\caption[]{Tableau des statistiques d'échantillonnage démontrées et utilisées en partie jusqu'à maintenant}
	\end{table}
	
	\pagebreak
	\paragraph{Intervalles de Confiance/Tolérance/Prédiction/Crédibilité}\mbox{}\\\\
	Nous y voici et afin d'éviter une confusion fréquente et avant de passer à d'autres sujets plus complexes, comparer l'intervalle de confiance, l'intervalle de tolérance (souvent appelé "intervalle de fluctuation" dans certains programmes scolaires), l'intervalle de prédiction et l'intervalle de crédibilité.

	\textbf{Définitions (\#\mydef):}
	
	\begin{enumerate}
		\item[D1.] "\NewTerm{L'intervalle de tolérance}\index{intervalle de tol\'erance}" (ou "intervalle de fluctuation") est du point de vue du statisticien\footnote{À l'opposé de celui de l'ingénieur en mécanique où l'intervalle de tolérance est l'intervalle de mesure d'une cote de pièce qui ne sera pas rejetée une fois sortie de la production.} un intervalle contenant un certain pourcentage (souvent $68.26$, $95.44$ ou $99.73\%$ pour une distribution Normale) des individus d'une population de mesures.
		
		\item[D2.] "\NewTerm{L'intervalle de confiance}\index{intervalle de confiance}" pour un indicateur statistique donné (moyenne, proportion, écart-type ou autre...) est un intervalle tel que si nous en construisons un certain nombre de structure similaire, alors un certain pourcentage d'entre eux (généralement $90$, $95$ ou $99\%$) contiendra peut-être la valeur vraie de l'indicateur statistique.
		
		\item[D3.] "\NewTerm{L'intervalle de crédibilité}\index {intervalle de cr\'edibilit\'e}" pour un indicateur statistique donné (moyenne, proportion, écart-type ou autre...) est un intervalle borné qui a un certain seuil de probabilité cumulée (généralement $90$, $95$ ou $99\%$) contient la valeur vraie de l'indicateur et pour lequel avons une connaissance a priori particulière de sa distribution (voir plus bas page \pageref{credibility interval}).
		
		\item[D4.]  "\NewTerm{L'intervalle de prédiction}\index{intervalle de prédiction}" permet de déterminer un intervalle d'une valeur individuelle basée sur la connaissance de la moyenne échantillonnale et de l'écart-type de la population.
	\end{enumerate}
	
	Nous avons déjà passé un long moment sur les intervalles de confiance et de crédibilité plus haut. Faisons alors un exemple pour les deux définitions restantes en prenant le cas où la moyenne et l'écart-type des prix de $49$ DVD sont donnés par:
	
	Nous avons alors:
	
	correspondant respectivement à des intervalles de tolérance selon une loi Normale de $68.26, 95.44$ et $99.73\%$.

	Par contre, un intervalle de confiance à $95\%$ basé sur la relation démontrée plus haut:
	
	donne:
	
	Donc $95\%$ de probabilité cumulée que la moyenne vraie (espérance) se trouve comprise entre 31.32 et 31.78.
	\begin{figure}[H]
		\centering
		\includegraphics{img/arithmetics/tolerance_confidence_interval.jpg}
		\caption[]{Histogramme de l'échantillon des prix de 49 DVD}
	\end{figure}
	Maintenant passons à une notion qui curieusement est rarement traitée dans les ouvrages de statistiques. L'idée de l'intervalle de prédiction est de plutôt que de s'intéresser à l'intervalle de confiance de l'espérance basé sur une moyenne expérimentale, d'utiliser cette moyenne expérimentale (échantillonnale) comme base pour prévoir l'intervalle d'une unique valeur (et non d'une moyenne!).
	
	Nous allons donc nous intéresser à la différence entre la moyenne et une valeur ponctuelle:
	
	que nous supposerons proche de zéro (il vaut mieux pour avoir un produit fiable et passer les tests d'autorisation des ventes...). Concernant la variance, ce qui nous intéresse ce n'est plus simplement l'écart-type de la moyenne mais l'écart-type de la différence... et comme l'échantillon est indépendant de la valeur unique nous avons:
	
	Donc nous pouvons écrire qu'en première approximation:
	
	Et bien sûr après ce que nous avons vu:
	
	Nous pouvons donc construire verbatim l'intervalle de prédiction:
	
	
	\subsubsection{Inférence Bayésienne}
	Pour comprendre l'idée "\NewTerm{d'inférence bayésienne}\index{inférence bayésienne}" (à ne pas confondre avec le concept de "test d'hypothèse bayésien"), rappelez-vous que dans l'approche non bayésienne (fréquentiste), nous mesurons par exemple un paramètre $\theta$. Ensuite, nous calculons l'erreur standard ou un intervalle de confiance sur lequel nous effectuons un test d'hypothèse.
	
	Dans l'approche bayésienne, nous supposons que le paramètre $\theta$ suit une distribution a priori non triviale. Nous calculons ensuite sa distribution de probabilité postérieure connaissant les mesures effectuées et nous calculons ensuite un intervalle de confiance sur cette dernière.
	
	Selon certains spécialistes, il existerait maintenant un grand nombre de théorèmes et une grande quantité  d'exemples numériques élaborés qui fournissent de fortes évidences de la supériorité de la méthode bayésienne dans une centaine de domaines différents! Les méthodes fréquentistes traditionnelles qui n'utilisent que des distributions d'échantillonnage sont utilisables et utiles dans de nombreux problèmes particulièrement simples et idéalisés; cependant, ils représentent les cas particuliers les plus proscrits de la théorie des probabilités, car ils supposent des conditions (répétitions indépendantes d'une expérience aléatoire mais pas d'informations pertinentes!) qui ne sont presque jamais rencontrées dans des problèmes réels.
	
	\begin{figure}[H]
		\centering
		\includegraphics[scale=0.5]{img/arithmetics/bayes_vs_frequentist.jpg}
		\caption[]{Comparaison des approches fréquentiste et bayésienne (Christophe Michel, 2018).}
	\end{figure}
	
	\begin{tcolorbox}[title=Remarque,colframe=black,arc=10pt]
	L'approche fréquentiste est le cas de ce que nous appelons également un "\NewTerm{a priori uniforme}". Mais le lecteur ne doit pas penser qu'une loi uniforme a priori est un exemple de a priori non informé (il y a un exemple mathématique détaillé plus loin juste après notre étude de la distribution bêta-binomiale postérieure) !!! En effet, les purs fréquentistes ne font AUCUN a priori (c'est-à-dire que la probabilité est complètement indéterminée $0/0$)! Le fréquentiste pur suppose toujours que les données de l'échantillon sont exactement représentatives de la population parce que c'est tout ce qu'il peut faire.
	\end{tcolorbox}
	
	Supposons que nous avons un modèle bayésien univarié avec une vraisemblance $P(y|\vec{\theta})$ (parfois aussi notée $f(y|\vec{\theta})$) et une distribution a priori univariée $P(\vec{\theta})$ (également parfois appelé $f(\vec{\theta})$).

	Rappelons que (selon ce que nous avons vu lors de notre étude des probabilités bayésiennes à la page \pageref{compound probabilities}) si nous multiplions la vraisemblance univariée $P(y|\vec{\theta})$ et la probabilité a priori univariée $P(\vec{\theta})$ nous obtenons la distribution a posteriori univariée $P(\vec{\theta}|y)$ (également notée $\pi(\vec{\theta}|y)$). En effet, en utilisant ce que nous avons vu plus haut nous obtenons (le dénominateur est pour rappel la "probabilité marginale"):
	
	Également souvent noté (toujours dans le cas univarié et en utilisant la définition de la distribution marginale):
	
	Où le ratio:
	
	est parfois appelé "\NewTerm{vraisemblance standardisée}\index{vraisemblance standardisée}".
	
	À l'exception de la probabilité marginale, les estimations a posteriori, a priori et la vraisemblance ne sont pas des estimations ponctuelles. Ce sont des courbes. Dans l'exemple ci-dessous, nous montrons la fonction de densité de probabilité sur différents emplacements d'une voiture:
	\begin{figure}[H]
		\centering
		\includegraphics[scale=0.3]{img/arithmetics/bayesian_inference_illustrated.jpg}
	\end{figure}
	Nous recommandons fortement au lecteur de voir les exemples ci-dessous pour une meilleure compréhension du dénominateur.
	
	\textbf{Définition (\#\mydef):} Un estimation antérieure $P(\vec{\theta})$ pour un modèle échantillonal est appelé "\NewTerm{l'estimation antérieure conjuguée}\index{estimation antérieure conjuguée}\label{conjugate prior}" si l'estimation postérieure résultant notée traditionnellement $\pi(\vec{\theta}|\vec{y})$ (pour les cas multivariés) appartient à la même famille de distribution que l'estimation antérieure.
	
	En d'autres termes: si notre distribution postérieure est de la même famille que la distribution antérieure, nous parlons alors "\NewTerm{distributions conjuguées}". Ensuite, nous disons également que la fonction antérieure est la "\NewTerm{conjuguée de la vraisemblance}" (en pratique c'est cependant l'exception plutôt que la règle...).
	
	Voici par exemple un tableau résumant les cas les plus classiques:
	\begin{table}[H]
		\resizebox{\textwidth}{!}{\begin{tabular}{|l|l|l|}
		\hline
		\rowcolor[HTML]{9B9B9B} 
		\multicolumn{1}{|c|}{\cellcolor[HTML]{9B9B9B}\textbf{Vraisemblance}} & \multicolumn{1}{c|}{\cellcolor[HTML]{9B9B9B}\textbf{Antérieure}} & \multicolumn{1}{c|}{\cellcolor[HTML]{9B9B9B}\textbf{Postérieure}} \\ \hline
		Binomiale & Bêta & Distribution (Bêta-)Binomiale  \\ \hline
		Négative Binomiale & Bêta & Distribution (Bêta-)Negative Binomiale  \\ \hline
		Poisson & Gamma & Distribution (Poisson-)Gamma \\ \hline
		Géométrique & Bêta & Distribution (Beta-)Geometric \\ \hline
		Normale (moyenne inconnue) & Normale & Distribution Normale \\ \hline
		Normale (variance inconnue) & Inverse Gamma & Distribution (Normal-)Gamma inverse \\ \hline
		Normale (tous les moments connus) & Normale/Gamma & Distribution Normale/Gamma \\ \hline
		Multinomiale & Dirichlet & Distribution (Multinomiale-)Dirichlet \\ \hline
		\end{tabular}}
	\end{table}
	Cependant (et comme toujours!) dans ce livre, nous ne détaillerons que les conjuguées qui ont des applications pratiques dans d'autres chapitres!
	
	Un dernier rappel avant de continuer. Si l'on connaît pour une distribution de vraisemblance donnée l'expression $P(y_i/\vec{\theta})$, alors pour un vecteur de réalisations indépendantes d'une variable aléatoire, la vraisemblance totale est égale au produit des probabilités:
	

	\paragraph{Antérieure Bêta conjuguée pour vraisemblance Binomiale (distribution bêta-binomial)}\mbox{}\\\\
	Introduisons maintenant la distribution bêta-binomiale qui est très utilisée en Ingénierie mais aussi en statistique bayésienne (surtout connue pour le fameux test A/B par exemple!).
	
	Nous décidons de commencer avec cette distribution ayant une vraisemblance Binomiale car c'est une loi avec un seul paramètre et pédagogiquement il semble plus judicieux de commencer avec cette dernière plutôt qu'avec une vraisemblance Normale qui a deux paramètres (en plus il y a trois conjugués possibles pour la vraisemblance Normale comme le montre le tableau ci-dessus).

	\textbf{Définition (\#\mydef):} La "\NewTerm{distribution bêta-binomiale distribution}\index{distribution b\^eta-binomiale distribution}\label{beta-binomial distribution}" est la distribution binomiale dans laquelle la probabilité $p$ de succès à chaque essai est fixe mais tirée au hasard à partir d'une distribution bêta avant les $n$ essais de Bernoulli.
	
	Prouvons maintenant que la distribution bêta est une distribution conjuguée de la distribution Binomiale, mais avant une définition!
	
	Supposons que nous observions $n$ variable aléatoire indépendante de Bernouilli $ x_1, \ldots, x_n $. Nous souhaitons estimer la probabilité de succès $p$ avec l'approche bayésienne. Nous utiliserons une version bêta $P_{a, b}$ a priori pour $ p $ (comme $P_{a,b} \in] 0,1 [$ cela peut être un bon choix ...) et prouverons que c'est un conjugué avant.
	
	Supposons que nous observions $n$ variables aléatoires indépendantes de Bernouilli $x_1,\ldots,x_n$. Nous souhaitons estimer la probabilité de succès $p$ avec l'approche bayésienne. Nous utiliserons une estimation antérieure (a priori) bêta $P_{a,b}$ pour $p$ (comme $P_{a,b}\in ]0,1[$ cela peut être un bon choix ...) et prouverons que c'est bien une estimation antérieure conjuguée.
	
	En utilisant d'abord la relation entre la fonction gamma d'Euler et la factorielle (\SeeChapter{voir section Calcul différentiel et Intégral \pageref{gamma euler function}}), nous avons en utilisant la notation conventionnelle (puisque la probabilité de l'a priori est supposée indépendante des essais eux-mêmes nous multiplions ensuite les deux probabilités):
	
	Notez que nous avons également le résultat équivalent suivant (juste la notation diffère!):
	
	Bien que cela ne soit pas vraiment nécessaire, déduisons la densité marginale de $y$ (dans le but de normaliser la densité postérieure plus loin!):
	
	Cette dernière densité marginale s'écrit aussi parfois:
	
	Donc une définition commune (fausse) de la distribution bêta-binomiale est donnée par ces deux relations équivalentes de la densité marginale:
	
	
	Concentrons-nous maintenant sur la postérieure $\pi(\vec{\theta},y)$ (c'est-à-dire l'expression "réelle" de la distribution bêta-binomiale), alors donnée par:
	
	Cette dernière égalité est parfois notée:
	
	Clairement, cette postérieure est une distribution bêta $P_{a,b}(y+a,n-y+b)$. On voit donc que seuls les paramètres de la bêta antérieure se transforment désormais en "hyperparamètres" d'une beta postérieure!
	
	Notez que nous obtenons ensuite rapidement par extension (lors tirage des éléments $N$ parmi les $n$):
	
	Cette dernière relation est parfois notée $\text{Bbin}(a,b,n)$.
	\begin{tcolorbox}[title=Remarque,colframe=black,arc=10pt]
	Le pire est que certains auteurs du domaine de fiabilité technique notent la probabilité cumulée que $ y <n-F $ où $ F $ est le nombre de défaillances comme suit (\SeeChapter{voir section Génie Industriel page \pageref{Beta-binomial sampling size}}):
	
	Ensuite, il ne devient pas du tout évident hors contexte de savoir si nous avons affaire à une densité de masse ou à une fonction de distribution.
	\end{tcolorbox}
	Un cas intéressant à considérer est celui où nous n'avons aucune prémonition particulière concernant la distribution antérieure. Par conséquent, nous pouvons prendre une fonction qui a la même probabilité pour toutes les valeurs dans l'intervalle $[0,1]$ du paramètre $p$ de la distribution binomiale.

	Par conséquent, la fonction bêta peut toujours être conservée comme une distribution a priori car elle est réduite au cas uniforme:
	
	qui est donc une distribution uniforme continue dans l'intervalle $[0,1]$. La fonction postérieure du paramètre $p$ est alors donnée par:
	
	Calculons maintenant la moyenne et la variance attendues de la distribution bêta-binomiale en utilisant\footnote{Il existe d'autres façons (compliquées) de déduire l'espérance et la variance, mais si vous vous souvenez de ces relations, vous avez pour autant que nous le savonss la dérivation la plus simple!} la loi de l'espérance totale (voir page \pageref{iterated conditional mean}) et celle de la loi de la variance totale (page \pageref{iterated condititional variance}):
	
	L'espérance et la variance pour la distribution binomiale sont supposées si bien connues que nous ne les mentionnerons pas ici (voir page \\pageref{binomial distribution}), et pour la distribution bêta que nous avons (voir page \pageref{beta distribution}):
	
	En utilisant la loi de l'espérance totale, nous avons:
	
	et en utilisant la loi de la variance totale, nous avons (dans la deuxième ligne, nous avons utilisé la relation de Huygens):
	
	Donc, pour résumer, nous avons pour l'espérance postérieure et la variance postérieure:
	
	Si nous réintroduisons $ p $, cela peut être écrit:
	
	On remarque que la variance est plus élevée que pour la distribution binomiale correspondante. On dit alors qu'il y a "surdispersion".
	
	\subparagraph{Antérieure  Uniforme vs Non-informative}\mbox{}\\\\
	Lorsque les distributions antérieures (probabilités à priori) n'ont pas de base de population, elles peuvent être difficiles à choisir, et il existe depuis longtemps un désir de distributions à priori dont on peut garantir qu'elles jouent un rôle minimal dans la distribution postérieure. Ces distributions sont parfois nommées "\NewTerm{distributions antérieures de référence}\index{distributions ant\'erieures de r\'ef\'erence}", et la densité antérieure est décrite comme "\NewTerm{à priori non informatif}\index{\`a priori non informatif}". On dit souvent que la raison pour l'usage des distribution à priori non-informatives est de \textit{laisser les données parler d'elles-mêmes}, de sorte que les inférences ne sont pas affectées par des informations extérieures aux données connues.
	
	Certaines personnes s'attendent à ce qu'une estimation antérieure uniforme soit un bon exemple d'un a priori non-informatif et qui permettrait de retomber sur le même résultat que l'approche fréquentiste\footnote{Pour des raisons que nous ignorons encore, même les physiciens de la théorie des cordes semblent utiliser la distribution uniforme comme "choix naturel" (critère de naturalité) pour un a priori non-informatif. Une première tentative pour justifier la distribution de probabilité uniforme pour le critère de naturalité pourrait être de dire qu'elle n'introduit pas de paramètres supplémentaires dans la théorie des cordes. Mais bien sûr, elle le fait: elle introduit le nombre $1$ comme largeur typique! Certains physiciens, cependant, soutiennent sans évidence scientifique pour l'instant, que $1$ satisfait le critère de naturalité...}. Cependant, ce n'est pas le cas!

	Par exemple, regardons le problème classique de Bernoulli - nous lançons  des pièces et $s$ sont les côtés faces, $f$ les côtés piles.
	
	Nous voulons prédire $p$, la probabilité de faces sur n'importe quel lancé d'une pièce donnée.
	
	L'approche fréquentiste consiste à dire que $p$ est simplement $s/(s+f)$. Dans cette approche, nous ne supposons rien sur la distribution de $p$ et nous trouvons simplement l'estimation du maximum de vraisemblance. C'est logique!
	
	Maintenant, adoptons l'approche bayésienne et nous disons:
	
	Comme nous l'avons prouvé plus haut à la page \pageref{beta-binomial distribution} si nous prenons comme antérieure la distribution bêta, nous obtenons:
	
	Supposons maintenant que nous voulons trouver l'espérance $p$:
	
	On retombe évidemment sur la même expression de l'espérance que la distribution bêta (car c'est un conjugué antérieur!). On peut donc faire la même simplification:
	
	Jusqu'à présent, cela fait sens! Supposons maintenant que nous voulions choisir une antérieure vraiment "objective" afin d'émuler l'approche fréquentiste, comme une distribution uniforme. Cela équivaut à $B(1,1)$ comme nous l'avons prouvé lors de notre étude de la distribution bêta. Dans ce cas, nous obtiendrions:
	
	Cela peut faire mal au cerveau de certaines personnes! Même si nous sommes aussi "objectif" que possible ("$p$" est vraisemblablement égal à n'importe quoi), nous obtenons un résultat différent de l'approche fréquentiste.
	
	D'un autre côté, si nous choisissons une antérieure non intuitive comme $ B(0,0)$ , nous obtenons le même résultat que l'approche fréquentiste:
	
	Qu'est-ce qui rend $B(0,0)$ plus similaire à l'approche fréquentiste que $B(1,1)$? Quelle hypothèse faisons-nous avec $B(1,1)$? Comment est-il possible que $B (0,0)$ soit moins informative? En fait, vous ne pouvez pas poser cette question car elle est tout simplement incohérente par définition!
	
	\paragraph{Antérieure Normale pour vraisemblance Normale}\mbox{}\\\\
	Nous ne nous concentrerons ici que sur le cas où la moyenne est inconnue car en pratique c'est la plus utilisée (à notre connaissance) et c'est aussi selon notre point de vue personnel le seul cas intuitif à expliquer pédagogiquement...

	Considérons que nous voulons faire l'inférence sur la moyenne. Nous supposons que la variable aléatoire mesurée $y$ suit probablement une distribution Normale de la moyenne et de l’écart type $\mathcal{N}(y|\mu,\sigma)$ (c’est-à-dire la vraisemblance de $y$) et avec l'antérieure que $\mu$ suive une distribution Normale $\mathcal{N}(\mu_0,\sigma_0)$, qui peut être considérée comme un bon choix tant que les valeurs possibles de $\mu$ correspondent au domaine de définition de la distribution Normale.

	On a alors (rappelez-vous que dans la distribution précédente c'est le paramètre d'intérêt qui est la variable aléatoire!):
	
	Nous regroupons et réorganisons certains termes:	
	
	Nous multiplions maintenant la partie intérieure des crochets par:
	
	dans le but d'éliminer $\mu^2$:
	
	où $k$ ne dépend pas de $\mu$. Continuons...:
	
	En supposant qu'il existe un $k$ tel que (...):
	
	\begin{tcolorbox}[title=Remarque,colframe=black,arc=10pt]
	En d'autres termes, nous devons vérifier que lors de l'utilisation de cet outil que:
	
	\end{tcolorbox}
	La distribution postérieure ressemble donc bien à l'apparence d'une loi Normale:
	
	\begin{tcolorbox}[title=Remarque,colframe=black,arc=10pt]
	A l'opposé de notre étude de la distribution postérieure bêta-binomiale, nous ne diviserons pas ici par la distribution marginale qui n'est de toute façon qu'une constante de normalisation et ne nous intéresse pas vraiment pour l'instant pour l'application pratique que nous aurons beaucoup plus tard dans ce livre.
	\end{tcolorbox}
	La moyenne et la variance de la distribution postérieure sont alors, par analogie, égales à:
	
	Analysons un peu plus en détail comment la moyenne antérieure $\mu_0$ et la moyenne postérieure $\mu_p$ sont liées l'une à l'autre:
	
	Nous pouvons observer que lorsque $n$ augmente, la moyenne empirique $\bar{y}$ (deuxième terme) domine la moyenne antérieure (a priori) $\mu_0$. Lorsque la variance antérieure $\sigma_0^2$ diminue, la moyenne a priori (premier terme) prend une importance prédominante.
	
	Rappelons qu'en pratique le spécialiste choisira, sur la base de son retour d'expérience (REX), les antérieurs $\{\mu_0,\sigma_0\}$ à propos de $\mu_p$ (puisque ce dernier en dépend) et que l'écart type $\sigma$est supposé connu (c'est l'ensemble de tous ces paramètres qui fait que beaucoup de praticiens préfèrent l'approche fréquentiste...).
	
	\pagebreak
	\paragraph{Intervalles de Crédibilité}\mbox{}\\\\
	Dans le domaine inférence bayésienne, un "\NewTerm{intervalle de crédibilité}\index{intervalle de crédibilité}\label{intervalle de crédibilité}"  est un intervalle dans lequel une valeur de paramètre non observée tombe avec une probabilité particulière. Il s'agit d'un intervalle dans le domaine d'une distribution de probabilité postérieure ou d'une distribution prédictive. La généralisation aux problèmes multivariés est la région crédible. Les intervalles crédibles sont analogues aux intervalles de confiance dans les statistiques fréquentistes, bien qu'ils diffèrent sur une base philosophique: les intervalles bayésiens traitent leurs limites comme fixes et le paramètre estimé comme une variable aléatoire, tandis que les intervalles de confiance fréquentistes traitent leurs limites comme des variables aléatoires et le paramètre comme un valeur fixe. En outre, les intervalles de crédibilité bayésiens utilisent (et nécessitent en fait) la connaissance de la distribution antérieure spécifique à la situation, contrairement aux intervalles de confiance fréquentistes.
	
	Un statisticien bayésien dirait que, compte tenu de ses données observées, il y a une probabilité cumulée de $95\%$ que la vraie valeur de  $\theta$ se situe dans la région crédible tandis que le statisticien fréquentiste dirait qu'il y a une probabilité de $95\%$ que lorsqu'il calcule un intervalle de confiance à partir de données de ce type, la vraie valeur de $\theta$ y tombera.
	
	Techniquement, "l'intervalle de crédibilité" bayésien de taille $1-\alpha$  est un intervalle $[x_1,x_2]$, où:
	
	Notez que c'est le paramètre et non les limites d'intervalle, conditionné par les données $D$, qui est aléatoire dans ce contexte! Cela rend un intervalle de crédibilité plus intuitif car maintenant nous pouvons interpréter $1-\alpha$ comme une probabilité (postérieure) étant donné un ensemble de données observées. Dans le cadre fréquentiste, un intervalle de confiance donné est correct ou incorrect puisque $\theta$ est fixe. Le niveau de confiance nous indique seulement quelle proportion d'intervalles de confiance serait correcte si nous répétions nos calculs en utilisant de nouveaux échantillons indépendants de la même distribution sous-jacente, ce que de nombreux non-statisticiens trouvent déroutant.
	
	Voyons un exemple compagnon bien illustré (par ailleurs...) dans notre livre compagnon sur le logiciel R pour obtenir plus d'intuition sur les différentes approches.
	
	Considérons un problème simple avec des essais de Bernoulli dans le domaine des réclamations d'assurance. Nous voulons dériver un certain intervalle de confiance pour la probabilité de réclamer une perte. Il y avait $n=1047$ polices d'assurance et $y=159$ réclamations.

	Considérons l'intervalle de confiance standard (fréquentiste) de la proportion. Qu'est-ce que cela signifie que:
	
	est l'intervalle de confiance (asymptotique) à $95\%$? La façon dont nous pouvons le voir est très simple. Générons quelques échantillons, de taille $n$, avec la même probabilité que l'empirique, c'est-à-dire $\hat{p}$ (qui est la signification de \textit {à partir de données de ce type}). Pour chaque échantillon, nous calculons l'intervalle de confiance avec la relation ci-dessus. Il s'agit d'un intervalle de confiance de $95\%$ car dans $95\%$ des scénarios, la valeur empirique se situe dans l'intervalle de confiance.
	
	Si nous écrivons un code R qui calcule 100 de ces intervalles, nous obtenons généralement:
	\begin{figure}[H]
		\centering
		\includegraphics[scale=0.7]{img/arithmetics/credibility_interval_r_example_100_ci.jpg}
	\end{figure}
	Maintenant, qu'en est-il de l'intervalle de crédibilité bayésien? Supposons que la distribution antérieure de la probabilité (notée $\hat{p}$ dans le cadre fréquentiste) de faire une réclamation ait une distribution $\mathcal{B}(a,b)$. Nous avons vu dans ce qui précède que, puisque la distribution bêta est le conjugué de celle de Bernoulli, la distribution postérieure sera également bêta. Plus précisément:
	
	Choisissons une antérieure (a priori) uniforme:
	
	et considérons que nous ne faisons l'expérience qu'une seule fois ($N=1$):
	
	Sur la base de cette propriété, l'intervalle de crédibilité à $95\%$ est basé sur des quantiles de cette distribution (postérieure) tels que:
		
	et a la forme suivante si tracée avec le logiciel statistique R:
	\begin{figure}[H]
		\centering
		\includegraphics[scale=0.7]{img/arithmetics/credibility_interval_r_example_ci_posterior.jpg}
	\end{figure}
	Il est simple avec un logiciel statistique d'obtenir les quantiles de la borne inférieure et de la borne supérieure de l'intervalle de crédibilité.
	
	Nous pouvons maintenant tracer à nouveau $100$ simulations pas générées en utilisant toujours la même proportion estimée comme auparavant, mais en utilisant cette fois-ci certaines probabilités possibles (en vert, ci-dessous, nous pouvons visualiser l'histogramme de ces valeurs choisies au hasard), sur la base de cette distribution postérieure (étant données les observations):
	\begin{figure}[H]
		\centering
		\includegraphics[scale=0.7]{img/arithmetics/credibility_interval_r_example_100_ci_posterior.jpg}
	\end{figure}
	Ci-dessus, il y a $95\%$ de probabilité cumulée que ces proportions empiriques se situent dans l'intervalle de crédibilité, défini à l'aide des quantiles de la distribution postérieure. On peut en fait visualiser toutes ces proportions: en noir la proportion utilisée pour générer l'échantillon, puis, en bleu ou rouge, les moyennes obtenues sur ces échantillons simulés.
	
	\subsection{Loi Faible des Grands Nombres}\label{weak law of large numbers}
	Nous allons maintenant nous attarder sur une relation très intéressante en statistiques qui permet de dire pas mal de choses tout en ayant peu de données et ce quelle que soit la loi considérée (ce qui est pas mal quand même!). C'est une propriété très utilisée en simulation statistique par exemple dans le cadre de l'utilisation de Monte-Carlo.
	
	Soit une variable aléatoire à valeurs dans $\mathbb{R}^+$. Alors nous allons démontrer la relation suivante appelée "\NewTerm{inégalité de Markov}\index{inégalité de Markov}":
	
	avec $\text{E}(X)\leq \lambda$  dans le contexte particulier des probabilités.
	
	En d'autres termes, nous proposons de démontrer que la probabilité qu'une variable aléatoire soit plus grande ou égale qu'une valeur $\lambda$ est inférieure ou égale à son espérance divisée par la valeur considérée $\lambda$ et ce quelle que soit la loi de distribution de la variable aléatoire $X$!
	
	\begin{dem}
	Notons les valeurs de $X$ par $(x_1,...,x_n)$, où $0\leq x_1 < x_2 < ... <x_n$ (c'est-à-dire triées par ordre croissant) et posons $x_0=0$. Nous remarquons d'abord que l'inégalité est triviale au cas où $\lambda \geq x_n \geq 0$. Effectivement, comme $X$ ne peut être compris qu'entre $0$ et $x_n$ par définition alors la probabilité qu'il soit supérieur à $x_n$ est nulle. En d'autres termes:
	
	et $X$ étant positif, $\text{E}(X)$ l'est aussi, d'où l'inégalité pour ce cas particulier dans un premier temps. 
	
	Sinon, nous avons $0<\lambda\leq x_n$ et il existe alors un $k\in (1,...,n)$ tel que $x_{k-1}<\lambda \leq x_k$. Donc:
	
	\begin{flushright}
		$\blacksquare$  Q.E.D.
	\end{flushright}
	\end{dem}
	
	\begin{tcolorbox}[colframe=black,colback=white,sharp corners]
	\textbf{{\Large \ding{45}}Exemple:}\\\\
	Nous supposons que le nombre de pièces sortant d'une usine donnée en l'espace d'une semaine est une variable aléatoire d'espérance $50$. Si nous souhaitons estimer la probabilité cumulée que la  production dépasse $75$ pièces nous appliquerons simplement:
	
	\end{tcolorbox}
	Considérons maintenant une sorte de généralisation de cette inégalité appelée "\NewTerm{inégalité de Bienaymé-Tchebychev}\index{inégalité de Bienaymé-Tchebychev}" (abrégée "\NewTerm{inégalité BT}") qui va nous permettre d'obtenir un résultat très très très intéressant et important un peu plus bas.

	Considérons une variable aléatoire $X$ réelle (donc nous ne nous limitons pas au seul cas où elle est dans $\mathbb{R}^+$). Alors nous allons démontrer l'inégalité de Bienaymé-Tchebychev suivante:
	
	qui exprime le fait que plus l'écart-type est petit, plus la probabilité que la variable aléatoire $X$ s'éloigne de son espérance est faible.
	
	\begin{dem}
	Nous obtenons cette inégalité en écrivant d'abord:
	
	où le choix du carré va nous servir pour une simplification future.

	Puis en appliquant l'inégalité de Markov (comme quoi c'est quand même utile...) à la variable aléatoire $Y=\left[X-\text{E}(X)\right]^2$ avec $\lambda=\varepsilon^2$ il vient automatiquement:
	
	Ensuite, en utilisant la définition de la variance:
	
	Nous obtenons bien:
	
	\begin{flushright}
		$\blacksquare$  Q.E.D.
	\end{flushright}
	\end{dem}

	Si nous posons:
	
	l'inégalité s'écrit aussi:
	
	et exprime que la probabilité cumulée qu'afin que $X$ s'éloigne de son espérance de plus que $t$ fois son écart-type, est inférieure à $1/t^2$. Mais pour $t=2$, $3$ et $4$, on peut déduire qu'il y a respectivement $75\%$, $89\%$ et $94\% $ des données entre l'intervalle défini par $2$, $3$ et $4$ d'écarts-types autour de la moyenne. Il y a, en particulier, moins de $1$ chance sur $9$ pour que $X$ s'éloigne de son espérance de plus de trois fois l'écart-type. C'est par ailleurs ce théorème qu'a utilisé le comité de Bâle pour définir le facteur de correction de la Value At Risk utilisé en finance (\SeeChapter{voir section Économie page \pageref{value at risk}}).
	
	\begin{tcolorbox}[colframe=black,colback=white,sharp corners]
	\textbf{{\Large \ding{45}}Exemple:}\\\\
	Nous reprenons l'exemple où le nombre de pièces sortant d'une usine donnée en l'espace d'une semaine est une variable aléatoire d'espérance $50$. Nous supposons en plus que la variance de la production hebdomadaire est de $25$. Nous cherchons à calculer la probabilité que la production de la semaine prochaine soit comprise entre $40$ et $60$ pièces.\\
	
	Pour calculer ceci il faut d'abord se souvenir que l'inégalité de BT est basée en partie sur le terme $|X-\text{E}(X)|$ donc nous avons:
	
	donc l'inégalité de BT nous permet bien de travailler sur des intervalles égaux en valeur absolue ce qui s'écrit aussi:
	
	Ensuite, ne reste plus qu'à appliquer simplement l'inégalité numériquement:
	
	\end{tcolorbox}
	Les deux dernières inégalités obtenues avant l'exemple vont nous permettre d'obtenir une relation très importante et puissante que nous appelons la "\NewTerm{loi faible des grands nombres}\index{loi faible des grands nombres}" (L.F.G.N.) ou encore "\NewTerm{théorème de Khintchine}\index{théorème de Khintchine}".

	Considérons une variable aléatoire $X$ admettant une variance et $(X_n)_{n \in \mathbb{N}^*}$ une suite de variables aléatoires indépendantes (donc non corrélées deux-deux) de même loi que $X$ et ayant toutes les mêmes espérances equation et les mêmes écarts-types $\sigma$.
	
	Ce que nous allons montrer est que si nous mesurons une même quantité aléatoire $X_n$ de même loi au cours d'une suite d'expériences indépendantes (alors dans ce cas, nous disons techniquement que la suite $(X_n)_{n \in \mathbb{N}^*}$ de variables aléatoires est définie sur le même espace probabilisé), alors la moyenne arithmétique des valeurs observées va se stabiliser sur l'espérance de $X$ quand le nombre de mesures est infiniment élevé.

	De manière formelle ceci s'exprime sous la forme:
	
	lorsque $n\rightarrow +\infty$ c'est cela le résultat très important dont nous faisions mention plus haut! L'estimateur empirique de la moyenne tend donc pour toute loi vers l'espérance vraie si $n$ est grand! Donc de par la même nous assurons que la moyenne empirique est un estimateur convergent de l'espérance! Ce résultat (assez intuitif) est parfois appelé "\NewTerm{théorème fondamental de Monte-Carlo}\index{théorème fondamental de Monte-Carlo}" car il est au centre du principe des simulations du même nom (\SeeChapter{voir section Méthodes Numériques page \pageref{monte carlo simulations}}) qui ont une importance cruciale dans l'étude des statistiques avancées.

	\begin{tcolorbox}[title=Remarque,colframe=black,arc=10pt]
	Cette propriété de convergence est également nommée "\NewTerm{régression vers la moyenne}\index{régression vers la moyenne}". Signifiant implicitement que si une variable est extrême lors de sa première mesure, elle aura tendance à être plus proche de la moyenne lors de sa deuxième mesure - et si elle est extrême lors de sa deuxième mesure, elle aura eu tendance à être plus proche de la moyenne lors de sa première mesure.\\

	L'un des premiers exemples de Galton était la taille moyenne des parents et de leurs enfants. Il a constaté que les parents de grande taille avaient (en moyenne) des enfants plus petits qu'eux et que les parents de petite taille avaient (en moyenne) des enfants plus grands qu'eux. Dans les deux cas, les enfants dont les parents se trouvaient aux extrémités de la distribution avaient des hauteurs plus proches de la taille moyenne de la population.
	\end{tcolorbox}
	Donc en d'autres termes la probabilité cumulée que la différence entre la moyenne arithmétique et l'espérance des variables aléatoires observées soit comprise dans un intervalle autour de la moyenne tend vers zéro quand le nombre de variables aléatoires mesurées tend vers l'infini (ce qui est finalement intuitif).
	
	Ce résultat nous permet d'estimer l'espérance mathématique en utilisant la moyenne empirique (arithmétique) calculée sur un très grand nombre d'expériences.
	\begin{dem}
	Nous utilisons l'inégalité de Bienaymé-Tchebychev pour la variable aléatoire (cette relation s'interprète difficilement mais permet d'avoir le résultat escompté):
	
	Et nous calculons d'abord en utilisant les propriétés mathématiques de l'espérance que nous avions démontrées plus haut:
	
	et dans un deuxième temps en utilisant les propriétés mathématiques de la variance aussi déjà démontrées plus haut:
	
	et puisque nous avons supposé les variables non corrélées entre elles alors la covariance est nulle dès lors:
	
	Donc en injectant cela dans l'inégalité BT:
	
	nous avons alors:
	
	et l'inégalité tend bien vers zéro quand $n$ au dénominateur tend vers l'infini.
	\begin{flushright}
		$\blacksquare$  Q.E.D.
	\end{flushright}
	\end{dem}
	Signalons que cette dernière relation est souvent notée dans certains ouvrages et conformément à ce que nous avons vu au début de ce chapitre:
	
	ou encore:
	
	Donc, pour  $\forall \varepsilon >0$:
	
	
	\subsection{Fonction Caractéristique}
	Dans la théorie des probabilités et des statistiques, la "\NewTerm{fonction caractéristique}\index{fonction caractéristique (théorie des probabilités)}\label{charactertistic function}" de toute variable aléatoire à valeur réelle définit complètement sa distribution de probabilité. Si une variable aléatoire admet une fonction de densité de probabilité, nous prouverons plus loin que la fonction caractéristique est alors la transformée de Fourier inverse de la fonction de densité de probabilité. Ainsi, elle fournit la base d'une voie alternative aux résultats analytiques par rapport au travail direct avec les fonctions de densité de probabilité ou les fonctions de distribution cumulative.
	
	Avant de donner une démonstration à la manière de l'ingénieur du théorème central limite, introduisons d'abord le concept de "fonction caractéristique" qui tient une place centrale en statistiques.
	
	Tout d'abord, rappelez-vous que la transformée de Fourier est donnée dans sa version physicienne (\SeeChapter{voir section Séquences et séries page \pageref {fourier transform}}) par la relation:
	
	Rappelons que la transformation de Fourier est un analogue de la théorie des séries de Fourier pour les fonctions non périodiques, et permet de leur associer un spectre en fréquences. Au facteur près, il s'agit d'une "\NewTerm{transformée de Laplace bilatérale}\index{transformée de Laplace bilatérale}" donnée par (\SeeChapter{voir section d'Analyse page \pageref{bilateral Laplace transform}}):
	
	avec $p$ qui est la variable complexe donnée dans le cas présent par (la partie réelle est nulle puisque la transformée de Fourier n'est que le cas particulier d'une transformée de Laplace dont la partie réelle de la variable est nulle: dont faire une transformée de Fourier c'est faire une transformée de Laplace sur l'axe des complexes uniquement):
	
	Nous souhaitons maintenant démontrer que si:
	
	En d'autres termes, nous cherchons une expression simplifiée de la transformée de Fourier de la dérivée de $f(x)$.
	
	\begin{dem}
	Nous partons donc de:
	
	Une intégration par parties donne: (\SeeChapter{voir section Calcul Différentiel et Intégral page \pageref{integration by parts}}):
	
	En imposant que, $f$ tend vers zéro à l'infini, nous avons alors:
	
	et:
	
	\begin{flushright}
		$\blacksquare$  Q.E.D.
	\end{flushright}
	C'est le premier résultat dont nous avions besoin!
	\end{dem}
	
	Maintenant, démontrons que si:
	
	
	\begin{dem}
	Nous partons donc de:
	
	\begin{flushright}
		$\blacksquare$  Q.E.D.
	\end{flushright}
	C'est le deuxième résultat dont nous avions besoin.
	\end{dem}
	
	Maintenant effectuons le calcul de la transformée de Fourier de la loi Normale centrée-réduite $\mathcal{N}(0,1)$ (ce choix n'est pas innocent...):
	
	Nous savons que cette dernière relation est trivialement solution de l'équation différentielle (ou bien elle vérifie):
	
	en prenant la transformée de Fourier des deux côté de l'égalité, nous avons en utilisant les deux résultats précédents:
	
	Nous avons:
	
	Ou encore:
	
	Donc après intégration:
	
	Puisque:
		
	nous avons donc:
	
	Nous avons démontré lors de notre étude de la loi Normale que:
	
	Donc:
	
	Nous avons alors (résultat très important!):
	
	Introduisons maintenant la fonction caractéristique telle que définie par les statisticiens:
	
	qui est un outil analytique important et puissant permettant d'analyser une somme de variables aléatoires indépendantes. De plus, cette fonction contient toutes les informations caractéristiques de la variable aléatoire $X$.
	\begin{tcolorbox}[title=Remarque,colframe=black,arc=10pt]
	La notation n'est pas innocente puisque le $\text{E}[...]$ représente une espérance de la fonction de densité par rapport à l'exponentielle complexe.
	\end{tcolorbox}
	Donc la fonction caractéristique de la variable aléatoire normale centrée réduite de distribution:
	
	devient simple à déterminer car:
	
	Raison pour laquelle la fonction caractéristique de la loi Normale centrée réduite est souvent assimilée à une simple transformée de Fourier (\SeeChapter{voir section Séquences et Séries page \pageref{fourier transform}}).
	
	Et grâce au résultat précédent:
	
	Donc:
	
	qui est le résultat dont nous avons besoin pour le théorème central limite. Cette fonction caractéristique est égale, à une constante près, à la densité de probabilité de la loi. Nous disons alors que la fonction caractéristique d'une gaussienne est gaussienne....
	
	Mais avant cela, regardons d'un peu plus près cette fonction caractéristique:
	
	En développement de Maclaurin nous avons  (\SeeChapter{voir section Séquences et Séries page \pageref{taylor series}}) et en changeant un peu les notations:
	
	et en intervertissant la somme et l'intégrale, nous avons:
	
	Cette fonction caractéristique contient donc tous les moments (terme général utilisé pour l'écart-type et l'espérance) de $X$.
	
	\pagebreak
	\subsubsection{Série génératrice de moments et cumulants}\label{moment generating function and cumulants}
	Dans la théorie des probabilités et  statistiques, la série génératrice de moments d'une variable aléatoire à valeur réelle est une spécification alternative de sa distribution de probabilité. Ainsi, elle fournit la base d'une voie alternative aux résultats analytiques par rapport au travail direct avec des fonctions de densité de probabilité ou des fonctions de distribution cumulative. Il existe des résultats particulièrement simples pour les séries génératrices des moments des distributions définies par les sommes pondérées de variables aléatoires. Cependant, toutes les variables aléatoires n'ont pas de séries génératrices de moments.
	
	Comme son nom l’indique, la série génératrice de moments (appelée autrefois "fonction génératrice") peut être utilisée pour calculer les moments d’une distribution: le $n$ème moment autour de $0$ est la $n$ème dérivée de la série génératrice de moments, évaluée en $0$.

	En plus des distributions à valeurs réelles (distributions univariées), des séries génératrices de moments peuvent être définies pour des variables aléatoires à valeurs vectorielles ou matricielles, et peuvent même être étendues à des cas plus généraux.

	La série génératrice de moments d'une distribution à valeur réelle n'existe pas toujours, contrairement à la fonction caractéristique. Il existe des relations entre le comportement de la série génératrice de moments d'une distribution et les propriétés de la distribution, telles que l'existence de moments.
	
	La série génératrice de moments d'une variable aléatoire $X$ est:
	
	partout où cette espérance existe. En d'autres termes, la série génératrice de moments de $X$ est l'espérance de la variable aléatoire $e^{t X}$.
	
	$M_{X}(0)$ existe toujours et est égal à $ 1 $ (voir la preuve ci-dessous). Cependant, un problème clé avec les séries génératrices de moments est que les moments et la série génératrice de moments peuvent ne pas exister, car les intégrales n'ont pas besoin de converger absolument. En revanche, la fonction caractéristique ou transformée de Fourier existe toujours (car elle est l'intégrale d'une fonction bornée sur un espace de mesure finie), et peut être utilisée à certaines fins à la place.
	
	La série génératrice de moments est ainsi nommée car elle peut être utilisée pour trouver les moments de la distribution. Le développement en série de $e^{t X}$ est:
	
	Dès lors:
	
	où $m_{n}$ est le $n$-ème moment. En différenciant $M_{X}(t)$ un nombre $i$ fois par rapport à $t$ et en fixant $t=0$, nous obtenons le $i$-ème moment sur l'origine, noté $m_{i}$.
	
	Si elle existe pour $t$ dans un voisinage de $0$, alors tous les "\NewTerm{moments}\index{moments}" de $X$ existent, et:
	
	La "\NewTerm{série génératrice cumulante}\index{série génératrice cumulante}\label{cumulant generating function}" est le logarithme de la série génératrice des moments:
	
	Cela génère les  "\NewTerm{cumulants}\index{cumulants}\label{cumulants}", qui sont définis par ce que la série génératrice de cumulants génère, c'est-à-dire, pour une variable aléatoire, le $k$-ème cumulant est:
	
	Les cumulants sont légèrement plus faciles à utiliser que les moments. Les quatre premiers sont:
	
	Le coefficient d'asymétrie et d'applatissement (voir page \pageref{skewness and kurtosis}) sont alors de simples fonctions des cumulants:
	
	Si $X$ est une variable aléatoire continue, la relation suivante entre sa série génératrice de moment $M_{X}(t)$ et la transformée de Laplace bilatérale sa fonction de densité de probabilité $f_{X}(x)$ est valable:
	
	étant donné que la transformée de Laplace de la fonction de densité de probabilité (\SeeChapter{voir section Analyse page \pageref{bilateral Laplace transform}}) est donnée par:
	
	et la "\NewTerm{série génératrice de moments}\index{série génératrice de moments}\label{moment-generating function}" (SGM) s'écrit alors (par la loi du statisticien inconscient...):
	
	La série génératrice de moments est l'espérance d'une fonction de la variable aléatoire, elle peut s'écrire:
	\begin{itemize}
		\item Pour une fonction de masse de probabilité discrète, $M_{X}(t)=\sum_{i=1}^{+\infty} e^{t x_{i}} p_{i}$ 
		
		\item Pour une fonction de masse de probabilité continue, $M_{X}(t)=\int_{-\infty}^{+\infty} e^{t x} f(x) \mathrm{d} x$
	\end{itemize}

	\begin{tcolorbox}[colframe=black,colback=white,sharp corners]
	\textbf{{\Large \ding{45}}Exemples:}\\\\
	E1. Prenons le cas de la distribution Normale.\\
	
	Supposons $X = \mathcal{N}\left(\mu, \sigma^{2}\right)$ (avec $\sigma^{2}>0$). Alors sa SGM est:
	$$
	\begin{aligned}
	M_{X}(t) &=\text{E}\left(e^{t X}\right)=\frac{1}{\sqrt{2 \pi} \sigma} \int\limits_{-\infty}^{+\infty} e^{t x} e^{-\frac{1}{2} \frac{(x-\mu)^{2}}{\sigma^{2}}} \mathrm{d} x =\frac{1}{\sqrt{2 \pi} \sigma} \int\limits_{-\infty}^{+\infty} e^{t x-\frac{1}{2} \frac{1}{\sigma^{2}}\left(x^{2}-2 \mu x+\mu^{2}\right)}\mathrm{d} x  \\
	&=\frac{1}{\sqrt{2 \pi} \sigma} \int\limits_{-\infty}^{+\infty} e^{-\frac{1}{2} \frac{1}{\sigma^{2}}\left(x^{2}-2\left(\mu+t \sigma^{2}\right) x+\mu^{2}\right)}\mathrm{d} x 
	\end{aligned}
	$$
	Dans l'exposant, complétons le carré par rapport au $x$ :
	$$
	x^{2}-2\left(\mu+t \sigma^{2}\right) x=\left(x-\left(\mu+t \sigma^{2}\right)\right)^{2}-\left(\mu+t \sigma^{2}\right)^{2}
	$$
	Alors:
	$$
	M_{X}(t)=e^{\frac{1}{2}\frac{1}{\sigma^{2}}\left(\left(\mu+t \sigma^{2}\right)^{2}-\mu^{2}\right)} \int\limits_{-\infty}^{+\infty} \frac{1}{\sqrt{2 \pi} \sigma} e^{-\frac{1}{2} \frac{1}{\sigma^{2}}\left(x-\left(\mu+t \sigma^{2}\right)\right)^{2}} \mathrm{d}x
	$$
	Notez que l'intégrande est la fonction de densité de probabilité de $\mathcal{N}\left(\mu+t \sigma^{2}, \sigma^{2}\right)$, ce qui signifie que l'intégrale vaut $1$. Étendre l'expression $\left(\mu+t \sigma^{2}\right)^{2}$ et en simplifiant un peu il vient:
	$$
	M_{X}(t)=e^{t \mu+\frac{1}{2} \sigma^{2} t^{2}}, t \in \mathbb{R}
	$$
	La série génératrice des cumulants est alors une simple quadrique:
	$$
	c_{X}(t)=t \mu+\frac{1}{2} \sigma^{2} t^{2}
	$$
	et il est facile de voir que:
	$$
	c_{X}^{\prime}(0)=\mu, \quad c^{\prime \prime}(0)=\sigma^{2}, \quad c^{\prime \prime \prime}(t)=0
	$$
	Ainsi, la moyenne est $\mu$ et la variance est $\sigma^{2}$ (sans surprise), et tous les autres cumulants sont $0$. En particulier, le coefficient d'asymétrie et d'aplatissement sont tous deux de $0$.\\
	
	E2. Voyons maintenant un exemple avec la distribution Gamma.\\
	
	La distribution Gamma a comme nous le savons deux paramètres: $\alpha>0$ est le paramètre de forme, et $\lambda>0$ est le paramètre d'échelle. Son support est $\mathcal{X}=[0, +\infty[$, et sa fonction de densité de probabilité est $c x^{\alpha-1} \exp (-\lambda x)$, où:
	$$
	\frac{1}{c}=\int\limits_{0}^{+\infty} x^{\alpha-1} e^{-\lambda x} \mathrm{d} x =\int\limits_{0}^{+\infty}(u / \lambda)^{\alpha-1} e^{-u}  \mathrm{d} u / \lambda=\frac{1}{\lambda^{\alpha}} \int\limits_{0}^{\infty} u^{\alpha-1} e^{-u} d u=\frac{1}{\lambda^{\alpha}} \Gamma(\alpha)
	$$
	\end{tcolorbox}
	
	\begin{tcolorbox}[colframe=black,colback=white,sharp corners]
	La fonction gamma est définie comme étant l'intégrale, telle que pour $\alpha>0$ (\SeeChapter{voir section Calcul Différentiel et Intégral page \pageref{gamma euler function}}):
	$$
	\Gamma(\alpha)=\int\limits_{0}^{+\infty} u^{\alpha-1} e^{-u} \mathrm{d} u
	$$
	Elle n'a généralement pas de solution analytique connue, mais il y a quelques faits pratiques que nous avons déjà prouvés au cours de son étude:
	$$
	\begin{aligned}
	\Gamma(\alpha+1) &=\alpha \Gamma(\alpha) \text { for } \alpha>0 \\
	\Gamma(n) &=(n-1) ! \text { for } n=1,2, \ldots ; \\
	\Gamma\left(\frac{1}{2}\right) &=\sqrt{\pi}
	\end{aligned}
	$$
	La fonction de densité de probabilité pour $\Gamma(\alpha, \lambda)$ est alors:
	$$
	f(x \mid \alpha, \lambda)=\frac{\lambda^{\alpha}}{\Gamma(\alpha)} x^{\alpha-1} e^{-\lambda x}, x \in(0, \infty)
	$$
	Si $\alpha=1$, nous retombons sur la "distribution exponentielle" $(\lambda)$, avec la fonction de densité de probabilité:
	$$
	g(x \mid \lambda)=\lambda e^{-\lambda x}, x \in \mathcal{X}=[0, +\infty[
	$$
	La SGM est alors:
	$$
	M_{X}(t)=\text{E}\left(e^{t X}\right) =\frac{\lambda^{\alpha}}{\Gamma(\alpha)} \int\limits_{0}^{+\infty} e^{t x} x^{\alpha-1} e^{-\lambda x}\mathrm{d}x =\dfrac{\lambda^{\alpha}}{\Gamma(\alpha)} \int\limits_{0}^{+\infty} x^{\alpha-1} e^{-(\lambda-t) x}\mathrm{d}x
	$$
	Cette intégrale a besoin de $(\lambda-t)>0$ pour être finie, donc nous avons besoin de $t<\lambda$, ce qui signifie que la SGM est finie pour un voisinage près de zéro, puisque $\lambda>0$. Maintenant, l'intégrale à la fin de la relation précédente est la même que celle rencontrée juste plus haut:
	$$\int\limits_{0}^{+\infty} x^{\alpha-1} \exp (-\lambda x) \mathrm{d} x$$
	à la différence que nous avons $\lambda-t$ à la place de $\lambda.$ Nous avons alors:
	$$
	\text{E}\left(e^{t X}\right) =\frac{\lambda^{\alpha}}{\Gamma(\alpha)} \frac{\Gamma(\alpha)}{(\lambda-t)^{\alpha}} =\left(\frac{\lambda}{\lambda-t}\right)^{\alpha}, t<\lambda
	$$
	Nous utiliserons la série génératrice de cumulants $c_{X}(t)=\log \left(M_{X}(t)\right)$ pour obtenir la moyenne et la variance, car c'est un peu plus facile. Ainsi:
	$$
	c_{X}^{\prime}(t)=\frac{\partial}{\partial t} \alpha(\log (\lambda)-\log (\lambda-t))=\frac{\alpha}{\lambda-t} \Longrightarrow \text{E}(X)=c_{X}^{\prime}(0)=\frac{\alpha}{\lambda}
	$$
	et:
	\end{tcolorbox}
	
	\begin{tcolorbox}[colframe=black,colback=white,sharp corners]
	$$
	c_{X}^{\prime \prime}(t)=\frac{\partial^{2}}{\partial t^{2}} \alpha(\log (\lambda)-\log (\lambda-t))=\frac{\alpha}{(\lambda-t)^{2}} \Longrightarrow \text{V}(X)=c_{X}^{\prime \prime}(0)=\frac{\alpha}{\lambda^{2}}
	$$
	En général, le $k$-ème cumulant est donné par:
	$$
	\gamma_{k}=\left.\frac{\partial^{k}}{\partial t^{k}} c_{X}(t)\right|_{t=0}=(k-1) ! \frac{\alpha}{\lambda^{k}}
	$$
	et en particulier:
	$$
	\begin{aligned}
	\text { Skewness }(X)&=\frac{2 \alpha / \lambda^{3}}{\alpha^{3 / 2} / \lambda^{3}}=\frac{2}{\sqrt{\alpha}} \\
	\text { Kurtosis }(X)&=\frac{6 \alpha / \lambda^{4}}{\alpha^{2} / \lambda^{4}} =\frac{6}{\alpha}
	\end{aligned}
	$$
	Ainsi, le coefficient d'asymétrie et d'aplatissement dépendent uniquement du paramètre de forme $\alpha$. En outre, ils sont positifs, mais tendent vers $0$ lorsque $\alpha$ augmente.\\
	
	E3. Voyons maintenant un exemple avec la distribution binomiale.\\
	
	La distribution binomiale est, comme nous le savons déjà, un modèle pour compter le nombre de succès dans  $n$ essais, par exemple, le nombre de faces dans dix lancés d'une pièce, où les essais sont indépendants et ont la même probabilité $p$ de succès:
	$$
	X = \mathcal{B}(n, p) \Longrightarrow f_{X}(x)=\left(\begin{array}{c}
	n \\
	x
	\end{array}\right) p^{x}(1-p)^{n-x}, \quad x \in \mathcal{X}=\{0,1, \ldots, n\}
	$$
	Le fait que la fonction de densité de probabilité soit égale à $1$ repose sur le théorème binomial (\SeeChapter{voir section Algèbre page \pageref{binomial theorem}}):
	$$
	(a+b)^{n}=\sum_{x=0}^{n} a^{x} b^{n-x}
	$$
	avec $a=p$ et $b=1-p$. Ce théorème aide également à trouver la SGM:
	$$
	\begin{aligned}
	M_{X}(t)&=\text{E}\left(e^{t X}\right)=\sum_{x=0}^{n} e^{t x} f_{X}(x)=\sum_{x=0}^{n} e^{t x}\left(\begin{array}{c}n \\x\end{array}\right) p^{x}(1-p)^{n-x} \\
	&=\sum_{x=0}^{n}\left(\begin{array}{c}n \\x	\end{array}\right)\left(p e^{t}\right)^{x}(1-p)^{n-x}=\left(p e^{t}+1-p\right)^{n}	
	\end{aligned}
	$$
	Qui est finie $t \in \mathbb{R}$, comme c'est le cas pour toute variable aléatoire bornée. Les deux premiers moments sont alors:
	$$
	\text{E}(X)=M_{X}^{\prime}(0)=\left.n\left(p e^{t}+1-p\right)^{n-1} p e^{t}\right|_{t=0}=n p
	$$
	et:
	$$
	\begin{aligned}
	\text{E}\left(X^{2}\right) &=M_{X}^{\prime \prime}(0)=\left.\left[n(n-1)\left(p e^{t}+1-p\right)^{n-2}\left(p e^{t}\right)^{2}+n\left(p e^{t}+1-p\right)^{n-1} p e^{t}\right]\right|_{t=0} \\
	&=n(n-1) p^{2}+n p
	\end{aligned}
	$$
	\end{tcolorbox}
	
	\begin{tcolorbox}[colframe=black,colback=white,sharp corners]
	Dès lors:
	$$
	\begin{aligned}
	\text{V}(X)&=\text{E}\left(X^{2}\right)-E(X)^{2} =n(n-1) p^{2}+n p-(n p)^{2}=n p-n p^{2}=n p(1-p)\end{aligned}
	$$
	\end{tcolorbox}
	
	\pagebreak
	\subsection{Théorème Central Limite}\label{central limit theorem}
	Le théorème central limite est un ensemble de résultats du début du 20ème siècle sur la convergence faible d'une suite de variables aléatoires en probabilité. Intuitivement, d'après ces résultats, toute somme  (implicitement: la moyenne de ses variables) de variables aléatoires indépendantes et identiquement distribuées tend vers une certaine variable aléatoire. Le résultat le plus connu et le plus important est simplement appelé "\NewTerm{théorème central limite}\index{théorème central limite}" qui concerne une somme de variables aléatoires indépendantes avec variance existante dont le nombre tend vers l'infini et c'est celui-ci que nous allons démontrer de manière heuristique ici.
	
	Dans le cas le plus simple, considéré ci-dessous pour la démonstration du théorème, ces variables sont continues, indépendantes et possèdent la même moyenne et la même variance. Pour tenter d'obtenir un résultat fini, il faut centrer cette somme en lui soustrayant sa moyenne et la réduire en la divisant par son écart-type. Sous des conditions assez larges, la loi de probabilité (de la moyenne) converge alors vers une loi Normale centrée réduite. L'omniprésence de la loi Normale s'explique par le fait que de nombreux phénomènes considérés comme aléatoires sont dus à la superposition de causes nombreuses.

	Ce théorème de probabilités possède donc une interprétation en statistique mathématique. Cette dernière associe une loi de probabilité à une population. Chaque élément extrait de la population est donc considéré comme une variable aléatoire et, en réunissant un nombre $n$ de ces variables supposées indépendantes, nous obtenons un échantillon. La somme de ces variables aléatoires divisée par n donne une nouvelle variable nommée la moyenne empirique. Celle-ci, une fois réduite, tend vers une variable Normale réduite lorsque $n$ tend vers l'infini comme nous le savons.
	
	Le théorème central limite nous dit à quoi il faut s'attendre en matière de sommes de variables aléatoires indépendantes. Mais qu'en est-il des produits? Eh bien, le logarithme d'un produit (à facteurs strictement positifs) est la somme des logarithmes des facteurs, de sorte que le logarithme d'un produit de variables aléatoires (à valeurs strictement positives) tend vers une loi Normale, ce qui entraîne une loi log-Normale pour le produit lui-même.
	
	En elle-même, la convergence vers la loi Normale ("\NewTerm{normalité asymptotique}\index{normalité asymptotique}") de nombreuses sommes de variables aléatoires lorsque leur nombre tend vers l'infini n'intéresse que le mathématicien. Pour le praticien, il est intéressant de s'arrêter un peu avant la limite: la somme d'un grand nombre de ces variables est presque gaussienne, ce qui fournit une approximation souvent plus facilement utilisable que la loi exacte.
	
	En s'éloignant encore plus de la théorie, on peut dire que bon nombre de phénomènes naturels sont dus à la superposition de causes nombreuses, plus ou moins indépendantes. Il en résulte que la loi Normale les représente de manière raisonnablement efficace.
	
	A l'inverse, on peut dire qu'aucun phénomène concret n'est vraiment Gaussien car il ne peut dépasser certaines limites, en particulier s'il est à valeurs positives.
	
	\begin{dem}
	Soit $\left\lbrace X_i \right\rbrace_{i=1...+\infty}$ une suite (échantillon) de variables aléatoires continues (dans notre démonstration simplifiée...), indépendantes (mesures de phénomènes physiques ou mécaniques indépendants par exemple) et identiquement distribuées, dont la moyenne $\mu_X$ et l'écart-type $\sigma_X$ existent (ce qui signifie que le théorème central limite fonctionne que pour les phénomènes à variance finie!!!).
	
	Nous avons vu au début de cette section que:
	
	sont les mêmes expressions d'une variable centrée réduite générée à l'aide d'une suite de $n$ variables aléatoires identiquement distribuées qui par construction a donc une moyenne nulle et une variance unitaire:
	
	Développons la première forme de l'égalité antéprécédente (les 2 sont de toute façon égales!):
	
	Maintenant utilisons la fonction caractéristique de la loi Normale centrée-réduite (nous allégeons par la même occasion l'écriture des estimateurs de la moyenne et de l'écart-type):
	
	Comme les variables aléatoires $X_i$ sont indépendantes et identiquement distribuées, il vient:
	
	Un développement de Taylor (\SeeChapter{voir section Séquences et Séries page \pageref{taylor series}}) du terme entre accolades donne au troisième ordre (développement en série de Maclaurin de l'exponentielle):
	
	Finalement:
	
	Posons:
	
	Nous avons alors:
	
	Et donc quand $x$ tend vers l'infini (\SeeChapter{voir section d'Analyse Fonctionnelle page \pageref{limits}}):
	
	Nous retrouvons donc la fonction caractéristique de la loi Normale centrée réduite!
	
	En deux mots, le Théorème Central Limite (TCL) dit que pour de grands échantillons, la somme centrée et réduite de $n$ variables aléatoires identiquement distribuées suit une loi Normale centrée et réduite. Et donc nous avons in extenso pour la moyenne empirique:
	
	Ou plus élégamment le TCL s'écrira:
	
	\begin{flushright}
		$\blacksquare$  Q.E.D.
	\end{flushright}
	\end{dem}
	
	Maintenant, nous allons illustrer le théorème central limite dans le cas d'une suite $\left\lbrace X_i \right\rbrace$ de variables aléatoires indépendantes discrètes suivant une loi de Bernoulli de paramètre $1/2$.
	
	Nous pouvons imaginer que $X_n$ représente le résultat obtenu au $n$-ème lancé d'une pièce de monnaie (en attribuant le nombre $1$ pour pile et $0$ pour face). Notons:
	
	la moyenne. Nous avons pour tout $n$ bien évidemment:
	
	et donc:
	
	Après avoir centré et réduit $\bar{X}_n$ nous obtenons:
	
	Notons $\Phi$ la fonction de répartition de la loi Normale centrée réduite.
	
	Le théorème central limite nous dit que pour tout $t \in \mathbb{R}$:
	
	A l'aide de Maple 4.00b nous avons tracé en bleu quelques graphiques de la fonction:
	
	pour différentes valeurs de $n$. Nous avons représenté en rouge la fonction $\Phi$.
	
	Pour $n=1$:
	\begin{figure}[H]
		\centering
		\includegraphics{img/arithmetics/clt_bernoulli_n_1.jpg}
		\caption[Approche de la distribution de Bernoulli par la distribution normale selon le CLT]{Première approche de la loi de Bernoulli par le loi Normale selon le TCL}
	\end{figure}
	Pour $n=2$:
	\begin{figure}[H]
		\centering
		\includegraphics{img/arithmetics/clt_bernoulli_n_2.jpg}
		\caption[]{Deuxième approche de la loi de Bernoulli par le loi Normale selon le TCL}
	\end{figure}
	Pour $n=5$:
	\begin{figure}[H]
		\centering
		\includegraphics{img/arithmetics/clt_bernoulli_n_5.jpg}
		\caption[]{Cinquième approche de la loi de Bernoulli par le loi Normale selon le TCL}
	\end{figure}
	Pour $n=30$:
	\begin{figure}[H]
		\centering
		\includegraphics{img/arithmetics/clt_bernoulli_n_30.jpg}
		\caption[]{Trentième approche de la loi de Bernoulli par le loi Normale selon le TCL}
	\end{figure}
	Ces graphiques ont été obtenus avec Maple 4.00b à l'aide des commandes suivantes:
	
	\texttt{>with(stats):\\
	>with(plots):\\
	>e1:=plot(Heaviside(t+1)*statevalf[dcdf,binomiald[1,0.5]](trunc((t+1)/2))\\
	,t=-2..2,y=0..1,color=blue):\\
	>e2:=plot(Heaviside(t+sqrt(2))*statevalf[dcdf,binomiald[2,0.5]]\\
	(trunc((t*sqrt(2)+2)/2)),t=-sqrt(2)-1..sqrt(2)+1,y=0..1,color=blue):\\
	>e3:=plot(Heaviside(t+sqrt(5))*statevalf[dcdf,binomiald[5,0.5]]\\
	(trunc((t*sqrt(5)+5)/2)),t=-sqrt(5)-1..sqrt(5)+1,y=0..1,color=blue):\\
	>e4:=plot(statevalf[cdf,normald](t),t=-5..5):\\
	>e5:=plot(Heaviside(t+sqrt(30))*statevalf[dcdf,binomiald[30,0.5]]\\
	(trunc((t*sqrt(30)+30)/2)),t=-sqrt(30)-1..sqrt(30)+1,y=0..1,color=blue):\\
	>display({e1,e4});\\
	>display({e2,e4});\\
	>display({e4,e3});\\
	>display({e5,e4});}
	
	montrent bien la convergence de $F_n$ vers $\Phi$.
	
	En fait nous remarquons que la convergence est carrément uniforme ce qui est confirmé par le "\NewTerm{théorème central limite de Moivre-Laplace}\index{théorème central limite de Moivre-Laplace}":
	
	Soit $X_n$ une suite de variables aléatoires indépendantes de même loi de Bernoulli de paramètre  $p$, $0<p<1$. Alors:
	
	tend uniformément vers $\Phi(t)$ sur $\mathbb{R}$ quand $n \rightarrow +\infty$.
	
	\subsection{Tests d'hypothèses univariés et d'adéquations (NHST)}\label{nhst}
	Lors de notre étude des intervalles de confiance, rappelons que nous sommes arrivés aux quelques relations suivantes (ce n'est que l'échantillon des plus importantes démontrées plus haut!):
	
	et:
	
	et:
	
	et finalement:
	
	qui permettaient donc de faire de l'inférence statistique en fonction de la connaissance ou non de la moyenne ou de la variance vraie sur la totalité ou sur un échantillon de la population. En d'autres termes de savoir dans quelles bornes se situait un moment (moyenne ou variance) en fonction d'un certain niveau de confiance $\alpha$ imposé. Nous avions vu que le deuxième intervalle ci-dessus ne peut être que difficilement utilisé dans la pratique (suppose la moyenne théorique connue) et nous lui préférons donc souvent le troisième dans la pratique.
	
	Nous allons également démontrer en détails plus loin les deux intervalles suivants:
	
	et:
	
	Le premier intervalle ci-dessus ne peut être lui aussi que difficilement utilisé dans la pratique (suppose la moyenne théorique connue) et nous lui préférons donc le deuxième.
	
	\textbf{Définition (\#\mydef):} Lorsque nous cherchons à savoir si nous pouvons faire confiance à la valeur d'une statistique (moyenne, médiane, variance, coefficient de corrélation, etc.) avec une certaine certitude, nous parlons de "\NewTerm{test d'hypothèse}\index{test d'hypothèse}" et plus particulièrement de "\NewTerm{test de conformité}\index{test de conformité}" (nous parlons de "\NewTerm{test d'adéquation}\index{test d'adéquation}" quand il s'agit de vérifier que des mesures semblent bien suivre une loi donnée et non juste une statistique).
	
	\begin{tcolorbox}[title=Remarque,colframe=black,arc=10pt]
	Le lecteur doit également se rappeler, comme déjà dit précédemment dans cette section, que nous avons mis de nombreuses autres preuves mathématiques détaillés de calculs d'intervalles de confiance liées par exemple aux techniques de régression dans la section de Méthodes Numériques.
	\end{tcolorbox}	
	
	Les tests d'hypothèses sont destinés à vérifier si un échantillon peut être considéré comme extrait d'une population donnée ou représentatif de cette population, vis-à-vis d'un paramètre comme la moyenne, la variance ou la fréquence observée. Ceci implique que la loi théorique du paramètre soit connue au niveau de la population. Les tests d'hypothèses ne sont pas faits pour démontrer l'hypothèse nulle (exprimant généralement une égalité ou une homogénéité entre différentes populations), mais pour éventuellement la rejeter (dispons pour être exact que le rejet est plus robuste). Au niveau de la communication des tests statistiques un certain nombre de spécialistes recommandent:
	
	\begin{enumerate}
		\item De toujours communiquer la $p$-valeur avec $4$ chiffres après la virgule (nous reviondrons plus loin sur ce concept).
		
		\item De ne jamais dire qu'une $p$-valeur faible montre une amplitude importante de l'effet étudié car cela n'est pas forcéement vrai (pour le vérifier il suffit de prendre un phénomène de très petite amplitude sur une gros échantillon et la $p$-valeur deviendra toute de suite très petite par construction).
		
		\item De toujours donner l'intervalle de confiance du test qu'il soit unilatéral ou bilatéral.
		
		\item De bien se garder de fixer un seuil de rejet au test excepté si une norme ou législation l'impose (dans ce dernier cas on précisera laquelle).
		
		\item De ne jamais dire que le test est "démontré", ou "significatif" ou même "statistiquement significatif". Juste dire que le résultat est "statistique" ou que nous avons la "probabilité des données connaissant l'hypothèse nulle" et c'est tout!
		
		 \item Si l'intérêt est de montrer l'hypothèse nulle et que cette dernière n'est pas rejetée, étant donné souvent sa puissance statistique faible, il faudra répéter l'expérience pour conforter la conclusion.
		 
		 \item Si l'intérêt est de rejeter l'hypothèse nulle et que cela se vérifie, une bonne pratique scientifique est de chercher des études supplémentaires qui mettraient en défaut la conclusion.
		 
		 \item S'il y a absence par exemple de différence statistique entre deux valeurs, cela ne signfie pas pour autant qu'il y ait présence statistique d'équivalence. Il faut alors procéder à des "tests d'équivalences".
		 
		 \item La rejet de l'hypothèse nulle ne signifie pas que le méchanisme du phénomène étudié a été mis en évidence mais indique juste pour rappel une information de taille sur les données a posteriori (ou en termes bayésiens: \textit{la vraisemblance des données n'est pas la crédence de la théorie.})!
		 
		 \item Nous communiquons la puissance a posteriori du test.
	\end{enumerate}
	Bref, les études doivent être diffisusées en respectant le principe de véracité, après avoir fait l'objet des vérifications de rigueur, et doivent être exposées, décrites et présentées avec impartialité (certains spécialistes parlent alors de "\NewTerm{a-théorique}\index{a-théorique}"). Il ne faut pas confondre résultats objectifs et spéculations. Les conclusions doivent être l'expression le plus fidèle possible du contenu des faites et des données.
	
	Par exemple, si nous souhaitons savoir avec une certaine confiance si une moyenne donnée d'un échantillon de population est réaliste par rapport à la vraie moyenne théorique inconnue, nous utiliserons le "\NewTerm{test-$Z$}\index{test-$Z$}" qui est simplement:
	
	Maintenant rappelons que nous avons démontré que si nous avions deux variables aléatoires de loi:
	
	alors la soustraction (différencier) des moyennes donne:
	
	Donc pour la différence de deux moyennes de variables aléatoires provenant de deux échantillons de population nous obtenons directement:
	
	Nous pouvons alors adapter le test-$Z$ sous la forme:
	
	La relation qui est très utile lorsque pour deux échantillons de deux populations de données, nous voulons vérifier s'il existe une différence statistiquement significative des différences des moyennes théoriques à un niveau de confiance equation fixé et la probabilité associée pour avoir cette différence:
	
	Donc:
	
	Nous parlons du "\NewTerm{test-$Z$ de la moyenne à deux échantillons}\index{test-$Z$ de la moyenne à deux échantillons}" et il est beaucoup utilisé dans l'industrie pour vérifier l'égalité de la moyenne de deux populations de mesures.
	
	Et si l'écart-type théorique n'est pas connu, nous utiliserons le "\NewTerm{test-$T$ de Student}\index{test-$T$ de Student}" (pas mal utilisé en pharmaco-économie) démontré plus haut:
	
	Dans la même idée pour l'écart-type, nous utiliserons le "\NewTerm{test du Khi-deux de la variance}\index{test du Khi-deux(de la variance}" aussi déjà démontré plus haut:
	
	Et lorsque nous voulons tester l'égalité de la variance de deux populations nous utilisons le "\NewTerm{Test $F$ de Fisher}\index{Test $F$ de Fisher}" de Fisher (démontré plus bas lors de notre étude de l'analyse de la variance):
	
	Dans la pratique il faut avoir conscience que le but d'un test est très très souvent de montrer que l'effet est significatif. Il est alors d'usage de dire que le test réussit si l'hypothèse nulle est rejetée au profit de l'hypothèse alternative. Lorsque le praticien sait que l'effet est significatif et pourtant que son test échoue à rejeter l'hypothèse nulle on parle parfois du "\NewTerm{dilemne du non rejet de l'hypothèse nulle}\index{dilemne du non rejet de l'hypothèse nulle}". Comme nous le verrons un peu plus loin, l'idée est alors de calculer à posteriori la puissance du test (celle-ci étant alors appelée par certains logiciels comme SPSS: "\NewTerm{puissance observée}\index{}") et d'adapter la taille de l'échantillon en conséquence pour avoir une puissance acceptable selon la tradition d'usage.
	
	\subsubsection{Orientation du test d'hypothèse et $p$-valeurs}\label{p value}\index{$p$-value}
	Le fait que nous obtenions l'ensemble des valeurs satisfaisant à un testborné à droite et (!) à gauche est ce que nous appelons dans le cas général un "\NewTerm{test bilatéral}\index{test bilatéral}" car il comprend le test unilatéral à gauche et unilatéral à droite. Ainsi, tous les tests susmentionnés sont dans une forme bilatérale mais nous pourrions en faire une utilisation unilatérale aussi! Nous utilisons un test unilatéral lorsque la différence attendue (ou à mettre en évidence) ne peut aller que dans un sens (typiquement dans le cas des essais cliniques ou lors d'un action corrective de contrôle qualité en industrie pour laquelle nous nous attendons à une amélioration allant dans une unique direction). Les test unilatéraux sont parfois nommés "\NewTerm{tests de non-infériorité}\index{tests de non-infériorité}" (unilatéral gauche) ou "\NewTerm{test de non-supériorité}\index{test de non-supériorité}" (unilatéral droite).
	
	Ci-dessous, nous avons représenté par exemple un test unilatéral à droite (car la région de rejet est à droite et donc la probabilité cumulée est unilatérale gauche) et un test bilatéral de $5\%$ (depuis le début du 21ème siècle, de plus en plus de praticiens recommandent de prendre une valeur de $ 1 \% $):
	\begin{figure}[H]
		\centering
		\includegraphics{img/arithmetics/unilateral_bilateral_test.jpg}
		\caption{ Illustration d'un test (ou intervalle de confiance) unilatéral à droite et bilatéral}
	\end{figure}
	Nous pouvons également résumer la manière de déterminer la $p$-valeur (sur laquelle nous reviendrons plus loin en détail) par le logigramme suivant:
	\begin{figure}[H]
		\centering
		\includegraphics{img/arithmetics/p_values_construction.jpg}
		\caption{Figure de résumé pour déterminer la $p$-valeur lors de tests paramétriques à distribution symétrique}
	\end{figure}
	
	\textbf{Définition (\#\mydef):} La "\NewTerm{$p$-valeur}\index{$p$-valeur}" est la probabilité d'obtenir un effet au moins aussi extrême que celui de vos données d'échantillon, en supposant l'hypothèse nulle vraie. En d'autres termes, si l'hypothèse nulle est vraie, la valeur $p$ est la probabilité d'obtenir vos données d'échantillon. Cela répond à la question: vos données d'échantillonnage sont-elles inhabituelles si l'hypothèse nulle est vraie? Si vous pensez que la valeur $p$ est la probabilité que l'hypothèse nulle soit vraie, la probabilité que vous commettiez une erreur si vous rejetez la valeur nulle, ou toute autre chose dans ce sens, alors vous tombé dans malentendu le plus courant des praticiens.
	
	Comme déjà mentionné, la valeur $p$-valeur est souvent considérée comme significative si elle est inférieure à $5\%$ ou même mieux quand ... en dessous de $1\% $. En physique des particules, selon la distribution de Gauss (pour laquelle il est facile de convertir des pourcentages en écart-type comme nous le savons déjà!), La valeur $p$-valeur est souvent communiquée en $\sigma$ comme le montre l'exemple aléatoire ci-dessous de études de découverte du pentaquark en  2003:
	\begin{figure}[H]
		\centering
		\includegraphics[scale=0.5]{img/arithmetics/sigma_pvalues_pentaquark.jpg}
	\end{figure}
	ou un autre exemple aléatoire (aléatoire bien connu ...) de la découverte de confirmation de Higgs Boson au LHC en 2012\footnote{Le standard de certitude dont les physiciens des particules ont besoin au début du 21ème siècle pour une découverte est d'environ $1$ dans $3.5$ millions de dollars, soit $5\sigma$.}:
	\begin{figure}[H]
		\centering
		\includegraphics[scale=0.4]{img/arithmetics/sigma_pvalues_higgs_boson.jpg}
	\end{figure}
	L'idée fausse commune est ce que nous aimerions vraiment savoir! Nous aimerions beaucoup savoir la probabilité qu'une hypothèse soit correcte, ou la probabilité de faire une erreur. Ce que nous obtenons à la place avec les méthodes fréquentistes, c'est la probabilité de notre observation, qui n'est tout simplement pas aussi utile.
	
	Ce serait formidable si nous pouvions avoir des évidences uniquement à partir d'un échantillon et déterminer la probabilité que l'échantillon soit faux. Malheureusement, cela ne semble pas possible avec les méthodes fréquentistes et cela pour des raisons logiques quand on y pense. Sans informations extérieures, un échantillon ne peut pas vous dire s'il est représentatif de la population.
	
	Les $p$-valeurs sont basées exclusivement sur les informations contenues dans un échantillon. Par conséquent, les $p$-valeurs  ne peuvent pas répondre à la question à laquelle nous voulons le plus répondre, mais il semble y avoir une tentation irrésistible de l'interpréter de cette façon.
	
	Notons également que les tests d'hypothèse sur l'écart type (variance), la moyenne ou la corrélation sont nommés "\NewTerm{tests paramétriques}\index{tests paramétriques}" à l'inverse des tests non paramétriques que nous verrons bien plus loin.
	
	\begin{tcolorbox}[title=Remarques,colframe=black,arc=10pt]
	\textbf{R1.}  Il existe également une autre définition du concept de test paramétrique et non-paramétrique (un peu différente car plus précise) à voir plus loin...\\
	
	\textbf{R2.} Attention! Certains auteurs ou professeurs parlent parfois de test "unilatéral à gauche" pour un "test unilatéral à droite"... Au fait il s'agit simplement d'un choix de vocabulaire. Si la référence pédagogique n'est pas la zone de rejet mais la zone d'acception, alors il est clair que les concepts de droite et gauche s'inversent...
	\end{tcolorbox}	
	Enfin, de nombreux logiciels calculent donc ce que nous appelons la "\NewTerm{$p$-valeur}" qui est le risque calculé (probabilité) $\alpha$ qu'aurait pu fixer le statisticien pour être à la limite entre l'acceptation de l'hypothèse nulle et son rejet (rappelons qu'un test qui réussit ne prouve rien). La $p$-valeur est donc une valeur fondamentale dans le domaine car elle permet de chiffrer la vraisemblance de l'hypothèse nulle equation (acception ou rejet).
	
	\textbf{Définition (\#\mydef):} Mais en toute rigueur la "\NewTerm{$p$-valeur}\index{$p$-valeur}" est la probabilité conditionnelle (bayésienne), que nos données satisfont l'hypothèse nulle (que $H_0$ est vraie\footnote{De manière équivalente, nous pouvons dire que c'est la probabilité conditionnelle d'obtenir des données aussi extrêmes que celle qui ont éé réellement observée, en supposant que l'hypothèse nulle est vraie! Une autre belle façon de décrire, c'est de dire que la valeur $p$ est la proportion de fois où vous verriez des preuves plus fortes que ce qui a été observé, par rapport à l'hypothèse nulle, si l'hypothèse nulle était vraie et que vous répétiez hypothétiquement l'expérience (échantillonnage d'individus d'une population) un grand nombre de fois!}), ie la vraisemblance des résultats:
	
	 et non la probabilité de l'hypothèse nulle connaissant les données (plausibilité de $H_0$, ie $P(H_0|\text{data})$)!:
		
	Même si la différence peut être faible comme nous l'avons vu dans le section de Probabilités, elle n'en est pas moins non nulle! Donc la $p$-valeur en réalité ne dit rien sur l'hyphothèse elle-même, mais elle donne une information sur les données expérimentales si l'expérience est menée un nombre infini de fois et que toutes les hypothèses pertinentes sont valables! Donc, comme une valeur $p$ ne donne des informations que dans le cas où nous répéterions l'expérience un grand nombre de fois, nous comprenons mieux pourquoi les valeurs $p$ sont souvent décriées dans des études à expérience unique...
	 
	 Comme l'a dit Ronald Fisher: \textit{Un fait scientifique ne doit être considéré comme expérimentalement établi que si une expérience correctement conçue ne donne que rarement ce niveau de signification}. Cela signifie que la $p$-valeur inférieure à $0.05$ indique seulement qu'il faut répéter l'expérience!
	 
	 Il est vrai que plus la $p$-valeur est petite, plus les données seraient inhabituelles si chaque hypothèse était correcte; mais une très petite $p$-valeur ne nous dit pas quelle hypothèse est incorrecte. Par exemple, la $p$-valeur peut être très petite car l'hypothèse ciblée est fausse; mais elle peut à la place (ou en plus) être très petite parce que les protocoles d'étude ont été violés, ou parce qu'il a été sélectionné pour présentation en fonction de sa petite taille. Inversement, une grande $p$-valeur indique seulement que les données ne sont pas inhabituelles sous le modèle, mais n'implique pas que le modèle ou l'un de ses aspects (comme l'hypothèse ciblée) est correct; elle peut à la place (ou en plus) devenir grande car (encore une fois) les protocoles d'étude ont été violés, ou parce qu'elle a été sélectionnée pour la présentation en fonction de sa grande taille!
	 
	 La définition générale d'une $p$-valeur  peut aider à comprendre pourquoi les tests statistiques nous en disent beaucoup moins que ce que beaucoup pensent qu'ils font: Non seulement la $p$-valeur ne nous dit pas si l'hypothèse ciblée pour le test est vraie ou pas; mais elle ne dit rien de spécifiquement lié à cette hypothèse, à moins que nous ne puissions être complètement assurés que toutes les autres hypothèses utilisées pour son calcul sont correctes, une assurance qui fait défaut dans beaucoup trop d'études.
	 
	Surtout gardez à l'esprit (parce que beaucoup de gens ont un biais cognitif naturel à ce sujet ...) qu'en général:
	
	Mais ces deux probabilités (vraisemblance à gauche et plausibilité à droite) sont cependant liées par le théorème de Bayes (\SeeChapter{voir section page Probabilités \pageref{bayes formula}}):
	
	Mais les deux termes $P(H_0)$ (probabilité a prorio du phénomène observé) et $P(\text{données}) $ sont en pratique difficiles voire impossibles à connaître.
	
	Pour un test d'hypothèse, par exemple, le $5\%$ de risque $\alpha$ est celui de rejeter l'hypothèse nulle  $H_0$ alors même qu'elle est vraie. Si le risque imposé/choisi est $5\%$ et que la $p$-valeur calculée est inférieure (dans la majorité des tests mais il faut être prudent car ce n'est pas une généralité!!!), le test échoue (rejet de l'hypothèse nulle) en faveur d'une hypothèse alternative notée $H_1$ ou parfois $H_a$. N'oubliez jamais que rejeter le test vaut toujours mieux en terme de pouvoir que de l'accepter. En d'autres termes, nous pouvons dire que la valeur $ p $ est la plus petite $\alpha$ à laquelle nous rejetterions $H_0$.
	
	L'hypothèse alternative $H_1$ a bien évidemment elle-même son propre risque que nous notons $\beta$ et sa propre $p$-valeur. Donc lorsque l'hypothse nulle n'est pas rejetée, le risque associé à cette décision est un "\NewTerm{risque de deuxième espèce}\index{risque de deuxième espèce}". Pour l'évaluer, il faudrait donc calculer le puissance du test considéré (voir les démonstrations mathématiques plus bas).
	
	Peut-être, pour mieux comprendre, voici une illustration d'un cas particulier d'un test d'hypothèse bilatéral de la moyenne pour une variable aléatoire suivant typiquement une loi Normale (en gros c'est le même principe pour tous les tests...):
	\begin{figure}[H]
		\centering
		\includegraphics{img/arithmetics/alpha_risk_beta_risk.jpg}
		\caption{Hypothèse nulle et alternative d'un test bilatéral particulier}
	\end{figure}
	Ainsi, dans le cas présenté ci-dessus, nous voyons mieux pourquoi l'hypothèse nulle peut donc être acceptée ou rejetée en faveur de l'hypothèse alternative (qui est de même loi que l'hypothèse nulle mais juste décalée) dépendant de la valeur de référence mesurée qui sera utilisée pour le test (en l'occurence dans le cas particulier il s'agit de la moyenne arithmétique des mesures).
	
	Nous remarquons aussi que la zone rouge de l'hypothèse alternative, correspondant à la probabilité cumulée $\beta$, est confondue en partie avec la partie jaune de l'hypothèse nulle. Raison pour laquelle nous pouvons parfois accepter l'hypothèse nulle à tort. Nous voyons cependant que plus $\beta$ serait petit, plus l'hypothèse alternative serait donc éloignée de la zone limite rouge de l'hypothèse nulle (cela correspondrait à une translation vers la droite dans le cas présent) et moins la probabilité de faire une fausse conclusion est grande. Raison pour laquelle nous parlons de "\NewTerm{risque $\beta$}\index{risque $beta$}" car plus celui-ci est petit, mieux c'est. In extenso, plus $1-\beta$ est grand, moins il y a de risque de confondre l'hypothèse nulle et alternative. Raison pour laquelle $1-\beta$ est appelé "\NewTerm{puissance du test}\index{puissance du test}" (voir plus bas la section qui est consacrée à cette notion).
	
	Nous ne rejettons pas (à contrecoeur) l'hypothèse nulle souvent si la $p$-valeur est plus grande que $5\%$ ($0.05$). Au fait, plus la $p$-valeur est grande, mieux c'est car l'intervalle de confiance est de plus en plus petit. Si l'intervalle de confiance vient à être énorme (très proche de $100\%$) car la $p$-valeur est très petite alors l'analyse n'a plus vraiment de sens physiquement parlant!
	
	Ainsi, si la $p$-valeur est petite, c'est qu'il faudrait prendre un risque $\alpha$ faible de se tromper, donc accepter (ie ne pas rejeter) dans presque tous les cas l'hypothèse testée ($H_0$)...
	
	\begin{tcolorbox}[title=Remarques,colframe=black,arc=10pt]
	\textbf{R1.} Nous ne devrions jamais dire que nous "acceptons" une hypothèse ou encore qu'elle est "vraie" ou "fausse" car ces termes sont trop forts et pourraient faire penser à une preuve scientifique. Nous devrions dire si nous "rejetons" ou "ne rejetons pas" l'hypothèse nulle et qu'elle est éventuellement "correcte" ou "non correcte".\\
	
	\textbf{R2.} Pour les test d'hypothèses bilatéraux, nous pouvons par exemple dire que nous avons (ou n'avons pas) une différence significative entre la valeur de référence mesurée et la valeur attendue. Pour les tests unilatéraux, nous pouvons dire que la valeur de référence mesurée est significativement plus grande ou plus petite que la valeur attendue.\\
	
	\textbf{R3.} Par ailleurs si le lecteur a bien compris la construction des tests d'hypothèses, le fait de rejeter une hypothèse à tort ("\NewTerm{TErreur de Type I}\index{erreur de Type I}" ou "\NewTerm{Erreur de première espèce}\index{erreur de première espèce}") est donc plus robuste que de l'accepter à tort ("\NewTerm{Erreur de type II}\index{erreur de type II}" ou "\NewTerm{Erreur de deuxième espèce}\index{Erreur de deuxième espèce}").\\
	
	\textbf{R4.} Le lecteur remarquera aussi en s'aidant de la figure précédent qu'un test unilatéral a une plus forte puissance qu'un test unilatéral (a même niveau de risque bien entendu!). Ainsi, une différence non statistiquement significative en test bilatéral, peut s'avérer statistiquement significative en unilatéral.\\
	
	\textbf{R5.} Si la $p$-valeur est proche de la valeur limite (de rejet), nous disons que "\NewTerm{l'effet est marginalement significatif}\index{effet est marginalement significatif}".
	\end{tcolorbox}	
	
	Un gros problème avec les tests d'hypothèses est ce que nous appelons le "\NewTerm {$p$-hacking}\index{$p$-hacking}": l'idée sous-jacente consiste à répliquer un test des dizaines de fois jusqu'à ce qu'il aboutisse à la conclusion que l'expérimentateur veut ... ou essayer toutes les combinaisons possibles de variables jusqu'à ce que nous trouvions quelque chose de significatif ou même que nous prenions simplement de gros échantillons (car nous savons que les tests NHST échouent systématiquement lorsque les échantillons sont gros).
	\begin{figure}[H]
		\centering
		\includegraphics[width=1.0\textwidth]{img/arithmetics/p_hacking.jpg}
		\caption[$p$-hacking]{$p$-hacking Open Science Collaboration, \textit{Estimating the reproducibility of physiological science}, Science, 349 (2015)}
	\end{figure}
	Le $p$-hacking est généralement considéré comme de la triche, mais que se passerait-il si nous le rendions obligatoire à la place? Si le but des études est de repousser les frontières de la connaissance, alors peut-être que jouer avec différentes méthodes ne devrait pas être considéré comme un sale tour, mais encouragé comme un moyen d'explorer les frontières. Un récent projet dirigé par Brian Nosek, fondateur du Center for Open Science, a offert un moyen intelligent de le faire.

	Vingt-neuf équipes avec un total de $61$ analystes y ont participé. Les chercheurs ont utilisé une grande variété de méthodes, allant - pour ceux d'entre vous intéressés par le gore méthodologique - des techniques de régression linéaire simples aux régressions multiniveaux complexes et aux approches bayésiennes. Ils ont également pris différentes décisions concernant les variables secondaires à utiliser dans leurs analyses.

	Malgré l'analyse des mêmes données, les chercheurs ont obtenu une variété de résultats. Vingt équipes ont conclu que les arbitres de football donnaient plus de cartons rouges aux joueurs à la peau foncée, et neuf équipes n'ont trouvé aucune relation significative entre la couleur de la peau et les cartons rouges.
	\begin{figure}[H]
		\centering
		\includegraphics[width=1.0\textwidth]{img/arithmetics/repetability.jpg}
	\end{figure}
	La variabilité des résultats n'est pas due à une fraude ou à un travail bâclé. Ils sont considérés comme des analystes hautement compétents qui sont motivés pour trouver l'évidence, a déclaré Eric Luis Uhlmann, psychologue à l'école de commerce Insead à Singapour et l'un des chefs du projet. Même les chercheurs les plus qualifiés doivent faire des choix subjectifs qui peuvent avoir un impact énorme sur les résultats qu'ils trouvent.

	Mais ces résultats disparates ne signifient pas que les études ne peuvent pas nous guider vers l'évidence. La leçon importante ici est qu'une seule analyse n'est pas suffisante pour trouver une réponse fiable au-delà de tout doute raisonnable.
	
	\begin{tcolorbox}[title=Remarque,colframe=black,arc=10pt]
	De plus, la signification statistique n'a littéralement aucune connexion avec la "\NewTerm{signification pratique}". Elle ne vous dira jamais si quelque chose est important ou non! En effet, ce n'est pas un indicateur connaissant le contexte! Les statistiques de test ne regardent que l'effet (différence, ratio, covariance, etc.) et la variance. En d'autres termes, elle regarde juste si le signal est suffisamment visible sur le bruit de fond. Elle prend la condition actuelle - taille de l'échantillon - en considération. Plus de données signifient que des effets plus petits seront indiqués comme statistiquement significatifs.
	\end{tcolorbox}	
	
	Gardez à l'esprit que, comme nous l'avons dit dans l'introduction: Ce qui rend la science si puissante, c'est qu'elle se corrige d'elle-même. Bien sûr, de fausses conclusions sont publiées, mais finalement de nouvelles études viennent les renverser, et la nouvelle évidence est révélée. Du moins, c'est ainsi que cela est censé fonctionner. Mais le monde de la publication scientifique n'a pas une grande expérience en matière de suivi des corrections, mais cela évolue avec les nouvelles technologies qui facilitent ce suivi avec des bases de données informatiques.
	
	La méthode scientifique est donc a posteriori de loin la méthode la plus fiable (moins pire) que toute autre façon d'acquérir du savoir que nous avons AUJOURD'HUI (mais ce n'est pas un argument pour ne pas chercher à en trouver une meilleure)!
	
	\begin{tcolorbox}[title=Remarques,colframe=black,arc=10pt]
	Soit dit en passant ... le lecteur peut essayer de se souvenir d'un adage nommé d'après l'économiste Charles Goodhart, qui a été formulé par Marilyn Strathern comme: \textit{Quand une mesure devient une objectif, elle cesse d'être une bonne mesure.} Mais le la formulation d'origine était: \textit{Toute régularité statistique observée aura tendance à s'effondrer une fois la pression exercée sur elle à des fins de contrôle.}
	\end{tcolorbox}	

	\textbf{Définitions (\#\mydef):}
	\begin{enumerate}
		\item[D1.] La "\NewTerm{probabilité $\alpha$ de l'erreur de Type I}" (de première espèce/faux négatif\footnote{Notez que selon le point de vue du scientifique, un "faux négatif" peut être nommé "faux positif", c'est pourquoi de nombreux manuels de Statistiques évitent d'utiliser ce vocabulaire spécifique!}) est la probabilité de rejet de l'hypothèse nulle alors qu'elle est vraisemblablement vraie\footnote{Veuillez garder à l'esprit que comme déjà mentionné dans l'introduction de ce livre, nous ne parlons pas ici de la "vérité philosophique", mais de la "vérité" comme la meilleure explication à notre niveau actuel de connaissance !!!}.
		
		\item[D2.] La "\NewTerm{probabilité $\beta$ de l'erreur de Type II}" (de deuxième espèce/faux positif) est la probabilité de maintien de l'hypothèse nulle alors qu'elle est vraisemblablement fausse. Puisque, par définition, la puissance est égale à un $ 1-\beta$, la puissance d'un test diminuera au fur et à mesure que le bêta augmentera.
	\end{enumerate}
	\begin{center}
	  \resizebox{\textwidth}{!}{\begin{tabular}{|l|c|c|c|}
	  \hline
	    \cellcolor{black!30}   & \multicolumn{2}{|c|}{\cellcolor{black!30}\textbf{État de la Nature}} \\ \hline
	\cellcolor{black!30}\textbf{Résultat du test} & $H_0$ \textbf{fausse} & $H_0$ \textbf{vraie} \\ \hline
	\textbf{Rejeter }$H_0$ & \cellcolor{green!30}\parbox{6cm}{Rejet correct de l'hypothèse nulle\\ \centering($1-\beta$: puissance du test)} & \cellcolor{red!30}\parbox{5.5cm}{Erreur de Type I (faux négatif)\\ \centering(Risque $\alpha$)} \\[3ex] \hline
	\textbf{Échec du rejet de }$H_0$ & \cellcolor{red!30}\parbox{5.5cm}{Erreur de Type II (faux positif)\\ \centering(Risque $\beta$)} & \cellcolor{green!30}\parbox{5.5cm}{\centering Acceptation correcte de l'hypothèse nulle\\ \centering ($1-\alpha$: seuil de confiance)}  \\[3ex] \hline
	  \end{tabular}}
	\end{center}
	Ou sous une forme plus mathématique:
	\begin{center}
	  \resizebox{\textwidth}{!}{\begin{tabular}{|l|c|c|c|}
	  \hline
	    \cellcolor{black!30}   & \multicolumn{2}{|c|}{\cellcolor{black!30}\textbf{État de la Nature}} \\ \hline
	\cellcolor{black!30}\textbf{Résultat du test} & $H_0$ \textbf{fausse} & $H_0$ \textbf{vraie} \\ \hline
	\textbf{Rejeter }$H_0$ & \cellcolor{green!30}\parbox{6cm}{Rejet correct de l'hypothèse nulle\\ \centering $P(\text{reject } H_0|H_0 \text{ est fausse})=1-\beta$} & \cellcolor{red!30}\parbox{5.5cm}{Erreur de Type I (faux négatif)\\ \centering $P(\text{rejeter } H_0|H_0 \text{ est vraie})=\alpha$} \\[3ex] \hline
	\textbf{Échec du rejet de }$H_0$ & \cellcolor{red!30}\parbox{6cm}{Erreur de Type II (faux positif)\\ \centering $P(\text{échec du rejet de } H_0|H_0 \text{ est fausse})=\beta$} & \cellcolor{green!30}\parbox{7cm}{\centering Acceptation correcte de l'hypothèse nulle\\ \centering $P(\text{échec du rejet de } H_0|H_0 \text{ est vraie})=1-\alpha$}  \\[3ex] \hline
	  \end{tabular}}
	\end{center}
	Ainsi, un critère traditionnel de sélection de test est d'utiliser le principe suivant: parmi tous les tests qui ont la même grandeur de l'erreur de type I, choisir celui qui a la plus petite grandeur de l'erreur de type II.
	
	Ainsi, on peut dire qu'un test d'hypothèse a simplement pour but de minimiser $\alpha$ l'erreur de Type I et, au contraire, de maximiser $1-\beta$, la puissance du test.
	
	En général, l'amplitude de l'erreur de type II augmente lorsque celle de l'erreur de type I diminue. Nous ne pouvons pas minimiser les deux erreurs à la fois. Pour cette raison, nous prenons souvent une valeur donnée pour $\alpha$, la grandeur de l'erreur de type I, et nous minimisons $\beta$, l'amplitude de l'erreur de type II (i.e. nous augmentons la puisse $1-\beta$).

	\begin{tcolorbox}[title=Remarque,colframe=black,arc=10pt]
	En 2017, le taux de faux négatifs dans la population générale des États-Unis pour le test du VIH est d'environ $0.003\%$, soit trois fois sur $100'000$ tests.\\

	Les taux de faux positifs sont bien inférieurs, entre $0.0004\%$ et $0.0007\%$ en raison de la pratique par laquelle le résultat positif initial est confirmé par un second test.\\
	
	Autre exemple, les faux négatifs se produisent sur $1$ des chaque $2'500$ femmes dépistées (Programme de dépistage du cancer du sein du NHS - dépliant \textit{Helping you decide} de Juillet 2013) et se produisent plus souvent chez les femmes plus jeunes que chez les femmes plus âgées parce que le tissu mammaire des femmes plus jeunes est plus dense.
	\end{tcolorbox}
	Pour clore, voici les trois situations types de tests d'hypothèses sur la statistique qu'est la moyenne dans le cadre d'une distribution sous-jacente Normale et dont l'espérance est dans ce cas particulier supposée nulle et de variance unitaire (car on peut très souvent ce ramener à ce cas particulier en centrant et réduisant la variable aléatoire sous-jacente):
	\begin{figure}[H]
		\centering
		\includegraphics[scale=0.9]{img/arithmetics/hypothesis_average_three_scenarios.jpg}
		\caption{Les trois scénarios possibles d'un test d'hypothèse sur la moyenne}
	\end{figure}
	Indiquons que cela n'a aucun sens (contrairement à ce que nous pouvons parfois lire sur certains supports papier ou électronique) d'avoir les hypothèses nulles suivantes dans le cas paticulier représenté ci-dessus:
	
	avec l'hypothèse alternative qui en découle automatiquement (je ne l'ai pas écrite car c'est inutile). La raison en est simple: comment pourriez-vous positioner votre distribution Normale centrée réduite si l'espérance n'est pas fixée...??? Raison pour laquelle l'hypothèse nulle dans le cadre des tests sur la moyenne (et d'un certain autre nombre de tests) est toujours une égalité!
	
	Pour résumer, nous pouvons dire que si nous prenons une décision, nous pouvons nous tromper et il vaut mieux éviter de se tromper souvent. En clair, la probabilité de dire une bêtise doit être connue et de préférence petite.
	\begin{figure}[H]
		\centering
		\includegraphics[width=1.0\textwidth]{img/arithmetics/null_hypothesis.jpg}
		\caption[BD illustrant l'hypothèse nulle et alternative]{BD illustrant l'hypothèse nulle et alternative (auteur: ?)}
	\end{figure}
	Mais gardez à l'esprit qu'une meilleure compréhension de la $p$-valeur  n'enlèvera pas l'impulsion humaine (biais cognitif) d'utiliser les statistiques pour créer un niveau de confiance impossible. La plupart des gens veulent quelque chose qu'ils ne peuvent pas vraiment obtenir et qu'ils ne comprennent pas vraiment à cause d'un manque d'éducation et de pensée critique. Ils veulent des certitudes! Et même si vous essayez de le leur expliquer, vous aurez un effet de retour de flamme car ils souffrent souvent du syndrome de l'anti-intellectualisme.
	
	\subsubsection{Méthode de Fisher pour $p$-values multiples}
	En statistique, la "\NewTerm{méthode de Fisher}\index{méthode de Fisher}, également connue sous le nom de "\NewTerm{test de probabilité combiné de Fisher}\index{test de probabilité combiné de Fisher}", est une technique de fusion de données ou "méta-analyse" (analyse des analyses). Elle a été développée par et nommée pour Ronald Fisher. Dans sa forme de base, elle est utilisé pour combiner les résultats de plusieurs tests indépendants portant sur la même hypothèse globale ($H_0$).

	Considérons un ensemble de $k$ tests indépendants, chacun d'eux pour tester une certaine hypothèse nulle $H_ {0 | i}, i = \{1, 2, \ ldots, k \}$. Pour chaque test, un niveau de signification $p_ {i}$, c'est-à-dire une $p$-valeur, est obtenue. Toutes ces $p$-valeurs  peuvent être combinées dans un test conjoint pour savoir s'il existe un effet global, c'est-à-dire si une hypothèse nulle globale $H_0$ peut être rejetée.
	
	Il existe plusieurs façons de combiner ces tests partiels indépendants. La méthode de Fisher est l'une de celles-ci et est peut-être la plus connue et la plus largement utilisée. Le test a été présenté dans le livre désormais classique de Fisher, \textit{Statistical Methods for Research Workers} (1932).

	Le test est basé sur le fait que la probabilité de rejeter l'hypothèse nulle globale est liée à l'intersection des probabilités de chaque test individuel, $\prod_i p_i$. Cependant, $\prod_i p_i$ n'est pas uniformément distribué (mais chaque $p_i$ est supposé être uniformément distribué!), même si l'hypothèse nulle est vrai pour tous les tests partiels, et ne peut pas être utilisé lui-même comme niveau de signification conjoint pour le test global. Pour remédier à ce fait, certaines propriétés et relations intéressantes entre les distributions de variables aléatoires ont été exploitées par Fisher et incorporées dans l'extrait succinct ci-dessus. La preuve est donnée ci-dessous:
	\begin{dem}
	Nous recherchons donc la loi de distribution de:
	
	mais cela semble assez difficile à faire car nous avons affaire à un produit. Dès lors, nous prenons d'abord le logarithme (le logarithme naturel comme nous savons par expérience que dans le domaine des statistiques que la plupart du temps c'est le $\ln$ qui apparaît):
	
	Nous avons donc maintenant une somme des logarithmes naturels de distributions uniformes $\mathcal{U}_{0,1} $. C'est un peu mieux mais maintenant nous avons le problème du $\ln\left(\mathcal{U}_{0,1}\right)$.

	Mais voyons maintenant quelque chose de sympa! N'oubliez pas que la fonction de distribution cumulative d'une distribution exponentielle est:
	
	L'inverse est alors donné par:
	
	Si $P$ est une variable aléatoire uniformément distribuée dans l'intervalle $[0,1]$, il en est de même de $1-P$. En conséquence, la relation précédente peut s'écrire de manière équivalente:
	
	où encore une fois: $P=\mathcal{U}_{0,1} $. Nous venons donc de trouver que le logarithme naturel d'une variable aléatoire uniforme dans l'intervalle $[0,1]$ suit une distribution exponentielle avec le paramètre $\lambda=1$!

	Mais maintenant, nous avons un petit problème. Comme nous avons une somme de lois exponentielles, nous n'avons jamais prouvé si la loi exponentielle est stable par addition...
	
	Rappelons maintenant que la distribution chi-carré avec $k$ degrés de liberté est donnée par:
	
	Si $k=2$ nous obtenons:
	
	En d'autres termes, une distribution chi-carré avec deux degrés de liberté est égale à une distribution exponentielle avec $\lambda=1/2$!
	
	Et rappelez-vous maintenant, lors de notre étude de la distribution Gamma, nous avons prouvé qu'elle était stable par addition. Et comme nous avons prouvé que la distribution du chi-carré découle d'un cas particulier de la distribution Gamma, et tout à l'heure que la distribution exponentielle découle d'un cas particulier de la distribution du chi carré, alors cette dernière est également stable par addition de degrés de liberté!!!!
	
	Par conséquent, en multipliant la somme des logarithmes par $ -2 $ pour se débarrasser des $-$ provenant de la loi exponentielle et des $ 1/2 $ provenant de ce qui limite la loi du chi-carré à la loi exponentielle, nous obtenons:
	
	à partir de laquelle une $p$-valeur pour l'hypothèse globale peut être facilement obtenue.
	\begin{flushright}
		$\blacksquare$  Q.E.D.
	\end{flushright}
	\end{dem}
	Selon la méthode de Fisher, deux petites valeurs  $p_1$ et $p_2$ se combinent pour former une $p$-valeur plus petite. La limite jaune-verte dans la figure ci-dessous définit la région où la $p$-valeur de la méta-analyse est inférieure à $0.05$. Par exemple, si les deux valeurs de $p$ sont d'environ $0.10$, ou si l'une est d'environ $0.04$ et l'autre d'environ $0.25$, la $p$-valeur de la méta-analyse est alors d'environ $0.05$:
	\begin{figure}[H]
		\centering
		\includegraphics[scale=0.7]{img/arithmetics/fisher_multiple_pvalues_test.jpg}
		\caption[Méthode de Fisher pour multiples $p$-valeurs]{Méthode de Fisher pour multiples $p$-valeurs (source: Wikipédia)}
	\end{figure}
	\begin{tcolorbox}[title=Remarque,colframe=black,arc=10pt]
	Dans la fameuse affaire de jugement de Lucia de Berk, le tribunal a largement utilisé des calculs statistiques pour obtenir sa condamnation. Dans une émission télévisée spéciale de NOVA en 2003, le professeur néerlandais de droit pénal Theo de Roos a déclaré: \textit{Dans l'affaire Lucia de Berk, les preuves statistiques ont été d'une importance considérable. Je ne vois pas comment on aurait pu arriver à une condamnation sans cela}. Le psychologue juridique Henk Elffers, qui a été utilisé par les tribunaux comme témoin expert sur les statistiques à la fois dans l'affaire initiale et en appel, a également été interrogé sur le programme et a déclaré que la possibilité qu'une infirmière travaillant les trois hôpitaux soit présente sur les lieux de tant de décès inexpliqués et de réanimation est des 1 chance sur 342 millions. Mais malheureusement, cette valeur a été mal calculée (c'est pourquoi tout calcul statistique important doit toujours être revu par des pairs au moins par trois autres personnes indépendantes) en faisant un simple calcul de la multiplication des $p$-valeurs... De plus, toujours dans le contexte de Lucai de Berk, les statisticiens Richard D. Gill et Piet Groeneboom ont calculé qu'il y avait finalement qu'une chance sur vingt-cinq qu'une infirmière puisse vivre une séquence d'événements du même type que Lucia de Berk.\\
	
	L'utilisation des arguments de probabilités dans le cas De Berk a été discuté dans un article de 2007 de \textit{Nature} par Mark Buchanan. Il a écrit \textit{Le tribunal doit peser deux explications différentes: meurtre ou coïncidence. L'argument selon lequel il était peu probable que les décès soient survenus par hasard (que ce soit $1$ chance sur $48$ ou $1$ chance sur $342$ millions) n'est pas si significatif en soi - par exemple, la probabilité que dix meurtres se produisent dans le même hôpital pourrait être encore plus improbable. Ce qui compte, c'est la probabilité relative (odds ratio) des deux explications! Cependant, le tribunal n'a reçu une estimation que pour le premier scénario.}\\
	
	Nous voyons donc ici que l'utilisation d'une simple multiplication de des $p$-valeurs selon la méthode de Fisher n'est pas le bon outil dans une telle situation!
	\end{tcolorbox}
	
	
	\paragraph{Paradoxe de Simpson (sophisme)}\label{Simpson paradox}\mbox{}\\\\
	Le "\NewTerm{paradoxe de Simpson}\index{le paradoxe de Simpson}" (en fait c'est un sophisme et non un paradoxe car c'est juste un cas particulier de biais de variable omis), est un paradoxe de probabilités et de statistiques, dans lequel une tendance apparaît dans différents groupes de données mais disparaît ou s'inverse lorsque ces groupes sont combinés \footnote{Edward H. Simpson a décrit ce phénomène pour la première fois dans un article technique en 1951, mais les statisticiens Karl Pearson et al., en 1899, et Udny Yule, en 1903, avait mentionné des effets similaires plus tôt. Le nom de paradoxe de Simpson a été introduit par Colin R. Blyth en 1972.}.
	
	Ce résultat est souvent rencontré en particulier dans les statistiques des sciences sociales (ressources humaines, marketing, psychologie) et des sciences médicales et est particulièrement déroutant lorsque les données fréquentielles reçoivent indûment des interprétations causales. Le paradoxe de Simpson disparaît lorsque les relations causales sont prises en compte.
	
	Voyons un exemple réel célèbre d'une véritable étude médicale (il y a aussi un autre cas célèbre avec le rapport \textit{Nation at Risk}, si le lecteur est intéressé, il suffit de le rechercher sur Internet!):
	
	Le tableau ci-dessous montre les taux de succès et le nombre de traitements pour les traitements impliquant à la fois de petits et grands calculs rénaux, où le traitement $A$ comprend toutes les interventions chirurgicales ouvertes et le traitement $B$ est la néphrolithotomie percutanée (qui n'implique qu'une petite ponction). Les nombres entre parenthèses indiquent le nombre de cas de réussite sur la taille totale du groupe (par exemple, $93 \% $ équivaut à $81$ divisé par $87$):
	
	\begin{center}
		\definecolor{gris}{gray}{0.85}
			\begin{tabular}{|p{3cm}|p{3cm}|p{3cm}|}
				\hline
				 & \multicolumn{1}{c}{\cellcolor{black!30}\textbf{Traitement $A$}} & 
  \multicolumn{1}{c}{\cellcolor{black!30}\textbf{Traitement $B$}} \\ \hline
				\multicolumn{1}{c}{\cellcolor{black!30}\textbf{Petits calculs}}  & Groupe 1\newline $93\%\; (81/87)$  & Groupe 2\newline $87\%\; (234/270)$ \\ \hline
				\multicolumn{1}{c}{\cellcolor{black!30}\textbf{Grands calculs}}   & Groupe 3\newline $73\%\; (192/263)$ & Groupe 4\newline $69\%\; (55/88)$  \\ \hline
				\multicolumn{1}{c}{\cellcolor{black!30}\textbf{Ensemble}} & \textbf{$78\%\; (273/350)$} & \textbf{$83\%\; (289/350)$} \\ \hline
		\end{tabular}
	\end{center}
	La conclusion paradoxale est que le traitement $A$ est plus efficace lorsqu'il est utilisé sur de petites calculs, et aussi lorsqu'il est utilisé sur de gros calculs, pourtant le traitement $B$ est plus efficace si l'on considère les deux tailles de calculs en même temps. Dans cet exemple, la variable "cachée" (ou "\NewTerm {variable de confusion}" \index{variable de confusion}) de la taille du calcul n'était pas connue auparavant pour être importante jusqu'à ce que ses effets soient inclus.
	
	Formellement, si on dénote (on change les notations !!!!) $A$ le résultat, $B$ le traitement, et $C$ les calculs, dans le cas général, avec le paradoxe de Simpson il est possible d'avoir en termes (formalisme) de probabilités bayésiennes:
	
	Le traitement considéré comme meilleur est déterminé par une inégalité entre deux ratios (succès / total). L'inversion de l'inégalité entre les ratios, qui crée le paradoxe de Simpson, se produit parce que deux effets se produisent ensemble:
	\begin{enumerate}
		\item Les tailles des groupes, qui sont combinées lorsque la variable cachée est ignorée, sont très différentes. Les médecins ont tendance à donner aux cas graves (gros calculs) le meilleur traitement $A$, et aux cas les plus bénins (petits calculs) le traitement inférieur $B$. Par conséquent, les totaux sont dominés par les groupes $3$ et $2$, et non par les deux groupes beaucoup plus petits $1$ et $4$.
		
		\item La variable cachée a un effet important sur les ratios, c'est-à-dire que le taux de réussite est plus fortement influencé par la gravité du cas que par le choix du traitement. Par conséquent, le groupe de patients avec de gros calculs utilisant le traitement $A$ (groupe $3$) fait moins bien que le groupe avec de petits calculs, même si ce dernier a utilisé le traitement inférieur $B$ (groupe $2$).
	\end{enumerate}
	Sur la base de ces effets, le résultat paradoxal semble provenir de la suppression de l'effet causal de la taille des calculs sur le succès du traitement. Le résultat paradoxal peut être reformulé plus précisément comme suit: Lorsque le traitement le moins efficace ($B$) est appliqué plus fréquemment à des cas plus faciles, il peut apparaître comme un traitement plus efficace.
	
	Le paradoxe de Simpson nous trompe généralement sur les tests de performance. Dans un exemple célèbre, les chercheurs ont conclu qu'un traitement plus récent pour les calculs rénaux était plus efficace que la chirurgie traditionnelle, mais il a été révélé plus tard que le nouveau traitement était plus souvent utilisé sur les petits calculs rénaux. Plus récemment, aux tests des écoles élémentaires, les élèves des minorités du Texas surpassent leurs pairs du Wisconsin, mais le Texas compte tellement d'élèves des minorités que le Wisconsin le bat dans le classement des États. Ce serait dommage si le paradoxe de Simpson conduisait les médecins à prescrire des traitements inefficaces ou les écoles du Texas à gaspiller de l'argent à copier le Wisconsin.
	
	Considérons un autre exemple illustré amusant dans le domaine du marketing! Considérons que nous avons les informations suivantes:
	\begin{figure}[H]
		\centering
		\includegraphics{img/arithmetics/simpson_paradox_normal.jpg}
		\caption[]{Performance de la campagne marketing}
	\end{figure}
	Évidemment, la campagne $B$ est la plus performante pour convertir un e-mail en clic ("taux de conversion")!
	
	Mais maintenant, en séparant les clics sur les liens de grande valeur (produits coûteux: conversion de ticket élevée) ou les liens de faible valeur (conversion de ticket bas), nous obtenons:
	\begin{figure}[H]
		\centering
		\includegraphics{img/arithmetics/simpson_paradox_ouch.jpg}
		\caption[]{Performance de la campagne marketing dans la covariable de conversion de ticket}
	\end{figure}
	Ou pour les personnes plus visuelles (avec des graphiques), voici un exemple de notre livre companion R. Il s'agit d'une simple régression de l'ensemble du jeu de données IRIS. La tendance de la régression linéaire semble négative:
	\begin{figure}[H]
		\centering
		\includegraphics[scale=0.8]{img/arithmetics/simpson_paradox_plot_before_clustering.jpg}
		\caption{Régression non groupée du jeu de données IRIS dans R pour illustrer le paradoxe de Simpson}
	\end{figure}
	Maintenant, si nous regroupons les régressions ... toutes les tendances sont positives ...:
	\begin{figure}[H]
		\centering
		\includegraphics[width=0.9\textwidth]{img/arithmetics/simpson_paradox_plot_after_clustering.jpg}
		\caption{Régression groupée du jeu de données IRIS dans R pour illustrer le paradoxe de Simpson}
	\end{figure}
	Certaines personnes pourraient critiquer que ce n’est pas un problème que l'on rencontre dans le business... Voici donc un exemple du paradoxe de Simpson typique dans les tableaux de bord de Business Intelligence:
	\begin{figure}[H]
		\centering
		\includegraphics{img/arithmetics/simpson_paradox_business_intelligence.jpg}
	\end{figure}
	Nous voyons ci-dessus que les ventes régionales agrégées semblent conformes aux objectifs. Mais au niveau de détail suivant (au niveau des territoires), nous pouvons voir que le territoire A se porte très mal ($25\%$ des objectifs) et le territoire B fait exceptionnellement bien ($400\%$ des objectifs). Ceci est un défi majeur dans le monde de la BI. Il est facile de se sentir satisfait de ce que les rapports et les tableaux de bord BI nous disent sans disposer d'un moyen simple et automatisé de se plonger dans les détails pour trouver les domaines de l'entreprise qui sont nettement sur ou sous-performants.
	
	\pagebreak
	\subsubsection{Puissance d'un test}\label{power of a text}
	Lorsque l'effet est concrètement important, on imagine bien qu'il faut moins d'observations pour le démontrer que lorsqu'il est petit... mais combien au juste? A-t-on les moyens, en termes de nombre de mesures, de démontrer ce que l'on cherche? Faut-il s'y prendre autrement et changer le dispositif de son observation/expérimentation?
	
	Nous avons vu mentionné plus haut que nous ne pouvons jamais "prouver" l'hypothèse nulle \footnote{Rappelez-vous qu'avec les tests statistiques basés sur l'hypothèse nulle, si la $p$-valeur est supérieure à un seuil donné, nous ne devons pas dire que nous pouvons accepter (ou prouver)  hypothèse nulle; nous utilisons plutôt la locution alambiquée que «\textit{Nous n'avons pas réussi à réfuter le nul}».}; tout ce que nous pouvons faire est de la rejeter ou de ne pas la rejeter. Cependant, il y a des moments où il est nécessaire d'essayer de "prouver" (ie avoir des évidences) l'inexistence d'une différence entre les groupes. Cela se produit le plus souvent dans le contexte de la comparaison d'un nouveau traitement avec un traitement établi et montrant que la nouvelle méthode n'est pas inférieure à la méthode standard en vigueuer (dans certaines circonstances, nous avons besoin de beaucoup plus de sujets pour obtenir des évidences de non-infériorité que lorsque nous devons montrer une différence). Pouvons-nous faire quelque chose comme ça?
	
	Pour répondre à toutes les questions ci-dessus, nous devons maintenant étudier plus en détail le concept de "\NewTerm{puissance d'un test}\index {puissance d'un test}" que nous n'avons jusqu'ici que mentionné. Pour cela, rappelez-vous la figure suivante que nous avons déjà recontré un peu plus haut:
	\begin{figure}[H]
		\centering
		\includegraphics{img/arithmetics/alpha_risk_beta_risk.jpg}
		\caption{Hypothèse nulle et alternative d'un cas particulier de test bilatéral}
	\end{figure}
	\pagebreak
	Dans l'exemple particulier ci-dessus, nous allons donc rejeter l'hypothèse nulle $H_0$ si $\bar{X}>1.96$ ou si $\bar{X}<1.96$. Imaginons que dans le cadre de l'hypothèse alternative, si nous avons mesuré $2.5$, nous aurons comme puissance du test:
	
	Donc le test est relativement puissant (dans la pratique, nous considérons un test comme étant puissant si sa valeur est au-delà de $80\%$, c'est-à-dire $\beta\leq 20\%$). Ainsi, nous remarquons que la puissance equation (a posteriori!) est d'autant plus grande que la $p$-valeur sera petite (et respectivement la puissance sera à posteriori d'autant plus petite que la $p$-valeur sera grande). Donc la puissance a posteriori est en correspondance décroissante avec la $p$-valeur (dans la pratique il est cependant un peu absurde de faire ces calculs a posteriori).
	
	Une excellente façon de jouer avec ce concept et de les visualiser est de jeter un oeil au widget Wolfram disponible sur \url{http://demonstrations.wolfram.com/StatisticalPower/}:
	\begin{figure}[H]
		\centering
		\includegraphics[scale=0.9]{img/arithmetics/wolfram_widget_power_errors_test.jpg}
	\end{figure}
	Avant de continuer, laissez-nous vous informer que un pourcentage significatif d'individus ne comprennent pas pourquoi la probabilité d'obtenir une erreur de type I lors d'un test d'hypothèse n'est pas affectée par la taille de l'échantillon?!
	
	Si nous utilisions un test d'hypothèse standard, nous pouvons définir le niveau de confiance $\alpha$ et comparer la  $p$-valeur du test à celui-ci. Dans ce cas, la taille de l'échantillon n'aura pas d'incidence sur la probabilité d'erreur de type I car votre niveau de confiance $\alpha$ est la probabilité d'erreur de type I, à peu près par définition. En d'autres termes, vous définissez la probabilité d'erreur de type I en choisissant le niveau de confiance! La probabilité d'erreur de type I n'est affectée que par votre choix du niveau de confiance et rien d'autre!
	
	Mais oui, nous pourrions également corriger le taux d'erreur de type II (mais ce n'est pas le but de la science, comme nous le savons, de "prouver" l'hypothèse nulle car elle est pratiquement assez difficile la plupart du temps!), Le taux d'erreur de type I devient généralement le paramètre inconnu et il dépend de la taille de l'échantillon, de la variance et de la taille de l'effet (delta).
	
	\begin{tcolorbox}[title=Remarque,colframe=black,arc=10pt]
	Beaucoup de gens disent que si une personne $A$ postule l'existence d'un phénomène, il incombe à la personne $A$ de prouver son existence plutôt que d'être le travail du personne $B$ de le réfuter. Par exemple, si nous affirmons qu'il y a des licornes roses à pois violets cachées au fond de la forêt, je ne peux pas simplement dire, «\textit{Eh bien, prouvez qu'elles n'existent pas}». Pour que le monde scientifique accepte notre demande, nous devons en ramener une, morte ou vivante. Est-il jamais légitime de violer cette injonction et d'essayer de "prouver" l'inexistence de quelque chose? Dans le texte ci-dessus, nous avons ensuite cherché à essayer de "prouver" (c'est-à-dire obtenir "suffisamment d'évidence) que quelque chose n'existe pas\footnote{Évidemment, si nous sommes sûrs d'avoir étudié TOUTES les combinaisons possibles, les lieux et que l'énorme majorité des propriétés majoritaires des licornes sont prouvées fausses ou impossibles, alors il est très probable au-delà de tout doute raisonnable que la licorne elle-même n'existe pas!}.
	\end{tcolorbox}
	
	\paragraph{Évidence ("preuve") de la négative}\mbox{}\\\\
	Le ministre géorgien Nelson L. Price affirme sur son site Web (et de nombreux autres vulgarisateurs scientifiques sur YouTube) que «\textit{l'une des lois de la logique est que vous ne pouvez pas prouver un négatif}». Le fait est, cependant, que cette prétendue "loi de la logique" n'est pas une loi!
	
	Remarquez, pour commencer, que le «Vous ne pouvez pas prouver un négatif» est elle-même une négation. Donc, si c'était vrai, elle serait elle-même impossible à prouver. Notez que toute affirmation peut être transformée en une négation par une petite reformulation - le plus évident, en niant l'affirmation puis en la niant à nouveau. «J'existe» équivaut logiquement à «Je n'existe pas», ce qui est négatif. Pourtant, voici une négation qu'il semble que nous pourrions peut-être prouver (à la manière de Descartes: «\textit{je pense, donc je n'existe pas!}»).
	
	Bien sûr, ceux qui disent «Vous ne pouvez pas prouver un négatif» insisteront sur le fait que nous avons mal compris leur point de vue. Comme le note Steven D. Hales, quand les gens disent: «Vous ne pouvez pas prouver un négatif», ce qu'ils veulent vraiment dire, c'est que «Vous ne pouvez pas prouver que quelque chose n'existe pas«. Si ce point était correct, il s'appliquerait non seulement aux êtres surnaturels se trouvant au-delà du voile cosmique, mais aussi aux choses qui pourraient être supposées exister de ce côté du voile, comme les licornes, les martiens, les lapins à 20 têtes, etc. Nous ne pourrions pas non plus prouver la non-existence de l'une de ces choses.
	
	Mais le point est-il correct? Est-il vrai que nous ne pouvons jamais prouver que quelque chose n'existe pas? Encore une fois, la réponse dépend \textbf{SI} nous avons suffisamment défini la portée de ce que nous regardez et le niveau d'évidence \footnote{Par exemple, la grande majorité des charactéristiques et des actes qui définissent certaines divinités dans des livres saints ont été montrés avec des évidences solides par la science moderne comme étant fausses. Cela nous amène à la conclusion très probable qu'il n'y a pas de divinités - en particulier les divinités qui sont devenues des mythologies nous en donnent une autre évidence solide.}. Si quelqu'un prétend qu'il y a une licorne ayant des caractéristiques matérielles similaires à celles d'un chevel dans le hangar à outils (voir \cite{streiner2003unicorns} pour un traitement relativement en profondeur de ce sujet), nous pouvons rapidement établir que cette personne se trompe en allant y jeter un oeil. Nous pourrions également établir qu'il n'y a pas de monstre du Loch Ness en drainant l'entiéreté du lac. Mais qu'en est-il de l'affirmation selon laquelle les licornes existaient autrefois? Nous ne pouvons pas voyager dans le temps et observer directement tout le passé comme nous le pouvons dans chaque coin de la remise à outils ou du lac du Loch Ness. S'ensuit-il que nous ne pouvons pas prouver que les licornes n'ont jamais existé?
	
	Cela dépend en partie, comme nous le savons déjà, de ce que nous entendons par "prouver". Le mot a une variété de significations. En disant que quelque chose est «prouvé», nous pourrions vouloir dire qu'il est établi au-delà de tout doute possible. Ou nous pourrions dire qu'il a été établi au-delà de tout doute raisonnable (c'est le genre de preuve en science expérimentale). Pouvons-nous établir au-delà de tout doute raisonnable que les licornes n'ont jamais vécu sur la Terre? Certes, cette partie de l'histoire de notre planète serait donc passée, nous ne pouvons donc plus l'inspecter directement. Mais sûrement, si les licornes parcouraient la Terre, nous nous attendrions à trouver des évidences de leur présence, telles que des fossiles de licornes ou du moins d'animaux étroitement apparentés à partir desquels les licornes auraient vraisemblablement évolué. Il n'y en a pas. Nous avons également de nombreuses évidences que les licornes sont une création fictive, auquel cas, il est sûrement raisonnable pour nous de conclure qu'il n'y a jamais eu de licornes. En effet, nous suggérons que nous puissions en donner la "preuve" au-delà de tout doute raisonnable (surtout si toutes les propriétés inhérentes aux licornes ont déjà été déconstruites).
	
	En réponse, on pourrait nous dire: «Mais vous ne pouvez pas prouver de manière concluante, au-delà de tout doute possible, que les licornes n'ont jamais vécu sur la Terre.» C'est indéniablement vrai! Cependant, ce point n'est pas propre aux négatifs. Il peut être fait à propos de toute affirmation sur l'inobservé, et donc de toute théorie scientifique, y compris des théories scientifiques sur ce qui existe. Nous pouvons "prouver" (avoir suffisamment d'évidences) au-delà de tout doute raisonnable que les dinosaures existaient, mais pas au-delà de tout doute possible!
	
	Malgré la montagne d'évidences que les dinosaures ont bien vécu sur Terre, il est toujours possible que, disons, tous ces fossiles de dinosaures soient des faux placés là par des extraterrestres farceurs il y a bien longtemps...
	
	Résumons! Si «Vous ne pouvez pas prouver un négatif» signifie que vous ne pouvez pas "prouver" (ie fournir des évidences) au-delà de tout doute raisonnable que certaines choses n'existent pas, alors l'affirmation est tout simplement fausse. Nous "prouvons" (ie fournissons des évidences) régulièrement la non-existence des choses. Si, d'un autre côté, «Vous ne pouvez pas prouver un négatif» signifie que vous ne pouvez pas "prouver" (ie fournir des évidences) au-delà de tout doute possible que quelque chose n'existe pas, eh bien, cela peut, sans doute, être vrai. Mais alors quoi? Ce point n'est pas pertinent dans la mesure où la défense des croyances en des entités surnaturelles contre l'accusation que la science et/ou la raison ont établi au-delà de tout doute raisonnable qu'elles n'existent pas.
	
	Considérons maintenant que quelque chose a un nombre fini de propriétés / caractéristiques observables (c'est-à-dire réfutables) dans un volume fini. Alors disons que nous avons un bol. Nous pouvons prouver la non-existence de lait dans notre bol avec une photo (ou avec un appareil plus complexe) d'un bol vide. Mais cette photo ne représente pas de lait négatif. Il représente un fond positif du bol, et de ce positif, nous pouvons déduire l'impossibilité de la présence du lait, sachant, à partir d'autres évidences, à quoi ressemble le lait. Nous utilisons donc des évidences positives pour obtenir l'évidence de l'impossibilité. L'impossibilité est l'épuisement des possibilités exclusives. Les possibilités sont nulles ou positives. Dans ce cas, lait et vide sont mutuellement exclusifs, où la probabilité de lait est de $0$ et la probabilité de vide est de $1$. C'est différent de "prouver" un négatif. Or, comme la plupart des religions ont des livres saints (au contenu moral plus que douteux) qui décrivent assez précisément certaines propriétés et actions de leurs dieux sur Terre (volume fini). Si l'on trouve des évidences dans ce volume fermé (Terre) qui réfutent au moins une des propriétés (c'est-à-dire caractéristiques) de leur dieu, alors ce dernier n'existe très probablement pas au-delà de tout doute raisonnable... Techniquement cela sera noté:
	
	et nommé la "\NewTerm{méthode de falisification Stenger-Neyman-Pearson}". Donc en appliquant cette relation aux livres religieux (vedas, bible, torah, coran, etc.) sachant que la bible a (sans compter les contradictions et erreurs historiques) au moins 21 erreurs scientifiques, le coran a (sans compter les contradictions et erreurs historiques) ) au moins 73 erreurs scientifiques, la torah a (sans compter les contradictions et erreurs historiques) au moins 20 erreurs scientifiques, les vedas ont (sans compter les contradictions et erreurs historiques) au moins 26 erreurs scientifiques nous amènent à une conclusion assez évidente sur la perfection - et donc l'existence - de leurs supposées divinités respectives...
	
	
	Le vieil adage «Vous ne pouvez pas prouver un négatif» est un argument typique de niveau d'évidence 02 émanant de personnes scientifiquement illettrées.
	\begin{figure}[H]
		\centering
		\includegraphics[scale=0.5]{img/arithmetics/universal_negative.jpg}
	\end{figure} 
	Donc, la preuve de la non-existence dépend de la précision avec laquelle la chose à vérifier n'existe pas. Si nous pouvons définir un attribut que la chose doit absolument avoir dans un volume ou une plage de temps finis, alors nous pourrons peut-être avoir des évidences solides que la chose avec cet attribut n'existe pas. Mais dès qu'il y a un flou dans la description, nous ne pouvons pas le faire. Donc, encore une fois, pour "prouver" que quelque chose n'existe pas, nous devons établir les propriétés de cette chose dans un volume fini et dans un temps fini au point de pouvoir tester sa négation.
	
	\subparagraph{Double test-$T$ unilatéral (DTU)}\mbox{}\\\\
	Si l'on veut tester l'égalité de deux valeurs $\hat{\theta}_1=\hat{\theta}_2$ (moyenne, proportion ou autre), c'est-à-dire l'inexistence d'un effet (ou d'un "négatif", comme étant l'hypothèse alternative $H_A$ nous ne pouvons pas faire cela avec le test classique de l'hypothèse nulle et alternative (car il est assez difficile de construire la statistique de l'hypothèse nulle $H_0$ dans ce cas, c'est-à-dire la distribution statistique de $\hat{\theta}_1 \neq\hat{\theta}_2$)!
	
	Une approche bien connue dans la pratique scientifique (en considérant le cas avec des moyennes!) est la "\NewTerm{Double test-$T$ unilatéral (DTU)}\index{double test-$T$ unilat\'eral (DTU)}". Il s'agit de regarder la différence entre les valeurs moyennes des deux populations puis de tester une hypothèse nulle conjointe : que cette différence est inférieure ou égale à $+\delta$ ET que cette différence est supérieure ou égale à $-\delta$ (pour un certain $\delta$ suffisamment petit et positif). Si nous pouvons rejeter cette hypothèse nulle conjointe, alors nous pouvonsconclure que la valeur absolue de la différence entre les moyennes est inférieure à $\delta$ (voir \cite{juzek2019set} pour une analyse sur la façon de choisir $\delta$).
	
	En d'autres termes, la procédure du double test-$T$ unilatéral (DTU) est utilisée pour montrer que deux échantillons sont similaires, contrairement aux tests statistiques classiques qui vérifient la dissimilarité. Le DTU repose sur un paramètre appelé delta, qui doit être défini par le chercheur à l'aide de son intuition. Cela peut être difficile, en raison des interactions complexes des paramètres pertinents.

	Dans certains domaines, il peut y avoir des normes industrielles guidant dans le choix de $\delta$. Si aucun tel outil existe, le critère d'acceptation de similarité est mieux développé à partir d'un consensus entre les parties prenantes telles que les décideurs, les ingénieurs système, les opérateurs et les experts en la matière. Décider quelle différence est « assez bonne » et déterminer $\delta$ pour le test de similarité n'est pas une tâche triviale et un temps suffisant doit être alloué pour cela dans la phase de planification. La valeur de $\delta$ doit être basée sur l'importance pratique d'une différence pour considérer que les moyennes ne sont pas équivalentes. $\Delta$ est similaire à la différence à détecter lorsque nous dimensionnons une expérience conçue traditionnellement. Une fois que $\delta$ est dérivé, l'hypothèse peut être énoncée comme suit:
	
	Un test-$T$ de similarité unilatéral (TSU) est approprié si l'équipe qui fait le test ne se soucie que si la réponse diffère dans un sens. Par exemple, si un nouveau composé de caoutchouc est utilisé pour fabriquer un pneu avec une garantie de kilométrage, les producteurs ne se soucieront peut-être que si le pneu a moins de kilomètres d'utilisation. Ceci est souvent appelé "\NewTerm{test de non-infériorité}\index{test de non-infériorité}". Dans ces situations, l'hypothèse nulle est souvent appelée hypothèse "d'infériorité" et l'hypothèse alternative est l'hypothèse de "non-infériorité". L'équation suivante est utilisée pour comparer la valeur de $t$ à la valeur critique.
	
	Comme toujours, nous n'utiliserons pas $\mu$, car il représente le véritable paramètre de population, que nous ne connaîtrons vraisemblablement jamais. À la place, nous utilisons $\hat{\mu}$, qui désigne la moyenne de l'échantillon. La valeur critique $t_{\alpha, \text{df}}$ peut être calculée avec un logiciel ou recherchée dans des tableaux statistiques.

	Nous rejettons $H_0$ si: 
	
	Dans les tests de similarité, il n'y a pas de test bilatéral. Au lieu de cela, nous effectuons deux tests de non-infériorité des deux côtés :
	
	Lorsque les hypothèses sont élargies pour créer les statistiques de test, nous obtenons le critère suivant. Nous rejettons $H_{0}$ si: 
	
	Notez que les deux moyennes sont considérées comme similaires si, et seulement si, les deux hypothèses nulles sont rejetées ! Il faut montrer que la différence des moyennes est inférieure à $+\delta$ et supérieure à $-\delta$ pour montrer que les moyennes sont équivalentes ; cela ne peut pas être vrai que dans une seule direction! Il est également prudent de souligner que cela est différent d'un test d'hypothèse bilatéral traditionnel. Lors de la réalisation d'un test bilatéral avec une probabilité de type I de $\alpha$, nous utilisons $\alpha / 2$ pour comparer à la $p$-valeur de chaque côté. Dans cette situation, nous utilisons la valeur totale de $\alpha$ sur les deux tests car ils sont tous les deux unilatéraux.
	
	\begin{tcolorbox}[colframe=black,colback=white,sharp corners]
	\textbf{{\Large \ding{45}}Exemple:}\\\\
	Supposons que nous analysions l'efficacité du mitrailleur de porte dans un hélicoptère. Habituellement le mitrailleur de porte utilise le M-249 mais une unité souhaite plutôt l'armer avec la M-4. La M-249 est une mitrailleuse capable d'une fréquence de tir plus élevé, mais elle est plus lourde, plus difficile à déplacer et plus sujette aux blocages. La M-4 peut-elle fournir une puissance de combat équivalente ? Nous avons les données suivantes :
	$$\begin{array}{|l|r|}
	\hline \textbf{ Type d'arme } & \begin{array}{c}
	\textbf{ Score de } \\
	\textbf{ dégâts de la cible }
	\end{array} \\
	\hline \text { M-249 } & 180 \\
	\hline \text { M-249 } & 143 \\
	\hline \text { M-4 } & 65 \\
	\hline \text { M-4 } & 112 \\
	\hline \text { M-4 } & 139 \\
	\hline \text { M-4 } & 112 \\
	\hline \text { M-4 } & 125 \\
	\hline \text { M-4 } & 78 \\
	\hline \text { M-4 } & 138 \\
	\hline \text { M-4 } & 84 \\
	\hline \text { M-249 } & 117 \\
	\hline \text { M-249 } & 169 \\
	\hline \text { M-249 } & 111 \\
	\hline \text { M-249 } & 114 \\
	\hline \text { M-249 } & 166 \\
	\hline \text { M-4 } & 134 \\
	\hline \text { M-249 } & 131 \\
	\hline \text { M-4 } & 69 \\
	\hline \text { M-249 } & 93 \\
	\hline \text { M-249 } & 177 \\
	\hline
	\end{array}$$
	Le lecteur peut voir l'arme utilisée et le score de dégâts pour chacun des $20$ essais. Cet exemple est suffisamment petit pour que nous puissions effectuer les calculs manuellement. Trois statistiques de base sont calculées et présentées dans le tableau ci-dessous :
	$$\begin{array}{|c|c|c|c|}
	\hline \text { Arme } & \text { Taille échantillon } & \text { Estimateur moyenne } & \text { Estimateur variance } \\
	\hline \text { M-249 } & n_{1}=10 & \hat{\mu}_{1}=140.1 & \hat{\sigma}_{1}^{2}=981.21 \\
	\hline \text { M-4 } & n_{2}=10 & \hat{\mu}_{2}=105.6 & \hat{\sigma}_{2}^{2}=849.6 \\
	\hline
	\end{array}$$
	En utilisant les valeurs calculées dans le tableau ci-dessus, nous pouvons calculer la statistique de test et la valeur critique. D'abord pour l'extrémité supérieure de la courbe :
	$$
	\text{ES}=\sqrt{\frac{\hat{\sigma}_{1}^{2}}{n_{1}}+\frac{\hat{\sigma}_{2}^{2}}{n_{2}}}=\sqrt{\frac{981.21}{10}+\frac{849.6}{10}}=\sqrt{183.0811} \cong 13.53
	$$
	\end{tcolorbox}
	
	\begin{tcolorbox}[colframe=black,colback=white,sharp corners]
	Nous rejettons $H_0$ si:
	$$\frac{\left|\hat{\mu}_{1}-\hat{\mu}_{2}\right|+\delta}{\text{ES}}>t_{\alpha, n_{1}+n_{2}-2}$$
	Dès lors:
	$$
	\frac{|140.1-105.6|+20}{13.53}>t_{0.05,18}
	$$
	C'est-à-dire:
	$$4.03>1.73 \stackrel{\text { dès lors }}{\longrightarrow} \text{Vrai}$$
	L'énoncé est vrai, nous rejetons donc $H_0$ dans le premier test unilatéral. Ensuite, nous effectuons le deuxième test:
	$$\frac{\left|\hat{\mu}_{1}-\hat{\mu}_{2}\right|-\Delta}{\text{ES}}<-t_{\alpha, n_{1}+n_{2}-2}$$
	Dès lors:
	$$\frac{|140.1-105.6|-20}{13.53}<-t_{0.05,18}$$
	C'est-à-dire:
	$$1.07<-1.734 \stackrel{\text { dès lors }}{\longrightarrow} \text{Faux}$$
	L'énoncé est faux et nous ne parvenons pas à rejeter $H_{0}$ dans le deuxième test unilatéral. Par conséquent, nous ne rejetons pas non plus l'hypothèse nulle globale selon laquelle la M-4 n'est pas équivalent à la M-249. Ce n'est pas un résultat surprenant car notre différence initiale de moyennes était de $34.5$.
	\end{tcolorbox}
	
	\pagebreak
	\paragraph{Puissance du test $Z$ à $1$ échantillon}\mbox{}\\\\
	En toute généralité, dans le cas d'un test bilatéral, la relation précédente s'écrira donc:
	
	Si l'écart-type de la moyenne n'est pas été unitaire, nous avons:
	
	Il vient donc:
	
	autrement écrit:
	
	C'est sous cette forme que nous retrouvons la puissance d'un test bilatéral de la moyenne (puissance du test $Z$ à $1$ échantillon):
	
	où $d_\sigma$ est parfois appelé la "\NewTerm{taille d'effet}\index{taille d'effet}" et est défini par:
	
	et $\delta$ est nommé la "\NewTerm{différence}"!
	
	Il va de soit que si la variance vraie n'est pas connue, il faut alors remplacer la loi Normale par la loi de Student tel que:
	
	avec:
	
	
	\begin{tcolorbox}[colback=red!5,borderline={1mm}{2mm}{red!5},arc=0mm,boxrule=0pt]
	\bcbombe Attention à un petit piège courant! Le développement ci-dessus correspond à un $\delta$ qui est donc négatif relativement à l'exemple de départ! La relation est un peu différente dans le cas où $\delta$ est positif mais cela n'a aucune importance car la puissance du test est identique valeur absolue!
	\end{tcolorbox}
	
	Pour avoir la taille de l'échantillon c'est assez simple. Nous avons en égalisant les centiles:
	
	et donc en bilatéral:
	
	où nous voyons que si la puissance du test est imposée comme étant égale à $50\%$, ayant $Z_{1-\beta}$ qui vaut alors $0$ nous retombons (!) sur la relation de l'effectif de l'échantillon pour loi Normale démontrée bien plus haut:
	
	Signalons aussi que nous retrouvons parfois dans la littérature la relation antéprécédente sous la forme suivante:
	
	Évidemement nous pouvons fixer d'autres paramètres pour déterminer la valeur de la variable restante. Nous pourrions par exemple chercher la valeur de la puissance du test en imposant l'écart-type, la taille de l'échantillon et le niveau de confiance, etc.
	
	Un lecteur nous a proposé une maniètre très élégante de retrouver le même résultat avec beaucoup moins de développements... Effectivement, il suffit de voir sur la figure précédente que nous avons:
	
	Dont nous tirons immédiatement une relation équivalement aux deux précédentes (qui donne bien évidemment le même résultat numérique):
	
	\begin{tcolorbox}[title=Remarque,colframe=black,arc=10pt]
	Le lecteur attentif aura peut-être remarqué que nous avons supposé dans les développements qui précédent que l'écart-type de la moyenne vraie et aternative (estimée) est implicitement supposée être la même... Dans la pratique cela est presque tout le temps le cas, raison pour laquelle les quasi totalité des logiciels de statistiques ne demandent qu'un seul écart-type pour le calcul de la puissance du test $Z$ à $1$ échantillon. Cependant, dans certains rares logiciels universitaires, on demande l'écart-type des deux moyennes. Mais dès lors les développements ci-dessus sont différents.
	\end{tcolorbox}
	
	Une analyse de puissance peut avoir plusieurs facettes:
	\begin{enumerate}
		\item Nous connaissons le niveau du test, la taille d'échantillon et la taille d'effet (implicitement la différence) et nous cherchons à calculer la puissance. Ceci permet de voir si notre dispositif expérimental est bien calibré.
		
		\item Nous connaissons la puissance voulue, le niveau du test et la taille d'effet à détecter. Nous cherchons alors à calculer la taille d'échantillon nécessaire pour monter un dispositif expérimental efficace.
		
		\item Nous connaissons la puissance voulue, le niveau du test et la taille d'échantillon et nous cherchons à vérifier qu'elle taille d'effet nous pouvons espérer mettre en évidence.
	\end{enumerate}
	Sauf exception, nous considèrerons qu'il est inutile de montrer un test si la puissance escomptée est inférieure à $80\%$. Cette puissance correspond à une probabilité de $80\%$ de ne pas rejeter l'hypothèse nulle à tort, ou, ce qui revient au même de $20\%$ d'erreur de type II.
	
	Évidemment, il est possible de faire le même raisonnement (analytiquement quand c'est possible, sinon numériquement) avec absolument TOUS les tests d'hypothèses que nous avons vus jusqu'à maintenant. Donc au même titre qu'il y a un peu plus d'une centaine de tests d'hypothèses dans le domaine des statistiques comme nous l'avons déjà mentionné... il est évident que nous n'allons pas nous... amuser... à faire les mêmes développements pour tous ces tests mais seulement pour les grands classiques. Tant que nous avons des ordinateurs à notre disposition avec les algorithmes intégrés par des informaticiens/scientifiques, nous pouvons nous passer de refaire tous les développements qui n'apporteraient pas grand chose. Par ailleurs, la majorité des logiciels comportement des outils pour calculer la puissance de $5$ à $10$ tests le plus souvent.
	
	\begin{tcolorbox}[title=Remarque,colframe=black,arc=10pt]
	Nous ne traiterons pas des tests statistiques paramétriques de détection des valeurs abérrantes sur ce site comme le test Q de Dixon ou de Grubb pour la simple raison qu'ils ont une origine trop empirique et qu'ils n'ont aucun intérêt analytiquement parlant. Par contre, si des lecteurs insistent, nous pourrons mettre les détails sur ces tests avec les algorithmes détaillés de calcul des valeurs critiques en utilisant un simple tableur et la technique de Monte-Carlo pour n'importe la distribution de leur votre choix (mais pas uniquement selon la loi Normale contrairement à ce qui est écrit dans la majorité des livres).
	\end{tcolorbox}
	
	\paragraph{Puissance du test $P$ à $1$ et $2$ échantillons}\mbox{}\\\\
	De même que l'intervalle de confiance de la loi Normale avec écart-type théorique connu (c'est-à-dire sur toute la population), nous pouvons déterminer le nombre d'individus (taille d'échantillon) si nous souhaitons imposer une puissance au test de la proportion à $1$ échantillon étudié plus haut. Pour cela, nous utilisons la même technique que pour la puissance du test $Z$. Nous écrivons alors dans un premier temps:
	
	D'où nous déduisons:
	
	Donc si la puissance est de $50\%$, nous retrouvons bien:
	
	Pour la puissance du test de la différence de deux proportions (test de la proportion à deux échantillons) dans l'objectif de déterminer la taille de l'échantillon nous sommes obligés de poser $n=n_1=n_2$. Dés lors, les développements obtenus lors de l'étude du test de la différence de deux proportions s'écrivent:
	
	avec:
	
	De la même manière que nous l'avons fait pour le test $Z$ et le test $P$ à $1$ échantillon, nous avons:
	
	Soit:
	
	Ce qui revient donc à supposer que la différence vraie des deux proportions est la moyenne (ce qui est discutable...).
	
	Mais nous avons aussi (comme les échantillons sont indépendants de par la propriété de la variance):
	
	Soit:
	
	ce qui nous donne:
	
	Nous avons alors après réarrangement:
	
	
	\begin{figure}[H]
		\centering
		\includegraphics[scale=0.18]{img/arithmetics/how_to_run_a_good_study.jpg}	
	\end{figure}
	
	\pagebreak
	\subsubsection{Tests $A$/$B$}
	Le "\NewTerm{test $A$/$B$}\index{test $A$/$B$}" (également appelés "\NewTerm{bucket tests}" ou "\NewTerm {split-run testing}") est une expérience aléatoire avec deux variantes à tester $A$ et $B$. Il inclut l'application de tests d'hypothèses statistiques ou de "tests d'hypothèses à deux échantillons" utilisés dans le domaine des statistiques. Le test $A$/$B$ est un moyen de comparer deux versions d'une même variable, généralement en testant la réponse d'un sujet à la variante $A$ par rapport à la variante $B$, et en déterminant laquelle des deux variantes est la plus efficace.
	
	En plein essor dans le domaine des affaires (business) et considéré commme une analyse de base dans les essais cliniques, les tests $A$/$B$ évaluent l'effet d'une intervention ou d'un traitement en comparant son taux de réussite à celui d'une condition témoin. Comme son nom l'indique, deux versions ($A$ et $B$) sont comparées, qui sont identiques à l'exception d'une variante qui pourrait affecter le comportement d'un utilisateur. La version $A$ pourrait être la version (contrôle) actuellement utilisée, tandis que la version $B$ est modifiée à certains égards (traitement). Par exemple, sur un site Web de commerce électronique, l'entonnoir d'achat est généralement un bon candidat pour les tests $A$/$B$, car même des améliorations marginales des taux de conversion peuvent représenter un gain significatif de ventes. Des améliorations significatives peuvent parfois être observées en testant des éléments tels que la police de texte, les mises en page, les images et les couleurs, mais pas toujours.
	
	\begin{tcolorbox}[title=Remarque,colframe=black,arc=10pt]
	Lorsque vous exécutez des tests $A$/$B$ sur quelque chose et que vous vérifiez régulièrement les expériences en cours pour des résultats significatifs, vous devez faire attention à ne pas tomber dans le piège que les statisticiens appellent les "\NewTerm{erreurs de test significatifs répétés}\index{erreurs de test significatifs répétés}". En effet, le calcul de la significance fait toujours l'hypothèse critique que nous avons que la taille de l'échantillon qui est fixée à \underline{l'avance}! Si nous attendons et que la taille de l'échantillon est trop grande, nous retombons dans le cas du $p$-acking où la taille de l'échantillon est trop grande et crée alors un faux test significatif.
	\end{tcolorbox}
	Voyons trois tests $A$/$B$ typiques  par ordre croissant de complexité mathématique (le premier peut être considéré comme un "test fréquentiste" et les deux autres comme des "tests bayésiens").
	
	
	\paragraph{Test de Fieller (ratio des moyennes)}\mbox{}\\\\
	Le but du "\NewTerm{test de Fieller}" est le calcul d'un intervalle de confiance pour le rapport de deux moyennes. C'est un test particulièrement tendance dans le domaine de la conception de sites Web lorsque deux design ou interface utilisateur d'un même site Web (ou le nombre de clics) doivent être comparés. Le test de Fieller est également connu sous le nom de "\NewTerm{test $A$/$B$ pour les moyennes}\index{test $A$/$B$ pour les moyennes}".
	
	Considérons $ X $ et $ Y $ deux variables aléatoires Normalement distribuées représentant des moyennes d'échantillons de taille respectivement $n$ et $m$ et ayant la même variance, telles que:
	
 	Il s'agit donc d'un cas homoscédastique et ensuite, en pratique, nous prenons l'écart type global.

	En réalité, cependant, nous n'aurons accès qu'aux estimateurs:
	
 	Ce qui nous intéresse, c'est de construire un intervalle de confiance pour le ratio:
	
 	Et c'est évidemment le but du test de Fieller!

	L'idée est alors de calculer et de regrouper les deux variables aléatoires en une seule:
	
 	qui par l'hypothèse nulle suivra donc une variable aléatoire du type:
	
 	On a alors in extenso:
	
 	Si l'écart type ne peut qu'être estimé, on sait qu'on a alors:
	
 	Il s'ensuit, bien entendu, que:
	
	Par symétrie de la loi de Student on peut donc écrire que:
	
	Après réarrangement, on obtient:
	
	Il y a maintenant deux situations intéressantes à considérer en fonction du signe du coefficient $\theta^2$ mais laissons cela de côté pour l'instant en nous concentrant uniquement sur les racines du polynôme entre parenthèses. On a alors (\SeeChapter{voir section Calcul Algébrique \pageref{second order polynomial roots}}):
	
	Ceci est immédiatement simplifié en:
	
 	Le développement dans la racine donne:
	 
	Ce qui peut être réécrit sous une forme qui nous sera utile:
	 
	Posons:
	
 	Nos racines peuvent alors s'écrire:
	
 	ou:
	
	ou autrement:
	
	Nous faisons une dernière simplification:
	
	Qui sont donc pour rappeler les racines de:
	
	Le coefficient de $\theta^2$ dans le polynôme est $Y^2-bS^2T_{\alpha/2}^2(k) $. Par conséquent:
	\begin{itemize}
		\item Si celle-ci est supérieure à zéro, alors la parabole est concave (vue de dessus) et l'intervalle de confiance se situe alors entre les deux racines de la parabole (car les valeurs y sont négatives en fonction de l'inégalité imposée).

		\item Si le coefficient de $\theta^2$ dans le polynôme est négatif alors la parabole est convexe (vue d'en haut) et l'intervalle de confiance est alors en dehors des deux racines de la parabole car les valeurs y sont négatives selon l'inégalité imposée) .

		\item Si le coefficient de $\theta^2$ a une probabilité proche de zéro d'être égal à zéro, nous pouvons ignorer ce scénario.
	\end{itemize}
	Evidemment seul le premier cas nous intéresse car le second est difficile à interpréter (les intervalles auraient des bornes infinies ce qui est irréaliste et surtout inutile). Il faut donc avoir:
	
	Ce qui peut s'écrire:
	
	Dès lors:
	
	\begin{tcolorbox}[colframe=black,colback=white,sharp corners]
	\textbf{{\Large \ding{45}}Exemple:}\\\\
	Considérons le cas où nous avons:
	
	Dès lors:
	
	Nous avons:
	
	et:
	
	Ainsi que (nous montrons ce calcul parce que certains logiciels statistiques - R en particulier - les donnent dans leur sortie alors que normalement il n'est utile que pour le test $T$ de Student de deux échantillons homoscédastiques indépendants):
	
	On a alors l'intervalle suivant entre les deux racines de la parabole:
	
	\end{tcolorbox}
	On peut aller un peu plus loin en considérant le cas où:
	
	Soit dans le contexte estimé:
	
 	Cela nous donne la distribution suivante (en supposant laNormalité) pour la variable aléatoire regroupant les deux premières selon $X-\theta Y$:
	
	Nous avons alors:
	
 	Et donc, comme auparavant, nous pouvons écrire:
	
	Ce que nous pouvons réécrire sous la forme suivante:
	
 	Ce qui développé donne:
	
	Nous regroupons les termes comme précédemment:
	
 	Nous avons alors pour racines après simplification des facteurs numériques:
	
 	Après la méthode de simplification vue ci-dessus (pfff ...) nous obtenons enfin:
	
	
	\paragraph{Test $A$/$B$ bayésien pour résultats binaires}\label{A/B testing for binary outcomes}\mbox{}\\\\
	Pour un test sur résultats binaires (par exemple un test de taux de conversion), la probabilité que $B$ batte $A$ à long terme est donnée par \footnote{La dérivation et le texte qui suivent viennent principalement de \url {https://www.evanmiller.org/bayesian-ab-testing.html}.}:
	
	où:
	\begin{itemize}
		\item $\alpha_A$ est un plus le nombre de succès pour $A$
		\item $\beta_A$ est un plus le nombre d'échecs pour $A$
		\item $\alpha_B$ est un plus le nombre de succès pour $B$
		\item $\beta_B$ est un plus le nombre d'échecs pour $B$
		\item $B$ est la fonction bêta comme vue à la page \pageref{beta function} (gardez à l'esprit que nous devons avoir $a>0,b>0$):
		
	\end{itemize}
	\begin{dem}
	La distribution bêta est une distribution antérieure pratique pour modéliser un paramètre binomial $p$ (car nous savons que c'est l'une des rares distributions à avoir un domaine de définition dans $[0,1]$). À partir d'un a priori non informatif, la distribution de $p$ après $S$ succès et $F$ échecs est donnée par $B(S + 1, F + 1)$. Pour le reste de la discussion, nous considérons $\alpha = S + 1$ et $\beta = F + 1$, qui ne sont que les deux paramètres de la distribution bêta pour le a priori.
	
	Supposons maintenant que nous ayons deux branches expérimentales ($A$ et $B$) et que nous ayons un a priori bayésien pour chacune:
	
	En utilisant la fonction de densité de probabilité de la distribution bêta comme a priori pour $p$, nous pouvons obtenir la probabilité totale que $p_B$ soit supérieur à $p_A$ en intégrant la distribution conjointe sur toutes les valeurs pour lesquelles $p_B>p_A$:
	
	parfois appelée "\NewTerm{intégrale hypergéométrique d'Euler}\index{intégrale hypergéométrique d'Euler}".
	
	En évaluant l'intégrale intérieure:
	
	la relation devient:
	
	où $I_x$ est la fonction bêta incomplète (\SeeChapter{voir section Statistiques page \pageref{incomplete Beta function}}). En utilisant l'identité $I_x(1,b)=1-(1-x)b$, la relation récursive:
	
	et le fait que $\alpha$ et $\beta$ sont des entiers, nous pouvons exprimer $I_x$ comme:
	
	Ou de manière équivalent:
	
	Notre intégrale de probabilité peut donc s'écrire:
	
	Finalement:
	
	\begin{flushright}
		$\blacksquare$  Q.E.D.
	\end{flushright}
	\end{dem}
	Le lecteur doit savoir que même s'il existe une formule fermée, au jour où l'on écrit ces lignes, la plupart des logiciels statistiques mettent en oeuvre des méthodes numériques (simulations) pour calculer cette probabilité!
	
	\begin{tcolorbox}[title=Remarques,colframe=black,arc=10pt]
	\textbf{R1.} Si nous voulons supposer que chaque valeur de $p$ a la même probabilité, c'est-à-dire que nous n'avons aucune information hors données (a priori) que nous aimerions inclure dans notre modèle, alors nous pouvons définir $\alpha_A = \beta_A = \alpha_B = \beta_B $, cela signifie une distribution uniforme pour toutes les valeurs de $p$ dans l'intervalle $[0,1]$.\\
	
	\textbf{R2.} Le lecteur peut se référer à notre livre compagnon électronique gratuit sur le logiciel open source R pour voir comment appliquer facilement ces équations sur des cas pratiques!
	\end{tcolorbox}
	
	\paragraph{Test $A$/$B$ bayésien pour données de comptage}\label{A/B testing for count datas}\mbox{}\\\\
	L'analyse des données de comptage (par exemple, si vous comparez le nombre de ventes par vendeur ou le nombre de ventes d'une semaine à l'autre) nécessite une relation différente. La probabilité que le groupe $1$ ait un taux d'arrivée plus élevé que le groupe $2$ est donnée par:
	
	Une dérivation et une mise en œuvre suivent; ils reflètent tous deux étroitement le cas du résultat binaire.
	
	\begin{dem}
	Soit $\lambda_1$ et $\lambda_2$ les paramètre de Poisson pour chaque groupe. Avec une croyance a priori distribuée gamma (\SeeChapter {voir section Statistiques page \pageref{gamma distribution}}), les vraisemblances postérieures sont données par:
	
	En utilisant la fonction de densité de probabilité de la distribution gamma, nous pouvons obtenir la probabilité totale que $\lambda_1$ soit supérieur à $ \lambda_2$ en intégrant la distribution conjointe sur toutes les valeurs pour lesquelles $\lambda_1>\lambda_2$:
	
	En évaluant l'intégrale interne, l'équation devient:
	
	où $Q$ est la fonction gamma régularisée incomplète supérieure (\SeeChapter{voir section de Calcul Différential et Intégral page \pageref{incomplete regularized Gamma function}}). En utilisiant l'identité $Q(1,z)=e^{-z}$ et la relation récursive:
	
	nous pouvons exprimer $Q$ comme (\SeeChapter{voir section de Calcul Différential et Intégral page \pageref{incomplete regularized Gamma function}}):
	
	L'intégrale de probabilité peut donc s'écrire (nous avons utilisé l'intégrale déjà vue lors de notre étude de la fonction Gamma à la page \pageref{gamma euler function}):
	
	En utilisant $\Gamma(z+1)=z\Gamma(z)$, nous avons:
	
	En remplaçant la fonction gamma par une fonction bêta (voir page \pageref{beta function}), nous avons:
	
	\begin{flushright}
		$\blacksquare$  Q.E.D.
	\end{flushright}
	\end{dem}
	\begin{tcolorbox}[title=Remarque,colframe=black,arc=10pt]
	Encore une fois, le lecteur peut se référer à notre livre compagnon électronique gratuit sur R pour voir comment appliquer ces équations facilement sur des cas pratiques avec un logiciel et un code open-source gratuits!
	\end{tcolorbox}
	
	\pagebreak
	\subsubsection{Analyse de la Variance (ANOVA)}\label{anova}
	"\NewTerm{L'analyse de la variance}\index{aanalyse de la variance}", en anglais "\NewTerm{Analysis Of VAriance}\index{analysis of variance}", est un ensemble de modèles statistiques utilisés pour analyser les différences entre les moyennes des groupes et leurs procédures associées (telles que les "variations" intra et inter groupes), développé par le statisticien et biologiste évolutionniste Ronald Fisher. Dans le cadre de l'ANOVA, la variance observée dans une variable particulière est divisée en composantes attribuables à différentes sources de variation.
	
	Les principales techniques d'analyse des variances sont les suivantes (certaines d'entre elles sont détaillées plus loin) dans l'ordre du plus utilisé au moins utilisé (et par ordre croissant de complexité):
	\begin{multicols}{2}
		\begin{itemize}
		\item ANOVA à un facteur et deux varaibles (test $T$ de Student)\footnote{Il existe un test de Friedman non paramétrique équivalent par une transformation en rangs que nous verrons plus loin..}
		
		\item ANOVA à un facteur fixe ($n$ variables fixes indépendantes)\footnote{Il existe un test équivalent non paramétrique par une transformation en rangs appelé test Kruskal-Wallis où le test post-hoc sera typiquement le test U de Mann-Whitney et nous les verrons tous deux plus loin ci-dessous.}
		
		\item ANOVA à facteurs fixes avec/sans répétitions\footnote{Dans les faits pour toutes les ANOVA, il existe une variante mathématique pour les mesures répétées.}
		
		\item ANOVA multifactorielle avec/sans répétitions 
		
		\item ANOVA à mesures répétées ANOVA (RM-ANOVA)\footnote{L'ANOVA à mesures répétées est l'équivalent de l'ANOVA à un facteur, mais pour des groupes liés, non indépendants, et est l'extension du test $T$ de Student pour données appariées. Une ANOVA à mesures répétées est également appelée ANOVA intra-sujets ou ANOVA pour les échantillons corrélés.}
		
		\item Conception en blocs\footnote{Il existe une variante où les blocs bloqués sont considérés comme fixes ou aléatoires ... Dans les faits, le bloc est considéré comme un autre facteur. Alors dans ce dernier cas il s'agit juste d'une ANOVA à deux facteurs} (randomisés) (ANOVA en blocs)
		
		\item ANOVA carré latin
		
		\item ANOVA carré gréco-latin
		
		\item ANOVA carré de Youden\footnote{Ce sont des carrés latins incomplets, dans lesquels le nombre de colonnes n'est pas égal au nombre de lignes.}
		
		\item ANOVA en parcelles divisées (Split-Plot)
		
		\item ANOVA en bandes croisées (Split-Block)
		
		\item ANOVA en blocs divisés (ANOVA)
		
		\item ANOVA en bandes doublement divisées (Split-Split-Plot)
		
		\item Analyse de la covariance (ANCOVA)
		
		\item Analyse de la variance imbriquée (hiérarchique) (HANOVA)
		
		\item ANOVA partiellement imbriquée (PANOVA)
		
		\item Analyse multivariée de la variance (MANOVA)
		
		\item ANOVA à effets aléatoires
		
		\item ANOVA à effets mixes (avec facteurs fixes ou aléatoires)
		
		\item ANOVA croisées
		
		\item ANOVA robustes (par permutations, basées sur M-estimateurs)
			
		\item ...
		\end{itemize}
	\end{multicols}
	et certaines des ANOVA ci-dessus sont dites "équilibrées" ou "non-équilibrées", selon que le nombre de mesures est égal ou non égal pour chaque variable et peut donc être calculé avec différents types de somme des carrés des erreurs carrées dans le cas déséquilibré nommés "Somme des carrés de type I", "Somme des carrés de type II" ou "Somme des carrés de type III". On peut donc dire qu'il existe au moins entre $20$ et $60$ types d'ANOVA différents (y compris les variations spéciales mais sans inclure les versions randomisées / bootstrapées...)!
	
	Il faut également savoir que la terminologie de l'ANOVA est en grande partie issue des plans statistiques des expériences (\SeeChapter{voir la section Génie industriel page \pageref{doe}}). L'expérimentateur ajuste les facteurs et mesure les réponses pour tenter de déterminer un effet. Les facteurs sont attribués aux unités expérimentales par une combinaison de randomisation et de blocage pour garantir la validité des résultats.

	\begin{tcolorbox}[title=Remarque,colframe=black,arc=10pt]
	Pour le lecteur intéressé à avoir un aperçu plus approfondi, mais sans trop de détails mathématiques, à l'opposé du présent livre, nous recommandons la référence mondiale actuelle suivante pour les experts en ANOVA et en conception d'expériences: \textit{Design and Analyis of Experiments} \footnote{Ensemble de trois volumes !!!} (auteurs: Hinkelmann, K. et Kempthorne, O.), environ $730$.- USD et $1,800$ pages (voir \cite{hinkelmann2007design},\cite{hinkelmann2005design} et \cite{hinkelmann2012design}) et \textit{Applied Linear Statistical Models} (voir \cite{kutner2005applied}).
	\end{tcolorbox}
	
	Le but de cette section aura principalement pour but d'éviter le type de conversation téléphonique suivante entre un spécialiste en ANOVA (S) et un chercheur scientifique ou ingénieur (R). R: «Bonjour, M. Stat, je me demande si vous avez juste une minute pour une question rapide de statistiques?» S: «Habituellement, je ne fais pas de consultation statistique par téléphone, mais laissez-moi voir si je peux vous aider. Quel est le problème?» R: «Nous développons de nouveaux supports de croissance pour les producteurs industriels pour la culture de plantes à fleurs. Nous avons trois de ces supports et nous les utilisons avec quatre variétés de fleurs. Nous avons cinq réplications pour chaque combinaison de médium et de fleur. Nous avons analysé les données sous forme d'une classification bidirectionnelle de $ 3 \times 4$ avec cinq observations par cellule. Mais mon assistant diplômé a parlé à l'un de vos étudiants et il est maintenant confus quant à la validité de cette analyse. Je veux juste que vous confirmiez que nous avons fait la bonne chose.» S: «Eh bien, je ne sais pas.» R: «Que voulez-vous dire, vous ne savez pas? Vous êtes expert pourtant?!» S: «J'ai vraiment besoin d'en savoir plus sur la façon dont vous avez réalisé l'expérience. Par exemple, comment avez-vous préparé les supports que vous avez utilisés dans les pots individuels? Je suppose que vous cultivez les fleurs dans des pots dans la serre.» R: «Oui, c'est vrai. Mon assistant diplômé a simplement mélangé chaque médium dans un grand récipient, que nous avons ensuite placé dans les pots individuels.» S: «Cela peut être un problème, car il se peut maintenant que vous n'ayez aucune réplication.» R: «Que voulez-vous dire, nous n'avons pas de réplication? Je viens de vous dire que nous avons cinq répliques.» S: «Oui, mais ... je pense qu'il serait préférable que vous veniez à mon bureau pour que je vous explique cela et pour examiner de plus près votre expérience.» R: «Mais nous avons déjà soumis l'article pour publication.» S: «Alors pourquoi ne viendriez-vous pas quand vous recevrez les retours?» Silence. R: «Oui, merci. Je ferai ainsi...»
	
	Ceci est, bien sûr, juste une conversation fictive \footnote{Un autre exemple réel typique peut être trouvé dans la vidéo YouTube suivante \url{https://youtu.be/ew2psDyhxb8}}. Mais de nombreux statisticiens consultants ont eu des conversations similaires (la plupart du temps, le document a déjà été rejeté). Le but de cette section est d'aider les ingénieurs non seulement à mieux se comprendre, mais aussi à mieux comprendre les subtilités de la conception et de l'analyse d'expériences.
	
	\pagebreak
	\paragraph{ANOVA à un facteur fixe}\label{anova one way fixed factor}\mbox{}\\\\
	L'objectif de l'analyse de la variance (contrairement à ce que son nom pourrait laisser penser) est une technique statistique permettant de comparer les moyennes de deux populations ou plus (très utilisé dans le pharma ou dans les labos de R\&D ou de bancs d'essais\footnote{Dans le cas des "facteurs aléatoires" (concept que nous verrons plus tard!) on compare cependant les variances ....!}). Cette méthode, néanmoins, doit son nom au fait qu'elle utilise des mesures de variance afin de déterminer le caractère statistiquement significatif, ou non, des différences de moyennes mesurées sur les populations.

	Plus précisément, la vraie signification est de savoir si le fait que des moyennes d'échantillons sont (légèrement) différentes peut être attribué au hasard de l'échantillonnage ou provient du fait qu'un facteur de variabilité engendre réellement des échantillons significativement différents (si nous avons les valeurs de toute la population, nous n'avons rien à faire!).
	 
	\begin{tcolorbox}[title=Remarque,colframe=black,arc=10pt]
	Pour plus d'informations au niveau du vocabulaire et la mise en application, l'ingénieur et le chercheur se reporteront à la norme ISO 3534-3:1999.
	\end{tcolorbox}
	
	Pour l'analyse de la variance appelée "\NewTerm{ANOVA à un facteur}\index{ANOVA!ANOVA \'a un facteur}" (ANalysis Of VAriance) ou "\NewTerm{ANAVAR à un facteur}" (ANAlyse de la VARiance), ou encore "\NewTerm{ANOVA à une voie}\index{ANOVA!ANOVA \'a une voie}" ou plus rigoureusement "\NewTerm{ANOVA à un facteur fixe avec répétitions}\index{ANOVA!ANOVA \'a un facteur fixe avec répétitions}" ou encore "\NewTerm{ANOVA à une variable catégorielle fixe avec répétitions}\index{ANOVA!ANOVA \'a une variable catégorielle fixe avec répétition}", nous allons d'abord rappeler, comme nous l'avons démontré, que la loi de Fisher-Snedecor est donnée par le rapport de deux variables aléatoires indépendantes qui suivent une loi du Khi-deux et divisée par leur degré de liberté tel que:
	
	et nous allons voir maintenant son importance.
	
	\begin{tcolorbox}[title=Remarque,colframe=black,arc=10pt]
	Lorsqu'un facteur peut avoir un très grand nombre de niveaux nous considérons le fait d'avoir choisi un niveau du facteur parmi une multitude de possibles comme une sélection aléatoire. Raison pour laquelle nous parlons alors dans ce derniers cas de "facteur aléatoire" qui fait l'objet d'ANOVA particulières étudiées une fois celles à facteurs fixes maîtrisées (par exemple les ANOVA mélengeant facteurs fixes et facteurs aléatoires sont appelées "\NewTerm{ANOVA mixes}\index{ANOVA!ANOVA mixes}").
	\end{tcolorbox}
	
	Considérons un échantillon aléatoire de taille $n$, disons $X_1,X_2, ...,X_n$ issu de la loi $\mathcal{N}(\mu_X,\sigma_X)$ et un échantillon aléatoire de taille $m$, disons $Y_1,Y_2, ...,Y_m$ issu de la loi $\mathcal{N}(\mu_Y,\sigma_Y)$.
	
	Considérons les estimateurs du maximum de vraisemblance de l'écart-type de la loi Normale traditionnellement notés dans le domaine de l'analyse de la variance par:
	
	Les statistiques ci-dessus sont celles que nous utiliserions pour estimer les variances si les moyennes théoriques $\mu_X,\mu_Y$ étaient connues. Donc nous pouvons utiliser un résultat démontré plus haut lors de notre étude des intervalles de confiance:
	
	Comme les $X_i$ sont indépendantes des $Y_j$  (hypothèse qui implique que la covariance est nulle, la réciproque n'étant pour rappel pas toujours vraie!), les variables:
	
	sont indépendantes l'une de l'autre.
	\begin{tcolorbox}[title=Remarque,colframe=black,arc=10pt]
	Il existe un type d'ANOVA prévu pour le cas où les variables ne sont pas indépendantes (on parle alors de "\NewTerm{covariable}\index{covariable}"). Il s'agit de "\NewTerm{l'ANCOVA}\index{ANCOVA}" qui signifie "\NewTerm{Analyse de la COvariance et de la VAriance}\index{Analyse de la COvariance et de la VAriance}" qui utilise un mix entre la régression linéaire (\SeeChapter{voir section Méthodes Numériques page \pageref{univariate regression variance analysis}}) et l'ANOVA. Le but de l'ANCOVA est de supprimer statistiquement l'effet indirect de la covariable. Nous verrons bien plus loin comment cela fonctionne dans les détails.
	\end{tcolorbox}
	Nous pouvons donc appliquer la loi de Fisher-Snedecor avec:
	
	Nous avons donc:
	
	Soit:
	
	Ce théorème nous permet de déduire l'intervalle de confiance du rapport de deux variances lorsque la moyenne théorique est connue. Puisque la fonction de Fisher n'est pas symétrique, la seule possibilité pour faire l'inférence c'est de faire appel au calcul numérique et nous noterons alors pour un intervalle de confiance donné le test de la manière suivante:
	
	Dans le cas où les moyennes $\mu_X,\mu_Y$ sont inconnues, nous utilisons les estimateurs sans biais des variances traditionnellement notés dans le domaine de l'analyse de la variance par:
	
	Pour estimer les variances théoriques, nous utilisons le résultat démontré plus haut:
	
	Comme les $X_i$ sont indépendantes des $Y_j$ (hypothèse!), les variables:
	
	sont indépendantes l'une de l'autre. Nous pouvons donc appliquer la loi de Fisher-Snedecor avec:
	
	Nous avons donc:
	
	Soit:
	
	Ce théorème nous permet de déduire l'intervalle de confiance du rapport de deux variances lorsque la moyenne empirique est connue. Puisque la fonction de Fisher n'est pas symétrique, la seule possibilité pour faire l'inférence c'est de faire appel au calcul numérique et nous noterons alors pour un intervalle de confiance donné le "\NewTerm{test de Fisher}\index{test de Fisher}" de la manière suivante:
	
	tout en se rappelant que son utilisation nécessite implicitement des contraintes de normalité des variables étudiées.
	
	R. A. Fisher (1890-1962) est, comme Karl Pearson, l'un des principaux fondateurs de la théorie moderne de la statistique. Fisher étudia à Cambridge où il obtint en 1912 un diplôme en astronomie. C'est en étudiant la théorie de l'erreur dans les observations astronomiques que Fisher s'intéressa à la statistique. Fisher est l'inventeur de la branche de la statistique appelée l'analyse de la variance.
	
	Au début du 20ème siècle, R. A. Fisher développe donc la méthodologie des plans d'expérience (\SeeChapter{voir section de Génie Industriel page \pageref{doe}}). Pour valider l'utilité d'un facteur, il met au point un test permettant d'assurer que des échantillons différents sont de natures différentes. Ce test est basé sur l'analyse de la variance (des échantillons), et nommé ANOVA pour "\NewTerm{analyse normalisée de la variance}\index{analyse normalis\'ee de la variance}".
	
	Prenons $k$ échantillons de $n$ valeurs aléatoires chacun. Chacune des valeurs étant considérée comme une observation ou une mesure de quelque chose ou sur la base de quelque chose (un lieu différent, ou un objet différent... bref: un seul et unique facteur de variabilité entre les échantillons!). Nous aurons donc un nombre total de $N$ d'observations (mesures) donné par:
	
	si chacun des échantillons a un nombre identique de valeurs $n$ (taille de l'échantillon) tel que $n_1=n_2=...=n_k$ nous parlons alors de "\NewTerm{plan équilibré}\index{plan \'equilibr\'e}" ou "\NewTerm{plan balancé}\index{plan balanc\'e}" à $k$ niveaux (ou $k$ modalités).
	
	\begin{tcolorbox}[title=Remarque,colframe=black,arc=10pt]
	 Si nous avons plusieurs facteurs de variabilité (par exemple: chaque lieu compare à lui-même plusieurs laboratoires), nous parlerons alors "\NewTerm{d'ANOVA multifactorielle}\index{ANOVA multifactorielle}". Dès lors, s'il n'y a que deux facteurs de variabilité, nous parlons "\NewTerm{d'ANOVA à deux facteurs}\index{ANOVA \'a deux facteurs}" (voir plus loin pour plus de détails sur les différentes ANOVA à deux facteurs).
	\end{tcolorbox}
	
	Nous considérerons que chacun des $k$ échantillons est issu (suit) d'une variable aléatoire suivant une loi Normale:
	\begin{table}[H]
		\begin{center}
		\begin{tabular}{ |l|l|l|l| }
			\hline
			\multicolumn{4}{ |c| }{Facteur $1$} \\
			\hline
			Échantillon $1$ & Échantillon $2$ & Échantillon ...$i$ & Échantillon $k$ \\ \hline
			 $x_{11}$ & $x_{12}$ & $\ldots$ & $x_{1k}$ \\
			 $x_{21}$ & $x_{22}$ & $\ldots$ & $x_{2k}$ \\
			 $\ldots$ & $\ldots$ & $x_{ij}$ & $\ldots$ \\
			 $x_{n1}$ & $x_{n2}$ & $\ldots$ & $x_{nk}$ \\ \hline
			 Moyenne: $\bar{x}_1$ &  Moyenne: $\bar{x}_2$  &  Moyenne: $\bar{x}_i$ & Moyenne: $\bar{x}_k$ \\
			\hline
		\end{tabular}
		\end{center}
		\caption{Structure typique dite "croisée" d'une analyse de la variance à un facteur}
	\end{table}
	Par conséquent, dans une analyse de variance à un facteur (ANOVA à un facteur), un seul facteur (traitement) à plusieurs niveaux est considéré.
	
	En termes de test, nous voulons tester si les moyennes des $k$ échantillons de taille $n$ sont égales sous l'hypothèse que leurs variances sont égales. Ce que nous écrivons sous forme d'hypothèse de la manière suivante:
	
	Autrement dit, les échantillons sont représentatifs d'une même population (d'une même loi statistique). C'est-à-dire que les variations constatées entre les valeurs des différents échantillons sont dues essentiellement au hasard. Pour cela nous étudions la variabilité des résultats dans les échantillons et entre les échantillons. 
	
	\begin{tcolorbox}[title=Remarque,colframe=black,arc=10pt]
	Le modèle statistique des observations peut aussi s'écrire sous la forme très importante suivante, nommée "\NewTerm{modèle d'effets factoriels}\index{mod\'ele d'effets factoriels}" ou même parfois "\NewTerm{modèle linéaire classificatoire}":
	
	Le paramètre  $\mu$ est la "\NewTerm{moyenne globale}", $\tau_i$ est le $i$-ème "\NewTerm{effet de traîtement}", et $\varepsilon_{ij}$ est une "\NewTerm {erreur aléatoire}". On suppose que les $\varepsilon_{ij}$ sont i.i.d selon $\mathcal{N}(0, \sigma^2) $. C'est un modèle complètement additif (il n'y a pas d'interactions!). Sous cette forme les hypothèses sont alors:
	
	Notez que nous avons alors:
	
	\end{tcolorbox}	
	Il revient exactement au même comme nous le verrons plus loin de poser que (formulation qu'on retrouve dans certains articles ou ouvrages):
	
	Nous noterons donc pour la suite $i$ l'indice d'échantillon (de $1$ à $k$) et $j$ l'indice de l'observation (de $1$ à $n$). Donc $x_{ij}$ sera la valeur de la $j$-ème observation de l'échantillon de données numéro $i$ (nous avons choisi d'inverser la notation d'usage donc attention à ne pas vous tromper par la suite... nous sommes désolés... c'était une bêtise!).
	
	Selon l'hypothèse susmentionnée, nous avons:
	
	Nous noterons par $\bar{x}_i$ la moyenne empirique/estimée (arithmétique) de l'échantillon $i$ (souvent appelée "\NewTerm{moyenne marginale}\index{moyenne marginale}" et parfois notée aussi $\bar{x}_{i.}$):
	
	et $\bar{\bar{x}}$ (aussi parfois notée $\bar{x}_{..}$) la moyenne empirique/estimée des $N$ valeurs (soit la moyenne des $\bar{x}_i$) donnée donc par:
	
	En utilisant  les propriétés de l'espérance et de la variance déjà démontrées plus haut nous savons que:
	
	avec $\mu$ qui est la moyenne des moyennes vraies $\mu_i$:
	
	\begin{tcolorbox}[title=Remarque,colframe=black,arc=10pt]
	Écrire le modèle comme modèle à effet de facteurs:
	
	nous voyons que les estimateurs du maximum de vraisemblance des paramètres du modèle sont $\hat{\mu}=\bar{\bar{x}}$ et $\hat{\tau}_i=\bar{x}_i-\bar{\bar{x}}$. Dès lors:
	
	\end{tcolorbox}
	Maintenant, introduisons trois variances importantes:
	\begin{enumerate}
		\item La "\NewTerm{variance totale}\index{variance totale}" comme étant intuitivement la variance estimée sans biais en considérant l'ensemble des $N$ observations comme un unique échantillon:
			
			où le terme au numérateur est appelé "\NewTerm{somme des carrés des écarts totaux}\index{somme des carrés des écarts totaux}".
			
			\item  La "\NewTerm{variance entre échantillons}\index{variance entre échantillons}" (c'est-à-dire entre les moyennes des échantillons) est aussi intuitivement l'estimateur de la variance des moyennes des échantillons:
			
			où le terme au numérateur est appelé "\NewTerm{somme des carrés des écarts entre échantillons}\index{somme des carrés des écarts entre échantillons}".
			
			Comme nous avons démontré que si toutes les variables sont identiquement distribuées (même variance) et indépendantes la variance des individus vaut $n$ fois celle de la moyenne:
			
			alors la  "\NewTerm{variance des observations}\index{variance des observations}" (variables aléatoires dans un échantillon) est donnée par:
			
			Nous avons donc ci-dessus l'hypothèse de l'égalité des variances qui est exprimée sous forme mathématique pour les développements à suivre.
			
			\item La "\NewTerm{variance résiduelle}\index{variance résiduelle}" est l'effet des facteurs dits "non contrôlés". C'est par définition la moyenne des variances des échantillons (en quelque sorte: l'erreur standard):
			
			où le terme au numérateur est appelé "\NewTerm{somme des carrés des écarts des résidus}\index{somme des carrés des écarts des résidus}" ou encore plus souvent "\NewTerm{erreur résiduelle}\index{erreur résiduelle}".
	\end{enumerate}
	Au final, ces indicateurs sont parfois résumés sous la forme suivante:
	
	Remarquons que si les échantillons n'ont pas la même taille (ce qui est rare dans la pratique), nous avons alors:
	
	\begin{tcolorbox}[title=Remarques,colframe=black,arc=10pt]
	\textbf{R1.} Le terme $Q_T$ est souvent indiqué dans l'industrie par l'abréviation SST signifiant en anglais "\NewTerm{Sum of Squares Total}\index{total sum of squares}" ou plus rarement TSS pour "\NewTerm{Total Sum of Squares}".\\
	
	\textbf{R2.} Le terme $Q_A$ est souvent indiqué dans l'industrie par l'abréviation SSB signifiant en anglais "\NewTerm{Sum of Squares Between (samples)}\index{sum of squares between samples}" ou plus rarement SSk pour "\NewTerm{Sum of Squares Between treatments}\index{sum of squares between treatments}" (aussi noté parfois: $\text{SS}_\text{tr}$).\\
	
	\textbf{R3.} Le terme $Q_R$ est souvent indiqué dans l'industrie par l'abréviation SSW signifiant en anglais "\NewTerm{Sum of Squares Within (samples)}\index{sum of squares within samples}" ou plus rarement SSE pour "\NewTerm{Sum of Squares due to Errors}\index{sum of squares due to errors}" (aussi noté parfois: $\text{SS}_E$).
	\end{tcolorbox}	
	Indiquons que nous voyons souvent dans la littérature (nous réutiliserons un peu plus loin cette notation):
	
	avec donc l'estimateur sans biais de la variance des observations:
	
	Avant d'aller plus loin, arrêtons-nous sur la variance résiduelle. Nous avons donc pour des échantillons qui ne sont pas de même taille:
	
	Cette écriture est souvent appelée "\NewTerm{variance groupée}\index{variance groupée}\index{variance combinée}\index{variance composite}\index{variance globale}" ("pooled variance" en anglais). La racine carrée d'un estimateur de variance groupée est évidemment connue sous le nom de "\NewTerm{écart-type groupé}\index{\'ecart-type group\'e}\index {\'ecart-type combin\'e}\index{\'ecart-type composite}\index{\'ecart-type global}\label{pooled standard deviation}". Nous voyons aussi que si nous assumons que toutes les moyennes $\bar{x}_i $ sont égales et également toutes les tailles d'échantillons $n_i$, nous retombons sur l'erreur standard!
	
	Ouvrons maintenant une petite parenthèse... Prenons le cas particulier de deux échantillons seulement:
	
	Soit en introduisant l'estimateur du maximum de vraisemblance de la variance:
	
	\begin{tcolorbox}[title=Remarque,colframe=black,arc=10pt]
	Il est alors facile de comprendre d'où vient la relation suivante:
	
	aussi nommé "\NewTerm{variance de variance de répétabilité de $p$ séries de données}".\\
	
	Dans les expériences de laboratoire, cela conduit également à la définition de la "\NewTerm {variance inter-séries}\index{variance inter-séries}" (différence entre la variance des valeurs moyennes de la série de données et la variance de répétabilité divisée par le nombre de mesures par série), où étant donné $p$ séries de données ($i$ variant de $1$ à $p$), chaque série de données contenant des valeurs $j$ ($j$ variant de $1$ à $n_i$):
	
	Dans la pratique, si la variance des moyennes est inférieure à la contribution de la variance de répétabilité, $S_R^2$, la variance inter-séries est définie égale à zéro!
	\end{tcolorbox}
	Nous pouvons d'ailleurs observer que dans le cas particulier où $n_1=n_2=n$:
	
	Donc:
	
	Supposons maintenant que nous souhaitions comparer avec un certain intervalle de confiance la moyenne de deux populations ayant une variance différente pour savoir si elles sont de natures différentes ou non.
	
	Nous connaissons pour le moment deux tests pour vérifier les moyennes. Le test-$Z$ et le test-$T$. Comme dans l'industrie il est rare que nous ayons le temps de prendre des grands échantillons, concentrons-nous sur le deuxième que nous avions démontré plus haut:
	
	Et rappelons aussi que:
	
	Maintenant rappelons que nous avons démontré que si nous avions deux variables aléatoires de loi:
	
	alors la soustraction (différencier) des moyennes donne (propriété de stabilité de la loi Normale):
	
	Donc pour la différence de deux  moyennes de variables aléatoires provenant de deux échantillons de population nous obtenons directement:
	
	Et maintenant l'idée est de prendre l'approximation (sous l'hypothèse que les variances sont égales):
	
	Cette approximation est appelée "\NewTerm{hypothèse homoscédastique}\index{hypothèse homoscédastique}".
	
	Nous avons alors l'intervalle de confiance (en supposant que nous n'avons à notre connaissance qu'un estimateur de la variance) suivant en se rappelant que la soustraction ou la somme de deux variables aléatoires indépendantes implique que leurs variances s'additionnent toujours (et donc il en va de même pour les degrés de liberté de la loi de Student y relative comme nous l'avons démontré plus haut suite à la liaison directe avec la loi du khi-2):
	
	avec:
	
	Comme l'idée dans la pratique est souvent de tester l'égalité des moyennes théoriques (et donc que leur différence est nulle) à partir des estimateurs connus alors:
	
	Dans la plupart des logiciels disponibles sur le marché, le résultat est uniquement donné à partir du fait que le quantile $T$ que nous avons est compris dans le $T_{\alpha/2}$ correspondant à l'intervalle de confiance donné rappelons-le par:
	
	dans le cas de l'hypothèse homoscédastique (égalité des variances/homogénéité des variances).
	\begin{tcolorbox}[title=Remarque,colframe=black,arc=10pt]
	Cette dernière relation est appelée "\NewTerm{independent two-sample $T$-test}\index{independent two-sample $T$-test}", ou "\NewTerm{test-T homoscédastique}\index{test-$T$ homoscédastique}\label{homoscedastic t test}" ou encore "\NewTerm{test-$T$ d'égalité des espérances de $2$ observations avec variances égales}" ou encore plus simplement mais un peu abusivement "\NewTerm{test-$T$ à $2$ échantillons}", avec taille des échantillons différentes et variances égales. Souvent dans la littérature, les deux moyennes théoriques sont égales lors de la comparaison. Il s'en suit que nous avons alors:
	
	\end{tcolorbox}
	Sinon, dans le cas plus général de l'hypothèse d'hétéroscédasticité (non égalité des variances), nous écrivons explicitement (nous reviendrons là-dessus lors de notre étude du test de Welch plus loin....):
	
	Donc:
	
	\begin{tcolorbox}[title=Remarque,colframe=black,arc=10pt]
	La relation antéprécédente est appelée "\NewTerm{independent two-sample $T$-test}", ou "\NewTerm{test-$T$ hétéroscédastique}\index{test-$T$ hétéroscédastique}" ou encore "\NewTerm{test d'égalité des espérances: deux observations avec variances différentes}". Si la taille des échantillons est égale et que les variances le sont aussi et que nous supposons les deux moyennes théoriques égales lors de la comparaison, il s'ensuit que nous avons alors:
	
	\end{tcolorbox}
	Si nous devons comparer deux moyennes $i,j$ parmi un nombre $k$ de plusieurs moyennes d'une ANOVA (comme les tests post-hoc pour identifier quelles moyennes sont différentes car le test omnibus ANOVA ne nous dit rien à ce sujet!), la relation précédente est alors généralisée à:
	
	et réécrit traditionnellement comme suit:
	
	et est nommé le "\NewTerm{test de Fisher de différence la moins significative}\index{test de Fisher de différence la moins significative}\index{test DMS}\index{DMS de Fisher}" ou simplement abrégé DMS\footnote{Les tests post-hoc ne nécessitent pas que l'ANOVA soit faite en premier, sauf le test DMS (l'ANOVA est ce que l'on appelle un "test protecteur" du DMS). Post-hoc ne signifie en aucun cas «post-ANOVA» (ceux qui le revendiquent, même les auteurs de manuels, n'ont clairement aucune idée de ce que font réellement les tests!). Post hoc (ou post factum) signifie «après la collecte des données».} (ou "least significant difference Fisher test" en anglais). Évidemment le test consiste à vérifier que $\bar{X}_i-\bar{X}_j>\text{DMS}$ à un seuil $\alpha$ donné et à répéter cette procédure pour toutes les $k(k-1)/2$ comparaisons.
	
	Bref, fermons cette parenthèse et revenons à nos moutons... Nous en étions donc au tableau suivant:
	
	où nous avons donc dans le cas d'échantillons de même taille:
	
	et ainsi que l'erreur totale qui est la somme de l'erreur des moyennes (interclasses) et de l'erreur résiduelle (intra-classes) et ce que les échantillons soient de même taille ou non:
	
	\begin{tcolorbox}[title=Remarque,colframe=black,arc=10pt]
	Rappelez-vous que cette décomposition de somme totale des carrés s'écrit aussi parfois:
	
	où SSTR est l'abréviation Somme des Carrés des Traîtements Résiduels (l'équivalent de $Q_A$) et Somme des Carrés des Erreurs (l'équivalent de $Q_R$).
	\end{tcolorbox}
	Effectivement:
	
	Or, nous avons:
	
	car:
	
	Donc:
	
	Maintenant, sous l'hypothèse forte (qui va nous être indispensable un peu plus loin) que les variances vraies sont liées par la relation:
	
	et donc que leurs estimateurs respectifs sont asymptotiquement égaux... ce qui dans la pratique n'est approximativement vrai que lorsque certaines conditions sont satisfaites (raison pour laquelle il faut absolument avant de faire une ANOVA exécuter un calcul de la puissance et de l'effectif d'une ANOVA!) nous avons:		
	
	ce qui découle immédiatement de la démonstration que nous avions faite lors de notre étude de l'inférence statistique avec la loi du Khi-deux où nous avions obtenu (pour rappel):
	
	Pour déterminer le nombre de degrés de liberté de la loi du Khi-deux de:
	
	nous allons utiliser le fait que (par le même raisonnement que pour la relation antéprécédente)
	
	et puisque $Q_T=Q_A+Q_R$,  nous devons alors avoir:
	
	Il s'ensuit de par la propriété de linéarité du Khi-deux:
	
	Donc pour résumer nous avons:
	
	C'est maintenant qu'intervient la loi de Fisher dans l'hypothèse où les variances sont égales (et les mesures Normalement distribuées)! Puisque:
	
	Ce que nous souhaitons faire c'est voir s'il y a une différence entre la variance des moyennes (interclasses) et la variance résiduelle (intra-classes). Pour comparer deux variances lorsque les moyennes vraies sont inconnues nous avons vu que le mieux était d'utiliser le test de Fisher. Or, nous avons démontré dans notre étude de la loi de Fisher un peu plus haut que:
	
	où dans notre cas d'étude:
	
	Comme il existe des dizaines de types différentes d'ANOVA il faut bien comprendre ce choix de la plus simple des ANOVA que nous sommes entrain d'étudier maintenant. Ainsi, si les moyennes sont les mêmes, l'hypothèse nulle est alors que ce rapport des variances est égal à l'unité (sous les conditions déjà susmentionnées bien plus haut). Si $F$ vient à être trop grand à un seuil donné, nous rejetons alors l'hypothèse nulle d'égalité des moyennes (car in extenso les variances vont être fortemement différentes aussi). Donc ici il semble cohérent de comparer les variances entre groupes (numératieur) avec celle dans les groupes (numérateur) mais comme nous le verrons ce n'est pas toujours ce choix qui sera fait (particulièrement dans les ANOVA hiérarchisées).
	
	Au vu de l'hypothèse de la première égalité de le relation ci-dessus (qui précède l'implication), nous comprenons en même temps aussi beaucoup mieux la très grande sensibilité des résultats de l'ANOVA à la non égalité des variances vraies!

	Indiquons encore que la relation précédente:
	
	est souvent indiquée dans la littérature sous la forme suivante:
	
	où MSk est appelé "\NewTerm{Mean Square for treatments}\index{mean square for treatments}" et MSE "\NewTerm{Mean Square for Error}\index{mean square for error}". Ce rapport va donc nous donner la valeur de la variable aléatoire $F$ (dont le support est pour rappel borné à zéro à gauche). Concernant le choix du test (unilatéral droite/gauche ou bilatéral), remarquons que si les moyennes sont vraiment égales, alors pour tout $i$:
	
	Donc dans ce cas:
	
	Ce qui nous amène évidemment à immédiatement adopter un test unilatéral droite!
	
	Sinon, en général l'interprétation de cette fraction est donc en gros la suivante: Il s'agit du rapport (normalisé au nombre de degrés de liberté) de la somme de l'erreur des moyennes (interclasses) et de l'erreur résiduelle (intra-classes) ou autrement dit le rapport de la variance interclasse par la variance résiduelle. Ce rapport suit donc une loi de Fisher à deux paramètres donnés par les degrés de liberté des classes respectives.
	
	\begin{tcolorbox}[title=Remarque,colframe=black,arc=10pt]
	S'il y a seulement deux populations (échantillons), il faut bien comprendre qu'à ce moment l'utilisation du test-$T$ de Student suffit amplement et est considéré comme équivalent! Au fait, l'ANOVA est une comparaison indirecte des moyennes, Student une comparaison directe... il est donc évident de deviner lequel est le mieux dans cette situation particulière!
	\end{tcolorbox}	
	
	Tous les calculs que nous avons faits sont très souvent représentés dans les logiciels sous la forme d'une table standardisée dont voici la forme et le contenu (c'est ainsi que le présente Microsoft Excel 11.8346 ou Minitab 15.1.1 par exemple): 
	\begin{table}[H]\small
		\centering
		\resizebox{\textwidth}{!}{\begin{tabular}{llcccc}\hline
		\textbf{Source} & \textbf{Somme des carrés (SCE)} & $\chi^2$ \textbf{ddl} & \textbf{Moyenne des carrés} & $F$ & $F$ \textbf{Critical} \\ \hline
		Inter-Classe & $Q_A=\displaystyle\sum_{i}n_i\left(\bar{x}_{i}-\bar{\bar{x}}\right)^2$ & $k-1$ & $\text{MCk}=\displaystyle\dfrac{Q_A}{k-1}$ &
		$\displaystyle\dfrac{\text{MCk}}{\text{MCE}}$ & $P(F> F_{k-1,N-k})$ \\
		Intra-Classe & $Q_R=\displaystyle\sum_{ij}\left(x_{ij}-\bar{x}_i\right)^2$ & $N-k$ & $ \text{MCE}=\displaystyle\dfrac{Q_R}{N-k}$  & & \\
		Total & $Q_T=\displaystyle\sum_{ij}\left(x_{ij}-\bar{\bar{x}}\right)^2$ & $N-1$ & & &\\ \hline
		\end{tabular}}
		\caption{Terminologie et paramètres traditionnels d'un Tableau ANOVA (TAV) à un facteur}
	\end{table}
	Ainsi, pour que l'hypothèse nulle ne soit pas rejetée, il faut que la valeur de:
	
	soit plus petite ou égale au centile de la même loi $F$ avec une probabilité cumulée correspondante à $1$ soustrait de niveau de confiance $\alpha$ et notée parfois $F_c$.
	
	La valeur choisie du $F$ critique est un peu malheureuse à notre avis dans les tableaux d'ANOVA (mais bon une fois que l'on sait que c'est ainsi...). Il est peut-être plus aisé de comprendre cette valeur si nous l'introduisons ainsi (le test unilatéral à droite ressort pédagogiquement mieux à mon avis):
	
	La $p$-valeur dans le tableau ANOVA et les résultats de comparaison multiples comme le test de Dunnett (voir plus loin ci-dessous) et le test de Tukey (voir également plus loin ci-dessous) sont basés sur des méthodologies différentes et peuvent parfois produire des résultats contradictoires. Par exemple, il est possible que la $p$-valeur de l'ANOVA puisse indiquer qu'il n'y a pas de différences entre les moyennes tandis que la sortie de comparaisons multiples indique que certaines moyennes sont différentes. Dans ce cas, il est d'usage à ce jour généralement faire confiance à la sortie de comparaisons multiples.
	
	Il faut cependant bien se rappeler que pour utiliser l'ANOVA, on doit donc supposer que les échantillons sont issus de populations différentes (données non-appariées) et suivent une loi Normale (même si l'ANOVA est relativement robuste à la non-normalité). Il est donc nécessaire de vérifier la normalité des distributions et l'homoscédasticité (test de Levene par exemple). Dans le cas contraire, il faut utiliser des variantes non paramétriques de l'analyse de variance (ANOVA de Kruskal-Wallis ou ANOVA de Friedman par exemple parmi d'autres que nous verrons plus loin).
	
	\begin{tcolorbox}[title=Remarques,colframe=black,arc=10pt]
	\textbf{R1.} À noter que dans la pratique, la variance inter-classe est très souvent nommée "\NewTerm{variance inter-laboratoires}\index{variance inter-laboratoires}" et la variance intra-classe est in extenso souvent nommée "\NewTerm{variance intra-laboratoire}\index{variance intra-laboratoire}".\\

	\textbf{R2.}  Il existe en ce début de 21ème siècle plus de $50$ test ou procédures de comparaison de variances. L'opinion varie parmi les auteurs quant à leur pertinence et l'efficacité des tests d'homogénéité de variance (\NewTerm{THV}). Certains affirment que ces derniers sont indispensables à réaliser avant toute ANOVA, d'autres disent que ces tests sont de toute façon de piètre performance, l'ANOVA étant plus robuste aux écarts d'homoscédasticité que ce qui peut être détecté par les THV, particulièrement en cas de non-Normalité. En fait, toutes ces questions se reportent au problème dit "\NewTerm{problème de Behrens-Fisher}\index{problème de Behrens-Fisher}", qui est celui de la comparaison de moyennes sans supposer l'équivariance. Cependant parmis la cinquantaine de tests existants, plusieurs études comparatives ont permis de dégager les tests suivants: Test de Bartlett, Levene et Brown-Forsythe.\\

	\textbf{R3.} Lorsque certains niveaux d'un facteur sont réunis en un seul pour être comparés à un niveau de référence les statisticiens parlent alors de création de "\NewTerm{contrastes}\index{contrastes}" (voir plus bas).
	\end{tcolorbox}
	La distribution de $\bar{x}_i$ est $\mathcal{N}(\mu_i,\sigma^2_\varepsilon/n)$. Dès lors, comme nous l'avons démontré plus haut, nous pouvons écrire:
	
	Donc:
	
	Un intervalle de confiance pour la différence de deux moyennes est alors comme d'habitude:
	
	
	\subparagraph{Test de Welch}\label{Welch test}\mbox{}\\\\
	Lors de notre étude de l'ANOVA à un facteur fixe ci-dessus, nous avions introduit le fait qu'il était possible de comparer la moyenne de deux populations dont les données suivaient une distribution Normale mais de variances inégales en utilisant le test hétéroscédastique $T$-test Student:
	
	avec:
	
	Mais en réalité nous avions un peu passé sous silence ... (certains l'auront peut-être remarqué) que nous ne sommes pas autorisés à faire cette dernière addition des degrés de liberté dès que les variances ne sont pas égales! En effet, revenons à la source de la distribution Student:
	
	où pour rappel, la variable $Z$ suit une loi Normale centrée réduite et la variable aléatoire $U$ suit une loi du $\chi^2$ avec $n-1$ degrés de liberté. Et rappelez-vous que nous avons prouvé que:
	
	Mais maintenant, considérons que cette même variable aléatoire $U$ est donnée par (nous travaillons pour rappel avec l'écart type de la moyenne!):
	
	où $ X $ et $ Y $ sont deux variables aléatoires indépendantes suivant des lois du $ \ chi ^ 2 $ avec respectivement $ f_1 $ et $ f_2 $ degrés de liberté tels que:
	
	où nous imposons que $ a $ et $ b $ sont positifs car sinon, comme prouvé plus haut, $ aX $ et $ bY $ ne suivraient pas des lois gamma.
	
	Rappelons-nous alors que (voir page \pageref{chi-square distribution}):
	
	Nous avons alors
	
	Cependant $ U = aX + bY $ ne suit pas une loi gamma si $ a \ neq b $ (comme nous l'avons démontré lors de notre étude de la loi Gamma à la page \pageref{gamma distribution}). En supposant cependant qu'elles soient suffisamment proches, les simulations numériques ont montré que l'on peut approcher la loi de $ U $ par la loi gamma suivante:
	
	qui a donc pour moyenne et variance  (voir page \pageref{gamma distribution}):
	
	et nous déterminerons les nombres positifs $ f $ et $ g $ pour que les moyennes et les variances espérées coïncident avec celles de $ U $.

	On a donc par l'indépendance des deux variables aléatoires:
	
	Pour que l'espérance et la variance de notre approximation correspondent à celle de $ U $, il faut avoir:
	
	En divisant la deuxième équation par la première, $ f $ disparaît et on trouve:
	
	En remplaçant dans la première on trouve ce qu'il est d'usage d'appeler le "\NewTerm{l'équation de Welch-Satterthwaite}\index{\'equation de Welch-Satterthwaite}":
	
	Ainsi explicitement:
	
	Comme on ne connaît pas en pratique les variances (et donc les vrais écarts-types), on prend en fait:
	
	Ainsi, l'expression ci-dessus arrondie à l'entier le plus proche est le nombre de degrés de liberté qui doivent réellement être pris pour le test hétéroscédastique de Student:
	
	Enfin, indiquons que dans le cas général avec $n$ termes, l'équation de Welch-Satterthwaite s'écrit (sans preuve):
	
	\begin{tcolorbox}[title=Remarque,colframe=black,arc=10pt]
	Pour le cas particulier $ n = 2 $ cette dernière relation se retrouve souvent dans la littérature sous la forme suivante (...):
	
	\end{tcolorbox}
	
	Dans le domaine de la finance lors de l'utilisation du modèle CAPM (\SeeChapter{voir section Économie page \pageref{capital asset pricing model}}) ou lors de l'utilisation de l'ANCOVA (\SeeChapter{voir section Statistiques page \pageref{ancova}}) et pas seulement... il est utile et important de comparer les pentes de la droite de régression pour savoir si elles sont significativement différentes ou non!

	Nous avons prouvé dans la section Méthodes Numériques (page \pageref{univariate linear regression gaussian model}) que pour une fonction linéaire dont les paramètres étaient estimés et qui s'exprimait sous la forme:
	
	et dont l'expression inconnue exacte est:
	
	nous avions:
	
	et que si la variance réelle est inconnue:
	
	ou parfois aussi écrit:
	
	Qui a la même forme que le test d'intervalle de confiance de Student pour la moyenne attendue!
	
	Ceci étant rappelé, l'objectif est maintenant de tester si deux pentes de deux régressions différentes sont égales (test très important pour l'étude de l'ANCOVA!), et alors écrivons que l'hypothèse nulle dans le cadre d'un test d'hypothèse évidemment comme suit:
	
	Puisque l'intervalle de confiance d'une seule pente d'une régression est similaire dans tous les aspects à un intervalle de confiance de la moyenne de Student, la différence de deux pentes est alors similaire dans tous les aspects au test d'hypothèse de Welch. On a alors puisque l'on fait l'hypothèse nulle que les deux vraies pentes ont une différence nulle (ie $ a_1 = a_2 $):
	
	avec:
	
	Ce qui est souvent noté par tradition:
	
	N'oubliez pas (cas très important dans l'application de l'ANCOVA) que si nous travaillons avec les estimateurs, alors $n_i\text{V}(x_i)$ devient $(n_i-1)\hat{\text{V}}(x_i)$.
	
	\subparagraph{Contrastes}\mbox{}\\\\
	De nombreuses méthodes de comparaison multiples utilisent l'idée de ce que nous appelons un "\NewTerm {contraste}". Qu'est-ce que c'est? Pour introduire ce concept, considérons une ANOVA à un facteur fixe où l'hypothèse nulle $H_0$ a été rejetée mais où en réalité nous ne savons pas quel niveau de la variable cause la différence qui nous pousse à rejeter l'hypothèse nulle.
	
	Considérons par exemple que nous soupçonnons le niveau parmi les niveaux $ i $, $ j $, $ k $, ... nous soupçonnons que les niveaux $ i $ et $ k $ sont différents. Par conséquent, nous aimerions tester l'hypothèse:
	
	Considérons par exemple que parmi les niveaux $ i $, $ j $, $ k $, ... nous soupçonnons que les niveaux $ i $ et $ k $ sont différents. Par conséquent, nous aimerions tester l'hypothèse:
	
	ou de manière équivalente:
	
	Si nous avions soupçonné au début de l'expérience que la moyenne des deux premiers niveaux (considérant une expérience imaginaire avec un facteur ayant $5$ niveaux fixes) ne différait pas de la moyenne des deux niveaux les plus élevés, alors l'hypothèse aurait été:
	
	ou:
	
	En général, un "\NewTerm{contraste}\index{contraste}" est une combinaison linéaire de paramètres de la forme:
	
	\begin{tcolorbox}[title=Remarque,colframe=black,arc=10pt]
	Deux contrastes $C_1,C_2$ avec coefficients $\{c_i\}$ et $\{d_i\}$, ie:
	
	sont dits "\NewTerm{contrastes orthogonaux}" si et seulement si $\sum c_id_i=0$.\\
	
	Par exemple, $C_1=\mu-2\mu_2+\mu_3$ et $C_2=\mu_1-\mu_2$ sont des contrastes orthogonaux puisque $(1)(1)+(-2)(0)+(1)(-1)=0$.
	\end{tcolorbox}
	Les deux hypothèses ci-dessus peuvent alors être exprimées en termes de contrastes:
	
	Dès lors dans le cas de notre exemple précédent:
	
	nous avons $c_1=+1$, $c_2=+1$, $c_3=-1$, $c_4=-1$, $c_5=0$.
	
	Les tests d'hypothèses impliquant des contrastes peut se faire de trois manières. Nous n'introduirons ici que celui qui donne au praticien la possibilité de construire un intervalle de confiance. Cette méthode utilise un test $T$ de Student. En effet, écrivons le contraste en termes d'estimateurs:
	
	comme sous l'hypothèse nulle on devrait avoir (ne pas l'oublier!):
	
	La variance de la somme peut s'écrire:
	
	Si l'hypothèse d'égalité de variance est satisfaite et que le plan est équilibré, nous avons:
	
	Mais avec des estimateurs, nous savons que cela s'écrit:
	
	et alors:
	
	Finalement:
	
	devrait suivre une loi normale centrée sous les hypothèses ci-dessus:
	
	Mais comme nous travaillons avec des estimateurs, nous savons que nous avons à la place:
	
	Cela est parfois noté dans le domaine de l'analyse de la variance:
	
	Par conséquent, l'intervalle de confiance de $100\cdot (1-\alpha)$ pourcent sur le contraste $\sum_{i=1}^k c_i\mu_i$ est:
	
	Il est clair que si cet intervalle de confiance inclut zéro, nous ne pourrions pas rejeter l'hypothèse nulle.
	\begin{tcolorbox}[title=Remarque,colframe=black,arc=10pt]
	Gardez à l'esprit que si vous souhaitez tester par exemple dans le cadre d'une ANOVA $H_0:\;\mu_2=(\mu_1+\mu_3)/2$, cela revient donc à écrire et tester le contraste suivant $\sum c_i\mu_i=\mu_1-2\mu_2+\mu_3=\bar{y}_1-2\bar{y}_2+\bar{y}_3$.
	\end{tcolorbox}

	\pagebreak
	\paragraph{ANOVA à deux facteurs fixes sans répétitions}\mbox{}\\\\
	Nous allons voir maintenant le concept d'interaction qui est fondamental pour bien comprendre ce qu'il y a derrière l'ANOVA à deux facteurs (fixes), aussi appelée "\NewTerm{ANOVA à deux variables catégorielles fixes}\index{ANOVA!ANOVA à deux variables catégorielles fixes}" ou "\NewTerm{plan d'expérience complétement randomisé avec blocs}\footnote{Comme déjà mentionné, un "bloc" peut être considéré comme un facteur et donc le traitement mathématique est exactement le même!}\index{ANOVA!plan d'expérience complétement randomisé}, sans et surtout plus tard avec (!) répétitions. Effectivement, c'est principalement avec l'ANOVA à deux facteurs avec répétitions – par construction mathématique - que l'on peut statistiquement (sous certaines hypothèses) étudier objectivement si deux ou plusieurs facteurs interagissent de manière significative ensemble.
	
	Il nous faut donc, avant de passer à la partie mathématique pure, introduire quelques notions!
	
	\textbf{Définitions (\#\mydef):}
	\begin{enumerate}
		\item[D1.] Nous disons qu'il y a  "\NewTerm{absence d'interaction}\index{ANOVA!absence d'interaction}" quand la moyenne des réponses d'un facteur en fonction de ses niveaux varie de la même amplitude et avec le même signe que la moyenne des réponses d'un autre facteur en fonction de ses niveaux. Nous disons alors que les courbes de réponses dans le diagramme des interactions sont parallèles.
		
		\begin{tcolorbox}[title=Remarque,colframe=black,arc=10pt]
		Le parallélisme des courbes de réponses est normal en situation d'absence d'interaction, car cela signifie que quel que soit le niveau de l'un ou l'autre des facteurs, la variation (si elle existe) de la réponse sera toujours de la même amplitude. Ce qui est caractéristique de l'indépendance (du moins localement).
		\end{tcolorbox}	
		
		\item[D2.] Nous disons que deux facteurs sont "\NewTerm{en interaction}\index{ANOVA!en interaction}" quand la moyenne des réponses d'un facteur en fonction de ses niveaux ne varie pas de la même amplitude ou/et pas avec le même signe que la moyenne des réponses d'un autre facteur en fonction de ses niveaux. Nous disons alors que les courbes de réponses dans le diagramme des interactions ne sont pas parallèles.
		
		\begin{tcolorbox}[title=Remarque,colframe=black,arc=10pt]
		L'absence d'interaction est une hypothèse très forte et une observation rare. Souvent, nous avons des interactions ou de fortes interactions.
		\end{tcolorbox}
	\end{enumerate}
	Pour comprendre le concept, nous aurons recours à de petits exemples sans répétition qui permettront de se faire une idée qualitative du phénomène mais en aucun cas une approche scientifique de l'interaction.

	À chaque fois nous visualiserons les situations au moyen de deux types de représentations: un graphique illustrant les effets principaux d'une part et un diagramme des interactions d'autre part.
	
	\pagebreak
	Considérons le petit tableau suivant avec deux facteurs à deux niveaux ("variables explicatives") comportant donc 4 cellules ("variables d'intérêt"):
	\begin{table}[H]
	\begin{center}
		\begin{tabular}{|c|c|c|}
			\hline
			{} & \multicolumn{2}{|c|}{\cellcolor{black!30}\textbf{Facteur $2$}} \\
			\hline
			\cellcolor{black!30}\textbf{Facteur $1$} & \cellcolor{black!30}Niveau $1$ & \cellcolor{black!30}Niveau $2$ \\
			\hline
			\cellcolor{black!30}Niveau $1$ & $3$ & $3$ \\
			\hline
			\cellcolor{black!30}Niveau $2$ & $3$ & $3$ \\
			\hline
		\end{tabular}
		\caption[]{Premier exemple d'une petite ANOVA à deux facteurs sans répétitions}
	\end{center}
	\end{table}
	Nous aurons comme représentations avec un logiciel comme Minitab:
	\begin{figure}[H]
		\centering
		\includegraphics{img/arithmetics/anova_principal_effects_no_interactions.jpg}
		\caption{Graphique des effets principaux avec Minitab 15}
	\end{figure}
	Nous voyons bien qu'aucun facteur n'a un effet principal sur quoi que ce soit. Ce qui est relativement intuitif étant donné le contenu de tableau précédent.

	Le diagramme des interactions (appelé souvent "\NewTerm{profileur}\index{diagramme profileur}" dans l'industrie) donne lui:
	\begin{figure}[H]
		\centering
		\includegraphics{img/arithmetics/anova_interactions_effects_no_interactions.jpg}
		\caption{Diagramme des interactions sans interactions... avec Minitab 15}
	\end{figure}
	où nous pouvons constater que les facteurs n'interagissent pas entre eux (ou se neutralisent c'est selon...). Nous disons alors qu'il n'y a "\NewTerm{(a priori) aucun effet ni aucune interaction (localement)}\index{ANOVA!absence d'interactions locales}". Au fait dans certaines expériences, l'absence d'interaction est une hypothèse très forte et donc souvent rare. Raison pour laquelle il faut faire attention aux mots choisis lors de l'interprétation des graphiques d'interaction (car ne pas passer par les calculs purs est délicat pour cette étape voire non scientifique!).
	
	Maintenant considérons le tableau suivant:
	\begin{table}[H]
	\begin{center}
		\begin{tabular}{|c|c|c|}
			\hline
			{} & \multicolumn{2}{|c|}{\cellcolor{black!30}\textbf{Facteur $2$}} \\
			\hline
			\cellcolor{black!30}\textbf{Facteur $1$} & \cellcolor{black!30}Niveau $1$ & \cellcolor{black!30}Niveau $2$ \\
			\hline
			\cellcolor{black!30}Niveau $1$ & $2$ & $2$ \\
			\hline
			\cellcolor{black!30}Niveau $2$ & $4$ & $4$ \\
			\hline
		\end{tabular}
		\caption[]{Second exemple d'une ANOVA à deux facteurs sans répétitions}
	\end{center}
	\end{table}
	Il nous paraît clair que le Facteur 1 à travers la prise en compte de son niveau semble avoir une influence sur la réponse. Mais voyons les différentes représentations:
	\begin{figure}[H]
		\centering
		\includegraphics{img/arithmetics/anova_principal_effects_with_interactions.jpg}
	\end{figure}
	\begin{figure}[H]
		\centering
		\includegraphics{img/arithmetics/anova_interactions_effects_with_interactions.jpg}
		\caption{Graphique des effets principaux et diagramme des interactions avec Minitab 15}
	\end{figure}
	Détaillons plus le premier graphique comme l'a proposé un lecteur:

	Ce graphique comporte $2$ parties: celle de gauche analyse les effets du facteur 1 à travers ses $2$ niveaux ; celle de droite en fait de même pour le facteur $2$.

	Examinons de plus près la partie de gauche:

	Nous y voyons $2$ points reliés par un segment de droite. Ici le premier point, celui pour le niveau 1, est situé à l'ordonnée $2$ alors que le deuxième point, celui pour le niveau 2, est situé à l'ordonnée $4$. Rappelons-nous maintenant que chaque point représente une moyenne. Ainsi l'ordonnée du premier point est bien située à la moyenne de $(2 + 2) / 2 = 2$.
	
	Ceci étant dit et en espérant que cela a aidé à une meilleure compréhension, revenons à nos moutons…

	Il apparaît assez clairement dans le graphique du dessus que seul le niveau du Facteur $1$ influence la réponse, alors que le Facteur $2$ n'influence en rien la réponse. Nous disons alors qu'il y a effet principal (localement) du Facteur $1$.
	
	Sur le diagramme des interactions, nous avons la même information, mais sous une forme différente. Nous voyons que quel que soit le niveau du Facteur 2, les réponses sont horizontales et donc que celui-ci n'influence en rien les résultats. Nous sommes alors dans une situation où "\NewTerm{(a priori) l'effet principal est (localement) le Facteur $1$ et en absence d'interactions entre les facteurs}".
	
	Voyons maintenant le tableau suivant:
	\begin{table}[H]
	\begin{center}
		\begin{tabular}{|c|c|c|}
			\hline
			{} & \multicolumn{2}{|c|}{\cellcolor{black!30}\textbf{Facteur $2$}} \\
			\hline
			\cellcolor{black!30}\textbf{Facteur $1$} & \cellcolor{black!30}Niveau $1$ & \cellcolor{black!30}Niveau $2$ \\
			\hline
			\cellcolor{black!30}Niveau $1$ & $4$ & $2$ \\
			\hline
			\cellcolor{black!30}Niveau $2$ & $4$ & $2$ \\
			\hline
		\end{tabular}
		\caption[]{Troisième exemple d'une petite ANOVA à deux facteurs sans répétitions}
	\end{center}
	\end{table}
	Nous pouvons cette fois observer que le Facteur $2$ a une influence mais pas le Facteur $1$. Mais voyons aussi cela avec nos $2$ types de représentations:
	\begin{figure}[H]
		\centering
		\includegraphics{img/arithmetics/anova_principal_effects_with_interactions_second_example.jpg}
	\end{figure}
	\begin{figure}[H]
		\centering
		\includegraphics{img/arithmetics/anova_interactions_effects_with_interactions_second_example.jpg}
		\caption{Graphique des effets principaux et diagramme des interactions avec Minitab 15}
	\end{figure}
	Nous observons bien sur le graphique que le Facteur $1$ n'a aucune influence. Sur le diagramme du dessous c'est moins évident mais la superposition des deux droites montre que le Facteur $1$ n'a pas d'influence. Nous disons alors qu'il y a "\NewTerm{(a priori) effet principal (localement) du Facteur 2 et absence d'interactions entre les facteurs}".
	
	Considérons maintenant le tableau suivant:
	\begin{table}[H]
	\begin{center}
		\begin{tabular}{|c|c|c|}
			\hline
			{} & \multicolumn{2}{|c|}{\cellcolor{black!30}\textbf{Facteur $2$}} \\
			\hline
			\cellcolor{black!30}\textbf{Facteur $1$} & \cellcolor{black!30}Niveau $1$ & \cellcolor{black!30}Niveau $2$ \\
			\hline
			\cellcolor{black!30}Niveau $1$ & $3$ & $1$ \\
			\hline
			\cellcolor{black!30}Niveau $2$ & $5$ & $3$ \\
			\hline
		\end{tabular}
		\caption[]{Quatrième exemple d'une petite ANOVA à deux facteurs sans répétitions}
	\end{center}
	\end{table}
	Nous voyons que les deux facteurs ont une influence sur la réponse. Ce que montrent bien les deux représentations ci-dessous:
	\begin{figure}[H]
		\centering
		\includegraphics{img/arithmetics/anova_principal_effects_with_interactions_third_example.jpg}
	\end{figure}
	\begin{figure}[H]
		\centering
		\includegraphics{img/arithmetics/anova_interactions_effects_with_interactions_third_example.jpg}
		\caption[]{Graphique des effets principaux et diagramme des interactions avec Minitab 15}
	\end{figure}
	Nous observons bien sur le graphique du dessus que le Facteur 1 a une influence sur la réponse et qu'il en est de même du Facteur 2 (et en plus de la même amplitude quel que soit le sens!). Sur le graphique du dessous c'est moins évident mais la même conclusion est valable. Nous disons alors que "\NewTerm{(a priori) les deux facteurs sont (localement) significatifs et sans interactions}\index{ANOVA!deux facteurs localement significatifs}".
	
	\pagebreak
	Passons au tableau suivant:
	\begin{table}[H]
	\begin{center}
		\begin{tabular}{|c|c|c|}
			\hline
			{} & \multicolumn{2}{|c|}{\cellcolor{black!30}\textbf{Facteur $2$}} \\
			\hline
			\cellcolor{black!30}\textbf{Facteur $1$} & \cellcolor{black!30}Niveau $1$ & \cellcolor{black!30}Niveau $2$ \\
			\hline
			\cellcolor{black!30}Niveau $1$ & $2$ & $4$ \\
			\hline
			\cellcolor{black!30}Niveau $2$ & $4$ & $2$ \\
			\hline
		\end{tabular}
		\caption[]{Cinquième exemple d'une petite ANOVA à deux facteurs sans répétition}
	\end{center}
	\end{table}
	qui sous cette forme n'est pas trivial à interpréter. Mais avec les représentations nous avons tout de suite des informations plus pertinentes:
	\begin{figure}[H]
		\centering
		\includegraphics{img/arithmetics/anova_principal_effects_with_interactions_fourth_example.jpg}
	\end{figure}
	\begin{figure}[H]
		\centering
		\includegraphics{img/arithmetics/anova_interactions_effects_with_interactions_fourth_example.jpg}
		\caption[]{Graphique des effets principaux et diagramme des interactions avec Minitab 15}
	\end{figure}
	Nous observons bien sur le graphique ci-dessus qu'aucun des facteurs n'a d'influence sur la réponse a priori (même graphique qu'au tout début avec la même moyenne). Le diagramme du dessous nous donne une information complémentaire par contre (!!!): Les facteurs ont une influence croisée et comme cette influence croisée est de même amplitude, les effets s'annulent. Nous disons alors que les "\NewTerm{(a priori) les deux facteurs sont (localement) en interaction F$1$ * F$2$}\index{ANOVA!deux facteurs localement en interaction}".
	
	Considérons maintenant le tableau suivant:
	\begin{table}[H]
	\begin{center}
		\begin{tabular}{|c|c|c|}
			\hline
			{} & \multicolumn{2}{|c|}{\cellcolor{black!30}\textbf{Facteur $2$}} \\
			\hline
			\cellcolor{black!30}\textbf{Facteur $1$} & \cellcolor{black!30}Niveau $1$ & \cellcolor{black!30}Niveau $2$ \\
			\hline
			\cellcolor{black!30}Niveau $1$ & $1$ & $3$ \\
			\hline
			\cellcolor{black!30}Niveau $2$ & $5$ & $3$ \\
			\hline
		\end{tabular}
		\caption[]{Sixième exemple d'une petite ANOVA à deux facteurs sans répétitions}
	\end{center}
	\end{table}
	qui, sous cette forme, n'est pas toujours évident à interpréter. Mais avec les schémas, nous avons immédiatement une information plus pertinente:
	\begin{figure}[H]
		\centering
		\includegraphics{img/arithmetics/anova_principal_effects_with_interactions_fifth_example.jpg}
	\end{figure}
	\begin{figure}[H]
		\centering
		\includegraphics{img/arithmetics/anova_interactions_effects_with_interactions_fifth_example.jpg}
		\caption[]{Graphique des effets principaux et diagramme des interactions avec Minitab 15}
	\end{figure}
	Nous observons bien sur le graphique du dessus que le Facteur $1$ semble avoir une influence et que le Facteur $2$ non (en moyenne!). Le diagramme des interactions du dessous nous donne, lui aussi, encore une fois, une information complémentaire (!!!): C'est que les facteurs sont en interaction. Nous disons alors que nous avons  "\NewTerm{(a priori) deux facteurs (localement) en interaction F$1$ * F$2$ où l'influence du Facteur $1$ est significative}".
	
	Considérons maintenant le tableau suivant:
	\begin{table}[H]
	\begin{center}
		\begin{tabular}{|c|c|c|}
			\hline
			{} & \multicolumn{2}{|c|}{\cellcolor{black!30}\textbf{Facteur $2$}} \\
			\hline
			\cellcolor{black!30}\textbf{Facteur $1$} & \cellcolor{black!30}Level $1$ & \cellcolor{black!30}Niveau $2$ \\
			\hline
			\cellcolor{black!30}Niveau $1$ & $3$ & $3$ \\
			\hline
			\cellcolor{black!30}Niveau $2$ & $5$ & $1$ \\
			\hline
		\end{tabular}
		\caption[]{Septième exemple d'une petite ANOVA à deux facteurs sans répétition}
	\end{center}
	\end{table}
	Nous voyons que les deux facteurs ont une influence sur la réponse. Ce que montrent bien les deux représentations ci-dessous:
	\begin{figure}[H]
		\centering
		\includegraphics{img/arithmetics/anova_principal_effects_with_interactions_sixth_example.jpg}
	\end{figure}
	\begin{figure}[H]
		\centering
		\includegraphics{img/arithmetics/anova_interactions_effects_with_interactions_sixth_example.jpg}
		\caption[]{Diagramme des effets principaux et interactions avec les interactions dans Minitab 15}
	\end{figure}
	Nous disons ici que nous avons "\NewTerm{(a priori) les deux facteurs (localement) en interaction F$1$*F$2$ où l'influence du Facteur $2$ est significative}".
	
	Et enfin un dernier tableau:
	\begin{table}[H]
	\begin{center}
		\begin{tabular}{|c|c|c|}
			\hline
			{} & \multicolumn{2}{|c|}{\cellcolor{black!30}\textbf{Facteur $2$}} \\
			\hline
			\cellcolor{black!30}\textbf{Facteur $1$} & \cellcolor{black!30}Level $1$ & \cellcolor{black!30}Niveau $2$ \\
			\hline
			\cellcolor{black!30}Niveau $1$ & $1$ & $5$ \\
			\hline
			\cellcolor{black!30}Niveau $2$ & $5$ & $1$ \\
			\hline
		\end{tabular}
		\caption[]{Huitième exemple d'une petite ANOVA à deux facteurs sans répétitions}
	\end{center}
	\end{table}
	qui nous donne les deux représentations:
	\begin{figure}[H]
		\centering
		\includegraphics{img/arithmetics/anova_principal_effects_with_interactions_seventh_example.jpg}
	\end{figure}
	\begin{figure}[H]
		\centering
		\includegraphics{img/arithmetics/anova_interactions_effects_with_interactions_seventh_example.jpg}
		\caption[]{Graphique des effets principaux et diagramme des interactions avec Minitab 15}
	\end{figure}
	Nous disons ici que nous avons "\NewTerm{(a priori) les deux facteurs (localement) en interaction F$1$*F$2$ où l'influence des deux facteurs est significative}".
	
	
	\begin{tcolorbox}[title=Remarques,colframe=black,arc=10pt]
	R1. Une croyance (communément répandue) de personnes qui manquent d'expérience dans les laboratoires consiste à penser que pour qu'une interaction soit significative il est nécessaire que les facteurs qui la composent le soient également.\\
	
	R2. Si une interaction est significative, les principaux effets sont objectivement rarement interprétables!
	\end{tcolorbox}
	Après tous ces tableaux, passons à partie mathématique:

	Nous avons vu précédemment comment effectuer une analyse de la variance à un facteur. Pour rappel, cela consiste donc à faire un test d'égalité des espérances pour $k$ échantillons indépendants de n variables aléatoires chacun (dans le cas où tous les échantillons ont donc le même nombre de mesures). Chaque échantillon étant considéré comme une expérience sur un sujet différent ou identique considéré alors comme un facteur variable indépendant!

		Cependant il arrive dans la réalité que pour chaque échantillon on fasse varier un deuxième paramètre, considéré alors comme un deuxième facteur variable. Nous parlons alors bien évidemment d'analyse de la variance à deux facteurs. De plus, nous allons considérer dans un premier temps pour simplifier les calculs que les variables aléatoires sont indépendantes! Donc un facteur n'a pas d'influence sur l'autre!!! En d'autres, termes il n'y a pas d'interaction entre les facteurs. Nous parlons alors d'une "\NewTerm{ANOVA à deux facteurs sans interactions}\index{ANOVA à deux facteurs sans interactions}".

	Afin de déterminer la formulation du test à effectuer, rappelons que pour l'analyse de la variance à un facteur, nous avions décomposé la variance totale en la somme de la variance des moyennes (interclasses) et de la variance résiduelle (intra-classes) telle que:
	
	en explicitant le fait que nous comparions les échantillons $i=1...k$:
	
	ce qui nous avait donné au final:
	
	Pour l'ANOVA à deux facteurs nous partirons du tableau suivant ("Éch." est l'abréviation de "Échantillon"):
	\begin{table}[H]
		\begin{center}
		\begin{tabular}{ |c|c|c|c|c|c| }
			\hline
			\multicolumn{1}{|c|}{\phantom} & \multicolumn{4}{ |c| }{Facteur $A$} & \\
			\hline
			Facteur $B$ & Éch. $1$ & Éch. $2$ & Éch. ...$j$ & Éch. $r$ & \\ \hline
			Éch. $1$ &  $x_{11}$ & $x_{12}$ & $\ldots$ & $x_{1r}$ & Moyenne $\bar{x}_1$ \\
			Éch. $2$ &  $x_{21}$ & $x_{22}$ & $\ldots$ & $x_{2r}$ & Moyenne $\bar{x}_2$\\
			 Éch. $i$ & $\ldots$ & $\ldots$ & $x_{ij}$ & $\ldots$ & Moyenne $\bar{x}_i$ \\
			 Éch. $k$ & $x_{n1}$ & $x_{n2}$ & $\ldots$ & $x_{kr}$ & Moyenne $\bar{x}_k$\\ \hhline{|=|=|=|=|=|=|}
			 \phantom & Moyenne: $\bar{x}_1$ &  Moyenne: $\bar{x}_2$  &  Moyenne: $\bar{x}_j$ & Moyenne: $\bar{x}_r$ & $\bar{\bar{x}}$ \\
			\hline
		\end{tabular}
		\end{center}
		\caption{Structure typique dite "croisée" d'une analyse de la variance à 2 facteurs sans répétition}
	\end{table}
	pour lequel dans un laboratoire, le facteur maintenu fixe pendant qu'on fera varier l'autre sera appelé le "\NewTerm{facteur bloc}\index{ANOVA!facteur bloc}" et l'autre sera appelé le "\NewTerm{facteur de traitement}\index{ANOVA!facteur de traitement}" et dans la pratique on fera en sorte que ce dernier ne soit pas effectué toujours dans le même ordre afin d'éliminer des éventuels effets d'inertie lors du passage d'un traitement à l'autre (les américains désignent les ANOVA à deux facteurs contrôlés sans interactions sous les termes: "\NewTerm{randomized block design}\index{randomized block design}" (GRBD)).
	
	Pour la suite, toute l'astuce consiste à décomposer la variance totale en comparant l'espérance des lignes (observations) indexées cette fois-ci avec $i=1...k$ et des colonnes (échantillons) indexées avec $j=1\ldots r$ par rapport à la moyenne totale telle que:
	
	Or, nous avons dans un premier temps:
	
	Donc il reste:
	
	Mais nous avons aussi:
	
	Pour la suite, indiquons d'abord que relativement à notre tableau, nous avons:
	
	Il s'ensuit alors que:
	
	et il vient alors immédiatement que nous avons de même:
	
	Donc il reste au final:
	
	ce que nous noterons dans ce livre de la manière condensée suivante:
	
	où $Q_A,Q_B$ sont bien évidemment associés aux effets principaux (comparaison des moyennes marginales avec la moyenne totale).
	
	Donc en comparaison à l'ANOVA à un facteur nous avons un terme supplémentaire pour la variance totale.

	Dans l'ordre il est évident que la première somme des écarts par rapport au premier facteur colonne:
	
	aura au même titre que l'ANOVA à un facteur  $k-1$ degrés de liberté. C'est-à-dire que sous les mêmes hypothèses que l'ANOVA à un facteur fixe:
	
	La deuxième somme des écarts par rapport au deuxième facteur ligne:
	
	est nouvelle mais cependant on démontre de manière parfaitement identique au premier qu'elle aura $k-1$ degrés de liberté. C'est-à-dire que sous les mêmes hypothèses que l'ANOVA à un facteur fixe:
	
	Pour la troisième somme qui suit obligatoirement aussi une loi du Khi-deux (étant donné que la variance totale suit une loi du Khi-deux et que les deux premiers termes de la somme aussi!):
	
	c'est un peu plus délicat... mais il y a une astuce à la sauce physicienne...! Nous savons de par notre étude de l'ANOVA à un facteur que la somme des degrés de liberté de chaque terme doit être égale au nombre total de degrés de libertés. En d'autres termes, nous devons avoir pour l'ANOVA à deux facteurs:
	
	Donc il manque bien évidemment:
	
	Ainsi:
	
	Nous avons alors le tableau suivant:
	
	Enfin, le reste est exactement le même que pour l'ANOVA à un facteur simplement que nous avons deux tests à effecteur cette fois-ci qui sont:
	
	Le choix ci-dessus semble intuitivement judicieux.
	
	Tous les calculs que nous avons faits précédemment sont très souvent représentés dans les logiciels sous la forme d'une table standardisée dont voici la forme et le contenu (c'est ainsi que le présente Microsoft Excel 11.8346 ou Minitab15.1.1 par exemple):
	\begin{table}[H]
		\resizebox{\textwidth}{!}{\begin{tabular}{lcccc}\hline
		\textbf{Somme des carrés (SCE)} & $\chi^2$ \textbf{ddl} & \textbf{Moyenne des carrés} & $F$ & \textbf{Valeur critique} $F$\\ \hline
		$Q_A=k\displaystyle\sum_{j}\left(\bar{x}_{j}-\bar{\bar{x}}\right)^2$ & $k-1$ & $\text{MCk}A=\dfrac{Q_A}{k-1}$ &
		$\dfrac{\text{MCk}A}{\text{MCE}}$ & $P(F> F_{k-1,(k-1)(r-1)})$ \\
		$Q_B=r\displaystyle\sum_{i}\left(\bar{x}_{i}-\bar{\bar{x}}\right)^2$ & $r-1$ & $\text{MCk}B=\dfrac{Q_B}{r-1}$ &
		$\dfrac{\text{MCk}B}{\text{MCE}}$ & $P(F> F_{r-1,(k-1)(r-1)})$ \\
		$Q_R=\sum_{ij}\left(x_{ij}-\bar{x}_i-\bar{x}_j+\bar{\bar{x}}\right)^2$ & $(k-1)(r-1)$ & $ \text{MCE}=\dfrac{Q_R}{(k-1)(n-1)}$  & & \\
		$Q_T=\displaystyle\sum_{ij}\left(x_{ij}-\bar{\bar{x}}\right)^2$ & $N-1$ & $\text{MCT}=\dfrac{Q_T}{N-1}$ & &\\ \hline
		\end{tabular}}
		\caption{Terminologie et paramètres traditionnels d'un Tableau ANOVA (TAV) à deux facteurs sans répétitions}
	\end{table}
	et la condition d'acception de l'hypothèse d'égalité des moyennes pour chaque facteur est la même que pour l'ANOVA à un facteur (voir le serveur d'exercice pour un exemple pratique et détaillé avec Microsoft Excel 11.8346).

	Nous avons donc deux tests de Fisher permettant chacun de savoir si le facteur $A$ (respectivement $B$) ont une influence significative ou pas sur les mesures.

	Évidemment, dans les développements ci-dessus, les facteurs $A$ et $B$ sont interchangeables dans les développements par symétrie!
	
	\begin{tcolorbox}[title=Remarque,colframe=black,arc=10pt]
	Le modèle d'effet de facteur correspondant est (notez que le modèle est supposé être purement additif, c'est-à-dire sans interactions!):
	
	avec $i=1\ldots a$, $j=1\ldots b$ où $\mu$ est la moyenne globale, $\tau_i$ est l'effet du $i$-ème traitement $A$, $\beta_j$ est l'effet du $j$-ème niveau du traitement $B$ (effet principal) et $\varepsilon_{ij}$ est l'erreur aléatoire habituelle qui suit une loi $\mathcal{N}(0,\sigma_\varepsilon^2)$.\\
	
	Les suppositions sur les termes d'effets sont $\sum_{i=1}^a\tau_i=0$, $\sum_{j=1}^b\beta_j=0$.\\
	
	L'hypothèse d'intérêt est $H_0:\;  \tau_i=0$ et $H_0:\; \beta_j=0$.\\
	
	On voit que les estimateurs du maximum de vraisemblance des paramètres du modèle sont $\hat{\mu}=\bar{\bar{x}}$ et $\hat{\tau}_i=\bar{x}_i-\bar{\bar{x}}$ et $\hat{\beta}_j=\bar{x}_j-\bar{\bar{x}}$. Dès lors:
	
	\end{tcolorbox}
	
	\pagebreak
	\paragraph{ANOVA à deux facteurs fixes avec répétitions}\mbox{}\\\\
	Jusqu'à présent nous avons examiné des ANOVA sur des expériences à un ou deux facteurs fixes (autrement dit: une ou deux variables catégorielles). Dans le cas à deux facteurs, nous avons considéré que pour chaque combinaison de facteurs nous n'avions qu'une seule mesure (cellule). Or, il peut arriver (et c'est préférable) que nous ayons plusieurs mesures pour une combinaison! Nous qualifions ce type d'étude de "\NewTerm{plan expérimental à mesures répétées}\index{plan expérimental à mesures répétées}" et les résultats seront traités avec une analyse de la variance à deux facteurs fixes à mesures répétées (MR-ANOVA) et avec interactions! Il s'agit d'un outil extrêmement important puisqu'il permet de valider des études menées par plusieurs laboratoires (ou employés) indépendants et il est également associé à de nombreux autres outils statistiques comme celui de l'étude de la reproductibilité et de la répétabilité (Étude R\&R) pour ne citer que le plus connu dans le domaine industriel.
	
	Il faut comprendre qu'il est obligatoire dans le domaine de la statistique d'associer les interactions entre facteurs systématiquement lorsque nous avons affaire à une expérience à mesures répétées. Ceci pour la simple raison que le terme mathématique d'interaction n'apparaît que dans cette situation.
	
	Ainsi, il peut être intuitif (avant même de le démontrer) qu'une ANOVA à deux facteurs (fixes) à mesures répétées (les américains désignent les ANOVA à deux facteurs contrôlés avec interactions sour les termes: "generalized randomized block design" (GRBD)) contient une interaction double, et deux effets principaux. Une ANOVA à trois facteurs (fixes) et à mesures répétées aura in extenso une interaction triple, trois interactions doubles et 3 effets principaux. Et ainsi de suite...
	
	Avant de commencer, nous allons considérer le tableau de mesures suivant où l'abréviation "Éch." fait référence au mot "échantillon":
	\begin{table}[H]
		\begin{center}
		\begin{tabular}{ |l|c|c|c|c|c| }
			\hline
			\multicolumn{1}{|c|}{\phantom} & \multicolumn{4}{ |c| }{Facteur $A$} & \\
			\hline
			Facteur $B$ & Éch. $1$ & Éch. $2$ & Éch. ...$j$ & Éch. $r$ & MOYENNE \\ \hline
			Éch. $1$ &  $x_{111}$ & $x_{112}$ & $\ldots$ & $x_{11r}$ & \\
			$\;$ Réplication $2$ &  $x_{211}$ & $x_{212}$ & $\ldots$ & $x_{21r}$ & \\			
			$\;$ Réplication $m$ &  $\ldots$ & $\ldots$ & $\ldots$ & $\ldots$ & \\			
			$\;$ Réplication $n$ &  $x_{n11}$ & $x_{n12}$ & $\ldots$ & $x_{n1r}$ & \\ \hline
			Moyenne Éch. $1$ &  $\bar{x}_{11}$ & $\bar{x}_{12}$ & $\ldots$ & $\bar{x}_{13}$ & $\bar{x}_{1.}$ \\ \hline		
			Moyenne Éch. $2$ &  $x_{121}$ & $x_{122}$ & $\ldots$ & $x_{12r}$ & \\
			$\;$ Réplication $2$ & $x_{221}$ & $x_{222}$ & $\ldots$ & $x_{22r}$ & \\			
			$\;$ Réplication $m$ & $\ldots$ & $\ldots$ & $\ldots$ & $\ldots$ & \\			
			$\;$ Réplication $n$ & $x_{n21}$ & $x_{n22}$ & $\ldots$ & $x_{n2r}$ & \\ \hline
			Moyenne Éch. $2$ & $\bar{x}_{21}$ & $\bar{x}_{22}$ & $\ldots$ & $\bar{x}_{2r}$ & $\bar{x}_{2.}$ \\ \hline			
			Éch. $i$ &  $\ldots$ & $\ldots$ & $x_{1ij}$ & $\ldots$ & \\
			$\;$ Réplication $2$ & $\ldots$ & $\ldots$ & $\ldots$ & $\ldots$ & \\			
			$\;$ Réplication $m$ & $\ldots$ & $\ldots$ & $\ldots$ & $\ldots$ & \\			
			$\;$ Réplication $n$ & $\ldots$ & $\ldots$ & $x_{nij}$ & $x_{nir}$ & \\ \hline	
			Moyenne Éch. $i$ & $\ldots$ & $\ldots$ & $\bar{x}_{ij}$ & $\ldots$ & $\bar{x}_{i.}$ \\ \hline		
			Éch. $k$ & $x_{2k1}$ & $x_{2k2}$ & $\ldots$ & $x_{2kr}$ & \\
			$\;$ Réplication $2$ &  $x_{21}$ & $x_{22}$ & $\ldots$ & $x_{2r}$ & \\			
			$\;$ Réplication $m$ &  $\ldots$ & $\ldots$ & $\ldots$ & $\ldots$ & \\			
			$\;$ Réplication $n$ &  $x_{nk1}$ & $x_{nk2}$ & $\ldots$ & $x_{nkr}$ & \\	\hline
			Moyenne Éch. $k$ &  $\bar{x}_{k1}$ & $\bar{x}_{k2}$ & $\ldots$ & $\bar{x}_{kr}$ & $\bar{x}_{k.}$ \\ \hhline{|=|=|=|=|=|=|}
			 MOYENNE & $\bar{x}_{.1}$ & $\bar{x}_{.2}$ & $\bar{x}_{.j}$ & $\bar{x}_{.r}$ & $\bar{\bar{x}}_{..}$ \\
			\hline
		\end{tabular}
		\end{center}
		\caption{Structure typique croisée d'une analyse de la variance à 2 facteurs avec répétition}
	\end{table}
	avec les propriétés habituelles des moyennes (pour rappel):
	
	Et rappelons que pour l'ANOVA à deux facteurs fixes sans réplications (et donc sans interactions), toute l'astuce avait consisté à décomposer la variance totale en comparant la moyenne des lignes indexées avec $i=1...k$ et des colonnes indexées avec $j=1...r$ par rapport à la moyenne totale.
	
	L'idée va maintenant être à peu près la même à la différence que nous allons comparer l'espérance des lignes indexées avec $i=1...k$ et des colonnes indexées avec $j=1...r$ non seulement par rapport à la moyenne totale mais aussi à celle de chaque ligne et de chaque colonne.
	
	Pour cela nous repartons de ce que nous avions obtenu pour l'ANOVA à deux facteurs sans réplication:
	
	mais dont la notation sera juste adaptée au contexte:
	
	Il est évident qu'avec cette écriture l'ANOVA à deux facteurs sans réplication deviendrait:
	
	Mais dans le cas présent, il nous faut rajouter une sommation pour les réplications et adapter la notation pour les mesures. Donc, sans refaire tous les développements (c'est un peu culotté mais bon...), nous obtenons déjà directement:
	
	où dans l'ordre, $m$ est la réplication de l'échantillon $i$ du facteur $A$ et de l'échantillon $j$ du facteur $B$.

	Il vient alors bien évidemment les variances interclasses pour les facteurs $A$ et $B$ qui sont immédiates:
	
	où $Q_A,Q_B$ sont bien évidemment encore une fois associées aux effets principaux (comparaisons des moyennes marginales avec la moyenne totale).
	
	Maintenant, nous allons  jouer un peu en introduisant sous la somme, en plus et en moins, dans le dernier terme:
	
	la moyenne des réplications:
	
	que nous retrouverons in fine dans la somme des carrés totale:
	
	Bien entendu, nous reconnaissons assez vite la variance intra-classes (appelée aussi souvent "\NewTerm{erreur résiduelle}\index{erreur résiduelle}" ou simplement dans le cas particulier de l'ANOVA à deux facteurs avec répétition "\NewTerm{erreur de répétabilité}\index{erreur de répétabilité}"):
	
	et le terme que nous pouvons interpréter (par comparaison avec l'ANOVA à deux facteurs sans répétitions) comme étant la variance d'interaction:
	
	Mais si notre hypothèse est vérifiée (c'est-à-dire que l'ANOVA est balancée),  le terme:
	
	doit s'annuler. Vérifions cela:
	
	et donc pour $i$ et $j$ fixés il vient:
	
	Et donc la sommation sur tous les $i$ et $j$ sera aussi nulle par extension. Ceux qui ont un doute quant à l'annulation des deux termes du développement ci-dessus, pourront peut-être se rassurer en faisant une application numérique.
	
	Donc au final:
	
	où pour rappel, $n$ est donc le nombre de réplications, $r$ le nombre d'échantillons du facteur $A$ et $k$ le nombre d'échantillons du facteur B (ces deux derniers paramètres sont souvent confondus par ceux qui font les calculs à la main). Résultat qui est parfois noté sous la forme suivante dans la littérature:
	
	Donc en comparaison à l'ANOVA à deux facteurs sans réplications, nous avons un terme supplémentaire pour la variance totale.
	
	Dans l'ordre il est évident que la première somme des écarts par rapport au premier facteur colonne:
	
	aura au même titre que l'ANOVA à un facteur et l'ANOVA à deux facteurs sans répétition $k-1$ degrés de liberté. C'est-à-dire que sous les mêmes hypothèses que ces deux ANOVA, nous avons:
	
	La deuxième somme des écarts par rapport au deuxième facteur ligne:
	
	aura sous les mêmes hypothèses la propriété:
	
	Grâce au raisonnement effectué à l'aide de l'ANOVA à deux facteurs sans répétition, nous savons que pour le terme d'interaction:
	
	Nous avons:
	
	Il reste à déterminer le nombre de degrés de liberté du dernier terme:
	
	Pour ce faire, nous procédons de la même manière qu'avec l'ANOVA à deux facteurs sans répétitions. Nous savons de par notre étude de l'ANOVA à un facteur que la somme des degrés de liberté de chaque terme doit être égale au nombre total de degrés de liberté. En d'autres termes, nous devons avoir pour l'ANOVA à deux facteurs:
	
	Donc il manque bien évidemment:
	
	Ainsi:
	
	Nous avons alors le tableau suivant:
	
	Enfin, le reste est exactement le même que pour l'ANOVA à deux facteurs sans réplication simplement que nous avons trois tests à effecteur cette fois-ci qui sont:
	
	Là encore le choix des ratios est relativement intuitif!
	
	Tous les calculs que nous avons faits précédemment sont très souvent représentés dans les logiciels sous la forme d'une table standardisée dont voici la forme et le contenu (c'est ainsi que le présente Microsoft Excel 11.8346 ou Minitab 15.1.1 par exemple):
	\begin{table}[H]
	\centering
	\resizebox{\textwidth}{!}{\begin{tabular}{lcccc}\hline
	\textbf{Somme des carrés (SCE)} & $\chi^2$ \textbf{ddl} & \textbf{Moyenne des carrés} & $F$ & \textbf{Valeur critique} $F$\\ \hline
	$Q_A=nk\displaystyle\sum_{j}\left(\bar{x}_{.j}-\bar{\bar{x}}_{..}\right)^2$ & $k-1$ & $\text{MCk}A=\dfrac{Q_A}{k-1}$ &
	$\dfrac{\text{MCk}A}{\text{MCE}}$ & $P(F> F_{k-1,Nk-r})$ \\
	$Q_B=nr\displaystyle\sum_{i}\left(\bar{x}_{i.}-\bar{\bar{x}}_{..}\right)^2$ & $r-1$ & $\text{MCk}B=\dfrac{Q_B}{r-1}$ &
	$\dfrac{\text{MCk}B}{\text{MCE}}$ & $P(F> F_{r-1,N-kr})$ \\
	$Q_{A\times B}=\displaystyle\sum_{mij}\left(\bar{x}_{mij}-\bar{x}_{i.}-\bar{x}_{.j}+\bar{\bar{x}}_{..}\right)^2$ & $(k-1)(r-1)$ & $ \text{MCk}AB=\dfrac{Q_{A\times B}}{(k-1)(n-1)}$  & $\dfrac{\text{MCk}AB}{\text{MCE}}$ & $P(F> F_{(k-1)(n-1),N-kr})$\\
	$Q_R=\displaystyle\sum_{mij}\left(\bar{x}_{mij}-\bar{x}_{ij}\right)^2$ & $N-kr$ & $ \text{MCE}=\dfrac{Q_R}{N-kr}$  &  & \\
	$Q_T=\displaystyle\sum_{mij}\left(x_{mij}-\bar{\bar{x}}_{..}\right)^2$ & $N-1$ &  $\text{MCT}=\dfrac{Q_T}{N-1}$  & &\\ \hline
	\end{tabular}}
	\caption{Terminologie et paramètres traditionnels d'un Tableau ANOVA (TAV) à deux facteurs avec répétition}
	\end{table}
	et la condition d'acception de l'hypothèse d'égalité des moyennes pour chaque facteur est la même que pour l'ANOVA à un facteur (voir le serveur d'exercice pour un exemple pratique et détaillé avec Microsoft Excel 11.8346).
	
	Nous avons donc trois tests de Fisher permettant chacun de savoir si le facteur $A$ (respectivement $B$ ou l'interaction $AB$) ont une influence significative ou pas sur les mesures.

	Évidemment, dans les développements ci-dessus, les facteurs $A$ et $B$ sont interchangeables dans les développements par symétrie!
	
	\begin{tcolorbox}[title=Remarque,colframe=black,arc=10pt]
	Le modèle d'effet de facteur correspondant est (notez que le modèle n'est pas purement additif!):
	
	avec $i=1\ldots a$, $j=1\ldots b$, $k=1\ldots n$ où $\mu$ est la moyenne globale, $\tau_i$ est l'effet sur le $i$-ème traitement de $A$, $\beta_j$ est l'effet du $j$-ème niveau de traitement de $B$ (effets principaux), $(\tau\beta)_{ij}$ est l'effet de l'interaction entre $\tau_i$ et $\beta_j$, et $\varepsilon_{ijk}$ est l'habituelle erreur aléatoire suivant une loi $\mathcal{N}(0,\sigma_\varepsilon^2)$.\\
	
	Les suppositions sur les termes d'effets sont $\sum_{i=1}^a\tau_i=0$, $\sum_{j=1}^b\beta_j=0$, $\sum_{i=1}^a (\tau\beta)_{ij}=0$ pour $j=1\ldots b$ et $\sum_{j=1}^b (\tau\beta)_{ij}=0$ pour $i=1\ldots a$.\\
	
	Les hypothès d'intérêt sont quant à elles $H_0:\;  \tau_i=0$, $H_0:\; \beta_j=0$ or $H_0:\; (\tau\beta)_{ij}=0$.
	\end{tcolorbox}
	
	\paragraph{ANOVA multifactorielle à mesures répétées}\mbox{}\\\\
	"\NewTerm{L'ANOVA multifactorielle à mesures répétées}\index{ANOVA multifactorielle à mesures répétées}" ou appelée aussi "\NewTerm{ANOVA multifactorielle à variables catégorielles et mesures répétées}" (et très rarement "\NewTerm{ANOVA équilibrée}") est simplement le nom sous lequel les spécialistes désignent les ANOVA suivantes:
	\begin{itemize}
		\item ANOVA à trois facteurs (fixes) avec ou sans répétition
		\item ANOVA à quatre facteurs (fixes) avec ou sans répétition
		\item ANOVA à cinq facteurs (fixes) avec ou sans répétition
		\item etc.
	\end{itemize}
	Évidemment, les ANOVA à un et deux facteurs (fixes) font aussi partie de la famille de l'ANOVA multifactorielle mais elles sont rarement signalées en tant que tel dans les logiciels de statistiques et sont souvent disponibles de façon explicite dans les menus de ces mêmes logiciels (car ce sont les deux plus utilisées dans les écoles). Il faut savoir aussi que la majorité des logiciels de statistiques gèrent des ANOVA multifactorielles jusqu'à $15$ facteurs fixes (variables catégorielles) à condition que le plan soit équilibré (c'est à dire que pour chaque niveau de chaque facteur, il y ait un nombre identique de mesures). Un tableur (comme Microsoft Excel) gère le plus souvent les ANOVA jusqu'à un maximum deux facteurs (fixes).

	Bon maintenant le lecteur risque d'être déçu (bon je suis aussi déçu de n'avoir qu'une seule vie...) car franchement je ne souhaite pas refaire les développements vus plus haut pour les ANOVA à un facteur et deux facteurs (fixes) pour $3$, $4$ et ce jusqu'à $15$ facteurs car cela prendrait plus de 100 pages A4 sous une forme pédagogique et claire et en plus c'est basé toujours sur la même mécanique de développement (la théorie généralisée de l'ANOVA bien qu'étant beaucoup plus courte, elle est à mon goût indigeste).
	
	\begin{tcolorbox}[title=Remarque,colframe=black,arc=10pt]
	Le praticien (et le lecteur) doit éviter à tout prix de faire la confusion entre L'ANOVA "multifactorielle" qui se réfère à des cas où nous avons plus d'un facteur (ie, variable explicative catégorielle) et l'ANOVA multivariée qui est aussi appelée MANOVA (notez le M initial) qui fait référence aux cas où nous avez plusieurs variables dépendantes / réponses (nous étudierons cette dernière plus loin à la page \pageref{MANOVA}).  
	\end{tcolorbox}
	
	\pagebreak
	\subparagraph{Interlude sur les PCR, PRB, PCRB, PBI, PBBI...}\mbox{}\\\\
	Jusqu'à présent, nous avons vu des ANOVA très courantes. Mais il est temps de discuter de la manière de faire les expériences avant d'aller plus loin. Mais avant il serait utile de faire d'abord un rappel entre deux grands types d'expériences scientifiques!
	
	\begin{fquote}[Ronald Fisher]Consulter le statisticien après la fin d’une expérience, cela revient souvent à lui demander seulement une autopsie. Il peut peut-être dire de quoi l’expérience est morte...
 	\end{fquote}
	
	\textbf{Définition (\#\mydef):} Les "\NewTerm{études observationelles}\index{tudes observationelles}" sont définies comme suit:
	\begin{itemize}
		\item Recueillir des données d'une manière qui n'interfère pas directement avec la façon dont les données sont générées, c'est-à-dire simplement "observer
		
		\item Sur la base d'une étude observationnelle, nous ne pouvons établir qu'une \textbf{association}, autrement dit une "\textbf{corrélation}", entre les variables explicatives et de réponse
		
		\item Si une étude observationnelle utilise des données du passé, elle est nommée "\NewTerm{étude rétrospective}\index{\'etude r\'etrospective}", alors que si les données sont collectées tout au long de l'étude, elle est nommée "\NewTerm{étude prospective}\index{\'etude prospective}".
	\end{itemize}
	\textbf{Définition (\#\mydef):} Une "\NewTerm{étude expérimentale}\index {\'etude expérimentale}" est définie comme suit:
	\begin{itemize}
		\item Attribuer au hasard les sujets (individus) à divers traitements
		
		\item Peut établir des connexions \textbf{causales} entre les variables explicatives et de réponse!
	\end{itemize}
	Comme exemple compagnon, considérons que nous voulons évaluer la relation entre un entraînement gymnastique régulier et le niveau d'énergie. Nous pouvons concevoir l'étude comme une étude observationnelle ou une expérience:
	\begin{itemize}
		\item Dans une étude observationnelle, nous échantillonnons deux types de personnes de la population, celles qui choisissent de s'entraîner régulièrement et celles qui ne le font pas. Ensuite, nous trouvons le niveau d'énergie moyen (dans le cas d'une analyse naive!) pour les deux groupes de personnes et comparons.

		\item Dans une expérience, nous échantillonnons un groupe de personnes de la population et nous répartissons au hasard (dans un cas simple et naif!) ces personnes en deux groupes: ceux qui s'entraîneront régulièrement à la gymnastique tout au long de l'étude et ceux qui ne le feront pas. La différence est que la décision de s'entraîner ou non n'est pas laissée aux sujets comme dans l'étude observationnelle, mais plutôt imposée par le chercheur. À la fin, nous comparons le niveau d'énergie des deux groupes.
	\end{itemize}
	\begin{figure}[H]
		\centering
		\includegraphics[width=1.0\textwidth]{img/arithmetics/observational_vs_experimental_experiment.jpg}
		\caption{Étude expérimentale vs observationnelle}
	\end{figure}
	Sur la base de l'étude observationnelle, même si nous trouvons une différence entre les niveaux d'énergie de ces deux groupes de personnes, nous ne pouvons vraiment pas attribuer cette différence uniquement à la pratique de la gymnastique, car il peut y avoir d'autres variables que nous n'avons pas contrôlées (variables de confusion) dans cette étude qui contribuent à la différence observée. Par exemple, les personnes en meilleure forme peuvent être plus susceptibles de s'entraîner régulièrement et ont également des niveaux d'énergie plus élevés. Cependant, dans l'expérience, les variables qui pourraient également contribuer au résultat sont probablement représentées de manière égale dans les deux groupes en raison de l'assignation aléatoire. Par conséquent, si nous trouvons une différence entre les deux moyennes, nous pouvons effectivement faire une déclaration probablement causale attribuant cette différence à l'élaboration.
	
	Dans la pratique, à partir du montage expérimental suivant (une unité expérimentale, $24$ unités d'observation):
	\begin{figure}[H]
		\centering
		\includegraphics[width=0.35\textwidth]{img/arithmetics/experiment_setup_general.jpg}
	\end{figure}
	que nous numérotons comme suit:
	\begin{figure}[H]
		\centering
		\includegraphics[width=0.35\textwidth]{img/arithmetics/experiment_setup_general_numbered.jpg}
	\end{figure}
	nous différencions les configurations suivantes \footnote{Sans tenir compte du fait que pour chacune, il existe une version pour les plans exérimentaux équilibrées et non équilibrées et aussi une version pour les valeurs manquantes} de type de plan d'expérience le plus courant au type le plus rare ou avancé:
	\begin{itemize}
		\item "\NewTerm{Plan complétement randomisé}\index{ANOVA!plan complétement randomisé}" (PCR):
		
		Les ANOVA à un facteur (mais pas seulement) sont généralement associées aux PCR car l'idée sous-jacente est de faire des mesures sur des unités expérimentales randomisées pour chaque niveau du facteur, afin d'éviter un biais possible (confusion). Une configuration correspondante typique sera:
		\begin{figure}[H]
			\centering
			\includegraphics[width=0.5\textwidth]{img/arithmetics/experiment_setup_general_crd.jpg}
		\end{figure}
		
		\item "\NewTerm{Plan randomisé par blocs}\index{ANOVA!plan randomisé par blocs}" (PRB):
		
		Lorsqu'un facteur est maintenu fixe, alors que nous varierons les autres, il sera nommé "\NewTerm {facteur de blocage}\footnote{Plus formellement, un bloc est un groupe d'individus appariés qui sont similaires ou identiques sur une variable nuisible.}\index{ANOVA!facteur de blocage}" et les autres seront nommés "\NewTerm{facteurs de traitement}\index{facteur de traitement}" (certains auteurs désignent l'ANOVA à deux facteurs fixes sans interactions par le nom "\NewTerm{plan randomisé par blocs}").
		
		Dans la figure ci-dessous, nous voyons quelques blocs (correspondant généralement à des facteurs difficiles à modifier tels que les jours de la semaine, les observateurs, les terrains, etc.):
		\begin{figure}[H]
			\centering
			\includegraphics[width=0.55\textwidth]{img/arithmetics/experiment_setup_general_rbd.jpg}
		\end{figure}
		En pratique nous allons faire en sorte que la randomisation dans les blocs ne soit pas toujours effectuée dans le même ordre pour éliminer l'inertie potentielle lors du passage d'un traitement aux autres effets (le lecteur peut voir que sur la figure précédente que les deux blocs \texttt{I} et \texttt{V} ont la même randomisation). Cela nous a conduit au prochain type de conception.
		
		\item "\NewTerm{Plan complétement randomisé par blocs}\index{ANOVA!plan complétement randomisé par blocs}" (PCRB):
		
		Les plans complétement randomisés par blocs (PCRB) utilisent également un schéma de randomisation restreint dans le but de réduire les erreurs qui expliquent les différences entre les observations au sein de chaque traitement. Dans chaque bloc (par exemple les localisations), les traitements sont randomisés dans les unités expérimentales (par exemple, des parcelles de terrain). Cela crée une différence entre les blocs et rend l'observation au sein d'un bloc plus similaire.
		
		La plan a le terme "complétement" parce que nous voyons l'ensemble complet des traitements dans chaque bloc et qu'il n'y aucun bloc randomisé de manière identique (nous en apprendrons également plus tard sur les conceptions de blocs incomplets où ce n'est plus le cas). Notez que le blocage existe déjà au moment de la randomisation (et pas seulement au moment de l'analyse).
		
		L'analyse de plans complétement randomisés par blocs est simple. Nous traitons le facteur de blocage comme "un autre" facteur dans notre modèle\footnote{Nous pouvons également bloquer sur plus d'un facteur. Un cas particulier est le soi-disant Plan Carré Latin} et il s'agit alors simplement d'une ANOVA à deux facteurs.
		
		Parce que dans le PCR précédent les blocs \texttt{I} et \texttt{V} avaient une randomisation identique, nous avons maintenant en configuration PCRB:
		\begin{figure}[H]
			\centering
			\includegraphics[width=0.6\textwidth]{img/arithmetics/experiment_setup_general_rcbd.jpg}
		\end{figure}
		Donc, si vous êtes en mesure de maintenir les conditions homogènes/ constantes tout au long de vos expériences, le PCR convient parfaitement pour les expériences en laboratoire (les expériences en laboratoire sont généralement menées dans des conditions contrôlées, le test PCR est donc généralement utilisé). Mais si vous avez des ressources comme du temps, de l'espace et des fonds et que vous êtes suffisamment confiant pour neutraliser la variation (erreur) en utilisant une technique de blocs appropriée, optez pour le PCRD pour réduire l'erreur expérimentale.
		
		\item "\NewTerm{Plan par blocs incomplets}\index{ANOVA!plan par blocs incomplets}" (PBI):
		
		Certaines expériences peuvent consister en un grand nombre de traitements et il peut ne pas être possible d'exécuter tous les traitements dans tous les blocs. Les plans où seuls certains des traitements apparaissent dans chaque bloc sont donc appelés plans par blocs incomplets.
		
		Voici un exemple où nous pouvons voir que dans certains blocs, certains traitements n'apparaissent jamais:
		\begin{figure}[H]
			\centering
			\includegraphics[width=0.6\textwidth]{img/arithmetics/experiment_setup_general_ibd.jpg}
		\end{figure}
		
		\item "\NewTerm{Plan par blocs balancés}\index{ANOVA!plan par blocs balancés}" (PBB):
		
		Dans une telle configuration, chaque bloc est sélectionné de manière équilibrée de sorte que toute paire de traitements se produise ensemble le même nombre de fois que toutes les autres paires.
		
		En considérant les quatre traitements de notre exemple précédent, il y a  $\begin{pmatrix} 4\\2 \end{pmatrix}=6$ façons d'en choisir trois parmi quatre:
		\begin{figure}[H]
			\centering
			\includegraphics[width=0.6\textwidth]{img/arithmetics/experiment_setup_general_bibd.jpg}
		\end{figure}
	
		\item "\NewTerm{Plan randomisé par blocs généralisé}\index{ANOVA!plan randomisé par blocs généralisé}" (PRBG):
		
		Comme une plan complétement randomisé par blocs (PCRB), un PRBG est randomisé. Au sein de chaque bloc, les traitements sont assignés aléatoirement aux unités expérimentales: cette randomisation est également indépendante entre les blocs. Dans un PCRB (classique), cependant, il n'y a pas de réplication des traitements au sein des blocs. Cette réplication permet l'estimation et le test d'un terme d'interaction dans le modèle linéaire associé.
		
		\item Et d'autres comme les PIPB (plans incomplets partiellement balancés), PICB (plans incomplets cycliques balancés), PICBG (plans incomplets cycliques balancés généralisés), etc.
	\end{itemize}
	
	\subparagraph{Interlude sur les ANOVA de Type I, Type II et Type III}\mbox{}\\\\
	Le but de cet interlude est de clarifier la signification des différents types de sommes de carrés (et les idées sous-jacentes) que l'on retrouve dans la plupart des logiciels statistiques dans le contexte de l'analyse de la variance (ANOVA). Pour cela, nous nous placerons dans le cadre de l'ANOVA avec deux facteurs fixes.
	
	En effet, lorsque les données sont déséquilibrées, il existe différentes manières de calculer les sommes des carrés pour l'ANOVA. Il existe au moins 3 approches, communément appelées sommes des carrés de type I, II et III (également appelées "\NewTerm{carrés pondérés des moyennes de Yates}"\index {carrés pondérés des moyennes de Yates}). Ces noms semblent avoir été introduits dans le monde des statistiques à partir du progiciel SAS, mais sont depuis des décennies assez répandus. Le type à utiliser a conduit à une controverse permanente. Cependant, il s'agit essentiellement de tester différentes hypothèses sur les données.
	
	Pour introduire ces trois types, considérons donc un modèle d'analyse de la variance à deux facteurs fixes dans lequel:
	\begin{itemize}
		\item Le premier facteur que nous noterons $F_1$, a $J$ niveaux ($J\geq 2$) qui seront indexés avec la lettre $j$
	
		\item Le deuxième facteur que nous noterons $F_2$, a $K$ niveaux ($K\geq 2$) qui seront indexés avec la lettre $k$
	
		\item A l'intersection du niveau $j$ de $F_1$ et du niveau $k$ de $F_2$, nous réalisons $n_{jk}$ observations ($n_{jk}\geq 1$) d'une variable aléatoire $y$ (et nous supposons cette fois que le plan expérimental n'est pas nécessairement balancé)
	
		\item Chaque observation est notée $y_{ijk}$ (avec $i=1,\ldots,n_{jk}$ et $j=1,\ldots,J$ et $k=1,\ldots,K$)
	
		\item Nous posons:
		\begin{itemize}
			\item $n_{j+}=\sum_{k=1}^K n_{jk}$ comme la taille marginale de l'échantillon du niveau $j$ de $F_1$
	
			\item $n_{+k}=\sum_{j=1}^J n_{jk}$ comme la taille marginale de l'échantillon du niveau $k$ of $F_2$
	
			\item $n=\sum_{j=1}^J\sum_{k=1}^K n_{jk}$ comme le nombre total d'observations (mesures)
		\end{itemize}
	\end{itemize}
	Introduisons les différentes moyennes partielles empiriques d'observations $y_{ijk}$:
	
	Considérons maintenant le total de la somme des carrés du modèle, quantité avec comme on le sait $n-1$ degrés de liberté:
	
	Nous allons tout d'abord à nouveau expliciter la décomposition de la quantité TSC. Pour cela, remarquons d'abord l'égalité suivante:
	
	Comme:
	
	où DP signifie implicitement "(termes) avec double produits". Donc avec notre problème nous avons:
	
	Par triple sommation, on obtient:
	
	Pour détailler les doubles produits $\text{DP}_l$, rappelons d'abord que si les quantités $x_ {jk}$ sont des réels indépendants de $i$, nous pouvons écrire:
	
	Mais nous savons aussi que:
	
	Dès lors:
	
	Il s'ensuit que les sommes des trois premiers produits doubles (ceux dans lesquels la quantité $ (y_ {ijk} -y_ {jk}) $ est en facteur) sont nulles (c'est-à-dire $2ab$, $2ac$, $2ad$). Par contre, les trois autres (c'est-à-dire $2bc$, $2bd$, $2cd$) sont, en général, différents de zéro. Le quatrième ($2bc$) s'écrit:
	
	produit double que nous noterons $\text{SDP}_1$. Les sommes des deux derniers produits doubles ($2bd$ et $2cd$ donc $2d(b+c)$) seront regroupées dans l'expression suivante:
	
	somme de produits double que nous noterons $\text{SDP}_2$.
	
	Explicitons maintenant les sommes des carrés en introduisant les grandeurs suivantes (nous connaissons les degrés de liberté grâce à notre étude de l'ANOVA à deux facteurs fixes pour rappel!):
	\begin{itemize}
		\item Somme des carrés due au facteur $F_1$ (quantité à $J-1$ degrés de liberté):
		
		
		\item Somme des carrés due au facteur $ F_2 $ (quantité à  $ K-1 $ degrés de liberté):
		
		
		\item Somme des carrés due au facteur due aux interactions des facteurs (quantité à $(J-1)(K-1)$ degrés de liberté):
		
		
		\item Somme des carrés due aux erreurs (ou résidus, quantité à $ n- JK $ degrés de liberté):
		
	\end{itemize}
	On peut enfin réécrire le total de la somme des carrés sous la forme:
	
	Mais le plus souvent écrit:
	
	Ou explicitement:
	
	Nous savons déjà, à partir de notre dérivation de la somme des carrés de l'ANOVA  équilibrée à deux facteurs fixes, que dans le cas équilibré:
	
	La décomposition du TSC est alors d'interprétation évidente. En revanche, ce n'est pas le cas des plans déséquilibrés ANOVA pour lesquels ces deux grandeurs sont en général non nulles.
	
	Lorsque les quantités $\text{SDP}_A$ et $\text{SDP}_B$ sont non nulles, il n'est pas possible de les affecter à une seule source de variation ($F_A$, $F_B$ ou $F_A \times F_B$). Ceci explique les difficultés rencontrées pour spécifier les sources de variation dans un modèle relatif à un plan déséquilibré. Pour cette raison, un autre raisonnement est utilisé pour préciser ces sources, ce qui explique l'existence de plusieurs types de sommes de carrés, selon la philosophie choisie.
	
	\begin{tcolorbox}[title=Remarque,colframe=black,arc=10pt]
	Rappelez-vous que évidemment les trois types de somme de carrés que nous allons voir maintenant sont toujours égaux, et surtout égaux à zéro comme nous le savons déjà! 
	\end{tcolorbox}
	Représentons maintenant le modèle complet par $\text{SC}(A, B, AB)$, soit:
	
	 D'autres modèles sont représentés similairement par: 
	 
	 indique le modèle sans interaction. Le prochain:
	 
	indique le modèle qui ne tient pas compte des principaux effets du facteur $A$. Nous avons également:
	
	 
	et ainsi de suite.
	 
	L'influence de facteurs particuliers (y compris les interactions) peut être testée en examinant les différences entre les modèles. Par exemple, pour déterminer la présence d'un effet d'interaction, un test $F$ de Fisher des modèles $\text{SC}(A, B, AB)$ et le modèle sans interactions $\text{SC}(A, B)$ serait effectuée. Il est pratique de définir des sommes incrémentielles de carrés pour représenter ces différences. Comme:
	 
	La notation montre les différences incrémentielles dans les sommes des carrés, par exemple $\text{SC}(AB|A, B)$ représente la somme des carrés pour l'interaction après la suppression des effets principaux, et $\text{SC}(A|B)$ est la somme des carrés de l'effet principal est la somme des carrés de l'effet principal $A$ après la suppression de l'effet principal $B$ et en ignorant les interactions. Les différents types de sommes de carrés apparaissent alors en fonction du stade de réduction du modèle auquel ils sont effectués. En particulier:
	\begin{itemize}
		\item Les "\NewTerm{ANOVA de type I}\index{ANOVA!Type I}", aussi appelées "\NewTerm{somme des carrés séquentielle}", sont calculées en utilisant $\text{SC}(A)$ pour le facteur $A$. $\text{SC}(B | A)$ pour le facteur $B$. $\text{SC}(AB | B, A)$ pour l'interaction $AB$. Ceci teste l'effet principal du facteur $A$, suivi par l'effet principal du facteur $B$ après l'effet principal $A$, suivi par l'effet de l'interaction $AB$ après les effets principaux. En raison de la nature séquentielle et du fait que les deux facteurs principaux sont testés dans un ordre particulier, ce type de sommes de carrés donnera des résultats différents pour des données non équilibrées en fonction de l'effet principal considéré en premier \footnote{Cependant dans le cas des plans balancés, le résultat sera le même et nous retombons sur l'ANOVA classique à deux facteurs fixes}. C'est pourquoi les sommes de carrés de type I, dans les plans non-balancés, sont réservées aux modèles dans lesquels il existe un ordre naturel entre les facteurs. Notez que pour le type I, l'effet SC individuel totalise l'effet total SC comme suit:
		
		
		\item Les "\NewTerm{ANOVA de type II}\index{ANOVA!Type II}" Le sont calculé en utilisant  $\text{SC}(A | B)$ pour le facteur $A$. $\text{SC}(B | A)$ pour le facteur $B$. Ce type teste chaque effet principal après l'autre effet principal. Notez qu'aucune interaction significative n'est supposée (en d'autres termes, nous devons d'abord tester l'interaction ($\text{SC}(AB | A, B)$ et seulement si $AB$ n'est pas significatif, nous continuons avec l'analyse pour les effets principaux). S'il n'y a effectivement aucune interaction, alors le type II est statistiquement plus puissant que le type III (voir ci-dessous). Notez la propriété importante que le type II est indépendant de l'ordre à l'opposé des autres types. Notez que pour le type II, il est évident que l'effet individuel de la SC ne correspond pas à l'effet total de la CS à l'opposé du type I. En effet:
		
		
		\item Les "\NewTerm{ANOVA de type III}\index{ANOVA!Type III}", aussi appelées "\NewTerm{somme marginale des carrés}", sont calculées en utilisant $\text{SC}(A | B, AB)$ pour le facteur $A$. $\text{SC}(B | A, AB)$ pour le facteur $B$. Ce type teste la présence d'un effet principal après l'autre effet principal et l'interaction. Cette approche est donc valable en présence d'interactions significatives. Cependant, il n'est souvent pas intéressant d'interpréter un effet principal si des interactions sont présentes (d'une manière générale, si une interaction significative est présente, les effets principaux ne doivent pas être analysés ultérieurement). Si les interactions ne sont pas significatives, le type II donne un test plus puissant. Notez également que comme pour le type II, il est évident que l'effet individuel de la SC ne se résume pas à l'effet total de la SC:
		
	\end{itemize}
	\begin{tcolorbox}[title=Remarques,colframe=black,arc=10pt]
	\textbf{R1.} Lorsque les données sont balancées, comme déjà montré ci-dessus, les facteurs sont orthogonaux et les types I, II et III donnent tous les mêmes résultats!\\
	
	\textbf{R2.} Au début du 21e siècle, les logiciels SAS et SPSS utilisent la SC de type III par défaut, tandis que les fonctions livrées avec le package de base du logiciel R utilisent le type I par défaut. Cela peut conduire à des résultats différents lors de l'analyse des mêmes données avec différents packages de statistiques.
	\end{tcolorbox}
	En résumé, l'hypothèse qui nous intéresse porte généralement sur l'importance d'un facteur tout en contrôlant le niveau des autres facteurs. Cela équivaut à utiliser des SC de type II ou III. En général, s'il n'y a pas d'effet d'interaction significatif, alors le type II est plus puissant et suit le principe de marginalité. En cas d'interaction, le type II est inapproprié alors que le type III peut encore être utilisé, mais les résultats doivent être interprétés avec prudence (en présence d'interactions, les principaux effets sont rarement interprétables).
	
	Le sujet est assez controversé depuis des décennies. C'est pourquoi, lorsque cela est possible, il vaudrait mieux éviter les plans non-balancés ou au besoin faire des analyses en comparant les résultat avec les trois types de SC!
	
	\paragraph{ANOVA (imbriquée) Hiérarchique (HANOVA)}\label{hierarchical anova}\mbox{}\\\\
	"\NewTerm{L'ANOVA hiérarchique à facteurs fixes}\index{ANOVA!ANOVA hiérarchique à facteurs fixes}" (HANOVA) ou "\NewTerm{ANOVA imbriquée avec facteurs fixes}\index{ANOVA!ANOVA imbriquée avec facteurs fixes}", plus souvent appelée "\NewTerm{plan imbriqué en deux étapes}\index {plan imbriqué en deux étapes}", consiste à analyser des échantillons où l'un des facteurs de contrôle n'est pas indépendant mais dépend d'un autre facteur de contrôle (c'est pourquoi souvent les débutants hésitent parfois quelque peu dans la pratique entre une ANOVA à deux facteurs et une ANOVA hiérarchique à deux facteurs).
	
	Considérons pour introduire le concept le cas historique de l'ANOVA hiérarchique (agriculture!), deux surfaces complètement indépendantes: la surface d'étude $A$ et la surface d'étude $B$. Chacune de ces surfaces (nommées traditionnellement "\NewTerm{parcelles entières}\index{parcelle entière}" ou "\NewTerm{effet de bloc}\index{effet de bloc}") est séparée en $4$ carrés d'étude distincts (nommés "\NewTerm{split-plots}\index {split-plots}" ou "\NewTerm{sous-parcelles}\index{sous-parcelles}" ou encore "\NewTerm{sous-facteurs}\index{sous-facteurs}") avec $5$ répliques pour chacun des carré. Nous avons donc un total de $2$ parcelles entières, avec $8$ sous-parcelles (surfaces de test) pour un total de $40$ individus \footnote{Afin de mieux comprendre la structure des plans split-plot (PSP), il est avantageux de des les visualiser en superposant un PCRB (pour les traitements en split-plot) sur un autre PCRB (pour l'ensemble des traitements). Nous référons à ce type de plan par PSP(PCRB, PCRB). Des variantes de cette forme de plans split-plot sont possibles en utilisant une autre option que les deux PCRB.}:
	\begin{table}[H]
		\centering
		\begin{tabular}{cccclcccc}
		\multicolumn{4}{c}{\textbf{Surface d'Étude $A$}} &  & \multicolumn{4}{c}{\textbf{Study d'Étude $B$}} \\ \cline{1-4} \cline{6-9} 
		\multicolumn{1}{|c|}{\cellcolor[HTML]{9B9B9B}$\pmb{S_1(A)}$} & \multicolumn{1}{c|}{\cellcolor[HTML]{9B9B9B}$\pmb{S_2(A)}$} & \multicolumn{1}{c|}{\cellcolor[HTML]{9B9B9B}$\pmb{S_3(A)}$} & \multicolumn{1}{c|}{\cellcolor[HTML]{9B9B9B}$\pmb{S_4(A)}$} & \multicolumn{1}{l|}{} & \multicolumn{1}{l|}{\cellcolor[HTML]{9B9B9B}$\pmb{S_1(B)}$} & \multicolumn{1}{l|}{\cellcolor[HTML]{9B9B9B}$\pmb{S_2(B)}$} & \multicolumn{1}{l|}{\cellcolor[HTML]{9B9B9B}$\pmb{S_3(B)}$} & \multicolumn{1}{l|}{\cellcolor[HTML]{9B9B9B}$\pmb{S_4(B)}$} \\ \cline{1-4} \cline{6-9} 
		\multicolumn{1}{|c|}{X} & \multicolumn{1}{c|}{X} & \multicolumn{1}{c|}{X} & \multicolumn{1}{c|}{X} & \multicolumn{1}{l|}{} & \multicolumn{1}{c|}{X} & \multicolumn{1}{c|}{X} & \multicolumn{1}{c|}{X} & \multicolumn{1}{c|}{X} \\ \cline{1-4} \cline{6-9} 
		\multicolumn{1}{|c|}{X} & \multicolumn{1}{c|}{X} & \multicolumn{1}{c|}{X} & \multicolumn{1}{c|}{X} & \multicolumn{1}{l|}{} & \multicolumn{1}{c|}{X} & \multicolumn{1}{c|}{X} & \multicolumn{1}{c|}{X} & \multicolumn{1}{c|}{X} \\ \cline{1-4} \cline{6-9} 
		\multicolumn{1}{|c|}{X} & \multicolumn{1}{c|}{X} & \multicolumn{1}{c|}{X} & \multicolumn{1}{c|}{X} & \multicolumn{1}{l|}{} & \multicolumn{1}{c|}{X} & \multicolumn{1}{c|}{X} & \multicolumn{1}{c|}{X} & \multicolumn{1}{c|}{X} \\ \cline{1-4} \cline{6-9} 
		\multicolumn{1}{|c|}{X} & \multicolumn{1}{c|}{X} & \multicolumn{1}{c|}{X} & \multicolumn{1}{c|}{X} & \multicolumn{1}{l|}{} & \multicolumn{1}{c|}{X} & \multicolumn{1}{c|}{X} & \multicolumn{1}{c|}{X} & \multicolumn{1}{c|}{X} \\ \cline{1-4} \cline{6-9} 
		\multicolumn{1}{|c|}{X} & \multicolumn{1}{c|}{X} & \multicolumn{1}{c|}{X} & \multicolumn{1}{c|}{X} & \multicolumn{1}{l|}{} & \multicolumn{1}{c|}{X} & \multicolumn{1}{c|}{X} & \multicolumn{1}{c|}{X} & \multicolumn{1}{c|}{X} \\ \cline{1-4} \cline{6-9} 
		\end{tabular}
		\caption{Illustration d'une ANOVA hiérarchique}
	\end{table}
	Nous sommes donc dans la situation particulière où les effets des niveaux du facteur du deuxième niveau (carrés) n'ont pas de signification concrète. Par exemple, ces carrés dépendent du niveau du facteur \textit{Surface} considéré et une étude des principaux effets du facteur \textit{Carré} n'est ici pas pertinente!
	
	On peut considérer, par exemple, que le facteur de la parcelle entière représente les fournisseurs, les types de lot d'objet le facteur de la sous-parcelle, et donc pour chaque fournisseur un niveau donné de la sous-parcelle représentera physiquement un lot spécifique pour chaque fournisseur et donc non interchangeable.
	
	Ou un exemple plus courant est le "\NewTerm{l'étude gage R\&R (Répétabilité et Reproductabilité)}\index{\'etude gage R\&R}". En effet, "\NewTerm{l'analyse des systèmes de mesure}\index{analyse des syst\'emes de mesure}" (ASM) sont essentiels au succès de toute analyse de données. Si vous ne pouvez pas vous fier à l'outil que vous utilisez pour prendre des mesures, pourquoi vous embêter à collecter des données pour commencer? Ce serait comme essayer de perdre du poids tout en comptant sur une balance qui ne fonctionne pas. Quel est l'intérêt alors de se peser (même si comme on le sait il est préférable d'avoir de piètres données que pas de données du tout)?
	
	Dans les études ASM pour des mesures continues (par ex. poids, longueur, volume) utilisant des tests non destructifs, chaque pièce peut être mesurée à plusieurs reprises. Dans ce cas, nous pouvons utiliser la "\NewTerm{l'étude gage croisée}\index{\'etude gage crois\'ee}". Cependant, nous devons parfois effectuer une ASM où le test requis pour prendre la mesure détruit l'objet ou modifie physiquement la caractéristique mesurée.
	
	Supposons à nouveau que nous menions une étude gage R\&R pour un test destructif avec $3$ opérateurs et $2$ répliques par pièce. Mais supposons qu'il ne soit pas possible d'obtenir $6$ spécimens qui soient suffisamment similaires pour être considérés comme la même pièce, plutôt que seulement $2$ spécimens. Alors, dans ce cas, nous devons utiliser une "\NewTerm{étude gage imbriquée}\index{\'etude gage imbriqu\'ee}".
	
	Donc gardons à l'esprit que:
	\begin{itemize}
		\item Les études R\&R croisées sont liées à des mesures non destructives (comme expliqué ci-dessus)

		\item Les études R\&R imbriquées sont liées aux mesures destructives (comme expliqué ci-dessus)
	\end{itemize}
	
	Chaque opérateur mesure un ensemble différent de pièces. Par conséquent, chaque partie est dite "imbriquée" dans l'opérateur plutôt que "croisée" puisque chaque partie est unique à un opérateur. Observons le schéma suivant pour une autre perspective des études croisées par rapport aux études imbriquées:
	\begin{figure}[H]
		\centering
		\includegraphics[width=1.0\textwidth]{img/arithmetics/cross_vs_nested_gage_studies.jpg}
		\caption[Étude gage R\&R croisée vs imbriquée]{Étude gage R\&R croisée vs imbriquée (source: Minitab)}
	\end{figure}
	L'étude de gage croisée est associée à une ANOVA à facteur fixe normal, ou à une ANOVA carré latin si les mesures ne sont pas répétées (voir plus bas page \pageref {latin Square ANOVA}) alors que  l'étude de gage imbriquée est associée à une ANOVA imbriquée!
	
	Il est habituel de dire que nous avons une "\NewTerm{ANOVA splipt-plot à deux facteurs}\index{split-plot à deux facteurs}" ou "\NewTerm{ANOVA hiérarchique à deux facteurs}\index {ANOVA!hiérarchique à deux facteurs}" car le comportement des carrés n'est pas indépendant de celui des surfaces.
	
	\begin{tcolorbox}[title=Remarques,colframe=black,arc=10pt]
	\textbf{R1.} Pour différencier facilement une ANOVA à deux facteurs fixes avec une ANOVA imbriquée à deux facteurs fixes, il suffit de noter que physiquement un niveau donné du premier facteur fixe représente le même objet pour chacun des niveaux du second facteur tandis que pour l-ANOVA hiérarchique c'est à chaque fois un objet différent spécifique au facteur sous lequel il se trouve!\\
	
	\textbf{R2.} Nous avons donné ci-dessus deux exemples avec un seul niveau imbriqué, mais il est possible de développer la théorie pour plusieurs niveaux d'imbrication (un niveau, dans un niveau qui est lui-même dans un niveau, etc.). On parle alors "\NewTerm{d'ANOVA hiérarchique multiple}\index{ANOVA!hiérarchique multiple}".
	\end{tcolorbox}
	Une expérience hiérarchique (imbriquée) et l'analyse correspondante doivent d'abord pouvoir nous dire s'il y a une différence significative entre les "parcelles entières" (facteurs contrôlés) et aussi s'il existe une différence significative entre les "sous-parcelles".
	
	Il est important de se rappeler que dans ce type de structure comme chaque niveau d'un facteur n'est présent qu'une seule fois au niveau d'un autre facteur, nous ne pouvons pas estimer les interactions entre les deux. En effet, si l'on considère que le facteur \textit{Surface} est un pays et le carré (sous-parcelle) un lac ... il est alors difficile de déplacer le lac dans un autre pays ... En reprenant notre exemple ci-dessus avec les surfaces ce n'est pas non plus forcément très trivial pour déplacer un carré sur une autre surface...
	
	\begin{tcolorbox}[title=Remarque,colframe=black,arc=10pt]
	Nous ne pouvons pas utiliser un modèle où les facteurs sont imbriqués sauf si nous avons des répétitions. Dans le cas inverse où les tests ne seraient pas répétés, l'effet dû au facteur imbriqué ne pourra être étudié et le modèle que nous devrons utiliser pour analyser les données sera l'un de ceux exposés dans notre étude de l'analyse de la variance à une voie.
	\end{tcolorbox}
	Dans le cas particulier où l'étude statistique de l'ANOVA hiérarchique ne nous permet pas de de mettre en évidence une différence significative entre les carrés (sous-parcelles) de deux surface (parcelles entières), mais en revanche a réussi à mettre en évidence une différence significative entre les deux surface (parcelles entières) cela devrait nous conduire à conclure à une influence environnementale particulière.
	
	En notant $i=1\ldots M$ le nombre de niveaux du facteur principal $A$, par $j=i\ldots m$ le nombre de niveaux du facteur imbriqué $B$ et par $k=1\ldots n $ le nombre de mesures répétées (donc un total de mesures égal à $M \cdot m \cdot n $) et les moyennes:
	
	et sous les mêmes hypothèses d'utilisation que les ANOVA à une voie ou ou à deux vois vue plus haut (indépendance des erreurs $\varepsilon_{ijk} $, les erreurs ont la même variance, les erreurs suivent une loi Normale), nous écrivons la décomposition de la variance sous la forme:
	
	Mais, nous avons:
	
	et:
	
	Il reste donc à la fin:
	
	où les indices de sommation ne sont pas traditionnellement les mêmes que l'ANOVA à deux facteurs avec répétition (mais c'est un détail). Si nous comparons avec la décomposition de la variance de l'ANOVA à deux facteurs (fixes) avec répétition qui était pour le rappel:
	
	On remarque que la principale différence réside dans le fait qu'on ne s'intéresse pas à la variance de l'interaction des deux facteurs (qui était le but ou disons plutôt une ... contrainte!) et qu'on n'a alors pas eu à la faire apparaître.
	
	Vient maintenant la quête de la détermination des degrés de liberté. Pour $Q_T$ c'est $\text{ddl}_T=Mmn-1$. Pour le terme $Q_A$ (facteur principal), le nombre de degrés de liberté est le même que pour l'ANOVA à une voie ou l'ANOVA à deux facteurs avec/sans répétitions. Soit $\text {ddl}_A = M-1$ degrés de liberté. Pour le terme résiduel, il vient aussi immédiatement que c'est la même chose que l'ANOVA à deux facteurs avec répétitions et donc que $\text{ddl}_R=Mm(n-1)=N-Mm$. Enfin, le terme restant$Q_{(A)B}$ est nouveau, mais on peut facilement déterminer son nombre de degrés de liberté puisqu'il faut avoir:
	
	Dès lors il vient:
	
	Alors:
	
	Nous avons alors le tableau suivant pour l'ANOVA hiérarchique (selon la représentation la plus courante dans les logiciels statistiques) que nous commenterons après:
	\begin{table}[H]
		\centering
		\resizebox{\textwidth}{!}{\begin{tabular}{lcccc}
		\hline 
		\textbf{Somme des carrés}	& \textbf{$\chi^2$ ddl} & \textbf{Moyennes des carrés} & $F$  & \textbf{$F$ Critique} \\ 
			\hline 
		\parbox{5cm}{$Q_A=mn\displaystyle\sum_{i=1}^{M}\sum_{ijk}(\bar{x}_{i..}-\bar{x}_{\dots})^2$\\ (de la surface/bloc)} & $M-1$ & $\text{MCk}A= \dfrac{Q_A}{M-1}$ &$\dfrac{\text{MCk}A}{\text{MCk}(A)B}$  & $P(F>F_{M-1,Mm(n-1)})$ \\
		\parbox{6cm}{$Q_{(A)B}=n\displaystyle\sum_{i=1}^{M}\displaystyle\sum_{j=1}^{m}(\bar{x}_{ij.}-\bar{x}_{i..})^2$\\ (de la sous-parcelle dans la parcelle)} &$M(m-1)$  &$\text{MCk}(A)B= \dfrac{Q_{(A)B}}{M(m-1)}$  &  $\dfrac{\text{MCk}(A)B}{\text{MCE}}$ & $P(F>F_{M(m-1),Mm(n-1)})$  \\
		\parbox{6cm}{$Q_R=\displaystyle\sum_{i=1}^{M}\displaystyle\sum_{j=1}^{m}\sum_{k=1}^n(x_{ijk}-\bar{x}_{ij.})^2$\\ (résidus)}	& $Mm(n-1)$ & $\text{MCE}=\dfrac{Q_R}{Mm(n-1)}$  &  &  \\
		\parbox{5cm}{$Q_T=\displaystyle\sum_{ijk}(x_{ijk}-\bar{x}_{\dots})^2$\\ (total)}	& $Mmn-1$ & $\text{MCT}=\dfrac{Q_T}{Mmn-1}$ &  &  \\ 
			\hline 
		\end{tabular}}
		\caption{Table de l'ANOVA à facteurs (aléatoires) hiérarchiques}
	\end{table}
	Nous avons donc deux tests de Fisher, tout comme l'ANOVA canonique à deux facteurs fixes ce qui est relativement intuitif. En revanche, ce qui surprend souvent les praticiens, c'est le premier test de Fisher où l'on a:
	
	Au lieu de ce que nous aurions pu écrire un peu trop vite:
	
	Donc ce qu'il faut comprendre ici, c'est qu'il n'y en a pas un qui a raison et l'autre qui est faux !!!!!!!!! Il faut savoir ce qui nous intéresse pour faire le meilleur choix. Ainsi, comparer la variance des blocs avec la variance résiduelle des traitements n'est pas forcément la plus intéressante pour juger de l'égalité des traitements moyens entre eux. C'est pourquoi il vaut mieux comparer la variance des blocs avec la variance des traitements.
	
	Par conséquent, dans la plupart des logiciels statistiques, le tableau ci-dessus est considéré comme le "tableau ANOVA pour les facteurs aléatoires" et le tableau ci-dessous est la version pour les facteurs fixes (au cas où vous auriez vraiment besoin de comparer la variance des blocs avec la variance résiduelle des traitements!):
	\begin{table}[H]
		\centering
		\resizebox{\textwidth}{!}{\begin{tabular}{lcccc}
		\hline 
		\textbf{Somme des carrés}	& \textbf{$\chi^2$ ddl} & \textbf{Moyennes des carrés} & $F$  & \textbf{$F$ critique} \\ 
			\hline 
		\parbox{5cm}{$Q_A=mn\displaystyle\sum_{i=1}^{M}\sum_{ijk}(\bar{x}_{i..}-\bar{x}_{\dots})^2$\\ (de la surface/bloc} & $M-1$ & $\text{MCk}A= \dfrac{Q_A}{M-1}$ &$\dfrac{\text{MCk}A}{\text{MCE}}$  & $P(F>F_{M-1,Mm(n-1)})$ \\
		\parbox{6cm}{$Q_{(A)B}=n\displaystyle\sum_{i=1}^{M}\displaystyle\sum_{j=1}^{m}(\bar{x}_{ij.}-\bar{x}_{i..})^2$\\ (de la sous-parcelle dans la parcelle)} &$M(m-1)$  &$\text{MCk}(A)B= \dfrac{Q_{(A)B}}{M(m-1)}$  &  $\dfrac{\text{MCk}(A)B}{\text{MCE}}$ & $P(F>F_{M(m-1),Mm(n-1)})$  \\
		\parbox{6cm}{$Q_R=\displaystyle\sum_{i=1}^{M}\displaystyle\sum_{j=1}^{m}\sum_{k=1}^n(\bar{x}_{ijk}-\bar{x}_{ij.})^2$\\ (résidus)}	& $Mm(n-1)$ & $\text{MCE}=\dfrac{Q_R}{Mm(n-1)}$  &  &  \\
		\parbox{5cm}{$Q_T=\displaystyle\sum_{ijk}(x_{ijk}-\bar{x}_{\dots})^2$\\ (total)}	& $Mmn-1$ & $\text{MCT}=\dfrac{Q_T}{Mmn-1}$ &  &  \\ 
			\hline 
		\end{tabular}}
		\caption{Table de l'ANOVA à facteurs fixes hiérarchiques}
	\end{table}
	\begin{tcolorbox}[title=Remarque,colframe=black,arc=10pt]
	Rappelons que pour une ANOVA à facteurs fixes sans réplications nous avions eu:
	
	avec $i=1\ldots a$, $j=1\ldots b$ où $\mu$ est la moyenne globale, $\tau_i$ est l'effet du $i$ème traitement $A$, $\beta_j$ est l'effet du $j$ème niveau de traitement de $B$ (effets principaux) et $\varepsilon_{ij}$ est l'erreur aléatoire habituelle qui suit une loi $\mathcal{N}(0,\sigma_\varepsilon^2)$.\\
	
	Pour l'ANOVA imbriquée, les $\beta_j$ seront développés en $\alpha_j+\gamma_{ij}$ et le nouveau modèle peut contenir un terme d'interaction représentant une interaction possible entre le facteur de blocage et les traitements. Ainsi, le modèle d'effet de facteurs peut s'écrire:
	
	\end{tcolorbox}
	Maintenant, avant de continuer avec d'autres types majeurs d'ANOVA (en particulier "l'ANOVA avec blocs" et "l'ANOVA en parcelles fractionnées"), donnons sous forme de figure ce que nous avons vu jusqu'à présent et ce que nous verrons juste après:
	\begin{figure}[H]
		\centering
		\includegraphics[width=1.0\textwidth]{img/arithmetics/anova_main_types.jpg}
		\caption{Comparaison de différentes structures d'ANOVA majeures}
	\end{figure}
	\begin{itemize}
		\item[$\pmb{(a)}$] Un plan croisé examine chaque combinaison de niveaux pour chaque facteur fixe
	
		\item[$\pmb{(b)}$] Un plan imbriquée peut progressivement sous-répliquer un facteur fixe avec des niveaux imbriqués d'un facteur aléatoire qui sont uniques au niveau dans lequel ils sont imbriqués
	
		\item[$\pmb{(c)}$] Si un facteur aléatoire peut être réutilisé pour différents niveaux du traitement, il peut être croisé avec le traitement et modélisé comme un bloc
	
		\item[$\pmb{(d)}$] Un plan à parcelles divisées (également appelé "plan mixte") est une structure dans laquelle les effets fixes (tissu, drogue) sont croisés (chaque combinaison de tissu et de drogue sont testés) mais eux-mêmes imbriqués dans des répliques.\\
		
		Ainsi, dans un modèle ANOVA à plan mixte, un facteur (un facteur à effets fixes) est une variable inter-sujets et l'autre (un facteur à effets aléatoires) est une variable intra-sujets. Ainsi, globalement, le modèle est un type de modèle à effets mixtes.
	\end{itemize}
	\begin{tcolorbox}[title=Remarque,colframe=black,arc=10pt]
	Rappelons que pour une ANOVA à facteurs fixes sans réplications nous avons eu:
	
	avec $i=1\ldots a$, $j=1\ldots b$ où $\mu$ est la moyenne globale, $\tau_i$ est l'effet du $i$ème traitement $A$, $\beta_j$ est l'effet du $j$ème niveau de traitement de $B$ (effet principal) et $\varepsilon_{ij}$ et l'erreur alátoire habituelle qui suit une loi $\mathcal{N}(0,\sigma_\varepsilon^2)$.\\
	
	Pour les ANOVA imbriquées et croisées, les $\beta_j$ sont développés en  $\alpha_j+\gamma_{ij}+(\alpha\gamma)_{ij}$ ainsi que certains termes d'interactions blocs-traitements. Ainsi, le modèle d'effet de facteurs peut s'écrire:
	
	\end{tcolorbox}
	
	\paragraph{ANOVA avec blocs}\mbox{}\\\\
	Avec un "\NewTerm{plan randomisé par blocs}\index{plan randomisé par blocs}" (PRB), nous avons une caractéristique des unités d'analyse que nous stratifions (bloquons) puis randomisons dans nos conditions de traitement au sein de chaque bloc. Par exemple, nous pourrions bloquer sur le sexe (homme et femme), puis attribuer au hasard un traitement et une condition de contrôle séparément pour les hommes et les femmes, en assurant l'équilibre entre les blocs dans le nombre attribué à chaque groupe. Dans ce plan, nous avons un facteur (traitement / contrôle) et un bloc (homme / femme). Cette conception contrôle également toute variance associée au bloc (vous ne voudriez utiliser qu'un bloc dont vous avez de bonnes raisons de penser qu'il est associé à la variable dépendante).
	
	Le but du facteur de blocage est de tenir compte d'un facteur de nuisance et/ou de réduire le terme d'erreur utilisé lors de la réalisation du test de significativité de l'effet du traitement. Pour cette raison, la significativité de l'effet de bloc lui-même n'est pas testée, et de multiples comparaisons ne sont pas effectuées entre les blocs fixes. Sinon, une ANOVA bloquée à une voie est analysée comme une ANOVA à deux voies sans interactions et sans réplications. Cela signifie que le tableau ANOVA suivant:
	\begin{table}[H]
		\resizebox{\textwidth}{!}{\begin{tabular}{lcccc}\hline
		\textbf{Somme des carrés (SCE)} & $\chi^2$ \textbf{ddl} & \textbf{Moyenne des carrés} & $F$ & $F$ \textbf{Critique} \\ \hline
		$Q_A=k\displaystyle\sum_{j}\left(\bar{x}_{j}-\bar{\bar{x}}\right)^2$ & $k-1$ & $\text{MCk}A=\dfrac{Q_A}{k-1}$ &
		$\dfrac{\text{MCk}A}{\text{MCE}}$ & $P(F> F_{k-1,(k-1)(r-1)})$ \\
		$Q_B=r\displaystyle\sum_{i}\left(\bar{x}_{i}-\bar{\bar{x}}\right)^2$ & $r-1$ & $\text{MCk}B=\dfrac{Q_B}{r-1}$ &
		$\dfrac{\text{MCk}B}{\text{MCE}}$ & $P(F> F_{r-1,(k-1)(r-1)})$ \\
		$Q_R=\displaystyle\sum_{ij}\left(x_{ij}-\bar{x}_i-\bar{x}_j+\bar{\bar{x}}\right)^2$ & $(k-1)(r-1)$ & $ \text{MCE}=\dfrac{Q_R}{(k-1)(n-1)}$  & & \\
		$Q_T=\displaystyle\sum_{ij}\left(x_{ij}-\bar{\bar{x}}\right)^2$ & $N-1$ & $\text{MCT}=\dfrac{Q_T}{N-1}$ & &\\ \hline
		\end{tabular}}
		\caption[]{Table de l'ANOVA à deux facteurs fixes sans répétitions}
	\end{table}
	devient:
	\begin{table}[H]
		\resizebox{\textwidth}{!}{\begin{tabular}{lcccc}\hline
		\textbf{Sommes des carrés (SCE)} & $\chi^2$ \textbf{ddl} & \textbf{Moyennes des carrés} & $F$ & $F$ \textbf{Critique} \\ \hline
		\parbox{5cm}{$Q_t=k\displaystyle\sum_{j}\left(\bar{x}_{j}-\bar{\bar{x}}\right)^2$\\traitements} & $k-1$ & $\text{MCk}t=\dfrac{Q_t}{k-1}$ &
		$\dfrac{\text{MCk}t}{\text{MCE}}$ & $P(F> F_{k-1,(k-1)(r-1)})$ \\
		\parbox{5cm}{$Q_b=r\displaystyle\sum_{i}\left(\bar{x}_{i}-\bar{\bar{x}}\right)^2$\\blocks}& $r-1$ & $\text{MCk}b=\dfrac{Q_b}{r-1}$ &
		$\dfrac{\text{MCk}b}{\text{MCE}}$ & $P(F> F_{r-1,(k-1)(r-1)})$ \\
		$Q_R=\displaystyle\sum_{ij}\left(x_{ij}-\bar{x}_i-\bar{x}_j+\bar{\bar{x}}\right)^2$ & $(k-1)(r-1)$ & $ \text{MCE}=\dfrac{Q_R}{(k-1)(n-1)}$  & & \\
		$Q_T=\displaystyle\sum_{ij}\left(x_{ij}-\bar{\bar{x}}\right)^2$ & $N-1$ & $\text{MCT}=\dfrac{Q_T}{N-1}$ & &\\ \hline
		\end{tabular}}
		\caption{Table de l'ANOVA à un facteur bloqué}
	\end{table}	
	Dans la pratique, il n'est parfois pas facile de décider si les effets de bloc doivent être traités comme des effets aléatoires. Par exemple, les blocs d'une expérience concernant des terrrains sont-ils choisis au hasard pour former une population de blocs plus importante? Ce sont probablement les seuls blocs disponibles pour l'expérience. Si l'expérience est reproduite sur deux ans (mise en place d'un PCRB avec des facteurs de blocage imbriqués), ces années sont-elles sélectionnées au hasard? Certainement pas, mais le chercheur voudra quand même considérer les années comme "aléatoires". Mais si une année s'avère être une année sèche et l'autre une année pluvieuse, alors nous avons clairement un facteur de blocage intrinsèque à effets fixes. Il existe évidemment de nombreuses variantes de cette discussion et cette question devient donc plutôt philosophique et souvent controversée. Généralement, on préfère faire l'analyse une fois en les considérant comme fixes et une fois comme aléatoires (car c'est très rapide avec un logiciel informatique), et voir si les conclusions sont les mêmes.
	
	Pour deux échantillons, une ANOVA bloquée à un facteur équivaut au test $T$ de Student pour deux échantillons  appariés!
	
	\paragraph{ANOVA carré Latin sans répétitions}\label{latin Square ANOVA}\mbox{}\\\\
	Une "\NewTerm{ANOVA carré latin}\index{ANOVA!carré latin}" est une ANOVA imbriquée à trois facteurs mais avec la subtilité que le troisième facteur imbriqué n'a pas chacun de ses niveaux répétés dans les niveaux des deux premiers mais une seule fois et de telle manière que chacun de ses niveaux n'apparaisse qu'une seule fois à chaque ligne et à chaque colonne du niveau parent:
	\begin{figure}[H]
		\begin{center}
		\begin{tabular}{|ccc|}
		\hline
		1&2&3\\
		3&1&2\\
		2&3&1\\
		\hline
		\end{tabular}
		\hspace{10pt}
		\begin{tabular}{|cccc|}
		\hline
		4&3&1&2\\
		3&4&2&1\\
		1&2&4&3\\
		2&1&3&4\\
		\hline
		\end{tabular}
		\hspace{10pt}
		\begin{tabular}{|ccccc|}
		\hline
		1&2&4&3&5\\
		4&5&2&1&3\\
		3&4&1&5&2\\
		2&3&5&4&1\\
		5&1&3&2&4\\
		\hline
		\end{tabular}
		\caption{Carrés Latin d'ordre 3, 4 et 5}
		\end{center}
	\end{figure}
	Le nombre de niveaus de chaque facteur est imposé comme étant en quantité égale, c'est pourquoi il y a le terme "carré" dans les plans "carré latin"! Le terme "latin" vient du fait qu'aucun terme n'est répété sur la même ligne ou sur la même colonne! On parle alors logiquement de "\NewTerm{plan carré latin équilibré}" ou simplement de "\NewTerm{plan carré latin}" (PCL).
	
	Ainsi, ce plan (attribuée à Leonard Euler en 1783) représente, dans un certain sens, la forme la plus simple d'un plan ligne-colonne car il est utilisée pour comparer $ K $ traitements dans $K$ lignes et $K$ colonnes, où les lignes et les colonnes représentent deux facteurs de blocage. Ils ont été proposés comme plan expérimental par Ronald Fisher en 1925, bien que François Cretté De Palluel (1788) ait déjà utilisé l'idée d'un plan de carré latin $4\times 4$ pour une expérience agricole.
	
	Une illustration complète d'un tel plan peut être donnée par l'exemple explicite suivant (typique dans les études croisées Gage R\&R sans réplications!):
	\begin{table}[H]
	\centering
	\begin{tabular}{rlllll}
	  \hline
	    & \multicolumn{5}{c}{Opérateurs} \\
	  \cline{2-6}
	  Lots  & 1 & 2 & 3 & 4 & 5 \\
	  \hline
	  1 & A=$24$ & B=$20$ & C=$19$ & D=$24$ & E=$24$ \\
	  2 & B=$17$ & C=$24$ & D=$30$ & E=$27$ & A=$36$ \\
	  3 & C=$18$ & D=$38$ & E=$26$ & A=$27$ & B=$21$ \\
	  4 & D=$26$ & E=$31$ & A=$26$ & B=$23$ & C=$22$ \\
	  5 & E=$22$ & A=$30$ & B=$20$ & C=$29$ & D=$31$ \\
	   \hline
	\end{tabular}
	\caption{Célèbre plan carré latin pour le modèle propulseur de Douglas C.Montgomery}
	\end{table}
	On voit pourquoi les ANOVA  carré latin sont relativement rares en pratique car elles supposent que pour chaque objet on a le même nombre de combinaisons de tests alors qu'en réalité on pourrait en avoir moins ou plus. De plus, les ANOVA carrés latins sont souvent désignés comme des plans incomplets, ou "\NewTerm{plans par blocs incomplets}\footnote {Gardez donc à l'esprit qu'il s'agit d'une conception par blocs randomisés dans laquelle chaque traitement n'est pas présent dans chaque bloc}\index{plans par blocs incomplets}", et donc in extenso économique car si on devait vraiment faire toutes les combinaisons pour un carré latin de dimension $n$, le nombre de combinaisons totales serait $n^3$ alors qu'en réalité le carré latin en représente seulement $n^2$.
	
	Par conséquent, nous pouvons déjà observer que:
	\begin{enumerate}
		\item L'ordre d'exécution des essais est aléatoire
		
		\item Nous avons de nombreux essais (comme l'exemple ci-dessus qui devrait avoir $5\cdot 5 \cdot 5 \cdot 5=125$ essais s'il était complet)
	\end{enumerate}
	Notons également que pour une dimension donnée, un carré latin n'est pas unique. En effet le carré latin suivant:
	\begin{table}[H]
		\centering
		\begin{tabular}{rllll}
		  \hline
		    & \multicolumn{3}{c}{Cars} \\
		  \cline{2-5}
		  Drivers & 1 & 2 & 3 & 4 \\
		  \hline
		  1 & A & B & D & C \\
		  2 & D & C & A & B \\
		  3 & B & D & C & A \\
		  4 & C & A & B & D \\
		   \hline
		\end{tabular}
	\end{table}
	est équivalent à celui-là:
	\begin{table}[H]
		\centering
		\begin{tabular}{rllll}
		  \hline
		    & \multicolumn{3}{c}{Voitures} \\
		  \cline{2-5}
		  Conducteurs & 1 & 2 & 3 & 4 \\
		  \hline
		  1 & A & B & D & C \\
		  2 & C & D & B & A \\
		  3 & B & A & C & D \\
		  4 & D & C & A & B \\
		   \hline
		\end{tabular}
	\end{table}
	L'intérêt réel des ANOVA carré latin est qu'elles sont certes économiques puisqu'elles sont incomplètes mais principalement qu'elles sont orthogonales en ce sens qu'elles contiennent la meilleure façon de faire les tests en minimisant la quantité de ces derniers. En effet, on voit que dans un carré latin que chaque lettre n'apparaît qu'une seule fois dans chaque ligne et chaque colonne (d'où l'origine du nom!) Afin de minimiser l'erreur (mathématiquement il y a donc "orthogonalité").
	
	Il est courant de représenter une ANOVA carré latin de la manière plus générale suivante:
	\begin{table}[H]
		\centering
		\begin{tabular}{r|llll|c}
		  \hline
		    & \multicolumn{4}{c}{1er Facteur} \\  \hline
		  \cline{2-5}
		  2ème Facteuer & 1 & 2 & 3 & 4 & Moyennes des lignes  \\
		  \hline
		  \texttt{I} & $y_{111}$ & $y_{122}$ & $y_{134}$ & $y_{143}$ & $\bar{y}_{1..}$ \\
		  \texttt{II} & $y_{214}$ & $y_{223}$ & $y_{231}$ & $y_{341}$ & $\bar{y}_{2..}$\\
		  \texttt{III} & $y_{312}$ & $y_{324}$ & $y_{333}$ & $y_{341}$ & $\bar{y}_{3..}$ \\
		  \texttt{IV} & $y_{413}$ & $y_{421}$ & $y_{432}$ & $y_{444}$ & $\bar{y}_{4..}$ \\ \hline
		  Moyennes des colonnes & $\bar{y}_{.1.}$ & $\bar{y}_{.2.}$ & $\bar{y}_{.3.}$ & $\bar{y}_{.4.}$ & $\bar{y}_{...}$ \\
		  \hline
		\end{tabular}
	\end{table}
	Et notons évidemment que le nombre de lignes est (évidemment!) toujours égal au nombre de colonnes et pour de tels plans, on parle aussi de "\NewTerm{plans symétriques}".
	
	Prouvons que nous avons:
	
	où $R$ désigne les "lignes", $C$ les "colonnes" et $r$ les "résidus".
	
	Eh bien c'est parti dans la joie et la bonne humeur ...! Nous écrivons d'abord:
	
	Mais:
	
	En effet, commençons par remarquer que:
	
	Les mêmes calculs montrent que:
	
	De plus:
	
	et encore une fois nous avons le même résultat pour les indices $ j $ et $ k $. C'est-à-dire:
	
	Dès lors:
	
	Maintenant, notez que nous avons:	
	
	étant donné que:
	
	Ici encore, nous avons également avec les mêmes calculs:
	
	De plus:
	
	Dès lors:
	
	Nous avons:
	
	étant donné que: 
	
	De plus:
	
	étant donné que:
	
	Le même résultat s'applique pour $j$ et $k$, dès lors:
	
	Dès lors:
	
	Retour au début...
	
	Ensuite, il nous reste:
	
	Évaluons les produits doubles dans la dernière somme:
	
	car comme vu précédemment, tout cela vaut $K^2(\bar{y})^2$.
	
	Les autres produits doubles valent également $0$ par les mêmes arguments. Donc:
	
	Pour finir, notons que:
	
	et de même pour $ j $ et $ k $. Donc:
	
	Si nous notons $ K $ la dimension d'un carré latin, pour chacun des termes ci-dessus, nous avons les degrés de liberté suivants:
	
	Nous avons alors le tableau ANOVA suivant (rappelez-vous que le $R$ est pour les lignes et le $C$ est pour les colonnes!):
	\begin{table}[H]
		\centering
		\resizebox{\textwidth}{!}{\begin{tabular}{lcccc}
		\hline 
		\textbf{Somme des carrés}	& \textbf{$\chi^2$ ddl} & \textbf{Moyenne des carrés} & $F$  & \textbf{$F$ critique} \\ 
			\hline 
		\parbox{5cm}{$Q_R=K\displaystyle\sum_{i}{\left(\bar{y}_{i..}-\bar{y}_{\dots}\right)}^{2}$\\ (lignes)} & $K-1$ & $\text{MCk}R= \dfrac{Q_R}{K-1}$ &$\dfrac{\text{MCk}R}{\text{MCE}}$  & $P(F>F_{K-1,(K-1)(K-2)})$ \\
		\parbox{6cm}{$Q_C=K\displaystyle\sum_{j}{\left(\bar{y}_{.j.}-\bar{y}_{\dots}\right)}^{2}$\\ (colonnes)} &$K-1$  &$\text{MCk}C= \dfrac{Q_{C}}{K-1}$  &  $\dfrac{\text{MCk}C}{\text{MCE}}$ & $P(F>F_{K-1,(K-1)(K-2)})$ \\
		\parbox{6cm}{$Q_{RC}=K\displaystyle\sum_{k}{\left(\bar{y}_{..k}-\bar{y}_{\dots}\right)}^{2}$\\ (traîtements)}	& $K-1$ & $\text{MCk}RC=\dfrac{Q_{RC}}{K-1}$  & $\dfrac{\text{MCk}RC}{\text{MCE}}$ & $P(F>F_{K-1,(K-1)(K-2)})$ \\
		\parbox{7cm}{$Q_r=\displaystyle\sum_{i,j,k}\left({y}_{ijk}-\bar{y}_{i..}-\bar{y}_{.j.}-\bar{y}_{..k}+2\bar{y}_{\dots}\right)^2$\\ (résidus)}	& $(K-1)(K-2)$ & $\text{MCE}=\dfrac{Q_r}{(K-1)(K-2)}$  &  &  \\ 
		\parbox{6cm}{$Q_T=\displaystyle\sum_{i,j,k}{\left({y}_{ijk}-\bar{y}_{\dots}\right)}^{2}$\\ (total)} & $K^2-1$ & $\text{MCT}=\dfrac{\mathrm{Q_T}}{K^2-1}$ &  &  \\ 
			\hline 
		\end{tabular}}
		\caption{Table de l'ANOVA carré latin}
	\end{table}
	\begin{tcolorbox}[title=Remarque,colframe=black,arc=10pt]
	Le modèle d'effet de facteurs pour ce plan est:
	
	où $i,j,k=1...N$ et $y_{ijk}$ est l'observation de la $i$ème ligne et $k$ème colonne pour $j$ème traitement, $\mu$ est la moyenne globale, $\alpha_i$ est l'effet de la $i$ème ligne, $\tau_j$ est l'effet du $j$ème traitement, $\beta_k$ est l'effet de la $k$ème colonne, et $\varepsilon_{ijk}$ et l'erreur aléatoire.\\
	
	Notez que seuls les trois indices sont nécessaires pour identifier une observation dans le dessin carré latin et que ce modèle est (évidemment) purement additif. Les hypothèses sur les termes d'effet sont $\sum_{i=1}^N \alpha_i=0$, $\sum_{j=1}^N \tau_j=0$ et $\sum_{k=1}^N \beta_k=0$.\\
	
	Nous sommes intéressés (principalement!) par un effet de traitement nul, c'est-à-dire, $H_0:\;\tau_1= \tau_2=\ldots=0$ versu $H_1:\; \tau_j\neq 0$.
	\end{tcolorbox}
	Gardons à l'esprit que les avantages des plans carrés latins sont:
	\begin{itemize}
		\item Ils permettent des expériences avec un petit nombre d'essais (plus économique que de croiser tous les facteurs)
		\item Facile à analyser
	\end{itemize}
	Les inconvénients sont:
	\begin{itemize}
		\item Le nombre de niveaux de chaque variable de blocage doit être égal au nombre de niveaux du facteur de traitement
	
		\item Ils supposent qu'il n'y a pas d'interactions entre les variables de blocage ou entre la variable de traitement et la variable de blocage (peut être irréaliste)
	
		\item On dit que les carrés inférieurs à $4\times 4$ ou $ 5 \times 5$ ont généralement trop peu de réplications pour un niveau de précision souhaitable (faible puissance)
		
		\item On peut prouver que les carrés $ 6 \times 6 $ sont impossibles (la preuve n'est en fait pas dans ce livre!)
	\end{itemize}
	\begin{tcolorbox}[title=Remarques,colframe=black,arc=10pt]
	R1. Les plans carrés latins font partie de ce que nous appelons aussi parfois "\NewTerm{plans à remplissage d'espace}" (famille qui contient également les plans sphériques, des plans uniformes avec les plans à potentiel minimum, ou les plans à entropie maximale ou les plans à processus Gaussiens EQMI (erreur quadratique moyenne intégrée)).\\
	
	R2. Il existe aussi des plans Latins rectangulaires pour information...
	\end{tcolorbox}
	
	\pagebreak
	\paragraph{ANOVA carré Gréco-Latin}\mbox{}\\\\
	Une "\NewTerm{ANOVA carré Gréco-Latin}\index{ANOVA!carré Gréco-Latin}" est une extension d'un carré Latin utilisé lorsque le nombre de blocs est supérieur à $ 2$. Concrètement, une ANOVA  carré Gréco-Latin peut contrôler jusqu'à deux facteurs de nuisance et un facteur de traitement (trois sources de variabilité exogènes ...) comme représenté ci-dessous:
	\begin{table}[H]
	\centering
	\begin{tabular}{rllll}
	  \hline
	    & \multicolumn{4}{c}{Colonne} \\
	  \cline{2-5}
	  Ligne & 1 & 2 & 3 & 4 \\
	  \hline
	  1 & A$\alpha$ & B$\beta$ & C$\gamma$ & D$\delta$ \\
	  2 & B$\delta$ & A$\gamma$ & D$\beta$ & C$\alpha$ \\
	  3 & C$\beta$ & D$\alpha$ & A$\delta$ & B$\gamma$ \\
	  4 & D$\gamma$ & C$\delta$ & B$\alpha$ & A$\beta$ \\
	   \hline
	\end{tabular}
	\caption{Carré Gréco-Latin d'ordre $4$}
	\end{table}
	L'utilisation de ces plans se traduit par un gain de temps et de moyens assez important, car le nombre total de points de conception a été considérablement réduit car si nous considérions $n$ niveaux pour chacun des trois facteurs, nous n'avons que $n^2$ points expérimentaux, lorsqu'il y a $n^4 $ (puisqu'il y a quatre facteurs !!!!) points de conception dans le plan factoriel complet. Bien que semblables aux carrés Latins, ces modèles ne peuvent être utilisés que lorsque les interactions sont statistiquement non significatives.
	
	Pour comprendre un carré Gréco-Latin, on considère d'abord un carré latin $K \otimes K$ sur lequel on a superposé un autre carré latin $K \otimes K$ qui est cette fois désigné par des lettres grecques. Ensuite dans un second temps lors de la superposition, il faut s'assurer que la propriété suivante est respectée: «chaque ligne et chaque colonne ne peuvent contenir que quelques lettres (latine, grecque) distinctes». Ainsi, si cette propriété est vérifiée (comme le montre le schéma ci-dessous) on dit que les deux carrés latins orthogonaux sont... orthogonaux:
	\begin{figure}[H]
		\centering
		\includegraphics[scale=0.9]{img/arithmetics/latin_square_to_greaco_square.jpg}
		\caption{Fusion de deux plans carré Latin}
	\end{figure}
	\begin{tcolorbox}[title=Remarque,colframe=black,arc=10pt]
	Les plans carrés Gréco-Latins font également partie de ce que nous appelons aussi parfois les "\NewTerm{plans à remplissage d'espace}" comme pour les plans carrés latins.
	\end{tcolorbox}
	Comme vous pouvez le constater, le plan carrée Gréco-Latin permet l'analyse de $4$ facteurs  (ligne, colonne, lettre latine, lettre grecque), chaque facteur ayant $K$ niveaux  pour seulement $ K \otimes K $ ou $K^2$ combinaisons possibles.

	Pour la nécessité de l'analyse de la variance, nous devons définir d'abord certaines variables:
	\begin{itemize}
		\item $K$ le nombre de niveaux de chaque facteur
		
		\item $N=K^2$, le nombre total d'observations $ y_ {i, j, k, l} $ dans le plan carrée Gréco-Latin, où $ i $ est le $ i$ème niveau de facteur représenté en ligne, $ j $ le $ j$ème niveau du facteur représenté en colonne, $ k $ le $ k$ème niveau du facteur représenté par des lettres latines et $ l $ le $ l$ème niveau du facteur représenté par des lettres grecques.
	\end{itemize}
	
	Ainsi, nous pouvons formuler comme suit:
	\begin{itemize}
	  \item La somme et la moyenne des observations:
	 
	  \item La somme et la moyenne de chaque ligne:
	   
	  \item La somme et la moyenne de chaque colonne:
	   
	  \item La somme et la moyenne pour chaque lettre latine:
	  
	  \item La somme et la moyenne pour chaque lettre grecque:
	   
	\end{itemize}
	Pour le développement, nous préférerons cette notation $\displaystyle\sum_{i,j,k,l}^{K}$ à celle-ci $\displaystyle \sum_{i=1}^{K}\sum_{j=1}^{K}\sum_{k=1}^{K}\sum_{l=1}^{K}$ afin de simplifier les écritures de ce qui va suivre.
	
	
	Traitons maintenant de la décomposition de la variance totale! Prouvons que cette dernière:
	
	peut être écrite:
	
	Pour cela, posons d'abord:
	 
	Mettons un peu de $\bar{y}_{....}$ dedans comme ceci:
	
	A ce niveau, nous retombons sur notre fameuse identité remarquable $(a + b) ^ 2 = a ^ 2 + b ^ 2 + 2ab$. Ainsi dans le cadre de notre développement, nous obtenons alors:
	
	Considérons maintenant le $ 3 $ ème membre de cette expression $2\displaystyle\sum ab$:
	
	Et montrons que le résultat de ce dernier est égal à $ 0 $. Commençons par le développer comme ci-dessous:
	
	Développons le premier terme ci-dessus désigné par l'exposant $^{(a)}$:
	
	Évaluons chacun des six termes ci-dessus séparément:
	
	Donc, finalement, nous obtenons que le terme original désigné par l'exposant $^{(a)}$ :
	
	est égal à:
	
	Ce qui donne:
	
	Ainsi, par le même processus mathématique ennuyeux, nous pouvons prouver que chacun des quatre autres termes restants est également égal à zéro. Donc:
	
	
	
	
	Maintenant que nous avons prouvé que$2\displaystyle\sum ab  = 0$, c'est-à-dire:
	
	Revenons au début (...):
	
	Puisque le $3$ème est égal, il ne reste évidemment que:
	
	Prenons le $2$ème terme de l'expression ci-dessus:
	
	Dans le développement ci-dessus, considérons l'expression suivante:
	
	Et prouvons qu'elle est égale à zéro:
	
	Par conséquent, chacune des expressions suivantes:
	
	
	
	est égale à $0$.
		
	Finalement il reste:
	
	Si nous désignons comme d'habitude par SCE la somme des carrés des écarts, nous pouvons réécrire notre résultat sous cette forme:
	
	où:
	
	et:
	
	Avec pour degrés de liberté:
	$$\text{ddl}_{T} = \text{ddl}_{R} + \text{ddl}_{C} + \text{ddl}_{l} + \text{ddl}_{g} +\text{ddl}_{r}$$
	$${K}^{2}-1 = \left(K-1 \right) + \left(K-1 \right) + \left(K-1 \right) + \left(K-1 \right) + \left(K-1 \right)\left(K-3 \right)$$
	Le degré de liberté pour les carrés des différences résiduelles est obtenu en posant:
	$${K}^{2}-1 -\left[\left(K-1 \right) + \left(K-1 \right) + \left(K-1 \right) + \left(K-1 \right) \right] = \left(K-1 \right)\left(K-3 \right)$$
	Nous avons alors le tableau ANOVA suivant:
	\begin{table}[H]
		\centering
		\resizebox{\textwidth}{!}{\begin{tabular}{lcccc}
		\hline 
		\textbf{Sommes des carrés (SCE)}	& \textbf{$\chi^2$ ddl} & \textbf{Mean squares} & $F$  & \textbf{$F$ critique} \\ 
			\hline 
		\parbox{5cm}{$Q_R=\displaystyle K\sum_{i}^{K}\left( \bar{y}_{i...}-\bar{y}_{....}\right)^2$\\ (lignes)} & $K-1$ & $\text{MCk}R= \dfrac{Q_R}{K-1}$ &$\dfrac{\text{MCk}R}{\text{MCE}}$  & $P(F>F_{K-1,(K-1)(K-3)})$ \\
		\parbox{6cm}{$Q_C=K\displaystyle\sum_{j}{\left(\bar{y}_{.j.}-\bar{y}_{\dots}\right)}^{2}$\\ (colonnes)} &$K-1$  &$\text{MCk}C= \dfrac{Q_{C}}{K-1}$  &  $\dfrac{\text{MCk}C}{\text{MCE}}$ & $P(F>F_{K-1,(K-1)(K-3)})$ \\
		\parbox{6cm}{$Q_l=\displaystyle K\sum_{k}^{K}\left( \bar{y}_{..k.}-\bar{y}_{....}\right)^2 $\\ (lettre latine)}	& $K-1$ & $\text{MCk}l=\dfrac{\mathrm{Q_l}}{K-1}$  & $\dfrac{\text{MCk}l}{\text{MCE}}$ & $P(F>F_{K-1,(K-1)(K-3)})$ \\
		\parbox{6cm}{$Q_g=\displaystyle K\sum_{l}^{K}\left( \bar{y}_{...l}-\bar{y}_{....}\right)^2 $\\ (lettre grecque)}	& $K-1$ & $\text{MCk}g=\dfrac{Q_l}{K-1}$ & $\dfrac{\text{MCk}g}{\text{MCE}}$ & $P(F>F_{K-1,(K-1)(K-3)})$ \\ 
		\parbox{7.5cm}{$Q_r$\\$=\displaystyle\sum_{i,j,k,l}^{K}\left({y}_{ijkl}-\bar{y}_{i...}-\bar{y}_{.j..}-\bar{y}_{..k.}-\bar{y}_{...l}+3\bar{y}_{....}\right)^2 $\\ (residuals)}	& $(K-1)(K-3)$ & $\text{MCE}=\dfrac{Q_r}{(K-1)(K-3)}$  &  &  \\ 
		\parbox{7.5cm}{$Q_T$\\$=\displaystyle\sum_{i=1}^{K}\sum_{j=1}^{K}\sum_{k=1}^{K}\sum_{l=1}^{K}{\left({y}_{ijkl}-\bar{y}_{....} \right)}^{2}$\\ (total)} & $K^2-1$ & $\text{MCk}T=\dfrac{Q_T}{K^2-1}$ &  &  \\ 
			\hline 
		\end{tabular}}
		\caption{Tableau de l'ANOVA Gréco-Latin}
	\end{table}
	Notez que dans le plan carré gréco-latin comme dans le plan carré latin
il est supposé qu'aucune interaction entre les facteurs n'existe. C'est une hypothèse encore plus forte dans le plan carré gréco-latin parce que nous avons un facteur supplémentaire et, par conséquent, plus d'interactions possibles.

	\begin{tcolorbox}[title=Remarque,colframe=black,arc=10pt]
	Le modèle d'effet de facteur pour ce plan (purement additif!) est :
	
	où $i,j,k,l=1\ldots N$ et où $y_{ijkl}$ est l'observation de la $i$ème ligne, lettre latine $j$, lettre grecque $k$, et $l$ème colonne, $\mu$ est la moyennes globale, $\theta_i$ est l'effet de la $i$ème ligne, $\tau_j$ est l'effet de la lettre latine $j$, $\omega_k$ est l'effet de la lettre grecque $k$, $\phi_l$ est l'effet de la $l$ème colonne, et $\varepsilon_{ijkl}$ sont les erreurs aléatoires et identiqeuemnt distribuées $\mathcal{N}(0,\sigma_\varepsilon^2)$. Notons que, comme pour le plan carré latin, deux indices suffisent pour identifier une observation et que les plans Gréco-Latins $6 \times 6$ sont impossibles.\\
	
	Nous sommes (principalement!) intéressés par tester pour les effets nuls de traitement c'est-à-dire, $H_0:\;\tau_1=\tau_2=\ldots=\tau_N=0$ versus $H_1:\; \tau_j\neq 0$.
	\end{tcolorbox}
	
	De nombreux logiciels ne renvoient que le premier test de Fisher (celui pour $ Q_R $, c'est-à-dire les traitements) car les blocs du plan Gréco-Latin représentent des restrictions sur la randomisation et que par conséquent les autres tests peuvent ne pas être appropriés.
	
	
	\paragraph{ANOVA Multivariée (MANOVA)}\label{MANOVA}\mbox{}\\\\
	"\NewTerm{L'ANOVA Multivariée}\index{ANOVA Multivariée}" est une extension de l'ANOVA où nous supposons que les variables catégorielles sont linéairement dépendantes et nous testons s'il y a une différence significative dans les moyennes (variables supposées être de type continues) à travers les différents groupes sous l'interdépendance des variables explicatives catégoriques.

	Si ce n'est pas clair (...), la notation mathématique peut être plus explicite pour le lecteur. Rappelons d'abord que dans le cas de l'ANOVA canonique à $1$ facteur fixe $k$ niveaux, l'hypothèse nulle était de la forme:
	
	Pour la MANOVA à facteurs fixes, l'hypothèse nulle est:
	
	où on a donc des "vecteurs moyens" pour un nombre donné de variables de groupes dépendants en une quantité $k$.
	
	Ainsi explicitement pour les $k$ groupes et les $p$ variables catégorielles $p$:
	
	Comme le lecteur le verra dans les développements ci-dessous, le MANOVA à facteurs fixes repose sur neuf hypothèses:
	\begin{itemize}
		\item[H1.] Nos deux variables dépendantes (ou plus!) doivent être mesurées à un niveau continu (c'est-à-dire qu'il s'agit de variables d'intervalles ou de ratios)
	
		\item[H2.] Nos variable indépendantes doivent être constituée d'au moins deux groupes catégoriels, indépendants (non liés)
		
		\item[H3.] Nous devrions avoir l'indépendance des observations, ce qui signifie qu'il n'y a pas de relation entre les observations dans chaque groupe ou entre les groupes eux-mêmes
		
		\item[H4.] Nous devrions avoir une taille d'échantillon adéquate. Il faut au moins avoir plus de cas (par exemple, participants) dans chaque groupe de la variable indépendante que le nombre de variables dépendantes que nous analysons
		
		\item[H5.] Il n'y a pas de valeurs aberrantes univariées ou multivariées. Premièrement, il ne peut y avoir de valeurs aberrantes (univariées) dans chaque groupe de la variable indépendante pour aucune des variables dépendantes
		
		\item[H6.] Il y a une Normalité multivariée
		
		\item[H7.] Il existe une relation linéaire entre chaque paire de variables dépendantes pour chaque groupe de la variable indépendante
		
		\item[H8.] Il existe une homogénéité des matrices de variance-covariance
		
		\item[H9.] Il n'y a pas de multicolinéarité
	\end{itemize}
	Rappelons maintenant que dans l'ANOVA à un facteur, nous calculons la $p$-valeur du rapport des variances estimées en supposant que les vraies variances sont égales:
	
	La prochaine étape dans la compréhension de MANOVA est de reconnaître le fait que les mathématiciens sont paresseux et font parfois des statistiques d'ingénierie ... Nous simplifions donc d'abord ce ratio en multipliant la gauche et la droite par $ (k-1) / (N-1) $. Cela donne ce que nous appelons une "$A$-statistiques":
	
	Mais cela ne suffit pas car nous sommes toujours avec des variables indépendantes! L'idée est donc d'utiliser la matrice variance-covariance et d'obtenir une expression similaire en tant que ratio, mais en prenant le déterminant au préalable ... Cela nous amène à écrire une multivariée comme le test de Fisher pour les variances (......):
	
	mieux connue sous la forme suivante:
	
	où $ W $ et $ T $ sont respectivement les déterminants des matrices carrées de la somme des carrés des écarts intérieurs (Within: $W$) et globaux (Total: $T$). Ce qui signifie que si la partie "intérieur" (Between: $ B $) est vraiment grande, alors $\Lambda $ tend vers zéro. En revanche, si $B$ est très petit, $\Lambda$ tend vers $1$.
	
	Pour prouver que ce rapport ne dépend pas de la matrice de covariance des variables dépendantes, rappelons qu'en considérant la structure de $ W $ et $ B + W $ nous pouvons faire une décomposition spectrale (voir plus loin notre étude de l'analyse en composantes principales ) sous la forme:
	
	Il peut être démontré avec quelques approximations que $\Lambda$ suit une distribution nommée "distribution de Wilk" (de "Samuel Stanley Wilks").
	
	Il faut noter que le $\Lambda$ de Wilk peut être exprimé en fonction des valeurs propres de $W^{-1}B$. De la définition de $\Lambda$, il s'ensuit en utilisant les propriétés du déterminant (\SeeChapter{voir section Algèbre Lináire page \pageref{determinant}}) que:
	
	On reconnaît ici le déterminant des "équations aux valeurs propres" (\SeeChapter{voir section Algèbre Linéaire page \pageref{eigenvalue equations}}) sur $W^{-1}B$ mais avec $\lambda=1$. En effet:
	
	Donc dans notre cas ici nous avons $\lambda=-1$ donc:
	
	Dès lors:
	
	et par conséquent:
	
	En outre, il s'ensuit que:
	
	Lorsque le $\Lambda$ de Wilk approche $1$, nous avons montré que cela signifie que la différence dans les moyennes est négligeable. C'est le cas lorsque $\ln(\Lambda)$ approche $0$. Cependant, lorsque $\Lambda$ approche $0$ ou $\ln(\Lambda)$ approche $1$, cela signifie que la différence est grande. Par conséquent, une valeur élevée de $\ln(\Lambda)$ (c'est-à-dire proche de $0$) est une indication de l'importance de la différence entre les moyennes.
	\begin{tcolorbox}[title=Remarque,colframe=black,arc=10pt]
	La mauvaise nouvelle est qu'il existe en fait quatre tests multivariés différents qui sont réalisés à partir de la relation ci-dessus. La raison de quatre statistiques différentes et quatre approximations est que les mathématiques de la MANOVA deviennent si compliquées dans certains cas que personne n'a jamais été en mesure de les résoudre à notre connaissance...
	\end{tcolorbox}
	Voyons maintenant un exemple compagnon détaillé:
	\begin{tcolorbox}[colframe=black,colback=white,sharp corners]
	\textbf{{\Large \ding{45}}Exemple:}\\\\
	Nous considérons l'ensemble suivant de deux variables dépendantes sur trois groupes de variables indépendantes:
	\begin{table}[H]
		\centering
		\begin{tabular}{|c|c|c|c|c|c|c|c|}
		\cline{1-2} \cline{4-5} \cline{7-8}
		\multicolumn{2}{|c|}{\cellcolor[HTML]{C0C0C0}$K_1$} & & \multicolumn{2}{c|}{\cellcolor[HTML]{C0C0C0}$K_2$} & & \multicolumn{2}{c|}{\cellcolor[HTML]{C0C0C0}$K_3$} \\ \cline{1-2} \cline{4-5} \cline{7-8} 
		\cellcolor[HTML]{EFEFEF}$x_1$ & \cellcolor[HTML]{EFEFEF}$x_2$ & & \cellcolor[HTML]{EFEFEF}$x_1$ & \cellcolor[HTML]{EFEFEF}$x_2$ & & \cellcolor[HTML]{EFEFEF}$x_1$  & \cellcolor[HTML]{EFEFEF}$x_2$  \\ \cline{1-2} \cline{4-5} \cline{7-8} 
		$2$ & $3$ & & $4$ & $8$ & & $7$ & $6$ \\ \cline{1-2} \cline{4-5} \cline{7-8} 
		$3$ & $4$ & & $5$ & $6$ & & $8$ & $7$ \\ \cline{1-2} \cline{4-5} \cline{7-8} 
		$5$ & $4$ & & $6$ & $7$ & & $10$ & $8$ \\ \cline{1-2} \cline{4-5} \cline{7-8} 
		$2$ & $5$ & & & & & $9$ & $5$ \\ \cline{1-2} \cline{4-5} \cline{7-8} 
		 & & & & & & $7$ & $6$ \\ \cline{1-2} \cline{4-5} \cline{7-8} 
		\multicolumn{1}{|l|}{$\bar{x}_{11}=3$} & \multicolumn{1}{l|}{$\bar{x}_{21}=4$} & \multicolumn{1}{l|}{} & \multicolumn{1}{l|}{$\bar{x}_{12}=5$} & \multicolumn{1}{l|}{$\bar{x}_{22}=7$} & \multicolumn{1}{l|}{} & \multicolumn{1}{l|}{$\bar{x}_{13}=8.2$} & \multicolumn{1}{l|}{$\bar{x}_{23}=6.4$} \\ \cline{1-2} \cline{4-5} \cline{7-8} 
		\end{tabular}
	\end{table} 
	avec:
	
	Maintenant nous calculons:
	
	Nous avons alors par exemple pour le premier groupe:
	
	Ce qui nous donne dans notre cas (très facile à vérifier avec n'importe quel tableur):
	
	Et nous avons (nous remarquons que nous supposons que sur tous les $ x_1 $ et $ x_2 $ le nombre de mesures sont $ N_1 = N_2 = N $):
	
	\end{tcolorbox}
	\begin{tcolorbox}[colframe=black,colback=white,sharp corners]
	et dès lors:
	
	Après le calcul de la $p$-valeur est plus délicat..., nous demanderons au lecteur de se référer aux livres compagnons sur Minitab ou R pour les détails de calcul et la conclusion finale.
	\end{tcolorbox}
	Nous pourrions donc utiliser une MANOVA à facteurs fixes pour déterminer si le rappel des faits à court et à long terme d'élèves différait en fonction de trois longueurs de cours différentes (c.-à-d. que les deux variables dépendantes sont «rappel de mémoire à court terme» et «rappel de mémoire à long terme», tandis que la variable indépendante est «durée du cours», qui comporte quatre groupes indépendants: «$30$ minutes ", «$60$ minutes», «$90$ minutes» et «$120$ minutes»). Alternativement, une MANOVA à facteurs fixes pourrait être utilisée pour déterminer s'il existe une différence de salaire et de primes en fonction du type de diplôme (c'est-à-dire que les deux variables dépendantes sont «salaire» et «primes», tandis que la variable indépendante est «type de degré» , qui comprend cinq groupes: «études commerciales», «psychologie», «sciences biologiques», «ingénierie» et «droit»).
	
	Lorsqu'il existe une différence statistiquement significative entre les groupes de la variable indépendante, il est possible de déterminer quels groupes spécifiques sont significativement différents les uns des autres à l'aide de tests post hoc. Nous devons effectuer ces tests post-hoc car la MANOVA à facteurs fixes est un test statistique omnibus\footnote{"omnibus" est dérivé du mot latin "pour tous". En langage clair, nous pouvez interpréter un test omnibus ou une statistique omnibus comme un "test global" - il teste un certain nombre de choses à la fois.}\index {test omnibus} et ne peut donc pas vous dire quels groupes spécifiques sont significativement différents l'un de l'autre; il vous indique seulement qu'au moins deux groupes étaient différents.
	
	\subparagraph{Test du $T$ carré de Hotelling}\mbox{}\\\\
	Le "\NewTerm{test du $T$ carré de Hotelling}\index{test du $T$ carré de Hotelling}" (Hotelling, 1931) est le pendant multivarié du test $T$ de Student. Par exemple, disons que nous voulions comparer les performances de deux groupes d'élèves différents à l'école. Nous pourrions comparer des données univariées (par exemple, les scores moyens des tests) avec un test $T$ de Student. Ou, nous pourrions utiliser le $T$ carré de Hotelling pour comparer des données multivariées (par exemple, la moyenne multivariée des scores aux tests, GPA et les notes de classe).
	
	Le test $T$-carré de Hotelling est basé sur la distribution du $T^2$ de Hotelling et constitue la base de diverses cartes de contrôle multivariées en Génie Industriel. Comme le lecteur le constatera à la fin du développement présenté ci-dessous, le test $T$ carré de Hotelling est simplement un cas particulier de MANOVA, dans lequel seuls deux lots (groupes) sont comparés!
	
	Plus techniquement, le test $T^2$ de Hotelling permet d'analyser si les mesures de variables aléatoires censées suivre conjointement une loi Normale multivariée et donc ayant une corrélation possible (matrice de variance-covariance) s'écartent considérablement de la Normalité ou non.
	
	Pour introduire ce test, rappelons que nous avons démontré dans la section de Statistiques que la Loi Normale multivariée pouvait s'écrire:
	
	On voit que le terme suivant entre parenthèses:
	
	donnant un scalaire peut être vu comme la variable aléatoire unique de la distribution bivariée (comme nous pouvons le faire avec le carré de la variable aléatoire dans la loi Normale univariée centrée réduite). Donc, pour une valeur donnée de $ D $, nous sommes "à une certaine hauteur" de la distribution gaussienne multidimensionnelle.
	
	Ce qui nous intéressera ici, c'est que lorsque nous travaillons avec une distribution multivariée, certaines variables ont moins de poids que d'autres en raison de leur grand écart-type et les variables corrélées ont tendance à réduire l'amplitude of $D$.
	
	Par conséquent, un sujet très intéressant en pratique industrielle (carte de contrôle Hotelling et régressions) est alors de savoir qu'elle est la loi de distribution qui décrit:
	
	Alors traitons maintenant de ce sujet!
	
	La "\NewTerm{distance de Mahalanobis}\index{distance de Mahalanobis}" entre deux points $\vec{x}$ et $\vec{y}$ est définie comme:
	
	Ainsi, la distance de Mahalanobis au carré d'un vecteur aléatoire $\vec{x}=\mathcal{N}(\vec{\mu},\Sigma)$ et le centre $\vec{\mu}$ d'une distribution gaussienne multivariée est défini comme:
	
	où $\Sigma$ est une matrice de covariance de dimensions $n\times n$ et $\vec{\mu}$ est le vecteur des moyennes. Afin d'obtenir une représentation différente de $D$, on peut d'abord effectuer une décomposition en valeurs propres sur $\Sigma^{-1}$ qui est donnée par (\SeeChapter{voir section Algèbre Linéaire page \pageref{inverse eigendecompsosition}}):
	
	où pour rappel $\vec{u}_k$ est le $k$ ème vecteur propre de la valeur propre correspondante $\lambda_k $. Injecter la relation ci-dessus dans:
	
	nous donne:
	
	En utilisant les propriétés des matrices transposées (\SeeChapter{voir section Algèbre Linéaire page \pageref{transposed matrix}}) nous obtenons:
	
	où $Y_{k}$ est une nouvelle variable aléatoire basée sur une transformation linéaire affine du vecteur aléatoire $\vec{x}$. En effet, nous avons:
	
	Si nous posons:
	
	dès lors nous obtenons:
	
	On peut alors que $ Y_k $ est le résultat d'une transformation affine de $Z$.
	
	Notons que $Y_{k}$ est maintenant une variable aléatoire tirée d'une distribution Normale univariée $Y_{k}= \mathcal{N}\left(0, \sigma_{k}^{2}\right)$, où, selon:
	
	Par conséquent, nous obtenons:
	
	Si nous insérons dans cette dernière relation la suivante:
	
	nous obtenons:
	
	puisque tous les vecteurs propres $\vec{u}_{i}$ sont orthonormés par paires, les produits scalairees $\vec{u}_{k}^T \vec{u}_{j}$ et $\vec{u}_{j}^T \vec{u}_{k}$ sera nul pour $j \neq k$. Uniquement pour le cas $ j = k $ on obtient:
	
	puisque la norme $\left\|\vec{u}_{k}\right\|$ d'un vecteur propre orthonormé est égale à $1$. La distance de Mahalanobis au carré peut alors être exprimée comme suit:
	
	Maintenant, la distribution du chi carré avec $n$ degrés de liberté est exactement définie comme étant la distribution d'une variable qui est la somme des carrés de $n$ variables aléatoires étant Normalement (loi Normale centrée réduite) distribuées. Par conséquent, $D$ est un chi carré distribué avec $n$ degrés de liberté.
	
	Donc, si les variables aléatoires sont réellement conjointement distribuées Normalement selon une matrice de corrélation $\Sigma$ nous avons alors:
	
	qui est alors nommée la "\NewTerm{statistique du $T^2$ de Hotelling}\index{statistique du $T^2$ de Hotelling}".
	
	Dès lors, à un seuil de confiance donné, nous avons:
	
	c'est-à-dire la probabilité associée à une certaine hauteur (isocline) dans le cas de la distribution Normale bivariée. Ainsi chaque quantile de la loi du $\chi^2$ correspond par exemple dans le cas bivarié à une isocline elliptique dans le cas général (vu de dessus) ou à un cercle lorsque la corrélation est nulle.
	
	Donc ici ce que testons est:
	\[
	H_{0}:\left(\begin{array}{c}
	\mu_{11} \\	\mu_{12} \\	\vdots \\ 	\mu_{1 p}
	\end{array}\right)=\left(\begin{array}{c}
	\mu_{21} \\	\mu_{22} \\	\vdots \\ 	\mu_{2 p}
	\end{array}\right) \text { contre } H_{a}:\left(\begin{array}{c}
	\mu_{11} \\ \mu_{12} \\ \vdots \\ \mu_{1 p}
	\end{array}\right) \neq\left(\begin{array}{c}
	\mu_{21} \\ \mu_{22} \\ \vdots \\ \mu_{2 p}
	\end{array}\right)
	\]
	Ou, en d'autres termes:
	\begin{center}
	$H_{0}: \mu_{11}=\mu_{21}$ et $\mu_{12}=\mu_{22}$ et $\ldots$ et $\mu_{1 p}=\mu_{2 p}$
	\end{center}
	Pour l'hypothèse nulle, cela ne convient que si les moyennes de population sont identiques pour toutes les variables. L'alternative est qu'au moins une paire de ces moyennes est différente. Ceci est exprimé ci-dessous:
	\begin{center}
		$H_{a}: \mu_{1 k} \neq \mu_{2 k}$ pour au moins un $k \in\{1,2, \cdots, p\}$
	\end{center}
	
	\begin{tcolorbox}[colframe=black,colback=white,sharp corners]
	\textbf{{\Large \ding{45}}Exemple:}\\\\
	Considérons le scénario suivant:
	\begin{figure}[H]
		\centering
		\includegraphics[scale=0.7]{img/arithmetics/hotelling_test.jpg}
	\end{figure}
	Les valeurs sont données par:
	
	et l'hypothèse nulle sera:
	
	En supposant que les données proviennent d'une distribution Normale multivariée et d'observations indépendantes (n'oubliez pas que $S$ est la notation traditionnelle pour l'estimation de $\Sigma$):
	
	Nous avons:
	
	Dès lors:
	
	La valeur dont nous avons besoin pour un seuil de $5\%$ est:
	
	Nous pouvons comparer notre $T^2$ au $5.991$. Nous voyons déjà que nous ne pouvons pas rejeter $H_0$ (voyez à quel point $\vec{\bar{x}}$ et $\vec{\mu}$ sont proches dans la figure ci-dessus).
	\end{tcolorbox}
	
	\begin{tcolorbox}[colframe=black,colback=white,sharp corners]
	Alternativement, nous pourrions calculer facilement la $p$-valeur avec n'importe quel tableur. Cela nous conduit alors à une $p$-valeur de $0.878$.
	\end{tcolorbox}

	
	\pagebreak
	\subsubsection{Tests d'équivalence}
	Comme le lecteur l'aura probablement compris à partir des paragraphes précédents, les tests d'hypothèses (NHST) appliqués à la recherche de différences entre groupes ne permettent pas techniquement de conclure une équivalence simplement parce que nous ne rejetons pas l'hypothèse nulle $H_0$. Pourtant, nous avons également montré que si la puissance du test était généralement supérieure à $80\%$, cela ne posait pas de problème de considérer l'hypothèse nulle $H_0$ comme vraie mais le problème est que pour cela, nous devons la plupart du temps considérer de grands échantillons et en pratique ce n'est pas toujours faisable et cela peut aussi conduire à des absurdités puisque les mathématiques montrent qu'avec de grands échantillons, on rejette presque systématiquement l'hypothèse nulle $H_0$.
	
	La $p$-valeur ne peut être utilisée que pour révéler statistiquement a posteriori le rejet de l'hypothèse nulle $H_0$ dans la grande majorité des cas. Pour résumer ce problème: \textit{Une absence de preuve n’est pas une preuve d’absence}. Exprimé en d'autres termes dans un cas très courant cela donne (exemple particulier!) que le non-rejet de l'hypothèse nulle $H_0$ d'égalité de deux moyennes (par exemple) n'implique pas l'égalité des moyennes!!!
	
	\begin{tcolorbox}[title=Remarque,colframe=black,arc=10pt]
	Certaines personnes déclarent assez souvent à tort que \og Une absence de preuve n’est pas une preuve d’absence. \fg{} Pourquoi font-ils ça ? Peut-être parce qu'ils ne savent pas mieux ? Et avec la preuve ci-dessous, ils ont la possibilité de savoir avec certitude que "L'absence de preuve EST une preuve d'absence !" et rien d'autre....\\
	
	Definitions:
	\begin{itemize}
		\item[D1.] $A$ est l'évidence de $B$: $P(B \mid A)>P(B \mid \neg A)$

		\item[D2.] Absence d'évidence: $a=\neg A$

		\item[D3.] Absence $b=\neg B$
	\end{itemize}
	
	\end{tcolorbox}
	
	Avec l'avènement des médicaments génériques des années 1960 dans le domaine pharmaceutique, l'importance de mettre en évidence les équivalences (on parle plutôt de "\NewTerm{bioéquivalences}\index {bioéquivalence}") a pris une place croissante et des organisations comme la F.D.A. (Federal Drug Administration \index{Federal Drug Administration} aux États-Unis), l'EMA (Agence Européenne des Médicaments\index{Agence européenne des médicaments}) ou l'OMS (Organisation Mondiale de la Santé\index {Organisation mondiale de la santé}) ont des directives communes qui mènent à l'approbation de générique seulement si l'équivalence est démontrée sous certaines conditions empiriques mais au moins normalisée selon un consensus d'experts.

	Ainsi, le plus souvent les trois conditions suivantes sont requises \footnote{Toujours en respectant les règles de publication scientifique introduites au début de ce livre}:
	\begin{enumerate}
		\item Un critère de comparaison est requis

		\item Un intervalle de confiance est requis

		\item Une limite a priori de bioéquivalence est requise
	\end{enumerate}
	Voyons un exemple compagnon pour en comprendre le concept:
	\begin{tcolorbox}[colframe=black,colback=white,sharp corners]
	\textbf{{\Large \ding{45}}Exemple:}\\\\
	Considérons le test $T$ de Student de la différence des moyennes de deux échantillons non appariés $\{$Test, Référence$\}$ que nous avions démontré ci-dessus:
	
	On connaît alors les deux bornes suivantes de l'intervalle de confiance:
	
	\begin{tcolorbox}[title=Remarque,colframe=black,arc=10pt]
	On parle dans ce cas particulier de test "TB" pour "\NewTerm{bi-test unilatéral}\index{bi-test unilatéral}".
	\end{tcolorbox}
	Nous supposerons que la F.D.A. exige:
	
	Considérons que les mesures nous ont donné:
	
	\end{tcolorbox}
	
	\begin{tcolorbox}[colframe=black,colback=white,sharp corners]
	Par conséquent, nous avons les limites d'équivalence qui seront:
	
	Les intervalles de confiance sont donnés par:
	
	Dès lors:
	
	Comme nous avons:
	
	L'équivalence est alors considérée par consensus d'experts comme "prouvée".
	\end{tcolorbox}
	

	\subsubsection{Test C de Cochran}
	Le test C de Cochran a pour but de vérifier l'homogénéité des variances pour plusieurs populations. Il s'agit d'un test préliminaire ou ultérieur (post hoc) utile à faire avant de faire une analyse ANOVA équilibrée (équilibrée) et recommandé par la norme ISO 5725 (comme le test de Tukey que nous verrons bien plus loin)!
	
	Bien que l'idée du test C de Cochran soit empirique, elle est néanmoins intuitive, tout comme les définitions des tests de Dixon et de Grubbs. Pourquoi alors présentons-nous dans ce livre en détail la preuve du test C de Cochran alors que nous avons mentionné que nous ne ferions pas cela pour le test de Grubbs et Dixon car l'approche de ces derniers était également empirique? La raison est simple en fait: le test de Grubbs et Dixon nécessite, au moins à notre connaissance, des simulations de Monte Carlo pour déterminer les valeurs critiques de rejet ou d'acceptation de l'hypothèse nulle, alors que la valeur critique du test C de Cochran peut être obtenue relativement facilement analytiquement.
	
	Cela dit ... nous définissons le test C de Cochran par le ratio:
	
	où les $ S_i $ sont les variances sans biais des $N$ différentes sources de données, chacune composée de $n$ échantillons et l'hypothèse nulle est intuitivement l'égalité des variances par rapport à l'hypothèse alternative que l'une des variances est trop grande ( c'est-à-dire: mauvais) et écartée car considérée comme une variance aberrante.
	
	L'ISO 5725 recommande de répéter ce test jusqu'à ce qu'il n'y ait plus de variance aberrante (donc trop grande ET loin des autres variances).

	Pour déterminer la valeur critique, inversons la définition du test C de Cochran et faisons quelques manipulations algébriques de base:
	
	Nous notons que le deuxième terme de la dernière égalité ressemble à peu près à une loi de Fisher. Comme la loi de Fisher n'est pas stable pour l'addition, nous devrions trouver un moyen de transformer le terme:
	
	en une seule variance. L'idée est alors relativement simple mais comme toujours fallait encore quelqu'un pour y penser ... Nous savons que les $S_i$ sont des équations de variances non biaisées avec un facteur $1/(n-1)$. Donc si les $N$ échantillons (niveaux) sont indépendants, la variance globale est alors  par stabilité de la loi Normale et en prenant les notations traditionnelles de l'ANOVA:
	
	Dès lors:
	
	On reconnaît dans la dernière égalité le rapport du carré de deux variances. Nous avons alors identiquement à ce que nous avons prouvé dans notre étude de l'ANOVA à un facteur fixe sans réplications:
	
	et alors il vient:
	
	qui est donc indépendant de $j$ et donc le test C de Cochran unilatéral gauche (puisque par définition le rapport de Cochran doit être le plus petit possible) aura pour valeur critique:
	
	\label{bonferroni-sidak multiple tests correction}Si nous considérons un test avec un niveau de signification de $1-\alpha$ (correspondant donc à la probabilité cumulée de ne pas commettre d'erreur de type I) et nous le réitérons indépendamment une seconde fois. Alors, si les tests sont indépendants, selon l'axiome des probabilités, la probabilité de ne pas commettre d'erreur de type I sera donnée par le produit des probabilités:
	
	et ainsi de suite pour $n$ tests. On remarque vite que la probabilité cumulée de ne pas commettre d'erreur de type I diminue très rapidement. Par exemple, pour $10$ tests répétés indépendants avec un niveau $5\%$, alors nous avons:
	
	ce qui est catastrophique! Donc si l'on veut un niveau de confiance sur des tests répétés d'une certaine valeur que l'on notera $\alpha_r$, il est clair qu'il faut résoudre l'équation suivante:
	
	Donc (relation parfois nommée "équation de Šidàk\index{\'equation de Sidak}"):
	
	et avec une développement de Taylor au second ordre, il vient (\SeeChapter{voir section Séquences et Séries page \pageref{taylor series}}):
	
	que nous appelons "\NewTerm{approximation de Bonferroni}\index{approximation Bonferroni}" ou parfois aussi "\NewTerm{approximation Boole}\index{approximation de Boole}" ou encore "\NewTerm{approximation de Dunn}\index{approximation de Dunn}". 
	\begin{tcolorbox}[title=Remarque,colframe=black,arc=10pt]
	En statistiques, le "\NewTerm{taux d'erreur de groupe}\index{taux d'erreur de groupe}" (TER) est la probabilité de faire une ou plusieurs fausses découvertes ou des erreurs de type I lors de l'exécution de plusieurs tests d'hypothèses\index{tests d'hypothèses multiple} (généralement pendant les méta-analyses).
	\end{tcolorbox}
	Donc au final, nous avons pour le test C de Cochran:
	
	que nous pouvons calculer avec la version anglaise de Microsoft Excel 14.0.6123 en utilisant la formule suivante:
	\begin{center}
	\texttt{=1/(1+(N-1)/FINV(ALPHA/N,n-1,(N-1)*(n-1)))}
	\end{center}
	\begin{tcolorbox}[title=Remarque,colframe=black,arc=10pt]
	Nous avons avantage ici à supposer que tous les tests sont indépendants les uns des autres. Dans les applications pratiques, ce n'est souvent pas le cas. En fonction de la structure de corrélation des tests, la correction de Bonferroni pourrait être extrêmement conservatrice, conduisant à un taux élevé de faux négatifs !!!
	\end{tcolorbox}
	
	\subsubsection{Taux d'erreur de groupe ($p$-valeurs ajustées)}
	Le "\NewTerm{taux d'erreur de groupe}\index{taux d'erreur de groupe}\label{family-wise error rate}" (TDEG), parfois aussi appelé "\NewTerm{taux d'erreur de famille} (ou encore "\NewTerm{taux d'erreur par expérience}\index{taux d'erreur par expérience}") est la probabilité de commettre des erreurs de type I dans l'ensemble (généralement pendant les méta-analyses).
	
	Le "\NewTerm{taux de fausses découvertes}\index{taux de fausses découvertes}" (TDFD) est la proportion attendue de faux rejets parmi tous les rejets (généralement aussi pendant les méta-analyses).
	
	Avant de passer à la partie mathématiques, rappelons le cas célèbre de l'équipe de Nosek qui a invité des chercheurs à participer à un projet d'analyse de données participatives. La configuration était simple. Les participants ont tous reçu le même ensemble de données et le même message: les arbitres de football donnent-ils plus de cartons rouges aux joueurs à la peau foncée qu'aux joueurs à la peau claire? Il leur a ensuite été demandé de soumettre leur approche analytique aux commentaires des autres équipes avant de plonger dans l'analyse.
	
	Vingt-neuf équipes totalisant $61$ analystes ont participé à cette étude. Les chercheurs ont utilisé une grande variété de méthodes, allant - pour ceux d'entre vous intéressés par le gore méthodologique - des simples techniques de régression linéaire aux régressions complexes à plusieurs niveaux et aux approches bayésiennes. Ils ont également pris différentes décisions sur les variables secondaires à utiliser dans leurs analyses.

	Malgré l'analyse des mêmes données, les chercheurs ont obtenu une variété de différents résultats. Vingt équipes ont conclu que les arbitres de football donnaient plus de cartons rouges aux joueurs à la peau foncée, et neuf équipes n'ont trouvé aucune relation significative entre la couleur de la peau et les cartons rouges.
	\begin{figure}[H]
		\centering
		\includegraphics[width=1.0\textwidth]{img/arithmetics/repetability.jpg}
		\caption{Mêmes données, conclusions différentes (objectif de la méta-analyse)}
	\end{figure}
	La variabilité des résultats n’était pas due à une fraude ou à un travail bâclé. Il s'agissait d'analystes hautement compétents, motivés pour trouver la la conclusion la plus vraisemblable, a déclaré Eric Luis Uhlmann, psychologue à l'école de commerce Insead de Singapour et l'un des chefs de projet. Même les chercheurs les plus qualifiés doivent faire des choix subjectifs qui peuvent ont un impact énorme sur le résultat qu'ils trouvent.
	
	C'est pourquoi il est essentiel de savoir comment gérer plusieurs $p$valeurs et surtout lorsque des vies sont concernées par le résultat (rappelez-vous la méta-analyse de l'influence de l'hydroxychloroquine pendant la période du COVID-19 en 2020).

	La méthode la plus couramment utilisée qui contrôle le TDEG au niveau $\alpha$ est la méthode de Bonferroni vue juste plus haut. Rappelons cette dernière!
	\begin{itemize}
		\item La probabilité de faire une erreur de Type I en ne traitant qu'un test spécifique est notée $\alpha[\text{PT}]$ (appelée "\NewTerm{alpha par test}"). Elle est également nommé "\NewTerm{alpha d'étendue test}".
	
		\item La probabilité de faire au moins une erreur de type I pour la globalité d'une famille de tests est notée $\alpha[\text{PF}]$ (appelée "\NewTerm {alpha par famille de tests}"). Elle est également appelée "\NewTerm{alpha d'étendue de famille de tests}" ou plus rarement "\NewTerm{alpha d'étendue expérimentale}".
	\end{itemize}
	Lors de notre étude du test C de Cochran précédemment (voir page \pageref{bonferroni-sidak multiple tests correction}) nous avons vu que la correction de Bonferroni pour plusieurs tests était notée par (en supposant l'indépendance!):
	
	où $n$ est le nombre de tests.
	
	Également au cours de notre étude du test C de Cochran plus haut, nous avons déduit l'approximation de Taylor du premier ordre suivante:
	 
	Les équations de Sidak-Bonferonni peuvent être utilisées pour trouver la valeur de $\alpha[\text{PT}]$ lorsque $\alpha[\text{PF}]$ est fixé. Par exemple, supposons que nous voulions effectuer quatre tests indépendants, et que nous voulions limiter le risque de faire au moins une erreur de type I à une valeur globale de $\alpha[\text{PF}]= 0.05$, nous considérerons un test comme significatif si sa probabilité associée est inférieure à (dans le cas de quatre tests associés):
	
	Avec l'approximation de Bonferroni, un test considéré comme significatif si sa probabilité associée est inférieure à:
	
	
	\begin{tcolorbox}[title=Remarque,colframe=black,arc=10pt]
	Parfois, nous pouvons avoir:
	 
	C'est pourquoi il n'est pas rare dans la littérature de retrouver cette dernière relation sous la forme:
	
	ou:
	
	Ce qui signifie simplement que si $n\alpha[\text{PT}]>1$, alors nous remplaçons le résultat par $1$.
	\end{tcolorbox}
	
	\begin{tcolorbox}[colframe=black,colback=white,sharp corners]
	\textbf{{\Large \ding{45}}Exemple:}\\\\
	Considérons quatre hypothès nulles $H_{1}, \ldots, H_{4}$ avec les $p$-valeurs respectives:
	
	à tester au niveau de signification $\alpha = 0.05$. Nous testons d'abord $0.01\cdot 4=0.04<0.05$: nous rejetons $H_{0(1)}$. Après avoir testé $0.04\cdot 4=0.16>0.05$: nous ne rejetons pas $ H_{0(2)}$. Après avoir testé $0.03\cdot 4=0.12>0.05$: nous ne rejetons pas $H_{0(3)}$. Enfin, nous testons $0.005\cdot 4=0.02<0.05$: nous rejetons $H_{0(4)}$.
	\end{tcolorbox}
	
	La méthode Bonferroni est garantie de contrôler le TDER, mais elle pose un gros problème. Elle réduit considérablement notre capacité à détecter les différences réelles. Par exemple, supposons que la taille de l'effet est de $2$ et que nous effectuons un test $T$ de Student, rejetant l'hypothès nulle pour $p<0.05$. Avec $10$ observations par groupe, la puissance est de $99\%$. Supposons maintenant que nous ayons $1'000$ tests et que nous utilisions la méthode Bonferroni. Cela signifie que pour rejeter l'hypothès nulle, nous avons besoin de $p<0000005$. La puissance n'est plus que de $29\%$. Si nous avons $10'000$ tests (ce qui est petit pour les études génomiques), la puissance n'est que de $10\%$.
	
	\paragraph{Correction de Holm-Bonferroni}\mbox{}\\\\
	La méthode de "\NewTerm{correction de Holm-Bonferroni}\index{correction de Holm-Bonferroni}" est la suivante:
	\begin{itemize}
		\item Soit $H_{1}, \ldots, H_{m}$ une famille (groupe) de $m$ hypothèses nulles et $p_{1}, \ldots, p_{m}$ les $p$-valeurs correspondantes.
		
		\item Nous commençons par classer les $p$-valeurs (de la plus petit à la plus grande) $p_{(1)} \ldots p_{(m)}$ et nous laissons les hypothèses associées être $H_{(1)}, \ldots, H_{(m)}$.
		
		\item Pour un niveau de signification donné $\alpha$, soit $k$  l'indice minimal tel que $p_{(k)}>\frac{\alpha}{m+1-k}$.

		\item Nous rejetons $H_{(1)}, \ldots, H_{(k-1)}$ et ne rejetons pas $H_{(k)}, \ldots, H_{(m)}$ $\cdot$ 
		
		\item Si $ k = 1 $ alors nous ne rejetons aucune des hypothèses nulles et si aucun $ k $ n'existe alors nous rejetons toutes les hypothèses nulles.
	\end{itemize}
	Cette méthode garantit que le TDEG $\leq \alpha$, où pour rappel TDEG est le taux d'erreur de groupe.
	 
	 \begin{dem}
	 Holm-Bonferroni contrôle le TDEG comme suit: Soit $H_{(1)}, \ldots, H_{(m)}$ une famille d'hypothèses, et $p_{(1)} \leq p_{(2)} \leq \ldots \leq p_{(m)}$ les $p$-valeurs triées. Soit $I_{0}$ l'ensemble des indices correspondant aux vraies (inconnues) hypothèses nulles, ayant $m_{0}$ membres.

	Supposons que nous rejetions à tort une hypothèse vraie. Nous devons prouver que la probabilité de cet événement est au plus de $\alpha$.
	
	Soit $ h $ la première hypothèse vraie rejetée (la première dans l'ordre donné par le test de Holm-Bonferroni). Alors  $h-1 \leq m-m_{0}$ sont toutes de fausses hypothèses rejetées et $h-1 \leq m-m_{0}$. De là, on obtient:
	 
	 étant donné que $h$ est rejeté nous avons: 
	 
	 par définition du test. En utilisant la relation précédente, le côté droit est au plus $\alpha/m_0$. Ainsi, si nous rejetons à tort une hypothèse vraie, il doit y avoir une hypothèse vraie avec une $p$-valeur au plus égale à $\alpha/m_0$.
	 
	 Définissons donc la variable aléatoire $A=\left\{P_{i} \leq \frac{\alpha}{m_{0}} \text { for } i \in I_{0}\right\}$ . Quel que soit l'ensemble (inconnu) d'hypothèses vraies $I_{0}$, nous avons $P(A) \leq \alpha$ (par les inégalités de Bonferroni). Par conséquent, la probabilité de rejeter une hypothèse vraie est au plus $\alpha$.
	 \begin{flushright}
		$\blacksquare$  Q.E.D.
	 \end{flushright}
	 \end{dem}
	 
	\begin{tcolorbox}[colframe=black,colback=white,sharp corners]
	\textbf{{\Large \ding{45}}Exemple:}\\\\
	Considérons quatre hypothèses nulles $H_{1}, \ldots, H_{4}$ avec $p$-valeurs:
	
	à tester au niveau de signification $\alpha=0.05$. Puisque la procédure est progressive (step-down), nous testons d'abord $H_{4}=H_{(1)}$, qui a la plus petite $p$-valeur$p_{4}=p_{(1)}=0.005$. La $p$-valeur est comparée à:
	 
	l'hypothèse nulle est rejetée et nous passons à la suivante. Puisque $p_{1}=p_{(2)}=0.01$ et:
	 	
	nous rejetons $H_{1}=H_{(2)}$ ainsi que $0.01<0.167$ et continuons. L'hypothèse suivante $H_{3}$ n'est pas rejetée puisque $p_{3}=p_{(3)}=0.03$ et:
	
	alors $0.03>0.025$. Nous avons notre $k$ qui est égal à $k=3$. Nous arrêtons de tester et concluons que $H_{1}$ et $H_{4}$ sont rejetés et que $H_{2}$ et $H_{3}$ ne sont pas rejetés lors du contrôle du taux d'erreur de groupe au niveau $\alpha=0.05$. Notez que bien que $p_{2}=p_{(4)}=0.04<0.05=\alpha$ s'applique, $H_{2}$ n'est pas rejeté. En effet, la procédure de test s'arrête une fois qu'un échec de rejet se produit.
	\end{tcolorbox}
	
	\paragraph{Correction de Holm–Šidák}\mbox{}\\\\
	Comme nous le savons, la correction de Bonferroni vue plus haut n'est qu'une approximation (série de Taylor du premier ordre) de la correction de Šidák.
	
	Il est donc possible de remplacer:
	
	avec: 
	
	résultant en un test légèrement plus puissant et nommé la "\NewTerm{Correction de Holm–Šidák}\index{correction de Holm–Šidák}".
	
	\subsubsection{Tests d'adéquation (tests de qualité d'ajustement)}\label{goodness of fit tests}
	La qualité de l'ajustement (QdA) d'un modèle statistique décrit dans quelle mesure il s'adapte à un ensemble d'observations. Les mesures de qualité de l'ajustement résument généralement l'écart entre les valeurs observées et les valeurs attendues dans le modèle en question. De telles mesures peuvent être utilisées dans le test d'hypothèses statistiques, par ex. pour tester la Normalité des résidus, pour tester si deux échantillons sont tirés de distributions identiques (voir le test de Kolmogorov - Smirnov plus loin), ou si les fréquences des résultats suivent une distribution spécifiée (voir le test du chi carré de Pearson ci-dessous).
	
	\paragraph{Test de QdA du khi-carré de Pearson}\mbox{}\\\\
	Nous étudierons ici notre premier test non paramétrique de QdA, certainement l'un des plus connus et des plus simples (qui ne s'applique qu'aux données non censurées à notre connaissance).
	
	Pour introduire ce test, supposons qu'une variable aléatoire suit une distribution de probabilité $P$. Si l'on tire un échantillon de la population correspondant à cette loi, la distribution observée, nommée "\NewTerm{distribution d'échantillonnage}\index{distribution d'échantillonnage}", s'écarte toujours plus ou moins de la distribution théorique, compte tenu des fluctuations d'échantillonnage.
	
	Généralement, on ne connaît pas la forme de la loi $ P $, ni la valeur de ses paramètres. C'est la nature du phénomène étudié et l'analyse de la distribution observée qui permettent de choisir une loi susceptible d'être adéquate et ensuite d'estimer les paramètres.
	
	Les différences entre la loi théorique et la distribution observée peuvent être attribuées soit aux fluctuations d'échantillonnage, soit au fait que le phénomène ne suit pas, en réalité, la loi supposée.
	
	Fondamentalement, si les écarts sont suffisamment petits, nous supposerons qu'ils sont dus à des fluctuations aléatoires et nous accepterons (nous ne rejeterons pas en fait!) la loi supposée. Au contraire, si les écarts sont trop importants, on en concluera qu'ils ne peuvent pas être expliqués uniquement par les fluctuations et que le phénomène ne suit pas la loi supposée (nous rejeterons l'hypothèse nulle).
	
	Pour évaluer ces lacunes et prendre une décision, nous avons besoin de:
	\begin{enumerate}
		\item Définir la mesure de la distance entre la distribution empirique et le théorique résultant de la loi de retenue.
		
		\item Déterminer la loi de probabilité suivie de la variable aléatoire étant donnée la distance (en fait malheureusement pas rejeter l'hypothèse nulle).
		
		 \item Énoncer une règle de décision pour dire à partir de la distribution observée, si la loi choisie est acceptable ou non.
	\end{enumerate}
	Dans un premier temps, nous aurons besoin à cet effet du théorème central limite et en second lieu rappelons que lors de la construction de la loi normale, nous avons prouvé que la variable:
	
	suit une distribution Normale centrée réduite lorsque $n$ s'approche de l'infini (condition de Laplace) et que la probabilité $p$ était très petite.
	
	En pratique, l'approximation est tout à fait acceptable.,. dans certaines entreprises ... et quand $np>5$ et $p\leq 0.5$ alors (c'était l'un des termes qu'il devait tendre vers zéro quand on avait fait la preuve):
	
	Par exemple dans les deux figures ci-dessous où nous avons représenté les lois binomiales approchées par les lois Normales associées, nous avons à gauche $n=60,p=1/6,np=10$ et à droite $n=40,p=0.05,np=2$:
	\begin{figure}[H]
		\centering
		\includegraphics{img/arithmetics/binomial_normal_approximation.jpg}
		\caption{Approche des fonctions binomiales par les fonctions Normales associées}
	\end{figure}
	Enfin, rappelez-vous que nous avons prouvé que la somme des carrés de $n$ variables aléatoires linéairement indépendante et distribuées selon une Normale centrée réduite suivait alors une loi de chi-carré avec $n$ degrés de liberté noté $\chi^2(n)$.
	
	Considérons maintenant une variable aléatoire $ X $ qui suit une fonction de distribution théorique (continue ou discrète) $ P $ et tirons un échantillon de taille $ n $ dans la population correspondant à cette loi $ P $.
	
	Les $n$ observations sont distribuées le long de $k$ termes (valeurs de classe) $C_1, C_2, ..., C_k$, dont les probabilités $p_1, p_2, ..., p_k$ sont déterminées par la fonction de distribution $ P $ (reportez-vous à l'exemple avec la ligne droite Henry  page \pageref{droite de Henry}).
	
	Pour chaque modalité $ C_i $, la taille de l'échantillon empirique est une variable aléatoire binomiale $k_i$ de paramètres:
	
	Ce nombre $k_i$ correspond en effet au nombre d'événements réussis: "résultat égal à la modalité $C_i$" de probabilité $p_i$, obtenu lors de l'échantillonnage sur $n$ éléments du lot expérimental (et non dans la population de la loi théorique comme avant!).
	
	Nous avons prouvé plus haut au cours de notre étude de la loi binomiale que son espérance était:
	
	et représente pour le coup la taille théorique d'échantillon de la modalité $C_i$ et que sa variance était donnée par (lorsque $n$ est très grand et $p_i$ très petite):
	
	Son écart-type est donc:
	
	Dans ces conditions, à condition que la modalité $C_i$ ait une taille $np_i$ au moins égale à  $5$, la variable centrée réduite:
	
	entre la taille de l'échantillon empirique et théorique peut être approximativement considérée comme une variable Normale centrée réduite comme nous l'avons prouvé plus haut. Cette relation est également nommée "\NewTerm{résiduel de Pearson}\index {résiduel de Pearson}" pour la valeur $i$, et elle compare les dénombrements observés aux dénombrements attendus. Le signe (positif ou négatif) indique si la fréquence observée $k_i$ est supérieure ou inférieure à la valeur ajustée sous le modèle, et la magnitude indique le degré d'écart. Lorsque les données ne correspondent pas à un modèle, l'examen des résidus de Pearson aide souvent à diagnostiquer où le modèle a échoué.
	
	Nous définissons maintenant la variable:
	
	où $k_i$ est souvent nommé la "\NewTerm{fréquence expérimentale}\index{fréquence expérimentale}" et $np_i$ la "\NewTerm{fréquence théorique}\index {fréquence théorique}".
	
	Si nous prenons le carré, c'est que dans une simple somme certains termes s'annuleraient par des effets opposés et masqueraient ainsi les différences, si nous prenons la somme des valeurs absolues, le tableau statistique de $D$ serait difficile à construire et le test ne serait pas très robuste en raison de la faible écart des distances. Le carré permet non seulement d'avoir un tableau statistique utilisable de $D$ qui est simple puisqu'il est basé sur une loi avec un seul paramètre, comme nous le verrons, et le carré augmente aussi suffisamment la robustesse du test.
	
	Notez que cette variable est aussi parfois (un peu malheureusement) désignée par:
	
	ou plus souvent (le $E_i$ ci-dessous n'est pas le même que le précédent $E_i$ juste au-dessus !!!):
	
	où:
	
	est communément appelé, comme déjà mentionné ci-dessus, un "\NewTerm{résidu de Pearson}\index{résidu de Pearson}".
	
	Cette variable $D$, somme des variables $E_i$ au carré, fournit une mesure d'une «distance» entre la distribution empirique et théorique et la distribution empirique. Notons cependant qu'il ne s'agit pas d'une distance au sens mathématique habituel (topologique).
	
	Rappelons que $ D $ peut donc aussi s'écrire:
	
	$D$ est donc la somme des carrés de $N$ variables aléatoires Normales centrées réduites liées par la relation linéaire simple:
	
	où $n$ est la taille de l'échantillon. Donc $D$ suit une distribution chi-carré, mais avec $N-1$ degrés de liberté, donc un degré de moins à cause de la relation linéaire unique entre eux! En effet, rappelons que le degré de liberté est le nombre de variables indépendantes dans la somme et pas seulement le nombre de termes additionnés!
	
	Dès lors:
	
	Nous appelons ce test le "\NewTerm{test non paramétrique du chi carré}\index{test non paramétrique du chi carré}" ou "\NewTerm{test du chi carré de Pearson}\index{test du chi carré de Pearson}" ou "\NewTerm{test d'ajustement du chi carré}\index{test d'ajustement du chi carré}" ou "\NewTerm{test de Karl Pearson}\index{test de Karl Pearson}" ou encore "\NewTerm {test de qualité de l'ajustement du chi carré}\index{test de qualité de l'ajustement du chi carré}"...
	
	\begin{tcolorbox}[title=Remarque,colframe=black,arc=10pt]
	Il existe de nombreux tests statistiques contenant le mot "chi-carré". Par exemple:
	\begin{itemize}
		\item Test de qualité de l'ajustement du chi-carré (nous permet de déterminer si une distribution de population spécifiée est valide)
		
		\item Test du chi carré d'association / indépendance (nous permet de déterminer si la distribution d'une variable a été influencée par une autre variable)
		
		\item Test d'homogénéité du chi-carré (nous permet de comparer deux ou plusieurs proportions de population)
		
		\item Test du chi carré pour les valeurs aberrantes (nous permet de détecter s'il existe au moins une valeur aberrante en comparaison d'une distribution spécifique)
		
		\item Test du chi carré pour la différence de deux données de comptage (cas particulier du test d'homogénéité avec une seule catégorie!)
	\end{itemize}
	Même si tous ces tests ont des objectifs différents, leur cadre mathématique et leur procédure sont EXACTEMENT les mêmes et donc l'un de ces quatre tests peut être exécuté pour obtenir en même temps les quatre conclusions avec les mêmes $p$-valeurs et la même valeur du chi-carré critique!!!
	\end{tcolorbox}	
	
	On voit aussi par ailleurs que le résidu de Pearson au carré est la contribution individuelle à la statistique du $\chi^2$ de Pearson.
	
	Ensuite, la traditions est de déterminer la valeur de la distribution du chi-carré avec $N-1$ degrés de liberté avec une probabilité  $5\%$ d'être dépassée. Ainsi, dans l'hypothèse que le phénomène étudié suit la distribution théorique $P$, il y a donc une probabilité cumulative de $95\%$ que la variable $D$ prenne une valeur inférieure à celle donnée par la distribution du khi-deux.
	
	Si la valeur de la loi du khi-carré obtenue à partir de l'échantillon est inférieure à celle correspondant à  $95\%$ de probabilité cumulée, on ne rejette pas l'hypothèse nulle que le phénomène suit la loi $P$.
	
	\begin{tcolorbox}[title=Remarques,colframe=black,arc=10pt]
	\textbf{R1.} Rappelons que le fait que l'hypothèse de la loi $P$ soit acceptée ne signifie pas que cette hypothèse est vraie, mais simplement que les informations données par l'échantillon ne permettent pas de la rejeter. De même, le fait que l'hypothèse de la loi $P$ soit rejetée ne signifie pas nécessairement que cette hypothèse est fausse mais que les informations fournies par l'échantillon conduisent plutôt à la conclusion que l'inadéquation d'une telle loi.\\
	
	\textbf{R2.} Pour que la variable $D$ suive une loi du khi-deux, il faut que les valeurs théoriques $np_i$ des différentes modalités $ C_i $ soient au moins égales à $ 5$, que l'échantillon ait été tiré au hasard (pas de corrélation) et que les probabilités $p_i$ ne soient pas proches de zéro.
	\end{tcolorbox}	
	Ce test de qualité d'ajustement souffre cependant d'un problème majeur: il nécessite de regrouper les mesures en classes $C_i$ et en pratique il n'y a pas de théorème absolu (du moins à ce que l'on sache) pour choisir le nombre de classes sauf la règle de Sturge prouvée plus tôt. C'est pour cette raison que le test d'adéquation du Khi-2 est réservé aux distributions discrètes où le problème du choix des classes ne se pose pas.
	
	Cependant, nous devrons créer des tests de qualité d'ajustement qui ne nécessitent pas l'utilisation de classes et nous verrons juste après des outils ad hoc à cet effet (Kolmogorov-Smirnov ou Anderson-Darling pour ne citer que les plus importants).
	\begin{tcolorbox}[colframe=black,colback=white,sharp corners]
	\textbf{{\Large \ding{45}}Exemple:}\\\\
	Supposons que les naissances dans un hôpital pendant un certain temps soient les suivantes:
	\begin{table}[H]
		\centering
		\definecolor{gris}{gray}{0.85}
				\begin{tabular}{|p{2.5cm}|c|c|c|c|c|c|c|c|}
					\hline
					\multicolumn{1}{c}{\cellcolor{black!30}\textbf{Jour}} & 
	  \multicolumn{1}{c}{\cellcolor{black!30}\textbf{L}} & 
	  \multicolumn{1}{c}{\cellcolor{black!30}\textbf{M}} & 
	  \multicolumn{1}{c}{\cellcolor{black!30}\textbf{M}} & 
	  \multicolumn{1}{c}{\cellcolor{black!30}\textbf{J}} & 
	  \multicolumn{1}{c}{\cellcolor{black!30}\textbf{V}} & 
	  \multicolumn{1}{c}{\cellcolor{black!30}\textbf{S}} & 
	  \multicolumn{1}{c}{\cellcolor{black!30}\textbf{D}}  & 
	  \multicolumn{1}{c}{\cellcolor{black!30}\textbf{Total}}\\ \hline
					\cellcolor{black!30}\textbf{Observations} & $120$ & $130$ & $125$ & $128$ & $80$ & $70$ & $75$ & $728$ \\ \hline
		\end{tabular}
	\end{table}	
	On note qu'il y a eu un total de $728$ naissances. Nous posons alors la question suivante: combien devrait-il y avoir de naissances, en théorie, chaque jour s'il n'y avait pas de différence entre les jours? Cela représente l'hypothèse nulle $H_0$. En fait, l'hypothèse nulle stipule que les différences entre les fréquences observées et les fréquences conceptuelles sont relativement faibles. Nous tenons pour acquis que s'il n'y a pas de différence, il devrait y avoir le même nombre de naissances chaque jour. Puisqu'il y a un total de $728$ de naissances pour $7$ de jours en théorie, il devrait y avoir $728/7 = 104$ de naissances chaque jour. Alors maintenant, nous avons le tableau suivant:
	\begin{table}[H]
		\centering
		\definecolor{gris}{gray}{0.85}
				\begin{tabular}{|p{2.5cm}|c|c|c|c|c|c|c|c|}
					\hline
					\multicolumn{1}{c}{\cellcolor{black!30}\textbf{Jour}} & 
	  \multicolumn{1}{c}{\cellcolor{black!30}\textbf{L}} & 
	  \multicolumn{1}{c}{\cellcolor{black!30}\textbf{M}} & 
	  \multicolumn{1}{c}{\cellcolor{black!30}\textbf{M}} & 
	  \multicolumn{1}{c}{\cellcolor{black!30}\textbf{J}} & 
	  \multicolumn{1}{c}{\cellcolor{black!30}\textbf{V}} & 
	  \multicolumn{1}{c}{\cellcolor{black!30}\textbf{S}} & 
	  \multicolumn{1}{c}{\cellcolor{black!30}\textbf{D}}  & 
	  \multicolumn{1}{c}{\cellcolor{black!30}\textbf{Total}}\\ \hline
					\cellcolor{black!30}\textbf{Observations} & $120$ & $130$ & $125$ & $128$ & $80$ & $70$ & $75$ & $728$ \\ \hline
					\cellcolor{black!30}\textbf{Attendu} & $104$ & $104$ & $104$ & $104$ & $104$ & $104$ & $104$ & $104$ \\ \hline
		\end{tabular}
	\end{table}	
	Le total des fréquences observées est égal au total des fréquences attendues. Le but est donc d'examiner la différence entre les fréquences observées et attendues (supposées suivre une loi uniforme dans ce cas particulier) en utilisant la relation Chi-carré. En d'autres termes, nous faisons un test d'ajustement entre une fonction de distribution empirique (observée) et la fonction de distribution uniforme. Ensuite nous avons:
	
	Le $\chi^2$ est donc de $43.49$. En tant que tel, ce nombre signifie peu. Ce résultat doit être interprété à l'aide de la table des valeurs critiques du  $\chi_N^2$. Sans utiliser le tableau, nous comprenons qu'il est très peu probable que la fréquence observée et la fréquence théorique soient identiques. On admet qu'il peut y avoir une différence (on rejette donc l'hypothèse nulle $H_0$ au profit des hypothèses alternatives $H_A$).\\
	
	N'oubliez donc pas que ce test ne s'applique qu'aux données non censurées, c'est-à-dire dont les intervalles sont fermés!
	\end{tcolorbox}
	
	\begin{tcolorbox}[title=Remarque,colframe=black,arc=10pt]
	Le test ci-dessus est également utilisé comme technique de sélection de caractéristiques dans le Machine Learning et est alors nommé "\NewTerm{sélection de caractéristiques du $\chi^2$}\index{sélection de caractéristiques du $\chi^2$}". L'idée est que si nous avons un ensemble de données avec trop de colonnes (catégorielles ou continues), nous devons en éliminer certaines en investiguant si elles sont statistiquement dépendantes (alors nous pouvons en éliminer une) ou indépendantes (alors nous ne pouvons pas éliminer n'importe laquelle d'entre elles). Notez que pour exécuter cette technique sur des variables continues (colonnes), nous pouvons simplement les regrouper dans des groupes!
	\end{tcolorbox}
	
	\paragraph{Test de QdA $G$ ($G$-test)}\mbox{}\\\\
	Nous utilisons le "\NewTerm{test de qualité d'ajustement $G$}\index{$G$–test de QdA}" ou également nommé plus simplement "\NewTerm {test du $G^2$}\index{test du $G^2$}" (c'est un cas particulier du "test du rapport de vraisemblance" ou du "test du rapport des log-vraisemblance" comme nous le verrons plus loin à la page \pageref{likelihood ratio tests}) quand nous avons une variable nominale et que nous voulons voir si le nombre d'observations dans chaque catégorie correspond à une attente théorique, et la taille de l'échantillon est grande et le total observé des dénombrements et le total attendu des dénombrements est égal!
	
	L'hypothèse statistique nulle est que le nombre d'observations dans chaque catégorie est égal à celui prédit par une théorie, et l'hypothèse alternative est que les nombres observés sont différents de ceux attendus. L'hypothèse nulle est généralement une hypothèse extrinsèque, dans laquelle nous connaissons les proportions attendues avant de faire l'expérience.
	
	La statistique du test $G$ est proportionnelle à la divergence de Kullback–Leibler (\SeeChapter{voir section Mécanique Statistique page \pageref{kullback-leibler divergence}}) de la distribution théorique de la distri
	
	Posons maintenant:
	
	avec la contrainte que:
	
	tel que le nombre total de comptages reste le même (ie $\sum_i O_i=\sum_i E_i=N$). Le test-$G$ est alors:
	
	Si nous développons en série de Taylor cela autour de:
	
	remarquant que cela suppose$E_i\gg \delta_i$ et donc $E_i\gg 0$ (comme pour le test d'adéquation du chi-carré vu juste plus haut), alors nous savons que pour le logarithme (\SeeChapter{voir section Séquences et Séries page \pageref{usual maclaurin developments}}):
	
	Dès lors nous obtenons:
	
	et alors, nous voyons que $G=\chi^2$ quand:
	\begin{itemize}
		\item $O_i$ est proche de $E_i$
		\item $\sum_i E_i=\sum_i O_i=N$
		\item $E_i\gg \delta_i$ ie $E_i\gg 0$
	\end{itemize}
	Cependant, plus $O_i$ et $E_i$ sont différents, moins cette approximation fonctionnera bien, et le $\chi^2$ aura tendance à calculer des réponses erronées. Les effets d'une seule valeur aberrante dans un petit ensemble d'échantillons seront plus prononcés, ce qui explique pourquoi le $\chi^2$  échoue souvent dans des situations avec peu de données.
	
	Puisque la valeur $\chi^2$ n'est qu'une approximation de la valeur $G$, la valeur $G$ peut également être utilisée dans le test de probabilité du chi carré. Il semble qu'actuellement la $p$-valeur du test $G$ doit être calculée en utilisant les techniques de Monte Carlo.
	
	\begin{tcolorbox}[title=Remarque,colframe=black,arc=10pt]
	Si le nombre attendu d'observations dans une catégorie est trop petit, le test $G$ peut donner des résultats inexacts, et nous devrions utiliser un test exact à la place (test exact de Fisher en cas d'indépendance comme vu à la page \pageref{exact Fisher test}, Test de McNemar en cas de dépendance comme vu à la page \pageref{mcnemar test}). En pratique, c'est une bonne idée de calculer à la fois le $\chi^2$ et le $G$ pour voir s'ils conduisent à des résultats similaires. Si les $p$-valeurs résultantes sont proches, alors nous pouvons être assez sûrs que l'approximation pour grand échantillon fonctionne bien. De toute évidence, le test de qualité d'ajustement $G$ suppose l'indépendance!
	\end{tcolorbox}

	\pagebreak
	\paragraph{Test de QdA de Kolmogorov-Smirnov}\mbox{}\\\\
	En statistique, le test de Kolmogorov-Smirnov est également un test de qualité d'ajustement basé sur une distance empirique utilisée pour déterminer si la distribution d'un échantillon suit une loi bien connue donnée par une fonction de distribution continue (ou pour comparer deux échantillons et vérifier s'ils sont dépendants ou pas ou similaires ou pas). Ce test, ainsi que celui du test QdA du khi-carré n'est valable que pour les données non censurées (du moins pas sans correction obtenue par des simulations numériques).
	
	Pour introduire ce test, nous avons choisi l'approche de Lilliefors qui permet d'éviter des calculs complexes. De plus, les logiciels qui fournissent le "\NewTerm{test QdA de Lilliefors}\index{test de qualité d'ajustement de Lilliefors}" n'offrent pas le test Kolmogorov-Smirnov puisque ce dernier n'est correct qu'asymptotiquement (ce qui est le cas du logiciel Tanagra 4.14).
	
	Imaginons que nous voulions construire un test QdA non paramétrique qui fonctionne à la fois pour les lois discrètes et continues sans subir le même problème que le test de QdA du khi-carré (regroupement en classes).
	
	Pour construire ce test, nous partons de la fonction de distribution empirique déjà définie au début de cette section et donnée pour rappel par:
	
	qui appartiennent évidemment à l'intervalle $[0,1]$.
	
	Notons maintenant $F(x)$, la vraie loi supposée dont l'expression analytique est connue et avec laquelle on voudrait comparer $\hat{F}(x)$ et construire la distance:
	
	La distribution de référence peut cependant également provenir d'un autre échantillon de mesure. L'idée est alors simplement de comparer deux distributions empiriques. On parle alors de "\NewTerm{test de Kolmogorov-Smirnov pour $2$ échantillons indépendants}\index{test de Kolmogorov-Smirnov pour $2$ échantillons indépendants}". Certains logiciels gèrent également empiriquement le cas où les deux échantillons n'ont pas la même taille.
	
	Le problème avec ce choix de distance est ... quel $x$ devrions-nous alors choisir pour faire un test? Eh bien pour répondre il est simple de voir qu'il serait insensé de prendre un $x$ pour lequel cette distance est minime, car avoir un $D_n$ qui peut être presque nul n'ajoute pas beaucoup d'informations (et ce n'est pas très robuste par la même occasion)... Par conséquent, nous nous reportons plutôt vers la plus grande déviation absolue. Ce qui nous amène à redéfinir la distance $D_n$ comme suit:
	
	où $D_n$ est nommé "\NewTerm{distribution empirique de Kolmogorov-Smirnov}\index{distribution empirique de Kolmogorov-Smirnov}" (bien sûr, nous devons prouver rigoureusement qu'il s'agit bien d'une distribution ... mais pour l'instant c'est trop complexe en termes du contenu de ce livre mais cela peut être vérifié par des simulations numériques!). Avant d'aller plus loin par rapport à la théorie, regardons un exemple pratique (parce que l'exemple est long on ne le mettra pas dans la case conventionnelle qui représente les exemples habituellement dans ce livre).
	
	Supposons que nous mesurions les cinq valeurs suivantes:
	
	c'est-à-dire une fois ordonnées:
	
	Nous voulons tester l'hypothèse nulle suivante:
	
	où $\Phi(x)$ est comme d'habitude la fonction de distribution (cumulative) de la distribution Normale centrée réduite.
	
	La fonction de distribution empirique sera donnée par:
	
	Ensuite, nous construisons traditionnellement le tableau suivant:
	\begin{table}[H]
		\centering
		\definecolor{gris}{gray}{0.85}
			\begin{tabular}{|c|c|c|c|}
			\hline
	\multicolumn{1}{c}{\cellcolor{black!30}$x$} & \multicolumn{1}{c}{\cellcolor{black!30}$\hat{F}(x)$} & \multicolumn{1}{c}{\cellcolor{black!30}$\Phi(x)$} & \multicolumn{1}{c}{\cellcolor{black!30}$|\hat{F}(x)-\Phi(x)|$} \\ \hline
			$-1.2^{-}$ & $0$ & $0.115$ & $0.115$\\ \hline
			$-1.2$ & $0.2$ & $0.115$ & $0.085$\\ \hline
			$-1.0^{-}$ & $0.2$ & $0.159$ & $0.041$\\ \hline
			$-1.0$ & $0.4$ & $0.159$ & $0.241$ \\ \hline
			$-0.6^{-}$ & $0.4$ & $0.274$ & $0.126$ \\ \hline
			$-0.6$ & $0.6$ & $0.274$ & \textbf{0.326}\\ \hline
			$+0.2^{-}$ & $0.6$ & $0.580$ & $0.020$\\ \hline
			$+0.2$ & $0.8$ & $0.580$ & $0.220$\\ \hline
			$+0.8^{-}$ & $0.8$ & $0.788$ & $0.012$\\ \hline
			$+0.8$ & $1$ & $0.788$ & $0.212$\\ \hline
		\end{tabular}
	\end{table}
	Souvent associé au graphique suivant comparant les distributions empiriques (en rouge) et théoriques (en bleu):
	\begin{figure}[H]
		\centering
		\includegraphics{img/arithmetics/ks_test_empirical_vs_theoretical.jpg}
		\caption{Représentation de l'approche du test de QdA de Kolmogorov-Smirnov}
	\end{figure}
	Nous observons alors que l'écart maximal est de $0.326$. Nous noterons cela pour après:
	
	que certains logiciels tels que Minitab ou R désignent par l'abréviation: KS.
	
	Le lecteur aura remarqué que le plus grand écart au-dessus de la courbe se mesure par:
	
	La plus grande déviation sous la courbe est mesurée par:
	
	Le plus grand écart est alors:
	
	Mais que pouvons-nous faire avec cette valeur? À quoi pouvons-nous le comparer? Eh bien, l'idée est relativement simple et consiste à générer des valeurs $n$ (donc $5$ dans notre exemple) à partir de la loi de distribution $F(x)$ de l'hypothèse nulle $H_0$ et de les comparer à elles-mêmes. En d'autres termes, faire une simulation de Monte Carlo (\SeeChapter{voir section de Méthodes Numériques page \pageref{monte carlo simulations}}).
	
	Ainsi, dans notre exemple, nous générons $5$ valeurs aléatoires de $\mathcal{N}(0,1)$ ce qui nous donne par exemple avec Microsoft Excel 11.8346:
	
	\begin{center}
	\texttt{=LOI.NORMALE.INVERSE.N(ALEA.ENTRE.BORNES(0,1000000)/1000000)}
	\end{center}
	
	On obtient alors $5$ valeurs de $Z$ (rappelez-vous que c'est la notation traditionnelle d'une variable aléatoire d'une distribution Normale réduite centrée!) qui ordonnées donneront par exemple:
	
	et nous répétons le même tableau que précédemment:
	\begin{table}[H]
	\begin{center}
		\definecolor{gris}{gray}{0.85}
			\begin{tabular}{|c|c|c|c|}
				\hline
\multicolumn{1}{c}{\cellcolor{black!30}$x$} & \multicolumn{1}{c}{\cellcolor{black!30}$\hat{F}(x)$} & \multicolumn{1}{c}{\cellcolor{black!30}$\Phi(x)$} & \multicolumn{1}{c}{\cellcolor{black!30}$|\hat{F}(x)-\Phi(x)|$} \\ \hline
		$-1.427^{-}$ & $0$ & $0.077$ & $0.115$\\ \hline
		$-1.427$ & $0.2$ & $0.077$ & $0.085$\\ \hline
		$+0.082^{-}$ & $0.2$ & $0.533$ & $0.041$\\ \hline
		$+0.082$ & $0.4$ & $0.533$ & $0.241$ \\ \hline
		$+0.162^{-}$ & $0.4$ & $0.564$ & $0.126$ \\ \hline
		$+0.162$ & $0.6$ & $0.564$ & \textbf{0.326}\\ \hline
		$+0.294^{-}$ & $0.6$ & $0.616$ & $0.020$\\ \hline
		$+0.294$ & $0.8$ & $0.616$ & $0.220$\\ \hline
		$+1.292^{-}$ & $0.8$ & $0.902$ & $0.012$\\ \hline
		$+1.292$ & $1$ & $0.902$ & $0.212$\\ \hline
	\end{tabular}
	\end{center}
	\end{table}
	Et donc, nous avons l'écart maximal qui est de $0.333$. Soit avec Microsoft Excel 14.0.6123:
	\begin{figure}[H]
		\centering
		\includegraphics{img/arithmetics/ks_ExcelValuesList.jpg}
		\caption[]{Calculs avec Microsoft Excel 14.0.6123}
	\end{figure}
	avec les formules explicites:
	\begin{figure}[H]
		\centering
		\includegraphics[width=1.0\textwidth]{img/arithmetics/ks_ExcelValuesListFormulas.jpg}
		\caption[]{Calculs explicites avec Microsoft Excel 14.0.6123}
	\end{figure}
	avec la petite routine VBA correspondante vite fait mal fait qui prendra le nombre d'itérations requises dans la cellule K1 et mettra la distribution empirique de Kolmogorov-Smirnov dans la colonne G de la feuille active:
	\begin{figure}[H]
		\centering
		\includegraphics[scale=0.9]{img/arithmetics/ks_VBA.jpg}
		\caption[]{Petit script VBA (MS Excel) pour le test QdA de K-S}
	\end{figure}
	Nous réitérons donc la procédure plusieurs milliers de fois et nous obtenons la fonction de distribution suivante (obtenue simplement en faisant un graphique de type nuage de points dans Microsoft Excel 14.0.6123 avec des $2'000$ simulations):
	\begin{figure}[H]
		\centering
		\includegraphics{img/arithmetics/ks_distribution_simulation.jpg}
		\caption[]{Simulation de distribution pour le QdA de K-S}
	\end{figure}
	et en appliquant un test unilatéral avec un seuil $\alpha=5\%$ nous obtenons pour le $95$ème centile:
	
	Le lecteur trouvera la même valeur dans les tables de Kolmogorov-Smirnov disponibles dans de nombreux ouvrages universitaires. Quelques milliers de simulations sont alors suffisantes pour retrouver les valeurs des tables de réference!
	
	Et maintenant nous comparons:
	
	et donc nous ne rejetons malheureusement pas l'hypothèse nulle.
	
	Mais... nous devons quand même faire attention car avec seulement cinq valeurs, il est vraisemblable que l'hypothèse nulle $H_0$ ne soit pas non plus rejetée pour d'autres lois que la loi Normale!
	
	Ainsi, comme le lecteur l'aura peut-être remarqué, pour chaque hypothèse nulle $H_0$ associée à une loi de distribution particulière, nous devons tabluler la distribution empirique de Kolmogorov-Smirnov pour différentes valeurs de $n$ et $\alpha$ en utilisant des méthodes numériques. Dans la majorité des livres universitaires dédiés au tests de Kolmogorov-Smirnov il y a un théorème puissant qui montre qu'en réalité, les valeurs critiques seront les mêmes!
	
	\begin{tcolorbox}[title=Remarque,colframe=black,arc=10pt]
	Kolmogorov et Smirnov ont prouvé que quand $n$ devient grand et que la loi de l'hypothèse nulle est continue, il n'est plus nécessaire de tabuler les valeurs de Kolmogorov et Smirnov pour chaque loi parce que nous avons alors:
	
	dès lors $D_n$ est indépendant de la loi de distribution de l'hypothèse nulle $H_0$. En faisant des simulations avec la méthode de Monte Carlo, nous observons une convergence quand $n$ atteint la centaine. Mais dans la pratique, la grande majorité du temps, il est inconcevable d'avoir une telle quantité de mesures. D'où le fait que ce résultat théorique et d'usage rare dans la pratique et justifie donc l'absence de la démonstration mathématique dans le présent ouvrage.
	\end{tcolorbox}
	Pour conclure sur le test QdA de K-S, indiquons aussi au lecteur que nous trouverons la preuve mathématique du test QdA de Anderson-Darling plus bas.
	
	\newpage
	\begin{center}
  		TEST UNILATÉRAL DE KOLMOGOROV-SMIRNOV
	\end{center}

	\begin{center}
	\renewcommand{\arraystretch}{1.1}
	\begin{tabular}{|c|ccccc|} \hline
	$n$&  0.2   &  0.1  & 0.05  &  0.02  & 0.01  \\ \hline
	 1 & 0.9000 & 0.9500 & 0.9750 & 0.9900 & 0.9950 \\
	 2 & 0.6838 & 0.7764 & 0.8419 & 0.9000 & 0.9293 \\
	 3 & 0.5648 & 0.6360 & 0.7076 & 0.7846 & 0.8290 \\
	 4 & 0.4927 & 0.5652 & 0.6239 & 0.6889 & 0.7342 \\
	 5 & 0.4470 & 0.5094 & 0.5633 & 0.6272 & 0.6685 \\
	 6 & 0.4104 & 0.4680 & 0.5193 & 0.5774 & 0.6166 \\
	 7 & 0.3815 & 0.4361 & 0.4834 & 0.5384 & 0.5758 \\
	 8 & 0.3583 & 0.4096 & 0.4543 & 0.5065 & 0.5418 \\
	 9 & 0.3391 & 0.3875 & 0.4300 & 0.4796 & 0.5133 \\
	10 & 0.3226 & 0.3687 & 0.4092 & 0.4566 & 0.4889 \\
	11 & 0.3083 & 0.3524 & 0.3912 & 0.4367 & 0.4677 \\
	12 & 0.2958 & 0.3382 & 0.3754 & 0.4192 & 0.4490 \\
	13 & 0.2847 & 0.3255 & 0.3614 & 0.4036 & 0.4325 \\
	14 & 0.2748 & 0.3142 & 0.3489 & 0.3897 & 0.4176 \\
	15 & 0.2659 & 0.3040 & 0.3376 & 0.3771 & 0.4042 \\
	16 & 0.2578 & 0.2947 & 0.3273 & 0.3657 & 0.3920 \\
	17 & 0.2504 & 0.2863 & 0.3180 & 0.3553 & 0.3809 \\
	18 & 0.2436 & 0.2785 & 0.3094 & 0.3457 & 0.3706 \\
	19 & 0.2373 & 0.2714 & 0.3014 & 0.3369 & 0.3612 \\
	20 & 0.2316 & 0.2647 & 0.2941 & 0.3287 & 0.3524 \\
	21 & 0.2262 & 0.2586 & 0.2872 & 0.3210 & 0.3443 \\
	22 & 0.2212 & 0.2528 & 0.2809 & 0.3139 & 0.3367 \\
	23 & 0.2165 & 0.2475 & 0.2749 & 0.3073 & 0.3295 \\
	24 & 0.2120 & 0.2424 & 0.2693 & 0.3010 & 0.3229 \\
	25 & 0.2079 & 0.2377 & 0.2640 & 0.2952 & 0.3166 \\
	26 & 0.2040 & 0.2332 & 0.2591 & 0.2896 & 0.3106 \\
	27 & 0.2003 & 0.2290 & 0.2544 & 0.2844 & 0.3050 \\
	28 & 0.1968 & 0.2250 & 0.2499 & 0.2794 & 0.2997 \\
	29 & 0.1935 & 0.2212 & 0.2457 & 0.2747 & 0.2947 \\
	30 & 0.1903 & 0.2176 & 0.2417 & 0.2702 & 0.2899 \\
	31 & 0.1873 & 0.2141 & 0.2379 & 0.2660 & 0.2853 \\
	32 & 0.1844 & 0.2108 & 0.2342 & 0.2619 & 0.2809 \\
	33 & 0.1817 & 0.2077 & 0.2308 & 0.2580 & 0.2768 \\
	34 & 0.1791 & 0.2047 & 0.2274 & 0.2543 & 0.2728 \\
	35 & 0.1766 & 0.2018 & 0.2242 & 0.2507 & 0.2690 \\
	36 & 0.1742 & 0.1991 & 0.2212 & 0.2473 & 0.2653 \\
	37 & 0.1719 & 0.1965 & 0.2183 & 0.2440 & 0.2618 \\
	38 & 0.1697 & 0.1939 & 0.2154 & 0.2409 & 0.2584 \\
	39 & 0.1675 & 0.1915 & 0.2127 & 0.2379 & 0.2552 \\
	40 & 0.1655 & 0.1891 & 0.2101 & 0.2349 & 0.2521 \\ \hline
	$>40$&
	$1.07/\sqrt{n}$&$1.22/\sqrt{n}$&$1.36/\sqrt{n}$&
	$1.52/\sqrt{n}$&$1.63/\sqrt{n}$ \\
	\hline
	\end{tabular}
	\end{center}
	
	\paragraph{Test de QdA de Ryan-Joiner}\mbox{}\\\\
	Considérons une variable aléatoire $X$ dont nous connaissons la distribution d'échantillonnage et pour laquelle nous aimerions vérifier la Normalité ou non et considérons une variable aléatoire ordonnée $Y$ générée par une distribution Normale centrée réduite centrée. Pour comparer $X$ et $Y$, nous centrerons $X$ et classerons ses valeurs par ordre croissant.
	
	Pour une même taille d'échantillon donnée, si les valeurs ordonnées de $X$ et $Y$ prises par paires suivent la même loi, une régression linéaire basée sur l'autre devrait donner un coefficient de corrélation assez proche de $1$. En prenant la définition du coefficient de corrélation de Pearson au carré, on a alors:
	
	On suppose que $Y$ suit une distribution Normale centrée réduite. Il vient alors:
	
	et si l'on prend l'estimateur du coefficient de corrélation:
	
	Donc après simplification:
	
	Il s'agit de l'approche Ryan-Joiner (implémentée dans Minitab par exemple) du test Shapiro-Wilk. Les résultats des deux tests sont très similaires. Les coefficients $a_i$ peuvent être facilement obtenus à l'aide de n'importe quel tableur en utilisant une simulation de Monte Carlo (\SeeChapter{voir section Méthodes Numériques page \pageref{monte carlo simulations}}). Sur demande des lecteurs, nous pouvons détailler comment obtenir les coefficients $a_i$ avec Microsoft Excel pour un $n$ donné.
	
	\begin{tcolorbox}[colframe=black,colback=white,sharp corners]
	\textbf{{\Large \ding{45}}Exemple:}\\\\
	Considérez les $10$ mesures suivantes de la colonne $\texttt{A}$ déjà triées par ordre croissant:
	\begin{figure}[H]
		\centering
		\includegraphics[scale=0.7]{img/arithmetics/rj_gof_table_excel_values.jpg}
		\caption[]{Exemple de mesures ordonnées, rangs, coefficient de RJ et score $Z$}
	\end{figure}
	\end{tcolorbox}
	
	\pagebreak
	\begin{tcolorbox}[colframe=black,colback=white,sharp corners]
	Les formules correspondantes sont:
	\begin{figure}[H]
		\centering
		\includegraphics[scale=0.63]{img/arithmetics/rj_gof_table_excel_formulas.jpg}
		\caption[]{Formulas in  Microsoft Excel 14.0.6123 of previous screen shot}
	\end{figure}
	et donc nous avons dans une feuille nommée \textit{CoeffMonteCarlo} des simulations Monte Carlo pour déterminer les $10$ coefficients $a_i$ traditionnellement notés dans le cas des $10$ mesures dans les tableaux comme suit: $\left\lbrace a_{1/10},a_{2/10},...,a_{10/10} \right\rbrace$. Tout d'abord, nous devons créer $10$ colonnes avec des variables Normales centrées réduites sur près de $10,000$ de lignes avec la formule Microsoft Excel suivante:
	\begin{figure}[H]
		\centering
		\includegraphics[scale=0.75]{img/arithmetics/rj_normal_centered_reduced_variables_for_coeff.jpg}
		\caption[]{Variables normales centrées réduites pour les coefficients RJ}
	\end{figure}
	et ensuite nous devons construire les rangs de toutes ces valeurs ligne par ligne telles que (donnée en anglais):
	\begin{center}
		\texttt{=LOI.NORMALE.STANDARD.INVERSE.N(RANDBETWEEN(1;99999999)/100000000)}
	\end{center}
	et ensuite nous devons construire les rangs (ordre croissant) de toutes ces valeurs ligne par ligne telles que:
	\begin{figure}[H]
		\centering
		\includegraphics{img/arithmetics/rj_normal_centered_reduced_variables_ordered.jpg}
		\caption[]{Tri croissant ligne par ligne des simulations pour déterminer les coefficients de RJ}
	\end{figure}
	\end{tcolorbox}
	\pagebreak
	\begin{tcolorbox}[colframe=black,colback=white,sharp corners]
	avec les formules suivantes (données seulement pour les 4 premiers i faute de place dans la capture d'écran):
	\begin{figure}[H]
		\centering
		\includegraphics{img/arithmetics/rj_normal_centered_reduced_variables_ordered_formulas.jpg}
		\caption[]{Microsoft Excel 14.0.6123 formules de la précédente capture d'écran}
	\end{figure}
	Pour finir, il n'y a plus qu'à calculer le coefficient de corrélation entre les colonnes \texttt{C} et \texttt{D} de la première capture d'écran:
	\begin{figure}[H]
		\centering
		\includegraphics{img/arithmetics/rj_correlation_coefficient.jpg}
		\caption[]{Calcul final du coefficient de corrélation de RJ avec Microsoft Excel 14.0.6123}
	\end{figure}
	Ce qui donne:
	\begin{figure}[H]
		\centering
		\includegraphics{img/arithmetics/rj_correlation_coefficient_value.jpg}
	\end{figure}
	où le carré de cette valeur est très très proche du test de Shapiro-Wilk.\\
	
	 Ensuite, pour savoir si on peut accepter ou rejeter l'hypothèse nulle $H_0$ de Normalité, il faudrait refaire la procédure avec en lieu et place mesures, des valeurs générées aussi aléatoires à partir d'une loi Normale et déterminer la valeur critique d'acceptation/rejet (normalement c'est très simple à faire mais on peut détailler sur demande).\\
	
	Après des calculs dans ce cas particulier, nous ne rejetons malheureusement pas l'hypothèse nulle.
	\end{tcolorbox}
	
	\pagebreak
	\paragraph{Test de QdA de Anderson-Darling}\label{anderson darling gof test}\mbox{}\\\\
	Il est surprenant qu'un test raisonnablement puissant (robuste) comme l'est le test de Kolmogorov-Smirnov puisse être conçu en ne s'appuyant que sur une unique observation et ce un seul point de la fonction de répartition candidate. Il semblerait, avec du recul, plus efficient de mesurer la différence entre les deux fonctions de répartition en comparant ces fonctions sur l'intégralité de leur domaine, c'est-à-dire de $-\infty$ à $+\infty$.

	Il existe une famille de tests dont les statistiques sont basées sur l'intégrale du carré de la différence (ces tests sont souvent considérés comme non paramétriques mais selon moi à tort et ce au même titre que le test de Kolmogorov-Smirnov est lui aussi considéré comme non paramétrique):
	
	entre la fonction de répartition empirique et la fonction de répartition de référence. La plus simple de ces statistiques est:
	
	qui est simplement la surface comprise entre la fonction de répartition empirique et la fonction de répartition de référence. Soit, en reprenant le graphique utilisé plus haut lors de notre étude du test d'ajustement de Kolmogorov-Smirnov:
	\begin{figure}[H]
		\centering
		\includegraphics{img/arithmetics/anderson_darling_concept.jpg}
		\caption{Représentation de l'approche du test d'ajustement d'Anderson-Darling}
	\end{figure}
	Cependant, arbitrairement, nous pouvons choisir autre chose que la mesure $x$ pour l'intégrale. Ainsi, un choix classique est de prendre la fonction de répartition théorique elle-même comme mesure de base de l'intégrale. Il vient ainsi:
	
	La statistique résultant de cet ajout s'appelle la "\NewTerm{statistique de Cramér-von Mises}\index{statistique de Cramér-von Mises}". Cependant elle souffre d'un gros défaut de robustesse lorsque des points de mesures se trouvent sur les queues de la distribution.
	
	Il a alors été proposé la mesure suivante qui est un peu moins sensible aux points de mesures se trouvant sur les queues:
	
	appelée "\NewTerm{statistique d'Anderson-Darling}\index{statistique d'Anderson-Darling}" qui a été la plus utilisée dans la fin du 20ème siècle et reste dominante au début du 21ème aussi (du moins tant que l'échantillon est d'une taille acceptable!). Elle est par construction plus robuste que les statistiques de Cramér-von Mises et de Kolmogorov-Smirnov mais des études par simulations ont montré qu'elle était moins robuste que le test de Shapiro-Wilk ou Ryan-Joiner.
	
	En se rappelant que la définition de la distribution empirique $\hat{F}_n(x)$ lors de notre étude du test d'ajustement (adéquation) de Kolmogorov-Smirnov implique que:
	
	si $x\in[-\infty,x_1]$ et:
	
	si $x\in[x_k,x_{k+1}]$ et:
	
	si $x\in[x_n,x_{+\infty}]$.  Nous avons alors en supposant en plus que $F$ est continue:
	
	Ensuite, nous faisons le changement de variable $u=F(x)$ et donc:
	
	et sans oublier les changement de bornes des intégrales puisque:
	
	Il vient alors:
	
	où nous avons bien évidemment posé:
	
	Il faut à présent calculer ces intégrales. Nous cherchons donc la primitive d'une fonction du type:
	
	Les primitives des deux expressions suivantes:
	
	ont été démontrées sous leur forme générale dans le chapitre de Calcul Différentiel Et Intégral et valent respectivement:
	
	car au vu des valeurs que peut prendre $u$, il est alors inutile d'indiquer les valeurs absolues.
	
	Il nous reste donc qu'à calculer la primitive de:
	
	où un changement de variable évident (si jamais vous souhaitez les détails n'hésitez pas à demander) nous donne la primitive sans la constante:
	
	Nous avons alors au final:
	
	Nous avons donc:
	\thickmuskip=0mu
	\medmuskip=0mu
		
	\thickmuskip=3mu
	\medmuskip=3mu
	Nous pouvons déjà remarquer que dans la dernière égalité:
	
	Il reste alors:
	
	Nous allons procéder maintenant à quelques manipulations algébrique astucieuses (mais simples) pour condenser l'écriture de cette dernière égalité.

	D'abord, remarquons que nous pouvons récrire la première somme ainsi (le lecteur pourra vérifier en développement les deux sommes pour une petite valeur de $n$):
	
	ce qui équivaut donc à poser $j=j-1$.

	Nous transformons aussi la deuxième somme:
	
	et le lecteur pourra vérifier que l'égalité ci-dessous pour la troisième somme est vérifiée:
	
	qui équivaut aussi à poser  $j=j-1$.
	
	Enfin, nous transformons la quatrième somme (puisque de toute façon lorsque $j$ vaut $n$ le terme de la somme est nul...):
	
	Alors, nous avons:
	
	Soit en éliminant les termes qui s'annulent:
	
	Et en regroupant les termes ayant la même forme de logarithme:
	
	Soit:
	
	Il s'agit d'un des formes du test d'Anderson-Darling et qui dans le cadre d'une loi Normale s'écrit par tradition sous la forme suivante:
	
	Mais il existe une autre expression simplifiée très courante. Pour l'établir, nous repartons de l'expression:
	
	En faisant le changement de variable $j=1+n-i$ dans la dernière somme, l'expression:
	
	devient:
	
	et les bornes de la somme deviennent:
	
	et dès lors:
	
	Donc:
	
	Enfin:
	
	\begin{tcolorbox}[colframe=black,colback=white,sharp corners]
	\textbf{{\Large \ding{45}}Exemple:}\\\\
	Supposons que nous ayons mesuré les cinq valeurs suivantes:
	
	soient ordonnées:
	
	Nous voulons tester l'hypothèse nulle suivante:
	
	où $\Phi(x)$ représente pour rappel la fonction de répartition de la loi Normale centrée réduite $\mathcal{N}(0,1)$.\\
	
	Pour mettre en place le calcul de l'indice AD dans un logiciel comme la version française de Microsoft Excel 14.0.6123 nous créons d'abord:
	\begin{figure}[H]
		\centering
		\includegraphics{img/arithmetics/ad_gof_excel_initial_values.jpg}
		\caption[]{Valeurs à tester avec colonnes habituelles dans le tableur}
	\end{figure}
	Soit explicitement:
	\begin{figure}[H]
		\centering
		\includegraphics[scale=0.9]{img/arithmetics/ad_gof_excel_initial_values_formulas.jpg}
		\caption[]{Formules explicites du tableau principal de la figure précédente}
	\end{figure}
	Nous obtenons donc la même valeur de l'indicateur AD que les logiciels de statistiques qui permettent de choisir la loi à comparer (et donc les paramètres y relatifs). Cependant pour de très petits échantillons les logiciels de statistiques utilisent la correction suivante (qui nous été impossible de réobtenir par simulation...):
	
	soit dans notre cas AD* vaut environ $0.789$ avec $\text{AD}=0.636$.
	\end{tcolorbox}
	
	\pagebreak
	\begin{tcolorbox}[colframe=black,colback=white,sharp corners]
	Ensuite pour calculer la $p$-valeur nous devons investiguer une curiosité... Effectivement si nous la déterminons en faisant une simulation de Monte Carlo comme nous l'avons fait lors de notre démonstration du test de Kolmogorov-Smirnov en changeant d'abord le contenu de la colonne \texttt{A} en y mettant des valeurs dynamiques triées:
	\begin{figure}[H]
		\centering
		\includegraphics{img/arithmetics/ad_gof_excel_initial_pvalue_sorting.jpg}
		\caption[]{Formules pour trier les valeurs simulées par Monte Carlo}
	\end{figure}
	Valeurs provenant donc de la colonne \texttt{O} où nous avons mis:
	\begin{figure}[H]
		\centering
		\includegraphics[scale=0.9]{img/arithmetics/ad_gof_excel_mc_simulated_values.jpg}
		\caption[]{Formules génératrices d'une loi Normale pour l'application de Monte Carlo}
	\end{figure}
	Le lecteur remarquera donc que cela revient finalement à comparer l'échantillon avec une distribution uniforme!!!\\
	
	En ayant ensuite préparé les colonnes suivantes \texttt{H}, \texttt{I} qui contiendront les valeurs simulées reportées par le code VBA donné un peu plus loin et les colonnes \texttt{L}, \texttt{M} qui nous permettent d'avoir la répartition des valeurs de AD et AD* pour en calculer le centile:
	\end{tcolorbox}
	
	\pagebreak
	\begin{tcolorbox}[colframe=black,colback=white,sharp corners]
	\begin{figure}[H]
		\centering
		\includegraphics[scale=0.8]{img/arithmetics/ad_gof_excel_reported_ad_values_formulas.jpg}
		\caption[]{Colonnes pour le reports du VBA et pour les différents centiles de AD/AD*}
	\end{figure}
	avec le petit code VBA ci-dessous vite fait mal fait:
	\begin{lstlisting}[language={[Visual]Basic}, caption={Code VBA du test d'AD de QdA}]
	Sub SimulAndersonDarling()

		Const intSimulations As Integer = 1000
		Dim i As Integer

		For i=1 To intSimulations
			Calculate
			Cells(i+1, 8).Value = Cells(9, 2).Value
			Cells(i+1, 9).Value = Cells(10, 2).Value
			Cells(8, 17).Value = i
		Next i
	End Sub
	\end{lstlisting}
	nous avons alors avec $10'000$ simulations la répartition suivante des valeurs de AD et AD*:
	\end{tcolorbox}
	
	\pagebreak
	\begin{tcolorbox}[colframe=black,colback=white,sharp corners]
	\begin{figure}[H]
		\centering
		\includegraphics[scale=1]{img/arithmetics/ad_gof_percentiles.jpg}
		\caption[]{Centiles du test de QdA d'Anderson-Darling}
	\end{figure}
	Donc que ce soit pour AD ou AD* la $p$-valeur est approximativement égale à  $1-40\%=60\%$ ce qui correspond aux valeurs tabulées par Peter A. W. Lewis chez IBM (1961).\\
	
	Ce qui est curieux et qu'il nous faut justement investiguer c'est que la grande majorité des logiciels utilisent les formules suivantes (R.B. D'Augostino et M.A. Stephens, Eds., 1986, Goodness-of-Fit Techniques, Marcel Dekker) permettant d'éviter les simulations de Monte-Carlo:
	
	et dans notre cas, l'application de ces formules donnent une $p$-valeurs d'environ $9\%$!!! Valeur que donnent effectivement les logiciels statistiques! Affaire à suivre pour trouver d'où vient cette énorme différence... Nous avons demandé au support technique d'un éditeur de progiciel statistique américain de nous expliquer la raison de la différence entre les valeurs tabulées Peter A.W. Lewis et celles R.B. D'Augostino et M.A. Stephens mais ils n'ont pas été capables de répondre. Nous avons également contacté M.A. Stephens lui-même pour qu'il nous communique comment il avait obtenu ces formules mais nous n'avons jamais eu de réponses...
	\end{tcolorbox}
	
	\pagebreak
	\paragraph{Test de QdA de Cramér–von Mises }\index{test de QdA de Cram\'er–von Mises}\mbox{}\\\\
	En statistiques, le test QdA de Cramér–von Mises (considéré par beaucoup comme un test non paramétrique) est utilisé pour juger de la qualité de l'ajustement d'une fonction de distribution cumulative $F^{*}$ par rapport à une fonction de distribution empirique donnée $F_{n}$, ou pour comparer deux distributions empiriques (le critère est nommé d'après Harald Cramér et Richard Edler von Mises qui l'ont proposé pour la première fois en 1928–1930). Il est également utilisé dans le cadre d'autres algorithmes, tels que l'estimation de la distance minimale. Il est généralement défini comme suit (comme nous l'avons déjà introduit juste plus haut):
	
	mais devrait mieux être écrit comme suit:
	
	Dans les applications à un échantillon, $F^{*}$ est la distribution théorique et $F_{n}$ est la distribution observée empiriquement. Les deux distributions peuvent également être toutes deux estimées empiriquement; c'est ce qu'on appelle le "cas à deux échantillons".
	
	Nous voulons dé,pmtrer la statistique du test de Cramér–von Mises (dit à $n$-degrés de liberté):
	
	où $U_i = F_0(X_{i:n})$ est la statistique d'ordre théorique de l'original (c'est pourquoi c'est un test paramétrique! Nous devons estimer les paramètres de la distribution d'origine):
	
	Notez que cette statistique est tabulée \footnote{Principalement et malheureusement uniquement pour la distribution Normale et ce également dans certains des logiciels statistiques les plus importants.} contre l'hypothèse nulle $H_0$ que $\hat{F}_{n}(t)=F_{0}(t)$.
	
	Soit $X_1,\ldots,X_n$ un échantillon aléatoire de taille $n$ tiré de $f(x)$. Le fonction cumulée empirique est:
	
	Soit  $U_i = F_0(X_{i:n})$, alors (le lecteur va voir qu'il y a des simplifications assez malignes...):
	
	Maintenant nous utilisons:
	
	Ainsi:
	
	Dès lors:
	
	\begin{tcolorbox}[colframe=black,colback=white,sharp corners]
	\textbf{{\Large \ding{45}}Exemple:}\\\\
	Considérons que nous avons les paramètres suivants d'un échantillon:
	
	En supposant une distribution Normale $\mathcal{N}(\bar{X},\sigma_X) $, nous obtenons pour chacune des valeurs de l'échantillon:
	
	Calculons maintenant la statistique de Cramér-von Mises en utilisant:
	
	Nous obtenons ici $W^2=0.1902842$ que l'on reporte dans la table de Cramér-von Mises pour $20$ degrés de liberté, ce qui donne une $p$-valuer de $0.006<5\%$. Par conséquent, nous rejetons $H_0$ et nous concluons que la fonction de distribution de $X$ n'est très probablement pas tirée d'une distribution Normale.
	\end{tcolorbox}
	
	\pagebreak
	\subsubsection{Estimation par densité de noyau}\label{kernel smoothing}
	En statistique, "\NewTerm{l'estimation par densité du noyau}\index{estimation par densité du noyau}" (EDN) ou "\NewTerm{lissage par noyau}\index{lissage par noyau}" est un moyen non paramétrique d'estimer la fonction de densité de probabilité d'un  variable aléatoire, mais aussi pour faire des régressions spéciales (\SeeChapter{voir la section Méthodes Numériques page \pageref{kernel regression}}). L'estimation par densité du noyau est un problème fondamental de lissage des données dans les affaires (business) et la science où des inférences sur la population sont faites, sur la base d'un échantillon de données fini. Dans certains domaines tels que le traitement du signal et l'économétrie, elle est également appelée "\NewTerm{méthode de la fenêtre de Parzen–Rosenblatt}\index{méthode de la fenêtre de Parzen–Rosenblatt}", d'après Emanuel Parzen et Murray Rosenblatt, qui sont généralement crédités pour l'avoir crée indépendamment dans sa forme actuelle.
	
	L'estimation non paramétrique d'une fonction de densité de probabilité est très importante dans tous les domaines utilisant des statistiques! On sait déjà que la technique utilisant les histogrammes est très délicate et discutable car il faut choisir la bonne largeur d'intervalle pour chacune des barres, ce qui est très subjectif et la règle de Sturges que nous avons vue précédemment peut être discutée!
	
	Pour introduire le sujet nous rappelons tout d'abord que nous avons dans le cas univarié:
	
	Ou ce qui s'écrit parfois plus élégamment:
	
	Sur la base de la méthode d'intégration numérique avec des rectangles (\SeeChapter{voir section Méthodes Numériques page \pageref{rectangle integration method}}), nous pouvons écrire:
	
	Par conséquent, nous avons:
	
	La probabilité ci-dessus peut être estimée par la fréquence relative des observations:
	
	Ce qui est parfois écrit:
	
	Une écriture alternative non triviale (et égale en valeur numérique!) est:
	
	où $ x_1, x_2, \ ldots, x_n $ sont les valeurs observées et:
	
	est appelée la fonction "\NewTerm{poids du noyau}" ou "\NewTerm{fonction de noyau de Rosenblatt univariée}\index{fonction de noyau de Rosenblatt}". La relation précédente est également souvent notée:
	
	Avec (notation un peu malheureuse à mon avis ...):
	
	et où $h$ est nommé la "\NewTerm{largeur de bande}".

	En général, une "\NewTerm {fonction noyau}\index{fonction noyau}" $K(x)$ est une fonction qui doit satisfaire les propriétés suivantes:
	\begin{enumerate}
		\item[P1.] La fonction du noyau est supérieure ou égale à zéro $K(x)\ge 0$
		
		\item[P2.] La fonction noyau est maximale en $x=0$
		
		\item[P3.] La fonction noyau est symétrique par rapport à l'axe vertical passant par $x=0$
		
		\item[P4.] La fonction noyau tend vers zéro loin de $x=0$ tel que $\lim_{|x|\rightarrow +\infty} K(x)=0$
		
		\item[P5.] La fonction noyau est une fonction de densité de probabilité
	\end{enumerate}
	La propriété P1 est évidente à contrôler et elle ne nécessite aucune preuve même dans des cas plus compliqués que la fonction noyau prise comme exemple précédemment.

	La propriété P2 est plus une contrainte d'implémentation qu'une propriété. Pour être satisfaiet, il faut simplement centrer les valeurs (mesures) avec lesquelles on travaille et supposer que la distribution est unimodale.
	
	La propriété P3 est également plus une contrainte d'implémentation qu'une propriété. Il nous sera utile pour simplifier certains calculs comme le biais de la moyenne attendue ou de l'écart type (voir plus bas).
	
	La propriété P4 s'applique par bon sens et découle de la propriété P2. Dans le cas des fonctions de densité de probabilité unimodales (raison pour laquelle cette propriété est souvent omise des énoncés).
	
	Pour vérifier la propriété P5, il suffit de montrer que la probabilité cumulée est unitaire.
	\begin{dem}
	On sait par construction de la fonction noyau précédente que sur chaque point est placé un rectangle de hauteur $1/2h$ et de largeur $2h$. Ensuite, l'idée est simplement d'écrire:
	
	\begin{flushright}
		$\blacksquare$  Q.E.D.
	\end{flushright}
	\end{dem}
	Quelques exemples de fonctions du noyau qui satisfont ces propriétés sont (les plus importantes ont déjà été disséquées dans les sous-sections précédentes) à une constante de normalisation près:
	\begin{itemize}
		\item Le noyau gaussien:
		
	
		\item Le noyau de Cauchy:
		
	
		\item Le noyau de Picard:
		
	
		\item Le noyau d'Epanechnikov:
		
	\end{itemize}
	Considérons maintenant quelques propriétés importantes de la fonction de densité empirique.

	Calculons son espérance (comme il s'agit de calculer l'espérance de la fonction de densité elle-même, cela nous donnera finalement le biais de l'espérance):
	
	Par exemple, pour un noyau gaussien, nous avons:
	
	À partir de ce moment, la variable dont nous prenons l'espérance, c'est-à-dire $\text{E}\left(K\left(\dfrac{x-x_i}{h}\right)\right)$, n'est plus une fonction de densité de probabilité mais une simple variable aléatoire transformée. Nous allons donc l'étendre au cas continu et le multiplier par sa vraie probabilité. L'idée très maligne étant que si le biais est nul, nous devrions trouver:
	
	et si on ne retombe pas sur ce résultat. La différence nous donnera le biais que nous recherchons!
	
	Vérifions cela:
	
	Posons pour simplifier les développements ultérieurs:
	
	et dès lors:
	
	Développons la fonction $f(x-hz)$ comme un développement de série de Taylor (\SeeChapter{voir section Séquences et Séries page  \pageref{taylor series}}) sur $z=0$:
	
	Dès lors nous avons:
	
	Mais, nous savons que si la fonction du noyau est symétrique et maximale à zéro (unimodale), alors nous pouvons inverser le signe de la première intégrale:
	
	La première intégrale est égale à $ 1 $ puisque la fonction noyau est une fonction de densité de probabilité. Ensuite, il reste:
	
	La deuxième intégrale calcule l'espérance. Or, puisque la fonction noyau est supposée être symétrique et maximale en zéro, il s'ensuit que son espérance est nulle et donc qu'il reste:
	
	Et donc le biais de la moyenne attendue est:
	
	Ainsi le biais diminue si on diminue $h$ (mais on va immédiatement montrer que c'est le contraire pour la variance).
	
	Calculons maintenant la variance:
	
	Parce que les $x_i$, $i=1,2,\ldots,n$, sont distribués indépendamment. Maintenant en utilisant le théorème de Huygens:
	
	Dès lors:
	
	Substitutant $z=\dfrac{x-t}{h}$ nous obtenons:
	
	Appliquer à nouveau une approximation de Taylor donne:
	
	Notez que si $n$ devient grand et $h$ devient petit alors l'expression ci-dessus devient approximativement:
	
	nous remarquons que la variance diminue à mesure que $h$ augmente.
	
	La sélection de la bande passante $h$ de l'estimateur de noyau est un sujet de recherche considérable. Il existent plus d'une méthode heuristique populaire mais au moins ils sont tous plus robustes que la règle de Sturge!
	
	Voici quelques exemples sans le script d'un lissage du noyau:
	\begin{figure}[H]
		\centering
		\includegraphics{img/arithmetics/kernel_smoothing_normal_random_variables.jpg}
		\captionsetup{width=0.7\linewidth}
		\caption[Estimation par densité du noyau de $100$ nombres aléatoires normalement distribués]{Estimation par densité du noyau de $100$ nombres aléatoires normalement distribués en utilisant différentes largeurs de bandes de lissage (source: Wikipédia)}
	\end{figure}
	et une comparaison entre un histogramme et un lissage par noyau:
	\begin{figure}[H]
		\centering
		\includegraphics[scale=0.8]{img/arithmetics/kernel_smoothing_histogram.jpg}
		\captionsetup{width=0.8\linewidth}
		\caption[]{Comparaison de l'histogramme (à gauche) et de l'estimation de la densité du noyau (à droite) construits à partir des mêmes données. Les noyaux individuels $ 6 $ sont les courbes en pointillés rouges, la densité du noyau estime les courbes bleues. Les points de données sont les petits tracé sur l'axe horizontal (source: Wikipédia)}
	\end{figure}
	Et une autre figure intéressante qui met en évidence le problème des histogrammes \cite{zucchini2003applied}:
	\begin{figure}[H]
		\centering
		\includegraphics[scale=0.9]{img/arithmetics/kernel_smoothing_histogram_issue.jpg}
		\captionsetup{width=0.8\linewidth}
		\caption{Histogrammes avec différentes largeurs de bandes et une estimation par noyaux de $f(x)$ pour le même échantillon}
	\end{figure}
	Le lecteur intéressé peut consulter nos livres compagnons sur R et MATLAB™ pour obtenir les détails sur la façon d'exécuter un lissage par densité de noyaux avec des logiciels scientifiques.
	
	\pagebreak
	\subsubsection{Tests de rapports des vraisemblances}\label{likelihood ratio tests}
	De nombreux logiciels de statistiques retournent, en complément des résultats statistiques classiques, une sortie qui est parfois nommée "\NewTerm{test de vraisemblance maximale}\index{test de vraisemblance maximale}" ou "\NewTerm{test du rapport de vraisemblance maximale}\index{test du rapport de vraisemblance maximale}"(TRV) pour un test dont le résultat est déjà donné. Ce test du maximum de vraisemblance est aussi parfois le résultat unique car les méthodes classiques ne fournissent pas de sortie calculable ou précise (c'est le cas du test G et du test de Poisson des moyennes). Comme nous le verrons ci-dessous, le principe est fondamentalement ... extrêmement simple!
	
	
	Rappelons que nous avons prouvé que pour une loi normale lors de notre étude des estimateurs de vraisemblance que:
	 
	Par exemple, si nous connaissons l'écart type mais que nous essayons d'estimer la moyenne, alors cette dernière relation peut s'écrire:
	 
 	Alors rien ne nous empêche d'écrire:
	
	Donc après quelques très petites manipulations élémentaires d'algèbre et d'arithmétique:
	
	Dans le cas de l'hypothèse suivante (test $T$ de Student en bilatéral) avec l'hypothèse d'homoscédasticité:
	
	Nous avons:
	
	Donc sous l'hypothèse nulle:
	
	Ce qui est souvent noté:
	
	et est le "\NewTerm{test du rapport de vraisemblance pour la comparaison des moyennes d'une loi Normale}" ou "\NewTerm{test du maximum de vraisemblance de la qualité d'ajustement pour la Normalité}"..
	\begin{tcolorbox}[title=Remarque,colframe=black,arc=10pt]
	Certains logiciels utilisent pour des raisons que j'ai été incapables de prouver une distribution chi carré $\chi_ {nd-1}^2$, où $d$ est le nombre de paramètres de la loi de distribution supposée implicite ($d=1$ pour une loi de Poisson ou $d=2$ pour une loi Normale, etc.).
	\end{tcolorbox}
	Nous voyons donc déjà, avant de continuer, qu'un test de rapport de vraisemblance est un test statistique utilisé pour comparer la qualité d'ajustement de deux modèles, dont l'un (le modèle nul) est un cas particulier de l'autre (le modèle alternatif). Le test est basé sur le rapport de vraisemblance, qui exprime le nombre de fois que les données sous un modèle soient plus vraisemblable par rapport à à un autre. Ce rapport de vraisemblance, ou de manière équivalente son logarithme, peut alors être utilisé évidemment pour calculer une $p$-valeur, ou comparé à une valeur critique pour décider de rejeter le modèle nul en faveur du modèle alternatif. Lorsque le logarithme du rapport de vraisemblance est utilisé, la statistique est appelée "\NewTerm{statistique du rapport de vraisemblance logarithmique}\index{statistique du rapport de vraisemblance logarithmique}", et la distribution de probabilité de cette statistique de test, en supposant que la valeur nulle modèle est vraie, peut être approximé en utilisant "\NewTerm{théorème de Wilks}\index{théorème de Wilk}" (approximation par une distribution chi-carré).
	
	\begin{tcolorbox}[title=Remarque,colframe=black,arc=10pt]
	Dans le cas de la distinction entre deux modèles, dont chacun n'a pas de paramètres inconnus, l'utilisation du test du rapport de vraisemblance peut être justifiée par le lemme de Neyman–Pearson, qui démontre qu'un tel test a la puissance la plus élevée parmi tous les concurrents. Mais cela sort du cadre de ce livre (du moins pour l'instant...).
	\end{tcolorbox}
	En pratique on retrouve plus souvent le test du rapport du maximum de vraisemblance avec la loi de Poisson. Nous rappelons que pour ces derniers nous avons également prouvé lors de notre étude des estimateurs de vraisemblance:
	
	Dès lors:
 	
	Comme pour le cas de la loi Normale, on peut écrire:
	
	Dès lors:
	
	Ce qu'il est coutume d'écrire:
	
	Supposons que nous voulions tester le fait que nos mesures ne s'écartent pas trop de la véritable attente, c'est-à-dire:
	
	Nous avons sous l'hypothèse nulle:
	
	Alors ... arrrrgh. Nous ne pouvons pas faire de déduction sur un tel résultat. C'est donc très ennuyeux de faire face à un tel ratio de vraisemblance ... Heureusement on sait que dans le cas où:
	
	la loi de Poisson tend vers une loi Normale (si $\mu$ est supérieur à environ $10$, alors la distribution Normale est une bonne approximation si une correction de continuité appropriée est effectuée). Ensuite, dans ces conditions, nous utilisons le résultat précédent de la loi Normale!
	
	Plus généralement, nous pouvons (malheureusement) trouver les relations finales ci-dessus sous les formes suivantes dans divers manuels universitaires:
	
	De nombreuses statistiques de test courantes telles que le test $Z$, le test $F$ de Fisher, le test $T$ de Student, le test du chi carré de Pearson et le test $G$ sont des tests pour des modèles imbriqués et peuvent être formulés sous forme de rapports de vraisemblance logarithmique ou d'approximations de ceux-ci.
	
	\begin{tcolorbox}[title=Remarque,colframe=black,arc=10pt]
	Certains logiciels proposent un autre test d'adéquation pour la distribution discrète, appelé "\NewTerm{estimation minimale du chi carré de qualité de l'ajustement}\index{estimation minimale du chi carré de qualité de l'ajustement}". L'idée est assez simple et nous l'illustrerons avec une distribution de Poisson. Prenons l'exemple suivant ($n=8$ pour une taille d'échantillon totale égale à $20$):
	
	L'estimation du chi carré minimum de la population du paramètre $\lambda$, est la valeur $\lambda$ qui minimise:
	
	Le calcul numérique montre que la valeur de $\lambda$ qui minimise la statistique du chi carré est d'environ $3.5242$. C'est l'estimation minimale du chi carré de $\lambda$. Pour cette valeur de $\lambda$, la statistique du chi carré est d'environ $3.062764$. Il y a $10$ cellules. Si l'hypothèse nulle avait spécifié une seule distribution, plutôt que d'exiger l'estimation de $\lambda$, alors la distribution de l'hypothèse nulle de la statistique de test serait une distribution du chi carré avec $10-1=9$ degrés de liberté.
	\end{tcolorbox}
	
	\subsection{Robustesse et Statistiques non paramétriques}
	Dans le domaine des statistiques inférentielles et des tests d'hypothèses, la robustesse est un concept récurrent (les banques sont obligées de faire des stress tests / crash tests de leurs modèles de risque par exemple). Nous avons déjà mentionné ce fait précédemment.
	
	\textbf{Définitions (\#\mydef):}
	\begin{enumerate}
		\item[D1.] Un test est qualifié de "\NewTerm{test robuste}\index{test robuste}" s'il reste valide alors que les hypothèses de l'application ne sont pas toutes satisfaites. Cela peut être une taille d'échantillon quelque peu faible ou une distribution de probabilité (distribution Normale pour les tests paramétriques) qui n'est pas très bien satisfaite, ou les valeurs aberrantes qui influencent trop le résultat du test. Par exemple, l'ANOVA est robuste par rapport à l'hypothèse de Normalité mais pas comparée à celle de l'homoscédasticité.
		
		\item[D2.] Un indicateur est nommé "\NewTerm{indicateur robuste}\index{indicateur robuste}" s'il n'est pas très sensible à la présence de valeurs aberrantes (le coefficient de corrélation ou la moyenne, par exemple, ne sont pas des indicateurs très robustes).
		
		\item[D3.] Plus généralement, un modèle est nommé "\NewTerm{modèle robuste} \index{modèle robuste}" lorsqu'il permet une extension des résultats (en temps ou pour une population). La robustesse est également applicable à une régression multiple, à des tests statistiques ou à un tableau de bord, ou à une planification de projets.
	\end{enumerate}
	Par conséquent, à moins d'être uniquement descriptif, vos analyses statistiques devront respecter certaines règles afin que leurs résultats soient généralisables.
	
	Première condition d'une bonne robustesse: les données! Intuitivement, nous savons tous que nous ne transformons pas un cas particulier en généralité (ce qui ne relèverait pas du domaine de la statistique mais des discussions de comptoir). Des données suffisantes créent des modèles fiables et solides. Par exemple, les prévisions dérivées d'une série chronologique montrant la saisonnalité nécessitent au moins trois ou quatre ans d'historique.
	
	Cependant, la quantité n'est pas suffisant, il faut aussi la qualité (pour la plupart des managers, la qualité est un concept très difficile à appréhender)!! En effet, il vaut mieux éviter de faire une étude sur des informations peu fiables qui peuvent conduire à des décisions coûteuses. De plus, il est recommandé d'être très prudent avec la manipulation des valeurs aberrantes. Si cela n'est pas possible, nous nous tournons vers des méthodes plus appropriées, telles que celles qui utilisent la médiane plutôt que la moyenne, les classent plutôt que les valeurs ou les techniques de filtrage.
	
	\textbf{Définition (\#\mydef):} La "\NewTerm{Statistique non paramétriques}\index {statistiques non paramétrique}" est la branche des statistiques qui ne repose pas uniquement sur des familles paramétrées de distributions de probabilités (des exemples courants de paramètres sont la moyenne et la variance).
	
	Le terme "statistiques non paramétriques" a été défini de manière imprécise des deux manières suivantes (entre autres ...):
	\begin{enumerate}
		\item Ils ne reposent pas sur des hypothèses selon lesquelles les données sont tirées d'une distribution de probabilité donnée. Nous disons alors qu'il s'agit de "\NewTerm{méthodes libre de distributions}\index{libre de distributions}".

		\item Ils ne reposent sur aucune opération arithmétique de toutes les données individuelles (comme les "rangs" ou les "étendues" par exemple!)
	\end{enumerate}
	Comme nous le verrons plus loin et comme nous l'avons déjà vu avec quelques cas particuliers plus haut, il existe pas mal de statistiques non paramétriques et de tests non paramétriques (gardez à l'esprit que même s'ils ne sont pas basés sur des familles de distributions paramétrées, ils ont d'autres hypothèses qui ne sont visibles que lorsque vous étudiez leur preeuve mathématique !!!). Plutôt que de les énumérer tous ici, nous voulons plutôt la réponse la plus courante à la question de nombreux praticiens qui est d'avoir une liste non exhaustive des équivalents non paramétriques à certains tests et statistiques paramétriques. Voici donc un tableau qui répond pour la part la plus importante à cette question:
	\begin{table}[H]
		\centering
		\resizebox{\textwidth}{!}{\begin{tabular}{|l|l|l|}
		\hline
		\rowcolor[HTML]{9B9B9B} 
		\multicolumn{1}{|c|}{\cellcolor[HTML]{9B9B9B}\textbf{Tests Paramétriques}} & \multicolumn{1}{c|}{\cellcolor[HTML]{9B9B9B}\textbf{Tests Non-paramétriques}} & \multicolumn{1}{c|}{\cellcolor[HTML]{9B9B9B}\textbf{Caractéristiques principales}} \\ \hline
		 & Test du signe à $1$ échantillon & \begin{tabular}[c]{@{}l@{}}Test sur la médiane de données \\ d'une distribution non-symétrique\end{tabular} \\ \cline{2-3} 
		\multirow{-2}{*}{Tests $T$ ou $Z$ à $1$ échantillon} & Test de Wilcoxon à $1$ échantillon & \begin{tabular}[c]{@{}l@{}}Test sur la médiane de données \\ d'une distribution symétrique\end{tabular} \\ \hline
		Test $T$ à $2$ échantillons & test de Mann-Withney & \begin{tabular}[c]{@{}l@{}}Test sur deux médianes en utilisant les rangs des\\ données échantillonnées\end{tabular} \\ \hline
		 & Test de Kruskal-Wallis & \begin{tabular}[c]{@{}l@{}}Test sur l'égalité des médianes de\\ deux populations ou plus. Plus\\ puissant que le test de la médiane de Mood,\\mais moins robuste aux valeurs aberrantes\end{tabular} \\ \cline{2-3} 
		\multirow{-2}{*}{ANOVA à une voie} & Test de la médiane de Mood & \begin{tabular}[c]{@{}l@{}}Test sur l'égalité des médianes de\\ deux populations ou plus. Plus\\ robuste aux valeurs aberrantes que le test de Kruskal-\\ Wallis, mais moins puissant.\end{tabular} \\ \hline
		ANOVA à deux voies randomisée par blocs & Test de Friedman & \begin{tabular}[c]{@{}l@{}}Test sur les médianes, utilisant des\\ expériences randomisées par blocs\end{tabular} \\ \hline
		Coefficient de corrélation de Pearson & Coefficient de corrélation de  Spearman & \begin{tabular}[c]{@{}l@{}}Mesure la dépendance statistique entre \\deux variables utilisant les rangs\end{tabular} \\ \hline
		\begin{tabular}[c]{@{}l@{}}Test de QdA d'Anderson-Darling, Shapiro-Wilk, \\ Kolmogorov-Smirnov\end{tabular} & Tests Mann–Whitney U ou Somme des rangs de Wilcoxon & \begin{tabular}[c]{@{}l@{}}Teste si deux échantillons sont tirés de  \\la même distribution, par rapport à\\  une hypothèse alternative donnée.\end{tabular} \\ \hline
		\end{tabular}}
		\caption{Quelques tests / statistiques paramétriques et leur équivalent non paramétrique}
	\end{table}
	
	\pagebreak
	\subsubsection{Statistiques de Rangs}
	\textbf{Définition (\#\mydef):} Les Statistiques de Rangs, également nommées "\NewTerm{statistiques d'ordre}\index{statistiques d'ordre}" sont définies comme l'ensemble des techniques de calculs statistiques et d'inférence statistique qui ont pour objectif principal de se débarrasser de la connaissance d'une distribution paramétrique et d'utiliser uniquement les rangs (ordre) des mesures. C'est un outil très puissant et beaucoup utilisé pour les statistiques non paramétriques!
	
	Comme nous l'avons déjà mentionné, nous parlons de tests paramétriques lorsque nous stipulons que les données proviennent d'une distribution connue donnée. Dans ce cas, les caractéristiques des données peuvent être résumées à l'aide des paramètres estimés sur l'échantillon, la procédure de test ultérieure se concentrant alors uniquement sur ces paramètres.
	
	\begin{tcolorbox}[title=Remarque,colframe=black,arc=10pt]
	Nous renovoyons le lecteur intéressé à en savoir plus mais sans détails mathématiques à l'excellent livre de Gopal K. Kanji qui contient une présentation sommaire de plus de $100$ tests paramétriques et non paramétriques avec pour chacun un petit exemple pratique. Malheureusement, ce livre devrait être publié dans une nouvelle édition avec les $200$ tests statistiques les plus utilisés pour être presque complet et parfait.
	\end{tcolorbox}	
	
	Les tests non paramétriques (comme les tests du chi carré déjà vus) éliminent l'étape nécessaire consistant à estimer les paramètres de la distribution avant de faire le test d'hypothèse. En général, les conclusions tirées des tests non paramétriques ne sont pas aussi puissantes que les conclusions paramétriques. Cependant, comme les tests non paramétriques font moins d'hypothèses, ils sont plus flexibles, plus robustes et applicables aux données non quantitatives.
	
	Lorsque les données sont quantitatives, les tests non paramétriques les plus courants transforment les valeurs en rangs. Le nom "\NewTerm{tests de rangs}\index{tests de rangs}" est alors souvent rencontré. Lorsque les données sont qualitatives, seuls les tests non paramétriques peuvent être utilisés.
	
	\paragraph{$L$-Statistiques}\label{L-statistics}\mbox{}\\\\
	Avant d'aborder les distributions et tests non paramétriques, donnons quelques définitions que le lecteur peut trouver dans la littérature hyperspécialisée et dont nous avons évité l'utilisation du moins jusqu'à présent...
	
	La médiane, la moyenne et l'étendue suggèrent l'utilisation de combinaisons linéaires des composantes du vecteur statistique d'ordre.
	
	\textbf{Définition (\#\mydef):} Ainsi, désignons par $X_{(1)}\geq X_{(2)},...,\geq X_{(n)}$, les statistiques d'ordre (valeurs ordonnées par ordre décroissant par leur rang et numérotées). On définit alors lA "\NewTerm{$L$-statistique}\index{$L$-statistique}" par:
	
	Le plus connu des "\NewTerm{$L$-estimateurs}\index{$L$-estimateurs}" est la moyenne arithmétique pour laquelle:
	
	Le deuxième $L$-estimateur le plus connu est la médiane pour laquelle nous avons lorsque $n$ est impair:
	
	et quand $n$ est pair:
	
	Le troisième $L$-estimateur le plus connu est l'étendue pour laquelle nous avons:
	
	Il existe d'autres $L$-estimateur empiriques mais nous nous arrêterons ici car la liste est assez longue.
	
	\paragraph{Lois de distributions de rangs}\mbox{}\\\\
	La statistique d'ordre $k$ est la $k$-ème plus petite valeur d'un échantillon de  taille $n$ d'une distribution aléatoire $F(x)$. Soit les variables aléatoires $X_1, X_2, ..., X_n$ indépendantes et identiquement distribuées. Étiquetons la plus petite valeur $X_{(1)}$, la plus petite suivante $X_{(2)} $ et ainsi de suite jusqu'à $X_{(n)}$. Alors la valeur de la statistique d'ordre $k$ est $X_(k)$, pour$1\leq k\leq n$.
	
	Soit $F(x)$ la fonction de distribution des $ X_i $ et $f(x)$ la fonction de densité. Déterminons d'abord la distribution de la statistique d'ordre $k$, c'est-à-dire:
	
	Pour avoir $X_{(k)}\leq x$, nous devons avoir au moins $k$ des $n$ $X_i$ inférieurs ou égaux à $x$. Chacun des $n$ essais indépendants a une probabilité $F(x)$ d'être inférieur ou égal à $x$, donc le nombre d'essais inférieur ou égal à $x$ a une distribution binomiale avec  $n$ essais et probabilité $F(x)$. C'est-à-dire:
	
	Pour trouver la fonction de densité, nous prenons la dérivée par rapport à $x$:
	
	Rédigé sous cette forme, il est évident que la majorité des termes s'annulent. Cela laisse:
	
	
	\paragraph{Intervalles de confiance pour les centiles libre de toute distribution}\mbox{}\\\\
	Plus tôt, nous avons appris (parmi beaucoup d'autres choses) comment calculer un centile d'échantillon comme une estimation ponctuelle d'un centile de population (ou de répartition). Tout comme il est judicieux de calculer les intervalles de confiance pour d'autres paramètres de population, tels que les moyennes et les variances, il est plus que judicieux d'apprendre à calculer un intervalle de confiance pour les centiles d'une population. C'est sur quoi nous allons travailler ci-dessous. Comme le suggère le titre du sujet, nous ne ferons aucune hypothèse sur la distribution des données, excepté sur le fait que la distribution doit être continue!
	
	Comme c'est généralement le cas, motivons la méthode de calcul d'un intervalle de confiance pour la médiane $M_e$ d'une population au moyen d'un exemple concret \footnote{C'est un exemple bien connu que vous pouvez trouver dans de nombreux manuels universitaire et sites Internet dont l'auteur original semble impossible déterminer.}. Supposons que $Y_1 < Y_2 < Y_3 < Y_4 < Y_5$ sont les statistiques d'ordre d'un échantillon aléatoire de taille $n = 5$ à partir d'une distribution continue. Ce que nous avons étudié plus haut nous dit que $Y_3$ sert de bon estimateur ponctuel de la médiane $M_e$. Voyons ce que nous pouvons construire comme intervalle de confiance étant donné que nous avons ces statistiques de rangs à notre disposition. Eh bien, supposons que nous suggerons un intervalle contraint par les statistiques du premier et du cinquième ordre, c'est-à-dire $[Y_1, Y_5]$, et qui servirait de bon intervalle. Dans quelle mesure pouvons-nous être sûrs que cet intervalle $[Y_1, Y_5]$ contiendrait la médiane de population inconnue $M_e$? Pour répondre à cette question, il suffit de calculer la probabilité suivante:
	
	Le calcul de la probabilité se réduit à un simple calcul binomial une fois que nous avons compris toutes les façons dont la médiane de population $M_e$ est prise en sandwich entre $Y_1$ et $Y_5$. Eh bien, la médiane de population $M_e$ est prise en sandwich entre $Y_1$ et $Y_5$, si la statistique de premier ordre est la seule statistique d'ordre inférieure à la médiane $M_e$:
	\begin{figure}[H]
		\centering
		\begin{tikzpicture}[scale=7]
		\draw[->, thick] (-0.1,0) -- (1.2,0);
		\foreach \x/\xtext in {0/$\color{red}{Y_1}$,0.2/$M_e$,0.4/$\color{red}{Y_2}$,0.6/$\color{red}{Y_3}$,0.8/$\color{red}{Y_4}$,1.0/$\color{red}{Y_5}$}
		    \draw[thick] (\x,0.5pt) -- (\x,-0.5pt) node[below] {\xtext};
		\end{tikzpicture}
	\end{figure}
	La médiane $M_e$ de la population est prise en sandwich entre $Y_1$ et $Y_5$, si les deux premières statistiques d'ordre sont les seules statistiques de commande inférieures à la médiane $M_e$:
	\begin{figure}[H]
		\centering
		\begin{tikzpicture}[scale=7]
		\draw[->, thick] (-0.1,0) -- (1.2,0);
		\foreach \x/\xtext in {0/$\color{red}{Y_1}$,0.2/$\color{red}{Y_2}$,0.4/$M_e$,0.6/$\color{red}{Y_3}$,0.8/$\color{red}{Y_4}$,1.0/$\color{red}{Y_5}$}
		    \draw[thick] (\x,0.5pt) -- (\x,-0.5pt) node[below] {\xtext};
		\end{tikzpicture}
	\end{figure}
	La médiane $M_e$ de la population est prise en sandwich entre $Y_1$ et $Y_5$, si les trois premières statistiques d'ordre sont inférieures à la médiane $M_e$, et les statistiques de quatrième et cinquième ordre sont supérieures à $M_e$:
	\begin{figure}[H]
		\centering
		\begin{tikzpicture}[scale=7]
		\draw[->, thick] (-0.1,0) -- (1.2,0);
		\foreach \x/\xtext in {0/$\color{red}{Y_1}$,0.2/$\color{red}{Y_2}$,0.4/$\color{red}{Y_3}$,0.6/$M_e$,0.8/$\color{red}{Y_4}$,1.0/$\color{red}{Y_5}$}
		    \draw[thick] (\x,0.5pt) -- (\x,-0.5pt) node[below] {\xtext};
		\end{tikzpicture}
	\end{figure}
	Et, la médiane $M_e$ de la population est prise en sandwich entre $Y_1$ et $Y_5$, si la statistique de cinquième ordre est la seule statistique d'ordre supérieure à la médiane $M_e$:
	\begin{figure}[H]
		\centering
		\begin{tikzpicture}[scale=7]
		\draw[->, thick] (-0.1,0) -- (1.2,0);
		\foreach \x/\xtext in {0/$\color{red}{Y_1}$,0.2/$\color{red}{Y_2}$,0.4/$\color{red}{Y_3}$,0.6/$\color{red}{Y_4}$,0.8/$M_e$,1.0/$\color{red}{Y_5}$}
		    \draw[thick] (\x,0.5pt) -- (\x,-0.5pt) node[below] {\xtext};
		\end{tikzpicture}
	\end{figure}
	Cela signifie que pour calculer la probabilité $P(Y_1 \leq M_e \leq Y_5)$, nous devons calculer la probabilité de chacun des événements ci-dessus. Maintenant, si nous désigner par $W$ le nombre de $X_i < M_e$, alors $W$ est une variable aléatoire binomiale avec $n$ essais mutuellement indépendants et probabilité de succès $p = P (X_i <M_e) = 0.5$  Et, en examinant les événements décrits ci-dessus, la probabilité souhaitée est calculée comme suit:
	
	La fonction de masse de probabilité binomiale (ou, alternativement, la table binomiale) rend le calcul simple:
	
	Ainsi, la probabilité que l'intervalle aléatoire $[Y1, Y5]$ contienne la médiane $M_e$ est de $0.9376$. Nous n'avons pas toujours autant de chance d'arriver à un coefficient de confiance décent dès notre premier essai. Parfois, nous devons réessayer en visant à obtenir un coefficient de confiance d'au moins $90\%$, mais aussi proche de $95\%$ que possible. Dans ce cas, le coefficient de confiance pour l'intervalle $[Y2, Y4]$ est:
	
	De toute évidence, nous ferions mieux de nous en tenir à l'intervalle $ [Y_1, Y_5] $ dans ce cas.
	
	Notez qu'en général nous avons alors:
	
	La méthode que nous avons apprise pour trouver un intervalle de confiance pour la médiane d'une distribution continue peut être facilement étendue afin que nous puissions trouver un intervalle de confiance pour tout centile $\pi_p$. La seule chose que nous devons changer est la probabilité de succès, c'est-à-dire que $X_i$ est inférieur à $\pi_p$:
	
	Ensuite, le coefficient de confiance exact est calculé comme avant en utilisant la distribution binomiale avec les paramètres $n$ et $p$:
	
	
	\paragraph{Test de la somme des rangs de Wilcoxon}\mbox{}\\\\
	L'idée du "\NewTerm{test de la somme des rangs de Wilcoxon}\index{test de la somme des rangs de Wilcoxon}\label{Wilcoxon rank sum test}" est la suivante: si nous collectons deux échantillons de mesure, et que nous stockons les valeurs dans l'ordre, l'alternance des $X_i$ (de taille $n_x$) et $Y_i$ (de taille $n_y$) devrait être assez stable si les deux lois de distribution  $F$ et $G$ d'échantillons suivent respectivement la même distribution de probabilité. C'est donc comme un "contrôle d'adéquation".
	
	Cela revient en fait à dire que l'hypothèse nulle est: \textit{La distribution des rangs dans les deux groupes est égale}.
	
	Formellement:
	
	En langage clair, l'hypothèse nulle est que la probabilité qu'une observation aléatoire du groupe $A$ dépasse l'observation appariée tirée du groupe $B$ est de moitié (c'est-à-dire qu'une observation aléatoire du groupe $A$ a autant de probabilité d'être supérieure que comme étant inférieure à son observation appariée dans le groupe $B$).
	
	La valeur pratique de ceci est difficile à voir, et donc dans de nombreux endroits, y compris les manuels de référence universitaire, l'hypothèse nulle est présentée comme les \og deux populations ont des médianes égales \fg{}. L'hypothèse nulle actuelle peut être exprimée comme l'hypothèse mentionnée avec la médiane, mais seulement sous l'hypothèse supplémentaire que les formes des distributions sont identiques dans chaque groupe!
	
	En d'autres termes, notre interprétation du test comme la comparaison des médianes des distributions exige le déplacement du paramètre de localisation uniquement. Comme cela est rarement vrai et jamais évalué, nous devrions probablement faire preuve d'une extrême prudence dans l'utilisation, et en particulier dans l'interprétation, du test de la somme des rangs de Wilcoxon.
	
	Il n'est donc pas comme le test d'adéquation du chi-carré de comparer des mesures à une loi théorique, mais à d'autres mesures d'une même loi hypothétique.
	
	\begin{tcolorbox}[title=Remarque,colframe=black,arc=10pt]
	Le test de la somme des rangs de Wilcoxon est un test non paramétrique, utilisé donc pour les "Tests de signification d'hypothèse nulle non paramétriques", car nous n'avons besoin d'aucune mesure de dispersion (ie variance) ou de localisation (ie moyenne) des variables aléatoires. De plus, il s'agit d'un test dit "robuste" dans le sens où il ne suppose pas la Normalité des données.
	\end{tcolorbox}	
	Prenons un exemple avant d'aborder l'aspect théorique. Voici deux échantillons de taille $10$ ($n_x = n_y$) de variables quantitatives:
	
	\begin{tcolorbox}[title=Remarque,colframe=black,arc=10pt]
	Le test de la somme des rangs de Wilcoxon peut très bien être utilisé pour les variables ordinales (mais seulement tant qu'elles sont dans un nombre acceptable). En règle générale, le test de la somme des rangs de Wilcoxon est également utilisé pour analyser les réponses aux enquêtes auprès des entreprises en utilisant une échelle de Likert en $7$ points.
	\end{tcolorbox}	
	Voici les statistiques d'ordre de l'échantillon de taille $20$ ($N=n_x+n_y$) regroupés et ordonnés (les $10$ valeurs du premier échantillon sont soulignées):
	
	Les valeurs du premier échantillon $X$ (nommé "\NewTerm{valeurs de traitement}\index{valeurs de traitement}") dans cet exemple semblent être plus petites que celles du deuxième échantillon $Y$ (nommé "\NewTerm{valeurs de contrôle}\index{valeurs de contrôle}") que nous représentons souvent dans un diagramme comme suit (en trichant un peu avec Microsoft Excel 11.8346):
	\begin{figure}[H]
		\centering
		\includegraphics{img/arithmetics/wilcoxon_rank_sum_test_ordered_values.jpg}
		\caption[]{Comparaison des valeurs ordonnées de deux échantillons dans Microsoft Excel 11.8346}
	\end{figure}
	\begin{tcolorbox}[title=Remarque,colframe=black,arc=10pt]
	Vous devez faire attention au test de Wilcoxon que nous étudions ici et au test de Kruskal-Wallis que nous verrons plus tard. En effet, beaucoup de gens pensent que parce qu'ils ne sont pas paramétriques, ils n'ont aucune hypothèse ... Cependant, comme nous le savons, c'est faux !!! Un test non paramétrique est en effet indépendant des paramètres de distribution mais il peut y avoir d'autres hypothèses. Dans le cas du test de Wilcoxon et du test de Kruskal-Wallis, l'hypothèse est (bien évidemment ...) que les paramètres d'échelle entre les deux échantillons sont égaux (sinon ce serait un non-sens de comparer deux choses que l'on sait déjà comme n'étant pas comparables!).
	\end{tcolorbox}	
	L'idée est alors de savoir si cette tendance est statistiquement significative? C'est-à-dire si nous avons une différence du genre $F<G$ entre leurs lois de distribution respectives:
	\begin{figure}[H]
		\centering
		\includegraphics{img/arithmetics/wilcoxon_rank_sum_distribution_comparison.jpg}
		\caption[]{Exemple générique de comparaison de deux distributions dans Microsoft Excel 11.8346}
	\end{figure}
	ou si elles peuvent être considérés comme identiques. Pour cela, nous devons examiner le concept de "\NewTerm{rang}\index{rang}":
	
	\textbf{Définition (\#\mydef):} Étant donné un échantillon aléatoire $X_1,X_2,...,X_n$ de taille $n$ de toute loi statistique continue, nous notons $R_i$ le rang des $X_i$ ordonnés dans un échantillon de population. Le rang $i$ est un entier non nul strictement positif et compris entre $1$ et $n$.
	
	\begin{tcolorbox}[colframe=black,colback=white,sharp corners]
	\textbf{{\Large \ding{45}}Exemple:}\\\\
	Dans:
	
	Nous avons respectivement les "statistiques d'ordre" suivantes:
	
	Une fois le concept de "rang" défini et calculé, regardons la somme dans le cadre de notre exemple avec deux échantillons:
	
	Le somme des  rangs notée traditionnellement $W_x$ ($W$ pour Wilcoxon) pour le premier échantillon est donc:
	
	et pour le deuxième échantillon:
	
	\end{tcolorbox}
	Les valeurs $W_x,W_y$ sont nommées "\NewTerm{statistiques de Wilcoxon}\index{statistique de Wilcoxon}".
	
	On voit donc déjà qu'il existe bien une différence qui semble a priori significative en termes de classement entre les deux échantillons. Le problème reste de construire un outil mathématique rigoureux pour inférer un fait avec une certaine certitude.
	
	Pour cela, introduisons d'abord la moyenne des rangs en utilisant le résultat présenté dans la section Séquences et Séries en ne considérant qu'un seul échantillon:
	
	En calculant cela, nous remarquons rapidement qu'il s'agit de l'espérance la la distribution discrète uniforme que nous avions étudié plus haut dans cette section pour une variable aléatoire discrète avec des valeurs comprises entre $1$ et $n$, ce qui est exactement la définition du rang! Ainsi, nous avons le rang qui aura comme valeur pour la moyenne et la variance pour l'ensemble de la population:
	
	Pour ceux qui trouvent cette analogie discutable, nous donnons juste en dessous la preuve de la variance en utilisant la relation de Huygens et la somme des carrés d'entiers positifs prouvée dans la section Séquences et Séries:
	
	Mais évidemment pour un seul échantillon cela n'a aucun intérêt! Reprenons nos deux séries $X_i,Y_i$ respectivement de taille égale $n_x=10,n_y=10$ sans distinction:
	
	On a alors les indicateurs statistiques de rangs sans distinction (il faut se rappeler que l'on ne sait toujours pas à ce niveau du développement mathématique si cela sera utile ou non):
	
	et les indicateurs statistiques des rangs mais cette fois avec distinction:
	
	Nous avons alors les indicateurs statistiques de localisation suivants:
	
	Ces calculs maintenant effectués, nous n'avons rien de concret encore rigoureux concernant le test de la somme des rangs de Wilcoxon qui a pour but, pour rappel, de vérifier si les deux échantillons suivent la même loi ou non (et ont donc les mêmes moments tels l'espérance, la variance, la médiane, etc.).
	
	Pour avancer, considérons les $n_x$ valeurs de l'échantillon $X$. On sait (\SeeChapter{voir section Probabilités page \pageref{binomial coefficient}}) alors qu'il y a:
	
	nombre d'arrangements possibles des $X_i$ dans la population des échantillons et si le test de la somme des rangs de Wilcoxon est vérifié (c'est-à-dire que les lois de probabilité sont les mêmes pour les deux échantillons), les divers arrangements devraient être également probables.
	
	Par exemple, si nous prenons $2$ échantillons avec respectivement chacun $2$  mesures ($2$ variables aléatoires de traitement et $2$ variables aléatoires de contrôle), nous avons:
	
	arrangements tous différents:
	
	Mais ... malheureusement ... ce n'est pas ce que nous voulons dans notre cas car nous aimerions déjà pouvoir distinguer les deux échantillons et aussi ne pas prendre en compte les arrangements qui consistent uniquement en une permutation des variables du même échantillon. On a alors (\SeeChapter{voir section Probabilités page \pageref {binomial coefficient}}):
	
	combinaisons possibles! En effet, avec deux échantillons ayant deux variables de traitement ($X$) et deux variables de contrôle ($Y$), nous avons ($W_S$ est la somme des rangs de la dernière colonne):
	\begin{table}[H]
		\centering
		\definecolor{gris}{gray}{0.85}
				\begin{tabular}{|c|c|c|}
					\hline
	\multicolumn{1}{c}{\cellcolor{black!30}\pbox{20cm}{\textbf{Rangs possibles} \\ \textbf{(contrôles)}}} & \multicolumn{1}{c}{\cellcolor{black!30}\pbox{20cm}{\textbf{Rangs possibles} \\ \textbf{(traitements)}}} & \multicolumn{1}{c}{\cellcolor{black!30}$W_S$} \\ \hline
			$1,2$ & $3,4$ & $7$\\ 
			$1,3$ & $2,4$ & $6$\\ 
			$1,4$ & $2,3$ & $5$\\ \hline
			$2,3$ & $1,4$ & $4$\\ 
			$2,4$ & $1,3$ & $4$\\ \hline
			$3,4$ & $1,2$ & $3$\\ \hline
		\end{tabular}
		\caption[]{Représentation par rangs de $2$ variables de traitement et de contrôle}
	\end{table}
	Si l'hypothèse nulle du test de la somme des rangs de Wilcoxon n'est pas rejetée, les classements des $6$ sont également probables. Nous concluons le tableau suivant:
	\begin{table}[H]
		\centering
		\definecolor{gris}{gray}{0.85}
		\begin{tabular}{|c|c|c|c|c|c|}
			\hline
\multicolumn{1}{c}{\cellcolor{black!30}\pbox{20cm}{\textbf{Valeur de $W_S$} }} & \multicolumn{1}{c}{\cellcolor{black!30}\pbox{20cm}{$3$}} & \multicolumn{1}{c}{\cellcolor{black!30}\pbox{20cm}{$4$}} & \multicolumn{1}{c}{\cellcolor{black!30}\pbox{20cm}{$5$}} & \multicolumn{1}{c}{\cellcolor{black!30}\pbox{20cm}{$6$}} & \multicolumn{1}{c}{\cellcolor{black!30}\pbox{20cm}{$7$}}\\ \hline
	\multicolumn{1}{c}{\cellcolor{black!30}\pbox{20cm}{\textbf{Probabilité}}} & $\dfrac{1}{6}$ & $\dfrac{1}{6}$ & $\dfrac{2}{6}$ & $\dfrac{1}{6}$ & $\dfrac{1}{6}$\\ 
	\multicolumn{1}{c}{\cellcolor{black!30}\pbox{20cm}{\textbf{Cumul}}} & $\dfrac{1}{6}$ & $\dfrac{2}{6}$ & $\dfrac{4}{6}$ & $\dfrac{5}{6}$ & $\dfrac{6}{6}$\\ \hline
		\end{tabular}
		\caption[]{Probabilités associées au test de la somme des rangs de Wilcoxon}
	\end{table}
	Ce tableau étant construit, supposons que l'on observe pour la somme des rangs des variables de traitement: $W_S = 7$. Le seuil d'un test unilatéral donnerait alors conformément au tableau ci-dessus:
	
	ou si nous obtenons $W_S=3$:
	
	On rejetterait donc l'hypothèse nulle d'une distribution identique entre les deux échantillons à toute limite supérieure (ou inférieure, respectivement) fixée à l'avance par la politique du laboratoire ... en test unilatéral ou bilatéral (raison pour laquelle certains logiciels statistiques donnent des valeurs de test unilatérales + valeurs de test bilatérales en même temps).
	
	Deux choses très importantes que vous devez noter pour ce qui va suivre sont les suivantes:
	\begin{enumerate}
		\item D'abord dans la construction du tableau ci-dessus (où nous reprenons la première partie ici):
		\begin{table}[H]
			\begin{center}
				\definecolor{gris}{gray}{0.85}
					\begin{tabular}{|c|c|c|c|c|c|}
						\hline
		\multicolumn{1}{c}{\cellcolor{black!30}\pbox{20cm}{\textbf{Valeurs de $W_S$} }} & \multicolumn{1}{c}{\cellcolor{black!30}\pbox{20cm}{$3$}} & \multicolumn{1}{c}{\cellcolor{black!30}\pbox{20cm}{$4$}} & \multicolumn{1}{c}{\cellcolor{black!30}\pbox{20cm}{$5$}} & \multicolumn{1}{c}{\cellcolor{black!30}\pbox{20cm}{$6$}} & \multicolumn{1}{c}{\cellcolor{black!30}\pbox{20cm}{$7$}}\\ \hline
				\multicolumn{1}{c}{\cellcolor{black!30}\pbox{20cm}{\textbf{Probabilité}}} & $\dfrac{1}{6}$ & $\dfrac{1}{6}$ & $\dfrac{2}{6}$ & $\dfrac{1}{6}$ & $\dfrac{1}{6}$\\ \hline
			\end{tabular}
			\end{center}
		\end{table}
		on voit qu'il y a une symétrie à la valeur $5$, ce qui signifie que la loi $W_S$ est symétrique dans ce cas particulier. Mais si nous prenons un autre exemple avec deux échantillons comprenant respectivement deux variables de contrôle et trois de traitements (deux variables aléatoires) nous aurions:
		\begin{table}[H]
			\begin{center}
				\definecolor{gris}{gray}{0.85}
					\begin{tabular}{|c|c|c|c|c|c|c|c|}
						\hline
		\multicolumn{1}{c}{\cellcolor{black!30}\pbox{20cm}{\textbf{Valeurs de $W_S$} }} & \multicolumn{1}{c}{\cellcolor{black!30}\pbox{20cm}{$6$}} & \multicolumn{1}{c}{\cellcolor{black!30}\pbox{20cm}{$7$}} & \multicolumn{1}{c}{\cellcolor{black!30}\pbox{20cm}{$8$}} & \multicolumn{1}{c}{\cellcolor{black!30}\pbox{20cm}{$9$}} & \multicolumn{1}{c}{\cellcolor{black!30}\pbox{20cm}{$10$}} & \multicolumn{1}{c}{\cellcolor{black!30}\pbox{20cm}{$11$}} & \multicolumn{1}{c}{\cellcolor{black!30}\pbox{20cm}{$12$}}\\ \hline
				\multicolumn{1}{c}{\cellcolor{black!30}\pbox{20cm}{\textbf{Probabilité}}} & $\dfrac{1}{10}$ & $\dfrac{1}{10}$ & $\dfrac{2}{10}$ & $\dfrac{2}{10}$ & $\dfrac{2}{10}$ & $\dfrac{1}{10}$ & $\dfrac{1}{10}$\\  \hline
			\end{tabular}
			\end{center}
		\end{table}
		le lecteur peut vérifier que quel que soit le nombre d'échantillons et le nombre de variables et de traitement de contrôle, le tableau de probabilité ci-dessus est toujours équilibré (enfin il y a une preuve mathématique de cela mais je la trouve inélégante...). Mais en fait c'est assez intuitif, comme les combinaisons $C_{n_x}^{n_x+n_y}$ sont indépendantes du fait que les rangs sont triés par ordre croissant ou décroissant, il faut donc qu'il y ait une symétrie.
		
		\item Deuxièmement, les valeurs des variables mesurées n'entrent pas en compte dans cette statistique paramétrique mais uniquement les valeurs tabulées des rangs avec leurs probabilités associées. En effet, comme vous l'avez peut-être remarqué, nous n'avions pas besoin de valeurs explicites des variables aléatoires pour construire le tableau ci-dessus!
	\end{enumerate}
	
	Maintenant, sachant que la loi $ W_S $ est symétrique et discrète nous aimerions calculer son espérance.
	
	La plus petite valeur possible de $W_S$ suppose qu'elle se trouve dans l'échantillon $X$ (les algorithmes informatiques déterminent automatiquement dans quel échantillon mais de toute façon en pratique, les échantillons ont presque toujours la même taille):
	
	La plus grande valeur possible est naturellement (rappelons que $N=n_x+n_y$):
	
	L'espérance de la somme des rangs de l'un des deux échantillons est alors:
	
	Alors finalement:
	
	Pour calculer la variance, qui nous sera utile pour faire si nécessaire une approximation que nous verrons plus loin, apparaît (malheureusement) la covariance car la connaissance des rangs donne des informations partielles sur les autres rangs. Nous avons donc:
	
	Nous savons déjà avec ce que nous venons de prouver ci-dessus que:
	
	Le problème reste le terme avec la covariance. Pour son calcul il existe des techniques rigoureuses en plusieurs pages et ... et aussi... un astuce bien plus courte. L'astuce consiste à utiliser la variable de classement global que nous notons $T_i$ avec $i=1...n+m$. Comme la somme des $T_i$ est une constante, alors nous avons:
	
	Il vient alors:
	
	On peut alors reprendre le calcul initial en remplaçant les covariances par leur expression, la dernière relation obtenue pour les covariances calculées sur le $T_i$ s'appliquant aussi (ce qui n'est pas forcément intuitif ... mais l'astuce fonctionne) aux $ R_i $:
	
	Finalement:
	
	C'est le même résultat que la méthodologie rigoureuse que l'on retrouve dans quelques rares manuels de référence.
	\begin{tcolorbox}[colframe=black,colback=white,sharp corners]
	\textbf{{\Large \ding{45}}Exemple:}\\\\
	Passons à un cas pratique pour le cas exact. Considérons donc $2$ échantillons  ayant $2$ variables de traitements $(X)$ et $2$ variables de contrôle $(Y)$ (c'est un peu simpliste et absurde comme exemple mais cela facilite l'aspect pédagogique ...) nous avons:
	
	Ainsi (la variable de traitement a donc les rangs $1$ et $3$ ce qui fait une somme des rang de $4$):
	
	Dès lors:
	
	Nous avons le tableau suivant comme nous l'avons montré ci-dessus:
	\begin{table}[H]
			\begin{center}
				\definecolor{gris}{gray}{0.85}
					\begin{tabular}{|c|c|c|c|c|c|}
						\hline
		\multicolumn{1}{c}{\cellcolor{black!30}\pbox{20cm}{\textbf{Valeurs de $W_S$} }} & \multicolumn{1}{c}{\cellcolor{black!30}\pbox{20cm}{$3$}} & \multicolumn{1}{c}{\cellcolor{black!30}\pbox{20cm}{$4$}} & \multicolumn{1}{c}{\cellcolor{black!30}\pbox{20cm}{$5$}} & \multicolumn{1}{c}{\cellcolor{black!30}\pbox{20cm}{$6$}} & \multicolumn{1}{c}{\cellcolor{black!30}\pbox{20cm}{$7$}}\\ \hline
				\multicolumn{1}{c}{\cellcolor{black!30}\pbox{20cm}{\textbf{Probabilité}}} & $\dfrac{1}{6}$ & $\dfrac{1}{6}$ & $\dfrac{2}{6}$ & $\dfrac{1}{6}$ & $\dfrac{1}{6}$\\  \hline
			\end{tabular}
			\end{center}
	\end{table}
	avec dans le cas présent:
	
	Si nous choisissons le seuil de confiance traditionnel à $5\%$ en bilatéral, nous avons selon le tableau ci-dessus que:
	
	Donc en d'autres termes nous voyons qu'il y a:
	\end{tcolorbox}
	
	\pagebreak
	\begin{tcolorbox}[colframe=black,colback=white,sharp corners]
	
	de probabilité cumulée que equationsoit compris entre $3$ et $7$ (la barre au-dessus du $6$ signifie pour rappel que ce chiffre se répète à l'infini). Donc forcément $4$ est compris dans l'intervalle bilatéral du $95\%$... et nous pouvons accepter l'hypothèse comme quoi les deux échantillons ne sont pas différents. La $p$-valeur correspondante en bilatéral est donc la moitié de $33.\bar{3}\%$.\\
	
	\begin{tcolorbox}[title=Remarque,colframe=black,arc=10pt]
	Au fait si nous voulions faire un exemple calculatoire manuel intéressant en jouant avec un seuil bilatéral de $5\%$ (soit de $2.5\%$ de chaque côté) il faudrait au moins $2$ échantillons avec $4$ variables aléatoires, soit $70$ combinaisons de rangs possibles. En-dessous de $4$ variables aléatoires par échantillons, il est évident que le test bilatéral à seuil de $95\%$ sera tel qu'on ne rejettera jamais l'hypothèse d'égalité...
	\end{tcolorbox}
	\end{tcolorbox}
	Si la taille des deux échantillons est assez grande (la majorité des praticiens considérent que chaque échantillon doit avoir au moins $20$ individus), il a été montré par simulations que nous pouvons faire l'approximation (utilisée par beaucoup de logiciels de statistiques):
	
	bien évidemment en déterminant ensuite toujours la $p$-valeur en bilatéral. Avec l'exemple précédent (n'ayant que $4$ individus au total), nous avons donc:
	
	Ce qui correspondant à une probabilité cumulée de $21.93\%$. Donc la $p$-valeur correspondante en bilatéral est d'environ $(1-22)\cong 44\%$  (à comparer à la valeur d'environ $33\%$ avec le cas exact).
	
	\paragraph{Test de la somme des rangs de Mann-Whitney}\mbox{}\\\\
	Le "\NewTerm{test de la somme des rangs de Mann-Withney}\index{test de la somme des rangs de Mann-Withney}" est au fait un test d'ajustement non-paramétrique très simple qui se déduit du test de la somme des rangs de Wilcoxon. Par ailleurs il en est inspiré à un tel point que nous l'appelons parfois dans l'industrie le \NewTerm{test de Wilcoxon-Mann-Withney}\index{test de Wilcoxon-Mann-Withney}" ou "\NewTerm{test d'ajustement de Wilcoxon-Mann-Withney}\index{test d'ajustement de Wilcoxon-Mann-Withney}" ou encore "\NewTerm{Mtest MWW}" (sans spécifier à chaque fois qu'il repose sur la somme des rangs).
	
	Le but de ce test, identiquement au test de la somme des rangs de Wilcoxon, est de trouver un moyen de vérifier que deux échantillons indépendants non nécessairement de même taille sont issus d'une même loi ou non (in extenso sont issus d'une même population ou non) mais avec une approche différente!
	
	\begin{tcolorbox}[title=Remarque,colframe=black,arc=10pt]
	Au même titre que le test de la somme des rangs de Wilcoxon, le test de la somme des rangs de Mann-Withney peut tout à fait être utilisé pour des variables ordinales (donc catégorielles mais à condition qu'elles soient en un nombre acceptable).
	\end{tcolorbox}
	
	Certains logiciels par ailleurs portent les choses à confusion car ils proposent le test de la somme des rangs de Wilcoxon sous le nom de test de de Mann-Withney... et inversement... et de plus n'indiquent pas ou ne proposent pas toujours le choix entre la version exacte ou approximative... Et en plus le test de la somme des rangs de Wilcoxon et celui de la somme des rangs signés que nous verrons plus loin n'est pas différencié.... donc attention! C'est typiquement un problème dont la source est l'absence d'une norme ISO définissant la terminologie et les options qui doivent être disponibles...

	Pour voir en quoi ce test consiste, construisons le tableau de rangs utilisant deux échantillons comprenant deux variables de contrôle et trois variables de traitement, nous avons alors:
	\begin{table}[H]
		\centering
		\definecolor{gris}{gray}{0.85}
		\begin{tabular}{|c|c|c|}
		\hline
\multicolumn{1}{c}{\cellcolor{black!30}\pbox{20cm}{\textbf{Rangs possibles} \\ \textbf{(Contrôles)}}} & \multicolumn{1}{c}{\cellcolor{black!30}\pbox{20cm}{\textbf{Rangs possibles} \\ \textbf{(Traitements)}}} & \multicolumn{1}{c}{\cellcolor{black!30}$W_S$} \\ \hline
		$1,2$ & $3,4,5$ & $12$\\ 
		$1,3$ & $2,4,5$ & $11$\\
		$1,4$ & $2,3,5$ & $10$\\  
		$1,5$ & $2,3,4$ & $9$\\ \hline
		$2,3$ & $1,4,5$ & $10$\\
		$2,4$ & $1,3,5$ & $9$\\  
		$2,5$ & $1,3,4$ & $8$\\ \hline
		$3,4$ & $1,2,5$ & $8$\\ 
		$3,5$ & $1,2,4$ & $7$\\ \hline
		$4,5$ & $1,2,3$ & $6$\\ \hline
		\end{tabular}
		\caption[]{Représentation des rangs de 3 variables de traitement et 2 de contrôle}
	\end{table}
	Dont nous déduisons le tableau suivant:
	\begin{table}[H]
		\centering
		\definecolor{gris}{gray}{0.85}
		\begin{tabular}{|c|c|c|c|c|c|c|c|}
		\hline
	\multicolumn{1}{c}{\cellcolor{black!30}\pbox{20cm}{\textbf{Valeurs de $W_S$} }} & \multicolumn{1}{c}{\cellcolor{black!30}\pbox{20cm}{$6$}} & \multicolumn{1}{c}{\cellcolor{black!30}\pbox{20cm}{$7$}} & \multicolumn{1}{c}{\cellcolor{black!30}\pbox{20cm}{$8$}} & \multicolumn{1}{c}{\cellcolor{black!30}\pbox{20cm}{$9$}} & \multicolumn{1}{c}{\cellcolor{black!30}\pbox{20cm}{$10$}} & \multicolumn{1}{c}{\cellcolor{black!30}\pbox{20cm}{$11$}} & \multicolumn{1}{c}{\cellcolor{black!30}\pbox{20cm}{$12$}}\\ \hline
			\multicolumn{1}{c}{\cellcolor{black!30}\pbox{20cm}{\textbf{Probabilités}}} & $\dfrac{1}{10}$ & $\dfrac{1}{10}$ & $\dfrac{2}{10}$ & $\dfrac{2}{10}$ & $\dfrac{2}{10}$ & $\dfrac{1}{10}$ & $\dfrac{1}{10}$\\  \hline
		\end{tabular}
	\end{table}
	Maintenant imaginons que nous ayons une autre expérience à analyser utilisant deux échantillons comprenant $3$ variables de contrôle et $2$ variables de traitement (le symétrique du précédent donc!), nous avons alors:
	\begin{table}[H]
		\centering
		\definecolor{gris}{gray}{0.85}
				\begin{tabular}{|c|c|c|}
					\hline
	\multicolumn{1}{c}{\cellcolor{black!30}\pbox{20cm}{\textbf{Rangs possibles} \\ \textbf{(Contrôles)}}} & \multicolumn{1}{c}{\cellcolor{black!30}\pbox{20cm}{\textbf{Rangs possibles} \\ \textbf{(Traitements)}}} & \multicolumn{1}{c}{\cellcolor{black!30}$W_S$} \\ \hline
			$3,4,5$ & $1,2$ & $3$\\ 
			$2,4,5$ & $1,3$ & $4$\\
			$2,3,5$ & $1,4$ & $5$\\  
			$2,3,4$ & $1,5$ & $6$\\ \hline
			$1,4,5$ & $2,3$ & $5$\\
			$1,3,5$ & $2,4$ & $6$\\  
			$1,3,4$ & $2,5$ & $7$\\ \hline
			$1,2,5$ & $3,4$ & $7$\\ 
			$1,2,4$ & $3,5$ & $8$\\ \hline
			$1,2,3$ & $4,5$ & $9$\\ \hline
		\end{tabular}
		\caption[]{Représentation des rangs de 2 variables de traitement et 3 de contrôle}
	\end{table}
	Dont nous déduisons le tableau suivant (le lecteur remarquera que c'est exactement le même que le précédent en ce qui concerne les probabilités!!):
	\begin{table}[H]
		\centering
		\definecolor{gris}{gray}{0.85}
		\begin{tabular}{|c|c|c|c|c|c|c|c|}
		\hline
		\multicolumn{1}{c}{\cellcolor{black!30}\pbox{20cm}{\textbf{Valeurs de $W_S$} }} & \multicolumn{1}{c}{\cellcolor{black!30}\pbox{20cm}{$3$}} & \multicolumn{1}{c}{\cellcolor{black!30}\pbox{20cm}{$4$}} & \multicolumn{1}{c}{\cellcolor{black!30}\pbox{20cm}{$5$}} & \multicolumn{1}{c}{\cellcolor{black!30}\pbox{20cm}{$6$}} & \multicolumn{1}{c}{\cellcolor{black!30}\pbox{20cm}{$7$}} & \multicolumn{1}{c}{\cellcolor{black!30}\pbox{20cm}{$8$}} & \multicolumn{1}{c}{\cellcolor{black!30}\pbox{20cm}{$9$}}\\ \hline
			\multicolumn{1}{c}{\cellcolor{black!30}\pbox{20cm}{\textbf{Probabilité}}} & $\dfrac{1}{10}$ & $\dfrac{1}{10}$ & $\dfrac{2}{10}$ & $\dfrac{2}{10}$ & $\dfrac{2}{10}$ & $\dfrac{1}{10}$ & $\dfrac{1}{10}$\\  \hline
		\end{tabular}
	\end{table}
	Eh bien l'idée du test de Mann-Withney est très simple. Plutôt que de tabuler des situations symétriques, il suffit de soustraire à chaque valeur de $W_S$, la valeur  $W_{S,\min}$ afin que chaque tableau soit identique et qu'un des deux seul soit utile. Voyons cela d'abord avec le premier tableau:
	\begin{table}[H]
		\centering
		\definecolor{gris}{gray}{0.85}
		\begin{tabular}{|c|c|c|}
					\hline
	\multicolumn{1}{c}{\cellcolor{black!30}\pbox{20cm}{\textbf{Rangs possibles} \\ \textbf{(Contrôles)}}} & \multicolumn{1}{c}{\cellcolor{black!30}\pbox{20cm}{\textbf{Rangs possibles} \\ \textbf{(Traitements)}}} & \multicolumn{1}{c}{\cellcolor{black!30}$W_s-\dfrac{1}{2}n_x(n_x+1)
$} \\ \hline
			$1,2$ & $3,4,5$ & $6$\\ 
			$1,3$ & $2,4,5$ & $5$\\
			$1,4$ & $2,3,5$ & $4$\\  
			$1,5$ & $2,3,4$ & $3$\\ \hline
			$2,3$ & $1,4,5$ & $4$\\
			$2,4$ & $1,3,5$ & $3$\\  
			$2,5$ & $1,3,4$ & $2$\\ \hline
			$3,4$ & $1,2,5$ & $2$\\ 
			$3,5$ & $1,2,4$ & $1$\\ \hline
			$4,5$ & $1,2,3$ & $0$\\ \hline
		\end{tabular}
		\caption[]{Symétrisation ($2$ contrôles/$3$ traitements)}
	\end{table}
	Dont nous déduisons le tableau suivant:
	\begin{table}[H]
		\centering
		\definecolor{gris}{gray}{0.85}
				\begin{tabular}{|c|c|c|c|c|c|c|c|}
					\hline
	\multicolumn{1}{c}{\cellcolor{black!30}\pbox{20cm}{\textbf{Valeurs de $W_{XY}$} }} & \multicolumn{1}{c}{\cellcolor{black!30}\pbox{20cm}{$0$}} & \multicolumn{1}{c}{\cellcolor{black!30}\pbox{20cm}{$1$}} & \multicolumn{1}{c}{\cellcolor{black!30}\pbox{20cm}{$2$}} & \multicolumn{1}{c}{\cellcolor{black!30}\pbox{20cm}{$3$}} & \multicolumn{1}{c}{\cellcolor{black!30}\pbox{20cm}{$4$}} & \multicolumn{1}{c}{\cellcolor{black!30}\pbox{20cm}{$5$}} & \multicolumn{1}{c}{\cellcolor{black!30}\pbox{20cm}{$6$}}\\ \hline
			\multicolumn{1}{c}{\cellcolor{black!30}\pbox{20cm}{\textbf{Probabilité}}} & $\dfrac{1}{10}$ & $\dfrac{1}{10}$ & $\dfrac{2}{10}$ & $\dfrac{2}{10}$ & $\dfrac{2}{10}$ & $\dfrac{1}{10}$ & $\dfrac{1}{10}$\\  \hline
		\end{tabular}
	\end{table}
	Maintenant imaginons que nous ayons une autre expérience à analyser utilisant deux échantillons comprenant $3$ variables de contrôle et $2$ variables de traitement (le symétrique du précédent donc!), nous avons alors:
	\begin{table}[H]
		\centering
		\definecolor{gris}{gray}{0.85}
			\begin{tabular}{|c|c|c|}
			\hline
	\multicolumn{1}{c}{\cellcolor{black!30}\pbox{20cm}{\textbf{Rangs possibles} \\ \textbf{(Contrôles)}}} & \multicolumn{1}{c}{\cellcolor{black!30}\pbox{20cm}{\textbf{Rangs possibles} \\ \textbf{(Traitements)}}} & \multicolumn{1}{c}{\cellcolor{black!30}$W_s-\dfrac{1}{2}n_x(n_x+1)
$} \\ \hline
			$3,4,5$ & $1,2$ & $0$\\ 
			$2,4,5$ & $1,3$ & $1$\\
			$2,3,5$ & $1,4$ & $2$\\  
			$2,3,4$ & $1,5$ & $3$\\ \hline
			$1,4,5$ & $2,3$ & $2$\\
			$1,3,5$ & $2,4$ & $3$\\  
			$1,3,4$ & $2,5$ & $4$\\ \hline
			$1,2,5$ & $3,4$ & $4$\\ 
			$1,2,4$ & $3,5$ & $5$\\ \hline
			$1,2,3$ & $4,5$ & $6$\\ \hline
		\end{tabular}
		\caption[]{Symétrisation ($2$ contrôles/$2$ traitements)}
	\end{table}
	Dont nous déduisons cette fois-ci exactement le même tableau que précédemment:
	\begin{table}[H]
		\centering
		\definecolor{gris}{gray}{0.85}
		\begin{tabular}{|c|c|c|c|c|c|c|c|}
			\hline
			\multicolumn{1}{c}{\cellcolor{black!30}\pbox{20cm}{\textbf{Valeurs de $W_{YX}$} }} & \multicolumn{1}{c}{\cellcolor{black!30}\pbox{20cm}{$0$}} & \multicolumn{1}{c}{\cellcolor{black!30}\pbox{20cm}{$1$}} & \multicolumn{1}{c}{\cellcolor{black!30}\pbox{20cm}{$2$}} & \multicolumn{1}{c}{\cellcolor{black!30}\pbox{20cm}{$3$}} & \multicolumn{1}{c}{\cellcolor{black!30}\pbox{20cm}{$4$}} & \multicolumn{1}{c}{\cellcolor{black!30}\pbox{20cm}{$5$}} & \multicolumn{1}{c}{\cellcolor{black!30}\pbox{20cm}{$6$}}\\ \hline
			\multicolumn{1}{c}{\cellcolor{black!30}\pbox{20cm}{\textbf{Probabilité}}} & $\dfrac{1}{10}$ & $\dfrac{1}{10}$ & $\dfrac{2}{10}$ & $\dfrac{2}{10}$ & $\dfrac{2}{10}$ & $\dfrac{1}{10}$ & $\dfrac{1}{10}$\\  \hline
		\end{tabular}
	\end{table}
	raison pour laquelle la littérature mentionne qu'on peut prendre celui que l'on veut!
	
	Donc pour résumer, la variante de Mann-Whitney (dans le cas concret ici présent il s'agit de la variante dite "\NewTerm{variante exacte de Mann-Whitney}\index{variante exacte de Mann-Whitney}") consiste à tabuler pour les situations symétriques une variable notée $W_{XY}$ définie naturellement par:
	
	Notée aussi très souvent dans la littérature:
	
	car alors $W_{XY}\in \left\lbrace 0,1,2,...\right\rbrace$ et donc:
	
	Dans les tables que l'on peut trouver dans les livres, les probabilités sont données avec la valeur normalisée de $U$. Ainsi, si nous reprenons notre exemple précédent mais avec les notations d'usage dans la pratique ($U$ au lieu de $W_{YX}$):
	\begin{table}[H]
		\centering
		\definecolor{gris}{gray}{0.85}
		\begin{tabular}{|c|c|c|c|c|c|c|c|}
			\hline
			\multicolumn{1}{c}{\cellcolor{black!30}\pbox{20cm}{\textbf{Valeurs de $U$} }} & \multicolumn{1}{c}{\cellcolor{black!30}\pbox{20cm}{$0$}} & \multicolumn{1}{c}{\cellcolor{black!30}\pbox{20cm}{$1$}} & \multicolumn{1}{c}{\cellcolor{black!30}\pbox{20cm}{$2$}} & \multicolumn{1}{c}{\cellcolor{black!30}\pbox{20cm}{$3$}} & \multicolumn{1}{c}{\cellcolor{black!30}\pbox{20cm}{$4$}} & \multicolumn{1}{c}{\cellcolor{black!30}\pbox{20cm}{$5$}} & \multicolumn{1}{c}{\cellcolor{black!30}\pbox{20cm}{$6$}}\\ \hline
			\multicolumn{1}{c}{\cellcolor{black!30}\pbox{20cm}{\textbf{Probabilité}}} & $\dfrac{1}{10}$ & $\dfrac{1}{10}$ & $\dfrac{2}{10}$ & $\dfrac{2}{10}$ & $\dfrac{2}{10}$ & $\dfrac{1}{10}$ & $\dfrac{1}{10}$\\  \hline
		\end{tabular}
	\end{table}
	Nous voyons que la probabilité cumulée que $U=2$ est de $0.4$. La table précédente se trouve dans la littérature parfois sous la forme suivante:
	\begin{table}[H]
		\centering
		\definecolor{gris}{gray}{0.85}
		\begin{tabular}{ |c|c|c|c| }
			\hline
			\multicolumn{4}{ |c| }{\cellcolor{black!30}\pbox{20cm}{Facteur $n_2=3$}} \\
			\hline
			$U/n_1$ & $1$ & $2$ & $3$ \\ \hline
			 $0$ & $0.250$ & $0.100$ & $0.050$ \\ \hline
			 $1$ & $0.500$ & $0.200$ & $0.100$ \\ \hline
			 $2$ & $0.750$ & \textbf{0.400} & $0.200$ \\ \hline
			 $3$ & $1$ & $0.600$ & $0.350$ \\ \hline
			 $4$ &  & $0.800$ & $0.500$ \\ \hline
			 $5$ &  & $0.900$ & $0.650$ \\ \hline
			 $6$ &  & $1$ & $\ldots$ \\ \hline
		\end{tabular}
		\caption[]{Représentation classique du test de Mann-Whitney}
	\end{table}
	où nous avons mis en rouge la colonne correspondant à notre exemple  ($U=2,n_1=2,n_2=3$).
	
	Ensuite il convient au praticien de choisir avec ces tableaux s'il souhaite faire un test bilatéral ou unilatéral.
	
	\begin{tcolorbox}[title=Remarques,colframe=black,arc=10pt]
	\textbf{R1.} Il est important de se rappeler que nous avons démontré par l'exemple que nous pouvons aussi bien prendre:
	
	que:
	
	puisqu'ils génèrent les mêmes tableaux!\\
	
	\textbf{R2.} $W_{XY}$ est traditionnellement noté $U$ par les praticiens comme nous l'avons vu, d'où le fait que l'on retrouve dans la littérature ce test sous le nom de "\NewTerm{test $U$ de Mann-Withney}\index{test $U$ de Mann-Withney}" avec les tables de probabilités associées sous le même nom. Cependant attention à ne pas confondre avec le "\NewTerm{test $U$ de Wilcoxon}\index{test $U$ de Wilcoxon}" appelé parfois "\NewTerm{test d'inversion de Wilcoxon}\index{test d'inversion de Wilcoxon}" qui se base sur les alternances d'apparition des valeurs des échantillons lorsque regroupés (test qui ne sera pas développé ici).
	\end{tcolorbox}
	Pour voir la version approximative (asymptotique) du test $U$ de Mann-Withney nous avons besoin de l'espérance et de la variance. Pour cela, rappelons que nous avons donc vu que la somme des rangs normalisés était donnée par:
	
	Mais nous pouvons aussi utiliser comme nous l'avons vu:
	
	et puisque:
	
	avec pour rappel:
	 
	nous avons donc:
	
	La moyenne des deux $U$ est donc la moyenne arithmétique de la somme. Nous avons donc:
	
	Ce qui signifie que $U_1$ ou $U_2$ doit être suffisemment différent de cette dernière moyenne pour que l'on rejette l'hypothèse nulle $H_0$ comme quoi les deux échantillons proviennet d'une même loi de distribution. Mais pour déterminer la $p$-valeur, nous avons qu'il nous faut aussi l'écart-type. Donc cherchons-le!
	
	L'écart-type est le même que pour le test de la somme des rangs de Wilcoxon (puisque le deuxième terme dans l'expression des $U$ est une constante dont la variance est nulle. Ainsi, il ne reste plus que la variance de la somme des rangs et nous avons déjà démontré plus haut qu'elle valait:
	
	\begin{tcolorbox}[colframe=black,colback=white,sharp corners]
	\textbf{{\Large \ding{45}}Exemple:}\\\\
	Reprenons l'exemple fait avec le test de la somme des rangs de Wilcoxon mais un peu modifié (pour que l'exemple soit plus parlant) c'est-à-dire:
	
	Soit groupé et ordonné:
	
	Nous avons alors:
	\end{tcolorbox}
	
	\pagebreak
	\begin{tcolorbox}[colframe=black,colback=white,sharp corners]
	
	Donc nous pouvons choisir n'importe lequel pour le test vu que les deux $U$ sont égaux. Si nous regardons le tableau créé plus haut, avec $(U=3,n_1=2,n_2=3)$, nous avons donc une probabilité cumulée de $60\%$ que $U$ soit égal à $3$. Donc nous ne rejetons pas l'hypothèse nulle (en unilatéral) comme quoi les deux échantillons proviennent de la même distribution.\\
	
	L'approximation en loi Normale donne alors:
	
	Donc la probabilité cumulée est de $50\%$ avec l'approximation Normale ce qui correspondant à une $p$-valeur de $50\%$. Là encore nous ne rejetons pas l'hypothèse nulle $H_0$.
	\end{tcolorbox}
	
	\paragraph{Traitement des rangs égaux dans les tests basés sur les rangs}\mbox{}\\\\
	Lorsque nous procédons à un test de la somme des rangs de type Wilcoxon-Mann-Withney ou autre, des égalités de rangs peuvent se produire.

	Reprenons pour l'exemple:
	
	avec les données suivantes:
	\begin{table}[H]
		\centering
		\definecolor{gris}{gray}{0.85}
			\begin{tabular}{|c|c|c|c|c|c|}
			\hline
			\multicolumn{1}{c}{\cellcolor{black!30}\pbox{20cm}{\textbf{Data}}} & $17$ & $17$ & $17$ & $19$ & $21$\\ 
			\multicolumn{1}{c}{\cellcolor{black!30}\pbox{20cm}{\textbf{Rank}}} & ? & ? & ? & $4$ & $5$\\ \hline
		\end{tabular}
		\caption[]{Exemple de problème en cas d'égalités}
	\end{table}
	Une solution conventionnelle consiste à attribuer à chaque "?" le rang moyen. Donc dans le cas présent, nous avons:
	
	Le tableau:
	\begin{table}[H]
		\centering
		\definecolor{gris}{gray}{0.85}
				\begin{tabular}{|c|c|c|}
					\hline
	\multicolumn{1}{c}{\cellcolor{black!30}\pbox{20cm}{\textbf{Rangs possibles} \\ \textbf{(Contrôles)}}} & \multicolumn{1}{c}{\cellcolor{black!30}\pbox{20cm}{\textbf{Rangs possibles} \\ \textbf{(Traitements)}}} & \multicolumn{1}{c}{\cellcolor{black!30}$W_S$} \\ \hline
			$1,2$ & $3,4,5$ & $12$\\ 
			$1,3$ & $2,4,5$ & $11$\\
			$1,4$ & $2,3,5$ & $10$\\  
			$1,5$ & $2,3,4$ & $9$\\ \hline
			$2,3$ & $1,4,5$ & $10$\\
			$2,4$ & $1,3,5$ & $9$\\  
			$2,5$ & $1,3,4$ & $8$\\ \hline
			$3,4$ & $1,2,5$ & $8$\\ 
			$3,5$ & $1,2,4$ & $7$\\ \hline
			$4,5$ & $1,2,3$ & $6$\\ \hline
		\end{tabular}
		\caption[]{Représentation des rangs de 3 variables de traitement et 2 de contrôle}
	\end{table}
	devient alors dans ce cas particulier:
	\begin{table}[H]
		\centering
		\definecolor{gris}{gray}{0.85}
				\begin{tabular}{|c|c|c|}
					\hline
	\multicolumn{1}{c}{\cellcolor{black!30}\pbox{20cm}{\textbf{Rangs possibles} \\ \textbf{(Contrôles)}}} & \multicolumn{1}{c}{\cellcolor{black!30}\pbox{20cm}{\textbf{Rangs possibles} \\ \textbf{(Traitements)}}} & \multicolumn{1}{c}{\cellcolor{black!30}$W_S^{*}$} \\ \hline
			$2,2$ & $2,4,5$ & $11$\\ 
			$2,2$ & $2,4,5$ & $11$\\
			$2,4$ & $2,2,5$ & $9$\\  
			$2,5$ & $2,2,4$ & $8$\\ \hline
			$2,2$ & $2,4,5$ & $11$\\
			$2,4$ & $2,2,5$ & $9$\\  
			$2,5$ & $2,2,4$ & $8$\\ \hline
			$2,4$ & $2,2,5$ & $9$\\ 
			$2,5$ & $2,2,4$ & $8$\\ \hline
			$4,5$ & $2,2,2$ & $6$\\ \hline
		\end{tabular}
		\caption[]{Représentation des rangs de 3 variables de traitement et 2 de contrôle}
	\end{table}
	où $W_S^{*}$ (remarquez la petite $^{*}$ en haut à droite!) représente la statistique de Wilcoxon lorsque nous sommes en présence d'égalités statistiques. La loi de $W_S^{*}$ peut être plus ou moins différente de celle de $W_S$. Effectivement:
	\begin{table}[H]
		\centering
		\definecolor{gris}{gray}{0.85}
			\begin{tabular}{|c|c|c|c|c|c|c|c|}
				\hline
\multicolumn{1}{c}{\cellcolor{black!30}\pbox{20cm}{\textbf{Statistique de Wilcsoxon} }} & \multicolumn{1}{c}{\cellcolor{black!30}\pbox{20cm}{$6$}} & \multicolumn{1}{c}{\cellcolor{black!30}\pbox{20cm}{$7$}} & \multicolumn{1}{c}{\cellcolor{black!30}\pbox{20cm}{$8$}} & \multicolumn{1}{c}{\cellcolor{black!30}\pbox{20cm}{$9$}} & \multicolumn{1}{c}{\cellcolor{black!30}\pbox{20cm}{$10$}} & \multicolumn{1}{c}{\cellcolor{black!30}\pbox{20cm}{$11$}} & \multicolumn{1}{c}{\cellcolor{black!30}\pbox{20cm}{$12$}}\\ \hline
		\multicolumn{1}{c}{\cellcolor{black!30}\pbox{20cm}{\textbf{Probabilité de $W_S$}}} & $\dfrac{1}{10}$ & $\dfrac{1}{10}$ & $\dfrac{2}{10}$ & $\dfrac{2}{10}$ & $\dfrac{2}{10}$ & $\dfrac{1}{10}$ & $\dfrac{1}{10}$\\  \hline
		\multicolumn{1}{c}{\cellcolor{black!30}\pbox{20cm}{\textbf{Probabilité de $W_S^{*}$}}} & $\dfrac{1}{10}$ & $0$ & $\dfrac{3}{10}$ & $\dfrac{3}{10}$ & $0$ & $\dfrac{3}{10}$ & $0$\\  \hline
		\end{tabular}
	\end{table}
	
	\pagebreak
	\paragraph{Test de la somme des rangs signés de Wilcoxon à un échantillon}\mbox{}\\\\
	Le but du test de la "\NewTerm{somme des rangs signés de Wilcoxon}\index{test de la somme des rangs signés de Wilcoxon}", appelé aussi parfois "\NewTerm{test de la médiane de Wilcoxon}\index{test de la médiane de Wilcoxon}", est d'utiliser une technique non paramétrique pour vérifier la symétrie ou non d'une distribution et donc in extenso faire une hypothèse sur la valeur de la médiane. L'idée est à la fois simple et subtile.
	
	Le principe et que si nous comparons les différences $D_i$ des individus d'un échantillon par rapport à la médiane, nous savons que si nous avons (par exemple) un nombre impair d'individus tous différents (non égaux), alors nous aurons $50\%$ des données au-dessus et en-dessous de la médiane. Ensuite, pour contrôler que la distribution des valeurs des individus vérifie une certaine symétrie, l'idée (simple mais astucieuse) consiste ensuite à:
	\begin{enumerate}
		\item Calculer les différences en valeur absolue $|D_i|$ par rapport à la médiane.
		
		\item Ranger ces différences absolues par ordre croissant et leur assigner leur rang respectif.
		
		\item  Calculer la somme des rangs des différences $D_i$ qui à la base sont négatives.
		
		\item Calculer la somme des rangs des différences $D_i$ qui à la base sont positives.
	\end{enumerate}
	et si l'échantillon a une distribution symétrique (donc la médiane est confondue alors avec la moyenne), il devrait y avoir une somme des rangs négatifs $S_{-}$ qui n'est pas statistiquement significativement différente de la sommes des rangs positifs $S^{+}$.
	
	Au passage nous remarquons donc qu'une hypothèse du test pour qu'il fonctionne est que la distribution statistique soit donc symétrique!!
	
	\begin{tcolorbox}[title=Remarque,colframe=black,arc=10pt]
	 Pour rappel, lors de notre étude des tests pour échantillons indépendants de Wilcoxon ou Mann-Withney vus plus haut (qui n'ont pas obligatoirement la même taille), nous ordonnons ensemble les valeurs des deux échantillons et nous faisons un calcul sur les rangs de ces valeurs. Dans les tests pour échantillons appariés (donc de même taille), nous ordonnons les différences de valeurs (pas les valeurs!) et nous travaille sur les rangs des \underline{différences}!!!
	\end{tcolorbox}
	Selon l'idée (principe) exposé plus haut, la somme des rangs qui portent le signe $-$ vaut alors en moyenne:
	
	Or, nous avons déjà démontré que l'espérance de la loi binomiale est:
	
	Et comme dans notre cas $N$  vaut $1$ (une seule valeur...) et $p$ vaut $1/2$ (une chance sur deux d'avoir un signe négatif), il vient immédiatement en utilisant les démonstrations du chapitre de Suites Et Séries:
	
	et pour la variance en utilisant aussi les résultats du chapitre Suites Et Séries:
	
	et à nouveau en utilisant la variance de la loi binomiale et les résultats du chapitre Suites Et Séries:
	
	Évidemment la somme des rangs des différences négatives (respectivement positif) sera au minimum nul et vaudra au maximum $n(n+1)/2$. Donc l'espérance dans le cas d'un test bilatéral ne doit pas être trop proche d'une de ces deux valeurs extrêmes.
	
	Dans le cas où $n$ est assez grand (supérieur à une trentaine), nous pouvons utiliser l'approximation de la loi Normale centrée réduite pour la variable:
	
	où $S_{-}$ est donc la somme des rangs de signe négatifs.
	
	Enfin signalons qu'empiriquement si des différences par rapport à la médiane sont nulles, elles ne seront pas prises en compte dans les rangs. Si des différences sont égales nous prendrons un rang moyen...
	
	\begin{tcolorbox}[colframe=black,colback=white,sharp corners]
	\textbf{{\Large \ding{45}}Exemple:}\\\\
	Commençons avec le cas à un échantillon comparé à sa médiane expérimentale (à l'opposé de la comparaison à une médiane hypothétisée lorsque nous considérons a priori la distribution symétrique et unimodale). Considérons que nous avons mesuré les valeurs suivantes pour le diamètre d'une pièce:
	
	Nous souhaitons donc savoir si la médiane expérimentale calculée (valant $40$ dans le cas présent) de cet échantillon peut ne pas être rejeté comme indicateur central à un niveau de confidence de $5\%$ en bilatéral (ce qui sera le cas si le nombre de différences positifs et négatifs est assez équilibré). Nous construisons alors le tableau suivant:
	\end{tcolorbox}
	
	\pagebreak
	\begin{tcolorbox}[colframe=black,colback=white,sharp corners]
	\begin{table}[H]
		\centering
		\definecolor{gris}{gray}{0.85}
		\begin{tabular}{|c|c|c|c|c|c|}
		\hline
	  \multicolumn{1}{c}{\cellcolor{black!30}\textbf{Mesures}} & 
	  \multicolumn{1}{c}{\cellcolor{black!30}\textbf{Différences}}  & 
	  \multicolumn{1}{c}{\cellcolor{black!30}\textbf{Valeurs absolues}} & 
	  \multicolumn{1}{c}{\cellcolor{black!30}\textbf{Rang}} & 
	  \multicolumn{1}{c}{\cellcolor{black!30}$R_{+}$} & 
	  \multicolumn{1}{c}{\cellcolor{black!30}$R_{-}$}\\ \hline
		$39$ & $-1$ & $1$ & $2$ & & $2$ \\ \hline
		$20.2$ & $-19.8$ & $19.8$ & $11$ & & $11$ \\ \hline
		$40$ & $0$ & $0$ &  & &  \\ \hline
		$32.2$ & $-7.8$ & $7.8$ & $6$ & & $6$  \\ \hline
		$30.5$ & $-9.5$ & $9.5$ & $8$ & & $9$  \\ \hline
		$26.5$ & $-13.5$ & $13.5$ & $10$ & & $10$  \\ \hline
		$42.1$ & $2.1$ & $2.1$ & $4$ & & $4$  \\ \hline
		$45.6$ & $5.6$ & $5.6$ & $6.5$ & & $6.5$  \\ \hline
		$42.1$ & $2.1$ & $2.1$ & $4$ & $4$ &  \\ \hline
		$45.6$ & $5.6$ & $5.6$ & $6.5$ & $6.5$ &  \\ \hline
		$42.1$ & $2.1$ & $2.1$ & $4$ & $4$ &  \\ \hline
		$45.6$ & $5.6$ & $5.6$ & $6.5$ & $6.5$ &  \\ \hline
		$42.1$ & $2.1$ & $2.1$ & $4$ & $4$ &  \\ \hline
		$29.9$ & $-10.1$ & $10.1$ & $9$ & & $9$  \\ \hline
		$40.9$ & $0.9$ & $0.9$ & $1$ & $1$ &  \\ \hhline{|=|=|=|=|=|=|}
		& & \textbf{Somme:} & & $26$ & $46$ \\ \hline
		\end{tabular}
	\end{table}
	A vue de nez l'égalité des rangs de ne s'annonce pas très bien mais allons quand même un peu plus loin...\\
	\begin{tcolorbox}[title=Remarque,colframe=black,arc=10pt]
	 Suivant certains manuels universitaires la somme des rangs ne donne pas la même valeur car il y a plusieurs techniques pour calculer des rangs de valeurs qui sont identiques... Nous avons cependant choisi celle utilisée par le logiciel Minitab qui est d'usage dans la communauté scientifique et qui correspond à celle dont nous avons déjà dictée les règles plus haut.
	\end{tcolorbox}
	Si nous considérons que le nombre d'individus est suffisant..., nous utilisons l'approximation (même si dans le cas présent les conditions ne sont pas satisfaites):
	
	Soit dans le cas présent:
	
	et respectivement:
	\end{tcolorbox}
	
	\pagebreak
	\begin{tcolorbox}[colframe=black,colback=white,sharp corners]
	
	Le premier cas correspond dans l'approximation à une loi Normale à une probabilité cumulée de $29.13\%$ obtenue avec la version française de Microsoft Excel 14.0.6123 à l'aide de la fonction:
	
	\begin{center}
	\texttt{=LOI.NORMALE.STANDARD.N(-0.549,VRAI)}
	\end{center}
	
	et donne une $p$-valeur en bilatéral d'environ $2\cdot 29.13\%\cong 58.26\%$.\\
	
	Le deuxième cas correspond dans l'approximation à une loi Normale à une probabilité cumulée à $84.62\%$ obtenue avec avec la version française de Microsoft Excel 14.0.6123 à l'aide de la fonction:
	\begin{center}
	\texttt{=LOI.NORMALE.STANDARD.N(1.02,VRAI)}
	\end{center}
	ce qui correspond à une $p$-valeur d'environ $2\cdot(1-0.8462)/2=30.76\%$ en bilatéral (un logiciel comme Minitab donne une $p$-valeur en bilatéral de $32\%$ puisqu'il ne fait pas l'approximation en loi Normale).\\
	
	Pour le deuxième cas nous sommes à la limite mais au seuil choisi plus haut nous ne pouvons malheureusement pas rejeter l'hypothèse nulle $H_0$ comme quoi $40$ est dans l'intervalle de confiance de la médiane (par ailleurs le test des signes amène à la même conclusion).
	\end{tcolorbox}
	\begin{tcolorbox}[title=Remarque,colframe=black,arc=10pt]
	Un logiciel comme Minitab bien que proposant le test de Wilcoxon à $1$ échantillon de la médiane donne pour médiane la valeur de $36.5$ et donne pour intervalle de confiance de la médiane les valeurs $31.1$ et $42.1$. Si nous appliquons la méthode de boostrapping présentée en détails dans la section de Méthodes Numériques nous obtenons comme médiane estimée $40$ (pour moyenne  $38.733$) et comme intervalle $30.50$ et $42.10$... Bon dans tous les cas nous arrivons de toute façon à ne pas rejeter l'hypothèse nulle $H_0$ mais quand même...
	\end{tcolorbox}
	
	\pagebreak
	\paragraph{Test de la somme des rangs signés de Wilcoxon pour deux échantillons appariés}\mbox{}\\\\
	Le "\NewTerm{test de la somme des rangs signés de Wilcoxon pour deux échantillons appariés}\index{test de la somme des rangs signés de Wilcoxon pour deux échantillons appariés}", ou appelé plus simplement "\NewTerm{test de la somme des rangs signés de Wilcoxon}\index{test de la somme des rangs signés de Wilcoxon}", est basé à $100\%$ sur le principe du test à 1 échantillon. La seule différence est que l'hypothèse nulle ou alternative est basée sur la différence de la médiane des données prises deux à deux de chacun des échantillons. Dans la majorité des cas, l'hypothèse nulle $H_0$ est que la médiane des différences est nulle contre l'hypothèse alternative $H_A$ qu'elle est statistiquement significativement différente de zéro.
	
	Comme le test $T$ pour les échantillons appariés, le test de la somme des rangs de Wilcoxon s'applique aux plans à deux échantillons impliquant des mesures répétées, des paires appariées ou des mesures «avant» et «après» !!
	
	Comme les développements mathématiques sont les mêmes que pour le test à $1$ échantillon, attaquons directement par un exemple.
	
	D'abord insistons juste que par extension, qu'une hypothèse du test pour qu'il fonctionne est que la distribution statistique des différences soit donc symétrique!!
	
	De nombreux textes introductifs motivent le test de la somme des rangs signés comme un test de différence des médianes, ou plus rarement selon notre expérience, de différence des moyennes, sans mentionner que deux hypothèses assez strictes sont nécessaires pour cette interprétation:
	\begin{itemize}
		\item La distribution des deux groupes doit avoir la même forme.

		\item La variance des deux groupes doit être égale.
	\end{itemize}
	Si ces deux hypothèses sont vraies, alors le test des rangs signés peut être valablement interprété comme ayant une hypothèse nulle des médianes égales (ou des moyennes égales).
	
	\begin{tcolorbox}[title=Remarque,colframe=black,arc=10pt]
	Notez que dans de nombreux logiciels statistiques, $R_+$ (voir l'exemple ci-dessous) est désigné par la lettre $V$.
	\end{tcolorbox}
	
	\begin{tcolorbox}[colframe=black,colback=white,sharp corners]
	\textbf{{\Large \ding{45}}Exemple:}\\\\
	Nous avons deux logiciels ($L1$, $L2$) différents à comparer que nous voulons soumettre à $12$ tâches ($T1, T2, T3, ..., T12$) de calculs spécifiques mais identiques pour chacun des logiciels. Nous souhaiterions savoir si les logiciels ont un temps de traitement statistiquement significativement différent ou non et si oui lequel est le plus performant (il s'agit des mêmes algorithmes mais le second logiciel a subi une modifiction par rapport au premier)!\\
	
	Nous avons alors le tableau suivant où le temps est en minutes et où les différences $x_i-y_i$ sont notées $d_i$:
	\begin{table}[H]
		\centering
			\definecolor{gris}{gray}{0.85}
				\begin{tabular}{|c|c|c|c|c|c|c|c|}
					\hline
					\multicolumn{1}{c}{\cellcolor{black!30}\textbf{Tâche}} & 
	  \multicolumn{1}{c}{\cellcolor{black!30}$L1$}  & 
	  \multicolumn{1}{c}{\cellcolor{black!30}$L2$} & 
	  \multicolumn{1}{c}{\cellcolor{black!30}$d_i$} & 
	  \multicolumn{1}{c}{\cellcolor{black!30}$|d_i|$} & 
	  \multicolumn{1}{c}{\cellcolor{black!30}\textbf{Rang}} & 
	  \multicolumn{1}{c}{\cellcolor{black!30}$R_{+}$}  & 
	  \multicolumn{1}{c}{\cellcolor{black!30}$R_{-}$}\\ \hline
		$T1$ & $24.0$ & $23.1$ & $0.9$ & $0.9$ & $1$ & $1$ & \\ \hline
		$T2$ & $16.7$ & $20.4$ & $-3.7$ & $3.7$ & $4$ & & $4$\\ \hline
		$T3$ & $21.6$ & $17.7$ & $3.9$ & $3.9$ & $5$ & $5$ & \\ \hline
		$T4$ & $23.7$ & $20.7$ & $3.0$ & $3.0$ & $2.5$ & $2.5$ & \\ \hline
		$T5$ & $37.5$ & $42.1$ & $-4.6$ & $4.6$ & $6$ & & $6$ \\ \hline
		$T6$ & $31.4$ & $36.1$ & $-4.7$ & $4.7$ & $7$ & & $7$ \\ \hline
		$T7$ & $14.9$ & $21.8$ & $-6.9$ & $6.9$ & $10$ & & $10$ \\ \hline
		$T8$ & $37.3$ & $40.3$ & $-3.0$ & $3.0$ & $2.5$ & & $2.5$ \\ \hline
		$T9$ & $17.9$ & $26.0$ & $-8.1$ & $8.1$ & $11$ & & $11$ \\ \hline
		$T10$ & $15.5$ & $15.5$ & $0.0$ & $0.0$ & $-$ & &  \\ \hline
		$T11$ & $29.0$ & $35.4$ & $-6.4$ & $6.4$ & $9$ & & $9$  \\ \hline
		$T12$ & $19.9$ & $25.5$ & $-5.6$ & $5.6$ & $8$ & & $8$  \\ \hhline{|=|=|=|=|=|=|=|=|}
		& & &  & & \textbf{Somme:} & $8.5$ & $57.5$ \\ \hline
			\end{tabular}
	\end{table}
	Nous voyons déjà que le logiciel $L1$ est globalement plus rapide que $L2$ et sans utiliser les tables exactes du test des signes de Wilcoxon, nous pouvons présenter que la différence est statistiquement significative.\\
	
	Si nous considérons que le nombre d'individus est suffisant..., nous utilisons l'approximation (même si dans le cas présent les conditions ne sont pas satisfaites):
	
	\end{tcolorbox}
	
	\pagebreak
	\begin{tcolorbox}[colframe=black,colback=white,sharp corners]
	Soit dans le cas présent:
	
	et respectivement:
	
	Le premier cas correspond dans l'approximation à une loi Normale à une probabilité cumulée de $0.836\%$ obtenue avec avec la versions française de Microsoft Excel 14.0.6123 à l'aide de la fonction:
	\begin{center}
	\texttt{=LOI.NORMALE.STANDARD.N(-2.392,VRAI)}
	\end{center}
	et  donc à un $p$-valEUR d'environ $2\cdot 0.836\%\cong 1.68\%$ en bilatéral.\\
	
	Le deuxième cas correspond dans l'approximation à une loi Normale à une probabilité cumulée à $92.68\%$ obtenue avec avec la versions française de Microsoft Excel 14.0.6123 à l'aide de la fonction:
	\begin{center}
	\texttt{=LOI.NORMALE.STANDARD.N(1.453,VRAI)}
	\end{center}
	ce qui correspond à une $p$-valeur d'environ $2\cdot (1-0.9268)\%\cong 14.452\%$ en bilatéral. Avec un logiciel comme Minitab 15.1.2 qui ne propose pas le test de Wilcoxon pour échantillons appariés mais pour lequel il existe une astuce pour l'exécuter quand même, nous obtenons un $p$-valeur de $3.3\%$ (même résultat qu'avec le logiciel statistique R qui lui a par contre ce test apparié de Wilcoxon implémenté!). D'autres logiciels donnent une $p$-valeur toujours inférieure à $5\%$ (mais les valeurs différent d'un logiciel à l'autre...).\\
	
	Par conséquent, avec les calculs manuels et en utilisant l'approximation Normale, nous rejeterions l'hypothèse nulle $H_0$ au seuil de $5\%$ (si on se concentre que sur le premier résultat). Il en est de même avec les logiciels R et Minitab, nous rejeterions aussi l'hypothèse nulle $H_0$ au seuil de $5\%$!
	\end{tcolorbox}
	
	\begin{tcolorbox}[title=Remarque,colframe=black,arc=10pt]
	Pour autant que nous le sachions, il existe au moins deux manières différentes de calculer la variance du test de la somme des rangs signés de Wilcoxon pour deux échantillons appariés (sans compter la correction de continuité qui rajoute un $+0.5$ au numérateur). C'est pourquoi la plupart du temps, les résultats diffèrent entre les logiciels statistiques.
	\end{tcolorbox}
	
	\pagebreak
	\paragraph{Test de Kruskal-Wallis}\mbox{}\\\\
	Le test de Kruskal-Wallis un test non paramétrique souvent assimilé (un peu rapidement...) à une ANOVA non paramétrique à une voie pour comparer si deux populations ou plus ont même médiane (hypothèse nulle) à la différence qu'il ne nécessite donc pas les hypothèses nécessaires au fonctionnement de l'ANOVA. Quand plusieurs populations comparées passent à travers ce test, ce dernier ne dit pas quelle population est statistiquement significativement différente mais uniquement qu'il y en a au moins une qui l'est. En réalité, comme nous allons le démontrer, le test de Kruskal-Wallis n'est qu'une extension du test U de Mann-Whitney vu plus haut pour un nombre de  populations supérieur ou égal à trois.

	Pour étudier ce test, nous allons supposer que nous n'avons que deux populations et nous allons en faire une généralisation intuitive. Cette démarche est celle qu'aurait utilisée Wilcoxon avant que Kruskal et Wallis n'en fassent la démonstration générale rigoureuse.

	Pour étudier ce test, rappelons d'abord que (relations dont l'origine et in extenso la démonstration ont déjà expliquées lors de notre étude du test de Mann-Withney vu plus haut) la moyenne de la somme des rangs et l'écart-type de la somme des rangs sont donnés par:
	
	dans le cas où il n'y pas de valeurs doubles. Sous cette hypothèse, rappelons que $\bar{R}$ peut être assimilé au rang de la valeur médiane (dans le cas d'un nombre impair de mesures).

	Rappelons que la moyenne des tirages de $n$ valeurs sans remplacement parmi $N$ sera proche d'une loi Normale, et nous avons déjà démontré tout au début de ce chapitre que:
	
	et que si la population n'est pas très grande, la variance doit être corrigée par le facteur de correction sur population finie que nous avions déjà aussi démontré:
	
	Dès lors, il vient:
	
	Nous avons alors dans le cas qui nous concerne avec les rangs (la variance des rangs étant la variance vraie: il n'y a pas d'estimateur!):
	
	Maintenant, de manière à former un variable Normale centrée réduite $Z$ nous pouvons centrer et réduire la variable aléatoire $\hat{\bar{R}}$ obtenue par échantillonnage en écrivant:
	
	où $\hat{\bar{R}}$ est donc la moyenne de la somme des  rangs d'un échantillon de la population. Et au fait toute l'idée astucieuse du test de Kruskal-Wallis se trouve ici: la distribution statistique de la moyenne de la somme des rangs d'un grand nombre d'échantillons de $N$ valeurs suit approximativement une loi Normale (revoir notre étude des limites des tirages sans remise)!
	
	Prenons le carré :
	
	L'approximation par la loi du Khi-deux n'étant valable que si n est assez grand comme nous en avons déjà parlé en détails lors de notre étude du test d'ajustement du Khi-deux.

	Et donc la parenthèse de la première égalité est égale au carré de l'écart du rang de la médiane. Raison pour laquelle on dit souvent qu'il s'agit d'un test de la médiane (mais c'est un raccourci abusif).
	
	Avant de continuer, insistons bien sur le fait que le scénario dans lequel nous nous trouvons est celui d'un tirage d'un échantillon $n$ parmi $N$, ce qui est équivalent à se retrouver avec deux échantillons (un de taille $n$ et l'autre de taille $N-n$) de même loi (provenant in extenso d'une même population). Il vient alors que (relations que nous allons utiliser un peu plus loin):
	
	et par extension du cas à un échantillon, si nous notons $R_i$ la somme des rangs des nombres de l'échantillon $i$, nous avons aussi:
	
	Il s'ensuit que si nous notons pour la suite:
	
	où $R$ est donc la somme des rangs de l'échantillon $i=1$, nous avons:\
	
	Si nous écrivons maintenant la relation démontrée plus haut:
	
	sous la forme suivante (il s'agit d'un développement astucieux en marche arrière... à partir de la troisième ligne):
	
	et nous retrouvons donc à la fin le fait que nous travaillions depuis le début avec deux échantillons, un de taille $n$ et donc l'autre (in extenso par tirage) de taille $N-n$.
	
	Le résultat précédent (qui était celui recherché depuis le début) peut être généralisé sous la forme suivante appelée "\NewTerm{test H de Kruskal-Wallis}\index{test H de Kruskal-Wallis}" à un niveau de confiance donné en unilatéral (parfois cette relation est écrite sans les parenthèses pour la sommation ce qui peut prêter à une mauvaise lecture):
	
	et si tous les $n_i$ sont égaux, nous retrouvons cette relation sous la forme fréquente:
	
	L'approximation suivant une loi du Khi-deux est cependant délicate lorsque la taille des échantillons $(c)$ est petite (se référer à notre étude de la loi du Khi-deux).
	
	\pagebreak
	\begin{tcolorbox}[colframe=black,colback=white,sharp corners]
	\textbf{{\Large \ding{45}}Exemple:}\\\\
	Reprenons l'exemple de l'article original de Kruskal-Wallis. Nous considérons que nous avons trois machines à l'origine identiques mais dont deux ont subi quelques modifications. Nous avons mesuré la production journalière un certain nombre de fois et avons obtenu le tableau suivant:
	\begin{table}[H]
		\centering
		\definecolor{gris}{gray}{0.85}
		\begin{tabular}{|c|c|c|c|c|c|c|c|}
		\hline
		\multicolumn{1}{c}{\cellcolor{black!30}\textbf{Tâche}} & 
	  \multicolumn{1}{c}{\cellcolor{black!30}\textbf{Standard}}  & 
	  \multicolumn{1}{c}{\cellcolor{black!30}\textbf{Rang}} & 
	  \multicolumn{1}{c}{\cellcolor{black!30}\textbf{Modifiée} $1$} & 
	  \multicolumn{1}{c}{\cellcolor{black!30}\textbf{Rang}} & 
	  \multicolumn{1}{c}{\cellcolor{black!30}\textbf{Modifiée} $2$} & 
	  \multicolumn{1}{c}{\cellcolor{black!30}\textbf{Rang}} \\ \hline
		$340$ & $5$ & $339$ & $4$ & $347$ & $10$ &\\ \hline
		$345$ & $9$ & $333$ & $2$ & $343$ & $7$ &\\ \hline
		$330$ & $1$ & $344$ & $8$ & $349$ & $11$ &\\ \hline
		$342$ & $6$ &  & & $355$ & $12$ &\\ \hline
		$338$ & $3$ &  &  &  &  &\\ \hline
		\multicolumn{7}{|c|}{- - - - - - - - - - - - - - - - - - - -}\\ \hline
		$n$ & $5$ &  & $3$  &  & $4$ & $12$\\ \hhline{|=|=|=|=|=|=|=|=|}	
		$R$ & $24$ &  & $14$  &  & $40$ & $78$\\ \hline
		$R^2/n$ & $11.5$ &  & $65.33$  &  & $400$ & $580.53$\\ \hline
		\end{tabular}
		\caption{Tableau d'exemple pour le test de Kruskal-Wallis}
	\end{table}
	Nous avons alors bien:
	
	et:
	
	Or, nous avons:
	
	Cela peut être obtenu facilement avec la version française Microsoft Excel 14.0.7166 avec la fonction:
	\begin{center}
	\texttt{=1-LOI.KHIDEUX.N(5.656,2,VRAI)}
	\end{center}
	Dans le cas présent, à un niveau de $5\%$, nous sommes donc à la limite avec l'approximation par une loi de Khi-deux. Comme l'ont montré Kruskal et Wallis, une simulation par Monte-Carlo donne une $p$-valeur de $0.049$.\\
	
	Bref, dans cette situation il conviendrait plutôt de rejeter l'hypothèse nulle comme quoi les productions sont similaires. Et donc privilégier le fait que celles-ci soient plutôt différentes. Une recommandation et de refaire le test par paire des mesures pour voir ce qui est statistiquement significativement différent deux par deux.
	\end{tcolorbox}
	
	\pagebreak
	\paragraph{Test de Friedman}\mbox{}\\\\
	Le test de Friedman, recommandé par la norme NF ISO 8587 pour l'analyse sensorielle (test de classement), considère une expérience avec deux facteurs (le premier étant considéré comme le traitement et le second comme les blocs de tests au même titre que l'ANOVA à deux facteurs contrôlés sans répétition) que l'on analyse à l'aide des rangs car les valeurs des mesures ne satisfont pas les conditions d'application d'ANOVA. Cependant, au contraire de l'ANOVA, le test de Friedman s'applique à des données appariées comme nous allons le voir de suite.

	Associons, comme nous l'avons déjà fait à plusieurs reprises, la théorie à un exemple en partant du tableau suivant où $8$ sujets (blocs) $B$ sous hypnose ont été soumis à $4$ émotions (traitements) $T$. Leur potentiel électrique épidermique a été mesuré (en millivolts) dans chaque cas (et l'ordre des traitements a été randomisé):
	\begin{table}[H]
		\centering
		\definecolor{gris}{gray}{0.85}
		\begin{tabular}{|c|c|c|c|c|c|c|c|c|c|}
		\hline
		& \multicolumn{8}{|c|}{\textbf{Sujets ($B$)}}\\ \hline
		\multicolumn{1}{c}{\cellcolor{black!30}\textbf{Émotions ($T$)}} & 
		\multicolumn{1}{c}{\cellcolor{black!30}\textbf{$1$}}  & 
		\multicolumn{1}{c}{\cellcolor{black!30}\textbf{$2$}} & 
		\multicolumn{1}{c}{\cellcolor{black!30}\textbf{$3$}} & 
		\multicolumn{1}{c}{\cellcolor{black!30}\textbf{$4$}} & 
		\multicolumn{1}{c}{\cellcolor{black!30}\textbf{$5$}} & 
		\multicolumn{1}{c}{\cellcolor{black!30}\textbf{$6$}} &
		\multicolumn{1}{c}{\cellcolor{black!30}\textbf{$7$}} &
		\multicolumn{1}{c}{\cellcolor{black!30}\textbf{$8$}} \\ \hline
		\cellcolor{black!30}\textbf{Peur} & $23.1$ & $57.6$ & $10.5$ & $23.6$ & $11.9$ & $54.6$ & $21.0$ & $20.3$ \\ \hline
		\cellcolor{black!30}\textbf{Joie} & $23.1$ & $57.6$ & $10.5$ & $23.6$ & $11.9$ & $54.6$ & $21.0$ & $20.3$ \\ \hline
		\cellcolor{black!30}\textbf{Tristesse} & $22.5$ & $53.7$ & $10.8$ & $21.1$ & $13.7$ & $39.2$ & $13.7$ & $16.3$ \\ \hline
		\cellcolor{black!30}\textbf{Calme} & $22.6$ & $53.1$ & $8.3$ & $21.6$ & $13.3$ & $37.0$ & $14.8$ & $14.8$ \\ \hline
		\end{tabular}
		\caption{Tableau d'exemple des mesures pour le test de Friedman}
	\end{table}
	L'idée centrale et subtile est de ne pas affecter un rang à l'ensemble de la population des mesures comme c'est le cas pour le test de Kruskal-Wallis (on perdrait alors le concept des blocs: in extenso du deuxième facteur) mais bien bloc par bloc tous supposés donc indépendants les uns des autres.
	\begin{tcolorbox}[title=Remarque,colframe=black,arc=10pt]
	Nous ne traiterons pas (au même titre que lors de notre étude du test de Kruskal-Wallis) de la situation où des mesures sont à égalité avec d'autres dans un même bloc, les démonstrations actuelles n'étant pas vraiment convaincantes de notre point de vue subjectif.
	\end{tcolorbox}
	Donc, à chaque valeur $\left\lbrace x_{tb}\right\rbrace_{T\times B}$ du tableau nous allons maintenant associer le rang $\left\lbrace r_{tb}\right\rbrace_{T\times B}$ correspondant à chaque traitement. Ce qui donnera:
	\begin{table}[H]
		\centering
		\definecolor{gris}{gray}{0.85}
		\begin{tabular}{|c|c|c|c|c|c|c|c|c|c|}
		\hline
		& \multicolumn{8}{|c|}{\textbf{Sujets ($B$)}}\\ \hline
		\multicolumn{1}{c}{\cellcolor{black!30}\textbf{Émotions ($T$)}} & 
		\multicolumn{1}{c}{\cellcolor{black!30}\textbf{$1$}}  & 
		\multicolumn{1}{c}{\cellcolor{black!30}\textbf{$2$}} & 
		\multicolumn{1}{c}{\cellcolor{black!30}\textbf{$3$}} & 
		\multicolumn{1}{c}{\cellcolor{black!30}\textbf{$4$}} & 
		\multicolumn{1}{c}{\cellcolor{black!30}\textbf{$5$}} & 
		\multicolumn{1}{c}{\cellcolor{black!30}\textbf{$6$}} &
		\multicolumn{1}{c}{\cellcolor{black!30}\textbf{$7$}} &
		\multicolumn{1}{c}{\cellcolor{black!30}\textbf{$8$}} \\ \hline
		\cellcolor{black!30}\textbf{Peur} & $4$ & $4$ & $3$ & $4$ & $1$ & $4$ & $4$ & $3$ \\ \hline
		\cellcolor{black!30}\textbf{Joie} & $3$ & $2$ & $2$ & $1$ & $4$ & $3$ & $1$ & $4$ \\ \hline
		\cellcolor{black!30}\textbf{Tristesse} & $1$ & $3$ & $4$ & $2$ & $3$ & $2$ & $2$ & $2$ \\ \hline
		\cellcolor{black!30}\textbf{Calme} & $2$ & $1$ & $1$ & $3$ & $2$ & $1$ & $3$ & $1$ \\ \hline
		\end{tabular}
	\end{table}
	Bon maintenant que nous avons construit une sorte tableau d'ANOVA à deux facteurs contrôlés sans répétition non paramétrique que faisons-nous? Quelle est l'idée? Eh ben l'idée de base est la même que le test de Kruskal-Wallis: nous allons utiliser la propriété de la moyenne de la somme des rangs mais tout en ayant en tête que cette fois-ci la numérotation ne s'est pas faite sur l'ensemble des mesures du tableau mais bloc par bloc.
	
	Dans le cadre de notre exemple particulier nous avons donc:
	\setlength\extrarowheight{5pt}
	\begin{table}[H]
		\centering
		\definecolor{gris}{gray}{0.85}
		\begin{tabular}{|c|c|c|c|c|c|c|c|c|c|c|c|}
		\hline
		& \multicolumn{8}{|c|}{\textbf{Sujets ($B$)}} & {} & {}\\ \hline
		\multicolumn{1}{c}{\cellcolor{black!30}\textbf{Émotions ($T$)}} & 
		\multicolumn{1}{c}{\cellcolor{black!30}\textbf{$1$}}  & 
		\multicolumn{1}{c}{\cellcolor{black!30}\textbf{$2$}} & 
		\multicolumn{1}{c}{\cellcolor{black!30}\textbf{$3$}} & 
		\multicolumn{1}{c}{\cellcolor{black!30}\textbf{$4$}} & 
		\multicolumn{1}{c}{\cellcolor{black!30}\textbf{$5$}} & 
		\multicolumn{1}{c}{\cellcolor{black!30}\textbf{$6$}} &
		\multicolumn{1}{c}{\cellcolor{black!30}\textbf{$7$}} &
		\multicolumn{1}{c}{\cellcolor{black!30}\textbf{$8$}} & {} & {}\\ \hline
		\cellcolor{black!30}\textbf{Peur} & $4$ & $4$ & $3$ & $4$ & $1$ & $4$ & $4$ & $3$ & $R_1=27$ & $\bar{R}_1=3.375$\\ \hline
		\cellcolor{black!30}\textbf{Joie} & $3$ & $2$ & $2$ & $1$ & $4$ & $3$ & $1$ & $4$ & $R_2=20$ & $\bar{R}_2=2.50$ \\ \hline
		\cellcolor{black!30}\textbf{Tristesse} & $1$ & $3$ & $4$ & $2$ & $3$ & $2$ & $2$ & $2$ & $R_3=19$ & $\bar{R}_3=2.375$ \\ \hline
		\cellcolor{black!30}\textbf{Calme} & $2$ & $1$ & $1$ & $3$ & $2$ & $1$ & $3$ & $1$ & $R_4=14$ & $\bar{R}_4=1.75$  \\ \hline
		\end{tabular}
	\end{table}
	\setlength\extrarowheight{0pt}
	et en cas de non influence des traitements, nous nous attendons à avoir:
	
	ou aussi (c'est équivalent):
	
	S'il y a non influence des traitements ces quatre dernières valeurs devraient être égales et fluctuer autour de:
	
	Nous pouvons pressentir que la fluctuation des $\bar{R}_t$ autour de $\mu_R$ doit suivre une loi Normale centrée s'il y a vraiment non influence (il existe une démonstration de cela dans l'article original de Friedman mais elle comporte des lacunes par moments et donc nous nous abstiendrons de la présenter). Nous pouvons également réduire la loi Normale telle que:
	
	Il n'est pas toujours intuitif que l'erreur standard soit obtenue par la division de la racine de $B$ (du nombre de blocs) car la majorité des praticiens ont pour intuition de diviser par la racine $T$ du nombre de traitements lorsqu'ils étudient l'aspect théorique du test de Friedman. Mais cela peut se vérifier avec une application numérique soit en se rappelant que le calcul de la variance $\sigma_R$ se fait à partir des $B$ rangs d'un traitement donné, rangs dont les valeurs (dans l'exemple ci-dessus ces valeurs sont comprises $8$ fois entre $1$ et $4$) sont bien évidemment supposées pour un traitement donné indépendantes et identiquement distribuées.
	
	Donce nous avons:
	
	Contrairement au test de Kruskal-Wallis, nous ne faisons pas d'échantillonnage, donc nous ne devons pas corriger l'écart-type à l'aide du facteur de correction sur population finie (fcp) pour diminuer sa valeur.

	L'idée de Friedmann (du moins c'est ainsi que nous allons le présenter) est de dire que l'écart-type de la somme des rangs des traitements obtenue de façon identique que lors du test de Kruskal-Wallis (dont l'origine a été détaillée lors de notre étude du test de Mann-Withney):
	
	n'est cette fois-ci qu'un estimateur de l'écart-type vrai et qu'il faut utiliser la relation entre l'estimateur non biaisé et biaisé pour corriger cette estimation (relation démontrée lors de notre étude des estimateurs):
	
	Dès lors:
	
	où nous avons retiré un degré de liberté au Khi-deux pour la raison déjà rencontrée maintes fois dans le présent chapitre.
	
	Soit après quelques simplifications élémentaires nous obtenons le "\NewTerm{test Q de Friedman}\index{test Q de Friedman}" (qui est donc un test non paramétrique):
	
	\begin{tcolorbox}[colframe=black,colback=white,sharp corners]
	\textbf{{\Large \ding{45}}Exemple:}\\\\
	Pour en revenir à notre exemple il vient alors:
	
	La valeur critique de $\chi^2(T-1)$ au seuil de $5\%$ est de $7.81$. Donc nous ne rejetons malheureusement pas l'hypothèse null comme quoi les traitements n'ont aucune influence (absence de différence entre les traitements). La probabilité cumulée correspondant à $6.45$ (donc la $p$-value) est de $90\%$.
	\end{tcolorbox}
	
	\pagebreak
	\paragraph{Coefficient de corrélation des rangs de Spearman}\mbox{}\\\\
	Le coefficient de corrélation des rangs Spearman, noté $ R_S $ est le coefficient de corrélation de la séquence $\left(R(i),S(i)\right),i=1...n$, des rangs naturellement inspirés du coefficient de corrélation linéaire de Pearson que nous avons vu au début de cette section:
	
	Prenons également un exemple avant d'aborder l'aspect théorique. Considérez que la mesure d'un échantillon de taille $10$ (nous avons pris les mêmes valeurs que celles prises pour les études des tests de rangs non-paramétriques précédentes):
	\begin{table}[H]
		\centering
		\definecolor{gris}{gray}{0.85}
		\begin{tabular}{|c|c|}
		\hline
		\multicolumn{1}{c}{\cellcolor{black!30}$X$}  & 
  \multicolumn{1}{c}{\cellcolor{black!30}$Y$} \\ \hline
		$5.7$ & $8.1$  \\ \hline
		$3.2$ & $5.5$  \\ \hline
		$8.4$ & $3.4$  \\ \hline
		$4.1$ & $7.9$  \\ \hline
		$6.9$ & $4.6$  \\ \hline
		$5.3$ & $1.6$  \\ \hline
		$1.7$ & $8.5$  \\ \hline
		$3.2$ & $7.1$  \\ \hline
		$2.5$ & $8.7$  \\ \hline
		$7.4$ & $5.7$  \\ \hline
		\end{tabular}
	\end{table}
	avec leurs rangs respectifs conformément à l'idée d'approche de Kendall (idée simple mais qu'il fallait trouver!):
	\begin{table}[H]
		\centering
		\definecolor{gris}{gray}{0.85}
		\begin{tabular}{|c|c|c|c|}
		\hline
		\multicolumn{1}{c}{\cellcolor{black!30}$X$}  & 
 \multicolumn{1}{c}{\cellcolor{black!30}$R(i)$}  & \multicolumn{1}{c}{\cellcolor{black!30}$Y$} & \multicolumn{1}{c}{\cellcolor{black!30}$S(i)$} \\ \hline
		$5.7$ & $7$ & $8.1$  & $8$\\ \hline
		$3.2$ & $3$ & $5.5$ & $4$  \\ \hline
		$8.4$ & $10$ & $3.4$ & $2$ \\ \hline
		$4.1$ & $5$ & $7.9$ & $7$\\ \hline
		$6.9$ & $8$ & $4.6$ & $3$ \\ \hline
		$5.3$ & $6$ & $1.6$ & $1$  \\ \hline
		$1.7$ & $1$ & $8.5$ & $9$  \\ \hline
		$3.2$ & $4$ & $7.1$ & $6$ \\ \hline
		$2.5$ & $2$ & $8.7$  & $10$ \\ \hline
		$7.4$ & $9$ & $5.7$ & $5$ \\ \hline
		\end{tabular}
	\end{table}
	Maintenant, montrons que la relation donnée ci-dessus est considérablement simplifiée car les valeurs de $R$, telles que celles de $S$, parcourent les premiers $n$ entiers. Pour cela, rappelez-vous que nous avons prouvé dans la section Séquences et Séries, que:
	
	Dès lors:	
	
	Ainsi:
	
	Nous avons également prouvé dans la section Séquences et Séries que:
	
	dès lors:
	
	Ainsi nous avons:
	
	Maintenant, jouons un peu plus pour obtenir une expression plus simplifiée en observant que:
	
	Dès lors il vient:
	
	Nous avons alors:
	
	Mais nous avions prouvé que:
	
	Dès lors:
	
	On retrouve ainsi la fameuse relation disponible dans tous les (bons) livres de statistiques:
	
	Le coefficient de Spearman a les mêmes propriétés essentielles que le coefficient de Pearson à savoir que:
	
	et est égal à $0$ lorsque les variables sont corrélées (sans oublier les subtilités importantes déjà évoquées dans notre étude du coefficient de Pearson !!!).
	
	Notez que ce coefficient semble être défini uniquement pour une paire de variables (je n'ai jamais vu de généralisation à un cas multivarié à ce jour).
	
	\paragraph{Coefficient de corrélation Tau de Kendall}\label{Kendall tau-correlation coefficient}\mbox{}\\\\
	Le "\NewTerm{coefficient de corrélation tau-b de Kendall }\index{coefficient de corrélation tau-b de Kendall}", $\tau_b$, est une mesure d'association non paramétrique (statistiques de rang) basée sur le nombre de concordances et de discordances par paires d'observations.
	
	Il est utilisé dans de nombreux domaines de la sociologie, de l'ingénierie et de la finance quantitative (pour un exemple en finance quantitative voir la référence \cite{meissner2013correlation}) et souvent comparé ou communiqué en même temps avec le coefficient de corrélation de Pearson et le coefficient de corrélation de Spearman!

	Supposons deux observations $ (X_i, Y_i) $ et $ (X_j, Y_j) $. Ils sont dits \textit{concordants} s'ils sont dans le même ordre par rapport à chaque variable. Autrement dit, si:
	\begin{enumerate}
		\item[(1)] $X_{i}<X_{j}$ et $Y_{i}<Y_{j}$, ou si
		
		\item[(2)] $X_{i}>X_{j}$ et $Y_{i}>Y_{j}$
	\end{enumerate}	
	Ils sont \textit{discordants} s'ils sont dans l'ordre inverse pour $X$ et $Y$, ou si les valeurs sont disposées dans des directions opposées. Autrement dit, si:
	\begin{enumerate}
		\item[(1)] $X_{i}<X_{j}$ et $Y_{i}>Y_{j}$, ou si
		
		\item[(2)] $X_{i}>X_{j}$ et $Y_{i}<Y_{j}$
	\end{enumerate}	
	Les deux observations sont liées si $ X_i = X_j $ et / ou $ Y_i = Y_j $.
	
	\begin{tcolorbox}[title=Remarque,colframe=black,arc=10pt]
	Le tau-c $\tau_c$ de Kendall ignorent les liens. Nous ne discuterons pas du tau-c de Kendall ici car il n'est plus souvent utilisé à notre connaissance. En l'absence de liens, les deux formules donnent des résultats identiques de tout façon.
	\end{tcolorbox}
	
	Le nombre total de paires qui peuvent être construites pour un échantillon de taille $n$ est:
	
	$N$ peuvent être décomposées en ces cinq quantités:
	
	où $P$ est le nombre de paires concordantes, $Q$ est le nombre de paires discordantes, $X_0$ est le nombre de paires liées uniquement sur la variable $X$, $Y_0$ est le nombre de paires liées uniquement sur la variable $Y$ et $(XY)_0$ est le nombre de paires liées à la fois sur $X$ et $Y$.
	
	Le Kendall tau-b pour mesurer l'association d'ordre entre les variables $X$ et $Y$ est donné par la formule suivante:
	
	S'il n'y a pas de liaisons $X_0=Y_0=0$ alors on retombe sur une expression courante que l'on retrouve dans de nombreux manuels du Kendall tau-b:
	
	Si nous utilisons la distance de différence symétrique (voir le premier exemple ci-dessous), cette dernière relation peut s'écrire:
	
	Cette valeur est alors mise à l'échelle et varie entre $-1$ (relation monotone négative parfaite entre deux variables: un score plus faible sur la variable $X$ est toujours associé à un score plus élevé sur la variable $Y$) et $+1$ (relation monotone positive parfaite entre deux variables: un score inférieur sur la variable $X$ est toujours associé à un score inférieur sur la variable $Y$).

	Contrairement au coefficient de corrélation des rangs de Spearman, le tau-b de Kendall estime la variance de population comme suit:
	
	De plus, si deux variables sont indépendantes, $\tau_b = 0$ mais l'inverse n'est pas toujours vrai: une relation curviligne ou autre non monotone peut encore exister.
	
	\begin{tcolorbox}[colframe=black,colback=white,sharp corners]
	\textbf{{\Large \ding{45}}Exemples:}\\\\
	E1. Supposons que deux experts commandent quatre vins notés $\{a, b, c, d\}$. Le premier expert donne ses préférences: $E_1=[a, c, b, d]$,  qui correspond aux rangs suivants $R_{1} = [1,3,2,4]$ et le deuxième expert a comme préférences $E_{2}=[a, c, d, b]$ ce qui correspond aux rangs suivants $R_{2}=[1,4,2,3]$.  L'ordre donné par le premier expert est composé des $6$ paires ordonnées suivantes:
	
	L'ordre donné par le deuxième expert est composé des 6 paires ordonnées suivantes:
	
	L'ensemble des paires qui sont dans un seul ensemble de paires ordonnées est:
	
	ce qui donne une valeur de $d_{\Delta}\left(\text{P}_{1}, \text{P}_{2}\right)=2$. Avec cette valeur de la différence symétrique, nous calculons la valeur du coefficient de corrélation des rangs de Kendall entre l'ordre donné par ces deux experts comme:
	
	\end{tcolorbox}
	
	
	\begin{tcolorbox}[colframe=black,colback=white,sharp corners]
	Notez que le résultat est égal à:
	
	
	E2. Deux intervieweurs ont classé $12$ candidats (de A à L) pour un poste. Les résultats du plus préféré au moins préféré sont:
	\begin{itemize}
		\item Intervieweur 1: ABCDEFGHIJKL
		
		\item Intervieweur 2: ABDCFEHGJILK
	\end{itemize}
	nous voulons calculer le $\tau_b$.\\
	
	Pour cela, nous faisons un tableau des classements. La première colonne, \textit{Candidat} est facultative et est présente à titre de référence uniquement. Le classement pour \textit{Intervieweur 1} doit être dans l'ordre croissant (du moins au plus préféré des candidats):
	\begin{center}
	\begin{tabular}{|c|c|c|}
		\hline \textbf{Candidat} & \textbf{Intervieweur 1 }& \textbf{Intervieweur 2} \\
		\hline A & 1 & 1 \\
		\hline B & 2 & 2 \\
		\hline C & 3 & 4 \\
		\hline D & 4 & 3 \\
		\hline E & 5 & 6 \\
		\hline F & 6 & 5 \\
		\hline G & 7 & 8 \\
		\hline H & 8 & 7 \\
		\hline I & 9 & 10 \\
		\hline J & 10 & 9 \\
		\hline K & 11 & 12 \\
		\hline L & 12 & 11 \\ \hline
	\end{tabular}
	\end{center}
	Après nous comptons le nombre de paires concordantes, en utilisant la deuxième colonne. Les paires concordantes correspondent au nombre de rangs plus importants inférieurs à un certain rang. Par exemple, le premier rang dans la deuxième colonne de l'intervieweur est un $1$, donc tous les $11$ rangs en dessous sont plus grands:
	\end{tcolorbox}
	
	\begin{tcolorbox}[colframe=black,colback=white,sharp corners]
	\begin{center}
	\begin{tabular}{|c|c|c|c|c|}
		\hline \textbf{Candidat} & \textbf{Intervieweur 1 }& \textbf{Intervieweur 2} & \textbf{Concordant} & \textbf{Discordant}  \\
		\hline A & 1 & 1 & 11 & \\
		\hline B & 2 & 2 & &  \\
		\hline C & 3 & 4 & &  \\
		\hline D & 4 & 3 & &  \\
		\hline E & 5 & 6 & &  \\
		\hline F & 6 & 5 & &  \\
		\hline G & 7 & 8 & &  \\
		\hline H & 8 & 7 & &  \\
		\hline I & 9 & 10 & &  \\
		\hline J & 10 & 9 & &  \\
		\hline K & 11 & 12 & &  \\
		\hline L & 12 & 11 & &  \\ \hline
	\end{tabular}
	\end{center}
	Cependant, en descendant la liste jusqu'à la troisième ligne (un rang de $4$), le rang immédiatement inférieur ($3$) est plus petit, donc il ne compte en tant que paire concordante:
	\begin{center}
	\begin{tabular}{|c|c|c|c|c|}
		\hline \textbf{Candidat} & \textbf{Intervieweur 1 }& \textbf{Intervieweur 2} & \textbf{Concordant} & \textbf{Discordant}  \\
		\hline A & 1 & 1 & 11 & \\
		\hline B & 2 & 2 & 10 &  \\
		\hline C & 3 & 4 & 8 &  \\
		\hline D & 4 & $\xcancel{3}$ & &  \\
		\hline E & 5 & 6 & &  \\
		\hline F & 6 & 5 & &  \\
		\hline G & 7 & 8 & &  \\
		\hline H & 8 & 7 & &  \\
		\hline I & 9 & 10 & &  \\
		\hline J & 10 & 9 & &  \\
		\hline K & 11 & 12 & &  \\
		\hline L & 12 & 11 & &  \\ \hline
	\end{tabular}
	\end{center}
	Lorsque toutes les paires concordantes ont été comptées, cela ressemble à ceci:
	\begin{center}
	\begin{tabular}{|c|c|c|c|c|}
		\hline \textbf{Candidat} & \textbf{Intervieweur 1 }& \textbf{Intervieweur 2} & \textbf{Concordant} & \textbf{Discordant}  \\
		\hline A & 1 & 1 & 11 & \\
		\hline B & 2 & 2 & 10 &  \\
		\hline C & 3 & 4 & 8 &  \\
		\hline D & 4 & 4 & 8 &  \\
		\hline E & 5 & 6 & 6 &  \\
		\hline F & 6 & 5 & 6 &  \\
		\hline G & 7 & 8 & 4 &  \\
		\hline H & 8 & 7 & 4 &  \\
		\hline I & 9 & 10 & 2 &  \\
		\hline J & 10 & 9 & 2 &  \\
		\hline K & 11 & 12 & 0 &  \\
		\hline L & 12 & 11 & &  \\ \hline
	\end{tabular}
	\end{center}
	\end{tcolorbox}
	
	\begin{tcolorbox}[colframe=black,colback=white,sharp corners]
	Maintenant, nous comptons le nombre de paires discordantes et les insérons dans la colonne suivante. Le nombre de paires discordantes est similaire aux étapes précédentes, seulement nous recherchons des rangs plus petits, pas des plus grands:
	\begin{center}
	\begin{tabular}{|c|c|c|c|c|}
		\hline \textbf{Candidat} & \textbf{Intervieweur 1 }& \textbf{Intervieweur 2} & \textbf{Concordant} & \textbf{Discordant}  \\
		\hline A & 1 & 1 & 11 & 0\\
		\hline B & 2 & 2 & 10 & 0 \\
		\hline C & 3 & 4 & 8 &  1\\
		\hline D & 4 & 4 & 8 &  0\\
		\hline E & 5 & 6 & 6 &  1\\
		\hline F & 6 & 5 & 6 &  0\\
		\hline G & 7 & 8 & 4 &  1\\
		\hline H & 8 & 7 & 4 &  0\\
		\hline I & 9 & 10 & 2 &  1\\
		\hline J & 10 & 9 & 2 &  0\\
		\hline K & 11 & 12 & 0 &  1\\
		\hline L & 12 & 11 & &  \\ \hline
	\end{tabular}
	\end{center}
	Et nous additionnons ces deux nouvelles colonnes:
	\begin{center}
	\begin{tabular}{|c|c|c|c|c|}
		\hline \textbf{Candidat} & \textbf{Intervieweur 1}& \textbf{Intervieweur 2} & \textbf{Concordant} & \textbf{Discordant}  \\
		\hline A & 1 & 1 & 11 & 0\\
		\hline B & 2 & 2 & 10 & 0 \\
		\hline C & 3 & 4 & 8 &  1\\
		\hline D & 4 & 4 & 8 &  0\\
		\hline E & 5 & 6 & 6 &  1\\
		\hline F & 6 & 5 & 6 &  0\\
		\hline G & 7 & 8 & 4 &  1\\
		\hline H & 8 & 7 & 4 &  0\\
		\hline I & 9 & 10 & 2 &  1\\
		\hline J & 10 & 9 & 2 &  0\\
		\hline K & 11 & 12 & 0 &  1\\
		\hline L & 12 & 11 & &  \\ 
		\hline  &  & \textbf{Total:} & 61 & 5 \\ \hline
	\end{tabular}
	\end{center}
	Et, en insérant les totaux dans la formule, nous obtenons:
	
	\end{tcolorbox}
	Nous n'avons pas besoin d'avoir deux juges pour se comparer. Par exemple, dans l'analyse de sensibilité, nous pouvons considérer qu'une entreprise souhaite mettre un nouveau produit alimentaire sur le marché. Les clients potentiels dégustent $10$  versions de ce produit qui ont différents niveaux de lipides. Les clients évaluent ensuite chaque variante. Une fois agrégées sur l'ensemble des dégustateurs, ces notes ne sont pas étudiées en tant que telles mais sont classées de manière à obtenir un ordre global de préférences. Nous voulons savoir si cette classification peut être liée au taux de lipides:
	\begin{center}
	$\begin{array}{|c|c|}\hline \textbf{Concentration de lipide} & \textbf{Préférences}  \\  \hline 13 & 10 \\  \hline 14 & 7 \\  \hline 15 & 9 \\  \hline 16 & 8 \\  \hline 17 & 5 \\  \hline 18 & 6 \\  \hline 19 & 4 \\  \hline 20 & 3 \\  \hline 21 & 1 \\ \hline  22 & 2 \\ \hline  \end{array}$
	\end{center}
	Ensuite, nous appliquons la même procédure que ci-dessus en comparant les deux colonnes et nous obtenons $\tau_b=-0.822$.
	
	\begin{tcolorbox}[title=Remarque,colframe=black,arc=10pt]
	Notez que alors que le coefficient de corrélation des rangs de Spearman peut être considéré comme le coefficient de corrélation de Pearson habituel calculé à partir des rangs, le Kendall tau représente plutôt une probabilité. Plus précisément, il s'agit de la différence entre la probabilité que les données observées soient dans le même ordre pour les deux variables contre la probabilité que les données observées soient dans des ordres différents pour les deux variables.
	\end{tcolorbox}
	
	Le coefficient de corrélation de Kendall ne dépend que de l'ordre des paires, et il peut toujours être calculé en supposant que l'un des rangs sert de point de référence (par exemple avec $N=4$ éléments nous supposons arbitrairement que le premier ordre est égal à $1234$). Par conséquent, avec deux ordres de classement fournis sur $n$ objets, il y a $n!$ différents résultats possibles (chacun correspondant à un ordre possible donné) à considérer pour calculer la distribution d'échantillonnage de $\tau_b$. On peut calculer la probabilité $p$ associée à chaque valeur possible de $\tau_b$. Par exemple, nous trouvons que la valeur $p$ associée à un test unilatéral pour une valeur de $\tau_b=\frac{2}{3}$ est égale à:
	
	Par conséquent, pour notre exemple des vins ci-dessus, nous ne pouvons pas rejeter l'hypothèse nulle, et nous ne pouvons pas conclure que les experts (juges) aient un désaccord significatif dans leur classiement des vins.
	
	Le calcul de la distribution d'échantillonnage est toujours théoriquement possible car fini. Mais cela nécessite de calculer $n!$ coefficients de corrélation, et il devient donc pratiquement impossible d'implémenter ces calculs pour des valeurs même modérément grandes de $n$. Ce problème n'est cependant pas aussi drastique qu'il n'y paraît car la distribution d'échantillonnage de $\tau_b$ converge vers une distribution Normale (la convergence est satisfaisante pour des valeurs de $n$ supérieures à $10$), avec une moyenne de $0$ et une variance égale à \footnote{Nous sommes allés aussi loin que possible (1933) pour trouver la preuve détaillée originale de cette variance, mais nous n'avons pas été en mesure de la trouver. Malheureusement, nous n'avons ni le temps ni le plaisir de réécrire la preuve par nous-mêmes ... Donc sans preuve - comme toujours en science (!) - utilisez cette relation à vos risques et périls!}:
	
	Par conséquent, pour $n$ supérieur à $10$, un test d'hypothèse nulle peut être effectué en transformant $\tau_b$ en une valeur $Z$ comme:
	
	Cette valeur $Z$ est Normalement distribuée avec une moyenne de $0$ et un écart type de $1$.

		
	\subsubsection{Statistiques de rangs}\label{range statistics}
	La "\NewTerm {statistique de rangs}\index{statistique de rangs}" est un outil très important en finance et en ingénierie qualité (pour ne citer que les deux exemples les plus connus). Comme le lecteur le verra dans ce qui suit ci-dessous, ces statistiques sont par construction un sous-domaine des statistiques d'ordre et le résultat que nous obtiendrons ici nous sera absolument utile dans le cadre des cartes de contrôle qualité (\SeeChapter{voir section Ingénierie page \pageref{quality control charts}}) et donc aussi dans les signaux de trading en finance.
	
	Étant donné $X_1,X_2,...,X_n$ supposées être variables aléatoires indépendantes et identiquement distribuées à partir d'une loi de distribution $F$ et de densité $f$. Rappelons que nous définissons la statistique d'ordre $X_{(i)} $ par:
	
	En notant:
	
	Les variables $ W_n $ et $ M_n $ définissent les statistiques d'ordre extrêmes et leur différence:
	
	s'appelle la "\NewTerm{déviation extrême}\index{déviation extrême}"
	
	Pour ce qui va suivre, nous considérerons comme évidente la relation:
	
	Déterminons maintenant la fonction de distribution du maximum $M_n$:
	
	car écrore que $M_n\leq$ équivaut à dire que pour chaque $X_{(i)}$  nous avons $X_{(i)}\leq n$  (pas facile à deviner qu'il faut avoir cette approche...).
	
	Nous avons alors puisque les variables sont indépendantes (\SeeChapter{voir section de Probabilités page \pageref{joint probability}}):
	
	et par suite nous avons évidemment la fonction de distribution:
	
	Respectivement en se basant sur la même idée:
	
	et par suite nous avons évidemment la fonction de distribution:
	
	Il vient alors:
	
	en ayant utilisé la linéarité de l'espérance et le fait que pour les deux fonctions de distribution nous travaillons sur la même variable aléatoire.
	
	En faisant une intégration par parties (\SeeChapter{voir section Calcul Différentiel et Intégral page \pageref{integration by parts}}):
	
	en n'oubliant pas que $F(-\infty)$ et $F(+\infty)=1$.

	Maintenant considérons le cas particulier où la fonction de répartition suit une loi Normale centrée réduite:
	
	Nous avons alors:
	
	Faisons un changement de variables:
	
	Nous avons alors:
	
	et nous trouvons alors la relation donnée ($99\%$ du temps sans démonstration) dans les livres de statistiques des procédés (SPC):
	
	appelée "\NewTerm{constante de Hartley}\index{constante de Hartley}\label{hartley constant}" et donc:
	
	Cette constante est impossible à ce jour et à notre connaissance à calculer formellement. Soit il faut passer par des approximations en série de Taylor des termes de l'intégrale, ce qui devient un cauchemar pour $n$ grand, soit par un calcul utilisant la méthode de Monte-Carlo (\SeeChapter{voir section Méthodes Numériques page \pageref{monte carlo simulations}}). Comme c'est relativement long à implémenter dans un tableur, les ingénieurs qualité préfèrent utiliser des tables dans lesquelles nous trouvons par exemple:
	\begin{table}[H]
		\begin{center}
			\definecolor{gris}{gray}{0.85}
			\begin{tabular}{|c|c|}
			\hline
			\multicolumn{1}{c}{\cellcolor{black!30}Valeurs de $n$}  & 
	  \multicolumn{1}{c}{\cellcolor{black!30}Valeurs de $d_2(n)$ pour une distribution Normale} \\ \hline
			$2$ & $1.128$  \\ \hline
			$3$ & $1.693$  \\ \hline
			$4$ & $2.059$  \\ \hline
			$5$ & $2.326$  \\ \hline
			$6$ & $2.534$  \\ \hline
			$7$ & $2.704$  \\ \hline
			$8$ & $2.847$  \\ \hline
			$9$ & $2.970$  \\ \hline
			$10$ & $3.078$  \\ \hline
			$\ldots$ & $\ldots$  \\ \hline
			\end{tabular}
			\caption{Valeurs tabuleés de la constante de Hartley $d_2(n)$}
		\end{center}
	\end{table}
	Voyons maintenant la variance de l'étendue en utilisant toujours le théorème de Huygens:
	
	Le calcul de $\text{E}(R^2)$ n'est vraiment pas sympa (du moins je n'ai rien trouvé qui satisfasse le but pédagogique de ce livre), la plus petite preuve complète tient sur $3-4$ pages A4 et formellement n'apporte pass grand chose car on finit par une intégrale qui ne se calcule pas à la main (par contre si quelqu'un a une preuve simple, élégante et détaillée n'hésitez pas à nous la faire parvenir!). C'est pour cette raison qu'après avoir écrit:
	
	si nous écrivons maintenant comme le font de nombreux livres techniques:
	
	Nous avons alors:
	
	Mais comme nous ne connaissons pas l'estimateur sans biais du maximum de vraisemblance de l'écart type $\sigma$, nous utiliserons la relation prouvée:
	
	Pour enfin avoir un estimateur biaisé de la variance de l'étendue:
	
	Voici quelques valeurs tabulées de $d_3(n)$:
	\begin{table}[H]
		\begin{center}
			\definecolor{gris}{gray}{0.85}
			\begin{tabular}{|c|c|}
			\hline
			\multicolumn{1}{c}{\cellcolor{black!30}Valeurs de $n$}  & 
	  \multicolumn{1}{c}{\cellcolor{black!30}Valeurs de $d_3(n)$ pour une distribution Normale} \\ \hline
			$2$ & $0.852$  \\ \hline
			$3$ & $0.888$  \\ \hline
			$4$ & $0.879$  \\ \hline
			$5$ & $0.864$  \\ \hline
			$6$ & $0.848$  \\ \hline
			$7$ & $0.833$  \\ \hline
			$8$ & $0.819$  \\ \hline
			$9$ & $0.807$  \\ \hline
			$10$ & $0.797$  \\ \hline
			$\ldots$ & $\ldots$  \\ \hline
			\end{tabular}
			\caption{Valeurs tabulées de la constante $d_3(n)$}
		\end{center}
	\end{table}
	
	\pagebreak
	\paragraph{Test (de l'étendue) de Tukey}\label{Tukey's range test}\mbox{}\\\\
	Supposons que nous avons $Z_k$ variables aléatoires centrées réduites et indépendantes. Et notons $U$ une variable aléatoire suivant une loi du Khi-deux à v degrés de liberté.
	
	Définissons maintenant pour des raisons qui paraîtront évidentes un peu plus loin, "\NewTerm{l'étendue Studentisée}\index{l'étendue Studentisée}" (l'origine du nom provient de sa ressemblance avec la définition de la loi de Student) par:
	
	et tentons de déterminer si cette relation suit une loi connue et une application possible (nous retrouvons au numérateur ce que nous avions défini plus haut comme étant la "déviation extrême" mais avec une autre notation).
	
	Pour cela, montrons que nous tombons sur la définition ci-dessus en considérant un cas un peu plus général où nous avons $X_i$ variables aléatoires indépendantes qui suivent une loi Normale $\mathcal{N}\left(\mu,\sigma\right)$  et avec l'écart type:
	
	Et étudions le ratio:
	
	Maintenant, procédons aux transformations classiques déjà vues et démontrées et utilisées maintes fois depuis le début de ce chapitre:
	
	Et nous avons alors:
	
	Donc voilà déjà pour la première étape. Pour l'instant, même si nous ne savons toujours pas si cette définition loi suit une distribution connue, nous pouvons déjà poser la définition très intéressante suivante (le terme de gauche est toujours positif):
	
	ou autrement écrite:
	
	et donc nous pouvons calculer quelle est la probabilité cumulée d'une étendue obtenue par mesures comparée à une étendue critique $R_{X,\text{crit}}$ correspondant directement à un seuil $\alpha$ imposé. Ce qui nous amène à pouvoir écrire que:
	
	Maintenant, rappelons que nous avons vu plus haut que la fonction de distribution de la déviation extrême était donnée par une relation à notre connaissance non calculable analytiquement:
	
	Donc la fonction de distribution $Q_{k,v,1-\alpha}$ n'est par conséquence pas assimilable à une loi connue quand $F(x)$ est une loi quelconque. Il faut donc malheureusement tabuler cette distribution par la méthode de Monte Carlo (\SeeChapter{voir section Méthodes Numériques page \pageref{monte carlo simulations}}) ou se réferer à des tables déjà existantes.
	
	Maintenant pour continuer, nous faisons un crochet par l'ANOVA à un facteur contrôlé que nous avions étudié. Rappelons d'abord que nous avons démontré que que pour des variables aléatoires Normales indépendantes et identiquement distribuées nous avions:
	
	et puisque l'ANOVA un facteur contrôlé est aussi basé sur l'hypothèse que:
	
	cela implique qu'asymptotiquement les estimateurs ont la même propriété:
	
	Nous savons aussi que l'écart-type de la moyenne d'un échantillon de l'ANOVA un facteur contrôlé est donc donné la cadre et les hypothèses de l'ANOVA à un facteur contrôlé par:
	
	Mais dans le cadre de l'ANOVA à un facteur contrôlé, nous avons aussi montré que sous les hypothèses imposées, nous avions:
	
	Il vient alors que:
	
	est un estimateur de:
	
	Et comme nous avions démontré que:
	
	Il vient alors que:
	
	Dès lors, nous sommes naturellement amenés à constater que la relation que nous avons définie plus  haut:
	
	peut-être utilisée dans l'étude de l'ANOVA à un facteur contrôlé sous la forme:
	
	pour faire un test préalable ou postérieur (post hoc) à une ANOVA à un facteur contrôlé pour vérifier l'hypothèse d'égalité des moyennes et identifier quelles sont les moyennes aberrantes (test de comparaisons multiples). Donc le test de Tukey est souvent accompagné du test C de Cochran que nous avons déjà étudié plus haut lorsque nous faisons une ANOVA à un facteur contrôlé.
	
	Soit, dans le cadre de l'ANOVA, nous devrions rejeter l'hypothèse d'égalité des moyennes des échantillons si:
	
	ou autrement écrit:
	
	Dans ce cas, il est alors quasi immédiat que nous pouvons construire l'intervalle de confiance suivant:
	
	Il s'agit du "\NewTerm{test de Tukey de l'étendue}\index{test de Tukey de l'étendue}" ou simplement nommé "\NewTerm{test de Tukey}".
	
	Il faut savoir maintenant qu'il existe un test post-hoc de l'ANOVA à un facteur qui lors de l'application de:
	
	ne va pas prendre les deux moyennes les plus extrêmes mais va comparer toutes les moyennes deux à deux (et pourquoi pas après tout!) avec la plus grande moyenne (bon on pourrait aussi s'amuser à faire toutes les combinaisons possibles comme le font certains logiciels de statistiques). Dans ce cas la relation à utiliser sera la même que ci-dessus à la différence que si nous avons par exemple une ANOVA à un facteur avec $4$ niveaux nous aurons alors $3$ comparaisons deux à deux (les différences des moyennes doivent toujours être positives). Ainsi, en imaginant que la troisième moyenne est la plus grande et que dans l'ordre décroissant les moyennes les plus grandes sont la 4ème, 2ème et 1ère (donc la 1ère est la plus petite) il vient alors:
	
	Cette façon de faire (d'étendre le prinicipe de base du test de Tukey), s'appelle le "\NewTerm{test de Newman-keuls}\index{test de Newman-keuls}" ou encore "\NewTerm{test de Student–Newman–Keuls}\index{test de Student–Newman–Keuls}" (SNK).
	
	\pagebreak
	\subsubsection{Théorie des Valeurs Extrêmes}
	La "\NewTerm{théorie des valeurs extrêmes}\index{théorie des valeurs extrêmes}" ou "\NewTerm{analyse des valeurs extrêmes}\index{analyse des valeurs extrêmes}" (AVE) est une branche de la statistique traitant des écarts extrêmes par rapport à la médiane de distributions de probabilité. Elle cherche à évaluer, à partir d'un échantillon ordonné donné d'une variable aléatoire donnée, la probabilité d'événements plus extrêmes que tout autre observé précédemment. L'analyse des valeurs extrêmes est largement utilisée dans de nombreuses disciplines, telles que l'ingénierie structurelle, la finance, les sciences de la terre, la prévision du trafic et l'ingénierie géologique. Par exemple, l'AVE pourrait être utilisé dans le domaine de l'hydrologie pour estimer la probabilité d'un événement d'inondation inhabituellement important, comme l'inondation de 100 ans. De même, pour la conception d'un brise-lames, un ingénieur côtier chercherait à estimer la vague de 50 ans et à concevoir la structure en conséquence.
	
	\StickyNote[2.5cm]{\LARGE À finir en fonction des dons}[6.5cm]
	
	\pagebreak
	\subsection{Statistiques Multivariées}
	De toute évidence, les statistiques multivariées sont une subdivision des statistiques englobant l'observation et l'analyse simultanées de plus d'une variable aléatoire.
	
	\begin{tcolorbox}[title=Remarque,colframe=black,arc=10pt]
	Certains types de problèmes impliquant des données multivariées, par exemple la régression linéaire simple et la régression multiple que nous avons vu plus haut, ne sont généralement pas considérés comme des cas particuliers de statistiques multivariées car l'analyse est traitée en considérant la distribution conditionnelle (univariée) d'une seule variable résultat compte tenu des autres variables.
	\end{tcolorbox}	 
	
	Il existe de nombreux modèles différents, chacun avec son propre type d'analyse que nous tenterons d'aborder dans ce livre comme toujours avec un maximum de détails et de manière le plus accessible possible (avec des exemples appliqués utilisant des logiciels de statistiques dans les livres compagnons bien sûr!):
	\begin{enumerate}
		\item L'Analyse en Composantes Principales (ACP) qui crée un nouvel ensemble de variables orthogonales contenant les mêmes informations que l'ensemble d'origine. Il fait pivoter les axes de variation pour donner un nouvel ensemble d'axes orthogonaux, ordonnés de manière à résumer les proportions décroissantes de la variation.
		
		\item L'Analyse Factorielle (AF) est similaire à l'ACP mais permet à l'utilisateur d'extraire un nombre spécifié de variables synthétiques, inférieur à l'ensemble d'origine, laissant la variation inexpliquée restante comme une erreur. Les variables extraites sont appelées variables ou facteurs latents; chacun peut être supposé tenir compte de la covariation dans un groupe de variables observées.

		\item L'Analyse Canonique de la Corrélation (ACC) qui trouve des relations linéaires entre deux ensembles de variables; c'est la version généralisée (c'est-à-dire canonique) de la corrélation bivariée.

		\item L'Analyse des Correspondances (AC), ou moyenne réciproque, qui trouve (comme l'ACP) un ensemble de variables synthétiques qui résument l'ensemble d'origine. Le modèle sous-jacent suppose des différences du chi-deux entre les observations.

		\item L'Analyse de Correspondance Canonique (ou «contrainte») (ACC) pour résumer la variation conjointe dans deux ensembles de variables; combinaison d'analyse de correspondance et d'analyse de régression multivariée. Le modèle sous-jacent suppose des différences du chi-deux entre les observations..

		\item Le Positionnement Multidimensionnel comprend divers algorithmes pour déterminer un ensemble de variables synthétiques qui représentent le mieux les distances par paires entre les enregistrements. La méthode originale est l'Analyse en Coordonnées Principales (ACoP basé sur l'ACP).

		\item L'Analyse Discriminante (linéaire ou quadratique) (AD), ou l'Analyse Canonique des Variables  (ACD), tente de déterminer si un ensemble de variables peut être utilisé pour distinguer deux ou plusieurs groupes d'observations.

		\item Les Modèles d'Équations Simultanées impliquent plus d'une équation de régression, avec différentes variables dépendantes, estimées ensemble.
		
		\item L'Analyse Multivariée de la Variance (MANOVA) qui étend l'analyse de la variance pour couvrir les cas où il y a plus d'une variable dépendante à analyser simultanément.

		\item L'Analyse Multivariée de la Covariance (MANCOVA) qui est une extension des méthodes d'analyse de la covariance (ANCOVA) pour couvrir les cas où il y a plus d'une variable dépendante et où le contrôle des variables indépendantes continues concomitantes (covariables) est nécessaire.
		
		\item Les Modèles Mixtes contenant à la fois des effets fixes et des effets aléatoires. Ces modèles sont utiles dans une grande variété de disciplines des sciences physiques, biologiques et sociales. Ils sont particulièrement utiles dans les contextes où des mesures répétées sont effectuées sur les mêmes unités statistiques (étude longitudinale).
	\end{enumerate}
	
	\subsubsection{Analyse en Composantes Principales}\label{principal component analysis}
	L'analyse en composantes principales (A.C.P.) est une méthode mathématique d'analyse graphique de données qui consiste à rechercher des directions dans l'espace qui représentent le mieux les corrélations entre des variables aléatoires $n$ (supposées avoir des relations linéaires entre elles). En d'autres termes, il s'agit d'un processus de réduction de dimension (comme pour le processus de décomposition en valeurs singulières - DVS - prouvé dans la section d'Algèbre Linéaire) car il donne la possibilité à l'analyste de choisir quelles sont les variables d'un modèle qui expliquent le mieux la variabilité.
	
	Simplement dit, une A.C.P. permet par exemple de trouver dans un jeu de données des similitudes de comportement d'achat entre des classes observées.
	
	Même si l'A.C.P. est principalement utilisée pour visualiser des données, il ne faut pas oublier que c'est aussi un moyen de:
	\begin{itemize}
		\item De décorréler ces données. Dans la nouvelle base, constituée des nouveaux axes, les points ont une corrélation nulle (nous le démontrerons).
		
		\item De classifier ces données en amas (clusters) corrélés (dans l'industrie c'est surtout cette possibilité qui est intéressante!).
	\end{itemize}
	\begin{tcolorbox}[title=Remarques,colframe=black,arc=10pt]
	\textbf{R1.} Il existe plusieurs versions de l'A.C.P. connues sous le nom de "\NewTerm{transformée de Karhunen-Loève}\index{transformée de Karhunen-Loève}" ou de "\NewTerm{transformée de Hotelling}\index{transformée de Hotelling}" et qui peuvent aussi bien être appliquées sans programmation V.B.A. dans Microsoft Excel que dans des logiciels spécialisés (où le temps de calcul sera par contre plus bref... et les résultats plus précis aussi...).\\
	
	\textbf{R2.} Selon les auteurs et le point de vue l'A.C.P appartient au domaine des statistiques nommé "Statistiques Exploratoires".
	\end{tcolorbox}
	Lorsque nous ne considérons que deux effets, il est usuel de caractériser leur effet conjoint via le coefficient de corrélation. Lorsque l'on se place en dimension deux, les points disponibles (l'échantillon de points tirés suivant la loi conjointe) peuvent être représentés dans un plan. Le résultat d'une A.C.P. dans ce plan consiste en la détermination des deux axes qui expliquent le mieux la dispersion des points disponibles.

	Lorsqu'il y a plus de deux effets, par exemple trois effets, il y a trois coefficients de corrélations à prendre en compte. La question qui a donné naissance à l'A.C.P. est: comment avoir une intuition rapide des effets conjoints?

	En dimension plus grande que deux, une A.C.P. va toujours déterminer les axes qui expliquent le mieux la dispersion du nuage des points disponibles.

L'objectif de l'A.C.P. est de décrire graphiquement un tableau de données d'individus avec leurs variables quantitatives de grande taille:
	\begin{table}[H]
		\centering
		\definecolor{gris}{gray}{0.85}
		\begin{tabular}{|c|c|}
			\hline
			\multicolumn{1}{c}{\cellcolor{black!30}\textbf{Individus/Variables}} & 
  \multicolumn{1}{c}{\cellcolor{black!30}\textbf{$\text{var}_1,\ldots,\text{var}_j,\ldots,\text{var}_n$}} \\ \hline
			$\text{ind}_1$ & \\	
			$\vdots$ & \\	
			$\text{ind}_i$ & $x_{ij}$ \\
			$\vdots$ & \\
			$\text{ind}_n$ & {} \\				
			\\ \hline
		\end{tabular}
		\caption{Représentation-type d'un tableau A.C.P.}
	\end{table}
	Afin de ne pas alourdir l'exposé de cette méthode et de permettre au lecteur de refaire complètement les calculs, nous travaillerons sur un exemple.
	
	Considérons pour l'exemple une étude d'un botaniste qui a mesuré les dimensions de $15$ fleurs d'iris (l'A.C.P. est aussi très utilisée en finance pour déterminer les éléments qui influencent le plus la volatilité d'un portefeuille). Les trois variables ($p=3$) mesurées sont:
	\begin{itemize}
		\item $x_1$: longueur du sépale

		\item $x_2$: largeur du sépale

		\item $x_3$: longueur du pétale
	\end{itemize}
	
	Les données sont les suivantes:
	
	\begin{table}[H]
		\begin{center}
			\definecolor{gris}{gray}{0.85}
			\begin{tabular}{|c|c|c|c|}
			\hline
			\multicolumn{1}{c}{\cellcolor{black!30}Fleur $n^{\circ}$}  & 
	  \multicolumn{1}{c}{\cellcolor{black!30}$x_1$} & 
	  \multicolumn{1}{c}{\cellcolor{black!30}$x_2$} & 
	  \multicolumn{1}{c}{\cellcolor{black!30}$x_3$}\\ \hline
			$1$ & $5.1$ & $3.5$ & $1.4$  \\ \hline
			$2$ & $4.9$ & $3.0$ & $1.4$  \\ \hline
			$3$ & $4.7$ & $3.2$ & $1.3$  \\ \hline
			$4$ & $4.6$ & $3.1$ & $1.5$  \\ \hline
			$5$ & $5.0$ & $3.6$ & $1.4$  \\ \hline
			$6$ & $7.0$ & $3.2$ & $4.7$  \\ \hline
			$7$ & $6.4$ & $3.2$ & $4.5$  \\ \hline
			$8$ & $6.9$ & $3.1$ & $4.9$  \\ \hline
			$9$ & $5.5$ & $2.3$ & $4.0$  \\ \hline
			$10$ & $6.5$ & $2.8$ & $4.6$  \\ \hline
			$11$ & $6.3$ & $3.3$ & $6.0$  \\ \hline
			$12$ & $5.8$ & $2.7$ & $5.1$  \\ \hline
			$13$ & $7.1$ & $3.0$ & $5.9$  \\ \hline
			$14$ & $6.3$ & $2.9$ & $5.6$  \\ \hline
			$15$ & $6.5$ & $3.0$ & $5.8$  \\ \hline
			\end{tabular}
			\caption[]{Exemple pratique de données tabulaires A.C.P.}
		\end{center}
	\end{table}
	Pour nous un tel tableau de données sera tout simplement une matrice réelle à $n$ lignes (les individus) et à $p$ colonnes (les variables):
	
	Par la suite l'indice $i$ correspondra à l'indice ligne et donc aux individus. Nous identifierons donc l'individu $i$ avec le point ligne $x_{i.}=(x_{i1},\ldots,x_{ip})$ qui sera considéré comme un point dans un espace affine (\SeeChapter{voir section Calcul Vectoriel page \pageref{affine space}}) de dimension $p$. L'indice $j$ correspondra à l'indice colonne donc aux variables. Nous identifierons la variable $j$ avec le vecteur-colonne:
	
	c'est donc un vecteur dans l'espace vectoriel de dimension $n$ dans $\mathbb{R}^n$.
	
	Nous nous placerons dans la suite suivant deux points de vue: soit nous prendrons le tableau de données comme $n$ points dans un espace affine de dimension $p$, soit nous prendrons ce tableau comme $p$ points d'un espace vectoriel de dimension $n$. Nous verrons qu'il y a des dualités entre ces deux points de vue.
	
	L'outil mathématique que nous allons utiliser ici est l'algèbre linéaire (\SeeChapter{voir section Algèbre Linéaire page \pageref{linear algebra}}), avec les notions de produit scalaire, de norme euclidienne et de distance euclidienne.
	
	Afin de simplifier la présentation, nous allons dans un premier temps considérer que chaque individu, comme chaque variable, a la même importance, le même poids. Nous ne considérerons aussi que le cas de la distance euclidienne.
	
	Nous allons commencer en centrant les données, c'est-à-dire mettre l'origine du système d'axes au centre de gravité du nuage de points. Ceci ne modifie pas l'aspect du nuage, mais permet d'avoir les coordonnées du point $M$ égales aux coordonnées du vecteur $\overrightarrow{GM}$ et donc de se placer dans l'espace vectoriel pour pouvoir y faire les calculs! Comme nous supposons dans toute la suite que les poids des individus sont identiques, nous prendrons donc $m_i=1/n$ avec $i=1\ldots n$.
	
	Nous considérons le repère orthonormé $(\text{O},\vec{e}_1,\vec{e}_2,\ldots,\vec{e}_p)$ dans la base canonique $(\vec{e}_1,\vec{e}_2,\ldots,\vec{e}_p)$ de $\mathbb{R}^p$. Soit donc $G$ le centre de gravité du nuage de points. Comme chaque variable ou chaque individu est supposé avoir le même poids, $G$ a alors pour coordonnées dans le repère $(\text{O},\vec{e}_1,\vec{e}_2,\ldots,\vec{e}_p)$:
	
	avec:
	
	Nous avons alors pour l'instant sous forme graphique:
	\begin{figure}[H]
		\centering
		\includegraphics{img/arithmetics/pca_gravitypoint.jpg}
		\caption[]{Points de mesures et centre de gravité}
	\end{figure}
	Nous appelons "\NewTerm{matrice centrée}\index{matrice centrée}" la matrice:
	
	\begin{tcolorbox}[title=Remarque,colframe=black,arc=10pt]
	 La matrice des données centrées contient les coordonnées centrées (que nous noterons $xc_{ij}$) des individus dans le repère $(G,\vec{e}_1,\vec{e}_2,\ldots,\vec{e}_p)$. Nous nous placerons dans la suite toujours dans ce repère pour le nuage de points des individus et nous prendrons  $\text{O}=G$.
	\end{tcolorbox}	
	Pour notre exemple, nous avons:
	
	et pour la matrice centrée:
	
	et sous forme graphique:
	\begin{figure}[H]
		\centering
		\includegraphics{img/arithmetics/pca_centered_measured_points.jpg}
		\caption[]{Points de mesures centrés}
	\end{figure}
	Pour donner une importance identique à chaque variable afin que le type d'unités des mesures n'influence pas l'analyse (et aussi comme nous l'avons prouvé page \pageref{correlation matrix} que la matrice de corrélation est égale à la matrice de variance-covariance!), nous travaillerons avec les données centrées réduites (\SeeChapter{voir section Statistiques page \pageref{reduced centered variable}}). Pour cela, nous noterons d'abord:
	
	où le lecteur aura remarqué que nous avons pris la variance biaisée. Mais dans la réalité on prendraévidemment l'estimateur de la variance et donc on divisera par $n-1$ plutôt que par $n$.
	
	La variance d'échantillon de la variable centrée est donc égale à un facteur $1 / n$ près à la norme de cette même variable mais centrée. La matrice des données centrées réduites (sans dimensions) est alors:
	
	Si nous notons $D_{1/\sigma}$ la matrice diagonale suivante:
	
	Nous avons alors:
	
	\begin{tcolorbox}[title=Remarque,colframe=black,arc=10pt]
	Chaque composante de la matrice $Y$ est donc de moyenne nulle et de variance unitaire (ce qui revient à dire que la norme de la variable centrée réduite est unitaire comme nous allons de suite le démontrer).
	\end{tcolorbox}	
	Nous définissons la "\NewTerm{matrice des données centrées normées}\index{matrice des données centrées normées}" par (nous parlons alors de "A.C.P. normée" ce qui n'est pas obligatoire mais simplifie l'interprétation):
	
	Soit encore (il s'agit simplement de l'erreur quadratique moyenne que nous avions introduite plus haut dans cette section):
	
	La terminologie vient bien évidemment du fait que la somme des carrées des composantes de chaque colonne de la matrice $Z$ est de norme unitaire. En effet:
	
	Ce qui donne:
	
	Nous avons graphiquement:
	\begin{figure}[H]
		\centering
		\includegraphics{img/arithmetics/pca_centered_reduced_measured_points.jpg}
		\caption[]{Points de mesures centrés et réduits}
	\end{figure}
	Représenter le nuage de points des données centrées réduites ou centrées normées ne modifie rien à la forme de celui-ci. En effet, la différence entre les deux n'est qu'un changement d'échelle.
	
	L'information intéressante pour les individus est la distance entre les points! En effet plus cette distance sera grande entre deux individus equation et equation plus les deux individus seront différents et mieux on pourra les caractériser. Mais il faut d'abord choisir une distance. Nous prendrons la distance euclidienne (\SeeChapter{voir section Topologie page \pageref{topology}}):
	
	Les figures suivantes montrent les projections orthogonales dans l'espace de ce nuage de points respectivement dans les plans $(\text{O},\vec{e}_1,\vec{e}_2),(\text{O},\vec{e}_2,\vec{e}_3)$ et enfin dans $(\text{O},\vec{u}_1,\vec{u}_3)$ qui est la meilleure projection, appelée "\NewTerm{plan factoriel}\index{plan factoriel}" (ou parfois "\NewTerm{diagramme des scores}\index{diagramme des scores}"), dans le sens où elle respecte le mieux les distances entre les individus (in extenso, elle déforme moins le nuage de points dans l'espace). L'objectif de l'A.C.P. est de déterminer ce meilleur plan et nous démontrerons comment.
	\begin{figure}[H]
		\centering
		\includegraphics{img/arithmetics/pca_projection_cloud_xy.jpg}
		\caption[]{Projection des points sur le plan horizontal du repère centré}
	\end{figure}
	\begin{figure}[H]
		\centering
		\includegraphics{img/arithmetics/pca_projection_cloud_yz.jpg}
		\caption[]{Projection des points sur le plan vertical du repère centré}
	\end{figure}
	\begin{figure}[H]
		\centering
		\includegraphics{img/arithmetics/pca_projection_cloud_best.jpg}
		\caption[]{Projection des points sur le plan factoriel}
	\end{figure}
	Et la vue plane de chacune des projections:
	\begin{figure}[H]
		\centering
		\includegraphics[scale=0.7]{img/arithmetics/pca_plane_views.jpg}
		\caption[]{Vue plane de chacune des projections}
	\end{figure}
	Avant de déterminer le plan factoriel, nous allons maintenant chercher à détecter les liens possibles entre les variables.
	
	Nous rappelons (\SeeChapter{voir section Statistiques page \pageref{covariance}}) que la covariance entre deux variables $x_{.j}$ et $x_{.j'}$ est donnée par:
	
	et que le coefficient de corrélation linéaire (\SeeChapter{voir section Statistiques page \pageref{linear correlation coefficient}}) est:
	
	Nous noterons par la suite:
	
	les matrices des variances-covariances et de corrélations carrées (toutes deux étant pour rappel des matrices carrées et symétriques) avec $j=1\ldots p,j'=1\ldots p$.
	
	Nous voyons relativement facilement que la matrice des variances-covariances est au coefficient $1/n$ près, la matrice des produits scalaires canoniques des vecteurs de la matrice des données centrées $X_c$ (en d'autres termes, chaque composante de la matrice des variances-covariances est égale au produit scalaire des variables centrées). Nous en déduisons la relation suivante:
	
	\begin{tcolorbox}[title=Remarque,colframe=black,arc=10pt]
	Pour une matrice $A$ donnée, l'expression $A^TA$ joue un rôle important dans les statistiques. Par exemple, c'est une expression importante de la solution analytique de la méthode des moindres carrés ordinaires. Ou, pour l'A.C.P., ses vecteurs propres sont les  composantes principales des données. Géométriquement, la matrice $A^TA$ est nommée "matrice des produits scalaires" (car elle contient tous les produits scalaires de toutes les colonnes dans $A$). Algébriquement, elle s'appelle "\NewTerm{matrice de la somme des carrés et des produits croisés}\index{matrice de la somme des carrés et des produits croisés}" (MSCPC). En divisant la matrice MSCPC par $n$, la taille de l'échantillon ou le nombre de lignes de $A$, nous obtenons ce que l'on appelle communément la matrice MCPC (moyenne-carré-et-produit-croisé). Si nous centrons les colonnes de $A$, $A^TA/(n-1)$ est la matrice de covariance. Si nous $Z$-standardisons les colonnes de $A$, $A^TA/(n-1)$ est alors la matrice de corrélation de Pearson.
	\end{tcolorbox}
	La matrice des variances-covariances (puisque comme nous l'avons vu dans le chapitre de Statistiques, la diagonale contient les variances... pour rappel!) est un outil connu d'interprétation sur ce site. Par contre, ce qui est nouveau et va nous être très utile pour déterminer le plan factoriel est la matrice des corrélations linéaires qui peut aussi être écrite sous la forme suivante:
	
	Ce qui donne pour notre exemple où nous avons trois variables (très facile à calculer avec un tableur comme Microsoft Excel), la matrice carrée suivante (que les données soient centrées ou non les composantes de la matrice sont identiques):
	
	Pour continuer, toujours dans le but de déterminer le plan factoriel, définissons le concept d'inertie d'un nuage de points.
	
	\textbf{Définition (\#\mydef):} Nous appelons "\NewTerm{inertie d'un nuage de points}\index{inertie d'un nuage de points}" la quantité:
	
	où $G$ est le centre de gravité du nuage de points et $M_i$ un point de $\mathbb{R}^p$ de coordonnées $x_i^T$.
	\begin{tcolorbox}[title=Remarque,colframe=black,arc=10pt]
	Le carré de la distance est pris par anticipation des développements qui vont suivre.
	\end{tcolorbox}
	Ensuite, démontrons que nous avons la relation suivante:
	
	\begin{dem}
	
	\begin{flushright}
		$\blacksquare$ Q.E.D.
	\end{flushright}
	\end{dem}
	Nous allons dans toute la suite travailler avec les données centrées normées, in extenso avec la matrice $Z$. Les points $M_i$ auront donc ici comme coordonnées $z_i^T$.
	
	Le problème est maintenant de trouver le meilleur espace affine de dimension $p$ dans le sens où il respecte au mieux les distances entre les points. Pour cela, nous allons rechercher la meilleure droite vectorielle $\Delta_{\vec{u}}$ qui est parfaitement déterminée par le vecteur $\vec{u}$. Appelons $H_i$ la projection orthogonale de $M_i$ sur la droite $\Delta_{u}$. Alors notre problème est de trouver la droite (in extenso le vecteur $\vec{u}$) qui fasse que la somme des carrés des distances entre les points $H_i$ soit maximale. Nous écrirons le problème sous la forme d'un problème de programmation quadratique (\SeeChapter{voir section Méthodes Numériques page \pageref{nonlinear optimization}}):
	
	Or ici, nous avons:
	
	En effet, le centre de gravité du nuage de points projetés est aussi l'origine. Par suite, notre problème peut s'écrire:
	
	Lui-même équivalant donc à:
	
	Résolvons donc ce problème:
	Tout d'abord, puisque $H_i$ est la projection orthogonale du point $M_i$ sur $\Delta_{\vec{u}}$ nous avons $\overrightarrow{OH}_i=\alpha_i \vec{u}$ pour tout $i$ avec $\alpha_i =\overrightarrow{OM}_i\circ \vec{u}$. Par suite les coordonnées des points $H_i$ sur la droite $\Delta_{\vec{u}}$ sont:
	
	Par suite, nous avons:
	
	Ici nous cherchons le vecteur unitaire $\vec{u}$. La matrice $Z$ nous est parfaitement connue. Or, nous avons:
	
	La matrice de corrélation $R$ est symétrique donc, selon le théorème spectral vu dans la section d'Algèbre Linéaire \pageref{spectral theorem}, elle est diagonalisable dans une base orthonormée de vecteurs propres. Ainsi, nous avions démontré dans le théorème spectral que:
	
	est diagonale (libre à nous d'en choisir le contenu) si $R$ est symétrique et $S$ orthogonale (qui est donc dans notre exemple une matrice carrée $3\times 3$!). Alors nous en déduisons la relation suivante qu'il est d'usage d'appeler la "\NewTerm{décomposition spectrale}\index{décomposition spectrale}" de $R$:
	
	et comme $S$ avait été démontrée comme étant orthogonale (et qu'il existe une famille de vecteurs propres pour cela!), nous avons \SeeChapter{voir section Algèbre Linéaire page \pageref{orthogonal matrix}}):
	
	Donc:
	
	où nous choisissons pour $\Lambda$  la matrice diagonale des valeurs propres rangées en ordre décroissant: $\lambda_1 \geq \lambda_2 \geq \ldots \geq \lambda_p$.
	
	Nous avons donc:
	
	Dans la littérature, cette somme est souvent notée de la manière suivante (très souvent mentionée dans les logiciels statistiques sous le nom de "\NewTerm{décomposition spectrale}"):
	
	Mais $S$ étant orthogonale, nous avons par conséquent:
	
	et ceci provient du fait que la matrice orthogonale est comme nous l'avions démontré dans la section d'Algèbre Linéaire une isométrie (elle conserve donc la norme!).
	
	Comme les valeurs propres sont dans l'ordre décroissant, nous écrirons:
	
	Or le terme entre parenthèses est strictement inférieur ou égal à $1$ de par l'implication précédente. Donc:
	
	Soit:
	
	Or rappelons que notre objectif est de maximiser cette inégalité. En d'autres termes de chercher $w_1$ tel que l'égalité soit respectée. Nous voyons assez vite qu'il en sera ainsi si $w_1=1$  et que les autres termes soient nuls. Ainsi, une solution triviale de notre problème de maximisation est donc:
	
	soit puisque:
	
	qui est alors le premier vecteur propre de $R$ (puisque $R$ se diagonalise dans cette base) associé à la plus grande valeur propre $\lambda_1$. D'où le fait que cette solution soit souvent notée sous la forme:
	
	toujours avec $\Lambda=S^{-1}RS$ (il est donc relativement aisé de déterminer $S$ avec des logiciels lorsque $R$ et $\Lambda$ sont connus).
	
	Une fois que l'on a trouvé la première droite vectorielle, nous cherchons une deuxième droite dans le sous-espace vectoriel orthogonal à la droite vectorielle qui maximise l'inertie du nuage de points projetés. Nous démontrons, et devinons, que la solution est donnée par la droite vectorielle dirigée par le vecteur propre associé à la deuxième valeur propre de la matrice de corrélation et ainsi de suite...
	
	Ainsi, nous obtenons une nouvelle base $(\vec{u}_1,\ldots,\vec{u}_p)$ dont un des plans constitue le plan factoriel. Cependant, il nous faut connaître les composantes de $Z$ dans cette base. Comme cette base a été construite sous la condition que $R$ y est diagonalisable via la matrice $S$ alors cette dernière matrice est l'application linéaire qui va nous permettre d'exprimer $Z$ dans la base $(\vec{u}_1,\ldots,\vec{u}_p)$ via la relation:
	
	Ainsi, dans notre exemple les trois valeurs propres de la matrice de corrélation $R$ sont (\SeeChapter{voir section Algèbre Linéaire page \pageref{eigenvector}}):
	
	et donc:
	
	\begin{tcolorbox}[title=Remarque,colframe=black,arc=10pt]
	Certains logiciels indiquent les poids en $\%$ respectifs et cumulés pour chacune des valeurs propres. Ainsi, nous avons dans le cas présent les poids respectifs suivants en $\%$ du total:
	
	Donc la première composante explique $66.67\%$ de l'effet. Les deux premières composantes en expliquent $96.15\%$, etc. C'est ainsi par exemple en finance que l'on va prendre parmi une dizaine ou plus de composantes, seulement celles qui mènent à "expliquer" le $95\%$.
	\end{tcolorbox}
	En ayant les trois valeurs propres, pour déterminer les trois vecteurs propres $(\vec{u},\vec{u}_2,\vec{u}_3)$ qui forment la base principale, il nous faut donc résoudre le système de trois équations à trois inconnues (\SeeChapter{voir section Algèbre Linéaire page \pageref{linear systems}}) suivant pour chaque valeur propre:
	
	Ce qui donne donc (nous nous passerons de ce calcul élémentaire qui peut être fait à la main ou avec un simple tableur) la matrice des vecteurs propres:
	
	qui vérifie donc:
	
	ou autrement écrit (suite à la remarque d'un lecteur qui a voulu vérifier les calculs et qui s'est fait piéger):
	
	Nous avons alors comme coordonnées des points $M_i$ dans la base $(\vec{u}_1,\vec{u}_2,\vec{u}_3)$ en utilisant:
	
	la matrice suivante:
		
	Les coordonnées des projections du nuage de points dans le meilleur plan défini par les vecteurs $(\vec{u}_1,\vec{u}_2)$ sont donc les deux premières colonnes de la matrice précédente (correspondant donc à la longueur du sépale et la largeur du sépale).

	Effectivement, nous voyons immédiatement que ce sont ces deux colonnes qui maximiseront la somme des normes dans le plan donné:
	\begin{figure}[H]
		\centering
		\includegraphics{img/arithmetics/pca_factorial_plane.jpg}
		\caption[]{Plan factoriel déjà montré plus haut...}
	\end{figure}
	Un logiciel comme Minitab 15.1 (référence dans l'industrie de la gestion de la qualité) donne les informations suivantes pour les valeurs propres (info pas très utile sous forme graphique... à mon avis):
	\begin{figure}[H]
		\centering
		\includegraphics[scale=0.8]{img/arithmetics/pca_eigenvalues.jpg}
		\caption[]{Valeurs propres pour l'ACP données par Minitab 15.1 ("scree plot")}
	\end{figure}
	et le plan factoriel suivant (resterait à savoir comment les valeurs sont calculées car elles ne sont pas identiques à celles que nous avons obtenues ici... mais la forme graphique est bien juste et c'est le principal!):
	\begin{figure}[H]
		\centering
		\includegraphics[scale=0.8]{img/arithmetics/pca_factorial_plane_minitab.jpg}
		\caption[]{Plan factoriel donné par Minitab 15.1}
	\end{figure}
	Si le lecteur a suivi attentivement toutes les étapes que nous avons faites jusqu'à présent, il a peut-être remarqué que nous avons fait:
	\begin{enumerate}
		\item Centrage des données à zéro (normalisation par la moyenne)
		\item Reduction des données
		\item Décorrélation des données en:
		\begin{enumerate}
			\item Calculant la matrice des covariances
			\item Calculant les vecteurs propres de la matrice des covariances
			\item Appliquant la matrice des vecteurs propres aux données (cela appliquera la rotation)
		\end{enumerate}
	\end{enumerate}
	De plus, certains logiciels et praticiens mettent à l'échelle la matrice non corrélée afin d'obtenir une matrice de covariance correspondant à la matrice identité (des $1$ sur la diagonale et des $0$ sur les autres cellules). Pour ce faire, nous mettons à l'échelle les données décorrélées en divisant chaque dimension par la racine carrée de sa valeur propre correspondante.
	
	Toutes les $7$ étapes énumérées ci-dessus sont nommées parfois "\NewTerm{blanchiment de l'ACP}\index{blanchiment de l'ACP}" ou simplement "\NewTerm{blanchiment}" (parfois aussi "\NewTerm{sphérisation des données}\index{sphérisation des données}"). Donc "blanchir" les données signifie que nous voulons les transformer de manière à avoir une matrice des covariances qui est la matrice identité ($1$ dans la diagonale et $0$ pour les autres cellules) et cette procédure est nommé "blanchiment" en référence au "bruit blanc".
	
	Pour clore ce sujet, signalons que de nombreux logiciels utilisent le fait que les vecteurs $z_{.j}$ sont unitaires pour faire le produit scalaire qui correspond alors dans ce cas particulier simplement au cosinus entre les différents vecteurs tel que:
	
	et comme nous avons démontré plus haut que:
	
	Il vient alors:
	
	et comme dans notre exemple, nous avons $3$ vecteurs $z_{.j}$, il y a donc $3$ produits scalaires possibles si nous omettons les produits scalaires des vecteurs avec eux-mêmes. Donc la matrice:
	
	contient aussi les angles entre les vecteurs $z_{.j}$.
	
	Enfin, signalons que l'A.C.P. étant sensible aux données aberrantes, il vaut mieux parfois transformer les valeurs du tableau d'origine en leurs rangs respectifs (voir notre étude des statistiques de rangs plus haut!) et appliquer exactement le même algorithme. Nous parlons alors d'une \NewTerm{A.C.P. de rangs}\index{A.C.P. de rangs}".
	
	\begin{tcolorbox}[title=Remarque,colframe=black,arc=10pt]
	Un logiciel convivial comme XL-Stats propose automatiquement dans le menu ACP cinq "types différents" d'Analyse en Composantes Principales qui sont respectivement: Pearson ($n$), Pearson ($n-1$), Spearman, Kendall, Covariance. Évidemment, la même chose peut être faite avec d'autres logiciels de statistiques, mais en réalité, ce n'est pas des choix visuellement simples disponibles (nous devons écrire un peu de code).\\
	
	Il existe également un groupe de méthodes ACP robustes qui ont été développées et qui sont mieux adaptées pour traiter des données de grande dimension dans une situation où la taille de l'échantillon est inférieure à la dimension. Ces méthodes sont: ACP robuste par projection-poursuite (PP-ACP), ACP sphérique (ACPS), ACP robuste (ACPRBO), ACP clairsemée robuste (ACPROC).
	\end{tcolorbox}	
	
	\paragraph{DVS et ACP}\mbox{}\\\\
	Nous allons maintenant prouver un théorème important qui n'est pas tout à fait évident.
	\begin{theorem}
	L'ACP ci-dessus (utilisant le théorème spectral, c'est-à-dire la décomposition en valeurs propres) est un cas particulier de la DVS - décomposition en valeurs singulières (cette dernière est donc moins restrictive)!!!
	\end{theorem}
	\begin{dem}
	Nous avons prouvé juste plus haut que:
	
	En utilisant les propriétés des matrices transposées (\SeeChapter{voir section Algèbre Linéaire page \pageref{transposed matrix}}):
	
	Et que:
	
	Et nous avons construit dans la section d'Algèbre Linéaire la décomposition de valeurs singulières (\SeeChapter{voir section Algèbre Linéaire page \pageref{spectral theorem}}):
	
	C'est-à-dire dans notre cas ici cela s'écrira:
	
	Dès lors:
	
	et comme $V$ est une matrice orthogonale ($V^TV=\mathds{1}$):
	
	Donc si nous comparons:
	
	la correspondance est assez facile à voir!
	 
	Cela signifie donc que pour que le théorème spectral de $R$ soit égal à sa DVS, nous avons besoin que $R$ ait:
	 \begin{itemize}
	 	\item Des vecteurs propres orthonormés (ce qui signifie que tout ce qu'il fait est de mettre à l'échelle son entrée le long de $n$ directions orthogonales, où $R$ est $n\times n$)
	 	
	 	\item Des valeurs propres positives (familièrement, cela doit être une matrice réelle et ne pas "retourner" quoi que ce soit)
	 \end{itemize}
	 Il s'avère que les conditions ci-dessus sont équivalentes aux suivantes:
	\begin{itemize}
	 	\item $R$ doit être une matrice réelle symétrique (cela équivaut à $ R $ ayant des valeurs propres réelles et des vecteurs propres orthonormés)
	 	
	 	\item Les valeurs propres (réelles) de $R$ doivent être positives
	 \end{itemize}
	 Donc, nous disons: les deux sont égaux quand $R$ est symétrique semi-positive définie (\SeeChapter{voir section Algèbre Linéaire page \pageref{positive semidefinite matrix}})! Cela peut être désigné techniquement par $R \succeq 0 \ \land \ R^\dagger = R$.
	 
	En fait, l'utilisation de la DVS pour effectuer l'ACP est beaucoup plus logique numériquement que la formation de la matrice de covariance pour commencer, car la construction de $ZZ^T$ peut entraîner une perte de précision. En effet dans certains cas leÁCP peut diverger très rapidement et donner des résultats inexacts! C'est pourquoi de nombreux logiciels statistiques offrent la possibilité d'exécuter une ACP basée sur la DVS (théorème de décomposition en valeurs singulières) ou sur le théorème spectral.
	\begin{flushright}
		$\blacksquare$  Q.E.D.
	\end{flushright}
	\end{dem}
	Une question courante sur les forums Internet est: \textit{Intuitivement, quelle est la différence entre le théorème spectral (décomposition en valeurs propres) et la décomposition en valeurs singulières?}. Également, si: \textit{L'un d'eux est-il plus général que l'autre? Est-ce que l'un ou l'autre est un cas particulier de l'autre?}
	
	Premièrement, d'après ce qui a été écrit précédemment, il devrait être tout à fait évident que la décomposition en valeurs propres résulte de la question de savoir dans quelles directions une \underline{forme quadratique} a le plus grand impact, lorsque la DVS par construction émerge de la demande dans quelles directions une \underline{transformation linéaire}\footnote{D'où le fait que l'ACP produise des transformations linéaires, capturant ainsi les relations linéaires entre les variables d'origine afin que nous puissions perdre des informations prédictives non linéaires!} a le plus grand impact!
	
	Maintenant, le lecteur doit garder à l'esprit que le théorème spectral décrit l'effet d'une matrice $A$ sur un vecteur comme le processus suivant en $3$-étapes $A =Q\Lambda Q^{-1}$:
	\begin{enumerate}
		 \item Une transformation linéaire inversible ($Q^{-1}$, donc $Q$ doit être inversible, c'est-à-dire carré!)
		 
		 \item Une mise à l'échelle ($\Lambda$ avec $\Lambda=\text{diag}(\vec{\lambda})$)
		 
		 \item L'inverse de la transformation initiale ($Q$)
	\end{enumerate}
	Cette décomposition n'existe pas toujours, mais le théorème spectral décrit les conditions dans lesquelles une telle décomposition existe.
	
	Quand pour la DVS, nous avons le processus suivant en $3$ étapes $A=U\Sigma V^{T}$:
	\begin{enumerate}
		 \item Une première rotation dans l'espace d'entrée ($V$ est une matrice de rotation étant donné que $V^TV=\mathds{1}$)
		 
		 \item Une simple mise à l'échelle positive qui prend un vecteur dans l'espace d'entrée vers l'espace de sortie ($\Sigma$ diagonale avec entrées positives!)
		 
		 \item Et une autre rotation dans l'espace de sortie ($U$ est une matrice de rotation étant donné que $U^TU=\mathds{1}$)
	\end{enumerate}
	Le théorème fondamental de l'algèbre linéaire dit qu'une telle décomposition existe toujours (surtout parce qu'aucun inverse n'est impliqué et ne s'applique donc pas uniquement aux matrices carrées).
	
	\pagebreak
	\subsubsection{Analyse Factorielle des Correspondances du Khi-deux (AFCK2)}\label{chi-square correspondence factor analysis}
	L'analyse factorielle des correspondances du khi-deux, en abrégé AFCK2, est une méthode d'analyse largement utilisée en biostatistique et en analyse d'enquêtes. La technique AFCK2 est principalement utilisée pour les grands tableaux pour comparer toutes les données (si possible toutes exprimées dans la même unité, comme une devise, une dimension, une fréquence ou toute autre grandeur mesurable). Elle peut notamment permettre l'étude de tableaux de contingence (ou tableaux croisés de cooccurrence) et décrire le lien entre deux variables. Elle sert à identifier et prioriser toutes les dépendances entre les lignes et les colonnes de la table.
	
	Si plus de deux variables doivent être prises en considération, nous parlons alors d'analyse des correspondances multiples du khi-deux (ACMK2).
	
	Abordons maintenant directement la théorie avec un exemple. Pour cela, nous considérons le tableau suivant (à deux variables) des superficies des types d'arbres qui se trouvent en Picardie (France) en 1984 en hectares:
	\begin{table}[H]
		\centering
		\definecolor{gris}{gray}{0.85}
		\begin{tabular}{|c|c|c|c|c|}
			\hline
			\multicolumn{1}{c}{\cellcolor{black!30}\textbf{}} & 
  \multicolumn{1}{c}{\cellcolor{black!30}\textbf{Feuillus}}  & 
  \multicolumn{1}{c}{\cellcolor{black!30}\textbf{Résineux}} & 
  \multicolumn{1}{c}{\cellcolor{black!30}\textbf{Mixtes}} & 
  \multicolumn{1}{c}{\cellcolor{black!30}\textbf{Total par dép.}}\\ \hline
		{\cellcolor{black!30}\textbf{L'Aisne (A)}} & $106,500$ & $3,380$ & $1,470$ & $111,350$ \\ \hline
		{\cellcolor{black!30}\textbf{L'Oise (O)}} & $101,700$ & $310,000$ & $0$ & $111,700$ \\ \hline
{\cellcolor{black!30}\textbf{La Somme (S)}} & $45,200$ & $4,350$ & $50$ & $49,600$ \\ \hline
{\cellcolor{black!30}\textbf{Total}} & $253,400$ & $17,730$ & $1,520$ & $272,650$ \\ \hline
		\end{tabular}
		\caption{Tableau de contingences (tableau croisé) de l'A.F.C.}
	\end{table}
	Les spécialistes du domaine appellent parfois les totaux des lignes et des colonnes respectivement les  "\NewTerm{marges de ligne}\index{marges de ligne}" et "\NewTerm{marges de colonne}\index{marges de colonne}". Lorsque l'ensemble du tableau est mis sous forme de pourcentage, par rapport au total des totaux, on parle de "\NewTerm{représentation en fréquences conjointes}\index{représentation en fréquences conjointes}":
	\begin{table}[H]
		\begin{center}
			\definecolor{gris}{gray}{0.85}
				\begin{tabular}{|c|c|c|c|c|}
					\hline
					\multicolumn{1}{c}{\cellcolor{black!30}\textbf{}} & 
	  \multicolumn{1}{c}{\cellcolor{black!30}\textbf{Feuillus}}  & 
	  \multicolumn{1}{c}{\cellcolor{black!30}\textbf{Résineux}} & 
	  \multicolumn{1}{c}{\cellcolor{black!30}\textbf{Mixtes}} & 
	  \multicolumn{1}{c}{\cellcolor{black!30}\textbf{Total par dép.}}\\ \hline
				{\cellcolor{black!30}\textbf{L'Aisne (A)}} & $39.06\%$ & $1.24\%$ & $0.54\%$ & $40.84\%$ \\ \hline
{\cellcolor{black!30}\textbf{L'Oise (O)}} & $37.30\%$ & $3.67\%$ & $0\%$ & $40.97\%$ \\ \hline
{\cellcolor{black!30}\textbf{La Somme (S)}} & $16.58\%$ & $1.60\%$ & $0.02\%$ & $18.19\%$ \\ \hline
{\cellcolor{black!30}\textbf{Total}} & $92.93\%$ & $6.50\%$ & $0.56\%$ & $100\%$ \\ \hline
			\end{tabular}
			\caption{Tableau des fréquences conjointes de l'A.F.C}
		\end{center}
	\end{table}
	Nous souhaitons analyser s'il existe les degrés de ressemblance et de différence entre les variables. Remarquons, que nous ne cherchons pas à comparer l'égalité des moyennes ou des variances donc les outils statistiques vus dans le chapitre du même nom ne sont pas adaptés à ce genre d'analyse.
	
	Si nous choisissons la distance euclidienne (\SeeChapter{voir section Calcul Vectoriel page \pageref{euclidean distance vector}}):
	
	sur les données brutes pour mesurer ces différences entre départements, nous obtenons les écarts suivants:
	
	et ainsi de suite pour les autres régions. Nous obtenons alors:
	
	Nous voyons en regardant le tableau et avant tout calcul que les départements de l'Aisne et l'Oise se ressemblent alors que le département de la Somme se diffère nettement. Les distances obtenues mettent en évidence cette observation.

	Mais dans le tableau ci-dessus les profils de l'Oise et de la Somme, avec une forêt mixte très faible, sont pourtant très proches en proportion.

	Dans ce contexte, nous voyons que la distance euclidienne transcrit les différences de masse entre les départements. En d'autres termes, l'Aisne et l'Oise se ressemblent car leurs superficies sont proches. Pour éliminer l'artefact lié aux ordres de grandeur, il nous faut transformer les données en pourcentage (pourcentages des régions). Nous obtenons alors:
	\begin{table}[H]
		\begin{center}
			\definecolor{gris}{gray}{0.85}
				\begin{tabular}{|c|c|c|c|c|}
					\hline
					\multicolumn{1}{c}{\cellcolor{black!30}\textbf{}} & 
	  \multicolumn{1}{c}{\cellcolor{black!30}\textbf{Feuillus}}  & 
	  \multicolumn{1}{c}{\cellcolor{black!30}\textbf{Résineux}} & 
	  \multicolumn{1}{c}{\cellcolor{black!30}\textbf{Mixtes}} & 
	  \multicolumn{1}{c}{\cellcolor{black!30}\textbf{\% Région}}\\ \hline
				{\cellcolor{black!30}\textbf{L'Aisne (A)}} & $95.6\%$ & $3.0\%$ & $1.3\%$ & $40.8\%$ \\ \hline
{\cellcolor{black!30}\textbf{L'Oise (O)}} & $91.0\%$ & $9.0\%$ & $0.0\%$ & $41.0\%$ \\ \hline
{\cellcolor{black!30}\textbf{La Somme (S)}} & $91.1\%$ & $8.8\%$ & $0.1\%$ & $18.2\%$ \\ \hline
			\end{tabular}
			\caption{Transformation du tableau de contingences en pourcents}
		\end{center}
	\end{table}
	où les spécialistes du domaine appellent parfois la colonne en pourcents des régions "\NewTerm{mprofil marginal des lignes}\index{profil marginal des lignes}" ou "\NewTerm{masse}" (et respectivement quand ils indiquent la ligne des pourcents des arbres).

	Si nous choisissons la distance euclidienne sur les proportions (données relatives), nous obtenons:
	
	soit:
	
	Cette fois, l'Oise et la Somme apparaissent bien comme se ressemblant le plus avec leurs forêts. Nous voyons que travailler avec les données relatives semble donc plus pertinent dans ce cas!
	
	Maintenant, nous allons emprunter une idée aux économistes qui, lorsqu'ils ont des tableaux du même genre que le précédent, calculent ce qu'ils appellent "\NewTerm{index}\index{index (économie)}" ou "\NewTerm{élasticité}" (souvent appelé "\NewTerm{index de spécificité}\index{index de spécificité} en statistiques et qui est donné par le rapport entre la fréquence conjointe et la fréquence marginale:
	
	Voici un exemple obtenu avec les tableaux croisés dynamiques de Microsoft Excel 11.8346 qui inclut la fonction Index. D'abord le tableau de départ:
	\begin{figure}[H]
		\centering
		\includegraphics{img/arithmetics/index_starting_pivottable.jpg}
		\caption[]{Tableau croisé dynamique Microsoft Excel 11.8346 de départ}
	\end{figure}
	et en activant la fonction Index:
	\begin{figure}[H]
		\centering
		\includegraphics{img/arithmetics/index_final_pivottable.jpg}
		\caption[]{Tableau croisé dynamique Microsoft Excel 11.8346 avec la fonction Index}
	\end{figure}
	Pour voir d'où viennent ces valeurs, regardons par exemple l'article \textit{Desk} dans la région \textit{Alberta}. Il a un rendement (fréquence conjointe) de:
	
	par rapport à toutes les régions ce qui est au-dessus de la valeur de $33.33\%$ qu'aurait comme rendement cet article toutes régions confondues s'il n'y avait pas de préférences de région!

	La région \textit{Alberta} a elle un rendement (fréquence marginale) de:
	
	par rapport à toutes les régions ce qui est en-dessous des $33.33\%$  de rendement qu'elle aurait s'il n'y avait pas de préférences de région. Ainsi, ce tableau d'index permet de savoir si les différences sont qualitativement significatives!!
	
	Le rapport donne donc:
	
	ce qui montre un fort décalage entre la valeur obtenue et la valeur que nous aurions si les proportions étaient respectées (surreprésentation de $283\%$).
	
	C'est donc une sorte de calcul de conformité: si le rapport valait $1$, c'est que le rendement régional des ventes de cet article particulier serait conforme par rapport à toutes les ventes de cette région relativement à un marché national. Il n'y aurait alors pas d'anomalies. Voyons cela par exemple pour nos arbres où nous avions les effectifs observés:
	\begin{table}[H]
		\begin{center}
			\definecolor{gris}{gray}{0.85}
				\begin{tabular}{|c|c|c|c|c|}
					\hline
					\multicolumn{1}{c}{\cellcolor{black!30}\textbf{}} & 
	  \multicolumn{1}{c}{\cellcolor{black!30}\textbf{Feuillus}}  & 
	  \multicolumn{1}{c}{\cellcolor{black!30}\textbf{Résineux}} & 
	  \multicolumn{1}{c}{\cellcolor{black!30}\textbf{Mixtes}} & 
	  \multicolumn{1}{c}{\cellcolor{black!30}\textbf{Total par dép.}}\\ \hline
				{\cellcolor{black!30}\textbf{L'Aisne (A)}} & $106,500$ & $3,380$ & $1,470$ & $111,350$ \\ \hline
				{\cellcolor{black!30}\textbf{L'Oise (O)}} & $101,700$ & $10,000$ & $0$ & $111,700$ \\ \hline
{\cellcolor{black!30}\textbf{La Somme (S)}} & $45,200$ & $4,350$ & $50$ & $49,600$ \\ \hline
{\cellcolor{black!30}\textbf{Total}} & $253,400$ & $17,730$ & $1,520$ & $272,650$ \\ \hline
			\end{tabular}
			\caption[]{Tableau de contingences (tableau croisé) de l'A.F.C.}
		\end{center}
	\end{table}
	et pour lequel nous obtenons le tableau croisé dynamique des index effectifs observés suivant dans Microsoft Excel 11.8346 :
	\begin{figure}[H]
		\centering
		\includegraphics{img/arithmetics/index_trees_forests.jpg}
		\caption[]{Tableau croisé dynamique Microsoft Excel 11.8346 avec l'analyse d'Index}
	\end{figure}
	et nous voyons encore clairement à l'aide de ce tableau que ce sont l'Oise et la Somme qui se ressemblent le plus!

	Avant de continuer, nous pourrions nous poser la question extrêmement importante suivante: Quels seraient les effectifs théoriques qui auraient été obtenus si les proportions des arbres dans les régions étaient rigoureusement équivalentes à la proportion d'ensemble (soit de telle manière à ce que les index soient tous unitaires)?

	Eh bien simplement en faisant le calcul suivant (il s'agit simplement d'une règle de trois calculée dans chaque cellule) dont il faut bien - si possible - comprendre le sens sans l'appliquer bêtement:
	\begin{table}[H]
		\begin{center}
			\definecolor{gris}{gray}{0.85}
				\begin{tabular}{|c|c|c|c|}
					\hline
					\multicolumn{1}{c}{\cellcolor{black!30}\textbf{}} & 
	  \multicolumn{1}{c}{\cellcolor{black!30}\textbf{Feuillus}}  & 
	  \multicolumn{1}{c}{\cellcolor{black!30}\textbf{Résineux}} & 
	  \multicolumn{1}{c}{\cellcolor{black!30}\textbf{Mixtes}}\\ \hline
				{\cellcolor{black!30}\textbf{L'Aisne (A)}} & \scriptsize\parbox{4cm}{$=(253,400/272,650)*111'350$\\$=103,488$} & \scriptsize\parbox{4cm}{$=(17,730/272,650)*111,350$\\$=7,241$} & \scriptsize\parbox{4cm}{$=(1,520/272,650)*111,350$\\$=621$} \\ \hline
				{\cellcolor{black!30}\textbf{L'Oise (O)}} & \scriptsize\parbox{4cm}{$=(253,400/272,650)*111'700$\\$=103,813$} & \scriptsize\parbox{4cm}{$=(17,730/272,650)*111'700$\\$=7,264$} & \scriptsize\parbox{4cm}{$=(1,520/272,650)*111'700$\\$=623$} \\ \hline
{\cellcolor{black!30}\textbf{La Somme (S)}} & \scriptsize\parbox{4cm}{$=(253,400/272,650)*49'600$\\$=46,098$} & \scriptsize\parbox{4cm}{$=(17,730/272,650)*49'600$\\$=3,225$} & \scriptsize\parbox{4cm}{$=(17,730/272,650)*49'600$\\$=276$} \\ \hline
{\cellcolor{black!30}\textbf{Total}} & \scriptsize $253,400$ & \scriptsize $17,730$ & \scriptsize $1,520$ \\ \hline
			\end{tabular}
			\caption[]{Respect des proportions de l'A.F.C.}
		\end{center}
	\end{table}
	Et nous obtenons avec ces nouvelles valeurs le tableau des index des effectifs théoriques suivant dans Microsoft Excel 11.8346:
	\begin{figure}[H]
		\centering
		\includegraphics{img/arithmetics/index_trees_forests_theoretical.jpg}
		\caption[]{Tableau croisé dynamique Microsoft Excel  11.8346 de l'index des effectifs théoriques}
	\end{figure}
	ce qui montre que les proportions sont maintenant respectées! Parenthèse fermée (mais sur laquelle nous reviendrons un peu plus loin)!
	
	Eh bien quand nous voulons faire de l'analyse factorielle des correspondances, notre relation:
	
	devient alors:
	
	soit:
	
	Cette fois encore, l'Oise et la Somme apparaissent bien comme se ressemblant le plus.
	
	La distance ci-dessus se nomme la "\NewTerm{métrique du Khi-deux}\index{métrique du Khi-deux}" car elle ressemble (mais c'est tout!) à la distance utilisée dans le test d'ajustement du même nom (\SeeChapter{voir page \pageref{chi-square test of independence}}) mais ici, elle permet seulement de mettre en place une hiérarchie dans le cadre d'un tableau de contingences et d'observer les variables similaires de manière plus aisée!!
	
	\begin{tcolorbox}[title=Remarque,colframe=black,arc=10pt]
	Il existe une autre manière de calculer une AFC en se basant sur une distance euclidienne mais en ayant pris soin au préalable de transformer les données du tableau de contingences de manière particulière et ce pour que le calcul soit identique qu'en lorsqu'on utilise la métrique du Khi-deux.
	\end{tcolorbox}	
	
	\subsubsection{Test d'indépendance du khi-deux}\label{chi-square test of independence}
	Le test du $\chi^2$ est appliqué lorsque vous avez deux variables catégorielles d'une seule population. Il est utilisé pour déterminer s'il existe une association significative entre les deux variables.
	
	Par exemple, dans une enquête électorale, les électeurs peuvent être classés par sexe (homme ou femme) et préférence de vote (démocrate, républicain ou indépendant). Nous pourrions utiliser un test du chi carré d'indépendance pour déterminer si le sexe est lié à la préférence de vote.
	
	Supposons qu'une variable catégorielle $A$ a $r$ niveaux et qu'une autre variable catégorielle $B$ a $c$ des niveaux. L'hypothèse nulle $H_0$ assume que connaître le niveau de la variable $A$ ne nous aide pas à prédire le niveau de la variable $B$. Autrement dit, les variables sont indépendantes.
	
	Nous présenterons ce test avec un exemple d'accompagnement car nous savons par expérience qu'il est plus efficace pour l'apprentissage et la compréhension.

	Rappelons que lors de l'introduction de la méthode précédente pour comparer des effectifs (valeurs) et détecter lesquels étaient les plus proches, nous avons donné le tableau des effectifs observés:
	\begin{table}[H]
		\begin{center}
			\definecolor{gris}{gray}{0.85}
				\begin{tabular}{|c|c|c|c|c|}
					\hline
					\multicolumn{1}{c}{\cellcolor{black!30}\textbf{}} & 
	  \multicolumn{1}{c}{\cellcolor{black!30}\textbf{Feuillus}}  & 
	  \multicolumn{1}{c}{\cellcolor{black!30}\textbf{Résineux}} & 
	  \multicolumn{1}{c}{\cellcolor{black!30}\textbf{Mixtes}} & 
	  \multicolumn{1}{c}{\cellcolor{black!30}\textbf{Total par dép.}}\\ \hline
				{\cellcolor{black!30}\textbf{L'Aisne (A)}} & $106,500$ & $3,380$ & $1,470$ & $111,350$ \\ \hline
				{\cellcolor{black!30}\textbf{L'Oise (O)}} & $101,700$ & $3,380$ & $1,470$ & $111,700$ \\ \hline
{\cellcolor{black!30}\textbf{La Somme (S)}} & $45,200$ & $4,350$ & $50$ & $49,600$ \\ \hline
{\cellcolor{black!30}\textbf{Total}} & $253,400$ & $17,730$ & $1,520$ & $272,650$ \\ \hline
			\end{tabular}
			\caption{Tableau de contingences pour l'analyse du $\chi^2$}
		\end{center}
	\end{table}
	et nous avons montré comment trouver le tableau des effectifs théoriques (arrondis à l'entier le plus proche) dans les cas où les proportions auraient dû éventuellement être respectées:
	\begin{table}[H]
		\begin{center}
			\definecolor{gris}{gray}{0.85}
				\begin{tabular}{|c|c|c|c|c|}
					\hline
					\multicolumn{1}{c}{\cellcolor{black!30}\textbf{}} & 
	  \multicolumn{1}{c}{\cellcolor{black!30}\textbf{Feuillus}}  & 
	  \multicolumn{1}{c}{\cellcolor{black!30}\textbf{Résineux}} & 
	  \multicolumn{1}{c}{\cellcolor{black!30}\textbf{Mixtes}} & 
	  \multicolumn{1}{c}{\cellcolor{black!30}\textbf{Total par dép.}}\\ \hline
				{\cellcolor{black!30}\textbf{L'Aisne (A)}} & $103,488$ & $7,241$ & $621$ & $111,350$ \\ \hline
				{\cellcolor{black!30}\textbf{L'Oise (O)}} & $103,813$ & $7,264$ & $623$ & $111,700$ \\ \hline
{\cellcolor{black!30}\textbf{La Somme (S)}} & $46,098$ & $3,225$ & $276$ & $49,600$ \\ \hline
{\cellcolor{black!30}\textbf{Total}} & $253,400$ & $17,730$ & $1,520$ & $272,650$ \\ \hline
			\end{tabular}
			\caption{Tableau de contingences avec effectifs théoriques pour l'analyse du $\chi^2$}
		\end{center}
	\end{table}
	La construction du dernier tableau ci-dessus présuppose par exemple que les trois régions sont dans des conditions identiques pour tout ce qui concerne la croissance et la multiplication des arbres et que le nombre d'arbres est en relation de cause à effet directe!!!! avec les régions et qu'il n'y a pas d'autres causes intermédiaires... ce qui est une hypothèse forte!
	
	Mais sous cette hypothèse, supposons que nous souhaiterions savoir si les différences observées entre le nombre d'arbres et les régions sont statistiquement significatives ou purement aléatoires à cause de l'échantillon expérimental? En d'autres termes, nous voulons savoir si le nombre d'arbres dépend réellement des régions dans lesquelles ils poussent ou si ces valeurs ne sont que dues au hasard de l'échantillon? Raison pour laquelle ce test s'appelle le "\NewTerm{test d'indépendance du Khi-deux}\index{test d'indépendance du Khi-deux}".

	Gardez donc à l'esprit que dans le test d'indépendance du chi carré, tous les sujets / unités sont collectés au hasard dans une population et deux variables catégorielles sont observées pour chaque unité.
	
	\begin{tcolorbox}[title=Remarque,colframe=black,arc=10pt]
	Le test d'indépendance du chi carré est recommandé en analyse sensorielle par la norme ISO 8588-1987 sous le nom de "test A-Non A".
	\end{tcolorbox}
	
	Pour répondre à cette question il faut d'abord une référence. Et cette référence est justement l'hypothèse de lien causal direct (proportions respectées) que nous avons donnée juste précédemment.

	Si nous considérons que chaque case du tableau des effectifs observés correspond à l'issue d'une variable aléatoire de loi inconnue et que chaque case du tableau théorique (du moins la classe d'effectifs) est considérée comme issue d'une variable aléatoire suivant une loi binomiale (et asymptotique d'une loi Normale) alors nous pouvons utiliser le test d'ajustement du Khi-deux:
	
	pour avoir une bonne idée (mais qui reste quand même très approximative au vu des hypothèses!) si les différences entre les valeurs des effectifs observés sont dues au hasard ou sont réelles. Or, si $D$ est petit, la probabilité que ce soit dû au hasard est grande mais si $D$ est grand alors nous avons une différence réelle (donc nous utilisons le test d'ajustement du Khi-deux mais dans le sens inverse!).
	
	Reste à déterminer le nombre de degrés de liberté de la loi du $\chi^2$ que suit cette somme dans ce type de configuration!
	
	Dans le cas particulier (mais facilement généralisable par récurrence) d'une table à deux entrées avec deux variables catégorisées $X$ avec $l$ niveaux et $Y$ avec $c$ niveaux nous aurons respectivement $l$ lignes et $c$ colonnes.
	
	Ainsi, la table aura bien évidemment $l\cdot c$ cellules. Dans la table des effectifs théoriques (dont chaque cellule est considérée comme une variable aléatoire) chaque cellule sera entièrement déterminée par la somme des autres de telle sorte que le nombre de degrés de liberté sera alors en toute logique comme nous l'avons vu lors de notre étude des degrés de liberté:
	
	Ainsi, en prenant notre exemple des forêts, c'est le total des totaux de $272,650$ qui nous permet d'écrire cette dernière relation et ainsi de déterminer la valeur d'une cellule éventuellement vide, toutes les autres étant données!

	Un test du Khi-deux sur ce type de table teste l'hypothèse d'indépendance contre l'hypothèse alternative de dépendance. Sous l'hypothèse d'indépendance nous estimons qu'il y a besoin de seulement:
	
	valeurs sur les $N$ pour pouvoir en déterminer la totalité (en supposant implicitement connues les sommes par ligne et par colonne).
	
	Ainsi, si vous avez une table de $2$ lignes par $2$ colonnes, il vous suffit si vous connaissez les totaux des lignes et des colonnes, d'avoir $2$ valeurs (soit $(2-1)+(2-1)$) pour déterminer les 2 manquantes. Le raisonnement s'applique aussi pour une table de $3$ lignes par $3$ colonnes où il vous suffit d'avoir au moins $4$ valeurs (soit $(3-1)+(3-1)$) pour déterminer les $5$ manquantes.
	
	Les degrés de liberté pour le Khi-deux sont alors:
	
	C'est cette relation qui nous dit (trivialement!) que si dans un tableau de $2$ lignes par $2$ colonnes comprenant donc $4$ cellules (totaux des lignes et colonnes étant aussi connus!) qu'étant donnée une seule des valeurs (ddl valant $1$), nous pouvons déterminer les $3$ autres valeurs manquantes.
	
	Voici donc une définition possible du nombre de degrés de liberté: C'est le nombre maximum de valeurs du modèle telles qu'aucune d'entre elles n'est calculable à partir des autres.
	
	De même, pour un tableau de $3$ lignes par $3$ colonnes comprenant $9$ cellules comme c'est le cas de notre exemple dans ce chapitre avec les forêts, la connaissance de $4$ cellules seules permet grâce aux totaux en ligne et colonnes de déterminer les $5$ autres qui seraient éventuellement non connues.
	
	D'où la relation dans le cadre de l'application du Khi-deux de la relation finale:
	
	en faisant usage des notations utilisées dans l'industrie. Le terme:
	
	est souvent appelé "\NewTerm{carré du résidu standardisé}\index{carré du résidu standardisé}". Et le rapport:
	
	est souvent appelé "\NewTerm{contribution au Khi-deux d'indépendance}\index{contribution au Khi-deux d'indépendance}".
	\begin{tcolorbox}[title=Remarque,colframe=black,arc=10pt]
	Pour utiliser ce test à bon escient, il faudrait donc vérifier d'abord que les différences (numérateur) suivent une loi Normale ou que l'ensemble des termes de la somme forment une loi du Khi-deux ou approximativement (asymptotiquement) une loi Normale centrée réduite et les effectifs de chaque cellule doit être supérieur à $5$ sinon des simulations de Monte Carlo doivent être utilisées pour déterminer la $p$-valeur.
	\end{tcolorbox}	
	Dans notre exemple nous avons:
	
	et la $p$-valeur de cette valeur avec la loi du Khi-deux à quatre degrés de liberté:
	
	est tellement proche de zéro (non statistiquement significatif) que nous n'avons aucune chance de nous tromper en affirmant que les différences observées dans le tableau sont statistiquement significatives entre les $3$ forêts et donc qu'il y a très probablement indépendance.
	
	\begin{tcolorbox}[title=Remarque,colframe=black,arc=10pt]
	Rappelons qu'il existe de nombreux tests statistiques dont le nom contient "khi-carré". Par exemple:
	\begin{itemize}
		\item Test de qualité de l'ajustement du khi-carré (nous permet de déterminer si une distribution de population spécifiée est valide)
		
		\item Test du khi-carré d'association / indépendance (nous permet de déterminer si la distribution d'une variable a été influencée par une autre variable)
		
		\item Test d'homogénéité du khi-carré (nous permet de comparer deux ou plusieurs proportions de population)
		
		\item Test du khi-carré pour les valeurs aberrantes (nous permet de détecter s'il existe au moins une valeur aberrante en comparaison d'une distribution spécifique)
		
		\item Test du khi-carré pour la différence de deux données de comptage (cas particulier du test d'homogénéité avec une seule catégorie!)
	\end{itemize}
	Même si tous ces tests ont des objectifs différents, leur cadre mathématique et leur procédure sont EXACTEMENT les mêmes et donc l'un de ces cinq tests peut être exécuté pour obtenir en même temps les cinq conclusions avec les mêmes $p$-valeurs et les mêmes valeurs critiques du khi-carré!!!
	\end{tcolorbox}	
	
	\pagebreak
	\paragraph{V de Cramér}\mbox{}\\\\
	Le "\NewTerm{V de Cramér}\index{V de Cramér}" (parfois appelé "\NewTerm{phi de Cramér}\index{phi de Cramér}" et noté $\varphi_c$) est une mesure d'association entre deux variables nominales, donnant une valeur comprise entre $0$ et $+1$ (inclus). Il est basé sur la statistique du khi-carré de Pearson et a été publié par Harald Cramér en 1946.
	
	Comme nous allons le prouver, $\varphi_c$ est l'intercorrélation de deux variables discrètes et peut être utilisé avec des variables ayant deux niveaux ou plus. $\varphi_c$ est une mesure symétrique, peu importe quelle variable nous plaçons dans les colonnes et laquelle dans les lignes. De plus, l'ordre des lignes / colonnes n'a pas d'importance, donc $\varphi_c$ peut être utilisé avec des types de données nominales, ordonnées, numériques ou autres...
	
	Nous avons vu plus haut que le test d'indépendance du khi-deux peut être utilisé pour mesurer le degré d'association de deux variables catégorielles dans une table de contingences de $l$ lignes et $c$ colonnes:
	
	et que cette distance suit une loi du khi-deux à $(l-1) (c-1)$ degrés de liberté. Nous allons démontrer de façon intuitive que la valeur maximum de la distance $D$ est donnée par:
	
	et que cette valeur maximale n'est atteinte que si et seulement si chaque ligne ou chaque colonne contient une et une seule valeur non nulle. Sous cette dernière condition, nous pouvons toujours réarranger le tableau de contingences de façon à avoir tous les termes non nuls sur la diagonale du tableau.
	
	Évidemment, si le tableau n'est pas carré comme ci-dessous:
	\begin{table}[H]
		\centering
		\definecolor{gris}{gray}{0.85}
		\begin{tabular}{|l|c|c|c|c|c|}
		\hline 
		{\cellcolor{black!30}\textbf{Ligne/Colonne}} & {\cellcolor{black!30}$\boldsymbol{c_1}$} & {\cellcolor{black!30}$\boldsymbol{c_2}$} & {\cellcolor{black!30}$\boldsymbol{c_3}$}  & {\cellcolor{black!30}$\boldsymbol{c_4}$}  & {\cellcolor{black!30}\textbf{Total}}\\ 
		\hline 
		{\cellcolor{black!30}$\boldsymbol{l_1}$} & $a_1$ & $0$ & $0$ & $0$ &  $a_1$ \\  \hline 
		{\cellcolor{black!30}$\boldsymbol{l_2}$} & $0$ & $a_2$ & $0$ & $0$ &  $a_2$ \\  \hline 
		{\cellcolor{black!30}$\boldsymbol{l_3}$} & $0$ & $0$ & $a_3$ & $0$ &  $a_3$ \\  \hline 
		{\cellcolor{black!30}$\boldsymbol{l_4}$} & $0$ & $0$ & $0$ & $a_4$ &  $a_4$ \\  \hline 
		{\cellcolor{black!30}$\boldsymbol{l_5}$} & $0$ & $0$ & $0$ & $0$ & $0$ \\ \hline  
		{\cellcolor{black!30}\textbf{Total}} & $a_1$ & $a_2$ & $a_3$ & $a_4$ & $N$ \\ 	\hline 
		\end{tabular} 
		\caption[]{Tableau de contingences rectangulaire}
	\end{table}
	Le cas qui maximise $D$ impose donc des termes en diagonale sur la dimension la plus petite en ligne ou colonne notée traditionnellement $q$. Dans le cas présent, nous avons donc:
	
	Dans ce cas extrême et bien évidemment théorique, les lignes ou colonnes qui n'ont que des zéros peuvent être mises de côté et dès lors tout tableau peut se résumer à:
	\begin{table}[H]
		\centering
		\definecolor{gris}{gray}{0.85}
		\begin{tabular}{|l|c|c|c|c|c|}
		\hline 
		{\cellcolor{black!30}\textbf{Ligne/Colonne}} & {\cellcolor{black!30}$\boldsymbol{c_1}$} & {\cellcolor{black!30}$\boldsymbol{c_2}$} & {\cellcolor{black!30}$\boldsymbol{c_3}$}  & {\cellcolor{black!30}$\boldsymbol{c_4}$}  & {\cellcolor{black!30}\textbf{Total}}\\ 
		\hline 
		{\cellcolor{black!30}$\boldsymbol{l_1}$} & $a_1$ & $0$ & $0$ & $0$ &  $a_1$ \\  \hline 
		{\cellcolor{black!30}$\boldsymbol{l_2}$} & $0$ & $a_2$ & $0$ & $0$ &  $a_2$ \\  \hline 
		{\cellcolor{black!30}$\boldsymbol{l_3}$} & $0$ & $0$ & $a_3$ & $0$ &  $a_3$ \\  \hline 
		{\cellcolor{black!30}$\ldots$} & $\ldots$ & $\ldots$ & $\ldots$ & $\ldots$ & $\ldots$ \\ \hline  
		{\cellcolor{black!30}$\boldsymbol{l_q}$} & $0$ & $0$ & $0$ & $a_q$ &  $a_q$ \\  \hline 
		{\cellcolor{black!30}\textbf{Total}} & $a_1$ & $a_2$ & $\ldots$ & $a_q$ & $N$ \\ 	\hline 
		\end{tabular} 
		\caption[]{Tableau de contingences rectangulaire transformé en carré}
	\end{table}
	Évidemment, faire abstraction des lignes ou colonnes qui n'ont que des valeurs nulles suppose que dans la distance $D$ du Khi-deux nous posions que:
	
	ce qui est tout à fait discutable... Pour la suite, nous allons avoir besoin des relations suivantes:
	
	et:
	
	Dès lors, il vient::
	
	Nous définissons alors la valeur suivante:
	
	comme étant le "\NewTerm{coefficient V de Cramér}\index{coefficient V de Cramér}" (la majorité des logiciels donnent cependant la valeur de V au carré). Ce dernier est tel qu'il ne dépasse jamais $1$ et permet une interprétation plus intuitive du degré d'association dans un tableau de contingences.
	
	Dans le cas où le tableau est de dimension $2$ en ligne et $2$ en colonne, la relation précédente se réduit alors immédiatement à:
	
	Relation qu'il est de tradition de noter dans ce cas particulier sous la forme suivante:
	
	et de nommer "\NewTerm{phi de Cramér}\index{phi de Cramér}" ou "\NewTerm{coefficient phi}\index{coefficient phi}".
	
	\begin{tcolorbox}[colframe=black,colback=white,sharp corners]
	\textbf{{\Large \ding{45}}Exemple:}\\\\
	Considérons le tableau suivant (même si les conditions ne sont pas satisfaites pour un test d'indépendance du Khi-deux):
	\begin{table}[H]
		\centering
		\definecolor{gris}{gray}{0.85}
		\begin{tabular}{|l|c|c|c|}
		\hline 
		{\cellcolor{black!30}\textbf{Projets}} & {\cellcolor{black!30}\parbox{3.5cm}{\textbf{Certifiés} \\\textbf{Responsables de projets}}} &  {\cellcolor{black!30}\parbox{3.5cm}{\textbf{Non-Certifiés} \\\textbf{Responsables de projets}}} & {\cellcolor{black!30}\textbf{Total}}\\ 
		\hline 
		{\cellcolor{black!30}\textbf{Délais respectés}} & $8$ & $1$ & $9$ \\  \hline 
		{\cellcolor{black!30}\textbf{Délais non-respectés}} & $4$ & $5$ & $9$  \\  \hline 
		{\cellcolor{black!30}\textbf{Total}} & $12$ & $6$ & $18$  \\ 	\hline 
		\end{tabular}
		\caption[]{Exemple de tableau de contingences pour le calcul du V de Cramér}
	\end{table}
	avec les effectifs théoriques:
	\begin{table}[H]
		\begin{center}
		\definecolor{gris}{gray}{0.85}
		\begin{tabular}{|l|c|c|c|}
		\hline 
		{\cellcolor{black!30}\textbf{Projets}} & {\cellcolor{black!30}\parbox{3.5cm}{\textbf{Certifiés} \\\textbf{Responsables de projets}}} &  {\cellcolor{black!30}\parbox{3.5cm}{\textbf{Non-Certifiés} \\\textbf{Responsables de projets}}} & {\cellcolor{black!30}\textbf{Total}}\\ 
		\hline 
		{\cellcolor{black!30}\textbf{Délais respectés}} & $(12/18)\cdot 9=6$ & $(6/18)\cdot 9=3$ & $9$ \\  \hline 
		{\cellcolor{black!30}\textbf{Délais non-respectés}} & $(12/18)\cdot 9=6$ & $(6/18)\cdot 9=3$ & $9$  \\  \hline 
		{\cellcolor{black!30}\textbf{Total}} & $12$ & $6$ & $18$  \\ 	\hline 
		\end{tabular} 
		\end{center}
		\caption[]{Tableau de contingences de l'exemple avec effectifs théoriques}
	\end{table}
	Nous avons alors:
	
	Et à un niveau de $95\%$, nous obtenons avec la version française de Microsoft Excel 14.0.6123:
	
	ou avec la $p$-valeur:
	
	Il vient alors immédiatement:
	
	Nous sommes un peu limite ici... étant donné que la $p$-valeur est toute proche du seuil de $0.05$. Cependant, prendre une décision dans le cas présent alors que de toute façon les effectifs sont si faibles reviendrait à conclure n'importe quoi et ce d'autant plus que l'outil est construit sur une cumulation d'approximations.
	\end{tcolorbox}
	
	\subsubsection{Coefficient phi de Pearson}\label{Pearson's phi coefficient}
	En statistiques, le "\NewTerm{coefficient phi de Pearson}\index{coefficient phi de Pearson}" est une autre mesure d'association pour deux variables binaires. Introduite par Karl Pearson, cette mesure est similaire au coefficient de corrélation de Pearson dans son interprétation. En réalité, un coefficient de corrélation de Pearson estimé pour deux variables binaires (variables de Bernoulli) donne mathématiquement le coefficient phi de Pearson comme nous allons le démontrer.
	
	Pour introduire ce coefficient et démontrer qu'il n'est qu'un cas particulier du coefficient de corrélation classique de Pearson, considérons la table suivante:
		
	\begin{table}[H]
		\centering
		\begin{tabular}{|l|l|c|c|c|}
		\hline
		\multicolumn{2}{|c|}{\cellcolor{black!30}} & \multicolumn{3}{c|}{\cellcolor{black!30}\textbf{$Y$}}                                             \\ \hline
		\cellcolor{black!30}{} & \cellcolor{black!30}{} & \cellcolor{black!30}\textbf{$Y=1$} & \cellcolor{black!30}\textbf{$Y=0$} & \cellcolor{black!30}\textbf{Total}\\ \hline
		\cellcolor{black!30}  & \multicolumn{1}{l|}{\multirow{1}{*}{\cellcolor{black!30}\textbf{$X=1$}}} & $n_{11}$ & $n_{12}$ & $n_{1.}$ \\ \cline{2-5} 
		\parbox[t]{2mm}{{\rotatebox[origin=c]{90}{\cellcolor{black!30}\textbf{$X$}}}} & \multirow{1}{*}{\cellcolor{black!30}\textbf{$X=0$}}                       & $n_{21}$ & $n_{22}$ & $n_{2.}$ \\ \cline{2-5}  
		 \cellcolor{black!30} & \multirow{1}{*}{\cellcolor{black!30}\textbf{Total}} {\cellcolor{black!30}} & $n_{.1}$ & $n_{.2}$ & $n$ \\ \cline{2-5}  \hline
		\end{tabular}
		\caption[]{Exemple dichotomique d'une table de contingences}
	\end{table}
	ou écrit autrement:
	\begin{table}[H]
		\centering
		\begin{tabular}{|l|l|c|c|c|}
		\hline
		\multicolumn{2}{|c|}{\cellcolor{black!30}} & \multicolumn{3}{c|}{\cellcolor{black!30}\textbf{$Y$}}                                             \\ \hline
		\cellcolor{black!30}{} & \cellcolor{black!30}{} & \cellcolor{black!30}\textbf{$Y=1$} & \cellcolor{black!30}\textbf{$Y=0$} & \cellcolor{black!30}\textbf{Total}\\ \hline
		\cellcolor{black!30}  & \multicolumn{1}{l|}{\multirow{1}{*}{\cellcolor{black!30}\textbf{$X=1$}}} & $a$ & $b$ & $a+b$ \\ \cline{2-5} 
		\parbox[t]{2mm}{{\rotatebox[origin=c]{90}{\cellcolor{black!30}\textbf{$X$}}}} & \multirow{1}{*}{\cellcolor{black!30}\textbf{$X=0$}} & $c$ & $d$ & $c+d$ \\ \cline{2-5}  
		 \cellcolor{black!30} & \multirow{1}{*}{\cellcolor{black!30}\textbf{Total}} {\cellcolor{black!30}} & $a+c$ & $b+d$ & $1$ \\ \cline{2-5}  \hline
		\end{tabular}
	\end{table}
	Maintenant rappelons que le coefficient de corrélation de Pearson est donné par:
	
	Maintenant laissons $X$ et $Y$ être des variables de Bernoulli (voir page \pageref{bernoulli distribution}). Nous ne supposons pas une distribution identique des deux variables ni indépendance, mais nous supposons que les $4$ probabilités suivant sont non-nulles.

	Soit:
	
	Nous avons alors dans un premier temps:
	
	Nous avons aussi:
	
	et:
	
	et:
	
	Enfin, par substitution dans l'équation pour $\rho_{XY}$:
	
	Faisant les bonnes substitutions, cela nous conduit à:
	
	Si nous remplaçons $n$ par  $n_{11}+n_{12}+n_{22}+n_{21}$ et $n_{1.}$ par $n_{11}+n_{12}$ et $n_{.1}$ par$n_{11}+n_{21}$ alors nous obtenons après quelques simplifications très simples:
	
	L'interprétation du coefficient phi est similaire au coefficient de corrélation de Pearson. La plage va de $ -1 $ à $ + 1 $, où:
	\begin{itemize}
		\item $0$ signifie aucune corrélation
		
		\item $+1$ est une corrélation positive parfaite: la majorité des données sont des les cellules diagonales
		
		\item $-1$ est une corrélation négative parfaite: la majorité des données sont en-dehors des cellules diagonales
	\end{itemize}
	Le Département de Sciences Politiques de l’Université Quinnipiac a publié cette liste utile de la signification des coefficients de corrélation de Pearson. Les mêmes règles empiriques peuvent être utilisées pour le coefficient $\phi$. Notez que ce sont des estimations brutes pour interpréter les forces des relations:
	\begin{table}[H]
		\begin{center}
			\definecolor{gris}{gray}{0.85}
			\begin{tabular}{|c|c|}
			\hline
			\multicolumn{1}{c}{\cellcolor{black!30}\textbf{Intervalle}}  & 
	  \multicolumn{1}{c}{\cellcolor{black!30}\textbf{Interprétation}} \\ \hline
			$0.7\leq \varphi \leq 1$ & Very strong positive relationship\\ \hline		
			$0.4\leq \varphi < 0.70$ & Strong positive relationship\\ \hline
			$0.3\leq \varphi < 0.4$ & Moderate positive relationship\\ \hline
			$0.2\leq \varphi < 0.3$ & Weak positive relationship\\ \hline
			$0.0< \varphi < 0.2$ & No or negligible positive relationship\\ \hline
			$\varphi=0$ & No relationship\\ \hline
			$0.0> \varphi > -0.2$ & No or negligible negative relationship\\ \hline
			$-0.2> \varphi > -0.3$ & Weak negative relationship\\ \hline
			$-0.3> \varphi > -0.4$ & Moderate negative relationship\\ \hline
			$-0.4> \varphi > -0.70$ & Strong negative relationship\\ \hline
			$-0.7> \varphi \geq 1$ & Very strong negative relationship\\ \hline
			\end{tabular}
			\caption{Interprétations coefficient $\varphi$ de Pearson}
		\end{center}
	\end{table}
	Bien que le coefficient de corrélation de Pearson se ramène au coefficient phi de Pearson dans le cas $ 2 \times 2 $, ils ne sont généralement pas les mêmes. Le coefficient de corrélation de Pearson va de $ -1 $ à $ + 1 $, où $ \ pm 1 $ indique un accord ou un désaccord parfait, et $ 0 $ indique aucune relation. Le coefficient phi a une valeur maximale qui est déterminée par la distribution des deux variables si une ou les deux variables peuvent prendre plus de deux valeurs.
	
	\begin{tcolorbox}[colframe=black,colback=white,sharp corners]
	\textbf{{\Large \ding{45}}Exemple:}\\\\
	Considérez le tableau suivant (même si les conditions ne sont pas remplies pour un test d'indépendance du chi carré):
	\begin{table}[H]
		\centering
		\definecolor{gris}{gray}{0.85}
		\begin{tabular}{|l|c|c|c|}
		\hline 
		{\cellcolor{black!30}\textbf{Projets}} & {\cellcolor{black!30}\parbox{3.5cm}{\textbf{Certifiés} \\\textbf{Responsables de projets}}} &  {\cellcolor{black!30}\parbox{3.5cm}{\textbf{Non-Certifiés} \\\textbf{Responsables de projets}}} & {\cellcolor{black!30}\textbf{Total}}\\ 
		\hline 
		{\cellcolor{black!30}\textbf{Échéances non respectées}} & $8$ & $1$ & $9$ \\  \hline 
		{\cellcolor{black!30}\textbf{Échéances respectées}} & $4$ & $5$ & $9$  \\  \hline 
		{\cellcolor{black!30}\textbf{Total}} & $12$ & $6$ & $18$  \\ 	\hline 
		\end{tabular} 
		\caption[]{Table de contingence carrée}
	\end{table}
	Nous avons alors:
	
	\end{tcolorbox}
	On peut prouver que le coefficient phi de Pearson est égal au coefficient phi de Cramèr vu juste avant (c'est d'ailleurs pour ça que la notation est la même!) et qui est donné pour rappel:
	
	Pour simplifier la notation, posons:
	
	On a alors pour un tableau de contingence de taille $2\times 2$:
	
	Avec de manière identique que pour notre étude du coefficient V de Cramèr:
	
	Ce qui implique:
	
	Mais nous avons
	
	et:
	
	et:
	
	et:
	
	Dès lors:
	
	Notez que:
	
	Ce dernier résultat nous amène à:
	
	Dès lors:
	
	Le coefficient phi de Pearson est la source d'inspiration du "\NewTerm{coefficient de corrélation de Matthews}\index{coefficient de corrélation de Matthews}" (M.C.C.) qui est utilisé en apprentissage automatique (machine learning) comme mesure de la qualité des classifications binaires (à deux classes). Seul le vocabulaire diffère. En effet, il est défini comme:
	
	Même s'il n'y a pas de manière parfaite de décrire la matrice de confusion $\mathcal{C}$ des vrais et faux positifs et négatifs par un seul nombre (nous verrons dans le chapitre des Méthodes Numériques ce que sont les "matrices de confusion"!), Le coefficient de corrélation Matthews est généralement considéré comme l'une des meilleures mesures de ce type en ce début du 21e siècle.
	
	Dans cette relation, TP est le nombre de vrais positifs, TN le nombre de vrais négatifs, FP le nombre de faux positifs et FN le nombre de faux négatifs. Si l'une des quatre sommes du dénominateur est nulle, le dénominateur peut être arbitrairement fixé à un; il en résulte un coefficient de corrélation Matthews de zéro, qui peut être démontré comme étant la valeur limite correcte.
	
	\subsubsection{Test exact de Fisher}\label{exact Fisher test}
	Lorsque les effectifs dans la table de contingences sont trop petits ou que les valeurs sont vraiment trop irrégulières, l'utilisation du Khi-deux (test de Pearson) n'est plus possible car les conditions d'application ne sont plus valables. Nous allons voir que le test exact de Fisher peut être formalisé analytiquement dans des tableaux de contingences de $2$ lignes par $2$ colonnes (la majorité des logiciels de statistiques ne gèrent que ce scénario particulier pour le test exact de Fisher) sinon quoi il faut recourir à des simulations de Monte-Carlo.
	
	Le principe du "\NewTerm{test exact de Fisher}\index{test exact de Fisher}" (utilisable aussi bien en unilatéral qu'en bilatéral même si ce dernier est largement plus courant dans la pratique), basé donc sur la fréquence de croisement, est de déterminer si la configuration observée dans le tableau de contingence est une situation extrême par rapport aux situations possibles. Comme nous allons le démontrer, ce test a pour propriété particulière que n'importe quelle case du tableau peut servir de référence pour le test car les distributions sous-jacentes (lois marginales) de probabilité sont équivalentes.
	
	Pour étudier ce test, comme souvent dans ce chapitre, passons directement par un exemple.
	
	 Considérons le tableau de contingence suivant (qui maintenant nous est connu...):
	\begin{table}[H]
		\centering
		\definecolor{gris}{gray}{0.85}
		\begin{tabular}{|l|c|c|c|}
		\hline 
		{\cellcolor{black!30}\textbf{Projets}} & {\cellcolor{black!30}\parbox{3.5cm}{\textbf{Certifiés} \\\textbf{Responsables de projets}}} &  {\cellcolor{black!30}\parbox{3.5cm}{\textbf{Non-Certifiés} \\\textbf{Responsables de projets}}} & {\cellcolor{black!30}\textbf{Total}}\\  
		\hline 
		{\cellcolor{black!30}\textbf{Échéances respectées}} & $8$ & $1$ & $9$ \\  \hline 
		{\cellcolor{black!30}\textbf{Échéances non-respectées}} & $4$ & $5$ & $9$  \\  \hline 
		{\cellcolor{black!30}\textbf{Total}} & $12$ & $6$ & $18$  \\ 	\hline 
		\end{tabular} 
		\caption[]{Tableau de contingences de départ pour l'étude du test exact de Fisher}
	\end{table}
	qui est donc en réalité pas adapté pour un test d'indépendance du Khi-deux puisque le contenu des cellules est inférieur à $10$ unités et le nombre de degrés de liberté serait lui égal à l'unité.
	
	Ce même tableau en pourcentages donnera (même si c'est inutile pour le test que nous étudions il arrive souvent que les logiciels de statistiques communiquent ces valeurs):
	\begin{table}[H]
		\centering
		\definecolor{gris}{gray}{0.85}
		\begin{tabular}{|l|c|c|c|}
		\hline 
		{\cellcolor{black!30}\textbf{Projets}} & {\cellcolor{black!30}\parbox{3.5cm}{\textbf{Certifiés} \\\textbf{Responsables de projets}}} &  {\cellcolor{black!30}\parbox{3.5cm}{\textbf{Non-Certifiés} \\\textbf{Responsables de projets}}} & {\cellcolor{black!30}\textbf{Total}}\\  
		\hline 
		{\cellcolor{black!30}\textbf{Échéances respectées}} & $88.88\%$ & $11.12\%$ & $50\%$ \\  \hline 
		{\cellcolor{black!30}\textbf{Échéances non-respectées}} & $44.44\%$ & $55.56\%$ & $50\%$  \\  \hline 
		{\cellcolor{black!30}\textbf{Total}} & $66.66\%$ & $33.34\%$ & $100\%$  \\ 	\hline 
		\end{tabular} 
	\end{table}
	Les effectifs théoriques sont donnés par (même si c'est aussi inutile pour le test que nous étudions il arrive souvent que les logiciels de statistiques communiquent ces valeurs):
	\begin{table}[H]
		\begin{center}
		\definecolor{gris}{gray}{0.85}
		\begin{tabular}{|l|c|c|c|}
		\hline 
		{\cellcolor{black!30}\textbf{Projets}} & {\cellcolor{black!30}\parbox{3.5cm}{\textbf{Certifiés} \\\textbf{Responsables de projets}}} &  {\cellcolor{black!30}\parbox{3.5cm}{\textbf{Non-Certifiés} \\\textbf{Responsables de projets}}} & {\cellcolor{black!30}\textbf{Total}}\\  
		\hline 
		{\cellcolor{black!30}\textbf{Échéances respectées}} & $(12/18)\cdot 9=6$ & $(6/18)\cdot 9=3$ & $9$ \\  \hline 
		{\cellcolor{black!30}\textbf{Échéances non-respectées}} & $(12/18)\cdot 9=6$ & $(6/18)\cdot 9=3$ & $9$  \\  \hline 
		{\cellcolor{black!30}\textbf{Total}} & $12$ & $6$ & $18$  \\ 	\hline 
		\end{tabular} 
		\end{center}
	\end{table}
	La question que nous allons commencer par nous poser est la suivante: connaissant les totaux pour chaque ligne et chaque colonne, quelle est la probabilité d'avoir les valeurs présentes actuellement dans chacune des cases?!
	
	Cette question peut être formalisée si nous changeons le tableau sous la forme générique suivante:
	\begin{table}[H]
		\begin{center}
		\definecolor{gris}{gray}{0.85}
		\begin{tabular}{|l|c|c|c|}
		\hline 
		{\cellcolor{black!30}\textbf{Projets}} & {\cellcolor{black!30}\parbox{3.5cm}{\textbf{Certifiés} \\\textbf{Responsables de projets}}} &  {\cellcolor{black!30}\parbox{3.5cm}{\textbf{Non-Certifiés} \\\textbf{Responsables de projets}}} & {\cellcolor{black!30}\textbf{Total}}\\  
		\hline 
		{\cellcolor{black!30}\textbf{Échéances respectées}} & $a=k$ & $b$ & $a+b=m$ \\  \hline 
		{\cellcolor{black!30}\textbf{Échéances non-respectées}} & $c$ & $d$ & $c+d$  \\  \hline 
		{\cellcolor{black!30}\textbf{Total}} & $a+c=p$ & $b+d$ & $a+b+c+d=n$  \\ 	\hline 
		\end{tabular} 
		\end{center}
	\end{table}
	Explicitement et relativement à notre exemple en adoptant la notation d'usage de la loi hypergéométrique, la question est de savoir quelle était la probabilité d'avoir $8$ ($a = k$) projets parmi les $18$ ($n$) dont les délais ont été respectés par des chefs de projets certifiés sachant qu'il y a $9$ projets ($m$) au total dont les délais ont été respectés et $12$ projets ($p$) au total gérés par des chefs de projets certifiés.
	
	Nous avons alors vu plus haut que dans ce cas il s'agit d'un tirage exhaustif, il faut donc utiliser la loi hypergéométrique donnée par:
	
	Soit avec la version française de Microsoft Excel 11.8346:
	\begin{center}
		\texttt{=LOI.HYPERGEOMETRIQUE(k,p,m,n,FAUX)}\\
		\texttt{=LOI.HYPERGEOMETRIQUE(8,12,9,18,FAUX)=0.06108597}
	\end{center}
	où pour rappel, $k$ est le nombre de succès dans l'échantillon, $p$ est la taille de l'échantillon, $m$ le nombre de succès dans la population et $n$ la taille de la population.

	Au fait les probabilités sont toutes égales quelle que soit la cellule choisie du tableau de contingences!! Cela peut se vérifier numériquement pour les sceptiques avec à nouveau un tableur comme la version française de Microsoft Excel 14.0.6123 en créant la structure suivante:
	\begin{figure}[H]
		\centering
		\includegraphics[scale=0.85]{img/arithmetics/exact_fisher_text_excel.jpg}
		\caption[]{Test exact de Fisher pour la symétrie avec Microsoft Excel 14.0.6123}
	\end{figure}
	et donc à chaque fois que le lecteur appuiera sur la touche F9 de son clavier il pourra constater que toutes les probabilités sont toujours égales.
	
	Cela peut se vérifier aussi formellement en choisissant une cellule du tableau et en écrivant:
	
	et pour une autre cellule de la même colonne, nous aurons:
	
	et donc:
	
	et ainsi de suite...
	
	Bref, ceci étant dit, nous avons donc dans la case supérieure gauche la valeur $8$ alors que l'effectif théorique est de $6$. La première chose à laquelle nous pouvons répondre est de savoir si cette valeur de 8 est anormalement grande ou pas par rapport à l'effectif théorique. Pour cela, nous calculons par exemple en unilatéral la probabilité cumulée d'être inférieur ou égal à $8$. Nous avons alors avec la version française de Microsoft Excel 14.0.6123 et ultérieur (le dernier paramètre à $1$ - équivalent à \texttt{VRAI} - de la fonction permettant d'indiquer au logiciel que nous voulons la probabilité cumulée):
	\begin{center}
	\texttt{=LOI.HYPERGEOMETRIQUE.N(8,12,9,18,1)=0.995475113}
	\end{center}
	Il apparaît donc avec un seuil de $5\%$ en unilatéral que cette valeur est anormalement grande. Nous sommes donc dans une situation extrême.
	
	Par contre, même si les probabilités sont égales pour toutes les cases, la probabilité cumulée elle ne l'est pas! Ainsi, nous avons par exemple pour la valeur de la case inférieure gauche (afin de vérifier si elle est anormalement petite par rapport à l'effectif théorique de $6$):
	\begin{center}
	\texttt{=LOI.HYPERGEOMETRIQUE.N(4,12,9,18,1)=0.06561086}
	\end{center}
	Donc c'est une valeur qui n'est pas anormalement petite. Cependant, nous souhaiterions avoir un test permettant de conclure si l'ensemble du tableau est dans une configuration extrême ou pas. Or, en faisant le calcul case par case, nous n'allons pas arriver à grand-chose...
	
	L'idée consiste alors à construire tous les tableaux dont les fréquences marginales sont de $9;9$ et $12;6$ et de calculer la probabilité d'une case donnée (l'avantage de cette technique est que la conclusion sera la même quelle que soit la case prise pour référence du calcul):
	\begin{table}[H]
		\centering
		\definecolor{gris}{gray}{0.85}
		\begin{tabular}{|l|c|c|c|}
		\hline 
		{\cellcolor{black!30}\textbf{Projets}} & {\cellcolor{black!30}\parbox{3.5cm}{\textbf{Certifiés} \\\textbf{Responsables de projets}}} &  {\cellcolor{black!30}\parbox{3.5cm}{\textbf{Non-Certifiés} \\\textbf{Responsables de projets}}} & {\cellcolor{black!30}\textbf{Total}}\\ 
		\hline 
		{\cellcolor{black!30}\textbf{Échéances respectées}} & $9$ & $0$ & $9$ \\  \hline 
		{\cellcolor{black!30}\textbf{Échéances non-respectées}} & $3$ & $6$ & $9$  \\  \hline 
		{\cellcolor{black!30}\textbf{Total}} & $12$ & $6$ & $18$  \\ 	\hline 
		\end{tabular} 
	\end{table}
	\begin{center}
	\texttt{=LOI.HYPERGEOMETRIQUE(9,12,9,18)=0.004524887}
	\end{center}
	
	\begin{table}[H]
		\begin{center}
		\definecolor{gris}{gray}{0.85}
		\begin{tabular}{|l|c|c|c|}
		\hline 
		{\cellcolor{black!30}\textbf{Projets}} & {\cellcolor{black!30}\parbox{3.5cm}{\textbf{Certifiés} \\\textbf{Responsables de projets}}} &  {\cellcolor{black!30}\parbox{3.5cm}{\textbf{Non-Certifiés} \\\textbf{Responsables de projets}}} & {\cellcolor{black!30}\textbf{Total}}\\ 
		\hline 
		{\cellcolor{black!30}\textbf{Échéances respectées}} & $8$ & $1$ & $9$ \\  \hline 
		{\cellcolor{black!30}\textbf{Échéances non-respectées}} & $4$ & $5$ & $9$  \\  \hline 
		{\cellcolor{black!30}\textbf{Total}} & $12$ & $6$ & $18$  \\ 	\hline 
		\end{tabular} 
		\end{center}
	\end{table}
	\begin{center}
	\texttt{=LOI.HYPERGEOMETRIQUE(8,12,9,18)=0.06108597}
	\end{center}
	
	\begin{table}[H]
		\begin{center}
		\definecolor{gris}{gray}{0.85}
		\begin{tabular}{|l|c|c|c|}
		\hline 
		{\cellcolor{black!30}\textbf{Projets}} & {\cellcolor{black!30}\parbox{3.5cm}{\textbf{Certifiés} \\\textbf{Responsables de projets}}} &  {\cellcolor{black!30}\parbox{3.5cm}{\textbf{Non-Certifiés} \\\textbf{Responsables de projets}}} & {\cellcolor{black!30}\textbf{Total}}\\ 
		\hline 
		{\cellcolor{black!30}\textbf{Échéances respectées}} & $7$ & $2$ & $9$ \\  \hline 
		{\cellcolor{black!30}\textbf{Échéances non-respectées}} & $5$ & $4$ & $9$  \\  \hline 
		{\cellcolor{black!30}\textbf{Total}} & $12$ & $6$ & $18$  \\ 	\hline 
		\end{tabular} 
		\end{center}
	\end{table}
	\begin{center}
	\texttt{=LOI.HYPERGEOMETRIQUE(7,12,9,18)=0.244343891}
	\end{center}
	
	\begin{table}[H]
		\begin{center}
		\definecolor{gris}{gray}{0.85}
		\begin{tabular}{|l|c|c|c|}
		\hline 
		{\cellcolor{black!30}\textbf{Projets}} & {\cellcolor{black!30}\parbox{3.5cm}{\textbf{Certifiés} \\\textbf{Responsables de projets}}} &  {\cellcolor{black!30}\parbox{3.5cm}{\textbf{Non-Certifiés} \\\textbf{Responsables de projets}}} & {\cellcolor{black!30}\textbf{Total}}\\  
		\hline 
		{\cellcolor{black!30}\textbf{Échéances respectées}} & $6$ & $3$ & $9$ \\  \hline 
		{\cellcolor{black!30}\textbf{Échéances non-respectées}} & $6$ & $3$ & $9$  \\  \hline 
		{\cellcolor{black!30}\textbf{Total}} & $12$ & $6$ & $18$  \\ 	\hline 
		\end{tabular} 
		\end{center}
	\end{table}
	\begin{center}
	\texttt{=LOI.HYPERGEOMETRIQUE(6,12,9,18)=0.380090498}
	\end{center}
	
	\begin{table}[H]
		\begin{center}
		\definecolor{gris}{gray}{0.85}
		\begin{tabular}{|l|c|c|c|}
		\hline 
		{\cellcolor{black!30}\textbf{Projets}} & {\cellcolor{black!30}\parbox{3.5cm}{\textbf{Certifiés} \\\textbf{Responsables de projets}}} &  {\cellcolor{black!30}\parbox{3.5cm}{\textbf{Non-Certifiés} \\\textbf{Responsables de projets}}} & {\cellcolor{black!30}\textbf{Total}}\\ 
		\hline 
		{\cellcolor{black!30}\textbf{Échéances respectées}} & $5$ & $4$ & $9$ \\  \hline 
		{\cellcolor{black!30}\textbf{Échéances non-respectées}} & $7$ & $2$ & $9$  \\  \hline 
		{\cellcolor{black!30}\textbf{Total}} & $12$ & $6$ & $18$  \\ 	\hline 
		\end{tabular} 
		\end{center}
	\end{table}
	\begin{center}
	\texttt{=LOI.HYPERGEOMETRIQUE(5,12,9,18)=0.244343891}
	\end{center}
	
	\begin{table}[H]
		\begin{center}
		\definecolor{gris}{gray}{0.85}
		\begin{tabular}{|l|c|c|c|}
		\hline 
		{\cellcolor{black!30}\textbf{Projets}} & {\cellcolor{black!30}\parbox{3.5cm}{\textbf{Certifiés} \\\textbf{Responsables de projets}}} &  {\cellcolor{black!30}\parbox{3.5cm}{\textbf{Non-Certifiés} \\\textbf{Responsables de projets}}} & {\cellcolor{black!30}\textbf{Total}}\\  
		\hline 
		{\cellcolor{black!30}\textbf{Échéances respectées}} & $4$ & $5$ & $9$ \\  \hline 
		{\cellcolor{black!30}\textbf{Échéances non-respectées}} & $8$ & $1$ & $9$  \\  \hline 
		{\cellcolor{black!30}\textbf{Total}} & $12$ & $6$ & $18$  \\ 	\hline 
		\end{tabular} 
		\end{center}
	\end{table}
	\makebox[\textwidth]{\texttt{=LOI.HYPERGEOMETRIQUE(4,12,9,18)=0.061085973}}
	
	\begin{table}[H]
		\begin{center}
		\definecolor{gris}{gray}{0.85}
		\begin{tabular}{|l|c|c|c|}
		\hline 
		{\cellcolor{black!30}\textbf{Projets}} & {\cellcolor{black!30}\parbox{3.5cm}{\textbf{Certifiés} \\\textbf{Responsables de projets}}} &  {\cellcolor{black!30}\parbox{3.5cm}{\textbf{Non-Certifiés} \\\textbf{Responsables de projets}}} & {\cellcolor{black!30}\textbf{Total}}\\ 
		\hline 
		{\cellcolor{black!30}\textbf{Échéances respectées}} & $3$ & $6$ & $9$ \\  \hline 
		{\cellcolor{black!30}\textbf{Échéances non-respectées}} & $9$ & $0$ & $9$  \\  \hline 
		{\cellcolor{black!30}\textbf{Total}} & $12$ & $6$ & $18$  \\ 	\hline 
		\end{tabular} 
		\end{center}
	\end{table}
	\makebox[\textwidth]{\texttt{=LOI.HYPERGEOMETRIQUE(3,12,9,18)=0.004524887}}
	
	Soit pour résumer avec les valeurs de $k$ (correspondantes à la cellule supérieure gauche):
	\begin{table}[H]
		\begin{center}
			\definecolor{gris}{gray}{0.85}
			\begin{tabular}{|c|c|c|}
			\hline
			\multicolumn{1}{c}{\cellcolor{black!30} $k$}  & 
	  \multicolumn{1}{c}{\cellcolor{black!30}\textbf{Probabilité}} \\ \hline
			$9$ & $0.00452489$ \\ \hline
			$8$ & $0.06108597$ \\ \hline
			$7$ & $0.24434389$ \\ \hline
			$6$ & $0.3800905$ \\ \hline
			$5$ & $0.24434389$ \\ \hline
			$4$ & $0.06108597$ \\ \hline
			$3$ & $0.00452489$ \\ \hline \hline
			\textbf{Somme:} & $1$ \\ \hline
			\end{tabular}
			\caption[]{Probabilités de la loi Hypergéométrique\\ correspondantes à la combinaison}
		\end{center}
	\end{table}
	Comme dans la colonne du tableau original avec laquelle nous venons de travailler la plus petite valeur est $4$ et la plus grande $8$, nous allons prendre les probabilités de queues pour savoir quelle est la $p$-valeur d'être au-dessus ou égal à $8$ et en-dessous ou égal $4$ (donc il s'agit d'un test en bilatéral). Nous avons alors:
	\begin{table}[H]
		\begin{center}
			\definecolor{gris}{gray}{0.85}
			\begin{tabular}{|c|c|c|}
			\hline
			\multicolumn{1}{c}{\cellcolor{black!30} $k$}  & 
	  \multicolumn{1}{c}{\cellcolor{black!30}\textbf{Probabilité}} \\ \hline
			$9$ & $0.00452489$ \\ \hline
			$8$ & $0.06108597$ \\ \hline
			$4$ & $0.06108597$ \\ \hline
			$3$ & $0.00452489$ \\ \hline \hline
			\textbf{Sum:} & $0.131221719$ \\ \hline
			\end{tabular}
			\caption[]{Sélection des valeurs d'intérêt}
		\end{center}
	\end{table}
	Donc la $p$-valeur est de $13.12\%$. Nous ne pouvons dès lors pas dire que notre tableau d'origine est dans une configuration extrême si par exemple nous choisissons un seuil de $5\%$. De nombreux logiciels ne donnent que la $p$-value.
	
	\begin{tcolorbox}[title=Remarque,colframe=black,arc=10pt]
	Le choix des bornes est discutable avec ce test car si nous choisissons par exemple de nous concentrer sur la probabilité cumulée d'être dans l'intervalle borné fermé de $4$ à $8$ (donc inclus!), nous aurions un résultat de $99.09\%$ et donc il faudrait considérer que nous sommes dans une configuration extrême à un seuil de $5\%$. Donc le choix des bornes avec une loi discrète est toujours délicat dans un test contrairement à un test basé sur une loi continue qui ne souffre absolument pas de ce problème. La majorité des logiciels de statistiques que nous connaissons prennent un intervalle borné ouvert pour l'intervalle (ce qui correspond alors premier calcul que nous avons effectué avec la $p$-valeur de $13.12\%$).
	\end{tcolorbox}	
	Enfin, pour clore, le lecteur pourra vérifier qu'il tombera sur le même résultat quelle que soit la case qu'il choisit comme référence.

	Indiquons que le test exact de Fisher peut bien évidemment être utilisé en machine learning pour mesurer la performance de classification binaire!
	
	\subsubsection{Kappa d'agrément de Cohen}
	Si nos jugements reflètent notre pensée, ils sont plus rarement en accord avec ceux d'autrui.

	Cette variabilité interindividuelle bénéfique pour l'Homme, est cependant pénalisante dans de nombreuses disciplines scientifiques,  où il est souvent nécessaire d'évaluer et d'améliorer l'accord entre des informations de même nature appliquées au même objet dans une exigence de contrôle de la qualité ou d'assurance qualité ou encore d'analyse sensorielle.

	Le test non paramétrique kappa de Cohen  permet par exemple de chiffrer l'accord binaire (dichotomique) entre deux ou plusieurs observateurs ou techniciens lorsque les jugements sont qualitatifs.
	
	Cette statistique kappa est fréquemment utilisée pour tester la "\NewTerm{concordance inter-juges}\index{concordance inter-juges}". L'importance de la concordance inter-juges réside dans le fait qu'elle représente la mesure de la manière dont les données collectées dans une étude sont des représentations correctes des variables mesurées. La manière avec laquelle les collecteurs de données (évaluateurs) attribuent le même score à la même variable est donc appelée "concordance inter-juges". Bien qu'il existe une variété de méthodes pour mesurer la fiabilité inter-juges (ou "inter-évaluateurs"), elle était traditionnellement mesurée en pourcentage de concordance, calculé comme le nombre de scores de concordance divisé par le nombre total de scores. En 1960, Jacob Cohen a critiqué l'utilisation du pourcentage de concordance en raison de son incapacité à rendre compte de l'accord fortuit. Il a alors présenté le kappa de Cohen, développé pour tenir compte de la possibilité que les évaluateurs devinent réellement au moins certaines variables en raison de l'incertitude. Comme la plupart des statistiques de corrélation, le kappa peut aller de $-1$ à $+1 $. Bien que le kappa soit l'une des statistiques les plus couramment utilisées pour tester la fiabilité inter-juges, il présente des limites. Les jugements sur le niveau de kappa qui devrait être acceptable pour la recherche en santé sont remis en question. L'interprétation suggérée par Cohen peut être trop indulgente pour les études liées à la santé, car elle implique qu'un score aussi bas que $0.41$ pourrait être acceptable. Le kappa et le pourcentage de concordance sont comparés, et les niveaux de kappa et de pourcentage de concordance qui devraient être exigés dans les études de santé sont suggérés.
	
	De nombreuses situations dans le secteur de la santé dépendent de plusieurs personnes pour collecter des données de recherche ou de laboratoire clinique. La question de la concordance ou de l'accord entre les individus collectant des données se pose immédiatement en raison de la variabilité parmi les observateurs humains. Des études de recherche bien conçues doivent donc inclure des procédures qui mesurent l'accord entre les différents collecteurs de données. La fiabilité inter-juges est une préoccupation à un degré ou à un autre dans la plupart des grandes études en raison du fait que plusieurs personnes collectant des données peuvent expérimenter et interpréter les phénomènes d'intérêt différemment!
	
	\begin{tcolorbox}[title=Remarque,colframe=black,arc=10pt]
	Il existe un certain nombre de statistiques qui ont été utilisées pour mesurer la fiabilité inter-juges et intra-juges. Une liste partielle comprend le pourcentage de concordance, le kappa de Cohen (pour deux évaluateurs), le Fleiss kappa (adaptation du kappa de Cohen pour trois évaluateurs ou plus) le coefficient de contingence, le Pearson $R$ et le Spearman rho, le coefficient de corrélation intra-classe, le coefficient de corrélation de concordance et l'alpha de Krippendorff (utile lorsqu'il y a plusieurs évaluateurs et plusieurs évaluations possibles). L'utilisation de coefficients de corrélation tels que le $R$ de Pearson peut être un mauvais reflet du degré d'accord entre les évaluateurs, entraînant une surestimation ou une sous-estimation extrêmes du niveau réel d'accord des évaluateurs.
	\end{tcolorbox}	
	Prenons le cas dans le domaine médical où deux ou plusieurs praticiens examinant le même patient proposent des diagnostics différents ou des décisions thérapeutiques différentes. En l'absence d'une référence, cette multiplication des avis n'apporte pas la sécurité attendue d'un parfait accord diagnostique ou thérapeutique pour le médecin traitant et le patient. Il est donc important que l'accord dans une équipe de travail ou entre plusieurs équipes soit le meilleur possible pour garantir la qualité et la continuité des soins.
	
	Une solution consiste ici à réaliser une séance de "mise en concordance" entre les médecins pour estimer leur taux d'accord par le coefficient kappa et d'étudier leurs désaccords pour y remédier.

	Pour illustrer le principe, considérons le cas très important où deux responsables qualité ont analysé $11$ pièces pour les rejeter ou les accepter. Ils ont obtenu:
	\begin{table}[H]
		\begin{center}
		\begin{tabular}{|l|l|c|c|c|}
		\hline
		\multicolumn{2}{|c|}{\cellcolor{black!30}} & \multicolumn{3}{c|}{\cellcolor{black!30}\textbf{Bob}}                                             \\ \hline
		\cellcolor{black!30}{} & \cellcolor{black!30}{} & \cellcolor{black!30}\textbf{Rejeté} & \cellcolor{black!30}\textbf{Accepté} & \cellcolor{black!30}\textbf{Total}\\ \hline
		\cellcolor{black!30}  & \multicolumn{1}{l|}{\multirow{1}{*}{\cellcolor{black!30}\textbf{Rejeté}}} & $3$ & $2$ & $5$ \\ \cline{2-5} 
		\parbox[t]{2mm}{{\rotatebox[origin=c]{90}{\cellcolor{black!30}\textbf{Alice}}}} & \multirow{1}{*}{\cellcolor{black!30}\textbf{Accepté}}                       & $1$ & $5$ & $6$ \\ \cline{2-5}  
		 \cellcolor{black!30} & \multirow{1}{*}{\cellcolor{black!30}\textbf{Total}} {\cellcolor{black!30}} & $4$ & $7$ & $11$ \\ \cline{2-5}  \hline
		\end{tabular}
		\caption[]{Tableau de contingences dichotomique d'exemple}
		\end{center}
	\end{table}
	Les fréquences théoriques étant obtenues par une règle de trois (même calcul de règle de trois que pour le test d'indépendance du Khi-deux et le test exact de Fisher vus plus haut):
	\begin{table}[H]
		\begin{center}
		\begin{tabular}{|l|l|c|c|c|}
		\hline
		\multicolumn{2}{|c|}{\cellcolor{black!30}} & \multicolumn{3}{c|}{\cellcolor{black!30}\textbf{Bob}}                                             \\ \hline
		\cellcolor{black!30}{} & \cellcolor{black!30}{} & \cellcolor{black!30}\textbf{Rejeté} & \cellcolor{black!30}\textbf{Accepté} & \cellcolor{black!30}\textbf{Total}\\ \hline
		\cellcolor{black!30}  & \multicolumn{1}{l|}{\multirow{1}{*}{\cellcolor{black!30}\textbf{Rejeté}}} &  $(4/11)\cdot 5=1.82$ & $(7/11)\cdot 5=3.18$ & $5$ \\ \cline{2-5} 
		\parbox[t]{2mm}{{\rotatebox[origin=c]{90}{\cellcolor{black!30}\textbf{Alice}}}} & \multirow{1}{*}{\cellcolor{black!30}\textbf{Accepté}}                       & $(4/11)\cdot 6=2.18$ & $(7/11)\cdot 6=3.82$ & $6$ \\ \cline{2-5}  
		 \cellcolor{black!30} & \multirow{1}{*}{\cellcolor{black!30}\textbf{Total}} {\cellcolor{black!30}} & $4$ & $7$ & $11$ \\ \cline{2-5}  \hline
		\end{tabular}
		\caption[]{Fréquences théoriques}
		\end{center}
	\end{table}
	Le Kappa de Cohen est défini par le rapport:
	
	Dès lors avec notre exemple:
	
	Cette valeur de $0.441$ indique un accord modéré entre les deux responsables.

	Si plutôt que d'avoir des fréquences, nous travaillons en pourcents (proportions) du total, le Kappa s'écrit alors:
	
	Ce qui donnera avec notre exemple:
	
	Ce qu'il est aussi d'usage d'écrire de manière plus condensée sous la forme:
	
	avec:
	
	où $+1$ correspond à un accord parfait et $-1$ à un désaccord parfait. Évidemment, pour qu'il y ait accord parfait, il faut que les cellules (Rejeté, Rejeté) et (Accepté, Accepté) soient égales et que les autres soient nulles.
	
	L'interprétation suivante est d'usage pour la partie positive de $\kappa_C$ (la négative ayant aucun intérêt):
	\begin{table}[H]
		\begin{center}
			\definecolor{gris}{gray}{0.85}
			\begin{tabular}{|c|c|}
			\hline
			\multicolumn{1}{c}{\cellcolor{black!30}\textbf{Intervalle}}  & 
	  \multicolumn{1}{c}{\cellcolor{black!30}\textbf{Interprétation}} \\ \hline
			$0.8\leq \kappa \leq 1$ & Très bon accord\\ \hline		
			$0.6\leq \kappa < 0.8$ & Bon accord\\ \hline
			$0.4\leq \kappa < 0.6$ & Accord modéré\\ \hline
			$0.2\leq \kappa < 0.4$ & Accord faible\\ \hline
			$0.0\leq \kappa < 0.2$ & Accord nul\\ \hline
			\end{tabular}
			\caption{$\kappa_C$ Interprétations d'usage}
		\end{center}
	\end{table}
	Il faut cependant être très critique (comme toujours!) en utilisant ce type d'outil. Comprendre sa construction permet aussi d'en identifier les faiblesses et hypothèses qui sont tout à fait discutables.

	Indiquons que le Kappa de Cohen peut bien évidemment être utilisé en machine learning pour mesurer la performance de classification binaire!
	
	\subsubsection{Test de McNemar}\label{mcnemar test}
	Le test de McNemar pourrait très bien se calculer en même temps que le Kappa de Cohen (le premier étant un test d'hypothèse statistiques et le deuxième un uniquement un estimateur ponctuel empirique de concordance). L'idée est que sous l'hypothèse nulle $H_0$ (appelée dans ce cas particulier "hypothèse de symétrie"), l'une des diagonales du tableau devrait avoir des valeurs égales.  En d'autres termes sous la forme des proportions et en ne nous concentrant que sur une des deux diagonales:
	
	ou de fréquences:
	
	Sachant que:
	
	et sous la condition que $n$ est suffisamment grand, nous pouvons écrire en nous basant sur une loi Binomiale dont le comportement est asymptotiquement Normal:
	
	Nous pouvons nous rendre compte que cela équivaut à écrire:
	
	Dans la littérature spécialisée, nous retrouvons souvent cette dernière relation sous la forme:
	
	Pour en revenir à notre relation initiale, certains en prennent le carré et approximent alors le carré de $Z$ comme une loi du Khi-deux à un degré de liberté (mais bon dans la réalité approximer une loi Normale centrée réduite par une loi du Khi-deux à un degré de liberté c'est un peu n'importe quoi...):
	
	qui est souvent la relation définie dans les livres (sans démonstration...) comme étant le "\NewTerm{test de McNemar}\index{test de McNemar}".

	Le test est normalement effectué en bilatéral. L'avantage du test de McNemar est la facilité avec laquelle nous pouvons construire un intervalle de confiance de la différence de la diagonale. Effectivement, partons de l'estimateur de la différence:
	
	Dès lors en utilisant aussi la variance (et la covariance) démontrée lors de notre étude de la loi multinomiale à la page \pageref{covariance trinomial distribution}, il vient:
	
	et donc nous pouvons faire un intervalle de confiance approximatif si les conditions habituelles sont respectées sous la forme:
	
	C'est-à-dire un petit plus explicitement:
	
	avec comme nous venons de le prouver:
	
	
	\pagebreak
	\begin{tcolorbox}[colframe=black,colback=white,sharp corners]
	\textbf{{\Large \ding{45}}Exemple:}\\\\
	Lors d'un audit social, un sondage est mené sur $200$ salariées à propose de l'organisation du travail. Après des réaménagements, la même question est posée. Peut-on considérer qu'il y a eu de réels changements?
	\begin{table}[H]
		\centering
		\definecolor{gris}{gray}{0.85}
		\begin{tabular}{|c|c|c|}
		\hline
		{\cellcolor{black!30}}  & \multicolumn{1}{c}{\cellcolor{black!30}\textbf{Oui (avant)}}  & 
  \multicolumn{1}{c}{\cellcolor{black!30}\textbf{Non (avant)}} \\ \hline
		\multicolumn{1}{c}{\cellcolor{black!30}\textbf{Oui (après)} }& $55$ & $38$ \\ \hline		
		\multicolumn{1}{c}{\cellcolor{black!30}\textbf{Non (après)}} & $38$ & $82$ \\ \hline
		\end{tabular}
		\caption[]{Sondage avant/après (apparié)}
	\end{table}
	Nous avons alors en prenant arbitrairement la diagonale $(25, 38)$ pour l'analyse:
	
	La majorité des logiciels de statistique ne vous donneront pas $2.683$ car ils appliquent une correction empirique à cause de l'approximation grossière qu'est la loi de Khi-deux. Vous aurez alors sur l'écran des logiciels la valeur de $2.285$. pour $Z^2$.\\

	La $p$-valeur sera elle donnée (sans la correction) avec un logiciel comme la version française de Microsoft Excel 14.0.6123 par :

	\begin{center}
	\texttt{= 1-LOI.KHIDEUX.N(2.683, 1, VRAI)} $= 10.14\%$
	\end{center}
	et avec la correction elle donnerait environ $13\%$. Donc dans les deux cas la $p$-valeur étant supérieure au seuil de $5\%$, on ne peut alors rejetter l'hypothèse nulle $H_0$ comme quoi la différence des deux valeurs est grande.\\

	Nous ne calculerons pas dans l'exemple ici présent l'intervalle de confiance de l'écart car les logiciels de statistiques ont presque tous des méthodes différentes de calcul pour cette valeur.\\
	
	Enfin, pour clore le sujet concernant le test de McNemar, signalons un indicateur empirique souvent utilisé est qui s'appelle le "\NewTerm{coefficient de Yule}\index{coefficient de Yule}" ou "\NewTerm{$Q$ de Yule}" défini par:
	
	où $a,b,c,d$ doivent être des entiers positifs non nuls correspondant aux valeurs de position des tables de contingence similaires à:
	\begin{table}[H]
		\begin{center}
			\definecolor{gris}{gray}{0.85}
			\begin{tabular}{|c|c|c|}
			\hline
			{\cellcolor{black!30}}  & \multicolumn{1}{c}{\cellcolor{black!30}\textbf{Oui}}  & 
	  \multicolumn{1}{c}{\cellcolor{black!30}\textbf{Non}} \\ \hline
			\multicolumn{1}{c}{\cellcolor{black!30}\textbf{Positif} }& $a$ & $b$ \\ \hline		
			\multicolumn{1}{c}{\cellcolor{black!30}\textbf{Négatif}} & $c$ & $d$ \\ \hline
			\end{tabular}
		\end{center}
	\end{table}
	\end{tcolorbox}
	
	\begin{tcolorbox}[title=Remarque,colframe=black,arc=10pt]
	Le $Q$ de Yule est juste la version a $2\times 2$ du coefficient gamma de Goodman et Kruskal. On peut aussi évidemment calculer un rapport de cotes (RC) en utilisant $(a\cdot d)/(b\cdot c)$ (puis calculer son intervalle de confiance en accord avec la démonstration disponible à la page \pageref{odds ratio confidence interval}). Nous pouvons également convertir ce rapport de cotes en une indice allant de $-1$ à $+1$ en utilisant $(\text{RC}-1)/(\text{RC}+1)$.
	\end{tcolorbox}
	Le $Q$ de Yule est toujours un nombre compris entre $-1$ et $+1$. Voici un tableau donnant un interprétation des valeurs des $Q$ de Yule:
	\begin{table}[H]
		\begin{center}
			\definecolor{gris}{gray}{0.85}
			\begin{tabular}{|c|c|}
			\hline
			\multicolumn{1}{c}{\cellcolor{black!30}\textbf{Intervalle}}  & 
	  \multicolumn{1}{c}{\cellcolor{black!30}\textbf{Interprétation}} \\ \hline
			$Q=0$ & Aucune association entre les variables\\ \hline		
			$0 \leq |Q|\leq 0.29$ & Une association négligeable ou très petite\\ \hline
			$0.30 \leq |Q|\leq 0.49$ & Une association modérée entre les variables\\ \hline
			$0.50 \leq |Q|\leq 0.69$ & Une association substantielle entre les variables\\ \hline
			$0.70 \leq |Q| < 1$ & Une association très forte\\ \hline
			$|Q| = 1$ & Une association parfaite entre les événements\\ \hline
			\end{tabular}
			\caption{Table d'interprétation de l'utilisation du $Q$ de Yule}
		\end{center}
	\end{table}
	
	Le "\NewTerm{test de Cochran-Armitage pour la tendance}\index{test Cochran-Armitage pour la tendance}" (voir juste ci-dessous), également nommé "\NewTerm{test du chi carré pour la tendance}\index{test du chi carré pour la tendance}" dans de nombreux logiciels statistiques, est utilisé dans l'analyse de données catégorielles lorsque le but est d'évaluer la présence d'une association entre une variable avec $2$ catégories et une variable avec $k$ catégories. Il modifie le test du chi carré de Pearson pour incorporer un effet d'ordre possible dans les effets des $k$ catégories de la deuxième variable. Par exemple, les doses d'un traitement peuvent être ordonnées comme étant 'faibles', 'moyennes' et 'élevées', et nous pouvons soupçonner que le bénéfice du traitement ne peut pas diminuer à mesure que la dose augmente. Le test de tendance est souvent utilisé comme test basé sur le génotype pour les études d'association génétique groupe test/groupe contrôle.
	
	\paragraph{Dérivation du test de Cochran-Armitage de tendance pour les tables génotypes $2\times 3$}\mbox{}\\\\
	Nous allons dériver ici un cas particulier car le cas général est hors du cadre de ce livre. Considérez pour cela que nous avons échantillonné $R$ cas et $S$ contrôles :
	\begin{table}[H]
		\centering
		\begin{tabular}{|l|c|c|c|c|}
			\hline 
			& \multicolumn{3}{c|}{\textbf{Nombre d'allèles malades}} &  \\ 
			\hline 
		      &  0 & 1 & 2 & \textbf{Totaux}\\
			\hline 
		\textbf{Groupe Test} & $r_0$ & $r_1$ & $r_2$& $R$\\
			\hline 
		\textbf{Groupe Contrôle} & $s_0$ & $s_1$& $s_2$& $S$\\
			\hline 
		\textbf{Totaux} &$n_0$& $n_1$ & $n_2$ & $N$ \\
			\hline
		\end{tabular} 
	\end{table}
	Nous voulons tester $H_0$ contre $H_1$. L'idée est:
	\begin{itemize}
		\item $H_0$ : Toutes les entrées du tableau sont proportionnelles, vs.
		
		\item $H_1$ : Dans une colonne, la valeur absolue de la différence entre la probabilité qu'une observation soit classée comme "Groupe Test" ou "Groupe Témoin" augmente de façon monotone dans le tableau.
	\end{itemize}
	Nous allons travailler avec la différence entre les valeurs de la colonne afin de normaliser d'abord les lignes pour avoir les mêmes sommes:
	\begin{table}[H]
		\centering
		\begin{tabular}{|l|c|c|c|c|}
		\hline 
		& \multicolumn{3}{c|}{\textbf{Nombre d'allèles malades}} &  \\ 
		\hline 
		&  0 &1& 2 & \textbf{Totals}\\
		\hline 
		\textbf{Groupe Test} & $S\cdot{r_0}$ & $S\cdot{r_1}$ & $S\cdot{r_2}$& $R\cdot S$\\
		\hline 
		\textbf{Groupe Contrôle} & $R\cdot{s_0}$ &$R\cdot{s_1}$& $R\cdot{s_2}$& $R\cdot S$\\
		\hline
		\end{tabular} 
	\end{table}

	Nous choisissons un ensemble de scores $ x_1 $, $ x_2 $ et $ x_3 $ et formons les statistiques du test:
	
	
	Sous $H_0$:
	
	En posant $S = N-R$, $s_i = n_i-r_i$, et en choisissant les sccores $x_0 = 0$, $x_1 = 1$, et $x_2 = 2$, la statistique du test devient:
	
	En revenant aux scores génériques $x_i$, nous calculons la variance de $U$ comme suit:
	
	Sous $H_0$ (nous utilisons la variance et covariance de la distribution hypergéométrique\footnote{Si le lecteur n'est pas en mesure de trouver où est la preuve de cette covariance, vous pouvez la trouver ici à la page \pageref{covariance trinomial distribution}} en ignorant le facteur de correction de la population):
	
	Posons $S = N - R$. Choisissons les scores $x_0 = 0$, $x_1 = 1$, et $x_2 = 2$ et la relation précédente se réduit alors à:
	
	Pour $N$ large (cela justifie le fait que nous avons ignoré le facteur de correction de population finie plus tôt), nous avons alors:
	
	Donc, en utilisant la notation de Sasieni (1997):
	
	Encore une fois, nous remarquons que le test de Cochran-Armitage pour la tendance utilise une statistqiue de Wald et appartient donc à la famille des tests de Wald.
	
	\begin{tcolorbox}[title=Remarque,colframe=black,arc=10pt]
	Pour un exemple pratique et la mise en œuvre de ce test, notre lecteur peut se référer au livre compagnon sur R.
	\end{tcolorbox}
	
	\pagebreak
	\subsection{Statistiques de survie}\label{survival analysis}
	L'analyse de survie est une branche des statistiques permettant d'analyser la durée attendue jusqu'à ce qu'un ou plusieurs événements se produisent, tels que la mort d'organismes biologiques, une défaillance des systèmes mécaniques, le délai de réapprovisionnement des clients, etc.
	
	Ce sujet est nommé "\NewTerm{théorie de la fiabilité}\index{théorie de la fiabilité}" ou "\NewTerm{analyse de fiabilité}\index{analyse de fiabilité}" dans le domaine de l'ingénierie (\SeeChapter{voir section Génie Industriel page \pageref{preventive maintenance}}), "\NewTerm{analyse de la durée}" ou "\NewTerm{modélisation de la durée}\index{modélisation de la durée}" dans le domaine de l'économie, et "\NewTerm{analyse de l'historique des événements}\index{analyse de l'historique des événements}" dans le domaine de la sociologie. L'analyse de survie tente de répondre à des questions telles que: quelle est la proportion d'une population qui survivra au-delà d'un certain temps? Parmi ceux qui survivent, à quelle vitesse mourront-ils ou échoueront-ils? Peut-on prendre en compte plusieurs causes de décès ou d'échec? Comment des circonstances ou des caractéristiques particulières augmentent-elles ou diminuent-elles la probabilité de survie?

	Pour répondre à ces questions, il est nécessaire de définir la "durée de vie". Dans le cas de la survie biologique, la mort est sans ambiguïté, mais pour la fiabilité mécanique, la défaillance peut ne pas être bien définie, car il peut bien y avoir des systèmes mécaniques dans lesquels la défaillance est partielle, ou non localisée dans le temps. Même dans les problèmes biologiques, certains événements (par exemple, une crise cardiaque ou une autre défaillance d'organe) peuvent avoir la même ambiguïté. La théorie exposée ci-dessous suppose des événements bien définis à des moments précis; d'autres cas peuvent être mieux traités par des modèles qui prennent explicitement en compte les événements ambigus.

	Plus généralement, l'analyse de survie implique la modélisation des données temporelles d'événements; dans ce contexte, la mort ou l'échec (rechute) est considéré comme un "événement" dans la littérature d'analyse de survie - traditionnellement, un seul événement se produit pour chaque sujet, après quoi l'organisme ou le mécanisme est mort ou brisé.
	
	Dans la pratique nous considérons:
	\begin{itemize}
		\item Les tables de mortalité (\SeeChapter{voir section Dynamique des populations page \pageref{life table}}) pour décrire les temps de survie des membres d'un groupe.
		
		\item La méthode Kaplan-Meier permet également de décrire les temps de survie des membres d'un groupe (voir ci-dessous).

		\item Le test Cochran-Mantel-Haenszel (test Log-Rank) pour comparer les temps de survie de deux ou plusieurs groupes (voir ci-dessous

		\item Le modèle de risque proportionnel de Cox (régression de Cox) pour décrire l'effet des variables qualitatives ou quantitatives sur la survie (voir ci-dessous).
	\end{itemize}
	Voici une taxonomie plus exhaustive (c'est-à-dire encore incomplète) des méthodes d'analyse des survie:
	\begin{figure}[H]
		\centering
		\includegraphics[width=1.0\textwidth]{img/arithmetics/taxonomy_survival_analysis_methods.jpg}
		\caption[Taxonomie des méthodes développées pour l'analyse de survie]{Taxonomie des méthodes développées pour l'analyse de survie\\ (source: ACM Computing Surveys, Vol.1, n ° 1, article 1, avril 2018)}
	\end{figure}

	\pagebreak
	\subsubsection{Taux de survie de Kaplan-Meier}
	OK ... nous n'aimons pas faire ça dans ce livre !!! Mais pour le confort de nos lecteurs et à cause des commentaires des lecteurs, copions / collons ici la théorie sur le taux de survie de Kaplan-Meier telle qu'introduite initialement dans la section de Génie Industriel à la page \pageref{Kaplan-Meier survival model}!
	
	Dans des domaines comme l'industrie, la médecine ou en biologie, nous intéressons souvent aux:
	\begin{enumerate}
		\item Durée de survie après un événement grave
		\item Durée de rémission après un traitement ou une opération
		\item Durée d'un symptôme après une anomalie
		\item Durée d'une infection sans symptôme
	\end{enumerate}
	Nous cherchons très souvent à distinguer au moins "l'événement d'intérêt":
	\begin{enumerate}
		\item Arrêt du système après l'événement grave
		\item Fin de la rémission
		\item Fin d'un symptôme après anomalie
		\item Début d'un symptôme lors d'une infection
	\end{enumerate}
	de la variable a expliquer "durée avant l'apparition de l'événement d'intérêt":
	\begin{enumerate}
		\item Temps écoulé avant l'arrêt du système
		\item Temps écoulé avant la fin de la rémission
		\item Temps écoulé avant la fin du symptôme
		\item Temps écoulé sans symptôme
	\end{enumerate}
	\textbf{Définitions (\#\mydef):}
	\begin{enumerate}
		\item[D1.] Nous appelons "\NewTerm{rémission}\index{rémission}", la diminution d'une maladie ou d'un dysfonctionnement de façon temporaire.
		
		\item[D2.]  La "\NewTerm{durée de survie}\index{durée de survie}" ou "\NewTerm{durée de vie}\index{durée de vie}" $T$ désigne le temps qui s'écoule depuis un instant initial (début d'un traitement, diagnostic, panne, etc.) jusqu'à la survenue d'un événement d'intérêt final (décès du patient, rechute, rémission, guérison, réparation, etc.). Nous disons que l'objet de l'étude "survit au temps $t$" si, à cet instant l'évènement d'intérêt final n'a pas encore eu lieu.
	\end{enumerate}
	
	\begin{tcolorbox}[title=Remarque,colframe=black,arc=10pt]
	Bien que ce type d'étude puisse être associé à la maintenance préventive (\SeeChapter{voir section Génie Industriel page \pageref{preventive maintenance}}), les statisticiens l'associent plutôt au domaine de "\NewTerm{l'analyse de la survie}\index{analyse de la survie}".
	\end{tcolorbox}
	
	Nous nous intéresserons dans le ce livre à un cadre particulier mais qui peut être facilement généralisé:
	\begin{itemize}
		\item Cohorte/Essai clinique où nous étudions la durée de survie de chaque patient (machine).
		
		\item Tous les patients (machines) n'ont pas la même durée d'observation (instants différents d'entrée dans l'étude).
		
		\item Nous avons des informations sur le temps de survie de chaque patient (machine) mais nous ne savons pas quand il arrive exactement.
	\end{itemize}
	Des deux derniers points, nous concluons que le temps de survie peut être censuré. Donc les techniques statistiques habituelles ne s'appliquent pas directement car les données censurées demandent un traitement particulier (bien évidemment, nous enlevons les données censurées nous perdons de l'information). Il va sans dire que ce type de situation est extrêmement fréquent et donc l'étude de l'estimation de Kaplan-Meier est très importante.
	
	\textbf{Définition (\#\mydef):}  La durée $T$ est dite censurée si la durée n'est pas intégralement observée. Les différents types de censures sont:
	\begin{itemize}
		\item La censure de type I: fixée à droite. Dans cette situation, la durée n'est pas observable au-delà d'une durée maximale, fixe appelée "\NewTerm{censure fixe}\index{censure fixe}" et imposée. Donc, soit nous avons l'opportunité d'observer la vraie durée de l'événement d'intérêt pour l'élément s'il a lieu avant la censure fixe, soit nous nous contenterons de la durée de la censure fixe s'il l'événement d'intérêt n'a pas eu lieu avant.
		
		\item  La censure de type I: fixée à gauche. Dans cette situation, la durée d'étude n'est pas observable avant une date connue appelée "\NewTerm{censure d'attente}\index{censure d'attente}" et imposée. Donc soit, l'événement d'intérêt a lieu au moment de la censure d'attente, soit après. S'il y a lieu après, nous considérerons la durée entre la date de la censure et la date de l'événement d'intérêt.
		
		\item  La censure de type II: attente. Dans cette situation, nous observons les durées de vie de $n$ individus jusqu'à ce que $m\leq n$ individus aient vu l'évènement se produire (décédés). La durée considérée est donc celle du début de l'expérience jusqu'à l'événement d'intérêt pour le  $m$-ème.
		
		\item  La censure de type III: aléatoire à gauche. Dans cette situation, nous ne connaissons pas quand l'événement d'intérêt a eu lieu (car nous avons commencé à observer le sujet d'étude trop tard). Nous ne pouvons alors pas traiter des "durées" dans le sens mesurable du terme et il nous faudra nous limiter à un simple comptage.
		
		\item La censure de type III: aléatoire à droite. Dans cette situation, nous ne connaissons pas quand l'événement aura lieu (car nous avons arrêté  d'observer le sujet avant qu'il y ait lieu pour une raison quelconque). Nous ne pouvons alors pas traiter des "durées" dans le sens mesurable du terme et il nous faudra nous limiter à un simple comptage.
		
		\item La censure de type IV: aléatoire par intervalle. Dans cette situation, nous avons un mélange de la censure aléatoire à gauche et à droite. C'est-à-dire que pour certains sujets d'étude, nous ne savons pas quand l'événement d'intérêt a commencé, et pour d'autres, nous ne savons pas quand il aura lieu (s'il a lieu...).
	\end{itemize}
	Dans l'industrie des machines, nous savons souvent affaire à la censure de type I: fixée  droite. Dans le domaine médical, lors d'essais cliniques, nous avons souvent affaire à une censure de type III aléatoire à droite. Dans le cas des pandémies, nous avons affaire à des censure de type III aléatoire à gauche.
	
	Pour introduire ce sujet, plutôt que de faire de la théorie obscure, comme toujours dans ce livre, nous préférons une approche pragmatique. Supposons que l'étude soit un essai clinique impliquant deux groupes de patients recevant deux types de traitements. Deux questions importantes se posent aux médecins lors d'un essai clinique de phase II (la phase I concerne l'approbation de non-toxicité pour l'homme et la phase 0 pour les animaux):
	\begin{enumerate}
		\item[Q1.]  L'un des deux traitements est-il plus efficace que l'autre en matière d'amélioration de la survie des patients?
		
		\item[Q2.] Peut-on mettre en évidence des facteurs prognostiques, c'est-à-dire qui améliorent/détériorent la survie?
	\end{enumerate}
	
	\begin{figure}[H]
		\centering
		\includegraphics[width=0.75\textwidth]{img/arithmetics/clinical_trials.jpg}
		\caption{Développement de médicaments et essais cliniques}
	\end{figure}
	
	Pour répondre à la question Q1 nous pouvons mettre en place des méthodes statistiques qui vont permettre de comparer les deux groupes de patients qui reçoivent les deux types de traitement.

	Pour répondre à la question Q2 nous pouvons proposer un modèle qui relie la durée de survie des patients à des variables explicatives et mettre en évidence des facteurs pronostiques.

	Accompagnons la théorie directement d'un exemple et considérons le tableau suivant où deux cohortes de patients (nous passons de l'ingénierie mécanique à l'ingénierie humaine...) de même taille initiale atteints de leucémie aiguë testent un médicament (6-MP) contre un placeboo\footnote{Rappelons que dans la science des médicaments, un "placebo pur" est un traitement sans aucune substance active; un "placebo impur" est un produit pharmacologiquement actif mais sans effet sur la pathologie traitée, ou dont l'efficacité n'a pas été \underline{suffisamment mise en évidence}.} bien évidemment en double aveugle.

	Nous avons les durées de rémission suivantes pour 21 patients (le tableau de 21 lignes indique donc le nombre de semaines pendant lesquelles un patient est considéré comme guéri après traitement avant de rechuter):
	\begin{table}[H]\centering
		\centering
		\definecolor{gris}{gray}{0.85}
		\begin{tabular}{|c|c|}
			\hline
			\multicolumn{1}{c}{\cellcolor{black!30}\textbf{Groupe 6-mercaptopurine (6-MP)}} & 
  \multicolumn{1}{c}{\cellcolor{black!30}\textbf{Groupe Placebo}} \\ \hline
			 $6$ & $1$ \\ \hline
			 $6$ & $1$ \\ \hline
			 $6$ & $2$ \\ \hline
			 $6+$ & $2$ \\ \hline
			 $7$ & $3$ \\ \hline
			 $9+$ & $4$ \\ \hline
			 $10$ & $4$ \\ \hline
			 $10+$ & $5$ \\ \hline
			 $11+$ & $5$ \\ \hline
			 $13$ & $8$ \\ \hline
			 $16$ & $8$ \\ \hline
			 $17+$ & $8$ \\ \hline
			 $19+$ & $8$ \\ \hline
			 $20+$ & $11$ \\ \hline
			 $22$ & $11$ \\ \hline
			 $23$ & $12$ \\ \hline
			 $25$ & $12$ \\ \hline
			 $32+$ & $15$ \\ \hline
			 $32+$ & $17$ \\ \hline
			 $34+$ & $22$ \\ \hline
			 $35+$ & $23$ \\ \hline
		\end{tabular}
		\caption[]{ Analyse de survie de deux cohortes avec censure à droite}
	\end{table}
	Le signe "$+$" correspond à des patients qui ont quitté l'étude à la semaine considérée. Ils sont donc censurés. Par exemple, le 4ème patient est perdu de vue pour une raison quelconque au bout de $6$ semaines de traitement avec le 6-MP: il a donc une durée de rémission \underline{supérieure} à $6$ semaines. Donc dans l'étude 6-MP, il y a $21$ patients avec et 12 données censurées.
	
	\begin{tcolorbox}[title=Remarque,colframe=black,arc=10pt]
	Le modèle théorique suppose que la censure est indépendant du temps de survie (censure non informative). Mais si la censure est du à l'arrêt du traitement, l'hypothèse d'indépendance n'est pas valide.
	\end{tcolorbox}
	Pour le groupe placebo il est simple de faire une courbe de survie. Il suffit de produire le tableau suivant (pour les semaines omises, on impose bien évidemment le nombre de rémission comme constantes):
	\begin{table}[H]\centering
		\centering
		\definecolor{gris}{gray}{0.85}
		\resizebox{\textwidth}{!}{\begin{tabular}{|c|c|c|}
				\hline
				\multicolumn{1}{c}{\cellcolor{black!30}\textbf{Semaine $i$}} & 
  \multicolumn{1}{c}{\cellcolor{black!30}\textbf{Nombre \# de rémissions à la semaine $i$}} & 
  \multicolumn{1}{c}{\cellcolor{black!30}\textbf{Proportion (probabilité)
de rémission à la semaine $i$}} \\ \hline
				 $0$ & $21$ & $100\%$ \\ \hline
				 $1$ & $19$ & $19/21=90\%$ \\ \hline
				 $2$ & $17$ & $17/21=81\%$ \\ \hline
				 $3$ & $16$ & $16/21=76\%$ \\ \hline
				 $4$ & $14$ & $14/21=67\%$ \\ \hline
				 $5$ & $12$ & $12/21=57\%$ \\ \hline
				 $8$ & $8$ & $8/21=38\%$ \\ \hline
				 $11$ & $6$ & $6/21=29\%$ \\ \hline
				 $12$ & $4$ & $4/21=19\%$ \\ \hline
				 $15$ & $3$ & $3/21=14\%$ \\ \hline
				 $17$ & $2$ & $2/21=10\%$ \\ \hline
				 $22$ & $1$ & $1/20=0.05\%$ \\ \hline
				 $23$ & $0$ & $0\%$ \\ \hline
		\end{tabular}}
	\caption[]{Analyse de survie de deux cohortes avec censure à droite}
	\end{table}
	Donc si les données ne sont pas censurées, la survie $S(t)$ peut être estimée par la proportion d'individus ayant survécu à l'instant $t$, qu'il est d'usage d'écrire sous la forme mathématique suivante:
	
	L'idée est donc d'estimer:
	
	par la proportion de patients ayant survécu jusqu'au temps $t$.
	
	Si les données sont censurées, l'estimation de la fonction de survie nécessite des outils spécifiques. Kaplan et Meier ont proposé dans ce cas particulier le calcul suivant:
	
	Voyons cela sous une forme un peu plus mathématique.  Si nous notons $X(1)\leq X(2)\leq ...\leq X(n)$ les instants (ordonnées) où un évènement s'est produit (décès ou censure), nous avons alors:
	
	Avec bien évidemment:
	
	Nous estimons:
	
	où $d_k$ est le nombre de décès observé au temps correspondant à l'événement $X(k)$ et $R_k$ est le nombre d'individus à risque (exposés au risque de décès) juste avant $X(k)$.
	
	Nous définissons "\NewTerm{l'estimateur de Kaplan-Meier}\index{estimateur de Kaplan-Meier}" pour tout $X(0)\leq t <X(k)$ par:
	
	où $\hat{\lambda}_k$ est le hasard discret (i.e. probabilité conditionnelle de décès).
	
	\begin{tcolorbox}[title=Remarque,colframe=black,arc=10pt]
	Il existe un estimateur beaucoup plus simple nommé "\NewTerm{estimateur de Nelson-Aalen}" et simplement défini par:
	
	Ce dernier estimateur n'est qu'une somme des temps de survie observés.
	\end{tcolorbox}
	
	Par conséquent, nous faisons maintenant pour le groupe 6-MP (pas le groupe placebo !!!!!) ce qui suit:
	\begin{table}[H]
		\centering
		\definecolor{gris}{gray}{0.85}
			\resizebox{\textwidth}{!}{\begin{tabular}{|c|c|c|c|}
				\hline
				\multicolumn{1}{c}{\cellcolor{black!30}\textbf{Durée de rémission}} & 
  \multicolumn{1}{c}{\cellcolor{black!30}\textbf{Sujets en rémission}} & 
  \multicolumn{1}{c}{\cellcolor{black!30}\textbf{Probabilité de ne pas rechuter}} & 
  \multicolumn{1}{c}{\cellcolor{black!30}\textbf{\textbf{Probabilité de survie de}}} \\ 
  				\multicolumn{1}{c}{\cellcolor{black!30}\textbf{(non censurées) observées}} & 
  \multicolumn{1}{c}{\cellcolor{black!30}\textbf{au \underline{début} de $k$}} & 
  \multicolumn{1}{c}{\cellcolor{black!30}\textbf{à $k$ sachant qu'on est en}} & 
  \multicolumn{1}{c}{\cellcolor{black!30}\textbf{Kaplan-Meier}} \\
   \multicolumn{1}{c}{\cellcolor{black!30}} & 
   \multicolumn{1}{c}{\cellcolor{black!30}} & 
  \multicolumn{1}{c}{\cellcolor{black!30}\textbf{rémission à $k-1$ ($\hat{P}_k$)}} &  \multicolumn{1}{c}{\cellcolor{black!30}} \\ \hline
				 $0$ & $21$ & $21/21=100\%$ & $100\%$\\ \hline
				 $6$ & $21$ & $18/21=1-3/21=85.7\%$ & $100\%\cdot85.7\%=85.7\%$ \\ \hline
				 $7$ & $17$ & $16/17=1-1/17=94.1\%$ & $85.7\%\cdot94.1\%=80.7\%$\\ \hline
				 $10$ & $15$ & $14/15=1-1/15=93.3\%$ & $80.7\%\cdot93.3\%=75.3\%$\\ \hline
				 $13$ & $12$ & $11/12=1-1/12=91.7\%$ & $75.3\%\cdot91.7\%=69\%$\\ \hline
				 $16$ & $11$ & $10/11=1-1/11=90.9\%$ & $69\%\cdot90.9\%=62.7\%$\\ \hline
				 $22$ & $7$ & $6/7=1-1/7=85.7\%$ & $62.7\%\cdot85.7\%=53.8\%$\\ \hline
				 $23$ & $6$ & $5/6=1-1/6=83.3\%$ & $53.8\%\cdot83.3\%=44.8\%$ \\ \hline
		\end{tabular}}
		\caption[]{6-MP analyse de survie}
	\end{table}
	Avec les courbes de survie correspondantes (y compris l'intervalle de confiance):
	\begin{figure}[H]
		\centering
		\includegraphics[scale=0.9]{img/arithmetics/kaplan_meier_survival_curve.jpg}
		\caption{Courbes de survie typiques de Kaplan-Meier avec le logiciel R}
	\end{figure}
	On retrouve ainsi les mêmes valeurs que celles données par exemple par Minitab  15.1.1 et R (voir les livres compagnons correspondats sur Minitab ou R pour les détails).
	
	Si nous supposons que les $\hat{\lambda}_k$ ne sont que des proportions binomiales étant donnés les $R_k$, nous avons la moyenne et variance normalisées suivantes de la variable aléatoire binomiale (voir page \pageref{normalized variance and mean of binomial distribution}):
	
	\begin{tcolorbox}[title=Remarque,colframe=black,arc=10pt]
	Malheureusement, c'est une tradition dans l'analyse de survie de désigner la variance normalisée de $\hat{\lambda}_k$ par $\text{V}(\hat{\lambda}_k)$ à la place de $\text{V}(\hat{\lambda}_k/r_k)$...
	\end{tcolorbox}
	Et nous supposerons aussi que les $\hat{\lambda}_k$ sont asymptotiquement indépendants (...).
	
	Puisque $\hat{S}_{\text{KM}}(t)$ est une fonction des $\hat{\lambda}_k$, nous pouvons estimer sa variance en utilisant la méthode Delta (\SeeChapter{voir section Méthodes Numériques page \pageref{delta method}}) qui a comme résultat important pour rappel:
	
	avec:
	
	Par conséquent, au lieu de traiter directement  $\hat{S}_{\text{KM}}(t)$, nous allons regarder son logarithme naturel:
	
	Ainsi, par l'indépendance approximative des $\hat{\lambda}_k$ et en gardant à l'esprit que nous venons de choisir $g()$ comme logarithme naturel et en utilisant la méthode delta:
	
	Et en y appliquant la méthode delta:
	
	puis en réorganisant et en injectant le résultat précédent, nous obtenons:
	
	L'approximation suivante:
	
	est nommée "\NewTerm{relation de Greenwood}\index{relation de Greenwood}". Si nous prenons la racine carrée alors le résultat est associé à l'erreur standard (à cause de la division par $ R_k $):
	
	Afin de construire une barre d'erreur pour $\hat{S}_{\text{KM}}(t)$, nous devons faire une sorte d'hypothèse de distribution. Le plus simple est de supposer que $\hat{S}_{\text{KM}}(t)$ est Normalement distribué de telle sorte que nous puissions écrire:
	
	Ou plus explicitement:
	
	Un hic avec ceci, bien sûr, est que la distribution de $\hat{S}_{\text{KM}}(t)$ n'est pas vraiment Normale. Des problèmes surviennent lorsque $\hat{S}_{\text{KM}}(t)$ est proche de $0$ ou $1$ car en effet l'intervalle de confiance peut donner des valeurs à $>1$ ou $<0$. Une approche empirique courante (entre autres!) Consiste alors à prendre pour l'intervalle de confiance la transformation log-log complémentaire:
	
	Étant donné que cette quantité n'est pas limitée, l'intervalle de confiance sera dans la bonne plage lorsque nous referons la transformation inverse:
	
	Pour voir pourquoi cela fonctionne:
	
	Pour obtenir un intervalle de confiance, nous devons maintenant aussi trouver l'erreur standard correspondant à cette transformation log-log, et pour cela nous devons calculer:
	
	De nos calculs précédents, nous savons:
	
	En appliquant à nouveau la méthode delta, nous obtenons:
	
	Nous prenons la racine carrée de ce qui précède pour obtenir l'erreur standard:
	
	Et donc nous avons:
	
	Puisque $\hat{S}_{\text{KM}}(t) = e^{-e^{\hat{W}(t)}}$, les limites de confiance ("modernes") pour l'intervalle de confiance of $\hat{S}_{\text{KM}}(t)$ sont finalement (notez que la limite supérieure et inférieure commutent!):
	
	En remplaçant $\hat{W}(t)=\ln\left(-\left(\hat{S}_{\text{KM}}(t)\right)\right)$ à nouveau dans les limites ci-dessus, nous obtenir la relation "moderne" définitive pour la borne de confiance de l'estimateur de Kaplan-Meier:
	
	
	\subsubsection{Tests de Cochran–Mantel–Haenszel}\label{cochran mantel test}
	En statistique, les tests Cochran–Mantel–Haenszel sont un ensemble de statistiques de test utilisées dans l'analyse de données catégorielles stratifiées. L'une de ces statistiques de test est le "\NewTerm{test de Cochran–Mantel–Haenszel test}\index{test de Cochran–Mantel–Haenszel}" (test (CMH)), qui permet la comparaison de deux groupes sur une réponse dichotomique / catégorielle.
	
	Ce test, également nommé "\NewTerm{test logarithmique des ranks}\index{test logarithmique des ranges}" ou "\NewTerm{test de Cochran-Mantel(-Haenszel) avec temps stratifié}\index{test de Cochran-Mantel(-Haenszel) avec temps stratifié}" ou encore "\NewTerm{test de Mantel-Cox}\index{test de Mantel-Cox}", ou "\NewTerm{Test du Khi carré de Cochran-Mantel-Haenszel}\index{Test du Khi carré de Cochran-Mantel-Haenszel}" ou encore (...) "\NewTerm{test de proportions multi-strates}\index{test de proportions multi-strates}" ... a pour objectif principal en pratique de tester l'hypothèse nulle $H_0$ comme quoi deux courbes de survie (groupe témoin par rapport au groupe test\footnote{L'ironie avec les antivaxistes disant qu'ils ne veulent pas faire partie d'une expérience c'est qu'ils ne se rendent pas compte qu'ils font partie du groupe de contrôle...}), tels que ceux visibles dans la figure ci-dessous (où la population survivante a été normalisée sur l'axe $y$), sont significativement différents ou pas sous les hypothèses que:
	\begin{enumerate}
		\item[H1.] Chaque strate (couche) est indépendante de la précédente.

		\item[H2.] Pour chaque strate, on s'attend à ce que la proportion attendue de survivants / décès soit la même toutes choses égales par ailleurs ... ou \og \textit{ceteris paribus} \fg{} pour ceux qui préfèrent le latin (voir ci-dessous si ce n'est pas clair).

		\item[H3.] Chaque strate est distribuée selon la loi hypergéométrique.

		\item[H4.] La distribution hypergéométrique peut être approchée par une distribution Normale (ce qui pour rappel, ne peut se faire que sous certaines conditions!!!).
	\end{enumerate}
	\begin{figure}[H]
		\centering
		\includegraphics{img/arithmetics/survival_typical_survival_curves.jpg}
		\caption[]{Courbes de survie typiques de Cox}
	\end{figure}
	Dans le test CMH, les données sont organisées dans une série de tables de contingence $2\times 2$ associées, l'hypothèse nulle est que la réponse observée est indépendante du traitement utilisé dans toute table de contingence $2\times 2$. L'utilisation par le test CMH de tables de contingence $2\times 2$ associées augmente la capacité du test à détecter des associations (la puissance du test est augmentée).
	
	En d'autres termes, l'hypothèse nulle $H_0$ du test est que le traitement (médical, mécanique ou autre) n'a aucune influence entre le groupe témoin et le groupe test dans une ou plusieurs strates (correspondant à différents hôpitaux, cohortes, laboratoires, etc. .), cela s'écrit:
	
	Pour introduire ce test, nous créons une table de contingence $2\times 2$ pour chaque strate, qui peut correspondre à un intervalle de temps $[t_i,t_{i+1}]$ où $i=1,2,\ldots,n$ et qui peut être également être assimilés à un hôpital / laboratoire différent pour une même étude clinique / de fiabilité, qui ont généralement la structure suivante:
	\begin{table}[H]
		\centering
		\definecolor{gris}{gray}{0.85}
		\begin{tabular}{|l|c|c|c|}
		\hline
		{\cellcolor{black!30}}  & \multicolumn{1}{c}{\cellcolor{black!30}\textbf{Décès Observés}}  & 
  \multicolumn{1}{c}{\cellcolor{black!30}\textbf{Survivants Observés}} & 
  \multicolumn{1}{c}{\cellcolor{black!30}\textbf{Total}} \\ \hline
		\multicolumn{1}{l}{\cellcolor{black!30}\textbf{Groupe I (contrôle)} }& $a_i$ & $b_i$ & $L_{1i}$\\ \hline		
		\multicolumn{1}{l}{\cellcolor{black!30}\textbf{Groupe II (test)}} & $c_i$ & $d_i$ & $L_{2i}$ \\ \hline
		\multicolumn{1}{l}{\cellcolor{black!30}\textbf{Total}} & $C_{1i}$ & $C_{2i}$ & $n_i$ \\ \hline
		\end{tabular}
		\caption{Table de contingence typique $ 2\times 2$ pour le test CMH}
	\end{table}
	Nous parlons alors de  "\NewTerm{table stratifiée générale $2\times 2\times n$}\index{table stratifiée générale $2\times 2\times n$}" ou de "\NewTerm{table de contingence en trois dimensions}\index{table de contingence en trois dimensions}"...
	\begin{tcolorbox}[title=Remarque,colframe=black,arc=10pt]
	Rappelons que si nous avons un seul tableau et que nous ne voulons comparer que la proportion de survivants ou de décès pour les deux groupes, un test des différences de proportions sera appliqué comme vu précédemment. Nous pouvons également faire un test du chi carré si nous n'avons toujours qu'une seule table (voir aussi plus haut) et que les conditions ad hoc sont remplies ou également un test exact de Fisher si les conditions du test du chi carré ne sont pas remplies. C'est pourquoi dans un bon logiciel statistique (comme Minitab par exemple), ces trois tests sont disponibles côte à côte.
	\end{tcolorbox}
	Ainsi, sous l'hypothèse que toutes choses restent égales par ailleurs (H2), et c'est le coeur de ce test (!!!!), le nombre attendu $E$ d'individus pour chaque cellule de la période $i$ sera comme pour le test exact de Fisher ou le coefficient d'accord kappa de Cohen être égal à \footnote{$L$ désigne la "Ligne" et $C$ la "Colonne"}:
	
	Il faut donc bien comprendre que, par exemple, $E_{a,i} $ représente alors le nombre attendu d'individus morts du groupe témoin s'il se comportait comme tous les individus des groupes Contrôle + Test. Ainsi, si les deux groupes (ou courbes de survie) se comportent de manière identique, la valeur attendue doit être égale à la valeur observée. Pour être sûr que le lecteur comprenne, illustrons ce concept par un exemple précis:
	\begin{tcolorbox}[colframe=black,colback=white,sharp corners]
	\textbf{{\Large \ding{45}}Exemple:}\\\\
	Nous considérons les valeurs observées suivantes:
	\begin{table}[H]
		\begin{center}
			\definecolor{gris}{gray}{0.85}
			\begin{tabular}{|l|c|c|c|}
			\hline
			{\cellcolor{black!30}}  & \multicolumn{1}{c}{\cellcolor{black!30}\textbf{Décès Observés}}  & 
	  \multicolumn{1}{c}{\cellcolor{black!30}\textbf{Survivants Observés}} & 
	  \multicolumn{1}{c}{\cellcolor{black!30}\textbf{Total}} \\ \hline
			\multicolumn{1}{l}{\cellcolor{black!30}\textbf{Groupe I (contrôle)} }& $200$ & $800$ & $1000$\\ \hline		
			\multicolumn{1}{l}{\cellcolor{black!30}\textbf{Groupe II (test)}} & $280$ & $1120$ & $1400$ \\ \hline
			\multicolumn{1}{l}{\cellcolor{black!30}\textbf{Total}} & $480$ & $1920$ & $2400$ \\ \hline
			\end{tabular}
			\caption[]{Cas particulier de valeurs observées pour le test CMH}
		\end{center}
	\end{table}
	Ici, les valeurs attendues donnent:
	\begin{table}[H]
		\begin{center}
			\definecolor{gris}{gray}{0.85}
			\begin{tabular}{|l|c|c|c|}
			\hline
			{\cellcolor{black!30}}  & \multicolumn{1}{c}{\cellcolor{black!30}\textbf{Décès Observés}}  & 
	  \multicolumn{1}{c}{\cellcolor{black!30}\textbf{Survivants Observés}} & 
	  \multicolumn{1}{c}{\cellcolor{black!30}\textbf{Total}} \\ \hline
			\multicolumn{1}{l}{\cellcolor{black!30}\textbf{Groupe I (contrôle)} }& $200$ & $800$ & $1000$\\ \hline		
			\multicolumn{1}{l}{\cellcolor{black!30}\textbf{Groupe II (test)}} & $280$ & $1120$ & $1400$ \\ \hline
			\multicolumn{1}{l}{\cellcolor{black!30}\textbf{Total}} & $480$ & $1920$ & $2400$ \\ \hline
			\end{tabular}
		\end{center}
	\end{table}
	\end{tcolorbox}
	On voit dans ce cas particulier ci-dessus, les valeurs attendues sont égales à celle observée (simplement pour la raison que les ratios $800/1'000$ et $1'120 / 1'400$ représentent dans le tableau le même pourcentage - la même proportion - de $80\%$). Évidemment, si les proportions de survivants observées pour les deux groupes sont toutes choses égales par ailleurs, il doit en être de même pour les morts observées. Donc ce que le lecteur doit comprendre lorsque nous faisons un test MCH, c'est que nous sommes libres de choisir la colonne à analyser (car finalement c'est la même chose!)
	
	Maintenant, nous pouvons remarquer que chacune des relations:
	
	représente en fait la moyenne d'une distribution binomiale ou hypergéométrique (voir les preuves ci-dessus lors de notre étude des lois correspondantes) puisque les deux lois ont la même expression pour la moyenne. Cependant, comme la taille des individus dans les cellules pouvait être significative par rapport au total des lignes ou des colonnes, l'échantillonnage ne peut pas être indépendant. Ensuite, il faut revenir sur la loi hypergéométrique (c'est-à-dire la troisième hypothèse énumérée précédemment).
	
	On voit aussi que par symétrie de la relation ci-dessus, que la variable d'intérêt soit l'attribut colonne ou l'attribut ligne ne change rien en réalité au résultat du calcul comme (par exemple):
	
	La variance pour chaque cellule sera celle de la loi hypergéométrique que nous avons déjà prouvée lors de son étude et qui a était donnée par:
	
	avec pour rappel $l=n-k$.
	
	Si nous adaptons la notation de la relation précédente à celle de notre tableau ci-dessus, cela nous fournit pour toutes les cellules (par la forme analytique de la variance de la loi hypergéométrique la variance a la même expression pour chaque cellule!):
	
	avec respectivement (pour ceux qui veulent faire l'analogie avec notre étude détaillée de la loi hypergéométrique):
	
	Là encore, on voit que qu'il s'agisse de la variable d'intérêt en ligne ou en colonne, la valeur calculée de la variance reste la même !!!
	\begin{tcolorbox}[title=Remarque,colframe=black,arc=10pt]
	Une question courante qui se pose fréquemment pour ceux qui étudient ce test est de savoir pourquoi on ne peut pas simplement faire la somme de toutes les strates (couches) dans un seul tableau (car ce serait bien plus facile, bien sûr ...) Bon on ne peut pas rassembler les tableaux en un seul car ils ne suivent pas forcément la même loi (pas distribués à l'identique dans le sens où la loi hypergéométrique a des réglages différents d'une table à l'autre) et malheureusement, la distribution hypergéométrique n'est pas stable par l'addition!
	\end{tcolorbox}	
	Et alors...??? Comment cela nous aidera-t-il à comparer si les deux courbes de survie sont identiques dans le temps (ou entre différents hôpitaux / laboratoires s'il s'agit d'un test clinique / de fiabilité)?!

	Eh bien... prenons comme exemple compagnon le tableau avec les valeurs observées suivantes:
	\begin{table}[H]
		\centering
		\definecolor{gris}{gray}{0.85}
		\begin{tabular}{|l|c|c|c|}
		\hline
		{\cellcolor{black!30}}  & \multicolumn{1}{c}{\cellcolor{black!30}\textbf{Décès Observés}}  & 
  \multicolumn{1}{c}{\cellcolor{black!30}\textbf{Survivants Observés}} & 
  \multicolumn{1}{c}{\cellcolor{black!30}\textbf{Total}} \\ \hline
		\multicolumn{1}{l}{\cellcolor{black!30}\textbf{Groupe I (contrôle)} }& $200$ & $900$ & $1100$\\ \hline		
		\multicolumn{1}{l}{\cellcolor{black!30}\textbf{Groupe II (test)}} & $280$ & $1120$ & $1400$ \\ \hline
		\multicolumn{1}{l}{\cellcolor{black!30}\textbf{Total}} & $480$ & $2020$ & $2500$ \\ \hline
		\end{tabular}
		\caption[]{Cas particulier des valeurs observées pour le test CMH}
	\end{table}
	On obtient alors les valeurs attendues:
	\begin{table}[H]
		\centering
		\definecolor{gris}{gray}{0.85}
		\begin{tabular}{|l|c|c|c|}
		\hline
		{\cellcolor{black!30}}  & \multicolumn{1}{c}{\cellcolor{black!30}\textbf{Décès Observés}}  & 
  \multicolumn{1}{c}{\cellcolor{black!30}\textbf{Survivants Observés}} & 
  \multicolumn{1}{c}{\cellcolor{black!30}\textbf{Total}} \\ \hline
		\multicolumn{1}{l}{\cellcolor{black!30}\textbf{Groupe I (contrôle)} }& $211.2$ & $888.8$ & $1100$\\ \hline		
		\multicolumn{1}{l}{\cellcolor{black!30}\textbf{Groupe II (test)}} & $280$ & $1120$ & $1400$ \\ \hline
		\multicolumn{1}{l}{\cellcolor{black!30}\textbf{Total}} & $268.8$ & $1'131.2$ & $2500$ \\ \hline
		\end{tabular}
		\caption[]{Valeurs attendues du tableau précédent}
	\end{table}
	Dès lors les différences:
	\begin{table}[H]
		\centering
		\definecolor{gris}{gray}{0.85}
		\begin{tabular}{|l|c|c|}
		\hline
		{\cellcolor{black!30}}  & \multicolumn{1}{c}{\cellcolor{black!30}\textbf{Décès Observés}}  & 
  \multicolumn{1}{c}{\cellcolor{black!30}\textbf{Survivants Observés}} \\ \hline
		\multicolumn{1}{l}{\cellcolor{black!30}\textbf{Groupe I (contrôle)} }& $-11.2$ & $11.2$ \\ \hline		
		\multicolumn{1}{l}{\cellcolor{black!30}\textbf{Groupe II (test)}} & $11.2$ & $-11.2$  \\ \hline
		\end{tabular}
		\caption[]{Différence entre les valeurs attendues et les valeurs observées}
	\end{table}
	Nous comprenons alors mieux pourquoi le choix d'une cellule particulière n'a aucune influence pour faire test. On prend simplement celle qui nous arrange le plus (en fonction de ce qui va suivre ...).
	
	Prenant donc au hasard (puisque le choix n'affecte pas - du moins pour l'instant .... - le résultat du test comme nous l'avons montré dans nos calculs précédents) la colonne Survivants et en particulier le groupe observé Test. Sa valeur attendue est alors:
	\begin{table}[H]
		\begin{center}
			\definecolor{gris}{gray}{0.85}
			\begin{tabular}{|l|c|}
			\hline
			{\cellcolor{black!30}}  & 
	  \multicolumn{1}{c}{\cellcolor{black!30}\textbf{Survivants Observés}} \\ \hline	
			\multicolumn{1}{l}{\cellcolor{black!30}\textbf{Groupe II (test)}} & $E_{d,1}=\dfrac{L_{21}C_{21}}{n_1}=\dfrac{1400\cdot 2020}{2500}=1131.2$   \\ \hline
			\end{tabular}
			\caption[]{Valeur attendue de la cellule Test-Survivants}
		\end{center}
	\end{table}
	La différence entre observé et attendu donne:
	
	Cette différence sera évidemment nulle si l'observé était égal à l'attendu! Avec pour variance:
	
	Maintenant, sous les conditions:
	
	comme on l'a vu lors de notre étude de la loi hypergéométrique, nous pouvons l'approcher par une loi Normale (hypothèse H4).
	
	Donc dans notre cas, cette approximation n'est pas acceptable (la troisième condition est disqualifiante) et donc le test ne peut pas être effectué (on utilise généralement le tableau la colonne et la ligne qui permettent cette approximation puisque le choix n'influence pas sur la valeur du test mais sur l'autorisation d'utiliser l'approximation mentionnée!). Mais si tel avait été le cas, nous pourrions approximer la distribution hypergéométrique par une distribution Normale telle que:
	
	Ce qui peut évidemment se réduire à une distribution Normale réduite centrée si l'on prend pour la cellule de notre tableau:
	
	Et donc il suffit qu'une strate (couche) $i$ pour savoir si nous sommes en dehors de l'intervalle de confiance que nous avons fixé. Mais ... nous avons plusieurs strates (couches)! L'idée est alors de faire la somme sur les strates indépendantes $T$ (alors on retombe sur l'hypothèse H1) telle que:
	
	ce qui n'était pas tout à fait pratique à l'époque où tout le monde n'avait pas un ordinateur car seules les tables de $\mathcal{N}(0,1)$ étaient disponibles. C'est pourquoi nous préférons faire la somme suivante:
	
	Ou sous forme condensée (là encore quel que soit le choix de la cellule!):
	
	Et si les différences entre observées et attendues ne sont pas trop importantes dans toutes les strates, la valeur de cette expression se trouve bien dans un certain intervalle de confiance de la distribution Normale. S'il est à l'extérieur, l'hypothèse d'égalité des deux courbes de survie sera rejetée.
	
	Cependant, la plupart des logiciels statistiques prennent le carré de cette relation!

	Il vient alors que le carré suit une loi du khi-2 avec un degré de liberté (nous l'avons prouvé plus haut dans cette section) de telle sorte que nous retombons sur la forme finale du test du log-rank que l'on  retrouve dans de nombreux livres:
	
	Il s'agit donc d'une statistique Wald (voir plus haut) et donc le test CMH appartient à la famille des tests de Wald\index{tests de Wald}!

	Pour des raisons pratiques, nous ajoutons un terme de $0.5$ à la somme, ce qui fournit une meilleure approximation de la distribution Normale. On peut donc voir parfois dans les livres:
	
	\begin{tcolorbox}[colframe=black,colback=white,sharp corners]
	\textbf{{\Large \ding{45}}Exemple:}\\\\
	Considérons les valeurs observées suivantes de deux hôpitaux $ A $ et $ B $ pour un essai clinique:
	\begin{table}[H]
		\begin{center}
			\definecolor{gris}{gray}{0.85}
			\begin{tabular}{|l|c|c|c|}
			\hline
			{\cellcolor{black!30}}$A$  & \multicolumn{1}{c}{\cellcolor{black!30}\textbf{Décès Observés}}  & 
	  \multicolumn{1}{c}{\cellcolor{black!30}\textbf{Survivants Observés}} & 
	  \multicolumn{1}{c}{\cellcolor{black!30}\textbf{Total}} \\ \hline
			\multicolumn{1}{l}{\cellcolor{black!30}\textbf{Groupe I (contrôle)} }& $288$ & $4$ & $292$\\ \hline		
			\multicolumn{1}{l}{\cellcolor{black!30}\textbf{Groupe II (test)}} & $400$ & $50$ & $450$ \\ \hline
			\multicolumn{1}{l}{\cellcolor{black!30}\textbf{Total}} & $688$ & $54$ & $742$ \\ \hline
			\end{tabular}
		\end{center}
	\end{table}
	\begin{table}[H]
		\begin{center}
			\definecolor{gris}{gray}{0.85}
			\begin{tabular}{|l|c|c|c|}
			\hline
			{\cellcolor{black!30}}$B$  & \multicolumn{1}{c}{\cellcolor{black!30}\textbf{Décès Observés}}  & 
	  \multicolumn{1}{c}{\cellcolor{black!30}\textbf{Survivants Observés}} & 
	  \multicolumn{1}{c}{\cellcolor{black!30}\textbf{Total}} \\ \hline
			\multicolumn{1}{l}{\cellcolor{black!30}\textbf{Groupe I (contrôle)} }& $300$ & $10$ & $310$\\ \hline		
			\multicolumn{1}{l}{\cellcolor{black!30}\textbf{Groupe II (test)}} & $450$ & $40$ & $490$ \\ \hline
			\multicolumn{1}{l}{\cellcolor{black!30}\textbf{Total}} & $750$ & $50$ & $800$ \\ \hline
			\end{tabular}
		\end{center}
	\end{table}
	Nous ne voulons pas savoir si les deux hôpitaux sont différents ou non (ce n'est pas le but!), mais si les différences entre le groupe contrôle et le groupe test dans tous les hôpitaux sont significativement différentes ou non!
	\end{tcolorbox}
	
	\begin{tcolorbox}[colframe=black,colback=white,sharp corners]
	Eh bien, nous devinons déjà intuitivement le résultat lorsque nous voyons les valeurs ... mais faisons quand même les calculs.\\

	On voit déjà que pour l'hôpital $ A $, la colonne Survivants remplit les trois conditions d'approximation par une distribution Normale (ce qui n'est pas le cas pour la colonne des Décès):
	
	De même pour l'hôpital $ B $ (et c'est heureux car une fois une colonne sélectionnée dans une strate, il faut que la condition d'approximation soit applicable à la même colonne de toutes les autres strates !!!):
	
	\begin{table}[H]
		\centering
		\definecolor{gris}{gray}{0.85}
		\begin{tabular}{|l|c|c|}
		\hline
		{\cellcolor{black!30}}$A$  & \multicolumn{1}{c}{\cellcolor{black!30}\textbf{Survivants Observés}}  & 
  \multicolumn{1}{c}{\cellcolor{black!30}\textbf{Total}} \\ \hline
		\multicolumn{1}{l}{\cellcolor{black!30}\textbf{Groupe I (contrôle)} }& $21.25$ & $292$ \\ \hline		
		\multicolumn{1}{l}{\cellcolor{black!30}\textbf{Groupe II (test)}} & $32.75$ & $450$ \\ \hline
		\multicolumn{1}{l}{\cellcolor{black!30}\textbf{Total}} & $54$ & $742$ \\ \hline
		\end{tabular}
	\end{table}
	\begin{table}[H]
		\centering
		\definecolor{gris}{gray}{0.85}
		\begin{tabular}{|l|c|c|}
		\hline
		{\cellcolor{black!30}}$B$  & \multicolumn{1}{c}{\cellcolor{black!30}\textbf{Survivants Observés}}  & 
  \multicolumn{1}{c}{\cellcolor{black!30}\textbf{Total}} \\ \hline
		\multicolumn{1}{l}{\cellcolor{black!30}\textbf{Groupe I (contrôle)} }& $19.38$ & $310$ \\ \hline		
		\multicolumn{1}{l}{\cellcolor{black!30}\textbf{Groupe II (test)}} & $30.62$ & $490$ \\ \hline
		\multicolumn{1}{l}{\cellcolor{black!30}\textbf{Total}} & $50$ & $800$ \\ \hline
		\end{tabular}
	\end{table}
	Ce qui donne pour les différences:
	\begin{table}[H]
		\centering
		\definecolor{gris}{gray}{0.85}
		\begin{tabular}{|l|c|}
		\hline
		{\cellcolor{black!30}}$A$  & \multicolumn{1}{c}{\cellcolor{black!30}\textbf{Survivants Observés}}   \\ \hline
		\multicolumn{1}{l}{\cellcolor{black!30}\textbf{Groupe I (contrôle)} }& $-17.25$ \\ \hline		
		\multicolumn{1}{l}{\cellcolor{black!30}\textbf{Groupe II (test)}} & $17.25$ \\ \hline
		\end{tabular}
	\end{table}
	\begin{table}[H]
		\centering
		\definecolor{gris}{gray}{0.85}
		\begin{tabular}{|l|c|}
		\hline
		{\cellcolor{black!30}}$B$  & \multicolumn{1}{c}{\cellcolor{black!30}\textbf{Survivants Observés}}  \\ \hline
		\multicolumn{1}{l}{\cellcolor{black!30}\textbf{Groupe I (contrôle)} }& $-9.38$  \\ \hline		
		\multicolumn{1}{l}{\cellcolor{black!30}\textbf{Groupe II (test)}} & $9.38$ \\ \hline
		\end{tabular}
	\end{table}
	On voit alors vite pourquoi une fois la colonne choisie, le choix de la ligne n'a plus d'importance (soit une somme de valeurs négatives, soit une somme de valeurs positives et comme on prend le carré de la somme, ça ne change rien finalement!). Choisissons arbitrairement la deuxième ligne. Ensuite nous avons:
	\end{tcolorbox}
	
	\begin{tcolorbox}[colframe=black,colback=white,sharp corners]
	
	Et nous avons avec un tableur comme Microsoft Excel 14.0.6123 en unilatéral gauche avec un seuil de risque de $5\%$:
	\begin{center}
		$\chi_{95\%}^2(1)$\texttt{=LOI.KHIDEUX.INVERSE(95\%,1)}$\cong 3.841$
	\end{center}
	Ainsi, sur le cumul des deux hôpitaux (strates), le groupe témoin est significativement différent du groupe test. La $p$-valeur du test est généralement donnée avec Microsoft Excel 14.0.6123 par:
	\begin{center}
		\texttt{=1-LOI.KHIDEUX.N(31.845,1,1)}$\cong 0.000002 \%$
	\end{center}
	\end{tcolorbox}
	
	\subsubsection{Modèle de hasard proportionnel de Cox}
	Le "\NewTerm{modèle de hasard proportionnel de Cox}\index{modèle de hasard proportionnel de Cox}" (à ne pas confondre avec les "transformations Box-Cox") en temps continu, également connu sous le nom de "\NewTerm{odèle semi-paramétrique à risques proportionnels}\index{odèle semi-paramétrique à risques proportionnels}" ou en version abrégée "\NewTerm{HPM Cox}" ou encore "\NewTerm{modèle de régression de Cox}\index{modèle de régression de Cox}" (RCM), est un modèle d'analyse du ratio de risque. Il exprime le rapport du risque instantané, le "\NewTerm{taux de hasard}\index{taux de hasard}", pour connaître l'événement (risqué) étudié après une durée d'exposition donnée selon une combinaison linéaire de facteurs explicatifs. Ce modèle est largement utilisé dans les études cliniques, ainsi que dans la maintenance préventive et le risque de crédits bancaires.
	
	Ce modèle appartient comme le modèle de Kaplan-Meier aux "modèles de durée" ou "modèles de survie" et fait appel à certains des résultats démontrés dans la section Génie Industriel dans le cadre de l'étude de la maintenance préventive (voir page \pageref{Kaplan-Meier survival model}).
	
	\begin{tcolorbox}[title=Remarque,colframe=black,arc=10pt]
	Le modèle de Kaplan-Meier permet de comparer plusieurs courbes de survie, le modèle de Cox permet non seulement de faire de même mais aussi de pouvoir le faire sur la base de multiples facteurs et aussi de faire des prédictions pour des valeurs non observées lors de l'expérience.
	\end{tcolorbox}
	
	Dans un souci de simplification et essentiellement pédagogique, nous ne nous intéresserons ici que dans le cas où les données sont non censurées, les covariables sans interactions et elles-mêmes indépendantes du temps (par exemple, la covariable "âge" serait indépendante ... dans le temps...). Nous rendrons la présentation plus complexe au fur et à mesure que ce livre évoluera.
	
	Il faut tout d'abord savoir que le modèle de régression de Cox est, comme l'un de ses autres noms l'indique, basé sur le taux de hasard instantané (risque) $h(t)$! Pourquoi ce choix alors qu'on pourrait reprendre les fonctions de survie ou de fiabilité ??? Tout simplement parce que les fonctions de survie $R(t)$ et de fiabilité $F(t)$ ont des contraintes qui compliqueraient la construction du modèle de régression. Alors que la seule chose que nous imposons au taux de risque est qu'il soit compris entre $[0,+\infty[$.
	
	L'objectif du modèle de Cox est de pouvoir quantifier et tester les effets de caractéristiques individuelles telles que le sexe, le niveau d'éducation, la classe sociale, la nationalité, l'expérience passée, la vaccination, etc. sur les risques de transition (vie / décès ou malade / non malade, etc.).
	
	L'idée de base est de pouvoir dire \og le taux de risque d'un tel groupe est $X$ fois supérieur au taux de risque de l'autre groupe \fg{} ou en d'autres termes \og le risque d'un tel groupe est dans une proportion $X$ par rapport à un tel autre groupe \fg{} et une première approche naïve et mathématiquement simple pour gérer cela est de dire que le taux de risque est de la forme (dans le cas univarié \footnote{Donc, une première hypothèse est que nous supposons que les covariables continues ont une forme linéaire et cela doit toujours être vérifié!}):
	
	où $x$ est une variable explicative le plus souvent catégorique (et plutôt de type binaire) et où $\beta_0$ est donc in extenso le taux de risque correspondant à $x=0$.
	
	Contrairement au "risque relatif" (ie ratio de risque) ou au "ratio des cotes", c'est un modèle plus prédictif et adapté à plus de deux variables (et pas seulement binaires!).
	
	Cependant, c'est une mauvaise conception pour le moment pour la simple raison que le domaine de définition de $h(x)$  n'est en fait pas limité à $[0,+\infty[$. Une idée serait de faire comme pour le modèle logistique, prendre l'exponentielle tel que:
	
	Évidemment, nous pouvons avoir plus d'une variable explicative. Dès lors, comme nous le savons déjà, le modèle s'écrirait:
	
	où $h_0$ est évidemment le risque de base lorsque toutes les covariables sont nulles. En pratique, nous verrons un peu plus loin que la détermination de ce risque de base n'a aucun intérêt car il est éliminé. Cette dernière relation s'écrit parfois sous la forme suivante pour mettre en évidence la part de risque des facteurs caractéristiques des individus étudiés:
	
	Cependant, nous devons considérer que le taux de risque est dépendant du temps, comme nous l'avons vu précédemment dans notre étude des techniques de maintenance, de sorte qu'il faut définir le "\NewTerm{risque instantané}":
	
	\begin{tcolorbox}[title=Remarque,colframe=black,arc=10pt]
	Par conséquent, la situation standard de l'application de cette méthode de survie dans les projets de recherche clinique suppose qu'une population homogène est étudiée lorsqu'elle est soumise dans des conditions différentes (par exemple, traitement expérimental et traitement standard). Le modèle de survie suppose alors que les données de survie des différents patients sont indépendantes les unes des autres et que la distribution individuelle des temps de survie de chaque patient est la même (temps d'échec indépendants et répartis de manière identique)!\\
	
	Cependant, dans le domaine des essais cliniques, on observe dans la majorité des situations pratiques que les patients diffèrent considérablement. L'effet d'un médicament, d'un traitement ou l'influence de diverses variables explicatives peuvent différer considérablement entre les sous-groupes de patients. Pour tenir compte de cette hétérogénéité non observée dans la population étudiée, Vaupel et al. (1979) ont introduit les "\NewTerm{modèles de fragilité univariés}" dans l'analyse de survie (disponible dans le logiciel R). L'idée clé est que les individus possèdent des fragilités différentes et que les patients les plus fragiles mourront plus tôt que les autres. Par conséquent, une sélection systématique d'individus robustes (c'est-à-dire des patients à faible fragilité) a lieu. Nous n'introduirons pas ce modèle mathématique dans ce livre!
	\end{tcolorbox}
	Il est important de noter que dans un cas simple, nous avons:
	
	c'est pourquoi le modèle de Cox a toujours l'ordonnée à l'origine de $y$ (ie $\beta_0$) qui est nulle car elle est intégré dans le hasard fondamental  $h_0(t,\vec{\alpha})$.
	
	En réalité, si l'on prend du recul, on pourrait voir dans l'expression précédente une analogie avec l'approche par la méthode de séparation des variables telle qu'elle se fait assez souvent lors de la résolution de certaines équations différentielles où la solution est construite de telle sorte que chaque facteur dépend uniquement sur certains paramètres.
	
	Ce qui est particulièrement apprécié par les praticiens de ce modèle empirique et qui expliquent l'origine de son nom, c'est qu'on n'a pas besoin de connaître $h_0(t,\vec{\alpha})$. En effet, si l'on considère le ratio du même modèle mais pour différentes valeurs de covariables en même temps, on a le "\NewTerm{ratio de hasard}\index{ratio de hasard}":
	
	La plupart du temps, le ratio est du type suivant:
	
	On voit donc qu'une hypothèse de cette approche est que les deux modèles sont supposés avoir les mêmes propriétés initiales (hypothèse souvent cachée dans les études cliniques)!!
	\begin{tcolorbox}[colframe=black,colback=white,sharp corners]
	\textbf{{\Large \ding{45}}Exemples:}\\\\
	E1. Pour une covariable dichotomique unique, disons avec des valeurs $0$ et $1$, le rapport de risque est:
	
	Ainsi, le risque relatif de deux individus avec des valeurs de covariables différentes devrait être indépendant du temps ou constant à tout moment. Il s'agit d'une hypothèse inhérente au modèle de Cox (et à tout autre modèle à risques proportionnels).\\
	
	E2. Avec deux covariables, nous avons:
	
	Considérons maintenant par exemple que nous gardons  $x_1$ fixe tel que $x_1^{\prime}=x_1$ et que nous prenons $x_2^{\prime}=x_2+1$ ($x_2$ peut être une valeur entière comme l'âge arrondi du patient par exemple). Ensuite, la relation précédente s'écrira:
	
	Et notez que quelle que soit la valeur de $x_2$ (et en gardant fixes les autres covariables), chaque incrémentation de cette covariable par $+1$ multipliera le risque instantané par un facteur $e^\beta$ (cette valeur est souvent écrite en pourcentage évidemment).
	\end{tcolorbox}
	Le consensus est de lire les valeurs du rapport de risque comme suit:
	\begin{itemize}
		\item Si $\text{H.R.}=0.5$: À un moment donné, la moitié des patients du groupe de test subissent un événement par rapport au groupe témoin.

		\item Si $\text{H.R.}=1.0$: À un moment donné, les taux d'événements sont les mêmes dans les deux groupes.

		\item Si $\text{H.R.}=2.0$: À un moment donné, deux fois plus de patients dans le groupe de test subissent un événement que dans le groupe témoin.
	\end{itemize}
	En supposant que l'hypothèse précédente soit satisfaite, il est donc très intéressant dans la pratique du \og s'amuser \fg{} à varier les caractéristiques des individus afin de calculer comment la différence $(\vec{x}_i-\vec{x}_j)$ augmente ou diminue le ratio de risque en pourcentages (c'est là que réside le pouvoir d'investigation et le plaisir du modèle Cox!).
	
	Donc non seulement le risque de base est éliminé quand on regarde le risque relatif (c'est par hypothèse!!!), mais aussi la dépendance au temps!

	Donc l'hypothèse fondamentale du modèle de Cox, comme déjà mentionné, est que nous supposons pour différents groupes que le hasard de base est égal !!! Cette hypothèse de proportionnalité est trivialement visible lorsque nous utilisons des graphiques de survie qualitatifs comme ceux présentés plus avec deux groupes \footnote{La population à l'étude est considérée comme constituée de deux sous-populations avec des risques différents.} (c'est-à-dire deux strates) et nous voyons qu'elles commencent au même point à $t=0$.
	
	Donc, pour rappel, les hypothèses jusqu'à présent sont:
	\begin{enumerate}
		\item[H1.] Le modèle est multiplicatif (ie le lien entre $h(t)$ et $h_0(t)$ doit être de la forme $h(t)=c^{te}h_0(t)$)
		
		\item[H2.] Nous supposons que les covariables continues ont une forme linéaire
	
		\item[H3.] Nous supposons que les covariables sont indépendantes du temps (les hasards proportionnels doivent donc tenir!)
	
		\item[H4.] Nous supposons que le taux de risque (hasard de base) initial est égal
	\end{enumerate}
	Un test classique d'indépendance temporelle consiste à transformer les courbes de survie en prenant:
	
	et de vérifier si les deux courbes linéarisées résultantes sont parallèles. En effet, rappelons que nous avons \footnote{On change juste la notation pour la faire correspondre aux usagse dans le domaine médical, avec $S(t)=R(t)$ et $h(u)=\lambda(u)$} (\SeeChapter{voir la section Génie Industriel page \pageref{reliablity hazard failue rate}}):
	
	Et l'hypothèse de Cox est:
	
	Par conséquent:
	
	Prenant le logarithme naturel:
	
	et encore une fois:
	
	Donc pour deux séries d'observations:
	
	Ensuite, l'espace entre le logarithme naturel des deux courbes doit être constant dans le temps!
	\begin{tcolorbox}[title=Remarque,colframe=black,arc=10pt]
	En pratique, un test courant pour vérifier l'hypothèse de proportionnalité consiste à utiliser les "résidus de Schoenfeld" par rapport au temps transformé.
	\end{tcolorbox}
	Voyons donc un exemple de deux strates de régression du modèle Cox à PH pour faire un jugement qualitatif basé sur les graphiques:
	\begin{figure}[H]
		\centering
		\includegraphics[scale=0.6]{img/arithmetics/cox_ph_regression_qualitative_analysis_00.jpg}
	\end{figure}
	Le graphique ci-dessus (régression) enfreint les hypothèses suivantes:
	\begin{enumerate}
		\item La distance entre les courbes n'est sûrement pas constante lorsque les courbes se croisent
		
		\item Nous ne pouvons pas juger de la Normalité sous-jacente des données
		
		\item La courbe bleue ne semble pas être basée sur un modèle linéaire (que l'on prenne le logarithme naturel ou pas!)
		
		\item Sûrement le ratio de risque de hasard est égal au début (c'est assez mauvais) mais même pire ... il semble sur le long terme ne pas avoir de hazard risk ratio asymptotique constant
	\end{enumerate}
	Le graphique ci-dessous est un peu meilleur par rapport au respect de l'hypothèse du modèle:
	\begin{figure}[H]
		\centering
		\includegraphics[scale=0.6]{img/arithmetics/cox_ph_regression_qualitative_analysis_01.jpg}
	\end{figure}
	En effet, les courbes ne se croisent plus. Mais malheureusement, la distance ne semble pas constante.
	
	La régression suivante est meilleure que les deux précédentes (semble avoir une distance asymptotique constante\footnote{Nous pensons que le lecteur comprend maintenant que si nous pouvons par translation verticale superposer parfaitement les deux régressions dès lors l'hypothèse H3 tient!}):
	\begin{figure}[H]
		\centering
		\includegraphics[scale=0.6]{img/arithmetics/cox_ph_regression_qualitative_analysis_02.jpg}
	\end{figure}
	Bon c'est bien de pouvoir comparer logiquement le ratio du taux de hasard à un même instant temporel et pour différentes valeurs de covariables mais comment déterminer les coefficients du modèle ??? Un premier réflexe serait de prendre le logarithme naturel du modèle pour faire une régression linéaire simple:
	
	Mais on voit tout de suite que la composante temps sera problématique! Vient ensuite le deuxième réflexe: travailler avec le maximum de vraisemblance (voir plus haut page \pageref{likelihood estimators})! Mais alors une question se pose ...: la fonction de hasard n'est pas à proprement parler une probabilité alors que va-t-on maximiser ???
	
	Eh bien pour cela nous allons utiliser des probabilités conditionnelles (voir plus haut page \pageref{bayesian inference}), même si à première vue il n'est pas facile de voir ce qui relie les probabilités conditionnelles à la fonction de hasard mais attendez un peu, en notant $x_{i}$ le caractère particulier qui identifie une certaine classe d'individus.
	
	Considérons pour cela $E$, l'ensemble des éléments comprenant les individus qui ont survécu $S$ et $\bar{S}$ l'ensemble de ceux qui n'ont pas survécu. Tel que pour un temps identique donné:
	
	et tel que:
	
	Ainsi par construction:
	
	Puisque l'intersection est vide, il sera difficile de révéler la fonction de hasard à laquelle nous pourrions penser ... Et bien c'est là que vient un truc!
	
	Il est vrai que si l'on prend l'ensemble $\bar{S}$ de ceux qui n'ont pas survécu toujours au même instant que ceux qui ont survécu $S$ les deux ensembles seront toujours disjoints, par exemple en impliquant la variable temps , on a:
	
	Par contre, si on mélange les temps, on a par exemple:
	
	qui ne sont plus forcément vides puisqu'ils peuvent se chevaucher
partiellement pris à des moments différents!

	Puisqu'il y a une superposition des deux ensembles, lorsqu'ils sont pris à des moments différents, nous pouvons maintenant écrire de manière astucieuse une probabilité conditionnelle:
	
	En supposant que le nombre de survivants diminue (...), cette dernière relation peut s'écrire:
	
	Ou dans le language de tous les jours:
	
	dit autrement:
	
	Cependant, on reconnaît ici au numérateur, si l'on reprend les notations habituelles de la maintenance préventive (\SeeChapter{voir section Génie Industriel page  \pageref{preventive maintenance}}),  $\mathrm{d}F(t)$ la fonction de défaillance cumulée! Par conséquent:
	
	et on reconnaît $R(t)$, la fonction de fiabilité, au dénominateur tel que:
	
	Dès lors:
	
	Et nous reconnaissons ici un ratio que nous avons déjà rencontré de notre étude de la maintenance préventive !!! On peut donc écrire:
	
	On pourrait penser qu'il serait possible d'utiliser cette probabilité conditionnelle pour appliquer la technique du maximum de vraisemblance, mais ... le problème est qu'il y a implicitement dans $h(t)$, une fonction dépendant du temps qu'il faut éliminer avant histoire de pouvoir en tirer quelque chose. Dès lors, l'idée est de faire un ratio des fonctions de hasard pour un même instant donné pour que la partie temporelle soit éliminée! Et qui dit ratio de probabilités, redit probabilité conditionnelle!

	Rappelons que nous venons d'obtenir:
	
	Qui est la probabilité conditionnelle qu'un individu sans aucune caractéristique particulière ne survit pas pendant la tranche de temps donnée tandis que tous les autres (n'ayant pas non plus de caractéristiques spéciales) ainsi que lui-même ont survécu jusqu'au temps $t+\Delta t$. Maintenant, si nous introduisons les caractéristiques, alors nous avons la possibilité d'introduire une autre couche de probabilité conditionnelle. Pour cela, prenons le cas particulier mais assez facilement généralisable d'une seule caractéristique (covariable) particulière et d'un événement d'intérêt sur un individu noté $i$ parmi tous les autres individus $j$ (y compris l'individu $i$) qui sont également suivis avec ces mêmes caractéristiques. La probabilité conditionnelle précédente peut alors s'écrire:
	
	Maintenant, comparons cette probabilité conditionnelle d'un individu particulier conditionnellement à tous les autres individus (y compris $i$ lui-même pour rappel!) avec toutes les mêmes caractéristiques:
	
	\begin{tcolorbox}[title=Remarque,colframe=black,arc=10pt]
	Le lecteur attentif aura peut-être remarqué que le modèle tel qu'il est construit pour le moment suppose que les non-survivants sont uniques pour une valeur donnée de la caractéristique. C'est pourquoi nous parlons de: "Modèle de régression à risque proportionnel de Cox à covariables non groupées, non censurées et indépendantes du temps". Cependant, nous verrons ci-dessous comment traiter le cas où deux individus aux caractéristiques différentes ou égales ne survivent pas en même temps.
	\end{tcolorbox}
	Mais dans le cas particulier qui nous concerne $\{\bar{S}_{t+\Delta t}/S_t\}_i\in \{\bar{S}_{t+\Delta t}/S_t\}_j$. Nous avons alors:
	
	Mais que nous pouvons aussi écrire plus explicitement (n'oubliez pas qu'ici $j$ représente un ensemble d'individus différents de $i$) car chaque événement individuel  y relatif est supposé comme une probabilité disjointe (\SeeChapter{voir section Probabilités page \pageref{disjoint probability}}):
	
	Mais le lecteur doit garder à l'esprit deux choses
	\begin{enumerate}
		\item La flèche du temps est orientée
		\item Par conséquent, les événements sont ordonnés!
	\end{enumerate}
	Qu'est-ce que cela signifie ou quelles en sont les implications? Juste que l'événement lié à un individu $i$ ne peut se produire qu'avant qu'un autre événement ne se soit produit pour un autre individu avec un indice de rang inférieur à $i$. Cela signifie que nous devrions écrire à la place:
	
	ou plus explicitement:
	
	Cette relation est souvent notée comme suit dans les manuels:
	
	et est la "probabilité de risque" de l'individu $i$ au point temporel $t_i$, relativement à la covariable $x$ (covariable unique dansce cas particulier!), où les événements individuels sont disjoints avec un risque initial proportionnel et où $\mathcal{R}(t_i)$ est l'ensemble de tous les individus survivants ou fonctionnant au temps $t_i$ (risque fixé à $t_i$).
	
	La dernière relation explique la probabilité que nous utiliserons pour maximiser la vraisemblance. Puisque le hasard de base dépendant du temps n'apparaît plus, il est d'usage de parler de "\NewTerm{vraisemblance partielle}\index{vraisemblance partielle}" de $n$ individus\footnote{Parce que nous supprimons les instants réels des événements même s'ils sont connus, nous utilisons le nom de "vraisemblance partielle"}:
	
	Ou dans certains manuels, cela est aussi noté (y compris dans le cas de plusieurs covariables et d'éventuelles données censurées):
	
	où la puissance $\delta_i$ signifie que les individus ayant les événements (échec / décès) contribuent réellement à la probabilité, mais pas les cas censurés à droite.
	
	Nous pouvons considérer la vraisemblance partielle comme la fonction de densité conjointe pour les rangs des sujets en termes d'ordre des événements, s'il n'y avait pas de censure et pas de temps d'événements liés.
	
	Puisque la vraisemblance ne peut être maximisée qu'avec des méthodes numériques (type méthode de Newton), il est d'usage de ne pas aller plus loin en prenant le logarithme naturel (mais cela ne changerait pas le résultat!). Mais en pratique, le coefficient est cependant estimé en minimisant le négatif de la log-vraisemblance:
	
	Cette notation n'est toujours pas facile à utiliser et à appréhender dans la pratique. Pour cette raison, nous allons faire un exemple numérique.
	\begin{tcolorbox}[colframe=black,colback=white,sharp corners]
	\textbf{{\Large \ding{45}}Exemple:}\\\\
	Pour voir comment, considérons les données ci-dessous avec une seule covariable (décès dans l'étude clinique dans les semaines après le début du test basé sur l'IMC\footnote{Indice de Masse Corporel}):
	\begin{table}[H]
		\centering
		\begin{tabular}{|c|c|c|}
		\hline
		\rowcolor[HTML]{C0C0C0} 
		\multicolumn{1}{|l|}{\cellcolor[HTML]{C0C0C0}\textbf{$\pmb{i}$}} & \multicolumn{1}{l|}{\cellcolor[HTML]{C0C0C0}\textbf{$\pmb{t_i}$ {[}semaines{]}}} & \multicolumn{1}{l|}{\cellcolor[HTML]{C0C0C0}\textbf{IMC $\pmb{x_i}$}} \\ \hline
		$1$ & $6$ & $31.4$ \\ \hline
		$2$ & $98$ & $21.5$ \\ \hline
		$3$ & $189$ & $27.1$ \\ \hline
		$4$ & $374$ & $22.7$ \\ \hline
		\end{tabular}
	\end{table}
	\begin{tcolorbox}[title=Remarque,colframe=black,arc=10pt]
	Comme il n'y a manifestement pas de variables qualitatives dans ce petit exemple, nous parlons de "modèle de risque proportionnel de Cox dépendant du temps non stratifié".
	\end{tcolorbox}
	Il y a donc quatre événements avec les données respectives de l'IMC $31.4$, $21.4$, $27.1$ et $22.7$. Comme nous l'avons vu, le temps n'intervient plus mais seulement le rang $i$, c'est-à-dire l'ordre dans lequel se déroulent les événements d'intérêt.\\
	
	Il correspond alors quatre probabilités conditionnelles que nous désignerons par souci de simplification par l'écriture $P_1$, $P_2$, $P_3$, $P_4$.
	\end{tcolorbox}
	
	\begin{tcolorbox}[colframe=black,colback=white,sharp corners]
	
	On a alors la vraisemblance partielle qui est donnée par:
	
	Et une application numérique qui maximise cela avec le solveur du tableur Microsoft Excel (ou autre feuille de calcul ou outil) donne immédiatement:
	
	Ceci est cohérent avec ce que renvoie des logiciels de statistiques comme MedCalc 12.7.5 ou R (voir dans le livre compagnon correspondant)! Cela nous donne:
	
	De là, on peut facilement comparer à un moment donné le rapport de vraisemblance entre deux individus aux caractéristiques différentes.\\
	
	Nous pouvons également donner le tracé de la courbe de survie correspondante (ici réalisé avec R et incluant l'intervalle de confiance que nous étudierons plus loin):
	\begin{figure}[H]
		\centering
		\includegraphics[scale=0.65]{img/arithmetics/cox_ph_r_application_confidence_interval.jpg}
		\caption{Courbes de survie typiques du modèle de Cox à hasards proportionnels}
	\end{figure}
	\end{tcolorbox}
	Nous remarquons après ce bref aperçu qu'à aucun moment nous n'avons eu besoin de connaître ou d'assumer la loi de distribution sous-jacente (Weibull, Exponentielle ou autre) et leurs paramètres associés. Le modèle de Cox est donc une technique non paramétrique.
	
	Il existe (bien qu'à la base ce ne soit pas le but du modèle de Cox), des estimateurs de $h_0(t)$, dont le plus connu est l'estimateur de Breslow, mais il aurait été comme inconsistent) et applicable uniquement pour le taux de risque cumulé et non le taux instantané (donc méfiez-vous des logiciels qui donnent $h_0(t)$...). Et l'intérêt du modèle de Cox est de s'intéresser au ratio de risque et non à estimer le hasard de la fonction survie / fiabilité complète (Kaplan-Meier est alors mieux adapté). Donc, en fait, il n'est pas prévu d'avoir une introduction détaillée de ces estimateurs (et de quelques autres) dans ce livre!
	
	\begin{tcolorbox}[title=Remarque,colframe=black,arc=10pt]
	Dans le logiciel R, la fonction \texttt{basehaz()} utilise par défaut l'estimation de Nelson-Aalen du hasard cumulatif avec une estimation de survie de type Breslow. Alternativement, un estimateur de Kaplan-Meier pour le risque cumulatif peut être sélectionné.
	\end{tcolorbox}
	
	Parlons maintenant un peu des événements simultanés! En principe, les durées sont supposées être les réalisations d'une variable continue et les covariables aussi (cette variable sous-jacente continue étant évidemment le temps), la probabilité que deux ou plusieurs individus connaissent l'événement au même moment est théoriquement nulle. En pratique, les données sont souvent enregistrées selon une certaine division temporelle (données hebdomadaires, mensuelles, etc.) de sorte qu'il est courant pour une même durée de survie de caractériser plusieurs individus différents.
	
	Fondamentalement, cela ne complique pas sérieusement l'écriture précédente, simplement le nombre de calculs peut devenir tel si l'on utilise la contribution exacte de ces individus à la probabilité partielle que des approximations ont été proposées (disponibles tous les trois les plus fameuses dans le logiciel SAS). Cependant, dans le cadre de ce livre, nous nous concentrerons sur la version exacte!
	
	Supposons donc que depuis un certain temps, deux individus aient connu l'intérêt de l'événement et qu'en même temps, trois autres individus soient également à risque. Si nous commençons par traiter le cas du premier individu, alors il y a $5$  individus à risque. Quand on traite le cas du deuxième individu, seuls $4$ individus sont à risque puisque le premier a déjà été pris en compte. Dans ce cas, la probabilité d'avoir les deux événements simultanément est consistante avec celle de $2$ événements non-incompatibles mais indépendants (\SeeChapter{voir section Probabilités page \pageref{disjoint probability}}):
	
	Naturellement, le fait de commencer par le premier individu est arbitraire puisqu'on peut tout aussi bien admettre de commencer par le second. Au total, la contribution de cette durée de deux individus à la vraisemblance partielle est alors de:
	
	Notons (pour ceux qui pourraient avoir un doute que c'est toujours plus petit ou égal à l'unité comme l'exige l'axiome des probabilités) que dans le cas où nous avons une population à deux individus cela se réduit à:
	
	ce qui est cohérent avec l'axiome de certains événements dans le domaine des probabilités! Et lorsque la population est plus importante, il est immédiat que le dénominateur de chaque terme aussi sera aussi plus grand, la probabilité totale est alors encore inférieure ou égale à l'unité!

	Si nous répétons la manœuvre avec trois individus simultanément, nous aurons $6$ termes qui s'additionneront. Plus généralement, s'il y a $k$ individus parmi $n$ qui vivent l'événement d'intérêt en même temps, le calcul de la contribution fait intervenir $k!$ termes. Le temps de calcul peut donc devenir très pénalisant, par exemple avec $k$ valant $10$, on a $3,628,800$ termes...
	
	Maintenant, comme pour l'estimateur de survie de Kaplan-Meier, voyons comment nous pouvons construire les intervalles de confiance pour le ratio hasard.

	Comme précédemment, nous utiliserons à nouveau la méthode Delta (\SeeChapter{voir section Méthodes Numériques page \pageref{delta method}}) pour obtenir l'erreur standard pour $e^{\hat{\beta}}$. Rappelons que cette méthode nous donne que:
	
	Cela donne immédiatement pour la variance:
	
	Par conséquent, en supposant que les $\beta$ sont normalement distribués (et sont un scalaire...) :
	
	Le problème ici est que nous pouvons avoir la borne inférieure de $e^{\beta}$ qui peut être négative ...
	
	Par conséquent, nous préférons utiliser:
	
	Si nous considérons des individus identiques à l'exception d'une covariable $x$, le rapport de risque devient comme nous le savons déjà:
	
	L'intervalle de confiance du ration de hasard est alors:
	
	\begin{tcolorbox}[colframe=black,colback=white,sharp corners]
	\textbf{{\Large \ding{45}}Exemple:}\\\\
	Un modèle de risque proportionnel de Cox a été utilisé pour modéliser les temps de survie des patients atteints de cancer. La taille de la tumeur (en [mm]) a été incluse comme covariable, avec le coefficient $\beta$. L'estimation du maximum de vraisemblance de $\beta$ a donné $\hat{\beta}= 0,0176$ avec une erreur standard de $0.004$.\\
	
	Nous voulons calculer une estimation du rapport de risque entre deux individus avec des tumeurs mesurant respectivement $46$ [mm] et $37$ [mm] qui sont identiques à d'autres égards et construire un intervalle de confiance de $95\%$ pour le ratio de hasard.\\
	
	Nous avons pour le ratio de hasard:
	
	En utilisant l'estimation:
	
	L'intervalle de confiance à $95\%$ pour $\beta$ est alors:
	
	C'est-à-dire:
	
	L'intervalle de confiance à $95\%$ pour le ratio de hasard est alors:
	
	\end{tcolorbox}
	
	\paragraph{Modèle de Cox stratifié}\index{Modèle de Cox stratifié}\mbox{}\\\\
	Supposons que les observations d'une variable définissent des sous-catégories de la population initiale. Un exemple évident est donné par une covariable de sexe codée $0$ ou $1$ (dans le cas simple...).
	
	Supposons encore que l'on ne s'intéresse pas à une mesure directe de l'impact de cette variable sur le risque ou que l'on pense que cette variable ne vérifie pas l'hypothèse des risques proportionnels de sorte que l'on refuse son inclusion dans la liste des covariables (autrement dit, à la place de supposer que la covariable binaire \textit{sexe} sur le risque de survie est constante dans le temps et multiplicative sur le risque, on peut supposer que c'est le risque de base qui est différent selon les hommes et les femmes et donc on a deux strates différentes!).
	
	Il est cependant possible que l'on veuille prendre en compte un effet du sexe des individus pour mesurer l'impact des autres variables. La solution comme déjà mentionnée dans la parenthèse précédente est alors de stratifier. Dans ce cas les coefficients explicatifs sont contraints d'être identiques dans chacune des strates mais une fonction de vraisemblance partielle est construite séparément sur chacune d'elles, la fonction maximisée pour trouver les estimateurs de paramètres étant le produit de ces vraisemblances partielles. En conséquence, évidemment, le risque de base diffère entre les strates. Par exemple, au lieu d'avoir (cas particulier simple avec deux covariables):
	
	où $x_2$ serait la covariable \textit{sexe}, nous avons plutôt le système suivant :
	
	Pour résumé:
	
	Ainsi, pour deux individus de la même strate, nous aurons plus de chance d'avoir un risque proportionnel (courbes de survie parallèles) car le rapport annule le hasard de base. En revanche, il est évident que la comparaison de deux individus de strates différentes aura très peu de chance d'être indépendante du temps puisque le rapport des hasards de base ne s'annulera pas.
	
	Cependant, si les deux s'annulent, nous avons une sorte d'évidence qu'ils sont proportionnels et qu'il est raisonnable de modéliser le terme de stratification de manière régulière!
	
	Dès lors les "\NewTerm{modèles de Cox stratifiés}":
	
	où $h_{0k}$ est le hasard de base pour la strate $k$ sont une extension utile des modèles de Cox standard (où pour le rappel $h_0$ est commun à tous les individus d'une étude):
	
	pour permettre des covariables avec des risques non proportionnels.
	
	La vraisemblance partielle que nous avons introduite plus haut :
	
	est ensuite calculé pour le modèle stratifié comme :
	
	Enfin, dans le cas de la stratification, la variable responsable de la stratification n'étant pas explicitement mesurée, on ne peut pas l'utiliser comme variable explicative.
	
	Alors, que pouvons-nous résumer jusqu'ici? Nous avons vu trois sortes de modèles avec pour chacun au moins un modèle mathématique particulier (mais il y en a bien plus en réalité quand on devient un expert dans le domaine comme nous l'avons déjà montré dans une figure plus haut résumant toutes les méthodes d'analyse de survie les plus connues ! ):
	\begin{itemize}
		\item Modèles paramétriques: où le hasard de base et les covariables sont tous supposés avoir les mêmes distributions (par exemple, les modèles de Weibull ou exponentiels la plupart du temps).
	
		\item Modèles non paramétriques : lorsque le hasard de base et les covariables n'ont pas de distribution spécifique supposée (par exemple, modèle de Kaplan-Meier).
	
		\item Modèles semi-paramétriques : lorsque le hasard de base n'a pas de distribution spéciale supposée mais que les covariables ont une certaine distribution supposée (par exemple, le modèle de risque proportionnel de Cox).
	\end{itemize}
	
	\pagebreak
	\subsection{Propagation des Erreurs (analyse d'incertitude expérimentale)}
	Il est impossible de connaître (mesurer) la valeur exacte d'une grandeur physique expérimentalement, il est donc très important de déterminer son incertitude.
	
	Nous nommons bien évidemment "\NewTerm{erreur}\index{erreur}", la différence entre la valeur mesurée et la valeur exacte. Cependant, comme nous ne connaissons pas la valeur exacte, nous ne pouvons de toute façon pas connaître l'erreur... Le résultat est encore incertain. C'est pourquoi on parle de "\NewTerm{incertitude de mesure}\index{incertitude de mesure}".
	
	On distingue deux grands types d'incertitudes:
	\begin{enumerate}	
		\item Les "\NewTerm{erreurs systématiques}\index{erreurs systématiques}" : elles affectent le résultat et ce constamment dans le même sens (erreurs des appareils de mesure, limites de précision, etc.). Il faut alors éliminer, ou corriger le résultat, si possible!

		\item Les "\NewTerm{erreurs accidenteles}\index{erreurs accidenteles}" (statistique) : nous devons répéter les mesures et calculer l'incertitude moyenne d'estimation à l'aide de techniques statistiques.
	\end{enumerate}	
	\begin{figure}[H]
		\centering
		\includegraphics{img/arithmetics/errors_propagation.jpg}
		\caption[]{Propagation des erreurs (source: xkcd)}
	\end{figure}
	
	\subsubsection{Incertitudes absolues et relatives (calcul direct du biais)}
	Si la vraie valeur d'une variable est $x$ (théoriquement supposée connue) et la valeur mesurée $x_0$, alors $\delta x_0$ est "\NewTerm{l'incertitude absolue}\index{incertitude absolue}" (incertitude due à des appareils de mesure) ou "\NewTerm{erreur absolue}\index{erreur absolue}".

	L'intervalle de fluctuation est donc noté :
	
	ou:
	
	"\NewTerm{L'incertitude}\index{incertitude}" ou "\NewTerm{erreur relative}\index{erreur relative}" est elle-même définie par:
	
	L'incertitude absolue est utilisée pour trouver une approximation du dernier chiffre significatif de celle-ci. Par contre, lorsque l'on veut comparer deux mesures ayant des incertitudes absolues pour identifier quelle était la plus grande marge d'erreur, on calcule l'incertitude relative de ce nombre en divisant l'incertitude absolue par la mesure elle-même, et on transforme le résultat typiquement en pourcentages.
	
	En d'autres termes, l'incertitude relative donne une idée de la précision de la mesure en \%. Si nous faisons une mesure avec une incertitude absolue de $1$ [mm], nous ne saurons pas si c'est une bonne mesure ou non. Cela dépend si on mesure la taille d'une pièce de monnaie, de notre voisin, ou de la distance Paris-Marseille de la distance Terre-Lune. Bref, elle dépend de l'incertitude relative (c'est-à-dire du rapport de l'incertitude absolue et de la mesure)!
	
	Donc dans le cas d'une loi du type $f(x,y,z,t)$ dont on cherche l'erreur on calculerait (une fois avec tous les termes positifs et une autre fois avec tous les termes négatifs) :
	
	Cette méthode est assez ennuyeuse car elle doit être calculée pour chaque valeur $(x,y,z,t)$.
	
	\subsubsection{Erreurs statistiques}
	Dans la plupart des mesures, nous pouvons estimer l'erreur due à des phénomènes aléatoires, nommés "\NewTerm{erreurs aléatoires}\index{erreurs aléatoires}", par une série de $n$ mesures $x_1,x_2,...,x_i, . .., x_n$ et ceci par opposition à "\NewTerm{l'erreur systématique}\index{erreur systématique}" qui est la partie non aléatoire de l'erreur.
	
	L'erreur aléatoire permet d'introduire les concepts de :
	\begin{itemize}
		\item Répétabilité : Elle est définie comme l'étroitesse de l'accord entre les résultats des mesures successives d'un même élément, effectuées avec la même méthode, par le même opérateur avec les mêmes instruments de mesure, dans le même laboratoire (conditions) et dans un intervalle de temps assez court (voir un peu plus loin une définition un peu plus rigoureuse conforme aux normes internationales).
		
		\item Reproductibilité (parfois appelée "justesse") : qui est définie comme l'étroitesse de l'accord entre les résultats des mesures successives de la même quantité, dans le cas où des mesures individuelles sont effectuées : par des méthodes différentes, en utilisant des instruments utilisés par différents opérateurs de laboratoires !
	\end{itemize}
	Ces deux types d'erreurs sont presque toujours regroupés sous les étiquettes "\NewTerm{R\&R}" ou "\NewTerm{étude R\&R}\index{\'etude R et R}" dans l'industrie. En général, l'accord est moins bon en matière de reproductibilité.
	
	\begin{tcolorbox}[title=Remarque,colframe=black,arc=10pt]
	Il existe des logiciels exécutant une ANOVA à facteurs fixse bidirectionnelle avec répétition comme Minitab qui génèrent des rapports très détaillés pour l'analyse R\&R.
	\end{tcolorbox}	
	Ces deux types d'erreurs peuvent être illustrés par le tir à la cible de façon plus générale :
	\begin{figure}[H]
		\centering
		\includegraphics{img/arithmetics/type_of_errors.jpg}
		\caption{Type d'erreurs de mesure}
	\end{figure}
	Comme nous l'avons vu précédemment, la valeur moyenne arithmétique sera dans le cas univarié :
	
	et l'écart-type (estimateur biaisé comme prouvé précédemment) toujours dans le cas univarié (le $n$ième cas a déjà été prouvé lors de notre étude détaillée de la variance) :
	
	et l'écart type non biaisé (comme cela a également été prouvé précédemment) :
	
	et nous avons prouvé que l'écart type de la moyenne était donné par (sous certaines hypothèses !) :
	
	et comme nous l'avons prouvé, après un grand nombre de mesures indépendantes et identiquement distribuées, la distribution des erreurs sur une mesure suit une distribution Normale de sorte qu'on peut écrire pour l'intervalle de fluctuation (si on n'a pas assez de mesure, on utilise alors l'intervalle de fluctuation sur la base de la loi de Student) :
	
	Bref, on peut utiliser tous les outils statistiques vus jusqu'à présent dans le domaine de la mesure en laboratoire ou ailleurs !
	
	Le résultat d'une mesure (ou d'une estimation et ce même dans le domaine de la gestion de projets !!!) doit comporter rigoureusement au moins quatre éléments. Par exemple pour une erreur symétrique :
	
	ou pour une asymétrique :
	
	Où nous avons :
	\begin{enumerate}
		\item La valeur numérique avec le nombre correct de décimales.

		\item Unité de mesure selon le système de mesure international.

		\item Incertitude élargie de $k\cdot \sigma$ (intervalle de fluctuation)

		\item La valeur entière de $k$ utilisée pour l'intervalle de fluctuation.
	\end{enumerate}
	Cette méthode est bien plus utile que la précédente car nous n'avons pas besoin de la recalculer à chaque fois pour chaque valeur expérimentale. C'est cette méthode qu'utilise par exemple le célèbre CERN (Organisation Européenne pour la Recherche Nucléaire)\footnote{Ajoutez à cela la méthodologie de mesure Six Sigma étudiée dans la section Génie Industriel}.
	
	\begin{tcolorbox}[colback=red!5,borderline={1mm}{2mm}{red!5},arc=0mm,boxrule=0pt]
	\bcbombe Attention! Il arrive de très nombreuses fois... dans les articles scientifiques et aussi dans les manuels scolaires qu'une mesure soit fournie sans indication de son approximation. Il est alors généralement admis en pratique que l'incertitude absolue n'est pas supérieure à une unité de l'ordre de puissance indiqué par le dernier chiffre.
	\end{tcolorbox}
	
	\subsubsection{Répétabilité}
	La répétabilité $r$, différence vraisemblable de mesure entre deux mesures d'objets similaires dans un même laboratoire dans des conditions opératoires similaires, est définie normativement (dans les normes ISO 5725:1987 et AFNOR NF X 06-041) dans le cas particulier unidimensionnel (univariée) par :
	
	où $p$ est une probabilité à valeur élevée, généralement égale à $95\%$ et $X_1,X_2$ deux variables aléatoires indépendantes et identiquement distribuées selon une distribution Normale avec une moyenne inconnue et une variance inconnue $\mathcal{N}(\mu ,\sigma)$. Par la stabilité de la distribution Normale, alors il vient :
	
	Mais, nous avons vu plus haut dans cette section dans le cadre de l'étude de l'intervalle de confiance de la moyenne que:
	
	Donc textuellement :
	
	et donc à l'aide des tables (ou logiciels), on a :
	
	et dès lors:
	
	Soit avec la notation respectant les normes des laboratoires :
	
	Mais dans le cas présent, nous avons une double variance. Il vient donc :
	
	\\
	On retombe donc sur la relation disponible dans les normes avec le fameux coefficient de $2.77$. Évidemment après il est clair que la valeur de $r$ doit être minimisée !
	
	\subsubsection{Propagation des erreurs (approximation linéarisée)}
	Soit une mesure $x+\delta x$ et $y=f(x)$ une fonction de $x$. Quelle est l'incertitude sur $y$ it si nous ne connaissons que l'incertitude d'un appareil de mesure, mais qui ne serait pas donnée comme un écart type statistique ?
	
	Dans ce type de situation on parle de "\NewTerm{mesure indirecte}\index{mesure indirecte}". C'est typiquement le cas si l'on veut mesurer une intensité électrique $I$ en la mesurant indirectement en faisant le rapport de la tension $U$ par la résistance $R$ utilisée pour la mesure comme $I=U/R$. Il est en effet évident que dans cette dernière situation on ne peut pas faire la somme de l'incertitude de tension et de l'incertitude de résistance car le système n'est pas homogène au niveau des unités !!!
	
	Lorsque $\delta x$ est petit, $f(x)$ est remplacé au voisinage de $x$ par sa tangente (c'est simplement la dérivée bien sûr !):
	
	mais si $y$ dépend de plusieurs variables $x, z, t$ mesurées avec les incertitudes $\delta x,\delta z,\delta t$ :
	
	l'erreur maximale possible est alors la différentielle totale exacte (\SeeChapter{voir section Calcul Différentiel et Intégral \pageref{total exact differential}}) :
	
	En utilisant les séries d'expansion de Taylor du premier ordre (\SeeChapter{voir section Séquences et Séries page \pageref{taylor series}}) nous pouvons écrire :
	
	Ce que nous désignons souvent comme la somme des dérivées partielles avec leur incertitude respective :
	
	et cela fonctionne très bien tant que les incréments $\Delta x$ sont suffisamment petits. Même les fonctions très courbées sont presque linéaires sur une région suffisamment petite!
	
	Cette dernière relation est la "\NewTerm{loi de propagation}\index{loi de propagation}" du problème étudié. La dérivée partielle en facteur de l'incertitude est dans la science de la mesure, nommée le "\NewTerm{coefficient d'incertitude}\index{coefficient d'incertitude}".
	
	Bien sûr, nous pouvons également écrire pour obtenir l'erreur relative :
	
	Ce qui nous amène à :
	
	et:
	
	L'incertitude relative d'un produit ou d'un quotient de valeurs incertaines est donc égale à la somme des incertitudes relatives. Il est donc clair qu'une opération mathématique ne peut pas améliorer l'incertitude des données.
	
	De ce qui précède on en déduit immédiatement :
	
	et dès lors:
	
	Voici un résumé de cela sous forme de tableau (avec un petit changement dans les notations):
	\begin{table}[H]
		\centering
		\label{my-label}
		\begin{tabular}{|c|c|c|c|c|}
		\hline
		\rowcolor[HTML]{C0C0C0} 
		$\pmb{z=}$ & $\pmb{x+y}$ & $\pmb{x-y}$ & $\pmb{xy}$ & $\pmb{\dfrac{x}{y}}$ \\ \hline
		\cellcolor[HTML]{C0C0C0}$\pmb{\Delta z=}$ & $\Delta x+\Delta y$ & $\Delta x+\Delta y$ & $x\Delta y+y\Delta x$ & $\dfrac{x \Delta y+y \Delta x}{y^2}$ \\ \hline
		\cellcolor[HTML]{C0C0C0}$\pmb{\dfrac{\Delta z}{z}=}$ & $\dfrac{\Delta x+\Delta y}{x+y}$ & $\dfrac{\Delta x+\Delta y}{x-y}$ & $\dfrac{x\Delta y+y\Delta x}{xy}=\dfrac{\Delta x}{x}+\dfrac{\Delta y}{y}$ & $\dfrac{\Delta x}{x}+\dfrac{\Delta y}{y}$ \\ \hline
		\end{tabular}
		\caption{Relations communes d'erreurs absolues et relatives}
	\end{table}
	\begin{tcolorbox}[title=Remarque,colframe=black,arc=10pt]
	Le résultat d'une multiplication, d'une division, d'une soustraction ou d'une addition est arrondi à autant de chiffres significatifs que la donnée qui en a le plus petit nombre.
	\end{tcolorbox}	
	Évidemment, cette loi de propagation (linéaire) n'est valable que dans la plage où la fonction peut être approchée comme linéaire. Soyez donc prudent dans son utilisation ! Sinon, nous devons prendre une approximation en série de Taylor d'ordre supérieur.
	
	Si l'incertitude des mesures est donnée sous forme statistique (écart-type), il est donc évident que l'on utilisera les propriétés de variance déjà vues au début de cette section... au moins pour les cas simples!
	
	Encore une fois le point négatif de cette méthode par rapport à la méthode statistique est qu'il faut calculer l'erreur pour chaque mesure...
	
	\begin{tcolorbox}[colframe=black,colback=white,sharp corners]
	\textbf{{\Large \ding{45}}Exemple:}\\\\
	On veut calculer l'incertitude de la mesure de la densité d'un cube d'un matériau donné. On sait alors que :
	
	\end{tcolorbox}
	
	\pagebreak
	\subsubsection{Propagation des erreurs (approche statistique)}
	Enfin, il y a un point supplémentaire à discuter : supposons que nous mesurions une quantité $u$ plusieurs fois, ou par plusieurs méthodes différentes, et pour chaque mesure $u_i$ nous estimons son incertitude $\sigma_i$. Les $\sigma_i$ ne sont pas nécessairement égaux ; certaines mesures seront meilleures que d'autres, en raison de la plus grande taille des échantillons (plus de répétitions), ou en raison d'autres facteurs, comme un meilleur appareil de messure. Comment déterminons-nous notre meilleure estimation de $u$ et comment trouvons-nous l'incertitude dans cette estimation ?
	
	Par exemple, supposons qu'une longueur $x$ soit mesurée par une personne $n_1$ fois et par une autre personne $n_2$ fois, de sorte que la première personne trouve :
	
	tandis que la deuxième personne trouve :
	
	Ici, $\sigma_1$ est l'incertitude de $u_1$, $\sigma_2$ est l'incertitude de $u_2$ et $\sigma^2$ est la variance de population des valeurs de $x$. Comment $u_1$ et $u_2$ doivent-ils être combinés pour obtenir un $u$ global, et quelle est l'incertitude dans ce $\bar{u}$ final ? Puisque $n_1=\sigma^2/\sigma_1^2$ et $n_2=\sigma^2/\sigma_2^2$, nous avons:
	
	avec par définition:
	
	En général, s'il y a $n$ valeurs de $u$, voici le résultat généralisé, sous une forme qui ne dépend que de chaque $u_k$ et de son incertitude $\sigma_k$ :
	
	Notez comment plus de mesures, ou des mesures plus précises, réduisent l'incertitude en augmentant sa réciproque.

	\pagebreak
	\begin{tcolorbox}[colframe=black,colback=white,sharp corners]
	\textbf{{\Large \ding{45}}Exemple:}\\\\
	Le domaine de la physique des particules offre de nombreuses situations où les relations précédentes ainsi que le $\chi^2$ peuvent être appliqués. Un exemple particulièrement simple\footnote{Cet exemple a été fourni par Patricia Burchat, membre de la faculté de physique de l'UCSC de 1988 à 1994, et maintenant à Stanford} implique des mesures de la masse $M_Z$ du boson $Z^0$ par des groupes expérimentaux au CERN. Les résultats des mesures de $M_Z$ effectuées par quatre détecteurs différents (L3, OPAL, Aleph et Delphi) sont les suivants\footnote{Des mesures plus récentes sont répertoriées sur \url{http://pdg.lbl.gov/2014/listings /rpp2014-list-z-boson.pdf}} :
	\begin{table}[H]
		\centering
		\begin{tabular}{|l|l|}
		\hline
		\rowcolor[HTML]{9B9B9B} 
		\textbf{Détecteur} & \textbf{Masse en GeV/$c^2$} \\ \hline
		L3 & $91.161 \pm 0.013$ \\ \hline
		OPAL & $91.174 \pm 0.011$ \\ \hline
		Aleph & $91.186 \pm 0.013$ \\ \hline
		Delphi & $91.188 \pm 0.013$ \\ \hline
		\end{tabular}
	\end{table}
	Les incertitudes répertoriées sont des estimations de $\sigma_i$, les écarts-types pour chacune des mesures. La figure ci-dessous montre ces mesures tracées sur une échelle de masse horizontale (tournée verticalement pour plus de clarté):
	\begin{figure}[H]
		\centering
		\includegraphics[scale=0.75]{img/arithmetics/propagation_error_mass_boson_cern.jpg}
	\end{figure}
	La question se pose : ces données peuvent-elles être bien décrites par un chiffre unique, à savoir une estimation de $M_Z$ faite en déterminant la moyenne pondérée des quatre mesures ?\\
	
	Nous savons que nous pouvons calculer la moyenne pondérée $\bar{M}_Z$, et son écart type $\sigma_{\bar{M}_Z}$ comme ceci :
	
	pour trouver:
	
	Alors nous pouvons construire le $\chi^2$:
	\end{tcolorbox}
	
	\begin{tcolorbox}[colframe=black,colback=white,sharp corners]
	
	Nous savons que nous devons nous attendre à ce que cette valeur de $\chi^2$ soit tirée d'une distribution du khi-2 $3$ degrés de liberté.\\
	
	Cela correspond à une $p$-valeur de $0.426803$. Le résultat n'est pas significatif à $p<0.05$. On ne peut donc pas rejeter l'hypothèse nulle selon laquelle les quatre mesures de la masse du boson $Z^0$ sont cohérentes entre elles.
	\end{tcolorbox}
	
	\subsubsection{Chiffres significatifs}
	Dans les petites écoles (et parfois aux niveaus supérieurs et dans les entreprises), il est nécessaire de transformer une mesure exprimée dans une certaine unité en une autre unité.
	
	Par exemple, en prenant les tables, on peut avoir la conversion suivante :
	
	Vient ensuite la question (que l'étudiant ou le praticien a peut-être oublié...). En partant d'une mesure avec une précision d'environ $1$ [lb] (donc de l'ordre de $0.5$ [kg]), une simple conversion d'unité pourrait-elle conduire à une précision de $1/10$ [mg] ????
	
	De cet exemple, il est nécessaire de retenir qu'une marge d'incertitude est associée à toutes les valeurs mesurées et à toutes les valeurs calculées à partir des valeurs mesurées.
	
	Dans les sciences exactes (et aussi les soft skills comme le management), tout raisonnement, toute analyse doit tenir compte de cette incertitude !!!
	
	Mais pourquoi certains chiffres sont-ils significatifs et d'autres non ? Car en sciences on ne rapporte que ce qui a été observé objectivement (principe d'objectivité). En conséquence, nous limitons l'écriture d'un nombre aux décimales  raisonnablement fiables malgré l'incertitude : les chiffres significatifs ! La précision que pourraient alors sembler apporter des chiffres supplémentaires est alors illusoire.
	
	Il faut alors connaître les arrondis selon quelques règles et conventions :
	\begin{itemize}
		\item Lorsque le chiffre du rang le plus élevé supprimé est supérieur à $5$, le dernier chiffre est augmenté de $1$ (exemple : $12,66$ s'arrondit à $12.7$). Dans la version francophone de Microsoft Excel 11.8346, ceci est faisable avec :
		\begin{center}
		\texttt{=ARRONDI(12.66,1)=12.7}
		\end{center}
		
		\item Lorsque le chiffre du rang le plus élevé supprimé est inférieur à $5$, le chiffre précédent reste inchangé (exemple : $12.64$ s'arrondit à $12.6$). Dans la version francophone de Microsoft Excel 11.8346 cela donne:
		\begin{center}
		\texttt{=ROUND(12.64,1)=12.6}
		\end{center}
		
		\item Lorsque le chiffre du rang le plus élevé supprimé est égal à $5$ alors si l'un des chiffres qui suivent n'est pas zéro, le chiffre précédent est augmenté de $1$ (exemple : $12.502$ arrondit à $12,7$). Dans la version francophone de Microsoft Excel 11.8346, ceci est faisable avec :
		\begin{center}
		\texttt{=ARRONDI(12.6502,1)=12.7}
		\end{center}
		
		\item Lorsque le chiffre du rang le plus élevé supprimé est égal à $5$ (qui n'est suivi d'aucun nombre !) ou n'est suivi que de zéros, on augmente de $1$ le chiffre précédent du nombre arrondi s'il est impair, sinon on le laissez inchangé (exemples : $12.75$ arrondit à $12.8$ et $12.65$ à $12.6$). Dans ce dernier cas, le dernier chiffre du nombre arrondi sera toujours un nombre pair. Les tableurs ne respectent pas vraiment cette dernière règle, en fait avec la version francophone de Microsoft Excel 11.8346 on a :
		\begin{center}
		\texttt{=ARRONDI(12.75,1)=12.8\\
		=ARRONDI(12.65,1)=12.7}
		\end{center}
	\end{itemize}
	En fait, dans la pratique ces règles sont rarement respectées car les logiciels (principalement les logiciels de type tableurs) n'intègrent pas de manière appropriée ces règles. Il est alors d'usage de se contenter d'arrondir à la valeur décimale la plus proche.
	
	Les chiffres significatifs d'une valeur comprennent tous ses chiffres déterminés avec certitude et le premier chiffre qui porte une incertitude (ce dernier significatif occupe le même rang que l'ordre de grandeur de l'incertitude!).
	
	Souvent, les sources de données ne mentionnent pas d'intervalle de fluctuation (c'est-à-dire l'indication $\pm \ldots$). Par exemple, lorsqu'on écrit $m=25,4\;\text{[kg]}$ on considère classiquement que l'incertitude est du même ordre de grandeur que le rang du dernier chiffre significatif (donc : la décimale incertaine!).
	
	En fait, seule la décimale de l'incertitude est implicite : la marge réelle est indéterminée.
	
	Cependant, des informations supplémentaires sur la précision peuvent être transmises par des notations supplémentaires. Il est souvent utile de savoir à quel point le ou les chiffres finaux sont exacts. Par exemple, la valeur acceptée de l'unité de charge électrique élémentaire peut être correctement exprimée sous la forme $1.602176487(40)\cdot 10^{-19}$ [C], qui est un raccourci pour $1.602176487\pm 0.000000040 \cdot 10^{-19} $ [C].
	
	\pagebreak
	\subsection{Statistiques spatiales}
	Les statistiques descriptives spatiales sont utilisées à diverses fins en géographie, en particulier dans les analyses de données quantitatives impliquant des systèmes d'information géographique (SIG).
	
	\subsubsection{Modèle de distance spatiale $2$D de Poisson}
	C'est au début des années 1920 qu'apparaissent les premières publications scientifiques (par des environnementalistes) concernant des tests statistiques de distribution spatiale aléatoire des organismes (analyse de motifs ponctuels spatiaux). Mais ce type de statistique est également utilisé en Marketing pour déterminer si la répartition des clients est aléatoire ou non ou en Sciences Sociales pour déterminer si les catégories sociales de la population humaine sont bien mélangées ou non.
	
	\textbf{Définition (\#\mydef):} La "\NewTerm{Structure spatiale totalement aléatoire}\index{structure spatiale totalement aléatoire}" (SPTA) décrit un processus ponctuel par lequel des événements ponctuels se produisent dans une zone d'étude donnée de manière complètement aléatoire. Elle est synonyme de "\NewTerm{processus de Poisson spatial homogène}\index{processus de Poisson spatial homogène}". Un tel processus est modélisé en utilisant un seul paramètre $\rho$ , c'est-à-dire la densité de points dans la zone définie.
	
	Dans les statistiques spatiales et les domaines connexes, un "\NewTerm{Processus de Poisson}\index{orocessus de Poisson}" (également nommé "\NewTerm{champ aléatoire ponctuel de Poisson}\index{champ aléatoire ponctuel de Poisson}") est un type d'objet mathématique aléatoire qui se compose de points situés au hasard sur un espace mathématique. Ce processus ponctuel a des propriétés mathématiques pratiques, ce qui l'a amené à être fréquemment défini dans l'espace euclidien et utilisé comme modèle mathématique pour des processus apparemment aléatoires dans de nombreuses disciplines.
	
	Les données sous la forme d'un ensemble de points, irrégulièrement répartis dans une région de l'espace, surviennent dans de nombreux contextes différents ; les exemples incluent les emplacements d'arbres dans une forêt, de nids d'oiseaux, de noyaux dans des tissus, de personnes malades dans une population à risque. Nous appelons un tel ensemble de données un "motif de points spatiaux" et appelons les emplacements des "événements", pour les distinguer des points arbitraires de la région en question. L'hypothèse de la structure spatiale totalement aléatoire pour un modèle de points spatial affirme que le nombre d'événements dans n'importe quelle région suit une distribution de Poisson avec un nombre moyen donné par subdivision uniforme. Les événements d'un motif sont répartis indépendamment et uniformément dans l'espace ; en d'autres termes, les événements sont également susceptibles de se produire n'importe où et n'interagissent pas les uns avec les autres!
	
	Les modèles typiques de distances moyennes aléatoires sont basés sur trois hypothèses fondamentales (hypothèse) :
	\begin{itemize}
		\item[H1.] La probabilité qu'un élément se trouve dans une région de surface $C$ incluse dans une zone plus grande $S$, telle que $C\subset S$, est proportionnelle à la surface $C$ telle que :
		
		cette hypothèse semble être nommée "\NewTerm{principe spatial de Laplace}\index{principe spatial de Laplace}".
	
	 	\item[H2.] La position des éléments dans une surface $S$ n'est pas influencée par ses voisins dans la même surface (pas de corrélation !).
	
		\item[H3.] Si la probabilité qu'un élément soit dans la surface $C$ est donnée selon H1 par :
		
		Alors la probabilité qu'il n'y soit pas est donnée par :
		
	\end{itemize}
	A partir de ces trois hypothèses, on peut alors affirmer (mais en pratique il faut vérifier que cela est satisfait!) que la probabilité de trouver $k$ éléments dans une surface $C\subset S$ sur un total de $n$ points est alors donnée par une loi binomiale :
	
	Dans une population de $n$ éléments inclus dans une surface $S$ avec donc une densité $\rho=n/S$ la distance $r_i$ entre chaque élément et son plus proche voisin est mesurée et on note $\bar{r} $ la distance arithmétique moyenne telle que :
	
	\begin{theorem}
	Nous allons prouver dans ce qui suit que si la distribution des éléments est réellement aléatoire et non corrélée, alors la moyenne attendue (espérance) de la distance est :
	
	Et qu'alors le "\NewTerm{critère de ratio de randomisation}\index{critère de ratio de randomisation}" :
	
	est utilisé comme mesure du degré d'écart de la distribution aléatoire. Si le ratio tend vers $1$, alors la distribution sera considérée comme aléatoire et non corrélée. Si les éléments sont très agrégés (corrélés), alors ce ratio tendra vers $0$.
	\end{theorem} 
	\begin{dem}
	Pour cela nous partons de :
	
	Nous rappelons que nous avons prouvé que la distribution binomiale tend vers une distribution de Poisson :
	
	avec $\mu=np$ et $n\rightarrow +\infty$ pour de petites valeurs de $k$.
	
	Si une surface principale $S$ supposée être un disque de rayon $R$ (oui ... en biologie les récipients passés au microscope sont souvent circulaires afin d'éviter les effets de coins) est découpée en $D$ surfaces secondaires égales, alors nous pouvons écrire:
	
	Ce qui donne, en analysant les unités, le nombre moyen de points par coupe. Dès lors:
	
	est la probabilité de trouver $k$ éléments sur une surface aléatoire de dimension $\pi R^2 D^{-1}$. Si nous posons $k=0$, alors nous avons la probabilité suivante de trouver zéro élément dans une surface aléatoire de dimension $\pi R^2 D^{-1}$ :
	
	Respectivement si on définit un disque centré sur un élément existant, alors on a la probabilité qu'aucun autre élément ne se trouve dans le disque de rayon $R$ qui est égale à :
	
	En conséquence:
	
	Est la probabilité qu'un voisin le plus proche se trouve au-delà de la distance $R$. En prenant la différentielle de cette dernière expression, on a alors la probabilité infinitésimale en fonction de la distance d'avoir un plus proche voisin :
	
	Par conséquent, la moyenne attendue de la distance au plus proche voisin d'un élément centré dans un disque de rayon $R$ allant de $0$ à $+\infty$ est :
	
	Posons:
	
	Nous avons alors:
	
	Nous avons déjà calculé cette intégrale lors de notre étude de la loi normale (Gauss-Laplace) et nous avons prouvé en intégrant par parties que :
	
	et dès lors:
	
	et comme en pratique, on a souvent $D=1$, il vient alors que :
	
	\begin{flushright}
		$\blacksquare$  Q.E.D.
	\end{flushright}
	\end{dem}
	
	\pagebreak
	\subsection{Sondages}
	La "\NewTerm{méthodologie des sondages}\index{méthodologie des sondages}" étudie l'échantillonnage d'unités individuelles d'une population et les techniques de collecte de données d'enquête associées, telles que la construction de questionnaires et les méthodes permettant d'améliorer le nombre et la précision des réponses aux enquêtes. La méthodologie d'enquête comprend des instruments ou des procédures qui posent une ou plusieurs questions auxquelles on peut ou non répondre.

	Les enquêtes statistiques sont entreprises dans le but de faire des inférences statistiques sur la population étudiée, et cela dépend fortement des questions et techniques d'enquêtes utilisées. Les sondages sur l'opinion publique, les enquêtes de santé publique, les études de marché, les enquêtes gouvernementales et les recensements sont tous des exemples de recherches quantitatives qui utilisent une méthodologie d'enquête contemporaine pour répondre à des questions sur une population. Bien que les recensements ne comprennent pas "d'échantillon", ils incluent d'autres aspects de la méthodologie d'enquête, comme les questionnaires, les enquêteurs et les techniques de suivi des non-réponses. Les enquêtes fournissent des informations importantes pour toutes sortes de domaines d'information publique et de recherche, par exemple, la recherche en marketing, la psychologie, les professionnels de la santé et la sociologie.
	
	Une enquête unique est constituée d'au moins un échantillon (ou de la population complète dans le cas d'un recensement), d'une méthode de collecte de données (par exemple, un questionnaire) et de questions individuelles ou d'éléments qui deviennent des données pouvant être analysées statistiquement. Une seule enquête peut se concentrer sur différents types de sujets tels que les préférences (par exemple, pour un candidat présidentiel), les opinions (par exemple, l'avortement devrait-il être légal ?), le comportement (tabagisme et consommation d'alcool) ou des informations factuelles (par exemple, le revenu), en fonction de son objectif. Étant donné que la recherche par sondage est presque toujours basée sur un échantillon de la population, le succès de la recherche dépend de la représentativité de l'échantillon par rapport à une population cible d'intérêt pour le chercheur. Cette population cible peut aller de la population générale d'un pays donné à des groupes spécifiques de personnes dans ce pays, à une liste de membres d'une organisation professionnelle ou à une liste d'élèves inscrits dans un système scolaire (voir également échantillonnage (statistiques) et échantillonnage d'enquêtes ). Les personnes qui répondent à une enquête sont appelées "répondants" et, selon les questions posées, leurs réponses peuvent se présenter en tant qu'individus, leurs ménages, employeurs ou autre organisation qu'ils représentent.
	
	La méthodologie d'enquête en tant que domaine scientifique cherche à identifier des principes concernant la conception de l'échantillon, les instruments de collecte de données, l'ajustement statistique des données et le traitement des données, et l'analyse finale des données qui peuvent créer des erreurs d'enquête systématiques et aléatoires. Les erreurs d'enquête sont parfois analysées en lien avec le coût de l'enquête. Les contraintes de coût sont parfois présentées comme l'amélioration de la qualité dans le cadre des contraintes de coût, ou alternativement, la réduction des coûts pour un niveau de qualité fixe. La méthodologie d'enquête est à la fois un domaine scientifique et une profession, ce qui signifie que certains professionnels du domaine se concentrent empiriquement sur les erreurs d'enquête et que d'autres conçoivent des enquêtes pour réduire ces mêmes erreurs. Pour les concepteurs d'enquêtes, la tâche consiste à prendre un grand nombre de décisions concernant des milliers de caractéristiques individuelles d'une enquête afin de l'améliorer.
	
	Les défis méthodologiques les plus importants d'un méthodologiste d'enquête comprennent la prise de décisions sur la façon de :
	\begin{itemize}
		\item Identifier et sélectionner les membres potentiels de l'échantillon
		\item Contacter les individus échantillonnés et collecter des données auprès de ceux qui sont difficiles à atteindre (ou réticents à répondre)
		\item Évaluer et tester les questions
		\item Sélectionnez le mode de formulation des questions et de collecte des réponses
		\item Former et superviser les enquêteurs (s'ils sont impliqués)
		\item Vérifiez l'exactitude et la cohérence interne des fichiers de données
		\item Ajuster les estimations de l'enquête pour corriger les erreurs identifiées
	\end{itemize}
	C'est une tendance courante que beaucoup de gens douteraient des résultats d'une enquête, à moins qu'ils ne trouvent la preuve que l'enquête a été réalisée « scientifiquement ». Eh bien, mener une enquête nécessite l'utilisation du processus scientifique, un cheminement qui est essentiellement suivi par toutes les types de recherchse. En gardant cela à l'esprit, une enquête qui a franchi de manière critique les étapes du processus scientifique postule un pourcentage plus élevé de validité et de fiabilité des résultats qu'un micro-troittor ou l'opinion des amis ou collègues au bar PMU, à la cantine ou sur les réseaus sociaux...

	Toutes les enquêtes ne peuvent pas être menées de manière à ce que chaque membre de la population puisse être étudié, car ce serait une manière très coûteuse et donc peu pratique de faire des recherches par enquête. Lors de l'exécution d'une enquête, le chercheur sélectionnera les participants au moyen d'une technique d'échantillonnage aléatoire particulière, et ces personnes seront les représentants de l'ensemble de la population cible. L'utilisation d'une méthode d'échantillonnage aléatoire ne signifie pas que l'enquête n'est pas scientifique ; au contraire, cela augmente la validité des résultats car le biais dans le choix des participants est éliminé, rendant ainsi le processus scientifique et les résultats valides. Évidemment, il y a toujours des pièges (le plus connu étant qu'un pourcentage important de personnes aiment répondre au hasard et bêtement à des sondages, de 3\%$ à 30\%$ pour le pire des cas liés à la politique, surtout lorsqu'il s'agit d'extrémisme).
	
	Il existe deux types de questions différentes qui peuvent être utilisées pour collecter des informations. Le premier est nommé "\NewTerm{question à réponse structurée ou fixe}" et le second est nommé "\NewTerm{question non structurée ou ouverte}" et fait souvent référence au concept de "témoignage". Il est important de comprendre quand et comment utiliser ces questions lors de la conception d'une enquête.

	La validité et la fiabilité sont également souvent discutées dans le domaine de la psychométrie, mais pas tellement dans les études de marché, même si on suppose qu'elles sont présentes.
	
	L'enquête mesure-t-elle ce qui doit être mesuré? C'est la question à laquelle on ne peut répondre qu'en vérifiant la "validité" de l'enquête. Dans la recherche scientifique, la validité nous indique la précision de l'enquête en vérifiant la représentativité de l'échantillon et la précision des questions. Il existe quatre types importants de validité de la recherche par sondage :
	\begin{itemize}
		\item Validité apparente : les questions semblent-elles raisonnables pour acquérir les données que vous souhaitez collecter ?
		
		\item Validité du contenu : les questions portent-elles uniquement sur le problème et d'autres sujets qui s'y rapportent ?
		
		\item Validité interne : les questions impliquent-elles le résultat que vous souhaitez obtenir de l'enquête ?
		
		\item Validité externe : les questions suscitent-elles des réponses généralisables (c'est-à-dire reflètent la réponse de l'ensemble de la population cible) ?
	\end{itemize}
	Dans la méthodologie d'enquête, la "fiabilité" fait référence au fait que les questions suscitent des informations similaires ou la même caractéristique même si les formulations ou les structures du questionnaire sont modifiées. La fiabilité de l'enquête se rapporte à la cohérence des questions et des déclarations dans un questionnaire.
	
	Un bon exemple de la pire chose qui puisse être faite comme l'a fait tout récemment l'un des principaux animateurs du téléjournal de la Suisse romande est d'utiliser Twitter pour obtenir des témoignages pour un documentaire (voir ci-dessous sa demande de témoignages et mes conseils pour éviter Twitter pour faire une telle chose car cette méthode se réduit uniquement à la cueillette des cerises et n'a absolument aucune valeur...) :
	\begin{figure}[H]
		\centering
		\includegraphics[scale=0.75]{img/arithmetics/testimonial_twitter.jpg}
	\end{figure}
	C'est pourquoi également pour éviter les fausses nouvelles, ou les informations très biaisées, le journaliste doit avoir un minimum de connaissances scientifiques. Surtout parce que la plupart d'entre eux pensent que collecter des informations auprès d'une population est quelque chose du niveau de la maternelle et que les enquêtes qui sont devenues depuis la fin du 20ème siècle d'une importance primordiale, notamment en marketing (Machine Learning), en médecine, en politique et en psychologie et de mauvaises méthodologies peuvent avoir des conséquences économiques ou sociales à l'échelle mondiale.
	
	\textbf{Définitions (\#\mydef):}
	\begin{enumerate}
		\item[D1.] Une "\NewTerm{population}\index{population}" est un ensemble fini d'objets sur lesquels une étude est effectuée. Ces objets sont nommés "\NewTerm{individus / unités statistiques}". Une population est notée $U=\{u_1,\ldots,u_N\}$ (l'origine du "$U$" est celle de "Univers" comme pour les probabilités) où $N$ est le nombre d'individus dans la population et $i\in \{1,\ldots,N\}$ et où $u_i$ est le $i$-ème individu.
	
		\item[D2.] Nous nommons "\NewTerm{base du sondage}\index{base du sondage}" une liste qui répertorie tous les individus d'une population.
	
		\item[D3.] Un "\NewTerm{caractère}" est une qualité qui est étudiée chez les individus d'une population. Un caractère est noté $Y$. Pour tout $i\in\{1,\ldots,N\}$, on note $y_i$ la valeur de $Y$ pour l'individu $u_i$.
	
		\item[D4.] Une statistique basée sur l'ensemble de la population est en méthodologie d'enquête notée par exemple $\bar{y}_U$ pour la moyenne, ou $\sigma_U$ pour l'écart type et ainsi de suite !
	\end{enumerate}
	Pour calculer/évaluer les paramètres de la population, deux méthodes courantes sont utilisées dans la méthodologie d'enquête :
	\begin{itemize}
		\item Le "\NewTerm{recensement}\index{recensement}" : nous avons accès à tous les individus et nous pouvons mesurer les valeurs de $Y$ pour chacun d'eux. Cependant, cela n'est pas toujours possible pour des raisons de coût, de temps ou à cause de certaines contraintes comme la destruction des individus étudiés.
	
		\item Le "\NewTerm{sondage}\index{sondage}" : les valeurs de $Y$ sont étudiées sur un sous-ensemble d'individus de la population.\\
	\end{itemize}
	
	\textbf{Définition (\#\mydef):} Un échantillon est un sous-ensemble d'individus d'une population. Un échantillon est noté dans la méthodologie d'enquête $\omega$!

	Pour constituer un échantillon représentatif de la population, au moins deux questions se posent :
	\begin{itemize}
		\item Comment procédons-nous?
		
		\item Combien d'individus faut-il choisir ?\\
	\end{itemize}
	Ce qui suit vise à apporter des réponses à ces questions !
	
	\pagebreak
	\subsubsection{Plans d'enquêtes}
	\textbf{Définitions (\#\mydef):}
	
	Un "\NewTerm{plan d'enquête}\index{plan d'enquête}" ou plus communément un "\NewTerm{plan d'échantillonnage}\index{plan d'échantillonnage}" est une procédure de sélection d'un échantillon dans une population. Un plan d'échantillonnage est dit :
	\begin{itemize}
		\item[D1.] Un "\NewTerm{échantillon aléatoire}" si chaque individu de la population a une probabilité connue d'être dans l'échantillon.
	
		\item[D2.] Un "\NewTerm{échantillon simple}" si chaque individu a la même probabilité qu'un autre d'être sélectionné. Les probabilités sont alors égales et un tel plan est noté EAPE pour "échantillon aléatoire avec probabilités égales".
	
		\item[D3.] Un "\NewTerm{échantillon aléatoire sans retirage}", abrégé EASR, si un seul individu ne peut apparaître qu'une seule fois dans l'échantillon.
	
		\item[D4.] Un "\NewTerm{échantillon aléatoire avec retirage}", abrégé EAAR, si le même individu peut apparaître plusieurs fois dans l'échantillon et si l'ordre dans lequel les individus apparaissent compte.
	\end{itemize}
	\begin{tcolorbox}[title=Remarque,colframe=black,arc=10pt]
	Les formules d'estimation habituelles dans les méthodologie d'enquête sont associées à un plan d'échantillonnage aléatoire avec retirage (EAAR). Pour simplifier la situation, ils sont généralement utilisés avec le cas sans retirage (EASR) lorsque $n$ est beaucoup plus petit que $N$. Une convention existante est $N\ge 10n$.
	\end{tcolorbox}
	En fait, il y a beaucoup plus de méthodes d'échantillonnage (effet Dunning-Kruger comme toujours...1). En voici une vue assez exhaustive :
	\begin{figure}[H]
		\centering
		\includegraphics[scale=0.65]{img/arithmetics/sampling_methods.jpg}	
		\caption{Méthodes d'échantillonnage}
	\end{figure}
	Plus en détails (ils utilisent presque tous es techniques mathématiques avancées) :
	\begin{itemize}
	\item Techniques d'échantillonnage non probabiliste\index{techniques d'échantillonnage non probabiliste}:
		\begin{itemize}
			\item "\NewTerm{Échantillonnage par quotas}\index{\'echantilonnage par quotas}": dans l'échantillonnage par quotas, une population est d'abord segmentée en sous-groupes mutuellement exclusifs. Ensuite, le jugement est utilisé pour sélectionner les sujets ou les unités de chaque segment en fonction d'une proportion spécifiée sans respecter les autres attributs du groupe (propriétés). Ce qui le rend non aléatoire, c'est que c'est l'humain qui sélectionnera les éléments des groupes.
			
			\item "\NewTerm{Échantillonnage au jugement}\index{\'echantillonnage au jugement}": Dans l'échantillonnage de jugement, un échantillon non aléatoire est sélectionné sur la base de l'opinion d'un expert. Les résultats obtenus à partir d'un échantillon de jugement sont sujets à un certain degré de biais, en raison du fait que la base et la population ne sont pas identiques (en proportion ET en attributs), mais cela peut également être une bonne méthode de double contrôle après un échantillonnage de type aléatoire. Le cadre est une liste de toutes les unités, éléments, personnes, etc., qui définissent la population à étudier. Il est également assimilé à un échantillonnage aléatoire car l'expert essaiera de le rendre aléatoire sans utiliser l'ordinateur, mais en fait l'échantillon sera très différent d'un échantillon aléatoire réel.

			\item "\NewTerm{Échantillonnage de commodité}\index{\'echantillonnage de commodit\'e}\index{\'echantillonnage par opportunit\'e}": Un échantillonnage de commodité est un type de méthode d'échantillonnage non probabiliste où l'échantillon est tiré d'un groupe de personnes faciles à contacter ou à atteindre. Par exemple, se tenir dans un centre commercial ou une épicerie et demander aux gens de répondre à des questions serait un exemple d'échantillon de commodité. Ce type d'échantillonnage est également connu sous le nom "d'échantillonnage instantané" ou "d'échantillonnage de disponibilité". Il n'y a pas d'autres critères à cette méthode d'échantillonnage si ce n'est que les gens soient disponibles et disposés à participer. De plus, ce type de méthode d'échantillonnage ne nécessite pas la génération d'un échantillon aléatoire simple, puisque le seul critère est de savoir si les participants acceptent de participer.
			
			\item "\NewTerm{Échantillonnage boule de neighe}\index{\'echantillonnage boule de neige}\index{\'echantillonnage par volontaires}":  Un échantillonnage boule de neige est une technique où les sujets d'étude existants recrutent de futurs sujets parmi leurs connaissances. Ainsi, on dit que le groupe échantillon grandit comme une boule de neige roulante. Au fur et à mesure que l'échantillon s'accumule, suffisamment de données sont recueillies pour être utiles à la recherche. Cette technique d'échantillonnage est souvent utilisée dans des populations cachées, comme les toxicomanes ou les professionnel(le)s du sexe, qui sont difficiles d'accès pour les chercheurs. Comme les membres de l'échantillon ne sont pas sélectionnés à partir d'une base de sondage, les échantillons boule de neige sont sujets à de nombreux biais.
		\end{itemize}
	\item Techniques d'échantillonnage probabiliste\index{techniques d'\'echantillonnage probabiliste}:
		\begin{itemize}
			\item "\NewTerm{Échantillonnage aléatoire simple}\index{\'echantillonnage al\'eatoire simple}": Un échantillon aléatoire simple est un sous-ensemble d'individus (un échantillon) choisi dans un ensemble plus large (une population). Chaque individu est choisi au hasard et entièrement par hasard, de sorte que chaque individu a la même probabilité d'être choisi à n'importe quelle étape du processus d'échantillonnage, et chaque sous-ensemble de $k$ individus a la même probabilité d'être choisi pour l'échantillon que n'importe quel autre sous-ensemble de $k$ individus.
			
			\item "\NewTerm{Échantillonnage systématique}\index{\'echantillonnage systématique}": L'échantillonnage séquentiel est une technique d'échantillonnage non probabiliste dans laquelle nous choisissons un seul ou un groupe de sujets dans un intervalle de temps donné, nous menons l'étude, analysons les résultats puis choisissons un autre groupe de sujets si nécessaire et ainsi de suite. Il est utilisé par les médecins mais aussi par les ingénieurs qualité industriels (notamment pour les cartes de contrôle).
			
			\item "\NewTerm{Échantillonnage aléatoire stratifié}\index{\'echantillonnage aléatoire stratifié}": L'échantillonnage aléatoire stratifié est une méthode d'échantillonnage qui implique la division d'une population en groupes plus petits appelés "\NewTerm{strates}\index{strates}". Dans l'échantillonnage aléatoire stratifié, les strates sont formées en fonction d'attributs ou de caractéristiques partagés par les membres. Un échantillon aléatoire de chaque strate est prélevé en nombre proportionnel à la taille de la strate par rapport à la population. Ces sous-ensembles de strates sont ensuite regroupés pour former un échantillon aléatoire.
			
			\item "\NewTerm{Échantillonnage en grappes}\index{\'echantillonnage en grappes}": L'échantillonnage en grappes est un plan d'échantillonnage utilisé lorsque des groupements mutuellement homogènes mais internes hétérogènes sont évidents dans une population statistique. Il est souvent utilisé dans les études de marché. Dans ce plan d'échantillonnage, la population totale est divisée en ces groupes (appelés grappes) et un échantillon aléatoire simple des groupes est sélectionné. Les éléments de chaque grappe sont ensuite échantillonnés. Si tous les éléments de chaque grappe échantillonnée sont échantillonnés, il s'agit alors d'un plan d'échantillonnage en grappe à "un niveau". Si un sous-échantillon aléatoire simple d'éléments est sélectionné dans chacun de ces groupes, on parle de plan d'échantillonnage en grappes à "deux niveaux". 
			\includegraphics[scale=0.8]{img/arithmetics/stratified_cluster_sampling.jpg}
			
			\item "\NewTerm{Échantillonnage aléatoire restreint}\index{\'echantillonnage al\'eatoire restreint}": Il n'y a pas de consensus général sur la définition au jour où nous écrivons ces lignes. Mais dans le cadre de ce livre, un échantillonnage aléatoire restreint sera lié à un échantillonnage avec retirage d'une matrice de plan d'expérience où le but est d'éviter les cofacteurs indésirables (variables cachées).
			
			\item "\NewTerm{Échantillonnage proportionnel à la taille}\index{\'echantillonnage proportionnel à la taille}": Lorsque des informations sur une mesure d'amplitude $G$ existent pour chaque élément de la population et que cette mesure de taille stocke des informations précieuses sur l'importance de l'élément $i$ à inclure dans l'échantillon, nous pouvons utiliser ces informations dans le plan d'échantillonnage. Les plans d'échantillonnage qui utilisent explicitement de telles mesures d'amplitude sont appelés "modèles d'échantillons probabilité proportionnelle à l'amplitude"\footnote{La probabilité sera alors de $\pi_i^{\text{ppa}}=\frac{G_i}{\sum_i G_i}$} (PPA). Les plans d'échantillonnage avec PPA sont souvent utilisés dans les enquêtes auprès des entreprises lorsqu'il est important d'inclure les plus grandes entreprises d'une industrie dans l'échantillon, car elles contribuent largement à la production de biens ou de services de l'industrie. 
			
			\item "\NewTerm{Échantillonnage à probabilité non-égale}\index{\'echantillonnage \`a probabilit\'e non-\'egale}": L'échantillonnage à probabilités non-égales est davantage une famille de techniques d'échantillonnage. Par exemple, les plans avec PPA sont un échantillonnage à probabilité non-égales, mais aussi un échantillonnage par jugement et un échantillonnage de commodité, etc.
			
			\item "\NewTerm{Échantillonnage en deux niveaux}\index{\'echantillonnage en deux niveaux}\index{double \'echantillonnage}": L'échantillonnage à deux niveaux fait également référence à l'échantillonnage en grappes mais dans l'industrie c'est l'idée d'échantillonner un lot une deuxième fois dans l'idée de lui donner une seconde chance si lors du premier échantillonnage il était rejeté.
			
			\item "\NewTerm{Échantillonnage spatialement équilibré}\index{\'echantillonnage spatialement \'equilibr\'e}": Spatialement équilibré fait référence à des échantillons uniformément répartis dans une zone d'étude. L'échantillonnage spatialement équilibré est beaucoup plus efficace que l'échantillonnage aléatoire simple si la population échantillonnée est plus ou moins uniformément répartie dans la zone échantillonnée. Bien qu'un plan d'échantillonnage systématique puisse atteindre un équilibre spatial complet, il manque la randomisation qui est souhaitable dans les plans d'échantillonnage statistiques et il est difficile à appliquer lorsque les unités sélectionnées pour l'échantillonnage ne sont pas contiguës dans la zone d'étude (par exemple, la sélection de lacs ou de terres humides à échantillonner). Il existe plusieurs techniques différentes pour créer des plans d'échantillonnage spatialement équilibrés, mais l'une des plus courantes est la conception stratifiée par tessellation aléatoire généralisée (CSTAL) décrite par Stevens et Olsen.
			
			\item "\NewTerm{Échantillonnage adaptatif}\index{\'echantillonnage adaptatif}": L'échantillonnage adaptatif fait référence à une technique dans laquelle le plan d'échantillonnage est modifié sur le terrain en fonction des observations faites dans un ensemble d'unités d'échantillonnage présélectionnées. La meilleure façon de décrire l'échantillonnage adaptatif est peut-être d'utiliser un exemple. Considérons l'échantillonnage pour la présence ou l'abondance de plantes rares. Une sélection aléatoire d'unités d'échantillonnage produira de nombreuses unités d'échantillonnage où la plante n'est pas détectée, mais la plante rare est susceptible de se trouver dans des unités d'échantillonnage à proximité des unités où elle a été détectée. Avec l'échantillonnage adaptatif, la détection de la plante rare sur un site déclenche la sélection et l'échantillonnage de sites voisins supplémentaires qui n'ont pas été sélectionnés à l'origine dans le cadre de l'ensemble d'échantillonnage. Ainsi, la plus grande différence entre l'échantillonnage adaptatif et de nombreuses autres techniques de sélection aléatoire est que les conditions observées dans une unité d'échantillonnage influencent la sélection d'autres unités d'échantillonnage.
			
			\item "\NewTerm{Suréchantillonnage}\index{Sur\'echantillonnage}" et "\NewTerm{sous-échantillonnage}\index{sous-\'echantillonnage}": Les données déséquilibrées sont parfois un problème dans le domaine du Machine Learning. Avec le sous-échantillonnage, nous sélectionnons au hasard un sous-ensemble d'échantillons de la classe avec plus d'instances pour correspondre au nombre d'échantillons provenant de chaque classe. Avec le suréchantillonnage, nous dupliquons au hasard des échantillons de la classe avec moins d'instances ou nous générons des instances supplémentaires en fonction des données dont nous disposons, afin de faire correspondre le nombre d'échantillons dans chaque classe.
			\begin{figure}[H]
				\centering
				\includegraphics[scale=0.7]{img/arithmetics/over_and_undersampling.jpg}
				\caption{Idée sous-jacente du sur et sous-échantillonnage}
			\end{figure}
			
			\item Outre le sur- et le sous-échantillonnage, il existe des méthodes hybrides qui combinent le sous-échantillonnage avec la génération de données supplémentaires. Deux des plus populaires sont ROSE (Random Over-Sampling Example) et SMOTe (Synthetic Minority Over-Sampling Technique).
			\begin{figure}[H]
				\centering
				\includegraphics[scale=0.7]{img/arithmetics/smote_and_rose_sampling.jpg}
				\caption{Idée sous-jacente de l'échantillonnage SMOTe et ROSE}
			\end{figure}
		\end{itemize}
	\end{itemize}
	
	\pagebreak
	\paragraph{Plan d'échantillonnage aléatoire simple sans remise et probabilités égales (EAPER)}\mbox{}\\\\
	Le plan d'échantillonnage aléatoire simple sans remise et probabilités égales est le plus facile à exécuter dans la réalité et le moins cher en termes de temps et de coûts. Cependant il a le problème (que nous devrons résoudre plus tard) de ne pas respecter les proportions originales de la population pour certains caractères (propriétés) que nous connaissons comme étant des variables potentiellement explicatives pour notre résultat d'enquête ! Par conséquent, ce type d'échantillonnage est fréquent dans les cas industriels mais lorsque l'enquête porte sur des sujets sociaux (politique, marketing, etc.), il doit être évité.
	
	Pour un tel plan, nous échantillonnons $n$ individus sans retirage à partir d'une population $U$ avec $N$ individus. Ensuite, l'égalité de probabilité sera notée en fonction du nombre de combinaisons sans retirage (\SeeChapter{voir section Probabilités page \pageref{choix fonction}}):
	
	avec $\omega\in W(\Omega)$ où $P$ désigne la probabilité égale et $W(\Omega)$ désigne l'ensemble de tous les échantillons de $n$ individus possibles avec un tel plan.
	
	Nous nommons "\NewTerm{taux d'échantillonnage} le rapport :
	
	Dans le cas de l'échantillonnage à probabilités égales sans remise, la probabilité que l'individu appartienne à $W$ est évidemment :
	
	pour tout $i\in\{1,\ldots,N\}$. Par extension, évidemment, si $(i,j)\in \{1,\ldots,N\}^2$ avec $i\neq j$, la probabilité que $u_i$ et $u_j$ appartiennent à $W$ est donc immédiatement donnée par (nous supposons que c'est suffisamment évident pour que nous n'ayons pas besoin de donner la preuve) :
	
	Ce qui va suivre maintenant est évident mais nous l'avons écrit pour montrer que la notation est un peu spécifique au domaine de la méthodologie d'enquête.

	L'estimateur de la moyenne sur l'échantillon est noté :
	
	ou ce qui est aussi parfois noté :
	
	où $S=\{(i_1,\ldots,i_n)\in\{1,\ldots,N\}^n,i_1\neq \ldots\neq i_n; u_{i_1}\in W,\ldots,u_{i_n}\in W\}$.
	
	Nous avons déjà prouvé bien plus tôt dans les textes ci-dessus que $\bar{y}_W$ est l'estimateur sans biais de $\bar{y}_U$.
	
	\subsubsection{Erreur d'échantillonnage et erreur non due à l'échantillonnag  pour les enquêtes}
	Les élections présidentielles américaines de 2016 et le référendum britannique sur l'Union européenne en 2016 ont rendu les sondages électoraux publics et les ont mis pour le coups sous la lumière des projecteurs. Et avec tant de rapports de sondages faussés, il est temps de parler de ce qui fait que les sondeurs et les agences de sondage font si mal les choses.

	Prenons le Brexit comme exemple. Le 23 juin 2006, la Grande-Bretagne a organisé un référendum et les électeurs ont finalement choisi de quitter l'Union européenne. De nombreuses agences de sondage ont prédit à tort que la Grande-Bretagne resterait dans l'UE, tandis que d'autres avaient raison. Pourquoi la disparité ? Que faisaient ces deux groupes différemment ?
	
	L'échantillonnage statistique - par opposition aux devinettes sauvages - fournit une mesure de la certitude de l'agence de sondage à propos de leurs prédictions. Lorsqu'ils font des projections basées sur des sondages, les statisticiens signalent une marge d'erreur afin de pouvoir indiquer un intervalle de confiance de $95\%$ ; un outil utile à part entière. Par exemple, YouGov a estimé que $52\%$ de la Grande-Bretagne voterait pour quitter l'UE, et a donné une marge d'erreur de $2\%$. Dans ce cas, YouGov disait qu'ils étaient certains à $95\%$ que le vrai pourcentage de personnes votant pour quitter l'UE était de $52\%$, plus ou moins $2\%$. Cependant, ces intervalles - y compris la marge d'erreur - sont souvent mal utilisés par les médias de masse et les réseaux sociaux.  
	
	Le secret de l'industrie est que la marge d'erreur signalée ne comprend que l'erreur d'échantillonnage (erreur aléatoire), mais ignore ce que l'on appelle commodément l'erreur non due à l'échantillonnage (erreur systématique). Imaginez cinq forces poussant et tirant simultanément l'estimation d'un sondage dans les deux sens – des forces qui ne sont pas signalées par le sondeur. Ces forces systématiques comprennent :
	\begin{figure}[H]
		\centering
		\includegraphics[scale=0.8]{img/arithmetics/survey_errors.jpg}
		\caption{Type d'erreurs dans la méthodologie des enquêtes}
	\end{figure}
	\begin{itemize}
		\item "\NewTerm{L'erreur systématique}\index{erreur syst\'ematique}", ou  le biais, qui résulte d'erreurs ou de problèmes dans la conception du plan de recherche ou de défauts dans l'exécution de la conception de l'échantillon. Une erreur systématique existe dans les résultats d'un échantillon si ces résultats montrent une tendance constante à varier dans une direction particulière (constamment supérieure ou inférieure) par rapport à la valeur réelle du paramètre de population. L'erreur systématique comprend toutes les sources d'erreur à l'exception de celles introduites par le processus d'échantillonnage aléatoire. Par conséquent, les erreurs systématiques sont parfois nommées "\NewTerm{erreurs de non-échantillonnage}\index{erreurs de non-\'echantillonnage}". Les erreurs non dues à l'échantillonnage qui peuvent systématiquement influencer les réponses à l'enquête peuvent être classées en erreurs de plans d'échantillonnages et erreurs de mesures.
	
		\item "\NewTerm{L'erreur de conception de l'échantillon}\index{erreur de conception de l'\'echantillon}" est une erreur systématique qui résulte d'un problème dans la conception de l'échantillon ou les procédures d'échantillonnage. Les types d'erreurs de plan d'échantillonnage comprennent les erreurs de base, les erreurs de spécification de la population et les erreurs de sélection et sont présentés ci-dessous.
	
		\item La base (trame) de sondage est la liste des éléments ou membres de la population à partir desquels les unités à échantillonner sont sélectionnées. "\NewTerm{L'erreur de trame}\index{erreur de trame}" résulte de l'utilisation d'une base de sondage incomplète ou inexacte. Le problème est qu'un échantillon tiré d'une liste sujette à une erreur de base peut ne pas être un échantillon représentatif de la population cible. Une source courante d'erreur de base dans la recherche marketing est l'utilisation d'un annuaire téléphonique publié comme base de sondage pour un sondage téléphonique (ou généralement des sondages en ligne sur Internet où un vaste groupe de personnes ayant un intérêt particulier - ou étant particulièrement concerné par le sujet de l'enquête - pourrait partager massivement le lien pour biaiser le résultat de l'enquête). De nombreux ménages ne sont pas répertoriés dans un annuaire téléphonique actuel parce qu'ils ne veulent pas être répertoriés ou ne sont pas répertoriés avec précision parce qu'ils ont récemment déménagé ou changé de numéro de téléphone. La recherche a montré que les personnes qui sont inscrites dans les annuaires téléphoniques sont systématiquement différentes de celles qui ne sont pas inscrites à certains égards importants, tels que les niveaux socio-économiques. Cela signifie que si une étude censée représenter les opinions de tous les ménages dans une zone particulière est basée sur des listes dans l'annuaire téléphonique actuel, elle sera sujette à une erreur de trame.
	
		\item "\NewTerm{L'erreur de spécification de population}\index{erreur de sp\'ecification de population}" résulte d'une définition incorrecte de la population ou de l'univers à partir duquel l'échantillon doit être sélectionné. Par exemple, supposons qu'un chercheur définisse la population ou l'univers d'une étude comme des personnes de plus de $35$ ans. Plus tard, il a été déterminé que les personnes plus jeunes auraient dû être incluses et que la population aurait dû être définie comme des personnes âgées de $20$ ans ou plus. Si les jeunes qui ont été exclus sont significativement différents en ce qui concerne les variables d'intérêt, alors les résultats de l'échantillon seront biaisés.
	
		\item "\NewTerm{L'erreur de sélection}\index{erreur de s\'election}" peut se produire même lorsque l'analyste dispose d'un cadre d'échantillonnage approprié et a défini correctement la population. Une erreur de sélection se produit lorsque les procédures d'échantillonnage sont incomplètes ou inappropriées ou lorsque les procédures de sélection appropriées ne sont pas correctement suivies. Par exemple, les enquêteurs en porte-à-porte pourraient décider d'éviter les maisons qui n'ont pas l'air propres et rangées parce qu'ils pensent que les habitants ne seront pas d'accord pour faire une enquête. Si les personnes qui vivent dans des maisons en désordre sont systématiquement différentes de celles qui vivent dans des maisons bien rangées, alors une erreur de sélection sera introduite dans les résultats de l'enquête.
	
		\item "\NewTerm{L'erreur de mesure}\index{erreur de mesure}" est souvent une menace beaucoup plus sérieuse pour l'exactitude de l'enquête que ne l'est l'erreur aléatoire. Lorsque les résultats des sondages d'opinion sont publiés dans les médias et dans les rapports de recherche marketing professionnels, un chiffre d'erreur est fréquemment signalé (disons, $\pm 5\%$). Le téléspectateur ou l'utilisateur d'une étude marketing a l'impression que ce chiffre se réfère à l'erreur totale d'enquête. Malheureusement, ce n'est pas le cas. Ce chiffre se réfère uniquement à l'erreur d'échantillonnage aléatoire. Il n'inclut pas l'erreur de conception de l'échantillon et ne parle en aucun cas de l'erreur de mesure qui peut exister dans les résultats de la recherche. L'erreur de mesure se produit lorsqu'il existe une variation entre l'information recherchée (valeur vraie) et l'information réellement obtenue par le processus de mesure. Notre principale préoccupation dans ce texte est l'erreur de mesure systématique. Divers types d'erreurs peuvent être causés par de nombreuses déficiences dans le processus de mesure. Ces erreurs comprennent l'erreur d'information de substitution, l'erreur de l'intervieweur, le biais de l'instrument de mesure, l'erreur de traitement, le biais de non-réponse et le biais de réponse que nous décrirons ci-dessous.
	
		\item "\NewTerm{L'erreur d'informations de substitution}\index{erreur d'informations de substitution}" se produit lorsqu'il y a un écart entre les informations réellement requises pour résoudre un problème et les informations recherchées par le chercheur. Elle concerne des problèmes généraux dans la conception de la recherche, en particulier l'incapacité à définir correctement le problème. Il y a quelques années, Kellogg's™ a dépensé des millions pour développer une gamme de 17 céréales pour petit-déjeuner contenant des ingrédients qui aideraient les consommateurs à réduire leur cholestérol. La ligne de produits s'appelait \textit{Ensemble}. Elle a lamentablement échoué sur le marché. Oui, les gens veulent réduire leur cholestérol, mais la vraie question était de savoir s'ils achèteraient une gamme de céréales pour petit-déjeuner pour accomplir cette tâche. Cette question n'a jamais été posée dans la recherche. De plus, le nom « \textit{Ensemble} » fait généralement référence à un orchestre ou à quelque chose que vous portez. Les consommateurs ne comprenaient ni la gamme de produits ni la nécessité de la consommer.
	
		\item "\NewTerm{L'erreur de l'interviewer}\index{erreur de l'interviewer}", ou "\NewTerm{biais de l'interviewer}\index{biais de l'interviewer}", résulte du fait que l'intervieweur influence un répondant - consciemment ou inconsciemment - à donner des réponses fausses ou inexactes. La tenue vestimentaire, l'âge, le sexe, les expressions faciales, le langage corporel ou le ton de la voix de l'enquêteur peuvent influencer les réponses données par certains ou tous les répondants. Ce type d'erreur est causé par des problèmes de sélection et de formation des enquêteurs ou par le non-respect des instructions par les enquêteurs. Les enquêteurs doivent être correctement formés et supervisés pour paraître neutres en tout temps. Un autre type d'erreur de l'intervieweur se produit lorsqu'une tricherie délibérée a lieu. Cela peut être un problème particulier dans les entretiens porte-à-porte, où les enquêteurs peuvent être tentés de falsifier les entretiens et d'être payés pour un travail qu'ils n'ont pas réellement effectué.
	
		\item Le "\NewTerm{biais d'insturement de mesure}\index{biais d'insturement de mesure}" (parfois nommé "\NewTerm{biais de questionnaire}\index{biais de questionnaire}") résulte de problèmes avec l'instrument de mesure ou le questionnaire. Des exemples de tels problèmes incluent des questions suggestives ou des éléments de la conception du questionnaire qui rendent l'enregistrement des réponses difficile et sujet à des erreurs d'enregistrement (l'erreur la plus courante étant le choix manquant "autre..."). Des problèmes de ce type peuvent être évités en accordant une attention particulière aux détails lors de la phase de conception du questionnaire et en utilisant des prétests de questionnaires avant le début des entretiens sur le terrain.
	
		\item Les "\NewTerm{erreurs de saisies}"\index{erreurs de saisies} peuvenet être dues à des erreurs qui se produisent lorsque les informations des documents d'enquête sont saisies dans l'ordinateur. Par exemple, un document peut être mal numérisé. Les personnes qui remplissent des sondages sur un smartphone ou un ordinateur portable peuvent appuyer sur les mauvaises touches.
	
		\item Si un échantillon de $400$ de personnes est sélectionné dans une population particulière, tous les $400$ personnes doivent être interrogées. En pratique, cela n'arrivera jamais. Des taux de réponse de $5\%$ ou moins sont courants dans les sondages par courrier. La question est : \og Est-ce que ceux qui ont répondu à l'enquête diffèrent systématiquement d'une manière importante de ceux qui n'ont pas répondu ? \fg{} De telles différences conduisent au "\NewTerm{biais de non-réponse}\index{biais de non-r\'eponse}". Nous avons récemment examiné les résultats d'une étude menée auprès des clients d'une grande association d'épargne et de crédit. Le taux de réponse au questionnaire, inclus dans les relevés mensuels des clients, était légèrement inférieur à $1\%$. L'analyse des professions des répondants a révélé que le pourcentage de retraités parmi les répondants était $20$ fois plus élevé que dans la région métropolitaine locale. Cette surreprésentation des retraités a soulevé de sérieux doutes quant à l'exactitude des résultats.
	
		\item S'il y a une tendance pour les gens à répondre à une question particulière d'une certaine manière, alors il y a un "\NewTerm{bias de réponse}\index{bias de r\'eponse}". Le biais de réponse peut résulter d'une falsification délibérée ou d'une fausse déclaration inconsciente. La falsification délibérée se produit lorsque des personnes donnent délibérément des réponses fausses à des questions. Il existe de nombreuses raisons pour lesquelles des personnes peuvent sciemment déformer des informations dans un sondage. Ils peuvent vouloir paraître intelligents, ils peuvent ne pas révéler des informations qu'ils jugent embarrassantes, ou ils peuvent vouloir cacher des informations qu'ils considèrent comme personnelles. Par exemple, dans une enquête sur le comportement d'achat de fast-food, les répondants peuvent avoir une assez bonne idée du nombre de fois qu'ils ont visité un fast-food au cours du mois dernier. Cependant, ils peuvent ne pas se souvenir des restaurants de restauration rapide qu'ils ont visités ou du nombre de fois qu'ils ont visité chaque restaurant. Plutôt que de répondre « Je ne sais pas » en réponse à une question concernant les restaurants qu'ils ont visités, les répondants peuvent simplement deviner. Une fausse déclaration inconsciente se produit lorsqu'un répondant essaie légitimement d'être véridique et précis mais donne une réponse inexacte. Ce type de biais peut se produire en raison du format de la question, du contenu de la question ou de diverses autres raisons.
	\end{itemize}

	Par conséquent, la vraie marge d'erreur est incommensurable ; nous l'estimons avec la marge d'erreur d'échantillonnage (borne inférieure). Comme elle est actuellement utilisée par les sondeurs, la marge d'erreur d'échantillonnage est une estimation très libérale de l'exactitude des prévisions.

	\pagebreak
	\subsection{Un monde sans statistiques}
	Les essais suivants illustrent l'importance des statistiques dans la recherche et la vie quotidienne. Pour ce faire, nous réfléchissons à la question de savoir à quoi ressembleraient les choses dans un monde (entreprise, travail, vie privée) sans statistiques\footnote{Les critiques (biais) de l'énoncé "\textit{les statistiques mentent toujours}" ont déjà été traitées dans l'introduction (et les arguments non constructifs de la haine des statistiques comme "\textit{vos statistiques sont drôles}" ou "\textit{Je vais préparer le repas de mes $1.2314$ enfants...}" appartient au même biais cognitif!)} ?
	
	La science serait à peu près Ok sans les statistiques. Isaac Newton n'avait pas besoin de statistiques pour ses théories de la gravité, du mouvement et de la lumière, pas plus qu'Albert Einstein n'avait besoin de statistiques pour la théorie de la relativité. La thermodynamique et la mécanique quantique sont fondamentalement statistiques, mais de nombreux progrès auraient pu être réalisés dans ces domaines sans statistiques. La deuxième loi de la thermodynamique est un fait observable, idem pour l'expérience des deux fentes et divers résultats expérimentaux révélant la nature de l'atome. La bombe A et, presque certainement, la bombe H, n'auraient peut-être jamais été inventées sans statistiques, mais dans l'ensemble, on peut penser que la plupart des gens penseraient que le monde serait un meilleur endroit sans ces développements scientifiques particuliers. Sans statistiques, nous pourrions oublier de découvrir le boson de Higgs etc, mais cela ne semble pas être une telle perte pour l'humanité.
	
	Cependant:
	\begin{itemize}
		\item Les statistiques ont aidé à gagner la Seconde Guerre mondiale, notamment en cassant le code Enigma (\SeeChapter{voir section Cryptographie page \pageref{enigma}})

		\item Les statistiques aident à évaluer les faits de données d'observation (changement de météo, approbation de nouveaux médicaments, nouvelles méthodes de gestion, nouvelles méthodes agricoles, etc.)

		\item Les statistiques avec des logiciels automatisés donnent la possibilité aux entreprises d'automatiser les décisions et les analyses et donc d'être plus compétitives et de vendre des services moins chers que les concurrents qui font le travail manuellement.

		\item Les statistiques aident à rechercher l'optimum en R\&D en réduisant le nombre d'expériences qui peuvent être énormes ou très coûteuses dans certains domaines industriels. Par conséquent, ceux qui utilisent des statistiques seront plus compétitifs et rapides à développer de nouveaux traitements.

		\item Les statistiques aident à constituer des pool d'enquêtes de manière efficace pour minimiser les coûts et maximiser la fiabilité des résultats (cela inclut également l'échantillonnage de qualité pour le rejet de lots dans l'industrie!). Ils aident également à calculer l'intervalle de tolérance (erreur) des enquêtes résultantes.
		
		\item Les statistiques sont utilisées en assurance car elles se sont avérées plus efficaces que les critères choisis par l'humain (\SeeChapter{voir section Techniques de Gestion Quantitatives page \pageref{insurance}}).

		\item Les statistiques sont utilisées dans la détection des fraudes (outils d'analyse des valeurs extrêmes) car manuellement l'analyse serait impossible sur des populations de plus de centaines de millions d'individus (\SeeChapter{voir section Méthodes Numériques page \pageref{data mining}})..

		\item Les statistiques sont utilisées dans la finance moderne car automatisées par ordinateur (donc moins chères qu'un courtier humain) et plus fiables et plus performantes sur le long terme (\SeeChapter{voir section Économie page \pageref{economy}}).

		\item Les statistiques sont utilisées pour la gestion de projets critiques de grande taille pour estimer les erreurs de marge sur la budgétisation et la planification du temps (\SeeChapter{voir section Technique Quantitative de Gestion page \pageref{probabilitic pert}}). Dans certains projets, le client n'acceptera pas une proposition de durée et de coûts sans méthodes d'estimation modernes (sinon le fournisseur sera suspecté d'être très amateur).

		\item Les statistiques sont utilisées en Data Mining (fouille de données) et Machine Learning  (apprentissage machine) pour anticiper les tendances sur les marchés financiers ou encore sur les comportements sur les réseaux sociaux ou le ciblage publicitaire (\SeeChapter{voir section Méthodes Numériques page \pageref{data mining}}).

		\item Les statistiques permettent de prévoir dans un certaine mesure les événements catastrophiques (tremblements de terre, tsunami, conditions météorologiques extrêmes, pannes de machines, fiabilité de moyens de transports) pour évacuer la population/utilisateurs et aussi pour optimiser les méthodes agricoles et industrielles.

		\item Les statistiques aident les ordinateurs à comprendre la saisie en langage naturel dans les moteurs de recherche ou également dans la correction des erreurs de frappe.

		\item Les statistiques auraient aidé à avoir de meilleurs claviers d'ordinateur modernes (car à l'origine les claviers étaient adaptés pour les machines à écrire mécaniques).

		\item Les statistiques sont utilisées pour automatiser l'analyse de la stabilité (capacité) des méthodes de production et ce pour être plus compétitif et vendre des produits moins chers que les concurrents qui le font manuellement (\SeeChapter{voir section Génie Industriel page \pageref{six sigma}}).

		\item Les statistiques sont utilisées en Mécanique Quantique et Mécanique Statistique (\SeeChapter{voir section Mécanique Statistique page \pageref{statistical mechanics}}) et ont aidé à la possibilité d'imaginer des dispositifs qui n'auraient pas été découverts aussi rapidement sans (LASER, transistor, etc.).

		\item Les statistiques sont utilisées en génétique dans le cadre de l'analyse des puces à ADN (voir notre livre d'accompagnement R) et permettent de faire découvrir et développer  plus rapidement des vaccins parfois très rapidement.

		\item Les statistiques sont utilisées dans les Réseaux de Neurones, le Deep learning (apprentissage machine profond) et donc l'Intelligence Artificielle (\SeeChapter{voir section Méthodes Numériques page \pageref{neural network}}) qui vont changer (ça commence déjà) radicalement le 21ème siècle et au-delà...

		\item Les statistiques sont utilisées dans certaines décisions de justice lorsque la preuve d'une corrélation improbable semble apparaître (voir l'affaire Lucie de Berk sur Internet par exemple).

		\item Les statistiques sont utilisées en astronomie pour soutenir statistiquement l'observation de nouvelles planètes ou de nouveaux types d'étoiles.

		\item Les statistiques sont utilisées pour les plans de démonstration des garanties de produits et le calcul de la garantie à vie optimale des produits neufs vendus sur le marché en grande quantité (\SeeChapter{voir section Génie Industriel page \pageref{design of reliability tests}}).

		\item Les statistiques sont utilisées dans les pays industrialisés pour prendre des décisions basées sur des évidences présentées par des économistes qui ne peuvent convaincre personne avec des anecdotes mais avec des conclusions dérivées de modèles économétriques ou de modèles de programmation mathématique ou des variantes des deux.
		
		\item Les statistiques sont utilisées à l'échelle nationale pour obtenir des informations sur ses habitants : chômage, proportion des sexes, répartition des salaires, tendances de la santé, vieillissement, etc., et il serait impossible de prendre des décisions politiques et de communiquer avec précision sans de telles statistiques dans les pays industrialisés.
	\end{itemize}
	Le lecteur doit cependant garder à l'esprit que l'affect, la peur et l'émotion sont nos balises embarquées ; toujours avec nous, ils sont opérationnels dans beaucoup de situations. Mais pour d'autres situations, c'est le cerveau rationnel qui devrait prendre le relais. Moyennes, médianes, fréquences, écarts-types, corrélations, expériences aléatoires, etc... ce sont des choses qu'aucun de nos capteurs biologiques n'est conçu pour reconnaître. C'est pourquoi un pourcentage important de la population rejette la méthode scientifique et la plupart rejettent les statistiques (ils sont mal à l'aise avec les preuves scientifiques).
	
	\begin{fquote}[Fred Mosteller]Il est facile de mentir avec les statistiques, mais plus facile de mentir sans elles.
 	\end{fquote}
	
	\subsubsection{Sophismes de données}
	Jusqu'à présent, nous avons présenté en détail certains problèmes humains communs et problèmes d'interprétation dans les statistiques comme (en plus de la manière déplorable dont les médias de masse communiquent les résultats scientifiques comme vu à la page \pageref{scientific mainstream media communication}) le fait que la corrélation n'est pas toujours causalité, une corrélation nulle ne signifie pas une absence de schéma, le paradoxe de Simpson, le quatuor d'Anscombe, l'utilisation de mauvais indicateurs ou tests statistiques, le $p$-hacking, etc.
	
	Certains d'entre eux peuvent être résumés dans l'affiche suivante (rappelez-vous cette affiche pour le moment où nous étudierons la fouille de données à la page \pageref{data mining})\label{data fallacies}:
	\begin{figure}[H]
		\centering
		\includegraphics[scale=0.2]{img/arithmetics/data_fallacies.pdf}	
		\caption[Sophismes de données communs]{Sophismes de données communs\\ (credit: Geckoboard, source: \url{https://www.geckoboard.com/learn/data-literacy/})}
	\end{figure}
	
	\begin{fquote}[?]Même une estimation approximative, même fausse, vaut mieux que pas d'estimation du tout.
 	\end{fquote}
	
	\begin{flushright}
	\begin{tabular}{l c}
	\circled{90} & \pbox{20cm}{\score{3}{5} \\ {\tiny 125 votes, 62.5\%}} 
	\end{tabular} 
	\end{flushright}
	
	
  \chapter{Algèbre}

	\textit{\textbf{Algebra is the science of calculating the quantities or structures represented by letters.}} (Larousse)
	\minitoc
	%to make section start on odd page
	\newpage
	\thispagestyle{empty}
	\mbox{}
	\section{Calcul Algébrique}\label{calculus}
	\lettrine[lines=4]{\color{BrickRed}D}ans la section d'Arithmétique de ce site, nous avons beaucoup écrit sur différents théorèmes utilisant les nombres abstraits afin de généraliser l'étendue de la validité de ces derniers. Nous avons cependant peu abordé la façon dont nous devions manipuler ces nombres abstraits. C'est ce que nous allons voir maintenant.
	
	Comme vous le savez peut-être déjà, le nombre peut être envisagé en faisant abstraction de la nature des objets qui constituent le groupement qu'il caractérise et ainsi qu'à la façon de codifier (chiffre arabe, romain, ou autre système...). Nous disons alors que le nombre est un "\NewTerm{nombre abstrait}\index{nombre abstrait}" et lorsque nous manipulons ces types d'objets nous disons que nous faisons du "calcul 	algébrique" ou encore du "\NewTerm{calcul littéral}".
	
	\textbf{Définition (\#\mydef):} Le "\NewTerm{calcul littéral}\index{calcul litt\'eral}" consiste à calculer avec des variables (c'est-à-dire avec des lettres) comme on le ferait avec des nombres.
	
	Pour les mathématiciens il n'est souvent pas avantageux de travailler avec des valeurs numériques (1,2,3...) car ils représentent uniquement des cas particuliers. Ce que cherchent les physiciens, ingénieurs ainsi que les mathématiciens, ce sont des relations applicables universellement dans un cadre le plus général possible.
	
	Ces nombres abstraits appelés aujourd'hui communément "\NewTerm{variable}\index{variables}" sont très souvent représentés par l'alphabet latin (pour lequel les premières lettres de l'alphabet latin $a, b, c, \ldots$ désignent souvent les nombres connus, et les dernières $x, y, z, \ldots$ les nombres inconnus), l'alphabet grec (aussi beaucoup utilisé pour représenter des opérateurs mathématiques plus ou moins complexes) et l'alphabet hébraïque (dans une moindre mesure).
	
	Bien que ces symboles puissent représenter n'importe quel nombre, il en existe cependant quelques-uns aussi bien en physique ou en mathématique qui peuvent représenter des constantes dites "\NewTerm{Universelles}\index{constantes universelles}" telles que la vitesse de la lumière $c$, la constante gravitationnelle $G$, la valeur de $\pi$, le nombre d'Euler $e$, etc.

	\begin{tcolorbox}[title=Remarque,colframe=black,arc=10pt]
	Il semblerait que les lettres pour représenter les nombres ont été employées pour la première fois par Viète au milieu du 16ème siècle.
	\end{tcolorbox}	

	Une variable est donc susceptible de prendre des valeurs numériques différentes. L'ensemble de ces valeurs peut varier suivant le caractère du problème considéré.
	
	Rappels (nous avions déjà défini cela dans le chapitre traitant des Nombres dans la section d'Arithmétique):

	\begin{enumerate}
		\item[R1.] Nous appelons "\NewTerm{domaine de définition}\index{domaine de d\'efinition}" d'une variable, l'ensemble des valeurs numériques qu'elle est susceptible de prendre entre deux bornes, ou dans un ensemble (tel que $\mathbb{N}, \mathbb{R},\mathbb{R}^+,$ etc.).
	
	Soit $a$ et $b$ deux nombres tel que $a<b$. Alors:
		
		\item[R2.] Nous appelons "\NewTerm{intervalle fermé d'extrémités $a$ et $b$}\index{intervalle ferm\'e}", l'ensemble de tous les nombres $x$ compris entre ces deux valeurs et nous le désignons de la façon suivante:
		
		
		\item[R3.] Nous appelons "\NewTerm{intervalle ouvert d'extrémités $a$ et $b$}\index{intervalle ouvert}", l'ensemble de tous les nombres $x$ compris entre ces deux valeurs non comprises et nous le désignons de la façon suivante:
		
		
		\item[R4.] Nous appelons "\NewTerm{intervalle fermé à gauche, ouvert à droite}\index{intervalle semi-ouvert}" la relation suivante:
		
		
		\item[R5.] Nous appelons "\NewTerm{intervalle ouvert à gauche, fermé à droite}" la relation suivante:
		
	\end{enumerate}

	\begin{tcolorbox}[title=Remarque,colframe=black,arc=10pt]
	Si la variable $x$ peut prendre toutes les valeurs négatives et positives possibles nous écrirons dès lors: $\left] -\infty,+\infty \right[$ où le symbole "$\infty$" signifie "infini". Evidemment il peut y avoir des combinaisons d'intervalles ouvert et infini à droite, fermé et limité à gauche et réciproquement.
	\end{tcolorbox}	

	\textbf{Définition (\#\mydef):} Nous appelons "\NewTerm{voisinage de $a$}\index{voisinage (analyse fonctionnelle)}", tout intervalle ouvert de $\mathbb{R}$ contenant $a$ (c'est un concept simple que nous reprendrons pour définir ce qu'est une fonction continue). Ainsi:
	
	est un voisinage de $a$.

	\subsection{Équations et inéquations}
	L'algèbre élémentaire consiste à partir des définitions de l'addition, soustraction, multiplication, et puissance et de leurs propriétés (associativité, distributivité, commutativité, élément neutre, inverse,...) - ce qui constitue selon l'ensemble sur lequel nous travaillons un corps ou un groupe commutatif abélien ou non (\SeeChapter{cf. chapitre Théorie des Ensembles \pageref{structures}}) - à manipuler selon un but fixé des "\NewTerm{équations algébriques}\index{\'equations alg\'ebriques}" mettant en relation des variables et constantes.

	Nous allons définir de suite après ce qu'est une équation et une inéquation mais nous souhaitons d'abord définir certaines de leurs propriétés:

%	Soit $A$ et $B$ deux polynômes (ou monômes) quelconques - voir définitions un peu plus loin - les expressions:
	
	Vérifient les propriétés suivantes :
	\begin{enumerate}
		\item[P1.] Nous pouvons toujours ajouter ou ôter aux deux membres d'une inéquation ou équation un même polynôme en obtenant une inéquation ou équation équivalente (c'est à dire avec les mêmes solutions ou réductions). Nous disons alors que l'égalité ou l'inégalité restent "vraies" par l'opération d'addition ou de soustraction membre à membre.
		
		\item[P2.] Si nous multiplions ou si nous divisons les deux membres d'une équation ou inéquation par un même nombre positif nous obtenons également une inéquation ou équation équivalente (nous avons déjà vu cela). Nous disons alors que l'égalité ou l'inégalité reste "vraie" par l'opération de multiplication ou division membre à membre.
		
		\item[P3.] Si nous multiplions ou si nous divisons les deux membres d'une inéquation par un même nombre négatif et si nous inversons le sens de l'inégalité, nous obtenons alors une inéquation ou équation équivalente.
	\end{enumerate}
	
	\subsubsection{Équations}
	\textbf{Définition (\#\mydef):} Une "\NewTerm{équation}\index{\'equation}" est une relation d'égalité entre des valeurs toutes abstraites (autrement dit: deux expressions algébriques) ou non toutes abstraites (dès lors nous parlons d'équations à une inconnue, deux inconnues, trois inconnues, ... ) reliées entre elles par des opérateurs divers.

	La maîtrise parfaite de l'algèbre élémentaire est fondamentale en physique-mathématique et dans l'industrie!!! Comme il existe une infinité de types d'équations, nous ne les présenterons pas ici. C'est le rôle de l'enseignant/formateur dans les classes d'entraîner le cerveau de son auditoire pendant plusieurs années (2 à 3 ans en moyenne) à résoudre énormément de configurations différentes d'équations algébriques (exposées sous forme de problèmes de tous les jours, géométriques ou purement mathématiques) et ce afin que les élèves manipulent ces dernières sans erreurs en suivant un raisonnement logique et rigoureux (ce n'est qu'en forgeant que l'on devient forgeron...)!!!

	En d'autres termes: Un professeur/formateur et un établissement ad hoc sont irremplaçables pour
acquérir un savoir et avoir un retour d'expérience!!!

	\begin{tcolorbox}[title=Remarque,colframe=black,arc=10pt]
Nous avons tenté, ci-dessous, de faire une généralisation simpliste des règles de base de l'algèbre élémentaire. Cette généralisation sera d'autant plus simple à comprendre que le lecteur aura l'habitude de manipuler des quantités abstraites.
	\end{tcolorbox}	

	Ainsi, soit $a, b, c, d, e, ..., x, y$ des nombres abstraits pouvant prendre n'importe quelle valeur numérique (nous restons dans le cadre des nombres classiques scolaires et industriels...).

	Soit $\Xi$ la lettre majuscule grecque se prononçant "Xi") représentant un ou plusieurs nombres abstraits (variables) opérants entre eux d'une façon quelconque tel que nous ayons des monômes (un seul nombre abstrait) ou polynômes (poly = plusieurs) algébriques différents distinguables ou non (nous faisons donc ici une sorte d'abstrait de l'abstraction ou si vous préférez une variable de plusieurs variables).

	Propriétés (il s'agit en fait plus d'exemples que de propriétés...):
	\begin{enumerate}
		\item[P1.] Nous aurons toujours $\Xi=\Xi$ si et seulement si le terme $\Xi$ à gauche de l'égalité est le même terme $\Xi$  que celui qui est à droite de l'égalité. Si cette condition est satisfaite nous avons alors :
		
		Sinon:
		
		où nous excluons donc les cas où tous les $\Xi$ sont identiques entre eux (sinon nous revenons à P1).
		
		\item[P2.] Nous avons, si $\Xi\neq 0$:
		
		qui vérifie la symbolique de l'équation $\Xi=\Xi$ dans le cas seulement où les éléments sont identiques entre eux (nous excluons bien évidemment le cas avec dénominateur nul).
	
		\item[P3.] \label{power rules calculations}Si tous les $\Xi$ sont strictement identiques, alors:
				
		Sinon nous avons:
		
		qui ne peut s'écrire sous forme condensée simple. Il peut aussi arriver que:
		
		avec le $\Xi$ à droite de l'égalité identique à aucun, un ou encore plusieurs $\Xi$ du membre gauche de l'égalité.

		\item[P4.] Nous pouvons avoir:
		
		sans que nécessairement les exposants du numérateur ou dénominateur soient égaux (nous excluons le dénominateur nul).

		Sinon nous pouvons avoir:
		
		Mais n'oubliez pas (\SeeChapter{voir section Opérateurs page \pageref {division}}) que dans le cas général où le numérateur et le dénominateur ne sont pas égaux que:
		
		
		\item[P5.] On a si tous les dénominateurs sont strictement égaux :
		
		mais il n'est cependant pas impossible d'avoir :
		
		avec le $\Xi$ à droite de l'égalité identique à un ou plusieurs membres de gauche de l'égalité ou encore il est tout à fait possible d'avoir :
		

		\item[P6.] Laissons $\star$ représenter exclusivement le symbole d'addition ou de soustraction, nous avons alors (à une variation de signe donné):
		
		si tous les $\Xi$ sont identiques les uns par rapport aux autres ou si la combinaison d'un nombre indéterminé de $\Xi$ est égal aux $\Xi$ sur la droite de l'égalité.

		Sinon on aura (c'est-à-dire que le résultat donnera n'importe quel monôme ou polynôme indéterminé) :
		

		\item[P7.] On a si tous les $\Xi$ (les bases) levés à la puissance sont strictement identiques :
		
	\end{enumerate}
	si et seulement si les bases $\Xi$ sont égales (ou pourraient être décomposables pour être égales) et les puissances $\Xi$ ne sont pas nécessairement égales.

	A partir de la connaissance des exemples de ces $7$ règles/propriétés de base, nous pouvons résoudre, simplifier ou montrer qu'une équation simple a des solutions ou non relativement à un problème ou un énoncé donné.

	Ainsi, étant donné un opérande ou une séquence d'opérations quelconques sur une ou plusieurs abstractions d'abstraits et parmi toutes, une (ou plusieurs) dont les valeurs numériques est ou sont inconnues (les autres sont connues), nous pourrions alors être en mesure de prouver qu'une relation (énoncé) comme :
	
	a des solutions existantes ou non (c'est-à-dire : est Vrai ou Faux).
	
	Dans le cas d'une équation avec valeur absolue (\SeeChapter{voir section Opérateurs page \pageref{absolute value}}) du type :
	
	avec le second membre étant strictement positif (sinon les relations précédentes seraient un non-sens) cela équivaut bien sûr à partir de la définition de la valeur absolue à écrire :
	

	\begin{tcolorbox}[title=Remarques,colframe=black,arc=10pt]
	\textbf{R1.} La présence de la valeur absolue dans une équation algébrique dans laquelle on cherche des solutions double souvent le nombre de solutions.\\
	
	\textbf{R2.} Une équation est nommée "\NewTerm{\'equation conditionnelle}\index{\'equation conditionnelle}" lorsqu'il y a des nombres dans l'ensemble de définition qui ne sont pas des solutions (ce qui est en fait le cas le plus courant). Inversement, si un nombre quelconque de l'ensemble de définition est la solution de l'équation, alors l'équation est nommée "\NewTerm{équation d'identité}\index{\'equation d'identit\'e}".
	\end{tcolorbox}	
	
	Nous devons parfois résoudre un "\NewTerm{système d'équations}\index{syst\'eme d'\'equations}" (voir par exemple les systèmes d'équations linéaires à la page \pageref{linear systems of equations}).
	\begin{itemize}
		\item Qu'est-ce que c'est ? : C'est un ensemble d'au moins deux équations à résoudre (la résolution n'est pas toujours équivalente à celle consistant à simplifier une expression!).

		\item Quelle est la spécificité d'un tel système ? : Toutes les solutions du système sont l'intersection de toutes les solutions des équations à résoudre (pour des exemples détaillés voir les sections Algèbre Linéaire et Méthodes Numériques). 

		\item Quelle sont les applications pratiques ? : C'est sans quasiment infini (voir les différents chapitres du livre), ces systèmes résolvent des problèmes impliquant des applications des mathématiques à d'autres domaines (finance, ingénierie, recherche opérationnelle, etc.).
	\end{itemize}
	En raison de la variété illimitée des applications, il est difficile d'établir des règles générales précises pour les solutions et les procédures existantes à suivre ne seront certainement utiles que si le problème peut être formulé en équations et les procédures suivantes peuvent aider un peu au moins à éviter certaines erreurs :
	\begin{enumerate}
		\item Si nous avons un l'énoncé d'un problème déjà sous forme écrite, nous le relisons plusieurs fois attentivement et nous considérons les faits donnés et la quantité d'inconnues à trouver et leur domaine de définition (résumer l'énoncé sur une feuille de papier ou ailleurs est souvent utile pour les gros problèmes!).
		
		\item Sélectionnez les lettres qui représentent les quantités inconnues. C'est une des étapes décisives dans la recherche de solutions. Les phrases contenant des mots tels que : trouver, quoi, comment, où, quand devraient vous aider à identifier la quantité inconnue.
		
		\item Faites un dessin (dans votre tête ou sur papier) avec des légendes. C'est à coup sûr possible la plupart du temps uniquement avec des problèmes ayant $1$, $2$ ou $3$ inconnues.
		
		\item Énumérez les faits connus et les relations concernant les quantités inconnues. Une relation peut être décrite par une équation dans laquelle apparaissent dans un ou les deux côtés du signe égale des déclarations (affirmations) écrites avec des phrases normales.
		
		\item Après l'étape précédente, écrivez une ou plusieurs équations qui décrivent avec précision ce qui a été dit avec des phrases.
		
		\item Résoudre l'équation ou le système d'équations formulés dans les étapes précédentes en utilisant les multiples techniques heuristiques existantes.
		
		\item Vérifier les solutions obtenues à l'étape précédente en se référant à l'énoncé initial du problème (vérifier que la solution est cohérente avec les conditions de l'énoncé).
	\end{enumerate}
	Certaines méthodes de résolutions de systèmes d'équations sont traitées en détail dans la section de Méthodes Numériques et également dans la section Algèbre Linéaire (vous y comprendrez donc mieux la procédure ci-dessus).
	 
	 \subsubsection{Inéquations}
	 Précédemment nous avons vu qu'une équation était composée d'une égalité de divers calculs avec des termes différents (avec au moins une "inconnue" ou un "nombre abstrait") et que :
 	\begin{enumerate}
 	  \item "Résoudre une équation" est un processus consistant à calculer la valeur de l'inconnue pour satisfaire l'égalité (quand une solution existe !)
 	  
 	  \item "Simplifier une équation" est le processus consistant à minimiser mathématiquement le nombre de termes (factoriser ou éliminer...)
 	  
 	  \item "Développer une équation" est le processus consistant à distribuer tous les termes.
 	\end{enumerate}
 	Pourquoi faut-il rappeler la définition d'une équation ? Tout simplement parce que pour l'inégalité, c'est presque le même processus intellectuel ! La différence ? Si l'équation est une égalité, l'inégalité est une inégalité (...) : Comme l'équation, l'inégalité est composée de divers calculs avec des termes différents interconnectés par des opérateurs quelconques avec au moins une inconnue.
	
	Principales différences entre les équations d'égalité et d'inégalité :
	
	\begin{enumerate}
		\item Égalité : Symbolisée par le symbole $=$
		
		\item Inégalité : Symbolisée par les relations d'ordre $<, \leq, \geq, >$
	\end{enumerate}
	
	Lorsque nous résolvons une inégalité, notre inconnue peut avoir une plage de valeurs qui satisfont l'inégalité. On dit alors que la solution de l'inégalité est un "\NewTerm{ensemble de valeurs}\index{ensemble de valeurs}". C'est la différence fondamentale entre l'égalité (\underline{plusieurs} solutions) et l'inégalité (\underline{plage} de solutions) !
	
	Faisons un rafraichissement sur les signes que l'on peut rencontrer dans une inégalité :
	\begin{itemize}
		\item $<$: Doit être lu "\NewTerm{strictement inférieur à}\index{symbole!strictement inf\'erieur \`a}" ou "\NewTerm{strictement plus petit que}\index{symbole!strictement plus petit que}". Dans ce cas, la valeur numérique cible n'est pas incluse dans la plage et nous pouvons alors représenter la plage (intervalle) avec un crochet gauche ouvert $] ...$ ou un crochet droit ouvert $... [$ à côté de la valeur cible.
		
		\begin{tcolorbox}[colframe=black,colback=white,sharp corners]
		\textbf{{\Large \ding{45}}Exemple:}\\\\
		Écrire $x<5$ signifie $x \in ]-\infty,5[$ et $x<-5$ signifie $x \in ]-\infty,-5[$ .
		\end{tcolorbox}
		
		\item $>$: Doit être lu "\NewTerm{strictement sup\'erieur \`a}\index{symbole!strictement plus grand que}" ou "\NewTerm{strictement plus grand que}\index{symbole!strictement plus grand que}". Dans ce cas, la valeur cible numérique n'est pas non plus incluse dans la plage (intervalle) et nous pouvons alors représenter la plage avec un crochet gauche ouvert $] ...$ ou un crochet droit $... [$ à côté de la valeur cible.
		
		\begin{tcolorbox}[colframe=black,colback=white,sharp corners]
	\textbf{{\Large \ding{45}}Exemple:}\\\\
		Écrire $x>5$ signifie $x \in ]5,+\infty[$ et $x>-5$ signifie $x \in ]-5,+\infty[$ .
		\end{tcolorbox}
		
		\item $\leq$: Doit être lu "\NewTerm{inférieur ou égal à}\index{symbole!inf\'erieur ou \'egal \`a}" ou "\NewTerm{plus petit ou égal à}\index{symbole!plus petit ou \'egal \`a}". Dans ce cas, la valeur cible numérique est dans la plage (intervalle) et nous pouvons alors représenter la plage avec un crochet gauche fermé $[...$ ou crochet droit $...]$.
		
		\begin{tcolorbox}[colframe=black,colback=white,sharp corners]
	\textbf{{\Large \ding{45}}Exemple:}\\\\
		Écrire $x\leq 5$ signifie $x \in ]-\infty,5]$ et $x<-5$ signifie $x \in ]-\infty,-5]$ .
		\end{tcolorbox}
		
		\item $\geq$: Doit être lu "\NewTerm{supérieur ou égal à}\index{symbol!sup\'erieur ou \'egal \`a}" ou "\NewTerm{plus grand ou égal à}\index{symbole!plus grand ou \'egal \`a}". Dans ce cas, la valeur cible numérique est dans la plage (intervalle) et nous pouvons alors représenter la plage avec un crochet gauche fermé $[...$ ou crochet droit $...]$.
		
		\begin{tcolorbox}[colframe=black,colback=white,sharp corners]
	\textbf{{\Large \ding{45}}Exemple:}\\\\
		Écrire $x\geq 5$ signifie $x \in [5,+\infty[$ et $x>-5$ signifie $x \in [-5,+\infty[$.
		\end{tcolorbox}
	
	\end{itemize}
	L'objectif des inégalités est la plupart du temps (hors but esthétique) d'avoir au moins une valeur numérique qui définit le domaine de solution (de tous les termes abstraits - variables - de l'inégalité) qui satisfait l'inégalité.
	
	Il existe de nombreuses façons de représenter les domaines de définition des variables qui satisfont une inégalité. Nous allons voir à travers de petits exemples quelles sont quelques-unes des possibilités les plus courantes :
	
	\begin{tcolorbox}[colframe=black,colback=white,sharp corners]
\textbf{{\Large \ding{45}}Exemple:}\\
	Soit une inégalité linéaire (du premier degré) sur $x$ avec une seule inconnue à laquelle on impose une contrainte particulière arbitraire par exemple (bien entendu, l'expression peut contenir plus de termes...) :
	
	où nous imaginons que cette inéquation résulte déjà d'une simplification de termes qui étaient superflus. Résoudre l'inégalité revient à rechercher les valeurs de $x$ inférieures à $2$. Bien sûr, il n'y a pas qu'une solution dans $\mathbb{R}$ mais un ensemble (intervalle) de solutions et c'est le principe des inégalités !\\
	
	Pour résoudre l'inégalité, on observe d'abord le type d'inégalité imposé ("strict" ou "égal"). Ensuite, dans les classes de lycées (et pas seulement parfois...) on représente l'ensemble $\mathbb{R}$ traditionnellement par un tableau tel que :
	\begin{table}[H]
		\centering
		\definecolor{gris}{gray}{0.85}
		\begin{tabular}{|l|c|r|}
		\hline 
		{\cellcolor{black!30}$-\infty$} & {\cellcolor{black!30}$0$} & {\cellcolor{black!30}$+\infty$}\\ 
		\hline 
		............... & .......|....... & ............... \\  
		\hline 
		\end{tabular} 
		\caption{Tableau de résolution des inégalités}
	\end{table}
	Nous savons intuitivement que la solution de notre inégalité comprend toutes les valeurs inférieures à $2$ ($2$ étant lui-même exclu des solutions) jusqu'à $-\infty$. Ensuite, nous écrivons cet intervalle ou domaine comme suit:
	
	On peut alors représenter sous forme de tableau l'ensemble des solutions (cela aide à comprendre et prépare l'étudiant à la résolution de systèmes d'équations et d'inéquations et à l'étude des variations de fonctions). Pour cela, on reprend le gabarit du tableau précédent et on y place notre valeur cible (on n'en a qu'une dans cet exemple particulier mais parfois il peut y en avoir plusieurs car il y a une singularité ou des racines pour certaines valeurs du domaine de définition), soit la valeur $2$ :
	\begin{table}[H]
		\begin{center}
		\definecolor{gris}{gray}{0.85}
		\begin{tabular}{|l|c|c|r|}
		\hline 
		{\cellcolor{black!30}$-\infty$} & {\cellcolor{black!30}$0$} & {\cellcolor{black!30}$2$} & {\cellcolor{black!30}$+\infty$}\\ 
		\hline 
		............... & .......|....... & .......[....... & ............... \\  
		\hline 
		\end{tabular} 
		\end{center}
		\caption{Table augmentée pour la résolution des inégalités}
	\end{table}
	et enfin, on délimite avec de la couleur (...) l'ensemble des solutions de $-\infty$ à $+2$ exclus :
	\begin{table}[H]
		\begin{center}
		\definecolor{gris}{gray}{0.85}
		\begin{tabular}{|l|c|c|r|}
		\hline 
		{\cellcolor{black!30}$-\infty$} & {\cellcolor{black!30}$0$} & {\cellcolor{black!30}$2$} & {\cellcolor{black!30}$+\infty$}\\ 
		\hline 
		{\cellcolor{green!30}...............} & {\cellcolor{green!30}.......|.......}  & {\cellcolor{green!30}.......[.......} & ............... \\  
		\hline 
		\end{tabular} 
		\end{center}
		\caption{Tableau en surbrillance pour la résolution des inégalités}
	\end{table}
	A la valeur $2$, on n'oublie pas de marquer le signe $....[$ pour montrer que cette valeur est exclue des solutions. Et c'est tout, et le concept peut être extrapolé à des inégalités beaucoup plus complexes!
	\end{tcolorbox}
	\begin{tcolorbox}[title=Remarques,colframe=black,arc=10pt]
	\textbf{R1.} Parfois au lieu de représenter des tableaux comme nous l'avons fait plus haut, certains professeurs (c'est un choix tout à fait artistique) demandent à leurs élèves d'ombrer les cases du tableau et de dessiner des petits cercles à l'intérieur..., ou ils utilisent aussi de petites flèches, ou dessinent le graphique de l'inégalité (cette dernière méthode est certes esthétique mais prend du temps dans des cas complexes mais un exemple est donné plus bas...).\\
	
	\textbf{R2.} Dans le cadre des inégalités de degré supérieur à $1$, il faut (voir plus loin ce que cela veut dire exactement) d'abord aussi déterminer les racines de l'inégalité qui déterminent les intervalles puis par tâtonnements, déterminer quels intervalles sont à rejeter ou à conserver.
	\end{tcolorbox}
	On peut aussi (tout comme les équations) parfois avoir à résoudre un "système d'inéquations". Qu'est-ce que c'est ? : C'est un ensemble d'au moins deux inégalités à résoudre. La particularité du système ? : L'ensemble des solutions du système est l'intersection de toutes les solutions de chaque inégalité.
	
	Par exemple le système suivant de trois inégalités :
	
	Autrement dit, la méthode est la même que la précédente, à la différence près que notre tableau (représentant les domaines de solutions) comportera une ligne supplémentaire par inéquation supplémentaire dans le système plus une ligne de synthèse qui est la projection des domaines de solutions possibles du système.
	
	Ainsi, un système à $n$ inéquations aura un tableau récapitulatif à $n+1$ lignes.
	
	Mathématiquement, les domaines (car il peut y en avoir plusieurs qui sont disjoints) peuvent s'écrire comme un ensemble de domaines:
	
	Les systèmes d'inéquations sont très fréquents dans beaucoup de problèmes de la mathématique, physique, économétrie, etc... Il est donc important de s'entraîner à les résoudre pendant vos études avec l'aide de votre professeur..
	
	Par exemple, voici une représentation possible du domaine de solution du système d'inéquations précédent :
	\begin{center}
	\begin{tikzpicture}

    \draw[gray!50, thin, step=0.5] (-1,-3) grid (5,4);
    \draw[very thick,->] (-1,0) -- (5.2,0) node[right] {$x_1$};
    \draw[very thick,->] (0,-3) -- (0,4.2) node[above] {$x_2$};

    \foreach \x in {-1,...,5} \draw (\x,0.05) -- (\x,-0.05) node[below] {\tiny\x};
    \foreach \y in {-3,...,4} \draw (-0.05,\y) -- (0.05,\y) node[right] {\tiny\y};

    \fill[blue!50!cyan,opacity=0.3] (8/3,1/3) -- (1,2) -- (13/3,11/3) -- cycle;

    \draw (-1,4) -- node[below,sloped] {\tiny$x_1+x_2\geq3$} (5,-2);
    \draw (1,-3) -- (3,1) -- node[below left,sloped] {\tiny$2x_1-x_2\leq5$} (4.5,4);
    \draw (-1,1) -- node[above,sloped] {\tiny$-x_1+2x_2\leq3$} (5,4);

	\end{tikzpicture}
	\end{center}
	
	\subsection{Identités remarquables}\label{calculus remarkable identities}
	Les identités remarquables sont des sortes de relations magiques, qui nous servent le plus souvent pour la factorisation ou la résolution d'équations algébriques.

	Rappelons certaines notions qui ont déjà été vues dans le chapitre de théorie des ensembles de la section d'arithmétique (nous supposons le concept d'élément neutre connu puisque déjà défini):
	\begin{itemize}
		\item Commutativité:
		
		
		\item Associativité:
		
		
		\item Distributivité:
		
	\end{itemize}
	Les mêmes observations sont valables avec l'opération de soustraction bien évidemment dans les domaines de définition adéquats.
	
	\pagebreak
	Nous pouvons vérifier avec des valeurs numériques (en remplaçant chaque nombre abstrait par un nombre choisi au hasard), ou par développement (ce serait mieux, ainsi vous êtes sûr d'avoir compris ce dont quoi nous parlions), que les identités algébriques suivantes sont vérifiées (ce sont les plus connues):
	\begin{enumerate}
		\item Identité du second degré:
		
		\begin{figure}[H]
			\centering
			\includegraphics[scale=0.5]{img/algebra/binomial_identity.jpg}
			\caption[]{Visuel de l'identité binomiale au carré (source: Wikipedia, Auteur:Drini)}
		\end{figure}
	
		\item Identité du troisième degré:
		
		\begin{figure}[H]
			\centering
			\includegraphics[scale=0.5]{img/algebra/binomial_cubic_identity.jpg}
			\caption[]{Visuel de l'identité binomiale au cube (source: Wikipedia, Auteur:Drini)}
		\end{figure}
	\end{enumerate}
	\begin{tcolorbox}[title=Remarque,colframe=black,arc=10pt]
	Nous pouvons très bien poser que $(a+(c+d))^2:=(a+b)^2$ où nous avons bien évidemment posé que  $b:=c+d$ (nous faisons un "abstrait d'abstraction" ou plus couramment: un "changement de variable"). Et dès lors:
	
	\end{tcolorbox}
	Nous pouvons remarquer ainsi qu'en toute généralité, pour calculer le développement de $(a+b)^n$,  nous utilisons le développement de $(a+b)^{n-1}$,  c'est-à-dire calculé avec la valeur précédente de $n$.
	
	Nous pouvons également voir une sorte de motif émerger si nous rassemblons tout cela:
	
	Nous remarquons alors les propriétés suivantes pour $a$ et $b$:
	\begin{enumerate}
		\item  Les puissances de $a$ décroissent de $n$ à $0$ ($a^0=1$, sdonc il n'est pas noté dans le dernier terme).
		
		\item  Les puissances de $b$ croissent de $0$ à $n$ ($a^0=1$, , donc il n'est pas noté dans le premier terme).
		
		\item Dans chaque terme, la somme des puissances de $a$ et $b$ est égale à $n$.
		
		\item Les coefficients multiplicateurs devant chaque terme se calculent en faisant la somme des coefficients multiplicateurs de deux termes du développement obtenu avec la valeur précédente de $n$ (voir la figure ci-dessous). 
	\end{enumerate}
	Les "\NewTerm{coefficients binomiaux}\index{coefficients binomiaux}" peuvent alors être obtenus par construction du "\NewTerm{triangle de Pascal}\index{triangle de Pascal}\label{Pascal's triangle}" ci-dessous:
	\begin{figure}[H]
		\centering
		\includegraphics{img/algebra/pascal_triangle.jpg}
		\caption{Triangle de Pascal}
	\end{figure}
	Où chaque élément est donné par (\SeeChapter{voir section Probabilités page \pageref{combinatorial analysis}}):
	
	avec $n,p\in \mathbb{N}^{+}$.
	\begin{theorem}
	Nous pouvons alors démontrer que:\label{binomial theorem}:
	
	ce qui constitue le fameux "\NewTerm{binôme de Newton}\index{bin\-ome de Newton}" (que nous réutiliserons dans de multiples autres chapitres et sections dans ce livre) aussi parfois nommé "\NewTerm{théorème binomial}\index{th\'eor\'eme binomial}".
	
	Plus explicitement\label{binomial coefficient development}:
	
	aussi appelé "\NewTerm{série binomiale}\index{s\'erie binomiale}".
	\end{theorem}
	\begin{dem}
	Cette relation peut être prouvée tout simplement par induction en supposant vraie la relation précédente et en la calculant pour le rang $1$ :
	
	Nous obtenons alors:
	
	La relation est vraie pour le rang $n + 1$, elle est donc vraie pour tout $n$.
	\begin{flushright}
		$\blacksquare$  Q.E.D.
	\end{flushright}
	\end{dem} 
	
	En appliquant deux fois la formule binomiale, on peut obtenir pour le plaisir la formule du trinôme suivante :
	
	Concernant les identités remarquables à valeurs négatives, il n'est pas nécessaire de mémoriser l'emplacement du signe "$-$". Il suffit de faire un changement de variable et une fois le développement fait on remplace la variable en arrière (changement de variable inverse).
	\begin{tcolorbox}[colframe=black,colback=white,sharp corners]
	\textbf{{\Large \ding{45}}Exemple:}\\
	
	et ainsi de suite pour toute puissance finie $n$.
	\end{tcolorbox}
	On peut bien sûr mélanger les genres (...) comme (exemple particulièrement célèbre) :
	
	et quelques relations pratiques supplémentaires remarquables qui sont souvent utilisées dans les petites classes pour les exercices :
	
	\begin{tcolorbox}[title=Remarque,colframe=black,arc=10pt]
	Lorsque, du côté droit (sous une forme numérique simplifiée) l'enseignant demande à ses élèves comme exercice d'obtenir la factorisation de la gauche de l'égalité, il n'y a pas d'autre moyen que de procéder par tests successifs.
	\end{tcolorbox}
	Pour information, nous obtenons également le développement célèbre suivant qui est immédiatement déductible de ce que nous avons vu auparavant :
	
	qui est valable pour toute valeur de $b$ et est nommée le "\NewTerm{développement binomial}\index{d\'eveloppement binomial}\label{binomial expansion}". 
	
	Pour les petits valeurs de $b$, nous pourrions négliger tous les termes impliquant $b^2$ ou une puissance supérieure de $b$, nous donnant la relation d'approximation :
	
	avec $b\ll 1$.
	
	Bien sûr, il y a encore un beaucoup plus grand nombre de relations utiles (dont une partie découle d'une généralisation de celles présentées ci-dessus) que le lecteur découvrira par ses propres raisonnements et en fonction de sa pratique.
	
	\begin{tcolorbox}[title=Remarque,colframe=black,arc=10pt]
	Il est bien sûr possible de multiplier des polynômes entre eux et de distribuer les termes multiplicatifs. Inversement, il est souvent demandé aux élèves des petites classes de faire la procédure inverse ("factoriser" ou "décomposer" un polynôme) afin qu'ils s'habituent à la manipulation des identités remarquables. Décomposer en un produit de facteurs est une opération importante en mathématiques, puisqu'il est ainsi possible de réduire l'étude d'expressions compliquées à l'étude de plusieurs expressions plus simples.
	\end{tcolorbox}
	
	\subsection{Polynômes}\label{polynomial}
	\textbf{Définition (version naïve \#\mydef):} Nous appelons "\NewTerm{polynôme algébrique $P(x)$}\index{polyn\-ome alg\'ebrique univari\'e}" une fonction de degré $n\in \mathbb{N}$ qui s'écrit:
	
	ou de façon plus condensée par:
	
	où le "\NewTerm{facteur dominant}\index{facteur dominant}" d'un polynôme est le coefficient du monôme de plus haut degré $a_n$ et le "\NewTerm{terme dominant}\index{terme dominant}" est simplement $a_nx^n$. Si $a_n=1$ est égal à $1$ nous parlons alors de "\NewTerm{polynôme normalisé}\index{polyn\-ome normalis\'e}".
	 
	 Si l'on considère chaque coefficient du polynôme comme la composante d'un vecteur\label{polynomial vector}, alors on peut manipuler les polynômes comme des espaces vectoriels notés $P_n(\mathbb{R})$ et de dimension $n+1$.
	 
	\begin{tcolorbox}[title=Remarques,colframe=black,arc=10pt]
	\textbf{R1.} L'indice $n$ de $P (x)$ est la plupart du temps omis comme explicitement défini dans l'expression du polynôme lui-même.\\
	
	\textbf{R2.} Le lecteur qui a lu la section de Théorie des Ensembles, se rappellera probablement que l'ensemble de tous les polynômes de degré $n$ ou inférieur forme une structure d'espace vectoriel !
	\end{tcolorbox}
	Nous supposerons sans preuve et évident (car nous sommes des ingénieurs ou physiciens mais pas des mathématiciens...) que si nous notons $\text{deg}(P_n(x))=n$ le degré d'un polynôme, et que $P_n(x) $ et $P_m(x)$ sont tous deux des polynômes, que alors :
	
	et:
	
	
	\textbf{Définition (version théorie des ensembles \#\mydef):} Soit $k$ un anneau (\SeeChapter{voir section Théorie des Ensembles page \pageref{ring}}) et $n\in \mathbb{N}^{*}$, "\NewTerm{l'anneau polynomial}\index{anneau polynomial}\label{polynomial ring}" en $n$ indéterminées (ou variables) $k[X_1,...,X_n]$ est construit à partir d'un polynôme élémentaire, nommé le "\NewTerm{monôme}\index{mon\-ome}" de la forme:
	
	où $\lambda \in k$ est le "\NewTerm{coefficient du monôme}\index{coefficient du mon\-ome}", $m_1,m_2,...,m_n$ sont des entiers positifs non nuls et où $X_1 ^{m_1}...X_m^{m_n}$ forment la "\NewTerm{partie littérale du monôme}\index{partie litt\'erale du mon\-ome}". Ainsi, par construction, un polynôme est une somme d'un nombre fini de monômes nommés les "\NewTerm{termes du polynôme}\index{termes d'un polyn\-ome}".
	
	Ainsi, le cas particulier courant utilisé dans les petites classes et présenté au début est $k[X]$, c'est-à-dire l'anneau de polynômes univariés à coefficients en $k$. En effet, la plupart du temps, les ingénieurs et étudiants travaillent avec des "anneaux de polynômes univariés à coefficient en $\mathbb{R}$" et notés $\mathbb{R}[X]$. Tout élément de $k[X]$ s'écrit donc :
	
	avec $a_i\in k$ (la majorité du temps $a_i\in \mathbb{R}$ ), $i=0...n$ et $n\in \mathbb{N}$.
	\begin{tcolorbox}[title=Remarques,colframe=black,arc=10pt]
	\textbf{R1.} Notez que les puissances de $i$ sont toujours positives ou nulles dans $k[X]$!!!\\
	
	\textbf{R2.} Nous disons que deux polynômes sont "similaires" s'ils ont la même partie littérale.
	\end{tcolorbox}
	\begin{figure}[H]
		\centering
		\includegraphics{img/algebra/polynomials.jpg}
		\caption[Quelques polynômes tracés avec R.3.2.1]{Quelques polynômes tracés avec R.3.2.1 (voir le livre compagnon sur R)}
	\end{figure}
	Le "\NewTerm{comportement limite}\index{comportement limite}" d'une fonction décrit ce qui arrive à la fonction quand $x\rightarrow \pm\infty$. Le degré d'un polynôme et le signe de son coefficient directeur (ie dominant) dictent son comportement limite. En particulier:
	\begin{itemize}
		\item Si le degré d'un polynôme $f(x)$ est pair et que le coefficient directeur est positif, alors $f(x)\rightarrow +\infty$ quand $x\rightarrow \pm\infty$.
		
		\item Si $f(x)$ est un polynôme de degré pair avec un coefficient directeur négatif, alors $f(x)\rightarrow -\infty$ quand $x\rightarrow \pm\infty$.
		
		\item Si $f(x)$ est un polynôme de degré impair avec un coefficient directeur positif, alors $f(x)\rightarrow -\infty$ quand $x\rightarrow +\infty$ et $f(x)\rightarrow +\infty$ quand $x\rightarrow +\infty$.
 
		\item Si $f(x)$ est un polynôme de degré impair avec un coefficient directeur négatif, alors $f(x)\rightarrow +\infty$ quand $x\rightarrow -\infty$ et $f(x)\rightarrow -\infty$ quand $x\rightarrow +\infty$.
	\end{itemize}
	Ces résultats sont résumés dans le tableau ci-dessous :	
	\begin{table}[H]
		\centering
		\begin{tabular}{|l|c|c|}
		\hline
		\rowcolor[HTML]{C0C0C0} 
		\textbf{Degré du polynôme}     & \multicolumn{2}{l|}{\cellcolor[HTML]{C0C0C0}\textbf{Coefficient directeur}}                                                                                                                                                                                                                                      \\ \hline
		                                      & $+$                                                                                                                                               & $-$                                                                                                                                               \\ \hline
		\cellcolor[HTML]{C0C0C0}\textbf{Pair} & $f(x)\rightarrow +\infty$ quand $x\rightarrow \pm\infty$                                                                                            & $f(x)\rightarrow -\infty$ quand $x\rightarrow \pm\infty$                                                                                             \\ \hline
		\cellcolor[HTML]{C0C0C0}\textbf{Impair}  & \begin{tabular}[c]{@{}c@{}}$f(x)\rightarrow -\infty$ quand $x\rightarrow -\infty$\\ $f(x)\rightarrow -\infty$ quand $x\rightarrow -\infty$\end{tabular} & \begin{tabular}[c]{@{}c@{}}$f(x)\rightarrow +\infty$ quand $x\rightarrow -\infty$\\ $f(x)\rightarrow -\infty$ quand $x\rightarrow +\infty$\end{tabular} \\ \hline
		\end{tabular}
		\caption{Comportement limite des polynômes}
	\end{table}
	\textbf{Définition (\#\mydef):} Nous nommons "\NewTerm{point tournant}\index{point tournant}" un point à partir duquel le graphique change de direction de croissant à décroissant ou de décroissant à croissant (ou autrement dit : où les dérivées sont égales à zéro).
	\begin{figure}[H]
		\centering
		\includegraphics[scale=0.5]{img/algebra/turning_points.jpg}
		\caption{Points tournants explicitement mis en évidence}
	\end{figure}
	La fonction $f$ ci-dessus est un polynôme de degré $4$ et a $3$ points tournants. Le nombre maximum de points de retournement d'une fonction polynomiale est toujours un de moins que le degré de la fonction.
	
	\textbf{Définition (\#\mydef):} Nous nommons "\NewTerm{racine}\index{racine d'un polyn\-ome}" ou "\NewTerm{zéro d'un polynôme (univarié)}\index{z\'ero d'un polyn\-ome univari\'e}", les valeurs $x$ telles que "\NewTerm{l'équation polynomiale}\index{\'equation polynomiale}" $P(x)=0$ est satisfaite à la condition qu'au moins un des $a_n$ avec $n>0$ ne soit pas nul.
	
	Si le polynôme est normalisé\footnote{En effet, si vous développez l'expression donnée, vous verrez que le facteur devant le terme d'ordre le plus élevé $x^n$ est $1$.} et admet une ou plusieurs racines $r_n$ nous pouvons alors évidemment le factoriser comme suit (nous le prouverons rigoureusement plus loin) :
	
	de sorte que lorsque $x$ prend la valeur d'une des racines $r_n$, l'expression ci-dessus est zéro. C'est ce que l'on nomme par convention "\NewTerm{factoriser un polynôme}\index{factoriser un polyn\-ome}".
	
	Si le polynôme n'est pas normalisé alors la factorisation sera évidemment donnée par :
	
	Les identités algébriques sont des formes particulières de fonctions polynomiales. En effet, considérons une constante $c$ et une variable $x$ et :
	
	On voit que si nous posons :
	
	nous retombons sur:
	
	\textbf{Définitions (\#\mydef):}
	\begin{enumerate}
		\item[D1.] Un polynôme à une indéterminée est nommé "\NewTerm{polynôme univarié}\index{polyn\-ome univari\'e}", un polynôme avec plus d'une indéterminée est nommé "\NewTerm{polynôme multivarié}\index{polyn\-ome multivari\'e}". Un polynôme à deux indéterminées est nommé "\NewTerm{polynôme bivarié}\index{polyn\-ome bivari\'e}".
		
		Un exemple célèbre en mathématiques pures d'un polynôme multivarié donné à plusieurs reprises dans les cours de premier cycle universitaire est (certains lecteurs le reconnaîtront probablement...):
		
	
		\item[D2.] Dans le cas de polynômes à plus d'une indéterminée, un polynôme est dit "\NewTerm{homogène de degré $n$}\index{polyn\-ome homog\'ene de degr\'e $n$} si tous ses termes non nuls sont de degré $n$ (l'exemple juste au-dessus est un tel polynôme !).
	\end{enumerate}
	
	\pagebreak
	\subsubsection{Relations générales entre racines et coefficients}
	Voyons maintenant une relation intéressante entre les racines d'un polynôme et ses coefficients.

	Pour cela, considérons le polynôme suivant :
	
	Et désignons ses racines par $r_1,r_2,\ldots,r_n$ ($n$ racines ne signifie pas nécessairement $n$ racines distinctes...). Nous avons alors :
	
	Nous en déduisons alors en développant la dernière égalité et en identifiant aux coefficients que :
	
	Nous pouvons le réécrire différemment en définissant les nombres $\sigma_1,\sigma_2,\ldots,\sigma_n$ comme suit :
	
	Inversement si les $r_1,r_2,\ldots,r_n$ satisfont le système précédent, alors ce sont les racines d'un polynôme normalisé de la forme :
	
	Remarquons que pour le fameux cas $n=2$ que nous avons trouvé :
	
	
	
	\subsubsection{Multiplication de polynômes}\label{polynomials multiplication}
	Le produit des deux polynômes $P$ (de degré $n$) et $Q$ (de degré $m$) est trivialement donné par :
	
	Il n'y a pas grand chose à dire de plus concernant la multiplication des polynômes pour les besoins de ce livre.
	
	\subsubsection{Division Euclidienne de Polynômes}\label{polynomials division}
	Plaçons nous à présent dans l'anneau $k[X]$. Si $P(x)\in k[X]$, nous notons $\text{deg}(P)$ le degré du polynôme $P(X)$ à coefficients dans un anneau $k$ (les réels ou les complexes... peu importe!)

	\begin{tcolorbox}[title=Remarque,colframe=black,arc=10pt]
	Par convention:
	
	\end{tcolorbox}
	\begin{theorem}
	Soit:
	
	avec $k,m>0$. Alors il existe deux polynômes uniques $q(X),r(X)\in k[X]$ tels que:
	
	et:
	
	où $q(X)$ est le "\NewTerm{polynôme quotient}\index{polyn\-ome quotient}" et $r(X)$ est le "\NewTerm{polynôme résiduel}\index{polyn\-ome r\'esiduel}".
	\end{theorem}
	\begin{dem}
	Si $u (X) = 0$ le résultat est évident. Supposons $u(X)\neq 0$ et prouvons l'existence par récurrence sur le degré $k$ de $u (X)$.
	
	Si $k = 0$ alors $q (X) = 0$ (puisque $m>0$) et donc $r (X) = u (X)$ fera le travail.
	
	Supposons maintenant ce résultat appliquable pour tout $k\leq n$... (cette supposition est totalement gratuite...):
	
	Soit $u(X)$ de degré $k=n+1$. Si $m>n+1$ alors $q (X) = 0$ et $r (X) = u (X)$ peuvent également faire le travail.
	
	Sinon, si $m\leq n+1$ alors en écrivant ($u_{n+1}$ est le $n+1$-ième coefficient du polynôme $u(X)$ et $v_m$ le $m$-ième coefficient de $v(X)$):
	
	on réduit alors $u(X)$ à un polynôme de degré $\leq n$ puisque $v(X)$ est de degré $m$ (et qu'il existe) !
	
	Effectivement le terme:
	
	supprime (au moins) le terme de plus haut degré $u_{n+1}X^{n+1}$.
	
	Par l'hypothèse d'induction, il existe $f(X)$ et $g(X)$ tels que :
	
	avec $\text{deg}(g)<m$. Donc après réorganisation:
	
	et dès lors:
	
	fait le travail!
	
	Donc par récurrence on voit que la division euclidienne existe dans l'anneau polynomial $k[X]$
	\begin{flushright}
		$\blacksquare$  Q.E.D.
	\end{flushright}
	\end{dem}
	\begin{tcolorbox}[title=Remarque,colframe=black,arc=10pt]
	Cette preuve nous a permis dans la section de Théorie des Ensembles de montrer que cet anneau est "principal".
	\end{tcolorbox}
	\begin{tcolorbox}[colframe=black,colback=white,sharp corners]
	\textbf{{\Large \ding{45}}Exemple:}\\\\
	Nous ne verrons qu'un seul exemple car l'idée est toujours la même. On veut diviser $x^3+x^2$ par $x-1$, nous obtenons:
		
	\end{tcolorbox}
	
	
	\pagebreak
	\subsubsection{Théorème de factorisation des polynômes}\label{factorization theorem}
	Nous allons maintenant prouver un théorème important qui est en fait illustré à l'origine (entre autres) par les identités remarquables que nous avons vues ci-dessus :
	
	\begin{theorem}
	Si un polynôme $P(X)\in k[\mathbb{K}]$ avec des coefficients dansn $k$ de degré $n\geq 1$ a une racine $x=r$ dans l'anneau $k$, alors on peut factoriser $P(x)$ par $(x - r)$ tel que :
	
	où $Q$ est un polynôme de degré $n-1$ (et peut donc être un simple monôme).
	
	Autrement dit, "\NewTerm{factoriser un polynôme}\index{factorisation d'un polyn\-ome}", c'est l'écrire comme un produit de monômes (dans le cas général : de polynômes). Lorsqu'elle n'est pas seulement appliquée aux polynômes ou aussi simplement aux nombres, la factorisation est une opération qui transforme une somme en un produit!
	\end{theorem}
	\begin{dem}
	L'idée est d'effectuer la division euclidienne de $P(x)$ par $(x-r)$. D'après le théorème précédent, il existe un couple $(Q, R)$ de polynômes tels que :
	
	et d'après le résultat du théorème précédent sur la division euclidienne :
	
	Mais $\text{deg}(x-r)=1$, donc $\text{deg}(R)=0$ (ou $-\infty$ par convention). $R(x)$ est donc une fonction polynomiale constante. De plus, par hypothèse, $r$ est une racine de $P(x)$. On a donc :
	
	Donc $R(r)=0$. Dès lors $R(x)$ est le polynôme zéro et le théorème est pratiquement prouvé. Il reste à prouver que $\text{deg}(Q)=n-1$, ce qui est une conséquence immédiate de la relation :
	
	Ainsi:
	
	\begin{flushright}
		$\blacksquare$  Q.E.D.
	\end{flushright}
	\end{dem}
	A partir de cette propriété de factoriser un polynôme, nommé parfois "\NewTerm{théorème de factorisation}\index{théorème de factorisation}", on peut donner un avant-goût d'un théorème beaucoup plus important :
	\begin{theorem}
	Montrons que si on a une fonction polynomiale $P(X)\in k[X]$ de degré $n\in \mathbb{N}$ avec des coefficients en $k$, alors elle a au plus un nombre fini $ n$ de racines (certaines pouvant être confondues) dans $k$.
	\end{theorem}

	\begin{dem}
	Premièrement, parce que $P(x)$ a un degré (ordre), $P(x)$ n'est pas une fonction polynomiale nulle. Alors, argumentons par l'absurde :
	
	Si la fonction $P(x)$ a $n$ racines avec $p>n$ (plus de racines que de degrés...), en notant ces racines $r_1,...,r_p$, on a, par le théorème de factorisation précédent (appliqué $p$ fois):
	
	où $Q$ est un polynôme de degré :
	
	Maintenant, puisque par définition un polynôme est un polynôme si et seulement si son degré (ordre) appartient à $\mathbb{N}$, le polynôme $Q$ doit être le polynôme nul tel que :
	
	Il s'ensuit que :
	
	Ceci contredit l'hypothèse initiale selon laquelle $P$ n'est pas le polynôme zéro, d'où :
	
	\begin{flushright}
		$\blacksquare$  Q.E.D.
	\end{flushright}
	\end{dem}
	
	\subsubsection{Équation diophantienne}\label{diophantine equation}
	Si l'on généralise la notion de polynôme univarié à plusieurs variables telles que :
	
	alors on nomme "\NewTerm{équation diophantienne}\index{\'equation diophantienne}" une équation de la forme :
	
	où $P$ est un polynôme à coefficients entiers (ou rationnels) dont on cherche les radicaux (racines) strictement dans $\mathbb{N}$ ou $\mathbb{Q}$. Les équations diophantiennes conventionnelles sont par exemple :
	\begin{itemize}
		\item Les équations diophantiennes linéaires :
		
		
		\item Les triplets pythagoriciens :
			
	\end{itemize}
	Pour la preuve générale de cette dernière, le lecteur devra attendre un peu le temps que les rédacteurs du présent livre aient le temps de comprendre la preuve traditionnelle avant de pouvoir la simplifier (...).
	
	\subsubsection{Polynômes et équations univarié du premier ordre}
	Étant donné la fonction linéaire :
	
	Si $a\neq 0$ alors la première l'équation :
	
	a une racine analytique simple donnée trivialement par :
	
	tel que $P_1(r)=0$.
	
	Si $b=0$ ce polynôme est nommé uen "\NewTerm{fonction affine}\index{fonction affine}".
	
	\begin{tcolorbox}[title=Remarques,colframe=black,arc=10pt]
	\textbf{R1.} Si les coefficients du polynôme univarié de degré $1$ sont tous tels que $a,b\in \mathbb{R}$ alors la racine appartient aussi à $\mathbb{R}$.\\
	
	\textbf{R2.} Si un des coefficients du polynôme univarié de degré $1$ appartient à $\mathbb{C}$ alors la racine appartient aussi à $\mathbb{C}$.\\
	
	\textbf{R3.} Si les deux coefficients du polynôme univarié de degré $1$ appartiennent à $\mathbb{C}$ alors la racine appartient aussi à $\mathbb{C}$ ou $\mathbb{R}$.\\
	
	\textbf{R4.} On dit que deux équations polynomiales sont "\NewTerm{équivalents}\index{polyn\-omes \'equivalents}" si elles admettent les mêmes solutions.
	\end{tcolorbox}
	Voici également quelques propriétés pour les polynômes univariés du premier ordre que nous donnons sans démonstration car elles nous semblent très très intuitives (sauf sur demande de lecteurs) :
	\begin{enumerate}
		\item[P1.] Si nous sommons (ou respectivement soustrayons) un même nombre à chaque membre de l'équatoin (à gauche du signe "$=$", et aussi à droite), on obtient une équation qui a les mêmes solutions que l'équation d'origine (et ce quelle que soit est son degré).
	
		\item[P2.] Si nous multiplions (ou divisons respectivement) chaque membre d'une équation (à gauche du signe "$=$", et aussi à droite) par un même nombre non nul, nous obtenons une équation qui a les mêmes solutions que l'équation d'origine (et ce quel que soit son degré).
	\end{enumerate}
	La méthode doit être suffisamment générale pour être appliquée à toutes les équations du même genre, être construite sur les quatre opérations arithmétique de base (addition, soustraction, multiplication et division) et l'extraction de racines. On peut trouver les solutions (racines), d'une équation grâce à ses coefficients, en n'utilisant que les opérations précédentes (c'est-à-dire sous une "\NewTerm{forme analytique}\index{forme analytique}"), on dit alors que le l'équation peut être résolue par "\NewTerm{radicaux}\index{radicaux}".
	
	\paragraph{Résolution par référence circulaire}\mbox{}\\\\
	De nombreux cadres dans les petites, moyennes et entreprises internationales ne savent plus résoudre les équations linéaires du premier degré (et au-delà aussi bien sûr...). Je pense donc qu'il serait intéressant pour l'étudiant de voir comment procède la majorité des gens qui ne sont pas ou plus capables de résoudre de tels problèmes de niveau secondaire (c'est particulièrement intéressant quand vous savez qu'ils sont payés assez cher pour résoudre de tels "problèmes compliqués" selon employeur). La façon la plus simple de voir cela c'est d'apprendre comment je les voirs utiliser le tableur Microsoft Excel. En effet, considérons l'exemple suivant:

	Une entreprise réalise un chiffre d'affaires (revenu) de $1'500$.- et doit payer $1'000$.- de frais fixes. Il a $40\%$ d'impôt payable après déduction. L'entreprise souhaite faire un don (déductible d'impôt) tel que le montant de ce don représente $10\%$ d'impôt après déduction.

	Ceci s'écrit mathématiquement simplement si on note $R$ le chiffre d'affaires, $C$ les frais fixes, $D$ le don, $t$ le taux d'imposition :
	
	Après quelques petits réarrangements triviaux, nous obtenons :
	
	Cela correspond bien à la taxe de $10\%$ après déduction puisque :
	
	Et voici comment les (mauvais) cadres utilisent un tableur comme Microsoft Excel et « résolvent » cela (vous pouvez me faire confiance, je vois vraiment des gens  faire cela en entreprise parce qu'ils ne sont plus capables de résoudre des problèmes niveau lycée sur papier). Ils écrivent d'abord dans le tableur (qui détecte automatiquement l'incohérence de l'utilisateur en l'indiquant par une flèche bleue pour mettre en évidence la "référence circulaire"):
	\begin{figure}[H]
		\centering
		\includegraphics{img/algebra/circular_reference_microsoft_excel_01.jpg}
		\caption{Résoudre une équation simple par référence circulaire dans Microsoft Excel}
	\end{figure}
	Validé cela donne:
	\begin{figure}[H]
		\centering
		\includegraphics{img/algebra/circular_reference_microsoft_excel_02.jpg}
	\end{figure}
	Pour résoudre cette équation, il faut activer dans les options du logiciel l'autorisation d'utiliser l'itération (la position de cette option dépend de la version du tableur).
	\begin{figure}[H]
		\centering
		\includegraphics[scale=0.8]{img/algebra/circular_reference_microsoft_excel_03.jpg}
		\caption[]{Option de résolution d'équations par récurrenc dans Microsoft Excel 14.0.7183}
	\end{figure}
	Ce qui donne alors:
	\begin{figure}[H]
		\centering
		\includegraphics{img/algebra/circular_reference_microsoft_excel_04.jpg}
	\end{figure}
	Donc, comme on peut le voir, c'est une façon assez peu efficace de résoudre un simple problème de niveau lycée (regrettable pour des gestionnaires ou analystes  qui ont un MBA...!).

	\subsubsection{Polynômes et équations univarisé d'ordre deux}\label{second order polynomials}
	Soit le polynôme univarié suivant à coefficient dans $\mathbb{R}$ (trinôme du second degré) :
	
	Si nous représentons ce polynôme univarié sur le plan, cela nous donne :
	\begin{figure}[H]
		\centering
		\includegraphics{img/algebra/polynomial_orientation.jpg}
		\caption{Représentation des polynômes en fonction du signe du terme de degré $2$}
	\end{figure}
	Si on prend la dérivée de cette fonction (\SeeChapter{voir section Calcul Différentiel et Intégral page \pageref{differential calculus}}) et qu'on cherche en quel point la dérivée est égale à zéro, on trouvera toujours l'optimum sur le point d'inflexion (\SeeChapter{voir section Calcul Différentiel et Intégral page \pageref{inflection point}}) de la parabole (qui correspond aussi à son axe de symétrie) :
	\begin{figure}[H]
		\centering
		\includegraphics{img/algebra/polynomial_optimum.jpg}
		\caption{Point d'inflexion de la tangente}
	\end{figure}
	Si $a\neq 0$, nous avons alors:
	
	Nous avons alors une "\NewTerm{racine double}\index{racine double}\label{double root}" (ou "\NewTerm{racine de multiplicité $2$}") que l'on note par :
	
	tel que $P(r_{1,2})=0$ et nous définissons un nouveau terme nommé parfois le "\NewTerm{déterminant du polynôme}" et le plus souvent le "\NewTerm{discriminant du polynôme}\index{discriminant du polyn\-ome}\label{discriminant}":
		
	Finalement\label{second order polynomial roots}:
	
	Si le polynôme univarié du second ordre en $x$ a deux racines, nous pouvons alors le factoriser sous une forme irréductible (suivant le théorème de factorisation démontré précédemment) sous la forme suivante :
	
	Nous prouvons aussi facilement à partir de l'expression de la racine en faisant de l'algèbre simple les "\NewTerm{relations de Viète}\index{relations de Vi\'ete}\label{vieta relations}" (sur demande des lecteurs nous pouvons détailler les développements nécessaires) :
	
	Selon le signe de $2a$ et celui du discriminant $\Delta$, nous avons:
	\begin{figure}[H]
		\centering
		\includegraphics{img/algebra/polynomial_second_order_signature.jpg}
	\end{figure}
	Dès lors:
	\begin{itemize}
		\item Si $\Delta<0$ notre polynôme n'a pas de racines réelles et ne peut pas être factorisé dans une multiplication de monômes avec des facteurs réels ($
\mathbb{R}$) mais avec un facteur complexe ($\mathbb{C}$). Par conséquent (il est recommandé d'avoir lu d'abord la partie sur les Nombres Complexes à la page 	\pageref{complex numbers} dans la section Nombres de ce livre) :
		
		et nous savons qu'on peut écrire n'importe quel nombre complexe sous une forme condensée (formule d'Euler) et comme les racines complexes d'un polynôme du second degré sont conjuguées (on connaît déjà ce jargon) nous avons:
		
		où (rappel) $r$ est le module des racines complexes (module qui est égal pour les deux) et $\varphi$ l'argument des racines complexes (égal en valeur absolue).
		
		\item Si $\Delta=0$ l'équation polynomiale a alors une seule solution qui est évidemment :
		
		
		\item Si $\Delta>0$ l'équation polynomiale a alors deux solutions définies par les relations générales que nous avons déjà données plus haut :
		
	\end{itemize}
	A propos du cas complexe, prenons comme exemple le polynôme quadratique suivant :
	
	qui n'admet que deux racines complexes que sont $\mathrm{i}$ et $-\mathrm{i}$. Dans le plan réel ce polynôme sera représenté avec Maple 4.00b par :
	
	\texttt{>plot(x\string^2+1,x=-5..5);}
	\begin{figure}[H]
		\centering
		\includegraphics{img/algebra/polynomial_complex_solutions_in_real_plane.jpg}
		\caption[]{Exemple de tracé d'un polynôme de degré $2$ qui n'admet que des solutions complexes}
	\end{figure}
	où l'on voit bien qu'il n'y a pas de solutions (zéros) réelles. En nous plaçant dans les complexe $\mathbb{C}$, nous avons par contre :
	
	\texttt{>plot3d(abs((re+I*im)\string^2+1),re=-2..2,im=0..2,view=[-2..2,-2..2,0..2],\\
	orientation=[-130,70],contours=50,style=PATCHCONTOUR,axes=frame,\\
	grid=[100,100],numpoints=10000);}
	\begin{figure}[H]
		\centering
		\includegraphics{img/algebra/polynomial_complex_solutions_in_complex_plane.jpg}
		\caption[]{Le même polynôme mais en jouant avec la représentation complexe}
	\end{figure}
	où les deux zéros sont visibles sur l'axe imaginaire à $-1$ et $+1$. Évidemment quand c'est la première fois que l'on voit une fonction représentée sur une figure prenant en compte les valeurs complexes on essaie de trouver où est la parabole correspondante au cas purement réel. Pour ce faire, on coupe simplement la surface ci-dessus en deux sur l'axe imaginaire et on obtient alors :
	
	\texttt{>plot3d(abs((re+I*im)\string^2+1),re=-2..2,im=0..2,view=[-2..2,-2..2,0..2],\\
	orientation=[-130,70],contours=50,style=PATCHCONTOUR,axes=frame,\\
	grid=[100,100],numpoints=10000);}
	\begin{figure}[H]
		\centering
		\includegraphics{img/algebra/polynomial_complex_solutions_in_complex_plane_cutted.jpg}
		\caption[]{A little zoom on the same polynomial}
	\end{figure}
	où l'on retrouve clairement notre parabole visible sur la surface découpée. On peut donc se demander si les nombres complexes sont une extension naturelle de notre espace conventionnel au-delà de nos sens physiques et de nos appareils de mesure communs... ?!

	\paragraph{Équations Irrationnelles}\mbox{}\\\\
	Le praticien doit toujours prendre l'habitude de vérifier la solution dans l'équation d'origine pour être sûr de la validation du domaine de définition de la fonction. En effet, il existe des solutions à la résolution d'équations qui ne satisfont pas l'équation d'origine et c'est ce que l'on nomme des  "\NewTerm{solutions étrangères}\index{solution polynomiale \'etrang\'ere}" et c'est typiquement le cas des équations irrationnelles.

	\textbf{Définition  (\#\mydef):} Une "\NewTerm{\'equation irrationnelle}\index{\'equation irrationnelle}" est une équation où l'inconnue est sous un radical (c'est-à-dire dans le cas typique : sous une racine carrée).
	
	\begin{tcolorbox}[colframe=black,colback=white,sharp corners]
	\textbf{{\Large \ding{45}}Exemple:}\\\\
	Considérons l'équation suivante :
	
	Pour le résoudre d'abord, nous pouvons redistribuer :
	
	Nous mettons le tout au carré:
	
	Nous simplifions un peu :
	
	et encore:
	
	Nous mettons au carré une deuxième fois:
	
	Nous simplifions une dernière fois:
	
	Nous obtenons deux solutions triviales qui sont $ x_1 = -2 $ et $ x_2 = 2 $. Mais seule la solution $ x_1 = -2 $ satisfait l'équation proposée. En effet, si on met la première solution dans l'équation d'origine nous obtenons :
	
	Mais si nous mettons la deuxième solution:
		
	\end{tcolorbox}
	
	\pagebreak
	\paragraph{Nombre d'or}\mbox{}\\\\
	Il existe un polynôme de degré deux dont la solution est fameuse de par le monde. Ce nombre est appelé la "\NewTerm{nombre d'or}\index{nombre d'or}\label{golden ratio}" ou "\NewTerm{divine proportion}\index{divine proportion}" et se retrouve en architecture, esthétique ou encore en phyllotaxie (c'est-à-dire dans la disposition des feuilles autour de la tige des plantes).
	
	Ce nombre vaut: 
	
	Le nombre d'or est aussi un nombre algébrique (tout nombre complexe qui est une racine d'un polynôme univarié non nul avec des coefficients rationnels) et même un entier algébrique (nombre complexe qui est une racine d'un polynôme monique avec des coefficients dans $\mathbb{Z}$) car c'est la solution de :
	
	Historiquement, le lecteur doit savoir que deux quantités $a$ et $b$ sont dites « dans le ratio du nombre d'or $\varphi$ » si leur rapport est le même que le rapport de leur somme à la plus grande des deux quantités :
	
	\begin{figure}[H]
		\centering
		\includegraphics[scale=1]{img/algebra/golden_ratio.jpg}
		\caption{Illustration du nombre d'or}
	\end{figure}
	Une méthode pour trouver la valeur de $\varphi$ consiste à commencer par la fraction de gauche. En simplifiant la fraction et en substituant dans $b/a = 1/\varphi$ :
	
	Dès lors:
	
	Multipliant par $\varphi$ donne:
	
	qui peut être réarrangé en :
	
	En utilisant la formule quadratique, deux solutions sont obtenues :
	
	Parce que $\varphi$ est le rapport entre des quantités positives, $\varphi$ est nécessairement positif :
	
	Introduisons maintenant une relation utile pour lorsque nous étudierons la fonction génératrice ordinaire (voir page \pageref{ordinary generating function}) relativement à la suite de Fibonacci. Rappelons que nous sommes partis de :
	
	we then have:
	
	C'est pourquoi $-1/\varphi$ est nommé le "\NewTerm{conjugué du nombre d'or}" !
	
	
	\subsubsection{Polynômes et équations univarisé d'ordre trois}
	Même s'il est rare de résoudre une telle chose en physique théorique ou en ingénierie, résoudre un polynôme univarié de degré $3$ est assez récréatif et montre un bon exemple d'un raisonnement mathématique déjà mature (nous avons ces développements grâce à Scipione del Ferro et Jérôme Cardan mathématiciens du XVIe siècle...).
	
	Étant donnée l'équation :
	
	avec les coefficients tous dans $\mathbb{R}$ (pour commencer...). Dans un premier temps, le lecteur pourra voir que les raisonnements que nous avons appliqués pour les polynômes de degrés inférieurs à $3$ coincent rapidement (excepté pour des cas particuliers simplistes bien sûr...).

	Nous allons contourner le problème par des changements de variables subtils mais tout à fait justifiés.
	
	Ainsi, rien ne nous empêche de poser que:
	
	et qu'en divisant le polynôme de degré $3$ par $a$ d'écrire:
	
	En regroupant les termes de même ordre:
	
	et posons (rien, mais alors absolument rien ne nous l'interdit):
	
	où (1) est connu si et seulement si $X$ est connu et où $p, q$ sont de toute façon connus.

	Le polynôme\footnote{La première fois que j'ai dû résoudre un tel polynôme, c'était pour calculer la nutation d'un gyroscope et la deuxième fois c'était pour le calcul de l'horizon d'un Trou Noir basé sur la métrique de Schwarzschild avec une constante cosmologique qui était en unités naturelles : $\dfrac{\Lambda}{3}r^3-r-2M=0$}:
	
	étant de degré impair, il admet - comme permet de le constater tout tracé visuel d'un tel polynôme à coefficients réels - au moins une racine réelle, appelée "\NewTerm{racine certaine}\index{racine certaine}" (vérifiez! Vous verrez bien par une représentation graphique d'un polynôme de degré impair que cela est trivial).
	
	Le membre de gauche de cette équation est un "\NewTerm{trinôme monique}\index{trin\-ome monique}" appelé aussi la "\NewTerm{cubique déprimée}\index{cubique d\'eprim\'ee}", car le terme quadratique a un coefficient $0$.
	
	Maintenant, nous faisons un autre changement de variable (nous en avons tout à fait le droit) subtil: 
	
	en imposant la condition que $u,v$ doivent être tels que $3uv=-p$ (rien ne nous empêche d'imposer une telle contrainte) et nous avons alors:
	
	Dès lors nous avons:
	
	Nous pouvons très bien faire une analogie entre les deux relations (1') et (2') et les relations de Viète que nous avions obtenues pour le polynôme de degré 2 qui rappelons-le étaient:
	
	excepté que nous avons maintenant (nous adoptons une autre notation pour ces racines intermédiaires):
	
	à la différence que nous avons maintenant (nous adoptons une autre notation pour ces racines intermédiaires): 
	
	dont $z_1,z_2$sont les racines.
	Cette dernière équation a pour discriminant:
	
	Prenons maintenant le cas par cas:
	\begin{enumerate}
		\item Si , l'$\Delta >0$ en $Z$ admet deux solutions $z_1,z_2$ dont la somme va nous donner indirectement la valeur de $X$ puisque par définition $X=u+v$ et $z_1=u^3$ et $z_2=v^3$. Nous voyons que nous avons tous les ingrédients pour trouver la première racine de l'équation initiale qui sera la racine certaine (ou "\NewTerm{zéro certain}\index{z\'ero certain}"). Ainsi:
		
		comme $\Delta>0$ et que les racines supérieures sont cubiques nous avons nécessairement $X_1\in \mathbb{R}$ si tous les coefficients de l'équation originale sont bien dans $\mathbb{R}$.
		
		\item Si $\Delta=0$, nous le savons, l'équation en $Z$ admet une racine double et puisque le discriminant comporte une puissance carrée de $q$ cela signifie nécessairement que $p$ est négatif.

		Le polynôme $P$ admet donc lui aussi une racine double et de même pour l'équation d'origine. Nous avons vu par ailleurs que pour un polynôme du second degré si le discriminant est nul les racines sont:
		
		alors par analogie:
		
		
		\item Si $\Delta<0$ nous devons à nouveau utiliser les nombres complexes comme nous l'avons fait lors de notre étude du polynôme de degré $2$. Ainsi, nous savons que l'équation en $Z$ admet deux solutions complexes telles que:
		
		et à nouveau comme les racines sont conjuguées nous pouvons écrire sous la forme condensée:
		
		et comme:
		
		nous avons donc:
		
		Comme $u_k,v_k$ sont conjugués, nous avons nécessairement.
	\end{enumerate}
	
	\begin{tcolorbox}[colframe=black,colback=white,sharp corners]
	\textbf{{\Large \ding{45}}Exemple:}\\\\
	Considérons l'équation:
	
	Nous avons donc:
	
	et alors:
	
	Nous avons donc:
	
	\end{tcolorbox}
	Les polynômes de degré trois sont donc bien résolubles par radicaux!
	
	Remarquez aussi une autre approche intéressante due à Viète ! On recommence à partir de :
	
	et posons $X = u\cos(\theta)$. L'idée est de choisir $u$ pour faire coïncider l'équation précédente avec l'identité trigonométrique suivante :
	
	Avant de continuer, démontrons cette identité en utilisant les identités trigonométriques prouvées dans la section Trigonométrie:
	
	En fait, en choisissant $u=2\sqrt{-{\dfrac {p}{3}}}$ et en divisant l'équation par $\dfrac{u^{3}}{4}$ nous obtenons:
	
	En comparant avec l'identité ci-dessus, nous obtenons :
	
	et donc les racines de l'équation cubique déprimée sont :
	
	
	\subsubsection{Polynômes et équations univarisé d'ordre quatre}
	L'équation polynomiale univariée à résoudre ici est :
	
	avec $a\neq 0$.
	\begin{tcolorbox}[title=Remarque,colframe=black,arc=10pt]
	Nous devons cette méthode de résolution à l'italien Ludovico Ferrari mathématicien italien du 16ème siècle également.
	\end{tcolorbox}
	Quitte à diviser par $a$ nous avons:
	
	Puis, en posant:
	
	l'équation se réduit:
	
	où nous voyons que le coefficient devant le $y^3$ s'annule. Ainsi, tout polynôme du type:
	
	peut être écrit sous la forme suivante:
	
	En posant:
	
	\begin{tcolorbox}[title=Remarque,colframe=black,arc=10pt]
	Si $d''=0$, l'équation à résoudre est en réalité une "\NewTerm{équation bicarrée}\index{\'equation bicarr\'ee}". Le changement de variable permet alors de se ramener à une équation polynomiale du deuxième degré (ce que nous savons facilement résoudre).
	\end{tcolorbox}
	Nous introduisons maintenant un paramètre $t$ (que nous choisirons judicieusement par la suite) et nous réécrivons l'équation polynomiale sous la forme suivante:
	
	\begin{tcolorbox}[title=Remarque,colframe=black,arc=10pt]
	Si le lecteur développe et distribue tous les termes de la relation précédente il retombera bien évidemment sur $x^4+c''x^2+d''x+e''=0$.
	\end{tcolorbox}
	L'idée sous-jacente est d'essayer de s'assurer que la partie entre crochets de l'expression précédente peut être écrite sous la forme d'un carré :
	
	Car dans ce cas, en utilisant :
	
	Notre équation polynomiale peut alors s'écrire :
	
	et nous n'aurions qu'à résoudre deux équations polynomiales du second degré (ce que nous savons déjà faire).
	
	Mais pour que nous puissions écrire :
	
	l'expression du second degré à gauche de l'égalité ne doit avoir qu'une seule racine. Mais nous avons vu dans notre étude des équations polynomiales du second degré cela signifiait puisque le discriminant est nul :
	
	et que la racine était alors donnée par :
	
	Ce qui correspond dans notre cas à :
	
	et donc que :
	
	avec:
	
	Donc finalement, si $t$ est tel que $4(2t-c'')(t^2-e'')={d''}^2$, alors nous avons :
	
	puisque le théorème fondamental des polynômes nous donne pour un polynôme du deuxième degré n'ayant qu'une seule racine:
	
	Pour conclure, il suffit de voir que trouver un nombre t vérifiant la relation:
	
	est un problème de degré $3$ que nous savons déjà résoudre par la méthode de Cardan.
	
	De telles méthodes générales n'existent plus pour les degrés égaux ou supérieurs à $5$ comme nous le verrons à l'aide de la théorie de Galois (\SeeChapter{voir section Algèbre Ensembliste page \pageref{galois theory}}).
	
	\subsubsection{Polynômes trigonométriques}\label{trigonometric polynomials}
	\textbf{Définition (\#\mydef):} Nous appelons "\NewTerm{polynôme trigonométrique}\index{polynôme trigonométrique}" de degré $N$ toute somme finie du type:
	
	où $c_n\in \mathbb{C}$.

	Un polynôme trigonométrique peut aussi être écrit en utilisant les fonctions trigonométriques usuelles grâce aux transformations suivantes:
	
	Soit en utilisant la formule d'Euler (\SeeChapter{voir section Nombres page \pageref{euler formula}}):
	
	Ce que nous pouvons réécrire aussi sous la forme:
	
	En posant alors:
	
	Il vient:
	
	Nous verrons longuement dans le chapitre des Suites Et Séries comment utiliser ces polynômes dans le cadre de l'étude des séries de Fourier.
	
	\subsubsection{Pôlynomes Cyclotomiques}
	\textbf{Définition (\#\mydef):} Si $n$ est un entier naturel (appartenant donc à $\mathbb{N}$) et $x$ un nombre complexe, nous appelons "\NewTerm{polynôme cyclotomique}\index{polynôme cyclotomique}" ce que nous notons traditionnellement et que nous définissons comme étant le produit de tous les monômes:
	
	où $\alpha$ est une racine primitive $n$-ème de l'unité de $\mathbb{C}$. En d'autres termes:
	
	Pour rappel une racine $n$-ème de l'unité (parfois appelée "\NewTerm{nombre de De Moivre}\index{nombre de De Moivre}") est un nombre complexe dont la puissance $n$-ème vaut $1$.
	
	Ainsi, l'ensemble des racines $n$-èmes de l'unité est l'ensemble:
	
	qui est un groupe cyclique (voir  section de Théorie des Ensebmles et aussi la section d'Algèbre Ensembliste page \pageref{set algebra}).
	
	Nous appelons alors "\NewTerm{racine primitive $n$-ème de l'unité}\index{racine primitive $n$-ème de l'unité}" ou "R.P.N." tout élément de ce groupe l'engendrant.
	
	Les éléments de $G_n$ sont donc du type:
	
	avec $k\in \mathbb{Z}$. Nous écrivons alors l'ensemble des $G_n$ sous la forme:
	
	Un petit exemple de polynôme cyclotomique (plus d'exemples seront donnés plus bas):
	
	avec:
	
	qui sont donc les racines quatrièmes de l'unité (autrement dit chacun de ces nombres mis à la puissance $4$ donne $1$). Elles forment le groupe et celui-ci ne peut-être engendré que par $\mathrm{i}$ et $-\mathrm{i}$ (générateur du groupe selon ce qui a été vu dans le chapitre de Théorie des Ensembles).
	
	Donc un polynôme cyclotomique est le produit de facteurs qui s'écrit:
	
	avec $k\in \{0,\ldots,n-1\}$.
	
	Nous verrons avec les exemples ci-dessous que si $n$ est pair alors :
	
	et si $n$ est impair :
	
	
	\pagebreak
	\begin{tcolorbox}[colframe=black,colback=white,sharp corners]
	\textbf{{\Large \ding{45}}Exemple:}\\\\
	Pour $n$ jusqu'à $30$, les polynômes cyclotomiques sont :
	
	\end{tcolorbox}
	
	\subsubsection{Polynômes de Legendre}\label{legendre polynomials}
	\textbf{Définition (\#\mydef):} Les polynômes de Legendre sont définis par (il est fortement recommandé de lire les sections de Calcul Différentiel et Intégral page \pageref{differential and integral calculus} et aussi d'Analyse Fonctionnelle \pageref{functional analysis} avant de poursuivre):
	
	où $P_n$ est donc un polynôme de degré $n$. Nous retrouverons ces polynômes dans la résolution d'équations différentielles en physique (propagation de la chaleur, physique quantique, chimie quantique, etc.). Nous retrouvons plus souvent l'écriture équivalente triviale:
	
	Nous nous concentrerons ici uniquement et uniquement sur les propriétés qui sont utilisées actuellement dans les autres sections sur la Physique de ce livre!!!
	
	Démontrons que selon la définition du produit scalaire fonctionnel (voir sections d'Analyse Fonctionelle page \pageref{functional dot product} et Calcul Vectoriel \pageref{dot product}) que les polynômes de Legendre sont orthogonaux. C'est une propriété très importante pour notre étude de la Chimie Quantique (voir page \pageref{quantum chemistry rigid rotator}) plus tard !
	
	\begin{dem}
	Soit $P$ un polynôme de degré $\leq n-1$. Il suffit de montrer que $\langle P_n | P \rangle =0$, c'est-à-dire que $P_n$ est orthogonal à l'espace des polynômes de degré inférieur à $n$. Nous avons en effet:
	
	en intégrant par parties nous obtenons:
	
	
	\begin{tcolorbox}[colback=red!5,borderline={1mm}{2mm}{red!5},arc=0mm,boxrule=0pt]
	\bcbombe Attention!!! Pour le terme nul ci-dessus, seulement le terme $(1-x^2)^n$ y est dérivé. Donc puisque $x$ est au carré, quelque soit la dérivée la valeur sera toujours la même. Ce qui justifie que le terme soit nul.
	\end{tcolorbox}
	
	En continuant de la sorte nous obtenons après $n$ intégrations par parties:
	
	\begin{tcolorbox}[title=Remarque,colframe=black,arc=10pt]
	Le terme dérivé est nul puisque le polynôme dérivé est de degré n-1 $n-1$.
	\end{tcolorbox}
	\begin{flushright}
		$\blacksquare$  Q.E.D.
	\end{flushright}
	\end{dem}	
	Voici quelques propriétés utiles pour le chapitre de Chimie Quantique des polynômes de Legendre:
	\begin{enumerate}
		\item[P1.] Nous avons $P_n(1)=1$:
		\begin{dem}
		
		et en utilisant la règle de différenciation de Leibniz pour les produits (\SeeChapter{voir section de Calcul Différentiel et Intégral page \pageref{Leibniz differentiation rule for products}}) nous avons:
		
		Dès lors:
		
		\begin{flushright}
			$\blacksquare$  Q.E.D.
		\end{flushright}
		\end{dem}

		\item[P2.] Nous avons $P_n(-x)=P_n(-x)$ si $n$ est pair:
		\begin{dem}
		Si $n$ est pair:
		
		est une fonction pair et dès lors:
		
		est pair.
		\begin{flushright}
			$\blacksquare$  Q.E.D.
		\end{flushright}
		\end{dem}

		\item[P3.] Nous avons $P_n(-x)=-P_n(x)$ si $n$ est impair:
		\begin{dem}
		Si $n$ est impair:
		
		est une fonction impair et dès lors:
		
		est impair.
		\begin{flushright}
			$\blacksquare$  Q.E.D.
		\end{flushright}
		\end{dem}
	\end{enumerate}
	\begin{theorem}
	Nous allons à présent démontrer la validité de la relation de récurrence suivante pour les $P_n$ (relations que nous utiliserons en physique):
	
	pour $n \geq 1$.
	\end{theorem}
	\begin{dem}
	$xP_n(x)$ est un polynôme de degré $n+1$, il existe dès lors des $a_j\in \mathbb{R}$ tel que ce polynôme peut s'exprimer comme combinaison linéaire de la famille de polynômes constituant la base orthonormale (base qui permet donc d'engendrer $xP_n(x)$):
	
	nous pouvons dès lors écrire:
	
	mais nous choisissons $k\leq n-2$ (parce que est dès lors $xP_k$ est de degré $n-1$):
	
	Dès lors:
	
	c'est-à-dire que nous devons avoir $a_k=0$. Il en suit que:
	
	Par les propriétés des polynômes de Legendre vues précédemment, nous pouvons écrire les égalités:
	
	et:
	
	dès lors:
	
	Le coefficient dominant de  $P_n$est défini (rappelons-le) par le coefficient du monôme du plus grand degré. Ainsi, il est donné par:
	
	Donc:
	
	\begin{tcolorbox}[title=Remarque,colframe=black,arc=10pt]
	Le lecteur vérifiera au besoin pour un $n$ donné que: 
	
	\end{tcolorbox}
	La relation:
	
	que nous avons obtenu ci-dessus nous impose que le coefficient dominant du polynôme de la combinaison linéaire soit égal au coefficient dominant du polynôme $xP_n$ (nous avons éliminé le $(-1)^n$ qui se simplifie):
	
	Après simplification, nous obtenons:
	
	et ce qui donne finalement facilement:
	
	La relation:
	
	devient dès lors:
		
	\begin{flushright}
		$\blacksquare$  Q.E.D.
	\end{flushright}
	\end{dem}
	Voici les six premiers polynômes de Legendre:
	
	Les graphiques de ces polynômes (jusqu'à $n = 5$) sont présentés ci-dessous :
	\begin{figure}[H]
		\centering
		\includegraphics{img/algebra/legendre_polynomials.jpg}
		\caption[Cinq premiers polynômes de Legendre]{Cinq premiers polynômes de Legendre (source: Wikipédia)}
	\end{figure}
	
	\subsubsection{Polynômes de Laguerre}\label{Laguerre polynomials}
	Avant d'aborder les mathématiques des polynômes de Laguerre, le lecteur doit savoir que la grande majorité du texte ci-dessous est un simple copier/coller du contenu de l'excellent manuel \cite{arfken2013mathematical}. Il est d'ailleurs fortement recommandé de lire les sections de Calcul Différentiel et Intégral à la page \pageref{differential and integral calculus} et aussi d'Analyse Fonctionnelle à la page \pageref{analyse fonctionnelle} avant de continuer !
	
	\textbf{Définition (\#\mydef):} "\NewTerm{L'équation différentielle ordinaire de Laguerre}\index{\'equation diff\'erentielle ordinaire de Laguerre}" (qui dérive de l'équation différentielle partielle radiale de Schrödinger pour l'atome d'hydrogène\footnote{Les polynômes de Laguerre apparaissent en mécanique quantique, dans la partie radiale de la solution de l'équation de Schrödinger pour un atome à un électron. Ils décrivent également les fonctions statiques de Wigner des systèmes oscillants en mécanique quantique dans l'espace des phases. Ils entrent en outre dans la mécanique quantique du potentiel de Morse et de l'oscillateur harmonique isotrope 3D.} comme nous le prouverons dans la section de Chimie quantique à la page \pageref{quantum chemistry non-rigid rotator}) est donnée par:
	
	Cette équation n'a de solutions non singulières que si $n$ est un entier non négatif.
	
	Dérivons une solution en utilisant la méthode de Frobenius (\SeeChapter{voir section Séquences et Séries page \pageref{Frobenius method}})! Pour cela nous supposons d'abord que la solution $y$ est donnée sous la forme :
	
	avec ses deux premières dérivées :
	
	et:
	
	En substituant ces expressions dans l'équation différentielle donnée, on obtient :
	
	En multipliant les facteurs par les sommations correspondantes, on obtient :
	
	Maintenant, nous ajustons l'indice des deux premières sommations pour tels que les $x$ aient l'exposant $r+n$ :
	
	Nous évaluons quelques termes pour que toutes les sommations commencent à l'indice $n=0$ et nous obtenons :
	
	En combinant les termes, nous avons :
	
	qui donne, après simplification :
	
	En égalant les coefficients des deux côtés, on obtient :
	
	En divisant la dernière équation par $r+n+1 \neq 0$ et en décalant les indices, on obtient "\NewTerm{l'équation de récurrence}":
	
	Puisque $a_{0} \neq 0$, nous obtenons "\NewTerm{l'équation indicatrice}" de l'équation initiale :
	
	ayant pour racines $r_{1,2}=0$.
	
	La relation de récurrence peut s'écrire :
	
	Dès lors:
	
	Par conséquent, en choisissant $a_{0}=1$, la solution $y=y_{1}$ est égale à :
	
	On voit que cette somme est finie car quelle que soit la valeur de $\lambda$ à un moment donné tout facteur aura $(\lambda-\ldots)$ égal à zéro. Donc, tous les termes s'annuleront. Ainsi la solution $y_1$ a un nombre fini de termes et est un polymonial!
	
	Pour $\lambda=n$, cette solution devient:
	
	On voit que cette solution est polynomiale si $\lambda=n$. Revenons maintenant à :
	
	Cette dernière relation nous amène à définir le polynôme de Laguerre comme :
	
	Cette relation se retrouve souvent dans les livres sous la forme suivante :
	
	On peut alors construire le tableau suivant :
	\begin{table}[H]
		\centering
		\begin{tabular}{|l|l|}
			\rowcolor[HTML]{C0C0C0}\hline$n$ & \multicolumn{1}{|c|} {$L_{n}(x)$} \\
			\hline 0 & 1 \\
			\hline 1 & $-x+1$ \\
			\hline 2 & $\frac{1}{2}\left(x^{2}-4 x+2\right)$ \\
			\hline 3 & $\frac{1}{6}\left(-x^{3}+9 x^{2}-18 x+6\right)$ \\
			\hline 4 & $\frac{1}{24}\left(x^{4}-16 x^{3}+72 x^{2}-96 x+24\right)$ \\
			\hline 5 & $\frac{1}{120}\left(-x^{5}+25 x^{4}-200 x^{3}+600 x^{2}-600 x+120\right)$ \\
			\hline 6 & $\frac{1}{720}\left(x^{6}-36 x^{5}+450 x^{4}-2400 x^{3}+5400 x^{2}-4320 x+720\right)$ \\
			\hline
			 $\ldots$ & $\ldots$\\
			\hline$n$ & $\frac{1}{n !}\left((-x)^{n}+n^{2}(-x)^{n-1}+\ldots+n(n !)(-x)+n !\right)$ \\
			\hline
		\end{tabular}
		\caption{Liste de quelques polynômes de Laguerre}
	\end{table}
	Et donnons quelques tracés de ces polynômes de Laguerre :
	\begin{figure}[H]
		\centering
		\includegraphics[width=0.8\textwidth]{img/algebra/laguerre_polynomials.jpg}
		\caption[Tracé des six premiers polynômes de Laguerre]{Tracé des six premiers polynômes de Laguerre (source : Wikipédia)}
	\end{figure}
	Maintenant, nous pouvons facilement prouver que :
	
	nommé "\NewTerm{formule  de Rodrigues'}\index{formule de Rodrigues}" pour les polynômes de Laguerre (il y a différentes manières de l'exprimer et différentes versions également selon le type de polynôme de Laguerre auquel nous avons affaire !).
	
	Si on utilise la représentation explicite des polynômes de Laguerre :
	
	alors le résultat découle facilement de la règle de différenciation de Leibniz pour les produits :
	
	Soit $f(x)=x^{n}$ et $g(x)=e^{-x}$. Alors $f^{(k)}(x)=x^{n-k} n ! /(n-k) !$ et $g^{(n-k)}(x)=(-1)^{n-k} e^{-x}$. Maintenant, si nous  le nettoyons et refaisons la somme dans l'ordre inverse, c'est-à-dire de $k \rightarrow n-k$ nous retombons sur la formule de Rodrigues ci-dessus.
	
	Montrons maintenant que la fonction génératrice fonctionnelle (\SeeChapter{voir section Séquences et Séries page \pageref{functional generating function}}) des polynômes de Laguerre est donnée par :
	
	La preuve courte est donnée par (on utilise simplement la série de Maclaurin de $e^{z}$ où $z=-xt/(1-t)$) :
	
	En différenciant la fonction génératrice ci-dessus par rapport à $x$ et $t$, nous obtenons les deux relations de récurrence pour les polynômes de Laguerre comme suit :
	
	Pour dériver la première relation de récurrence, nous partons de :
	
	On différencie les deux côtés par rapport à $t$ pour obtenir :
	
	En multipliant les deux membres par $(1-t)^{2}$ et en simplifiant, on obtient :
	
	Nous égalisons maintenant les coefficients de $t^{n}$ des deux côtés dans la relation ci-dessus pour obtenir :
	
	Pour obtenir la deuxième relation de récursivité, nous repartons de la fonction génératrice :
	
	Nous différencions maintenant les deux côtés de l'égalité ci-dessus par rapport à $x$ pour obtenir :
	
	en utilisant la fonction génératrice cette dernière peut être réécrite :
	
	Dès lors:
	
	En égalisant les coefficients $t^{n}$ des deux côtés de ce qui précède nous donne :
	
	Dès lors:
	
	et après réarrangement :
	
	De de la première relation de récurrence :
	
	Nous différencions ce qui précède par rapport à $x$ pour obtenir :
	
	En substituant les valeurs de $L_{n-1}^{\prime}(x)$ et $L_{n+1}^{\prime}(x)$ dérivées plus tôt dans la relation précédente, nous obtenons :
	
	et après simplification on obtient la seconde relation de récurrence :
	
	Pour la preuve de l'orthogonalité des polynômes de Laguerre, le lecteur peut se référer à la section Analyse page \pageref{orthogonality of Laguerre polynomial}.
		
	\paragraph{Polynômes de Laguerre associés}\label{Associated Laguerre polynomials}\mbox{}\\\\
	\textbf{Définition (\#\mydef):} Dans de nombreuses applications, notamment en mécanique quantique (encore !), on a besoin des "\NewTerm{polynômes de Laguerre associés}\index{polyn\-omes de Laguerre associ\'es}" aussi nommés "\NewTerm{polynômes de Laguerre généralisés}\index{polyn\-omes de Laguerre g\'en\'eralis\'es}" définis par:
	
	A partir de la forme série de $L_{n}(x)$ nous vérifions que les plus petits polynômes de Laguerre associés sont donnés par :
	
	Dès lors en général:
	
	Les premiers polynômes de Laguerre associés sont :
	\begin{table}[H]
		\centering
		\begin{tabular}{|l|l|}
			\rowcolor[HTML]{C0C0C0}\hline$n$ & \multicolumn{1}{|c|} {$L^k_{n}(x)$} \\
			\hline $L_0^k$ & $1$ \\
			\hline $L_1^k$ & $-x+(\alpha+1)$ \\
			\hline $L_2^k$ & $\frac{x^2}{2}-(\alpha+2)x+\frac{(\alpha+1)(\alpha+2)}{2}$ \\
			\hline $L_3^k$ & $-\frac{x^3}{6}+\frac{(\alpha+3)x^2}{2}-\frac{(\alpha+2)(\alpha+3)x}{2}+\dfrac{(\alpha+1)(\alpha+2)(\alpha+3)}{6}$ \\
			\hline
		\end{tabular}
		\caption{Liste de quelques polynômes de Laguerre associés}
	\end{table}
	Et un tracé de certains d'entre eux:
	\begin{figure}[H]
		\centering
		\includegraphics[width=0.8\textwidth]{img/algebra/associated_laguerre_polynomials.jpg}
		\caption[Tracés de quelques polynômes de Laguerre associés]{Tracés de quelques polynômes de Laguerre associés (source: Wikipédia)}
	\end{figure}
	Ou en utilisant une relation de récurrence pour tout $k\leq 1$ :
	
	Une fonction génératrice peut être développée en différenciant la fonction génératrice de Laguerre $k$ fois pour donner :
	
	A partir des deux derniers membres de cette équation, en annulant le facteur commun $z^{n}$, nous obtenons:
	
	De là, pour $x=0$, le développement binomial :
	
	nous donne:
	
	Les relations de récurrence peuvent être dérivées de la fonction génératrice ou en différenciant les relations de récurrence polynomiales de Laguerre. Parmi les nombreuses possibilités, nous avons :
	
	Ainsi, nous obtenons en dérivant une fois l'équation différentielle ordinaire de Laguerre :
	
	et finalement en différenciant l'équation différentielle ordinaire de Laguerre $k$ fois :
	
	En ajustant l'indice $n \rightarrow n+k$, nous avons "\NewTerm{l'équation différentielle ordinaire de Laguerre associée}\index{\'equation diff\'erentielle ordinaire de Laguerre associ\'ee}":
	
	Plus connue sous la forme suivante :
	
	Lorsque des polynômes de Laguerre associés apparaissent dans un problème physique, c'est généralement parce que ce problème physique implique l'équation différentielle ci-dessus. L'application la plus importante concerne les états liés de l'atome d'hydrogène, qui en sont dérivés ! La "\NewTerm{représentation de Rodrigues du polynôme de Laguerre associé}\index{Représentation de Rodrigues du polynôme de Laguerre associé}" :
	
	peut être obtenue par:
	
	Les polynômes de Laguerre associés présentent les mêmes propriétés d'orthogonalité que les polynômes de Laguerre !
	
	Dérivons à nouveau un résultat utile pour notre étude de la chimie quantique : la constante de normalisation du polynôme de Laguerre associé !
	
	En utilisant la formule de Rodrigue pour le polynôme de Laguerre associé, nous obtenons :
	
	Par conséquent:
	
	pour $m \leqslant n$ (par induction sur $m$ et intégration par parties). En prenant $m=n$ (rappelons que $L_{n}^{k}(x)$ est un polynôme de degré $n$ avec le coefficient dominant égal à $(-1)^{n} / n !$ ) , on obtient le nécessaire :
	
	
	\paragraph{Fonctions de Laguerre}\label{Laguerre functions}\mbox{}\\\\
	En posant:
	
	$\psi_{n}^{k}(x)$ satisfait l'équation différentielle ordinaire auto-adjointe :
	
	Les $\psi_{n}^{k}(x)$ sont parfois nommés "\NewTerm{fonctions de Laguerre}\index{fonction de Laguerre}".
	
	Une autre forme utile est donnée en définissant (cela correspond à modifier la fonction $\psi_{n}^{k}(x)$ pour éliminer la dérivée première dans l'équation différentielle ordinaire précédente) :
	
	La substitution dans l'équation de Laguerre associée donne (correspondant à l'équation différentielle ordinaire sans dimension du terme radial du rotateur non rigide que nous rencontrons lors de notre étude de l'atome hydrogénoïde dans la section de Chimie Quantique)\label{radial term dimensionless non-rigid rotator}:
	
	Prouvons cette affirmation ! Pour atteindre la dérivée seconde, nous avons besoin de la dérivée première, et utilisons la notation :
	
	où $v=L_{j}^{k}(x)$, car les indices ne changent pas et ne servent qu'à ajouter de l'encombrement, et nous pouvons nous rappeler que la variable indépendante est $x$. La dérivée première est alors :
	
	Similairement nous obtenons:
	
	En substituant la dérivée seconde et la fonction dans l'équation différentielle ordinaire de Laguerre associée :
	
	et en divisant par le facteur commun de $e^{-x / 2} x^{(k+1) / 2}$, les termes restants sont :
	
	qui est l'équation de Laguerre associée. Puisque $v=L_{j}^{k}(x)$, et les $L_{j}^{k}(x)$ sont des solutions de l'équation de Laguerre associée. La dernière ligne ci-dessus équivaut à :
	
	qui est l'équation de Laguerre associée, que nous savons être un énoncé vrai, donc :
	
	en sont des solutions ! La preuve est terminée.
	
	L'intégrale de normalisation correspondante :
	
	est (sans démonstration):
	
	Notons que les $\Phi_{n}^{k}(x)$ ne forment pas un ensemble orthogonal (sauf avec $x^{-1}$ comme fonction de pondération) à cause de $x^{-1}$ dans le terme $(2 n+k+1) / 2 x$.

	\begin{flushright}
	\begin{tabular}{l c}
	\circled{90} & \pbox{20cm}{\score{3}{5} \\ {\tiny 70 votes,  56.29\%}} 
	\end{tabular} 
	\end{flushright}
	
	%to make section start on odd page
	\newpage
	\thispagestyle{empty}
	\mbox{}
	\section{Algèbre Ensembliste}\label{set algebra}
	\lettrine[lines=4]{\color{BrickRed}N}ous allons aborder maintenant dans ce livre l'étude des structures ensemblistes de manière très pragmatique (puisque rappelons que ce site est dédié aux ingénieurs). Ainsi, il sera fait usage du minimum de formalisme et seulement les démonstrations des éléments que nous considérons comme absolument essentiels à l'ingénieur seront présentées. Par ailleurs, de nombreuses démonstrations seront faites par l'exemple et nous nous focaliserons en grande partie sur la théorie algébrique des groupes car elle a une place presque prédominante en physique plus que pour les autres structures ensemblistes.

	\subsection{Algèbre et Géométrie Corporelle}
	Les symétries des figures géométriques, des cristaux et de tous les autres objets de la physique macroscopique font l'objet depuis des siècles d'observations et d'études. En termes modernes, les symétries d'un objet donné forment un groupe. 

	Depuis le milieu du 19ème siècle, la théorie des groupes a pris une extension énorme, et ses applications à la mécanique quantique et à la théorie des particules élémentaires se sont développées tout au long du 20ème siècle.
	
	Dans une lettre de 1877 au mathématicien Adolph Mayer, Sophus Lie écrit qu'il a créé la théorie des groupes en janvier 1873. Il s'agit bien sûr des groupes qu'il appelait "\NewTerm{groupes continus}\index{groupes continus}" et qui sont appelés aujourd'hui "\NewTerm{groupes de Lie}\index{groupes de Lie}\label{lie group}". Lie cherchait à étendre l'usage des groupes du domaine des équations algébriques, où Galois les avait introduites, à celui des équations différentielles.
	
	Dès 1871, la notion de générateur infinitésimal d'un groupe à un paramètre de transformations était apparue dans son oeuvre. C'est l'ensemble des générateurs infinitésimaux des sous-groupes à un paramètre d'un groupe continu qui forme ce que nous appelons aujourd'hui une \NewTerm{algèbre de Lie}\index{alg\'ebre de Lie}".
	
	Ce furent Wigner et Weyl qui montrèrent le rôle prééminent de la théorie des groupes, et de leurs représentations en particulier, dans la nouvelle mécanique quantique que développaient Heisenberg et Dirac. L'idée générale de la théorie des représentations est d'essayer d'étudier un groupe en le faisant agir sur un espace vectoriel de manière linéaire: nous essayons ainsi de voir le groupe comme un groupe de matrices (d'où le terme "représentation"). Nous pouvons ainsi, à partir des propriétés relativement bien connues du groupe des automorphismes de l'espace vectoriel (\SeeChapter{voir section Théorie des Ensembles page \pageref{automorphism}}), arriver à déduire quelques propriétés du groupe qui nous intéressent.
	
	Nous pouvons considérer la théorie des représentations de groupes comme une vaste généralisation de l'analyse de Fourier. Son développement est continu et elle a, depuis le milieu du 20ème siècle, des applications innombrables en géométrie différentielle, en théorie ergodique, en théorie des probabilités, en théorie des nombres, dans la théorie des formes automorphes, dans celle des systèmes dynamiques ainsi qu'en physique, chimie, biologie moléculaire et traitement du signal. À l'heure actuelle, des branches entières des mathématiques et de la physique en dépendent.
	
	Avant de commencer, nous renvoyons le lecteur au chapitre traitant de la Théorie Des Ensembles pour qu'il se rappelle de la structure et des propriétés fondamentales qui constituent le groupe et également au chapitre d'Algèbre Linéaire (car nous en utiliserons quelques résultats).
	
	\subsubsection{Groupes Cycliques}\label{cyclic groups}
	Le groupe cyclique (dont la définition a déjà été vue dans le chapitre de Théorie des Ensembles) va nous servir de base dans le cadre de l'étude des groupes finis. Par ailleurs, plutôt que de faire des développements généralisés nous avons préféré prendre des exemples particuliers afin de présenter l'idée de groupe cyclique (approche plus adaptée à l'ingénieur).

	Nous allons donc prendre l'exemple fort sympathique des heures de la montre... avec trois approches différentes qui successivement (!) permettront d'aborder un groupe cyclique simple.
	\begin{enumerate}
		\item Première approche:
		
		Imaginons donc une horloge avec une aiguille qui peut prendre $12$ positions possibles (mais pas de positions intermédiaires). Nous noterons de manière spéciale les $12$ positions possibles: $\overline{0},\overline{1},\overline{2},...,\overline{11}$ (le trait au-dessus des nombres n'est pas innocent!).
		
		Rien ne nous empêche sur l'ensemble de ces positions de définir une addition, par exemple:
		
	  	ce qui est similaire aux résultats que nous obtenons lorsque dans notre quotidien nous faisons des calculs avec notre montre.
	  	
	  	\item Deuxième approche (première extension):
	  	
	  	Si nous observons bien une montre ou une horloge, nous remarquons qu'à chaque fois que nous rajoutons $12$ (ou retirons $12$...) à une valeur des heures de notre montre alors nous tombons sur un ensemble de nombres bien déterminé qui sont aussi dans $\mathbb{Z}$. Ainsi (évidemment dans le cadre d'une montre/horloge seules les premières valeurs positives nous intéressent la plupart du temps mais ici nous faisons des maths alors nous généralisons un peu...):
	  	
		Nous retrouvons ici un concept que nous avions déjà vu dans le chapitre de Théorie Des Nombres. Il s'agit de classes de congruences et l'ensemble de ces classes forme l'ensemble quotient $\mathbb{Z}/12\mathbb{Z}$. Si nous munissons cet ensemble quotient d'une loi d'addition, il est normalement facile d'observer que celle-ci est une loi interne à l'ensemble quotient, qu'elle est associative, qu'il existe un élément neutre et que chaque élément possède un symétrique (inverse).
		
		Ainsi, cet ensemble quotient muni uniquement de la loi d'addition (sinon en ajoutant la multiplication nous pouvons former un anneau) est un groupe commutatif.

		\item Troisième approche (deuxième et dernière extension):
		
		Voyons une troisième et dernière approche qui explique pourquoi le groupe quotient est cyclique. 
		
		Si nous projetons la rotation des aiguilles de notre montre (toutes les rotations dans l'algèbre ensembliste se font traditionnellement dans le sens des aiguilles d'une montre!) dans $\mathbb{C}$ et que nous définissons:
		
		Nous avons alors $x^{12}=x^0=1$ et:
		
		ce qui explique pourquoi le groupe quotient $(\mathbb{Z}/12\mathbb{Z},+)$ est appelé "\NewTerm{groupe cyclique}\index{groupe cyclique}" (par isomorphisme de groupe selon ce qui a été vu dans la section de Théorie des Ensembles). Son isomorphe est noté $C_{12}$ et tous les éléments sont modulus $1$. Il est commun de noter tous les nombres complexes de module $1$ comme suit:
		
		Si nous représentons dans $\mathbb{C}$ l'ensemble isomorphe nous obtenons alors sur le cercle unité un polygone ayant  $n$ sommets comme le montre la figure ci-dessous:
		\begin{figure}[H]
			\centering
			\includegraphics{img/algebra/c_12_cyclic_group.jpg}
			\caption{$C_{12}$ Groupe cyclique}
		\end{figure}
		Par ailleurs, le nombre d'éléments composants $\mathbb{Z}/12\mathbb{Z}$ étant fini, $(\mathbb{Z}/12\mathbb{Z},+)$ est fini. Contrairement au groupe qui $(\mathbb{Z},+)$ est lui un groupe discret infini. 
		
		Ce concept de finitude sera peut-être plus évident avec l'exemple que nous ferons de suite après avec où le lecteur observera que cet ensemble a le même nombre d'éléments que $C_4$.
	\end{enumerate}
	
	\begin{tcolorbox}[title=Remarque,colframe=black,arc=10pt]
	Les mathématiciens appellent $C_n$ le "groupe des racines $n$-èmes de l'unité". Une racine $n$-ème de l'unité (parfois appelée "\NewTerm{nombre de De Moivre}\index{nombre de De Moivre}") est donc un nombre complexe dont la puissance $n$-ème vaut $1$. Par ailleurs, pour un entier $n$ donné, toutes les racines $n$-èmes de l'unité sont situées sur le cercle unité et sont les sommets d'un polygone régulier à $n$ côtés ayant un sommet d'affixe $1$.
	\end{tcolorbox}
	
	\textbf{Définition (\#\mydef):} Un "\NewTerm{groupe fini}\index{groupe fini}\label{finite group}" est un groupe mathématique avec un nombre fini d'éléments comme les "groupes de permutations", les "groupes symétriques", les "groupes cycliques", etc.
	
	Ce qui intéresse les physiciens particulièrement dans un premier temps ce sont les représentations des groupes finis (aussi les groupes continus que nous verrons plus loin). Ainsi, la représentative de $\mathbb{Z}/n\mathbb{Z}$ nous est connue puisque la rotation dans le plan complexe est donnée comme nous l'a montrée notre étude des complexes dans le section sur les Nombres:
	
	avec $k\in[0,n-1]$. Cette représentative est un sous-groupe du groupe des rotations$\text{O}(2)$ sur lesquelles nous reviendrons plus loin. Le groupe des rotations du plan étant lui-même un sous-groupe du groupe linéaire $\text{GL}(2)$ (nous en donnerons une définition précise et un exemple plus loin).
	
	Au fait, les mathématiciens sont capables de démontrer que tous les groupes quotients $\mathbb{Z}/n\mathbb{Z}$ sont cycliques à isomorphisme près avec $C_n$ et ils disent alors que est un quotient fini du groupe monogène $\mathbb{Z}$...
	
	Cette approche est par contre peut-être un peu abstraite. Alors, si le lecteur se rappelle de la section de Théorie Des Ensembles nous avons vu une définition bien précise de ce qu'était la cyclicité d'un groupe: Un groupe $G$ est dit cyclique si $G$ est engendré par la puissance d'au moins un de ses éléments appelé "générateur" tel que:
	
	Vérifions que ce soit bien le cas pour le groupe:
	
	qui constitue un cas scolaire.
	
	Nous noterons les éléments qui constituent ce groupe:
	
	Ceci étant fait, il convient de faire attention que dans la définition ensembliste du groupe cyclique nous parlons de "puissance" si la loi interne du groupe est la multiplication mais si la loi interne est l'addition, nous avons alors:
	
	Le premier élément générateur du groupe $G=\left\lbrace \mathbb{Z}/4\mathbb{Z},+ \right\rbrace$ est $1$. Effectivement:
	
	Le deuxième élément générateur du même groupe est $3$:
	
	Par contre, le lecteur pourra vérifier que $2$ n'est pas générateur de ce groupe!
	
	Au fait, en ce qui concerne les groupes $G=\left\lbrace \mathbb{Z}/n\mathbb{Z},+ \right\rbrace$ les mathématiciens arrivent à démontrer que seuls les éléments du groupe qui sont premiers avec $n$ sont générateurs (c'est-à-dire les éléments dont le plus grand commun diviseur est $1$).
	
	\begin{tcolorbox}[title=Remarque,colframe=black,arc=10pt]
	Un peu similaire à la façon dont les nombres premiers sont les blocs de construction de base des entiers, il existe un type spécial de groupes finis qui sont les groupes de construction de base de tous les groupes communs finis. Ces types particuliers de groupes sont appelés "groupes simples finis". La classification des groupes finis et simples est sans précédent dans l'histoire des mathématiques, car sa preuve fait $15'000$ pages et a nécessité près de 50 ans pour être achevée et a impliqué quelques centaines de mathématiciens. C'est pourquoi on l'appelle le "théorème énorme".
	\end{tcolorbox}
	
	Ce sera tout pour notre introducton aux groupes cycliques pour ingénieurs. Maintenant passons à une introduction succinte à un autre type de groupes importants.
	
	\subsubsection{Groupes de Transformations}
	Le groupe des rotations est celui qui intéresse le plus les physiciens surtout dans les domaines des matériaux, de la chimie, de la physique quantique et de l'art... Les mathématiciens apprécient eux l'étude des groupes de rotations dans le cadre de la géométrie bien évidemment (mais pas seulement) et les informaticiens tout autant les groupes linéaires. Nous avons d'ailleurs vu un exemple de groupe de rotations juste précédemment.
	
	\textbf{Définition (\#\mydef):} Nous appelons "\NewTerm{groupe linéaire d'ordre $n$}\index{groupe lin\'eaire d'ordre $n$}"" et nous notons $\text{GL}(n)$ les matrices inversibles $n\times n$ ou dites aussi "\NewTerm{matrices régulières}\index{matrices r\'eguli\'eres}" (donc le déterminant est non nul selon ce que nous avons vu dans le chapitre d'Algèbre Linéaire) dont les coefficients sont dans un corps quelconque $\mathbb{K}$: (le corps $\mathbb{R}$ ou le groupe $\mathbb{C}$ la majorité du temps):
	
	Le groupe est ainsi nommé car les colonnes d'une matrice inversible sont linéairement indépendantes.
	
	Nous considérerons comme évident que $\text{GL}(n)$ est un groupe: la multiplication des matrices est associative et chaque matrice de $\text{GL}(n)$ possède un inverse par définition (étant donné que le déterminant est non nul). D'autre part, le produit de deux matrices régulières est encore une matrice régulière.
	
	Un exemple simple et important de groupe linéaire est celui du sous-"\NewTerm{groupe des transformations affines}\index{groupe des transformations affines}" du plan qui est traditionnellement noté (c'est intuitif):
	
	avec  $a,b,c,d,\alpha,\beta\in \mathbb{R},ad-bc\neq 0$ (nous verrons le pourquoi du comment de l'inégalité un peu plus loin).
	
	\textbf{Définition (\#\mydef):} Le "\NewTerm{groupe affine}\index{groupe affine}" ou "\NewTerm{groupe affine général}\index{groupe affine g\'en\'eral}" de tout espace affine $A$ sur un corps $\mathbb{K}$ est le groupe noté $\text{Aff}(A)$ ou $\text{Aff}(n,\mathbb{K})$ de toutes les transformations affines inversibles de l'espace vers lui-même. C'est un "\NewTerm{Groupe de Lie}\index{Groupe de Lie}" si $\mathbb{K}$ est le corps des quaternions réels ou complexes.
	
	Prenons un exemple pratique:
	
	ce qui appliqué à un cercle donnerait:
	\begin{figure}[H]
		\centering
		\includegraphics{img/algebra/affine_group_transformation_example.jpg}
		\caption{Transformations affines sur un cercle}
	\end{figure}
	Cette transformation est une manière de définir les ellipses comme images d'un cercle par une transformation affine.
	
	Les coefficients $\alpha,\beta$ sont sans importance pour la forme de l'image. En fait, ils induisent bien évidemment des translations sur les figures. Nous pouvons donc nous en passer si nous cherchons seulement à la déformer.
	
	Ainsi, il nous reste:
	
	ce qui peut s'écrire sous forme matricielle:
	
	La transformation se réduit donc à la matrice:
	
	et comme nous l'avons vu en algèbre linéaire, la multiplication matricielle est associative mais n'est pas commutative, donc la transformation linéaire ne l'est pas non plus. 
	
	L'élément neutre est la matrice:
	
	et l'inverse de $F$ est:
	
	et comme nous avons imposé $ad-bc\neq 0$ tout élément y possède donc un inverse. Ainsi, le groupe linéaire affine est non commutatif et... forme bien un groupe....
	
	Comme nous allons le voir, tous les groupes de Lie "classiques" sont des sous-groupes de $\text{GL}(n)$.
	
	\textbf{Définition (\#\mydef):} Nous appelons "\NewTerm{groupe spécial linéaire d'ordre $n$}\index{groupe sp\'ecial lin\'eaire d'ordre $n$}" et nous notons $\text{SL}(n)$ les matrices inversibles (carrées $n\times n$) dont les coefficients sont dans un corps quelconque et dont le déterminant est égal à l'unité:
	
	Il s'agit évidemment d'un sous-groupe de $\text{GL}(n)$.
	
	En reprenant l'exemple précédant et en se rappelant que le déterminant d'une matrice carrée bidimensionnelle est  (\SeeChapter{voir section d'Algèbre Linéaire \pageref{determinant}}):
	
	nous remarquons bien géométriquement ce que signifie d'avoir un déterminant unitaire dans ce cas! Effectivement nous avons vu dans la section d'Algèbre Linéaire lors de notre interprétation géométrique qu'avoir un déterminant équivaut à une surface. Ainsi, le fait d'avoir $ad-bc$ unitaire permet donc que quel que soit l'ordre de la transformation, nous avons l'aire qui vaut toujours $1$. Ainsi, le groupe spécial linéaire conserve les surfaces.
	
	\textbf{Définition (\#\mydef):} Nous appelons "\NewTerm{groupe orthogonal réel d'ordre $n$}\index{groupe orthogonal r\'eel d'ordre $n$}" et notons $\text{O}(n)$ les matrices orthogonales (voir section d'Algèbre Lináire pour un rappel de ce que sont les matrices orthogonales) données par:
	
	Par ailleurs, nous avons démontré dans la section d'Algèbre Linéaire lors de notre étude des matrices de rotations que $A^TA=I_n$  implique $\det(A)=\pm 1$.
	
	C'est le cas par exemple de la matrice de $\text{O}(2)$ vue précédemment (elle appartient au groupe orthogonal mais aussi au groupe des rotations que nous verrons plus loin):
	
	qui est orthogonale comme il est facile de le vérifier (multipliez simplement la matrice par sa transposée pour vérifier si vous obtenez la matrice identité).
	\begin{tcolorbox}[title=Remarque,colframe=black,arc=10pt]
	$\text{O}(1)$ est constitué aussi de l'ensemble des matrices triviales.... $[+1] [-1]$ qui sont simplement des vecteurs à une composante... c'est à dire de simples scalaires.
	\end{tcolorbox}
	\textbf{Définition (\#\mydef):} Si $A\in \text{O}(n)$ et que $\det(A)=1$ nous obtenons alors un sous-groupe de $\text{O}(n)$ appelé "\NewTerm{groupe spécial orthogonal réel d'ordre $n$}\index{groupe sp\'ecial orthogonal r\'eel d'ordre $n$}\label{special real group orthogonal}" et alors défini par:
	
	La matrice de rotations donnée précédemment fait partie de ce groupe puisque son déterminant est égal à l'unité! Par ailleurs, ce groupe occupe une place très spéciale en physique et nous le retrouverons maintes fois lors de notre étude de la physique quantique.
	
	Le sous-groupe $\text{SO}(2)$, appelé aussi parfois "\NewTerm{groupe cercle}\index{groupe cercle}" et noté $S^1$, que nous avions aussi étudié dans la section de Géométrie Euclidienne a une représentative donnée par la matrice:
	
	et occupe une place à part dans la famille des groupes $\text{SO}(n)$ avec $n$ supérieur à l'unité. Effectivement il est le seul à être commutatif. Par ailleurs, il est isomorphe à $e^{\mathrm{i}\theta}$ soit à $\text{U}(1)$ le groupe multiplicatif des nombres complexes de module $1$. C'est aussi le groupe de symétrie propre d'un cercle et l'équivalent continu de $C_n$.
	
	Le sous-groupe $\text{SO}(3)$ donné par la matrice (\SeeChapter{voir section Géométrie Euclidienne page \pageref{3d rotation matrix around x}}):
	
	pour la rotation autour de l'axe $x$ dans l'espace tridimensionnel n'est pas commutatif (les matrices de rotation dans le plan étant elles pour rappel commutatives!). Par ailleurs les quaternions, dont la représentative est donc $\text{SO}(3)$, forment un groupe non commutatif aussi (par rapport à la loi de multiplication) comme nous l'avons vu dans la section sur les Nombres.
	
	Par rapport à un vecteur unitaire on se rend facilement compte visuellement parlant que $\text{SO} (3)$ est un sous-groupe fermé de $\text{GL}(3)$, c'est-à-dire de l'ensemble des groupes linéaires de dimension $3$.
	
	\begin{tcolorbox}[title=Remarque,colframe=black,arc=10pt]
	$\text{SO}(1)$  est constitué de la matrice $[1]$ (c'est-à-dire un simple scalaire unitaire!).
	\end{tcolorbox}	
	\textbf{Définition (\#\mydef):} Nous appelons "\NewTerm{groupe unitaire d'ordre $n$}\index{groupe unitaire d'ordre $n$}\label{unitary linear group}" et nous notons $\text{U}(n)$ les matrices dont les composantes sont complexes (dans le cadre de ce site le plus souvent) ou réelles et qui sont orthogonales:
	
	Remarquons par ailleurs que toute matrice unitaire à coefficients complexes et à une dimension... (de $\text{U}(n)$ donc...) est un nombre complexe de module unitaire, qui peut toujours s'écrire sous la forme $e^{\mathrm{i}\mathbb{R}}$.
	
	
	Nous en avons déjà vu un exemple aussi dans ce livre lors de notre étude des spineurs (voir section du même nom page \pageref{spinors}). Il s'agit des matrices de Pauli (utilisées dans la section de Physique Quantique Relativiste) données par:
	
	\textbf{Définition (\#\mydef):} Nous appelons "\NewTerm{groupe spécial unitaire d'ordre $n$}\index{groupe sp\'ecial unitaire d'ordre $n$}\label{special unitary group}" et nous notons $\text{SU}(n)$ les matrices dont les coefficients sont complexes et qui sont orthogonales et dont le déterminant est unitaire:
	
	\begin{tcolorbox}[title=Remarque,colframe=black,arc=10pt]
	$\text{U}(1)$ est égal à $\text{SU}(1)$ et il s'agit donc du cercle unité complexe égal à $e^{\mathrm{i}\mathbb{R}}$. Par ailleurs, $\text{SO}(2)$ est commutatif et isomorphe à $\text{U}(1)$ car c'est l'ensemble des rotations du plan.
	\end{tcolorbox}
	Un exemple connu est toujours celui des matrices de Pauli mais simplement écrites sous la forme utilisée en Physique Quantique Relativiste (voir section du même nom page \pageref{pauli matrices}):
		
	qui font partie de $\text{SU}(2)$ et qui comme nous l'avons montré (implicitement) au début de la section Calcul Spinoriel est isomorphe au groupe des quaternions $\text{SO} (3)$ de module $1$ sur la sphère de dimension $3$. Relations que les mathématiciens appellent dans le cas présent un "homomorphisme de revêtement"...
	\begin{tcolorbox}[title=Remarque,colframe=black,arc=10pt]
	Le groupe spécial unitaire possède une importance particulière en physique des particules. Si le groupe unitaire $\text{U}(1)$ est le groupe de jauge de l'électromagnétisme (pensez au nombre complexe apparaissant dans les solutions de l'équation d'onde!),  $\text{SU}(2)$ est le groupe associé à l'interaction faible, et $\text{SU} (3)$ celui de l'interaction forte. C'est par exemple grâce à la structure des représentations de $\text{SU}(3)$ que Gell-Mann a conjecturé l'existence des quarks.
	\end{tcolorbox}
	Avec une approche différente de celle vue dans le chapitre de Calcul Spinoriel comment montrer que les matrices de Pauli sont les bases de $\text{SU}(2)$?
	
	D'abord, rappelons que nous avons montré dans de Calcul Spinoriel que toute rotation dans l'espace de trois dimensions pouvait s'exprimer à l'aide de la relation:
	
	Et nous avons vu dans le chapitre d'Informatique Quantique qu'une formulation explicitée et décomposée de la relation précédente était:
	
	et donc que tout élément de $\text{SU}(2)$ est produit de ces trois matrices qui font chacune décrire à l'extrémité d'un vecteur dans l'espace une courbe!
	
	Maintenant, nous remarquons que ces trois matrices sont égales lorsque $\theta=0$:
	
	Nous obtenons alors la matrice identité. Donc si nous cherchons la tangente en ce point conjoint, nous pouvons dès lors construire une base ($3$ vecteurs orthogonaux).
	
	Regardons ceci:
	
	Ainsi,  $\text{SU}(2)$ admet pour base:
	
	et ce sont en d'autres termes les générateurs infinitésimaux du groupe $\text{SU}(2)$. $\text{SU}(2)$ a donc une base qui est une Algèbre de Lie selon le vocabulaire des mathématiciens.
	
	Ce résultat est assez remarquable... Puisque $\text{SU}(2)$ et $\text{SO}(3)$ sont isomorphes, nous pouvons alors obtenir la base de l'Algèbre de Lie de $\text{SO}(3)$ alors avec la même méthode!!!
	
	Voyons ceci! Nous avons vu dans le chapitre de Géométrie Euclidienne que les matrices de rotations étaient données par (nous changeons le $R$ par un $U$ afin de ne pas confondre avec les matrices précédentes):
	
	Nous remarquons à nouveau qu'en $\theta=\gamma=\phi=0$ la courbe que fait décrire à un vecteur les trois matrices de rotations passe par:
	
	Alors de la même manière que pour $\text{SU}(2)$, nous calculons les dérivées en ces angles pour déterminer les matrices de base génératrices de $\text{SO}(3)$:
	
	L'algèbre de Lie de $\text{SO} (3)$ admet donc pour base:
	
	En physique, on préfère travailler avec des matrices complexes. Nous introduisons alors les matrices:
	
	Il faut alors remarquer que si nous définissons:
	
	nous avons trivialement pour la complexe conjuguée de la matrice transposée:
	
	et au fait... nous avons aussi les relations de non-commutation (ce que nous pouvons développer sur demande):
	
	et aussi la relation de commutation:
	
	ce que satisfont aussi les matrices de Pauli et... pour rappel (ou information pour ceux qui n'ont pas encore lu la section de Physique Quantique Ondulatoire) les $J_i$ sont les opérateurs du moment cinétique total du système de couplage spin-orbite!!!
	
	La plupart des groupes que nous avons vus jusqu'à présent peuvent être résumés avec la figure suivante :
	\begin{figure}[H]
		\centering
		\includegraphics{img/algebra/special_linear_group.jpg}
	\end{figure}
	Les rotations avec les quaternions indiquées dans la figure ci-dessus sont étudiées dans la section Nombres du chapitre Arithmétique.
	
	\pagebreak
	\subsubsection{Groupes de Symétries}
	Le groupe de symétries d'un objet noté $X$ (image, signal etc. en 1D, 2D, 3D ou autre) est le groupe de toutes les isométries (une isométrie est une transformation qui conserve les longueurs) sous lesquelles il est invariant avec la composition en tant qu'opération.
	
	Tout groupe de symétries dont les éléments ont un point fixe commun, ce qui est vrai pour tous les groupes de symétries de figures limitées, peut être représenté comme un sous-groupe du groupe orthogonal O(n) en choisissant l'origine pour point fixe. Le groupe de symétries propre est alors un sous-groupe du groupe orthogonal spécial $\text{SO}(n)$, et par conséquent, il est aussi appelé le "groupe de rotations" de la figure.
	
	Dans ce qui suit, nous allons interpréter la composée de deux opérations de symétries ou de rotations comme une multiplication au même titre que pour les permutations.
	
	Voyons d'abord deux définitions fondamentales:
	
	\textbf{Définitions (\#\mydef):}
	\begin{enumerate}
		\item[D1.] Le "\NewTerm{groupe des sym\'etries}\index{groupe des sym\'etries}", appelé aussi "\NewTerm{groupe des invariants}\index{groupe des invariants}", de $X$ est l'ensemble des symétries de $X$, muni de la structure de multiplication donnée par composition qui laisse $X$ invariant.
		
		\item[D2.] "\NewTerm{L'ordre}\index{ordre d'un groupe}" d'un groupe est le nombre total de toutes ses symétries uniquement (y compris l'identité!).
	\end{enumerate}
	\begin{tcolorbox}[colframe=black,colback=white,sharp corners]
	\textbf{{\Large \ding{45}}Exemples:}\\\\
	E1. Le coeur (...):
	\begin{figure}[H]
		\centering
		\includegraphics{img/algebra/symmetries_heart.jpg}
	\end{figure}
	a un groupe de symétries total à $2$ éléments, à savoir l'application identité $\text{id}$ et l'application qui est la réflexion dans l'axe vertical $r_v$ (sous-groupe de symétries à $1$ élément). Nous observons que le symétrique est donné aussi via la relation:
	
	\end{tcolorbox}
	
	\pagebreak
	\begin{tcolorbox}[colframe=black,colback=white,sharp corners]
	E2. La lettre phi (...):
	\begin{center}
	\[ \scalebox{8}{$\Phi$} \]
	\end{center}
	a un groupe de symétrie total à $4$ éléments, à savoir l'application identité $\text{id}$, les deux réflexions $r_h$ et $r_v$ et la rotation par l'angle $\pi$ que nous noterons $t_\pi$ (sous-groupe de rotations à $1$ élément). Cette forme possède donc un groupe de symétries d'ordre $3$.\\
	
	Dans ce groupe nous avons:
	
 	(et c'est commutatif!), $t_{\pi}\circ t_{\pi}$ est la rotation par un angle  $2\pi$, ce qui est la même application que l'application identité, donc $t_{\pi}\circ t_\pi=\text{id}$.\\
 	
 	Ainsi, le groupe de symétries de cette lettre est commutatif et la loi de composition est bien interne. C'est donc bien un groupe!\\
 	
 	E3. Le pentagone régulier:
 	\begin{figure}[H]
		\centering
		\includegraphics{img/algebra/symmetries_pentagon.jpg}
	\end{figure}
	a un groupe de symétries total à $10$ éléments à savoir les $5$ rotations $\text{id},t_{2\pi/5},t_{4\pi/5},t_{6\pi/5},t_{8\pi/5}$ ainsi que les $5$ réflexions dans les $5$ axes de symétries. C'est donc un groupe de symétries d'ordre $5$ correspondant au groupe cyclique $\mathbb{Z}/ 5\mathbb{Z}$.
	\begin{tcolorbox}[title=Remarque,colframe=black,arc=10pt]
	Plus généralement, le groupe de symétries d'un $n$-gone régulier (si $n$ est impair) a exactement $2n$ éléments. Ce groupe s'appelle le "\NewTerm{groupe diédral d'ordre $n$}\index{groupe di\'edral d'ordre $n$}" et est noté le plus souvent $D_{2n}$ (il faut faire attention car certains auteurs ne multiplient pas $n$ par le facteur $2$ ce qui fait que l'indice représente alors directement l'ordre et non le nombre d'éléments).
	\end{tcolorbox}	
	Le pentagone a donc $D_{10}$ pour groupe diédral et $\mathbb{Z}/ 5\mathbb{Z}$ en est un "\NewTerm{sous-groupe distingué}\index{sous-groupe distingu\'e}" (nous reviendrons plus tard sur cette notion de sous-groupe distingué).
	\end{tcolorbox}
	
	\pagebreak
	\begin{tcolorbox}[colframe=black,colback=white,sharp corners]
	E4. Le groupe diédral $D_6$ d'ordre $3$ des isométries d'un triangle équilatéral (polygone régulier) a $6$ éléments que nous noterons (afin que l'écriture soit moins lourde):
	
	où $\sigma_1,\sigma_2,\sigma_3$ sont les symétries par rapport aux trois bissectrices (respectivement médiatrices). La table de compositions de ce groupe diédral montre aussi que ce groupe est non-commutatif:
	\begin{figure}[H]
		\centering
		\includegraphics{img/algebra/equilateral_dihedral_group_representation.jpg}
	\end{figure}
	\begin{table}[H]
		\begin{center}
		\begin{tabular}{>{\columncolor[gray]{0.75}}c||c|c|c|c|c|c|}
	\hline
	\rowcolor[gray]{0.75}$\nearrow D_6$ & $id$ & $t_{2\pi/3}$ & $t_{4\pi/3}$ & $\sigma_1$ & $\sigma_2$ & $\sigma_3$ \\
	  \hline \hline
	  % after \\: \hline or \cline{col1-col2} \cline{col3-col4} ...
	 $\text{id}$ & $\text{id}$ & $t_{2\pi/3}$ & $t_{4\pi/3}$ & $\sigma_1$ & $\sigma_2$ &$\sigma_3$ \\
	 \hline
	 $t_{2\pi/3}$ & $t_{2\pi/3}$ & $t_{4\pi/3}$ & $\text{id}$ & $\sigma_3$ & $\sigma_1$ &$\sigma_2$ \\\hline
	 $t_{\pi/3}$ & $t_{4\pi/3}$ & $\text{id}$ & $t_{2\pi/3}$ & $\sigma_2$ & $\sigma_3$ &$\sigma_1$ \\  \hline
	 $\sigma_1$ & $\sigma_1$ & $\sigma_2$ &$\sigma_3$ &$\text{id}$ & $t_{2\pi/3}$ & $t_{4\pi/3}$ \\\hline
	 $\sigma_2$ & $\sigma_2$ & $\sigma_3$ &$\sigma_1$  & $t_{4\pi/3}$ & $\text{id}$ & $t_{2\pi/3}$ \\\hline
	 $\sigma_3$ & $\sigma_3$ & $\sigma_1$ &$\sigma_2$  & $t_{2\pi/3}$ & $t_{4\pi/3}$ & $\text{id}$  \\
	  \hline
		\end{tabular}
		\end{center}
		\caption{Symétries du groupe diédral d'ordre $3$}
	\end{table}
	Nous reviendrons sur cet exemple lorsque nous introduirons un peu plus loin le concept de groupe distingué lors de notre étude des groupes de permutations et la définition des groupes distingués.\\
	
	E5. Regardons un dernier exemple appliqué à la chimie en énumérant les opérations de symétries qui laissent la molécule $\mathrm{NH}_3$ (tétraèdre) invariante.
	\begin{figure}[H]
		\centering
		\includegraphics{img/algebra/nh3.jpg}
	\end{figure}
	\end{tcolorbox}
	
	\pagebreak
	\begin{tcolorbox}[colframe=black,colback=white,sharp corners]
	Le groupe de transformations contient $6$ éléments: l'identité $\text{id}$, $C_3$ qui est la rotation de $2\pi/3$, la rotation de (que nous noterons par la suite $C_{-3}$) toutes deux selon l'axe $z$ (perpendiculaire au plan $xy$ donc...) et $3$ axes  $\sigma_1,\sigma_2,\sigma_3$ de symétrie/réflexion passant chacun par le milieu d'une des arêtes de base au milieu de l'arête opposée comme le montre la figure ci-dessous (pyramide vue du dessus):
	\begin{figure}[H]
		\centering
		\includegraphics{img/algebra/tetrahedral_group_representation.jpg}
		\caption{Opérations laissant invariant un tétraèdre}
	\end{figure}
	La combinaison des différents éléments de symétries montre que la table de compositions est (ce qui prouve que la loi est interne et que nous travaillons donc bien dans un groupe):
	\begin{table}[H]
		\begin{center}
		\begin{tabular}{>{\columncolor[gray]{0.75}}c||c|c|c|c|c|c|c|c|}
		\hline
		\rowcolor[gray]{0.75}$\nearrow $ & $\text{id}$ & $\sigma_1$ & $\sigma_2$ & $\sigma_3$ &$C_3$&$C_{-3}$\\
		  \hline \hline
		  % after \\: \hline or \cline{col1-col2} \cline{col3-col4} ...
		 $\text{id}$ & $\text{id}$ & $\sigma_1$ & $\sigma_2$ & $\sigma_3$ &$C_3$&$C_{-3}$ \\
		 \hline
		 $\sigma_1$ & $\sigma_1$ & $\text{id}$ & $C_{-3}$& $C_3$& $\sigma_3$ &$\sigma_2$ \\\hline
		 $\sigma_2$ & $\sigma_2$ & $C_3$ & $\text{id}$  & $C_{-3}$& $\sigma_1$  & $\sigma_3$ \\\hline
		 $\sigma_3$ & $\sigma_3$ & $C_{-3}$ & $C_3$  & $\text{id}$ & $\sigma_2$ &$\sigma_1$ \\  \hline
		$ C_3$ &  $C_{3}$ &  $\sigma_2$ &$\sigma_3$ & $\sigma_1$ &$C_{-3}$&$id$\\  \hline
		$C_{-3}$&$C_{-3}$& $\sigma_3$ &$\sigma_1$ & $\sigma_2$ &$\text{id}$&$C_3$\\
		  \hline
		\end{tabular}
		\end{center}
		\caption{Compositions de transformations du tétraèdre}
	\end{table}
	Attention à l'ordre des opérations dans le tableau ci-dessus, nous appliquons d'abord l'élément de ligne puis l'élément de colonne!\\
	
	Nous constatons que le groupe n'est donc pas commutatif.
	\end{tcolorbox}

	\pagebreak
	\paragraph{Orbite et Stabilisateur}\mbox{}\\\\
	Nous allons voir maintenant deux définitions que nous retrouverons en cristallographie (leur nom n'est pas innocent!).
	
	\textbf{Définition (\#\mydef):} L'orbite d'un élément $x$ de $E$ est donnée par:
	
	L'orbite de $x$ est l'ensemble des positions (dans $E$) susceptibles d'être occupées par l'image de $x$ sous l'action de $G$. Les orbites forment évidemment une partition de $E$. 
	
	\begin{tcolorbox}[colframe=black,colback=white,sharp corners]
	\textbf{{\Large \ding{45}}Exemple:}\\\\
	Considérons un ensemble $E$ sur lequel agit un groupe $G$, par:
	
	l'ensemble des $6$  sommets d'un hexagone sur lequel nous faisons agir le groupe $G=\{\text{id},t_{2\pi/3},t_{4\pi/3}\}$. Nous observons déjà trivialement que $G$ est bien un groupe!\\
	
	Maintenant, prenons un élément de $E$, par exemple $S_0$.\\
	
	Son orbite va donc être par définition:
	
	\end{tcolorbox}
	\textbf{Définition (\#\mydef):} Le stabilisateur $x$ d'un élément de $E$ est l'ensemble:
	
	des éléments qui laissent $x$ invariant sous leur action. C'est un sous-groupe de $G$.
	\begin{tcolorbox}[colframe=black,colback=white,sharp corners]
	\textbf{{\Large \ding{45}}Exemple:}\\\\
	Pour reprendre notre exemple précédent. Son stabilisateur va être réduit à:
	
	\end{tcolorbox}
	
	
	\pagebreak
	\subsubsection{Groupes de Permutations}
	Les groupes symétriques ont une importance non négligeable dans certains domaines de la physique quantique mais aussi en mathématiques en Algèbre Linéaire (pour la théorie du déterminant comme nous le verrons dans la section correspondante) et dans le cadre de la théorie de Galois (voir plus bas) et dans les codes correcteurs d'erreurs (voir la section du même nom et en particulier l'exemple avec les cartes de crédti). Il convient donc d'y porter aussi une attention toute particulière.
	
	Rappelons d'abord (\SeeChapter{voir section Probabilités page \pageref{simple permutation without repetitions}}) que dans un ensemble $\{1,...,n\}$ il y a $n!$ permutations possibles. Les mathématiciens disent, à juste titre, qu'il y a $n!$ bijections et appellent ce nombre "\NewTerm{ordre du groupe de permutations}\index{ordre du groupe de permutations}".
	
	\begin{tcolorbox}[colframe=black,colback=white,sharp corners]
	\textbf{{\Large \ding{45}}Exemple:}\\
	Prenons par exemple l'ensemble $\{1,2,3\}$. Cet ensemble a $3!$ permutations possibles qui sont notées dans le cadre des groupes de permutation de la manière suivante:
	
	Ce qui se lit dans l'ordre: application identité id, $1$ amène sur $2$ ou $2$ sur un $1$ (en termes de position!), $1$ amène sur $3$ ou $3$ sur $1$, $2$ amène sur $3$ ou $3$ sur $2$, $1$ amène sur $2$ qui amène sur $3$ qui amène sur $1$, $1$ amène sur $3$ qui amène sur $2$ qui amène sur $1$.
	
	Nous pouvons observer facilement que la composition de deux permutations n'est pas commutative:
	
	et que la composition de deux permutations est une loi interne:
	
	avec un élément neutre qui est bien l'identité id. Nous avons donc bien un groupe non commutatif. Rappelons également au lecteur que certains éléments du groupe, s'ils sont bien choisis, peuvent former un sous-groupe. C'est l'exemple de:
	
	qui est un sous-groupe de $S_3$ (il est facile de vérifier qu'il possède toutes les propriétés d'un groupe).
	\end{tcolorbox}
	
	\textbf{Définition (\#\mydef):} Un sous-groupe $H$ d'un groupe $G$ est appelé "\NewTerm{groupe distingué}\index{groupe distingu\'e}" si, pour tout $g$ de $G$ et tout $h$ de $H$, nous avons que $ghg^{-1}$ est élément de $H$. Les mathématiciens appellent cela un "\NewTerm{automorphisme intérieur}\index{automorphisme intérieur}"...
	
	Voyons d'abord un exemple géométrique parlant après quoi nous reviendrons à cette définition avec $S_3$.

	\begin{tcolorbox}[colframe=black,colback=white,sharp corners]
	\textbf{{\Large \ding{45}}Exemple:}\\
	Nous avons vu plus haut les éléments du groupe de symétrie diédral d'ordre $3$ du triangle équilatéral. Géométriquement ils correspondent tous à des déplacements dans le plan dans lequel se trouve le triangle. Nous avions obtenu pour rappel le tableau de compositions suivant:
	\begin{figure}[H]
		\centering
		\includegraphics[scale=0.8]{img/algebra/equilateral_dihedral_group_representation.jpg}
	\end{figure}
	\begin{table}[H]
		\centering
		\begin{tabular}{>{\columncolor[gray]{0.75}}c||c|c|c|c|c|c|}
		\hline
		\rowcolor[gray]{0.75}$\nearrow D_6$ & $id$ & $t_{2\pi/3}$ & $t_{4\pi/3}$ & $\sigma_1$ & $\sigma_2$ & $\sigma_3$ \\
		  \hline \hline
		  % after \\: \hline or \cline{col1-col2} \cline{col3-col4} ...
		 $\text{id}$ & $\text{id}$ & $t_{2\pi/3}$ & $t_{4\pi/3}$ & $\sigma_1$ & $\sigma_2$ &$\sigma_3$ \\
		 \hline
		 $t_{2\pi/3}$ & $t_{2\pi/3}$ & $t_{4\pi/3}$ & $\text{id}$ & $\sigma_3$ & $\sigma_1$ &$\sigma_2$ \\\hline
		 $t_{\pi/3}$ & $t_{4\pi/3}$ & $\text{id}$ & $t_{2\pi/3}$ & $\sigma_2$ & $\sigma_3$ &$\sigma_1$ \\  \hline
		 $\sigma_1$ & $\sigma_1$ & $\sigma_2$ &$\sigma_3$ &$\text{id}$ & $t_{2\pi/3}$ & $t_{4\pi/3}$ \\\hline
		 $\sigma_2$ & $\sigma_2$ & $\sigma_3$ &$\sigma_1$  & $t_{4\pi/3}$ & $\text{id}$ & $t_{2\pi/3}$ \\\hline
		 $\sigma_3$ & $\sigma_3$ & $\sigma_1$ &$\sigma_2$  & $t_{2\pi/3}$ & $t_{4\pi/3}$ & $\text{id}$  \\
		  \hline
		\end{tabular}
	\end{table}
	D'abord, nous constatons facilement à l'aide de ce tableau que nous avons:
	\begin{itemize}
		\item Le sous-groupe formé de $\{\text{id}\}$ d'ordre $1$
	
		\item Le sous-groupe formé de $\{\text{id},t_{2\pi/3},t_{4\pi/3}\}$ d'ordre $3$

		\item Le sous-groupe formé de $\{\text{id},\sigma_1\}$ d'ordre $2$

		\item Le sous-groupe formé de $\{\text{id},\sigma_2\}$ d'ordre $2$

		\item Le sous-groupe formé de $\{\text{id},\sigma_3\}$ d'ordre $2$
	\end{itemize}
	Parmi ces cinq sous-groupes, voyons lesquels sont distingués (cela est relativement facile à visualiser à l'aide du tableau de compositions):
	\begin{itemize}
		\item Le sous-groupe formé de $\{\text{id}\}$
	
		\item Le sous-groupe formé de $\{\text{id},t_{2\pi/3},t_{4\pi/3}\}$ 
	\end{itemize}
	\end{tcolorbox}
	
	\pagebreak
	\begin{tcolorbox}[colframe=black,colback=white,sharp corners]
	Nous allons voir maintenant une chose remarquable! En numérotant par $1$, $2$ et $3$ les sommets du triangle équilatéral et en prenant les rotations dans le sens des aiguilles d'une montre, nous pouvons identifier les éléments de $D_6$ aux éléments suivants de $S_3$:
	
	et reconstruire la même table de compositions (copie de la précédente mais juste avec le changement d'écriture... hé hé!):
	\begin{figure}[H]
		\centering
		\includegraphics{img/algebra/equilateral_dihedral_group_representation.jpg}
	\end{figure}
	\begin{table}[H]
		\begin{center}
		\begin{tabular}{>{\columncolor[gray]{0.75}}c||c|c|c|c|c|c|c|c|}
		\hline
		\rowcolor[gray]{0.75}$\nearrow S_3$ & $ (1) $ & $ (1\ 2\ 3) $ & $(1\ 3\ 2)$ & $(1\ 2)$ & $(1\ 3)$ & $(2\ 3)$ \\
		  \hline \hline
		  % after \\: \hline or \cline{col1-col2} \cline{col3-col4} ...
		 $(1)$ & $(1)$ & $(1\ 2\ 3)$ & $(1\ 3\ 2)$ & $(1\ 2)$ &$(1\ 3)$&$(2\ 3)$ \\
		 \hline
		 $(1\ 2\ 3)$ & $(1\ 2\ 3)$ & $(1\ 3\ 2)$ & $(1)$& $(2\ 3)$ & $(1\ 2)$ & $(1\ 3)$ \\ \hline
		$(1\ 3\ 2)$ & $(1\ 3\ 2)$ & $(1)$ & $(1\ 2\ 3)$& $(1\ 3)$ &$(2\ 3)$ &$(1\ 2)$ \\ \hline
		 $(1\ 2)$ &$(1\ 2)$ &$(1\ 3)$& $(2\ 3)$ &  $(1)$&$(1\ 2\ 3)$  & $(1\ 3\ 2)$  \\ \hline
		$(1\ 3)$ &$(1\ 3)$ &$(2\ 3)$ &$(1\ 2)$ &$(1\ 3\ 2)$ & $(1)$ & $(1\ 2\ 3)$  \\ \hline
		$(2\ 3)$ &$(2\ 3)$ &$(1\ 2)$ &$(1\ 3)$ &$(1\ 2\ 3)$ & $(1\ 2\ 3)$ & $(1)$   \\\hline
		\end{tabular}
		\end{center}
	\end{table}
	\end{tcolorbox}
	Bon... ce petit interlude fermé, revenons au groupe distingué de $S_3$ (car il va être important pour notre introduction aux groupes de Galois) et rappelons d'abord que:
	
	et nous voyons que le sous-groupe distingué est formé de:
	
	\textbf{Définition (\#\mydef):} Pour tout sous-groupe $H$ stable par les automorphismes intérieurs d'un groupe $G$, nous appelons "\NewTerm{indice de $H$ dans $G$}" le quotient de l'ordre du groupe $G$ par l'ordre du sous-groupe $H$ et nous l'écrivons $[G/H]$.
	
	\begin{tcolorbox}[colframe=black,colback=white,sharp corners]
	\textbf{{\Large \ding{45}}Exemple:}\\\\
	L'indice du sous-groupe $\{(1), (1 2)\}$ dans le groupe $S_3$ est $6/2$ c'est-à-dire $3$.
	\end{tcolorbox}
	 Ce concept nous sera très utile lors de notre introduction aux corps de Galois plus loin.
	
	Considérons maintenant, la permutation particulière $\sigma$ pour aborder le sujet sous un angle différent mais équivalent:
	
	Les mathématiciens ont pour habitude de noter cela, dans un premier temps, sous la forme:
	 
	avec:
	
	Dès lors:
	 
	Etant donné $\sigma$ et $\tau$, deux permutations, il est naturel de regarder leur composition (rappelons que cela signifie d'abord $\sigma$, puis $\tau$ comme pour la composition de fonctions).
	
	Ainsi, si:
	
	Alors:
	
	et:
	
	Maintenant, l'idée est d'interpréter la composition comme une multiplication de permutations. Cette multiplication est alors non-commutative comme nous venons de le constater dans l'exemple précédent. Nous avons en général $\sigma \circ \tau \neq \tau \circ \sigma$.
	
	Chaque bijection a un inverse (une fonction réciproque). Dans notre exemple il s'agit de évidemment de:
	
	Géométriquement, pour calculer l'inverse $\sigma^{-1}$ d'un élément $\sigma$, il suffit de prendre la réflexion du dessin de $\sigma$ dans un axe horizontal comme le montre la partie gauche de la figure ci-dessous:
	\begin{figure}[H]
		\centering
		\includegraphics{img/algebra/permutations.jpg}
		\caption{Exemples de composées et d'inverses de permutations}
	\end{figure}
	Nous pouvons représenter une permutation $\sigma$ comme une matrice de permutation, c'est-à-dire la matrice qui mappe :
	
	Une matrice $n\times n$ est une matrice de permutation si et seulement si toutes ses entrées sont des zéros et des uns, chaque colonne a exactement un $1$ et chaque ligne a exactement un $1$. Dans ce cas, nous avons :
	 
	 \textbf{Définitions (\#\mydef):}
	 \begin{enumerate}
		\item[D1.] L'ensemble des permutations d'un ensemble avec $n$ éléments, muni de cette structure de multiplication, s'appelle le "\NewTerm{groupe des permutations d'ordre $n$}\index{groupe des permutations d'ordre $n$}" ou  "\NewTerm{groupe des substitutions d'ordre $n$}\index{groupe des substitutions d'ordre $n$}", et se note $S_n$ ou encore $S(n)$.
		
		\item[D2.] Nous disons qu'un élément $\sigma$ de $S_n$ est un "\NewTerm{cycle d'ordre $k$}\index{cycle d'ordre $k$}", ou un "\NewTerm{$k$-cycle}", s'il existe $a_1,a_2,...,a_k\in \{1,...,n\}$ tel que:
		\begin{itemize}
			\item $\sigma$ envoie $a_1$ sur $a_2$, $a_2$ sur $a_3$, ..., $a_{k-1}$, et $a_k$ sur $a_1$.

			\item $\sigma$ fixe tous les autres éléments de $S_n$
		\end{itemize}
		et nous notons le cycle ainsi:
		
		\begin{tcolorbox}[colframe=black,colback=white,sharp corners]
		\textbf{{\Large \ding{45}}Exemple:}\\\\
		Pour mieux comprendre reprenons notre exemple de $S_4$:
		
		Ce groupe symétrique est un $3$-cycle noté $\sigma=(1\; 3\; 4)$ car dans l'ordre: $1$ envoie sur $3$, $3$ envoie sur $4$ et $4$ envoie sur $1$ (et le $2$ n'étant pas mentionné il reste fixe). Nous pouvons noter cela aussi des façons suivantes équivalentes: $\sigma=(3\; 4\; 1)$ ou encore $\sigma=(4\; 1\; 3)$.
		\end{tcolorbox}

		\item[D3.] L'ordre d'un $k$-cycle est $k$ (d'où le nom!).
		\begin{tcolorbox}[colframe=black,colback=white,sharp corners]
		\textbf{{\Large \ding{45}}Exemple:}\\\\
		Effectivement si nous reprenons $\sigma=(1\; 3\; 4)$, nous avons alors:
		
		\end{tcolorbox}

		\item[D4.] Nous disons qu'une permutation $\sigma$ est un "\NewTerm{cycle}\index{cycle (permutation)}" s'il existe $k\in \mathbb{N}$ tel que $\sigma$ est un $k$-cycle.
		
		Attention! Toute permutation doit s'écrire comme un produit de cycles disjoints (c'est-à-dire qu'un nombre qui apparaît dans un cycle ne doit pas apparaître dans un autre cycle).
		
		\begin{tcolorbox}[colframe=black,colback=white,sharp corners]
		\textbf{{\Large \ding{45}}Exemple:}\\\\
		Par exemple, dans $S_9$, nous avons:
		
		Donc cette permutation est un produit d'un $4$-cycle et d'un $3$-cycle disjoint.\\
		
		Nous laisserons d'ailleurs le lecteur vérifier par lui-même que le groupe cyclique engendré par $\sigma$ (qui dans le cas présent est sous-groupe de $S_9$) est d'ordre $12$ ($12$-cycle)...
		\end{tcolorbox}
		\begin{tcolorbox}[title=Remarque,colframe=black,arc=10pt]
		Les mathématiciens peuvent démontrer que si $\sigma$ est un élément qui a une décomposition en $c$ cycles disjoints de longueur $n_1,n_2,...,n_c$ alors l'ordre $\sigma$ de est le plus petit commun multiple des ordres de tous les cycles disjoints qui le composent.
		\end{tcolorbox}	
	\end{enumerate} 
	Nous supposerons également intuitif que dans le vocabulaire commun, un $2$-cycle dans $S_n$ s'appelle aussi une "\NewTerm{transposition}\index{transposition}".
	
	Allons un petit peu plus loin. Nous nous proposons de montrer par l'exemple que l'ensemble des transpositions engendre $S_n$. Autrement, dit, toute permutation s'écrit comme un produit de transpositions!
	
	\begin{tcolorbox}[colframe=black,colback=white,sharp corners]
	\textbf{{\Large \ding{45}}Exemple:}\\\\
	Reprenons notre exemple (il s'agit d'une permutation paire):
		
	\end{tcolorbox}
	En général, un $k$-cycle s'écrit donc comme produit de $k-1$ transpositions.
	
	\begin{theorem}
	Comme les permutations d'un ensemble fini constituent un groupe. Cela signifie (entre autres choses) qu'il existe donc toujours un entier $k$, tel que $p$ opéré $k$ fois est la transformation identité (c'est-à-dire l'opération qui ne change rien).
	\end{theorem}
	\begin{dem}
	Si $G$ est un groupe fini et que $g\in G$, nous considérons la suite d'éléments (se rappeler que dans un groupe il ,n'y a, qu'une opération et donc le carré, le cube, etc. signifie que nous composons cette opération!):
	
	Par exemple, dans les groupes de permutations, l'opération est la composition des permutations.
	
	Étant donné que $G$ est fini et que cette suite est infinie, il existe forcément deux éléments égaux dans la suite... Il existe donc deux indices différents n et m tels que:
	
	En supposant que $n<m$, l'égalité précédente se simplifie et nous obtenons:
	
	où $e$ est l'élément neutre du groupe.
	\begin{flushright}
		$\blacksquare$  Q.E.D.
	\end{flushright}
	\end{dem}
	Nous allons voir que les permutations étant bijectives, nous pouvons créer sur des groupes finies des compositions d'opération de permutation qui finissent toujours par ramener à l'état initial (application identité id).
	\begin{tcolorbox}[colframe=black,colback=white,sharp corners]
	\textbf{{\Large \ding{45}}Exemples:}\\\\
	E1. Dans une liste de $5$ objets, nous échangeons le premier et le troisième, et, en même temps, nous faisons passer le deuxième en position $4$, celui qui est en position $4$ est mis en position $5$ et celui qui est en position $5$ est mis en position $2$. En nous réitérons. Cela donne:
	
	Nous sommes revenus au point de départ après $6$ étapes.
	\end{tcolorbox}
	
	\pagebreak
	\begin{tcolorbox}[colframe=black,colback=white,sharp corners]
	E2. Considérons la "\NewTerm{transformation du photomaton}\index{transformation du photomaton}" d'une image de Mona Lisa de dimension de $256$ par $256$ pixels:
	\begin{figure}[H]
		\centering
		\includegraphics{img/algebra/mona_lisa.jpg}
		\caption{Transformation du photomaton}
	\end{figure}
	Nous pouvons avoir l'impression que chaque image a été obtenue à partir de la précédente en réduisant la taille de l'image de moitié, ce qui a donné quatre morceaux analogues que nous avons placés en carré pour obtenir une image ayant la même taille que l'image d'origine. Mais en fait il n'est est rien! Le nombre de pixels a été conservé (aucun pixel n'est dupliqué!!!) et en fait nous avons seulement déplacé les pixels par permutation pour avoir quatre images qui ne contiennent pas réellement toute l'information de l'image d'origine mais seulement une partie. \\
	
	En réitérant la procédure $8$ fois, nous retombons toujours sur l'image d'origine quelle que soit l'image de départ. La question est de comprendre alors pourquoi?\\
	
	Considérons que l'image d'origine est un carré d'une taille de $16$ pixels de large par $16$ pixels de haut (mais vous pouvez appliquer ce qui va suivre avec une image rectangulaire de n'importe quelle taille et vous verrez que cela marche aussi!). Chaque pixel d'une ligne (la démarche est exactement la même pour les colonnes!) est identifié par une coordonnée selon l'axe $X$ allant de $0$ à $15$.\\
	
	Nous avons ainsi une suite de nombres au début où les coordonnées de pixels correspondant à la leur coordonnée $x$ :
	
	Nous faisons alors la permutation qui consiste à noter $k$ la position d'un pixel et de faire:
	
	\end{tcolorbox}

	\pagebreak
	\begin{tcolorbox}[colframe=black,colback=white,sharp corners]
	Cela donne alors à la première permutation:
	
	Ainsi, pour une image de $16$ par $16$ pixels, il faut quatre permutations, ce qui correspond à $2^4=16$. Donc, pour une image de $256$ pixels, nous avons $256=2^8$, d'où le fait qsu'il faille $8$ permutations pour retrouver la Mona Lisa d'origine avec:
	
	Ainsi, dans le cas général d'une image de largeur $L$ en nombre de pixels en comptant à partir de $1$, la transformation est:
	
	où $E^+[...]$ est la valeur entière supérieur la plus proche au cas où $L$ serait impair.\\
	
	Le lecteur aura aussi peut-être remarqué quelque chose d'intéressant si nous reprenons notre exemple avec l'image de $16$ pixels... Effectivement, prenons le troisième pixel depuis la gauche de coordonnée $x$ égale à $2$. En binaire, sa position initiale est alors $0010$. Après la première permutation, sa coordonnée $x$ est égale à $1$, soit en binaire: $0001$. Après la deuxième permutation, sa coordonnée $x$ est égale $8$, soit en binaire: $1000$, etc. Au fait nous voyons que chaque permutation se résume en binaire à décaler les bits vers la droite.
	\end{tcolorbox}
	\textbf{Définition (\#\mydef):} Soit $\sigma \in S_n$ une permutation. Nous disons que $\sigma$ est "\NewTerm{permutation paire}\index{permutation paire}" si, dans une écriture de $\sigma$ comme produit de transpositions, il y a un nombre pair de transpositions. Nous disons évidemment que $\sigma$ est une "\NewTerm{permutation impaire}\index{permutation impaire}" si, dans une écriture de $\sigma$ comme produit de transpositions, il y a un nombre impair de transpositions.
	
	Finissons par un petit complément... Nous avons que $S_3$ est un groupe des permutations d'ordre $3$ avec donc $3!=6$ permutations possibles.
	
	Si nous énumérons les $6$ permutations nous avons vu que nous obtenons:
	
	Parmi ceux-ci, seuls certains peuvent être écrits comme un produit pair de transpositions :
	
	Les permutations paires forment avec la permutation identité, un sous-groupe (non commutatif !) que l'on nomme le "\NewTerm{groupe alternatif d'ordre $n$}\index{groupe alternatif d'ordre $n$}" et que l'on note $AN $. C'est facile à vérifier avec l'exemple précédent.
		
	\subsection{Théorie de Galois}\label{galois theory}
	En algèbre abstraite, la théorie de Galois, du nom d'Évariste Galois, établit un lien entre la théorie des corps et la théorie des groupes. En utilisant la théorie de Galois, certains problèmes de la théorie des corps peuvent être réduits à la théorie des groupes, qui est, dans un certain sens, plus simple et mieux comprise.
	
	À l'origine, Galois utilisait des groupes de permutations pour décrire comment les diverses racines d'une équation polynomiale donnée sont liées les unes aux autres.
	
	La naissance et le développement de la théorie de Galois ont été provoqués par la question suivante, dont la réponse est connue sous le nom de "\NewTerm{théorème d'Abel-Ruffini}\index{théorème d'Abel-Ruffini}" : Pourquoi n'y a-t-il pas de formule pour les racines d'une équation polynomiale de degré cinq (ou supérieur) en termes de coefficients du polynôme, en utilisant uniquement les opérations algébriques habituelles (addition, soustraction, multiplication, division) et l'application de radicaux (racines carrées, racines cubiques, etc.)?.
	
	La théorie de Galois ne fournit pas seulement une belle réponse à cette question, elle explique également en détail pourquoi il est possible de résoudre des équations de degré quatre ou moins de la manière ci-dessus, et pourquoi leurs solutions prennent la forme qu'elles prennent. De plus, cela donne un moyen conceptuellement clair et souvent pratique de dire quand une équation particulière de degré supérieur peut être résolue de cette manière.
	
	La théorie de Galois trouve son origine dans l'étude des fonctions symétriques : les coefficients d'un polynôme monique (polynômes dont le coefficient affecté au degré le plus haut du dénominateur doit être $1$) sont (au signe près) les polynômes symétriques élémentaires dans les racines. Par exemple: roots. For instance:
	
 	où $1$, $a + b$ et $ab$ sont les polynômes élémentaires de degré $0$, $1$ et $2$ à deux variables.
 	
 	Nous avons essayé de rendre cette partie du livre aussi simple que possible. Nous espérons donc que notre objectif est atteint si vous comprenez ce qui suit. Commençons maintenant !!!
 	
 	\subsubsection{Polynômes élémentaires symétriques et invariants}
 	\textbf{Définition (\#\mydef):} Le "\NewTerm{$k$-ième polynôme symétrique élémentaire à $n$ variables}\index{polyn\-ome sym\'etrique \'el\'ementaire}", noté $s_k(x_k,\ldots , x_n)$ est la somme de tous les monômes de degré $k$ possibles avec $n$ variables avec chaque $x_i$ apparaissant PAS PLUS D'UNE FOIS DANS CHAQUE MONÔME. Formellement, pour $k\leq n$ :
	
	Un polynôme est dit "\NewTerm{invariant sous $S_n$}\index{polyn\-ome invariant}" si et seulement si c'est un polynôme dans les fonctions symétriques élémentaires $s_1,\ldots, s_n$.

	Dès lors:
	
	\begin{tcolorbox}[colframe=black,colback=white,sharp corners]
	\textbf{{\Large \ding{45}}Exemple:}\\\\
	Pour $n = 1$:
	
	Pour $n = 2$:
	
	Pour $n = 3$:
	
	Pour $n = 4$:
	
	Considérons maintenant l'équation :
	
	Elle peut être réécrite comme :
	
	ou comme nous connaissons les relations de Viète à deux racines, ces dernières peuvent être réécrites :
	
	\end{tcolorbox}
	
	\pagebreak
	\begin{tcolorbox}[colframe=black,colback=white,sharp corners]
	C'est-à-dire:
	
	De même, pour un polynôme du troisième degré :
	
	et dès lors:
	
	\end{tcolorbox}
	Les polynômes symétriques élémentaires apparaissent lorsque nous développons une factorisation linéaire d'un polynôme monique. Nous avons l'identité :
	
	C'est-à-dire que lorsque nous substituons des valeurs numériques aux variables $r_1,r_2,\dots,r_n$, nous obtenons le polynôme univarié monique (avec la variable $x$) dont les racines sont les valeurs substituées à $r_1,r_2,\dots, r_n$ et dont les coefficients sont à leur signe près les polynômes symétriques élémentaires. Ces relations entre les racines et les coefficients d'un polynôme sont nommées "\NewTerm{formules générales de Viète}\index{formules générales de Vi\'ete}" pour lesquelles nous avons déjà vu deux cas particuliers dans la section Calcul et que nous généraliserons plus loin.
	\begin{theorem}
	Si $r_1,r_2,\ldots, r_n$ sont les racines d'un polynôme de degré $n$, alors:
	
	\end{theorem}
	\begin{dem}
	Nous allons prouver ceci par récurrence sur le degré du polynôme. Si notre polynôme est de degré $n = 1$ avec la racine $r$, le côté gauche est $x-r$, et le côté droit est $x-s_1(r)= x-r$, donc l'équation est vraie pour $n = 1$. Supposons que l'équation soit vraie pour tous les polynômes de degré $n$. Soit $P(x)$ de degré $n+1$ de racines $r_1,\ldots,r_{n+1}$.

    Dès lors, nous pouvons écrire:
    
	où nous laissons $s_i$ dénoter $s_i(r1, \ldots , a_n)$ par souci de concision. En multipliant le membre de droite :
	
	Étant donné que:
	
	et:
	
	Nous avons alors:
	
	Reste maintenant que si l'on peut prouver que :
	
	pour tous les autres $i$, nous pourrons conclure que l'équation est vraie pour $n+1$, donc pour tous les $n$.
	
	Rappelez-nous que par définition :
	
	Alors:
	
	En séparant la somme par rapport aux monômes divisibles par $r_{n+1}$, nous voyons que ce qui précède est égal à (la plupart du temps le mieux est de vérifier cela en utilisant l'un des exemples précédents donnés plus tôt):
	
	il est donc clair que la relation que nous voulions tient!
	\begin{flushright}
		$\blacksquare$  Q.E.D.
	\end{flushright}
	\end{dem}
	
	\subsubsection{Formules générales de Viète}
	Si nous écrivons l'équation du second degré comme suit :
	
	Nous savons déjà que:
	
	Et aussi pour un polynôme du troisième degré nous avons vu juste plus haut dans les exemples que si nous avons :
	
	alors:
	
	Nous pouvons facilement voir émerger un motif qui est:
	
	
	\begin{flushright}
	\begin{tabular}{l c}
	\circled{90} & \pbox{20cm}{\score{4}{5} \\ {\tiny 16 votes,  81.25\%}} 
	\end{tabular} 
	\end{flushright}

	%to make section start on odd page
	\newpage
	\thispagestyle{empty}
	\mbox{}
	\section{Calcul Différentiel et Intégral}\label{differential and integral calculus}
	\lettrine[lines=4]{\color{BrickRed}L}e calcul différentiel est un des domaines les plus passionnants et vastes de la mathématique, et il existe une littérature considérable (colossale) sur le sujet. Les résultats initiés par des scientifiques comme Fermat, Newton, Leibniz, Euler et compagnie depuis la fin du 17ème siècle retrouvent des implications dans absolument tous les domaines de la physique, de l'informatique, de l'électronique, de la chimie, de la finance, de la biologie et de la mathématique elle-même.

	Les mathématiciens ont rédigé une telle quantité de théorèmes depuis sa naissance au milieu du 16ème siècle sur le sujet que la validation d'un échantillon de ceux-ci est parfois délicate car nécessitant à eux-seuls la vie d'un homme pour être parcourus (c'est un problème que la communauté des mathématiciens reconnaît) et vérifiés (ce qui fait que parfois personne ne les vérifie...).

	Ce constat fait, nous avons choisi de ne présenter ici que les points absolument nécessaires à la compréhension des outils fondamentaux de l'ingénieur. Les puristes nous excuseront donc pour l'instant de ne pas présenter certains théorèmes qui peuvent leur sembler indispensables mais que nous rédigerons une fois le temps venu...

	Nous allons principalement étudier dans ce qui va suivre ce que les mathématiciens aiment bien préciser (et ils ont raison): les cas généraux des fonctions réelles à une variable réelle. Les fonctions plus complexes (à plusieurs variables réelles ou complexes, continues ou discrètes) viendront une fois cette partie terminée.

	\begin{tcolorbox}[title=Remarque,colframe=black,arc=10pt]
	Nous ne nous attarderons pas à démontrer les dérivées et primitives de toutes les fonctions car comme il y a une infinité de fonctions possibles, il y a également une infinité de dérivées et de primitives. C'est le rôle des professeurs dans les instituts scolaires d'entraîner les élèves à appliquer et à comprendre le raisonnement de dérivation et d'intégration par des applications sur des fonctions connues (l'Internet ne remplacera très probablement jamais l'école à ce niveau).
	\end{tcolorbox}	
	
	\subsection{Calcul Différentiel}\label{differential calculus}
	Soit une fonction $f$ réelle à une variable réelle $x$ notée $f(x)$ (nous nous limitons à ce cas de figure pour l'instant et étudierons les dérivées partielles dans des espaces à un nombre de dimensions quelconques plus loin) continue au moins dans un intervalle où se situe l'abscisse $a$.

	\pagebreak
	\textbf{Définitions (\#\mydef):} 	
	\begin{enumerate}
		\item[D1.] Nous appelons "\NewTerm{pente moyenne}\index{pente moyenne}", ou encore "\NewTerm{coefficient directeur}\index{coefficient directeur}" le rapport de la projection orthogonale de deux points $x_1 \neq x_2$ de la fonction $f$ non nécessairement continue sur l'axe des abscisses et des ordonnées tel que:
		
		Ce qui se représente sous forme graphique de la manière suivante avec une fonction particulière:
	
		\begin{figure}[H]
			\centering
			\begin{tikzpicture}[>=stealth',
	                    dot/.style={circle,draw,fill=white,inner sep=0pt,minimum size=4pt},scale=1.25]
	
		    % draw axis lines
		    \draw[->,thick] (-0.5,0) -- ++(11,0) node[below left]{$x$};
		    \draw[->,thick] (0,-0.5) -- ++(0,7) node[below right]{$y=f(x)$};
		    \coordinate (O) at (0,0);
		
		    % create path for function curve
		    \path[thick,red] (-0.3,2) to[out=-25, in=200] coordinate[pos=0.2] (p) coordinate[pos=0.6] (q) (9,5);
		    % fill area
		    \fill[blue, opacity=.1] (p) -| (q);
		    % draw the secant line with fixed length
		    \draw[shorten <=-1.5cm] (p) -- ($ (p)!7.5cm!(q) $) node[below right, pos=0.9]{Sécante};
		    % draw function curve
		    \draw[thick,red] (-0.3,2) to[out=-25, in=200] (9,5);
		
		    % draw all points
		    \node[dot,label={above:$P$}] (P) at (p) {};
		    \node[dot,label={above:$Q$}] (Q) at (q) {};
		    \node[dot] (p1) at (P |- O) {};
		    \node[dot] (p2) at (Q |- O) {};
		    \node[dot] (p3) at (P -| Q) {};
		
		    % draw lines between nodes and place text
		    \draw (P) -- node[left]{$f(x_{1})$} (p1) node[dot,label={below:$x_{1}$}]{};
		    \draw (p2) node[dot,label={below:$x_2=x_{1} + h$}]{} -- (p3);
		    \path (p1) -- node[below]{$h$} (p2);
		
		    % draw blue arrows between nodes
		    \draw[<->,blue,thick] (P) -- node[below]{$\Delta x=h$} (p3);
		    \draw[<->,blue,thick] (Q) -- node[right]{$f(x_{1} + h) - f(x_{1})=\Delta y=\Delta f$} (p3);
		
		    % draw the explanation for the y-value of point Q
		    \draw[help lines] (Q) -- (Q -| {(9.5,0)}) ++(-0.5,0) coordinate (p4);
		    \draw[help lines, <->] (p4) -- node[fill=white,text=black]{$f(x_{1} + h)=f(x_2)$} (p4 |- O);
			\end{tikzpicture}
		\end{figure}
		
		\begin{tcolorbox}[title=Remarque,colframe=black,arc=10pt]
	$\Delta$ signifiant "un delta" exprime le fait que nous sous-entendons une différence d'une même quantité.
		\end{tcolorbox}
		
		Nous supposerons comme évident (sans démonstration) que deux fonctions dont les pentes sont les mêmes dans un même intervalle de définition, y sont parallèles (ou confondues).
		
		\begin{tcolorbox}[title=Remarque,colframe=black,arc=10pt]
		Nous démontrerons dans la section de Géométrie Analytique que deux fonctions dont la multiplication des pentes vaut $-1$ sont perpendiculaires.
		\end{tcolorbox}
		
		\item[D2.] Nous appelons "\NewTerm{nombre dérivé en $a$}\index{dériv\'ee}" ou "\NewTerm{pente instantanée}" ou encore "\NewTerm{dérivée première}\index{dériv\'ee premi\'ere}", la limite quand $h$ tend vers $0$ (si elle existe) du rapport de la projection orthogonale de deux points infiniment proches $x_1\neq x_2$ de la fonction $f$ continue (dans le sens qu'elle ne contient pas de "trous") sur l'axe des abscisses $x$ et des ordonnées $y$ tel que:
		
		Cette relation est parfois nommée "\NewTerm{quotient de différence de Newton}\index{quotient de diff\'erence de Newton}".
		
		Une interprétation graphique donne donc bien que $f'(a)$ est le coefficient directeur (la pente de la tangente au point d'abscisse $a$).
		
		\begin{tcolorbox}[title=Remarques,colframe=black,arc=10pt]
		\textbf{R1.} La lettre $\mathrm{d}$ (écrite à la verticale, non en italique comme le recommande la norme internationale ISO 80000-2 : \textit{Quantités et unités - Partie 2 : Signes et symboles mathématiques à utiliser dans les sciences naturelles et la technologie} ) signifie ici un "\NewTerm{différentiel}\index{diff\'erentiel}" et exprime le fait que nous prenons une différence infinitésimale de la quantité concernée.\\
	
		\textbf{R2.} Nous renvoyons le lecteur à la section d'Analyse Fonctionnelle  (page \pageref{limits}) pour la définition de ce qu'est exactement une fonction continue.
		\end{tcolorbox}
		\item[D3.] Soit $f$ une fonction définie sur un intervalle $I$ et dérivable en tout point $a$ de $I$, la fonction qui à tout réel $a$ de $I$ associe le nombre $f'(a)$ est appelée "\NewTerm{fonction dérivée de $f$ sur $I$}" et est notée $f'$.
		
		\begin{tcolorbox}[title=Remarque,colframe=black,arc=10pt]
		Au niveau des notations les physiciens adoptent suivant leur humeur différentes notations possibles pour les dérivées. Ainsi, considérons la fonction réelle $f(x)$ à une variable réele $x$, vous trouverez dans la littérature ainsi que dans le présent livre les différentes notations suivantes pour la dérivée première: 
		
		ou encore en considérant implicitement que $f$ est fonction de $x$ (ceci permet d'alléger un petit peu la tailles des développements):
		
		\end{tcolorbox}
	\end{enumerate}
	
	Nous pouvons de la même manière définir les dérivées d'ordre 2 (dérivée d'une dérivée), les dérivées d'ordre 3 (dérivée d'une dérivée d'ordre 2) et ainsi de suite. Nous rencontrerons par ailleurs très fréquemment de telles dérivées en physique (et même en maths pour l'analyse fonctionnelle).
	
	Précisons que les dérivées d'ordre 2 ont une interprétation très importante en physique et dans le domaines de l'optimisation (\SeeChapter{voir section Méthodes Numériques page \pageref{newton raphson method}}). Effectivement, si le signe de la dérivée première est positif puis devient négatif quand $x$ croît, alors nous devinons facilement que nous parcourons un maximum local d'une fonction (point où la dérivée est nulle) et que si le signe de la dérivée première est négatif puis devient positif quand $x$ croît, alors nous parcourons un minimum local de la fonction (point où la dérivée est aussi nul). En d'autres termes, lorsque la pente change de signe (s'annule en changeant de signe) la fonction passe par un extremum (minimum ou maximum) et la tangente est "horizontale" en ce point: parallèle à l'axe des abscisses. Nous parlons alors de "\NewTerm{point tournant}\index{point tournant}\label{turning point}":
	\begin{figure}[H]
		\centering
		\includegraphics[scale=0.5]{img/algebra/turning_points.jpg}
		\caption{Points tournants mis en évidence}
	\end{figure}
	Mais lorsque la dérivée d'ordre 2 est nulle, cela signifie que la courbure de la fonction s'inverse. On parle alors d'un "\NewTerm{point d'inflexion}\index{point d'inflexion}\label{inflection point}" :
	\begin{figure}[H]
		\centering
		\begin{tikzpicture}[line cap=round,line join=round,>=triangle 45,x=3cm,y=2cm]
            \draw[->,color=black] (-0.9,0.) -- (1.9,0.);
            \foreach \x in {}
            \draw[shift={(\x,0)},color=black] (0pt,2pt) -- (0pt,-2pt) node[below] {\footnotesize $\x$};
            \draw[->,color=black] (0.,-2.4) -- (0.,2.4);
            \foreach \y in {}
            \draw[shift={(0,\y)},color=black] (2pt,0pt) -- (-2pt,0pt) node[left] {\footnotesize $\y$};
            \draw[color=black] (0pt,-10pt) node[right] {};
            \clip(-0.9,-2.4) rectangle (1.9,2.4);
            \draw[<->,line width=1.2pt,color=red,smooth,samples=100,domain=-0.9:1.9] plot(\x,{3.0*(\x)^(3.0)-5.0*(\x)^(2.0)+(\x)+1.0});
            \draw[line width=1.2pt,color=blue,smooth,samples=100,domain=-0.9:1.9] plot(\x,{9.0*(\x)^(2.0)-10.0*(\x)+1.0});
            \draw [dash pattern=on 1pt off 1pt] (0.11111111164602015,-4.279272269869239E-9)-- (0.11111111164602015,1.0534979423868314);
            \draw (-0.32,0.05) node[anchor=north west] {$\frac{1}{3}$};
            \draw (0.06,0.07) node[anchor=north west] {$\frac{1}{9}$};
            \draw (0.52,0.03) node[anchor=north west] {$\frac{5}{9}$};
            \draw [dash pattern=on 1pt off 1pt] (0.54,0.54)-- (0.54,0.);
            \draw (0.24,0.70) node[anchor=north west] {\tiny{point d'inflection}};
            \draw (0.11,1.25) node[anchor=north west] {\tiny{maximum (point tournant)}};
            \draw (0.8,1.47) node[anchor=north west] {$f'(x)$};
            \draw (1.15,1.21) node[anchor=north west] {$f(x)$};
            \draw (0.98,-0.20) node[anchor=north west] {\tiny{minimum (point tournant)}};
		\end{tikzpicture}
		\caption{Point tournant et point d'inflexion}
	\end{figure}
	Donc chose très importante qu'il faudra toujours avoir en tête (!!!) quand vous poserez que la dérivée d'une fonction est nulle, c'est que nous pouvons avoir une dérivée qui s'annule en un point sans que ce soit un extremum (nous appelons cela un point d'inflexion). Pour contrôler que ce soit bien un maximum, nous pouvons calculer la dérivée seconde\label{second derivative} afin d'éliminer le cas d'un point d'inflexion. Sinon il faut recourir à un tableau de variation pour s'assurer que nous avons affaire à un maximum ou à un minimum comme par exemple avec la fonction $x^3-3x^2+2$:

	\begin{minipage}{\linewidth}\centering
    \begin{variations}
     x      & \mI &    & 0 &    & 2 &    & \pI  \\
     \filet
     f'     & \ga +    & 0    &  -  &  0   & \dr+      \\
     \filet
     \m{f}  & ~  & \c  & \h{~} & \d & ~    &  \c       \\
     \end{variations}
	\end{minipage} 	
	
	Dont le tracé correspondant est :
	\begin{figure}[H]
		\centering
		\includegraphics[scale=1]{img/algebra/variation_plot_example_1.jpg}
		\caption[]{Plot of  function $x^3-3x^2+2$}
	\end{figure}
	Un autre exemple avec $f(x)=x^4-4x^3+11$:
	\begin{figure}[H]
		\centering
		\includegraphics[scale=0.8]{img/algebra/variation_plot_example_2.jpg}
		\caption[]{Tracé de la fonction $x^4-4x^3+11$}
	\end{figure}
	Avec une table de variation plus détailée:
	
	\begin{center}
		\begin{tikzpicture}[t style/.style={solid}]
		\tkzTabInit[espcl=2]{$x$/.5,$f'(x)$/.5,$f''(x)$/.5,$f(x)$/3} {$-\infty$,$0$,$2$,$3$,$+\infty$}
		\tkzTabLine{,-,0,-,t,-,0,+, }
		\tkzTabLine{,+,0,-,0,+,t,+, }
		
		\node [below] (n1) at (N13){$+\infty$};
		\node [below=1cm](n2) at (N23){$11$};
		\node [below=2cm] (n3) at ([yshift=1em]N33){$-5$};
		\node [above] (n4) at ([yshift=1em]N44){$-16$};
		\node [below] (n5) at (N53){$+\infty$};
		
		\node[below=1ex]at(n2){$ \mathrm{\Sigma.K.} $};
		\node[below=1ex]at([xshift=.5ex]n3){$ \mathrm{\Sigma.K.} $};
		\node[below=1ex]at([xshift=1ex]n4){$ \mathrm{T.E.} $};
		
		\draw[>->] (n1) to [out=-90,in=180] (n2);
		\draw[>->] (n2) to [out=0,in=90] (n3.west);
		\draw[>->] (n3.east) to  [out=-90,in=180] (n4);
		\draw[>->] (n4) to [out=0,in=-90] (n5);
		
		\end{tikzpicture}
	\end{center}

	Voici un exemple d'une fonction avec ses dérivées première et seconde avec Maple 4.00 :

	\texttt{>plot([tanh(x),diff(tanh(x),x),diff(tanh(x),x\$2)],x=-5...5,\\color=[red,green,blue]);}

	\begin{figure}[H]
		\centering
		\includegraphics[scale=0.75]{img/algebra/derivatives.eps}
		\caption{Tracé de la fonction tangente hyperbolique, sa dérivée première et seconde}
	\end{figure}

	\begin{tcolorbox}[title=Remarque,colframe=black,arc=10pt]
	Deux bonnes fonctions pour se souvenir facilement de la propriété de la dérivée seconde sont $f(x)=x^2$ et $f(x)=-x^2$. Comme vous le savez, la première a un minimum global et la second un maximum global et si vous calculez la dérivée seconde pour la première, vous obtenez une constante positive et une constante négative pour la seconde.
	\end{tcolorbox}
	
	Un point d'inflexion très important dans les affaires et la science est le point d'inflexion de la distribution Normale (\SeeChapter{voir section Statistiques page \pageref{gauss distribution}}).
	
	Étant donné la fonction de la distribution de Gauss univariée:
	
	Les points d'inflexion se produisent lorsque la dérivée seconde $f''(x)=0$. Donc, nous devons le calculer. D'abord:
	
	Dès lors:
	
	Nous la posons comme étant égale à zéro:
	
	Nous avons donc deux situations possibles :
	
	Par conséquent, les deux points d'inflexion se produisent à :
	
	ou un écart-type au-dessus et au-dessous de la moyenne!

	Maintenant, suite à un problème de compréhension de la part d'un lecteur dans un des chapitres de ce livre, précisons une technique utilisée fréquemment par les physiciens. Considérons une dérivée d'ordre $2$ telle que:
	
	Si nous regardons le  $\dfrac{\mathrm{d}}{\mathrm{d}x}$ comme un opérateur différentiel (ce qu'il est!) nous pouvons bien évidemment écrire:
	
	Finalement nous avons:
	
	et donc il vient après simplification par $f(x)$:
	
	sinon quoi nous ne pouvons pas avoir cette égalité si l'opérateur agit explicitement sur une fonction dans une relation mathématique ou physique quelconque.

	Cela peut paraître évident pour certains mais parfois moins pour d'autres... et il était visiblement utile de préciser cela car c'est souvent utilisé dans les sections de Relativité Restreinte, Relativité Générale, Physique Quantique Corpusculaire et Physique Quantique Ondulatoire.

	Indiquons et démontrons maintenant deux propriétés intuitivement évidentes des dérivées et qui nous seront plusieurs fois indispensables pour certaines démonstrations sur ce site (comme par exemple dans le chapitre de méthodes numériques ou ici même...).

	\begin{theorem}
	Considérons d'abord deux nombres réels $a<b$ et $f$ une fonction à valeurs réelles continue sur $[a, b]$ et dérivable sur l'intervalle ouvert  $]a, b[$ telle que $f(a)=f(b)$. Alors nous voulons démontrer qu'il existe bien évidemment au moins un élément $c \in ]a, b[$ tel que $f'(c)=0$ (c'est typiquement le cas des fonctions polynômiales!).
	
	Cette propriété est appelée "\NewTerm{théorème de Rolle}\index{théorème de Rolle}\label{rolle theorem}" et donc explicitement elle montre qu'il existe au moins un élément où la dérivée de $f$ est nulle si en la parcourant nous revenons à la même valeur des images pour deux valeurs distinctes des abscisses, c'est-à-dire qu'il existe au moins un point où la tangente est horizontale.
	\end{theorem}

	\begin{dem}
	Si $f$ est constante, le résultat est immédiat... Dans le cas contraire, comme $f$ est continue sur l'intervalle fermé borné $[a, b]$ elle admet au moins un minimum global ou maximum global compte tenu que nous nous basons sur l'hypothèse que et que $f$ n'est pas constante. L'extrema est atteint en un point $c$ appartenant à l'intervalle ouvert $] a, b [$ (le fait de prendre l'intervalle ouvert permet dans certains cas d'éviter d'avoir un extrema à nouveau en $a$ ou en $b$).
	
	Supposons comme premier cas que $f(c)$ est maximum global. La dérivée de la fonction $f$ entre $c$ et un deuxième point $a$ alors un signe connu.
	\begin{itemize}
		\item Pour $h$ strictement positif et tel que $c + h$ appartienne à l'intervalle $[a, b]$:
		
		En considérant la limite quand $h$ tend vers $0$, la valeur de la dérivée $f'(c)$ est négative.
	
		 \item Pour $h$ strictement négatif et tel que $c + h$ appartienne à l'intervalle $[a, b]$:
		
		En considérant la limite quand $h$ tend vers $0$, le nombre dérivé $f'(c)$ est positif.
	\end{itemize}
	Au bout du compte, la dérivée de $f$ est nulle au point $c$.
	
	La démonstration est analogue si $f(c)$ est un minimum global, avec les signes des dérivées qui sont les opposés.
	\begin{flushright}
		$\blacksquare$  Q.E.D.
	\end{flushright}
	\end{dem}
	Maintenant, considérons deux réels $a,b$ et $f(x)$ une fonction continue sur $[a, b]$ (c'est-à-dire $\mathcal{C}([a,b])$) et dérivable sur $] a, b [$.

	\begin{theorem}
	Alors, nous nous proposons de montrer qu'il existe au moins un réel $c \in ]a,b[ $ tel que:
	
	Ce qui peut aussi s'écrire sous la forme suivante:
	
	avec $s\in ]0,1[$.
	
	Puisque le terme de gauche représente une augmentation finie du terme de droite, alors ce résultat est nommé "\NewTerm{théorème de la valeur moyenne}\index{th\'eor\'eme de la valeur moyenne}" ou mieux "\NewTerm{théorème des incréments finis}\index{th\'eor\'eme des incr\'ements finis}".
	
	Géométriquement cela signifie qu'en au moins un point $c$ du graphe de la fonction $f(x)$, il existe une tangente de coefficient directeur:
	
	Graphiquement cela donne:
	\begin{figure}[H]
	\centering
	\includegraphics{img/algebra/mean_value_theorem.eps}
	\caption{Représentation graphique du théorème de Rolle}
	\end{figure}
	
	\end{theorem}
	\begin{dem}
		Nous avons premièrement:
		
		parce que la pente de $h(x)$ est trivialement:
		
		et comme nous devons avoir $f(a)$ lorsque $x=a$ il s'ensuite la relation donnée précédemment.
		
		Ensuite, pour montrer qu'une telle valeur $c$ existe, l'idée est de ramener les deux points $a$ et $b$ dans la même ordonnée ce qui nous ramène au théorème de Rolle et pour cela, nous définissons une fonction $g (x)$ par :
		
		qui est telle qu'en effet $g(a)=g(b)$ ... et est dans ce cas égal à $0$ (mais cette valeur n'est pas pertinente).
	
		Par conséquent, le théorème de Rolle discuté ci-dessus indique qu'il existe un point entre $a$ et $b$ où la dérivée de $g(x)$ est nulle telle que $g'(c)=0$. Et en voyant ça :
		
		Nous avons alors
		
		Dès lors après simplification:
		
		\begin{flushright}
			$\blacksquare$  Q.E.D.
		\end{flushright}
	\end{dem}
	Puisque le terme de gauche représente un accroissement fini du terme de droite, alors ce résultat est appelée "\NewTerm{théorème des accroissements finis}\index{th\'eor\'eme des accroissements finis}"" (TAF). 
	
	A l'aide de ce petit théorème et des outils mathématiques introduits précédemment, nous pouvons construire un petit théorème fort utile et puissant en physique.
	
	\textbf{Définition (\#\mydef):} Nous appelons "\NewTerm{règle de L'Hôpital}\index{r\'egle de L'H\-opital}" (également appelée "règle de l'Hospital"\label{Hospital rule} ou "règle de Bernoulli") le procédé qui utilise la dérivée dans le but de déterminer les limites difficiles à calculer de la plupart des quotients et qui apparaissent souvent en physique.

	\begin{dem}
	Considérons deux fonctions $f(x)$ et $g(x)$ et telles que $f(a)=g(a)=0$ alors nous pouvons écrire:
	
	Alors selon la définition de la dérivée:
	
	\begin{flushright}
		$\blacksquare$  Q.E.D.
	\end{flushright}		
	\end{dem}
	Nous pouvons généraliser ce résultat précédent initialement basé sur la contrainte un peu trop forte:
	
	\begin{dem}
	Rappelons donc que selon le théorème des accroissements finis, si $f(x)$ est dérivable sur un intervalle $]a, b[$ et continue sur $[a, b]$ alors il existe un réel $c$ dans l'intervalle $[a, b]$ tel que:\
	
	Si le théorème se vérifie pour deux fonctions satisfaisant aux mêmes contraintes alors nous avons deux fonctions telles que:
	
	Si $g'(c)$ est non nul nous avons alors tout à fait le droit d'écrire le rapport (certains appellent cela le "\NewTerm{théorème des accroissements fini généralisé}\index{th\'eor\'eme des accroissements fini g\'en\'eralis\'e}"...):
	
	ce qui sans perdre en validité tant que $c$ est dans l'étau $[a, x]$ peut s'écrire:
	
	Ainsi, lorsque $x \rightarrow a$ ce qui implique que l'étau $[a, x]$ se referme et donc $c \rightarrow a$, nous avons:
	
	Ainsi, nous venons de prouver quand dans la démonstration précédente de la règle de l'Hôpital la relation:
	
	que nous avions est vraie en toute généralité et qu'il n'est pas nécessaire que $f(a)=g(a)=0$ soit vrai pour que le résultat soit juste!
	\begin{flushright}
		$\blacksquare$  Q.E.D.
	\end{flushright}
	\end{dem}
	\begin{tcolorbox}[title=Remarque,colframe=black,arc=10pt]
	Le résultat peut être généralisé, toujours sous la même condition, avec :
	
	\end{tcolorbox}
	
	\begin{tcolorbox}[colframe=black,colback=white,sharp corners]
	\textbf{{\Large \ding{45}}Exemples:}\\\\
	E1. Nous voulons calculer:
	
	Il y a deux façons de résoudre ça ! On peut d'abord factoriser :
	
	Soit en utilisant la règle de l'Hôpital (nous voyons que l'hypothèse $f(2)=g(2)=0$ est satisfaite !) :
	
	Mais nous voyons que nous ne pouvons pas appliquer la règle de l'Hôpital deux fois car les hypothèses ne seraient pas remplies.\\
	
	E2. Nous voulons calculer:
	
	Nous ne pouvons évidemment pas appliquer la règle de l'Hôpital puisque $f(+\infty)\neq g(+\infty) \neq 0$. La technique de factorisation échouera également. Nous devons utiliser une astuce de puissance.
	\end{tcolorbox}
	
	\begin{tcolorbox}[colframe=black,colback=white,sharp corners]
	On divise par la puissance du plus grand dénominateur : 
	
	\end{tcolorbox}
	Nous mettons également en garde contre la précédente "règle de division par la puissance la plus élevée" qui est applicable dans le scénario spécifique où nous prenons une limite à l'infini ou moins l'infini ET l'expression limite est de la forme $P(x)/Q(x)$ où $P$ et $Q$ sont des combinaisons linéaires de puissances de $x$ et la plus grande puissance de $x$ apparaissant dans l'expression entière est non négative.
	
	\subsubsection{Différentielles}
	
	Nous avons indiqué précédemment ce qu'était un différentiel $\mathrm{d}$. Mais il existe en fait plusieurs types de sortes de différentielles d'une fonction (remarquez que nous distinguons le genre masculin et féminin du terme):
	\begin{enumerate}
		\item Les différentielles
		\item Les différentielles partielles
		\item Les différentielles totales exactes 
		\item Les différentielles totales inexactes 
	\end{enumerate}
	Rappelons que nous appelons "\NewTerm{différentiel $\mathrm{d}f$}\label{differential}" d'une fonction univariée à une variable la relation donnée par (voir texte précédent):
	
	Cependant, pour exprimer l'effet d'un changement de toutes les variables d'une fonction $f$ de plusieurs variables (ie multivariée), nous devons utiliser une autre type de différentielle que nous appelons la "\NewTerm{différentielle totale}\index{diff\'erentielle totale}" (dérivée en deux sous-familles: différentielle totale exacte et différentielle totale inexacte que nous détaillerons plus loin).
	
	Soit par exemple, une fonction $f(x, y)$ des deux variables $x$ et $y$. L'accroissement $\mathrm{d}f$ de la fonction $f$, pour un accroissement fini de $x$ à $x+\Delta x$ et $y$ à $y+\Delta y$ est évidemment donné par:
	
	Nous pouvons aussi écrire:
	
	Ou encore:
	
	Pour des accroissements infiniment petits de $x$ et $y$:
	
	Intéressons-nous dès lors aux deux termes au passage à la limite:
	
	Le premier terme de gauche, nous le voyons, ne donne finalement que la variation en $x$ de la fonction $f(x, y)$ en ayant $y$ constant sur la variation. Nous notons cela dès lors (si la connaissance des variables constantes est triviale, nous ne les indiquons plus):
	
	et de même:
	
	\begin{tcolorbox}[title=Remarque,colframe=black,arc=10pt]
	Quand une variable est fixée pour étudier la variation de l'autre, certains auteurs ou professeurs des anciennes générations aiment à dire: "toutes choses égales par ailleurs $f$ varie en fonction de ... de façon ....". Bref, c'est un usage que l'on retrouve dans d'autres domaines (comme les régressions linéaires à plusieurs variables explicatives) mais qui se perd...
	\end{tcolorbox}
	Les deux expressions:
	
	sont ce que nous appelons des "\NewTerm{différentielles partielles}\index{diff\'erentielles partielles}" ou plus simplement "\NewTerm{dérivée partielle}\index{d\'eriv\'ee partielle}\label{partial derivative}" (dont le cas d'application pratique le plus simple et probablement le plus intéressant et pédagogiquement pertinent disponible à ce jour sur l'entier du site est le modèle d'approvisionnement de Wilson avec rupture présenté dans la section de Techniques De Gestion).
	
	Nous avons alors:
	
	qui est la "\NewTerm{différentielle de $f$}". Les thermodynamiciens parlent souvent eux de la "\NewTerm{différentielle totale exacte de $f$}\label{total exact differential}" ou plus simplement "\NewTerm{différentielle exacte de $f$}" ou encore de "\NewTerm{dérivée totale}" mais aussi de "\NewTerm{dérivée extérieure}\index{d\'eriv\'ee ext\'erieure}".
	
	La relation précédente est un cas particulier de ce que les mathématiciens appellent en toute généralité une "\NewTerm{forme différentielle}\index{forme diff\'erentielle}":
	
	nous y reviendrons un peu plus loin... 
	
	Il est d'usage de noter:
	
	donc sous forme d'un champ vectoriel.
	
	Il est important de se rappeler de la forme de la différentielle totale car nous la retrouverons partout dans des opérateurs particuliers en physique, dans la mécanique des fluides, dans la thermodynamique, etc.
	
	Géométriquement, les dérivées partielles peuvent être interprétées comme suit: la fonction $f(x, y)$ définit une surface dans $\mathbb{R}$, dont l'intersection avec la plan $y=y_0$ est une courbe $f(x,y_0)$.
	
	La dérivée partielle $\partial_x f$  est alors la pente de cette courbe en tout point $x$. Nous avons alors naturellement la fonction suivante pour la pente au point $(x_0,y_0)$:
	
	De la même manière, la tangente à la courbe $f(x_0,y)$ sera donnée par:
	
	Le plan localement tangent au point $(x_0,y_0)$ déterminé pas ses deux tangentes est alors donné par:
	
	En réorganisant les termes tel que:
	
	Nous reconnaissons:
	
	Ainsi, par exemple, la surface représentée par la fonction:
	
	est représentée ci-dessous avec les deux tangentes passant par le point:
	
	et dont les équations respectives sont:
	
	et:
	
	\begin{figure}[H]
		\centering
		\includegraphics[scale=0.75]{img/algebra/total_derivative_two_tangents.jpg}
		\caption[]{Les deux tangentes de la fonction au point d'intérêt}
	\end{figure}
	Nous avons le "\NewTerm{tangent plane}\index{tangent plane}" en ce point qui est alors donné par:
		
	\begin{figure}[H]
		\centering
		\includegraphics[scale=0.75]{img/algebra/total_derivative_tangent_plane.jpg}
		\caption[]{Les deux tangentes de la fonction au point d'intérêt avec la plan tangent}
	\end{figure}
	
	\begin{tcolorbox}[title=Remarque,colframe=black,arc=10pt]
	De la même manière, pour une fonction de plus de deux variables, par exemple $f(x,y,z)$, la différentielle totale $\mathrm{d}f$ est:
	
	Dans l'équation ci-dessus, la différentielle $\mathrm{d}f$ a été calculée à partir de l'expression de la fonction $f$. Puisqu'il existe une fonction $f$ qui vérifie l'expression de $\mathrm{d}f$, la différentielle $\mathrm{d}f$ est dite alors aussi "différentielle totale exacte".
	\end{tcolorbox}	
	Profitons pour faire une indication importante sur l'utilisation des dérivées partielles par les physiciens (et donc dans les nombreux chapitres y relatifs du site). Nous avons vu plus haut que si $f$ dépend de deux variables $x, y$ nous avons:
	
	et s'il ne dépend que d'une variable nous avons alors:
	
	Et alors dans le cas univarié:
	
	raison pour laquelle les physiciens mélangent allègrement les deux notations...
	
	Maintenant, il faut cependant savoir qu'il existe également des différentielles totales exactes qu'aucune fonction ne vérifie. Dans ce cas, nous parlons de "\NewTerm{différentielle totale inexacte}\index{diff\'erentielle totale inexacte}\label{total inexact derivative}" et pour déterminer si une différentielle totale est exacte ou inexacte, nous utilisons les propriétés des dérivées partielles (cas très important en thermodynamique!!!).
	
	Soit la fameuse forme différentielle générale (cela fait appel à de la géométrie différentielle):
	
	où $M (x, y)$ et $N (x, y)$ sont des fonctions des variables $x$ et $y$. Si $\mathrm{d}z$ est une différentielle totale exacte, alors:
	
	Il faut donc que in extenso:
	
	ou encore, en effectuant une seconde dérivation, que: 
	
	pour que la forme différentielle, soit une différentielle totale exacte.
	
	Avant de continuer, nous avons besoin d'un résultat donné par le "\NewTerm{théorème de Schwarz}\index{th\'eor\'eme de Schwarz}\label{Schwarz theorem}" (mais qui a été démontré à la fin du 17ème siècle par un des frères Bernoulli) qui s'énonce de la manière suivante:
	
	\begin{theorem}
	Soit une fonction $f$, si:
	
	sont continues alors nous avons (il faut vraiment vérifier que ce soit le cas!) un résultat très important dans la pratique:
	
	pour tout $(x_0,y_0)\in U$ où $U$ est le domaine de définition où $f$ est continue (et donc dérivable).
	\end{theorem}
	\begin{dem}
	Nous considérons l'expression:
	
	Posons:
	
	Nous avons alors:
	
	Par le théorème des accroissements finis:
	
	avec $s,t \in ]0,1[$. En reprenant les définitions de $g$ et $w$ nous obtenons:
	
	en appliquant à nouveau le théorème des accroissements finis aux deux membres entre parenthèses nous trouvons:
	
	avec $\tilde{s},\tilde{t} \in ]0,1[$. Pour finir nous voyons que nous avons:
	
	et par continuité lorsque $k,h\rightarrow 0$, nous avons:
	
	Plus simplement écrit:
	
	Donc si $f$ s'exprime sous forme différentielle totale exacte alors les différentielles croisées sont égales (la réciproque n'est pas forcément vraie).
	\begin{flushright}
		$\blacksquare$  Q.E.D.
	\end{flushright}
	\end{dem}
	Par récurrence sur le nombre de variables nous pouvons démontrer le cas général (c'est long mais c'est possible, nous le ferons si besoin il y a...). 
	
	Donc finalement pour en revenir à notre problème initial, nous avons donc:
	
	Ce qui nous donne finalement la "\NewTerm{condition de Schwarz}\index{condition de Schwarz}":
	
	C'est donc la condition que doit satisfaire une forme différentielle pour être une différentielle totale exacte et la condition qu'elle ne doit pas satisfaire pour être une différentielle totale inexacte!!! C'est une propriété très important pour l'étude de la Thermodynamique!
	
	Afin de ne pas confondre les deux types de différentielles, nous utilisons le symbole $\delta$ pour représenter une différentielle totale inexacte:
	
	et $\mathrm{d}$ pour une différentielle totale exacte:
	
	La distinction est extrêmement importante car seules les différentielles totales exactes qui satisfont donc:
	
	ont une intégrale qui ne dépend \underline{que} des bornes d'intégration (puisque toutes les variables changent en même temps). Dès lors les différentielles totales inexactes dépendent \underline{que} des bornes d'intégration, ce qui signifie que:
	
	et donc sur un chemin fermé on peut avoir :
	
	Alors que pour les différentielles totales exactes:
	
	soit (voir la démonstration détaillée plus loin lorsque nous traiterons des intégrales curvilignes):
	
	Autrement dit, la variation d'une fonction dont la différentielle est totale exacte, ne dépend pas du chemin suivi, mais uniquement des états initiaux et finaux car elle s'exprime comme le gradient d'une fonction (voir la démonstration par l'exemple dans la section d'Électrostatique quand nous vérifions que la différence de potentiel est indépendante du chemin). Nous appelons une telle fonction qui satisfait à une différentielle totale exacte en physique, une \NewTerm{fonction d'état}\index{fonction d'\'etat}" et en mathématique une "\NewTerm{fonction holomorphe}\index{fonction holomorphe}" (\SeeChapter{voir la section d'Analyse Complexe page \pageref{holomorphic functions}}), c'est-à-dire une fonction dont la valeur ne dépend que de l'état présent et futur, et non de son histoire.
	
	Cette distinction est très importante et particulièrement en thermodynamique où il convient de déterminer si une quantité physique est une différentielle totale exacte (une "fonction d'état" donc) ou non afin de savoir comment évoluent les systèmes.
	
	\begin{tcolorbox}[colframe=black,colback=white,sharp corners]
	\textbf{{\Large \ding{45}}Exemple:}\\\\
	Un exemple important de forme différentielle en thermodynamique, est le travail élémentaire $\delta W$ d'une force exercée sur un corps en mouvement dans le plan $\text{O}xy$, nous avons:
	
	Les forces $F_x,F_y$ et ne dérivent pas nécessairement d'un même potentiel $U(x, y)$ tel que:
	
	dès lors  $\delta W$ est donc une différentielle totale inexacte!
	\end{tcolorbox}
	
	Enfin quelque chose peut-être important à garder à l'esprit ?! :
	\begin{itemize}
		\item Supposons $y=f(x,w)$, tandis qu'à leur tour $x=g(t)$ et $w=h(t)$. Comment $y$ change-t-il lorsque $t$ change ?
		
		Notez que si nous supposons $y=f(x,w(x))$ (une dépendance directe!) mais avec $x$ n'ayant aucune dépendance implicite, la relation ci-dessus devient\label{tota differential with direct dependency}:
		
		
		\item Supposons $y=f(x,w)$, tandis qu'à leur tour $x=g(t,s)$ et $w=h(t,s)$. Comment $y$ change-t-il lorsque $t$ change ?
		
	\end{itemize}
	Notez que le premier point est la "\NewTerm{dérivée totale}\index{d\'eriv\'ee totale}" alors que le deuxième point est la "\NewTerm{dérivée partielle totale}\index{d\'eriv\'ee partielle totale}".
	 
	 \subsubsection{Dérivée usuelles}\label{usual derivatives}
	 
	 Nous allons démontrer ici les dérivées les plus fréquentes (une petite trentaine) que nous puissions rencontrer en physique théorique et mathématique ainsi que certaines de leurs propriétés (en fait, nous allons toutes les appliquer dans les sections relatives à la Mécanique, l'Ingénierie, l'Atomistique, les Mathématiques Sociales, etc.). La liste est pour l'instant non exhaustive mais les démonstrations étant généralisées, elles peuvent s'appliquer à un grand nombre d'autres cas similaires (que nous appliquerons/rencontrerons tout au long de ce site).
	 \begin{enumerate}
	 	\item Dérivée de $f(x)=x^n$:
	 	
	 	Partons d'abord d'un cas particulier, la dérivée de $x^3$:
	 	
		La dérivée de la fonction cubique est donc $3a^2$.
		
		On peut généraliser ce résultat pour tout entier positif ou négatif $n$ et on verra que la fonction $f$ définie sur $\mathbb{R}$ par $f(x)=x^n$ est dérivable et que sa dérivée $f'$ est donné par $f'(x)=nx^{n-1}$.
		
		Effectivement:
		 
		 Ainsi, nous avons (quelques exemples peuvent être utiles pour comprendre la portée de ce résultat):
		 
		 Nous voyons donc qu'en ayant déterminé la dérivée d'une fonction de la forme $x^n$, nous avons également déterminé la dérivée de toute fonction qui est mise sous cette forme tel que:
		 
		 Cependant, les fonctions:
		 
		 ne sont pas dérivables en $x=0$ puisque la fonction n'y est plus définie (division par zéro). De plus, en ce qui concerne la fonction comportant la racine (puissance non entière), la dérivée n'est pas définie dans $\mathbb{R}_{-}^{*}$.
		 
		 \begin{tcolorbox}[title=Remarque,colframe=black,arc=10pt]
		Il ressort immédiatement des résultats ci-dessus que si $f(x)=x^n$ pour $n$ étant un entier positif $f^{(k)}(x) = x^{n-k}\ n!/(n-k)!$ pour $k\leq n$.
		\end{tcolorbox}	
		 
		 \item Dérivée de $f(x)=c^{te}$:
		 
		 Le résultat précédent donne un résultat immédiat intéressant pour des fonctions constantes telles que :
		 
		 il n'est alors pas difficile de déterminer que la dérivée est simplement :
		 
		 La dérivée de toute fonction constante est donc nulle (il est important de se souvenir de ce résultat lorsque nous étudierons les propriétés des intégrales) !!!
		 
		 \item Dérivée de $f(x)=\cos(x)$:
		 
		 Soit donc $a$ un réel quelconque fixé, alors (attention! il est utile de connaître les relations trigonométriques remarquables que nous démontrons dans la section de Trigonométrie):
		 
		 Puisque en utilisant la règle de l'Hospital (ou en voyant que $\sin(x)$ peut être assimilié à une droite $f(x)=x$ à proximité de $x=0$):
		 
		 Donc pour résumer:
		 
		 
		 \item Dérivée de $f(x)=\sin(x)$:
		 
		 Soit donc $a$ un réel quelconque fixé, alors  (attention! il est utile de connaître les relations trigonométriques remarquables que nous démontrons dans la section de Trigonométrie):
		 
		 Puisque en utilisant la règle de l'Hospital (ou en voyant que $\sin(x)$ peut être assimilié à une droite $f(x)=x$ à proximité de $x=0$):
		 
		 Donc pour résumer:
		 
		 
		 \item Dérivée de $f(x)=\log_b(x)$:
		 
		 Nous commençons en écrivant:
		 
		 Dès lors:
		 
		 Ainsi:
		 
		 Multiplions et divisons par  $x$ l'expression figurant dans le membre droit de la dernière égalité:
		 
		 Désignons la quantité $\dfrac{\Delta x}{x}$ par $\alpha$. Il est évident que quand $\alpha \rightarrow 0$ quand $\Delta x$ pour un $x$ donné. Par conséquent:
		 
		 Or, nous retrouvons ici une autre provenance historique de la constante d'Euler (voir la section d'Analyse fonctionnelle page \pageref{Euler number} pour la preuve) où:
		 
		 Ainsi:
		 
		 Un cas particulier important est le cas où $b = e$. Nous avons alors:
		  
		 
		 \item Dérivée d'une somme de fonctions:
		 
		 Soient $u$ et $v$ deux fonctions. La fonction somme $s=u+v$ est dérivable sur tout intervalle où $u$ et $v$ sont dérivables, sa dérivée est la fonction $s'$ somme des fonctions dérivées $u'$ et $v'$ de $u$ et $v$.
		 
		 Soit $a$ un nombre réel et $u,v$ deux fonctions définies et différentiables sur $a$ :
		 
		 
		 Donc la dérivée d'une somme est la somme des dérivées.
		 
		 Ce résultat se généralise pour une somme d'un nombre quelconque fixé de fonctions.
		 
		 \item Dérivée d'un produit de fonctions (nommé souvent "\NewTerm{règle du produit}\index{r\'egle du produit}"):
		 
		 Soient $u$ et $v$ deux fonctions. La fonction produit $p=uv$ est dérivable sur tout intervalle où $u$ et $v$ sont dérivables, et $p'$ sera la notation pour la dérivée du produit. Voyons maintenant quelle est son expression.
		 
		 Soient $a$ un réel fixé et $u,v$ deux fonctions définies et dérivables en $a$:		 
		 
		 Nous rajoutons à cette dernière relation deux termes dont la somme est nulle tels que:
		 
		D'où le résultat fameux:
		
		Mais il existe une formulation plus générale que la dérivée première d'un produit:
		 \begin{theorem}
		 	Considérons pour cela toujours nos deux fonctions $u$ et $v$, $n$ fois dérivables sur un intervalle $I$. Alors le produit $uv$ est $n$ fois dérivable sur $I$ et:
		 	
		 	et ceci constitue la "\NewTerm{règle de différentiation de Leibniz pour le produit}\index{r\'egle de diff\'erentiation de Leibniz pour le produit}\label{Leibniz differentiation rule for products}" que nous avons utilisée dans la section de Calcul Algébrique pour l'étude des polynômes de Legendre (qui nous sont eux-mêmes indispensables pour l'étude de la Chimie Quantique).
		 	
		 	La démonstration de cette expression est très proche de celle faite pour le binôme de Newton (\SeeChapter{voir section Calcul Algébrique page \pageref{binomial theorem}}).
		 \end{theorem}
		 \begin{dem}
		 	Soit:
		 	
		 	D'autre part:
		 	
		 	La relation est ainsi bien initialisée.
		 	
		 	La démonstration se fait par récurrence. Ainsi, le but est de montrer que pour $\forall n \geq 0 \in \mathbb{N}$ que si:
		 	
		 	alors:
		 	
		 	Nous avons donc:
		 	
		 	Nous allons procéder à un changement de variable dans la première somme pour ne plus avoir le terme en $k + 1$. Nous posons pour cela $j=k+1$:
		 	
		 	Si nous revenons à la lettre $k$, nous avons donc:
		 	
		 	Nous avons donc:
		 	
		 	Nous voulons réunir les deux sommes. Pour cela, nous écartons les termes en trop dans chacune d'elles:
		 	
		 	Ce qui donne donc:
		 	
		 	D'après la formule de Pascal (\SeeChapter{voir section Probabilities page \pageref{pascal formula}}), nous avons:
		 	
		 	Donc:
		 	
		 	Mais nous avons en même temps:
		 	
		 	Dès lors:
		 	
		 	\begin{flushright}
				$\blacksquare$  Q.E.D.
			\end{flushright}
		 \end{dem}
		 
		 \item Dérivée d'un ratio de deux fonctions différerntiables (nomméel a  "\NewTerm{règle du quotient}\index{r\'egle du quotient}\label{quotient rule}"):
		 
		 Soit $f(x)=g(x)/h(x)$. L'application de la définition de la dérivée et des propriétés des limites donne la preuve suivante :
		 
		 Nous pouvons également faire la preuve en utilisant la différenciation implicite. En effet, soit $f(x)=g(x)/h(x)$, donc $g(x)=f(x)h(x)$. La règle du produit prouvée juste plus haut nous donne alors :
		
		En résolvant pour $f'(x)$ et en substituant $f(x)$ donne :
		
		 
		 \item Dérivée d'une fonction composite univariée:
		 
		 Considérons la fonction composée $f=g \circ u=g(u(x))$ de deux fonctions $g$ et $u$ différentiables, la première dans $u(a)$, la seconde dans $a$. On a donc :
		 
		 Posons maintenant $k=u(a+h)-u(a)$ alors nous avons :
		  
		 Continuons notre développement précédent :
		  
		 Ainsi, la dérivée d'une fonction composite est donnée par la dérivée de la fonction, multipliée par le "\NewTerm{dérivée intérieure}\index{d\'eriv\'ee int\'erieure}". De plus, ce type de dérivée est très importante car souvent utilisée en physique sous le nom de "\NewTerm{dérivation (univariée) en chaîne}\index{dérivation (univariée) en chaîne}" ou simplement "\NewTerm{règle de dérivation en chaîne}\index{r\'egle de d\'erivation en cha\-ine}".
		 
		 Voyons de quoi il s'agit. La relation précédente obtenue peut être réécrite d'une autre manière plus courante :
		 
			Ou typiquement lorsque nous avons plusieurs fonctions qui se multiplient :
		 
		 Nous parlons alors parfois de "\NewTerm{règle de puissance en chaîne}\index{r\'egle de puissance en cha\-ine}".
		 
		 \item Dérivée d'une fonction composée bivariée :
		 \begin{theorem}
		 La dérivée en $t$ d'une fonction composite $z=f(x(t),y(t))$ est donnée par :
		  
		 Nous supposons dans ce théorème et ses applications que $x=x(t)$ et $y=y(t)$ ont des dérivées premières en $t$ et que $z=f(x,y)$ a des dérivées continues du premier ordre dans un cercle ouvert centré sur $(x(t),y(t))$.
		 \end{theorem}
		 \begin{dem}
		 	Nous fixons $t$ et posons $(x,y)=(x(t),y(t))$. Nous considérons $\Delta t$ non nul et si petit que $(x(t+\Delta t),y(t+\Delta t))$ est dans le cercle où $f$ a des dérivées premières continues et posons $\Delta x=x (t+\Delta t)-x(t)$ et $\Delta y=y(t+\Delta t)-y(t)$. Alors par définition de la dérivée :
		 	
			Nous pouvons appliquer le "théorème de la valeur moyenne" qui dit que (voir la preuve plus loin lors de notre étude du Calcul Intégral):
						
			 à l'expression dans le premier ensemble de crochets à droite de la dernière égalité ci-dessus où $y$ est constant et à l'expression dans le deuxième ensemble de crochets où $x$ est constant. Nous en concluons qu'il existe un nombre $c_1$ entre $x$ et $x+\Delta x$ et un nombre $c_2$ entre $y$ et $y+\Delta y$ tels que :
			
			Nous ajoutons les deux relations ci-dessus et divisons par $\Delta t$ pour obtenir :
			
			Les fonctions $x=x(t)$ et $y=y(t)$ sont continues en $t$ car elles ont des dérivées en ce point. Par conséquent, comme $\Delta t\rightarrow 0$, les nombres $\Delta x$ et $\Delta y$ tendent tous les deux vers zéro et le cercle incluant les constantes $c_i$ s'effondre jusqu'au point $(x,y)$, Comme les dérivées partielles de $f$ sont continues, le terme $\partial_x f(c_1,y+\Delta y)$ tend vers $\partial_x f(x,y)$ et le terme $\partial_y f(x,c_2) $ tend vers $\partial_y f(x,y)$ comme $\Delta t\rightarrow 0$. De plus:
		
		quand $\Delta t\rightarrow 0$, ainsi la relation ci-dessus devient:
			
			 qui est parfois nommée "\NewTerm{règle en chaîne multivariée}\index{r\'egle en cha\-ine multivari\'ee}\label{multivariate chain rule}" (mais en réalité il s'agit seulement du cas bivarié...) et est une relation trèeeees importante pour l'étude de la physique!
			 
			 Cette dernière relation est parfois écrite de la manière suivante: 
			 
		 	\begin{flushright}
			$\blacksquare$  Q.E.D.
			\end{flushright}
		 \end{dem}
		 
		 \item Dérivée d'une fonction réciproque:
		 \begin{theorem}
		 	Si la fonction $f$ est continue, strictement monotone sur un intervalle $I$, dérivable sur $I$, alors la fonction réciproque est dérivable sur l'intervalle $f(I)$ et admet pour fonction dérivée:
		 	
		 \end{theorem}
		 \begin{dem}
		 	En effet nous pouvons écrire :
		 	
		 	C'est-à-dire (application identité) :
		 	
		 	Par application de la dérivation des fonctions composées:
		 	
		 	d'où:
		 	
		 	Pour une variable $x$, nous poserons pour la dérivée de la fonction réciproque:
		 	
		 	\begin{flushright}
			$\blacksquare$  Q.E.D.
			\end{flushright}
		 \end{dem}
		 \item Dérivée de la fonction $\arccos (x)$:
		 
		 	En utilisant le résultat précédent de la fonction réciproque et de la dérivée de $\cos (x)$  démontré plus haut, nous pouvons calculer la dérivée de la fonction $\arccos (x)$:
		 	
		 	
		 	\item Dérivée de $\arcsin (x)$:
		 
		 	En utilisant le résultat précédent de la fonction réciproque et la dérivée de $\sin (x)$, nous pouvons calculer la dérivée de la fonction $\arcsin (x)$:
		 	
		 	
		 	\item Dérivée d'un quotient de deux fonctions:
		 	
		 	Considérons que la fonction:
		 	
		 	est dérivable sur tout intervalle où les fonctions $u$ et $v$ sont dérivables et où la fonction $v$ est non nulle.
		 	
		 	La fonction $f$ peut être considérée comme le produit de deux fonctions: la fonction $u$ et la fonction $1/v$. Une produit de deux fonctions est dérivable si chacune d'elle est dérivable, il faut donc que la fonction $u$ soit dérivable et que la fonction $1/v$soit également dérivable ce qui est le cas quand $v$ est dérivable non nulle.
		 	
		 	
		 	\item Dérivée de la fonction $\tan(x)$:
		 	
		 	Par définition (\SeeChapter{voir section Trigonométrie page \pageref{definition trigonometric functions}}), pour $\forall x \neq k\dfrac{\pi}{2},k\in \mathbb{Z}$ nous avons:
		 	
		 	et en appliquant donc la dérivée d'un quotient vue précédemment, nous avons:
		 	
		 	ou:
		 	
		 	
		 	\item Dérivée de la fonction $\cot(x)$:
		 	
		 	Par définition (\SeeChapter{voir section Trigonométrie page \pageref{definition trigonometric functions}}), $\forall x \neq k\pi,k\in \mathbb{Z}$:
		 	
		 	et en appliquant donc la dérivée d'un quotient vue précédemment, nous avons:
		 	
		 	ou:
		 	
		 	
		 	\item Dérivée de la fonction $\arctan(x)$:
		 	
		 	Nous utilisons les propriétés dérivées des fonctions réciproques démontrées plus haut:
		 	
		 	
		 	\item Dérivée de la fonction $\text{arccot}(x)$:
		 	
		 	Selon la même méthode que précédemment:
		 	
		 	
		 	\item Dérivée de la fonction $e^x$:
		 	
		 	Nous démontrerons lors de notre étude des Méthodes Numériques (voir section du même nom \pageref{euler number computation}) que le "nombre d'Euler" peut être calculé selon la série:
		 	
		 	qui converge sur $\mathbb{R}$. En dérivant terme à terme cette série qui converge, il vient:
		 	
		 	Ainsi l'exponentielle est sa propre dérivée. Ainsi, nous pouvons nous permettre d'étudier les dérivées de quelques fonctions trigonométriques hyperboliques (\SeeChapter{voir section Trigonométrie page \pageref{hyperbolic trigonometry}}) et d'autres nombreux cas spécifiques (voir tous les autres chapitres de ce livre).
		 	
		 	\item Dérivée de la fonction $\sinh(x)$:
		 	
		 	Rappelons que (\SeeChapter{voir section Trigonométrie page \pageref{definition trigonometric functions}}):
		 	
		 	Donc trivialement:
		 	
		 	
		 	\item Dérivée de la fonction $\cosh(x)$:
		 	
		 	Rappelons que (\SeeChapter{voir section Trigonométrie page \pageref{definition trigonometric functions}}):
		 	
		 	Donc trivialement:
		 	
		 	
		 	\item Dérivée de la fonction $\tanh(x)$:
		 	
		 	Rappelons que (\SeeChapter{voir section Trigonométrie page \pageref{definition trigonometric functions}}):
		 	
		 	Donc en appliquant la dérivée d'un quotient nous obtenons:
		 	
		 	ou:
		 	
		 	
		 	\item Dérivée de la fonction $\coth(x)$:
		 	
		 	Rappelons que (\SeeChapter{voir section Trigonométrie page \pageref{definition trigonometric functions}}):
		 	
		 	Donc en appliquant la dérivée d'un quotient nous obtenons:
		 	
		 	
		 	\item Dérivée de la fonction $\text{arcsinh}(x)$:
		 	
		 	Nous appliquons les propriétés des dérivées des fonctions réciproques démontré plus haut:
		 	
		 	Mais (\SeeChapter{voir section Trigonométrie page \pageref{definition trigonometric functions}}):
		 	
		 	et dès lors:
		 	
		 	Etant donné que $\cosh(x)$ ne prend que des valeurs positives, nous avons:
		 	
		 	Donc finalement:
		 	
		 	
		 	\item Dérivée de la fonction $\text{arccosh}(x)$:
		 	
		 	Nous appliquons les propriétés des dérivées des fonctions réciproques:
		 	
		 	Or selon la même méthode que précédemment (\SeeChapter{voir section Trigonométrie page \pageref{definition trigonometric functions}}):
		 	
		 	d'où: 
		 	
		 	Etant donné que $\text{arccosh}(x)$ ne prend que des valeurs positives il en est de même pour $\sinh(x)$, nous avons alors:
		 	
		 	Donc finalement:
		 	
		 	
		 	\item Dérivée de la fonction $\text{arctanh}(x)$:
		 	
		 	En appliquant à nouveau les propriétés des dérivées des fonctions réciproques:
		 	
		 	
		 	\item Dérivée de la fonction  $\text{arccoth}(x)$:
		 	En appliquant à nouveau les propriétés des dérivées des fonctions réciproques:
		 	
		 	
		 	\item Dérivée de la fonction $a^x$ avec $a>0$:
		 	
		 	Donc (dérivée d'une fonction composée):
		 	
		 	
		 	\item Nous connaissons les règles de la chaîne pour la dérivée d'un produit de fonction. Mais qu'en est-il de la règle équivalent pour la puissance qui est assez fréquente en physique ?
	
			En effet, la dérivée de $f(x)^{g(x)}$ est considérée par beaucoup comme l'une des dérivées les plus polyvalentes !
			
			Premièrement, pour faire la preuve, il devrait être évident que :
			
			et:
			
		 	ne sont que des cas particuliers de la dérivée de $f(x)^{g(x)}$.
		 	
		 	Pour résoudre:
		 	
			nous devons utiliser un petit truc pour faciliter la différenciation. Et rien n'est aussi simple que :
			 
			Donc, puisque $x = e^{\ln(x)}$ nous l'utiliserons à notre avantage.
			
			Nous commençons par :
			
			Nous savons que:
			
			donc nous injectons cela pour obtenir ceci:
			
			Il ne nous reste plus qu'à différencier le membre de droite.
N'oublions pas que :
			
			Dès lors:
			
			Cela équivaut à écrire :
			
		   Il ne nous reste plus qu'à différencier le membre de droite. Voici la règle de multiplication simple :
			
			Et puis nous devons différencier le logarithme naturel :
			
		  	Donc, nous injectons cela pour obtenir ceci:
			
			Et puis nous distribuons $f(x)^{g(x)}$ pour obtenir :
			
			Alors si nous simplifions :
			
			Quelques exemples d'applications nous donnent donc :
			
			En effet, il suffit de poser $f(x) = x$ et $g(x) = n$ et nous obtenons:
		   	
			Ou une autre application :
			
			tout ce que nous avons à faire est de définir $f(x) = a$ et $g(x) = x$ et nous obtenons :
			
			
			
			\item Pour le plaisir (nous n'utilisons cette dérivée nulle part dans ce livre sur aucun cas pratique), calculons la dérivée de $x^x$.
			
		 \end{enumerate}
		 
	\pagebreak
	\subsubsection{Dérivation implicite}
	Jusqu'à présent, les fonctions dont nous nous sommes occupés sont des fonctions qui ont été définies explicitement. Une fonction est définie explicitement si {\it sortie est donnée directement en termes d'entrée}. Par exemple, dans la fonction :
	
	la valeur de $f(x)$ est donnée explicitement ou directement en termes d'entrée. Juste en connaissant l'entrée, nous pouvons immédiatement trouver la sortie. Un deuxième type de fonction qu'il nous est également utile de considérer est une "\NewTerm{fonction implicitement définie}\index{fonction implicitement d\'efinie}". Une fonction est définie implicitement si {\it sortie ne peut pas être trouvée directement à partir de l'entrée}. Par exemple (exemple simple et stupide) :
	
	est une fonction définie implicitement, car pour chaque valeur $x$ positive, il existe une valeur $f$ correspondante, mais nous ne pouvons pas la trouver directement à partir de la fonction. Nous aurions besoin de mettre les deux côtés au carré, puis nous aurions la fonction explicitement définie :
	 	
	Il nous est également possible d'avoir des fonctions implicitement définies que nous ne pouvons pas réécrire en tant que fonction explicitement définie !!!
	
	Par exemple, nous pourrions avoir la fonction :
	
	Pour une valeur $x$ donnée, il {\it peut} y avoir une valeur de sortie correspondante $f(x)$ qui en fait une déclaration vraie. De cette façon, une fonction $f(x)$ serait définie pour tous ces $x$ où il existe une solution. Par exemple, nous avons $f(0) = 0$, car la définition de $x=f(x)=0$ dans l'équation ci-dessus est une déclaration vraie. À l'heure actuelle, nous n'avons pas les outils appropriés pour résoudre une telle équation, mais le concept important ici est que nous pouvons avoir une fonction définie de cette manière!
	
	Lorsque nous parlons de fonctions, nous voulons dire que nous avons une règle qui nous fournit au plus une sortie pour une entrée donnée (il n'y a pas de sortie pour des entrées pour lesquelles la fonction n'est pas définie). Dans un sens plus général, nous pourrions vouloir examiner les règles qui nous fournissent plusieurs sorties pour une entrée donnée. Un tel exemple serait l'équation :
	
	qui est l'équation du cercle unité (\SeeChapter{voir section Géométrie Analytique page \pageref{equation of a circle}}). Il s'avère que cet objet est constitué de deux fonctions, à savoir :
	
	L'équation de ce cercle ne définit pas de fonction (parce que la sortie est multivaluée !!), mais elle définit un certain type de courbe dans le plan $x$-$y$. En général, nous devrions être capables de décrire une courbe arbitraire comme une combinaison d'un certain nombre de fonctions. Il est judicieux (et utile) de considérer la pente de certains points de la courbe, qui correspondrait simplement à la pente de la fonction spécifique qui définit cette partie de la courbe.
	
	Pour résoudre le problème de trouver la dérivée d'une fonction définie d'une manière telle que $\sin(f(x)) + f(x) = x$ ou par une courbe comme un cercle, nous utilisons à nouveau la règle de la chaîne. La manière dont nous allons l'employer est nommée "\NewTerm{différenciation implicite}\index{diff\'erenciation implicite}". Le processus fonctionne comme suit : nous différencions les deux côtés de l'équation par rapport à $x$ (ou la variable indépendante), puis nous résolvons la dérivée de la variable dépendante. Partout où nous trouvons la fonction $f$ (ou la variable dépendante), nous utiliserons la règle de la chaîne pour trouver la dérivée. Commençons par quelques exemples simples :
	
	\begin{tcolorbox}[colframe=black,colback=white,sharp corners]
	\textbf{{\Large \ding{45}}Exemples:}\\\\
	E1. Nous voulons trouver $\mathrm{d}y/\mathrm{d}x$ si $y^2 = x$.\\
	
	Une méthode pour résoudre ce problème serait de le réécrire en termes de fonction explicite de $y$, et de le différencier. Puisque nous avons $y = \pm \sqrt{x}$, nous avons en fait deux fonctions, et nous trouverions :
	
	Cela fonctionne suffisamment bien dans cette situation, mais qu'en est-il d'une fonction que nous ne pouvons pas réécrire explicitement ? Il faudrait une différenciation implicite. Appliquons une différenciation implicite à cette situation, comme exercice: 
	
	Si nous ne pouvions pas réécrire $y$ explicitement en termes de $x$, c'est aussi loin que nous pourrions aller, mais nous savons que $y = \pm \sqrt{x}$, et en injectant ce résultat, nous trouvons que :
	$$\frac{\mathrm{d}y_1}{\mathrm{d}x} = \frac{1}{2\sqrt{x}} \quad \text{et} \quad \frac{\mathrm{d}y_2}{\mathrm{d}x} = -\frac{1}{2\sqrt{x}}$$
	encore une fois. De cette façon, nous avons pu calculer les deux dérivées en une fois.\\
	
	E2. Nous voulons trouver $\mathrm{d}f/\mathrm{d}x$ pour $\sin(f) + f = x$.\\
	
	Ici, nous n'avons pas d'autre choix que d'appliquer une différenciation implicite :
	
	\end{tcolorbox}
	Un cas célèbre d'application en mathématiques pures de la différenciation implicite (nous verrons plus loin pour la physique) est celui à l'origine de cette technique : le "\NewTerm{folium de Descartes}\footnote{Le nom vient du mot latin folium qui signifie "feuille".}" défini par :
	
	La courbe a été proposée pour la première fois par René Descartes en... 1638.
	\begin{figure}[H]
		\centering
			\includegraphics[scale=0.9]{img/algebra/folium_of_descartes.jpg}
		\caption[Le folium de Descartes avec asymptote]{Le folium de Descartes (vert) avec asymptote (bleu) (source : Wikipédia)}
	\end{figure}
	Son titre de gloire réside dans un incident dans le développement du calcul. Descartes a défié Pierre de Fermat de trouver la ligne tangente à la courbe à un point arbitraire puisque Fermat avait récemment découvert une méthode pour trouver des lignes tangentes. Fermat a résolu le problème facilement, ce que Descartes n'a pas pu faire. Depuis l'invention de l'algèbre, la pente de la ligne tangente peut être trouvée facilement en utilisant la différentiation implicite en n'importe quel point comme nous allons le montrer !
	
	Considérons maintenant que nous voulons trouver la pente de la courbe au point $(2,4)$. Ensuite, nous utilisons la différentiation implicite :
	
	En évaluant la dérivée au point $(2,4)$ nous trouvons :
	
	Maintenant que nous avons la pente de la ligne tangente au point d'intérêt, nous utilisons la forme point-pente\index{forme point-pente} :
	
	Maintenant, tant que c'est la ligne normale à la courbe en ce point qui est d'intérêt, nous devons trouver la ligne perpendiculaire à la ligne tangente. Cette ligne passera par le même point, mais la pente sera l'inverse négatif de la pente de la ligne tangente. Il s'ensuit que :
	
	Bon maintenant concentrons-nous sur un exemple appliqué à la physique (mise à part le fameux cas de la recherche de la tangente d'une ellipse). Nous prouverons dans la section de Mécanique des Continuums l'équation de Van der Waals (voir page \pageref{Van der Waals state equation})::
	
	Si $T$ reste constant, considérons que l'on veut trouver le taux de variation du volume par rapport à la pression, soit $\mathrm{d}V/\mathrm{d}P$! Nous pouvons nous défier ici de calculer ce taux de variation pour transformer cette équation en une fonction explicitement définie... Nous avons donc qu'il faut utiliser la différentiation implicite. D'abord:
	
	Cela donne immédiatement :
	
	Maintenant, nous distribuons :
	
	Après simplification et réarrangement nous obtenons :
	
	Finalement:
	
	Le cas particulier des exemples ci-dessus avec l'équation de Van der Waals et le folium de Descartes est introduit dans certains manuels comme suit avec un cas à deux variables :
	
	Dès lors:
	
	Se réduit à:
	
	Finalement:
	
	C'est tout pour le moment. Nous nous arrêterons ici sur ce sujet car nous n'avons pas besoin de plus de techniques ou d'exemples pour notre étude de la physique et de l'ingénierie comme présenté dans ce livre.
	
		\pagebreak
		\subsubsection{Régularité}\label{smoothness}
		La régularité a à voir avec le nombre de dérivées d'une fonction qui existent et sont continues. Le terme "fonction régulière" est souvent utilisé techniquement pour désigner une fonction qui a des dérivées de tous les ordres partout dans son domaine.
		 
		 \textbf{Définition (\#\mydef):} Une "\NewTerm{classe de différentiabilité}\index{classe de diff\'erentiabilit\'e}" est une classification des fonctions selon les propriétés de leurs dérivées. Les classes de différentiabilité d'ordre supérieur correspondent à l'existence de plus de dérivées.
		 
		 Considérons un ensemble ouvert sur la ligne réelle $\mathbb{R}$ et une fonction $f$ définie sur cet ensemble avec des valeurs réelles ou complexes. Soit $k$ un entier non négatif. La fonction $f$ est dite de classe (de différentiabilité) $\mathcal{C}^k$ si les dérivées $f', f'', ..., f(k)$ existent et sont continues (la continuité est impliquée par la différentiabilité pour toutes les dérivées sauf pour $f(k)$). La fonction $f$ est dite de classe $\mathcal{C}^\infty$, ou "lisse", si elle a des dérivées de tous les ordres. La fonction $f$ est dite de classe $\mathcal{C}^\omega$, ou simplement "analytique", si $f$ est lisse et si elle est égale à son développement en série de Taylor autour d'un point quelconque de son domaine (\SeeChapter{voir section Séquences et séries page \pageref{taylor series}}). $\mathcal{C}^\omega$ est donc strictement contenu dans $\mathcal{C}^\infty$.

		\pagebreak
		 \subsection{Calcul Intégral}\label{integral calculus}
		 Nous discuterons ici des principes de base du calcul intégral dans $\mathbb{R}^n$. Les sujets les plus avancés viendront en fonction du temps dont disposent les rédacteurs de ce livre mais le lecteur peut déjà se référer à la section Analyse Complexe (voir page \pageref{complex analysis}) pour les techniques d'intégration basées sur le théorème des résidus qui est très puissant et utile surtout pour certaines intégrales en physique quantique.
		 
		 \begin{fquote}[Anonymous]La différentiation est une science, l'intégration est un art.
 		\end{fquote}
		 
		 \subsubsection{Intégrale Définie}\label{definite integral}
		 L'origine du calcul intégral semble venir d'Archimède qui était fasciné par le calcul des aires de différentes formes. Il a utilisé un processus connu sous le nom de \textit{Méthode d'exhaustion}, qui utilisait des formes de plus en plus petites, dont les aires pouvaient être calculées exactement, pour remplir une région irrégulière et obtenir ainsi des approximations de plus en plus proches de la superficie totale. Dans ce processus, une zone délimitée par des courbes est remplie de rectangles, de triangles et de formes avec des formules de zone exactes. Ces aires sont ensuite additionnées pour approximer l'aire de la région incurvée. Cette sous-section présente les intégrales définies.
		 		 
		 La première idée du concept d'intégrale est de calculer l'aire algébrique (surface positive si au-dessus de l'axe $x$ ou négative en dessous) entre une courbe et son support. Voir la figure ci-dessous avec une zone positive et les notations pour les développements qui suivront :
		 \begin{figure}[H]
			\centering
			\includegraphics{img/algebra/integral_all_positive.jpg}
			\caption[]{Aire $A$ à calculer sous une courbe continue bornée}
		\end{figure}
		Ou avec des aires algébriques différentes (la différence entre la zone bleue et la zone jaune est nommée "\NewTerm{zone nette signée}\index{zone nette sign\'ee}") :
		 \begin{figure}[H]
			\centering
			\includegraphics{img/algebra/integral_all_positive_and_negative.jpg}
			\caption[]{Aire $A$ à calculer dans une fonction continue bornée positive et négative}
		\end{figure}
		Une valeur approchée de l'aire sous une courbe peut être obtenue par un découpage en $n$ bandes rectangulaires verticales de même largeur. En particulier on peut réaliser un encadrement de cette aire à l'aide d'une somme majorante $A_M$ et d'une somme minorante $A_m$ pour un découpage donné:
		\begin{figure}[H]
			\centering
			\begin{subfigure}{0.4\textwidth}
				\includegraphics[width=\textwidth]{img/algebra/integral_minorant_sum.jpg}
				\caption{Somme minorante des aires $A_m$}
			\end{subfigure}
			\begin{subfigure}{0.4\textwidth}
				\includegraphics[width=\textwidth]{img/algebra/integral_majorant_sum.jpg}
				\caption{Somme majorante des aires $A_M$}
			\end{subfigure}				
		\end{figure}
		Supposons que le nombre $n$ de bandes tende vers l'infini. Comme les bandes sont de même largeur, la largeur de chaque bande tend vers $0$ (objectivement il n'est pas nécessaire que la largeur des sous-intervalles du découpage soit la même partout).
		
		Si les sommes $A_m$ et $A_M$ ont toutes deux une limite lorsque le nombre $n$ de bandes tend vers l'infini, alors l'aire $A$ sous la courbe est comprise entre ces deux limites. Nous notons cela:
		
		Évidemment si ces deux limites sont égales, leur valeur est celle de l'aire sous la courbe.
		
		D'où une première définition directe de l'intégrale définie ou dite  "\NewTerm{intégrale de Riemann}\index{intégrale de Riemann}\label{riemann integral}".
		
		\textbf{Définition (\#\mydef):}	Soit un intervalle $[a, b]$, divisé en $n$ parties égales, soit $f$ une fonction continue sur l'intervalle $[a, b]$, soit $A_m$, la somme algébrique minorante et soit $A_M$, la somme algébrique majorante. Nous appelons "\NewTerm{intégrale définie}\index{intégrale définie}" de $f$, depuis $a$ jusqu'à $b$, notée:
		
		le nombre $A$ tel que:
		
		pourvu que cette limite existe. Si cette limite existe, alors nous disons que $f$ est "intégrable" sur $[a, b]$ et l'intégrale définie existe. 
		
		Le symbole:
		
		n'est que que le symbole de la somme discrète $\sum$ mais appliquée aux cas d'éléments infiniments petits.
		
		Les nombres $a$ et $b$ (qui peuvent parfois être des fonctions!) de l'intégrale sont appelés les "\NewTerm{bornes d'intégration}\index{bornes d'int\'egration}": $a$ est la "\NewTerm{borne inférieure}", $b$ est la  "\NewTerm{borne supérieure}".
		
		\begin{center}
		\begin{tikzpicture}[scale=2.3]
		  \shade[top color=blue,bottom color=gray!50] 
		      (0,0) parabola (1.5,2.25) |- (0,0);
		  \draw (1.05cm,2pt) node[above] 
		      {$\displaystyle\int_0^{3/2} \!\!x^2\mathrm{d}x$};
		
		  \draw[style=help lines] (0,0) grid (3.9,3.9)
		       [step=0.25cm]      (1,2) grid +(1,1);
		
		  \draw[->] (-0.2,0) -- (4,0) node[right] {$x$};
		  \draw[->] (0,-0.2) -- (0,4) node[above] {$f(x)$};
		
		  \foreach \x/\xtext in {1/1, 1.5/1\frac{1}{2}, 2/2, 3/3}
		    \draw[shift={(\x,0)}] (0pt,2pt) -- (0pt,-2pt) node[below] {$\xtext$};
		
		  \foreach \y/\ytext in {1/1, 2/2, 2.25/2\frac{1}{4}, 3/3}
		    \draw[shift={(0,\y)}] (2pt,0pt) -- (-2pt,0pt) node[left] {$\ytext$};
		
		  \draw (-.5,.25) parabola bend (0,0) (2,4) node[below right] {$x^2$};
		\end{tikzpicture}
		\end{center}
		
		Intuitivement, il est évident que lorsque $a=b$, nous étendons la définition ainsi: 
		
		Enfin, signalons qu'il est tout à fait possible que l'intégrale soit négative ou même complexe puisqu'il s'agit d'une surface algébrique! C'est-à-dire qu'en général, le résultat peut être dans $\mathbb{C}$.
	\begin{tcolorbox}[title=Remarques,colframe=black,arc=10pt]
	\textbf{R1.} D'autres lettres que $x$  peuvent être employées dans la notation de l'intégrale définie. Ainsi si $f$ est intégrable sur $[a, b]$, alors $\int\limits_a^bf(x)\mathrm{d}x=\int\limits_a^bf(t)\mathrm{d}t=\int\limits_a^bf(s)\mathrm{d}s$, etc. C'est la raison pour laquelle la variable $x$ de la définition est dite "\NewTerm{variable muette}\index{variable muette}".\\

	\textbf{R2.} Comme nous le verrons plus loin, il est essentiel de ne pas confondre "\NewTerm{int\'egrale d\'efinie}\index{int\'egrale d\'efinie}" et "\NewTerm{intégrale indéfinie}\index{int\'egrale ind\'efinie}". Ainsi, une intégrale indéfinie, notée $\int\limits f(x)\mathrm{d}x$ est une fonction, ou, plus précisément, une famille de fonctions appelées aussi "\NewTerm{primitives de $f$}\index{primitie d'une fonction}" (voir plus bas) alors qu'une intégrale définie, notée $\int\limits_a^b f(x)\mathrm{d}x$ est une constante. 
	\end{tcolorbox}
	Voyons une deuxième approche de définition de l'intégrale, un peu plus rigoureuse que la précédente (suite à la demande de plusieurs lecteurs). Nous utiliserons par tradition cette fois-ci, le $S$ de la surface au lieu du $A$ de l'aire.
	
	Soit f une fonction bornée sur $[a, b]$. Nous considérons une subdivision de son support $[a, b]$ que nous notons:
	
	où les intervalles ne sont pas obligatoirement de tailles équivalentes.

	Nous notons pour $i=1,2,3,...,n$:	
	
	
	\textbf{Définitions (\#\mydef):}
	\begin{enumerate}
		\item[D1.] Nous appelons "\NewTerm{somme de Darboux inférieure}\index{somme de Darboux inf\'erieure}" associée à $f$ et $\sigma$ la surface:
		
		\begin{figure}[H]
			\centering
			\includegraphics{img/algebra/darboux_inferior_sum_concept.jpg}
			\caption{Principe du calcul de la somme de Darboux inférieure }
		\end{figure}
		
		\item[D2.] Nous appelons "\NewTerm{somme de Darboux supérieure}\index{somme de Darboux sup\'erieure}" associée à $f$ et $\sigma$ la surface:
		
		\begin{figure}[H]
			\centering
			\includegraphics{img/algebra/darboux_superior_sum_concept.jpg}
			\caption{Principe du calcul de la somme de Darboux supérieure }
		\end{figure}
	\end{enumerate}
	Une fonction $f$  est dite "\NewTerm{Riemann-Intégrable sur $[a, b]$}" si et seulement si les deux surfaces susmentionnées coïncident lorsque les intervalles deviennent infiniment petit.
	
	L'ensemble des fonctions Riemann-intégrables sur $[a, b]$ sont notées $\mathcal{R}_{[a,b]}$.
	
	Les sommes de Darboux ne sont pas très utiles pour le calcul effectif d'une intégrale, par exemple à l'aide d'un ordinateur, car il est en général assez difficile de trouver les $\inf$ et $\sup$ sur les sous-intervalles. On considère plutôt:
	
	La "\NewTerm{somme de Riemann}\index{somme de Riemann}" est défini par le fait que si nous désignons une "\NewTerm{partition}\index{partition}" (ou "\NewTerm{partition régulière}" s'ils sont tous ont la même largeur) de l'intervalle $[a,b]$ par:
	
	et que:
	
	où $\xi_i \in [x_{i-1},x_i]$, alors:
	
	Mais comme il faut bien choisir un $\xi_i$, souvent nous prenons soit celui à droite, soit celui à gauche, dès lors en prenant au hasard la "méthode des rectangles à gauche":
	
	Ce qui nous donnerait pour l'exemple ci-dessous:
	\begin{figure}[H]
		\centering
		\includegraphics{img/algebra/left_rectangle_integral.jpg}
		\caption{Principe du calcul de la méthodes des rectangles à gauche}
	\end{figure}
	Soit:
	
	Mais c'est facile pour une fonction en escalier... mais cela l'est moins pour une fonction continue pour laquelle nous n'obtiendrions qu'une valeur approchée de la surface réelle! L'idée est alors de prendre des intervalles de plus en plus petits:
	\begin{figure}[H]
		\centering
		\includegraphics{img/algebra/riemann_integral_left_rectangle.jpg}
		\caption{Principe du calcul de l'intégrale de Riemann avec la méthode des rectangles à gauche}
	\end{figure}
	Et dès lors, à la limite, nous obtenons la quantité voulue:
	
	Le fait de chercher cette limite s'appelle "calculer l'intégrale", et plus spécifiquement de la méthode choisie: "intégrale de Riemann".
	
	\begin{tcolorbox}[colframe=black,colback=white,sharp corners]
	\textbf{{\Large \ding{45}}Exemple:}\\\\
	Nous voulons utiliser la construction de l'intégrale définie pour évaluer :
	
	en utilisant une approximation de l'extrémité droite pour générer la somme de Riemann.\\
	
	Pour cela, nous avons d'abord établi la somme de Riemann. Sur la base des limites d'intégration, nous avons $a=0$ et $b=2$. Pour $i=0,1,2,\ldots,n$, soit $P=\{x_i\}$ une partition régulière de $[0,2]$. Puis:
	
	Ainsi, la valeur de la fonction à l'extrémité droite de l'intervalle est :
	
	Alors la somme de Riemann prend la forme :
	
	En utilisant la relation de sommation de Gauss de $\displaystyle\sum_{i=1}^n i^2$ (\SeeChapter{voir section Séquences et Séries page \pageref{sum of squares integers}}), nous avons :
	
	Maintenant, pour calculer l'intégrale définie, nous devons prendre la limite comme $n\rightarrow+\infty$. Nous avons alors:
	 
	\end{tcolorbox}
	
	\subsubsection{Intégrale Indéfinie}
	Nous avons vu précédemment lors de notre étude des dérivées, le problème suivant: étant donnée une fonction $F(x)$, trouver sa dérivée, c'est-à-dire la fonction algébrique:
	
	\textbf{Définition (\#\mydef):} Nous disons que la fonction $F (x)$ est une "\NewTerm{primitive}\index{primitive}" ou "\NewTerm{int\'egrale ind\'efinie}\index{int\'egrale ind\'efinie}" de la fonction $f (x)$ sur le segment $[a, b]$, si en tout point de ce segment nous avons l'égalité $F'(x)=f(x)$.
	
	Deux définitions alternatives plus explicites et moins techniques sont : 
	\begin{itemize}
		\item Une "intégrale indéfinie" est une FONCTION de $x$ (ou une autre variable), tandis qu'une "intégrale définie" est une VALEUR !
		
		\item La collection de toutes les primitives de $f(x)$ est appelée "l'intégrale indéfinie" de $f$ par rapport à $x$.
	\end{itemize}
	
	Une autre manière de voir le concept d'intégrale indéfinie est de passer par le "\NewTerm{théorème fondamental du calcul intégral (et différentiel)}\index{th\'eor\'eme fondamental du calcul int\'egral (et diff\'erentiel)}" appelé aussi parfois "\NewTerm{théorème fondamental de l'analyse}\index{th\'eor\'eme fondamental de l'analyse}\label{fundamental theorem of calculus}" dont les deux propriétés s'énoncent ainsi:
	\begin{enumerate}
		\item Si $A$ est la fonction définie par $A(X)=\displaystyle\int\limits_a^X f(t)\mathrm{d}t$ pour tout $X$ dans $[a, b]$, alors $A$ est la primitive de $f$ sur $[a, b]$ qui s'annule en $a$ (ou en d'autres termes: $f (t)$ est la dérivée de $A$).
		
		\item Si $F$ est une primitive de $f$ sur $[a, b]$, alors:
		
		qui peut aussi s'écrire évidemment comme suit :
		
		Certaines personnes disent alors que la somme des changements à l'extérieur d'une fonction (le côté gauche de l'égalité) est égale aux changements à l'intérieur de la fonction (le côté droit de l'égalité). C'est typiquement un cas particulier à $1$ dimension du théorème de Stokes!
	\end{enumerate}
	Démontrons la première propriété du théorème fondamental:
	\begin{dem}
	Soit la fonction:
	
	Si $f$ est positive et $h>0$ (la démonstration dans le cas où $h<0$ est  similaire) et comme $X>0$, nous pouvons nous représenter $A(X)$ comme l'aire sous la courbe de $f$ de $t=a$ jusqu'à $t=X$.
	\begin{figure}[H]
		\centering
		\includegraphics{img/algebra/area_primitive_representation.jpg}
		\caption{Représentation graphique de l'aire}
	\end{figure}
	Pour démontrer que $A$ est une primitive de $f$, nous allons prouver que $A'=f$. Selon la définition de la dérivée:
	
	Etudions ce quotient: $A(x+h)-A(x)$ est représentée par l'aire de la bande de largeur $h$, prise en sandwich entre deux rectangles de largeur $h$.
	
	Soit $M$ le maximum de $f$ sur l'intervalle $[X,X+h]$ et $m$ le minimum de $f$ sur ce même intervalle. Les aires respectives des deux rectangles sont $Mh$ et $mh$.
	
	Nous avons alors la double inégalité suivante:
	

	Comme $h$ est positif, on peut diviser par $h$ sans changer le sens des inégalités:
	
	Lorsque $h\rightarrow 0^+$ et si $f$ est une fonction continue, alors $M$ et $m$ ont pour limite $f(X)$, et le rapport:
	
	qui est compris entre $m$ et $M$, a bien pour limite $f(X)$.
	
	Comme $A'(X)=f(X)$ pour tout $X$, ceci nous montre que la dérivée de la fonction aire est $f$. Ainsi $A$ est une primitive de $f$. Comme $A(a)=0$, $A$ est bien la primitive de f qui s'annule en $a$.
	\begin{flushright}
		$\blacksquare$  Q.E.D.
	\end{flushright}
	\end{dem}
	Avant de commencer la démonstration de la deuxième propriété du théorème fondamental, donnons et démontrons le théorème suivant qui va nous être indispensable: Si $F_1(x)$ et $F_2(x)$ sont deux primitives de la fonction $f(x)$ sur le segment $[a, b]$, leur différence est une constante (ce théorème est très important en physique pour ce qui est de l'étude de ce que nous appelons les "conditions initiales").
	\begin{dem}
	Nous avons en vertu de la définition de la primitive:
	
	Nous avons en vertu de la définition de la primitive: $\forall x \in [a,b]$.
	
	Posons:
	
	Nous pouvons écrire:
	
	Il vient donc de ce que nous avons vu pendant notre étude des dérivées que:
	
	Nous avons alors:
	
	\begin{flushright}
		$\blacksquare$  Q.E.D.
	\end{flushright}
	\end{dem}
	Il résulte de ce théorème que si nous connaissons une primitive quelconque $F(x)$ de la fonction $f(x)$, toute autre primitive de cette fonction sera de la forme:
	
	où $c^{te}$ est nommée la "\NewTerm{constant d'intégration}\index{constant d'int\'egration}\label{constant of integration}".
	
	Donc finalement, nous appelons "\NewTerm{intégrale indéfinie}\index{int\'egrale ind\'efinie}" de la fonction $f(x)$ et nous notons:
	
	toute expression de la forme où  $F (x)+c^{te}$ est une primitive de $f(x)$. Ainsi, par convention d'écriture:
	
	si et seulement si $F'(x)=f(x)$.
	
	Dans ce contexte, $f(x)$ est également appelée "\NewTerm{fonction à intégrer}\index{fonction \`a int\'egrer}" et $f(x)\mathrm{d}x$, la "\NewTerm{fonction sous le signe somme}".
	\begin{tcolorbox}[title=Remarque,colframe=black,arc=10pt]
	Une "\NewTerm{antidérivée}\index{antid\'eriv\'ee}" (primitive) d'une fonction $f$ est une fonction $F$ dont la dérivée est $f$. L'intégrale indéfinie de $f$ est l'ensemble de TOUTES les antidérivées (primitives) de $f$. Donc l'intégrale indéfinie et l'antidérivée (primitive) ne sont pas les mêmes chosese car la première est l'ensemble de toutes les secondes ! Si $f$ et $F$ sont tels que décrits tout à l'heure, l'intégrale indéfinie de $f$ a la forme $\{F+c^{te}|c^{te}\in\mathbb{R}\}$ quand une antidérivée (primitive) n'est qu'un élément de cet ensemble !! Mais tous les enseignants ne sont pas d'accord sur la définition des antidérivées (primitives)...
	\end{tcolorbox}
	
	Géométriquement, nous pouvons considérer l'intégrale indéfinie comme un ensemble (famille) de courbes telles que nous passons de l'une à l'autre en effectuant une translation dans le sens positif ou négatif de l'axe des ordonnées.
	
	Revenons-en à la démonstration du point (2) du théorème fondamental de l'analyse:
	\begin{dem}
	Soit $F$ une primitive de $f$. Puisque deux primitives diffèrent d'une constante, nous avons bien:
	
	ce que nous pouvons écrire aussi:
	
	pour tout $X$  dans $[a, b]$. Le cas particulier $X=a$ donne $\int\limits_a^a f(t)\mathrm{d}t$ et donc $F(a)+c^{te}=0$ et nous otenons trivialement $c^{te}=-F(a)$. En remplaçant, nous obtenons:
	
	Comme cette identité est valable pour tout $X$ de l'intervalle $[a,b]$, elle est vraie en particulier pour $X=b$. D'où:
	
	\begin{flushright}
		$\blacksquare$  Q.E.D.
	\end{flushright}
	\end{dem}
	Ce dernier résulétat montre aussi quelque chose d'utile!: Il n'est pas nécessaire lorsque nous évaluaons une intégrale de prendre en compte la constante de la primitive générale puisque celle-ci s'annule de par la différence des deux primitives!!
	\begin{tcolorbox}[title=Remarques,colframe=black,arc=10pt]
	\textbf{R1.} Le théorème fondamental qui montre le lien entre primitive et intégrale a conduit à utiliser le même symbole $\int$ pour écrire une primitive (introduit par Leibniz à la fin du 17ème siècle), qui est une fonction, et une intégrale, qui elle, est un nombre.\\
	
	\textbf{R2.} Nous avons également démontré dans le chapitre de Mécanique Analytique comment calculer à l'aide d'une intégrale la longueur d'une courbe dans le plan si la fonction $f(x)$ est explicitement connue.
	\end{tcolorbox}
	Voici quelques propriétés triviales de l'intégration qu'il est bon de se rappeler car souvent utilisées ailleurs sur le site (si cela ne vous semble pas évident, contactez-nous et nous le détaillerons):
	\begin{enumerate}
		\item[P1.] La dérivée d'une intégrale indéfinie est égale à la fonction à intégrer:
		
		
		\item[P2.] La différentielle d'une intégrale indéfinie est égale à l'expression sous le signe somme:
		
		
		\item[P3.] L'intégrale indéfinie de la différentielle d'une certaine fonction est égale à la somme de cette fonction et d'une constante arbitraire:
		
		
		\item[P4.] L'intégrale indéfinie de la somme (ou soustraction) algébrique de deux ou plusieurs fonctions est égale à la somme algébrique de leurs intégrales (ne pas oublier que l'on travaille avec l'ensemble des primitives et non des primitives particulières!):
		
		\begin{dem}
		Pour démontrer cela nous allons prouver que la dérivée du membre de gauche permet de trouver le membre de droite et inversement (réciproque) à l'aide des propriétés précédentes.
		
		D'après P1 nous avons:
				
		Vérifions s'il en est de même avec le membre de droite (nous supposons connues les propriétés des dérivées que nous avons démontrées au début de ce chapitre):
		
		\end{dem} 
		\begin{flushright}
			$\blacksquare$  Q.E.D.
		\end{flushright}
		
		\item [P5.] Nous pouvons sortir un facteur constant de sous le signe somme, c'est-à-dire:
		
		Nous justifions cette égalité en dérivant les deux membres (et d'après les propriétés des dérivées):
		
		
		\item[P6.] Nous pouvons sortir un facteur constant de l'argument de la fonction intégrée (plutôt rarement utilisée):
		
		En effet, en dérivant les deux membres de l'égalité nous avons d'après les propriétés des dérivées:
		
		
		\item[P7.] L'intégration d'une fonction dont l'argument est sommé (ou soustrait) algébriquement est la primitive de l'argument sommé (respectivement soustrait):
		
		Cette propriété se démontre également identiquement à la précédente à l'aide des propriétés des dérivées.
		
		\item[P8.] La combinaison des propriétés P6 et P7 nous permet d'écrire:
		
		
		\item[P9.] Soit $f$ une fonction continue sur $[a,b]$, nous avons pour tout $c$ appartenant à cet intervalle:
		
		Ce théorème, appelé parfois "\NewTerm{relation de Chasles}\index{relation de Chasles}" (de par son pendant vectoriel), découle immédiatement de la définition de l'intégrale indéfinie. $F$ étant une primitive de $f$ sur $[a, b]$ nous avons:
		
		
		\item[P10.] Voilà une propriété souvent utilisée dans la section de Statistiques du site (nous ne trouvons pas de moyen d'exprimer cette propriété par le langage courant donc...):
		
		 Voyons deux propriétés qui nous seront parfois utiles pour calculer des intégrales difficiles:
		 
		\item[P11.] Si une fonction est paire (\SeeChapter{voir section Analyse Fonctionnelle page \pageref{even function}}), l'intégrale sur des bornes symétriques équivaut à:s
		
		
		\item[P12.] Si une fonction est paire (\SeeChapter{voir section Analyse Fonctionnellea page \pageref{odd function}}), l'intégrale sur des bornes symétriques équivaut à:
		
		
		\item[P13.] L'intégrale d'une fonction périodique est invariante sous un décalage de son intégration. C'est une propriété que nous utiliserons plus loin pour finaliser la preuve de la représentation intégrale de la fonction de Bessel d'ordre zéro du premier type.
	
		Si $f$ est une fonction périodique de période $T$ nous savons que pour toute valeur $a$:
		
		Considérons donc maintenant :
		
		Nous faisons le changement de variable $y=t-T$, alors nous avons pour la dernière intégrale :
		
		Dès lors:
		
	\end{enumerate}
	
	\pagebreak
	\subsubsection{Intégrale Double}\label{double integral}
	L'idée des intégrales doubles est de mesurer le volume de la zone délimitée par le graphe d'une fonction de deux variables, au-dessus d'un domaine $D$ du plan (ci-dessous $D$ est rectangulaire).
	\begin{figure}[H]
		\centering
		\includegraphics{img/algebra/double_integral_square_domain.jpg}
		\caption{Exemple d'une fonction à deux variables au-dessus d'un domaine}
	\end{figure}
	Il va sans dire que les intégrales doubles sont extrêmement importantes aussi dans tout le domaine des mathématiques appliquées!
	
	Là encore, l'idée est la même que l'intégrale définie. Si nous adaptons une approche simpliste, nous décomposons la fonction continue en un escalier et le volume à calculer se réduit alors à faire la somme des volumes de parallélépipèdes:
	\begin{figure}[H]
		\centering
		\includegraphics{img/algebra/double_integral_square_domain_decomposition_into_big_parallelepipeds.jpg}
		\caption{Décomposition du volume en parallélépipèdes grossiers}
	\end{figure}
	Dès lors nous avons la double somme:
	
	Pour une fonction continue, nous procèdons par approximations successives: nous calculons des sommes de Riemann pour des subdivisions de plus en plus fines du domaine $D$:
	\begin{figure}[H]
		\centering
		\includegraphics{img/algebra/double_integral_square_domain_decomposition_into_thin_parallelepipeds.jpg}
		\caption{Décomposition du volume en parallélépipèdes de plus en plus fins}
	\end{figure}
	et donc à la limite : 
	
	Par contre, quand on veut intégrer sur un domaine qui n'est pas rectangulaire, les choses se compliquent à priori... Voyons comment contourner le problème.
	
	Pour cela, nous allons construire le domaine $D$ fermé borné de la façon suivante.
	
	où le lecteur aura remarqué que le support de $y$ est la variable $x$ par l'intermédiaire de deux fonctions $u$ et $v$. C'est ce que nous appelons alors un "\NewTerm{domaine du type I}\index{domaine de d\'efinition!domaine du type I}" (et donc si c'est $y$ qui paramétrise $x$ alors il s'agit "\NewTerm{domaine du type II}\index{domaine de d\'efinition!domaine du type II}").

	Ce qui peut s'illustrer par la figure ci-dessous:
	\begin{figure}[H]
		\centering
		\includegraphics{img/algebra/double_integral_type_I_domain_example.jpg}
		\caption{Exemple d'un domaine de type I }
	\end{figure}
	où nous remarquons que cette approche simpliste (il existe d'autres approches possibles mais qui nécessitent de faire appel à la théorie de la mesure) nécessite que le domaine soit simplement connexe (qu'il n'y ait pas de trous hors du domaine $D$ entre $u (x)$ et $v (x)$) ou décomposé en sous-domaines disjointes simplement connexes.
	
	Bref, nous pouvons donc intégrer de la manière suivante:
	
	Donc nous transformons l'intégrale double en deux intégrales simples emboîtées.
	
	\paragraph{Théorème de Fubini}\label{fubini theorem}\mbox{}\\\\
	Nous allons voir un théorème important utilisé à de nombreuses reprises dans différents chapitres du site et qui permet d'inverser l'ordre d'intégration.

	En se rappelant que:
	
	nous pouvons aussi utiliser:
	
	Ainsi avec cette paramétrisation nous pouvons écrire:
	
	Nous pouvons ainsi changer l'ordre d'intégration, le calcul est différent, mais le résultat est le même. Mais ce n'est pas cela qui nous intéresse en réalité ici.
	
	Considérons une fonction telle que (nous disons que la fonction est à "variables séparables"):
	
	Alors:
	
	Supposons que le domaine est un rectangle (nous faisons cette simplification sinon la démonstration se complique nettement). C'est-à-dire:
	
	Dès lors par la propriété de linéarité des intégrales:
	
	
	\subsubsection{Intégration par changement de variables}
	Lorsque nous ne pouvons facilement déterminer la primitive d'une fonction donnée, nous pouvons nous débrouiller par un changement de variable astucieux (parfois même très subtil) pour contourner la difficulté. Cela ne marche pas à tous les coups (car certaines fonctions ne sont pas intégrables formellement) mais il vaut la peine d'essayer avant d'avoir recours à l'ordinateur.

	À nouveau, nous ne donnons que la forme générale de la méthode. C'est le rôle des professeurs dans les écoles d'entraîner les élèves à comprendre et maîtriser ce genre de techniques. De plus, les chapitres traitant des sciences exactes sur le site (physique, informatique, astrophysique, chimie, ...) regorgent d'exemples utilisant cette technique et servent ainsi implicitement d'exercices de style.

	Soit à calculer l'intégrale (non bornée pour l'instant):
	
	bien que nous ne sachions pas calculer directement la primitive de cette fonction $f(x)$ (en tout cas nous imaginons être dans une telle situation) nous savons (d'une manière ou d'une autre) qu'elle existe (nous ne traitons pas encore des intégrales impropres à ce niveau).

	La technique consiste alors dans cette intégrale à effectuer le changement de variable:
	
	où $\varphi (t)$ est une fonction continue ainsi que sa dérivée, et admettant une fonction inverse. Alors $\mathrm{d}x=\varphi' (t)\mathrm{d}t$, démontrons que dans ce cas l'égalité:
	
	est satisfaite.
	\begin{dem}
	Nous sous-entendons ici que la variable $t$ sera remplacée après intégration du membre droit par son expression en fonction de $x$. Pour justifier l'égalité en ce sens, il suffit de montrer que les deux quantités considérées dont chacune n'est définie qu'à une constante arbitraire près ont la même dérivée par rapport à $x$. La dérivée du membre gauche est:
	
	Nous dérivons le membre droit par rapport à$x$ en tenant compte que $t$ est une fonction de $x$. Nous savons que:
	
	Nous avons par conséquent:
	
	\begin{flushright}
		$\blacksquare$  Q.E.D.
	\end{flushright}
	\end{dem}
	Les dérivées par rapport à x des deux membres de l'égalité de départ sont donc égales.
	
	Bien évidemment, la fonction $x=\varphi (t)$ doit être choisie de manière à ce que nous sachions calculer l'intégrale indéfinie figurant à droite de l'égalité.
	
	\begin{tcolorbox}[title=Remarque,colframe=black,arc=10pt]
	Il est parfois préférable de choisir le changement de variable sous la forme $t=\varphi (x)$ au lieu de $x= \psi(t)$ car cela à une large tendance à simplifier la longueur de l'équation au lieu de l'allonger.
	\end{tcolorbox}
	Il est évident que ce théorème s'écrira plus explicitement :
	
	
	\pagebreak
	\paragraph{Jacobien}\label{jacobian}\mbox{}\\\\
	Considérons un domaine $D$ du plan $u,v$ limité par une courbe $L$. Supposons que les coordonnées $x, y$ soient des fonctions des nouvelles variables $u, v$ (toujours dans le cadre d'un changement de variables donc) par les relations de transformations:
	
	où les fonctions $\varphi (x,y)$ et $\phi(u,v)$ sont univoques, continues et possèdent des dérivées continues dans un certain domaine $D'$ que nous définirons par la suite. Il correspond alors d'après les relations précédentes à tout couple de valeurs $u, v$ un seul couple de valeur $x, y$ et réciproquement.
	
	Il résulte de ce qui précède qu'à tout point $P(x,y)$ du plan $\text{O}xy$ correspond univoquement un point $P '(u, v)$ du plan $\text{O}uv$ de coordonnées $u, v$ définies par les relations précédentes. Les nombres $v$ et $u$ seront appelées "\NewTerm{coordonnées curvilignes}\index{coordonn\'ees curvilignes}\label{curvilinear coordinates}" de $P$ et nous verrons des exemples concrets et schématisés de ceux-ci dans la section de Calcul Vectoriel.
	
	Si dans le plan $\text{O}xy$ le point $P$ décrit la courbe fermée $L$ délimitant le domaine $D$, le point correspondant décrit dans le plan $\text{O}uv$ un certain domaine $D'$. Il correspond alors à tout point de $D'$ un point de $D$. Ainsi, les relations de transformations établissent une correspondance biunivoque entre les points des domaines $D$ et $D'$.
	
	Considérons maintenant dans $D'$ une droite d'équation $u=c^{te}$. En général, les relations de transformation lui font correspondre dans le plan $\text{O}xy$ une ligne courbe (ou inversement). Ainsi, découpons le domaine $D'$ par des droites d'équations $u=c^{te}$ et$v=c^{te}$  en de petits domaines rectangulaires (nous ne prendrons pas en compte dans la limite, les rectangles empiétant sur la frontière de $D'$). Les courbes correspondantes du domaine $D$ découpent alors ce dernier en quadrilatère (définis par des courbes donc). Évidemment, l'inverse est applicable.
	
	Considérons dans le plan $\text{O}uv$ le rectangle $\Delta s'$ limité par les droites:
	
	et le quadrilatère curviligne $\Delta s$ correspondant dans le plan $\text{O}xy$. Nous désignerons les aires de ces domaines partiels également par $\Delta s'$ et $\Delta s$. Nous avons évidemment:
	
	Les aires $\Delta s$ et $\Delta s'$ peuvent être en général différentes.
	 \begin{figure}[H]
		\centering
		\includegraphics[scale=0.8]{img/algebra/non_linear_map.jpg}
		\caption{Une application non linéaire $f : \mathbb{R}^2\mapsto \mathbb{R}^2$ envoie un petit carré à un parallélogramme déformé}
	\end{figure}
	Supposons donc dans $D$ une fonction continue $z=f(x,y)$. Il correspond à toute valeur de cette fonction du domaine $D$ la même valeur $z=F(u,v)$ (ce qu'il faut vérifier) dans $D'$, où:
	
	Considérons les sommes intégrales de la fonction $z$ dans le domaine $D$. Nous avons évidemment l'égalité suivante:
	
	Calculons $\Delta s$, c'est-à-dire l'aire du quadrilatère curviligne $P_1,P_2,P_3,P_4$ dans le plan $\text{O}xy$:
	
	Déterminons les coordonnées de ses sommets:
	
	Nous assimilerons dans le calcul de l'aire du quadrilatère $P_1,P_2,P_3,P_4$ les arcs $P_1P_2,P_2P_3,P_3P_4,P_4P_1$ à des segments de droites parallèles. Nous remplacerons en outre les accroissements des fonctions par leurs différentielles. C'est dire que nous faisons abstraction des infiniment petits d'ordre plus élevé que $\Delta u$ et $\Delta v$. Les relations précédentes deviennent alors:
	
	Sous ces hypothèses, le quadrilatère curviligne $P_1P_2P_3P_4$ peut être assimilé à un parallélogramme. Son aire est approximativement égale au double de l'aire du triangle  $P_1P_2P_3$, aire que nous pouvons calculer en utilisant les propriétés du déterminant (comme nous le démontrerons dans le chapitre d'Algèbre Linéaire, le déterminant dans $\mathbb{R}^2$ représente un parallélogramme alors que dans $\mathbb{R}^3$ celui-ci représente le volume d'un parallélépipède):
	
	Tel que (c'est là qu'il faut faire le meilleur choix pour que l'expression finale soit la plus simple et la plus esthétique, nous procédons par essais successifs et faisons enfin le choix ci-dessous):
	\begin{figure}[H]
		\centering
		\includegraphics{img/algebra/graphical_representaton_of_determinant.jpg}
		\caption{Représentation graphique du déterminant}
	\end{figure}
	Ainsi, nous avons:
	
	Par conséquent la relation suivante (contenant ce qu'il est d'usage d'appeler le "\NewTerm{déterminant fonctionnel}\index{d\'eterminant fonctionnel}):
	
	avec:
	
	qui est la "\NewTerm{matrice jacobienne}\index{matrice jacobienne}" (alors que son déterminant est appelé le "jacobien" (tout court)) de la transformation de coordonnées de $\mathbb{R}^2 \rightarrow \mathbb{R}^3$. En appliquant exactement le même raisonnement pour $\mathbb{R}^3$, la matrice jacobienne s'écrit alors (en changeant un peu les notations car sinon cela devient illisible):
	
	Bref, à quoi cela sert-il concrètement ? Eh bien revenons à notre relation:
	
	qui n'est finalement qu'approximatiion étant donné que dans les calculs de l'aire $\Delta s$ nous avons négligé les infiniment petits d'ordre supérieur. Toutefois, plus les dimensions des domaines élémentaires $\Delta s$ et $\Delta s'$ sont petites, et plus nous nous approchons de l'égalité. L'égalité ayant finalement lieu quand nous passons à la limite (finalement en maths aussi on fait des approximations... hein !), les surfaces des domaines élémentaires tendant vers zéro:
	
	Appliquons maintenant l'égalité obtenue au calcul de l'intégrale double (nous pouvons faire de même avec la triple bien sûr). Nous pouvons donc finalement écrire (c'est la seule manière de poser la chose qui ait un sens):
	
	Passant à la limite, nous obtenons l'égalité stricte:
	
	Telle est la relation de transformation des coordonnées dans une intégrale double. Elle permet de ramener le calcul d'une intégrale double dans le domaine $D$ au domaine $D'$, ce qui peut simplifier le problème. De même, pour une intégrale triple, nous écrirons:
	
	
	\begin{tcolorbox}[colframe=black,colback=white,sharp corners]
	\textbf{{\Large \ding{45}}Exemples:}\\\\
	Déterminons maintenant le Jacobien pour les systèmes de coordonnées les plus courants (nous renvoyons à nouveau le lecteur au chapitre de Calcul Vectoriel pour plus d'informations concernant ces systèmes):\\
	
	E1. Coordonnées polaires\label{jacobian in polar coordinates} $x=r\cos(\phi),y=r\sin(\phi)$:
	
	Comme $r$ est toujours positif, nous écrivons simplement:
	
	\end{tcolorbox}
	
	\pagebreak
	\begin{tcolorbox}[colframe=black,colback=white,sharp corners]
	E2. En coordonnées cylindriques $x=r\cos(\phi),y=r\sin(\phi),z=z$ (voir section Algèbre Linéaire page \pageref{determinant} pour le calcul détaillé d'un tel déterminant):
	
	Comme $r$ est toujours positif, nous écrivons simplement:
	
	
	E3. \label{jacobian spherical coordinates}En coordonnées sphériques  $x=r\sin(\theta)\cos(\phi),y=r\sin(\theta)\sin(\phi),z=r\cos(\theta)$ (voir section Algèbre Linéaire page \pageref{determinant}  pour le calcul détaillé d'un tel déterminant):
	
	Comme $r^2$ est toujours positif, nous écrivons simplement:
	
	\end{tcolorbox}
	La définition de la différentiabilité dans le calcul multivariable est un peu technique. Il y a des subtilités à surveiller, car il faut se rappeler que l'existence de la dérivée est une condition plus stricte que l'existence de dérivées partielles. Mais, au final, si notre fonction est assez sympa pour qu'elle soit dérivable, alors la dérivée elle-même n'est pas trop compliquée. C'est une généralisation assez simple de la dérivée à variable unique.
	
	Dans le calcul univaré, nous avons appris que la dérivée d'une fonction $f : \mathbb{R} \rightarrow \mathbb{R}$ en un seul point n'est qu'un nombre réel, le taux d'augmentation de la fonction (c'est-à-dire la pente du graphique) à ce stade. Nous pourrions considérer ce nombre comme une matrice de $1 \times 1$, donc si nous le souhaitons, nous pourrions désigner la dérivée de $f(x)$ à $x=a$ comme :
	
	Pour une fonction à valeur scalaire de plusieurs variables, telles que $f(x, y)$ ou $f(x, y, z)$, nous pouvons considérer les dérivées partielles comme les taux d'augmentation de la fonction dans la directions des coordinées. Si la fonction est dérivable, alors la dérivée est simplement une matrice ligne contenant toutes ces dérivées partielles, que nous appelons la "matrice des dérivées partielles" (qui correspond à la matrice jacobienne). Pour $f : \mathbb{R}^{n} \rightarrow \mathbb{R}$, vu comme un $f(\vec{x})$, où $\vec{x}=\left(x_{1 }, x_{2}, \ldots, x_{n}\right)$, la matrice $1 \times n$ des dérivées partielles à $\vec{x}=\vec{a}$ est :
	
	La dernière généralisation concerne les fonctions vectorielles, $f : \mathbb{R}^{n} \rightarrow \mathbb{R}^{m}$. Ici, $f(\vec{x})$ est une fonction du vecteur $\vec{x}=\left(x_{1}, x_{2}, \ldots, x_{n}\right)$ dont la sortie est un vecteur de $m$ composants. Nous pourrions écrire $f$ en termes de ses composants comme :
	
	(Rappelez-vous que lorsque nous considérons les vecteurs comme des matrices, nous les considérons comme des matrices de colonnes, de sorte que les composants sont empilés les uns sur les autres.)
	
	Pour former la matrice des dérivées partielles, nous considérons $f(\vec{x})$ comme une matrice de colonnes, où chaque composant est une fonction à valeur scalaire. La matrice des dérivées partielles de chaque composant $f_{i}(\vec{x})$ serait une matrice de lignes de $1 \times n$, comme ci-dessus. Nous empilons simplement ces matrices de lignes les unes sur les autres pour former une matrice plus grande. Nous obtenons que la matrice $m \times n$ complète des dérivées partielles à $\vec{x}=\vec{a}$ est :
	
	Bien que nous devrions probablement nous référer à la dérivée de $f$ comme la transformation linéaire associée à la matrice $\mathrm{D}f(\vec{a})$, c'est bien à ce niveau de se référer à la "\ NewTerm{matrice des dérivées partielles}" $Df(\vec{a})$ comme "la dérivée" de $f$ au point $\vec{a}$ (en supposant que $f$ est dérivable en $\vec{ a}$, bien sûr).
	
	\begin{tcolorbox}[colframe=black,colback=white,sharp corners]
	\textbf{{\Large \ding{45}}Exemple:}\\\\
	Soit $f(x, y)=x^{2}+y^{2}$. Nous voulons trouver $\mathrm{D} f(1,2)$ et l'équation pour le plan tangent (en utilisant le développement de Taylor multivarié tel que démontré à la page \pageref{multivariate taylor series}) à $(x, y)=(1,2)$.\\\\
		
	Nous avons dans un premier temps:
	
	Ainsi $\mathrm{D} f(1,2)=\left[2\quad 4\right]$.\\
	
	Puisque les deux dérivées partielles $\frac{\partial f}{\partial x}(x, y)$ et $\frac{\partial f}{\partial y}(x, y)$ sont des fonctions continues, nous savons que $f(x, y)$ est dérivable. Par conséquent, $\mathrm{D}f(1,2)$ est la dérivée de $f$, et la fonction y a un plan tangent.\\
	
	Pour calculer l'équation du plan tangent, le seul calcul supplémentaire est la valeur de $f$ à $(x, y)=(1,2)$, soit $f(1,2)=1^{2} +2^{2}=5$. L'équation pour le plan tangent est donc (toujours en utilisant l'approximation multivariée de Taylor) :
	
	\end{tcolorbox}

	\pagebreak
	\subsubsection{Intégration par parties}\label{integration by parts}
	Lorsque nous cherchons à effectuer des intégrations, il est très fréquent que nous ayons à utiliser un outil (ou méthode de calcul) appelé "\NewTerm{int\'egration par parties}\index{int\'egration par parties}". Il existe différents degrés d'utilisation de cet outil et nous allons commencer par le plus simple et qui est le plus utilisé dans tous les chapitres traitant de physique sur le présent site.

	Nous partons d'abord de la dérivée du produit de deux fonctions démontrée plus haut:
	
	nous avons donc:
	
	et il vient:
	
	après une dernière simplification nous avons enfin la fameuse relation très importante:
	
	Mais nous allons parfois avoir besoin de la généralisation de cette dernière relation. Nous pouvons démontrer que si $f$ et $g$ sont deux applications (fonctions) de classe $\mathcal{C}^n$ (dérivables $n$ fois) sur $[a, b]$ dans $\mathbb{C}$, alors :
	
	\begin{dem}
	Procédons par récurrence sur $n$ (attention ce n'est pas forcément facile à comprendre comme souvent avec les démonstrations par récurrence!).
	
	Tout en sachant la relation vraie pour $n = 1$, nous la supposons vraie pour $n$ (comme donnée dans la relation précédente!) et nous la démontrons pour $n + 1$ (donc nous devons retomber sur la relation précédente mais avec $n + 1$ au lieu de $n$):
	
	\begin{tcolorbox}[title=Remarque,colframe=black,arc=10pt]
	L'astuce (proposée par un internaute) dans cette démonstration est de bien voir que $-(-1)^n$ donne un signe moins quand $n$ est pair et un plus quand $n$ est impair et de même $+(-1)^{n+1}$ donne un signe moins quand $n$ est pair et un plus quand $n$ est impair.
	\end{tcolorbox}
	Pour $n = 1$ nous retrouvons la formule bien connue et qui sera très très souvent utilisée dans tout le présent livre:
	
	\begin{flushright}
		$\blacksquare$  Q.E.D.
	\end{flushright}
	\end{dem}

	\pagebreak
	\subsubsection{Primitives usuelles}\label{usual primitives}
	Il existe en mathématique et en physique un grand nombre de primitives ou de fonctions définies sur des intégrales que nous retrouvons assez fréquemment (mais pas exclusivement). Par ailleurs, toutes les primitives démontrées ci-dessous seront utilisées dans les sections relatives à la Mécanique, l'Ingénierie, l'Atomistique, les Mathématiques Sociales, etc. Donc, comme dans n'importe quel formulaire, nous vous proposons les primitives connues mais avec les démonstrations! 

	Cependant, nous omettrons les primitives qui découlent déjà des dérivées que nous avons démontrées plus haut. Ce qui signifie par exemple que nous supposerons connues les deux primitives très importantes (certainement les plus utilisées dans l'ensemble des pages de ce livre): 
		
	Sinon voici déjà une liste de quelques intégrales fréquentes (le lecteur en rencontrera de toute façon bien d'autres - développées dans les détails - lors de son parcours du présent livre): 
	\begin{enumerate}
		\item Primitive de $f(x)=\tan (x)$:
		
		Par définition nous avons donc:
		
		Nous utilisons le changement de variable $u=\cos(x),\mathrm{d}u=-\sin(x)\mathrm{d}x$ et dès lors:
		
		Donc:
		
		
		\item Primitive de $f(x)=\cot(x)$:
		
		Par définition nous avons donc:
		
		Nous utilisons le changement de variable $u=\sin(x),\mathrm{d}u=\cos(x)\mathrm{d}x$ et dès lors:
		 
		Donc:
		
		
		\item Primitive de $f(x)=\arcsin(x)$:
		
		Nous intégrons par parties:
		
		Si nous posons $u=1-x^2$, ce qui nous donne $\mathrm{d}u=-2x\mathrm{d}x$, nous obtenons:
		
		Donc:
		
		
		\item Primitive de $f(x)=\arccos(x)$:
		
		Nous intégrons à nouveau par parties:
		
		Si nous posons $u=1-x^2$, ce qui nous donne $\mathrm{d}u=-2x\mathrm{d}x$, nous obtenons:
		
		Donc:
		
		
		\item Primitive de $f(x)=\arctan(x)$:
		
		Nous intégrons encore une fois par parties:
		
		Si nous posons $u=1+x^2$, ce qui nous donne $\mathrm{d}u=2x\mathrm{d}x$, nous obtenons:
		
		Donc:
		
		
		\item Primitive de $f(x)=\text{arccot}(x)$:
		
		Encore une fois... nous intégrons par parties:
		
		Si nous posons  $u=1+x^2$, ce qui nous donne $\mathrm{d}u=2x\mathrm{d}x$, nous obtenons:
		
		Donc:
		
		
		\item Primitive de $f(x)=xe^{ax}$ avec $a\in \mathbb{R}\left\lbrace 0 \right\rbrace$:
		
		Une intégration par parties nous donne:
		
		Donc:
		
		La généralisation est de notre point de vue immédiate et on obtient la formule de réduction suivante :
		
		\begin{tcolorbox}[title=Remarque,colframe=black,arc=10pt]
		Une autre intégrale très importante avec l'exponentielle en physique est celle que nous avions démontrée lors de notre étude de la loi de Gauss-Laplace en statistiques et probabilités (détermination de la moyenne).
		\end{tcolorbox}
		
		\item Primitive de $f(x)=\ln(x)$:
		
		Nous écrivons:
		
		En intégrant par parties nous trouvons:
		
		Finalement:
		
		
		\item Primitive de $f(x)=x\ln(ax)$ avec $a\in \mathbb{R}\left\lbrace 0 \right\rbrace$:
		
		Une intégration par parties nous donne:
		
		Donc:
		
		
		\item Primitive de $f(x)=a^x$ pour $a>0,a\neq 1$:
		
		Pour commencer nous écrivons:
		
		Ainsi il vient:
		
		et:
		
		Finalement:
		
		
		\item Primitive de $f(x)=\log_a(x)$:
		
		Pour $a>0,a\neq 1$ sachant que (voir les propriétés des logarithmes dans la section d'Analyse Fonctionnelle):
		
		nous avons en utilisant la primitive de  $\ln(x)$:
		
		
		\item Primitive de $f(x)=\tanh(x)$:
		
		Nous avons:
		
		Nous utilisons le changement de variable $u=\cosh(x),\mathrm{d}u=\sinh(x)\mathrm{d}x$ et obtenons:
		
		Donc:
		
		
		\item Primitive de $f(x)=\coth(x)$:
		
		Nous avons donc:
		
		Nous utilisons le changement de variable $u=\sinh(x),\mathrm{d}u=\cosh(x)\mathrm{d}x$ et obtenons:
		
		Donc:
		
		\item Primitive de $f(x)=\text{arcsinh}(x)$:
		
		Nous intégrons par parties:
		
		Si nous posons $u=1+x^2,\mathrm{d}u=2x\mathrm{d}x$ nous obtenons:
		
		Donc:
		
		
		\item Primitive de $f(x)=\text{arccosh}(x)$:
		
		Nous intégrons par parties:
		
		Si nous posons $u=x^2-1,\mathrm{d}u=2x\mathrm{d}x$, nous obtenons:
		
		Donc finalement:
		
		
		\item Primitive de $f(x)=\text{arctanh}(x)$:
		
		Nous intégrons par parties:
		
		Si nous posons $u=1-x^2,\mathrm{d}u=-2x\mathrm{d}x$, nous obtenons:
		
		Donc finalement:
		
		
		\item Primitive de $f(x)=\text{arccoth}(x)$:
		
		Nous intégrons par parties:
		
		Si nous posons  $u=1-x^2,\mathrm{d}u=-2x\mathrm{d}x$, nous obtenons:
		
		Donc finalement:
		
		
		\item Primitive de $f(x)=\sin ^{n}(x)$ avec $n \geq 2$:
		
		Posons $I_n=\int \sin ^n(x)\mathrm{d}x$. Une intégration par partie donne:
		
		en remplaçant $\cos ^2(x)$ par $1-\sin ^2{x}$ dans la dernière intégrale, nous obtenons:
		
		et donc:
		
		Toutes les primitives de même forme (relation de récurrence) sont nommées "\NewTerm{formules de réduction}\index{formules de r\'eduction}".
		
		\item Primitive de $f(x)=\cos ^{n}(x)$ avec $n \geq 2$:
		
		Dans ce cas nous avons la formule de récurrence:
		
		cela se démontre exactement de la même manière que la relation récursive précédente (le lecteur peut demander les détails s'il le souhaite).
		
		\item Primitive de $f(x)=\tan ^{2}(x)$:
		
		Sachant que $\tan'(x)=1+\tan^2(x)$, nous avons:
		
		Donc:
		
		
		\item Primitive de $f(x)=\cot ^{2}(x)$:
		
		Sachant que $\cot'(x)=-1-\cot^2(x)$, nous avons:
		
		Donc:
		
		
		\item Primitive de $f(x)=\sin ^{-2}(x)$:
		
		En utilisant les relations trigonométriques remarquables (\SeeChapter{voir section Trigonométrie page \pageref{remarkable trigonometric identities}}), , nous avons: 
		
		selon la primitive $\cot^2(x)$. Donc:
		
		
		\item Primitive de $f(x)=\cos ^{-2}(x)$:
		
		En utilisant encore une fois les relations trigonométriques remarquables (\SeeChapter{voir section Trigonométrie page \pageref{remarkable trigonometric identities}}), nous avons:
		
		grâce à la primitive de $\tan^2(x)$. Donc:
		
		
		\item Primitive de $f(x)=\sin^{-1}(x)$:
		
		Nous faisons la substitution $x=2\arctan(t),t=\tan(x/2)$. Sachant que: (\SeeChapter{voir section Trigonométrie page \pageref{remarkable trigonometric identities}}):
		
		nous obtenons alors:
		
		Donc:
		
		Finalement:
		
		
		\item Primitive de $f(x)=\cos^{-1}(x)$:
		
		Sachant que $\cos(x)=\sin(x+\pi/2)$ (\SeeChapter{voir section Trigonométrie page \pageref{remarkable angles}}), nous avons:
		
		Nous faisons le changement de variable $x+\pi/2=u, \mathrm{d}u=\mathrm{d}x$:
		
		grâce à la connaissance de la primitive de $\sin^{-1}(x)$. Finalement:
		
		
		\item Primitive de $f(x)=\dfrac{1}{1+\cos(x)}$:
		
		Nous faisons la substitution $x=2\arctan(t),t=\tan(x/2)$, sachant que (\SeeChapter{voir section Trigonométrie page \pageref{remarkable trigonometric identities}}):
		
		nous obtenons:
		
		Dès lors:
		
		Finalement:
		
		
		\item  Primitive de $f(x)=\dfrac{1}{1-\cos(x)}$:
		
		Nous faisons à nouveau la substitution $x=2\arctan(t),t=\tan(x/2)$. Dès lors nous trouvons:
		
		Tel que finalement:
		
		
		\item  Primitive de $f(x)=\dfrac{1}{1+\sin(x)}$:
		
		Sachant que:
		
		Nous avons alors:
		
		En faisant le changement de variable:
		
		nous obtenons:
		
		Finalement:
		
		
		\item  Primitive de $f(x)=\dfrac{1}{1-\sin(x)}$:
		
		Par le même raisonnement que précédemment en utilisant le cosinus nous obtenons:
		
		
		\item Primitive de $f(x)=\sinh^n(x)$ avec $n \geq 2$:
		
		Posons:
		
		Une intégration par partie donne (nous avons démontré lors des dérivées usuelles que la primitive du sinus hyperbolique était le cosinus hyperbolique):
		
		en remplaçant $\cosh^2(x)$ avec $1+\sinh^2(X)$ ans la dernière intégrale, nous obtenons:
		
		et donc:
		
		Ainsi nous pouvons obtenir pour ce cas particulier:
		 
		
		\item Primitive de $f(x)=\cosh^n(x)$ avec $n \geq 2$:
		
		Dans ce cas nous avons aussi la relation de récurrence:
		
		qui se démontre de la même façon que ci-dessus. Ainsi:
		
		
		\item Primitive de $f(x)=\tanh^2(x)$:
		
		Sachant que (démontré lors des dérivées usuelles):
		
		nous avons:
		
		Donc:
		
		
		\item Primitive de $f(x)=\coth^2(x)$:
		
		Sachant que (démontré lors des dérivées usuelles):
		
		nous avons:
		
		Donc:
		
		
		\item Primitive de $f(x)=\dfrac{1}{\sinh^2(x)}$:
		
		Nous avons en utilisant la primitive de  $f(x)=\coth^2(x)$:
		
		Donc:
		
		
		\item Primitive de $f(x)=\dfrac{1}{\cosh^2(x)}$:
		
		Nous avons en utilisant la primitive de $f(x)=\tanh^2(x)$:
		
		Donc:
		
		
		\item Primitive de $f(x)=\dfrac{1}{\sinh(x)}$:
		
		Nous faisons la substitution: 
		
		Nous obtenons en utilisant la dérivée de $\text{arctanh}(x)$:
		
		
		et finalement:
		
		
		\item Primitive de $f(x)=\dfrac{1}{\cosh(x)}$:
		
		Nous faisons la substitution:
		
		Nous avons en utilisant la primitive de $\arctan(x)$:
		
		et finalement:
		
		
		\item Primitive de $f(x)=\dfrac{1}{1+\cosh(x)}$:
		
		Nous faisons la substitution:
		
		Nous obtenons:
		
		Finalement nous obtenons:
		
		
		\item Primitive de $f(x)=\dfrac{1}{1-\cosh(x)}$:
		
		Nous faisons la substitution: 
		
		Nous obtenons:
		
		Finalement:
		
		
		\item Primitive de $f(x)=\dfrac{1}{1+\sinh(x)}$:
		
		Nous faisons la substitution:
		
		Nous obtenons: 
		
		D'où:
		
		Donc:
		
		Finalement:
		
		
		\item Primitive de $f(x)=\dfrac{1}{1-\sinh(x)}$:
		
		Nous faisons la substitution habituelle:
		
		Nous obtenons: 
		
		Or:
		
		D'où:
		
		Finalement:
		
		
		\item Primitive de $f(x)=e^{ax}\sin(bx)$ avec $a,b\in \mathbb{R},a^2+b^2\neq 0$:
		
		Une première intégration par parties donne:
		
		Une deuxième intégration par parties donne:
		
		d'où l'égalité:
		
		Ainsi en redistribuant la relation précédente: 
		
		
		\item Primitive de $f(x)=e^{ax}\cos(bx)$ avec $a,b\in \mathbb{R},a^2+b^2\neq 0$:
		
		Un raisonnement analogue à celui d'avant montre que:
		
		
		\item Primitive de $f(x)=x\sin(ax)$ avec $a \in \mathbb{R}^*$:
		
		Une intégration par parties nous donne:
		
		
		\item Primitive de $f(x)=x\cos(ax)$ avec $a \in \mathbb{R}^*$:
		
		Une intégration par parties nous donne:
		
		
		\item Primitive de $f(x)=\dfrac{1}{(x-a)(x-b)}$ avec $a \neq b$:
		
		Nous avons la relation suivante (en calcul intégral, nous appelons une telle décomposition "\NewTerm{décomposition en fraction partielle}\index{d\'ecomposition en fraction partielle}"):
		
		Dès lors:
		
		
		Finalement:
		
		
		\item Primitive de $f(x)=\dfrac{1}{a^2-x^2}$ avec $a \neq 0$:
		
		Nous avons en utilisant le résultat précédent:	
		
		Dès lors:
		

		\item Primitive de $f(x)=\dfrac{1}{a^2+x^2}$ avec $a \neq 0$:
		
		En faisant le changement de variable:
		
		Nous obtenons en utilisant la dérivée de $\arctan (x)$:
		
		
		\item Soit:
		
		avec $n \in \mathbb{N}$. Nous obtenons:
		
		Or cette dernière intégrale se résout par parties:
		
		Dès lors:
		
		Or cette dernière intégrale se résout par parties:
		
		Identiquement au développement suivant, nous avons pour (le signe change):
		
		la relation suivante:
		
		Vous pourrez trouver une application de ces deux primitives dans le modèle cosmologique newtonien de l'univers dans le chapitre d'Astrophysique ainsi que dans le chapitre de Relativité Générale dans le cadre de l'étude de l'effet Shapiro!
		
		\item Primitive de $f(x)=\dfrac{1}{(1-x^2)^2}$:
		
		Nous avons en utilisant les primitives de $\dfrac{1}{(1-x^2)^n}$ (vue avant) et de  $\dfrac{1}{1-x^2}$ (vue plus haut):
		
		
		\item Primitive de $f(x)=\dfrac{1}{(1+x^2)^2}$\label{black hole primitive}:
		
		Nous avons en utilisant les primitives de $\dfrac{1}{(1+x^2)^n}$ (vue avant) et de $\dfrac{1}{1+x^2}$ (vue plus haut):
		
		Vous pouvez trouver une application de cette primitive dans la section de  Relativité Générale pour le temps de chute propre dans un Trou Noir non rotatif.
		
		\item Primitive de $f(x)=\sqrt{x^2-a^2}$ avec $a\in \mathbb{R}^*$ (cas relatif à la surface sous une hyperbole):
		
		Nous pouvons sans perte de généralité supposer  $a>0$. Remarquons que le domaine de définition de $f$ est $]-\infty,-a] \cup [a,+\infty[$.
		
		Nous allons déterminer une primitive de $f$ uniquement sur l'intervalle $[a,+\infty[$ (car c'est celle dont nous aurons besoin dans certains chapitres).
		
		Faisons le changement de variable:
		
		avec donc:
		
		où nous considérons la fonction $\cosh: \mathbb{R}^+ \rightarrow [1,+\infty[$ avec pour réciproque la fonction  $\text{arccosh}:[1,+\infty[ \rightarrow \mathbb{R}^+$ donnée par (\SeeChapter{voir section Trigonométrie page \pageref{inverse hyperbolic to logarithm}}):
		
		Nous obtenons alors en utilisant la primitive de $\sinh^2(x)$:
		
		mais (\SeeChapter{voir section Trigonométrie page \pageref{hyperbolic trigonometry}}) comme:
		
		Donc:
		
		et en utilisant un autre résultat de la section de Trigonométrie:
		
		nous avons alors:
		
		étant donné que les primitives sont données à une constante près, nous pouvons écrire:
		
		pour $x\geq a$. $F$ est donc une primitive de $\sqrt{x^2-a^2}$ sur l'intervalle $[a,+\infty[$.
		
		\item Primitive de $f(x)=\sqrt{a^2-x^2}$ avec $a\in \mathbb{R}^*$:
		
		Nous pouvons sans perte de généralité supposer  $a>0$. Remarquons que le domaine de définition de $f$ est $[-a, a]$.
		
		Nous faisons les substitutions:
		
		nous obtenons:
		
		où nous avons utilisé la primitive de $\cos^n (x)$ avec $n=2$ démontrée plus haut. Maintenant nous avons:
		
		Donc:
		
		et:
		

		\item Primitive de $f(x)=\sqrt{x^2+a^2}$ avec $a\in \mathbb{R}^*$:
		
		Nous pouvons sans perte de généralité supposer $a>0$.
		
		Faisons le changement de variable:
		
		avec:
		
		Nous obtenons:
		
		Ainsi:
		
		Mais comme nous avons vu dans la section de Trigonométrie page \pageref{hyperbolic trigonometry}:
		
		et:
		
		Donc nous avons finalement:
		
		où le $ln (a)$ a encore une fois été omis car les primitives sont données à une constante près.
		
		\item Primitive de $f(x)=\left(\sqrt{a^2-x^2}\right)^{-1}$ avec $a\in \mathbb{R}^*$:
		
		Nous pouvons sans perte de généralité supposer $a>0$.
		
		Nous faisons la substitution:
		
		Nous obtenons:
		
		
		\item Primitive de $f(x)=\dfrac{1}{\sqrt{a^2+x^2}}$ avec $a\in \mathbb{R}^*$:
		
		Nous pouvons sans perte de généralité supposery $a>0$.
		
		Faisons le changement de variable:
		
		Nous obtenons de la même manière que pour les intégrales usuelles précédentes:
		
		et sachant que (\SeeChapter{voir section Trigonométrie page \pageref{inverse hyperbolic to logarithm}}):
		
		Nous obtenons alors au final la primitive importante suivante:
		
		En procédant de même, mais en utilisant le cosinus hyperbolique au lieu du sinus hyperbolique, nous avons bien évidemment:
		
		Nous réutiliserons ces deux dernières relations dans des cas pratiques importants des sections de Mécanique Analytique, Génie Civil (où la constante $a$ valant $1$, $\ln(a)$ est de toute façon nul!) et de Relativité Générale (où $a$ sera non nul et donc il ne sera pas possible d'omettre la constante $\ln(a)$).
		
		\item En mathématiques, il existe plusieurs intégrales connues sous le nom de "\NewTerm{intégrales de Dirichlet}\index{intégrale de Dirichlet}\label{Dirichlet integral}". L'une d'entre elles est l'intégrale impropre de la fonction sinc sur la droite réelle positive :
	
	De par le théorème de Fubini (voir \pageref{fubini theorem}):
	
	Puis en utilisant la primitive de $e^{ax}$:
	
	et en utilisant la primitive de $e^{ax}\sin(bx)$:
	
	Dès lors:

	En utilisant la primitive de $\frac{1}{x^2+a^2}$ vu plus tôt:
	
	Comme:
	
	Alors:
	
		
		\item Considérons une intégrale de la forme suivante que nous pouvons utiliser pour améliorer la formule de Stirling (\SeeChapter{voir section Méthodes Numériques page \pageref{stirling}}) et aussi dont nous avons absolument besoin pour l'étude de la diffraction de Fresnel circulaire (\SeeChapter{voir section Optique Ondulatoire page \pageref{wave optics}}):
		
		où:
		\begin{itemize}
			\item $\lambda$ est grand;
			\item $g(y)$ est une fonction lisse qui a un minimum local à $y^*$ à l'intérieur de l'intervalle $[a, b]$ ;
			\item $h(y)$ est lisse.
		\end{itemize}
		L'intégrale peut être la fonction génératrice caractéristique de la distribution de $g(Y)$ lorsque $Y$ a une densité $h$ (\SeeChapter{voir section Statistiques page \pageref{charactertistic function}}), cela pourrait être une espérance postérieure de $h(Y)$, ou juste une intégrale "simple".
		
		Lorsque $\lambda$ est grand, la contribution à cette intégrale est essentiellement entièrement par construction provenant d'un voisinage autour de $y^*$.
		
		Nous formalisons cela par le développement de Taylor de la fonction $g$ autour de $y^*$ :
		
		Puisque $y^*$ est un minimum local, nous avons :
		
		et donc:
		
		Dès lors:
		
		Le fait que ci-dessus les limites aient changé de $[a,b]$ à $]-\infty,+\infty[$ est du au fait que nous supposons que la zone d'intérêt est au voisinage de $y^*$ et cela parce que $\lambda $ est tellement grand que  quand on s'éloigne rapidement un peu de $y^*$, on peut considérer la courbe comme négligeable !
		
		Si on approxime $h(y)$ linéairement autour de $y^*$, soit :
		
		tel que:
		
		Dès lors:
		
		Le lecteur ne doit pas oublier qu'ici $\lambda g''(y^*)$ est une constante et que si on pose :
		
		Dès lors, pour la première intégrale ci-dessus, nous voyons que nous nous rabattons sur l'intégrale de quelque chose de très similaire à la distribution de Gauss (avec une moyenne $y^*$ et un écart type $\lambda g''(y^*)$) et ensuite il vient presque immédiatement (\SeeChapter{voir section Statistiques page \pageref{Gauss integral}}) :
		
		Pour la deuxième intégrale :
		
		le changement de variable $y-y^*=x$ nous donne :
		
		Il est immédiat que primitive est de la forme :
		
		Donc par symétrie la seconde intégrale est nulle ! On a enfin :
		
		Ce calcul est nommé "\NewTerm{méthode d'intégration de Laplace}\index{m\'ethode d'int\'egration de Laplace}\label{laplace method of integration}" ou simplement "\NewTerm{intégration de Laplace}".
	\end{enumerate}
	
	\begin{tcolorbox}[title=Remarque,colframe=black,arc=10pt]
	Les personnes intéressées par une lecture plus approfondie sur des intégrales communes et belles peuvent jeter un oeil au très bon livre de P.J. Nahin : \textit{Inside Interesting Integrals} \cite{nahin2014inside}.
	\end{tcolorbox}	
	
	\subsubsection{Règle d'intégration de Leibniz}
	En algèbre, la "\NewTerm{règle de l'intégrale de Leibniz}\index{règle intégrale de Leibniz}\label{Leibniz's integral rule}", du nom de Gottfried Leibniz, indique que pour une intégrale de la forme :
	
	où $-\infty <a(x),b(x)<+\infty$, la dérivée de cette intégrale s'exprime par :
	
	Cette relation peut être utile pour évaluer certaines intégrales définies. Lorsqu'elle est utilisée dans ce contexte, la règle de Leibniz pour différencier sous le signe intégral est également connue sous le nom de "\NewTerm{astuce de Feynman}\index{astuce de Feynman}".
	
	Notons que si $a(x)$ et $b(x)$ sont des constantes plutôt que des fonctions de $x$, nous avons un cas particulier de la règle de Leibniz :
	
	Par ailleurs, si $a(x)=a$ et $b(x)=x$, ce qui est aussi une situation courante (par exemple, dans la preuve de la formule d'intégration répétée de Cauchy), nous avons :
	
	\begin{dem}
	Dans un premier temps, autorisons-nous à écrire:
	
	Nous partons de la définition même de la dérivée :
	
	Maintenant, puisque les limites d'intégration dépendent (en général) de $\alpha,$ alors un $\Delta \alpha$ provoquera un $\Delta a$ et un $\Delta b$ et nous devons donc écrire :
	
	Comme $\Delta \alpha \rightarrow 0$ nous avons $\Delta a \rightarrow 0$ et $\Delta b \rightarrow 0,$ aussi, et ainsi :
	
	où les deux derniers termes découlent du fait que comme $\Delta a \rightarrow 0$ et $\Delta b \rightarrow 0$ la valeur de $x$ sur tout l'intervalle d'intégration reste pratiquement inchangée à $x=a$ ou à $x= b$ respectivement. Ainsi:
	
	ou, en prenant le $1/\Delta \alpha$ à l'intérieur de l'intégrale (l'intégrale de Riemann elle-même est définie comme une limite, donc ce que nous faisons est d'inverser l'ordre de deux opérations limites, ce qu'un pur mathématicien voudrait justifier mais, comme d'habitude dans ce livre, nous ne nous en soucierons pas !):
	
	\begin{flushright}
		$\blacksquare$  Q.E.D.
	\end{flushright}
	\end{dem}
	La différenciation sous le signe intégral est mentionnée dans les mémoires à succès du regretté physicien Richard Feynman \textit{Surely You're Joking, Mr. Feynman!} dans le chapitre \textit{A Different Box of Tools}. Il décrit l'avoir appris, alors qu'il était au lycée, à partir d'un ancien texte, \textit{Advanced Calculus} (1926), de Frederick S. Woods (qui était professeur de mathématiques au Massachusetts Institute of Technology). La technique n'était pas souvent enseignée lorsque Feynman a reçu plus tard son éducation formelle en Algèbre, mais en utilisant cette technique, Feynman a pu résoudre des problèmes d'intégration autrement difficiles à son arrivée à l'école supérieure de l'Université de Princeton :
	
	«\textit{Une chose que je n'ai jamais apprise était l'intégration des contours. J'avais appris à faire des intégrales par diverses méthodes décrites dans un livre que mon professeur de physique au lycée, M. Bader, m'avait donné. Un jour, il m'a dit de rester après les cours. "Feynman, dit-il, tu parles trop et tu fais trop de bruit. Je sais pourquoi. Tu t'ennuies. Alors je vais te donner un livre. Tu montes là-bas au fond, dans le coin, et étudies ce livre, et quand vous saurez tout ce qu'il y a dans ce livre, vous pourrez reparler." Donc à chaque cours de physique, je ne prêtais aucune attention à ce qui se passait avec la loi de Pascal, ou quoi qu'ils fassent. J'étais à l'arrière avec ce livre : "Advanced Calculus", de Woods. Bader savait que j'avais un peu étudié "Calculus for the Practital Man", alors il m'a donné les vrais exercices. C'était pour un cours junior ou senior au collège. Il y avait des séries de Fourier, des fonctions de Bessel, des déterminants, des fonctions elliptiques, toutes sortes de trucs merveilleux dont je ne connaissais rien. Ce livre montrait aussi comment différencier des paramètres sous le signe intégral, c'est une certaine opération. Il s'avère que cela n'est pas très enseigné dans les universités; ils ne le soulignent pas. Mais j'ai compris comment utiliser cette méthode, et j'ai utilisé ce foutu outil encore et encore. Donc, parce que j'étais autodidacte en utilisant ce livre, j'avais des méthodes particulières pour faire des intégrales. Le résultat était que lorsque les gars du MIT ou de Princeton avaient du mal à faire une certaine intégrale, c'était parce qu'ils ne pouvaient pas le faire avec les méthodes standard qu'ils avaient apprises à l'école. Si c'était l'intégration des contours, ils l'auraient trouvée ; s'il s'agissait d'une simple extension en série, ils l'auraient trouvée. Ensuite, je viens et j'essaie de différencier sous le signe intégral, et souvent cela a fonctionné. J'avais donc une excellente réputation pour faire des intégrales, uniquement parce que ma boîte à outils était différente de celle des autres, et ils avaient essayé tous leurs outils dessus avant de me donner le problème.}»
	
	
	\begin{tcolorbox}[colframe=black,colback=white,sharp corners]
	\textbf{{\Large \ding{45}}Example:}\\\\
	Let us compute the following integral with variables limits:
	
	As both $a(x)$ and $b(x)$ are constants rather than functions of $x$, we will use the special case of Leibniz's integral rule:
	
	Therefore differentiating under the integral with respect to $\alpha$ , we have (at the end we use one of the previous usual primitive and trigonometric relations):
	
	\end{tcolorbox}
	
	\begin{tcolorbox}[colframe=black,colback=white,sharp corners]
	Therefore:
	
	But $I(\pi/2)=0$ by definition, so we must have $c^{te}=\pi^2/8$ and finally:
	
	\end{tcolorbox}
	
	\subsubsection{Integral representation of first kind Bessel's function}\label{integral representation of first kind Bessel's function}
	A particularly useful and powerful way of treating Bessel functions employs their integral representation as we will see in the section of Wave Optics page \pageref{fresnel circular aperture}.
	
	Remember that in the section of Sequences and Series we have proved that the generating function of Bessel's function was (see page \pageref{generating function for bessel function of first kind}):
	
	That is:
	
	Now remember that (\SeeChapter{see section Numbers page \pageref{euler formula}}):
	
	So if we return to the generating function, and substitute $t=e^{\mathrm{i}\theta}$, we get:
	
	In which to condensate the result we have used first the property proved during our study of the generating function of Bessel's functions:
	
	and also:
	
	and so on...
	Now remember that:
	
	Therefore identifying real and imaginary part, we get:
	
	Remember also that we have proved during our study of Fourier series in the section of Sequences and Series that:
	
	\begin{center}
	\begin{tabular}{ccc}
	$\text{with }n,k\in \mathbb{N}\text{ and }n\ne k$
	&$\qquad$&
	$\text{with }n,k\in \mathbb{N}\text{ and }n = k$
	\end{tabular}
	\end{center}
	That is:
	
	where $\delta_{nm}$ is the Kronecker symbol (\SeeChapter{see section Tensor Calculus page \pageref{kronecker symbol}}).
	
	Now let us write:
	
	Let us focus on the first integral:
	
	So we see above that whatever the value for any $n$, excepte of $n=0$:
	
	Therefore in only remains for $n>0$:
	
	and we see above that if $n>0$ is odd all the integrals vanish but if $n>0$ is even, only the corresponding $J_n(x)$ remains!
	
	Exactly the same analysis can be done for can be done:
	
	We have therefore for each of the integrals above, especially for each the left term that (recall that $n=0,2,4,\ldots$ is even and $n=1,3,5,\ldots$ is odd):
	
	If these two equations are added together we have using trigonometric identities (\SeeChapter{see section Trigonometry page \pageref{remarkable trigonometric identities}}):
	
	for $n=0,1,2,3,\ldots$.
	
	If we put $n=0$ in the above relation, we get:
	
	If we plot $\cos(x\sin(\theta))$ we see that it repeats itself in all four quadrants (it's an even function):\\\\
	\texttt{>plot([cos(sin(theta)),cos(2*sin(theta)),cos(3*sin(theta)),cos(5*sin(theta))]\\
	,theta=-2*Pi..2*Pi);}
	\begin{figure}[H]
		\centering
		\includegraphics[scale=0.55]{img/algebra/cos_sin_maple.jpg}
	\end{figure}
	So we can write:
	
	This is the real integral representation of the zero order Bessel function of the first type.
	
	But in many developments we don't use the above expression as there is not phasor that is visible. So the trick is to notice that $\sin(x\sin(\theta))$ reverses its sign in the third an fourth quadrant (it's and odd function):\\\\
		\texttt{>plot([sin(sin(theta)),sin(2*sin(theta)),sin(3*sin(theta)),sin(5*sin(theta))],\\
	theta=-2*Pi..2*Pi);}
	\begin{figure}[H]
		\centering
		\includegraphics[scale=0.6]{img/algebra/sin_sin_maple.jpg}
	\end{figure}
	So we have:
	
	Adding the both relations by multiplying the second by $\mathrm{i}$,  we have:
	
	Finally we get the complex representation of the zero order Bessel function of the first type:
	
	Let us do a change of variable $\theta=\varphi+\pi/2$, then:
	
	But we have proved earlier that for any periodic function:
	
	Therefore:
	
	This integral representation my be obtained in various ways but this one seems the most easy one to us. Many other integral representation exists.

	\subsubsection{Dirac Function}
	The Dirac function, also named "\NewTerm{Dirac peak}\index{Dirac peak}\label{dirac function}" or "\NewTerm{delta function}\index{delta function}", plays a very important practical role both in electronic and computer also  wave mechanics and quantum field theory (this allows to discretize a continuum!).
	
	Before going further we could notice that it is wrong to speak about a "function" because a function is an application of a start set (usually the set of real or complex number with one or more dimensions) in an arrival set (usually the set of real or complex numbers in one or more dimensions). While the domain of definition of the Dirac function is not a set of numbers but strictly speaking a set of functions!
	
	More technically the Dirac delta function, or $\delta$ function, is a generalized function, or distribution, on the real number line that is zero everywhere except at zero, with an integral of one over the entire real line. The delta function is sometimes thought of as an infinitely high, infinitely thin spike at the origin, with total area one under the spike, and physically represents the density of an idealized point mass or point charge. It was introduced by theoretical physicist Paul Dirac. In the context of signal processing it is often referred to as the unit impulse symbol (or function). Its discrete analogue is the Kronecker delta function, which is usually defined on a discrete domain and takes values $0$ and $1$.
	
	As always in this book we will focus here only on the properties we will need to study Applied Mathematics stuffs of other sections of the book.
	
	To represent mentally in an easy way this function, first consider the function defined by:
	
	The representation of $y=f(x)$ above is a rectangle of width $a$, and of height $1/ a$ and unit surface. The Dirac function can be considered as the boundary when $a\leftarrow 0$ of the  $f (x)$. So we have:		
	
	That is to say:
	\begin{figure}[H]
		\centering
		\includegraphics[scale=0.5]{img/algebra/common_dirac_peak_representation.jpg}
		\caption[Schematic representation of the Dirac delta function by a line surmounted by an arrow]{Schematic representation of the Dirac delta function by a line surmounted by an arrow (source: Wikipedia)}
	\end{figure}
	with:
	
	where $\varepsilon$ is a number greater than $0$ and as small as we want.
	\begin{tcolorbox}[title=Remark,colframe=black,arc=10pt]
		As the reader will have probably noticed it when we introduced the initial function $f (x)$, the resulting delta Dirac function has therefore the dimension of the inverse of a length!
	\end{tcolorbox}
	For a function $g (x)$ continues in $x = 0$ we have:
	
	By extension we have:
	
	and for a function $g (x)$ coninue on $x_0$:
	
	It is then relatively easy to define the Dirac function in 3-dimensional space by:
	
	As as already mentioned we will prove properties of the Dirac function only if we will need them in other sections of this book.
	
	To see now the integral representation of the Dirac function (very useful in Quantum Field Theory) let us recall the Fourier transform of a function $f(t)$ (\SeeChapter{see section Sequences and Series page \pageref{fourier transform}}) using another common notation and convention used by physicists and mathematicians (some of them invert the definition... the Fourier transform is designated as the Inverse Fourier transform...) :
	
	This transform is reversible, i.e., you can go back from $\tilde{f}(s)$ to $f(t)$ by:
	
	If we set $f(t)=\delta(t)$ in the above equations, we find:
	
	In other words, the delta function and a constant $1 / \sqrt{2 \pi}$ are Fourier-transform of each other.
	
	Another way to see the integral representation of the delta function is again using the limits. For example, using the limit of the Gaussian (and the fact the the Fourier transform of a Gaussian is another Gaussian as prove at page \pageref{fourier transform gaussian function}):
	
	Or more generally:
	
	
	\subsubsection{Gamma Euler Function}\label{gamma euler function}
	We define the Euler Gamma function (Eulerian integral of the second kind) by the following integral:
	
	with $x$ belonging to the set of complex numbers whose real part is positive and non-zero (thus the positive real number are also included in the domain of definition)! Indeed, if we take complex numbers with a zero or negative real part, the integral diverges and is then undefined!
	
	\begin{tcolorbox}[title=Remark,colframe=black,arc=10pt]
		We have already met this integral and some of its properties (which will be proved here) in our study of the Beta, Gamma, Chi-square, Fisher and Student statistical distribution functions (\SeeChapter{see section Statistics page \pageref{statistical distributions}}). We will also use this integral in maintenance (\SeeChapter{see section Industrial Engineering page \pageref{preventive maintenance}}), in String theory and other engineering and physics fields (see the corresponding chapters) as for the canonical negative binomial generalized linear regression (\SeeChapter{see section Numerical Methods page \pageref{regression techniques}}).
	\end{tcolorbox}
	Consider we want to solve a quite common related practical case:
	
	Set:
	
	Hence:
	
	Arranging we get:
	
	That is:
	
	
	Here is a graphical plot of the module of the Euler Gamma function $\Gamma_{-1}(x)$ for $x$ browsing an interval of real numbers (take care in Maple 4.00b to write GAMMA capitalized!!!):
	
	\texttt{>with(plots):\\}
	\texttt{>plot(GAMMA(x),x=-Pi..Pi,y=-5..5);}
	\begin{figure}[H]
		\centering
		\includegraphics{img/algebra/maple_gamma_euler_2d_plot.jpg}
		\caption{Plot of the Euler Gamma function in Maple 4.00b}
	\end{figure}
	and always the same function with Maple 4.00b but now in the complex plane and always with in ordinate the module of the Gamma Euler function:
	
	\texttt{>with(plots):\\}
	\texttt{>plot3d(abs(GAMMA(x+y*I)),x=-Pi..Pi,y=-Pi..Pi,view=0..5, grid=[30,30],orientation=[-120,45],axes=frame,style=patchcontour);}
	
	\begin{figure}[H]
		\centering
		\includegraphics{img/algebra/maple_gamma_euler_3d_plot.jpg}
		\caption{Plot of the Euler Gamma function in the complex plane with Maple 4.0}
	\end{figure}
	This function is interesting if we impose the variable $x$ to belong to the set of integer numbers and that we write it as follows:
	
	Let us integrate by part the latter function:
	
	Since the exponential function decreases much faster than $t^x$ then we have:
	
	In literature, we frequently find the following notations (there are confusing):
	
	Which brings us to write the result in a more traditional form:
	
	From the relation $\Gamma_{0}(x)=x\Gamma_{0}(x-1)$, it comes by induction:
	
	But:
	
	That gives:
	
	Therefore:
	
	or written in another way for $x\in \mathbb{N}^*$
	
	Another interesting and useful result of the Euler gamma function is obtained when we replace $t$ by $y^2$ and calculate this latter for $x=0.5$.
	
	First we have:
	
	and after:
	
	But, as we have proved it in the section Statistics during our study of the Normal distribution (see page \pageref{Gauss integral}), we recognize here the Gauss integral that is equal for recall to:
	
	Various kinds of relations can be derived using the recurrence relation :
	
	and from the previous result. For example the Gamma function for $n+\frac{1}{2}$, where $n$ in an integer will be very useful to us in the section Statistics for the derivation of the noncentral chi-square distribution and can be derived as follows:
	
	
	\subparagraph{Incomplete regularized Gamma function}\label{incomplete regularized Gamma function}\mbox{}\\\\
	The "\NewTerm{(upper) incomplete regularized Gamma function}\index{incomplete regularized Gamma function}", useful in the study of the A/B testing for count data (\SeeChapter{see section Statistics page \pageref{A/B testing for count datas}}), is defined by:
	
	Obviously the non-regularized version is defined as:
	
	And factoring one $\beta$ it is also sometimes defined as (with the conventional notation) and putting $t=\beta\lambda$:
	
	So whatever which version we choose for the next developments, the results remains the same!
	\begin{tcolorbox}[title=Remark,colframe=black,arc=10pt]
	We also define the (lower) incomplete regularized Gamma function by:
	
	These functions were first investigated by the mathematician Friedrich Prym in 1877, and $Q(\alpha,\beta\lambda)$ has also been named "\NewTerm{Prym's function}\index{Prym's function}".
	\end{tcolorbox}
	Before continuing consider the following special case that will be useful to further below:
	
	And we need also to prove as intermediate result a special recursive relation. For this thanks to an integration by part we prove the following relation:
	
	Indeed:	
	
	and multiplying by $\alpha$ in both side of the equal we proved the previous equality!
	
	Knowing that $\Gamma(\alpha+1) = \alpha\Gamma(\alpha)$ and that by definition $Q(\alpha,t)=\Gamma(\alpha,t)/\Gamma(\alpha)$ and dividing:
	
	 by $\Gamma(\alpha + 1)$ we get:
	
	From the above relation, we can calculate $Q(\alpha+2,t)$ :
	
	and $Q(\alpha+3,t)$ for the fun...:
	
	We show easily by recurrent that the following power recursive relation:
	
	Using the identity $Q(1,t)=e^{-t}$ and the previous recursive relationship we can express $Q$ as:
	
	Indeed, let us put first $\alpha=1$:
	
	We put $n=n+1$ (hence we required now that $n>1$), therefore:
	
	We do another change of variable $k=k+1$:
	
	
	\pagebreak
	\paragraph{Euler-Mascheroni Constant}\mbox{}\\\\
	This small text is a just curiosity regarding to Euler's constant $e$ and to almost every Differential and Integral calculus tools that we have seen until now. This is a very nice example (almost artistic) of what we can do with mathematics as soon as we have enough tools at our disposal.
	
	Moreover, this constant is useful in some differential equations which we see later.
	
	Remember that we saw in the section of Functional Analysis that the Euler constant $e$ is defined by the limit:
	
	
	In a more general case we can easily demonstrate in the same way that (you can ask us the details if needed):	
	
	This obviously suggests:
	
	by a change of variable $t=nu$ we write:
	
	And we use the definition of the Beta function (\SeeChapter{see section Statistics page \pageref{beta function}}):
	
	Therefore:
	
	To transform this expression we can write:
	
	But the quantity:
	
	tends to the limit $\gamma=0.5772$, named "\NewTerm{Euler-Mascheroni constant}\index{Euler-Mascheroni constant}" or also "\NewTerm{Euler Gamma constant}\index{Euler Gamma constant}" when $n$ tends to infinity.
	
	Therefore:
	
	We divide each term of the product $(x+1)...(x+n)$ by the corresponding integer taken into $n!$, so we get (according to a reader request we have put a maximum of details!):
    
	
	\pagebreak
	\subsubsection{Curvilinear Integrals}\label{curvilinear integral}
	The line integrals (curvilinear integrals) are also very important in physics. The reader will thus find see them again in the section of Classical Mechanics, Electrodynamics Magnetostatics and to calculate the work of a force or the "flow field", or in the chapter of Euclidean Geometry to calculate the center of gravity of curves (functions), or in the section of Geometric Shapes to calculate the surface of some bodies of revolution but also in Corpuscular Quantum Physics for the famous "path integral" (which is only the term used by physicists to say "line integral") or for the specific calculation of integrals using the residue theorem proved in the section of Complex Analysis or for transformation states in the section of Thermodynamics. This is why there will not be here examples of line integrals because they are already so many applications in the other chapters of this book.
	
	With the definition of these integrals, we can prove two very important results detailed in section of Vector Calculus that are respectively the Green's theorem and Stokes' theorem or even the theorem of residues proved in the section of Complex Analysis and already mentioned in the preceding paragraph (this is important enough to mention it twice!).
	
	More technically a line integral is an integral where the function to be integrated is evaluated along a curve. The terms "\NewTerm{path integral}\index{path integral}", "\NewTerm{curve integral}\index{curve integral}", and "\NewTerm{curvilinear integral}\index{curvilinear integral}" are also used; "\NewTerm{contour integral}\index{contour integral}" as well, although that is typically reserved for line integrals in the complex plane.
	
	\paragraph{Curvilinear Integral of a scalar field}\mbox{}\\\\
	Consider a parametrized curve $C$ (\SeeChapter{see section Differential Geometry page \pageref{parametric curves}}) by a vector function $\vec{r}(t)$ with $t \in [a,b]$ of class $\mathcal{C}^1$ piecewise (this condition is necessary so that we can integrate the curve without problems).
	
	\textbf{Definitions (\#\mydef):}
	 \begin{enumerate}
	 	\item[D1.] The curve is said to be a "\NewTerm{closed curve}\index{closed curve}" if $\vec{r}(a)=\vec{r}(b)$
	 	
	 	\item[D2.] The curve is said to be a "\NewTerm{smooth curve}\index{smooth curve}" if $\forall t \in [a,b]\; \vec{r}'(t)\neq 0$
	 \end{enumerate}
	 Recall that a parametric curve can be written as follows (all vector function can be written in this form):
	 
	Consider a function or a "\NewTerm{scalar field}\index{scalar field}" $f(x,y)$ defined in a neighborhood of $C$. We subdivide $[a,b]$ into $n$ subintervals of equal length $\Delta t$ as:
	
	We choose on each subinterval a point $t_i^*\in [t_i,t_{i+1}]$. Given $\delta s_i$ the length of the arc $C$ connecting the point $(x(t_i),y(t_i))$ and $(x(t_{i+1}),y(t_{i+1})$, the integral of $f$ along $C$ is defined as the "\NewTerm{line integral}\index{line integral}" or "\NewTerm{path integral}\index{path integral}":
	
	Which as we know, can be written (see section of Differential Geometry page \pageref{curvilinear abscissa helix} of Analytical Mechanics page \pageref{parametric curve length} and many others):
	
	and that can obviously be immediately extended to the case to 3 variables and more.
	
	Or in vector form:
	
	The line integral is linear, that is to say, if $C=C_1 \cup C_2$ and that $C_1 \cap C_2$ is a point, then (without going into the strict definition of the union of two curves...):
	
	
	\paragraph{Curvilinear Integral of a vector field}\mbox{}\\\\
	Consider a vector field (e.g. a force field) as:
	
	and an infinitesimal element of a curve (path) $\mathcal{C}^1$ piecewise as:
	
	The idea is then to consider that the dot product (vector field projection on the path element) represents the work along the differential element:
	
	Therefore the work on all the path will be given by (using the property of linearity of the curvilinear integral):
	
	This can obviously be generalized to $n$ dimensions. Let us indicate that when the line integral (path integral) of a vector field is extended to a closed curve, then we speak of "\NewTerm{circulation of the vector field}\index{circulation of the vector field}".
	
	As:
	
	We then have write a fairly common notation:
	
	\begin{tcolorbox}[title=Remark,colframe=black,arc=10pt]
	In physics many times problems are often in the plane and require the transition to polar coordinates because many academic physic problems are centro-symmetric, which also facilitates the calculations of path integrals.
	\end{tcolorbox}
	
	\begin{tcolorbox}[colframe=black,colback=white,sharp corners]
	\textbf{{\Large \ding{45}}Example:}\\\\
	Let us calculate the work of the force of gravity moving a mass $M$ of the point $M_1(a_1,b_1,c_1)$ to the point $M_2(a_2,b_2,c_2)$ along an arbitrary path $C$. The projections of the force of gravity on the coordinate axes are:
	
	The work accomplish is then:
	
	so we find a very known result of the section of Classical Mechanics.
	\end{tcolorbox}
	A line integral of a vector $\vec{F}$ field along a curve $C_1$ is independent of the path of integration if:
	
	for any non-null curve $C_2$ having only the same points of departure and arrival. Furthermore if the vector field satisfied (where $U$ in physics is typically a potential):
	
	as (the reader will recognize an exact total differential form):
	
	Then the integral path on an arbitrary curve only depends only on the difference of the values the function $U$ at the two ends! This is the "\NewTerm{fundamental theorem for line integrals}\index{fundamental theorem for line integrals}" or "\NewTerm{gradient theorem for line integrals}\index{gradient theorem for line integrals}".
	\begin{dem}
	If the differential form of the vector field satisfies an total exact differential, we have:
	
	That is:
	
	\begin{flushright}
		$\blacksquare$  Q.E.D.
	\end{flushright}
	\end{dem}
	So the line integral of an exact total differential does not depend on the path of integration but only the ends! We also conclude that if $\vec{F}$ isderived from a scalar potential and $A = B$, the line integral is then zero.
	
	In physics this result is interpreted by saying that the work provided by a force $\vec{F}$ derived from a scalar potential acting on an elementary particle in a finite displacement does not depend on the path followed.
	
	\textbf{Definitions (\#\mydef):}
	\begin{enumerate}
		\item[D1.] When the curve (path) $C$ is closed and the path integral has a result independent result of the direction in which this path is traveled, we use the following notation (the letter below the symbol representing the path can of course vary...):
		
		If this closed integral is always zero, we say that the integrated vector field is a "\NewTerm{conservative vector field}\index{conservative vector field}" and "\NewTerm{derives from a scalar potential}" (and therefore satisfies the Schwarz's theorem to be written as exact total differential) since this stems from the proof given already just above.
		
		\item[D2.] \label{closed path orientation}When the value of the integral of a closed path depends on the orientation (clockwise not equal to counterclockwise) we use the following notation (the letter below the symbol representing the path can of course vary...):
		
		Thus if the direction is direct (that is to say "counterclockwise" or "trigonometric") as the notation on the above, its sign will be positive; if on the contrary the direction is clockwise his sign will be negative (see the proof in the section of Complex Analysis). Therefore we often speak about respectively "negative direction" or "positive direction".
		
		Thus, to summarize, a line integral (path integral) is fully defined by the expression under the symbol of the integral, the form of integration path and the direction of the integration.
	\end{enumerate}
	\begin{tcolorbox}[title=Remark,colframe=black,arc=10pt]
	The reader will find some the proofs of very important properties of curvilinear integrals  in section of Vector Calculus as the Green-Riemann theorem or a particular application to study holomorphic functions in the section of Complex Analysis.
	\end{tcolorbox}
	
	\subsubsection{Integrals involving parametric equations}
	Now that we have seen how to calculate the derivative of a plane curve, the next question is this: How do we find the area under a curve defined parametrically? 
	
	To derive an expression for the area under a parametric curve defined by the functions:
	
	with $a\leq t\leq b$.

	We assume that $x(t)$ is differentiable and start with an equal partition of the interval $a\leq t\leq b$. Suppose:
	
	and consider the following figure: $x(t_0),x(t_1),x(t_n)$
	\begin{figure}[H]
		\centering
		\includegraphics{img/algebra/integral_parametric_curve.jpg}
	\end{figure}
	We use rectangles to approximate the area under the curve. The height of a typical rectangle in this parametrization is $y(x(\overline{t_i}))$ for some value $\overline{t_i}$ in the $i$-th subinterval, and the width can be calculated as $x(t_i)-x(t_{i-1})$. Thus the area of the $i$-th rectangle is given by:
	
	Then a Riemann sum for the area is:
	
	Multiplying and dividing each area by $t_i-t_{i-1}$ gives:
	
	Taking the limit as $n$ approaches infinity gives:
	
	And it is obvious that applying Pythagoras's theorem on an infinitesimal length of the parametric curve we have:
	
	with $x=x(t)$, $y=y(t)$ and $t_1<t<t_2$. This gives the arc length\index{arc length} of the parametric curve between two points on the curve.
	
	It comes also immediately:
	
	The chain rule gives:
	
	Therefore:
	
	We will meet again this relation in the section of Analytical Mechanics.
	
	In astronomy we often have to deal with closed curved and calculated the distance travel on that curve. But working in cartesian coordinates is not always is the best. This is why it is better to change in polar coordinates.

	The idea is to suppose that we are able to express our curve of interest in the following form:
	
	where $\alpha\leq \theta \leq \beta$. In order to adapt the arc length relation for a polar curve, we use the relations:
	
	and we replace the parameter $t$ by $\theta$ Then:
	
	we replace $\mathrm{d}t$ by $\mathrm{d}\theta$, and the lower and upper limits of integration are $\alpha$ and $\beta$ respectively. Then the arc length formula becomes:
	
	So finally in polar coordinates:
	
	
	\subsubsection{Improper Integrals}
	Improper integrals are definite integrals where one or both of the boundaries is at infinity, or where the integrand has a vertical asymptote in the interval of integration. As crazy as it may sound, we can actually calculate some improper integrals using some clever methods that involve limits.
	
	By abuse of notation, improper integrals are often written symbolically just like standard definite integrals, perhaps with infinity among the limits of integration. When the definite integral exists (in the sense of either the Riemann integral or the more advanced Lebesgue integral), this ambiguity is resolved as both the proper and improper integral will coincide in value and this is what will the most occur through all applications of integrals in physics, chemistry and engineering through this book!
	
	For the Riemann integral (or the Darboux integral, which is equivalent to it as we have seen earlier above), improper integration is necessary both for unbounded intervals (since one cannot divide the interval into finitely many subintervals of finite length) and for unbounded functions with finite integral (since, supposing it is unbounded above, then the upper integral will be infinite, but the lower integral will be finite)!
	
	An "\NewTerm{improper integral}\index{improper integral}\label{improper integral}" of a function $f(x) > 0$ is given basically by:
	
	And we say that the improper integral converge if this limit exists and diverges otherwise.
	
	More generally, improper integrals are given by the following pairs of possibilities:
	
	or:
	
	in which one takes a limit in one or the other (or sometimes both) endpoints.
	\begin{tcolorbox}[title=Remark,colframe=black,arc=10pt]
	By abuse of notation, improper integrals are often written symbolically just like standard definite integrals, perhaps with infinity among the limits of integration. When the definite integral exists (in the sense of either the Riemann integral or the more advanced Lebesgue integral), this ambiguity is resolved as both the proper and improper integral will coincide in value.
	\end{tcolorbox}
	
	Geometrically then the improper integral represents the total area under a curve stretching to infinity. If the integral $\int_a^{+\infty} f(x)\mathrm{d}x$ converges the total area under the curve is finite; otherwise it's infinite.
	
	How can an area that extends to infinity be finite?  Obviously the area between $a$ and $N$ (i.e. $\int_a^N f(x)\mathrm{d}x$) is finite.  As $N$ goes to infinity this quantity will either grow without bound or it will converge to some finite value. 
	
	The domains where improper integrals are the most used, without even be noticeable most of time by students or engineering practitioners, are respectively:
	\begin{itemize}
		\item Statistics (see corresponding section) when we normalize or check the condition of covergence to a cumulated probability of $1$ of a density function (most of time in statistics one or the both bounderies are equal to infinity)

		\item Wave Quantum Physics (see corresponding section) where we deal sometimes with free propagating particles from infinity to infinity (this also happens sometimes in General Relativity)

		\item Differential equations, especially when we solve them by using Fourier Transform or Laplace transform (we have many examples using this transforms accross the book) for practical application in physics and high level financial engineering.

		\item In astrophysics or electrostatics when dealing with any punctual potential field source of the type $1/r^2$  for which we want to calculate the work necessary to bring an object from infinity to that source (calculation of the type $\int_{+\infty}^{r} f(r)\mathrm{d}r$).
	\end{itemize}
	\begin{tcolorbox}[colframe=black,colback=white,sharp corners]
	\textbf{{\Large \ding{45}}Examples:}\\\\
	E1. We want compute the very important integral (to introduce Laplace Transforms):
	
	with $a\in\mathbb{R}$.\\
	
	Following the definition above we need to first compute a definite integral and the take a limit. So from the definition:
	
	We first compute the definite integral. We start with the case $a=0$:
	
	therefore for $a=0$ the improper integral $I$ does not exist. When $a\neq 0$ we have:
	
	In the case $a<0$, that is $a=-|a|$, we have that:
	
	therefore for $a<0$ the improper integral $I$ does not exist. In the case $a>0$ we have:
	
	\end{tcolorbox}
	
	\begin{tcolorbox}[colframe=black,colback=white,sharp corners]
	
	E2. We want to compute:
	
	The integrand is not continuous at $x=0$ and so we’ll need to split the integral up at that point:
	
	Now we need to look at each of these integrals and see if they are convergent
	
	At this point we're done.  One of the integrals is divergent that means the integral that we were asked to look at is divergent.  We don't even need to bother with the second integral.
	\end{tcolorbox}
	On a side note, notice that the area under the curve of this infinite interval was not infinity as the reader may have suspected it to be (perhaps).  In fact, it was a surprisingly small number.  Of course this won't always be the case, but it is important enough to point out that not all areas on an infinite interval will yield infinite areas.
 
	Let's now get some definitions out of the way.  We will name these integrals "\NewTerm{convergent integrals}\index{convergent integral}" if the associated limit exists and is a finite number (i.e. it's not plus or minus infinity) and "\NewTerm{divergent integrals}\index{divergent integral}" if the associated limit either doesn't exist or is (plus or minus) infinity.
	
	Notice that the similar integral:
	
	cannot be assigned a value in the previous way, as the integrals above and below zero do not independently converge. However there is another possibility that we will see in the section of Analysis during our study of the Hilber transform (see page \pageref{hilbert transform}).
	
	\pagebreak
	\subsubsection{Elliptic Integrals}\label{elliptic integrals}
	"\NewTerm{Elliptic integrals}\index{elliptic integrals}" originally arose in connection with the problem of giving the arc length of an ellipse. They were first studied by Giulio Fagnano and Leonhard Euler. Modern mathematics defines an "elliptic integral" as any function $f$ which can be expressed in the form:
	
	where $R$ is a rational function of its two arguments, $P$ is a polynomial of degree $2$ to $4$ with no repeated roots, and $c$ is a constant.

	In general, integrals in this form cannot be expressed in terms of elementary functions. Exceptions to this general rule are when $P$ has repeated roots, or when $R(x,y)$ contains no odd powers of $t$. However, with the appropriate reduction formula, every elliptic integral can be brought into a form that involves integrals over rational functions and the three Legendre canonical forms (i.e. the elliptic integrals of the first, second and third kind).

	Let us recall that and inform you that we have, and we will in this book, encounter the following Elliptic Integrals:
	\begin{itemize}
		\item Non-Euclidean Geometry page \pageref{elliptic integral riemann space} (complete elliptic integral of second kind):
		
		
		\item Geometric shapes page \pageref{elliptic integral ellipse perimeter} (second-order elliptic integral):
		
		
		\item Classical Mechanics Pendulum \pageref{elliptic integral pendulum} (elliptic integral of the first kind):
		
	\end{itemize}
	
	\paragraph{Incomplete Elliptic Integrals}\label{incomplete elliptic integrals}\mbox{}\\\\
	It is common first to consider three types integrals that we name "\NewTerm{incomplete\footnote{The term "incomplete integral" seems to be an old fashioned way to call a primitive function} elliptic integrals}":
	
	\textbf{Definitions (\#\mydef):}
	\begin{enumerate}
		\item[D1.] The "\NewTerm{incomplete elliptic integral of the first kind}" depends of two parameters, that are the amplitude $\phi$ and the angle $\alpha$:
		
		We can do the following simple change of variables:
		
		Hence:
		
		We then use also sometimes instead the parameter $k=\sin(\alpha)$, ($\leq m \leq 1$), by writing:
		
		named the "\NewTerm{Jacobi's form}" of the elliptic integral.
		
		\begin{tcolorbox}[colframe=black,colback=white,sharp corners]
		\textbf{{\Large \ding{45}}Example:}\\\\
		Using (\SeeChapter{see section Trigonometry page \pageref{remarkable trigonometric identities}}):
		
		Then:
		
		\end{tcolorbox}
		
		\item[D2.] The "\NewTerm{incomplete elliptic integral of the second type}" is define we the same both parameters:
		
		We can also do the following simple change of variable:
		
		and $k=\sin(\alpha)$ to get same as previously:
		
	
		\item[D3.] The "\NewTerm{incomplete elliptic integral of the third kind}":
		
		or:
		
		for $n>0$. We can do the same type of change of variable, and therefore:
		
	\end{enumerate}
	
	\paragraph{Complete Elliptic Integrals}\label{complete elliptic integrals}\mbox{}\\\\
	When the amplitude is equal to $\pi/2$ (hence $t=1$), we write the following "\NewTerm{complete elliptic integrals}":
	
	\textbf{Definitions (\#\mydef):}
	\begin{enumerate}
		\item[D1.] The "\NewTerm{complete elliptic integral of the first kind}" may thus be defined from the incomplete elliptic integral of the first kind:
		
	
		\item[D2.] The "\NewTerm{complete elliptic integral of the second kind}" may thus be defined from the incomplete elliptic integral of the second kind also:
		
	
		\item[D3.] The "\NewTerm{complete elliptic integral of the third kind}" may thus be defined from the incomplete elliptic integral of the third kind also:
		
	\end{enumerate}
	
	\pagebreak
	\subsection{Differential Equations}
	\textbf{Definition (\#\mydef):} In mathematics, a "\NewTerm{differential equation}\index{differential equation}" (D.E.)  is a relationb etween one or more unknown functions and their derivatives up to order $n$. The "\NewTerm{order}\index{order of a differential equation}" of a differential equation corresponding to the maximum degree of differentiation which one of the functions is subjected.
	
	Compared to our goal of trying to see how the math describes the sensible reality, differential equations are a great success but are also the source of many troubles. First there are modeling difficulties (see for example the differential equation system of General Relativity in the corresponding section of the book...) resolution difficulties (there is no general method even with numerical computer methods as you can see in the corresponding section!), then their are proper mathematical difficulties (that's why some D.E. have million dollar price in case or resolution), finally difficulties related to the fact that certain differential equations are unstable by nature and give chaotic solutions (see the section Population Dynamics or Meteorology for flagrant simple examples!).
	
	\begin{tcolorbox}[title=Remark,colframe=black,arc=10pt]
	The differential equations are used to construct mathematical models of physical orbiological phenomena, such as for the study of radioactivity, celestial mechanics, electronic circuits, populatin development or even financial stochastic process. Therefore, differential equations represent a vast field of study, both in pure and Applied Mathematics.
	\end{tcolorbox}
	\begin{fquote}[Steven Strogatz]Since Newton, mankind has come to realize that the laws of physics are always expressed in the language of differential equations.
 	\end{fquote}
	The differential equation of order $n$ the more general can always be written as:
	
	We consider in this book only the cases where $x$ and $y$ have their values in $\mathbb{R}$. A solution to such a D.E. on the interval $I \in \mathbb{R}$ is a function $y \in \mathcal{C}^n (I,\mathbb{R})$ (a function $y:I \rightarrow \mathbb{R}$ which is $n$ times continuously differentiable) such that for any $x \in I$, we have:
	
	\begin{tcolorbox}[title=Remarks,colframe=black,arc=10pt]
	\textbf{R1.} For reasons that will be developed later, we also say "integrate a D.E." instead of saying "finding a solution to the D.E.". The first expression is particularly found in the American literature.\\
	
	\textbf{R2.} Since all this book is full examples of differential equations with initial conditions (we speak then about a "\NewTerm{Cauchy problem}\index{Cauchy problem}") and methods of resolutions in the section of Classical Mechanics, Atomic Physics, Cosmology, Econometry, Sequences and series, Industrial Engineering, Statistics, etc., we will not give any application examples here and will focus only on the minimum theoretical useful aspect.
	\end{tcolorbox}
	
	\pagebreak
	\subsubsection{First order differential equations}\label{first order differential equations}
	A differential equation of the first order is therefore a D.E. which involves only the first derivative $y'$.
	
	\textbf{Definition (\#\mydef):} A first order differential equation is named "D.E. of order 1 with separate variables" if it can be written as:
	
	Such a differential equation can be easily integrated. Indeed, we write:
	
	Then symbolically:
	
	\begin{tcolorbox}[title=Remark,colframe=black,arc=10pt]
	We write explicitly here the arbitrary integration constant $c^{te} \in \mathbb{R}$ (which is normally implicitly present in the indefinite integrals) to not forget it!
	\end{tcolorbox}
	So the purpose is first to find the primitives $F$ and $G$ of $f$ and $g$, and then to express it in terms of $x$:
	
	The integration constant is fixed when asked for a given $x=x_0$, we get a particular value of $y(x)=y(x_0)=y_0$. We speak then of "\NewTerm{initial value problem}\index{initial value problem}" or of "\NewTerm{initial conditions}\index{initial conditions}\label{initial conditions}" also abbreviated: IC. So, in other words, initial conditions are values of the solution and/or its derivative(s) at specific points.  As we will see eventually, solutions to ... nice enough ... differential equations are unique and hence only one solution will meet the given conditions. The number of initial conditions that are required for a given differential equation will depend upon the order of the differential equation as we will see.
	
	\subsubsection{Linear differential equations}\label{linear differential equations}
	\textbf{Definition (\#\mydef):} A differential equation of order $n$ is named "\NewTerm{linear differential equation L.D.E.}\index{linear differential equation}" if and only if it is of the form:
	
	with:
	
	Let us now see a property that may seem insignificant at first glance but which will become very important later!
	
	We will prove now that $L$ is a linear application:
	
	and for all $\lambda \in \mathbb{R}$
	
	Then we say that the linear D.E. represent a linear model if the multiples of this function (or any linear combination) are also a solution. Thus, in physics, for a linear system, the amplification of the cause involves an amplification of the effect (the systems are often linear in high-school problems but in reality they are rather the exception!).
	
	For example, the ordinary differential equation of order $2$ of the simple pendulum proved in the section of Classical Mechanics is not linear because it contains a sine term that is not separable.

	\textbf{Definition (\#\mydef):} The linear differential equation (which is the most common in physics):
	
	is named "\NewTerm{homogeneous equation H.E.}\index{homogeneous equation}" or " \NewTerm{equation without second member E.W.S.M.}\index{equation without second member}" (and sometimes "\NewTerm{complementary equation}\index{complementary equation}") associated to:
	
	\begin{theorem}
	We will now prove an important property of H.E.: the set $\left\lbrace S_0 \right\rbrace$ of solutions of the H.E. is the kernel of the linear application $L$ (which means for refresh: $L(S_0)=0$) and the set  $\left\lbrace S \right\rbrace$ of solutions to $L(y)=f(x)$ is given by:
	
	that is to say that the solutions of the form:
	
	where $y_p$ is a "particular/specific solution" to $L(y)=f(x)$ and $y_h$ the "\NewTerm{homogeneous solution}\index{homogeneous solution}" give all the D.E. solutions.
	\end{theorem}
	\begin{dem}
	The first statement will be assumed obvious.
	
	As regards to the second part, any function of the form $y_p+y_h$ is solution of $L(y)=f(x)$.
	
	Indeed it is trivial and it follows from the definition of the kernel concept (\SeeChapter{see section Set Theory page \pageref{kerr}}):
	
	\begin{flushright}
		$\blacksquare$  Q.E.D.
	\end{flushright}
	\end{dem}
	What is important also to understand with the linear D.E. with second member, it is that if we find solutions to $L(y)$ with a second given member and solutions to the same D.E. with another different second member, then the sum of all these solutions will be a solution of the D.E. with the sum of the second members!!!\label{sum solutions of differential equations}
	
	\subsubsection{Resolution methods of differential equations}
	There are many ways to solve accurately or approximately linear or non-linear differential equations. Let us give the list of the few methods we will analyze further below by the example (but who are already many, many times in the chapters of Mechanics, Cosmology, Social Sciences and Quantum Physics):
	\begin{itemize}
		\item The "\NewTerm{method of characteristic polynomial of D.E.}\index{Differential equations!method of characteristic polynomial}" (see below) used a bit in every section of the chapter of Mechanics/Quantum Physics/Cosmology/Chemistry and Social Sciences of this book.
		
		\item The "\NewTerm{method of integrating factor}\index{Differential equations!method of integrating factor}" (see also below) for general knowledge but used to this date for practical cases in this book.
		
		\item The "\NewTerm{method of variation of the constant}\index{Differential equations!method of variation of the constant}" (see below) and used to this date only in the section of Industrial Engineering in this book.
		
		\item The "\NewTerm{method of disturbances of D.E.}\index{Differential equations!method of disturbances}" (see below) useful for the wave quantum physics and quantum field theory.
	\end{itemize}
	
	Note also other widely used methods (classical high-school technics) but that are treated case by case in the individual sections of this book because solving approaches are too numerous and specific to each problem:
	
	\begin{itemize}
		\item The "\NewTerm{separation of variables method of D.E.}\index{Differential equations!separation of variables}" (the heat equation in the section of Thermodynamics, wave equation in the section Marine \& Weather Engineering, Schrödinger evolution equation in the section of Wave Quantum Physics, vibration of a drum in the section of Wave Mechanics, etc.), whose we will see a very specific and simple case lower but for which it is best to refer to the sections mentioned for concrete examples.
		
		\item The "\NewTerm{matrix method for solving D.E.}\index{Differential equations!matrix method}" and "\NewTerm{trivial solution of D.E.}" (Lotka-Volterra model in the section of Populations Dynamics, electron or nuclear spin resonance in the section of Relativistic Quantum Physics, Lorenz model in the section of Marine \& Weater Engineering, etc.).
		
		\item The "\NewTerm{spectral method}\index{Differential equations!spectral method}" using the spectral theorem proved in the section of Linear Algebra page \pageref{spectral theorem} (see the section of Industrial Engineering for the calculation of system reliability by Markov chains for a concrete example).
		
		\item The "\NewTerm{method of the Fourier transform of the D.E.}\index{Differential equations!Fourier transform method}" or "\NewTerm{method of the Laplace transform of the D.E.}\index{Differential equations!Laplace transform method}" (heat equation in the section Thermodynamics, resolution of the Black \& Scholes equation in the section of Economy, beam equation  under point load in the section of Civil Engineering).
		
		\item "\NewTerm{Numerical methods for D.E.}\index{Differential equations!numerical method}" to solve the differential equations using computer when the D.E. have no known analytic solutions or when they have but we need a visual three dimensional view of the solutions (heat equation in the section of Theoretical Computing).
		
		\item The "\NewTerm{Frobenius method}\index{Differential equations!Frobenius method}" named after Ferdinand Georg Frobenius and  also "\NewTerm{power series solutions}\index{Differential equations!power series solutions}", that is a way to find an infinite series solution for a second-order ordinary differential equation of a special form. We will use this technique in the section of Sequences and Series for our study of the Bessel series and also introduce Bessel series by solving in the section of Mechanical Engineering the probel of the self-buckling column with the power series solutions.
	\end{itemize}
	\begin{tcolorbox}[title=Remark,colframe=black,arc=10pt]
	The first differential equations were solved around the end of the 17th century and beginning of the 18th century. By the middle of the 18th century people realized that the first methods we listed above had reached a dead end. One reason was the lack of functions to write the solutions of differential equations. The elementary functions we use in calculus, such as polynomials, quotient of polynomials, trigonometric functions, exponentials, and logarithms, were simply not enough. People even started to think of differential equations as sources to find new functions. It was matter of little time before mathematicians started to use power series expansions to find solutions of differential equations. Convergent power series define functions far more general than the elementary functions from calculus.
	\end{tcolorbox}
	
	\paragraph{Method of characteristic polynomial}\label{method of characteristic polynomial}\mbox{}\\\\
	Solving simple differential equations (with constant coefficients and without second member most of the time...) uses a technique using a characteristic polynomial of the differential equation which we will see the details in the developments that follow on  few special cases very frequent in physics.
	
	It is a relatively simple method to implement when we seek solutions to the homogeneous differential equation without second member (E.W.S.M.). In the contrary case, the presence of a second member, we add the solutions of the homogeneous equation to the particular solutions.
	
	\subparagraph{Resolution of the H.E. of the first order L.D.E. with constant coefficients}\label{first order lde with constant coefficients}\mbox{}\\\\

	Consider the following L.D.E. with constant coefficients:
	
	which is a simplified version of the following general L.D.E.  with constant coefficients:
	
	where:
	
	We write its associated homogeneous equation (E.W.S.M.):
	
	Which can be written:
	
	Therefore:
	
	There is behind this homogeneous solution infinite solutions: to each value given to the constant $C$ there is a solution.
	
	We still need to add to this homogeneous solution the particular solution $y_p$ and for that we have a collection of recipes, depending on the type of the function $f (x)$ of the second member of the differential equation. We will see in each case in the various chapters this book as already mentioned.
	
	\subparagraph{Resolution of the H.E. of the first order L.D.E. with non-constant coefficients}\mbox{}\\\\
	The general solution of homogeneous linear differential equations (E.W.S.M.) of order $1$ with non constant coefficients:
	
	can always be reduced to the following form:
	
	where:
	
	Well obviously there is the solution $y=0$... but let us try to do better. So we have:
	
	It comes therefore:
	
	where $G (x)$ is a primitive of $g (x)$. Since then:
	
	It is also common to find these developments in another notation a little bit more explicit.
	
	So we start again of the differential equation without second member with non-constant coefficients:
	
	after rearrangement:
	
	And then:
	
	Therefore:
	
	This result will be very useful to calculate the Fourier transform of a Gaussian function (\SeeChapter{see section Sequences and Series page \pageref{fourier transform gaussian function}}), Fourier transform, which is essential to resolve in a fairly general way the heat equation (\SeeChapter{see section Thermodynamics page \pageref{heat equation}}) resolution that will finally allow us to prove the Black \& Scholes equation (\SeeChapter{see section Economy page \pageref{solving black and scholes}}). 
	
	\subparagraph{Resolution of the H.E. of the second order L.D.E. with constant coefficients}\mbox{}\\\\\label{second order differential equations}
	Consider the L.D.E. with constant coefficients:
	
	which is a simplified version of the following general L.D.E. with constant coefficients:
	
	where:
	
	We write is homogeneous associated equation (E.W.S.M.):
	
	wherein the function of the second member is zero. We can quite quickly consider a solution of the type (inspiring of the form of the solutions of the first order differential equations):
	
	where $\tau$ is a constant. Which give us therefore:
	
	What we can simplify by:
	
	If our starting assumption is correct, we only have to solve in $K$ this "\NewTerm{characteristic equation (CHARE)}\index{characteristic equation}\label{characteristic equation}"  or "\NewTerm{characteristic polynomial}\index{characteristic polynomial}" of the homogeneous equation to find the homogeneous solution:
	
	whose solutions depend on the sign of the discriminant of the characteristic polynomial\label{discriminant differential equation}:
	
	\begin{itemize}
		\item If the discriminant is strictly positive, that is to say $\Delta>0$:
		
		So we know that the characteristic polynomial has two distinct roots and then we have:
		
		where $K_1\tau=c^{te}$ and $K_2\delta=c^{te'}$. Then we say that the solution is "\NewTerm{delayed}\index{Differential equations!delayed solution}" or "\NewTerm{advanced}\index{Differential equations!advanced solution}" by the values of these constants. But the key is to note that if $y_h(x)$ is a solution, then $y_h(x\pm \Delta x)$ is always a solution!
		
		We then speak of "\NewTerm{general solution of the homogeneous equation}\index{general solution of the homogeneous equation}". There is behind this result an infinity of solutions: to each value given to the constants $A, B$ corresponds a solution.
		
		Physicists also write sometimes this in a particular form by putting
		
		with then:
		
		And using the hyperbolic trigonometric functions (\SeeChapter{see section Trigonometry page \pageref{hyperbolic trigonometry}}):
		
		where finally the possibility to write the homogeneous solution in the form (when we omit the advance or delay $\delta=\tau=0$):
		
		In addition, let us show that the solutions of the E.W.S.M. form a vector space of dimension $2$ (corresponding to the order of our differential equation)!
		
		Indeed:
		\begin{itemize}
			\item The zero function: $y=0$ is a solution of the E.W.S.M. (this is unnecessary to prove because obvious...!)
			
			\item The sum or subtraction of solutions remains a solution (this we have already proved it before)
			
			\item The elements of the basis of a vector space (the solutions of the E.W.S.M.) are linearly independent (that's an interesting property that we will need later!)
		\end{itemize}
		Let us put:
		
		Then:
		
		These relations injected into the E.W.S.M. in generalized form:
		
		Then gives:
		
		We do have indeed a vector space structure.
		
		Let us recall that conversely two functions are linearly dependent if:
		
		\item If the discriminant is strictly positive, that is to say $\Delta=0$: 
		
		The characteristic equation has a real double root $K$.
		
		By going a little fast we would say then:
		
		and that it is over... but that it is forget that the vector base must be formed of two independent solutions!
		
		So the second option is probably... of the form:
		
		Then:
		
		If we inject it into the E.W.S.M. in generalized form:
		
		Then:
		
		That is to say in our case:
		
		But, both actual real values of $K$ are precisely solutions of:
		
		The prior-previous relation is reduced to:
		
		and as we are in the case of study where the discriminant is zero, we have:
		
		Therefore and finally the prior-previous relation reduces to:
		
		We deduce from it that:
		
		Therefore finally:
		
		Which gives for the general solution of the E.W.S.M.:
		
		
		\item If the discriminant is strictly positive, that is to say $\Delta<0$:
		The characteristic equation has two complex conjugate roots (\SeeChapter{see section Calculus page \pageref{second order polynomial roots}}):
		
		Therefore:
		
		But if we look instead for real solutions, we can always put $A$ and $B$ as being equal such as:
		
		And if we set the delay and advance respectively as being zero ($\delta=\tau=0$), so we find the relation available in most books without proof:
		
		where $A'$ and $B'$ are any two real constants. There is another important form of this last relation (often used in electronics, for example). Indeed, it is possible for any $A'$ and $B' \in \mathbb{R}$ , to find $C'$ and $\phi$ also in $\mathbb{R}$ such as the following equality holds:
		
		We put:
		
		Then:
		
		It is then possible to find $\phi$ such as:
		
		Therefore our initial expression (proposition) can be written as:
		
		Finally:
		
	\end{itemize}
	So we can make the following summary\label{summary LDE with constant coefficients}:
	\begin{table}[H]
		\begin{center}
			\definecolor{gris}{gray}{0.85}
				\begin{tabular}{|c|c|c|}
					\hline
					\multicolumn{1}{c}{\cellcolor{black!30}\textbf{Discriminant}} & 
	  \multicolumn{1}{c}{\cellcolor{black!30}\textbf{Roots}}  & 
	  \multicolumn{1}{c}{\cellcolor{black!30}\textbf{Homogeneous solution}} \\ \hline
					$\Delta>0$ & \centering\arraybackslash\ $K_{1,2}=\dfrac{-a\pm \sqrt{a^2-4b}}{2}$ & $y_h=Ae^{K_1(x+\tau)}+Be^{K_2(x+\delta)}$ \\ \hline
					$\Delta=0$ & \centering\arraybackslash\ $K_{1,2}=\dfrac{-a\pm \sqrt{a^2-4b}}{2}$  & $y_h=(C_1 x+C_2)e^{K(x+\tau)}$ \\ \hline
					$\Delta<0$ & \centering\arraybackslash\ $K_{1,2}=\dfrac{-a\pm \sqrt{a^2-4b}}{2}$  & $y_h=e^{\alpha x}(A' \cos(\beta x)+B'\sin(\beta x))$ \\ \hline
			\end{tabular}
		\end{center}
		\caption{Typical solutions of H.E. of The L.D.E. with constant coefficients}
	\end{table}
	
	\paragraph{Integrating Factor Method (Euler's Method)}\mbox{}\\\\
	The technique of integrating factor is useful when it comes to solve differential equations of the form:
	
	We have not to this day practical application of this technique in the other chapters of this book. You must therefore see this as a presentation for general culture.
	
	The basic idea is to find a function $M(x)$, named "\NewTerm{integration factor}\index{integration factor}", by which can be multiplied by our differential equation to bring the left-hand side of equality to a simple derivative. For example, for a linear differential equation as the one above, we choose often the following integration factor (but this is by far not the only possibility and this choice does not solve everything!):
	
	Therefore we have:
	
	or by distributing:
		
	Which can therefore be seen as:
	
	or even better (and therein lies the whole trick)...:
	
	We can then take the primitive with respect to $x$:
	
	We can then take the primitive with respect to $x$:
	
	and trivially (!) we have the left primitive that is immediate:
	
	Which is sometimes written as:
	
	\begin{tcolorbox}[colframe=black,colback=white,sharp corners]
	\textbf{{\Large \ding{45}}Example:}\\\\
	Consider the following differential equation:
	
	That we will rewrite as:
	
	We see then that (assuming $x$ is positive):
	
	Then we have:
	
	Hazard making  sometimes things good (the example is purposely very simple), we have this equality that simplified as:
	
	in:
	
	Which may be condense in:
	
	By integrating:
	
	It then comes immediately:
	
	Finally:
	
	\end{tcolorbox}
	
	\pagebreak
	\paragraph{Method of separation of variables}\label{separation vaiables method}\mbox{}\\\\
	The method of separation of variables (also known as the "\NewTerm{Fourier method}\index{Fourier method}") is any of several methods for solving ordinary and partial differential equations, in which algebra allows one to rewrite an equation so that each of two variables occurs on a different side of the equation.
	
	In mathematics, a "\NewTerm{partial differential equation PDE}\index{partial differential equation}" is a differential equation that contains unknown multivariable functions and their partial derivatives (a special case are ordinary differential equations, which deal with functions of a single variable and their derivatives). PDEs are used to formulate problems involving functions of several variables, and are either solved by hand, or used to create a relevant computer model.
	
	PDEs can be used to describe a wide variety of phenomena such as sound, heat, electrostatics, electrodynamics, fluid dynamics, elasticity, or quantum mechanics. These seemingly distinct physical phenomena can be formalized similarly in terms of PDEs. Just as ordinary differential equations often model one-dimensional dynamical systems, partial differential equations often model multidimensional systems. PDEs find their generalization in stochastic partial differential equations.
	
	A partial differential equation (PDE) for the function $U(x_{1},\cdots ,x_{n})$ is an equation of the form:
	
	If $f$ is a linear function of $U$ and its derivatives, then the PDE is named a "\NewTerm{linear partial differential equation}\index{linear partial differential equation}". Common examples of linear PDEs include the heat equation, the wave equation, Laplace's equation, Helmholtz equation, Klein–Gordon equation, and Poisson's equation (see the chapters of Mechanics, Electrodynamics and Quantum Physics for the study of most of them!).
	
	The method of separation of variables is a very common technique used in physics when we have second-order differential equations. Many useful examples and very detailed are already in the various chapters already mentioned above. Here we will just present a special case just for doing things good but with the minimum subsistence level!
	
	Consider the common case of physical partial differential equation of the type:
	
	The solution of this equation therefore requires finding a function $U$ which depends on $x$ and $y$ such that:
	
	In physics, the idea is then to put that we can always find a said a "separable" solution of the form:
	
	Thus, the differential equation can be written as:
	
	Which can be rewritten as:
	
	Or:
	
	After rearrangement it is use in physics to note this last equality in condensed form:
	
	This equality can only take place if each term is a constant since $X$ depends only on $x$ and $Y$ only on $y$. It comes then:
	
	And each differential equation can then be solved independently of the other and once the solutions found we multiplied them determine the expression of $U$.
	\begin{tcolorbox}[title=Remark,colframe=black,arc=10pt]
	Probably the most beautiful example is the section of Quantum Chemistry.
	\end{tcolorbox}
	How do we know that this technique of separation of variables is valid? When the differential operators in various variables are additive in the partial differential equation, that is, when there are no products of differential operators in different variables, the separation method usually works. We are proceeding in the spirit of \textit{let’s try and see if it works}... If our attempt succeeds, then this technique will be justified. If it does not succeed, we shall find out soon enough and then we shall try another attack, such as Green’s functions, integral transforms, or brute-force numerical analysis!
	
	\paragraph{Method of constant variation}\mbox{}\\\\
	The variation of constants, is a general method to solve inhomogeneous linear ordinary differential equations.

	For first-order inhomogeneous linear differential equations it is usually possible to find solutions via integrating factors or undetermined coefficients with considerably less effort, although those methods leverage heuristics that involve guessing and don't work for all inhomogeneous linear differential equations.

	Variation of parameters extends to linear partial differential equations as well, specifically to inhomogeneous problems for linear evolution equations like the heat equation, wave equation, and vibrating plate equation. In this setting, the method is more often known as "\NewTerm{Duhamel's principle}\index{Duhamel's principle}", named after Jean-Marie Duhamel who first applied the method to solve the inhomogeneous heat equation.
	
	The idea of the method of variation of the constant is as follows: if we have a particular solution affected by constants, we know that depending of the initial conditions thereof are well determined. The idea is then to generalize by putting that these constants are functions. In some cases obviously mathematical developments will show that the functions are necessarily constant.
	
	The idea behind this method is to say that the solutions of the (linear) differential equation with the second member will look like the solutions of the homogeneous equation. As the term on the right will disrupt this solution, we vary only constants (which will therefore no longer be constants), but we remain on the "base" of homogeneous solutions, to seek close solutions. After we check that this "physicist reasoning" gives out all the solutions of the differential equation.
	
	Let us see before studying to the general case a simple example by considering the following differential equation:
	
	for which the particular solution of the homogeneous equation (E.W.S.M.) is (if you need the details not hesitate to ask!):
	
	The method of variation of the constant consist then to put that:
	
	and therefore:
	
	But by the differential equation with the second member, we have:
	
	
	and it follows that:
	
	where we eliminated the integration constant because what we want is a particular solution! The particular general solution (pg) is then the sum of the particular solution and the homogeneous one and this with the variation of the constant:
	
	Thus, generalizing the previous example, so we have a differential equation of the form:
	
	General particular solution will be:
	
	Then we have:
	
	hence injected into the original differential equation:
	
	Therefore after factoring similar terms:
	
	So we have the above relation and the particular solution to the homogeneous differential equation (therefore without second member):
	
	Therefore we find:
	
	and it sufficient then to integrate this equation to find $C_0(t)$. Then, the particular general solution (pg) will be the sum of the particular homogeneous solution and that with the variation of the constant.
	
	\pagebreak
	\subsubsection{Classification of partial differential equations}
	Before we begin, the reader wonders what classifying differential equations can be used for, well, here are the two main arguments for the usefulness of a classification in the order of importance:
	\begin{itemize}
		\item Many books have authors who systematically speak of a differential equation by categorizing it, so it is more pleasant to know what they are talking about

		\item Some finite element modeling software (including MATLAB™) requires that the differential equation category be chosen before anything else can be done as a calculation.
	\end{itemize}
	So this being said, formally, we name "\NewTerm{partial differential equation PDE}\index{partial differential equation}" of order less than or equal to $2$ in a domain $\Omega \subset \mathbb{R}^n$ and of unknown:
	
	an equation of the following general type:
	
 	where $s(x)$ is often named a "\NewTerm{source term}\index{source term}" in analogy with the main situation in physics concerned.
	\begin{figure}[H]
		\centering
		\includegraphics[scale=0.5]{img/algebra/ode_vs_pde.jpg}
	\end{figure}
	It is now important to generalize the latter equation in vector form. Thus, we introduce the following notations:
	\begin{itemize}
		\item $A=[a_{ij}]$ the $n\times n$ symmetric matrix of the coefficients in front of the terms of order $2$

		\item $B=(f_i(x))$ The vector of size $n$ of the coefficients in front of the terms of order $1$

		\item $[H\Phi(x)]$ the $n\times n$ symmetric Hessian matrix (\SeeChapter{see section Sequences and Series page \pageref{hessian matrix}}) of $\Phi$:
		

		\item $\vec{\nabla}(\Phi(x))$ the vector of size $n$ of $\Phi$:
		

		\item The notation (named "\NewTerm{Frobenius (matrix) dot product}\index{Frobenius (matrix) dot product}"):
		
		also sometimes denoted as the classical dot product: $\langle A,B\rangle$.
		\begin{tcolorbox}[title=Remark,colframe=black,arc=10pt]
		The "\NewTerm{Frobenius norm}\index{Frobenius norm}" of a matrix $\mathbb{R}^{m\times n}$ is given by:
		
		\end{tcolorbox}
	\end{itemize}
	With this, the previous relation:
	
	can be rewritten as:
	
	What is often (abusively) written in a very condensed way:
	
	\begin{tcolorbox}[colframe=black,colback=white,sharp corners]
	\textbf{{\Large \ding{45}}Example:}\\\\
	The following PDE:
	
	First, it is easy to determine (because it is a definition) that:
	
	It is also trivial that:
	
	and that:
	
	It is also trivial that:
	
	Finally, the minor difficulties are to find:
	
	\end{tcolorbox}
	For linear PDE of order $2$, the matrix $A$ is non-zero and is symmetric. It is therefore diagonalizable with real eigenvalues (\SeeChapter{see section Linear Algebra page \pageref{eigenvector}}) and their study provides elements of classification of linear PDE systems of order $2$ under the denomination of "Elliptic", "hyperbolic" or "parabolic" PDE (\SeeChapter{see section Analytical Geometry page \pageref{classification of conical by the determinant}}).

	It is often customary to say in physics that the elliptics PDE characterize problems of equilibrium or stationarity, that hyperbolics PDE characterize problems of wave propagations and finally that parabolic PDE characterize diffusion problems (see examples further below).
	
	That latter terminology comes from the fact that when the matrix $A$ is constant, the curves:
	 
	are respectively ellipsoids, hyperboloid and paraboloid (\SeeChapter{see section Analytical Geometry page \pageref{type of conics matrix approach}}).

	Indeed, we have proved in the section of Analytic Geometry that following the determinant of $A$, we had:
	\begin{itemize}
		\item If $\det(A)=ac-b^2>0$, the curve $\Gamma$ is either empty, reduced to a point, or an ellipse.

		\item If $\det(A)=ac-b^2<0$, the curve $\Gamma$ is either the union of two intersecting lines, that is to say an hyperbola.

		\item If $\det(A)=ac-b^2=0$, the curve $\Gamma$ is either empty, a line, or two distinct parallel lines, or a parabola.	
	\end{itemize}
	More explicitly what we have seen above can be reformulated as following. If we have the following PDE:
	
	where $a$, $b$, $c$, $d$, $e$ and $f$ are real constants is said to be:
	\begin{itemize}
		\item An "\NewTerm{elliptical partial differential equation}\index{elliptical partial differential equation}" if $ac-b^2>0$ that is to say if the eigenvalues are all positive or all negative (\SeeChapter{see section Analytical Geometry page \pageref{type of conics determinant}}).

		\item An "\NewTerm{hyperbolic partial differential equation}\index{hyperbolic partial differential equation}" if $ac-b^2<0$ that is to say there only one negative eigenvalue and all the rest are positive, or there is only one positive eigenvalue and all the rest are negative (\SeeChapter{see section Analytical Geometry page \pageref{type of conics determinant}}).

		\item A "\NewTerm{parabolic partial differential equation}\index{parabolic partial differential equation}\label{parabolic partial differential equation}" if $ac-b^2=0$ that is to say if the eigenvalues are all positive or all negative, save one that is zero (\SeeChapter{see section Analytical Geometry page \pageref{type of conics determinant}}).
	\end{itemize}

	\begin{tcolorbox}[colframe=black,colback=white,sharp corners]
	\textbf{{\Large \ding{45}}Example:}\\\\
	The Laplace's equation:
	
	is an elliptic PDE.\\

	The wave equation:
	
	is a hyperbolic PDE.\\

	The heat equation:
	
	is a parabolic PDE.
	\end{tcolorbox}
	
	\pagebreak
	\subsection{Systems of Differential Equations}
	Let us study now special developments that will also be useful for the study of quantum physics or for the resolution of particular systems of differential equations (see corresponding section in this book for the details on these examples) and especially one which is well known in chaos theory!
	
	Let us first indicate to the reader before going further that more complex inhomogeneous case (with second member) and with unknown coefficients is treated directly by an example in the section of Industrial Engineering during the study of the reliability of a repairable system as a Markov chain with resolution using the determinants and eigenvalues/eigenvectors.
	
	To start this first approach, we will have to introduce the concept of exponentiation of a matrix:

	The set of matrices $n \times n$ with coefficients in $\mathbb{C}$ denoted $M_n(\mathbb{C})$ is a vector space for the addition of matrices and multiplication by a scalar. We will as always denote by $\mathds{1}_n$ the identity matrix of dimension $n$.
	
	We will admit that a sequence of matrices $A_n$ converges to a matrix $A$ if and only if the sequences of coefficients of the matrices $A_n$ converge towards the corresponding coefficients of $A$.
	
	\begin{tcolorbox}[colframe=black,colback=white,sharp corners]
	\textbf{{\Large \ding{45}}Example:}\\\\
	In $M_2(\mathbb{C})$ the sequence of matrice:
	
	converge to:
	
	when $n\rightarrow +\infty$.
	If $x\in \mathbb{C}$, we saw in our study of complex numbers (\SeeChapter{see section on Numbers page \pageref{taylor expansion complex exponential}}) that the series:	
	
	does converge and its limit is denoted by $e^x$. In fact here there is no difficulty in replacing $x$ by a matrix $A$ since we know (we have proved it during our study of complex numbers) that any complex number can be written as follows (the body of complex numbers is isomorphic to the field of real matrices of square dimensions $2$ having this form):
	
	\end{tcolorbox}
	\pagebreak
	\begin{tcolorbox}[colframe=black,colback=white,sharp corners]
	and that a complex number is equivalent to put his matrix form also to the square:
	
	Indeed:
	
	\end{tcolorbox}
	We then define the "\NewTerm{exponential of a matrix}\index{exponential of a matrix}\label{exponential of a matrix}" $A\in M_n(\mathbb{C})$ as the matrix limit of the sequence:
	
	If the matrix $A$ is diagonal obviously its exponential is easy to calculate. Indeed, if:
	
	It follows:
	
	This property of diagonal matrices will also be very useful to us in the section of General Relativity during our study of the variation of the metric determinant.
	
	However, it is clear that a non-diagonal matrix will be much more complicated to deal with! We will then use a diagonalization technique or an endomorphisms reduction\index{endomorphism reduction}\footnote{In linear algebra, an "endomorphism reduction" aims to express matrices and endomorphisms in a simpler form, for example to facilitate calculations. This consists essentially in finding a decomposition of the vector space into a direct sum of stable subspaces on which the induced endomorphism is simpler. Less geometrically, this corresponds to finding a basis of the space in which the endomorphism is simply expressed like for example the spectral theorem (\SeeChapter{see section Linear Algebra page \pageref{spectral theorem}}).} (\SeeChapter{see section Linear Algebra page \pageref{matrix endomorphism}}) to simplify our problem.
	
	So notice that if $S\in M_n(\mathbb{C})$ is reversible and if $A\in M_n(\mathbb{C})$ then:
	
	This follows from the fact that (think to the change of basis of a linear application as it has been studied in the section of Linear Algebra page \pageref{change of basis}... perhaps it may help):
	
	Therefore:
	
	This development will enable us to bring the computing of the exponential of a diagonalizable matrix in search of its eigenvalues and its eigenvectors.
	\begin{tcolorbox}[colframe=black,colback=white,sharp corners]
	\textbf{{\Large \ding{45}}Examples:}\\\\
	Let us calculate $e^A$ where:
	
	The eigenvalues of $A$ are $\lambda_1=-3,\lambda_2=7$, and associated eigenvectors are:
	
	Indeed:
	
	By putting:
	
	We get:
	
	with:
	
	Therefore:
	
	\end{tcolorbox}
	Now, let us recall that in the case of real numbers we know that if $x,y\in \mathbb{R}$ then:
	
	In the case of matrices we can prove that if $A,B\in M_n(\mathbb{C})$ are two matrices that commute with one another, that is to say such that $AB=BA$, then:
	
	The condition of commutativity comes from the fact that the addition in the exponential is itself commutative. The proof is therefore intuitive.
	
	An important corollary of this proposition is that for any matrix $A\in M_n(\mathbb{C})$, $e^A$ is reversible. Indeed the matrices $A$ and $-A$ commute and therefore:
	
	We recall that a matrix $A$ with complex coefficients is unitary if:
	
	The following theorem will serve us later:
	\begin{theorem}
	Let us prove that if $A$ is a Hermitian matrix (also named "self-adjoint") (\SeeChapter{see section Linear Algebra page \pageref{self-ajdoint matrix}}) then for any $t\in\mathbb{R}$, $e^{\mathrm{i}tA}$ is unitary.
	\end{theorem}
	\begin{dem}
	
	Therefore:
	
	\begin{flushright}
		$\blacksquare$  Q.E.D.
	\end{flushright}
	\end{dem}
	Remember that this condition for a self-adjoint matrix is linked to the definition of linear group of order $n$ (\SeeChapter{see section Set Algebra page \pageref{unitary linear group}}).
	
	One of the first applications of the exponentation of matrices is the resolution of ordinary differential equations. Indeed, from the linear differential equation below using as initial condition $y(0)=0$ and where $A$ is a matrix:
	
	the solution is given by (as seen previously):
	
	We frequently find that kind of systems of differential equations in biology (population dynamics), astrophysics (study of plasmas) or in fluid mechanics (chaos theory) and in classical mechanics (coupled systems), astronomy (coupled orbits), in electrical engineering, etc.
	
	\begin{tcolorbox}[colframe=black,colback=white,sharp corners]
	\textbf{{\Large \ding{45}}Example:}\\\\
	Suppose we have the following homogeneous system of differential equations (without constant terms):
	
	The associated matrix is then:
	
	and its exponential (see developments made above):
	
	The general solution of the system is:
	
	So we have:
	
	By calculating the derivative of the previous relations and comparing to:
	
	we easily determine the constants to get:
	
	which finally gives us:
	
	\end{tcolorbox}

	\subsection{Regular Methods of Perturbations}\label{regular methods of perturbations}
	Very frequently in physics (high level physics) or financial engineering, a mathematical problem can not be solved exactly. Even if the solution is known sometimes there are such a dependency of parameters that the solution is difficult to use as such.
	
	Sometimes, however, it happens that an identified parameter of the differential equation, which we denote by tradition with the Greek letter $\varepsilon$, is such that the solution is available and reasonably simple for $\varepsilon=0$.
	
	The problem then is to know how the solution is altered for a non-zero $\varepsilon$ but still small. This study is the center of "\NewTerm{perturbation theory}\index{perturbation theory}" that we will use, for example in the section of General Relativity to calculate the precession of the perihelion of Mercury.
	
	As the perturbation theory within the general framework is too complex for this book purpose, we propose an approach by example with first a simple algebraic equation and then with what interests us: a differential equation.
	
	\subsubsection{Perturbation theory for algebraic equations}
	Consider the following polynomial equation:
	
	We know from our study of the section Functional Analysis, that this polynomial equation has two roots that are trivially:
	
	For small $\varepsilon$, these roots can be approximated by the first term of the Taylor series expansion (\SeeChapter{see section Sequences and Series page \pageref{taylor series}}):
	
	The question is whether we can get the two previous relation without a priori knowledge of the exact solution of the initial polynomial equation? The answer is obviously: YES with the help of perturbation theory.
	
	The technique is based on four steps:
	\begin{enumerate}
		\item In the first step, we assume that the solution of the polynomial equation is an expression of the type Taylor series on $\varepsilon$. Then we have:
		
		where $X_1,X_2,X_3$ are obviously to be determined.
		\item In the second step, we inject the solution in our hypothetical polynomial equation:
		
		As:
		
		and:
		
		It finally comes that the polynomial equation can be written as:
		
		
		\item In the third step we successively equalize the terms with 0 such as:
		
		\item Fourth and last step, we solve successively the polyinomial equations above to get:
		
		By injecting these results in the hypothetical solution:
		
		it is obvious to observe that we fall back on the certain solution:
		
	\end{enumerate}
	
	\pagebreak
	\subsubsection{Perturbation theory of differential equations}
	Perturbation theory is therefore also often used to resolve numerous differential equations. This is the case for example in fluid mechanics, in General Relativity or quantum physics.
	
	Again, rather than doing a super abstract and general theory, we will see the concept with an example as previously.
	
	Consider the following ordinary differential equation with second member and constant coefficients:
	
	Or written in another way?
	
	with the boundary conditions:
	
	The exact resolution is relatively easy to obtain:
	
	First we start with the homogeneous equation:
	
	So it is a linear differential equation of order 2 with constant coefficients, equation that it is relatively easy to solve in the general case. Given the equation:
	
	Assume that the function $y$ which satisfies this differential equation is the of the form $y=e^{Kx}$ where $K$ may be a complex number. Then we have:
	
	provided, of course, that $e^{Kx}\neq 0$. This last relation is the auxiliary quadratic equation of the differential equation (characteristic polynomial in other words). It has two solutions/roots (it's a simple resolution of a polynomial of the second degree) which we denote in the general case: $K_1,K_2$. Which means that:
	
	are satisfied for the two roots. If we do the sum, since both are equal to the same constant:
	
	Thus, it is immediate that the general solution of the homogeneous equation is of the type:
	
	where $A, B$ are obviously constants to be determined. Now we solve the characteristic polynomial:
	
	It comes immediately that:
		
	Therefore:
	
	Now a particular solution to:
	
	is relatively trivially a solution of the type:
	
	where $B$ is of course a constant to be determined and which is equal (once injected into the differential equation):
	
	Therefore:
	
	Hence finally the general solution:
	
	Then, with the initial conditions that are a for reminder:
	
	it is very easy to find A:
	
	We also have:
	
	We are free to choose that $c^{te}=0$ which gives us:
	
	Then:
	
	becomes:
	
	Now that we have the general solution if $\varepsilon$ is small we can take the development of order $4$ in Maclaurin series of the exponential (\SeeChapter{see section Sequences and Series page \pageref{euler maclaurin expansion}}). Such as:
	
	Injected into $y$ this gives (you will notice that we sometimes express explicitly... the term of order 5 by anticipation...):
	
	Now that we have this development, what we want to show is that from a perturbative expansion we can find the same result in series and this without any prior knowledge on the solution.
	
	Again, the development is done in 4 steps:
	\begin{enumerate}
		\item In the first step, we assume that the solution of differential equation is an expression of the type Taylor series on  $\varepsilon$. Then we have:
		
		where $y_0,y_1,y_2$ are obviously to be determined.
		
		\item In the second step, we inject the hypothetical solution of our differential equation in itself with the initial conditions and we develop the whole.
		
		then the initial conditions:
		
	
		\item In the third step we equalize successively the terms with $0$ such as:
		
		
		\item In the fourth step we solve the differential equations listed above (if you do not see how we solve them do not hesitate to contact us!):
		
		By injecting relations in the supposed solution developed in Taylor series and injected into the differential equation:
		
		We fall back on:
		
	\end{enumerate}
	
	\begin{flushright}
	\begin{tabular}{l c}
	\circled{95} & \pbox{20cm}{\score{4}{5} \\ {\tiny 119 votes,  75.45\%}} 
	\end{tabular} 
	\end{flushright}
	
	%to make section start on odd page
	\newpage
	\thispagestyle{empty}
	\mbox{}
	\section{Sequences and Series}
	
\lettrine[lines=4]{\color{BrickRed}S}equences and series have a great importance in Applied Mathematics and that is why we devote to them a whole section. We will also see them often in various sections of the Mechanics chapters when we need to make some minor approximations (...) as well as in the sections of Economy and Quantitative Management Techniques. The reader should try to not confuse in what follows the concept of "sequences" with that of "series" which, while being similar in substance, are not always analyze mathematically in the same way.

We wanted to study in this section simple things without going to far within the topological concepts of sequences and series. However, those interested in more rigorous definitions can read the sections Fractals (\SeeChapter{see chapter of Theoretical Computing page \pageref{fractals}}) and Topology where many concepts about series are  (supremum, infimum, subsequence, Bolzano-Weierstrass' theorem, etc.).

\subsection{Sequences}

\textbf{Definition (\#\mydef):} A "\NewTerm{sequence}\index{sequence}" of a set is a family of elements indexed by the set of natural numbers (\SeeChapter{see section Numbers page \pageref{natural numbers}}) or by a part of it. In a vulgarized way we say that a sequence is a list of objects put in order, each with a order number. We typically write a sequence as:
	
where indexing is sometimes (by tradition...) without the 0.

For some sequences, we provide the first term $u_1$ (if indexing starts with 1 instead of 0), and a formula for any term $u_{n+1}$ from the previous term $u_n$ regardless $n \geq 1$. We call such a formulation a "\NewTerm{recurring definition}\index{recurring definition}" and the sequence is defined "\NewTerm{recursively}\index{recursively defined sequence}" (and even if it is indexed from 0 instead of 1).

Before seeing some examples of sequences families that will be used in the various sections of the book (Population Dynamic, Economy, Nuclear Physics, etc.) let us see a small set of definitions as it is the tradition in mathematics...

\textbf{Definitions (\#\mydef):}
	\begin{enumerate}
		\item[D1.] Numbers (as sequence) are in "\NewTerm{arithmetic progression}\index{Sequence!arithmetic progression}" if the difference of two consecutive terms is equal to a constant $r$ named the "\NewTerm{reason}".
		
		\item[D2.] Numbers (as sequence) are in "\NewTerm{geometric progression}\index{Sequence!geometric progression}" if the ratio of two consecutive terms is equal to a constant $r$ also named the "\NewTerm{reason}".
		
		\item[D3.] Numbers (as sequence) are in "\NewTerm{harmonic progression}\index{Sequence!harmonic progression}" if the inverse of two consecutive terms are in arithmetic progression.
	\end{enumerate}

	Therefore, a number $b$ is respectively the arithmetic mean, geometric, harmonic of $a$ and $c$ if the numbers $a, b, c$ are respectively in  an arithmetic, geometric or harmonic progression.

	\begin{tcolorbox}[title=Remark,colframe=black,arc=10pt]
For the definitions of the averages listed above see the section Statistics.
	\end{tcolorbox}
	
\textbf{Definitions (\#\mydef):}
	\begin{enumerate}
		\item[D1.] A "\NewTerm{majorated sequence}\index{majorated sequence}" is a sequence such that there is a real number $M$ such that:\\ $\forall n \in \mathbb{N}, \; u_n \leq M$
		
		\item[D2.] A "\NewTerm{minorated sequence}\index{minorated sequence}" is a sequence such that there is a real number $m$ such that:\\ $\forall n \in \mathbb{N}, \; u_n \geq m$
		
		\item[D3.] A "\NewTerm{bounded sequence}\index{bounded sequence}" is a sequence that is both majorated and minorated.
		
		\item[D4.] A sequence $(u_n)$ is named  "\NewTerm{increasing sequence}\index{increasing sequence}" if $\forall n \in \mathbb{N}, \; u_{n+1}-u_n > 0$
		
		\item[D5.]  A sequence $(u_n)$ is named  "\NewTerm{decreasing sequence}\index{decreasing sequence}" if $\forall n \in \mathbb{N}, \; u_{n+1}-u_n < 0$
		\begin{tcolorbox}[title=Remark,colframe=black,arc=10pt]
	If a sequence is increasing or decreasing, we sometimes just say it is a "\NewTerm{monotonous sequence}" (without specifying if its increasing or decreasing).
		\end{tcolorbox}
		
		\item[D6.]  A sequence $(u_n)$ is named  "\NewTerm{constant sequence}\index{constant sequence}" if $\forall n \in \mathbb{N}, \; u_{n+1}=u_n$
	\end{enumerate}
	
	We will now see some practical important arithmetic and geometric sequences that will be used later in other sections of this book.
	
	But keep in mind during the lecture that we can express sequences mot of time in two ways:
	\begin{itemize}
		\item A "closed formula" or "closed relation" for a sequence $(u_n)_{n\in\mathbb{N}}$ is a relation for $u_n$ using a fixed finite number of operations on $n$ only (and nothing else!). This is what we normally think of as a function of $n$ only,
	
		\item A "recursive definition" for a sequence $(u_n)_{n\in\mathbb{N}}$ consists of a recurrence relation: a relation relating a terme of the sequence to previous terms (terms with smaller index $n$) and an initial condition.
	\end{itemize}
	
	\pagebreak
\subsubsection{Arithmetic Sequences}

\textbf{Definition (\#\mydef):} We say that numbers or "\NewTerm{terms}\index{term of a sequence}" are in an "\NewTerm{arithmetic sequence}" when the difference of their sequential value is equal to a constant $r$ named the "\NewTerm{reason}\index{reason of an arithmetic sequence}" of the sequence so that (recursive relation):
	
where $r$ is the "reason" of the progression. We then obviously have if the indexing starts from $0$:
	
	\begin{tcolorbox}[colframe=black,colback=white,sharp corners]
\textbf{{\Large \ding{45}}Examples:}\\\\
E1. The sequence:
	
where $n$ is a constant and the reason is equal $1$.\\\\
E2. The sequence:
	
	is an arithmetic sequence of reason $x$.
	\end{tcolorbox}
	Thus, if we write $u_n$ any term of the sequence $(u_n)$ of reason $r$, we have:
	
	We have the following properties for this type of sequences:
	\begin{enumerate}
		\item[P1.] A term whose rank is the average of the ranks of the other two terms is the arithmetic mean of these two terms.
		\begin{dem}
		Consider now $(u_n)$ an arithmetic sequence of reason $r$ given by the previous development:
			
and $a,b,k \in \mathbb{N}$ such as $a+b=2k$, then we have:
			
and so:
		
with $k=\dfrac{a+b}{2}$.
		\begin{flushright}
			$\blacksquare$  Q.E.D.
		\end{flushright}
\end{dem}
	\item[P2.] For three consecutive terms $u_n,u_{n+1},u_{n+2}$ in an arithmetic sequence of reason $r$, the second term is the arithmetic mean of the other two.
		\begin{dem}
			Let us write:
				
				\begin{flushright}
					$\blacksquare$  Q.E.D.
				\end{flushright}
		\end{dem}
	\item[P3.] If $u_1,u_2,u_3,...,u_n,...$ is an arithmetic sequence of ratio $r$, then the $n$-th partial sum $S_n$ (that is to say, the sum of the first $n$ terms to the power of $1$) is given by:
		
when indexing is starts from $1$.
		\begin{dem}
			We can write the sequence:
				
		Playing with the second line, we get:
			
What can be simplified even more:
			
Considering that we will prove a little bit later that the simple following Gauss series:
			
is equal to:
			
We then have for:
			
the following relation:
			
We thus get:
			
We see with the latter relation that if $u_1=r=1$ we fall back on the simple Gauss series.

As:
			
when the indexation starts from 1 we thus get:
			
			\begin{flushright}
				$\blacksquare$  Q.E.D.
			\end{flushright}
		\end{dem}
We will see other types of summations a little bit further below during our study of series!
	\end{enumerate}
	
	\subsubsection{Harmonic Sequences}
	\textbf{Definition (\#\mydef):} We say that numbers $\dfrac{1}{a}, \dfrac{1}{b}, \dfrac{1}{c},...$ generates an "harmonic progression" when their inverses are in arithmetical progression (also with a "reason" $r$. We represent this progress by:
	
We then obviously if the indexing starts from 0:
	
Moreover, we assume, in what follows, that there is no zero denominator.

By sharing this type of sequences successively in groups containing $2^n$ terms, we observe that each of them is bigger than the last of his group. For example:
	
And we can see that the sum of the terms of each group is larger than 1/2.

We can also see that each term is the harmonic mean of the previous and consecutive one:

\begin{dem}
	
Thus:
	
So finally:
		
	\begin{flushright}
		$\blacksquare$  Q.E.D.
	\end{flushright}
\end{dem}

	\subsubsection{Geometric Sequences}
\textbf{Definition (\#\mydef):} A "\NewTerm{geometric sequence}\index{geometric sequence}\label{geometric sequence}" is a sequence of numbers such that each of them is equal to the previous $n$ multiplied by a constant number $q$ that we also name the "\NewTerm{reason}\index{term of a geometric sequence}" or "\NewTerm{common ratio}\index{common ratio}" of the sequence. We will denote by:
	
Thus, if we denote by $u_n$ any term of the sequence $u_n$, we have (trivial):
	
Here are some properties for such a type of sequence (without proof until now... except if some readers ask for them because most are really trivial):
	\begin{enumerate}
		\item[P1.] (trivial) The quotient of two terms of the same sequence is a power of the reason $q$ whose exponent equals the difference in rank of the two terms chosen (simple ratio of two same bases with different powers).
		\item[P2.] (trivial) If we multiply or divide term by term two geometric sequences, we get a third geometric sequence whose raison equal the product (respectively the quotient) of the reasons of the two chosen sequences (simple operation with the reasons of the two original sequences).
		\item[P3.] In a geometric sequence, a term whose rank is the average of the ranks of the other two terms is the geometric mean (\SeeChapter{see section Statistics page \pageref{geometric mean}}) of these two terms (reread many times if needed...).
	\end{enumerate}
Let us prove the property P3:
\begin{dem}
	Given a geometric sequence with real positive reason $q$, we have for recall:
	
Let $a, b$ be the ranks of two terms of the geometric sequence, then we have:
	
and thus:
	
	\begin{flushright}
		$\blacksquare$  Q.E.D.
	\end{flushright}
\end{dem}

	\begin{corollary}
	We have as corollary that for three consecutive terms $n,n+1,n+2$in a geometric progression, the second term is the geometric mean of the other two.
	\end{corollary}
	\begin{dem}
		We have:
		
Thus:	
		
	\begin{flushright}
		$\blacksquare$  Q.E.D.
	\end{flushright}
	\end{dem}
However, there are some special sequences that have special properties that we find very frequently in mathematical or theoretical physics in this book. Without going into  too much detail, here's a partial list of with the important proofs that we will have to use later:

	\subsubsection{Cauchy Sequence}\label{cauchy sequence}

It is often interesting for the mathematician, as much as for the physicist, to know the properties of a sequence with a given type of progression. The most important property is the limit to which it tends.

	\begin{tcolorbox}[title=Remark,colframe=black,arc=10pt]
The reader who is not comfortable with topology can skip the text that follows... and whoever wants to know more about Cauchy sequences may read the section Topology or also particularly the section on Fractals (\SeeChapter{chapter Theoretical Computing}).
	\end{tcolorbox}

\textbf{Definition (\#\mydef):} Let $(X, d)$ a metric space (\SeeChapter{see section Topology page \pageref{metric space}}), we say that the sequence:
	
converges to $x \in X$ if (by definition!):
	
In other words, more we go far in the sequence, the more points are close (in the sense of the metric $d$) to each other.

If we chose a particular metric (the Euclidean one for example) and a discrete sequence the above definition will look like this:
	
	Where the convergence point is therefore $a$ and we have:
	
	In the example of the figure below where the sequence seems to converge to $1.13$ we observe that for a given non-zero positive $\varepsilon$, there there is a particular $n$ which we denote $N$ ($n=17$ in the figure below) from which the sequence converges:
	\begin{figure}[H]
		\centering
		\includegraphics[scale=0.75]{img/analysis/cauchy_convergence.eps}
		\caption{Illustration of the principle of convergence of a sequence}
	\end{figure}

	However, the above definition of the convergence makes problem because the number $x$ should be known. In most cases of interest $x$ is unfortunately not known. To break this deadlock, Cauchy had the idea to propose the following definition:

	\textbf{Definition (\#\mydef):} We say by definition that a sequence $(x_n)_{n \in \mathbb{N}}$ of elements of $X$ is a "\NewTerm{Cauchy sequence}\index{Cauchy sequence}" if:
	
	The reader must notice that it is not sufficient for each term to become arbitrarily close to the preceding term. This is why require that $|a_{N+1} - a_{N}| < \varepsilon$ is not sufficient!

	It is almost obvious then that any convergent sequence is a Cauchy sequence (well there are some subtleties that we will not reference for now).

	\begin{tcolorbox}[title=Remark,colframe=black,arc=10pt]
This criterion therefore facilitates some proofs because it helps to show the existence of a limit without involving its value, generally unknown.
	\end{tcolorbox}
	
	\begin{theorem}
	Let us now prove that the theorem that asses that any convergent sequence is Cauchy sequence.
	\end{theorem}
	\begin{dem}
	Consider a sequence $u_n$ converging to the value $l$ (which is unknown to us) and $\varepsilon>0$ (randomly selected). Then there exists according to the definition of a convergent sequence, $n \in \mathbb{N}$ such that:
	
	The choice to write $\dfrac{\varepsilon}{2}$ is completely arbitrary but in fact we anticipate the result of the demonstration so that it is more aesthetic...
	
	Therefore for $p,q>N$ (in fact know the value of $N$ is irrelevant, since it should work for any value... well don't forget that $N$ depends on $\dfrac{\varepsilon}{2}$) we have using the triangle inequality (\SeeChapter{see section Vector Calculus page \pageref{triangle inequality}}):
	
	and because $d(u_n,l)\leq\dfrac{\varepsilon}{2}$ we can write:
	
	Therefore:
	
	That may be a bit abstract so let's see an example with the harmonic sequence as an example to close the proof:
	
	First, nothing it is not vorbidden to us to take $n \geq 2$ (otherwise it will be hard to make a difference between two terms...).
	Therefore we take the Euclidean distance:
	
	First, the reader will note that in all cases since $k\leq 2n$ is between $n+1$ and $2n$. Which brings us to write:
	
	So from this inequality it comes automatically that each term of the sum on the left below will be greater than each term of the sum on the right:
	
	With (just do a particular example)
	
	Therefore:
		
	Now the idea is to see if the sum on the left is therefore greater than or equal to $\varepsilon=\dfrac{1}{2}$ and this for any $n$. Thus the suite is not convergent!
	
	Thus, the idea is that we found an $\varepsilon$ for which the Cauchy criterion is deficient.
	\begin{flushright}
		$\blacksquare$  Q.E.D.
	\end{flushright}
	\end{dem}
	So it is not because the points are always closer to each other that they converge to a given point, because this point may not exist.
	
	The best example is probably the following (it is also a little bit stupid example but...):
	\begin{tcolorbox}[colframe=black,colback=white,sharp corners]
\textbf{{\Large \ding{45}}Example:}\\\\
Let us take $X=\mathbb{Q}$ and the absolute difference as distance:
	
	Given $z$ an irrational number and $q_j \in \mathbb{Q}$ with $j \in \mathbb{N}^{*}$ such that:
	
	The idea is that greater is $j$, more the rational number $q_j$ is near the irrational $z$ and we know we can found such a sequence.
	Let us show that the $q_1,q_2$ we could be able to build form a Cauchy sequence! Indeed using triangle inequality:
	
	and therefore is a Cauchy sequence if and only if $\vert q_m-q_n \vert\leq \varepsilon$ if:
	
	We have thus found a $N$ (equal to $\dfrac{1}{2\varepsilon}$) which satisfies our definition of a Cauchy sequence. But this sequence does not converge in $\mathbb{Q}$ otherwise $z$ would be rational.
	You can check this with $\pi$ and the sequence:
	
	\end{tcolorbox}
	
	\begin{tcolorbox}[title=Remark,colframe=black,arc=10pt]
Mathematicians use such results to define the set of irrational and also by using some additional topological concepts.
	\end{tcolorbox}
	We have just seen that a Cauchy sequence is not necessarily a convergent sequence in $X$. The inverse is however true: any convergent sequence is a Cauchy sequence!!
	
	\subsubsection{Fibonacci Sequence}\label{Fibonacci Sequence}
	
	If we calculate a sequence of numbers starting with 0 and 1, such that each term is equal to the sum of the two previous ones, we can form the following sequence:
	
	therefore, if we designate the different terms by:
	
	We build therefore the following sequence law:
	
	More often written in the following form:
	
	The Fibonacci sequence has many interesting strong properties, which will be developed later. However, it seems to be the first "\NewTerm{recurring sequence}\index{recurring sequence}" known in history (hence the fact that we are talking about it in this book). 
	
	The origin of this sequence seem to come from a rabbit problem asked to Fibonacci in 1202. Starting with a couple of rabbits, how much couples of rabbits will we get after a given number of months knowing that each pair produces a new pair every month (and no couples die...), which becomes productive only after two months. Therefore we have:
	\begin{itemize}
		\item Beginning: We have nothing $(0)$
		\item 1st month: We buy a couple of baby rabbits $(1)$.
		\item 2nd month: The couple of rabbits are now adults $(1)$.
		\item 3rd month: We have the couple of rabbits that make a new couple of baby rabbits. We have two couples $(2)$.
		\item 4th month: We have two couple of adults with a new couple of babies. We have three couples $(3)$.
		\item 5th month: We have three couples of adults rabbits and two new couples of baby rabbits. We have five couples $(5)$
		\item and so on...
	\end{itemize}
	
	Let us take now a "in real life" example (this is typically a biased scientific example because you will always finish to find in Nature what your are looking for to argue your theories with a least on particular example...): the heart of some flowers! The scales of a pineapple or pinecone form two families of spirals wound in opposite directions. On a pine cone, you will count 5 spiral in one direction and 8 in the other, on pineapples, 8 and 13, on sunflowers 21 and 34. Each time we get Fibonacci numbers!
	
	A famous illustration of this is to do draw the following simple figure (named"Fibonacci Spiral") which reproduces the Fibonacci numbers on a grid plan with squares and corners connect with arc circles:
	
	\begin{figure}[H]
		\centering
		\includegraphics[scale=0.75]{img/algebra/fibonacci.jpg}
		\caption{Fibonacci spiral}
	\end{figure}
	\begin{tcolorbox}[title=Remark,colframe=black,arc=10pt]
	The Fibonnacci spiral (quarter-circles tangent to the interior of each square) must not be confused with the "Golden Spiral" (a logarithmic spiral whose growth factor is the Golden ratio $\varphi$)!
	\end{tcolorbox}
	Let us now consider the limit of the Fibonacci sequence, let:
	
	by the properties of limits:
	
	As the construction of the Fibonacci sequence is given by:
	
	Using this equation to substitute, we get:
	
	and so we get the equivalent well known equation (see page \pageref{golden ratio}):
	
	and finally solving for $L$ using the quadratic formula yields and keeping the only solution that makes sense we have:
	
	that is the Golden ratio!
	
	We will come back later (see page \pageref{ordinary generating function}) on this sequence to prove another relation it has with the Golden ratio (see page \pageref{golden ratio}) through the direct determination of the $n$ term value!

	\subsubsection{Logic Sequences/Psychologist Sequences}
	\textbf{Definition (\#\mydef):} Psychologists name "\NewTerm{logical sequences}\index{logical sequence}", sequences that they write with an idea in mind, and they call "logical" people who find their idea, although there are other possibilities mathematically speaking (but psychologist don't know anything about real logic).
	
	For example if you have to find the next number $X$ to the logic sequence:
	
	In fact, you make the difference between the last and prior-previous number then you multiply by $10$ therefore the next number is $X=31000$.
	
	From a mathematical point of view, any number is suitable to replace $X$, also exists for each value of $X$, a polynomial in $n$ that takes the values $4, 5, 10, 50, 400, 3500$ for $n = 0, 1 , 2 ..6$.
	
	For the example above we can take for example:
	
	and is such that $P(0)=4, P(1)=5, P(2)=10, ...,P(6)=0$.

	\pagebreak
	\subsection{Series}

The physicists often needs to simply and formally solve problems, to approximate some given "terms" of their equations. For this purpose, they will use the properties of some given series. Also statisticians and financial analysts often face to series they need to simplify.

\textbf{Definition (\#\mydef):} Let be given an infinite number sequence:
	
	The expression:
	
	is named a "numeric series".
	
	\textbf{Definition (\#\mydef):} The partial sum of the first $n$ terms of the series is named "\NewTerm{partial sum}\index{partial sum}\label{partial sum}" and denoted by:
	
	If the following limit denoted $S$ exists and is finite:
	
	we name it "\NewTerm{sum of the series}\index{sum of a series}" and we say that the "\NewTerm{series converges}\index{convergent series}" (it is therefore a  Cauchy series). However, if the limit does not exist, we say that the "\NewTerm{series diverges}\index{divergent series}" and has no sum (for details see further below when we will deal with some empirical convergence criterias).
	\begin{theorem}
		Also let us prove for fun (because it is almost trivial) that if $\displaystyle \sum_{k\geq 0} u_k $ is a convergent numerical series then:
		
		But the opposite is not necessarily true!! In fact remember the example during our study above of Cauchy sequences with the harmonic series $\sum_{k=1}^n \dfrac{1}{k}$ that is not convergent even if the terms tends to zero when $k \rightarrow +\infty$.
	\end{theorem}
	\begin{dem}
		We assume first that $\displaystyle \sum_{k\geq 0} u_k $ is a convergent series and denote its limit by $S$. Let:
		
		Therefore:
		
		However, if the series is really convergent:
		
		So finally:
		
		\begin{flushright}
			$\blacksquare$  Q.E.D.
		\end{flushright}
	\end{dem}
	Following a reader request, let us prove in detail (not only assuming that it is intuitive) that the harmonic series\index{harmonic series}\label{harmonic series}:
	
	diverges.
	\begin{dem}
		Suppose the series converges to $H$, i.e.:
		
		Then:
		
		Calculating the underbraces, we get:
		
		This contradiction concludes the proof.	
		\begin{flushright}
			$\blacksquare$  Q.E.D.
		\end{flushright}
	\end{dem}
	\begin{tcolorbox}[title=Remark,colframe=black,arc=10pt]
	Many people are struggling understanding intuitively why the harmonic series diverges but the $p$-harmonic series converges. For sure there are methods and applications to prove convergence and divergence, but many people have trouble understanding intuitively why it is. We know we must never trust too much or intuition, but this is hard for many to grasp. In both cases, the terms of the series are getting smaller, hence are approaching zero, but they both result in different answers.
	
	Perhaps a convincing way is to make an analytical prolongation to change the sum into an integral (\SeeChapter{see section Differential and Integral Calculus page \pageref{usual primitives}}):
	
	converges but:
	
	\end{tcolorbox}
	
	In the next subsection we will how to calculate the partial sum of some classic series that are important in physics, statistics and finance. We will start with Gauss arithmetic series that are an expression of the sum of the $n$ first nonzero integers raised to a given power $k$ in a condensed form. The application of this condensed form of a series has an important practical use in physics, statistics and finance when we wish to simplify the expression of certain results.
	
	\subsubsection{Gauss Series}\label{gauss series}\index{Gauss series}
	
	Gauss arithmetic series are an expression of the sum of the $n$ first nonzero integers raised to a given power $k$ in a condensed form $S_k$. The application of this condensed form of a series has an important practical use in physics, statistics and finance when we wish to simplify the expression of certain results.
	
	It is said that Gauss have found an attractive method in 1786 to determine the arithmetic sum of the first $n$ integers at the power of 1 when he was nine years old (...):
	
	To simplify, we find easily the following closed formula (notice that this is not a recursive relation!):
	
	for $n \geq 0$. Let us indicate that each intermediate sum of the series (1, 3, 6, 10, 15, etc.) is named "\NewTerm{triangular number}\index{triangular number}" since it is possible to represent it in the following form:

	\begin{figure}[H]
		\centering
		\includegraphics[scale=1]{img/algebra/triangular_number.jpg}
		\caption{Triangular number}
	\end{figure}

	We can continue with higher powers bit not as exercises because these relations are very useful!

	Now let us calculate the very important case that we find ourselves in a number of other sections (Economy, Quantum wave Physics, etc.) and that  the sum of the first $n$ square integer numbers (still non-zero!).

	Let us write for this\label{sum of squares integers}
	
	We know from Newton's binomial theorem (\SeeChapter{see Section Calculus page \pageref{binomial theorem}}):
	
	so we can write and add a member to member the $n$ following equalities:
	
	And the sum can be simplified as:
	
	After some elementary algebra manipulations we get:
	
	Therefore:
	
	Finally:
	
	We continue with the sum of the first $n$ cubes (non-zero) integers. The principle is the same as before, we write:
	
	We know from Newton's binomial theorem (\SeeChapter{see Section Calculus page \pageref{binomial theorem}}):
	
	We get by varying $k$ from $1$ to $n$, $n$ relations that we can add a member to member
	
	And the sum can be simplified as:
	
	Giving after development:
	
	And after an fist simplification:
	
	And a second simplification:
	
	The result result is therefore:
	
	or written differently:
	
	For sure, we can continue like this during a long time, but from a certain value of the power things get a bit more complicated (furthermore the method is a little bit boring). Thus, one of the members of the Bernoulli family (it was a family of very talented mathematicians... as you can see int the Biographies chapter) founded a general relation working for any power by defining what we name the "Bernoulli polynomial" (see further below).
	
	Let us conclude with one last case we will need during our study of Fourier series. We put:
	
	We want express this expression (series) as rational fraction. To do this, we multiply all by $x^2$. So we have two expressions:
	
	We subtract the first from the second:
	
	Finally:
	
	Most of times, to indicate that this is for odd powers we prefer to write:
	
	Similarly, for the needs of the section of Economy, we have:
	
	Therefore:
	
	Finally:
	
	Most of times, to indicate that this is for even powers we prefer to write:
	
	\paragraph{Bernoulli's Numbers and Polynomials}\mbox{}\\\\
	As we have seen above, it is possible to express the sum of the first $n$ nonzero integers to a given power (the first four have been proved previously) following the below relations where we put now $n:=n+1$ as we want now $n$ to be the number of terms we want the sum including 0 (hence the negative sign in the relations below that we did not have earlier):
	
	It is said that Jacob Bernoulli then noticed that the polynomials $S_p$ had the form:
	
	In this expression, the numbers $(1,-1/2,1/12,0,...)$ seem not to depend on $p$. More generally, after trial and error we see that the polynomial can be written as:
	
	Giving by identification the "\NewTerm{Bernoulli's numbers}\index{Bernoulli's numbers}":
	
	\begin{theorem}
	Thereafter, it seems that mathematicians in their research fell randomly (???) on the fact that the Bernoulli numbers could be expressed by the series:
	
	with $\vert z\vert<2\pi$.
	\end{theorem}
	\begin{dem}
	We have seen during our study of complex numbers (\SeeChapter{see section Numbers page \pageref{taylor expansion complex exponential}}) that:
		
		Therefore:
		
		Let us write now:
		
		Then we must have:
		
		We see (by distributing) that:
		
		for all this to be equal to unity we must have:
		
		From the second equation we get:
		
		and from the third equation we get:
		
		etc. Continuing this way we show that:
		
		It is obvious that this method allows us to calculate by hand only the first terms of this series.
		
		Thus, based on (as we will see later, this is the expression of an "ordinary generating function"):
		
		we find that the first Bernoulli numbers are:
		\begin{table}[H]
			\begin{center}
			\definecolor{gris}{gray}{0.85}
			\begin{tabular}{|p{1cm}|p{2.5cm}|}
					\hline
					\multicolumn{1}{c}{\cellcolor{black!30}\textbf{$k$}} & 
  \multicolumn{1}{c}{\cellcolor{black!30}\textbf{$B_k$}} \\ \hline
					\centering\arraybackslash\ 0 & \centering\arraybackslash\ 1 \\ \hline
					\centering\arraybackslash\ 1 & \centering\arraybackslash\ -1/2 \\ \hline
					\centering\arraybackslash\ 2 & \centering\arraybackslash\ 1/6  \\ \hline
					\centering\arraybackslash\ 3 & \centering\arraybackslash\ 0  \\ \hline
					\centering\arraybackslash\ 4 & \centering\arraybackslash\ -1/30  \\ \hline
					\centering\arraybackslash\ 5 & \centering\arraybackslash\ 0  \\ \hline
					\centering\arraybackslash\ 6 & \centering\arraybackslash\ 1/42  \\ \hline
					\centering\arraybackslash\ 7 & \centering\arraybackslash\ 0  \\ \hline
					\centering\arraybackslash\ 8 & \centering\arraybackslash\ -1/30  \\ \hline
					\centering\arraybackslash\ 9 & \centering\arraybackslash\ 0  \\ \hline
					\centering\arraybackslash\ 10 & \centering\arraybackslash\ 5/66  \\ \hline
					\centering\arraybackslash\ 11 & \centering\arraybackslash\ 0  \\ \hline
					\centering\arraybackslash\ 12 & \centering\arraybackslash\ -691/2730  \\ \hline
					\centering\arraybackslash\ 13 & \centering\arraybackslash\ 0  \\ \hline
					\centering\arraybackslash\ 14 & \centering\arraybackslash\ 7/6  \\ \hline
					\centering\arraybackslash\ ... & \centering\arraybackslash\ $...$  \\ \hline
			\end{tabular}
			\end{center}
			\caption{Example of tabular representation of a game}
		\end{table}	
		The reader will have noticed that $B_k=0$ when $k$ is odd and different from $1$. That why sometimes the above relation is given by (remember that $0$ is an even number!):
		
		\begin{flushright}
			$\blacksquare$  Q.E.D.
		\end{flushright}
	\end{dem}
	We see easily that the values of the Bernoulli numbers can not be described in a simple way. In fact, they are essentially values of the zeta Riemann function (see below) for negative integer values of the variable, and these numbers are associated with profound theoretical properties that go beyond the study of this book. Furthermore, the Bernoulli numbers also appear in the Taylor series expansion of  circular and hyperbolic trigonometric tangent functions in the Euler-Maclaurin formula (see below).
	
	With a small modification it is possible to define the "\NewTerm{Bernoulli polynomials $B_k(x)$}\index{Bernoulli polynomials}\label{bernoulli polynomials}" by:
	
	with:
	
	\begin{theorem}
	Furthermore, it is normally easy to observe that:
	
	and therefore it normally easy to deduce that:
	
	\end{theorem}
	\begin{dem}
	On one side we have:
	
	and another we have:
	
	So:
	
	\begin{flushright}
		$\blacksquare$  Q.E.D.
	\end{flushright}
	\end{dem}
	And by identification of the coefficients we deduce:
	
	and for $k \geq 1$:
	
	It is then easy to deduce that the polynomials $B_k(x)$ are of degree $k$:
	
	Here is a plot of these polynomials:
	\begin{figure}[H]
		\centering
		\includegraphics{img/algebra/bernoulli_polynomials.jpg}
		\caption[Some Bernoulli polynomials]{Some Bernoulli polynomials (source: Wikipedia)}
	\end{figure}
	What is remarkable is that using the Bernoulli polynomials, we see that it is possible to write the $S_p$ as follows after some trials:
	
	
	Some write this relations even otherwise. Indeed, from previous relation, we can write:
	
	using:
	
	We have:
	
	So we just demonstrated in an engineer way that:
	
	However, we can now ask ourselves what happens to the partial sum of arithmetic and geometric sequences as presented earlier in this section. 
	
	\paragraph{Euler-Maclaurin formula}\label{Euler-MacLaurin formula}\mbox{}\\\\
	The Euler–Maclaurin formula is a relation for the difference between an integral and a closely related sum. It can be used to approximate integrals by finite sums, or conversely to evaluate finite sums and infinite series using integrals and the machinery of calculus. For example, many asymptotic expansions are derived from the formula\footnote{It will be useful to us during our study of the electromagnetic effect in the section of Wave Quantum Physics.}.
	
	If $m$ and $n$ are natural numbers and $f(x)$ is a real or complex valued continuous function for real numbers $x$ in the interval $[m, n]$, then the integral:
	
	can be approximated by the sum (or vice versa):
	
	(see rectangle method page \pageref{rectangle integration method}). The Euler-Maclaurin formula provides expressions for the difference between the sum and the integral in terms of the higher derivatives $f^{(k)}(x)$ evaluated at the endpoints of the interval, that is to say when $x=m$ and $x=n$.
	
	Explicitly, for $p$ a positive integer and a function $f(x)$ that is $p$ times continuously differentiable on the interval $[m, n]$, we have:
	
	where $B_{k}$ is the $k$ th Bernoulli number (with $B_{1}=1 / 2$ ) and $R_{p}$, is an error term which depends on $n, m, p$, and $f$ and is usually small for suitable values of $p$ The formula is often written with the subscript taking only even values, since the odd Bernoulli numbers are zero except for $B_{1}$. In this case we have:
	
	or alternatively:
	
	\begin{dem}
	One of the fundamental concepts of calculus is the correspondence between sums and integrals. Given any sufficiently well-behaved continuous function $f(x)$, consider the sum the integral defined by:
	
	These two functions obviously have some similarities, but they also have significant differences. For example, the summation is taken over the values of $f(k)$ at discrete integer arguments $\mathrm{k}=0,1,2, \ldots,n$, whereas the integral is taken over the values of $\mathrm{f}(\mathrm{x})$ for arguments x varying continuously from $0$ to $n$. Strictly speaking, the function $s(n)$ is defined only for integer arguments, while $I(\mathrm{n})$ is defined for any non-negative real value of $n$. Nevertheless, we can derive an interesting and useful relation between these two functions.
	
	First, notice that $I(n)$ can be written in the form:
	
	The Taylor series expansion of each of the individual terms of the integrand is of the form:
	
	where $f^{(j)}$ denotes the $j$th derivative of $f$. The integral of this term from $x=0$ to $1$ is:
	
	Inserting all these expressions into the original integral, we get:
	
	where:
	
	Now, if we had started with the derivative of $f$ in place of $f$, we would have arrived at the same expression:
	
	except that each $f^{(j)}$ would be replaced with $f^{(j+1)}$, and hence each $S^{(j)}$ would be replaced with $S^{(j+1)},$ and $I(n)$ would be replaced with $I^{(1)}(n)$ where:
	
	Thus beginning with successively higher derivatives of $f$, we get the infinite sequence of relations:
	
	Each of these equations has (in general) infinitely many terms, but we can solve the system of equations for any finite numbers of terms by taking just a restricted portion of these equations. For example, taking just the terms up to $\mathrm{S}^{(6)}$ from the first four equations, we have (in matrix form):
	
	Solving this system of equations gives
	
	where:
	
	and so on are the Bernoulli numbers. Thus we have
	
	Inserting the expressions for each $I^{(j)}$ from relation:
	
	this gives:
	
	Recall that $S^{(0)}$ is defined as the sum of $f(k)$ for $k=0,1,2, \ldots, n-1$. Therefore, if we wish to express this in terms of a summation from $k=0$ to $n$, we must add $f(n)$ to both sides. Noting that $B_{1}=-1 / 2$, this leads to the result:
	
	This is named the "\NewTerm{Euler-Maclaurin formula}\index{Euler-MacLaurin formula}", a generalization of Bernoulli's formula for the sum of powers of the first $n$ integers. Often used in the following form:
	
	or using a condensed notation:
	
	We should note that this series is divergent for most applications, but the error is less than the first neglected term, so the Euler-Maclaurin formula is often a useful method of approximation, relating the series summation to the continuous integral of a function.
	\begin{flushright}
		$\blacksquare$ Q.E.D.
	\end{flushright}
	\end{dem}
	
	\begin{tcolorbox}[colframe=black,colback=white,sharp corners]
	\textbf{{\Large \ding{45}}Example:}\\\\
	The Bernoulli numbers from $B_{1}$ to $B_{7}$ are $1/2, 1/6, 0,-1/30, 0, 1/42, 0$. Therefore the low-order cases of the Euler-Maclaurin formula are:
	
	\end{tcolorbox}
	
	\subsubsection{Arithmetic Series}
	We have shown above that the partial sum of a Gauss series (analogous to the sum of the terms of an arithmetic progression of reason $r = 1$) was given by:
	
	if not denote the value of the $n$-th term by $u_n$ instead of $n$, the development that we made for the series of Gauss then brings us to:
	
	and if we denote the first term $1$ of the Gauss series $u_0$ then we have:
	
	which gives us simple the partial sum of the $n$-terms of an arithmetic sequence of reason $r$.
	\begin{tcolorbox}[colframe=black,colback=white,sharp corners]
	\textbf{{\Large \ding{45}}Examples:}\\\\
	E1. A simple Gauss series with of reason 1 starting at 4, finishing at 6:
	
	E2. Now an arithmetic partial sum series of reason 2 starting at 4, finishing at 8:
	
	\end{tcolorbox}
	\begin{tcolorbox}[title=Remark,colframe=black,arc=10pt]
	The reader will have observed that the reason $r$ does not appear in the latter relation. Indeed, by taking (always) the same development that for the Gauss series, the term $r$ is simplified and vanish.
	\end{tcolorbox}
	
	\subsubsection{Geometric Series}\label{geometric series}
	Similarly, with a geometrical sum where we have for recall:
	
	we have therefore:
	
	The last relation is written (after simplification):
	
	and if $q\neq 1$ we get:
	
	which can be written by factoring $u_0$:
	
	If $q$ is positive and less than $1$, as $n$ approaches infinity we have the result that will be used extensively in the section Economy:
	
	That we found often in the following form (special case):
	
	
	\begin{tcolorbox}[colframe=black,colback=white,sharp corners]
	\textbf{{\Large \ding{45}}Examples:}\\\\
	E1. Consider the following geometric series of reason $q = 2$:
	
	to calculate the sum of the first four terms $\left\lbrace 1,2,3,4 \right\rbrace$, we take the power of $2$ equivalent of $n=2$ (zero is not taken into account). We then get well: $S_3=15$.\\
	
	E2. Consider the quite famous geometric series of reason $q=1/2$. We then have:
	
	\end{tcolorbox}
	
	\paragraph{Zeta function and Euler's product formula}\label{zeta function}\mbox{}\\\\
	The German mathematician Riemann named "zeta" a function already studied before him, but that he examined when the value is a complex number (\SeeChapter{see section Numbers page \pageref{complex numbers}}). This function is represented as a series of inverse powers of integers. This how typically the series looks like for a special case:
	
	This may seems to the reader as pure abstract mathematics. However as we will see in the next chapters of this book, this function will appear for example during our study of the Black body (Stefan-Boltzmann law), or during our study of the Casimir force in Quantum Physics, but also during our study of the recombination of the Cosmological Microwave Background.
	\begin{tcolorbox}[title=Remarks,colframe=black,arc=10pt]
	\textbf{R1.} It is traditional to denote by $s$ the variable upon which this series depends.\\
	
	\textbf{R2.} The Riemann zeta function is a special case of the "\NewTerm{zeta function}\index{zeta function}" defined by:
	
	where $s=\Re(s)+\mathrm{i}\Im(s)=\sigma+\mathrm{i}t$ but when $a_n=n$.
	\end{tcolorbox}
	This series has an interesting property but if we remain within the framework of positive integer powers here is how we can introduce its origin:
	
	when $n\longrightarrow +\infty$ then we have:
	
	If we put $x=2^s$, we obtain the sum of the inverse of powers of $2$ and similarity with $x=3^s$ such that:
	
	If we do the product of these two expressions, we obtain the sum of the powers of all fractions whose denominator is a product of $2$ and $3$:
	
	If we take all primes left, we'll get on the right all integers, since every integer is the product of prime numbers according to the fundamental theorem of arithmetic (\SeeChapter{see section Number Theory page \pageref{fundamental theorem of arithmetic}}), and this is Euler fundamental identity: what we now name "\NewTerm{Riemann zeta function}\index{Riemann zeta function}" is both a finished product involving prime numbers and an infinite sum of inverse powers of all integers:
	
	In condensed notation, the "\NewTerm{Euler's product formula}\index{Euler's product formula}" of the Riemann zeta function is given by:
	
	where $p$ as indicated below the product symbol are the prime numbers.
	\begin{tcolorbox}[title=Remark,colframe=black,arc=10pt]
	The zeta Riemann function is a special case of "\NewTerm{Dirichlet series}\index{Dirichlet series}" that are defined by:
	
	When the real part of $s$ is greater than $1$, the Dirichlet series converges. On the other hand, the Dirichlet series diverges when the real part of s is less than or equal to $1$, so, in particular, the series $1 + 2 + 3 + 4 + \ldots$ that results from setting $s=-1$ does not converge.
	\end{tcolorbox}
	We now recommend most readers to skip what follows on the Riemann zeta function and return back here once the Fourier series and the Gamma-Euler function presented later in this section are mastered and understood...
	
	We assume in what follows that the Fourier series and Gamma-Euler function are known and mastered and that the Parseval's equality (see below page \pageref{Parseval theorem}) was studied (since it is also proved further below). 
	
	We will seek to determine first the Riemann zeta function for two positive values ($s$ respectively having values $2$ and $4$) that will be useful in the valuation of integrals in some sections of the chapter about Mechanics. As the two values are positive the series converge and there won't be too many issues evaluating the Riemann zeta function\footnote{For the section of Cosmology we should need also the value of the Riemann zeta function for $s=3$ but however the result give an irrational constant (ie the result cannot be written as a fraction $p/q$ where $p$ and $q$ are integers) named the "Apéry's constant". There is a theorem that proves that $\zeta(3)$ is irrational named the "Apéry's theorem".}. 
	
	Once done we will determine the value of the Riemann zeta function for any negative integer value! And here obviously the series seems to be divergent. In mathematics and theoretical physics for that kind of situation, we use the "\NewTerm{zeta function regularization}\label{zeta riemann regularization}" that is a type of regularization or summability method that assigns finite values to divergent sums or products, and that in particular can be used to define determinants and traces of some self-adjoint operators. The technique is now commonly applied to problems in physics, but has its origins in attempts to give precise meanings to ill-conditioned sums appearing in Number Theory. As we will see just below and during our study of Grandi's series (see page \pageref{Grandi series}) there are several different summation methods named "zeta function regularization" for defining the sum of a possibly divergent series!
	
	So let's start!
	
	\begin{itemize}
		\item To determine the value of $\zeta (2)$, we will express the function:
		
		in Fourier series form (see a little further below in this section). During our study of Fourier series we will see that there are two traditional ways to define a Fourier series and we have done here the choice of the definition of the most commonly used among physicists and engineers:
		
		As we prove it in our study of Fourier series, Fourier coefficients $a_0,a_n,b_n$ are obtained by solving:
		
		and using the integration by parts (\SeeChapter{see section Differential and Integral Calculus page \pageref{integration by parts}}) we have:
		
		It comes then:
		
		But the Parseval's theorem that we will prove in our study of Fourier series a little bit further below gives us too (depending on the choice of the definition of the Fourier series and associated coefficients, the Parseval's theorem is expressed a little bit differently!):
		
		Therefore we get immediately:
		
		But we will also see during our proof of Parseval's theorem that:
		
		Therefore it comes in our case:
		
		Therefore:
		
		and finally:
		
		The problem of finding the value this series converges to is known as the "\NewTerm{Basel problem}". Leonhard Euler found the solution in 1735...
		
		\item To determine the value of $\zeta(4)$, we will do the same, but with the function:
		
		in the form of Fourier series:
		
		For this purpose, we will calculate Fourier coefficients using the integration by parts (\SeeChapter{see section Differential and Integral Calculus page \pageref{integration by parts}}):
		Then we have:
		
		Therefore we have:
		
		But the Parseval's theorem that we will prove below gives us also:
		
		It then comes immediately:
		
		But we will see also see later below during our Parseval's theorem proof that:
		
		Then it comes in our case:
		
		Therefore:
		
		That is to say:
		
		Finally:
		
		
		\item Now let us focus on the first divergent case $\zeta(-1)$. 
		
		There are a few ways to prove that:
		
		One method, along the lines of Euler's reasoning, uses the relationship between the Riemann zeta function and what we define as the "\NewTerm{Dirichlet eta function}\index{Dirichlet eta function}" or "\NewTerm{alternating zeta function}\index{alternating zeta function}\footnote{Yes there are a lot of different kind (variations) of the zeta function. You can found a list of the 34 most important of them on Wikipedia.}":
		
		This Dirichlet series is the alternating sum corresponding to the Dirichlet series expansion of the Riemann zeta function, $\zeta(s)$.
		
		Where both Dirichlet series converge, one has the identities:
		
		The identity$(1-2^{1-s})\zeta (s)=\eta (s)$ continues to hold when both functions are extended by analytic continuation to include values of s for which the above series diverge. Substituting $s=-1$, one gets $-3\zeta(-1) = \eta(-1)$. Now, computing $\eta(-1)$ is an easier task, as the eta function is equal to the Abel sum of its defining series, which is a one-sided limit:
		
		Dividing both sides by $-3$, one gets:
		
		 The first example in which zeta function regularization is available appears in the Casimir effect, which is in a flat space the bulk contributions of the quantum field in three space dimensions. In this case we must calculate the value of Riemann zeta function at $-3$:
		
		  which diverges explicitly. However, it can be analytically continued to $s=-3$ where hopefully there is no pole (\SeeChapter{see section Complex Analysis page \pageref{residue theorem}}), thus giving a finite value to the expression.
		 \begin{tcolorbox}[title=Remark,colframe=black,arc=10pt]
		The reason why we see $\zeta(-1)$ in many YouTube videos and physics textbook, rather than $\zeta(-3)$ is that when you imagine the Casimir effect as happening in one dimension (along a line rather than in $3$D), the energy density you calculate is $\zeta(-1)$ rather than $\zeta(-3)$.
		\end{tcolorbox}
		
		Let us see another and more rigorous way to prove that $\zeta(-1)=-\dfrac{1}{12}$.
		
		We need a result first:
		
		Let us now multiply the above result by $x+1$, then integrate by parts twice, to we get:
		
		Now we can plug in $x=-1$ into the above result to get:
		
		Since $(1-2^2)\Gamma(1)=-3$ the above result can be rearrange to:
		
		We will see a third method just below to get that same result and even a fourth one during our study later of Grandi's series (see \pageref{Grandi series}).
		
		\item Let us prove now that and elegant and fast way to determine the value of the zeta function for all negative integers (there are other methods but involving heavy and boring mathematics for most engineers).
		
		For that purpose we will first need to prove the following lemma:
		\begin{lemma}
		The Riemann zeta function in the critical strip is given by:
		
		where $[x]$ denotes the integer part of $x$.
		\end{lemma}
		For the proof we need first for $\Re(s) > 1$:
		 
		Since $[x]=n$ for any $x$ in the interval $[n,n+1]$, we have:
		
		allowing the following simplification:
		
		Because $0\leq \{x\}<1$, the last integral converges and is holomorphic (\SeeChapter{see section Complex Analysis page \pageref{holomorphic functions}}) on $\Re(s)>0$. But that means the full equation is meromorphic (\SeeChapter{see section Complex Analysis page \pageref{meromorphic function}}) on $\Re(s)>0$, and thus provides an analytic continuation of $\zeta(s)$ on the half plane $\Re(s)>0$. The $s/(s-1)$ term gives a simple pole at $s=1$ with residue $1$ (\SeeChapter{see section Complex Analysis page \pageref{residue theorem}}).
		
		Injecting again back $[x] = x-\{x\} \Leftrightarrow \{x\}=x-[x]$, we get:
		
		That finish the proof of the lemma.
		
		Now remember that we have proved just earlier above that:
		
		where $\{x\}$ denotes the fractional part of $x$. This gives the analytic continuation of $\zeta(s)$ for $\Re(s)>0 .$ We can now proceed inductively. Writing:
		
		and integrating the last integral by parts. we get:
		
		and the latter integral converges for $\Re(s)>-1$. That is:
		
		Thus, inductively, we deduce:
		
		and the infinite sum on the right hand side converges for $\Re(s)>-m$. 
		
		If in the relation above we put $s=1-m,$ and note that for $r=m$ that $\zeta(s+m)$ has a simple pole at $s=1-m,$ we obtain the recurrence:
		
		The first few values at nonpositive integers are immediate and given by:
		
		And that's it!
	\end{itemize}
		
	\subsubsection{Telescoping Series}
	A "\NewTerm{telescoping series}\index{telescoping series}" is a series in which most of terms cancel in each of the partial sums, leaving only some of the first terms and some of the last terms:
	
	For example, the series:
	
	simplifies as:
	
	We will encounter such as series for business purposes (management) in our study of Queuing Theory in the section of Quantitative Management!!!
	
	\subsubsection{Grandi's Series}\label{Grandi series}
	The "\NewTerm{Grandi's series}\index{Grandi's series}" (after Italian mathematician, philosopher, and priest Guido Grandi, who gave a memorable treatment of the series in 1703) is defined as the following arithmetic series:
	
	It is a very famous series in mathematics and physics because:
	\begin{itemize}
		\item It highlights in a very simple way the fact (see below) that it is dangerous to manipulate infinite series
		
		\item Its result seems completes non-intuitive but in fact it opens the door to a more general definition of what is a "sum"
		
		\item It is a beautiful example of a series that seems useless and purely mathematics but that has in fact important application in quantum physics (Casimir Effect as seen in the section of Wave Quantum Physics) and String Theory (number of dimensions as seen in the section of String Theory).
	\end{itemize}
	and this is why we dedicate to it a special subsection in this book!
	
	It seem quite obvious at a first glance that it is a divergent series, meaning that it lacks a sum in the usual sense (the sequence of partial sums of Grandi's series clearly does not approach any number). But the other hand, its Cesàro sum is $1/2$!!?? So what the hell is a Cesàro sum?
	
	\textbf{Definition (\#\mydef):} In mathematical analysis a "\NewTerm{Cesàro sum}\index{Cesàro sum} assigns values to some infinite sums that are not convergent in the usual sense. The Cesàro sum is defined as the limit of the arithmetic mean of the partial sums of the series.
	
	Let $\{a_n\}$ be a sequence, and let:
	
	be the $k$th partial sum of the series:
	
	The series $\sum _{n=1}^{+\infty}a_{n}$ is say to be "Cesàro summable", with Cesàro sum $S\in\mathbb{R}$, if the average value of its partial sums $s_k$ tends to $S$:
	
	In other words, the Cesàro sum of an infinite series is the limit of the arithmetic mean (average) of the first $n$ partial sums of the series, as $n$ goes to infinity. If a series is convergent, then it is Cesàro summable and its Cesàro sum is the usual sum. For any convergent sequence, the corresponding series is Cesàro summable and the limit of the sequence coincides with the Cesàro sum.
	
	One obvious method to attack the Grandi's series:
	
	is to treat it like a telescoping series and perform the subtractions in place:
	
	On the other hand, a similar bracketing procedure leads to the apparently contradictory result:
	
	Thus, by applying parentheses to Grandi's series in different ways, one can obtain either $0$ or $1$ as a "value". It can be shown that it is not valid to perform many seemingly innocuous operations on a series, such as reordering individual terms, unless the series is absolutely convergent. Otherwise these operations can alter the result of summation.

	Treating Grandi's series as a divergent geometric series we may use the same algebraic methods that evaluate convergent geometric series to obtain a third value:
	
	so:
	
	Therefore:
	
	Finally:
	
	The same conclusion results from calculating $-S$, subtracting the result from $S$, and solving $2S = 1$.

	The above manipulations do not consider what the sum of a series actually means. Still, to the extent that it is important to be able to bracket series at will, and that it is more important to be able to perform arithmetic with them, one can arrive at two conclusions:
	\begin{itemize}
		\item The series $1-1 + 1-1 + \ldots$ has no sum

		\item ...but its sum should be $1/2$ (see further below)
	\end{itemize}
	In fact, both of these statements can be made precise and formally proven, but only using well-defined mathematical concepts that arise in the 19th century. After the late 17th-century introduction of calculus in Europe, but before the advent of modern rigor, the tension between these answers fueled what has been characterized as an "endless" and "violent" dispute between mathematicians. The funnies is that the violent discussions still continue today... a YouTube video on this subject have more than $5,000$ comments... and blog post more than $200$ comments and a forum thread more than $600$... (and by the way the comments show the worst of what the human behavior and respect can be...) So this is quite a hot topic...
	
	Let us also recall that at the beginning of our study of Geometric series we have proved that:
	
	Therefore if $u_0=1$ this reduce to:
		
	where as $n$ goes to infinity, the absolute value of $|q|$ must be less than one for the series to converge!

	Now notice that if $q=-1$ we fall back on Grandi's series and therefore that latter is a special case of the geometric series $1+q^1+q^2+q^3+\ldots$ and then we would perhaps write a bit too quick:
	
	But as we have just mention it, we are not authorized to write the latter fraction if $q=\pm 1$ otherwise the series diverge excepted... if we work in a special axiomatic framework. This is almost as chocking as when imaginary numbers were introduced in the 16th century, as well as negative numbers that were poorly understood and regarded by some as fictitious or useless, much as zero once was...
	\begin{tcolorbox}[title=Remark,colframe=black,arc=10pt]
	There have been very interesting studies about the reaction of high-school level students to Grandi's series presentation. The reactions and analysis are very interesting and I can personally only recommend every teacher to introduce this series in classes but without giving the result in a first time!
	\end{tcolorbox}
	But there is a "one more thing"... We will now calculate a sum, thinking it really gives infinity as a result:
	
	To do this, let's do another trick of mathematical magician:
	
	Therefore:
	
	So we can compute:
	
	Our first concrete result, squared, can be rewritten as follows:

	
	Or well:
	
	Explicitly:
   \begin{eqnarray*}
	   (-1 + 1-1 + 1-1 + \ldots)\\
	   \underline{\times (-1 + 1-1 + 1-1 + \ldots)} \\
	    =1-1 + 1-1 + 1-1 + 1-1 + \cdots \\ -1 + 1-1 + 1-1 + 1-1 + \cdots\\ + 1-1 + 1-1 + 1-1 + \cdots\\ \cdots
	\end{eqnarray*}
	Summing each column we see that we fall back on:
   
	Therefore:
	
   But as $S = 1/2$, then:
   
   Therefore:
   
   That is (to freak out a last time), we have shown that:
   
	Quite interesting! But don't forget that we get this result in a special framework that we have build and that generalize the conventional sum that we learn at high-school. It seems that all generalizations of the conventional sum (as mathematicians do definitions with coffee as we know...) lead to the same result of $-1/12$...
	
	It is also quite chocking that as natural numbers are closed under addition to that the sum of all naturals is $-1/12$...
	
	Furthermore, all this stuff is not new! It was known by many people long time before (few hundred years) and it was especially Srinivasa Ramanujan and later Godfrey Harold Hardy in a book titled \textit{Divergent Series} that formalized the subject in a more elegant and technical way than the one you have here above.
	\begin{figure}[H]
		\centering
		\includegraphics{img/algebra/ramanujan_series.jpg}
		\caption[]{Piece of Ramanujan's first notebook about $-1/12$}
	\end{figure}
	
	\pagebreak
	\subsubsection{Taylor and Maclaurin Series}\label{taylor series}
	Taylor and Maclaurin series provide a convenient and powerful tool to simplify theoretical models and computer calculations (fluid modeling or fields in space). They are used heavily in all fields of physics but they are also found in the industry including engineering (design of experiments, numerical methods, quality management), statistics (integral approximations), finance (stochastic processes ), complex analysis... We strongly advise the reader to read carefully the developments that follow. There is also a serious funny quote on the subject:
	\begin{fquote}Everything is just some form of first order Taylor Expansion!\end{fquote}
	To start, consider a polynomial (with one variable/univariate):
	
	We trivially have for this latter:
	
	Given now the derivative of the polynomial $P (x)$:
	
	Therefore:
	
	and so on with $P''(x), P'''(x), ...$ such that:
	
	Then:
	
	Therefore:
	
	relation that we name "\NewTerm{limited Maclaurin series}\index{limited Maclaurin series}" or simply "\NewTerm{Maclaurin series}\index{Maclaurin series}" of order $k + 1$.
	\begin{tcolorbox}[title=Remark,colframe=black,arc=10pt]
	In practice, as we will see in many other sections of this book, we often use limited developments of order $1$ (also named "\NewTerm{affine approximations}\index{affine approximation}", or "\NewTerm{affine tangent approximations}\index{affine tangent approximations}"), which can facilitate the calculations, when we do not expect too much precision in the final result.
	\end{tcolorbox}
	Now by applying the same reasoning but by centering the value of the polynomial on $x=x_0$, we have:
	
	and so the previous development becomes more general:
	
	which is no other than the general expression of a polynomial expression in a form named "\NewTerm{limited Taylor series}\index{limited Taylor series}" of order $k + 1$. This function can be assimilate to a polynomial as $n$ is finite. But if $n$ is infinite, as we shall see later, this series converges to the function we are seeking the representation in the form of a sum of terms.
	
	Thus, some functions $f (x)$ of class $\mathcal{C}^n$ that can be approximated by a polynomial $P(x)$ (a sum of powers in other words...) centered on the value $x_0$ can be expressed as:
	
	Result often referred to as "\NewTerm{Taylor's theorem}\index{Taylor's theorem}".
	
	But this last relation is not correct for all functions that can not be expressed as a polynomial. Therefore we say that the series is not convergent for them. We will see an example later below.
	
	The latter relations is sometimes also written ... more conventionally:
	
	In finance (and not only!), we will often use the following rearrangement:
	
	Let us return briefly to the approximation of $f (x)$ near and centered in $x_0$:
	
	Another very common notation in physics and financial engineering of the above relation is\label{differential expression of Taylor series}:
	
	Some people do not like using this formulation because we have the risk to forget that the approximation for a few terms is only good as long as we are not too far from $x_0$ with $x$. This is why it often happens that we write:
	
	with $x_0$ fixed and a $h$ variable but small (!) and so it then comes a current form of notation of Taylor series:
	
	with $x_0$ fixed and $h$ variable but small and therefore it comes a common other notation of Taylor series (!):
	
	Let's see an application example with Maclaurin series (with $x_0$ being zero) of the function $\sin (x)$ and Maple 4.00b:
	
	\texttt{>p[n](x) = sum((D@@i)(f)(a)/i!*(x-a)\string^i,i=0..n);\\
	>p11:= taylor(sin(x),x=0,12);\\
	>p11:= convert(p11,polynom);\\
	>with(plots):\\
	>tays:= plots[display](sinplot):\\
	for i from 1 by 2 to 11 do\\
	tpl:= convert(taylor(sin(x), x=0,i),polynom):\\
	tays:= tays,plots[display]([sinplot,plot(tpl,x=-Pi..2*Pi,y=-2..2,\\
	color=black,title=convert(tpl,string))]) od: \\
	>plots[display]([tays],view=[-Pi..2*Pi,-2..2]);}

	\begin{figure}[H]
		\centering
		\includegraphics{img/algebra/maclaurin_sinus_serie.jpg}
		\caption{Approximation of the sine function by a Maclaurin development Maple 4.00b}
	\end{figure}
	We see well in this example that the Maclaurin series only allows to approach a function at a point with a limited number of points. But more terms we take (put $100$ terms in the Maple code above) more the validity is big on the whole domain of definition of the function. In fact it is possible to prove that the function $sin (x)$ is exactly expressible in Maclaurin series when the number of terms is infinite. We say then that its "rest" is zero.
	
	But this is not true for all functions! For example the function:
	
	
	\texttt{>p[n](x) = sum((D@@i)(f)(a)/i!*(x-a)\string^i,i=0..n); \\
	>p10:= taylor(1/(1-x\string^2),x=0,10);\\
	>p10:= convert(p10,polynom);\\
	>with(plots):\\
	>tays:= plots[display](xplot):\\
	for i from 1 by 2 to 10 do\\
	tpl:= convert(taylor(1/(1-x\string^2), x=0,i),polynom):\\
	tays:= tays,plots[display]([xplot,plot(tpl,x=-2..2,y=-2..2,
	color=black,title=convert(tpl,string))]) od: \\
	>plots[display]([tays],view=[-2..2,-2..2]);}
	\begin{figure}[H]
		\centering
		\includegraphics{img/algebra/maclaurin_nonconvergent_serie.jpg}
		\caption{Example of non-convergent Maclaurin serie Maple 4.00b}
	\end{figure}
	We see above that regardless of the number of terms that we take the Maclaurin series converges only in one area of definition between $] -1,1 [$. This interval is named the "\NewTerm{radius of convergence}\index{radius of convergence}" and it determination (the singularity) is crucial in many areas of engineering, physics and analysis. We will return in mure more detail on this example in the section of Complex Analysis.
	
	But we can shift the Maclaurin series of the previous function to approximate the function with a Taylor series in other non-singular point such as in $x_0$ having for value $2$:
	
	\texttt{>p[n](x) = sum((D@@i)(f)(a)/i!*(x-a)\string^i,i=0..n);\\
	>p10:= taylor(1/(1-x\string^2),x=2,10);\\
	>p10:= convert(p10,polynom);\\
	>with(plots):\\
	>tays:= plots[display](xplot):\\
	for i from 1 by 2 to 10 do\\
	tpl:= convert(taylor(1/(1-x\string^2), x=2,i),polynom):\\
	tays:= tays,plots[display]([xplot,plot(tpl,x=0..5,y=-2..2,\\
	color=black,title=convert(tpl,string))]) od: \\
	>plots[display]([tays],view=[-0..5,-2..2]);}
	
	\begin{figure}[H]
		\centering
		\includegraphics{img/algebra/maclaurin_nonconvergent_serie_shifted.jpg}
		\caption{Shift possibility of Maclaurin serie in Maple 4.00b}
	\end{figure}
	
	We will study a generalization to the complex plane of Taylor series in the section of Complex Analysis to get a veeeeery powerful result for physicists to calculate complicated curvilinear integrals.
	
	\pagebreak
	\paragraph{Usual Maclaurin developments}\label{usual maclaurin developments}\mbox{}\\\\
	We will prove here the most frequent Maclaurin developments (about ten) to the second order that we can meet in theoretical and mathematical physics (in fact we heve developed here only use almost everywhere in the book). The list is not exhaustive for the time being but as the proof below are generalized, they can be applied to many other cases (that we will apply/meet throughout this book).
	
	\begin{tcolorbox}[title=Remark,colframe=black,arc=10pt]
	The Taylor expansions (that is to say elsewhere than on zero) are very rare (there are one or two in this entire book but they are detailed in their respective sections), we will omit them.
	\end{tcolorbox}
	
	\begin{enumerate}
		\item Taylor-Maclaurin development of $f(x)=e^x$:
		
		First remember that we have proved in the section of Differential and Integral Calculus that:
		
		Therefore we have:
		
		More generally:
		
		And therefore we have the famous result for $x=1$\label{euler maclaurin expansion}:
		
		that is sometimes named the "\NewTerm{exponential sequence}\index{exponential sequence}".
		
		\item  Taylor-Maclaurin development of $f(x)=\sin(x)$:
		
		First remember that we have proved in the section of Differential and Integral Calculus that:
		
		Therefore we have:
		
		
		\item  Taylor-Maclaurin development of $f(x)=\cos(x)$:
		
		First remember that we have proved in the section of Differential and Integral Calculus that:
		
		Therefore we have\label{cosine maclaurin dev}
		
		\item  Taylor-Maclaurin development of $f(x)=\tan(x)$:
		
		First remember that we have proved in the section of Differential and Integral Calculus that:
		
		Therefore we have:
		
		
		\item  Taylor-Maclaurin development of $f(x)=\arctan(x)$:
		
		First remember that we have proved in the section of Differential and Integral Calculus that:
		
		Therefore we have:
		
		
		\item  Taylor-Maclaurin development of $f(x)=\displaystyle\frac{1}{1+x}$:
		
		First remember that we have proved in the section of Differential and Integral Calculus that:
		
		Therefore we have:
		
		It then follows immediately another Taylor series we will also meet again many  number of times:
		
		For our study of the CMB in the section of cosmology (especially recombination era), we will need a result related to the two series above named the "\NewTerm{Binomial theorem for negative integer exponents}\index{binomial theorem for negative integer exponents}\label{binomial theorem for negative integer exponents}". Given the binomial coefficient given for recall by:
		
		and $n$ be a positive integer. Then (without proof but using "engineer intuition"):
		
		for $|x|<1$.
		
		\item  Taylor-Maclaurin development of $f(x)=\sqrt{1+x}$:
		
		First remember that we have proved in the section of Differential and Integral Calculus that:
		
		Therefore we have:
		
		It then also follows immediately another Taylor series we will also meet again many  number of times:
		
		
		\item  Taylor-Maclaurin development of $f(x)=\ln(1+x)$\label{maclaurin dev natural logarithm}:
		
		First remember that we have proved in the section of Differential and Integral Calculus that:
		
		Therefore we have:
		
		
		\item  Now consider the important case for the Langevin model of paramagnetism that is approximated Taylor expansion of the hyperbolic cotangent function (\SeeChapter{see section Trigonometry page \pageref{hyperbolic trigonometry}}), that is for refresh defined by the relation:
		
		For this, we will use the Landau notation, with expressions like $\mathcal{O}(x^n)$ remembering that we proved a little before above:
		
		when $x \rightarrow 0$.
		For the hyperbolic cotangent we have then:
		
		Now we must be remember as we have just proved a little earlier that:
		
		for $\vert x \vert < 1$. Therefore:
		
		and finally replacing this in the previous expression we find:
		
		
		\item  Another famous Maclaurin series used thousand of times in the world for business application is the computation of the numerical values of the Normal distribution:
		
		So first to simplify this integral, we typically let:
		
		that we know already (\SeeChapter{see section Statistics page \pageref{reduced centered variable}}) as being the $z$ score of a data value. With this simplification, the integral above becomes:
		
		The Maclaurin series for $e^{-x^2/2}$ is given by:
		
		Therefore:
		
		Therefore:
		
		Is is obvious that the constant will eliminate itself. Therefore!
		
		and in the common case in business where $a=0$ we get (with two terms only):
		
	\end{enumerate}
	
	\pagebreak
	\paragraph{Taylor series of bivariate functions (multivariate Taylor series)}\label{multivariate taylor series}\mbox{}\\\\
	We will see now how to approach a function $f (x, y)$ of two real variables by a sum of powers (Taylor series). This type of approximation is widely used in many fields of engineering (see sections of Industrial Engineering page \pageref{bivariat taylor expansion doe} and Numerical Methods page \pageref{newton raphson method}).
	
	We are looking for an approximation of $f (x, y)$ at point $f(x_0+h,y_0+h)$. For this, let us write (a priori nothing prohibits us from doing so) that:
	
	Then we have:
	
	The value of (the trick is here!):
	
	can be approximated using a Taylor series around the value $0$ such that:
	
	But we have:
	
	and:
	
	According to Schwarz's theorem (\SeeChapter{see section Differential and Integral Calculus page \pageref{Schwarz theorem}}):
	
	Then we have:
	
	and we show by induction that:
	
	Therefore we finally get:
	
	or in another equivalent simplified form:
	
	Or if we define a matrix $H$ named "\NewTerm{Hessian matrix}\index{Hessian matrix}\label{hessian matrix}" given by:
	
	we can also write:
	
	In Maple 4.00b we use the following command to make a development of order $3$ around $0$:
	
	\texttt{>readlib(mtaylor):\\
	>mtaylor(f(x,y), [x,y], 3);}
	
	\begin{tcolorbox}[colframe=black,colback=white,sharp corners]
	\textbf{{\Large \ding{45}}Example:}\\\\
	Let us see an example with the famous humpback whale:\\
	
	\texttt{>with(plots): with(plottools):\\
	>readlib(mtaylor):\\
	>fct:=x\string^2*(4-2.1*x\string^2+1/3*x\string^4)+x*y+y\string^2*(-4+4*y\string^2);\\
	>poly2 :=mtaylor(fct,[x=1,y=1],6);\\
	>\#Convert all the coefficients to floating point numbers\\
	>poly2n := map(evalf,poly2):\\
	>gr1:= plot3d(poly2n,x=-2..2,y=-1..1,color=red):\\
	>gr2:= plot3d(fct,x=-2..2,y=-1..1,color=blue):\\
	>display3d({gr1,gr2},view=-3..8,axes=framed);
	}
	\begin{figure}[H]
		\centering
		\includegraphics{img/algebra/taylor_multivariate.jpg}
		\caption{Bivariate Taylor example with Maple 4.00b}
	\end{figure}
	\end{tcolorbox}
	
	The previous relation can also be written in the multivariate case in a point $\vec{x}$ near the origin of the coordinate system $\vec{p}$ as following (common notation in computer science):
	
	where:
	
	
	\pagebreak
	\paragraph{Quadratic Form}\mbox{}\\\\
	Now we will need for the section of Theoretical Computing to state an important property (which would have also its place only in the section of Differential and Integral Calculus):
	
	Let $f$ be a function defined and derived over an interval $I$ and given $a$ an element of $I$. If $f$ is such that $f'(a)=0$ then we say it has a "\NewTerm{local extremum}\index{local extremum}\label{local extremum}" on $a$.
	
	\begin{tcolorbox}[title=Remark,colframe=black,arc=10pt]
	The reciprocal is false, the function $x^3$ is such an example. Its derivative is zero at $0$ but there is no local extreme at this point. So be careful!
	\end{tcolorbox}
	
	However, let $f$ be a function defined and derived over an interval $I$ and given $a$ an element of $I$. If $f$ is such that $f'(a)=0$ and if $f'$ changes sign in $a$ then $f$ has a local extremum at $a$.
	
	To return now to our bivariate development Taylor, we know that if $(x_0,y_0)$ is a local extremum of $f$ then we have in a first time (\SeeChapter{see section Differential and Integral Calculus page \pageref{local extremum}}):
	
	However we have seen that this condition is not sufficient to ensure that $(x_0,y_0)$ is a local extremum.
	
	Reconsider the Taylor expansion of $f$ above taking into account the above condition. Development simplifies then to:
	
	Then we know that by definition so that the $(x_0,y_0)$ is a local minimum (respectively a local maximum) it is sufficient that the expression in brackets is positive (respectively negative). Since the second derivatives of $f$ are continuous, it is sufficient that the expression:
	
	to be positive (negative resp.) regardless of $h$ or $k$ and it is zero only if $h=k=0$. Then we say that $q$ is a "\NewTerm{positive definite quadratic form (resp. negative definite)}\index{positive definite quadratic form}".
	
	To simplify writing and to comply with traditions we put now:
	
	Then we can rewrite $q$ as follows:
	
	where $H$ remains the Hessian matrix of $f$ evaluated on $(x_0,y_0)$.
	
	So we see that $q$ is positive definite (local minimum) if $a>0$ and $\det(H)>0$, negative definite (local maximum) if $a<0$ and $\det(H)>0$.
	
	Returning to the partial derivatives these conditions are described as follows:
	\begin{itemize}
		\item Positive definite (local minimum) if:
		
		
		\item Negative definite (local maximum) if:
		
		
		\item Indefinite if:
		
	\end{itemize}
	Finally we see that the sign of the determinant of Hessian matrix and that of $\dfrac{\partial^2 f}{\partial x^2}(x_0,y_0)$ allow us to obtain a sufficient condition to determine if we are in the presence of a local extremum.
	
	If we now consider a general $2\times 2$ symmetric matrix:
	
	It should be almost obvious to the reader that this matrix induces, in the sens of the Hessian matrix, the quadratic form:
	
	If $y = 0$, then we have $\mathrm{Q}_A(x,0) = ax^2$ (parabola), so we must certainly have $a > 0$ in order for $A$ to be positive definite.
	
	The following examples illustrate that in general, it cannot easily be determined whether a symmetric matrix is positive definite from inspection of the entries.
	\begin{tcolorbox}[colframe=black,colback=white,sharp corners]
	\textbf{{\Large \ding{45}}Examples:}\\\\
	E1. As first example, consider the matrix:
	
	Then:
	
	and we have whatever $(x_0,y_0)$:
	
	Therefore, even though all of the entries of $A$ are positive, $A$ is not positive definite.
	\end{tcolorbox}
	\begin{tcolorbox}[colframe=black,colback=white,sharp corners]
	E2. Consider the matrix:
	
	Then:
	
	which can be seen to be always nonnegative. Furthermore, $ \mathrm{Q}_A(x,y) =0 $ if and only if $x = y$ and $y =0$, so for all nonzero vectors $(x,y)$, $ \mathrm{Q}_A(x,y) >0 $ and $A$ is positive definite, even though $A$ does not have all positive entries.\\
	
	E3. Let:
	
	\begin{figure}[H]
		\centering
		\includegraphics[scale=0.7]{img/algebra/hessian_exp_x_y.jpg}
	\end{figure}
	We have:
	
	which yields the critical points $(x, x)$ for all $x \in \mathds{R}$. We also have:
	
	which yields:
	
	That is,$\mathrm{H}f(x,x)$, is positive semidefinite, making $(x,x)$ a global minimizer of $f(x,y)$.
	\end{tcolorbox}
	
	\begin{tcolorbox}[colframe=black,colback=white,sharp corners]
	E4. Let:
	
	\begin{figure}[H]
		\centering
		\includegraphics[scale=0.7]{img/algebra/hessian_x3-12xy.jpg}
	\end{figure}
	We have:
	
	which yields the critical points $(0, 0)$ and $(2, 1)$. We also have:
	
	and therefore:
	
	We see that $\mathrm{H}f(2, 1)$ is positive definite, because its principal minors are positive, but $\mathrm{H}f(0, 0)$ is not, as $\Delta_1 = 0$ and $\Delta_2 = -144$. That is, $\mathrm{H}f(0, 0)$ is indefinite, so $(0, 0)$ is a saddle point.
	\newline
	Furthermore:
	
	so $f(x, y)$ has no global minimizer on $\mathds{R}^2$. We can conclude, however, that $(2, 1)$ is a strict local minimizer.\\
	
	It should be emphasized that if the Hessian is positive semidefinite or negative semidefinite at a critical point, then it cannot be concluded that the critical point is necessarily a minimizer, maximizer or saddle point of the function.
	\end{tcolorbox}
	
	
	\paragraph{Lagrange Remainder}\label{Lagrange Remainder}\mbox{}\\\\
	There may be an interest for some numerical applications (\SeeChapter{see section Numerical Methods page \pageref{numerical methods}}) to know the approximation error of the polynomial $P_n(x)$ in relation to the function $f(x), \forall x$.
	
	Let us define for this purpose a "remainder", such that:
	
	\begin{theorem}
	The function $R_n(x)$ is named "\NewTerm{Lagrange rest}\index{Lagrange rest}" or "\NewTerm{Lagrange remainder}\index{Lagrange remainder}" or "\NewTerm{Lagrange error}\index{Lagrange error}".
	
	\end{theorem}
	
	\begin{dem}
	Given a function $g(t)$ defined by the difference of a function $f(x)$ assumed to be known and a Taylor approximation of the same function:
	
	with, of course:
	
	We see that $g (t)$ vanishes as expected for value $t=x$.
	
	Now let us derive $g(t)$ with respect to $t$, we find:
	
	After simplification:
	
	According to Rolle's theorem (\SeeChapter{see section Differential and Integral Calculus page \pageref{rolle theorem}}), there exist a value $t=z$ for which the derivative $g'(t)$ is zero. So:
	
	We can simplify the equation by $(x-z)^n$:
	
	which can also be written as:
	
	so we find for the maximum of $R_n$:
	
	\begin{flushright}
		$\blacksquare$  Q.E.D.
	\end{flushright}
	\end{dem}
	We see that as the polynomial $P_n(x)$ is of high degree, the more it approximates the function $f (x)$ with accuracy. What will happen when $n\rightarrow +\infty$?:
	
	Suppose that $f (x)$ has derivatives of all orders (what we denote for reminder $\mathcal{C}^n$) for all values of any interval containing $x_0$ and let the rest of Lagrange $R_n$ of f (x) of $f(x)$ on $x_0$. If, for any $x$ in the range:
	
	then $f (x)$ is exactly represented by P $(x)$ on the interval.
	\begin{dem}
	The proof simply stems from the expression of $P_n(x)$ when $n\rightarrow +\infty$.
	
	Indeed, if we take an infinity of terms for $P_n(x)$, the correspondence with the approximated function is perfect and so the rest is zero.
	\begin{flushright}
		$\blacksquare$  Q.E.D.
	\end{flushright}
	\end{dem}
	The polynomial:
	
	is named "\NewTerm{Taylor polynomial}\index{Taylor polynomial}\label{Taylor polynomial}" or "\NewTerm{Taylor series}\index{Taylor series}". If $x_0=0$, it is named "\NewTerm{Maclaurin polynomial}\index{Maclaurin polynomial}" or "\NewTerm{Maclaurin series}\index{Maclaurin series}".
	
	\paragraph{Taylor Series with Integral Remainder}\mbox{}\\\\
	We'll see if a theorem that will be useful in the section of Statistics to link the Poisson and Chi-2 laws and that is used in statistical software for Poisson test of rare events (that is the only business practical application that is known to us at this day).
	\begin{tcolorbox}[title=Remark,colframe=black,arc=10pt]
	If anyone has a more educational proof whose beginning is a little less "formula fell from the sky", we are takers!
	\end{tcolorbox}
	\begin{theorem}
	Let $f(x)$ be $n + 1$ times differentiable on the interval $[a, b]$. Then we have:
	
	where it is important (for the good understanding of what we will do in the section of Statistics) that the reader notices in the development that when the derivative stops at the $n$-th term in the series, the integral (the remainder) has a factor of $1 / n !$, a power $n$ and a derivative of order $n + 1$. So verbatim, as we shall prove it below, if we stop the development of the terms to $n-1$, the integral (the remainder) will have a factor of $1 / (n-1) !$, a power $n-1$ and a derivative of order $n$-th.
	\end{theorem}
	\begin{dem}
	The proof is mady by induction. We first consider the formula fallen from the sky:
	
	We show that it is correct for $k = 0$, then we do an induction on $k$ for $k\in \mathbb{N}$.
	
	For $k = 0$, we have the well-known relation (\SeeChapter{see section Differential and Integral Calculus page \pageref{fundamental theorem of calculus}}):
	
	Suppose the property true for $k<n$:
	
	We integrate by parts (\SeeChapter{see section Differential and Integral Calculus page \pageref{integration by parts}}) the term:
	
	Then we have:
	
	Therefore:
	
	\begin{flushright}
		$\blacksquare$  Q.E.D.
	\end{flushright}
	\end{dem}
	
	\subsubsection{Fourier Series (trigonometric series)}\label{fourier series}
	We name by definition "\NewTerm{trigonometric series}\index{trigonometric series}" a series of the form:
	
	or in a more compact form:
	
	The constants $a_n,b_n$ with $n\in \mathbb{N}^{*}$ are the coefficients of the trigonometric series usually named "\NewTerm{Fourier coefficients}\index{Fourier coefficients}".
	
	\begin{tcolorbox}[title=Remark,colframe=black,arc=10pt]
	We have already mentioned this type of series in our study of the types of existing polynomials since Fourier series are in fact only trigonometric polynomials (\SeeChapter{see section Calculus page \pageref{trigonometric polynomials}}). Furthermore, we saw as example in the section of Functional Analysis during our study of scalar functional product that the sine and cosine functions were the bases of a vector space!
	\end{tcolorbox}
	If the series converges, its sum is a periodic function $f (x)$ of period $T=2\pi$, since $\sin (nx)$ and $\cos (nx)$ are periodic functions of period $2\pi$. So that:
	
	Let us now state the following problem: We give ourselves a known periodic function $f(x)$, piece-wise continuous of period $2\pi$. We ask ourselves if there is a trigonometric series converging to $f (x)$ under some conditions that must be satisfied on this series.
	
	Suppose now that the function $f (x)$, periodic and of period $2\pi$, can be effectively represented by a trigonometric series converging to $f (x)$ in the interval $[0, T]$, that is to say it the sum of this series:
	
	Suppose that the integral of the function of the left member of this equality is equal to the sum of the integral of all the terms of the above series. This will occur, for example, if we assume that the proposed trigonometric series converges absolutely, that is to say, the numerical series converges (by the property that the trigonometric functions are bounded):
	
	The serie:
	
	is then majorable and can be integrated term by term from $0$ to $T$ (where $T=2\pi$) which allows us to determine the different Fourier coefficients. But before we start let us present the following integrals that will be very useful later:
	
	\begin{center}
	\begin{tabular}{ccc}
	$\text{with }n,k\in \mathbb{N}\text{ and }n\ne k$
	&$\qquad$&
	$\text{with }n,k\in \mathbb{N}\text{ and }n = k$
	\end{tabular}
	\end{center}
	Before continuing, let us prove the value taken by these six integrals (following the request of readers). But first, remember that as $n,k \in \mathbb{N}$ then:
	
	
	\begin{enumerate}
		\item We proceed using the remarkable trigonometric relations (\SeeChapter{see section Trigonometry page \pageref{remarkable trigonometric identities}}) and the primitive of elementary trigonometric functions (\SeeChapter{see section Differential and Integral Calculus page \pageref{usual primitives}}):
		
		because as we have seen it in the section Trigonometry $\sin(k\pi)=0,k\in\mathbb{Z}$ and as $T=2\pi$ the two previous differences have all terms equal to zero such that at the end:
		
		
		\item For the second integral, we proceed using the same techniques and the same properties of trigonometric functions:
		
		
		\item And we continue like this also for the third one, according to the same properties:
		
		
		\item Once gain using the same methods (this becomes routine ...) first for $k\neq 0$:
			
			and for $k=0$ it comes immediately:
			
			
			\item Again ... (soon finish...) first for $k\neq 0$:
			
			and for $k=0$ it comes immediately:
			
			
			\item And finally the last (...):
				
		\end{enumerate}
		This small work done let us now come back on our topic... To determine the coefficients $a_n$ both members of equality:
	
	by $\cos(kx)$:
	
	The series of the second member of equality is majorable, since its terms do not exceed in absolute to the terms of the positive convergent series. So we can integrate term by term on every bounded segment $0$ to $T$:
	
	We have proved above that whatever the integer values that take $k$ or $n$ the second term in the parenthesis is always zero. It then remains only:
	
	But we have proved above that the integral on the right is always zero if $n$ and $k$ are different. This leaves only the case where $n$ and $k$ are equal. Meaning:
	
	In this situation, we first the special case where $k$ is zero. In that case:
	
	Therefore:
	
	It is obvious that the coefficient $a_0$ represents the average of the signal or of its DC component, if it exists.
	
	In the case where $k$ it is not zero, we have:
	
	Therefore:
	
	To determine the coefficients $b_n$ we proceed the same way but this time multiplying both members of equality by $\sin(kx)$:
	
	The series of the second member of equality is majorable because its terms are not higher in absolute values to the terms in the convergent positive series. So we can integrate term by term on every bounded segment from $0$ to $T$:
	
	We have shown proved before that whatever are the integer values that taken by $k$ or $n$ the first term of the parenthesis is always zero. It remains then only:
	
	
	But we have proved before that the integral on the right is always zero if $n$ and $k$ are different. This leaves only the case where $n$ and $k$ are equal. Meaning:
	
	In this situation, we first have the special case where $k$ is zero. But we see now that we have a zero indeterminacy. It is better to consider the general case from which have:
	
	Hence we easily derive that:
	
	Therefore, for the situation where $k$ is zero the coefficient is therefore equal to zero!
	
	So finally the Fourier coefficients are determined by the integrals:
	
	But as it's annoying to have three results for the coefficients we'll play a little with the definition of the Fourier series.
	
	Indeed by summing from $1$ to $+\infty$, rather than $0$ to $+\infty$, we have:
	
	This then allows us only to have to remember ($a_0$ therefore included!):
	
	Physicists have for habit to write the last two relations as follows:
	
	The possible decomposition of any periodic piecewise continuous function approximated by an infinite sum of trigonometric functions (sine or cosine) consisting of a basic function and its harmonics is named "\NewTerm{Fourier theorem}\index{Fourier theorem}" or "\NewTerm{Fourier-Dirichlet theorem}\index{Fourier-Dirichlet theorem}".
	\begin{figure}[H]
		\centering
		\includegraphics{img/algebra/fourier_series_examples.jpg}
		\caption[Examples of some Fourier series]{Examples of some Fourier series (source: Mathworld)}
	\end{figure}
	It can also happen that sometimes we know the Fourier series and we are looking for the original function $f(x)$. As a companion example consider that we want to calculate:
	
	So this is like searching the original $f(x)$ of the above Fourier series.

	It follows therefore that $a_0=0$ and $b_n=0$ and:
	
	But as far as we know there is no easy way to extract $f(x)$ that seems accurate! So using hyperbolic trigonometry (\SeeChapter{see section Trigonometry \pageref{hyperbolic trigonometry}}), we write:
	
	Now these power series may be identifies as Maclaurin expansions of $-\ln(1-z)$ (see proof above) with $z=e^{\mathrm{i}x}$ for the first term and $z=e^{-\mathrm{i}x}$ for the second term.

	Therefore:
	
	
	The Fourier series allows implicitly to represent all the frequencies in a periodical signal whose function is known mathematically (closed form). We can wonder why talk about Fourier series when, in practice, we do not really know the mathematical representation of a signal? This will bring us to a better understanding of the concept of the Fourier transform in discrete-time that we will see a little further, which does not need a mathematical representation of a continuous and periodic signal.
	\begin{figure}[H]
		\centering
		\includegraphics[scale=0.5]{img/algebra/time_frequency_domain.jpg}
		\caption{Time\index{time domain} and frequency domain\index{frequency domain}}
	\end{figure}
	We notice also that if $f (x)$, that is to say the periodic function of which we seek expression in trigonometric Fourier series, is even then the series will also be even and thus contain only cosine terms (the cosine function being an even function) implying that $b_n=0$ and otherwise in the case of an odd function $a_n=0$ (the sine being for reminder an odd function)!
	
	It should be noted, and this is important for what will follow, that as we have seen in the section of Calculus during our study of trigonometric polynomials, Fourier series could be written in the following complex form (by changing some notations and passing the sum to infinity):
	
	and we have seen that (always in the section of Calculus) that:
	
	Therefore:
	
	This gives us:
	
	Therefore:
	
	Or more generally:
	
	So if we take the famous case where $t_0=-T/2$ we get:
	
	Obviously the two relation above are the Fourier coefficient in the "time space" (or "time domain") point of view. We will see in the section of Electrical Engineering that there is an equivalent if we work with Fourier series in the frequency space (or "frequency domain").
	
	\begin{tcolorbox}[colframe=black,colback=white,sharp corners]
	\textbf{{\Large \ding{45}}Examples:}\\\\
	E1. Upon decomposition of a continuous signal, we say (improperly at our point of view) that the coefficients $a_n,b_n$ are each (implicitly) a separate frequency associated with an amplitude that we visualize on a graph by vertical lines. This graph shows the "\NewTerm{frequency spectrum}\index{frequency spectrum}" of the decomposed signal. We can also add another representation which is named "\NewTerm{phase spectrum}\index{phase spectrum}". This spectrum gives us the phase of the harmonic signal (in phase advance or delay).
	\begin{figure}[H]
		\centering
		\includegraphics{img/algebra/fourier_spectrum_graph.jpg}
		\caption{Example of amplitudes and frequencies associated to the different coefficients}
	\end{figure}
	Let us see now how to decompose a known periodic signal into several distinct amplitudes and frequencies signals
	Let us take for example, a periodic square wave signal defined over a period $T = 2$ and of amplitude $A$ such that:
	
	At period $T = 2$ corresponds as we know a pulsation:
	
	\end{tcolorbox}
	
	\pagebreak
	\begin{tcolorbox}[colframe=black,colback=white,sharp corners]
	Let us calculate first the coefficients $c_k$ thanks to the integral that determine the coefficients (the choice of the bounds of the integral is therefore assumed that the signal is periodic by construction!):
	
	Taking $k = 2$, we have:
	
	Similarly for $k = 4,6,8$ and for any even number.\\
	
	About odd numbers, we will have:
	
	The coefficients will then be:
	
	There is only problem in this relations, the coefficient $c_0$ cannot be calculated according to this relation because you can see that if $k = 0$ in the result above, we have an infinite value and it is at least impossible. The coefficient is null or not null  but never infinite (at least in physics because this implies and infinite energy).\\
	
	To find the coefficient $c_0$, we must calculate the integral for $k = 0$. The coefficient $c_0$ is then determined by:
	
	\end{tcolorbox}
	
	\pagebreak
	\begin{tcolorbox}[colframe=black,colback=white,sharp corners]
	The "frequency" spectrum (caution to the abuse of language!) and amplitude will be of the following form for $k=-5...+5$ and $A=1$ null frequencies not being shown:
	\begin{figure}[H]
		\centering
		\includegraphics{img/algebra/fourier_coefficients_example.jpg}
		\caption{Frequency spectrum of the example Fourier series coefficients}
	\end{figure}
	E2. Let us consider another famous case! The signal decomposition of the  square plus of amplitude $A$ and width $T_p$ (this function is sometimes denoted $\Pi_P(t)$) and of period $T$:
	\begin{figure}[H]
		\centering
		\includegraphics{img/algebra/square_pulse.jpg}
	\end{figure}
	
	\end{tcolorbox}
	\begin{tcolorbox}[colframe=black,colback=white,sharp corners]
	
	So we fall back on a discrete version of the famous cardinal sine function (\SeeChapter{see section Trigonometry page \pageref{sinc cardinal}})!
	\end{tcolorbox}
	The abuse to talk about "frequencies" for Fourier coefficients thus leads us to have negative frequencies on the  $x$-axis... but it's only a question of vocabulary (there is no direct relation with the real frequencies) with which you must be familiar.
	
	The amplitude spectrum and phase is calculated according to the following relations:
	
	It is then relatively easy to notice that if $T$ tends to a larger and larger number, the spectrum peaks approach increasingly. So when $T$ tends to infinity the spectrum becomes continuous!!!
	
	The phase spectrum of the above example will give the following for the odd values:
	\begin{figure}[H]
		\centering
		\includegraphics{img/algebra/fourier_phase_diagram.jpg}
		\caption[]{Phase spectrum for the Fourier Series}
	\end{figure}
	It is even possible for example to obtain relatively easily the frequency spectrum in a software like Microsoft Excel 11.8346 (the reader will find an example much more detailed and interesting on the companion exercise server in the section Sequences and Series) !!!
	
	Indeed, it is enough for this purpose to sample for example our signal $128$ times (Microsoft Excel 11.8346 needs $2^n$ samples and works only under this condition!). Then we divide the interval $-1<t<0$ in $64$ samples and ditto for the interval $0>t>1$:
	\begin{figure}[H]
		\centering
		\includegraphics{img/algebra/signal_sample_01.jpg}\\
		\includegraphics{img/algebra/signal_sample_02.jpg}\\
		\includegraphics{img/algebra/signal_sample_03.jpg}
		\caption[]{Signal sample}
	\end{figure}
	Which gives in graphical form (be careful because for the discrete Fourier transform works well in Microsoft Excel 11.8346, it is necessary that the sampling frequency - corresponding to the number of measurements in a second - is at least 100 times higher than the frequency of the original signal otherwise the result can be absurd!):
	\begin{figure}[H]
		\centering
		\includegraphics{img/algebra/signal_fourier_transform_excel.jpg}
		\caption[]{Graphical representation of the data series in Microsoft Excel 11.8346}
	\end{figure}
	Afterwards in Microsoft Excel you simply go to the menu \textbf{Tools/Utility Analysis} and choose the \textbf{Fourier Analysis} option:
	\begin{figure}[H]
		\centering
		\includegraphics{img/algebra/excel_data_analysis_tool.jpg}
		\caption[]{Screenshot of \textbf{Utility Analysis} dialog box of Microsoft Excel 11.8346}
	\end{figure}
	Then comes the following dialog box that must be fill-in as indicated below (we see that the $x$-axis does not matter!):
	\begin{figure}[H]
		\centering
		\includegraphics{img/algebra/excel_fourier_analysis_dialog_box.jpg}
		\caption[]{Parameters of the \textbf{Fourier Analysis} tool in Microsoft Excel 11.8346}
	\end{figure}
	Then comes the following generated list for the coefficients:
	\begin{figure}[H]
		\centering
		\includegraphics{img/algebra/fourier_coefficients_excel_list_01.jpg}
	\end{figure}
	\begin{figure}[H]
		\centering
		\includegraphics{img/algebra/fourier_coefficients_excel_list_02.jpg}
	\end{figure}
	\begin{figure}[H]
		\centering
		\includegraphics{img/algebra/fourier_coefficients_excel_list_02.jpg}
		\caption[]{Corresponding Fourier coefficients to sampled signal with Microsoft Excel 11.8346}
	\end{figure}
	It remains to calculate the module of the complex numbers with the native function  \texttt{IMABS( )} function in Microsoft Excel 11.8346 and divide the result by $128$ for each of the coefficients $c_n$ but we already see that each pair even coefficient  is zero and this match well the theoretical result obtained previously.
	
	We then have putting the index $n$ in front of each module:
	\begin{figure}[H]
		\centering
		\includegraphics{img/algebra/fourier_coefficients_excel_list_completed_01.jpg}
	\end{figure}
	\begin{figure}[H]
		\centering
		\includegraphics{img/algebra/fourier_coefficients_excel_list_completed_02.jpg}
	\end{figure}
	\begin{figure}[H]
		\centering
		\includegraphics{img/algebra/fourier_coefficients_excel_list_completed_03.jpg}
		\caption[]{Module of complex coefficients of the example Fourier Transform with Microsoft Excel 11.8346}
	\end{figure}
	By plotting a customized scatter diagram (still with  Microsoft Excel 11.8346) of columns D and E, we finally get (we restricted the $x$-axis to $[-5, +5]$ for easier reading:
	\begin{figure}[H]
		\centering
		\includegraphics{img/algebra/excel_fourier_spectrum_frequencies.jpg}
		\caption[]{Frequency spectrum of the transformed with Microsoft Excel 11.8346}
	\end{figure}
	To compare with the theoretical calculations (chart already presented previously) ...:
	\begin{figure}[H]
		\centering
		\includegraphics{img/algebra/fourier_coefficients_example.jpg}
		\caption{Theoretical frequency spectrum of the example Fourier series coefficients}
	\end{figure}
	Or in a more general and pedagogical way:
	\begin{figure}[H]
		\centering
		\includegraphics[scale=0.8]{img/algebra/frequencies_fourier_series.jpg}
	\end{figure}
	\begin{tcolorbox}[colframe=black,colback=white,sharp corners]
	\textbf{{\Large \ding{45}}Example:}\\\\
	Let us now consider another example identical to the previous with a different approach. We define a periodic function of period $T=2\pi$ as follows:
	
	Let us calculate the Fourier coefficients (we translate the bounds of the integral since the function is periodic this change nothing but facilitate the calculations!):
	
	and:
	
	We notice that $b_n$ is equal to $0$ for $n$ even and equal to $4\pi/n$ when $n$ is odd.
	The Fourier series of the function under consideration is thus written:
	
	What in Maple 4.00b will be written:\\
	
	\texttt{>S:=(4/Pi)*Sum(sin((2*n+1)*x)/(2*n+1),n=0..N);}
	\end{tcolorbox}
	
	\pagebreak
	\begin{tcolorbox}[colframe=black,colback=white,sharp corners]
	and that we can plot using the command:\\
	
	\texttt{>plot({subs(N=4,S),subs(N=8,S),subs(N=16,S)},x=-Pi..Pi,\\
	color=[red,green,blue],numpoints=200);}\\
	
	What gives three plots for $4$, $8$ and $16$ of the series in red, green and blue:
	\begin{figure}[H]
		\centering
		\includegraphics{img/algebra/fourier_series_with_various_terms.jpg}
		\caption{Example of Fourier series in Maple 4.00b with $4$, $8$ and $16$ terms}
	\end{figure}
	For $50$ terms we get:\\
	
	\texttt{> plot(subs(N=50,S),x=-Pi..Pi,numpoints=800);}\\
	\begin{figure}[H]
		\centering
		\includegraphics{img/algebra/fourier_series_with_fifty_terms.jpg}
		\caption{Example of Fourier series in Maple 4.00b with $50$ terms}
	\end{figure}
	\end{tcolorbox}
	We see on the example above the side effects named "\NewTerm{Gibbs phenomenon}\index{Gibbs phenomenon}". It is possible to prove they occur to the value of the abscissa corresponding to $x=\pi/2n$ and matching equation and the peak rises to $\pm 1.179$ for all values of $n$. Let's see this!
	
	We proved just before for our example that:
	
	Which can be written:
	
	using the proof made much more earlier at the beginning of this section as that:
	
	Then we have:
	
	Remember that during our study of complex numbers (\SeeChapter{see section Numbers page \pageref{complex numbers}}) we proved that:
	
	Which brings us to:
	
	We will now focus on small values of $x$. So we can then make a Maclaurin  development of first order at the denominator (but not the numerator because of the presence of the $n$):
	
	We make a change of variable:
	
	where we used the traditional notation of the "cardinal sinus" in the last relation as defined in the section trigonometry (remember that this fraction is common in physics this is why it has a specific notation).
	
	As what interest us is to determine the maximum of the Gibbs phenomenon (the disturbance), we see that it takes place in this particular case that we presented (see figure above) for each multiple of $\pi$ and as the denominator of the expression of the integral will decrease as the multiple is higher, it follows that the greatest maximum is at the point where $2nx=\pi$  (the point $0$ at the opposite will cancel the integral therefore we must this latter of our choice). Then we have:
	
	the evaluation of this integral can be done only numerically as far as we know, therefore we get:
	
	That is about $18\%$ above the expected threshold value.
	
	\paragraph{Fourier series generative art}\mbox{}\\\\
	Before continuing on hard mathematical topics about Fourier series, let us open a parenthesis where the purpose here take the science out of this quote\index{Fermi's elephant}:
	\begin{fquote}[J. Von Neumann]With four parameters i can fit an elephant.
 	\end{fquote}
 	With computer it is easy to drawing stuffs using points moving on a circle, that are themselves centered on another circle like illustrated below (an epicycloid):
 	\begin{figure}[H]
		\centering
		\includegraphics[width=0.5\textwidth]{img/algebra/generative_art_epycloid.jpg}
	\end{figure}
	or a more complicated example:
	\begin{figure}[H]
		\centering
		\includegraphics[width=0.5\textwidth]{img/algebra/generative_art_epycloid_three_points.jpg}
	\end{figure}
	Obviously as the examples above, the results change vary depending on the rotating speed of each point on its circle:
	\begin{figure}[H]
		\centering
		\includegraphics[width=0.7\textwidth]{img/algebra/generative_art_epycloid_three_points_various_speed.jpg}
	\end{figure}
	or a case with eight circle (each of them have a point moving at a specific rotation speed):
	\begin{figure}[H]
		\centering
		\includegraphics[width=0.7\textwidth]{img/algebra/generative_art_fermi_elephant.jpg}
	\end{figure}
 	is based in fact on Fourier series!
 	
	To understand why consider the following example:
	\begin{figure}[H]
		\centering
		\includegraphics[width=0.5\textwidth]{img/algebra/generative_art_two_points_with_equations.jpg}
	\end{figure} 
	The first point is described  obviously by the parametric equation:
	
	and the second point therefore by (if it turns two time faster than the angle in the main circle):
	
	This is the parametric equation of the following epicycloid:
	\begin{figure}[H]
		\centering
		\includegraphics[width=0.6\textwidth]{img/algebra/generative_art_epycloid_two_points.jpg}
	\end{figure} 
	In the general case we have:
	
	where $a_i$ are the radius of the circles and $n$ are the various circles speed (pulsation).
	
	But the question is rather the opposite. From a given curve, how can found the values of the parameters?
	
	The idea is quite simple. It's to follow respectively each coordinate in function of another one as illustrated below, where the closed shape is transformed into two curves:
	\begin{figure}[H]
		\centering
		\includegraphics[width=0.8\textwidth]{img/algebra/generative_art_epycloid_conversion_into_curves.jpg}
	\end{figure} 
	and as seen during our study of the Fourier series, we know how to found the Fourier series of such curves!
	
	We know that the Fourier coefficients are given by:
	
	If we do that numerically for example with $x(t)$:
	\begin{figure}[H]
		\centering
		\includegraphics[width=1\textwidth]{img/algebra/generative_art_epycloid_x_t_numerical_integration.jpg}
	\end{figure} 
	That gives:
	
	Therefore:
	
	and using the same procedure, we could find:
	
	Therefore:
	
	We see however that $x(t)$ doesn't have only cosine terms and that $y(t)$ doesn't have only  sine terms. Then playing a bit with elementary trigonometric identities we can found (it's not absolutely a necessary transformation but like this it will be in adequation with the general expression given earlier above):
	
	Now let us come back on the four parameter elephant challenge. A satisfying result is:
	\begin{figure}[H]
		\centering
		\includegraphics[width=0.5\textwidth]{img/algebra/generative_art_fermi_elephant_4_parameters.jpg}
	\end{figure} 
	with:
	
	with the four parameters:
	
	Some people have fun doing other things:
	\begin{figure}[H]
		\centering
		\includegraphics[width=0.8\textwidth]{img/algebra/generative_art_woman_with_a_perl.jpg}
	\end{figure} 
	\begin{figure}[H]
		\centering
		\includegraphics[width=0.7\textwidth]{img/algebra/simpson_generative_art.jpg}
	\end{figure} 
	
	\begin{tcolorbox}[title=Remark,colframe=black,arc=10pt]
	If the reader want to see how we plot the Fermi's elephant, he can refer to our R companion book.
	\end{tcolorbox}
	
	
	\paragraph{Power of a signal}\mbox{}\\\\
	A periodic signal has an infinite energy and null average power (\SeeChapter{see secton Electrokinetics}). Its average power over a period is then defined by:
	
	If we develop this equation, we have:
	
	This means that the power of a continuous-time periodic signal is equal to the sum of the squared Fourier coefficients. This is what we name the "\NewTerm{Parseval's theorem}\index{Parseval's theorem}\label{Parseval theorem}". This means that if we have any signal that can be decomposed in Fourier series, we can know the power of that signal using only the spectral coefficients.
	
	In reality, we can't mathematically determine the expression of this signal, we use therefore discretization or sampling and then we use a discrete Fourier transform, we can calculate the power of this signal using only the spectral coefficients. This gives us a characteristic of the signal.
	
	Let us also indicate the following result which will be very useful to us in the section of Thermodynamics for the study of the black-body and is also very closely related to important properties of the Riemann zeta function:
	
	The following relation:
	
	is named "\NewTerm{Parseval's equality}\index{Parseval's equality}".
	
	According to the definition of the Fourier series and the definition of the coefficient $a_0$ which follows immediately, we also have frequently in the literature:
	
	
	
	\pagebreak
	\paragraph{Fourier Transform}\label{fourier transform}\mbox{}\\\\
	Fourier series are a very powerful tool for the analysis of periodic signals for example, but the set of periodic functions is small compared to all the functions that we encounter in physical and engineering problems. So, will we introduce a new extremely powerful analytical tool that extends to a class of more general functions that have very important applications in signal processing, image processing, sound processing, statistics, finance (fourier option pricing techniques) and markets advanced statistics!!!
	
	\begin{tcolorbox}[title=Remark,colframe=black,arc=10pt]
	Many teachers and authors associate Fourier Series and Fourier Transform in the field of Functional Analysis. This is right in fact but it seemed to us more appropriate to put the study of the basics of this two subjects in this section because closely related to Sequences and Series. However, the subjects that follows normally the study of Fourier Transform, that is to say: Laplace Transfoms, Hilbert Transform and others will be given in the section of Functional Analysis of this book (see page \pageref{fourier transform analysis}). The Fast Fourier Transform study can be found in the section of Theoretical Computing.
	\end{tcolorbox}
	The Fourier transform (FT) is then used for both periodic signals and for aperiodic signals.
	
	For this, we start from study of Fourier series with the complex notation of a periodic function of period $T$ by considering that the period is becoming increasingly big to such that $T\rightarrow +\infty$. Therefore the spectral lines gradually approach to turn into a continuous spectrum.
	
	Therefore, let us resume the expressions proved just earlier:
	
	that we can write equivalently in the following traditional form (wherein it is customary to put the factor $1/T$ rather in $f(t)$):
	
	and let us write this for future needs in the following form:
	
	and let us put naturally that:
	
	Thus, when $T\rightarrow +\infty$, the pulsation tends to zero and we have $\omega_n\rightarrow \omega$ because we move from discrete values in continuous values that browse through the set of real number $\mathbb{R}$ (for all $k$). Therefore:
	
	we pass to the limit that is to say:
	
	This implies that:
	
	Therefore we obtain for the coefficients (we change the notation because the previous one is inadequate)
	
	and for the infinite series (the sum becomes an integral)
	
	Caution!!! To make the difference between the given function and its equivalent in which we seek expression in infinite sum, we will denote them differently now. Thus, we get:
	
	Thus the discrete Fourier series becomes a continuous function.
	
	\textbf{Definitions (\#\mydef):}
	\begin{enumerate}
		\item[D1.] We name "\NewTerm{Fourier Transform (FT)}\index{Fourier transform }" of $f$, or more rigorously the "\NewTerm{Continuous Time Fourier Transform (CTFT)}\index{Continuous Time Fourier Transform }", the relation:
		
		sometimes also denoted as follows:
		
		sometimes also named "\NewTerm{spectral density amplitude}\index{spectral density amplitude}".
		
		\item[D2.] We name "\NewTerm{Inverse Fourier Transform (IFT)}\index{inverse Fourier Transform}", or more rigorously the \NewTerm{Continuous Time Inverse Fourier Transform (CTIFT)}\index{Continuous Time Inverse Fourier Transform }",  of $F$ the relation:
		
	\end{enumerate}
	Any such transformation technique (as there are many as we will see in the section of Functional Analysis!) is named an "\NewTerm{integral transformation}\index{integral transformation}".
	\begin{tcolorbox}[title=Remark,colframe=black,arc=10pt]
	There are many ways of writing the Fourier transform according to the choice of the initial value of $T$! The reader must also know that it is quite common in physics to invert the definitions...the Fourier transform may be designated as the Inverse Fourier transform and reciprocally...
	\end{tcolorbox}
	Some physicist and engineers prefer to make the two previous relations symmetrical by putting the same coefficient in both directions, which will be for example $1/\sqrt{2}$. This will give:	
	
	Let us also give the corresponding three-dimensional version that will serve us many times in wave mechanics, electrodynamics, wave optics or in the various sections of quantum physics of this book:
	
	To make things perhaps clearer (at least we hope so), let us prove generally that the previous Fourier transform $\mathcal{F}$ is isometric (retains the "norm" - or "modulus" if you prefer...)
	
	\begin{theorem}
	For any functions $f, g$ we have the functional inner product:
	
	But since the functions are in the complex space, as we saw in the section of Vector Calculus, then we must use the notation of the hermitian product:
	
	Remember that:
	
	\end{theorem}
	\begin{dem}
	Then we want to prove the equality:
	
	Explicitly:
	
	But the variable to integrate but must be the same and for $\mathcal{F}(g)$ to be implicitly dependent of $\vec{r}$ it is necessary to take the Fourier transform on $\vec{k}$. Such as:
	
	Therefore:
	
	Therefore using the Fubini's theorem (\SeeChapter{see section Differential and Integral page \pageref{fubini theorem}}):
	
	Thanks to this result, we have also proved (this is immediate)
	
	We have not specified the bounds: they are infinite in every definition (we include all possible $\vec{k}$ or $\vec{r}$).
	\begin{flushright}
		$\blacksquare$  Q.E.D.
	\end{flushright}
	\end{dem}
	Let us now see and prove three interesting properties of the Fourier transform:
	\begin{enumerate}
		\item[P1.] If the function $f$ is an even function (\SeeChapter{Functional Analysis}), it comes a simplification of the Fourier Transform such that:
		
		
		\item[P2.] If $f$ is odd, we proceed in the same manner as above, and we get:
		
		
		\item[P3.] Very important property of the Fourier transforms which will be useful to us in finance (\SeeChapter{see section Economy page \pageref{economy}}), and also as part of the study of the heat equation (\SeeChapter{see section of Thermodynamics page \pageref{fourier transform heat equation}}).
		
		First remember that the Fourier transform is given by:
		
		We want to see what happens if:
		
		By doing an integration by parts (\SeeChapter{see section Integral and Differential page \pageref{integration by parts}}):
		
		we get:
		
		where we put ourselves in the situation where:
		
		Therefore:
		
		More generally:
		
	\end{enumerate}
	\begin{tcolorbox}[title=Remark,colframe=black,arc=10pt]
	The branch of "\NewTerm{harmonic analysis}\index{harmonic analysis}", or "\NewTerm{2D Fourier analysis}\index{2D Fourier analysis}", is the branch of mathematics that studies the representation of functions or signals as a superposition of basic waves. It deepens and generalizes the notions of Fourier series and Fourier transform. The basic waves are named "harmonics", hence the name of discipline. During the last two centuries it has had numerous applications in physics and economics under the name "spectral analysis" and knows recent applications including signal processing, quantum mechanics, neuroscience, stratigraphy, statistics, etc.
	\end{tcolorbox}
	\begin{tcolorbox}[colframe=black,colback=white,sharp corners]
	\textbf{{\Large \ding{45}}Examples:}\\\\
	\label{fourier transform pulse square}E1. Let us see now a first example (among the three fundamentals) of a Fourier transform that we use again in the sections about quantum physics as well as in wave optics. We will calculate the Fourier transform (spectrum) of the following function (rectangular pulse):
	\begin{figure}[H]
		\centering
		\includegraphics[scale=0.5]{img/algebra/fourier_transform_rectangular_pulse.jpg}
		\caption{Rectangular pulse example for Fourier Transform}
	\end{figure}
	We have therefore:
	
	where $\text{sinc}$ is the sine cardinal as we already know (\SeeChapter{see section Trigonometry page \pageref{sinc cardinal}}). So we fall back on the $\text{sinc}$ and if we take the squared modulus squared we therefore get the decomposition of a theoretical monochromatic wave diffracted by a rectangular slot!!! Thus, it seems possible to study the diffraction phenomena using the Fourier transform and this field is named "\NewTerm{Fourier optics}\index{Fourier optics}". We'll come back on this later in the section of Functional Analysis when we deepen the Fourier transforms.\\
	
	We know that the spectrum (described by the $\text{sinc}$ function) crosses zero every time that the sine function is zero, that is to say, every time the frequency is a multiple of $1 / a$.\\
	
	The spectrum of this pulse illustrates two important points regarding the limited time signals:
	\begin{enumerate}
		\item[P1.] A short signal has a broadband spectrum.
		\item[P2.] To a narrow spectrum correspond a long-term signal.
	\end{enumerate}
	\end{tcolorbox}

	\pagebreak
	\begin{tcolorbox}[colframe=black,colback=white,sharp corners]
	E2. The Fourier transform of an integrable function $f$ is given as we know now by:
	
	Consider the integrable Gaussian function of the type \label{fourier transform gaussian function}
	
	with $a>0$ define on $\mathbb{R}$.\\
	
	We want to compute its Fourier transform because it is a very important case and particularly useful for solving the heat equation that we will treat in the section of Thermodynamics and also to solve the differential equation of Black \& Scholes in the section Economy.\\
	
	The brilliant trick, if we want to avoid making complex analysis on $3$ A4 pages, is to notice that $F(\omega)$ is the solution of the following linear differential equation:
	
	where $y$ is a function of $\omega$.\\
	
	Indeed deriving  $F(\omega)$  we get:
	
	Integration by parts gives us:
	
	\end{tcolorbox}
	
	\pagebreak
	\begin{tcolorbox}[colframe=black,colback=white,sharp corners]
	We recognize the expression of the Fourier transform of $f$. Therefore:
	
	This shows that the $F(\omega)$ is solution of the differential equation above.\\
	
	We have proved in the section of Differential and Integral Calculus  that the general solution of this differential equation is given by:
	
	where $A \in \mathbb{R}$. And as in the present case:
	
	The primitive $G (x)$ is therefore easy to calculate and we get:
	
	Therefore:
	
	To determine the constant $A$ it suffices to notice that:
	
	To determine the constant $A$ it suffices to note that:
	
	and therefore:
	
	It is then usual to say that the Fourier transform of a Gaussian is another Gaussian!!
	\end{tcolorbox}
	
	\pagebreak
	\begin{tcolorbox}[colframe=black,colback=white,sharp corners]
	E3.\label{inverse fourier transform square pulse} This last example is very important to understand the Nyquist sampling. We consider a square pulse in the frequency domain:
	\begin{figure}[H]
		\centering
		\includegraphics[scale=0.5]{img/algebra/inverse_fourier_transform_rectangular_pulse.jpg}
		\caption{Rectangular pulse example for Inverse Fourier Transform}
	\end{figure}
	Let us calculate its inverse Fourier transform:
	
	There is something important to notice here (or to remember whatever...)!!! First, we know that sinc is equal to $1$ at $t=0$. But we also know that every time $Wt=k\pi$ with $k\in\mathbb{N}^{*}$ we have $\mathrm{sinc}=0$. Therefore, every time $t=k\pi/W$ we have:
	
	To make things more nice, signal processing engineers like to write rather the previous result into the form:
	
	and as you have almost surely noticed... this introduce a new type of sine cardinal (sic...!).  You may ask yourself why....????? That's quite easy. Written this way the $\mathrm{sinc}_\pi$ cancels for:
	
	This is prettier.... and even has a name: the "\NewTerm{normalized sinc function}\index{normalized sinc function}" and result in the following well known plot in signal processing for the study of the Nyquist sampling theorem:
	\end{tcolorbox}
	\begin{tcolorbox}[colframe=black,colback=white,sharp corners]
	\begin{figure}[H]
		\centering
		\includegraphics[scale=1]{img/algebra/sinc_inverse_fourier_transform.jpg}
	\end{figure}
	\end{tcolorbox}
	We will see plenty of other Fourier transform in the section of Function Analysis page \pageref{usual Fourier transforms}.
	
	\pagebreak
	\subsubsection{Bessel Series}
	Bessel functions are very useful in many advanced fields of physics involving delicate differential equations to solve. The areas in which we find them most often are calorimetry (heat conduction), nuclear physics (physics of reactors), optics and fluid mechanics.
	
	This series are still not study too much in the graduate curriculum and it is often the role of the student to seek the additional information it needs on this subject in the library of his school. We wanted to present here the developments that avoid this approach while staying at home in front of our computer (furthermore books on the subject are quite rare...).
	
	\begin{tcolorbox}[title=Remark,colframe=black,arc=10pt]
	We usually speak by abuse of language of "\NewTerm{Bessel functions}\index{Bessel functions}\label{bessel functions}" instead of "\NewTerm{Bessel series}\index{Bessel series}".
	\end{tcolorbox}
	There is a significant amount of Bessel functions but we will restrict ourselves to the study of those most used one in physics.
	
	\paragraph{Zero order Bessel's Functions}\mbox{}\\\\
	The function known as the "\NewTerm{Zero order Bessel's function}\index{zero order Bessel's function}", is defined by the power series:
	
	It is during the study of the properties of derivation and integration that Friedrich Bessel found that this series of power is a solution to a differential equation that is found frequently in physics. That is why it bears his name.
	
	If $u_r$ represents the $r$-th term of the series, we easily see that:
	
	which tends to zero as $r\rightarrow +\infty$, regardless of the value of $x$. This has for consequence that the series converges for all values of $x$. Since this is a series of positive power, the function $J_0(x)$ and all its derivatives are continuous function for all values of $x$, real or complex.
	
	\paragraph{$n$ order Bessel's Functions}\mbox{}\\\\
	The function $J_n(x)$, known as the "\NewTerm{$n$ order Bessel's function}\index{$n$ order Bessel's function}", is defined, when $n$ is a positive integer, by the power series:
	
	which converges for all values of $x$, real or complex.
	\begin{figure}[H]
		\centering
		\includegraphics{img/algebra/bessel_functions.jpg}
		\caption[Plot of few Bessel functions]{Plot of few Bessel functions (source: Wikipedia)}
	\end{figure}
	and in Microsoft Excel or Maple 4.00b the previous function can be found under the name \texttt{BESSELJ( )}. For example for the previous graph in Maple, we just write:
	
	\texttt{>plot([BesselJ(0,x),BesselJ(1,x),BesselJ(2,x),BesselJ(3,x)],x=0..20);}
	
	Let us see in particular, that for $n=1$ we have:
	
	and when $n=2$:
	
	We can notice that $N_n(x)$ is an even function of $x$ when $n$ is even and odd if $n$ is odd (\SeeChapter{see section Functional Analysis page \pageref{even function}}).
	
	If we play to do engineering maths we notice by trial and errors that:
	
	Based on this trial and error approach we have using the above expression by factorizing the $(x/2)^{2k}$ term only:
	
	So finally we see after trial and errors again (instead than doing 5 pages of mathematical developments as do mathematicians) that\label{condensed expression of Bessel series}:
	
	More generally for $n$ that is non-integer we use the Euler Gamma function (\SeeChapter{see section Differential and Integral Calculus page \pageref{gamma euler function}}):
	
	That is also sometimes written by pure mathematicians (...):
	
	Now by differentiating the function $J_0(x)$ and comparing the result with the series $J_1(x)$, we see that without much pain that:
	
	We also find without too much difficulty, the following relation\label{bessel differential reccurence relation}:
	 
	\begin{tcolorbox}[title=Remark,colframe=black,arc=10pt]
	In general by recursive reasoning we get:
	
	Therefore:
	
	This last relation will be useful to us in the section of Wave Optics for our study of the circular aperture diffraction (Airy disk).
	\end{tcolorbox}
	Using the fact that:
	
	and including it in the previous relation, we find:
	
	written in another way:
	
	$y=J_0(x)$ is therefore a solution of the differential equation of the second order:
	
	written otherwise:
	
	or even:
	
	A solution to a an equation of parameter $n$ which is not a multiple of $J_n(x)$ is named "\NewTerm{Bessel function of the second kind}\index{Bessel function of the second kind}". Let us suppose now that $u$ is such a function and let us put $v=J_0(x)$; then according to the relation:
	
	we have:
	
	Multiplying the first relation by $v$ and the second by $u$ and after subtracting, we get:
	
	we therefore also have:
	
	we can therefore write:
	
	Indeed, because if we develop, we find:
	
	For the equality:
	
	is satisfied, we have:
	
	Dividing by $xv^2$, we have:
	
	which is equivalent to:
	
	immediately, by integrating it comes:
	
	where $A$ is a constant. Consecutively we have since $v=J_0(x)$:
	
	where we recall, $A$ and $B$ are constants, and $B\neq 0$ if $u$ is not a multiple of $J_0(x)$ by definition!
	
	If in the last relation, $J_0(x)$ is replaced by its expression in terms of series we have:
	
		For those who want to check this last relation (I do not like this kind of algebraic calculations) with Maple 4.00b just write:
	
	\texttt{>1/x*taylor(1/(series(BesselJ(0,x),x))\string^2,x=0,5);}
	
	Therefore:
	
	consecutively if we put:
	
	where $Y_0(x)$ is a particular Bessel function of the second type named "\NewTerm{Bessel-Neumann function of the second kind of zero order}\index{Bessel-Neumann function of the second kind of zero order}".

	Identically to the fact that when $J_0(x)\rightarrow 0$ when $x \rightarrow 0$, the expression $Y_0(x)$ because of the term $\log(x)$ when $x$ is small approaches $Y_0(x) \rightarrow -\infty$ when $x\rightarrow +0$.
	
	Finally, it comes from what we have seen that $J_0(x)$ and $Y_0(x)$ are independent solutions of the differential equation:
	
	The general solution being therefore:
	
	where $A$, $B$ are arbitrary constants and $x>0$ so that $Y_0(x)$ is real.

	If we replace $x$ by $k$, where $k$ is a constant, the differential equation becomes:
	
	by multiplying the whole by $k^2$ we find the general form of the differential equation:
	
	whose general solution is:
	
	where $k>0$ such that $Y_0(kx)$ is real when $x>0$.
	
	In fact, the Bessel functions are solutions of the differential equation previously studied and solved by the "\NewTerm{Frobenius method}\index{Frobenius method}\label{Frobenius method}". Indeed, let us write:
	
	and let us make the substitution:
	
	substituting in $Ly$, we get:
	
	Now let us choose the $c_i$ to satisfy the differential equation such as:
	
	Therefore, unless $\rho$ is a negative integer, we have:
	
	By substituting these values in the relation:
	
	we get:
	
	Therefore:
	
	If we put $\rho=$ in the prior-previous relation, we get:
	
	
	\paragraph{Bessel's Differential Equations of order $n$}\mbox{}\\\\
	We have defined the Bessel series as:
	
	Let us put:
	
	and let us derivate as follows:
	
	But we also have:
	
	By subtraction:
	
	Which finally gives:
	
	This is also written:
	
	which is named the "\NewTerm{Bessel differential equation of order $n$}\index{Bessel differential equation of order $n$}" or more simply "\NewTerm{Bessel equation}\index{Bessel equation}". In fact, most schools or Internet sites give this differential equation as a definition but now it is clear that there is rigorous reasoning behind this equation.
	
	The solution is therefore of the type:
	
	which is still sometimes written using the gamma Euler function (\SeeChapter{see section Differential and Integral Calculus page \pageref{gamma euler function}}):
	
	It follows that:
	
	and therefore that $y=J_n(x)$ is the solution of this differential equation.
	
	We will fall back on such a differential equation during our study of the wave equation of a circular drum (\SeeChapter{see section Wave Mechanics page \pageref{circular drum}}), during our study of the physics of nuclear reactors (\SeeChapter{see section Nuclear Physics}) and finally during our study of self-buckling (\SeeChapter{see section Mechanical Engineering page \pageref{self-buckling}}).
	
	\subsection{Convergence Criteria}
	When we study a series, one of the fundamental questions is that of the convergence or divergence of this series.
	
	If a series converges, its general term approaches zero as $n$ approaches infinity:
	
	or obviously generally:
	
	This criterion is necessary but insufficient to establish the convergence of a series. By cons, if this criterion is not met, we are absolutely sure that the series does not converge (so it diverges!).
	
	Three methods are proposed to deepen the convergence criteria:
	\begin{enumerate}
		\item The integral test

		\item The d'Alembert rule

		\item The Cauchy rule
	\end{enumerate}
	In the following paragraphs, we will assume the series with positive terms. The case of alternating series will be seen later.
	
	\subsubsection{Integral Test}
	The integral test for convergence is a method used to test infinite series of non-negative terms for convergence. It was developed by Colin Maclaurin and Augustin-Louis Cauchy and is sometimes known as the "\NewTerm{Maclaurin–Cauchy test}\index{Maclaurin–Cauchy test}".
	
	Given the series with decreasing positive (monotone decreasing) terms:
	
	That is to say:
	
	and given a continuous decreasing function such that:
	
	Then the infinite series
	
	converges to a real number if and only if the improper integral:
	
	is finite. In other words, if the integral diverges, then the series diverges as well.
	\begin{tcolorbox}[title=Remark,colframe=black,arc=10pt]
	In no case does the integral gives the value of the sum of the series! The full test only gives an indication of the convergence of the series. Before making the test of the integral, it is important to check that the terms of the series are strictly decreasing to fill the condition $a_1\ge a_2\ge a_3\ge \cdots\ge a_n\ge \cdots$.
	\end{tcolorbox}
	\begin{tcolorbox}[colframe=black,colback=white,sharp corners]
	\textbf{{\Large \ding{45}}Example:}\\\\
	The harmonic series:
	
	diverges because (\SeeChapter{see section Differential and Integral Calculus page \pageref{usual primitives}}):
	
	So this harmonic series does not converge.
	\end{tcolorbox}
	
	\subsubsection{D'Alembert Rule}
	The "\NewTerm{ratio test}\index{ratio test}" is also a test (or "criterion") for the convergence of a series of the type:
	
	where each term is a real or complex number and $a_n$ is nonzero when $n$ is large. The test was first published by Jean le Rond d'Alembert and is sometimes known as "\NewTerm{d'Alembert's ratio test}\index{d'Alembert's ratio test}" or as the "\NewTerm{Cauchy ratio test}\index{Cauchy ratio test}".
	
	The usual form of the test makes use of the limit:
	
 	The ratio test states that:
	\begin{enumerate}
		\item if $L < 1$ then the series converges absolutely;
		\item if $L > 1$ then the series does not converge;
		\item if $L = 1$ or the limit fails to exist, then the test is inconclusive, because there exist both convergent and divergent series that satisfy this case
	\end{enumerate}
	and we define the radius of convergence by:
	
	For the proof suppose that:
	
	\begin{tcolorbox}[title=Remark,colframe=black,arc=10pt]
	In reality this rule is normally without the absolute value. The case with the absolute value as above is named the "\NewTerm{absolute convergence test}\index{absolute convergence test}" and applied for the more general case of an alternate series such that:
	
	If an alternating series of terms is absolutely convergent, the absolute series that follows also converge.\\
	
	Therefore the absolute convergence test is a generalization of the d'Alembert rule but most of time we don't make any distinctions between the both.
	\end{tcolorbox}
	We can then show that the series converges absolutely by showing that its terms will eventually become less than those of a certain convergent geometric series. To do this, let:
	
	Then $r$ is strictly between $L$ and $1$, and:
	
	 for sufficiently large $n$ (say, $n$ greater than $N$).

	Hence:
	
	for each $n > N$ and $i > 0$, and so:
	
	That is, the series converges absolutely.

	On the other hand, if $L > 1$, then:
	
	for sufficiently large $n$, so that the limit of the sum is non-zero. Hence the series diverges.
	\begin{tcolorbox}[colframe=black,colback=white,sharp corners]
	\textbf{{\Large \ding{45}}Example:}\\\\
	Given the following geometric series:
	
	We get quotient:
	
	Therefore the series converge!
	\end{tcolorbox}
	Obviously some practical applications will (can) give for example:
	
	and some practitioners name this the "\NewTerm{Cauchy convergence rule}\index{Cauchy convergence rule}"... (do not confuse with the Cauchy convergence test that will be study in the section Fractals).
	
	\begin{tcolorbox}[colframe=black,colback=white,sharp corners]
	\textbf{{\Large \ding{45}}Example:}\\\\
	Let us do now a last important example. We have study earlier above Bessel series. But it can be not obvious that these series converge. Let us prove that this is indeed the case for $J_0$, that is to say for:
	
	Therefore:
	
	\end{tcolorbox}
	
	\subsubsection{Alternating Series Test}
	The alternating series test is a method used to prove that an alternating series with terms that decrease in absolute value is a convergent series. The test was used by Gottfried Leibniz and is sometimes known as "\NewTerm{Leibniz's test}\index{Leibniz's test}", "\NewTerm{Leibniz's rule}\index{Leibniz's rule}, or the "\NewTerm{Leibniz criterion}\index{Leibniz criterion}".
	
	A series of the form
	
	where either all an are positive or all an are negative, is named an "\NewTerm{alternating series}\index{alternating series}".

	The alternating series test then says: if $a_n$ decreases monotonically and:
	
 	then the alternating series converges.
	
	There a lot of other tests as the Raabe's test, the Bertrand's test, the Gauss's test, the Kummer's test.
	
	\subsubsection{Fixed Point Theorem}\label{fixed point theorem}
	The fixed point theorem is not really useful in physics and for the engineers (but implicitly it is essential but physicists and engineers often use math tools whose properties have already been approved in advance by mathematicians), however we find it in chaos theory and in theoretical computing (see the sectiona on Fractals especially the topic on the Sierpinski triangle). We can therefore only recommend the reader to take the time to read and understand the explanations and developments that follow.
	
	Let $(X,d)$, be a complete metric space (\SeeChapter{see sections Topology page \pageref{metric space} or Fractals page \pageref{fractal metric space}}) and $T:X\mapsto X$ an application strictly contracting of constant $L$ (see the Lipschitz functions in the section Topology page \pageref{lipschitz functions}), then there exists a unique point $\omega\in X$ such that:
	
	$\omega$ is then named the "\NewTerm{fixed point}\index{fixed point}" of $T$ (think to the case $\cos(x)=x)$. Furthermore, if denote by:
	
	the image of $x$ by the $n$-th iterate of $T$, then we have:
	
	and the convergence speed can also be estimated by:
	
	By the fact that we restrict our study to iterating a function, we speak of "\NewTerm{Banach fixed-point theorem}\index{Banach fixed-point theorem}\label{banach fixed point theorem}" that gives a general criterion guaranteeing that, if it is satisfied, the procedure of iterating a function yields a fixed point.
	\begin{tcolorbox}[title=Remark,colframe=black,arc=10pt]
	You can have fun with your pocket calculator or your operating system by choosing a random number and taking the cosine iteratively. You will find you will tend to $0.74$ and therefore it is verbatim the solution of $\cos (x) = x$.
	\end{tcolorbox}
	\begin{dem}
	Given $x\in X$. We consider the following sequence $(T^n(x))_{n\in \mathbb{N}}$ as defined above. First we will prove that this sequence is a Cauchy sequence (see above what is a Cauchy sequence).

	Applying the triangle inequality (\SeeChapter{see section Vector Calculus page \pageref{triangle inequality}}) several times we have:
	
	But:
	
	Therefore:
	
	and as:
	
	Therefore:
	
	To finish:
	
	that is to say, that in a first time $(T^n(x))_{n\in \mathbb{N}}$ converge, and we put:
	
	Now we check that $\omega$ is a fixed point of $T$. Indeed $T$ is uniformly continuous (as Lipschitz - see section Topology page \pageref{lipschitz functions}) therefore a fortiori continues:
	
	It remains to check that $\omega$ is the only fixed point (therefore we will have proved that $\omega$ does not depend on the choice of $x$). Suppose that we also have $T(y)=y$ then:
	
	An estimate of the speed of convergence is given by:
	
	$\mathrm{d}(,)$ is continuous with respect to each variable so:
	
	and the limits preserve the inequalities (not strict one) thus:
	
	\begin{flushright}
		$\blacksquare$  Q.E.D.
	\end{flushright}
	\end{dem}
	
	\pagebreak
	\subsection{Generating Functions (transformation of a sequence into a series)}
	For some mathematical financial risk management models (\SeeChapter{see section Economy page \pageref{economy}}) and also integration transformation of Bessel functions (\SeeChapter{see section Differential and Integral Calculus page \pageref{integral representation of first kind Bessel's function}}) we will need in this book a gentle introduction to generating functions.
	
	\subsubsection{Ordinary Generating Functions (transformation of a sequence into a series)}
	Remember first that we have proved earlier above that the general Maclaurin expansion of a function was given by:
	
	That is for $x=\cong 0$
	
	In the case of our study, let us write the latter relation as:
	
	We say then that the above relation is the "\NewTerm{ordinary generating function}\index{ordinary generating function}\label{ordinary generating function}" of the sequences of numbers $a_0,a_1,a_2,a_3,\ldots$ in the formal parameter $x$. And we don't care if it diverge or not!
	
	Because the generating function is an algebraic expression that encodes the sequence and allows us to manipulate it in ways that are not possible in other forms. Many times if the sequence you are looking at is "interesting" (and this word has lots of interpretations), the generating function has a short simple form.

	The generating function allows us to derive formulas for the sequence, identities involving the sequence, estimate the values and so much more as we will see further below.	
	
	One thing to never forget: The generating function is not the sequence and the sequence is not the generating function. They are not the same thing. One is a sequence, the other is an algebraic expression!

	If you have a sequence you can say "the generating function of the sequence" to refer to the algebraic object. If you have a generating function you might say "the sequence of coefficients of the generating function" in order to refer to the sequence.
	
	Let us try now a companion example, the sequence consisting of all $1$:
	
	The generating function is therefore the geometric series that we have proved earlier above implicitly (explicitly you can see the proof at page \pageref{sum of powers}):
	
	The next simple companion example would be the positive integers:
	
	This has generating function:
	
	Now we observe that the derivative of this generating function is the derivative of the generating function:
	
	Indeed:
	
	That is we have therefore:
	
	We conclude that:
	
	We can't use the exactly same trick to figure out the generating function for the sequence:
	
	because if we take the derivative of:
	
	then we do not quite have the square integers. But the attentive reader will notice that if we first multiply the latter relation by $x$ and then take the derivative then we have:
	
	Therefore:
	
	It is easy with a software like Maple to control that the Maclaurin series of $(1+x)/(1-x)^3$ is equal to the generating function and therefore give us the coefficients $a_i$ of the sequence.
	
	Let us see now a non-trival example... It's the Fibonacci sequence given for recall (see page \pageref{Fibonacci Sequence}) by (each term is equal to the sum of the previous two):
	
	We will give the generating function for this sequence a name $F(x)$ so then:
	
	where $F_0=F_1=1$ and $F_k=F_{k-1}+F_{k-2}$ for $k\geq 2$. Then we can see that:
	
	Since we have figured out that:
	
	then:
	
	and this can be rewrittten as:
	
	and hence:
	
	It is always surprising that the generating function for the Fibonacci numbers has such a compact formula. But once again, using for example a software like Maple and doing the Maclaurin expansion of the above function, you will get a series whose coefficients $a_k$ corresponds to the Fibonacci sequence!
	
	Bu let us continue to investigate...! We have therefore the denominator:
	
	that can be factorized assuming the equation:
	
	has solutions! We quickly found that the roots are given by:
	
	We recognize here the negative values of the Golden ratio $\varphi$ and its conjugate (see earlier above page \pageref{golden ratio}).
	
	Therefore the denominator can be written as (using the relation between the Golden ratio and its conjugate):
	
	Thus we can write:
	
	Let us decompose this fraction (the $-$ sign is now the in the constants $A$ and $B$):
	
	where $A$, $B$ need to be found out. Clearly we have:
	
	By identification, we have the following linear system:
	
	Hence:
	
	Or written explicitly:
	
	The solution to this linear system is straightforward and given by:
	
	Thus we have:
	
	Next we use another result the famous Taylor development:
	
	for $|x|<1$. Using this result we can see that:
	
	Hence:
	
	Hence:
	
	This result is sometimes named the "\NewTerm{Fibonacci Binet's Formula}\index{Fibonacci Binet's Formula}".
		
	\paragraph{Composition of Generating functions}\mbox{}\\\\
	Now if we have two generating functions:
	
	for two sequences of integers $a_0,a_1,a_2,a_3,\ldots$ and $b_0,b_1,b_2,b_3,\ldots$ then there are several ways that we can combine the sequences and get generating functions fr new sequences.

	\begin{itemize}
		\item Sum: If we add the generating functions we have that:
		
		is a generating function for the sequence:
		
		
		\item Product: However if we multiply the two generating function, we ha that:
		
		This can be summarized in the expression:
		
		Another special case of the product of generating function is the product:
		
		It is the product of two generating functions, the first one being the generating function:
		
		By:
		
		the product of these is a generating function for the sequence (as all $a_i=1$):
		
	
		\item Derivative: We have already seen a couple examples of the
use of the derivative in previous examples.
	\end{itemize}
	Remember now that we have proved that:
	
	Therefore by the fact that:
	
	has for generating sequence:
	
	we have then that:
	
	is then a generating function for the sequence of the sum of the first $n$ positive integers:
	
	In particular, the coefficient of $x^k$ is:
	
	We also know by taking the derivative of $1/(1-x)^2$ we have:
	
	Therefore if we divide this equation by two we have:
	
	It must be that the coefficient of $x^k$ in $1/(1-x)^3$ is equal to $(k + 1)(k + 2)/2$ and it is equal to the sum of the first $k+1$ integers, so:
	
	
	\subsubsection{Multivariate Generating Functions}
	Remember that in the section of Calculus we have proved that:
	
	can be expressed by a condensed form that involves the binomial coefficient:
	
	then this is what we name a "\NewTerm{multivariate generating function}\index{multivariate generating function}". It works just as the other generating functions we have previously worked with except that it has two parameters.
	
	\subsubsection{Functional Generating Functions}
	So far we have seen generating functions that gives scalar values. But a more general family are the "\NewTerm{functional generating functions}\index{functional generating functions}\label{functional generating function}" that gives functions instead of simple scalars.
	
	A generating function for a sequences of functions $\{f_n(x)\}$ is obviously a power series of the type:
	
	whose coefficient are now functions of $x$.
	
	Let us see the both examples that will be useful to us in the sections of Economy, Wave Optics and also Differential and Integral Calculus!.
	
	Let us start with the most important example involving probabilities and especially the Poisson distribution as used in the First Boston Credit Risk Metric model in the section Economy!
	
	Let us recall that the Poisson distribution mass, that is, the probability that $N$ is equal to $n$, is given by (\SeeChapter{see section Statistics page \pageref{poisson distribution}}):
	
	for $n=0,1,2,\ldots$.

	The probability generating function is equal to:
	
	We know also the following Maclaurin infinite series expansion:
	
	And if we rewrite:
	
	in the following way:
	
	And further assuming the above Maclaurin expansion we can write:
	
	Which implies:
	
	This is the probability generating function of a Poisson distribution that we will use for our study of the CreditRisk model.
	
	And now for our study of Wave Optics (especially the Fresnel diffraction for a circular aperture!) let us found the functional generating function of the first kind Bessel functions!
	\begin{theorem}
	The generating function for the sequence of Bessel function of first kind, on integer order, is:\label{generating function for bessel function of first kind}
	
	\end{theorem}
	\begin{dem}
	To obtain an expression for $J_n(x)$, we use the Maclaurin series for $e^x$ to get:
	
	Now let us  make the change of variable: $n=r-s$.

	Therefore the expression in the sum becomes:
	
	Then it is obvious that as the range of $r$ and $s$ is $]-\infty,+\infty[$ we have for the range of $n$ also: $]-\infty,+\infty[$. So the first sum is easy to determine:
	
	But as we can see in the expression in the sum above, we don't get rid of the summation variable $s$. So there is obviously a second missing sum on the variable $s$! The interval of summation for $s$ is not obvious at a first glance...

	What is sure is that $s$ is still in the range $[0,+\infty]$ (positivie values) if we look at the term $s!$ in the denominator. But what makes problem is the lower bound of the sum. So if we look closely to the relation in the sum above we have a term $(n+s)!$. Obviously $n+s$ must never be negative! It comes therefore that if $n<0$, we must have $n+s\geq 0$, that is when $n<0$, $s$ must start $-n$. Finally:
	
	That is:
	
	where the $J_n$ are the Bessel function of the first kind.
	\begin{flushright}
		$\blacksquare$  Q.E.D.
	\end{flushright}
	\end{dem} 
	Therefore, for $n\geq 0$:
	
	And for $n<0$:
	
	Using an index shift, we obtain:
	
	
	\begin{flushright}
	\begin{tabular}{l c}
	\circled{95} & \pbox{20cm}{\score{3}{5} \\ {\tiny 44 votes,  64.09\%}} 
	\end{tabular} 
	\end{flushright}
	
	%to make section start on odd page
	\newpage
	\thispagestyle{empty}
	\mbox{}
	\section{Vector Calculus}\label{vector calculus}

	\lettrine[lines=4]{\color{BrickRed}V}ector calculus or "vector analysis" is a branch of mathematics that studies the scalar or vector fields that are sufficiently regular of Euclidean spaces (see definition further below).\\\\

The importance of vector calculus comes from its extensive use in physics and in the engineering sciences. It is from this perspective that we will present it, and this is why we limit ourselves mostly to the case of the usual three-dimensional spaces. In this context, a vector field associates to each point of the space a vector (with three real components), while a scalar field associates just a unique real number to such a point.

There is a phenomenal amount of series and theories about these, but we will mention especially the Taylor series (used almost everywhere in applied science), Fourier series (signal theory, statistics, wave mechanics or quantum physics) and Bessel series functions (very important in nuclear physics!) that we will make a brief study here and that will continue in the section of Functional Analysis.

	\begin{tcolorbox}[colframe=black,colback=white,sharp corners]
\textbf{{\Large \ding{45}}Example:}\\\\
For example, imagine the water from a lake. The temperature data at each point forms a scalar field, that of his speed at each point, a vector field (see definition further below).
	\end{tcolorbox}

Physical concepts such as strength or speed are characterized by a direction, an orientation and an intensity. This triple character is highlighted by arrows. These are the source of the concept of vector and are the most suggestive example. Although their nature is essentially geometric, it is their ability to bind to each other, so their algebraic behavior, which mostly retain our attention. Split into equivalence classes their set represents the classic model of a "\NewTerm{vector space}\index{vector space}" (\SeeChapter{see section Set Theory page \pageref{poisson distribution}}). One of our primary goals here is the detailed description of this model.

	\begin{tcolorbox}[title=Remarks,colframe=black,arc=10pt]
	\textbf{R1.} Before reading what follows, the reader is advised to have at least read diagonally the section on Set Theory in the Arithmetic chapter. We define there what is a "vector space" using the tools of Set Theory. Even if this concept  is although not absolutely essential it is still interesting to see how two areas of mathematics fit together and also just for the reason... of introducing vector stuffs with a least a little bit rigour.\\
	
	\textbf{R2.} Vectorial analysis contains many terms and definitions that must be learned by heart. This work is hard but unfortunately necessary...
	\end{tcolorbox}
	
	\pagebreak
	\subsection{Concept of Arrow}	

\textbf{Definition (\#\mydef):} We denote by $U$ the ordinary space of elementary geometry and $P, Q,...$ its points. We will call "\NewTerm{arrow}\index{arrow}" all directed line segment (in space). The arrow of origin $P$ (origin point) and extremity $Q$ (terminal point) will be denoted $\overrightarrow{PQ}$ or abbreviated by a single letter (Latin or Greek) arbitrarily chosen as example: $\overrightarrow{F}$.

	\begin{tcolorbox}[title=Remark,colframe=black,arc=10pt]
In the norm ISO 80000-2:2009 it is authorized to represent the vectors with a letter in bold.
	\end{tcolorbox}	

We will consider as obvious that any arrow is characterized by its direction, its orientation (because of a given direction it can point in both directions), its intensity or magnitude (length) and its origin.

In vector (or multivariable) calculus, we will deal with functions of two or three variables (usually $x, y$ or $x, y, z$, respectively). The graph of the arrow of coordinates $(x, y)$, lies in Euclidean space, which in the Cartesian coordinate system consists of all ordered doublets of real numbers $(a,b)$. Since Euclidean can be 3-dimensional (and more or less for sure!), we denote it by $\mathbb{R}^3$.

The graph of the arrow consists of the points $(a, b, c)$. The 3-dimensional coordinate system of Euclidean space can be represented on a flat surface, such as this page or a blackboard, only by giving the illusion of three dimensions, in the manner shown in the figure below:

\begin{figure}[H]
\centering
\includegraphics[scale=0.75]{img/algebra/euclidian_vector.eps}
\caption{Example of arrow in $\mathbb{R}^3$ euclidian space}
\end{figure}

Euclidean space has three mutually perpendicular coordinate axes ($x$, $y$ and $z$), and three mutually perpendicular coordinate planes: the $xy$-plane, $yz$-plane and $xz$-plane:

\begin{figure}[H]
\centering
\includegraphics[scale=0.75]{img/algebra/euclidian_planes.eps}
\caption{Mutually perpendicular planes in $\mathbb{R}^3$}
\end{figure}

The coordinate system shown above is known as a right-handed coordinate system, because it is possible, using the right hand, to point the index finger in the positive direction of the $x$-axis, the middle finger in the positive direction of the $y$-axis, and the thumb in the positive direction of the $z$-axis, as below:

\begin{figure}[H]
\centering
\includegraphics[scale=0.75]{img/algebra/right_hand.eps}
\caption{Right hand system}
\end{figure}


	\subsection{Set of Vectors}	

\textbf{Definitions (\#\mydef):}

	\begin{enumerate}
		\item[D1.] We say that two arrows are "\NewTerm{equivalent arrows}" \index{equivalent arrows}  if they have the same direction, the same orientation and the same intensity.
		\item[D2.] We say that two arrows are "\NewTerm{colinear arrows}"\index{colinear arrows} only if they have the same direction.
	\end{enumerate}
Let us now split the set of all arrows in equivalence classes: two arrows belong to the same class if and only if they are equivalent.

So:

\textbf{Definitions (\#\mydef):} 

\begin{enumerate}
	\item[D1.]Each equivalence class of arrows whose origin point and terminal point are distinct is a "\NewTerm{vector}"\index{vector}\label{vector}" or rather a "\NewTerm{free vector}"\index{free vector} because its origin is not taken into account (if its origin is well defined, then we have a "\NewTerm{bounded vector}")\index{bounded vector}.
	
	\item[D2.]Degenerated arrows (that is to say of the form $\overrightarrow{PP}$) are named "\NewTerm{zero vector}"\index{zero vector} and written $\vec{0}$ when they have an undefined direction and orientation and zero intensity (origin and terminal point are not distinct).
\end{enumerate}	

The set of vectors will be commonly referred  by $\mathbb{V}$. Note that the elements of $\mathbb{V}$ are (equivalence) classes arrows and not individual arrows. It is however clear that any arrow is sufficient to determine the class of equivalence to which it belongs and it is natural to name the corresponding class: "\NewTerm{representative class}"\index{representative class} of the vector.

Let us now draw a representative of a vector $\vec{y}$ from the end of a representative vector $\vec{x}$. The arrow whose origin is that of the representative $\vec{x}$ and the end of representative $\vec{y}$ determines a new vector which we write: $\vec{x}+ \vec{y}$. The operation that combines any two vectors by their sum is named "\NewTerm{vector addition}"\index{vector addition}.

\begin{figure}[H]
\centering
\includegraphics[scale=1]{img/algebra/vector_addition.jpg}
\caption{Example of a sum of two vectors}
\end{figure}

\begin{figure}[H]
\centering
\includegraphics[scale=1]{img/algebra/vector_addition_robotics.jpg}
\caption{Example of a sum of two vectors with robot dynamics notation}
\end{figure}

Using a figure, it is easy to show that the operation of vector addition is associative and commutative, i.e. that:
	
and:
	
It is also evident that the zero vector $\vec{0}$ is the neutral element of the vector addition. Formally:
	
where $-\vec{x}$ means the opposite of vector $\vec{x}$, that is to say the vector whose representatives have the same direction and the same intensity as those of $\vec{x}$, but the opposite orientation. 

\textbf{Definitions (\#\mydef):}
	\begin{enumerate}
		\item[D1.] Two vectors whose sum is zero are then named "\NewTerm{opposed vectors}"\index{opposed vectors} since the only thing that differentiates them is their orientation...
		\item[D2.] It follows that if two or more vectors have the same direction, the same intensity and the same orientation then  we say that they are "\NewTerm{equal vectors}"\index{equal vectors}.
	\end{enumerate}
As we can see the reverse operation of vector addition is the vector subtraction. Subtract a vector is equivalent to adding the opposite vector.

	\begin{tcolorbox}[title=Remarks,colframe=black,arc=10pt]
\textbf{R1.} The addition extends, by induction, to the case of any finite family of vectors. Under associativity, these successive additions can be performed in any order, which justifies the writing without brackets.\\\\
\textbf{R2.} The multiplication between two vectors is a concept that does not exist. But, as we shall see it a little further, we can multiply the vectors by some other vector properties that which we call the "norm" or simply by scalars and still by some other things...
	\end{tcolorbox}	

\subsubsection{Pseudo-Vectors}

In physics, in the statement named the "\NewTerm{Curie's principle}"\index{Curie's principle} (\SeeChapter{see section Principia page \pageref{curie principle}}), physicists mention of what they name "\NewTerm{pseudo-vectors}"\index{pseudo-vector}\label{pseudo vector}. This is the simple vocabulary to talk about something equally trivial but basically only a few people actually do use. But it can still be useful to present what it is.

	In fact, vectors and pseudo-vectors are transformed in the same way for a rotation or translation (we will see in our study of Linear Algebra how mathematically perform this type transformations). It is not the same in symmetry with respect to a plane or at one point. In these transformations we have by definition the following properties:
	\begin{enumerate}
		\item[P1.] A vector is transformed into its symmetrical.
		\item[P2.] A pseudo-vector is converted into the opposite of its symmetrical.
	\end{enumerate}
	Here is a figure with typical examples (the choice of letters representing the vectors and pseudo-vectors is not due to chance; they are a wink to the properties of electric and magnetic fields as studied in the Electromagnetism chapter):
	\begin{figure}[H]
	\centering
	\includegraphics[scale=0.75]{img/algebra/pseudo_vector.eps}
	\caption{Differences of transformations between a vector and a pseudo-vector}
	\end{figure}

A well known practical example is a pseudo-vector that we will study in detail much further below and resulting of an operation named "\NewTerm{cross product}"\index{cross product}\label{cross product}:
	
		And to see why the result is a pseudo-vector, consider the special simple case:
	
	Now if we do a symmetric operation of the $X\text{O}Z$ plan we get:
	
	So as we can see the vector resulting of the cross product is a pseudo-vector because under transformation of the plan it's orientation change!

A vector resulting of a mathematical operation of symmetry that does not change its orientation is named a "\NewTerm{polar vector}"\index{polar vector} but in fact almost everybody say just "vector".

Now that we have an idea of what vectors are, we can start to perform some of the usual algebraic operations on them and this is what we name "\NewTerm{vector algebra}"\index{vector algebra}.

	\subsubsection{Normal vector}
	The "\NewTerm{normal vector}\index{normal vector}\label{normal vector}", denoted $\vec{N}$ or $\vec{n}$, to a surface is a vector which is perpendicular to the surface or curve at a given point. When normal vectors are considered on closed surfaces or closed curves, the inward-pointing normal (pointing towards the interior of the surface) and outward-pointing normal are usually distinguished.

	The unit vector obtained by normalizing the normal vector (i.e., dividing a nonzero normal vector by its vector norm) is the unit normal vector, often known simply as the "\NewTerm{unit normal}"\footnote{Care should be taken to not confuse the terms "vector norm" (length of vector), "normal vector" (perpendicular vector) and "normalized vector" (unit-length vector)}.
	\begin{figure}[H]
		\centering
		\includegraphics[scale=0.8]{img/algebra/normal_vector_field.jpg}
		\caption{A vector field of normals to a surface}
	\end{figure}
	In the three-dimensional case a "\NewTerm{surface normal}\index{surface normal}", or simply normal, to a surface at a point $P$ is a vector that is perpendicular to the tangent plane to that surface at $P$. The word "normal" is also used as an adjective: a line \textit{normal} to a plane, the \textit{normal} component of a force, the \textit{normal} vector, etc. 
	
	We will see and prove in details in this section and many others, how to calculate explicitly various normal vectors to curves and surfaces.
	
	For example that the normal vector to two vectors $\vec{a}$ and $\vec{b}$ is given by the cross product (see further below page \pageref{cross product}):
	
	Or the normal to a parametric curve in Differential Geometry page \pageref{first Frenet formula}:
	
	Or the normal vector of a plane of equation $ax+by+cz+d=0$ in the section of Analytical Geometry page \pageref{vector normal plane}:
	
	Or the gradient that is perpendicular to the isolines of a surface given by the cartesian function $f(x,y,z)=0$ of smoothness $\mathcal{C}^1$ as we will see further below page \pageref{gradient normal}:
	
	and so on...
	
	

\pagebreak
\subsubsection{Multiplication by a scalar}
The vector expression $\alpha\cdot \vec{x}$ named "\NewTerm{product of vector $\vec{x}$ by scalar $\alpha$}"\index{vector scalar product} is defined as follows:

Take a representative arrow $\vec{x}$ and construct a same direction arrow in the same or opposite orientation , depending on whether $\alpha$ (scalar) is positive or negative, and of intensity $\mid \alpha \mid$ times the intensity of the initial arrow. The arrow thus obtained is a representative of the vector of relation:
	
If $\alpha=0$ or $\vec{x}=0$ we write:
	
The operation consisting of performing the product of a scalar by a vector is named "\NewTerm{scalar multiplication}\index{scalar multiplication}".

We easily check that the scalar multiplication is associative and distributive with respect to the vector numerical addition, formally:
	

	The multiplication of a vector by non-null scalar doesn't change its direction if the scalar is positive but if the scalar is negative the vector will still have the same direction but it orientation will be opposite.
	
	From this definition he have that two vectors $\vec{v}$ and $\vec{w}$ are parallel (denoted by $\vec{v}\mid\mid\vec{w})$) if one is a scalar multiple of the other. You can think of scalar multiplication of a vector as stretching or shrinking the vector, and as flipping the vector in the opposite direction if the scalar is a negative number.
	
	
	Let's see a concrete example worldwide known of use of vectors with scalars (probably also the simplest example):
	
	\paragraph{Rule of three}\mbox{}\\\\
	Let us go back to the "\NewTerm{rule of three}"\index{rule of three} (sometimes also named "\NewTerm{rules of ratios and proportions}\index{rules of ratios and proportions}" or "\NewTerm{unit reduction method}\index{unit reduction method}") often define in small classes (middle-school) intuitively but without nice proof. This rule is probably the most widely used algorithm in the world that identifies a fourth number when are given three and the four numbers are linearly dependent.

The rule of three is derived most of time in two versions:
	\begin{enumerate}
		\item[V1.] Simple an direct if the magnitudes are directly proportional.
		\item[V2.] Simple and reverse if the quantities are inversely proportional.
	\end{enumerate}
and when two variables $X$ and $Y$ are proportional we note for recall:
	
\begin{theorem}
Suppose now that $X$ can take the values $x_1,x_2$. $Y$ will take  the values linearly dependent $y_1,y_2$ then the following proportional relation applies:
	
is named "\NewTerm{simple and direct ratio}\index{simple and direct ratio}".
\end{theorem}
\begin{dem}
	Given two collinear vectors $\vec{x}=(x_1,x_2),\vec{y}=(y_1,y_2)$ and therefore proportional to a given factor $\lambda$ such that:
	
	\begin{flushright}
		$\blacksquare$  Q.E.D.
	\end{flushright}
\end{dem}
	\begin{tcolorbox}[title=Remarks,colframe=black,arc=10pt]
If this ratio is not equal (thus: not proportional), then we must switch to other tools such as simple and inverse ratio, or regression techniques and verbatim: extrapolation.
	\end{tcolorbox}	
	\begin{tcolorbox}[colframe=black,colback=white,sharp corners]
\textbf{{\Large \ding{45}}Example:}\\\\
In Lausanne (Switzerland), in 2011, garbage bags ar taxed and following rates apply: a bag of $17 [L]$ is $1.-$ and the bag of $110 [L]$ is $3.80.-$. Reported to $17 [L]$ the price of the garbage bag of 110 [L] is thus:
	
That is to say approximately $60\%$ of the price of the bag of $17 [L]$ (then go search for an explanation... ???).
	\end{tcolorbox}
\begin{theorem}
The following proportional relation:
	
is named "\NewTerm{simple and inverse ratio}\index{simple and inverse ratio}".
\end{theorem}
\begin{dem}
	Given two collinear vectors $\vec{x}=(x_1,x_2),\vec{y}=(y_1,y_2)$ and therefore proportional to a given factor $\lambda$ such that:
	
	\begin{flushright}
		$\blacksquare$  Q.E.D.
	\end{flushright}
\end{dem}
	\begin{tcolorbox}[title=Remark,colframe=black,arc=10pt]
\textbf{R1. }If this ratio is not equal (thus: not inverse proportional), then we must switch to other tools such as simple and direct ratio, or regression techniques and verbatim: extrapolation.\\
\textbf{R2.} We also name "\NewTerm{simple or inverse joint rule}\index{simple or inverse joint rule}", a series of direct or inverse rule of three.
	\end{tcolorbox}	

	Basically, it is enough that we knew three of the four variables to solve this simple equation of the first degree.

	In such calculations, the agents of the exchange market have noticed that most of the time the ratio values were close to unity. They were thus naturally led to define the "percentage" as the proportion of a quantity or magnitude relative to another, measured with hundred (at least most of time...). Remember (\SeeChapter{see section Numbers page \pageref{percentage}}):

	\begin{itemize}
		\item Given a scalar $x \in \mathbb{R}$ then expressed in percentage it will denoted by:
			
		\item Given a scalar $x \in \mathbb{R}$ then expressed in per-thousand it will denoted by:
			
	\end{itemize}

	\subsection{Vector Spaces}
	\textbf{Definition (\#\mydef):} We name "\NewTerm{vector space}\index{vector space}" a set $E$ of elements designated by $\vec{x},\vec{y},...$ and named (as we know) "vectors", with a "vector algebraic structure" defined by the operation of vector addition (and thus vector subtraction) and scalar multiplication. These two operations satisfy the laws of associativity, commutativity, distributivity, neutral element and opposing element as we have already seen in the section of Set Theory.

	For more information about what a vector space set is exactly  the reader will have therefore to refer to the section of Set Theory where this concept is defined more strictly (it would be redundant to repeat it here an anyway it is not crucial because the properties are intuitive).

	For every positive integer $n$, $a_i$ means all the $n$-tuples of numbers arranged in a vector column:
	
	or as line vector (vector column that has been \textbf{T}ransposed):
	
	and $\mathbb{R}^n$ provides clearly a vector space structure. The vectors of this space will be named as we already know: "vectors". They are often denoted more briefly by:
	
	or even more briefly by:
	
	The number $a_i$ is sometimes name "\NewTerm{term}\index{vector term}" or "\NewTerm{component of index $i$}\index{vector component}" of $(a_i)$.

Now, unless stated otherwise, the vectors will always be the elements of a vector space $E$.

	\subsubsection{Linear Combinations}\label{linear combinations}
	\textbf{Definition (\#\mydef):} We name "\NewTerm{linear combination}\index{linear combination}\label{vector linear combination}" of vectors any vector relation of the form:
	
When a vector can be expressed in the above way we say that the vector is in "\NewTerm{component form}\index{vector component form}".

	The null vector $\vec{0}$ is a linear combination of the $\alpha_1\vec{x}_1+\alpha_2\vec{x}_2+...+\alpha_n\vec{x}_n$ with all coefficients equal to zero. We speak therefore of "\NewTerm{trivial linear combination}\index{vector trivial linear combination}".

	\textbf{Definition (\#\mydef):} We name "\NewTerm{convex combination}\index{convex combination}", any linear combination whose coefficients are non-negative and sum equal to 1. The set of convex combinations of two points $P$ and $Q$ of a punctual space $P_0$ (with an origin) is the line segment $P$ and $Q$. To realize this, we just write:
	
	and we make $\alpha$ vary from 0 to 1 and to find that all the points of the segment are thereby obtained.
	
	If the vector $\vec{v}$ is a linear combination of $\alpha_1\vec{x}_1+\alpha_2\vec{x}_2+...+\alpha_n\vec{x}_n$ and each of these vectors $\vec{x_i}$ is a linear combination of a set of independent vectors $\vec{y}_1,\vec{y}_2,...,\vec{y}_n$, then it could be obvious that $\vec{v}$ is also a linear combination of  $\vec{y}_1,\vec{y}_2,...,\vec{y}_n$.
	
	\textbf{Definition (\#\mydef):} A number $n$ of non-zero vector are "\NewTerm{coplanar}\index{coplanar}" if one of them is a linear combination of the others. For example, three vectors are coplanar if one of them is in the plane defined by the two others.
	
	\subsubsection{Sub-vector spaces}
	\textbf{Definition (\#\mydef):} We name "\NewTerm{vectorial subspace $V$ of $E$}\index{vectorial subspace}" any subset of $E$ that is itself a vector space for the operations of addition and scalar multiplication defined in $E$.
	
	A vectorial sub-space $V$, as a vectorial space, can not be empty as it includes at least one vector, i.e. its zero vector, this being also necessarily also the zero vector of $E$. In addition, together with the vectores $\vec{x}$ and $\vec{y}$ (if it contains other vectors than the zero vector), it also includes all their linear combinations $\alpha\vec{x}+\beta\vec{y}$.
	
	Conversely, as soon as we see any subset having these properties is a vectorial subspace. We have thus established the following proposition:
	
	A subset $V$ of $E$ is a subspace of $E$ if and only if $V$ is not empty and $\alpha\vec{x}+\beta\vec{y}$ belongs to $V$ for every pair $(\vec{x},\vec{y})$ of $V$ and all any pair $(\alpha,\beta) \in \mathbb{R}$.
	
	\subsubsection{Generating families}
	It follows that if we have a family of vectors $(\vec{x}_1,\vec{x}_2,...,\vec{x}_k)$ the set of linear combinations of $\vec{x}_1,\vec{x}_2,...,\vec{x}_k$ with $k<n$ can be a subspace $S$ of $E$, more specifically the smallest subspace of $E$ consisting of $\vec{x}_1,\vec{x}_2,...,\vec{x}_k$.
	
	The $\vec{x}_1,\vec{x}_2,...,\vec{x}_k$ vectors that satisfy the above condition are named "\NewTerm{generators}\index{generators of a family of vectors}" of $S$ and the family $(\vec{x}_1,\vec{x}_2,...,\vec{x}_k)$ the "\NewTerm{generating family}\index{generating family}" of $S$. We also say that these vectors or family "generate $S$".
	
	\begin{tcolorbox}[title=Remark,colframe=black,arc=10pt]
The subspace generated by a nonzero vector consists of all multiples of this vector. We name such a subspace a "\NewTerm{vector line}\index{vector line}". A subspace generated by two vector non multiple of each other is named a "\NewTerm{vector map}\index{vector map}" or "\NewTerm{vector plane}\index{vector plane}".
	\end{tcolorbox}
	
	\subsubsection{Linear Dependence or Independence}
	What follows is very important in physics: we advise future physicists or engineer really take the time to read the developments below.
	
	If $(\vec{e}_1,\vec{e}_2,\vec{e}_3)$ are three vectors of $e^3$ whose representatives are not parallel to the same plane (by convention a zero-vector is parallel to any plane), so any vector $\vec{x}$ of $E^3$ can be written by the linear combination:
	
	where $\alpha_1,\alpha_2,\alpha_3$ are typically in $\mathbb{R}$.
	For example the above vector $\vec{x}$ (but can also be obtained for different values of $\alpha_i$!):
	\begin{figure}[H]
		\centering
		\includegraphics[scale=0.75]{img/algebra/vector_linear_combination.jpg}
		\caption{Example of a construction of a vector in a three-dimensional space}
	\end{figure}
	In particular, the only possibility to get the zero vector $\vec{0}$ as a linear combination of $(\vec{e}_1,\vec{e}_2,\vec{e}_3)$ is to assign the trivial value $0$ to the coefficients $\alpha_1,\alpha_2,\alpha_3$.

	Conversely, if for three vectors  $\vec{e}_1,\vec{e}_2,\vec{e}_3$ of $E^3$ the relation:
		
	implies $\alpha_1=\alpha_2=\alpha_3=0$, any vectors may be linear combination of the other two, in other words, their representatives are not parallel to the same plane.
	
	Based on these observations, we will extend the notion of absence of parallelism to a same plane in the case of any number of vectors of a given vector space $E$.
	
	We say that the vectors $\vec{x}_1,\vec{x}_2,...,\vec{x}_k$ are "\NewTerm{linearly independent}\index{linearly independent vectors}" if the relation:
	
	necessarily implies  $\alpha_1=\alpha_2=...=\alpha_k=0$, in other words, if the trivial linear combination is the only linear combination of $\vec{x}_1,\vec{x}_2,...,\vec{x}_k$ which is zero. Otherwise, we say that the vectors $\vec{x}_1,\vec{x}_2,...,\vec{x}_k$ are "\NewTerm{linearly dependent}\index{linearly dependent vectors}".
	
	If the intention is fixed on the family $(\vec{x}_1,\vec{x}_2,...,\vec{x}_k)$ rather than the terms of which it is made, we say that the latter is a "\NewTerm{free family}\index{free family (vector calculus)}" or "\NewTerm{linked family}\index{linked family (vector calculus)}" following that the vectors are linearly independent or dependent.
	
	\subsubsection{Base of a vectorial space}
	\textbf{Definition (\#\mydef):} We say that a family of finite vectors is a basis of $E$\index{vector basis}\label{vector basis} if and only if:
	\begin{enumerate}
		\item If it is free.
		
		\item It generates $E$.
	\end{enumerate}
	Following this definition, every free family $\vec{x}_1,\vec{x}_2,...,\vec{x}_k$ is a basis of the subspace it generates.
	
	\begin{tcolorbox}[colframe=black,colback=white,sharp corners]
	\textbf{{\Large \ding{45}}Example:}\\\\
	If we consider $\mathbb{C}$ as a $\mathbb{R}$-vector space (\SeeChapter{see section Set Theory page \pageref{vector space}}), then since all the elements of $\mathbb{C}$ are written $a+\mathrm{i}b$, the elements that generate $\mathbb{C}$ are $1$ and $\mathrm{i}$ (both are free).\\
	
	A base of $\mathbb{C}$ (which is 2-dimensional) as a $\mathbb{R}$-vector space is therefore the free finite set $\left\lbrace 1, i \right\rbrace$.
	\end{tcolorbox}
	For a family of vectors $(\vec{e}_1,\vec{e}_2,...,\vec{e}_n)$ to be a basis of $E$, then it is necessary and sufficient that every vector $\vec{x}$ of $E$ is expressed uniquely as a linear combination of the vectors $(\vec{e}_1,\vec{e}_2,...,\vec{e}_n)$:
	
	The above relation is decomposition of $\vec{x}$ following the base $(\vec{e}_1,\vec{e}_2,...,\vec{e}_n)$ where the coefficients $x_1,x_2,...,x_n$ are the components of $\vec{x}$ in this base. In the presence of a base, each vector is determined entirely by its components.
	
	Proposition:
	
	If $x_1,x_2,...,x_n$ are the components of $\vec{x}$ and $y_1,y_2,...,y_n$ those of equation then: 
	
	are the components of $\vec{x}+\vec{y}$.
	
	In other words, add two vectors is equivalent to add their components and multiply a vector by a scalar obviously equivalent to multiplying its components by the same scalar. The basis is an important tool because it allows you to perform operations on vectors through operations on numbers.
	
	\begin{tcolorbox}[colframe=black,colback=white,sharp corners]
	\textbf{{\Large \ding{45}}Example:}\\\\
	The following column vectors of $\mathbb{R}^n$:
	
	generate a base that we name "\NewTerm{canonical basis}\index{canonical basis}\label{canonical basis}" of $\mathbb{R}^n$ (we will work in complex spaces in another section of this book).
	\end{tcolorbox}
	\begin{tcolorbox}[title=Remark,colframe=black,arc=10pt]
	As part of the three-dimensional space, bases are very often treated as a triad (actually if you connect the ends of the three vectors by features you will get an imaginary triad).
	\end{tcolorbox}	
	
	\pagebreak
	\subsubsection{Direction Angles}
	It is clear that only one standard angle cannot describe the direction of a vector in space. We then use the concept of "\NewTerm{direction angles}\index{direction angles}". This is to measure the angle of the vector $\vec{U}$ with respect to each of the positive axis of the base:
	\begin{figure}[H]
		\centering
		\includegraphics{img/algebra/direction_angles.jpg}
		\caption{Representation of direction angles}
	\end{figure}
	if:
	
	Then by definition:
	
	The values:
	
	are named the "\NewTerm{cosines directions}\index{cosines directions}\label{cosines directions}" of $\vec{x}$.
	
	The three angles mentioned are not completely independent. Indeed, two are enough to completely determine the direction of a vector in space, the third can be deduced from the following equality (obtained from the calculation of the sum of squares of previous relations):
	
	Therefore the direction cosines are the scalar components of a unit standard vector  $\vec{u}$ having the same direction as $\vec{U}$:
	
	
	\subsubsection{Dimensions of a vector space}
	We say that a basis $E$ is of "\NewTerm{finite size}\index{finite size basis}" if it is generated by a finite family of vectors. Otherwise, we say that $E$ is of "\NewTerm{infinite dimension}\index{infinite dimension basis}" (we'll discuss this type of spaces in another section). Any finite dimensional vector space and not reduced to the zero vector has a basis. In fact, from any generating family of such a vector space we can extract a basis.
	
	The dimension of a vector space is denoted by:
	
	Any vector space $E$ of nonzero finite dimension $n$ can be mapped in one to one correspondence (that is to say in bijection) with $\mathbb{R}^n$. We just need to choose a basis of $E$ and to match to any vector $\vec{x}$ of $E$ the column vector whose terms are the components of $\vec{x}$ in the chosen basis (this is  mathematician blah blah but it will be useful when we will discuss more complex spaces):
	
	This correspondence preserves the operations of addition and multiplication by a scalar as we have already seen; in other words, it can perform operations on vectors by operations on numbers.
	
	\begin{tcolorbox}[title=Remark,colframe=black,arc=10pt]
For "classic" resolution methods of such systems, we refer the readers to the section on Numerical Methods of the chapter on Computing Science.
	\end{tcolorbox}	
	Then we say that $E$ and $\mathbb{R}^n$ are "\NewTerm{isomorphic}\index{isomorphic basis}" or that the correspondence is an isomorphism (\SeeChapter{see section Set Theory page \pageref{isomorphism}}).
	
	\subsubsection{Extension of a free family}
	\begin{theorem}
	Given $(\vec{x}_1,\vec{x}_2,...,\vec{x}_k)$ a free family and $(\vec{v}_1,\vec{v}_2,...,\vec{v}_m)$ a generating family of $E$. If $(\vec{x}_1,\vec{x}_2,...,\vec{x}_k)$ is not a basis of $E$, we can extract a subfamily $$(\vec{v}_{i1},\vec{v}_{i2},...,\vec{v}_{il})$$ of $(\vec{v}_1,\vec{v}_2,...,\vec{v}_m)$ so that the family $(\vec{x}_1,\vec{x}_2,...,\vec{x}_k,\vec{v}_{i1},\vec{v}_{i2},...,\vec{v}_{il})$ is a basis of $E$.
	\end{theorem}
	\begin{tcolorbox}[title=Remark,colframe=black,arc=10pt]
	Such a theorem is useful when going from a mathematical space passage having given properties to another space with different mathematical properties.
	\end{tcolorbox}	
	\begin{dem}
	We assume that at least one of the vectors $\vec{v}_i$ is not a linear combination of vectors $(\vec{x}_1,\vec{x}_2,...,\vec{x}_k)$, otherwise $(\vec{x}_1,\vec{x}_2,...,\vec{x}_k)$ would generate $E$ and would therefore be a possible basis of $E$. Let us note that vector $\vec{v}_{i1}$. The family $(\vec{x}_1,\vec{x}_2,...,\vec{x}_k,\vec{v}_{i1})$ is then a free family. Indeed, the relation:
	
	then implies first that $\beta_1=0$, otherwise $\vec{v}_{i1}$ would be a linear combination of the vectors $\vec{x}_1,\vec{x}_2,...,\vec{x}_k$, and then all $\alpha_i=0$ since the vectors $\vec{x}_1,\vec{x}_2,...,\vec{x}_k$ are linearly independent.
	
	If the family $(\vec{x}_1,\vec{x}_2,...,\vec{x}_k,\vec{v}_{i1})$ generates $E$, it is then a possible base for $E$ and the theorem is proved. Otherwise, the same reasoning ensures the existence of another vector $\vec{v}_{i2}$ .... If the new resulting family is not a basis of $E$, then the extraction process vectors $\vec{v}_i$ of $(\vec{v}_1,\vec{v}_2,...,\vec{v}_m)$ continues. When it stops, we will get an "extension" of $(\vec{x}_1,\vec{x}_2,...,\vec{x}_k)$ in a free family generating $E$, that is to say a base of $E$.
	\begin{flushright}
		$\blacksquare$  Q.E.D.
	\end{flushright}
	\end{dem}
	It returns a corollary: 

	Every finite dimensional vector space and not reduced to zero vector has a basis! In fact, from any generating family of such a space, we can extract a base.
	
	\subsubsection{Rank of a finite family}
	\textbf{Definition (\#\mydef):} We name "\NewTerm{rank of a family of vectors}\index{rank of a family of vectors}" and denote by $\text{rk}(S)$ the dimension of the subspace $S$ of $E$ it creates.
	
	\begin{theorem}
	The rank of a family of vector $(\vec{x}_1,\vec{x}_2,...,\vec{x}_k)$ is less than or equal to $k$ and is equal to $k$ if and only if the family is free.
	\end{theorem}
	\begin{dem}
	Let us set aside the trivial first case where the rank of the family $(\vec{x}_1,\vec{x}_2,...,\vec{x}_k)$ is zero. By the previous corollary, then we can extract from this family a base of the subspace it generates. The rank is than less or equal to $k$ following that $(\vec{x}_1,\vec{x}_2,...,\vec{x}_k)$ is a linked family or not.
	\begin{flushright}
		$\blacksquare$  Q.E.D.
	\end{flushright}
	\end{dem}
	
	\pagebreak
	\subsubsection{Direct Sums}
	\textbf{Definition (\#\mydef):} We say that the sum $S + T$ of two subspaces $S$ and $T$ of $E$ is a "\NewTerm{direct sum}\index{direct sum of subspaces}\label{direct sum}" if (special case applied to a 2 dimensional space!):
	
	In this case, we note it:
	
	In other words, the sum of two vector subspaces $S$ and $T$ of $E$ is direct if the decomposition of all element $S + T$ into a sum of an element of $S$ and of $T$  is unique.
	
	\begin{tcolorbox}[colframe=black,colback=white,sharp corners]
	\textbf{{\Large \ding{45}}Example:}\\\\
	For example, the $XY$-plane, a two-dimensional vector space, can be thought of as the direct sum of two one-dimensional vector spaces, namely the $X$ and $Y$ axes. In this direct sum, the $x$ and $y$ axes intersect only at the origin (the zero vector). Addition is defined coordinate-wise, that is:
	 
	which is the same as vector addition.
	\end{tcolorbox}	
	This concept of trivial decomposition will be very useful in some theorems, the most important in this book is definitely the spectral theorem (\SeeChapter{see section of Linear Algebra page \pageref{spectral theorem}}) that has important implications in statistics!!!
	
	From the direct sum we can introduce the concept of "\NewTerm{complementary subspace}\index{complementary subspace}" also named "\NewTerm{subspace}\index{subspace}" (depending on countries ...):
	\begin{theorem}
	Suppose that $E$ is of finite dimensions. For any subspace $S$ of $E$, there exists a subspace $T$ (not unique) of $E$ such that $E$ is the direct sum of $S$ and $T$. We say then that $T$ is a "\NewTerm{supplementary subspace}\index{supplementary subspace}" of $S$ into $E$.
	\end{theorem}
	\begin{dem}
	First let us set aside the trivial case where $S=\left\lbrace \vec{0} \right\rbrace$  and $S = E$. The subspace $S$ admits a basis $(\vec{e}_1,\vec{e}_2,...,\vec{e}_k)$, where $k$ is less than the dimension $n$ of $E$. By the theorem of extension of a free family, this basis can be extended in a basis $(\vec{e}_1,\vec{e}_2,...,\vec{e}_k,\vec{e}_{k+1},...,\vec{e}_n)$ of $E$. Let $T$ be the subspace vector generated by the family $(\vec{e}_{k+1},...,\vec{e}_n)$ . If $\vec{x}$ is any vector of $E$, then $\vec{x}=\vec{s}+\vec{t}$, where $\vec{s}$ is a vector of $S$ and $\vec{t}$ a vector of $T$. In addition $S\cap T=\left\lbrace \vec{0} \right\rbrace$, because no vector, excepted the zero vector may be a linear combination of the vectors $\vec{e}_1,...,\vec{e}_k$ and of the vectors $\vec{e}_{k+1},...,\vec{e}_n$. We therefore conclude that:
	 
	\begin{flushright}
		$\blacksquare$  Q.E.D.
	\end{flushright}
	\end{dem}
	
	\subsubsection{Affine spaces}\label{affine space}
	In mathematics, an affine space $G=\text{A}\mathbb{R}^n$ is a geometric structure that are independent of the concepts of distance and measure of angles, keeping only the properties related to parallelism and ratio of lengths for parallel line segments as their is not origin point $\text{=}(0,0)$.
	
	The space $G$ of elementary geometry is both common and the source of the concept of "affine space" that we will introduce because when high-school student begins learn geometry they learn it without any reference point $\text{=}(0,0)$.

	In an affine space, there is therefore no distinguished point that serves as an origin. Hence, no vector has a fixed origin and no vector can be uniquely associated to a point. In an affine space, there are instead "\NewTerm{displacement vectors}\index{displacement vectors}", also named "\NewTerm{translation vectors}\index{translation vectors}\label{translation vector}" or simply translations, between two points of the space.
	
	This space $G$ is associated with the "\NewTerm{geometric vector space}\index{geometric vector space}" $V$ by the correspondence between vectors and arrows studied so far! The following definition is only to highlight the main common points of this correspondence:
	
	\textbf{Definition (\#\mydef):} Let $G$ be a non-empty set of elements that we name "\NewTerm{points}\index{points in a vector space}" and let us  denote them by the letters $P, Q, ...$; given also $E$ a vector space. Suppose that to any two points $(P, Q)$ corresponds a vector denoted $\overrightarrow{PQ}$ (typically the point $P$ is chosen as fictive origin). We say then that $U$ is an "\NewTerm{affine space}\index{affine space}" of directed space $E$ if the following conditions are met:
	\begin{enumerate}
		\item[C1.] For any fixed point $P$, the correspondence between pairs $(P, Q)$ and vectors $\vec{x}$ defined by only one point plus the origina point is bijective, ie, for every vector $\vec{x}$ it exists a point $Q$ such that we can define a vector $\overrightarrow{PQ}$.
		
		\item[C2.] For each triple of points $(P, Q, R)$:
		
		This is the famous "\NewTerm{Chasles relation}\index{Chasles relation}" (which we will see later a pseudo-equivalent in the section of Differential and Integral Calculus).
		
		\item[C3.] If $P$ is a point and a $\vec{x}$ vector, to express that $Q$ is the unique point such as $\vec{x}=\overrightarrow{•}{PQ}$, we write:
		
		Although being a bit excessive, this writing is consistent with the usage and suggests well the idea of the operation it designates.
	\end{enumerate}
	\begin{tcolorbox}[colframe=black,colback=white,sharp corners]
	\textbf{{\Large \ding{45}}Example:}\\\\
	Below an "artistic" example of an affine space $G=\text{A}\mathbb{R}^2$ where there is no origin and any extremity of a line can be considered as the origin of a vector:
	\begin{figure}[H]
		\centering
		\includegraphics{img/algebra/affine_space.jpg}
		\caption{Artistic but real example of an $G=\text{A}\mathbb{R}^2$ affine space}
	\end{figure}
	\end{tcolorbox}
	
	The following properties follow directly from the definition of affine space:
	\begin{enumerate}
		\item[P1.] For any point $P$, $P+(\vec{x}+\vec{y})=(P+\vec{x})+\vec{y}$
		
		\item[P2.] For any point $P$, $\overrightarrow{PP}=\vec{0}$. This results from the $\overrightarrow{PQ}+\overrightarrow{QR}=\overrightarrow{PR}$ provided in the case where we have $P=Q=R$.
		
		\item[P3.] $\overrightarrow{PQ}=-\overrightarrow{QP}$. Just put $R = P$ in the De Chasles relation $\overrightarrow{PQ}+\overrightarrow{QR}=\overrightarrow{PR}$.
		
		\item[P4.] Parallelogram rule:
		Given the polygon with the vertices (clockwise) $P,P',Q,Q'$ and edges $\overrightarrow{PP'},\overrightarrow{P'Q'},\overrightarrow{QQ'},\overrightarrow{PQ}$:
		\begin{figure}[H]
			\centering
			\includegraphics{img/algebra/affine_parallelogram.jpg}
			\caption{Vector polygon in $\text{A}\mathbb{R}^2$}
		\end{figure}
		We have:
		
		if and only if:
		
		which would then give a parallelogram!
	
		Indeed, replacing $R$ with $Q'$ in the Chasles relation we have:
		
		and by doing the same but replacing $R$ with $Q$ and $Q$ by $P'$ we get:
		
		We then have by equalizing the last two relations:
		
	\end{enumerate}
	Earlier we saw what made that a space $G$ could be provided with a vector space structure (we saw then that it was therefore "vectorialized"). In the general case of an affine space $G$, the process is the same:
	
	We choose any point $\text{O}$ of $G$. The correspondence between pairs $(\text{O},P)$ and vectors of director space $E$ being therefore biunivocal we define then the addition of points and multiplication of a point by a scalar by the corresponding operations on the vectors of $E$. Armed with these two operations, $G$ becomes a vector space, named "\NewTerm{vectorialized space $G$ regarding to $Q$}\index{vectorialized space}". We denote this space by $V$ and named the point $\text{O}$ "\NewTerm{origin}\index{origin of a vector space}".
	
	Given how operations have been defined, it follows that $V$ is isomorphic to the space vector $E$:
	
	However, this isomorphism depends on the choice of the origin $\text{O}$ and in practice this origin is selected on the basis of the data inherent to the studied problem. For example, if an affine transformation allows an invariant point (which does not move), it is advantageous to select that point as the origin.
	\begin{tcolorbox}[title=Remarks,colframe=black,arc=10pt]
	\textbf{R1.} When we talk about dimension of an affine space, we talk about the size of its director space.\\
	
	\textbf{R2.} The space $G$ of elementary geometry is an affine space of type $\text{A}\mathbb{R}^2$. Indeed, its direction is the geometrical space $G$ and the conditions of definition of affine space are met.\\
	
	\textbf{R3.} An affine space is a set of elements with a difference function. This difference is a binary function, which takes two points $p$ and $q$ (both in $G$) and yields an element (a vector) $\vec{v}$ of a vector space $E$. We write $\vec{v}=p-q.$ Additionally, this difference function must ensure that, for any point $p$ in $E$, it holds$p-p=0$, where $\vec{0}$ is the null vector of $E$.
	\end{tcolorbox}
	
	\pagebreak
	\subsection{Euclidean Vector Spaces}
	Before defining what is an Euclidean vector space, let us first define some mathematical tools and some concepts.
	
	We can, by choosing a unit length, measure the intensity of each arrow, in other words, determine its length. We can also measure the angular distance of two arrows (or vectors) of any common origin (not necessarily distinct) taking as the unit of angle measurement for example the radian (\SeeChapter{see section Trigonometry page \pageref{radian}}). The measurement of this difference is then a number between $0$ and $\pi$ named "\NewTerm{angle}\index{angle between two vector}" of the two arrows (see the section of Euclidean Geometry for more details). If both arrows have same direction and orientation, their angle is zero and if they have same direction and the opposite orientation, this same angle is $\pi$.
	
	The representing arrows of a same vector $\vec{x}$ all have the same length. We denote this length (distance), named also "\NewTerm{norm}\index{norm}\label{vector norm}" or "\NewTerm{module}\index{module}", by the notation:
	
	given in the three-dimension case explicitly by the following Euclidean distance\label{euclidean distance vector} if the origin of the vector coincides with the origin O of the vector basis:
	
	\begin{figure}[H]
		\centering
		\includegraphics{img/algebra/details_norm_calculation.jpg}
		\caption{Details of the calculation of the norm in an orthogonal coordinate system $\mathbb{R}^3$}
	\end{figure}
	If $\vec{x}$ is a nonzero vector we can build an unit norm vector $\vec{u}$ of same direction and orientation (colinear) by the following operation that is used a lot in physics:
	
	We will name "\NewTerm{non-zero angle of vectors $\vec{x}$ and $\vec{y}$}" the angle of two arrows of common origin representing one being $\vec{x}$ and the other $\vec{y}$.
	
	However, more strictly speaking a "norm" is defined on a real vector space (or complex) $E$, so that we speak then of "\NewTerm{normalized vector space}\index{normalized vector space}" is an application:
	
	satisfying the following properties:
	\begin{enumerate}
		\item[P1.] Positivity:
		
		\item[P2.] Linearity:
		
		where we take the modulus of the constant if this is not in the set of real numbers $\mathbb{R}$ but in the set of complex numbers $\mathbb{C}$.
		\item[P3.] Nullity (often associated with the property P1):
		
		\item[P4.] Minkowski inequality (triangle inequality):
		
		That we will prove further below.
	\end{enumerate}
	\begin{tcolorbox}[title=Remarks,colframe=black,arc=10pt]
	\textbf{R1.} These properties are mainly imposed by our intuitive approach of Euclidean space (vector space of finite dimension over the field of real number $\mathbb{R}$ and with a scalar product that we will see later) and its geometric interpretation (through the fact that it is also an affine space $\text{A}\mathbb{R}^n$).\\
	
	\textbf{R2.} We will prove a little further below the property P4 under the name of "triangle inequality" and we will do a little more general study of this inequality under the name "Minkowski inequality" in the section Topology.
	\end{tcolorbox}	
	
	\pagebreak	
	\subsubsection{Scalar Product (Dot Product)}\label{dot product}
	\textbf{Definition (\#\mydef):} An "\NewTerm{Euclidean vector space}\index{Euclidean vector space}" of dimension $n$ is a vector space (real and of finite dimensional for the purists) with a specific operation, the "\NewTerm{scalar product}\index{scalar product}" also named "\NewTerm{dot product}\index{dot product}" which we denote (notation specific to this website) regarding to the special case of vectors:
	
	\begin{tcolorbox}[title=Remarks,colframe=black,arc=10pt]
	\textbf{R1.} We find in some books (for information) the notation $\left( \vec{x}|\vec{y}\right)$ or $\langle \vec{x} | \vec{y} \rangle$ in the generalization of this definition as we shall see a little further below. According to the standard ISO 80000-2:2009 we should write the dot product as $a\cdot b$. Using the Linear Algebra notation that we will see later, the dot product will be written: $\vec{x}^T\vec{y}$.\\
	
	\textbf{R2.} The scalar product has a huge importance in the whole field of mathematics and physics; and we will see it again in all following chapters of this book. It is therefore necessary to carefully understand what follows.\\
	
	\textbf{R3.} The scalar product may be viewed as a projection of the length of a vector along the length of another one as we will see later.
	\end{tcolorbox}	
	This scalar product has the following properties (most of which stem from the definition itself) in a Euclidean space:
	\begin{enumerate}
		\item[P1.] Commutativity: $\vec{x}\circ\vec{y}=\vec{y}\circ\vec{x}$
		\item[P2.] Associativity: $\alpha(\vec{x}\circ\vec{y})=(\alpha\vec{x})\circ\vec{y}=\vec{x}\circ(\alpha\vec{y}) $
		\item[P3.] Distributivity: $\vec{x}\circ(\vec{y}+\vec{z})=\vec{x}\circ\vec{y}+\vec{x}\circ\vec{z}$
		\item[P4.] Non-degerated: $\vec{x}\circ\vec{y}=0$ then $\forall\vec{x}\neq\vec{0}\Rightarrow \vec{y}=\vec{0}$
		\item[P5.] Squared scalar: $\|\vec{x}\|^2=\vec{x}\circ\vec{x}$ and $\vec{x}\circ\vec{x}>0$ if $\vec{x}\neq \vec{0}$
		\item[P6.] Bi-linearity: $(\alpha\vec{x}+\beta\vec{y})\circ\vec{z}=\alpha(\vec{x}\circ	\vec{z})+\beta(\vec{y}\circ\vec{z})$
	\end{enumerate}
	Only the latter property requires perhaps a proof (and one of the results of the proof  will be useful to us later to prove another very important property of the scalar product):
	\begin{dem}
	Given:
	
	which is the "\NewTerm{orthogonal projection vector}\index{orthogonal projection vector}\label{orthogonal projection vector}" (the $x$ on index of $\text{project}$ meaning "the vector $\vec{x}$") of the vector $\vec{y}$ of standardization at the unit of vector $\vec{x}$.
	
	Using the scalar product, the vector $\text{proj}_x\vec{y}$ can be expressed otherwise, we just need to take the relation that we have seen above:
	\begin{figure}[H]
		\centering
		\includegraphics{img/algebra/dot_product.jpg}
		\caption{Geometrical representation of the dot product (projection)}
	\end{figure}
	
	and introduce it into $\text{proj}_x\vec{y}$ with to obtain:
	
	The norm of $\text{proj}_x\vec{y}$ is written:
	
	If $\vec{x}$ has a unit norm, the relations of previous projections are simplified and become obviously:
	
	By elementary geometric considerations (distributivity of the scalar product), it is easy to realize that:
	
	If we now return to the proof of the bi-linearity property:
	
	We have in a first time:
	
	and, from the definition of the property of the orthogonal projection, it comes immediately by a one-to-one correspondence:
	
	hence the property $P6$ that follows by multiplying the two members of equality by $\vec{z}\circ\vec{z}$ and after by simplification by $\vec{z}$.
	\begin{flushright}
		$\blacksquare$  Q.E.D.
	\end{flushright}
	\end{dem}
	\begin{enumerate}
		\item[D1.] A vector space $E$ is said to be an "\NewTerm{proper Euclidean vector space}\index{proper Euclidean vector space}" if $\forall \vec{x} \in E \qquad \|\vec{x}\|>0$.
		
		\item[D2.] We say that the vectors $\vec{x}$ and $\vec{y}$ are "\NewTerm{orthogonal vectors}\index{orthogonal vectors}" if they are non-null and that their scalar product is equal to zero (their angle is equal to $\pi/2$).
		
		\item[D3.] A basis of vectors $(\vec{e}_1,\vec{e}_2,...,\vec{e_n})$ is said to be an "\NewTerm{orthonormal basis}\index{orthonormal basis}" if all the vectors $\vec{e}_1,\vec{e}_2,...,\vec{e_n}$ are pairwise orthogonal and their norm is equal to the unit (thus constituting a: free family).
	\end{enumerate}
	\begin{tcolorbox}[title=Remark,colframe=black,arc=10pt]
	We will see in the section Tensor Calculus (we could have done here too but we don't use it for a practical case in this section therefore...) how from a set of independent vectors build an orthogonal basis. This is what the reader will find under the name  "(Gram-)Schmidt orthogonalization method".
	\end{tcolorbox}	
	By a simple geometric argument, we see that every vector is the sum of its orthogonal projections on the vectors of an orthonormal basis, that is, if $(\vec{e}_1,\vec{e}_2,\vec{e}_2)=(\vec{u},\vec{v},\vec{w})$ is an orthonormal basis in $\mathbb{R}^3$ for example:
	
	\begin{figure}[H]
		\centering
		\includegraphics{img/algebra/orthonormal_basis_projection.jpg}
		\caption{Orthonormal basis projection example}
	\end{figure}
	This decomposition is also obtained by the P6 property of the scalar product. Indeed, consider the components $(\vec{x}_1,\vec{x_2},\vec{x}_3)$ of a vector $\vec{x}$ in our orthonormal basis:
	
	since $\vec{e}_1\circ\vec{e}_1=1$ and $\vec{e}_1\circ\vec{e}_2=0$. Therefore we get immediately:
	
	hence the decomposition.
	
	Given the respective components $(\vec{x}_1,\vec{x_2},\vec{x}_3)$ and $(\vec{y}_1,\vec{y_2},\vec{y}_3)$ of the vectors  $\vec{x}$ and $\vec{y}$ vectors in a canonical orthonormal basis $(\vec{e}_1,\vec{e_2},\vec{e}_3)$ we know now that we can write the scalar product in the form:
	
	by the property P6 of the scalar product:
	
	using the properties P1 and P6 again:
	
	Which finally gives us the very famous and important decomposition:
	
	This is one of the most important relation in the field of vector calculus, which we name "\NewTerm{canonical scalar product}\index{canonical scalar product}" or "\NewTerm{canonical dot product}\index{canonical dot product}".
	
	\begin{tcolorbox}[title=Remark,colframe=black,arc=10pt]
	Then angle $\theta$ of the dot product is sometimes denoted by:
	
	\end{tcolorbox}
	Now let us prove with a simple two dimensional case a property that physicist like a lot to characterize an orthogonal linear application (\SeeChapter{see section Linear Algebra page \pageref{orthogonal matrix}}): that the dot product is invariant under any orthogonal transformation (then abusively said "invariant under basis change...").

	For this let us consider a vector $\vec{x}=(x_1,x_2)$ and the 2D rotation matrix (\SeeChapter{see section Numbers page \pageref{rotation matrix in the plane}}):
	
	Now let us calculate:
	
	So now let us consider two vectors $\vec{a}$ and $\vec{b}$ we have proved just above that their dot product is given by:
	
	And after the chosen orthogonal transformation we get:
	
	So the dot product is indeed invariant under this rotation that is a special 2D case of orthogonal transformation  and in fact under any other orthogonal transformation.
	
	\begin{tcolorbox}[title=Remark,colframe=black,arc=10pt]
	The dot product, must not  be confused with the "\NewTerm{element-wise multiplication}\index{element-wise multiplication}\label{element-wise multiplication}" of two vectors defined by:
	
	\end{tcolorbox}		
	
	\pagebreak
	\paragraph{Cauchy–Schwarz inequality}\mbox{}\\\\
	In mathematics, the Cauchy–Schwarz inequality is a useful inequality encountered in many different settings, such as linear algebra, analysis, probabilities, statistics and other areas (just read this book entirely to have an idea...). It is considered to be one of the most important inequalities in all of mathematics!!!
	
	The relation
	
	 can also be trivially written as follows if we use the concept of the norm and the definition of the scalar product:
	 
	 It is interesting to notice that if both $\vec{x}$ and $\vec{y}$ are orthogonal vectors, we fall back on the result of a famous theorem: the Pythagorean theorem!

	Indeed, therefore we have if the two vectors are orthogonal:
	 
		This gives us:
	 
	This relations is very important in physics and mathematics. It must be remembered!
	 
	\begin{theorem}
	We name "\NewTerm{Cauchy-Schwarz inequality}\index{Cauchy-Schwarz inequality}\label{cauchy-schwarz inequality}", the inequality, valid for any choice of the vectors $\vec{x}$ and $\vec{y}$, the relation:
	
	Which can also be written as:
	
	\end{theorem}
	First we will consider as obvious that equality only occurs when the two vectors are collinear.	
	\begin{dem}
	We put ourself in the case where $\vec{x},\vec{y}\neq\vec{0}$. So then $\lambda\in\mathrm{R}$ we have obviously according to the properties of the scalar product:
	
	So this is a simple equation of the second degree where variable is $\lambda$. Remembering what we saw in our study of polynomials of second degree (\SeeChapter{see section Calculus page \pageref{polynomial}}), the previous relation (that it is always greater than or equal to zero) is satisfied if the discriminant $b^2-4ac$ is negative or zero. In other words, if:
	
	Thus after simplification:
	
	\begin{flushright}
		$\blacksquare$  Q.E.D.
	\end{flushright}
	\end{dem}
	When $E$ is in $\mathbb{R}^n$, the Cauchy-Schwarz inequality is written with the vector components:
	
	In the particular case where $\forall i,b_i=1$ it becomes:
	
	or even:
	
	which shows that the square of the arithmetic mean is less than or equal to the arithmetic mean of the squares. This result is important for the study of Statistics!
	
	Furthermore, using the property of the cosine and the Cauchy-Schwarz inequality we can write immediately:
	
	relation that we will see again in the context of the study of Statistics (\SeeChapter{see section Statistics page \pageref{coefficient of correlation}}).
	
	\paragraph{Triangle Inequalities}\mbox{}\\\\
	By majoring $2\vec{x}\circ\vec{y}$ by $2\|\vec{x}\|\cdot\|\vec{y}\|$ (using the Cauchy-Schwarz inequality!) in the relation already establish previously:
	
	we get:
	
	which take us immediately by taking the root square the "\NewTerm{triangle inequality}\index{triangle inequality}\label{triangle inequality}" that is very useful for the study of Sequences and Series and also in Topology:
	
	\begin{tcolorbox}[title=Remark,colframe=black,arc=10pt]
	The generalization of this inequality relatively to the choice of the norm (that is to say: the way we define a distance) as we will see in the section of Topology, gives what we name the "\NewTerm{Minkowski inequality}\index{Minkowski inequality}".
	\end{tcolorbox}	
	By applying one time the triangle inequality to the vectors $\vec{x}$ and $(\vec{y}-\vec{x})$ and another time to vectors $(\vec{y})$ and $(\vec{x}-\vec{y})$ we get the variant:
	
	
	\paragraph{General Scalar/Dot Product}\mbox{}\\\\
	Let us see now another and little more general, formal and abstract way to define the dot product while trying to stay as simple as possible (caution! in the general case the notation of the scalar product changes! ).
	
	First the reader must know that, as we will see it in the section of Functional Analysis, all the concepts studied until now in this section can also be applied to a special category of functions! Yeeesss!!! Add, subtract functions like vector is obvious but you must know that some more or less complicated functions are colinear or orthogonal (think to affine functions!) and furthermore there are not limits to $\mathbb{R}$ but can be extended to $\mathbb{C}$ easily and hence the scalar product departure set. 
	
	To make thinks to too much complicate we will focus here only on a gentle generalization of the dot product to vectors (we will come back on functions in the section of Functional Analysis later).
	
	\textbf{Definition (\#\mydef):} Let $E$ be a real vector space (once again we focus here only on simple vectors for the moment). A "\NewTerm{positive symmetric bilinear form}\index{positive symmetric bilinear form}" on $E$ also named "\NewTerm{inner product}\index{inner product}\label{inner product}", is an application:
	
	\begin{enumerate}
		\item[P1.] Positivity: 
		
	  	
	  	\item[P2.] Nullity (defined): 
	  	 
	  	
	  	\item[P3.] Symmetry (defined): 
	  	
	  	
	  	\item[P4.] The bilinearity (bilinear form) with, in order, the "\NewTerm{linearity on the left}\index{linearity on the left}" and "\NewTerm{linearity on the right}\index{linearity on the right}":
	  	
		
		\item[P...] And so on... we have the same six properties as the scalare product as the both are the same if we focus only on vector in $\mathbb{R}$.
	\end{enumerate}
	\begin{tcolorbox}[title=Remark,colframe=black,arc=10pt]
	Again, these properties are mainly imposed by our intuitive approach of the Euclidean space and its geometric interpretation.
	\end{tcolorbox}	
	
	\textbf{Definition (\#\mydef):} A space $E$ provided with a scalar product is named in general (with the departure set in $\mathbb{C}$) a "\NewTerm{pre-Hilbert space}\index{pre-Hilbert space}" or "\NewTerm{inner product space}\index{inner product space}". If $E$ is of finite dimension, then we speak of "Euclidean space".
	\begin{tcolorbox}[title=Remark,colframe=black,arc=10pt]
	If $E$ is a Euclidean space, then if the determinant $\det(\vec{e}_1,\vec{e}_2,\ldots,\vec{e}_n)$ is equal $\pm 1$ (\SeeChapter{see section Linear Algebra page \pageref{determinant}}) we speak of "\NewTerm{oriented Euclidean space}\index{oriented Euclidean space}\label{oriented Euclidean space}".
	\end{tcolorbox}	
	
	We will see in our study of Topology (see section of the same name page \pageref{topology}) that the properties of the scalar product are the foundation bricks to set a norm and therefore a distance in a metric space. This distance will be given according to what we will see in the section of Topology:
	
	
	\textbf{Definition (\#\mydef):} We say that a space $E$ having a dot product (inner product) $ \langle \cdot | \cdot \rangle$ is a "\NewTerm{Hilbert space}\index{Hilbert space}" if this space is complete for the metric defined above.
	
	In other words, having a metric space provided with a distance generated by a scalar product is one thing. Then having a measurable distance is another one!!! A Hilbert space has thus distances measurable in the topological sense because the set we are working on is continuous and any point can be approached indefinitely (imagine having a rule and you can not approach ont this rule the points that define the dimensions of your object... it would be embarrassing...). So without complete space a lot of theorems of functional analysis (that is strongly linked to vector calculus) could not be used in the study of vector spaces and this would be very embarrassing in quantum wave physics for example...
	
	Formally, remember that a metric space is complete if all Cauchy sequences (\SeeChapter{see section Sequences and Series page \pageref{cauchy sequence}}) of this space are converging (\SeeChapter{see section Fractals page \pageref{complete space cauchy sequence}}) in a metric space (\SeeChapter{see section Topology page \pageref{metric space}}).
	
	\subsubsection{Cross Product}
	The cross product of two vectors is a proper operation to the dimension $3$. To introduce it, it is first necessary to orient the space intended to receive it. The orientation is defined by the concept of "determinant", therefore we will begin with a brief introduction to the study of this concept. This study will be repeated later in more details in the analysis of linear systems in the section of Linear Algebra.
	
	\textbf{Definition (\#\mydef):} We name basically "\NewTerm{determinant}\index{determinant}" of two column vectors of $\mathbb{R}^2$ (for the general form of the determinant see the section of Linear Algebra page \pageref{determinant}):
	
	and we denote it:
	
	the number:
	
	We name determinant of three column vectors of $\mathbb{R}^2$ (once again see the section Linear Algebra for a generalization):
	
	and we denote it:
	
	the number:
	
	Thus, the function that associates to each pair of column vectors of $\mathbb{R}^2$ (or respectively to each triplet of column vectors of $\mathbb{R}^3$) has a determinant named "determinant of order $2$" (respectively "determinant of order $3$")
	
	As we will prove it in the section of Linear Algebra the determining has the property of being multiplied by $-1$ if one of the column vectors is replaced by its opposite or two of its column vectors are exchanged. In addition, the determinant is nonzero if and only if its column vectors are linearly independent (the proof - that has a great importantce in Applied Mathematics - is a few lines further below and a generalization  can be found in the section of Linear Algebra).
	
	\textbf{Definition (\#\mydef):} Given $x_1,x_2,x_3$ and $y_1,y_2,y_3$the respective components of the vectors $\vec{x}$ and  $\vec{y}$ in the orthonormal basis $(\vec{e}_1,\vec{e}_2,\vec{e}_3)$. We name "\NewTerm{cross-product}\index{vector cross-product}" of $\vec{x}$ and $\vec{y}$, and we denote it in most books by:
	
	and in a minority of books:
	
	the vector:
	
	or as components\label{cross product matrix form}:
	
	The matrix form above will be very useful to us in the section Mechanics for the construction of the Inertial Matrix.
	\begin{tcolorbox}[title=Remarks,colframe=black,arc=10pt]
	\textbf{R1.} The first notation is the international notation due to Gibbs (which we will use throughout this book), the second is the French notation due to Burali-Forti (quite annoying because confusing with the notation of the operator AND in Proof Theory or Logical Systems).\\
	
	\textbf{R2.} It is usually quite annoying to remember by heart the relations that form the cross product. But fortunately there are at least three good mnemonics techniques:
	\begin{enumerate}

		\item The first and probably the fastest method is to remember by heart only one of the expressions of the components of the cross product and after by decrement of the indices (by starting again from $3$ when we reaches $0$) get all the other components. But we must still find a simple way to remember by heart one of the components... A good way is the following mathematical property of two collinear vectors giving an easy with to find back the third component (the one along the $z$-axis):
		
		Given two colinear vectors in the same plane, then:
		
		We fall back on the expression of the third component of the cross product of two vectors.\\
		
		Or if you want to remember only the first component given in letters by $z_x = x_y y_z - x_z y_y$ the indices gives "xyzzy" (like a name of a person...). The second and third equations can be obtained from the first by simply vertically rotating the subscripts, $x\rightarrow y \rightarrow  z \rightarrow x$. 
		
		\item The second method that we will see in details during our study of the section of Tensor Calculus is to use the Levi-Civita antisymmetric symbol. This method is certainly the most aesthetic of all but not necessarily the fastest to develop and the easiest to remember. We give here just the expression without explanations at the moment as we will study this later (but t is also useful to get the general expression of the determinant):
		
		
		\item The latter method is quite simple and trivial but it implicitly uses the first method as you must remember how the calculate at least a $2\times 2$ determinant. The idea is the following: the $i$-th component of $\vec{x}\times\vec{y}$ is the determinant of the two column vectors from which we have removed the $i$-th term, the second determinant is, however, with a "$-$" sign such that:
			  
	\end{enumerate}
	\end{tcolorbox}
	
	\pagebreak
	It is important, even if it is relatively simple to remember, that the different cross products for orthogonal basis vectors are (and especially for the canonical basis) first:
	
	and also:
	
	These relations can be condensed using the Dirac function symbol\label{orthogonal basis} (\SeeChapter{see section Tensor Calculus page \pageref{kronecker symbol}}):
	
	The vector product also has the following properties that we will prove just now:
	  \begin{enumerate}
	  	\item[P1.] Antisymmetry:
	  		
	  		
	  	\item[P2.] Linearity\label{cross product linearity}:
	  		
	  		
	  	\item[P3.] If and only if $\vec{x}$ and $\vec{y}$ are linearly independent (very important!):
	  	
	  	
	  	\item[P5.] Non associativity:
	  	
	  	
	  	\item[P4.] Distributivity over the sum:
	  	
	  \end{enumerate}
	  The first two properties directly derived from the definition and the property P4 is easily verified by developing the components and comparing the results.
	  
	 Then let us prove the third property which is very important in linear algebra (next section) and the fifth one (because requested by a reader).
	\begin{theorem}
	If and only if $\vec{x}$ and $\vec{y}$ are linearly independent (very important!):
	
	\end{theorem}
	\begin{dem}
	Given two vectors $\vec{x}(x_1,x_2,x_3)$ and $\vec{y}(y_1,y_2,y_3)$. If the two vectors are linearly dependent then there exists an $\alpha \in \mathbb{R}$ such that we can write:
	
	If we develop the cross product of two vectors that a dependent to a given factor, we get:
	
	It goes without saying that the above result is equal to the zero vector $\vec{0}$ if indeed the two vectors are linearly dependent.
	\begin{flushright}
		$\blacksquare$  Q.E.D.
	\end{flushright}
	\end{dem}
	
	\begin{theorem}
	The cross product is distributive over the sum.
	\end{theorem}
	\begin{dem}
	
	\begin{flushright}
		$\blacksquare$  Q.E.D.
	\end{flushright}
	\end{dem}
	
	If we now assume that both vector $\vec{x}$ and $\vec{y}$ are linearly independent and non-zero vector, we must prove that the cross product two properties:
	\begin{theorem}
	The resulting of a cross product results in a vector orthogonal (perpendicular) to $\vec{x}$ and $\vec{y}$ if they are not null.
	\end{theorem}
	\begin{dem}
	To prove this we simply write the development using the dot product:
	
	This equation shows that the vector $\vec{x}$ is perpendicular to the resulting vector of cross product between $\vec{x}$ and $\vec{y}$.
	\begin{flushright}
		$\blacksquare$  Q.E.D.
	\end{flushright}
	\end{dem}
	\begin{theorem}
	The cross product has for norm (module):
	
	where $\theta$ is the angle between $\vec{x}$ and $\vec{y}$.
	\end{theorem}
	\begin{dem}
		To prove this we simply write the development of the norm of the cross product:
		
		Finally:
		
	\begin{flushright}
		$\blacksquare$  Q.E.D.
	\end{flushright}
	\end{dem}
	\label{cross product as surface parallelogram}We then notice that in the case where $E$ is the Euclidean vector space, the norm of the vector product is the area (surface) of the parallelogram constructed on representatives of vectors $\vec{x}$ and $\vec{y}$ of common origin:
	\begin{figure}[H]
		\centering
		\includegraphics{img/algebra/cross_product_parallelogram.jpg}
		\caption{Geometrical representation of the cross product}
	\end{figure}
	If $\vec{x}$ and $\vec{y}$ are linearly independent, the triplet $(\vec{x},\vec{y},\vec{x}\times \vec{y})$ and also the triplet $(\vec{x},\vec{y},\vec{y}\times \vec{x})$ are direct.
	
	Indeed, $(\vec{z}_1,\vec{z}_2,\vec{z}_3)$ being the components of $\vec{x}\times \vec{y}$ (in the basis $(\vec{e}_1,\vec{e}_2,\vec{e}_3)$), the determinant of passage of $(\vec{e}_1,\vec{e}_2,\vec{e}_3)$ to $(\vec{x}\times \vec{y},\vec{x},\vec{y})$ (form example) will be written:
	 
	 This determinant is positive, as at least one of the $z_i$ is not zero, according to the third property of linear independence of the cross product.
	 
	 Here are a few very important properties of practical utility of the cross product (particularly in the different section of physics in this book!) that are trivial to check whether the developments with explicit components are done (we can make them on request if needed!):
	\begin{enumerate}
		\item[P1.] $\vec{x}\times(\vec{y}\times\vec{z})=(\vec{x}\circ\vec{z})\vec{y}-(\vec{x}\circ\vec{y})\vec{z}$
		\begin{tcolorbox}[title=Remark,colframe=black,arc=10pt]
		The latter relation is sometimes named the "\NewTerm{Grassman rule}\index{Grassman rule}\label{grassman rule}", or more commonly "\NewTerm{dual vector product}\index{dual vector product}", or even by physicists "\NewTerm{triple cross product}\index{triple cross product}",  and it is important to note that without the parentheses the result is not unique!
		\end{tcolorbox}	
		Let us prove this identity! First the left part is equal to:
		
		And for the right part:
		
		and then we see that there is indeed equality!
		\item[P2.] $(\vec{x}\times\vec{y})\circ (\vec{z}\times\vec{v})=(\vec{x}\circ\vec{z})(\vec{y}\circ\vec{v})-(\vec{x}\circ\vec{v})(\vec{y}\circ\vec{z})$
		
		\item[P3.] $(\vec{x}\times\vec{y})\circ\vec{z}=-(\vec{x}\times\vec{z})\circ \vec{y}$
		
		\item[P4.] $(\vec{x}\times\vec{y})\circ\vec{z}=\vec{x}\circ(\vec{y}\times\vec{z})$
		
		\item[P5.] $\|\vec{x}\times\vec{y}\|^2=(\vec{x}\circ\vec{x})^2(\vec{y}\circ\vec{y})^2-(\vec{x}\circ\vec{y})^2$
	\end{enumerate}
	The last identity is related to the Pythagorean theorem (\SeeChapter{see section Euclidean Geometry page \pageref{pythagorean theorem}}). Indeed, we see it better by rewriting:
	
	This can be seen from the definitions of the cross product and dot product, as
	
	
	\pagebreak
	\subsubsection{Mixed Product (triple product)}
	We can extend the definition of the vector product to another type of mathematical tool we name the "\NewTerm{mixed product}".
	
	\textbf{Definition (\#\mydef):} We name "\NewTerm{mixed product}\index{mixed product}\label{mixed product}", also sometimes named  "\NewTerm{triple scalar product}\index{triple scalar product}", of vectors $\vec{x},\vec{y},\vec{z}$ the double product:
	 
	 often condensed under the following notation:
	  
	 From what we saw in the definition of the dot and cross product, mixed product can also be written:
	 
	 We note that in the case where $E$ is the Euclidean vector space $\mathbb{R}^3$, the absolute value of the mixed product symbolize the oriented volume of the parallelepiped, built on the representatives $\vec{x},\vec{y},\vec{z}$ of common origin.
	 
	It is quite trivial that the mixed product is an extension to the three-dimensional case of the cross product. Indeed, in the expression of the mixed product, the vector product is the base surface of the parallelepiped and the scalar product project the vectors on the resulting vector from the cross product which gives the height $h$ of the parallelepiped.
	
	\begin{figure}[H]
		\centering
		\includegraphics{img/algebra/mixed_product.jpg}
		\caption{Mixed product illustration}		
	\end{figure}
			
	If we develop:
	
	so triple product can also be understood as the determinant of a $3\times 3$ matrix (thus also its inverse) having the three vectors either as its rows or its columns (\SeeChapter{see section Linear Algebra page \pageref{3x3 matrix determinant}}):
	
	 By the commutative properties of the scalar product, we have:
	 
	and the reader will check without any trouble (we can write the details on request) that by developing components we have:
	
	The triple product has also the following properties that the reader could be able to check easily by developing just the components of each expression excepted perhaps the third one (we can detailed as always on request if needed):
	\begin{enumerate}
		\item[P1.]  $[\vec{x},\vec{y},\vec{z}] =
	   [\vec{z},\vec{x},\vec{y}] =
	   [\vec{y},\vec{z},\vec{x}] =
	  -[\vec{y},\vec{x},\vec{z}] =
	  -[\vec{z},\vec{y},\vec{x}] =
	  -[\vec{x},\vec{z},\vec{y}]$
	  
	  \item[P2.] $[\alpha\vec{x}+\beta\vec{y},\vec{z},\vec{v}] =
	  \alpha[\vec{x},\vec{z},\vec{v}] +
	    \beta[\vec{y},\vec{z},\vec{v}]$
	    
	  \item[P3.] $[\vec{x},\vec{y},\vec{z}] \ne 0$ if and only if $\vec{x},\vec{y},\vec{z}$ are independent.
	  
	  \item[P4.] $(\vec{x}\times\vec{y})\times(\vec{z}\times\vec{v}) =
	  [\vec{x},\vec{y},\vec{v}] \cdot \vec{z} - [\vec{x},\vec{y},\vec{z}] \cdot \vec{v}$ 
	\end{enumerate}
	\begin{tcolorbox}[title=Remark,colframe=black,arc=10pt]
	We will come back on the triple product during our study on tensor calculus as it gives the opportunity to get a very interesting result concerning a future application in General Relativity.	
	\end{tcolorbox}
	
	\pagebreak
	\subsection{Vectorial Functional Space}
	Given $\mathcal{C}_{[a,b]}^k$ the set of real functions that can be $k$-times derivates (\SeeChapter{see section Differential and Integral Calculus page \pageref{smoothness}}) in the closed bounded interval $[a,b]$. We will designate the elements of this set by the letters $\vec{f},\vec{g},...$.
	
	The value of $\vec{f}$ at the point $t$ will be obviously denoted by $\vec{f}(t)$. Say that $\vec{f}=\vec{g}$ is therefore equivalent than to say that:
	 
	 In a condensed way some practitioners denote this $\vec{f}(t) \equiv \vec{g}(t)$, the symbol $\equiv$ indicating obviously that the two members are equals for any $t$ in the bounded interval $[a,b]$.
	 
	 Consider the two following operations:
	\begin{itemize}
		\item $\vec{f}(t)+\vec{g}(t)$ defined by the relation $(\vec{f}+\vec{g})(t)\equiv \vec{f}(t)+\vec{g}(t)$
		\item $\alpha\vec{f}$ defined by the relation $(\alpha \vec{f})(t)=\alpha\vec{f}(t)$
	\end{itemize}
	These two operations satisfy to all conditions of the vectors of a vector space as we have already defined at the beginning of this section (associativity, commutativity, null vector, opposed vector, distributivity, neutral element) and therefore gives us the possibility to assign to $\mathcal{C}_{[a,b]}^k$ of a vector space structure! Le null vector of this space being obviously the null function (equal to zero everywhere) and the opposite of $\vec{f}$ being $-\vec{f}$.

	It is interesting to notice that $\mathcal{C}_{[a,b]}^k$  as a vector space is a generalization of $\mathbb{R}^n$ to the continuous case. Indeed, we can consider any vector $\vec{v}=(a_i)$ of $\mathbb{R}^n$ in the form of a real function defined on the set $\left\lbrace 1,2,...,n \right\rbrace
$: the value of this function at the point $i$ is simply $a_i$.
	
	\begin{tcolorbox}[title=Remark,colframe=black,arc=10pt]
	The polynomials of order $n$ with one unknown form as an example of functional vector space of dimension $n + 1$ such that for each coefficient of the polynomial corresponds a vector component such that:
		
	\end{tcolorbox}
	The preferred application field of the abstract theory of the dot product (inner product) is formed by the functional vector spaces. We name therefore "\NewTerm{canonical scalar product}\index{canonical dot product}" in $\mathcal{C}_{[a,b]}(\mathbb{R}^2)$ the operation defined by the relation:
	
	This operation defines indeed a scalar product, the properties of the latter being verified (on reader request we can add the proof if necessary), and furthermore, the integral:
	
	is positive if the continuous function $\vec{f}$ is not identically zero.
	
	Technically the latter relation is written when in $\mathbb{R}^2$:
	
	We will give more precision about this norm and its associated scalar product and with example in the section of Functional Analysis.
	
	\pagebreak
	\subsection{Hermitian Vector Space}
	The purpose of what will follow is, as always in this book, not to give a detailed study about vector spaces in $\mathbb{C}$ but just to give the minimal knowledge and vocabulary necessary to the lecture of some theoretical models in physics and especially those presented in this book in the section of Wave Quantum Physics.
	
	When the scalars that appears in the definition of vector spaces are complex numbers (in $\mathbb{C}$ as seen in the section Numbers), and not only real numbers, then we speak obviously about "\NewTerm{complex vectorial spaces}\index{complex vectorial spaces}".
	
	\begin{tcolorbox}[title=Remark,colframe=black,arc=10pt]
	Rigourlsy in the common communication, people should always precise if we speak of real vectorial space or complex vectorial space...
	\end{tcolorbox}
	Let us give some expamples of famous complex vectorial spaces (as many people think the are useless):

	\begin{tcolorbox}[colframe=black,colback=white,sharp corners]
	\textbf{{\Large \ding{45}}Examples:}\\\\
	E1. The space vector $\mathbb{C}^n$ of column-vectors with $n$ components ($\mathbb{C}^1$ being obviously identified to $\mathbb{C}$). We will meet, among others, such vector space in the section of Relativistic Quantum Physics.\\
	
	E2. The vectorial space of univariate polynomial with coefficient in $\mathbb{C}$. We will meet such spaces in the section of Wave Quantum Physics or even Quantum Chemistry.\\
	
	E3. The vectorial space of complex functions of one real or comple variables continuous or not. We will meet such vector spaces frequently in the section of Wave Mechanics and especially in the section of Electrodynamics.
	\end{tcolorbox}
	The purpose here is to adapt what we have seen so far to complex vectorial space. The following example, show us that we can transpose as it the previous definitions. Indeed, let us consider the vector space $\mathbb{C}^n$. As for $\mathbb{R}^n$, we could have the tentation to define a dot product on $\mathbb{C}^n$ by:
	
	with $x_i,y_i\in \mathbb{C}$.
	Sadly, we see that this definition is not satisfactory because we could have therefore:
	
	and this quantity is in general not a real number in a complex vector space and violates the property of positivity of the dot product and therefore prevent us to introduce and use the concept of distance. What is obviously a big problem in our actual perception of the world.
	
	We could therefore not define anymore a norm in $\mathbb{C}^n$ by writing:
	
	For $\langle \vec{x}|\vec{x}\rangle$ to be a positive real number we see that it would be better to define the scalar product like this:
	
	In this case we have therefore:
	
	which is well a positive real number. From there, we can once again define a norm for complex vector space $\mathbb{C}^n$ by putting:
	
	We will now show how to define an inner product on a complex vector space in the general case.
	
	\subsubsection{Hermitian Inner Product}\label{hermitian inner product}
	\textbf{Definition (\#\mydef):} Let $\mathcal{H}$ be a complex vector space (!). We name "\NewTerm{scalar product}\index{scalar product}" or more accurately "\NewTerm{Hermitian inner product}\index{Hermitian inner product}" on $\mathcal{H}$ (that is to say: a dot product in complex space...), an application:
	
	That satisfies (they are more properties but we will focus only about what we need for practical application in this book and especially quantum physics):
	\begin{enumerate}
		\item[P1.] Positivity:
		
	  	
		\item[P2.] Nullity:
		 
	  	
		\item[P3.] Hermitian symmetry:
		
	  	
	  	\item[P4.] Bilinearity (bilinear form) changes a little bit too ... so that we speak now of "\NewTerm{sesquilinearity}\index{sesquilinearity}". We speak then, in order, of left anti-linearity and of right linearity such as:
	  	
	\end{enumerate}
	\begin{tcolorbox}[title=Remarks,colframe=black,arc=10pt]
	\textbf{R1.} It seems that some mathematicians put the anti-linearity on the right. It's just a matter of agreement that does not matter and exists because of a lack of international norms in mathematics.\\
	
	\textbf{R2.} The reader may notice easily that if the elements of the above definitions are all in the set $\mathbb{R}$ then the sesquilinearity is reduced to the bilinearity and the hermitian character to a simple symmetry. So the hermitian inner product reduces to the scalar product.\\
	
	\textbf{R3.} We want to give for now only the minimum on the vast subject that is complex vector spaces so that the reader can read without too much trouble the beginning of the section of Wave Quantum Physics.
	\end{tcolorbox}	
	When we join to a complex vector space a scalar product then just as a real vector space becomes an Euclidean vector space or Prehilbertien vector space, the complex vector space becomes what we name a "\NewTerm{Hermitian vector space}\index{Hermitian vector space}" (term often used in the section of Wave Quantum Physics).
	
	\textbf{Definition (\#\mydef):} Again, we say that a space $\mathcal{H}$ provided with an Hermitian product $\langle \vec{x}|\lambda \vec{y} +\mu \vec{z}\rangle$ is a "\NewTerm{Hilbert space}\index{Hilbert space}" if this space is complete for the metric defined above.
	
	Thus, Hilbert spaces is a generalization of spaces including dot products and Hermitian dot product of Euclidean and Prehilbertien spaces.
	
	\subsubsection{Types of Vectors Spaces}
	To sum it all up:
	\begin{itemize}
		\item We name "pre-Hilbert space" (real or complex) any vector space of finite dimension or not, provided with a dot (scalar) product.
		
		\item We name "Hilbert space" (real or complex) any complete prehilbertian space (as space provide with a norm).
		
		\item We name "Euclidean space" any real vector space of finite dimension with a dot (scalar) product and denoted by $\mathcal{E}^n$.
		
		\item We name Hermitian space any complex vector space of finite dimension with a dot (scalar) product and denoted by $\mathcal{H}^n$.
	\end{itemize}
	
	\pagebreak
	\subsection{System of Coordinates}\label{system of coordinates}
	We will address here the aspect of coordinates changes of vector components not from a basis to another one (for that you need to go see the section of Linear Algebra) but from one coordinate system to another. That means that in any case we will stay in an Euclidean space. This type of transformation has strong implication in physical (and a little bit less in pure mathematics) when we want to simplify the study of physical systems whose equations become easier to handle in other coordinate systems.
	
	\textbf{Definition (\#\mydef):} In mathematics, a "\NewTerm{coordinate system}\index{coordinate system}" is used to match to each point of an $n$-dimensional space, a $m$-tuple of scalars.
	
	Although we are in a chapter and a section of this book that is suppose to be pure maths oriented..., we will allow ourselves in what follows to make a direct connection with physics relatively the terms of the speed and acceleration in different coordinate systems (sorry for the "math skills only" people...). Our teaching experience has show that this helps the readers (most of time students) to better understand the various abstract concepts.
	
	
	\subsubsection{Cartesian (rectangular) Coordinate System}
	We do not want to take too much time on this system as it is well known to everyone usually. However, let us recall that most of the time, in physics, the Cartesian systems in which we are working are in $\mathbb{R}^2 $(two real spatial dimensions), or $\mathbb{R}^3$ (three real spatial dimensions) or even $\mathbb{R}^4$ or $\mathbb{C}^4$ (three spatial dimensions and one of time) when we work in relativity. The number of dimensions can be higher as for example with the Kaluza-Klein theory (five dimensions) merging General Relativity and Electromagnetism or much more with String Theory (above 20 dimensions!!!).
	
	In $\mathbb{R}^3$ (the most common case), there are three basic vectors traditionally denoted by:
	
	Or more explicitly:
	
	
	In this system, the position of a point $P$ (identifiable by a vector $\vec{x}$ for example) is defined by the three numbers named  "\NewTerm{coordinates}\index{coordinates}" (more generally "\NewTerm{components}") denoted (typically in Tensor Calculus):
	
	and in physics denoted more conventionally by:
	
	where usually the component $(z)$ represents the height (vertical), the component $(x)$ is the width and the component  $(y)$ is the length (obviously these are completely arbitrary choice).
	
	This point $P$ can be spotted by a vector arbitrarily designated $\vec{r}$ in the basis  $\vec{e}_i$ by the relation (using Tensor notation):
	
	and if the basis is canonical (orthonormal) such that:
	
	we write:
	
	In physics, if we work with coordinates, it is always to be able to determine the location of an item. Or, as we shall see it more rigorously in the section of Analytical Mechanics, the physicist works with the following concepts (each element being often time-dependent):
	\begin{itemize}
		\item Positions: $\vec{r}=\biggl(x(t),y(t),z(t),t\biggr)$
		
		\item Velocity: $\displaystyle\frac{\mathrm{d}\vec{r}}{\mathrm{d}t}=\dot{\vec{r}}=\vec{v}
	=\biggl(\dot{x}(t),\dot{y}(t),\dot{z}(t),t\biggr)$
	
		\item Acceleration: $\displaystyle\frac{\mathrm{d}\vec{v}}{\mathrm{d}t}=\dot{\vec{v}}=\vec{a}=
	\biggl(\ddot{x}(t),\ddot{y}(t),\ddot{z}(t),t\biggr)$
	\end{itemize}
	Now let us see how the different concepts are expressed in systems such as spherical, cylindrical and polar coordinates (remember that we remains for all of them in a flat Euclidean space!!!).
	
	\pagebreak
	\subsubsection{Spherical Coordinate System}\label{spherical coordinates}
	The choice to start with this coordinate system is not a coincidence. It has the advantage of being a generalization of cylindrical and polar systems that we will meet thereafter and will help us easier to determine the expressions of position, velocity and acceleration\footnote{Two commonly-used sets of "\NewTerm{orthogonal curvilinear coordinates}\index{orthogonal curvilinear coordinates}" are cylindrical polar coordinates and spherical polar coordinates.}\label{orthogonal curvilinear coordinates}.
	
	We traditionally represent (in Switzerland ... and in accordance with the standard ISO 31-11) a spherical coordinate system as follows:
	\begin{figure}[H]
		\centering
		\includegraphics{img/algebra/coordinate_system_spherical.jpg}
		\caption{Representation of the spherical coordinate system}
	\end{figure}
	We see very clearly if we know the basic trigonometric relations and identities (see the section of the same name in the Geometry chapter) we have the transformations:
	
	
	where the two angles $\theta, \phi$ are respectively the latitude and colatitude (longitude):
	\begin{figure}[H]
		\centering
		\includegraphics[scale=0.7]{img/algebra/latitude_longitude.jpg}
		\caption[Latitude and Longitude concepts illustrated]{Latitude and Longitude concepts illustrated (source: OpenStax)}
	\end{figure}
	We have inversely:
	
	Now we must find the expressions that connect the vectors of the spherical basis that we choose to denote by $\vec{e}_r,\vec{e}_\theta,\vec{e}_\phi$ with the vectors of the Cartesian basis $\vec{e}_x,\vec{e}_y,\vec{e}_z$:
	
	Let us indicate that by dividing by $\sin(\theta)$ the second basis vector $\vec{e}_\phi$, we make sure that by the properties of the norm of the vector product that:
	
	will be well normalized to unity as expected (as we know from start that as we toke a direct orthogonal coordinate system the product of norms of the basis vectors must be equal to $1$)!
	
	We will also use later (for the study of vector operators further below and the geodesic of the sphere in the section of Analytical Mechanics) the variation $\mathrm{d}\vec{r}$ expressed in spherical coordinates:
	
	Or more explicitly:
	
	
	All this can be put in a famous matrix form as (this also applies automatically to the component of the unitary vector):
	
	To express the velocity and acceleration in spherical coordinates, we will also need the derivatives with respect to time:
	
	So if we do now a little bit of physics, we have:
	
	This brings us to (we will need this relation mainly in the chapter of Astrophysics):
	
	It is interesting that we get the same result through the following method that may be less intuitive:
	
	and substituting the derivative obtained above:
	
	Concerning the acceleration we get:
	
	But we have:
	
	Therefore it comes:
	

	Thus finally:
	
	
	\paragraph{Orthodromic distance}\mbox{}\\\\
	We will now see a powerful and simple application of vector calculus (especially of the dot procuct with spherical coordinates!). The reader will see later in the section of Analytical Mechanics that the geodesic of the sphere is the equation of the great circles (circles on the sphere whose centers coincide with the center of the sphere).

	But without knowing this result, and even without needing it, we want to calculate the "\NewTerm{great-circle distance}\index{great-circle distance}" or "\NewTerm{orthodromic distance}\index{orthodromic distance}" that is the shortest distance between two points on the surface of a sphere, measured along the surface of the sphere:
	\begin{figure}[H]
		\centering
		\includegraphics{img/algebra/orthodromic_distance.jpg}
		\caption{Orthodromic distance (great-circle distance)}
	\end{figure}
	Through any two points on a sphere that are not directly opposite each other, there is a unique great circle. The two points separate the great circle into two arcs. The length of the shorter arc is the great-circle distance between the points. A great circle endowed with such a distance is named a "\NewTerm{Riemannian circle}\index{Riemannian circle}" in Riemannian geometry.
	
	Let us consider two points on the surface of a sphere $P_1$ and $P_2$ and their respective latitude and longtitude such that they are defined by:
	
	Therefore in spherical coordinates:
	
	Hence taking the vectors having for origin the center of the sphere:
	
	The Earth is nearly spherical, so great-circle distance formulas give the distance between points on the surface of the Earth (as the crow flies) correct to within $0.5\%$ or so.
	
	We do now the dot product:
	
	In function of the angle theta we know that the dot product has also for expression:
	
	In function of the an angle that we will denote $\alpha$ we know that the dot product has also for expression:
	
	By identification of that latter relation with prior-previous one we get:
	
	Hence after simplification:
	
	As we know that the circonference of circle is given by $P=2\pi R$ then the arc lengths by unit of angle is given by:
	
	So finally:
	
	\begin{tcolorbox}[colframe=black,colback=white,sharp corners]
	\textbf{{\Large \ding{45}}Example:}\\\\
	Consider $P_1$ as being the Kourou in French Guiana $(\theta_1,\phi_1)=(+5^\circ.23,52^\circ.75)$ and the second point $P_2$ as being Toulouse in France $(\theta_2,\phi_2)=(43^\circ.63,-1^\circ.37)$. Then we get $\cos(\alpha)=0.485348$ and therefore $l=6,786$ [km].
	\end{tcolorbox}

	\pagebreak
	\subsubsection{Cylindrical Coordinate System}\label{cylindrical coordinates}
	The cylindrical coordinate system (very useful in the study of helical motion systems) is quite similar to spherical coordinates as it can be seen as a slice of the sphere. 

	Given the figure:
	\begin{figure}[H]
		\centering
		\includegraphics{img/algebra/coordinate_system_cylindrical.jpg}
		\caption{Representation of the cylindrical coordinate system}
	\end{figure}
	Warning!!! The vector $\vec{r}$ is unlike the previous spherical case defined only in the $XY$ plane or a plane which is parallel to it!
	
	It comes easily in cylindrical coordinates for $r>0$:
	
	and vice versa:
	 
	Now we must find the expressions that connect the vectors of the cylindrical base that we choose to denote by $\vec{e}_r,\vec{e}_\phi,\vec{e}_z$ (instead of $\vec{r}, \vec{\phi},\vec{z}$ as it is done traditionally) with the vectors of the Cartesian base $\vec{e}_x,\vec{e}_y,\vec{e}_z$. We have identically to what we did for the spherical coordinates:
	
	Or more explicitly:
	
	All this can be put in a famous matrix form as (this also applies automatically to the component of the unitary vector):
	
	Let us indicate that by dividing by $\sin(\phi)$ the second vector base $\vec{e}_\phi$, we ensure us that by the properties of the norm of the cross product we have:
	
	will be well normalized to unity as expected (as we know from start that as we toke a direct orthogonal coordinate system the product of norms of the basis vectors must be equal to $1$)!! In the case of cylindrical coordinates the angle being anyway right, we would not be obliged to indicate this division, but we have made this choice for consistency with previous developments...
	
	\begin{tcolorbox}[title=Remark,colframe=black,arc=10pt]
	It is important to notice that the cross product of two basis vectors always gives the third perpendicular  basis vector (like the Cartesian and spherical coordinates so!).
	\end{tcolorbox}
	For future needs, let us determine the partial differential of each of these coordinates:
	
	We will also use later (for the study of vector operators) the variation $\mathrm{d}\vec{r}$ expressed in cylindrical coordinates:
	
	To express the speed and acceleration in cylindrical coordinates, we will also need the derivatives with respect to time:
	
	So if we now do a little bit physics, we get (let us recall that the $z$ component is independent of other cylindrical components):
	
	which brings us to:
	
	For acceleration we get (exactly the same approach as for the expression of the speed):
	
	
	\subsubsection{Polar Coordinate System}\label{polar coordinates}
	The polar coordinate system is very similar to the cylindrical coordinates as it can be seen as an entrenchment of one dimension (the height) of the cylindrical system (we will often encounter this system in the section of Classical Mechanics, Corpuscular Quantum Physics and Astronomy).

	Given the figure:
	\begin{figure}[H]
		\centering
		\includegraphics{img/algebra/coordinate_system_polar.jpg}
		\caption{Representation of the polar coordinate system}
	\end{figure}
	Thus, it comes easily in polar coordinates for $r>0$:
	
	and vice versa:
	
	\begin{tcolorbox}[title=Remark,colframe=black,arc=10pt]
	In physics we frequently use the previous parametric equation of the circle in the following way involving time $t$ and pulsation (as a constant) $\omega$\label{vector parametric circle equation}
	
	Hence:
	
	\end{tcolorbox}
	Now we must find the expressions that connect the vectors of the polar base that we choose to denote by $\vec{e}_r,\vec{e}_\phi$ (instead of $\vec{r}, \vec{\phi}$ as it is done traditionally) with the vectors of the Cartesian base $\vec{e}_x,\vec{e}_y$. We have identically to what we did for the spherical coordinates:
	
	Or more explicitly:
	
	Once again, dividing by $\sin(\theta)$ the second basis vector $\vec{e}_\phi$, we ensure the properties of the norm of the vector product that:
	
	will be well normalized to unity (as we know from start that as we toke a direct orthogonal coordinate system the product of norms of the basis vectors must be equal to $1$)!. In the case of polar coordinate the angle being a anyway right, we would not be obliged to indicate this division, but we have made this choice for consistency with the previous developments.
	
	For future needs, Let us determine the partial differential of each of these coordinates:
	
	We will also use later (for the study of vector operators) the variation $\mathrm{r}$ expressed in polar coordinates:
	
	To express the speed and acceleration in polar coordinates, we will also need the derivatives with respect to time:
	
	So if we now do a little bit physics, we have:
	
	and therefore:
	
	where the first term is the radial velocity component and the second term the tangential component of the (angular) velocity. The velocity expression in polar coordinates is very important in astronomy as it allows quite easily calculate the calculation of the kinetic energy:
	
	For the acceleration we get:	
	
	where the first term is the radial acceleration, the second term the centripetal acceleration, the third the Coriolis acceleration and finally the fourth one the tangential acceleration.
	
	\pagebreak
	\subsection{Differential Operators}\label{differential operators}
	\textbf{Definition (\#\mydef):} Define a scalar field, vector field or tensor field in a volume $V$, it is define an application that for any point $\vec{x}$ of this volume $V$ associates respectively a scalar, a vector or a tensor.
	
	Thus, the application $f$ that at any point $\vec{x}$ of $V$ of spatial coordinates $(x, y, z)$ associates the scalar value $f(\vec{x})=f(x,y,z)$ is a scalar field in $V$.
	
	At each point of a volume traversed by a moving fluid, the vector that coincides at every moment with the speed of the particle which pass through this point at this same time defines a 3D vector field, optionally variable in time. The fields thus defined are a basic mathematical tool in physics.
	\begin{tcolorbox}[title=Remark,colframe=black,arc=10pt]
	When we plot a scalar field, the set of continuous dots of equal value is so-named "\NewTerm{isolines}\index{isolines}" or more commonly "\NewTerm{contours}\index{contours}".
	\end{tcolorbox}
	Vector fields are especially an important tool for describing many physical concepts, such as gravitation and electromagnetism, which affect the behavior of objects over a large region of a plane or of space. They are also useful for dealing with large-scale behavior such as atmospheric storms or deep-sea ocean currents.

	For example the figure below shows a gravitational field exerted by two astronomical objects, such as a star and a planet or a planet and a moon. At any point in the figure, the vector associated with a point gives the net gravitational force exerted by the two objects on an object of unit mass. The vectors of largest magnitude in the figure are the vectors closest to the larger object. The larger object has greater mass, so it exerts a gravitational force of greater magnitude than the smaller object.
	\begin{figure}[H]
		\centering
		\includegraphics[scale=0.5]{img/algebra/vector_field_gravitation.jpg}
		\caption[Gravitation vector field example]{Gravitation vector field example (source: OpenStax)}
	\end{figure}
	Another example is the figure below that shows the velocity of a river at points on its surface. The vector associated with a given point on the river's surface gives the velocity of the water at that point. Since the vectors to the left of the figure are small in magnitude, the water is flowing slowly on that part of the surface. As the water moves from left to right, it encounters some rapids around a rock. The speed of the water increases, and a whirlpool occurs in part of the rapids.
	\begin{figure}[H]
		\centering
		\includegraphics[scale=0.5]{img/algebra/vector_field_speed.jpg}
		\caption[Velocity vector field example]{Velocity vector field example (source: OpenStax)}
	\end{figure}
	The gradient, divergence and the rotational are the three main linear differential operators of the first order that will be introduced here. This means they only involve the partial first derivatives (or simply "differentials") of the fields, unlike, for example, the Laplace operator which involves partial derivatives of order $2$.	
	
	\pagebreak
	\subsubsection{Gradients of Scalar Field}\label{gradient scalar field}
	The gradient is an operator that applies to a scalar field and transforms it into a vector field. Intuitively, the gradient indicates the direction of the greater variation of the scalar field, and the intensity of this variation. For example, the gradient of the altitude is directed along the maximum slope line and its norm increases with the slope.
	
	Given a three-dimensional scalar field $f(x,y,z)$, wherein x and y and z are the Cartesian coordinates of a point $M$ in space. When $M$ moves in space according to the $\mathrm{d}\vec{r}$ of components $\mathrm{d}x, \mathrm{d}y$ and $\mathrm{d}z$, the scalar field $f$ varies according to the total differential $\mathrm{d}f$:
	
	From this relation, we can define the "\NewTerm{gradient operator}\index{gradient operator}" of a scalar field such as:
	
	where:
	
	is a vector operator named "\NewTerm{gradient of the scalar field $f$}\index{gradient of a scalar field}". To condense the writing, we sometimes use the symbol $\vec{\nabla}$ named the "\NewTerm{nabla of the scalar field $f$}\index{nabla of a scalar field }".
	
	The vector obtained by the gradient calculation has the following three fundamental properties:
	\begin{enumerate}
		\item[P1.] The components of the gradient represent by construction the variation (slope) of the function $f$ in the different directions of space.
		
		\item[P2.] The gradient is perpendicular to the isolines of the function $f$.

		\item[P3.] The direction of the gradient (and therefore its norm) is the maximum variation of $f$.

		\item[P4.] The direction of the gradient shows the values where $f$ increases.
	\end{enumerate}
	\label{gradient normal}Following the request of some readers let us prove some of these properties.
	
	Given $t\mapsto C(t)$ an isoline. Then $t\mapsto f(c(t))=c^{te}$ and therefore:
	  
	which proves that the gradient is orthogonal to the tangent of the isoline (property P2).
	
	Let us come back on:
	
	That it is tradition to write as:
	
	named the "\NewTerm{directional derivative}\index{directional derivative}". Its value is maximum obviously if $\theta=0$, that is to say that the gradient is colinear to the variation $\mathrm{d}\vec{r}$. Hence, the direction (and therefore the norm) of greatest increase of $f$, often named direction of "\NewTerm{deepest ascent}\index{deepest ascent}\label{eepest ascent}", is the same direction as the gradient vector!! Thus we proved the property P3.
	
	Obviously the directional derivative takes on its greatest negative value if $\theta=\pi$ (or $180$ degrees). Hence, the direction of greatest decrease of $f$ is the direction opposite to the gradient vector and often named direction of "\NewTerm{deepest descent}\index{deepest descent}\label{eepest descent}".
	\begin{figure}[H]
		\centering
		\includegraphics[width=1.0\textwidth]{img/algebra/gradient_deepest_descent_deepest_ascent.jpg}
		\caption{Gradient deepest ascent and deepest descent}
	\end{figure}
	If $\theta=\pi/2$ then there is no change in altitude in this direction and we are on an "\NewTerm{isocline}\index{isocline}".
	
	The property P4 can be explained without formalism as following:
	\begin{itemize}
		\item If the function is decreasing in one variable, then the partial derivative is negative, so the component vector of the gradient for that variable points in the negative direction - which means increasing function value.

		\item If the function is increasing in one variable, then the partial derivative is positive, so the component vector of the gradient for that variable points in the positive direction - which means increasing function value.
	\end{itemize}
	Then is doesn't matter how the function profile is, the gradient, by definition, points in the increasing direction. Indeed, when 	
$f(x,y)$ is decreasing in $x$, the function decreases as you move forward in $x$. But because the partial derivative with respect to $x$ is negative, the $x$-component of the gradient points towards origin (move backward in $x$), that is to say in the direction which makes f to increase.
	
	From the definition and from the total differential, we get
	
	This leads us to put that:
	
	and so that finally the operator of the "\NewTerm{gradient in Cartesian coordinates}\index{gradient in Cartesian coordinates}" is given by:
	
	Finally we see that the gradient of a scalar field $f(x,y,z)$ is the vector field whose components at each point are the three derivatives of the scalar field $f$ with respect to the three-dimensional coordinates, denoted here by $x, y, z$ and that by its direction, and its norm, the gradient vector of a scalar field at a point therefore includes indications on how the field varies around this point.
	
	\begin{tcolorbox}[title=Remark,colframe=black,arc=10pt]
	One of the necessary and sufficient conditions for a vector field to be the gradient of a scalar field $f$ is that this vector field is irrotational (see below the rotational operator of a vector field).
	\end{tcolorbox}
	\begin{tcolorbox}[colframe=black,colback=white,sharp corners]
	\textbf{{\Large \ding{45}}Example:}\\\\
	Let us find the direction for which the directional of:
	
	at $(-2,3)$ is a maximum and what is its maximum value?
	\begin{figure}[H]
		\centering
		\includegraphics{img/algebra/directional_derivative_search_plot_maple.jpg}
		\caption[]{Maple plot of $f(x,y,)=3x^2-4xy+2y^2$}
	\end{figure}
	The maximum value of the directiona derivative occurs as we have just proved it when $\vec{\nabla}f$ and the unit vector $\mathrm{d}\vec{r}$ point in the same direction.\\

	Therefore we start by calculating $\vec{\nabla}f(x,y)$:
	
	Next we evaluate the gradient at $(-2,3)$:
	
	\end{tcolorbox}
	\begin{tcolorbox}[colframe=black,colback=white,sharp corners]
	We need to find a unit vector that points in the same direction as $\vec{\nabla}f(-2,3)$, so the next step is to divide $\vec{\nabla}f(-2,3)$ by its norm (the value of that norm being also the maximum value of the directional derivative at point $(-2,3)$), which gives:
	
	As the unit vector above is a vector in the plane, nothing avoid us to calculate the angle it does with the axis. By applying elementary trigonometry, we get if we denote this angle by $\theta$:
	
	Since cosine is negative and sine is positive, the angle must be in the second quadrant. Therefore:
	
	that is to say approximately $114.5916$ degrees (or seen from the point of view of the first quadrant: $39.792$ degrees).\\
	
	With Maple 4.00b we can have a more detailed and general investigation of the calculation we just did:\\

	\texttt{>with(plots):\\
	>with(linalg):\\
	>Pa:=contourplot(3*x\string^3-4*x*y+2*y\string^2,x=-3..-1,y=2..4,filled=true):\\
	>Pb:=fieldplot(grad(3*x\string^3-4*x*y+2*y\string^2,vector([x,y])),x=-3..-1,y=2..4):\\
	>display(Pa,Pb);
	}
	\begin{figure}[H]
		\centering
		\includegraphics[scale=0.9]{img/algebra/directional_derivative_gradient_plot_maple.jpg}
	\end{figure}
	\end{tcolorbox}
	After having defined the gradient in Cartesian coordinates $x, y, z$ we have to address the expression of this operator in other coordinate systems. It is common in physics to have to use cylindrical, polar and spherical coordinates to simplify the formal study of physical systems. So if we refer to the previous study of coordinate systems, we have (recall) first in polar coordinates:
	
	But, with the definition of gradient in Cartesian coordinates, in polar coordinates we have the following definition:
	
	If we express the total exact differential (\SeeChapter{see section of Differential and Integral Calculus page \pageref{total exact differential}}) of $\mathrm{d}f$ we obtain the following relation:
	
	This allows us to get the relation:
	
	therefore:
	
	which bring us to:
	
	Thus the "\NewTerm{gradient in polar coordinates}\index{gradient in polar coordinates}\label{gradient in polar coordinates}" is expressed as
	
	Let us now tackle the expression of the gradient in cylindrical coordinates. Let us recall that  during our study of different coordinate systems we obtained for cylindrical coordinates:
	
	So we already know that the expression of the gradient in cylindrical coordinates will be the same in polar coordinates with the exception of the addition of the vertical $z$ component that is independent of other coordinates. Thus we get the "\NewTerm{gradient in cylindrical coordinates}\index{gradient in cylindrical coordinates}":
	 
	Let us now tackle on the expression of the gradient in spherical coordinates. Let us recall that during our study of the different coordinate systems we obtained for the spherical coordinates:
	
	But, with the definition of gradient in Cartesian coordinates, we have in spherical coordinates the following definition:
	
	If we express the total differential of $\mathrm{d}f$ we get the following relations:
	
	This allows us to obtain the relation (we now use the notation that uses the operator "nabla"):
	
	The relation:
	
	requires that:
	
	Thus the "\NewTerm{gradient in spherical coordinates}\index{gradient in spherical coordinates}" is expressed as:
	 
	So we finally saw all the expressions of the gradient operator in the Cartesian, polar, cylindrical and spherical systems.
	
	\begin{tcolorbox}[colframe=black,colback=white,sharp corners]
	\textbf{{\Large \ding{45}}Example:}\\\\
	Let us see now a visual example of the previous developements with Maple 4.00 and a special case function $f(x,y)=\sin(x)\sin(y)$.\\
	
	\texttt{> with(linalg):\\
	> with(plots):\\
	> plot3d(sin(x)*sin(y),x=-3..3,y=-3..3,axes=framed);}\\
	\begin{figure}[H]
		\centering
		\includegraphics{img/algebra/gradient_function_of_example.jpg}
		\caption[]{Plot of the function taken as example}
	\end{figure}
	And now we show the isolines:\\
	
	\texttt{>contourplot3d(sin(x)*sin(y),x=-3..3,y=-3..3,filled=true,\\
	axes=framed,coloring=[red,blue],style=patch);}
	\begin{figure}[H]
		\centering
		\includegraphics{img/algebra/gradient_function_with_isolines.jpg}
		\caption[]{Function with is isolines}
	\end{figure}
	\end{tcolorbox}
	
	\pagebreak
	\begin{tcolorbox}[colframe=black,colback=white,sharp corners]

	And now we plot the gradient vector with a plane projection and we see they are indeed perpendicular to the isolines:\\\\
		\texttt{>Pa:=contourplot(sin(x)*sin(y),x=-3..3,y=-3..3,contours=10,\\
		coloring=[red,blue],filled=true):
\\
	>Pb:=fieldplot(grad(sin(x)*sin(y),vector([x,y])),x=-3..3,y=-3..3,\\
	arrows=THICK):\\
	>display(Pa,Pb);}
	\begin{figure}[H]
		\centering
		\includegraphics{img/algebra/gradient_function_with_isolines_and_projected_gradient.jpg}
		\caption[]{Function with its isolines and projected gradient}
	\end{figure}
	And with a 3D perpsective:\\
	
	\texttt{>campo:=fieldplot3d([diff(sin(x)*sin(y),x),diff(sin(x)*sin(y),\\
	y),0],x=-3..3,y=-3..3,
z=-3..3,axes=framed,arrows=THICK);\\
>superf:=plot3d(sin(x)*sin(y),x=-3..3,y=-3..3):
\\
	> display({campo,superf});}
	\begin{figure}[H]
		\centering
		\includegraphics[scale=0.8]{img/algebra/gradient_function_with_isolines_and_gradient.jpg}
		\caption[]{Function with its isolines and gradient in 3D}
	\end{figure}
	\end{tcolorbox}

	\pagebreak
	\begin{tcolorbox}[colframe=black,colback=white,sharp corners]
	Seen from above with a small rotation:\\
	\begin{figure}[H]
		\centering
		\includegraphics{img/algebra/gradient_function_with_isolines_and_projected_gradient_rotation.jpg}
		\caption[]{Function rotation with its isolines and gradient}
	\end{figure}
	\end{tcolorbox}
	
	\subsubsection{Gradients of Vector Field}\label{gradient of vector field}
	The gradient of a vector field $\vec{f}(x,y,z): \mathbb{R}^3\mapsto\mathbb{R}^3$ is the field named "\NewTerm{tensor field}\index{tensor field}" defined by the following nine relations in Cartesian coordinates:
	 
	 We will use such a gradient in our study in the section of Marine \& Weather Engineering of the Papillon effect whose origin comes from the determination of the Navier-Stokes equations of the section of Continuum Mechanics and we will also use this type of gradient in the section of Theoretical Computing in our study of the Gauss-Newton optimization method.
	
	We have the following 4 components in polar coordinates:
	 
	We have the following 9 components in cylindrical coordinates:
	 
	We have the following 9 components in spherical coordinates:
	 
	 So we finally saw all the expressions of a gradient vector field in Cartesian , polar, cylindrical and spherical system.
	 
	\subsubsection{Divergences of a Vector Field}\label{divergence vector field}
	The diverence is applied to a vector field and turns it into a scalar field. Therefore it is an application from $\mathbb{R}^3\mapsto \mathbb{R}$. Intuitively, and in the most common case, the divergence of a vector field expressed its tendency to come from or converge to some points.
	\begin{tcolorbox}[title=Remark,colframe=black,arc=10pt]
	Non-initated people often confuse the gradient and divergence operator. To make the difference we must remember that the divergence of a vector is a number and that the gradient is a vector! The gradient indicates the direction in which the change is the most important and its amplitude. The divergence simply say what comes in or out from a given point.
	\end{tcolorbox}	
	However, we must distinguish two contributions to the divergence that we will rigorously define a little further below: one due to the variations named the "\NewTerm{directional divergence}\index{directional divergence}" and the other due to variations in modules (norm) named the "\NewTerm{modular divergence}\index{modular divergence}". Thus, for simple fields, we can imagine cases where the divergence would only be modular and others, where it would only be directional. We could also build a field where the two types of divergence coexist, but having adverse effects.
	
	Let us consider for example a vector $\vec{f}$ of space and we make it pass through any surface $S$. Physicists then assimilate the quantity $\vec{f}$ which moves along the normal vector to the surface as a flow of $\vec{f}$ through $S$.
	
	To be convinced of this analogy we can imagine a fluid flowing on a flat surface, the flow through the surface is obviously zero in this cas, by cons if the fluid flows vertically through a horizontal surface the flow will be maximal. It is then immediate that we want to represent the flow by the scalar product of $\vec{f}$ with the normal $\vec{n}$ to  the surface $S$.
	\begin{tcolorbox}[title=Remark,colframe=black,arc=10pt]
	We must always pay attention to the direction of $¨\vec{n}$ because at any point of a surface $S$ we have in general two normal vectors $\vec{n}$ that are colinear but of opposite directions.
	\end{tcolorbox}	
	If the surface is planar  then the normal $\vec{n}$ is the same everywhere, but if it changes from place to place, then we will look at a small surface element $\mathrm{d}S$.
	
	If a small flow element is defined by:
	
	then the total flow will be given by:
	
	which is sometimes written (it's a little bit abusive but why not ...)
	
	Let us now suppose that our vector $\vec{f}$ moves a point $M(x,y,z)$ in space to  $M'(x+\mathrm{d}x,y+\mathrm{d}y,z+\mathrm{d}z)$ through a rectangular parallelepiped of sides of $\mathrm{d}x, \mathrm{d}y$ and $\mathrm{d}z$:
	
	\begin{figure}[H]
		\centering
		\includegraphics{img/algebra/ostrogradsky_box_vector_displacement.jpg}
		\caption[]{Move of a vector through a parallelepiped}
	\end{figure}
	We can decompose the movement (flow) through each face of the parallelepiped (decompositions in the orthonormal basis). For example, if we are interested at the decomposed part of the flow through the face $(\mathrm{d}y, \mathrm{d}z)$ described by the peaks vertices $BCFG$ we have obviously $\vec{n}=(1,0,0)$.
	
	We still need to determined how to represent the flow $\vec{f}$ for this direction. As the flow is a function, that is to say that each of its components may be dependent of the three components of the space (if we take the case of a function in $\mathbb{R}^3$) we have:
	
	\begin{tcolorbox}[title=Remark,colframe=black,arc=10pt]
	Those who are not convinced can go read the beginning of section Electrodynamics where we take the electric field as an (excellent) example.
	\end{tcolorbox}	
	While the variation of the flow according to $x$ is given by:
	
	which give us:
	
	therefore by summing:
	
	Compared to the first expression of $\Phi$, the term $\mathrm{d}x\mathrm{d}y\mathrm{d}z$ is then a volume element and not more of surface. We also have an interesting result:
	
	whose more explicit and rigorous writing  should be (to highlight well that the considered closed surface is the boundary of the closed studied volume):
	
	or more commonly written:
	
	\begin{tcolorbox}[title=Remark,colframe=black,arc=10pt]
	See the practical examples in the section Electrodynamics where for example the electric field divergence is zero for a free spherical charge as the vectors point in different directions (directional divergence) and where the norms decrease as the inverse of the square of the radius (modular convergence). Both contributions are in opposition and so the total divergence is zero.
	\end{tcolorbox}
	The development above is named "\NewTerm{Ostrogradsky theorem}\index{Ostrogradsky theorem}" or "\NewTerm{Gauss-Ostrogradsky theorem}\index{Gauss-Ostrogradsky theorem}\label{gauss ostrogradsky theorem}" or more simply "\NewTerm{divergence theorem}\index{divergence theorem}" and actually defines the total divergence of $\vec{f}$ in a volume as the flow $\vec{f}$ through the "walls" of the closed volume (Gauss closed surface), which expresses well the name "divergence".
	
	Now reconsider the previous relation but extracting and unitary vector from the vector field $\vec{f}$ such that:
	
	where now $f$ is a scalar field. This can be rewritten obviously using chain rule derivatives an dot product commutativity:
	
	In the special case of a uniform field we have:
	
	Then it remains:
	
	The dot product being distributive on the sum of vectors we can rewrite this as:
	
	and therefore we get the "\NewTerm{gradient theorem\footnote{As it is applicable only for uniform field it is not used a lot in practice}}\index{gradient theorem}":
	
	
	We define the operator "\NewTerm{divergence}\index{divergence operator}" by the following relation (the tensor notation has been used to shorten the writing) in an $n$-dimensional space:
	
	Thus we have for the operator "\NewTerm{divergence in Cartesian coordinates}\index{divergence in Cartesian coordinates}":
	
	If the divergence of a vector field is identically zero in all the points of an Eulerian frame \footnote{The Eulerian specification of the flow field is a way of looking at fluid motion that focuses on specific locations in the space through which the fluid flows as time passes.[1][2] This can be visualized by sitting on the bank of a river and watching the water pass the fixed location.}, the triple integral flux of this field through a volume $V$ will be:
	
	It follows that the flow of this vector field through the edges of the volume is zero, that is to say that the incoming flow compensates the output flow. We say that such a field vectors having a null divergence has a "\NewTerm{conservation flow}\index{conservation flow}".
	
	To determine the expression of the divergence operator in polar coordinates let us recall the relations proved earlier above:
	
	Given now a vectorial function $\vec{f}:\mathbb{R}^2\rightarrow \mathbb{R}^2$. We have:
	
	Knowing the expression of $\vec{e}_r,\vec{e}_\phi$ depending on $\vec{e}_x,\vec{e}_y$, from the expression above we deduce:
	
	The divergence of $\vec{f}$ being defined in the two dimensional case by:
	
	we then have:
	
	The first term is (application of the gradient in polar coordinates!):
	
	in the same way we get for the second term (we can as always give more details on request):
	
	By adding the two terms and expressing the partial derivatives of the functions $f_x,f_y$ in function of the partial derivatives of the functions $f_r,f_\phi$ using the relations:
	
	We get:
	
	After simplification:
	
	The expression of the operator "\NewTerm{divergence in polar coordinates}\index{divergence in polar coordinates}" is then:
	
	To determine the expression of the divergence operator in cylindrical coordinates let us recall the relations:
	
	Given now a vector function $\vec{f}:\mathbb{R}^3\rightarrow \mathbb{R}^3$. We have:
	
	As we know the expressions of $\vec{e}_r,\vec{e}_\phi,\vec{e}_z$ in function of the $\vec{e}_x,\vec{e}_y,\vec{e}_z$, from the above expressions we deduce:
	
	The divergence of $\vec{f}$ being defined in the three dimensional case by:
	
	we then have:
	
	The first term is equal to (application of the gradient in cylindrical coordinates):
	
	in the same way we get for the second component (we can give the details on request):
	
	and for the last one:
	
	By adding the three terms and expressing the partial derivatives of the functions $f_x,f_y,f_z$ in function of the partial derivatives of the functions $f_r,f_\phi,f_z$ using the relations:
	
	we get:
	
	After simplification:
	
	The expression of the operator "\NewTerm{divergence in cylindrical coordinates}\index{divergence in cylindrical coordinates}" is then:
	
	To find the expression of the divergence in spherical coordinates, let us recall the relations:
	
	Given now a vector function $\vec{f}:\mathbb{R}^3\rightarrow \mathbb{R}^3$. We have:
	
	Knowing the expression of $\vec{e}_r,\vec{e}_\theta,\vec{e}_\phi$ depending on $\vec{e}_x,\vec{e}_y,\vec{e}_z$, from the expression above we deduce:
	
	The divergence of $\vec{f}$ being defined in the three dimensional case by:
	
	we then have:
	
	The first component is equal to (application of the gradient in spherical coordinates):
	
	in the same way we get for the second component (we can give the details on request):
	
	and finally for the third and last one:
	and:
	
	By adding the three terms and expressing the partial derivatives of the functions $f_x,f_y,f_z$ in function of the partial derivatives of the functions $f_r,f_\theta,f_\phi$ using the relations:
	
	we get (we can develop the intermediate details on request):
	
	We notice that we can regroup terms depending on the same variable using the property of the derivative, so we get for the expression of the divergence in spherical coordinates:
	
	and therefore the operator of "\NewTerm{divergence in spherical coordinates}\index{divergence in spherical coordinates}" is:
	
	So we finally we saw all the expressions of the divergence operator of a vector field in Cartesian, polar, cylindrical and spherical systems.
	
	\pagebreak
	\subsubsection{Rotationals of a Vector Field (Curl)}\label{rotational}
	The "\NewTerm{curl}\index{curl}" or "\NewTerm{rotationnal}\index{rotational}" of a vector field can be seen (this is a simplification!) as the vector field whose field lines are perpendicular to those we have calculated the rotational as shown in the special example below (we will see more academic detailed further below):
	\begin{figure}[H]
	\centering
		\includegraphics{img/algebra/curl.jpg}
		\caption{Example of rotational of a vector field}
	\end{figure}
	In a little bit more technical way the rotational is a vector operator that describes the infinitesimal rotation of a $3$-dimensional vector field. At every point in the field, the rotational of that point is represented by a vector. The attributes of this vector (length and direction) characterize the rotation at that point. he direction of the rotational is the axis of rotation, as determined by the right-hand rule, and the magnitude of the rotational is the magnitude of rotation. 
	
	The rotational transforms a vector field in another vector field. For most people it is more difficult to accurately represent than the gradient and divergence, it intuitively reflects the tendency of a field to rotate around a point (the way it is twisted).
	
	Let us give before tackling with the mathematical stuff and also mathematical examples two every-day life examples:
	\begin{tcolorbox}[colframe=black,colback=white,sharp corners]
	\textbf{{\Large \ding{45}}Examples:}\\\\
	E1. In a tornado, the wind turns around the eye of the storm and the wind velocity vector field has a non-zero rotational around the eye.\\

	E2. The rotational of the velocity field of a disc that rotates at a constant speed is constant, directed along the axis of rotation and oriented such that the rotation takes place, in relation to it, in the direct sense.\\
	\end{tcolorbox}
	A vector field is said to by "\NewTerm{irrotational}\index{irrotational}\label{irrotational}" when the rotational of this field is identically zero at all points of space. Otherwise, we say it is a "\NewTerm{vortex}\index{vortex}".
	
	In the usual case where $\mathrm{d}x$ is an element of length, the measurement unit of the rotational is then the unit of the considerated field divided by a unit length. For example, in fluid mechanics: the unity of the rotational of a velocity field is radians per unit time, as an angular velocity as we divide a velocity ([ms$^{-1}$] by the length [m]!
	
	The divergence gives some indication of the behavior of a vector or a vector field: how it moves in relation to the normal and how it crosses the surface, but it is insufficient. Take a field which would have the shape of a cylinder and another field which have a helicoidal form of the same diameter as the cylinder. If the move  in the same direction the divergence will be the same even if the movements are quite different This requires that we determine how the field is bent as it passes through a surface: this will be determined by the circulation (as the work of a force, for example) of the vector along a closed curve, obtained with the sum of dot products $\vec{f}\circ \mathrm{d}\vec{r}$ on the closed contour (\SeeChapter{see section Differential and Integral Calculus page \pageref{curvilinear integral}}):
	
	in fact it's the same to look at how twisted is the vector with respect to the normal vector of the surface which leads us to define the "rotational" or "swirl vector" by writing:
	
	that thus establishes a relation between the line integral and the surface integral (we then transform a line integral on a closed path in a surface integral delimited by the given path).
	
	\begin{theorem}
	In other words, the rotational is calculated by using the fact that the flow around a closed basic path of a vector field is equal to the flux of its rotational through the immediate elementary surface generated by this path.

	This is the "\NewTerm{Stokes' theorem}\index{Stokes' theorem}\label{stokes theorem}" (which is more rigorously demonstrable with a heavy mathematical formalism) which is in fact a definition of the rotational operator which we will seek the explicit mathematical expression right now!
	\end{theorem}
	\begin{dem}
	Given $\vec{f}$ a vector field defined in a given space. We want to calculate the circulation of $\vec{f}$ around a closed path (contour) $C$:
	
	We choose for contour $C$ the edged of an infinitesimal rectangle $(\mathrm{d}x,\mathrm{d}y)$ that is into $\mathbb{R}^3$and parallel to the $xy$-plane (note that we travel the contour so as to always have the surface to our left!):
	\begin{figure}[H]
		\centering
		\includegraphics{img/algebra/rotational_contour_path.jpg}
		\caption[]{Contour (path) of integraton}
	\end{figure}
	For the two horizontal sides (edges), the contribution to the circulation is:
	
	Which authorize us to write:
	
	Same for the vertical sides  (edges) we have also:
	
	Therefore we have the circulation following $z$:
	
	Which can also be written in the following more general and important form:
	
	and that is nothing less than the famous "\NewTerm{Green's theorem}\index{Green's theorem}\label{green theorem}" or also known as the "\NewTerm{Green-Riemann theorem}\index{Green-Riemann theorem}" that we will see again in the section of Complex Analysis.
	
	Green's theorem is a special case of Stokes (general manifold) theorem, but a generalisation to 2-dimension of the fundamental theorem of calculus. But stated in another way: The total spin outside of a function (closed curve) is equal to the sum of little spins on the inside of the function (closed curve).
	
	And that we will write in the situation that interest us:
	
	By circulation permutation we then get:
	
	Either in vector condensed form:
	
	This allows us to better understand the notation, or the non intuitive definition of the rotational in many books and that is:
	
	that is to say the cross product of the gradient operator by the vector field!
	
	So finally we have proved the Stokes theorem or also named in this form "\NewTerm{curl theorem}\index{curl theorem}" that gives well:
	
	and at the same time the rotational in Cartesian coordinates.
	
	The above classical Stokes theorem can be stated in one sentence: The line integral of a vector field over a loop is equal to the flux of its curl through the enclosed surface.
	\begin{flushright}
		$\blacksquare$  Q.E.D.
	\end{flushright}
	\end{dem}
	
	\begin{tcolorbox}[title=Remark,colframe=black,arc=10pt]
	The above relation is in fact not the "Stokes theorem" but a special case of it named the "\NewTerm{Kelvin–Stokes theorem}\index{Kelvin–Stokes theorem}". The stokes theorem is given by (without proof):
	
	and says that the integral of a differential form $\omega$ over the boundary of some orientable manifold $\Omega$ is equal to the integral of its exterior derivative $\mathrm{d}\omega $ over the whole of $\Omega$. It is the generalization of the fundamental theorem of calculus to $N$ dimensions.\\
	
	This  is  the  most  general  and  conceptually  pure  form  of  Stokes’  theorem,  of  which  the  fundamental  theorem  of calculus (1D case), the fundamental theorem of line integrals, Green’s theorem (2D case), Stokes’ (original) theorem (3D case), and the divergence theorem are all special cases!
	\end{tcolorbox}	
	
	\begin{tcolorbox}[colframe=black,colback=white,sharp corners]
	\textbf{{\Large \ding{45}}Examples:}\\\\
	E1. Let us take the vector field, which depends on $x$ and $y$ linearly:
	
	Its plot look like this in Maple 4.00b:\\
	
	\texttt{
	>with(DEtools): with(plots):\\
	>fieldplot([y, -x], x=-5..5, y=-5..5,arrows = medium, \\
	color = sqrt(x \string^2 + y\string^2),thickness=2,labels=[`x`,`y`],\\
	title=`Simple vector field`);	
	}
	\begin{figure}[H]
		\centering
		\includegraphics{img/algebra/vector_field_01.jpg}
		\caption[]{Vector field example with Maple 4.00b}
	\end{figure}
	Simply by visual inspection, we can see that the field is rotating. If we place a paddle wheel anywhere, we see immediately its tendency to rotate clockwise. Using the right-hand rule, we expect the rotational to be into the page. If we are to keep a right-handed coordinate system, into the page will be in the negative $z$ direction. The lack of $x$ and $y$ directions is analogous to the cross product operation.\\

	If we calculate the rotational:
	
	\end{tcolorbox}
	
	\pagebreak
	\begin{tcolorbox}[colframe=black,colback=white,sharp corners]
	With Maple 4.00b we get this algebraic result with the following commands:\\
	
	\texttt{>with(linalg):\\
	>f:=[y,-x,0];v:=[x,y,z];\\
	>curl(f,v);\\
	}
	
	As we did yet not have the time to find an easy way to plot the resulting rotational vector field in the release 4.00b of Maple we will take the picture provided by Wikipedia:
	\begin{figure}[H]
		\centering
		\includegraphics[scale=0.65]{img/algebra/rotational_of_vector_field_01.jpg}
		\caption[]{Rotational of previous vector field (source: Wikipedia)}
	\end{figure}
		
	E2. Suppose we now consider a slightly more complicated vector field:
	
	Its plot:
	\begin{figure}[H]
		\centering
		\includegraphics[scale=0.65]{img/algebra/vector_field_02.jpg}
		\caption[]{second vector field example (source: Wikipedia)}
	\end{figure}
	\end{tcolorbox}
	
	\begin{tcolorbox}[colframe=black,colback=white,sharp corners]
	We might not see any rotation initially, but if we closely look at the right, we see a larger field at, say, $x=4$ than at $x=3$. Intuitively, if we placed a small paddle wheel there, the larger "current" on its right side would cause the paddlewheel to rotate clockwise, which corresponds to a rotational in the negative $z$ direction. By contrast, if we look at a point on the left and placed a small paddle wheel there, the larger "current" on its left side would cause the paddlewheel to rotate counterclockwise, which corresponds to a rotational in the positive $z$ direction. Let's check out our guess by doing the math:\\
	
	Indeed the rotational is in the positive $z$ direction for negative $x$ and in the negative $z$ direction for positive $x$, as expected. Since this rotational is not the same at every point, its plot is a bit more interesting:
	\begin{figure}[H]
		\centering
		\includegraphics[scale=0.65]{img/algebra/rotational_of_vector_field_02.jpg}
		\caption[]{Rotational of previous vector field (source: Wikipedia)}
	\end{figure}
	\end{tcolorbox}
	\pagebreak	
	Let us now determine the expression of the rotational in cylindrical coordinates (the rotational in polar coordinates is not defined in $\mathbb{R}^2$).
	
	Using the same technique as for the rotational in Cartesian coordinates, we write the circulation of $\vec{f}$ along a contour corresponding to a small piece $P_1P_2P_3P_4$ of an orthogonal cylinder (oriented in the direction of the $z$-axis):
	\begin{figure}[H]
		\centering
		\includegraphics{img/algebra/rotational_cylindrical.jpg}
		\caption[]{Representation of the cylinder piece $P_1P_2P_3P_4$}
	\end{figure}
	We have then by fixing $z$ (caution! the $\mathrm{d}\vec{r}$ has nothing to do with the cylinder radius $r$... the notation can be confusing I'm sorry!):
	
	the total circulation thus gives after regrouping terms:
	
	We can not at this stage compare with the rotational because it is difficult to us to make appearing the differential of the surface if we look at the differentials that currently appear in circulation. The best is then to divide everything by $r\mathrm{d}\phi\mathrm{d}r$:
	
	Therefore:
	
	Now we determine the rotational by fixing $\varphi$. The problem is like having a rectangle in the space that we travel to determine the circulation. But we already know what is the result of the rotational for a rectangle in Cartesian coordinates following the $z$-axis:
	
	except that in cylindrical coordinates we have to replace $z$ by $\varphi$, $x$ by $y$, $y$ by $r$ and $f_y$ by $f_r$ and finally $f_x$ by $f_z$ (this choice is always appropriate simply because the circulation is such that the surface is always on our left). This gives us:
	
	It therefore only remains to us to find the component of the rotational on $r$ (therefore when $r$ is fixed). The calculation is more difficult as we have to follow (positively always!) a curved surface by the variation of the angle $\varphi$.
	
	We then have by fixing $r$:
	
	the total circulation thus gives after regrouping terms:
	
	We can not at this stage compare with the rotational because it is difficult to us to make appearing the differential of the surface if we look at the differentials that currently appear in circulation. The best is then to divide everything by $r\mathrm{d}\phi\mathrm{d}z$:
	
	Then finally:
	
		And finally we have the "\NewTerm{rotational in cylindrical coordinates}\index{rotational in cylindrical coordinates}" given in its globality by:
	
	The reader can check verify that this result is simply the gradient in cylindrical coordinates applied to the vector field $\vec{f}$.
	
	To be convinced, let us now show directly the expression of the rotational in spherical coordinates by showing this through the cross product of the gradient in spherical coordinates with the vector field $\vec{f}$.
	
	First let us recall that we have obtained for the gradient in spherical coordinates:
	
	Therefore we have:
	
	what we can write using the decomposition in basis vectors:
	
	Thanks to the partial derivatives that we proved earlier during our introduction to the spherical coordinates, it comes:
	
	The cross products with the colinears vectors canceled. Therefore it remains:
	
	As the cross product of two basis vectors give the corresponding orthogonal vector (positively or negatively) then we have:
	
	By regrouping the terms it comes:
	
	Thus by simplifying:
	
	Thus finally we get the "\NewTerm{rotational in spherical coordinates}\label{rotational in spherical coordinates}":
	
	
	\pagebreak
	\subsubsection{Laplacians of Scalar Fields (Laplace Operator)}\label{scalar laplacian}
	The Laplacian of a scalar field $\phi(x_1,x_2,x_3)$ give also a scalar field that gives the difference between the value of the function $\phi$ on one point and it average around this point. In other words: the second partial derivative measure the variations of the slope on the study point in its immediate neighborhood and following one direction at a time. If the second partial derivative is null following $x$, then the slope is constant in its immediate neighborhood and following this dimension (direction), this implies that the value of the function at the study point is the average of its neighborhood (following one dimension).
	
	The reader will be able to see again major practical applications of this differential operator in the sections of Complex Analysis, Quantum Chemistry, Astronomy, Electrodynamics, Weather \& Marine Engineering, Wave Mechanics, Wave Quantum Physic and Quantum Field Theory.
	
	This operator is defined from the divergence and the gradient and we denote it by (tensor notations):
	
	The Laplacian is null, or quite small, when the function varies. The functions satisfying the "\NewTerm{Laplace equation}\index{Laplace equation}":
	
	are named "\NewTerm{harmonic function}\index{harmonic function}".
	
	Thus the "\NewTerm{scalar Laplacian operator in cartesian coordinates}\index{scalar Laplacian operator in cartesian coordinates}" is by this definition, given by:
	
	The Laplacian of a scalar field in other coordinate systems is a little bit more hard to get that for the other differential operators. There are more than one possible proof but among the existing one we have try to choose (as always) this that seem to us the most interesting in the point of view of the tools used (and not of simplicity!).

	Given the Laplacian in cartesian coordinates in $\mathbb{R}^2$ of a scalar field $f$:
	
	To determine this expression in polar coordinates, we will use the total exact differential and the multivariate chain rule in polar coordinates (\SeeChapter{see section Differential and Integral Calculus page \pageref{multivariate chain rule}}):
	
	therefore for a second derivative:
	
	but we know that we have in polar coordinates:
	
	hence for the first derivative:
	
	and for the second derivative:
	
	therefore:
	
	and given that the second partial derivatives are continuous, then the cross derivatives are equal according to the Schwarz's theorem (\SeeChapter{see section Differential and Integral Calculus page \pageref{Schwarz theorem}}):
	
	Therefore:
	
	Similarly, we will have:
	
	hence the expression of the Laplace operator in polar coordinates by adding the last two expressions:
	
	Therefore the "\NewTerm{scalar Laplacian in polar coordinates}\index{scalar Laplacian in polar coordinates}" is finally given by:
	
	To find the expression of the Laplacian operator in spherical coordinates, we will use the intuition of the physicist and the concepts of similarity.
	
	We will first of all help us with the below figure to find out what we mean:
	\begin{figure}[H]
		\centering
		\includegraphics{img/algebra/coordinate_system_spherical_for_Laplacian_study.jpg}
		\caption[]{Recall of the spherical coordinate system representation}
	\end{figure}
	Recall that the relation between cartesian and spherical coordinates are given by the relations:
	
	We will now consider the following similarities:
	\begin{enumerate}
		\item Cylindrical coordinates:
			
		
		\item Spherical coordinates:
			
	\end{enumerate}
	Let us build a correspondence table:
	
	The goal is to play with this correspondence with in a first time the Laplacian in cylindrical coordinates where we have subtracted from both sides of the equality the term $\dfrac{\partial^2 f}{\partial z^2}$. Therefore:
	
	let us now use our small correspondence table and we get:
	
	The second term of the equality of the latter relation is the spherical equivalent of the term \#1 of the Laplacian in cylindrical coordinates:
	
	Now let us examine and focus on the term: $\dfrac{1}{\rho}\dfrac{\partial f}{\partial \rho}$
	
	Identically as when we determined the relation:
	
	we get:
	
	with:
	
	Which give us the possibility to write:
	
	If we play again with our small correspondence table we get:
	
	We divide the latter relation by $\rho$ and we get:
	
	We have therefore above the spherical equivalent of the second term \#2 of the Laplacian in cylindrical coordinates:
	
	The last and third term is quite simple to determine. We just have to replace $\rho$ by $r\sin(\theta)$ to get:
	
	By bringing together all terms obtained previously, we finally get the extended form of the Laplacian in spherical coordinates used so much in physics (see corresponding sections of this book):
	
	We can shorten this expression by factoring the terms:
	
	If we condense even a little bit more, we get the final expression of the "\NewTerm{scalar Laplacian in spherical coordinates}\index{scalar Laplacian in spherical coordinates}\label{scalar laplacian in spherical coordinates}" named also "\NewTerm{spherical Laplacian}":
	
	
	\subsubsection{Laplacians of Vector Fields}\label{laplacian of vector fields}
	As to the Laplacian of a scalar field, the Laplacian of a vector field is only a very convenient notation system for condensing the writing of the components of a vector field.
	
	The reader will also find practical applications of this operator in the sections of Electrokinetics, Electrodynamics and Continuum Mechanics.
	
	Thus, the vector Laplacian is often defined by:
	
	We also prove that in the specific case of the Cartesian coordinates, the Laplacian of a vector field has the components the scalar Laplacian of each of the components.
	
	We also prove that in the specific case of the Cartesian coordinates, the Laplacian of a vector field has the components the scalar Laplacian of each of the components.
	
	So therefore have in Cartesian coordinates:
	
	So we therefore in Cartesian coordinates:
	
	and thus the "\NewTerm{vector Laplacian of a vector field in Cartesian coordinates}\index{vector Laplacian of a vector field in Cartesian coordinates}" is indeed the scalar Laplacian of each component:
	
	Or more explicitly:
	
	The Laplacian of a vector field, frequently named "\NewTerm{vectorial laplacian}\index{vectorial laplacian}", in other coordinates systems is quite simple to get once we know the Laplacian of a scalar field in the same coordinates!
	
	We have first in cylindrical coordinates:
	
	To simplify (because of a lack of space) let us focus first on the first line:
	
	and after for the second line:
	
	and finally the third one:
	
	So what gives for us for the "\NewTerm{vector Laplacian in cylindrical coordinates}\index{vector Laplacian in cylindrical coordinates}" as we can find it in tables or forms:
	
	To finish and in the joy (...) let us make the merry and detailed calculations of the vector Laplacian operator in spherical coordinates (it's quite long but it's just to make sure that we fall back on what is in tables and forms):
	
	Let us focus on the first line (caution! this will be quite long...):
	
	\pagebreak
	
	That's it for the first line ... Let us get on to the second line always with joy...:
	
	
	\pagebreak
	and finally a last effort for the third and last line:
	

	\pagebreak
	
	So what gives for the "\NewTerm{vector Laplacian in spherical coordinates}\index{vector Laplacian in spherical coordinates}" as we can find it in tables or forms:
	
	that's it... for the skeptics...
	
	\subsubsection{Remarkable Identities}\label{differential operators identities}
	The scalar and vector differential operators have some very simple remarkable identities that we will find very often in physics in this book.
	
	Let us first see the relation that make no sense (in case you would fall on them without purpose...) :
	
	For the relation above, the rotational (curl) of a divergence does not exist since the rotational operator applies to a vector field while the divergence is a scalar!
	
	For the above relation the rotational (curl) of a scalar Laplacian does not exist since the rotational operator applies to a vector field while by construction, the Laplacian is a scalar.
	
	Let us now see some remarkable identities without proof for the majority (if there is a proof this is because one reader did the request to have all the details...):
	\begin{enumerate}
		\item By construction the scalar Laplacian is the divergence of the gradient of the scalar field:
		
		
		\item The rotational (curl) of the gradient is equal to zero:
		
		Therefore if the rotational (curl) of a vector variable (vector field) is zero, this same variable can be expressed as the gradient of a scalar potential!!!!!!!!!!!! This is a veeeeerrrrry important property (or trick depending of the point of view...) in Electromagnetics and Fluid Mechanics and Quantum Physics!
		\begin{dem}
		
		\begin{flushright}
			$\blacksquare$  Q.E.D.
		\end{flushright}
		\end{dem}
		
		\item The dot product of two rotational is equal to something boring to say just with words...:
		
		
		\item The divergence of the rotational (curl) of a vector field is always equal to zero:
		
		\begin{dem}
		
		\begin{flushright}
			$\blacksquare$  Q.E.D.
		\end{flushright}
		\end{dem}
		
		\item The rotational (curl) of the rotational  of a vector field is equal to the gradient of the divergence of this vector field less its vector Laplacian:
		
		\begin{dem}
		
		It is then easy to check that this last equality is equal to:
		
		\begin{flushright}
			$\blacksquare$  Q.E.D.
		\end{flushright}
		\end{dem}
		
		\item The multiplication of the nabla operator by the dot product of two vectors is equal to ... (see below), which provides a very useful relation in Fluid Mechanics:
		
		
		\item The scalar product of the rotational (curl) of a vector is the difference of the commutated operators such that (we can provide the detail proof on request):
		
		We will use this last relation in our study of electromagnetic radiation pressure in the section of Electrodynamics (among others...).
		
		\item The gradient of a cross product is the difference of the commutated operators such that (we can provide the detail proof on request)
		
		We will use this last relation in our study of superconductors in the section of Electrokinetics page \pageref{superconductivity}.
		
		\item Let us consider a vector $\vec{v}$ which each component depends on $x,y,z$. We have (the left term is named the "material derivative", see section Continuum Mechanics page \pageref{material derivative}):
		
		Thus explicitly:
		
		Or even more explicit:
		
		So now to prove that the equality holds, we will develop the left part and right part. So let us begin with the left part:
		
		Now let us develop the right part:
		
		and then we see that the equality holds!
	\end{enumerate}
	
	\pagebreak
	\subsubsection{Summary}
	As part of this book, we will use the different notations presented and summarized in the table below. Their usage gives us the possibility in the context of the different theories to avoid confusion with other mathematics being (tools). It's annoying but we have to do with it.
	
	\begin{table}[H]
		\begin{center}
			\definecolor{gris}{gray}{0.85}
				\begin{tabular}{|p{8cm}|p{6cm}|}
					\hline
					\multicolumn{1}{c}{\cellcolor{black!30}\textbf{Type}} & 
	  \multicolumn{1}{c}{\cellcolor{black!30}\textbf{Expression}} \\ \hline
					Gradient of a scalar field $A:\mathbb{R}\mapsto \mathbb{R}^3$ & \centering\arraybackslash\ $\vec{\nabla}(f) \quad \vec{\nabla}\cdot f \quad \vec{\nabla}f \quad \overrightarrow{\text{grad}}(f)$ \\ \hline
					Gradient of a vector field $A:\mathbb{R}^n\mapsto \mathbb{R}^n\times \mathbb{R}^n$ & \centering\arraybackslash\ $\vec{\nabla}(\vec{f})$  \\ \hline
					Divergence of a vector field $A:\mathbb{R}^n\mapsto \mathbb{R}$ & \centering\arraybackslash\ $\vec{\nabla}\circ \vec{f}\quad \text{div}(\vec{f})$  \\ \hline
					Rotational of a vector field $A:\mathbb{R}^3\mapsto \mathbb{R}^3$ & \centering\arraybackslash\ $\vec{\nabla}\times\vec{f}\quad \text{rot}(\vec{f})$  \\ \hline
					Lapalacian of a scalar field $A:\mathbb{R}^n\mapsto \mathbb{R}$ & 		\centering\arraybackslash\ $\Delta \cdot f\quad \vec{\nabla}^2 f\quad \vec{\nabla}\circ(\vec{\nabla}\cdot f)$  \\ \hline
					Laplacian of a vector field $A:\mathbb{R}^3\mapsto \mathbb{R}^3$& \centering\arraybackslash\  $\Delta\vec{f}$  \\ \hline
			\end{tabular}
		\end{center}
		\caption{Summary of vector differential operators}
	\end{table}
	Now let us do a quick summary of main differential operators:
	\begin{itemize}
		\item The gradient can be assimilated to the "slope" (example: the electric field is the "slope" of the electrostatic potential).
		
		The various expressions of the gradient operator (placed under the form of the nabla operator) in Cartesian, polar, cylindrical, spherical coordinates are the following:
		
		
		\item The divergence characterizes a flow of something that comes from somewhere, a source, or who goes to it. If the divergence is different from zero, it means that there is concentration around a point, so the density increases (or decreases, it depends on the sign). It could be the density of electric charges or the mass density. Hence the famous theorem that says that the flow (that which passes through a surface) is equal to the integral of the divergence (what remains).
		
		The various expressions of the divergence operator (placed under the form of the nabla operator) in Cartesian, polar, cylindrical, spherical coordinates are the following:
		
		
		\item The rotational characterizes the existence of a vortex (Widely used in fluid mechanics). If there is a whirlwind, we can follow a flow line on a closed curve (closed: in the differential point of view, not in the geometrical one!) without it change of direction: the circulation will the not be equal to zero (it is equal to integral of the rotational (curl)).
		
		The various expressions of the rotational (curl) operator (placed under the form of the nabla operator) in Cartesian, cylindrical, spherical coordinates are the following:
		
		
		\item The Laplacian of a scalar field gives a scalar field that measures the difference between the value of the function at a point and its average around that point. In other words, the partial second derivative measure the variations of the slope at the point examined in the immediate surroundings and in one dimension at a time. If the partial second derivative is zero in one direction, then the slope is constant in the immediate surroundings and according to this dimension, this means that the value of the function on the study is equal to the average of his neighborhood (in one dimension).
		
		The different expressions of the scalar Laplacian operator (placed under the form of the nabla operator) in Cartesian, polar and spherical coordinates are:
		
		
		\item As for the Laplacian of a scalar field, the Laplacian of a vector field is only a very convenient notation system for condensing the writing of the components of a vector field.
		
		The different expressions of the scalar Laplacian operator (placed under the form of the nabla operator) in Cartesian, polar and spherical coordinates are:
		
	\end{itemize}
	And also we have the following list of remarkable identities:
	
	
	\pagebreak
	Finally let us finis this summary with all the theorem that we have obtained so far in this section that are named "\NewTerm{$1$st order integral theorems}\index{first order integral theorems}":
	\begin{itemize}
		\item Gradient theorem (only for uniform field):
		
	
		\item Ostrogradsky theorem (divergence theorem):
		

		\item Green theorem (Green-Riemann theorem):
		
	
		\item Stokes theorem (curl theorem):
		
	\end{itemize}
	Keep in mind that there are all special case of the Stokes (general manifold theorem\footnote{If one day we have the time to write the \LaTeX to provide a detailed example we will! But actually that doesn't bring too much interest for an engineer.}) including the fundamental theorem of calculus!
	
	Then the fundamental theorem of calculus (1D), Green's theorem (2D) and the divergence theorem (3D) (ie Stokes original theorem) can all be stated respectively by the following parametrised sentence (still special case of Stokes general manifold theorem): The total [change, spin, flow] outside is equal to the sum of little [changes, spins, flows] on the inside.
	
	\begin{flushright}
	\begin{tabular}{l c}
	\circled{95} & \pbox{20cm}{\score{4}{5} \\ {\tiny 99 votes,  84.44\%}} 
	\end{tabular} 
	\end{flushright}
	
	%to make section start on odd page
	\newpage
	\thispagestyle{empty}
	\mbox{}
	\section{Linear Algebra}\label{linear algebra}

	\lettrine[lines=4]{\color{BrickRed}T}here are several approaches to learn Linear Algebra. First a pragmatic way (we'll start with this one because our experience has shown us that it is the one that seemed to work best for students) and a more formal way that we will present in after. We should first warn the reader that linear algebra is a powerful calculation tool that we use enormous in economic and industrial practice in the following areas (see the respective chapters of the book for specific examples): Statistics , Electrotechniques, Finance market, Numerical optimization methods, Optics, Quantum Physics, Electrodynamics, Relativity, Fluid mechanics, etc. It is then necessary to attach a special attention to this subject.
	
	First let us answer at  two questions from a reader: Why Linear Algebra is named like this? And is there a Non-Linear Algebra?
	
	Here are my answers:

	\begin{enumerate}
		\item This is named "Linear Algebra" because it was first necessary to choose a name... and also because it is a generalization of the scalar algebra but with vectors where applications are no longer scalar functions but matrix applications whose effect is to act as a linear sum of a base vectors (at least this can be interpreted as such).
		\item Officially and to my knowledge there is no non-linear algebra in the same philosophy as what we will see in this section. It seem that they are mathematicians who have created "non linear algebra" but these have nothing to do with matrices.
	\end{enumerate}

Now, remember that we studied in section Calculus how to determine the intersection (if any exist) of the equation of two lines in $\mathbb{R}^2$  (we can extend the problem obviously to more than two lines), as it is equivalent to solve a polynomial of order $1$, given by:
	
where $a_i,b_i \in \mathbb{R}$.

Thus seeking the value for which:
	
leads us to write:
	
However, there is another way of presenting the problem as we have seen in the Numerical Methods section (\SeeChapter{Theoretical Computer chapter}). Indeed, we can write the problem in the form of a block of equations:
		
and as we seek $y_1=y_2=y$, we have:
		
This writing is named as we have seen in the chapter of Theoretical Computing (\SeeChapter{see section of Numerical Methods page \pageref{linear systems of equations}}) a "\NewTerm{linear system}\index{linear system}" that we can solve by subtracting or adding the lines between them (all the solutions are always equal), which gives us:
	
and we see that we fall back on the solution:
	
So there are two ways to present a problem of intersection of lines:
	\begin{enumerate}
		\item In the form of an equation
		\item In the form of a system
	\end{enumerate}
We will focus in a part of this section to the second method that will allow us with tools seen in the section of Vector Calculus to resolve not only the intersections of one or more straight lines but one or more straight, plans, hyperplanes, etc. in respectively $\mathbb{R},\mathbb{R^2},...,\mathbb{R^n}$. 

	Obviously we will see that Linear Algebra is not only use for this purpose but can also be used the generalized some physics mathematical models or to express geometrical transformations of vectors or figures in 2D or 3D or also express Markov Chains, special properties of Multivariate Statistics, calculation of some differential equation (for reliability engineering for example) and much more! Many examples are given in this book about the application of Linear Algebra in real life.

Before attacking the theoretical part, let us presented a very interesting example but that requires a concept - the determinant - that we will prove rigorously much further in detail in this section (it seemed to us more pedagogical to approach this subject now rather than the reader must wait to browse dozens of mathematical developments of pages before reaching the rigorous definition of the determinant).

Consider the system of two linear equations with two unknowns (system of intersection of planes):
	
If we solve this, we quickly get (technique named "\NewTerm{substitution method}\index{substitution method}"):
	
It comes then:
	
and so at the end:
	
and if we define a little bit fast something that is named the "\NewTerm{determinant}\index{determinant}" that we will see further below rigorously as follows:
	
or with another very more common notation:
	
we thus have:
	
And by proceeding in the same way we get:
	
It comes then:
	
and so finally we get:
	
It then appears clear that if:
		
the system has infinitely many solutions. In contrast, the system has no solution if:
	
And if the reader repeats (happily...) the procedure for a system of three equations with three unknowns of the type (intersection of hyperplanes):
	
We then get (after some basic boring algebraic operations):
	
with:
	
	It then appears clear that if:
	
	the system has infinitely many solutions. In contrast, the system has no solution if:
	
	and so on for $n$ equations with $n$ unknowns.

	However, there was a condition to satisfy: as we have seen in the previous example, we could not solve a system of equations with two unknowns if we have only one equation. That is why it is necessary and sufficient for a system of equations with $n$ unknowns to have at least $n$ equations. Thus, we speak of "\NewTerm{systems with $n$ equations in $n$ unknowns}\index{systems with $n$ equations in $n$ unknowns}". We also prove that it is necessary and sufficient that the determinant is non-zero for a linear system, whose matrix equivalent is square, to have a unique solution (the concept of "determinant" and "matrix" will be defined further below robustly) and therefore that the corresponding matrix to the system is invertible (non-singular).

\pagebreak
\subsection{Linear Systems}\label{linear systems}

\textbf{Definitions (\#\mydef):}
\begin{enumerate}
	\item[D1.] We name "\NewTerm{linear system}\index{linear system}", or simply "\NewTerm{system}" any family of equations of the form:
	
where each line represents the equation of a line, plane or hyperplane (\SeeChapter{see section Analytical Geometry page \pageref{equation of the plane}}), and $a_{mn}$ are the "\NewTerm{system coefficients}\index{linear system coefficients}", the $b_m$ the "\NewTerm{coefficients of the second member}\index{coefficients of the second member}" and the $x_n$ the "\NewTerm{unknowns of the system}\index{unknowns of the system}".

	\item[D2.] If the system has $n$ unknown and $n$ equations and has a unique solution, we then name it a "\NewTerm{Cramer system}\index{Cramer system}" (1750).\label{unique solution linear system}
	
	\item[D3.] If the coefficients of the second member are all zero, then we say that the system is a "\NewTerm{homogeneous system}\index{homogeneous system}" so it has at least the trivial solution where all $x_n$ are equal to zero. Otherwise we say that we have an "\NewTerm{inhomogeneous system}\index{inhomogeneous system}".
	
	\item[D4.] We name "\NewTerm{homogeneous system associated to the system}\index{homogeneous system associated to the system}", the system of equations we get by substituting zeros to the coefficients of the second member ($b_m$).
	\end{enumerate}
Let us now recall the following items:
	\begin{itemize}
		\item The linear equation (\SeeChapter{see section Functional Analysis page \pageref{straight line}}) is given by:
			
		by defining $x=x_1$ and $y=x_2$.
		
		\item The equation of a plane (\SeeChapter{see section Analytical Geometry page \pageref{equation of the plane}}) is given by:
			
		by defining $x=x_1, y=x_2, z=x_3$.
	\end{itemize}
We often write a linear system in the following condensed form:
	
We name "\NewTerm{system solution}\index{linear system solution}" or "\NewTerm{vector system solution}\index{vector system solution}" any $t$-uple $(x_1^0,x_2^0,...,x_n^0)$ such as:
	
Solve a system means finding all the solutions of this system (we find many such systems in economic, operational research or design of experiments). Two system with $n$ unknowns are named  "\NewTerm{equivalent systems}\index{equivalent linear systems}" if all of the solution of one system is the solution of the other, i.e., if they have the same set of solutions. We sometimes say that the equations of a system are "\NewTerm{compatible equations}\index{compatible equations}" or "\NewTerm{incompatible equations}\index{incompatible equations}", depending on whether the system has at least one solution or not admit any.

We can also give for sure a geometric interpretation to these systems. Suppose that the first members of the equations of the system are not zero. So we know that each of these equations represents a hyperplane of an affine space (\SeeChapter{see section Vector Calculus page \pageref{affine space}}) of dimension $n$. Therefore, all solutions of the system, considered as set of $n$-tuples of coordinates is a finite intersection of hyperplanes.

	\begin{tcolorbox}[colframe=black,colback=white,sharp corners]
	\textbf{{\Large \ding{45}}Example:}
	
	denoted conventionally in high-school classes in the form:
	
	This system has for solutions the points representing the intersection of the three planes defined by the three equations. But as we can see it visually with Maple 4.00b using the following commands:\\

	\texttt{>with(plots):}\\
	\texttt{>implicitplot3d({x-3*z=-3,2*x-5*y-z=-2,x+2*y-5*z=1},x=-3..3,}
	\texttt{y=-3..3,z=-3..3);}
	
	\begin{figure}[H]
	\centering
	\includegraphics[scale=0.35]{img/algebra/system_3_equations_3_unknowns.eps}
	\caption{Graphical representation of the linear system}
	\end{figure}
	
	There is no solutions. You can be checked by hand or with Maple 4.00b by writing:\\
	
	\texttt{>solve({x-3*z=-3,2*x-5*y-z=-2,x+2*y-5*z=1},{x,y,z});}
	\end{tcolorbox}

	\begin{tcolorbox}[title=Remark,colframe=black,arc=10pt]
For "classic" resolution methods of such systems, we refer the readers to the section on Numerical Methods of the chapter on Computing Science.
	\end{tcolorbox}	
	
	Finally, note an important case in practice: "\NewTerm{overdetermined systems}\index{overdetermined systems}" where we have more equations than unknowns. 
	\begin{figure}[H]
		\centering
		\includegraphics[scale=0.45]{img/algebra/overdetermined_system.jpg}
		\caption{Overdetermined system}
	\end{figure}
	The first situation of this type dates from the 18th century through the study of the lunar oscillations but we also find this frequently in R\&D laboratory in the context of design of experiments (\SeeChapter{see section Industrial Engineering page \pageref{doe}}) or in structural equation models (\SeeChapter{see section Numerical Methods}).
	
	An overdetermined system may be written as:
	
	where $X$ is $n \times m$ and $\operatorname{rank}(X \mid \vec y)>m ;$ that is, the system is not consistent. We have changed the notation slightly from the consistent system $A x=b$ that we have been using because now we have in mind statistical applications, and in those the notation $y \approx X \beta$ is more common. The problem is to determine a value of $b$ that makes the approximation close in some sense. In applications of linear systems, we refer to this as "fitting" the system, which is referred to as a "model".

	\begin{tcolorbox}[colframe=black,colback=white,sharp corners]
\textbf{{\Large \ding{45}}Example:}\\\\
Consider the special case but very telling following example of three equations with two unknowns:
	
system that will be written in the form of matrix and vector as following:
	\end{tcolorbox}

	\begin{tcolorbox}[colframe=black,colback=white,sharp corners]
	
	or said in an "\NewTerm{augmented form}\index{linear system augmented form}" as follows:
	
	and with Maple 4.00b:\\
	
	\texttt{>with(plots):}\\
	\texttt{>implicitplot({2*x+3*y=-1,-3*x+y=-2,-x+y=1},x=-3..3,y=-3..3);}
	\begin{figure}[H]
		\centering
		\includegraphics[scale=0.7]{img/algebra/overdetermined_system.eps}
		\caption{Representation of a system of 3 equations with two unknowns with Maple 4.00b}
	\end{figure}
	
	We can see on the chart above that this particular system has no complete solution but thus has a  solution if we take only the problem by pair of equations... which does not necessarily help in facts...
	\end{tcolorbox}

	Notice, that we have just seen that the system could be written as following:
	
	This looks like a multiple linear regression system (\SeeChapter{see section of Theoretical Computing page \pageref{linear systems of equations}}) whose column vector of unknowns can be viewed as the coefficient vector $\vec{\beta}$ of the regression line such that:
	
We then proved in detail in the section of Theoretical Computing that:
	

at the condition that the square matrix $X^TX$ is, as we will see further below, invertible (non-singular). Since the works of Gauss on the subject it is considered that this approach (hence implicitly based on ordinary least squares regression\footnote{Such that we found $\vec{\beta}$ that minimize $\|X\vec{\beta}-\vec{y}\|^2$}) remains even today the best approach to solve such systems!

	If the invertible property is satisfied, we find a "\NewTerm{pseudo-solution}\index{pseudo-linear solution}"\index{pseudo-solution} (this is the official terminology...) by making calculations quickly by hand (or with a spreadsheet software like Microsoft Excel):
	
and injecting these values in the initial overdetermined system, the reader will quickly understand why we talk about "pseudo-solution"...

	\begin{tcolorbox}[title=Remark,colframe=black,arc=10pt]
A reader asked us why we don't use for the example above just the relation $\vec{\beta}=\vec{y}X^{-1}$? The answer is simple! Because $X$ is not a square matrix or in other words... it is in our example over-determined this is why we can only found a "pseudo-solution".
	\end{tcolorbox}

This was the pragmatic way of looking at it ... now let us turn to the second slightly more mathematical way ... (but still relatively simple):

	\subsection{Linear Transformations}
	\textbf{Definition (\#\mydef):} A "\NewTerm{linear transformation}\index{linear transformation}" or "\NewTerm{linear application}\index{linear application}" $A$ is a mapping of a vector space $E$ to a vector space $F$ such that with $K$ being in $\mathbb{R}$ or in $\mathbb{C}$\label{linear application}:
	
	this constitutes, for recall, an endomorphism\label{matrix endomorphism} (\SeeChapter{see section Set Theory page \pageref{endomorphism}}).

	The first property specifies that the transformation of a sum of vectors must be equal to the sum of the transformed vectors, so that it is linear. The second property specifies that the transformation of a vector to  which we applied a scale factor (scaling) must also be equal to the same factor applied on the transformation of the original vector. If either of these two properties is not met, the transformation is therefore not linear.

	We now show that any linear transformation can be represented by a matrix!

	Given the basis vectors $\left(\vec{v}_1,\vec{v}_2,...,\vec{v}_n \right)$ of $E$ and $\left(\vec{u}_1,\vec{u}_2,...,\vec{u}_n \right)$ for those of $F$ with $m \geq n$. With these bases, we can represent any vectors $\vec{x} \in E, \vec{y} \in F$ with the following linear combinations (\SeeChapter{see section Vector Calculus page \pageref{linear combinations}}):
	
	Consider the linear transformation $A$ which applies $E$ to $F$ ($A:E\mapsto F$). So:
			
	that we can rewrite as follows:
	
	But since $A$ is a linear operator by definition, we can also write:
	
	Considering now that the vectors $A(\vec{v}_j)$ are elements of $F$, we can rewrite them as a linear combination of its basis vectors:
	
	Therefore, we get:
	
	By reversing the order of summations, we can write:
	
	and rearranging the latter relation, we produce the result:
	
	Finally, remembering that the basis vectors $\vec{u}_i$  must be independent, we can conclude that their coefficients must necessarily be zero, so:
	
	Which corresponds to the "\NewTerm{matrix product}\index{matrix product}":
	
	That we can write:
	
	In other words, any linear transformation can be described by a matrix $A$ that is multiplied with the vector that we want to transform, to obtain the vector resulting from the transformation.
	
	\pagebreak
	\subsection{Matrices}
	So we name a "\NewTerm{matrix}\index{matrix}" with $m$ rows and $n$ columns, or "\NewTerm{type $m\times n$ matrix}" (the first term always correspond to the rows and the second to columns to the second, to remember this there is a good mnemonic trick: President Lincoln - abbreviation of Lin(e) and Col(um)n ...), any array of numbers in a ring $\mathbb{K}$ (that is most of times $\mathbb{R}$):
	
	For example:
	
	is a $5\times 5$ matrix.
	
	We often denote a matrix of type $m\times n$ briefly by:
	
	or simply $(a_{ij})$. In a more formal way:
	
	
	The number $a_{ij}$ is named "\NewTerm{term or coefficient of index $i, j$}\index{term of a matrix}". The index $i$ is named the "\NewTerm{line index}\index{row index}" and the index $j$ the "\NewTerm{column index}\index{column index}".
	
	We denote by $M_{mn}(\mathbb{K})$ all matrices $m\times n$ whose coefficients take values in $\mathbb{K}$ (typically $\mathbb{R}$ or $\mathbb{C}$ for example).
	
	When $m=n$, we say that $(a_{ij})$ is a "\NewTerm{square matrix of order $n$}\index{square matrix}\label{square matrix}":
		
	In this case, the terms $a_{11},a_{22},...,a_{nn}$ are named "\NewTerm{diagonal terms}\index{matrix diagonal terms}" denoted by: 
	
	
	We will assign to the matrices special symbols, ie uppercase Latin letters: $A, B, ...$ for matrices and for columns-matrices symbols that will be vectorial lowercase letters $\vec{a},\vec{b},...$.
	
	We also name a matrix with a single row a "\NewTerm{line-matrix}\index{line -matrix}" and a matrix with a single column a "\NewTerm{column-matrix}\index{column-matrix}". It is clear that a matrix column is nothing but a "\NewTerm{column vector}\index{column-vector}" or simply a "\NewTerm{vector}\index{vector}" (\SeeChapter{see section Vector Calculus page \pageref{vector}}). Thereafter, the rows of a matrix will be treated as lines-matrices and the columns to columns-matrices.
	
	More formally, let $M\in \mathbb{R}^{m\times n}$ then a "\NewTerm{submatrix}" of $M$ is a matrix which is made of some rectangle of elements in $M$. Rows and columns are submatrices. In particular.
	\begin{itemize}
		\item An  $m\times 1$ submatrix is named a "\NewTerm{column vector}" of $M$. The $j$-the column vector is denoted $\text{col}_j(M)$ or $\vec{M}_j$ and $\vec{M}_{ji}=M_{ij}$ for $1\leq i\leq m$. In other words:
		
		
		\item An $1\times n$ submatrix of $M$ is named a "\NewTerm{row vector}" of $M$. The $i$-th row vector is denoted $\text{row}_i(M)$ and $(\text{row}_i(M))_j=M_{ij}$ for $1\leq j\leq n$ in other words:
		
	\end{itemize}
	
	We name "\NewTerm{zero matrix}\index{zero matrix}", and we denote by $0_{mn}$ or simply $\mathbf{0}$ or $\mathds{O}$ (this last notation is preferred in this book!) any matrix in which each term is zero:
		
	Null columns-matrices are designated by the symbol vector: $\vec{0}$.
	
	We name "\NewTerm{identity matrix of order $n$}\index{identity matrix}" or "\NewTerm{unit matrix of order $n$}\index{unit matrix}", and denote it by $I$, or simply $\mathds{1}$, the square matrix of order $n$:
	
	It can also be written using the Kronecker delta notation $\delta_{ij}$ (\SeeChapter{see section Tensors Calculus page \pageref{kronecker symbol}}).
	
	\begin{tcolorbox}[colback=red!5,borderline={1mm}{2mm}{red!5},arc=0mm,boxrule=0pt]
	\bcbombe Caution! When we work with matrices having complex coefficients we must always use the term "identity matrix" rather than "unitary matrix" because in the field of complex numbers the unitary matrix is another mathematical object that should not be confused!
	\end{tcolorbox}

	We will see later that the zero matrix acts as a neutral element of the matrix addition and the unit matrix as neutral element for the matrix multiplication.
	
	The $ij$-th "\NewTerm{standard basis matrix}\index{standard basis matrix}" for $\mathbb{R}^{m\times n}$ is denoted $E_{ij}$ for $1\leq i \leq m$ and $1\leq j\leq n$. The matrix $E_{ij}$ is zero in all entries except for the $(i,j)$-th slot where it has a $1$. In other words, we define $(E_{ij})_{kl}=\delta_{ik}\delta_{jl}$. Hence every matrix in $\mathbb{R}^{m\times n}$ is a linear combination of the $E_{ij}$. 
	
	Indeed, let $M\in \mathbb{R}^{m\times n}$ then:
	
	
	The purpose of the concept of matrix will appear throughout the texts that will follow but the immediate reason for this notion is simply to allow some finite families of numbers to be designed as a rectangular array and to generalize physics theorems to multidimensional spaces.
	
	\subsubsection{Rank of a matrix}\label{rank of a matrix}
	We will now briefly review the definition of "\NewTerm{rank of a finite family}\index{rank of a finite family}" that we saw in the section of Vector Calculus.
	
	As a reminder we name "\NewTerm{rank}" of a free family of vectors the dimension (positive integer number) of the vector subspace $S$ of $E$ that it generates.
	
	\textbf{Definition (\#\mydef):} Given $(\vec{a}_1,\vec{a}_2,...,\vec{a}_n)$ the columns of a matrix $A$, we name "\NewTerm{rank of $A$}\index{rank of a matrix}", and denote by $\text{rk}(A)$, the rank of the family of vectors $(\vec{a}_1,\vec{a}_2,...,\vec{a}_n)$. More precisely, in linear algebra, the rank of a matrix $A$ is the dimension of the vector space generated (or spanned) by its columns. This is the same as the dimension of the space spanned by its rows as we will see late. It is a measure of the "nondegenerateness" of the system of linear equations and linear transformation encoded by $A$ (see earlier above!). There are multiple equivalent definitions of rank. A matrix's rank is one of its most fundamental characteristics!
	
	In a slightly more familiar language (...) the rank of a matrix is given by the number of columns-matrices that can't be expressed by the combination and scalar multiplication of other columns-matrices of the same matrix!!
	
	Given this definition it is almost obvious that the rank of a matrix is zero if and only if the matrix is the zero matrix $\mathds{O}$.
	
	Before we enter in more formal calculations let us see already some introducing examples:
	\begin{tcolorbox}[colframe=black,colback=white,sharp corners]
	\textbf{{\Large \ding{45}}Examples:}\\\\
	E1. The matrix:
	
	has rank $2$: the first two rows are linearly independent, so the rank is at least $2$, but all three rows are linearly dependent (the first is equal to the sum of the second and third) so the rank must be less than $3$.\\
	
	E2. The matrix:
	
	has rank $1$: there are nonzero columns, so the rank is positive, but any pair of columns is linearly dependent. Similarly, the transpose:
	
	of $A$ has rank $1$. Indeed, since the column vectors of $A$ are the row vectors of the transpose of $A$, the statement that the column rank of a matrix equals its row rank is equivalent to the statement that the rank of a matrix is equal to the rank of its transpose, i.e.:
	
	\end{tcolorbox}
	The above example show the statement that the column rank of a matrix equals its row rank is equivalent to the statement that the rank of a matrix is equal to the rank of its transpose.
	
	Before continuing we would like to indicate to the reader that later we will prove that if the rows of a matrix are independent, its determinant is non zero $\det(A)\neq 0$ and therefore $\text{rk}(A)=n$ and obviously $\text{rk}(A)=1$ when $\det(A)=0$.
	
	\textbf{Definition (\#\mydef):} A matrix is said to have "\NewTerm{full rank}\index{full rank}" if its rank equals the largest possible for a matrix of the same dimensions, which is the lesser of the number of rows and columns. A matrix is said to be "\NewTerm{rank deficient}\index{rank deficient}" if it does not have full rank.
	
	\begin{tcolorbox}[title=Remark,colframe=black,arc=10pt]
	If there is difficulty in determining the rank of a matrix there is a technique named "\NewTerm{matrix scaling}\index{matrix scaling}" that we will see just below that can do this work very quickly.
	\end{tcolorbox}
	\textbf{Definition (\#\mydef):} We name "\NewTerm{system associated matrix}\index{system associated matrix}":
	
	The mathematical object define by:
	
	that is to say, the matrix $A$ in which the terms are the coefficients of the system. We name "\NewTerm{matrix of the second member of the linear system}\index{matrix of the second member of the linear system}", or simply "\NewTerm{second member of the system}", the matrix-column $\vec{b}=(b_i)$ whose terms are the coefficients of the second member of this system. We also name "\NewTerm{augmented matrix associated to a system}\index{augmented matrix associated of a system}" the matrix $A$ obtained by adding $\vec{b}=(b_i)$ as the $(n + 1)$-th column.
	
	If we now consider an associated  system matrix $A$ and of second member $\vec{b}$. Let us always denote the columns of $A$ by $(\vec{a}_1,\vec{a}_2,...,\vec{a}_n)$. The system can then be written equivalently as a linear vector equation:
	
	Now remember a theorem that we saw and proved in the section of Vector Calculus: for the rank of a family of vectors $(\vec{x}_1,\vec{x}_2,...,\vec{x}_n)$ to be the equal to rank of augmented family $(\vec{x}_1,\vec{x}_2,...,\vec{x}_n,\vec{y})$, it is necessary and sufficient that the vector $\vec{y}$ is a linear combination of the vectors $\vec{x}_1,\vec{x}_2,...,\vec{x}_n)$.
	
	It follows that our linear system in a vector form has at least one solution $(\vec{x}_1^0,\vec{x}_2^0,...,\vec{x}_n^0)$ if the rank of the family $(\vec{a}_1,\vec{a}_2,...,\vec{a}_n)$ is equal to the rank of augmented family $(\vec{a}_1,\vec{a}_2,...,\vec{a}_n,\vec{b})$ and this solution is unique if and only if the rank of the family $(\vec{a}_1,\vec{a}_2,...,\vec{a}_n)$ is equal to $n$.
	
	Thus, for a linear system matrix of associated matrix $A$ and of second member $\vec{b}$ admits at least one solution, it is necessary and sufficient that the rank of $A$ is equal to the rank of the augmented matrix $(A|\vec{b})$. If this condition is met, the system admits a unique solution if and only if the rank of $A$ is equal to the number of unknowns in other words: if the columns of $A$ are linearly independent!!!
	
	We say that a matrix is "staggered" if its rows (lines) meet the following two conditions:
	\begin{enumerate}
		\item[C1.] Any null line is followed by lines full of zeros
		
		\item[C2.] The leading coefficient (the first nonzero number from the left, also named the "\NewTerm{pivot}") of a nonzero row is always strictly to the right of the leading coefficient of the row above it 
	\end{enumerate}
	These two conditions imply that all entries in a column below a leading coefficient are zeros.
	
	A non-zero row echelon matrix is therefore of the form (by adding and subtracting rows between them):
	
	where $j_1<j_2<...<j_r$ and $a_{1j_1},a_{2j_2},...,a_{rj_r}$ are nonzero terms. The terminal zero lines may be missing.
	
	The columns of index $j_1,j_2,...,j_r$ of an echelon matrix are clearly linearly independent. Considered as vectors-columns of $\mathbb{R}^2$, so they form a basis of this vector space. Considerating the other columns also as vectors-columns of $\mathbb{R}^n$, we deduce that they are necessarily linear combination of those of index  $j_1,j_2,...,j_r$ and therefore that the rank of the echelon matrix $M$ is $\text{rk}(A)=r$.
	
	We will note that $r$ is also the number of nonzero lines of the echelon matrix and also the rank of the lines of this matrix, since the nonzero lines are therefore clearly independent (we proved in the section Vector Calculus that the rank of lines and columns has same value with the same properties of independence).
	
	We can therefore allow ourselves to do a given number of elementary (extra) operations on the lines of matrices that we will be very useful, without changing its rank:
	\begin{enumerate}
		\item[P1.] We can swap the lines.
		\begin{tcolorbox}[title=Remark,colframe=black,arc=10pt]
		As we know the matrix can be seen just an aesthetic graphic representation of a linear system. So swap two lines does not change the system.
		\end{tcolorbox} 
		
		\item[P2.] Multiply a row by a nonzero scalar.
		\begin{tcolorbox}[title=Remark,colframe=black,arc=10pt]
		This obvioiusly not altering the linear independence of the vectors lines.
		\end{tcolorbox} 
		
		\item[P3.] Adding to an original line, a multiple of another line.
		\begin{tcolorbox}[title=Remark,colframe=black,arc=10pt]
		The original line will disappear in favor of the new that is independent of all (former) others. The system thus remains linearly independent.
		\end{tcolorbox} 
	\end{enumerate}
	
	Any matrix can be transformed into a row echelon form by a finite sequence of the previous properties here's how.
	\begin{enumerate}
		\item Pivot the matrix
		\begin{enumerate}
			\item Find the pivot, the first non-zero entry in the first column of the matrix.
			\item Interchange rows, moving the pivot row to the first row.
			\item Multiply each element in the pivot row by the inverse of the pivot, so the pivot equals $1$.
			\item Add multiples of the pivot row to each of the lower rows, so every element in the pivot column of the lower rows equals $0$.				
		\end{enumerate}
		\item To get the matrix in row echelon form, repeat the pivot.
		\begin{enumerate}
			\item Repeat the procedure from Step 1 above, ignoring previous pivot rows.
			\item Continue until there are no more pivots to be processed.
		\end{enumerate}
		\item To get the matrix in reduced row echelon form, process non-zero entries above each pivot.
		\begin{enumerate}
			\item Identify the last row having a pivot equal to 1, and let this be the pivot row.
			\item Add multiples of the pivot row to each of the upper rows, until every element above the pivot equals $0$.
			\item Moving up the matrix, repeat this process for each row.
		\end{enumerate}
	\end{enumerate}
	It is therefore obvious that the elementary operations on the rows of a matrix do not change the rank of the rows of the matrix. However, we know that the rank of lines of a row matrix matrix is equal to the rank of the columns, that is to say to the rank of this matrix (once again see the section Vector Calculus for the proof). We conclude that the column rank of any matrix of the type $m\times n$ is also equal to the rank of the rows of this matrix.
	
	As a corollary of this conclusion, it appears that:
	
	When solving linear systems of $m$ equation with $n$ unknowns it appears, as we have already noted at the beginning of this section (and with practical example in the section of Theoretical Computing), there there must be at least an equal number of equations than unknowns or more rigorously: the number of unknowns must be less or equal to the number of equations such as:
	
	
	\pagebreak
	\subsubsection{Matrix Algebra}
	Remember that we saw during our study of Vector Calculus that the algebraic operations of multiplication of a vector by a scalar, addition or subtraction of vectors between them and the operation of scalar product formed in the context of set theory a "vector space" (\SeeChapter{see section Set Theory page \pageref{vector space}}) possessing therefore a "vector algebraic structure". This under the condition that the vectors of course have the same dimensions (this observation, for recall, is not valid if instead of the scalar product we take the cross product).
	
	Just as vectors, we can multiply a matrix by a scalar and add (subtract) them together (as long as they have the same dimensions..) but in addition, we can also multiply two matrices together under certain conditions which we will define below. This will also make the set of matrices in the set-sense, a vector space on $K$ (being most of time $\mathbb{R}$) having also therefore a "vector algebraic structure".
	
	Thus, a vector may also be viewed as a particular matrix of dimension $m\times n$  and operate in the vector space of matrices. Basically..., vector calculus is only a special case of linear algebra!!! This is way at school people learn (after Calculus) Vector Calculus first and Linear Algebra later (and some will lean after Tensor Calculus).
	
	\begin{enumerate}
		\item[D1.] Given $A,B\in M_{mn} (\mathbb{R})$. We name "sum of $A$ and $B$" the matrix $C\in M_{mn}(\mathbb{R})$ whose coefficients are:
		
		That is to say explicitly:
		
		
		\item[D2.] Given $A\in M_{mn} (\mathbb{R})$ a matrix and a $\lambda\in \mathbb{R}$ a scalar (we can also take in $\mathbb{C}$ if we want). We name "\NewTerm{product of $A$ by $\lambda$}" the matrix whose coefficients are:
		
		That is to say explicitly:
		
		In two previous definitions so we can actually conclude that the space / set of matrices is a vector space and thus has a vector algebraic structure.
		
		\item[D3.] Let $E, F, G$ be three  vector spaces of basis respectively $\mathcal{E},\mathcal{F},\mathcal{G}$ and two linear application $f$ and $g$ (see the section Set Theory for a refresh).
		
		Let us denote by $A$ the matrix of $f$ with respect to basis $\mathcal{E},\mathcal{F}$, and $B$ the matrix of $g$ with respect to basis $\mathcal{F},\mathcal{G}$. Then the matrix $C$ of $g\cdot g$ (see the definition of a composite function in the section of Functional Analysis) relating to the basis $\mathcal{E},\mathcal{G}$ is equal to the product of $B$ by $A$ denoted simply by $BA$.
		
		So let $B\in M_{mn} \in \mathbb{R}$ and $A\in M_{np} \in \mathbb{R}$, we name "\NewTerm{matrix product}\index{matrix product}" or "\NewTerm{matrix multiplication}" of $A$ and $B$ and we denote it by $BA$, the matrix $C\in M_{mp}\mathbb{R}$ whose components are:
		
		It is important to notice that at the opposite to the addition, $A$ and $B$ may have different dimensions. However! the number of rows of $A$ must be equal to the number of columns of $B$, as indicated by the index $n$ of the two matrices. So in the product $BA$, if $B$ is a matrix $m\times n$, $A$ must be a matrix $n\times p$, for any $p$!!!
		
		Schematically:
		{\Huge{
		\[
		\framebox[2.5cm]{\clap{\raisebox{0pt}[1.5cm][1.5cm]{$\mat C$}}\subdims{-2.5cm} n p} =
		\framebox[1.5cm]{\clap{\raisebox{0pt}[1.5cm][1.5cm]{$\mat B$}}\subdims{-2.5cm} m n} \ 
		\framebox[2.5cm]{\clap{\raisebox{5mm}[1.5cm]{$\mat A$}}     \subdims{-1cm} n p} 
		\]}}\\\\
		
		or even more explicity (\NewTerm{Falk's scheme}\index{Falk's scheme}):
		\begin{figure}[H]
		\centering
			\includegraphics[scale=0.9]{img/algebra/falks_scheme.jpg}
			\caption[Matrix Product Falk's scheme]{Matrix Product Falk's scheme (credit: Alain Matthes)}
		\end{figure}
	\end{enumerate}
	By denoting with a capital Latin letters matrices and with lowercase Greek letters scalars, the reader can easily verify with what we have seen \underline{until now} (we can add proofs on request) the following properties of matrix algebra (the matrices are assumed to have adequate dimensions)\label{non-commutativity matrices}:
	\begin{enumerate}
		\item[P1.] Left distributivity: $A(B+C)=AB+AC$
		\item[P2.] Right distributivity: $(A+B)C=AC+BC$
		\item[P3.] Scaling association: $(\lambda A)B=A(\lambda A)$
		\item[P4.] Associativity: $(AB)C=A(BC)$
		\item[P5.] Non-commutativity: $BA\neq AB$
		\item[P6.] Absorbing-Element: $A\mathds{O}=\mathds{O}$
		\item[P7.] Neutral Element for addition: $A+\mathds{O}=A$
	\end{enumerate}
	It is especially important to remember of the property P5 that shows that the multiplication is obviously not commutative (for dimensions greater than $1$ of course!) and also the property P4 such that the matrix multiplication is associative.
	
	Concerning the general proof that the assertion of commutativity if false we must pass through a numerical example (because even the general case without replacing algebraic terms by numerical values will not show you much in our point of view...).
	
	\begin{tcolorbox}[title=Remark,colframe=black,arc=10pt]
	The set of square matrices $M_{nn}\in \mathbb{R}$ of order $n$ with components in $\mathbb{R}$ provided with the sum and the usual matrix multiplication forms a ring (\SeeChapter{see section Set Theory page \pageref{ring}}). This is true more generally if the coefficients of the matrices are taken in any ring: for example, all the matrices $M_{nn}\in \mathbb{Z}$  with integer components is a ring.
	\end{tcolorbox}
	One reader asked us to prove the property of associative. So let us begin!
	
	Let $A\in M_{kl}(\mathbb{C}),B\in M_{lm}(\mathbb{C}),C\in M_{mn}(\mathbb{C})$, then for $1\leq i\leq k,1\leq j\leq n$, we have well (we use the explicit expression of matrix components multiplication seen above many times as you can see):
	
	
	\subsubsection{Type of Matrices}
	To simplify the notations and length of calculations we introduce now the most common types of matrices that the reader will encounter throughout his reading of this book (and not just in the chapters on pure mathematics!).
	
	Some definitions will only be recalls!
	
	We denote by $M_{mn}(\mathbb{K})$ all matrices $m\times n$ whose coefficients take values in $K$ (typically $\mathbb{R}$ or $\mathbb{C}$ for example).
	
	\textbf{Definitions (\#\mydef):}
	\begin{enumerate}
		\item[D1.] When $m=n$, we say that $(a_{ij})$ is a "\NewTerm{square matrix of order $n$}\index{square matrix}":
			

		\item[D2.] We name "\NewTerm{zero matrix}\index{zero matrix}", and we denote by $0_{mn}$ or simply $\mathbf{0}$ or $\mathds{O}$ (this last notation is preferred in this book!) any matrix in which each term is zero:
			
		Null columns-matrices are designated by the symbol vector: $\vec{0}$.	
	
		\item[D3.] We name "\NewTerm{identity matrix of order $n$}\index{identity matrix}" or "\NewTerm{unit matrix of order $n$}\index{unit matrix}", and denote it by $I$, or simply $\mathds{1}$, the square matrix of order $n$:
		
		It can also be written using the Kronecker delta notation $\delta_{ij}$ (\SeeChapter{see section Tensors Calculus page \pageref{kronecker symbol}}).
		
		\begin{tcolorbox}[title=Remark,colframe=black,arc=10pt]
		Even if we use it actually for only two practical applications in this book (Principal Component Analysis and Multidimensional Scaling), and that we don't found it often in other textbooks, the latter definition must not be confuse we the matrix named "\NewTerm{matrix of ones}\index{matrix of ones}", and defined for a square $n\times n$ matrix by:
		
		\end{tcolorbox}
		
		\item[D4.] We name "\NewTerm{diagonal matrix}\index{diagonal matrix}" any square matrix $A\in M_{nn}\in\mathbb{C}$ which only the diagonal has non-null elements:
		
		Formally:
		
		The usual notation of a diagonal matrix is:
		
		
		\item[D5.] We name "\NewTerm{lower triangular matrix}\index{lower triangular matrix}" a square matrix if all the entries above the main diagonal are zero:
		
		Formally:
		 
		 Similarly, a square matrix is named "\NewTerm{upper triangular matrix}\index{upper triangular matrix}" if all the entries below the main diagonal are zero:
		 
		Formally:
		 
		A "\NewTerm{triangular matrix}\index{triangular matrix}" is one that is either lower triangular or upper triangular. A matrix that is both upper and lower triangular is named a "diagonal matrix".
		
		If an upper triangular matrix was obtained by a "staggered" matrix we will write it as following:
		
		
		\item[D6.] A "\NewTerm{Toeplitz matrix}\index{Toeplitz matrix}\label{Toeplitz matrix}" or diagonal-constant matrix (named after Otto Toeplitz), is a matrix in which each descending diagonal from left to right is constant. For instance, the following matrix is a Toeplitz matrix:
		
		Any $n\times n$ matrix $A$ of the form:
		
		is a Toeplitz matrix. If the $i,j$ element of $A$ is denoted $A_{i,j}$, then we have:
		
		We will meet such a matrix in our study of time series analysis and especially when we will prove the Yule-Walker equations (\SeeChapter{see section Social Sciences page \pageref{Yule-Walker equations}}).
		
		\item[D7.] Given $M_{nn}$ a square matrix. The matrix $M_{nn}$ is named "\NewTerm{invertible matrix}\index{invertiable matrix}" or "\NewTerm{regular matrix}\index{regular matrix}" or "\NewTerm{non-singular matrix}\index{non-singular matrix}" if and only if $M_{nn}^{-1}$ is such that:
		
		If this is not the case, we say that $M_{nn}$ is a "\NewTerm{singular matrix}\index{singular matrix}". We will prove later that for a square matrix to invertible (non-singular) on sufficient condition is that its determinant is equal to zero.
		
		This definition is fundamental, it has extremely important consequences in all linear algebra and also in physics (solving linear systems, determining, eigenvectors and eigenvalues, etc.), statistics and finance and it is appropriate to remember it.
		
		Let us prove a useful property of invertible matrices associated with the property of associativity of matrix multiplication\label{inverse matrix property}:
		
		Indeed:
		
		
		\item[D8.] Given a matrix $A_{mn}:=A$:
		
		We name "\NewTerm{transposed matrix}\index{transposed matrix}\label{transposed matrix}" of $A=A_{mn}$, the matrix denoted by $A^T=A_{nm}$  (the superscript $T$ is depending of the books and teachers uppercase or lowercase and either left or right but the standard ISO 80000-2:2009 recommends the capital and superscript on the top right\footnote{Sadly many textbooks, especially from USA denoted the transpose with a prime: $A'$}) the matrix for which we transpose the rows into columns and the into rows:
		
		Here are some interesting properties of the transpose of a matrix (which we will be useful to us later in this section for a famous theorem and also in the study of the multiple linear regression methods in the section of Numerical Methods!):
		\begin{enumerate}
			\item[P1.] $(A^T)^T$
			\item[P2.] $(\lambda A+B)^T=\lambda A^T+B^T,\lambda\in \mathbb{R},\lambda\in \mathbb{R}$
			\item[P3.] $(AB)^T=B^TA^T$
			\item[P4.] $(A^{-1})^T=(A^T)^{-1},\exists A^{-1}$
			\item[P5.] $A\vec{x}\circ\vec{y}=\vec{x}A^T\vec{y}$
		\end{enumerate}
		The transposed matrix is very important in physics, statistics and finance and obviously in the field of mathematics for example in the context of the theory of groups and symmetries! So it is also worth remembering its definition.
		
		As the third property is the most used one in the various sections of this book let us demonstrate it by considering $A\in M_{lm}\in\mathbb{C},B\in M_{mn}\in \mathbb{C}$:
		\begin{dem}
		 Remembering the explicit relation of matrix multiplication seen earlier:
		
		But in this last equality, we note that we browse $B$ by line and $A$ in column for a $i$ and a $j$ fixed and this we know then corresponds to the matrix multiplication $AB$, therefore:
		
		Finally we have well:
		
		\begin{flushright}
			$\blacksquare$  Q.E.D.
		\end{flushright}
		\end{dem}
		And for the same reasons let us prove the before last property.
		\begin{dem}
		First, it is trivial that if $A$ is invertible:
		
		and taking the transpose on both sides of the equality we find (we use the property proved just before):
		
		The latter equality shows obviously that $(A^{-1})^T$ is the inverse of $A^T$, that is to say:
		
		\begin{flushright}
			$\blacksquare$  Q.E.D.
		\end{flushright}
		\end{dem}
		Finally, a last and important property of transposed matrices, is that given a matrix $A$ (square or not, symmetric or not), the multiplication by it's own transposed gives a symmetric matrix. 
		
		The proof is quite straightforward and very useful of the SVD (Singular Value Decomposition) theorem:
		
		
		\item[D9.] Given:
		
		a matrix of $M_{mn}(\mathbb{C})$. We name "\NewTerm{adjoint matrix}\index{adjoint matrix}\label{adjoint matrix}" of $A$, the matrix of $M_{mn}(\mathbb{C})$ defined by:
		
		which is the complex conjugate of the transposed matrix or if you prefer ... the transposed matrix of the complex conjugate (in the case of real components... we obviously don't need to take the conjugate!). To simplify the notations we simply note this matrix $A^\dagger$ (notation frequently use in Quantum Physics and Set Algebra).
		\begin{tcolorbox}[title=Remark,colframe=black,arc=10pt]
		Trivial relation (which is often used in Quantum Field Physics) already prove juste before and obviously right when the component are in $\mathbb{C}$:
		
		\end{tcolorbox}
		
		\item[D10.] By definition, a matrix is named "\NewTerm{Hermitian matrix}\index{Hermitian matrix}" or "\NewTerm{self-adjoint matrix}\index{self-adjint matrix}\label{self-ajdoint matrix}"... if it is equal to its own adjoint (conjugate transpose matrix) such that:
		
		
		\item[D11.] Given $A$ as square matrix $M_{nn}(\mathbb{R})$, the "\NewTerm{trace}\index{trace of a matrix}" of $A$ denoted $\text{tr}(A)$ is defined by the sum of the terms of the diagonal (very useful in some statistical techniques):
		
		Some useful related relations (we can add the detailed proof on the demand of the readers):
		
		and:
		

		\item[D12.] A matrix $A$ is named "\NewTerm{nilpotent matrix}\index{nilpotent matrix}" if by multiplying it successively by itself it can give zero. Explicitly, if there exists an integer $k$ such that:
		
		If the matrix $A$ multiplied by itself gives $A$ ... then we talk about an "\NewTerm{idempotent matrix}\index{idempotent matrix}".
		
		Such matrices are for example very common in Markov Chains when the transition matrix contains probabilities (see the sections of Probabilities and Graph Theory).
		\begin{tcolorbox}[title=Remark,colframe=black,arc=10pt]
		To remember this name, we can decompose it into "nil" that means "zero" and "potent" that means "potential". So something "nilpotent" is therefore something that is potentially zero....
		\end{tcolorbox}
		
		\item[D13.] A matrix $A$ is named "\NewTerm{orthogonal matrix}\index{orthogonal matrix}\label{orthogonal matrix}" if its elements are real and if it obeys to:
		
		which can be translate into (where $\delta_{ij}$ the Kronecker symbol):
		
		The matrix column vectors $A_i$ are thus orthogonal to each other as the resulting operation above can be seen as a row-column dot product such that for a $n\times n$ squared matrix:
		
		Therefore an orthogonal matrix represents also an orthogonal basis!
		
		\begin{tcolorbox}[title=Remarks,colframe=black,arc=10pt]
		When numerical data are stored in the usual waay in a matrix $X$, the matrix $X^TX$, and its properties often plays an important role in statistical analysis and especially in regression techniques. A matrix of this form is named a "\NewTerm{Gramian matrix}\index{Gramian matrix}".
		\end{tcolorbox}
		
		A typical mathematical example is the matrix of the canonical orthonormal basis (\SeeChapter{see section Vector Calculus page \pageref{canonical basis}}):
		
		or a well known matrix in quantum physics (\SeeChapter{see section Quantum Computing page \pageref{hadamard quantum gate}}):
		
		\begin{tcolorbox}[title=Remarks,colframe=black,arc=10pt]
		\textbf{R1.} Therefore this is typically the case of the canonical basis matrix, or any diagonalized matrix.\\
		
		\textbf{R2.} If instead of just taking a matrix with real coefficients, we take complex coefficients with its complex transposed matrix (adjoint matrix). So we say (sadly... because it makes confusion with the name of another matrix already define) that $A$ is a "unitary matrix" if it satisfies the previous relation!
		\end{tcolorbox}
		We will come back later, after having introduced the concepts of eigenvectors and eigenvalues, a particular and very important case of orthogonal matrices (named "translations matrices").
		
		Let us also mention another important property in geometry, physics and statistics of orthogonal matrices.
		
		\begin{theorem}
		Given $f(\vec{x})=A\vec{x}+\vec{b}$, where $A$ is an orthgonal matrix and $\vec{b}\in \mathbb{R}^n$. Then $f$ (respectively $A$) is an isometry. That is to say:
		
		So in other words: Orthogonal matrices are linear mappings which preserve the norm (the distance)!!!
		\end{theorem}
		\begin{dem}
		Remembering that $\vec{x}\circ\vec{y}$ is the same in linear algebra notation as $\vec{x}^T\vec{y}$ we then have:
		
		and we have well:
		
		\begin{flushright}
			$\blacksquare$  Q.E.D.
		\end{flushright}
		\end{dem}
		
		\item[D14.] Given a square matrix $A \in M_{nn}$. The matrix $A$ is named "\NewTerm{symmetric matrix}\index{symmetric matrix}\label{symmetric matrix}" if and only if:
		
		We will meet this definition again in the section of Tensor Calculus.
		
		\item[D15.] Given a square matrix $A \in M_{nn}$. The matrix $A$ is named "\NewTerm{antisymmetric matrix}\index{antisymmetric matrix}" (also sometimes named "\NewTerm{skew-symmetric matrix}"...) if and only if:
		
		which requires that:
		
		In electromagnetism the electromagnetic field tensor has components which can be written as an antisymmetric $4\times 4$ matrix. In classical mechanics, a solid's propensity to spin in various directions is described by the inertia tensor which is symmetric. The energy-momentum tensor from electrodynamics is also symmetric. Most metric in Einstein field equations in General Relativity are symmetric but we can also build antisymmetric metrics. Matrices are everywhere if look for them.
		\begin{tcolorbox}[title=Remark,colframe=black,arc=10pt]
		We will prove in the section of Tensor Calculus (page \pageref{decomposition square matrix symmetric and antisymmetric}) that every square matrix are formed by the sum of a symmetric and antisymmetric matrix.
		\end{tcolorbox}
		
		
		\item[D16.] Let $E$ be a vector space of dimension $n$ and $ \mathcal{B},\mathcal{B}'$ two basis of $E$:
		
		We name "\NewTerm{transition matrix}\index{transition matrix}" of the basis $\mathcal{B}$ to the basis  $\mathcal{B}'$, and we denote by $P$ the square matrix of $M_{nn}(\mathbb{K})$ which columns are formed of components of vectors of the basis $\mathcal{B}'$ on the basis $\mathcal{B}$ (see further below the detailed treatment of basis changes for more information).
		
		We consider the vector $\vec{x}(x_1,x_2,...,x_n)$ of $E$ which is written in the basis $\mathcal{B}(\vec{e}_1,\vec{e}_2,...,\vec{e}_n)$ and $\mathcal{B}'(\vec{e}_1^{\prime},\vec{e}_2^{\prime},...,\vec{e}_n^{\prime})$ following the relations:
		
		With:
		
		the vector of $K^n$ formed of the components of $\vec{x}$ in the basis $\mathcal{B}$ and of the vector $\vec{x}$ formed of the components in the basis $\mathcal{B}^{\prime}$. So:
		
		relation for which the detailed proof will be given later in our study of basis changes. We also have obviously:
		
		\begin{tcolorbox}[title=Remarks,colframe=black,arc=10pt]
		\textbf{R1.} When a vector is given and its basis is not specified, remember that it is therefore implicitly in the canonical basis:
		
		which remains invariant by the multiplication by any vector and when the basis used is denoted by $(\vec{e}_i)$ and is not specified, then it is also that of the canonical basis.\\
		
		\textbf{R2.} If a vector is given relative to the canonical basis, its components are named "\NewTerm{covariant}\index{covariant components}\label{covariant components}",  if they are expressed in another noncanonical base, then we say that the components are "\NewTerm{contravariant}\index{contravariant components}" (for details on the subject see the section of Tensor Calculus).
		\end{tcolorbox}
		
		\item[D17.] A matrix is named "\NewTerm{positive-definite matrix}\index{positive-definite matrix}\label{positive definite matrix}" (which will be useful in the section of Theoretical Computing for some important engineering techniques and in quantitative finance for qualitative estimation of the correlation matrix) if:
		
		or more simply if $\vec{x}\neq \vec{0}$:
		
		and "\NewTerm{semi-positive matrix}\index{semi-positive matrix}" or "\NewTerm{positive-semidefinite matrix}\index{positive-semidefinite matrix}\label{positive semidefinite matrix}" if:
		
		We have already proved that a semi-positive matrix has its eigenvalues which are all positive OR null, while if it is positive definite its eigenvalues are all positive AND not null (\SeeChapter{see section Analytical Geometry page \pageref{classification of conical by the determinant}}).
		
		Notice that this implies that if a matrix $A$ is invertible, then $B=A^TA$ is positive definite.

		Indeed (even if we have already indirectly seen this proof in the section of Statistics), $B$ is positive definite if for every non-zero $\vec{x}$ we have:
		
		Now if we have $A$ such that $B=A^TA$, then using the properties of transposed matrices:
		
		But:
		
		If $A$ in invertible, $\|A\vec{x}\|=0$ only when $\vec{x}=0$.
		
		\label{inverse definite positive matrix is also definite positive} Let us prove also that the inverse of a positive definite matrix is also positive definite (important property for the Mahalonobis distance seen at page \pageref{Mahalanobis distance} for example!).

		First, let us recall that if $A$ is positive definite, then we have proved (\SeeChapter{see section Statistics page \pageref{positive semi-definite matrix not always invertible}}) that it's determinant is never null and that therefore it's always invertible. 
		
		Secondly, we will prove further below at page \pageref{spectral theorem} during our study of spectral theorem that for a definite positive matrix $M$ we have:
		
		where $\lambda_i$ are the eigenvalues of $i$ (positive and non-null for definite positive matrices!). But also further below during our study of determinant at page \pageref{inverse determinant}, we will prove for a matrix $M$ that:
		
		Therefore:
		
		And as for a definite positive matrix we have all $\lambda_i>0$, then $M^{-1}$ is also positive definite since all $1/\lambda_i>0$ and hence $\det(M^{-1})$ is also strictly positive.
		
		We can extend the above definition so that in general a matrix $M$ is said to by:
		\begin{itemize}
			\item "\NewTerm{positive definite}" if for any vector $\vec{x}\neq\vec{0}$, $\vec{x}^TM\vec{x}>0$. This is sometimes denoted $M\succ \vec{0}$
			\item "\NewTerm{positive semidefinite}" if $\vec{x}^TM\vec{x}\geq 0$. This is sometimes denoted $M\succeq \vec{0}$
			\begin{itemize}
			\item "\NewTerm{nonnegative definite}" if it is either positive definite or positive semi definite\footnote{Nonnegative definite and positive semidefinite are the same but as you can reat the expression from time to time in some textbooks...}
			\end{itemize}
			
			\item "\NewTerm{negative definite}" if for any vector $\vec{x}\neq\vec{0}$ $\vec{x}^TM\vec{x}<\vec{0}$. This is sometimes denoted $M\prec \vec{0}$
			\item "\NewTerm{negative semidefinite}" if $\vec{x}^TM\vec{x}\leq\vec{0}$. . This is sometimes denoted $M\preceq \vec{0}$
			\begin{itemize}
				\item "\NewTerm{nonpositive definite}" if it is either negative definite or negative semi definite
			\end{itemize}
			\item "\NewTerm{indefinite}" if it is nothing of those.
		\end{itemize}
		
		\item[D18.] The "\NewTerm{adjugate}\index{adjugate}\label{adjugate}", "\NewTerm{classical adjoint}\index{classical adjoint}", or "\NewTerm{adjunct}\index{adjunct}" of a square matrix is the transpose of its cofactor (see further below page \pageref{cofactor}) matrix:
		
		
		\begin{tcolorbox}[colframe=black,colback=white,sharp corners]
		\textbf{{\Large \ding{45}}Example:}\\\\
		Consider a $3\times 3$ matrix:
		
		Its cofactor matrix is:
		
		where:
		
		Its adjugate is the transpose of its cofactor matrix:
		
		\end{tcolorbox}
			
		\item[D19.] A symmetric square matrix having all it components being positive and only zeros on the diagonal is named a "\NewTerm{distance matrix}\index{distance matrix}\label{distance matrix}" or even sometimes a "\NewTerm{pairwise similarity matrix}\index{pairwise similarity matrix}" (we will meet several times this type of matrix in Data Mining techniques during our study of the Numerical Methods section). Most of times this matrix contains the distances, taken pairwise, between the elements of a set (indexed by the row and column number of the corresponding matrix). Depending upon the application involved, the distance being used to define this matrix may or may not be a metric.
		
		A famous type of such matrix is the Euclidean distance matrix. If $D_x^2$ is a Euclidean distance matrix and the points $x_{1},x_{2},\ldots ,x_{n}$ are defined on $m$-dimensional space, then the elements of $A$ are given by:
	
	where $\lVert \ldots \rVert _{2}$ denotes the $2$-norm on $\mathbb{R}^m$ and:
	

		
		\item[D20.] A matrix is named "\NewTerm{sparse matrix}\index{sparse matrix}" if it contains a significant number of null values. In numerical methods, there are algorithms that use this specificity to optimize the storage of this type of matrices (used in OLAP\footnote{Acronym of OnLine Analytical Processing.} cubes and financial engineering).
		
		The above sparse matrix contains only $9$ nonzero elements, with $26$ zero elements. Its sparsity is $74\%$, and its density is $26\%$.
		
		\item[D21.] A "\NewTerm{block matrix}\index{block matrix}"(also called partitioned matrix) is a matrix of the kind:
		
		where $A$, $B$, $C$ and $D$ are matrices, called "\NewTerm{blocks}", such that:
		\begin{itemize}
			\item $A$ and $B$ have the same number of rows
		
			\item $C$ and $D$ have the same number of rows
		
			\item $A$ and $C$ have the same number of columns
		
			\item $B$ and $D$ have the same number of columns
		\end{itemize}
		Ideally, a block matrix is obtained by cutting a matrix two times: one vertically and one horizontally. Each of the four resulting pieces is a block.
		
		\begin{tcolorbox}[colframe=black,colback=white,sharp corners]
		\textbf{{\Large \ding{45}}Examples:}\\\\
		E1. The matrix:
		
		can be written as a block matrix:
		
		where:
		
		E2. The matrix:
		
		can be written as a block matrix:
		
		where:
		
		\end{tcolorbox}
		An important fact about block matrices is that their multiplication can be carried out as if their blocks were scalars, by using the standard rule for matrix multiplication:
		
		The only caveat is that all the blocks involved in a multiplication (e.g., $AE$, $BG$, $CE$) must be conformable. For example, the number of columns of $A$ and the number of rows of $E$ must coincide.
		
		Many proofs in linear algebra are greatly simplified if one can easily deal with the determinants of block matrices, that is, matrices that are subdivided into blocks that are themselves matrices.
	\end{enumerate}
	
	\subsubsection{Determinant}\label{determinant}
	We will look at determinants in the point of view of the physicist or of the engineer (the mathematician point of view is rather off-putting ...). In physics (whether in classical mechanics and quantum field physics), chemistry or engineering, we frequently have to solve linear systems. But we have now seen that a linear system:
	
	can be written as:
	
	and we know that the only soluble linear systems (in the sense that they have a unique solution !!!) are those that have as many equations as unknowns and their determinant is not zero! Thus, the matrix must be a square matrix $M_{mm}$.
	
	If a solution exists, then there is a column matrix (or "vector") $X$ such that $AX=B$ which involves:
	
	What imposed this relation? Well this is relatively simple, but at the same time very important: for a linear system to have a unique solution, it is necessary that the matrix is invertible (non-singular)! What relation with the  concept of "determinant" then? It's simple: mathematicians have sought how to write inverse matrices of linear systems for which they knew there was a unique solution and they arrived after trial and error to determine a kind of formula to assess if the matrix is invertible (non-singular) or not. Once this formula found, they formalized (as they know so well how to do it...), with a great rigor, the concept surrounding this formula that they named the "\NewTerm{determinant}\index{determinant}". They did it so well that in fact we sometimes forgot that they have found it by trial and error...
	\begin{tcolorbox}[title=Remark,colframe=black,arc=10pt]
	If a matrix of a linear system is not invertible (non-singular), this has the consequence that there is no solution or an infinity of solutions (as usual what ...)
	\end{tcolorbox}
	We below first focus on how to build the determinant bydefining a particular type of application. Then, after seeing a simple and readable example of the calculation of a determinant, we will focus on determining the formula of it in the general case. Finally, once this is done, we will see what is the relation between the inverse of a matrix and the determinant!!!
	
	In what will follow all vector spaces will be considered of finite dimension and the field of complex numbers $\mathbb{C}$ (those who prefer the can take $\mathbb{R}$ as basis field, in fact we could take any field).
	
	First of all we will do a little bit of pure math (a bit off putting) before moving on  concrete stuff.
	
	Given $V$ a vector space, we write will write as usual $V^n$ instead of $V\times V\times... \times V$. $(\vec{e}_1,\vec{e}_2,...,\vec{e}_3)$ designate the canonical basis of $\mathbb{R}^n$. $M_n(\mathbb{R})$ is the set of square matrices $n\times n$ with coefficients in $\mathbb{R}$.
	
	\textbf{Definitions (\#\mydef):}
	\begin{enumerate}
		\item[D1.] A "\NewTerm{multilinear application}\index{multilinear application}" on a space $V$ is defined by an $\varphi: V^n \rightarrow \mathbb{R}$ which is linear in each of its components. Meaning:
		
		for any $\lambda,\mu\in \mathbb{K}$ and $\vec{x}_i,\vec{u},\vec{v}\in V$ where the $\vec{x}_i$ are vectors.
		\begin{tcolorbox}[title=Remark,colframe=black,arc=10pt]
		A non-null multilinear application is not a linear application of the space $V^n$ in $\mathbb{R}^n$. Excepted if $n=1$. Indeed, this can be verified by the definition of a linear application  versus the one of the multilinear application:
		
		\end{tcolorbox}
		\item[D2.] An "\NewTerm{alterneted multilinear application}\index{alterneted multilinear application}" on $V$ is by definition a multilinear application that satisfies:
		
		for any $j=1...n,\vec{x}_j\in V$. Therefore the permutation of two vectors that follows change the sign of $\varphi$.
		\begin{theorem}
		Therefore, if $\varphi$ is a multilinear alterneted application, then $\varphi$ is multilinear if and only if $\forall \vec{j}\in V,j=1...n$ we have:
		
		or in a more simple case:
		
		\end{theorem}
		\begin{dem}
		If $\varphi$ is alternated we have by definition:
		
		Therefore by rearranging:
		
		And if $\varphi$ is a multilinear application we can write:
		
		\begin{flushright}
			$\blacksquare$  Q.E.D.
		\end{flushright}
		\end{dem}
		Now comes the interesting stuff:
		
		\item[D3.]  A "\NewTerm{determinant}\index{determinant}" is by definition a multilinear alterneted application:
		
		satisfying as well:
		
		\begin{tcolorbox}[title=Remark,colframe=black,arc=10pt]
		The columns of a matrix form $n$ vectors and we then see that a determinant $D$ on $\mathbb{R}^n$ inducte an application $D$ of $M_n(\mathbb{R})\mapsto \mathbb{R}$ (where ase we know $M_n(\mathbb{R})$ is the set of squatre matrices $n\times n$ with components in $\mathbb{R}$) defined by:
		
		where $\vec{m}_i$ is the $i$-th column of $M$. 
		\end{tcolorbox}	
	\end{enumerate}
	Let us study the case $n=2$. If $D$ is a determinant, for any vector:
	
	we have:
	
	As $D$ is multilinear, we have:
	
	as it is alterneted:
	
	it remains:
	
	and we have:
	
	And finally:
	
	In fact, we just prove that if the a determinant exists, it is unique and of the form indicated previously, we should also check that the defined application satisfy the properties of a determinant, but the latter is immediate.
	
	Thus, if:
	
	is a matrix we have then (notice the three common notations for the determinant that you can found in various textbooks and even in the actual one depending on the context and of the traditions):
	
	Let us give now a geometric interpretation of the determinant. Given $\vec{v}_1,\vec{v}_2$ two vectors of $\mathbb{R}^2$:
	\begin{figure}[H]
		\centering
		\includegraphics{img/algebra/determinant_parallelelogram.jpg}
		\caption{Geometric interpretation of the determinant}
	\end{figure}
	The vector $\vec{w}$ is obtained by projecting $\vec{v}_1$ on $\vec{v}_2$ and we have therefore:
	The vector $\vec{w}$ is obtained by projecting $\vec{v}_1$ on $\vec{v}_2$ and we have therefore:
	
	The area of the above parallelogram is therefore:
		
	if:
	
	then:
	
	and finally:
	
	Therefore the determinant, in absolute value, represents the area of the parallelogram defined by the vectors $\vec{v}_1,\vec{v}_2$ when these vectors are linearly independent. We can generalize this result to a $n$-dimensional space, in particular, for $n=3$, the determinant of three linearly independent vectors represents the volume of the parallelepiped defined by these as we already prove it during our study of the mixed product in the section of Vector Calculus page \pageref{mixed product}.
	
	The more general case of the expression of the determinant is a little trickier to ascertain. This requires that we define a particular bijective application but simple that we have already met in the section Statistics.
	
	\textbf{Definition (\#\mydef):} Given $F_n=\left\lbrace 1,2,...,n\right\rbrace,n\in \mathbb{N}^{*}$ we name "\NewTerm{permutation}\index{permutation}" of $F_n$ any bijective application of $F_n$ in $F_n$:
	
	Given $\mathcal{S}_n$ the set of possible permutations (bijective applications) of $\left\lbrace 1,2,...,n\right\rbrace$.  $\mathcal{S}_n$ obviously contains ... (see our study of Combinatorics in the section of Probabilities) $n!$ elements. The information an element $\sigma$ of $\mathcal{S}_n$ is defined by the successive information of:
	
	Given an ordered sequence of elements (ascending) $\left\lbrace 1,2,...,n\right\rbrace,n \in \mathbb{N}^{*}$ we name "inversion", any permutation of elements in the ordered sequence (so the result will not be ordered at all...). We denote by $I(\sigma)$ the number of inversions.
	
	We say that the permutation $\sigma$ is even (odd) if $I(\sigma)$ is even (odd). We name "\NewTerm{signature}\index{signature of a matrix}" of $\sigma$, the number $\varepsilon(\sigma)$ defined by $\varepsilon(\sigma)=(-1)^{I(\sigma)}$, that is to say:
	
	We now have the necessary tools to set up the general relation of the determinant:
	
	\textbf{Definition (\#\mydef):} Given:
	
	We name "\NewTerm{determinant of a square matrix $A$}\index{determinant of a square matrix}" of dimension $n$, and we denote by $\det (A)$ (or sometimes by $D(A)$ or especially in the field of Statistics by $|A|$), the scalar defined by (we'll see a few examples further below):
	
	sometimes named "\NewTerm{Leibniz formula of determinants}\index{Leibniz formula of determinants}\label{leibniz formula}" or "\NewTerm{Laplace's formula}\index{Laplace's formula}". This relation seems to have been obtained in the past by trial and error and by induction for larger dimensions.
	
	\pagebreak
	\begin{tcolorbox}[colframe=black,colback=white,sharp corners]
	\textbf{{\Large \ding{45}}Examples:}\\\\
	E1. Given $A=(a_{ij})_{1\leq i,j\leq 2}\in M_{22}(\mathbb{K})$, let us consider the $2!=2$ permutations of the second indices (of integers $1,2$) taken in their whole\label{determinant of two by two matrix}:
	
	We calculate the signature of $\sigma$. Here is the scheme of this rule (recall: we say that there is an "inversion", if in a permutation, a greater integer precede a smallest integer):
	\begin{table}[H]	
		\begin{center}
			\begin{tabular}{|c|c|c|}
			\hline
			  \rowcolor[gray]{0.75}$\sigma(1)\sigma(2)$ & $\underbrace{12}_{\sigma(1)\sigma(2)}$ & $\underbrace{21}_{\sigma(1)\sigma(2)}$ \\
			  \hline
			  \cellcolor{black!30}Number of inversions & $0$ & $1$ \\\hline
			  \cellcolor{black!30}Permutation & even & odd  \\\hline
			  \cellcolor{black!30}$\varepsilon(\sigma)=(-1)^{I(\sigma)}$ & $+1$ & $-1$ \\\hline
			\end{tabular}
		\end{center}
		\caption{Inversions and permutations of a determinant of order $2$}
	\end{table}
	Therefore we have:
	
	This corresponds well to what we saw initially. Remember also on the way that we will soon prove that the determinant of a square matrix must be zero so that the matrix is invertible (non-singular)!\\
	
	E2. Given $A=(a_{ij})_{1\leq i,j\leq 3} \in M_{33}(\mathbb{K})$, let us consider the $3!=6$ permutations of the second indices (integers $1,2,3$) taken in their whole\label{determinant of three by three matrix}:
	\begin{align*}
	123 \quad 132 \quad 213 \quad 31 \quad 312 \quad 321
	\end{align*}
	We calculate the signatures of $\sigma$. Here is a scheme of this rule (recall: we say that there is an "inverse", if in a permutation, a greatest integer precedes a lower integer):
	\begin{table}[H]	
		\begin{center}
			\begin{tabular}{|c|c|c|c|c|c|c|}
			\hline
			  \rowcolor[gray]{0.75}$\sigma(1)\sigma(2)$ & $123$ & $132$ & $213$ & $231$ & $312$ & $321$\\
			  \hline
			  \cellcolor{black!30}Number of inversions & $0$ & $1$ & $1$ & $2$ & $2$ & $3$\\\hline
			  \cellcolor{black!30}Permutation & even & odd & odd & even & even & odd  \\\hline
			  \cellcolor{black!30}$\varepsilon(\sigma)=(-1)^{I(\sigma)}$ & $+1$ & $-1$ & $-1$ & $+1$ & $+1$ & $-1$\\\hline
			\end{tabular}
		\end{center}
		\caption{Inversions and permutations of a determinant of order $3$}
	\end{table}
	\end{tcolorbox}
	
	\pagebreak
	
	\begin{tcolorbox}[colframe=black,colback=white,sharp corners]
	Therefore we have\label{3x3 matrix determinant}:
	
	From the above formula, each term is in the form of $a_{1j_1}a_{2j_2}a_{3j_3}$ where the second indices $j_1,j_2,j_3$ is a permutation of $1,2,3$. For example, the first term has $1,2,3$ as the second indices; the second term has $1,3,2$; the third term has $2,1,3$, etc.\\
	
	The sign of each term is the sign of the permutation. For example, the sign of $1,2,3$ is clearly $1$. For the second term, $1,3,2$ is obtained from $1,2,3$ by one transposition $2,3\rightarrow3,2$, so the sign is $-1$, etc.
	\end{tcolorbox}
	\begin{tcolorbox}[title=Remark,colframe=black,arc=10pt]
	Some people learn by heart a method named "\NewTerm{rule of Sarrus}\index{rule of Sarrus}" to calculate the determinants of order three as the previous one given by:
	\begin{center}
	\begin{tikzpicture}
    \matrix [%
      matrix of math nodes,
      column sep=1em,
      row sep=1em
    ] (sarrus) {%
      a_{11} & a_{12} & a_{13} & a_{11} & a_{12} \\
      a_{21} & a_{22} & a_{23} & a_{21} & a_{22} \\
      a_{31} & a_{32} & a_{33} & a_{31} & a_{32} \\
    }; 

    \path ($(sarrus-1-3.north east)+(0.5em,0)$) edge[dotted] ($(sarrus-3-3.south east)+(0.5em,0)$)
          (sarrus-1-1)                          edge         (sarrus-2-2)
          (sarrus-2-2)                          edge         (sarrus-3-3)
          (sarrus-1-2)                          edge         (sarrus-2-3)
          (sarrus-2-3)                          edge         (sarrus-3-4)
          (sarrus-1-3)                          edge         (sarrus-2-4)
          (sarrus-2-4)                          edge         (sarrus-3-5)
          (sarrus-3-1)                          edge[dashed] (sarrus-2-2)
          (sarrus-2-2)                          edge[dashed] (sarrus-1-3)
          (sarrus-3-2)                          edge[dashed] (sarrus-2-3)
          (sarrus-2-3)                          edge[dashed] (sarrus-1-4)
          (sarrus-3-3)                          edge[dashed] (sarrus-2-4)
          (sarrus-2-4)                          edge[dashed] (sarrus-1-5);

    \foreach \c in {1,2,3} {\node[anchor=south] at (sarrus-1-\c.north) {$+$};};
    \foreach \c in {1,2,3} {\node[anchor=north] at (sarrus-3-\c.south) {$-$};};
  \end{tikzpicture}
	\end{center}
	We prefer in this book the general formulation of the determinant because applicable to all orders.
	\end{tcolorbox}
	Let's us see some properties and corollaries of this formulation of the determinant:
	\begin{enumerate}
		\item[P1.] Given a square matrix of order $n$, we do not change the value of its determinant if:
		\begin{enumerate}
			\item By performing an elementary operation on the columns of $M_n$.

			\item By performing an elementary operation on the rows of $M_n$.
		\end{enumerate}
		\begin{dem}
		If $M_n=(a_{ij})_{i,j=1,...,n}$ the $M_n$ is composed of $n$ column vectors:
		
		Doing an elementary operation on the columns of $M_n$ is equivalent to add $\lambda v_i,i \in \{1,...,n\}$ to one of the columns $v_n$ of $M_n$. Given $M_n^{\prime}$ the matrix obtained by adding $\lambda v$ to the $j-th$ column of $M_n$, we get:
		
		By multilinearity (finally the proof in not really difficult):
		
		and as the determinant is alternated:
		
		About the elementary operations on the rows we just need to consider the transpose (that is to cry such it is simple, but we had to thing about this trick).
		\begin{flushright}
			$\blacksquare$  Q.E.D.
		\end{flushright}
		\end{dem}
		
		\item[P2.] Given $M_n(\mathbb{K})$ a squared matrix of order $n$ and given $\lambda \in \mathbb{K}$:
		
		\begin{dem}
		As before, it is enough to simply noticed that if $v_1,...,v_n$ are the column vectors forming the matrix $M_n$ then $\lambda v_1,...,\lambda v_n$ are those that constitute $\lambda M_n$ and:
		
		The application being $n$-linear, we arrive at the equality:
		
		\begin{flushright}
			$\blacksquare$  Q.E.D.
		\end{flushright}
		\end{dem}
		
		\item[P3.] Given is a square matrix of order $n$. We change the sign of the determinant of $M_n$ if:
		\begin{itemize}
			\item We permute two of its columns
	
			\item We permute two of its rows
		\end{itemize}
		\begin{dem}
		$M_n$ is constituted by $n$ vectors $v_1,..,v_n$. The determinant of $M_n$ is equal to the determinant of these $n$. Permute two columns of $M_n$ is same as permuting the two corresponding vectors. Let us suppose that the permuted vectors are the $i$-th and $j$-th, the determinant being an alternate application, we have:
		
	About the rows, we just have to consider the transposed of $M_n$ to arrive to the same result!
		\begin{flushright}
			$\blacksquare$  Q.E.D.
		\end{flushright}
		\end{dem}
		
		\item[P4.] Given $A,B\in M_n\in (\mathbb{C})$ then:\label{determinant product}
		
		As far as we know the proof can be done in at least two ways, the first is rather indigestible and abstract... so we will let it to mathematicians (...) even if it has the advantage of being general, the second one (much more easier) is to check this assertion for various squared matrices (the way engineers would do it...):
		\begin{dem}
		
		and:
		
		The calculations therefore produce results that are identical. We can check for square matrices of higher dimensions.
		\begin{flushright}
			$\blacksquare$  Q.E.D.
		\end{flushright}
		\end{dem}
		
		\item[P5.] A square matrix $A\in M_n(\mathbb{C})$ is invertible (non-singular) if and only if $\det(A)\neq 0$ (this is the most important property among all).
		\begin{dem}
		If $A$ is invertible (non-singular), we have:
		
		\begin{flushright}
			$\blacksquare$  Q.E.D.
		\end{flushright}
		\end{dem}
		Notice that as $\det(\mathds{1})=1$, if follow immediately that \label{inverse determinant}:
		
		As we already said it, this is the most important property of matrices as part of theoretical physics because if $A$ is a linear system, the calculation of the determinant indicates whether it has unique solutions or not. Otherwise, as we have already mentioned and study it, the system has no solution, or an infinity of solutions!
		
		We must consider an important special case! Given the following system:
		
		where $A\in M_n(\mathbb{K})\neq 0$ and $B\in M_n(\mathbb{K})\neq 0$ are to be determined. It is obvious that $A$ is invertible (non-singular) or not, the trivial solution is $A\cdot B=0$. However, let us imagine a case of theoretical physics where we have $A\cdot B=0$ but for which we know that $A\in M_n(\mathbb{K})\neq 0$ for which we impose $B\in M_n(\mathbb{K})\neq 0$. In this case, we must eliminate the trivial solution $B=0$. Furthermore, calculate the inverse (if it exists) of the matrix $A$ will bring us to nothing concrete except that $B=0$ which obviously does not satisfy us. The only solution is then to play such that the coefficients $a_{ij}$ of the matrix $A$ are such that its determinant is zero and therefore the matrix in invertible! The advantage? Just to have an infinite number of possible solutions (of $B$ then!) that satisfy $A\cdot B=0$. We will need this methodology in the section of Wave Quantum Physics, when we will determine the existence of antiparticles through the linearized Dirac equation. It must therefore be remember.
		
		\item[P6.] Two "\NewTerm{conjugated}\index{conjugated matrices}" matrices (be careful! it is not the "conjugate" in the complex sense) have the same determinant.
		\begin{dem}
		
		
		\begin{flushright}
			$\blacksquare$  Q.E.D.
		\end{flushright}
		\end{dem}
		
		\item[P7.] For any matrix $A\in M_n(\mathbb{C})$:
		\begin{dem}
		
		But as (trivial... simple product of all coefficients):
		
		As (trivial) $\varepsilon(\sigma^-1)=\varepsilon(\sigma)$ and that $x,y\in\mathbb{C}:\bar{x}\cdot \bar{y}=\overline{x\cdot y}$ (\SeeChapter{see section Numbers page \pageref{module product complex numbers}}) then we can write:
		
		\begin{flushright}
			$\blacksquare$  Q.E.D.
		\end{flushright}
		\end{dem}
		
		\item[P8.] For any matrix $A\in M_n(\mathbb{R})$:
		
		\begin{dem}
		Well... it's the same as the previous property but without the conjugate values... In fact, we prove in the same way, the same property for $A\in M_n(\mathbb{C})$.
		\begin{flushright}
			$\blacksquare$  Q.E.D.
		\end{flushright}
		\end{dem}
		
		\item[P9.] Given a matrix $A=(a_{ij})\in M_n(\mathbb{C})$ we denote by $A_{ij}$ the matrix obtained from $A$ by removing the $i$-th row and the $j$-th column (very important notation to remember for what will follow!!!). The $A_{ij}$ belongs therefore to $M_{n-1}(\mathbb{C})$. Then for any $i=1...n$:
		
		where the term:
		
		is named the "\NewTerm{cofactor}\index{cofactor}\label{cofactor}\footnote{From all the cofactors we can build a "\NewTerm{cofactor matrix}\index{cofactor matrix}" containing all $C_{ij}$.}" or "\NewTerm{partial determinant}\index{partial determinant}\label{partial determinant}", $\det(A_{ij})$ is named the "\NewTerm{minor}\index{minor}", and the whole boxed relation above is named the "\NewTerm{cofactor expansion theorem}\index{cofactor expansion theorem}".
		\begin{dem}
		For the proof let us define the application:
		
		It could be almost easy to see that $\varphi$ is multilinear (you just have to consider that $(-1)^{i+j}a_{ij}$ as a simple constant and after by extension of the definition of the determinant... too easy...).

		Let us show however that this application is alternated (in this case it is a determinant hat has all the properties of a... determinant!).
	
		Given $a_k,a_{k+1}$ two column vectors $A$ that follows each other. Let us suppose that $a_k=a_{k+1}$, we have to show in this case that $\varphi(A)=0$ (which comes from the definition of an alternate application).
	
		We have first (it is mandatory by the definition itself) if we don't erase any of the columns $j$ being $k$ or $k+1$:
		
		and we have obviously if we don't remove respectively the column $k$ and the column $k+1$:
		
		Therefore:
		
		It is therefore OK. The application $\varphi$ is alternated and multilinear, it is indeed well a determinant.
		
		We have just prove that $\varphi$ is a determinant and by unicity we have $\varphi(A)=\det(A)$ for any $A\in M_n(\mathbb{C})$.
		\begin{flushright}
			$\blacksquare$  Q.E.D.
		\end{flushright}
		\end{dem}
		
		\begin{tcolorbox}[title=Remark,colframe=black,arc=10pt]
		So finally we have seen now in details the concepts of cofactor that should permit the reader to understand the concept of "adjugate matrix" (or "comatrix") that we had introduce at the beginning of this section at the page \pageref{adjugate}.
		\end{tcolorbox}
		If we take the following $2\times 2$ matrix:
		
		we get:
		
		Or if we take the following $3\times 3$ matrix (already used in an examples earlier!):
		
		Then the cofactor matrix is given by:
		
		And we notice something that can be generalized: if $A$ is symmetric, then $C=C^T$.
	
		\begin{tcolorbox}[colframe=black,colback=white,sharp corners]
		\textbf{{\Large \ding{45}}Example:}\\\\
		Let us see an example of this method by calculating the determinant of: 
		
		Let us develop the second line ($i=2$). We get:
		\end{tcolorbox}
		
		\pagebreak
		\begin{tcolorbox}[colframe=black,colback=white,sharp corners]
			
		Let us develop following the first column for verification (we never know...):
		
		Or the reader can check with Maple 4.00 by using:\\
	
		\texttt{>with(linalg):}\\
		\texttt{>det(array([[1,2,3],[2,1,0],[1,-1,I]]));}\\
		
		The calculation determined above is therefore "exponential" as if for example we must calculate the determinant of a square matrix of order (dimension) $n=10$, then the determinant will be developed in a sum of $10$ terms, which each contains the determinant of a matrix of dimension $n=9$, which is a cofactor of the starting matrix. If we develop any of this determinant, we get a sum of $9$ determinants where each contains the determinant of a matrix of dimension $n=8$. At this level, there is therefore $90$ determinants of matrices of dimension $8$ to calculate. The process could continue until it remains only determinant of order $2$. And therefore we guess that the number of determinants of order $2$ is important!
		\end{tcolorbox}
		
		Notice that as the adjugate of the $2\times 2$ matrix:
		
		
		\item[P10.] As we will prove it during our study of the spectral theorem further below at page \pageref{spectral theorem}, for a positive definite matrix $M$ we have:
		
		where the $\lambda_i$ are the eigenvalues of the matrix $M$. And for a definite semi-positive matrix $M$ we have:
		
		 
	\end{enumerate}

	\textbf{Definition (\#\mydef):} Given $m,n$ two positive integers and $A$ a $m\times n$ matrix with coefficients in $\mathbb{C}$. For any $k\leq \min(m,n)$ a "\NewTerm{minor of order $k$}\index{minor}\label{minor}" is a determinant of the type:	
	
	with $1\leq i_1< \ldots \leq m$ and $1\leq j_1 <\ldots <j_k\leq n$.
	
	In the particular case of a matrix of order $n>1$ the definition is simpler: the minor $M_{ij}$ of the element $a_{ij}$ is the determinant of the matrix of order $n-1$ that we get by remove the row $i$ and the column $j$. Therefore, to calculate the minor of an element, we remove the line and the column to which the element belongs to, and we calculate the determinant of the remaining square matrix.
	
	\begin{tcolorbox}[title=Remark,colframe=black,arc=10pt]
	The Leibniz formula of determinants is not the way you computer libraries calculate determinants. It is more of a theoretical tool, in fact it is probably one of the slowest ways to calculate the determinant. Gaussian Elimination is a fast and relatively simple way to do such a calculation!
	\end{tcolorbox}
	
	\paragraph{Derivative of a Determinant relatively to a parameter}\label{derivative of a determinant}\mbox{}\\\\
	Let us see now a result that will be quite useful to us in the section General Relativity:

	Given a square matrix $n\times n$ with functions $g_{ij}:\mathbb{R}\mapsto \mathbb{R}$ that can be derivate at least one time. Let us put $g:=\det(G)$ with $G=(g_{ij})$. We want to calculate $\mathrm{d}_t g$. Given $g_i$ the $i$-th column vector of the matrix $G$. Let us use the formula:
	
	Knowing that the derivative of $g_{\sigma(1),1}\cdot \ldots \cdot g_{\sigma(n),n}$ is (derivative of $n$ products):
	
	Therefore we have:
	
	If we take a look closely to the first above sum, we notice that:
	
	where $g_1^{\prime}$ is the derivative of the vector $g_1$. Same for the following sums. Therefore:
	
	Let us develop again. Let us consider the term $\det(g_1^{\prime},g_2,\ldots,g_n)$ above. If we develop it relatively to the first column, we get:
	
	 Also, by developing the $j$-term of the above sum relatively to the $j$-th column, we get:
	
	If we put:
	
	We get:
	
	Which is written in tensor notation (\SeeChapter{see section Tensor Calculus page \pageref{tensor notation}}):
	
	We also have:
	
	where $b_{ji}$ is the coefficient being at the row $j$-th, columen $i$-th of the matrix $G^{-1}$. If we denoted $g^{ij}$ the coefficient $i,j$ of the matrix $(G^{-1})^t$ then:
	
	The expression of the derivative is then finally:
	
	which is written in tensor notation:
	
	This result, finally quite simple, we will be helpful to us in the section of Tensor Calculus to build the tools necessary for the study of General Relativity and in the context of the determination of Einstein field equations. It is therefore appropriate to remember it!
	
	\paragraph{Derivative of logarithm of a determinant}\label{derivative of logarithm of a determinant}\mbox{}\\\\
	Let $M \in \mathbb{R}^{n \times n}$ be a square matrix. For a function $f: \mathbb{R}^{n \times n} \mapsto \mathbb{R}$, define its derivative $f'$ as an $n \times n$ matrix where the entry in row $i$ and column $j$ is $\partial f/\partial m_{ij}$.

	For some functions $f$, the derivative $f'$ has a nice form. We will show now a matrix derivation that we will meet in the section of Numerical Methods during our study of Gaussian Mixture Model and Factor Analysis:
	
	Here, we restrict the domain of the function to $M$ with positive determinant (typically the case of variance-covariance matrices!). Notice that it is not intuitive that the derivative of a scalar function of a matrix is equal to a matrix...
	
	Before we get move to the proof, we need to recall some terms. For a matrix $M$:
	\begin{itemize}
		\item The $(i,j)$ minor of $M$, denoted $\det(M_{ij})$, is the determinant of the $(n-1) \times (n-1)$ matrix that remains after removing the $i$th row and $j$th column from $M$
	
		\item The cofactor matrix of $M$, denoted $C$, is an $n \times n$ matrix such that $C_{ij} = (-1)^{i+j} \det(M_{ij})$
	
		\item The adjugate matrix of $M$, denoted $\text{adj}(M)=C^T=\left((-1)^{i+j}\det(M_{ij})\right)^T_{1\leq i,j\leq n}$, is simply the transpose of $C$.
	\end{itemize}
	These terms are useful because they related to both matrix determinants and inverses. If $M$ is invertible, then as we have proved earlier above:
	
	so\label{eqn:inverse matrix determinant cofactor relation}:
	
	On the other hand, by the cofactor expansion theorem of the determinant proved earlier above (as for a symmetric matrix $C=C^T$):
		
	So by the derivative product rule:
	
	If $k \neq j,$ then $\frac{\partial m_{ij}}{\partial X_{ik}}=0,$ otherwise it is equal to $1$. This means that the first term above reduces to $C_{ik}$. For any $k$, the elements of $M$ which affect $C_{ij}$ are those which do not lie on row $i$ or column $j$. Hence, $\partial C_{ij}/\partial m_{ik}=0$ for all $j$! Therefore:
	
	Putting all this together with an application of the derivative chain rule, we get (using the for the last equality the relation \ref{eqn:inverse matrix determinant cofactor relation}):
	
	as required!
	
	\paragraph{Cramer's rule}\label{Cramer's rule}\mbox{}\\\\
	Let us first solve a simple general $2$ by $2$ linear system using substitution to see how Cramer's rule out pops (this will be considered as an informal proof obviously!).

	For this purpose we start with the following system (don't forget that such a system is traditionally written $A\vec{x}=\vec{b}$):
	
	Multiplying both sides of the first equation by $a_2$, and both sides of the second equation by $a_1$, then subtracting, we find that:
	
	Assuming that $a_1b_2-b_1a_2$ is not $0$, we find that:
	
	The formula for $x$ can be derived similarly so that we have:
	
	If we denoted $x$ by $x_1$ and $y$ by $x_2$, then we see that we have:
	
	where we define $A_k$ to be the $n\times n$ matrix obtained by replacing the $k$-th column of $A$ by the inhomogeneous term $\vec{b}$.
	
	The rules for $3\times 3$ matrices are similar. Given:
	
	which in matrix format is:
	
	Then the values of $x$, $y$ and $z$ (after some boring algebra) can be found as follows:
	
	Then denoting again $x$ by $x_1$, $y$ by $x_2$ and $z$ by $x_3$ we have again (don't forget that such a system is traditionally written $A\vec{x}=\vec{b}$):
	
	where we define $A_k$ to be the $n\times n$ matrix obtained by replacing the $k$-th column of $A$ by the inhomogeneous term $\vec{b}$.
	
	But now let us deal with the general proof. But before... some readers may argue that it is useless! In fact not really! They are some quite important applications in Tensor Calculus (for example the divergence of a tensor field that has some application in advanced General Relativity\footnote{Don't look for it in this book!}), or in computing derivatives implicitly\footnote{Actually also not needed to read this book entirely!}. But this are all quite abstract applications. Indeed! So the reader has to know that the most well known practical application is the use in forecasting and times series analysis in business (more generally in any advanced finance field) for the study of the partial autocorrelation coefficient of autoregressive process (through the Yule-Walker equations).
	
	\begin{theorem}
	If $A\vec{x}=\vec{b}$ is a linear system of equations with $\vec{x}=[x_1\; x_2\; \ldots\; x_n]^T$ and $A\in \mathbb{R}^{n\times n}$ such that $\det(A)\neq 0$ then we find the solutions:
	
	where we define $A_i$ to be the $n\times n$ matrix obtained by replacing the $i$-th column of $A$ by the inhomogeneous term $\vec{b}$.
	\end{theorem}
	
	As there a lot of way to make the proof, the best an easiest one according to us is that provided by PlanetMath.org and written by the pseudonyme rmilson the 2013-03-22 for which we provide here a simple copy paste (as everything seems perfect to us and very clever way to do the proof...!).
	\begin{dem}
	Since we assume that $\det(A)\neq 0$, by the properties of the determinant we know that $A$ is invertible.
	
	We claim that this implies that the equation $A\vec{x}=\vec{b}$ has a unique solution. Note that $A^{-1}\vec{b}$ is a solution since:
	
	so we know that a solution exists.
	
	Let $\vec{s}$ be an arbitrary solution to the equation, so $A\vec{s}=\vec{b}$. But then:
	
	so we see that $A^{-1}\vec{b}$ is the only solution.
	
	For each integer $i$, $1\leq i\leq n$, let $a_i$ denote the $i$th column of $A$, let $e_i$ denote the $i$th column of the identity matrix $I_n$, and let $X_i$ denote the matrix obtained from $\mathds{1}_n$ by replacing column $i$ with the column vector $\vec{x}$.

	We know that for any matrices $A$, $B$ that the $k$th column of the product $AB$ is simply the product of $A$ and the $k$th column of $B$. Also observe that:
	
	for $k=1,...,n$. Thus, by multiplication, we have:
	
	Since $X_i$ is $\mathds{1}_n$ with column $i$ replaced with $\vec{x}$, we will compute the determinant of $X_i$ with the cofactor expansion gives. For this, let us recall the cofactor expansion relation that we have just proved earlier and that was for recall:
	
	can be used to calculated the determinant only for the cofactor (row) $j=i$ such that for $X_i$:
	
	If we choose to calculate this sum only for a given row $i$, we know that $x_{ij}\neq 0$ only for $i=j$, then the previous relation reduce to:
	
	Thus by the multiplicative property of the determinant:
	
	we get:
	
	and if we denote $M_i$ by $A_i$ we get the famous "\NewTerm{Cramer's rule}\index{Cramer's rule}":
	
	as required!
	\begin{flushright}
		$\blacksquare$  Q.E.D.
	\end{flushright}
	\end{dem}
	
	\paragraph{Determinant Cofactor and Matrix Inverse}\label{determinant matrix inverse}\mbox{}\\\\
	Let us finish our study of the determinants with the "icing on the cake" by giving a very important relation in many fields of engineering, physics and mathematics that connects the inverse of the coefficients of a matrix with miners of order $n$ (we will use this relation further below).
	
	Given $A\in M_n(\mathbb{C})$ an invertible matrix (non singular). Let us write $A=(a_{ij})$ and $A^{-1}=(b_{ij})$. Then:
	
	\begin{dem}
	Let us denote by $a_k$ the $k$-th column vector of the matrix $A$. Knowing that $A\cdot A^{-1}=\mathds{1}$ (under known assumptions), we have (trivial):
	
	Let us calculate now $\det(a_1,\ldots,a_{k-1},e_j,a_{k+1},\ldots,a_n)$. First by developing relatively to the $k$-the column we found (as only one of the coefficient of $e_j$ is not null and that the unique non-null one is equal to the unit):
	
	Furthermore (properties of the determinant):
	
	Therefore:
	
	That is to say\label{inverse matrix}:
	
	That later relation is sometimes denoted:
	
	where "adj" means adjacent.
	\begin{flushright}
		$\blacksquare$  Q.E.D.
	\end{flushright}
	\end{dem}
	\label{some matrix inverse}
	\begin{tcolorbox}[colframe=black,colback=white,sharp corners]
	\textbf{{\Large \ding{45}}Example:}\\\\
	E1. Inverse of a general $2\times 2$ matrix:
	
	
	E2. Inverse of a general $3\times 3$ matrix:
		
	
	E3. Inverse of a general $4\times 4$ matrix:
	
	and:
	
	then there exists an inverse matrix of $A$, and it is:
	
	\end{tcolorbox}
	
	\begin{tcolorbox}[colframe=black,colback=white,sharp corners]
	where:
	
	\end{tcolorbox}
	
	For a simple, detailed, and important practical application in the industry (because otherwise in this entire book we will rarely inverse small matrices), the reader can refer to the section of Theoretical Computing in the part concerning the multiple linear regression.
	
	Let us also indicate the following important properties where $A$ and $B$ are both square invertible matrices $M_{n}(\mathbb{C})$ and $\lambda\in\mathbb{C}$ (the first should be obvious, the second has already been presented earlier but unproven and the third one is important for the proof of the variance inflation factor that we will prove in the section of Theoretical Computing on page \pageref{variance inflation factor}) :
	
	Let us prove the last property using the property of associativity:
	\begin{dem}
	
	Which prove that $B^{-1}A^{-1}$ is indeed the inverse of $AB$ where $I_n$ (also denoted $\mathds{1}$ for recall) is a diagonal matrix (also square) of dimension $n$.
	\begin{flushright}
		$\blacksquare$  Q.E.D.
	\end{flushright}
	\end{dem}
	
	\paragraph{Determinant of block-diagonal or block-triangular matrices}\label{determinant of block-diagonal or block-triangular matrices}\mbox{}\\\\
	For our study of Factor Analysis we will need to compute the determinant of two particular type of block-triangular matrices.
	
	First we will need to compute the determinant of block-upper-triangular matrix of the form:
	
	where where $A$ and $D$ are square matrices. And secondly of a block-lower-triangular matrix of the form:
	
	
	However to calculate the determinant of these matrices we should first prove how to calculate the determinant of a diagonal block matrix of the following type:
	
	where $\mathds{1}$ is in this special case a $1\times 1$, that is, $\mathds{1}=1$. Suppose $A$ is $k \times k$. Then $\Gamma$ is of dimension $(k+1)(k+1)$.  
	
	For the proof, we will use the relation derived earlier above:
	
	\begin{dem}
	So we need to compute:
	
	where $\mathcal{S}$ is the set of all permutations of the first $k+1$ natural numbers.
	
	In the special case above we can rewrite this as:
	
	\begin{tcolorbox}[title=Remark,colframe=black,arc=10pt]
	Let us recall to help to understand what we just did that we are typically in a situation similar to the following below where $a_{33}=1,a_{31}=a_{13}=a_{23}=a_{32}=0$:
	
	From the above formula, each term is in the form of $a_{1j_1}a_{2j_2}a_{3j_3}$ where the second indices $j_1,j_2,j_3$ is a permutation of $1,2,3$. For example, the first term has $1,2,3$ as the second indices; the second term has $1,3,2$; the third term has $2,1,3$, etc.\\
	
	The sign of each term is the sign of the permutation. For example, the sign of $1,2,3$ is clearly $1$. For the second term, $1,3,2$ is obtained from $1,2,3$ by one transposition $2,3\rightarrow3,2$, so the sign is $-1$, etc.\\
	
	We see obviously that in the case $a_{33}=1,a_{31}=a_{13}=a_{23}=a_{32}=0$, the above relation reduces to:
	
	\end{tcolorbox}
	The result for the case in which $\mathds{1}$ is not $1\times 1$ is proved recursively. For example, if $\mathds{1}$ is $2\times 2$, we have:
	
	and analogously for larger dimensions.
	\begin{flushright}
		$\blacksquare$  Q.E.D.
	\end{flushright}
	\end{dem}
	The proof for the case in which:
	
	is similar to the one just provided.
	
	Ok now that we have this result, let us prove the following determinant:
	
	
	\begin{dem}
	Let us assume that $A$ is $k \times k$ and $D$ is $l \times l$, so that $C$ is $l \times k$ and $\mathds{O}$ is $k \times l$. In what follows, we will denote by $\mathds{1}_k$ a $k \times k$ identity matrix and by $\mathds{O}_{lk}$ an $l\times k$ zero matrix. Note that (in the purpose to decompose the above matrix $\Gamma$ in product of triangular block matrices!):
	
	Thus, similarly to the previous proof with the diagonal block matrices and triangular block matrices:
	
	\begin{flushright}
		$\blacksquare$  Q.E.D.
	\end{flushright}
	\end{dem}
	
	\paragraph{Inverse and Determinant of a partitioned symmetric matrix}\label{inverse of a partitioned symmetric matrix}\mbox{}\\\\
	We will prove here a set of relations that we will need (once again...) for our study of Factor Analysis in the section of Statistics.
	
	Let us divide an $n\times n$ symmetric matrix $A$ into four blocks:
	
	The inverse matrix ${ B}={ A}^{-1}$ can also be divided into four blocks:
	
	Here we assume the dimensionalities of these blocks are:
	\begin{itemize}
		\item $A_{11}$ and $B_{11}$ are $p\times p$
		\item $A_{22}$ and $B_{22}$ are $q\times q$
		\item $A_{12}=A_{21}^T$ and $B_{12}=B_{21}^T$ are $p\times q$
	\end{itemize}
	with $p+q=n$. 
	
	Then what we want to prove first (and after we will be able to focus on the inverse determinant problem) is that the components of the inverse of a partitioned symmetric matrix are given by:
	
	ie (it is in this form that we found it in most textbooks):
	
	\begin{tcolorbox}[title=Remark,colframe=black,arc=10pt]
	The term $A_{11}-A_{12} A_{22}^{-1} A_{21}$ seems to be named the "\NewTerm{Schur complement}\index{Schur complement}" of $A_{22}$ in $A$. Same for the term $A_{22}-A_{21} A_{11}^{-1} A_{12}$ that seems also be named "Schur complement" of $A_{11}$ in $A$.
	\end{tcolorbox}
	\begin{dem}
	We start with:
	
	Let us equate each of the four blocks to get:
	
	Plug $B_{21}$ into $B_{11}$ to get:
	
	Solve for $B_{11}$ to get:
	
	Applying the Woodburry matrix identity (see below page \pageref{Woodbury matrix identity}) given for recall by:
		
	to this expression we also get the other expression in the theorem, ie:
		
	Similarly we can get:
	
	
	\begin{flushright}
		$\blacksquare$  Q.E.D.
	\end{flushright}
	\end{dem}
	Let us see now how to calculate the determinant of a partitioned symmetric matrix\label{determinant of a partitioned symmetric matrix}, given by:
	
	\begin{dem}
	First, let us consider a matrix of the form:
	
	where $A,B,C,D$ have size $n\times n$. We start from consider multiplication on the left by a matrix such that:
	
	Now let us consider the following left multplication:
	
	Therefore (starting from the beginning) in one line, that means we have:
	
	So that means after rearranging:
	
	And it is therefore quite immediate that:
	
	So that finally:
	 
	Similarly we get:
	
	To summarize:
	
	This is a result we needed before attacking the real deal!
	
	Indeed, according to the result above, this means that we can write:
	
	
	As seen above (page \pageref{determinant product}), we have:
	
	and (page \pageref{determinant of block-diagonal or block-triangular matrices}):
	
	Therefore the equality:
	
	is proved.
	\begin{flushright}
		$\blacksquare$  Q.E.D.
	\end{flushright}
	\end{dem}
	
	\pagebreak
	\subsection{Change of basis (frames)}\label{change of basis}
	A basis for a vector space of dimension $n$ is a sequence of $n$ vectors $(\vec{e}_1, …, \vec{e}_n)$ with the property that every vector in the space can be expressed uniquely as a linear combination of the basis vectors (\SeeChapter{see section Vector Calculus page \pageref{vector basis}}). The matrix representations of operators are also determined by the chosen basis! Since it is often desirable to work with more than one basis for a vector space, it is of fundamental importance in linear algebra to be able to easily transform coordinate-wise representations of vectors and operators taken with respect to one basis to their equivalent representations with respect to another basis. Such a transformation is named a "\NewTerm{change of basis}\index{change of basis}".
	
	Let us now suppose that we move from a frame $\mathcal{E}=(\vec{e}_1,\vec{e}_2,...,\vec{e}_n)$ of a space $V^n$ to another space $\mathcal{F}=(\vec{f}_1,\vec{f}_2,...,\vec{f}_n)$ of this same space sharing the same origin O. Thus in two dimension:
	\begin{figure}[H]
		\centering
		\includegraphics[scale=0.75]{img/algebra/basis_change.jpg}
		\caption[A vector can be represented in two different bases (purple and red arrows]{A vector can be represented in two different bases (purple and red arrows) (source: Wikipedia)}
	\end{figure}
	Let us decompose the $\vec{f}_i$ in the basis $\mathcal{E}$:
	
	\textbf{Definition (\#\mydef):} We name "\NewTerm{transition matrix}\index{transition matrix}" the matrix (linear application) that allows to pass from $\mathcal{E}\mapsto \mathcal{F}$ given by:
	
	\begin{theorem}
	Now let us consider the vector given by:
	
	So we intend to prove that the components of $y_1,y_2,...,y_n$ of $\vec{v}$ in the basis $\mathcal{F}$ are given by:
	
	Thus explicitly:
	
	\begin{tcolorbox}[title=Remark,colframe=black,arc=10pt]
	The matrix $P$ is invertible (non-singular), because its columns are linearly independent (they are the vectors $\vec{f}_i$ decomposed in the basis $\mathcal{E}$ and the $\vec{f}_i$ base and the $\vec{f}_i$ are linearly independent as they form a base!).
	\end{tcolorbox}
	\end{theorem}
	\begin{dem}
	Let us take the case to simplify the case $n=2$ (the proof being quite easily generalized) with $\mathcal{E}=(\vec{e}_1,\vec{e}_2)$ and $\mathcal{F}=(\vec{f}_1,\vec{f}_2)$.
	Then we have:
	
	We therefore have $\vec{v}=x^i\vec{e}_i$ and we seek to express $\vec{v}$ in the basis $\mathcal{F}$ as $\vec{v}=y^i\vec{f}_i$. We'll search the linear application that link these two relation such that:
	
	Thus written in an explicit way:
	
	Therefore:
	
	That is to say:
	
	So $P$ (if it exists) is indeed the matrix that can express the components of a vector of a basis in those of another basis such that we write in vector notation:
	
	\begin{flushright}
		$\blacksquare$  Q.E.D.
	\end{flushright}
	\end{dem}
	\begin{theorem}
	Let us now consider a linear application $g:V^n\mapsto V^n$. Let $A$ be the matrix in the basis $\mathcal{E}$, and $B$ its matrix in the basis $\mathcal{F}$ (of same dimension). Then we might have:
	
	which is equivalent:
	
	or even:
	
	If there exists such a matrix $P$ satisfying these relations, we say that $A$ and $B$ are "\NewTerm{similar matrices}\index{similar matrices}".
	\end{theorem}
	\begin{dem}
	Let us take back the fact that we proved that it was eventually possible to build a transition matrix $P$ from the fact that:
	
	and let us put:
	
	We have then a function that bring us to write:
	
	On the other hand, we have (that we proved earlier):
	
	Therefore:
	
	hence:
	
	and as we saw it in our study of the determinant, the determinants of $A$, $B$ are equal and therefore invariant. We will  return later back on a similar formulation in our study of the Spectral Theorem below.
	\begin{flushright}
		$\blacksquare$  Q.E.D.
	\end{flushright}
	\end{dem}
	At the vocabulary level we say when we are in the presence of a such a matrix relation that: the matrix $A$ is "\NewTerm{conjugated}" to the matrix $B$.
	
	\pagebreak
	\subsection{Eigenvalues and Eigenvectors}\label{eigenvector}
	\textbf{Definition (\#\mydef):} An "\NewTerm{eigenvalue}\index{eigenvalue}" is by definition (we will find again this definition in the introduction to quantum algebra in the section of Wave Quantum Physics) a value $\lambda$ belonging to a field $\mathbb{K}$ such that given a squared matrix $A\in M_{mm}(\mathbb{K})$ we have:
	
	and conversely a vector $\vec{X}\in M_{m1}(\mathbb{K})$ is an "\NewTerm{eigenvector}\index{eigenvector}" if and only if:
	
	The major advantage of these concepts will be able the possibility to study a linear application, or any other item linked to a matrix representation, in a simple representation through a basis change on which the restriction of $A$ is a single homothetic transformation (typically solving simple systems of differential equations). 
	\begin{tcolorbox}[title=Remark,colframe=black,arc=10pt]
	This definition can be generalized to functional analysis (therefore it does not only applied to Matrix Algebra) too as by defining the "\NewTerm{spectrum of an endomorphism}\index{spectrum of an endomorphism}\label{spectrum of an endomorphism}" as the set:
	
	\end{tcolorbox}
	Thus, all the eigenvalues of a matrix $A\in M_{mm}(\mathbb{K})$ is named "\NewTerm{spectrum of $A$}\index{spectrum of a matrix}" and satisfies the following homogeneous system named a "\NewTerm{matrix-eigenvalue equation}\index{matrix-eigenvalue equation}\label{matrix-eigenvalue equation}":
	
	or (whatever it is the same!):
	
	where $I_n$ (also sometimes denoted by the symbol: $\mathds{1}$) is for recall a diagonal unit matrix (and therefore also square) of dimension $n$. This system we know (proved above) has nontrivial solutions, therefore $\vec{X} \neq\vec{0}$ or $(\lambda I_n-A)\neq \vec{0}$, if and only if (we'll see many examples in various section related to physics in this book):
	
	that is to say that the matrix $A-\lambda I_n$ is not inversible (singular).
	
	The determinant $\det(A-\lambda I_n)$ is a polynomial on $\lambda$ of degree $n$ and can have at maximum $n$ solutions/eigenvalues as we have proved it in our study of polynomials (\SeeChapter{see section Calculus page \pageref{polynomial}}) and is named "\NewTerm{characteristic polynomial}\index{characteristic polynomial}\label{characteristic polynomial determinant}" of $A$ and the equation $\det(A-\lambda I_n)$ is named "\NewTerm{characteristic equation}\index{characteristic equation}" of $A$ or "\NewTerm{eigenvalues equation}\index{eigenvalues equation}\label{eigenvalue equations}".
	
	For the small parenthesis, it is nice to notice that we always have in the development of $\det(A-\lambda I_n)$ the trace of the matrix $\text{tr}(A)$ and the determinant $\det (A)$ that appear. Let us see two examples of this:
	
	\begin{tcolorbox}[colframe=black,colback=white,sharp corners]
	\textbf{{\Large \ding{45}}Examples:}\\\\
	E1. Let us begin with the case $n=2$:
	
	Therefore for a square matrix of dimension $2$, the eigenvalues are (simple resolution of a polynomial of the 2nd degree):
	
	E2. For a matrix of dimension $n=3$, we have:
	
	and here... the final solution (roots) are quite less easy in the general case...
	\end{tcolorbox}
	On the path let us notice (we will generalize the result coming from this during our study of the spectral theorem) that as multiplying the homogeneous system:
	
	by $-1$ on the both sides of the equality doesn't change anything to the problem, then we get:
	
	So we can see that is multiplication doesn't change anything to the final result!
	
	Thus by a term by term correspondence it comes the very important result in Statistics (and also in Numerical Methods!) that we will prove later in a more general way with the spectral theorem:
	
	If we look at $(\lambda I_n-A)$ as a linear application $f$, as it is non-trivial solutions that interest us, we can say that the eigenvalues are the elements $\lambda$ such that:
	
	and that the kernel constitutes the eigenspace  of $A$ of the eigenvalue $\lambda$ from which non-null elements are the eigenvectors!
	
	It corresponds to the study of the main axes, according to which the application behaves like an expansion (homothetic application) multiplying the vectors by the same constant. This homothetic ratio is then the "eigenvalue", the vectors to which it applies the "eigenvectors" are asembled in a "eigenspace".
	
	Another way of looking at it:
	\begin{itemize}
		\item A vector is said to be an "eigenvector" by a linear application if it is not zero and if the application does only change its size (norm) without changing its direction.

		\item An "eigenvalue", associated to an "eigenvector", is the size modification factor (homothetic ratio), ie the number by which we must multiply the vector to get its image. This factor can be negative (reverse direction of the vector) or zero (vector transformed into a vector of zero length).
		
		We can say that therefore that the eigenvalue $\lambda$ "scalar" the application $A$ for the eigenvector $\vec{X}$.

		\item An "eigenspace" associated to an "eigenvalue" is the set of eigenvectors that have the same eigenvalue value and a zero vector. They suffer all from the multiplication by the same factor.
	\end{itemize}
	\begin{tcolorbox}[title=Remark,colframe=black,arc=10pt]
	In mechanics, we study the eigenfrequencies and eigenmodes of oscillating systems (\SeeChapter{see section Wave Mechanics page \pageref{eigenmodes}}). In Functional Analysis, an eigenfunction is an eigenvector for a linear operator, that is to say a linear application acting on a space of function (\SeeChapter{see section Functional Analysis page \pageref{functional analysis}}). In geometry and optics, we speak of eigendirections to take into account the curvature of the surfaces (\SeeChapter{see section Non-Euclidean Geometry page \pageref{non-euclidean geometry}}). In graph theory, an eigenvalue is simply an eigenvector of the adjacency matrix of the graph (\SeeChapter{see section Graph Theory page \pageref{adjacency matrix}}).
	\end{tcolorbox}
	So as the determinant of $\det(A-\lambda I_n)$ is a polynomial on $\lambda$ then the $\lambda_i$ are also the roots of the characteristic polynomial:
	
	Therefore:
	
	This is a relation used sometimes in some statistical models (form example the MANOVA!). 
	
	Before closing this short introduction to the eigenvalues and eigenvectors (we will discussed this further below), let us indicate that since an eigenvector must satisfy the homogeneous system:
	
	Nothing avoid us then from multiplying the eigenvector by a constant $k$ which normalizes it to the unit (technique often used in statistics and numerical methods to improve the floating precision of the algorithms) since:
	
	Thus in practice it is customary that if the eigenvector is given for example by:
	
	To normalize it at the unit by writing:
	
	
	For the section of Wave Quantum Physics and more especially for the study of the angular momentum and spin we need to prove that given an operator acting on an eigenvector, then if we square the operator, this result in squaring the eigenvalue.
	\begin{dem}
	Given:
	
	Then:
	
	\begin{flushright}
		$\blacksquare$  Q.E.D.
	\end{flushright}
	\end{dem}
	
	\subsubsection{Rotation Matrices and Eigenvalues}\label{rotation matrix in linear algebra}
	Now that we have seen what was an eigenvalue and an eigenvector, let us come back on a particular type of orthogonal matrices that we will be particularly useful to us in our study of quaternions (\SeeChapter{see section Numbers page \pageref{quaternions}}), of groups and symmetries (\SeeChapter{see section Set Algebra page \pageref{set algebra}}) and particle physics (\SeeChapter{see section Elementary Particle Physics page \pageref{elementary particle physics}}).
	
	We denote, as what has been seen in the section of Set Algebra, $\text{O}(n)$ the set of $n\times n$ (square) orthogonal matrices with coefficients in $\mathbb{R}$, that is to say, satisfying:
	
	That will denote also sometimes for recall sometimes as:
	
	The columns and rows of an orthogonal matrix the form the orthonormal basis of the usual space $\mathbb{R}^2$ for the usual dot product.
	
	The determinant of an orthogonal matrix is equal to $\pm 1$ (rotation conserves angles and volumes), indeed $A^T A=I$  leads to:
	
	A rotation matrix with determinant $+1$ is a "\NewTerm{proper rotation}\index{eigenvalue equations}", and one with a negative determinant $-1$ is an "\NewTerm{improper rotation}\index{improper rotation}", that is a reflection combined with a proper rotation.
	
	\pagebreak
	\begin{tcolorbox}[colframe=black,colback=white,sharp corners]
	\textbf{{\Large \ding{45}}Example:}\\\\
	Let us calculate now explicitly the determinant of a $2\times 2$ rotation matrix (\SeeChapter{see section Numbers page \pageref{2d rotation matrix}}) and $3\times 3$ rotation matrix (\SeeChapter{see section Euclidean Geometry page \pageref{3d rotation matrix around}}) as it is ask by many student on various Internet forums.\\

	So first we consider the $2\times 2$ rotation matrix and using the relation of the determinant proved earlier, we get:
	
	And for one randomly chosen rotation matrix of the three $3\times 3$ rotation matrices (\SeeChapter{see section Euclidean Geometry page \pageref{3d rotation matrix around}}), we get:
		
	\end{tcolorbox}
	
	We denote by $\text{SO}(n)$ the set of orthogonal matrices of determinant $1$ (for more details see the section of Set Algebra page \pageref{set algebra}). Let us show in three points that if $A\in \text{SO}(3,\mathbb{R})$  then $A$ is the rotation matrix relative to an axis passing through the origin.
	
	\begin{enumerate}
		\item Any eigenvalue of a rotation matrix $A$ (real or complex) is of module $1$. In other words it conserve the norm:

		Indeed, if $\lambda$ is an eigenvalue of eigenvector $\vec{X}$, we have:
		
		or noting the dot product with the book usual notation:
		
		
		\item  It exists a straight line in the space that is used a rotation axes and any vector on this line is not modified by any rotation.

		Let us denote by $\vec{X}$ an eigenvector  of eigenvalue $1$ (that is to say such that $A\vec{X}=\vec{X}$). As the reader may have perhaps already understand it (read until the end please!), the straight line generated by $\vec{x}$ that we will denote by $\langle \vec{X} \rangle$ constitutes our rotation axes.
	
		Indeed, any vector $\langle \vec{X} \rangle$ is send on itself by the application $A$. In this case, the orthonormal space denoted by $\langle \vec{X} \rangle^\perp$ that is of dimension $2$ is the perpendicular plane to the rotation axes.
	
		\item Any vector perpendicular to the rotation axes remains, after rotation, perpendicular to this axes. In other words,  $\langle \vec{X} \rangle$ in invariant through the application of $A$.
		
		Indeed, if $\vec{w}\in \langle \vec{x} \rangle$ the, $w=A^TA w=A^Tw$ and for all $\vec{y}\in \langle \vec{X} \rangle^\perp$:
		
		that is to say $A \vec{y} \langle \vec{X} \rangle^\perp$. Therefore $\langle \vec{X} \rangle^\perp$ is invariant by $A$.
		
		Finally, the restriction of $A$ to the space $\langle \vec{X} \rangle^\perp$ is a rotation!
	\end{enumerate}
	\begin{tcolorbox}[colframe=black,colback=white,sharp corners]
	\textbf{{\Large \ding{45}}Example:}\\\\
	Given $e^{\mathrm{i}\alpha}$ (see the section Numbers where the rotation by the complex number is proven) an eigenvalue (which module is $1$ as we proved it during our study of complex numbers) of $A$ restraint to $\langle \vec{X} \rangle^\perp$.\\
	
	Let us write $\vec{w}=\vec{u}+\mathrm{i}\vec{v}$ an eigenvector with $\vec{u},\vec{v}\in \mathbb{R}^2$ such as:
	
	with (as we already proved it in our study of complex numbers):
	
	where we know by our study of complex numbers, that the vectors $\vec{u},\vec{v}$ generate an orthogonal basis (not necessarily normalized at the unit!) of $\langle \vec{X} \rangle^\perp$.
	\begin{tcolorbox}[title=Remark,colframe=black,arc=10pt]
	We think that it could by easy at this level of the reader to check that this matrix is orthogonal (if it not the case contact us and this will be detailed!).
	\end{tcolorbox}
	\end{tcolorbox}
	
	\pagebreak
	\subsection{Spectral Theorem}\label{spectral theorem}
	Let us now see a very important theorem relatively to the eigenvalues and eigenvectors which is named the "\NewTerm{spectral theorem}\index{spectral theorem}" which will be very useful to us for the various sections of physics of this book and also the section  Statistics as well as in the section of Theoretical Computing and Industrial Engineering.
	
	To summarize, mathematicians say in their language that the spectral theorem give the possibility to affirm the diagonalization  of endomorphism\footnote{What the also named an "endomorphism reduction"}\index{diagonalization  of endomorphism}\index{endomorphism reduction} (of matrices) and also justify the decomposition in eigenvalues (also named "\NewTerm{singular value decomposition S.V.D.}\index{singular value decomposition}") that we will see further below during our presentation of some matrices decomposition techniques.
	
	\begin{tcolorbox}[title=Remark,colframe=black,arc=10pt]
	The singular value decomposition theorem (S.V.D.) is however very general, in the sense that it applies to any rectangular matrices. The eigenvalue decomposition (see further below the details), however, only works for some square matrices.
	\end{tcolorbox}
	
	To simplify the proof, we will deal here only real matrices (component in $\mathbb{R }$) and also avoiding up the language of mathematicians.
	
	We will denote in a first time $M_n(\mathbb{R})$ the set of all $n\times n$ (square) matrices with real coefficients.
	
	We will confuse the matrix $M\in M_n (\mathbb{R})$ with the linear application on the vector space $\mathbb{R}^n$ by:
	
	with $\vec{v}\in \mathbb{R}^n$.
	
	Reminder: We have seen above during our study of basis changes that if $(\vec{c}_1,\ldots,\vec{c}_n)$ is a basis of $\mathbb{R}^n$ and $M\in M_n(\mathbb{R})$ then the matrix of the linear map $M$ in the basis $(\vec{c}_1,\ldots,\vec{c}_n)$  is:
	
	where $S$ is the matrix formed by the column vectors $\vec{c}_1,...,\vec{c}_n$.
	
	First, we simply check that if $A$ is a symmetric matrix then (this should be trivial but it can be verified with an example of dimension $2$ very quickly):
	
	\begin{enumerate}
		\item[P1.] All eigenvalues of $M$ are reals.
		\begin{dem}
		Given:
		
		an a priori complex eigenvector of the eigenvalue $\lambda \in \mathbb{C}$. Let us denote:
				
		the conjugate vector of $\vec{z}$. Then we have:
		
		On the other hand since $M=\overline{M}$ we have:
		
		As $\vec{z} \neq \vec{0}$, we have $\lambda=\vec{\lambda}$ and therefore, $\lambda\in \mathbb{R}$.
		\begin{flushright}
		$\blacksquare$  Q.E.D.
		\end{flushright}
		\end{dem}
		Before going further, we also have to prove that if $M\in M_n(\mathbb{R})$ is a symmetrical matrix and $V$ and vectorial subspace of $\mathbb{R}^n$ invariant relatively to $M$ (that is to say that satisfies for any $\vec{v}\in V: \; M\vec{v}\in V$) then we have the following properties:
		
		\item[P3.] The orthogonal of $V$ denoted obviously by $V^\perp$ (obtained by applying the Gram-Schmidt method seen in the section of Vector Calculus page \pageref{gram-schmidt procedure}) is also invariant through $M$.
		
		\begin{dem}
		Given $\vec{v}\in V$ and $\vec{w}\in V^\perp$ then:
		
		this shows well that $M\vec{w}\in V^\perp$.
		\begin{flushright}
		$\blacksquare$  Q.E.D.
		\end{flushright}
		\end{dem}

		\item[P4.] If $(\vec{w}_1,\ldots,\vec{w}_k)$ is an orthonormal basis of $\vec{V}^\perp$ then the restriction matrix of $M$ to $V^\perp$ in the basis $(\vec{w}_1,\ldots,\vec{w}_k)$  is also symmetrical.
		\begin{dem}
		
		Let us denote $A=(a_{ij})_{1\leq i,j\leq k}$ the matrix of the restriction of $M$ to $V^\perp$ in the basis $(\vec{w}_1,\ldots,\vec{w}_k)$. We have by definition for any $j=1...k$ (as the vector resulting of a linear application such as $M$ can be express in its basis):
		
		Or:
		
		as:
		
		if $i\neq m$ in the orthonormal basis.
		
		On another side:
		
		Therefore:
		
		This shows that:
				
		\begin{flushright}
		$\blacksquare$  Q.E.D.
		\end{flushright}
		\end{dem}
	\end{enumerate}
	\begin{theorem}
		We will now be able to show that any symmetric matrix $M \in M_n(\mathbb{R})$ is diagonalizable. That is to say that there is an invertible matrix $S$ such that the result of the calculation:
	
	gives a diagonal matrix!
	\begin{tcolorbox}[title=Remark,colframe=black,arc=10pt]
	In fact we will see, to be more precise, that there exists an orthogonal matrix $S$ such that $S^{-1}MS$ is diagonal.
	\end{tcolorbox}
	Reminder: Say that $S$ is "orthogonal" means that $SS^T=I$ (where $I$ is the identity matrix) which is equivalent to say that the columns of $S$ form an orthonormal basis of $\mathbb{R}^n$.
	\end{theorem}
	\begin{dem}
	We prove the assertion by induction on $n$. If $n=1$ there is nothing to prove. Let us suppose that the assertion is satisfied for $k\leq n$ and let us prove it for $k=n+1$. Then given $M\in M_{n+1}(\mathbb{R})$ a symmetric matrix and $\lambda$ an eigenvalue of $M$.
	
	We easily verify that the eigenspace:
	
	is invariant by $M$ (just take any numerical application) and that by the proof seen earlier, that $W^\perp$ is also invariant by $M$. Moreover, we know (\SeeChapter{see section Vector Calculus page \pageref{direct sum}}) that $\mathbb{R}^{n+1}$ can be decomposed into a direct sum:
	
	If:
	
	then:
	
	and it is sufficient to take an orthonormal basis of $W$ to diagonalise $M$. Indeed, if $(\vec{w}_1,\ldots,\vec{w}_{n+1})$ is such a basis, the matrix $S$ formed by the column vectors $\vec{w}_j$ ($j=1\ldots n+1$) is orthogonal and satisfies:
	
	and $S^{-1}MS$ is indeed diagonal.
	
	Let us now suppose that $\dim(W^\perp)>0$ and given $(\vec{u}_1,\ldots,\vec{u}_m)$ with $m\leq n$ an orthonormal basis of $W^\perp$. Let us denote by $A$ the restriction matrix of $M$ to $W^\perp$ in the basis  $(\vec{u}_1,\ldots,\vec{u}_m)$ . $A$ is also  symmetric (as proved in one of the preceding properties).
	
	By induction hypothesis there exists an orthogonal matrix $H\in M_m(\mathbb{R})$ such that $H^{-1}AH$ is diagonal.
	
	Let us denote by $(\vec{w}_1,\ldots,\vec{w}_{n+1-m})$ an orthonormal basis of $W$ and $G$ the matrix formed by the column vectors: $\vec{w}_1,\ldots,\vec{w}_{n+1-m},\vec{u}_1,\ldots\vec{u}_m$. So we can write that:
	
	and $G$ is also orthogonal by construction.
	
	Let us consider the following block matrix (matrix of matrices):
	
	and let us put:
	
	It is almost obvious that $S$ is orthogonal as $G$ and $L$ are also orthogonal. Indeed, if:
	
	then (remember that matrix multiplication is associative !!!):
	
	Also $S$ satisfies:
	
	and then:
	
	is indeed diagonal.
	\begin{flushright}
		$\blacksquare$  Q.E.D.
	\end{flushright}
	\end{dem}
	Finally here is finally the famous "\NewTerm{spectral theorem}\index{spectral theorem}" (real case):
	\begin{theorem}
	Given $M\in M_n(\mathbb{R})$ a symmetric matrix, then there exists an orthonormal basis made of eigenvectors of $M$.
	\end{theorem}
	\begin{dem}
	So we have seen in the preceding paragraphs that there exists an orthogonal matrix $S$ such that $S^{-1}MS$ is diagonal if $M$ is symmetric! Let denote by $\vec{c}_1,\ldots,\vec{c}_n$ the columns of $S$. The basis $(\vec{c}_1,\ldots,\vec{c}_n)$ is an orthonormal basis of $\mathbb{R}^2$ as $S$ is orthogonal. Denoting the $\vec{e}_i$ the $i$-th vector of the canonical basis of $\mathbb{R}^n$ and $\lambda_i$ and the $i$-th diagonal coefficient of $S^1{M}S$ we have without directly supposing that $\lambda_i$ is an eigenvalue for now:
	
	by multiplying by $S$ on both sides of the equality we have:
	
	and therefore:
	
	\begin{flushright}
		$\blacksquare$  Q.E.D.
	\end{flushright}
	\end{dem}
	To finish about the spectral theorem in this book, let us so reprove a result we get earlier but that was presented in a quite ugly way and poorly  rigorous (the sum of the eigenvalues equals the trace of a matrix):
	
	Remember that spectral theorem therefore tells us that for any symmetric matrix $M$, there exists an orthogonal matrix $S$ such that:
	
	is diagonal. Nothing prevents us to choose the resulting diagonal matrix as a matrix of eigenvalues in the diagonal. What we denote usually:
	
	and as $S$ is a real orthogonal matrix and that by definition we have that a matrix is orthogonal if and only if $A^{-1}=A^T$, then we find the following relation:
	
	Commonly named a "\NewTerm{spectral decomposition}\index{spectral decomposition}" or "\NewTerm{eigendecomposition}\index{eigendecomposition}".
	
	So obviously have we will have to find $S$ if $M$ is known or vice versa. Anyway, let us come back on our topic and take track of this relation:
	
	Then by using the property of the trace $\text{tr}$, of the associativity of the matrix multiplication, and the orthogonality of $S$ we have:
	
	This reprove the results seen earlier above with a condition that was not trivial at this time: the matrix must be symmetrical (or symmetrizable)!
	
	We also have by extension:
	
	and therefore by using the proven property relatively to the determinant (during our proofs of the main determinant properties) and  the conjugated matrices we get:
	
	and therefore if $M$ is symmetric we have the important property:
	
	
	\begin{tcolorbox}[title=Remark,colframe=black,arc=10pt]
	 This result is in this text a particular form of the more general case (thus also applicable to rectangular or also non-orthogonal matrices) and that seems to be named the "\NewTerm{Eckart-Young theorem}\index{Eckart-Young theorem}".
	\end{tcolorbox}
	
	\begin{tcolorbox}[colframe=black,colback=white,sharp corners]
	\textbf{{\Large \ding{45}}Example:}\\\\
	We want to show that:
	
	we assume that we know that the eigenvalue-eigenvector pairs are:
	
	We therefore introduce $S$ and $\Lambda$ as follows:
	
	We must show that $M=S\Lambda S^{-1}$. This is indeed the case, since:
	
	Therefore $M$ is indeed diagonalizable.
	\end{tcolorbox}
	
	\pagebreak
	\subsection{Matrix Decompositions}
	In the mathematical discipline of linear algebra, a "\NewTerm{matrix decomposition}\index{matrix decomposition}" or "\NewTerm{matrix factorization}\index{matrix factorization}" is a factorization of a matrix into a product of matrices in the purpose to simplify problems (typically to simplify the resolution of linear systems especially when the matrix to invert are huge!) or to make emerge some interesting properties in various cases.
	
	There are many different matrix decompositions. Each finds use among a particular class of problems. There is almost two dozens of matrix decomposition techniques as far as we know. However in the texts that will follow below we will focus only on techniques that are used for practical explicit applications in this book (mainly in the field of Statistics and by extensions in financial engineering and also Machine Learning).
	
	\subsubsection{Singular Value Decomposition (SVD)}\label{singula value decomposition}
	The singular value decomposition of a matrix $M$ ($m\times n$) is the factorization of $M$ into the product of three matrices:
	
	where the columns of $U$ ($m\times r$) and $V$ ($r\times n$) are orthonormal such that\footnote{In other words, $U$ and $V$ are rotation matrices such that $U^TU=\mathds{1}_{m\times m}$ and $V^TV=\mathds{1}_{r\times r}$.}:
	
	and the matrix $D$ ($n\times n$) is diagonal with positive real entries.  
	\begin{tcolorbox}[title=Remark,colframe=black,arc=10pt]
	If $V$ is orthogonal, then as we have seen at page \pageref{orthogonal matrix}, we have $V^{-1}=V^T$, then that latter relation can also be written:
	
	\end{tcolorbox}
	The SVD is useful in many tasks as Data Mining, Image Processing and Advanced Numerical Methods.

	To gain insight into the SVD, we treat the rows of an $m\times n$ matrix $M$ as $m$ points in a $n$-dimensional space and consider the problem of finding the best $k$-dimensional subspace with respect to the set of points. Here "best" means minimize the sum of the squares of the perpendicular distances of the points to the subspace. 

	Let us begin with a special case of the problem where the subspace is 1-dimensional: a line through the origin. We will see later that the best-fitting $k$-dimensional subspace can be found by $k$ applications of the best fitting line algorithm. Finding the best fitting line through the origin with respect to a set of points $\{x_i|1 \leq i \leq m\}$ in the plane means minimizing the sum of the squared distances of the points to the line. Here distance is measured perpendicular to the line. The problem is then named as we know: the "best least squares fit" (\SeeChapter{see section Numerical Methods page \pageref{least squares method}}).
	
	Consider projecting a point $\vec{x}_1$ onto a line through the origin:
	\begin{figure}[H]
		\centering
		\includegraphics[scale=1]{img/algebra/svd.jpg}
		\caption{The projection of the point $\vec{x}_i$ onto the line through the origin in the direction of $\vec{v}$}
	\end{figure}
	Then us Pythagorean theorem we get:
	
	That is:
	
	Therefore (see figure):
	
	To minimize the sum of squares for the distance to the line, one could minimize:
	
	minus the sum of the square of the lengths of the projections of the points to the line. However,  $\sum_{i=1}^m (x_{i1}^2+x_{i2}^2+\ldots+x_{in}^2)$ is constant! (independent of the line), so minimizing the sum of the squares of the distances is equivalent to maximizing the sum of the squares of the lengths of the projections onto the line. Similarly for best-fit subspaces, we could maximize the sum of the squared lengths of the projections onto the subspace instead of minimizing the sum of squared distances to the subspace.
	
	\paragraph{Singular Vectors and Values}\mbox{}\\\\
	We will now build the "\NewTerm{singular vectors}\index{singular vectors}" (and also "\NewTerm{singular values}\index{singular values}") of a $m\times n$ matrix $M$. 

	Consider the rows of $M$ as $m$ points in a $d$-dimensional space. Consider the best fit line through the origin. Let $\vec{v}$ be a unit vector along this line.

	The length of the projection of $\vec{x}_i$, the $i$-th row $M$, onto $\vec{v}$ is (\SeeChapter{see section Vector Calculus page \pageref{dot product}}):
	
	That we will denote for what follows as (\SeeChapter{see section Vector Calculus page \pageref{dot product}}):
	
	So in our case:
	
	From this we denote the sum of all lengths of the projections by:
	
	The best fit line is the on maximizing $\|M\vec{v}\|^2$ (ie: $\|M\vec{v}\|$) and hence minimizing the sum of the squared distances of the points to the line.
	
	With this in mind, we define the "\NewTerm{first singular vector $\vec{v}_1$}\index{first singular vector}", of $M$, which is a vector, as the best fit through the origin for the $m$ points in $n$-space that are the rows of $M$. Thus:
	
	The scalar value:
	
	is named the "\NewTerm{first singular value}\index{first singular value}" of $M$. Notice that $\sigma_1^2$ is therefore implicitly the sum of the squares of the projections of the points to the line determined by $\vec{v}_1$.
	
	The greedy approach to find this time not the best fit $1$-dimensions but $2$-dimensional subspace for a matrix $M$, takes $\vec{v}_1$ as the first basis vector for the $2$-dimensional subspace and finds the best $2$-dimensional subspace containing $\vec{v}_1$.

	Thus, instead of looking for the best $2$-dimensional subspace containing $\vec{v}_1$, look for a unit vector, denoted $\vec{v}_2$, perpendicular to $\vec{v}_1$ that maximizes $\|M\vec{v}\|^2$ among all such unit vectors.

	Using the same strategy to find the best three and higher dimensional subspaces, defines $\vec{v}_3,\vec{v}_4,\ldots$ in similar manner.
	
	The "\NewTerm{second singular vector $\vec{v}_2$}", is defined by the best fit line perpendicular to $\vec{v}_1$:
	
	The value:
	
	is named the "\NewTerm{second singular value}" of $M$. 

	The "\NewTerm{third singular vector $\vec{v}_3$}" is defined similarly by:
	
	and so on...
	
	The process stop theoretically when we have found $\vec{v}_1,\vec{v}_2,\ldots,\vec{v}_r$ as singular vector that satisfies:
	
	
	As the $\vec{v}_i$ are perpendiculars, if we apply $M$ on all this vectors, the resulting vectors will also be perpendicular between them!  

	Therefore we build the vectors:
	
	that are all perpendiculars vectors between them as already mentioned and named "\NewTerm{left singular vectors}\index{left singular vectors}" of $M$ when the $\vec{v}_i$ will be named "\NewTerm{right singular vectors}\index{right singular vectors}". The SVD theorem will fully explain the reason for these terms.
	
	\begin{theorem}
	Let $M$ be an $m\times n$ matrix with right singular vector $\vec{v}_1,\ldots,\vec{v}_r$, left singular vectors $\vec{u}_1,\ldots,\vec{u}_r$, and corresponding singular values $\sigma_1,\ldots,\sigma_n$. Then the "\NewTerm{singular value decomposition theorem}\index{singular value decomposition theorem}" states that:
	
	\end{theorem}
	\begin{dem}
	We start naturally from:
	
	\begin{tcolorbox}[title=Remark,colframe=black,arc=10pt]
	Don't forget that $\vec{x}^T\vec{x}$ in Linear Algebra gives a scalar that is equivalent to the "dot product" ("inner product"), when instead $\vec{x}\vec{x}^T$ gives a square matrix named the "\NewTerm{outer product}\index{outer product}\label{outer product}", or sometimes "\NewTerm{Gram matrix}\index{Gram matrix}\label{Gram matrix}", defined by:
	
	\end{tcolorbox}
	Now let us take the a special case with (as I don't like the general proof):
	

	and let us wee what gives:
	
	So if we look closely the result is the same as if we define the matrix;
	
	Therefore we can see that:
	
	leads to the same result. Therefore we can write:
	
	But we must not forget that if  $V$ is an orthogonal matrix, then it represents an orthonormal basis and then we have proved already earlier above that in this case:
	
	Therefore:
	
	And this finish the proof!
	\begin{flushright}
		$\blacksquare$  Q.E.D.
	\end{flushright}
	\end{dem}
	The reader should also notice that:
	
	and:
	
	also often denoted $M=U\Sigma V^T$ (as $\Sigma$ is the upper case letter of $\sigma$), are equivalent notation for the same thing (just develop the last one explicitly and you will see you fall back on the same result\footnote{On request we can write the details})! The difference is that the notation with the sum is most used in Data Mining and that with the matrices in Statistics.
	\begin{figure}[H]
		\centering
		\includegraphics[scale=1]{img/algebra/svd_multiplication.jpg}
		\caption{The SVD decomposition of a $m\times n$ matrix}
	\end{figure}
	It is usage to build the matrix $D$ such that the diagonal is in descending order of amplitude and to order the vectors $\vec{v}_i$ in the corresponding order. The reason is quite easy to understand as you can see in the example further below.
	
	This is important to know that this not the only possible decomposition of a matrix. The are many other one but we will focus in this book only the decomposition that are directly useful for engineering topics presented in this book.
	\begin{tcolorbox}[title=Remark,colframe=black,arc=10pt]
	Some authors prefers to work with the norm of the singular values, that is, with $\sqrt{\sigma_i}$, therefore the left singular value are defines as:
	
	Without that it change our previous proof result that will just be:
	
	That in Europe is frequently written (but can bring to confusion with the notation of eigenvalues...):
	
	or in matrix form (...):
	
	\end{tcolorbox}
	Let us take a practical example in image processing of SVD (made by Jason Liu in MATLAB™). The image below is an image made of $400$ unique row vectors (the reader can found the equivalent example in our R companion book):
	\begin{figure}[H]
		\centering
		\includegraphics[scale=0.6]{img/algebra/svd_feynman_original.jpg}
		\caption{SVD MATLAB™ example original image}
	\end{figure}
	What happens if in the sum:
	
	we take only the first biggest singular vector?:
	\begin{figure}[H]
		\centering
		\includegraphics[scale=0.6]{img/algebra/svd_feynman_r_equal_1.jpg}
		\caption[]{SVD MATLAB™ SVD with $r=1$}
	\end{figure}
	What happens if we take the first two singular vectors?:
	\begin{figure}[H]
		\centering
		\includegraphics[scale=0.6]{img/algebra/svd_feynman_r_equal_2.jpg}
		\caption[]{SVD MATLAB™ SVD with $r=2$}
	\end{figure}
	...and if we take the first ten singular vectors?:
	\begin{figure}[H]
		\centering
		\includegraphics[scale=0.6]{img/algebra/svd_feynman_r_equal_10.jpg}
		\caption[]{SVD MATLAB™ SVD with $r=10$}
	\end{figure}
	...and if we take the first fity singular vectors?:
	\begin{figure}[H]
		\centering
		\includegraphics[scale=0.6]{img/algebra/svd_feynman_r_equal_50.jpg}
		\caption[]{SVD MATLAB™ SVD with $r=50$}
	\end{figure}
	There we have it! Using $50$ unique values and we get a decent representation of what $400$ unique values look like.
	
	So as we can see SVD is a great space reduction technique!
	
	\paragraph{Special case of SVD for symmetric matrices}\mbox{}\\\\
	The SVD for symmetric matrices is the unique case that interest us for practical purposes in this book (for the PCA: Principal Component Analysis). Therefore we will do a special focus on this latter case.
	
	We will prove here in two steps that first, for a symmetric matrix, the eigenvectors are orthogonal and secondly that the Singular vectors and Singular values are respectively equal to the eigenvectors and eigenvalues for symmetric matrices.
	
	Let us see the first step! In general, for any matrix, it's eigenvectors (when the exist) are NOT always orthogonal to each other. But for a special type of matrix, symmetric matrix, the eigenvalues are always real and the corresponding eigenvectors are always orthogonal!
	
	\begin{dem}
	Let us take a real symmetric matrix $M$ and two distinct eigenvalues of $M, \lambda_{1}$ and $\lambda_{2}$, such that $M \vec{x}_{1}=\lambda_{1} \vec{x}_{1}$ and:
	
	where $\vec{x}_{1}$ and $\vec{x}_{2}$ are obviously eigenvectors.

	From $M \vec{x}_{1}=\lambda_{1} \vec{x}_{1},$ we get:
	
	From $M \vec{x}_{2}=\lambda_{2} \vec{x}_{2},$ we similarly get:
	
	But since $M$ is symmetric:
	
	Also, clearly $\vec{x}_{1}^{T} \vec{x}_{2}=\vec{x}_{2}^{T} \vec{x}_{1} .$ Thus: 
	
	But $\lambda_{1}-\lambda_{2} \neq 0$. Hence:
	
	as required.
	\begin{flushright}
		$\blacksquare$  Q.E.D.
	\end{flushright}
	\end{dem}
	Now we treat the second step, for this let us consider $A$ be a symmetric $n \times n$ matrix. Then the maximum value of $|A \vec{x}|,$ where $\vec{x}$ ranges over unit vectors in $\mathbb{R}^{n},$ is the largest singular value $\sigma_{1},$ and this is achieved when $\vec{x}$ is an eigenvector of $A^{T} A$ with eigenvalue $\sigma_{1}^{2}$.
	
	\begin{dem}
	Let $\vec{v}_{1}, \ldots, \vec{v}_{n}$ be an orthonormal basis for $\mathbb{R}^{n}$ consisting of eigenvectors of $A^{T} A$ (indeed as $A^{T} A$ is symmetric as $(AA^T)^T=(A^T)^TA^T=AA^T$ - this even if $A$ is not symmetric itself - then according to the result just seen below, the eigenvectors are orthogonal) with eigenvalues $\sigma_{i}^{2} .$ If $\vec{x} \in \mathbb{R}^{n},$ then we can expand $\vec{x}$ in this basis as:
	
	for scalars $\{c_{1}, \ldots, c_{n}\}\in \mathbb{R}$ since $\vec{x}$ is a unit vector, $\|\vec{x}\|^{2}=1$, which (since the vectors $\vec{v}_{1}, \ldots, \vec{v}_{n}$ are orthonormal) means that:
	
	On the other hand:
	
	By:
	
	since the $\vec{v}_{i}$ are eigenvectors of $A^{T} A$ with eigenvalues $\sigma_{i}^{2},$ we have:
	
	Taking the dot product with:
	
	and using the fact that the vectors $\vec{v}_{1}, \ldots, \vec{v}_{n}$ are orthonormal, we get:
	
	Since $\sigma_1$ is the largest singular value, we get:
	
	Equality holds when $c_{1}=1$ and $c_{2}=\ldots=c_{n}=0$. Thus the maximum value of $\|A \vec{x}\|^{2}$ for a unit vector $\vec{x}$ is the eigenvalue $\sigma_{1}^{2},$ which is achieved when $\vec{x}=\vec{v}_{1}$ (equality between the vector and one of the eigenvector of $A^{T} A$).
	\begin{flushright}
		$\blacksquare$  Q.E.D.
	\end{flushright}
	\end{dem}
	One can similarly show that $\sigma_{2}$ is the maximum of $\|A \vec{x}\|$ where $\vec{x}$ ranges over unit vectors that are orthogonal to $\vec{v}_{1}$. Likewise, $\sigma_{3}$ is the maximum of $\|A \vec{x}\|$ where $\vec{x}$ ranges over unit vectors that are orthogonal to $\vec{v}_{1}$ and $\vec{v}_{2} ;$ and so forth.
	
	So that why in the case of symmetric (obviously square) matrices, the SVD denoted often:
	
	is then denoted (remember that $\Lambda$ is the upper case letter of $\lambda$, that latter being the traditional notation of eigenvalues):
	
	A common question that we can found on Internet forums is: \textit{Why are singular values always non-negative?}

	To answer let us take some, non zero singular value $\sigma_{i}$. We can reverse the sign if it is positive. That is, $-\sigma_{i}=-\sqrt{\lambda_{i}^{2}}=-\lambda_{i}$ where $\lambda_{i}$ is an eigenvalue of $A^{T} A$ corresponding to an eigenvector $v_{i}$ . That is $A^{T} A v_{i}=\lambda_{i}^{2} v_{i}$. 
	
	Who can stop us to write instead $A^{T} A\left(-v_{i}\right)=\lambda_{i}^{2}\left(-v_{i}\right) ?$ What this means is that we can reverse the sign of a singular value, but then we need to go to the matrix $V$ and reverse the sign of its corresponding eigenvector column.

	Hence, there is not a unique way to write $A=U D V^{T}$. But if we decide that all $\sigma_{i}$ are non-negative, then "yes" there is a unique way to write $A=UD V^{T}$. Of course all $\sigma_{i}$ are sorted from largest to smallest (otherwise there would be a bunch of possibilities by permuting any two columns of $U$ and $V$ and their corresponding eigenvalues).
	
	\paragraph{Orthogonal decomposition (eigenvalue spectral decomposition)}\mbox{}\\\\
	Let us come back to:
	
	If $M$ is symmetric, than it is a square matrix. Then obviously $U$ and $V$ are also square matrices have the same dimensions as those of $M$.
	
	If $M$ symmetric, then it is also obviously equal to it's transpose $M^T$, so if we do the SVD of the transpose we should get the same decomposition as $M^T$ such that:
	
	But as we have proved during our study of matrix transposition, we have:
	
	But we should have the equality:
	
	Therefore:
	
	So that for symmetric matrices $M$ we have:
	
	As before $V$ (so was $U$) is orthogonal but now it is also a square matrix. Hence it is invertible. In this case we have proved in the section of Linear Algebra that:
	
	So that finally for symmetric matrices $M$, we get the special case of SVD, named the "\NewTerm{orthogonal decomposition}\index{orthogonal decomposition}\label{orthogonal decomposition}" or "\NewTerm{eigenvalue spectral decomposition}\index{eigenvalue spectral decomposition}":
	
	Also often denoted for the obvious reason already mentioned above:
	
	
	\begin{tcolorbox}[colback=red!5,borderline={1mm}{2mm}{red!5},arc=0mm,boxrule=0pt]
\bcbombe Be careful to not make a confusion between Spectral Value Decomposition (SVD), also named Singular Value Decomposition, and Eigenvalue Spectral Decomposition (EVD) that is the special case of SVD but for square symmetric matrices. So that means in order for the SVD of a matrix $M$ to be equal to its eigendecomposition we need $M$ to:
	\begin{itemize}
		\item Have orthonormal eigenvectors (hence be a symmetric real matrix)
	
		\item Have positive eigenvalues (colloquially, it must be a real matrix and not "flip" anything)
	\end{itemize}	
	\end{tcolorbox}
	Notice that also the square root of $M$ (written here as $M^{1/2}$) - such that $M^{1/2}M^{1/2}=M$ - can be easily found to be:
	
	Indeed, we can easily check that is equality is correct:
	
	Another useful result for later (study of the statistical distribution of the Mahalanobis distance) is the eigenvalue decomposition of the inverse of a matrix $M$ using the relation about invertible matrices proved at page \pageref{inverse matrix property} and the associative property of the matrix product\label{inverse eigendecompsosition}:
	
	
	\subsubsection{$LU$ Decomposition}\label{lu decomposition}
	As an alternative to Gaussian elimination, a non-singular matrix $A$ may be written in the form $A = LU$ where $L$ is a lower triangular matrix (all entries above the main diagonal are zero) and all the entries on the main diagonal are unity:	
	
	This is known as "\NewTerm{$LU$ decomposition}\index{$LU$ decomposition}" or "\NewTerm{$LU$ factorization}" or "\NewTerm{triangular triangularization}".
	
	\begin{tcolorbox}[title=Remark,colframe=black,arc=10pt]
	$LU$ decomposition is not unique: if $A = LU$, then $A = LDD^{-1}U=(LD)(D^{-1}U) = L' U'$ is again an $LU$ decomposition, if $D$ is a diagonal matrix. An additional assumption $l_{ii} = 1$, $\forall i = 1, \ldots, n$, (impose that $L$ is a lower triangular unit matrix) guarantees the uniqueness.
	\end{tcolorbox}
	
	For example	(three by three case):
	
	$U$ is an upper triangular matrix (all entries below the main diagonal are zero). For example (three by three case):
	
	The method of LU decomposition involves writing the system $A\vec{x} = \vec{b}$ as $LU\vec{x} = \vec{b}$. If $U\vec{x}$ is written as $\vec{y}$, the system becomes $L\vec{y} = \vec{b}$. Due to the form of $L$ (triangular matrices make solution straightforward), $\vec{y}$ may be found quickly. Once this is done, the equation $U\vec{x} = \vec{y}$ may be solved for $\vec{x}$. 
	
	The tricky bit is probably finding $L$ and $U$. For a $3 \times 3$ matrix $A$.
	
	Let:
	
	Equating coefficients:
	
	These equations may be solved as follows:
	
	\begin{itemize}
		\item Solve [1] for $u_{11}$ 
		\item Solve [2] for $l_{21}$
		\item Solve [3] for $l_{31}$
		\item Solve [4] for $u_{12}$
		\item Solve [7] for $u_{13}$
		\item Solve [5] for $u_{22}$
		\item Solve [6] for $l_{32}$
		\item Solve [8] for $u_{23}$
		\item Solve [9] for $u_{33}$ 
	\end{itemize}
	This procedure seems to be named the "\NewTerm{Crout's algorithm}\index{Crout's algorithm}".
	
	As companion example let's solve:
	
	\hspace{6.1cm} $A$ \hspace{1.6cm} $\vec{x}$ \hspace{1.4cm}
	$\vec{b}$
	
	Let:
	
	i.e.:
	
	so, comparing coefficients:
	
	Or written in a more general way for $U$:
	
	and for $L$:
	
	We see that there is a calculation pattern, which can be expressed as the following relations:
	
	\begin{tcolorbox}[title=Remark,colframe=black,arc=10pt]
	We notice in the second relation that to get the $l_{ij}$ below the diagonal, we have to divide by the diagonal element (pivot)$u_{jj}$ , so we get problems when $u_{jj}$ is either $0$ or very small, which leads to numerical instability.
	\end{tcolorbox}
	
	Therefore:
	
	The equation $LU\vec{x} = \vec{b}$ becomes $L\vec{y} =
	\vec{b}$ i.e.:
	
	So (this procedure is named "forward substitution"):
	
	Where in general we can see that the pattern is:
	
	The equation $U\vec{x} = \vec{y}$ becomes 
	
	So (this procedure is named "backward substitution":
	
	Where in general we can see that the pattern is:
	
	For large matrices, this procedure involves fewer operations than Gaussian elimination. However, Gaussian Elimination is probably easier to program.
	
	The matrix $U$ is identical to the triangular matrix found in Gaussian Elimination;	$\vec{y}$ is identical to the right hand side immediately before the back-substitution.
	
	$LU$ decomposition can be viewed as the matrix form of Gaussian elimination. Computers usually solve square systems of linear equations using $LU$ decomposition, and it is also a key step when inverting a matrix or computing the determinant of a matrix. $LU$ decomposition was introduced by mathematician Tadeusz Banachiewicz in 1938.
	

	\subsubsection{Cholesky's Decomposition}\label{Cholesky decomposition}
	The Cholesky is another way of solving systems of linear equations. It can be significantly faster and uses less memory than the $LU$ decomposition by exploiting the property of symmetric matrices. However, it is required that the matrix being decomposed be Hermitian (or real-valued symmetric and thus square) and positive definite.
	
	For many symmetric matrices (strictly speaking for positive definite matrices\footnote{Don't forget that the variance-covariance matrix is a positive definite matrix!} $\vec{x}^{T} A \vec{x} > 0$ for all $x \ne 0$ as detailed at page \pageref{positive definite matrix}), $LU$ decomposition can be carried out using $U = L^{T}$. Therefore:
		
	This is known as "\NewTerm{Cholesky's decomposition}\index{Cholesky's decomposition}". While the Cholesky decomposition only works for symmetric, positive definite matrices, the more general LU decomposition works for any square matrix!
	
	\pagebreak
	As companion example, let us use Cholesky's method to solve:
	
	\hspace{6.5cm} $A$ \hspace{1.4cm} $\vec{x}$ \hspace{1.4cm}
	$\vec{b}$ \\
	Let:
	
	i.e.:
	
	So:
	
	Plus duplicates of [2], [3] and [5]
	\begin{itemize}
		\item From [1], $l_{11} = 2$,
		\item From [2], $l_{21} = -1$,
		\item From [3], $l_{31} = 1$,
		\item From [4], $l_{22} = 4$,
		\item From [5], $l_{32} = 0$,
		\item From [6], $l_{33} = 3$ 
	\end{itemize}
	So:
	
	The system becomes:
	
	i.e:
	
	so:
	
	$\vec{x}$ satisfies i.e.:
	
	
	
	
	We can see that from the above example that the diagonal elements $l_{kk}$ of $L$ there is a calculation pattern:
	
	For the elements below the diagonal ($l_{ik}$ , where $i>k$) there is also a calculation pattern:
	
	We see that the bothe case have a pattern that be written respectively:
	
	We can now understand for what the positive definiteness was necessary! Indeed, the positive definiteness of $A$ guarantees that all square roots are real. We can also notice from the above developments that $L$ needs to have positive diagonal entries\footnote{The reader should however not forget that from what we have seen during our study of positive definite matrices, the fact that all entries are positive makes an that the matrix is symmetric is not enough to make it positive definite}!
	
	Suppose now a matrix $A$ factors as $A=L^TL$. Then:
	
	This shows that $A$ is positive semidefinite.
	
	If we further assume that $L$ is square and triangular with positive real diagonal entries, then $L$ is invertible, so $L\vec{x}=\vec{0}\Leftrightarrow \vec{x}=\vec{0}$. In this case, we see that $A$ is positive definite.
	
	Some textbooks use Cholesky decomposition to define what makes a matrix positive definite. Indeed, such textbooks only say that any matrix $A$ that can \underline{uniquely} be written under a Cholesky decomposition $A=L^TL$ are positive definite matrices.
	
	\begin{tcolorbox}[title=Remark,colframe=black,arc=10pt]
	The above algorithms show that every positive definite matrix $A$ has a Cholesky decomposition. This result can be extended to the positive semi-definite matrix! However for positive semi-definite, it is possible but not for sure the Cholesky decomposition is not unique.
	\end{tcolorbox}
	
	
	\subsubsection{$QR$ Decomposition}\label{QR decomposition}
	A "\NewTerm{QR decomposition}\index{$QR$ decomposition}", also named a "\NewTerm{$QR$ factorization}", of a matrix is a decomposition of any real matrix $A$ into a product $A = QR$ of an orthogonal matrix $Q$ and an upper triangular matrix $R$:
	
	with $\text{col}_i(Q)\circ \text{col}_j(Q)=0, \forall i\neq j$ and (we will show if further below) $r_{ij}=\text{col}_i(Q)\circ \text{col}_j(A), \forall 1\leq 1\leq i\leq n$.
	
	There are several methods for actually computing the $QR$ decomposition, but we will focus here on the Gram–Schmidt process whose idea is described on page \pageref{gram-schmidt procedure}.
	
	So Consider the Gram–Schmidt process applied to the columns of the matrix $A$ and let us recall the result that we have derived during our study of the Gram–Schmidt process:
	
	That we will rewrite in our case:
	
	And in the $QR$ decomposition it is traditional to take:
	
	Therefore with this tradition we have ($Q$ is then not only orthogonal but orthonormal!):
	
	Before we continue, let us show an companion example. Let us consider the matrix:
	
	Let us carry out the Gram-Schmidt process with the columns $\text{col}_1(A)$ and $\text{col}_2(A)$:
	
	That is how the Gram-Schmidt process produces the matrix $Q$. Here are one of the methods to find $R$.
	
	For this the reader must understand (and see!) that at each step of Gram-Schmidt procedure, the operations on the vectors correspond to column operations $A$, which correspond to multiplying by elementary matrices on the right. Let us write all those matrices:
	
	From this we get (using the properties on the inverse matrices):
	
	We have found:
	
	Let us check the factorization:
	
	Let us rewrite the equalities using symbols, in order to obtain the general relation:
	
	From this we get:
	
	This relation for $R$ generalizes to any value of $n$. For example, if $n$ were $3$, $R$ would be given
by:
	
	Knowing this, there is no need to rewrite the whole computation every time. We can just
build the matrix $R$ as we go along the Gram-Schmidt process. So more generally:
	

	There is also another way to get $R$.  Indeed, the fact that $Q$ has orthonormal columns can be restated as we know as $Q^TQ=\mathds{1}$. In particular, $Q$ has a left inverse, namely $Q^T$. From this find $R$:
	
	In other words, the relation:
	
	holds, no matter what the dimensions of the matrix.
	
	If we take bake our example, let us recall that we had:
	
	Then:
	
	
	So to summarize:
	\begin{table}[H]
		\begin{tabular}{lll}
		\rowcolor[HTML]{C0C0C0} 
		\textbf{Factorization} & \textbf{Restrictions} & \textbf{Properties of Factors} \\ \hline
		\begin{tabular}[c]{@{}l@{}}SVD\\ $A_{nm}=U_{nn}D_{nm}V^T_{mm}$\end{tabular} & \multicolumn{1}{c}{none} & \begin{tabular}[c]{@{}l@{}}$U$ orthogonal\\ $V$ orthogonal\\ $D$ non-negative\end{tabular} \\
		\multicolumn{3}{c}{variations: for symmetric $A$, $A=VCV^T$} \\ \hline
		\begin{tabular}[c]{@{}l@{}}$LU$\\ $A_{nn}=L_{nn}U_{nn}$\end{tabular} & $A$ square, (others) & \begin{tabular}[c]{@{}l@{}}$L$ full-rank lower triangular\\ $U$ upper triangular\end{tabular} \\
		\multicolumn{3}{c}{\begin{tabular}{ll}
variations: & with partial pivoting, $A=LP$ \\
 & with full pivoting, $P_1AP_2=LU$ \\
 & $A=LDU$, with $D$ diagonal and $U_{ii}=1$
\end{tabular}} \\ \hline
		\begin{tabular}[c]{@{}l@{}}$QR$\\ $A_{nm}=Q_{nn}R_{nm}$\end{tabular} & \multicolumn{1}{c}{none} & \begin{tabular}[c]{@{}l@{}}$Q$ orthogonal\\ $R$ upper triangular\end{tabular} \\
		\multicolumn{3}{c}{variations: skinny $QR$ for $n>m$, $A=Q_1R_1$} \\ \hline
		\begin{tabular}[c]{@{}l@{}}Cholesky\\ $A_{nn}=L_{nn}U_{nn}$\end{tabular} & $A$ non-negative definite & \begin{tabular}[c]{@{}l@{}}$L$ full-rank lower triangular\\ $U$ upper triangular\end{tabular} \\ \hline
		\begin{tabular}[c]{@{}l@{}}Diagonal (ie orthogonal)\\ $A_{nn}=V_{nn}C_{nn}V^T_{nn}$\end{tabular} & $A$ symmetric & \begin{tabular}[c]{@{}l@{}}$V$ orthogonal\\ $C$ diagonal\end{tabular} \\ \hline
		\begin{tabular}[c]{@{}l@{}}Square root\\ $A_{nn}=(A^{1/2}_{nn})^2$\end{tabular} & $A$ non-negative definite & $A^{1/2}_{nn}$ non-negative definite \\ \hline
		\end{tabular}
		\caption{Matrix Factorizations}
	\end{table}
	
	\pagebreak
	\subsection{Woodbury Matrix identity and Sherman-Morrison relation}
	The "\NewTerm{Woodbury matrix identity}\index{Woodbury Matrix Identity}\label{Woodbury matrix identity}" gives the inverse of an $n\times n$ square matrix ${ A}$ modified by a perturbation term ${ UBV}^T$:
	
	It is a very useful useful result for our study on the inverse and determinant of a partitioned symmetric matrix (itself useful for the Factor Analysis statistical tool). It has also other practical application like for PPCA (Probabilistic Principal Component Analysis) or for hierarchical linear models!
	
	The proof is straightforward:
	\begin{dem}
	
	\begin{flushright}
		$\blacksquare$  Q.E.D.
	\end{flushright}
	\end{dem}
	Consider some special cases:
	\begin{itemize}
		\item If ${ U}={ V}={ \mathds{1}}$, then we get:
		
		
		\item If ${ B}={ \mathds{1}}$, and let ${ U}=[{\vec u}_1,\ldots,{\vec u}_m]$ and ${ V}=[{\vec v}_1,\ldots,{\vec v}_m]$, then we get the inverse of a rank-$n$ modified matrix:
		
		
		\item More specially, when $m=1$, ${ U}={\vec u}$ and ${ V}={\vec v}$ we get the inverse of rang-$1$ modified matrix:
		
	\end{itemize}
	Let us see two possible derivations of that later case! First possible proof:
	\begin{dem}
	We first show the following is an identity ($A=\mathds{1}$):
	
	Pre-multiplying ${ \mathds{1}}+{\vec w}{\vec v}^T$, the right side becomes ${ \mathds{1}}$ as well as the left side:
	
	We next let ${\vec u}={ A\vec w}$ in:
	
	and the left side of the relation to be proven becomes:
	
	Substituting $\vec{w}={A}^{-1}{\vec u}$ we get the "\NewTerm{Sherman-Morrison relation}\index{Sherman-Morrison relation}\label{Sherman-Morrison relation}":
	
	\begin{flushright}
		$\blacksquare$  Q.E.D.
	\end{flushright}
	\end{dem}
	And second possible proof:
	\begin{dem}
	This second proof is based on the following problem. Assuming a linear equation system ${\ A}{\vec y}={\vec b}$ is solve to get ${\vec y}={ A}^{-1}{\vec b}$, we want to solve this system:
	
	We first pre-multiply both sides of this equation by ${\ A}^{-1}$ to get:
	
	If we define ${\vec w}={ A}^{-1}{\vec u}$ and $\alpha={\vec v}^T{\vec x}$, the above equation can be written as:
	
	Pre-multiplying both sides by ${\vec v}^T$, we get:
	
	Solving for $\alpha$ we get:
	
	Substituting $\alpha$ into the previous equation for ${\vec x}$ we get:
	
	But solving the equation $({\vec A}+{\vec u}{\vec v}^T){\vec x}={\vec b}$ we also get:
	
	we therefore have:
	
	\begin{flushright}
		$\blacksquare$  Q.E.D.
	\end{flushright}
	\end{dem}
	
	\begin{flushright}
	\begin{tabular}{l c}
	\circled{95} & \pbox{20cm}{\score{4}{5} \\ {\tiny 36 votes,  79.94\%}} 
	\end{tabular} 
	\end{flushright}
	
	%to make section start on odd page
	\newpage
	\thispagestyle{empty}
	\mbox{}
	\section{Tensor Calculus}\label{tensor calculus}
	\lettrine[lines=4]{\color{BrickRed}T}he conventional vector calculus is a simple and effective technique that adapts perfectly to the study of mechanical and physical properties of matter in an Euclidean space of three dimensions. However, in many fields of physics, it appears experimental quantities that can't be easily represented by simple column vectors of Euclidean vector spaces. This is the case for example in continuum mechanics (fluids or solids), electromagnetism, General Relativity, etc.
	
	Thus, since the late 19th century, the analysis of forces acting within a continuous medium led to highlight the physical quantities characterized by nine numbers representing the pressure forces or internal stress (see section Continuum Mechanics page \pageref{continuum mechanics} for the details). The representation of these quantities required the introduction of a new mathematical tool who was named "\NewTerm{tensor}\index{tensor}", by reference to its physical origin. Subsequently, starting the years 1900, it was R. Ricci and T. Levi-Civita who developed the tensor calculus; then the study of tensor allowed a deepening of the theory of vector spaces and contributed to the development of differential geometry (see section of the same name page \pageref{differential geometry}).
	
	Tensor calculus, also sometimes named "\NewTerm{absolute differential geometry}\index{absolute differential geometry}" also has the advantage to free itself from all coordinate systems and the results of  the mathematical developments are thus invariant (huge simplification in calculations but in compensation we have a huge increase in abstraction and notation complexity). We therefore don't need to be concerned in what type of referential frame we work and this is very interesting in General Relativity.
	
	We advise strongly the reader to very well master the basics of vector calculus and linear algebra as they have been presented before in previous sections (especially because linear algebra forms the skeleton of tensor calculus!). If necessary, we have chosen when writing this section to come back on certain points seen in the section of Vector Calculus and Linear Algebra (covariant components, contravariant components, etc.).
	
	Furthermore, if the reader has already covered the study of constraints in solids (\SeeChapter{see section Continuum Mechanics page \pageref{constraints}}) or of the Faraday tensor (\SeeChapter{see section Electrodynamics page \pageref{faradey tensor}}) or the energy-momentum tensor (\SeeChapter{see section General Relativity page \pageref{energy momentum tensor}}) this will be a practical advantage before reading what follows. Furthermore, the redaction of the above items (tensors) was made so that the concept of tensor is  introduced if possible (...) intuitively.
	
	We will only do a very few practical examples in this section. Indeed the examples, you have probably already guess it..., will come when we will study the continuum mechanics, General Relativity, quantum field theory, electrodynamics, etc.
	
	An advice maybe: you have seen many time in the section of Statistics that write with vectors and after think with matrix was a powerful way to generalize some important results. For this section on tensors remember that the idea is the same but we think matrix and we write tensor! (you will better understand this little adage once you will be finished to read this whole section).
	
	\subsection{Tensor}
	\textbf{Definition (simplistic \#\mydef):} The "\NewTerm{tensors}\index{tensor}" are mathematical objects generalizing the concepts of vectors and matrices. They were introduced in physics to represent the state of stress and deformation of a volume subjected to forces, hence their name (tensions).
	
	The rigorous definition requires (I personally think...) to have first read this section in its whole. But you must know that in fact a tensor is roughly like a determinant... (\SeeChapter{see section Linear Algebra page \pageref{determinant}}). Eh yes! It is simply a multilinear application on a space of a given size (corresponding to the number of columns of the matrix/tensor) which finally gives a scalar (of a given field).
	
	\label{tensor notation}For example, we have proved in the section of Continuum Nechanics that normal and tangential forces in a fluid were given by the relation:
	
	what was noted in the traditional condensed form as following (where we no longer distinguish what is tangential to what is normal so there is a loss of clarity):
	
	We thus make appear a mathematical quantity $\sigma_{ij}$ with $9$ components, while a vector in the same space $\mathbb{R}^3$ has $3$ components.
	
	This notion is also much used in the section of General Relativity where we have proved that the energy-momentum tensor in a particularly simple case was given by:
	
	and satisfies the non-less important equation of conservation:
	
	Or otherwise, still in the section of General Relativity, we have shown that the tensor of the Schwarzschild metric was given by:
	
	and therefore gives us the "equation of the metric"\index{metric equation} or "invariant interval"\index{invariant interval} (\SeeChapter{see section Special Relativity page \pageref{interval invariant}}):
	
	Note also that in the section of Special Relativity we have shown that the Lorentz transformation tensor is given by:
	
	which in a condensed form gives the following components transformation:
	
	As regards the transformation of the electromagnetic field we have also proved that the Faraday tensor is given by:
	
	and therefore permits switch from one repository frame to another using the relation:
	
	But these are very simple tensor that can be represented in the form of matrices. You should also remember that it is not because you are reading a variable with indices suggests that we are dealing with a tensor that it is necessarily one. For example, the famous relation (widely used in the section of General Relativity and we will prove far further below):
	
	might suggest that the first member of the far left is a tensor but in fact it is not... this is just a symbol... hence its name: Christoffel \underline{symbol} (not: Christoffel \underline{tensor}).
	
	The interest of tensors in physics is that their characteristics are independent of the chosen coordinates. Thus, a relation between tensors in a base will be true regardless of the base used thereafter. This is a fundamental and powerful characteristic of General Relativity (among others)!
	
	\pagebreak
	\subsection{Indicial Notation}
	We will use thereafter many mathematical symbols: coordinates, components of vectors and tensors, matrix components, etc., whose number in each category is large or indeterminate. To distinguish the various symbols of a category we use indices. For example, instead of the traditional variables $x, y, z$ we will use the variables $x_1,x_2,x_3$ (as we have already done in the section of Linear Algebra). This rating becomes essential when we an undetermined number of variables.
	
	Thus, if we have $n$ variables, we denote them by $x_1,x_2,...,x_n$.
	
	We will also use superscripts, when required; eg $x^1,x^2,x^3$. To avoid confusion with writing powers, the quantity $xî$ to the power $p$ will be written $(x^i)^p$. When the context eliminates any potential ambiguity, the use of parentheses however is not fundamentally necessary.
	
	In tensor calculus there is a summation convention using the fact that the repeated index, below for example the index $i$, will become itself an indication of the summation. We write then, with this convention:
	
	thereby this condense relatively well the notations!
	
	Thus, to represent the linear system:
	
	we will write (notice carefully how are written the components of the associated matrix!):
	
	specifying that's for $n=3$.
	
	We see in this example, how the summation convention allows a condensed and thus powerful writing.
	
	The summation convention covers all the mathematical symbols having repeating indices. Thus the decomposition of a vector $\vec{x}$ on a basis $(\vec{e}_1,\vec{e}_2,\vec{e}_3)$ will be therefore written for $n=3$:
	
	In summary, any term that has a repeated index represents a sum over all possible values of the repeated index.
	
	\begin{tcolorbox}[title=Remark,colframe=black,arc=10pt]
	We name, for obvious reasons we will detail below, the $x^i$ "\NewTerm{contravariant component}\index{contravariant components}" of the vector $\vec{x}$.
	\end{tcolorbox}
	
	\pagebreak
	\subsubsection{Summation on multiple index}\label{einstein summation convention}
	The summation convention (due to Einstein) extends to the case where we have, in general, several repeated indices in upper and lower positions so-named "\NewTerm{dumb indices}\index{dumb indices}" in the same monomial (often physicists omit the rule of set their position opposite as it will be the case often on this book too!). Thus, for example, the quantity $A_i^jx^ix^j$, represents the following sum for $i$ and $j$ taking the values from $1$ to $2$:
		
	Thus we see easily that an expression with two summation indices that take values respectively $1,2,...,n$, will have $n^2$; will have $n^2$ terms, $n^3$ if there are three sommation indices, etc.
	
	However, we must be careful to substitutions with this kind of notation because if we assume that we have the relation:
	
	then to get the expression of $A$ only in function of the variables $y^j$ we cannot write:
	
	because it does not return to the same expression as the dumb indices after development are systematically sum in an identical and rigid way (we leave to the reader make a little application case to see this, if need you can contact us and we will do an example). In other words, a dumb index can not be repeated more than $2$ times.
	
	\subsubsection{Kronecker Symbol}\label{kronecker symbol}
	This symbol introduced by the mathematician Kronecker, is the following (often used in physics and in many other fields):
	
	This symbol is named "\NewTerm{Kronecker symbol}\index{Kronecker symbol}". It conveniently allows you to write, for example, the dot product of two vectors $\vec{e}_1$ and $\vec{e}_2$, of unit norm and orthogonal to each other, in the form:
	
	We will find this symbol in many examples of theoretical physics in this book (wave quantum physics, quantum field theory, general relativity, fluid mechanics, etc.).
	
	It should be noted that there is a generalized version of the Kronecker symbol:
	
	We have also, for example:
	
	where $\varepsilon_{ijk}$ is the Levi-Civita symbol that will define right now:
	
	\subsubsection{Antisymmetric Symbol (Levi-Civita symbol)}\label{levi civita symbol}
	Another useful symbol is the "\NewTerm{symbol of antisymmetry}\index{symbol of antisymmetry}" or also named "\NewTerm{antisymmetry tensor}\index{antisymmetry tensor}" that we will find in the sections of Electrodynamics,  General Relativity and Relativistic Quantum Physics in this book.
	
	In mathematics, particularly in linear algebra, tensor analysis, and differential geometry, the Levi-Civita symbol represents a collection of numbers; defined from the sign of a permutation of the natural numbers $1, 2, \ldots, n$, for some positive integer $n$.
	
		 The values of the Levi-Civita symbol are independent of any metric tensor and coordinate system. Also, the specific term "symbol" emphasizes that it is not a tensor because of how it transforms between coordinate systems, however it can be interpreted as an antisymmetric tensor (a tensor is antisymmetric on (or with respect to) an index subset if it alternates sign (+/-) when any two indices of the subset are interchanged).
	 
	 In the case $n=2$ the Levi-Civita symbol is defined by:
	 
	The values can be arranged into a $2\times 2$ antisymmetric matrix (we can see we fall back on the definition of an antisymmetric tensor):
	
	Use of the 2D symbol is relatively uncommon, although in certain specialized topics like supersymmetry and twistor theory it appears in the context of 2-spinors. The 3D and higher-dimensional Levi-Civita symbols are used more commonly.
	
	In three dimensions, the Levi-Civita symbol is defined as follows:
	
	i.e.  $\varepsilon_{ijk}$  is $1$ if $(i, j, k)$ is an even permutation of $(1,2,3)$ or in the natural order $(1,2,3)$, $-1$ if it is an odd permutation, and $0$ if any index is repeated. In three dimensions only, the cyclic permutations of $(1,2,3)$ are all even permutations, similarly the anticyclic permutations are all odd permutations. This means in 3D it is sufficient to take cyclic or anticyclic permutations of $(1,2,3)$ and easily obtain all the even or odd permutations.

	It can also be express with the Kronecker symbol:
	
	
	An illustrative representation gives:
	\begin{figure}[H]
		\centering
		\includegraphics[scale=0.75]{img/algebra/levi_civita_symbol.jpg}
		\caption[3D Levi-Civita symbol illustration]{3D Levi-Civita symbol illustration (source: Wikipedia)}
	\end{figure}
	By using this symbol, a determinant of order two (\SeeChapter{see section Linear Algebra page \pageref{determinant}}) is then written in the advantageous form:
	
	and the vector cross product:
	
	where of course, $j$ and $k$ are summed and where the dummy index $i$ is the line number of the resulting vector (if requested we will make the developments). In particular, the rotational (curl) of a vector field (\SeeChapter{see section Vector Calculus page \pageref{rotational}}) is then:
	   
	As an example, let us calculate in index notation the double vector product $\vec{A}\times\vec{B}\times\vec{C}$:
	 
	where again, the dumy index $i$ is the line number of the resulting vector. Let us eee detailed demonstration of these equalities (the order of the equalities below does not need to follow the sequence of equalities of the previous relation).
	\begin{dem}
	We have proved in the section of Vector Calculus the following identity:
	
	\begin{tcolorbox}[title=Remark,colframe=black,arc=10pt]
	The latter relation is sometimes, and for recall, named the "\NewTerm{Grassmann rule}\index{Grassmann rule}" or more commonly "\NewTerm{dual vector product}\index{dual vector product}".
	\end{tcolorbox}	
	To prove the relation:
	
	to a change of indices let us first prove that:
	
	which give us:
	
	We do the development only for the first line (this is already bor... euh long enough...):
	
	This is the first step that was necessary to be proven.
	
	Now let us prove that for the $n$th line we have well:
	
	with the help of a result obtained in the section of Vector Calculus (vector product of three different vectors) we have the first term (the first line of the vector resulting from the calculation):
	
	It is immediate that ($i$ being equal to $1$):
	
	Let us show now that for $i$ equal $1$ we also have:
	
	Indeed:
	
	\begin{flushright}
		$\blacksquare$  Q.E.D.
	\end{flushright}
	\end{dem}
	As a second example, letus prove that the divergence of a curl vanishes:
	
	By the Schwarz's theorem (\SeeChapter{see section Differential and Integral Calculus page \pageref{Schwarz theorem}}) $\partial_i\partial_j$ is symmetric (so invert the indices has no impact) in the indices and that $\varepsilon_{ijk}$ is antisymmetric (by definition) in the same indices, the sum on $i$ and $j$ must necessarily cancel. For example, the contribution to the sum of the $i=1,j=2$ is the opposite of this with $i=2,j=1$.
	
	\begin{tcolorbox}[title=Remarks,colframe=black,arc=10pt]
	\textbf{R1.} The symbol of antisymmetry is often named "\NewTerm{Levi-Civita tensor}\index{Levi-Civita tensor}" in the literature. In fact, although it is a tensor in the form of its notation, it's more of a mathematical tool that a mathematical "being" hence the preference of some physicists to name it "symbol" rather than "tensor". But it's up to you ...\\
	
	\textbf{R2.} By abuse of writing we do not write the basic vector but rigorously, and to avoid forgetting it, remember that in order to balance the members of the equality and in order to clarify that the vectors are expressed in the same base, we should write:
	
	\end{tcolorbox}
	Let us now see the first simple practical applications of this index notation using the example of the base change that we have seen in the section of Vector Calculus:
	
	Given two bases $(\vec{e}_1,\vec{e}_2,\ldots,\vec{e}_n)$ and $(\vec{e}_1^{\prime},\vec{e}_2^{\prime},\ldots,\vec{e}_n^{\prime})$ of an Euclidean vector space $\mathcal{E}^n$. Each vector of a base can be decomposed on the other base in the form of a linear application (base change matrix - see section Linear Algebra page \pageref{change of basis}):
	
	where we obviously use the summation convention for $i,k=1,2,\ldots,n$.
	
	Let us recall that the base change matrix (or "\NewTerm{transformation matrix}\index{transformation matrix}") should have as many columns as the basic vector have lines (dimensions or components). Small example with three dimensions gives:
	
	and obviously it is much more funny to write this as:
	
	so where on $A$, we have the $k$ that represents the column of the matrix and $i$ the row of the matrix.

	Any vector $\vec{x}$ of $\mathcal{E}^n$ can be decomposed (we have already prove this in the section of Vector Calculus) on each basis $\mathcal{E}^n$ under the form:
	
	If we seek for the relations between the components $x^i$ and ${x'}^k$ it is enough to take again the relations prove in the section of Linear Algebra and we have then:
	
	Immediately by the uniqueness of the decomposition of a vector on a base, we can equalize the components of the basis vectors and we get (the must be careful to rearrange again the order of the terms because the matrix multiplication is in general, not commutative as we already know!):
	
	By construction we also have the trivial relation:
	
	As:
	
	A way to prove in a quite general way the previous relation using tensor calculus notation is to remember the following result proved in the section of Linear Algebra:
	
	and by using:
	
	We therefore have:
	
	The basis vectors being linearly independent, this last relation implies that when $i\neq j$:
	
	and when $i=j$:
	
	Therefore it comes:
	
	And for the dot product, the results obtained with the index notation are very interesting and extremely powerful. We have already defined the scalar product in the section of Vector Calculus but let us see how we handle this with the index notation:

	Let us consider an Euclidean vector space $\mathcal{E}^n$ reported any basis ${\vec{e}_i}$. We already know that vectors are written on this basis:
	
	\label{metric tensor euclidean space}The scalar product with respect to its properties and the index notation is then written:
	
	This is a fundamental relation for advanced physics (General Relativity and String Theory) that makes appear the "\NewTerm{metric covariant tensor}\index{metric covariant tensor}" (\SeeChapter{see section Non-Euclidean Geometry page \pageref{riemann spaces}}):
	
	and to satisfy the commutative property of the dot product (\SeeChapter{see section Vector Calculus page \pageref{dot product}}) we must obviously have the equality (at least in Euclidean space or approximated as...):
	
	The prior-previous relation is sometimes written in the form:
	
	\begin{tcolorbox}[title=Remark,colframe=black,arc=10pt]
	When the basis vectors $\vec{e}_i$ form an orthogonal vector space (not necessarily orthonormal) the quantities:
	
	are obviously zero when $i \neq j$. The dot product of two vectors $\vec{x}$ and $\vec{y}$ is then reduce to:
	
	We the have in this particular case\label{condensed flat metric space notation}:
	
	and therefore when the basis vectors form an orthonormal vector space it is clear that $g_{ij}$ is equal to the Kronecker symbol alone such that:
	
	\end{tcolorbox}
	
	\subsection{Metric and Signature}\label{metric and signature}
	As we have seen in it in the section of Vector Calculus (and Topology), the dot product of a vector $\vec{x}$ can be used to define the concept of norm of a vector (and also the concept of distance).
	
	Let us recall that we have by definition the norm of a vector which is given by (\SeeChapter{see section Vector Calculus page \pageref{dot product}})\label{norm tensor notation}:
	
	where the numbers $g_{ij}$ define somehow a "measure" of the vectors; we then say in the language of tensor calculus that they are the "metrics" of the selected vector space.
	
	In the space of classical geometry, the norm is a number that is always strictly positive and which becomes zero if the measured vector is also zero. By cons the previous expression of the norm of a vector, may eventually be negative for any numbers $g_{11},g_{12},\ldots,g_{nn}$ (complex spaces for example). So we can distinguish two kinds pre-Euclidean vector spaces\label{pre euclidean vector space} (Euclidean space in which we define a scalar product for recall) depending on the fact that the norm is positive or not. However when in theoretical physics we want to make the analogy with a vector space structure we need that the condition:
	
	is satisfied ($g_{ij}$ can be written as a matrix, nothing avoid us to do it).
	
	Explanations: We know that the dot product must satisfy the commutative property such that:
	
	On the other hand, if for any nonzero $y^j$ we have:
	
	this implies $x^i=0$ (that is one of the properties of the norm we saw in the section of Vector Calculus). We can then write:
	
	We are here simply with a system of $n$ equations with $n$ unknowns (having to admit by hypothesis that only the solution $x^i=0$), it is necessary and sufficient for this that the determinant of the system, denoted $g$, to be is different from zero (\SeeChapter{see section Linear Algebra page \pageref{determinant matrix inverse}}). So we must have:
	
	It is one of the condition for an expression comparable to a norm under a tensor index notation form in the context of theoreticla physicsa vector space of the states of the system !!
	
	\begin{tcolorbox}[title=Remarks,colframe=black,arc=10pt]
	\textbf{R1.} The number of $+$ and $-$ signs found in the expression of the dot product is a is a characteristic of a given vector space $E^n$. It is named the "\NewTerm{signature of the vector space}\index{signature of a vector space}".\\
	
	\textbf{R2.} A practical application of calculation of the metric is presented in details in the section of General Relativity.\\
	\end{tcolorbox}
	From the coefficients of the covariant metric tensor $g_{ij}$ defining the metric of the space $E^n$, we can introduce the coefficients of the "\NewTerm{contravariant metric tensor $g^{ij}$}\index{contravariant metric tensor}" defining the metric of a "\NewTerm{dual space}\index{dual space metric}" $E_{*}^n$ by the relation:
	
	In other words, the metric tensor twice covariant is its own inverse by its equivalent twice contravariant tensor. We will prove it explicitly later by showing during our study of the of the Gram's determinant that contravariant and covariant components of a Euclidean space are equal and both space have the same number of dimensions.

	A well known special case that meets the above equality is the Minkowski's metric tensor of (\SeeChapter{see section Special Relativity page \pageref{minkowski metric}}), where we have:
	 
	\begin{tcolorbox}[title=Remark,colframe=black,arc=10pt]
	The space $E^n$ is also named "primal space" and if it is of Euclidean type let us recall that it is denoted $\mathcal{E}^n$.
	\end{tcolorbox}
	The dual space is underpinned by $n$ basis vectors $\vec{e}^i$ constructed from the vectors $\vec{e}_i$ such that:
	
	It is therefore easy to see that the scalar product of the vectors $\vec{e}^i$ defines the metric $g^{ij}$ of the dual space:
	
	while the vectors $\vec{e}^i$ (contravariant) and $\vec{e}_j$ (covariates) are orthogonal:
	
	We can also express a vector in the dual base by the next writing by noting that obviously the position of the dummy indices is reversed:
	
	\begin{tcolorbox}[title=Remark,colframe=black,arc=10pt]
	The components $x_i$ (orthogonal projections of the vector on the axes) are named, for reasons that we will see further below, the "covariant" components.
	\end{tcolorbox}
	So we finally have the possibility to change the vectors of a base to another one:
	
	where it is important to remember that to make a contravariant a covariant  component, we bring up the index:
	
	and conversely, to make it covariant:
	
	A "\NewTerm{covariant vector}\index{covariant vector}" or "\NewTerm{cotangent vector}\index{cotangent vector}" (often abbreviated as "\NewTerm{covector}\index{covector}") has components that co-vary with a change of basis. That is, the components must be transformed by the same matrix as the change of basis matrix. The components of covectors (as opposed to those of vectors) are said to be "covariant".
	
	So, still in the case of the example of the Minkowski metric, if we consider the contravariant four-vector:
	
	Then we have:
	
	
	\subsection{Gram's Determinant}
	Let us see another approach to determine the base vectors of the dual space that can allow also a better understanding of the concept and will allow us to get an interesting result that we will use during certain calculations of General Relativity and String Theory (mainly its study using to the Lagrangian formalism).
	
	So we have for $i=j=1$:
	
	This scalar product can be seen as a normalization condition for the two bases and two scalar products $\vec{e}^2\circ\vec{e}_1=0$,$\vec{e}^2\circ\vec{e}_1=0$ as orthogonalization conditions . Thus, as $\vec{e}_1$ is perpendicular to $\vec{e}^2,\vec{e}^3$ we can write:
	
	where $c^{te}$ is a constant of proportionality. Now let us play around with the prior previous relation:
	
	Then we get:
	
	where we see appear the mixed product as we had defined it in the section of Vector Calculus.
	
	Thus we get very easily:
	
	and even for contravariant vectors (without proof as may be too obvious?):
	
	\begin{tcolorbox}[title=Remarks,colframe=black,arc=10pt]
	\textbf{R1.} The reader will have perhaps noticed that the relations above are only valid for a three-dimensional space.\\
	
	\textbf{R2.} The notation of the two previous relations is mathematically a bit unfair because in reality it is not an equality between two vectors but an application of a vector space in the other one!\\
	
	\textbf{R3.} As in physics is very frequently considered Cartesian , cylindrical and spherical orthonormal base and that denominator of the two previous relations is always equal to $1$ in these bases than the contravariant basis vectors are identified with covariant basis vectors (and vice versa ). So the covariant coordinates are equal to the contravariant coordinates for these special cases!!!!!!!!!! 
	\end{tcolorbox}	
	Now let us come back on something that will seem very old of us... In the section of Vector Calculus, we have defined and studied what were the cross product and mixed product. We will now see another way of representing them and see that this representation provides a result for the less quite relevant!
	We saw in the section of Vector Calculus that the vector product was given by:
	
	But what we did not see and we will now that we can trivially this expression is only the vector determinant of the following matrices:
	
	Yes ... so the result does not give a scalar! It is just a usage rating.
	
	But as we do the tensor calculus, we must now properly distinguish covariant and contravariant components. We'll rewrite it properly with contravariant components:
	
	Similarly, the mixed product can be written using this relation and notation:
	
	Or, looking at the expression of the determinant we see quite easily, without having to do developments, that
	
	Indeed (we calculate the determinant making use of the demonstration of the determinant with three components proved in the section of Linear Algebra):
	
	The prior-previous relation is also is frequently written:
	
	with obviously:
	
	named "\NewTerm{Euclidean volume}\index{Euclidean volume}" (indeed let us recall that the mixed product is a volume as we show it in the section of Vector Calculus!)
	\begin{tcolorbox}[title=Remark,colframe=black,arc=10pt]
	Let us recall again that if the basis vectors are orthonormal, whether they are expressed in Cartesian , cylindrical or spherical coordinates then:
	
	\end{tcolorbox}
	Moreover, we also have the important relation:
	
	Moreover, we also have the important relation:
	
	Indeed, using the relation see in the of Vector Calculus:
	
	But we have seen previously that $\vec{e}^i\circ\vec{e}_j=\delta_j^i$:
	
	and thus finally:
	
	This having been done let us come back our relation of the vector product:
	
	and let express the components of the 1st line of the determinant in their dual basis (in contravariant coordinates):
	
	Obviously, if the cross product is expressed in covariant components then we have:
	
	Now let us apply the mixed product:
	
	knowing the expression of the determinant of a square $3\times 3$ matrix  (\SeeChapter{see section Linear Algebra page \pageref{3x3 matrix determinant}}) it comes immediately (we can detail on request as always in this book):
	
	Conversely, it comes almost immediately:
	
	But, we have proved in the section of Vector Calculus that $x_i=\vec{x}\circ\vec{e}_i$. It comes then:
	
	and therefore:
	
	The latter relation is often named "\NewTerm{Gram determinant}\index{Gram determinant}". A special very interesting case gives us (we use the relation between the metric components and the dot products of the vector basis we proved during our study of the metric just earlier above):
	
	written in another way:
	
	Thus, the Euclidean volume  is given by what name call the "\NewTerm{functional determinant}\index{functional determinant}" of the system (expression that we will see again and use in the section of General relativity to calculate the real volume and also in the section of String Theory):
		
	that is without units (so you have to multiply it by a factor of elementary volume to get volume units). If we note in another way the determinant:
		
	We get the common relation we can found in many books on General Relativity and String Theory but given without proof and named the "\NewTerm{Riemannian volume}\index{Riemannian volume}\label{Riemannian volume}" form or simply "\NewTerm{volume form}":
		
	or written in the following "\NewTerm{invariant volume element}\index{invariant volume element}" form:
	
	The reader can verify normally easily enough that for the orthonormal Cartesian reference frame we fall back on the volume of a cube and that for the spherical case we fall back well on the expression of the infinitesimal volume of the sphere as used in the section of Geometric Shapes (but on request we can add the details here).
	
	If we use the result we get in the section of Differential Geometry we have therefore for any surface patch:
	

	
	\subsection{Contravariant and Covariant Components}\label{contravariant and covariant components}
	So far we wrote the dummy indices arbitrarily on superscript or subscript at our discretion. However, this is not always allowed and sometimes the fact that a dummy index is in superscript or subscript has a special significance! This is often the major difficulty in the study of some theorems, because if we do not study those indices from the beginning, we do not really know how to interpret the position of the dummy indices. The reader should then be extremely careful at this level.

	For an Euclidean vector space $\mathcal{E}^n$ reported to any base $(\vec{e}_i)$, the scalar product of a vector $\vec{x}=x^i\vec{e}_i$ by a vector its base is written as we know by (remember that this is equivalent as projecting the components on the axis corresponding to $\vec{e}_j$):
	
	Therefore:
	
	This relation is of major importance in theoretical physics and tensor calculus. It is important to remember it when we will study the contraction of indexes later (you can observe in the previous relation that we have "lowered" in the left side the index of the component of the right member of the equality).
	
	These scalar products denoted $x_j$, are named "\NewTerm{covariant components}\index{covariant components}" in the base $(\vec{e}_i)$, of the vector $\vec{x}$. These components are therefore defined by:
	
	They will denoted by lower indices !!! We will see later that these components are naturally introduced for some vectors of physics, for example the gradient vector. Moreover, the notion of covariant component is essential for tensors.
	
	\begin{tcolorbox}[title=Remarks,colframe=black,arc=10pt]
	\textbf{R1.} Never forget that this is therefore the projection of a vector on a vector of its own base!!!!\\
	
	\textbf{R2.} The basic vectors always have their indices noted down because they are their own covariant components (they project on themselves by scalar product). This is the main trick used by beginners to remember when to put lower indices (and therefore they know when to put the upper one...)!
	\end{tcolorbox}
	Conversely, the "\NewTerm{contravariant components}\index{contravariant components}" (in other words: the non-projected components) can be calculated by solving with respect to the $n$ unknowns, the system of $n$ equations:
	
	The previous relations show that the covariant components $x_j$ are related to the conventional components $x^i$ and that the contravariant components $x^i$ are therefore numbers such that:
	
	They will be indicated with superscripts !! The study of the basis changes will further justify the appellation of these different components.
	
	In a canonical orthonormal basis (very special case and corresponding to the classical cartesian, polar, cylindric and shperical coordinates), the covariant and contravariant components are the same as we already know after our study of Gram's determinant. Indeed:
	
	\begin{tcolorbox}[title=Remark,colframe=black,arc=10pt]
	We see above, that the incessant writing  of dummy superscript or subscript indices  can sometimes lead to some confusion and serious headaches ...
	\end{tcolorbox}
	Incidentally, when we refer to a vector (or, more generally, a tensor) as being either contravariant or covariant we're abusing the language slightly, because those terms really just signify two different conventions for interpreting the components of the object with respect to a given coordinate system, whereas the essential attributes of a vector or tensor are independent of the particular coordinate system in which we choose to express it. In general, any given vector or tensor (in a metrical manifold) can be expressed in both contravariant and covariant form with respect to any given coordinate system!
	
	Furhtermore It may seem that the naming convention is backwards, because the "contra" components go with the axes, whereas the "co" components go against the axes, but historically these names were given on the basis on the transformation laws that apply to these two different interpretations.
	
	Most of the physical definitions refer to the contravariant components. But we must be aware in physics that we have a for example a singularity in a tensor, if it is physical or only due to the choice of coordinate system! There is nothing inherent in the contravariant, covariant, or mixed components of for example on stress energy tensor that cannot be changed with a coordinate transformation, EXCEPT for the scalars. 
	
	\pagebreak
	\subsection{Operation in Basis}
	The interest of physicist for the tensor calculus, is passing parameters from one base to another for some given reasons (often the aim is to simplify the study of problems or simply because the studied states depend - or may depend - on the geometry of the space in question). It is therefore necessary to introduce the main tools relating thereto. We will also take this opportunity to present the developments that could have been already addressed in the section of Vector Calculus.
	\begin{tcolorbox}[title=Remark,colframe=black,arc=10pt]
	As I like to say... Tensor Calculus is to physics with is XML to computing science. A background independent language!
	\end{tcolorbox}
	
	\subsubsection{Gram-Schmidt Orthogonalization Method}
	The "\NewTerm{Schmidt orthogonalization method}\index{Schmidt orthogonalization method}\label{gram-schmidt procedure}" (also named "\NewTerm{Gram-Schmidt orthogonalization method}\index{Gram-Schmidt orthogonalization method}") allows the actual determination of an orthogonal basis for any pre-Euclidean vector space $\mathcal{E}^n$ (we could introduce this method in the section of Vector Calculus but it seemed more interesting to us to present this method in a general and aesthetic framework using tensor calculus).
	
	For this, let us consider a set of $n$ linearly independent vectors $(\vec{x}_1,\vec{x}_2,\ldots,\vec{x}_n)$ of $\mathcal{E}^n$ and suppose that for each vector we have the dot product (square of the norm):
	
	Let us seek $n$ vectors $\vec{e}_i$ orthogonal between them. Let us start for this with $\vec{e}_1=\vec{x}_1$ and let us seek for $\vec{e}_2$ orthogonal to $\vec{e}_1$ under the form (this is a choice!!!):
	
	The mental visualization of the process is not quit easy so the reader has to trust (anyway...) the mathematical results (if once we have the time we will draw the process of the classical three dimension case).
	
	The coefficient $\lambda_1$ is calculated by writing the orthogonality relation:
	
	We deduce without too much troubles:
	
	The parameter $\lambda_1$ being determined, we get the vector $\vec{e}_2$ that is orthogonal to $\vec{e}_1$ and not zero, since the system is linearly independent $(\vec{e}_1,\vec{x}_2,\ldots,\vec{x}_n)$.
	
	Thus so far we have:
	
	The parameter $\lambda_1$ being determined, we get the vector $\vec{e}_2$ that is orthogonal to $\vec{e}_1$ and not zero, since the system $(\vec{e}_1,\vec{x}_2,\ldots,\vec{x}_n$ is linearly independent.
	
	The vector $\vec{e}_2$ is sought in the form:
	
	The two relations of orthogonality: $\vec{e}_1\circ\vec{e}_e=0$ and $\vec{e}_2\circ\vec{e}_3=0$, enables the calculation of the coefficients $\mu_1$ and $\mu_2$ . We get therefore:
	
	what determines the vector $\vec{e}_3$, orthogonal to $\vec{e}_1$ and $\vec{e}_2$, and not zero, since the $(\vec{e}_1,\vec{e}_2,\vec{x}_3,\ldots,\vec{x}_n)$ are independent. By continuing the same type of calculation, we get step by step a system of orthogonal vectors $(\vec{e}_1,\vec{e}_2,\vec{x}_3,\ldots,\vec{e}_n)$ between them and none of them are zero.
	
	So finally:
	
	
	In case where some vectors are such like $\vec{x}_i\circ\vec{x}_i=0$ (their norm is zero), we replace then $\vec{x}_i$ by $\vec{x'}_i+\lambda\vec{x}_j$, choosing a vector $\vec{x}_j$ so that we get $\vec{x'}_i\circ\vec{x'}_i\neq 0$.
	
	We therefore conclude that any pre-Euclidean space admits orthogonal bases!
	\begin{tcolorbox}[colframe=black,colback=white,sharp corners]
	\textbf{{\Large \ding{45}}Example:}\\\\
	Suppose:
	
	We want to find an orthonormal set of vector that spans $\{\vec{x}_1,\vec{x}_2,\vec{x}_3\}$. For this we use the Gram-Schmidt followed by a normalisation, let $\vec{e}_1=\vec{x}_1=(1,1,1)$ then we calculate:
	
	As a quick check we have indeed:
	
	Next:
	
	again it's good to check that $\vec{e}_1\circ\vec{e}_2=0$ and $\vec{e}_1\circ\vec{e}_3=0$ as we desire. Finally we notice that:
	
	Hence:
	
	are orthonormal vectors.
	\end{tcolorbox}
	
	This system of calculation of bases is of primary importance! It can be used to study physical systems from a pre-Euclidean repository whose properties change over time. Which is typical in General Relativity. 
	
	The reader interested in computer science and numerical methods can refer to the $QR$ matrix decomposition further below (see \pageref{QR decomposition}) to see another application of the Gram-Schmidt rationalization method.

	\subsubsection{Change of Basis}\label{change of basis tensor calculus}
	Given two bases $(\vec{e}_1,\ldots,\vec{e}_n)$ and $(\vec{e'}_1,\ldots,\vec{e'}_n)$ of a vector space $\mathcal{E}^n$. Each vector of a base can be decomposed on the other basis as follows (we have already prove it):
	
	A vector $\vec{x}$ of $\mathcal{E}^n$, when its contravariant components known, can be decomposed in each base in the form:
	
	and we have already proved that:
	
	We notice that the transformation relations of the components of a contravariant vare the opposite of those of the basis vectors, the quantities $A$ and $A '$ being permutted, hence also the origin of the name "\NewTerm{contra}"-"\NewTerm{variants}" of these components!

	Let $x_i$ and ${x'}_k$ be the covariant components of a vector $\vec{x}$ respectively in the bases $(\vec{e}_i)$ and $(\vec{e'}_k$. Let us replace the basis vectors expressed by the relations:
	
	in the expression of the definition of the covariant components, therefore we get:
	
	Hence the relation between the covariant components in each base:
	
	We get also:
	
	We notice that the covariant components transform as the basis vectors, hence also the name of these components!
	
	Once again, unless the basis is orthonormal, never forget that the covariant and contravariant components are different!!!
	
	\subsubsection{Reciprocal Basis (Dual Basis)}
	Now let us come back on the concept of dual space but as seen in the vector calculation point of view. This second approach can perhaps help some readers to better understand the concepts seen previously but against hides the underlying reasoning for the origin of the names "covariant" and "contravariant". But it is still the most common presentation used in the literature...
	
	Given a basis ($\vec{e}_i$) of an Euclidean vector space $\mathcal{E}^n$. By definition, $n$ vectors $\vec{e}^k$ which satisfy the following relations:
	
	are named "\NewTerm{reciprocal vectors}\index{reciprocal vectors}" of the vectors $\vec{e}_i$. They will be denoted with higher indices. By definition, each reciprocal vector $\vec{e}^k$ must therefore be orthogonal to all the vectors $\vec{e}_i$, except for $k=i$.
	
	Let us first show that the reciprocal vectors $\vec{e}^k$ of a given base $(\vec{e}_i)$ are linearly independent. For this, we must show that a linear combination $\lambda\vec{e}^k$ gives a zero vector if and only if each coefficient $\lambda_k$ is zero.
	\begin{dem}
	Given $\vec{e}=x^i\vec{e}_i$ any vector of $\mathcal{E}_n$. Let us make a dot product by $\vec{x}$ the previous linear combination $\lambda_k \vec{e}^k$, we get:
	
	The latter equality must be verified whatever the $x^i$, it is therefore necessary that each $\lambda_i$ is zero and thus the vectors $\vec{e}^k$ are indeed linearly independent vectors.
	\begin{flushright}
		$\blacksquare$  Q.E.D.
	\end{flushright}
	\end{dem} 
	The system of $n$ reciprocal vectors forms a basis named the "\NewTerm{reciprocal basis}\index{reciprocal basis}" (which is just the dual basis) of the vector space $\mathcal{E}^n$.
	\begin{tcolorbox}[colframe=black,colback=white,sharp corners]
	\textbf{{\Large \ding{45}}Example:}\\\\
	Given three vectors $\vec{e}_1,\vec{e}_2,\vec{e}_3$ forming a basis (not necessarily orthonormal!) of an euclidean space. We decide to define first:
	
	where, for recall, the symbol $\times$ represents the cross product (\SeeChapter{see section Vector Calculus page \pageref{cross product}}) and the whole is the  mixed product as also seen in the section Vector calculus and represents an oriented volume.\\
	
	The following vectors:
	
	form the dual basis!\\
	\begin{tcolorbox}[title=Remark,colframe=black,arc=10pt]
	We recognize here the relations we have just proved earlier above during our study of the Gram's determinant!!!
	\end{tcolorbox}
	\end{tcolorbox}
	Now, let us consider a vector on the original base $\vec{e}_1,\vec{e}_2,\vec{e}_3$ that we will denote by (as seen above):
	
	with therefore by definition the contravariant components of the vector that appear as we defined earlier above (and that we had at the same time explained the origin of the name). We also saw above that each contravariant component will also (naturally and by extension) be given by:
	
	Similarly, so we have the covariant components that appear:
	
	In this approach, we then define the contravariant  metric tensor and respectively covariant:
	
	It comes therefore for example for the contravariant components (in the case of a three-dimensional space), knowing that the approach is the same for the covariant components:
	
	And so we find the transformation relations between the  contravariant and covariant components already seen above with the difference that it seems coming out of a hat by successive definitions and that therefore hides the origin of the name of these components (at least in our point of view). But perhaps some readers prefer this approach...???
	
	\begin{tcolorbox}[colframe=black,colback=white,sharp corners]
	\textbf{{\Large \ding{45}}Example:}\\\\
	As an example, consider the basis:
	
	Notice that it is not orthogonal because $\vec{e}_1\circ \vec{e}_2=4\neq 0$.\\

	In this case we have by applying the previous Gram's relations:
	
	As we can see in the figure below where $\vec{e}_1$ and $\vec{e}_2$ are shown in green, and the reciprocal vectors  $\vec{e}^1$ and  $\vec{e}^2$ are shown in blue:\\
	\begin{figure}[H]
		\centering
		\includegraphics[scale=0.75]{img/algebra/reciprocal_basis_contravariant_components.jpg}
		\caption[]{Basis vectors, reciprocal basis vectors and contravariant components}
	\end{figure}
	Notice that by construction we have indeed that $\vec{e}^1$ is orthogonal to $\vec{e}_2$ and $\vec{e}^2$ is orthogonal to $\vec{e}_1$!
	\end{tcolorbox}
	
	\begin{tcolorbox}[colframe=black,colback=white,sharp corners]
	For a given vector $\vec{a}$, say:
	
	we can use the relation proved earlier to find its contravariant components in the basis $(\vec{e}_1,\vec{e}_2,\vec{e}_3)$. We get:
	
	So that (see figure above):
	
	Now observe that the original basis vectors are reciprocal of the reciprocal ones. Thus we can just as well expand the same vector $\vec{a}$ along the basis vectors:
	
	with $a_i=\vec{a}\circ\vec{e}_i$.\\

	The components $a_i$ are the covariant components of $\vec{a}$. In our example, we obtain:
	
	so we have:
	
	\begin{figure}[H]
		\centering
		\includegraphics[scale=0.75]{img/algebra/reciprocal_basis_covariant_components.jpg}
		\caption[]{Basis vectors, reciprocal basis vectors and covariant components}
	\end{figure}
	\end{tcolorbox}
	
	\subsection{Euclidean Tensors (cartesian tensor)}
	The generalization of the concept of vector has led us to the study of vector spaces to $n$ dimensions. Tensors are also one-dimensional vectors but possess additional properties compared to vectors.

	For the theoretical physicist, tensor calculus is interesting primarily in how the components of the tensor are transformed during a change of basis vector spaces from which they come. We will begin to study them vis-à-vis the properties of bases changes (because it is the most interesting case).

	A tensor is in practice often only defined and used in the form of its components. These can be expressed in covariant or contravariant form like any vector. But a new type of components will appear in the tensor, it is the "mixed components". These three types of components are decomposition of Euclidean tensor on different bases.
	
	\textbf{Definition (\#\mydef):} A "\NewTerm{Cartesian tensor}\index{Cartesian tensor}" uses an orthonormal basis to represent a tensor in a Euclidean space in the form of components.

	Use of Cartesian tensors occurs in physics and engineering, such as with the Cauchy stress tensor (\SeeChapter{see section Continuum Mechanics page \pageref{cauchy stress tensor}}) and the moment of inertia tensor in rigid body dynamics (\SeeChapter{see section Classical Mechanics page \pageref{inertia tensor}}). Sometimes general curvilinear coordinates are convenient, as in high-deformation continuum mechanics, or even necessary, as in General Relativity (\SeeChapter{see section General Relativity page \pageref{general relativity}}).
	
	\subsubsection{Fundamental Tensor}
	During the theory view earlier above, we used the quantities $g_{ij}$, defined from the scalar product of the basis vectors $(\vec{e}_i)$ of a $n$-dimensional pre-Euclidean  vector space $\mathcal{E}^n$, by:
	
	These $n^2$ quantities are the covariant components of a tensor named the "\NewTerm{fundamental tensor}\index{fundamental tensor}" or as we already know the "\NewTerm{metric tensor}\index{metric tensor}".
	
	Let us study how vary the quantities $g_{ij}$ when we make a basis change:

	Given $({e'}_k)$ another based linked to the previous by the known relation:
	
	Substituting the relation $\vec{e}_i={A'}_i^k\vec{e'}_k$ in the expression of $g_{ij}$, it come (we change the indices as it should be done during a substitution):
	
	In the new base $(\vec{e'}_k)$, the dot products of the basis vectors are therefore quantities such that:
	
	So we finally have for the expression of the covariant components $g_{ij}$ in a basis change:
	
	Identically we have:
	
	In general, a sequence of $n^2$ quantities $t_{ij}$ that transforms, during a base change of $\mathcal{E}^n$, according to the two previous relations, namely:
	
	are, by definition, the "\NewTerm{covariant components of a tensor of order two}" (with two indices) on $\mathcal{E}^n$.

	We can therefore manipulate quantities expressing the intrinsic properties of bases as standard tensors!
	
	\subsubsection{Tensor product (dyadic) of two vectors and matrices}\label{tensor product}
	Let us consider an Euclidean vector space $\mathcal{E}^n$ of base $(\vec{e}_i)$ and given two vector of $\mathcal{E}^n$:
	
	Let us form the two by two products of contravariant components $x^i$ and $y^j$, namely:
	
	We thus get $n^2$ quantities, if the two vectors have the same number of components, which are also the contravariant components of a tensor of order two named the "\NewTerm{tensor product}\index{tensor product}" of the vector $\vec{x}$ by the vector $\vec{y}$.
	
	For example for $\vec{x}$ of dimension $2$ and $\vec{y}$ of dimension $3$ we have:
	
	We can also tensorally multiply two matrices $A$ and $B$. Then, the matrix describing the tensor product $A\otimes B$ is the "\NewTerm{Kronecker product}\index{Kronecker product}\label{kronecker product}" of the two matrices as we use it in Relativistic Quantum Physics.
	
	For example, if:
	
	Then:
	
	The most famous case is the covariant tensor of rank $2$ in a space of $4$ dimensions as it is the most used one in tensor calculus:
	
	The reader can also now better understand the origin of the name of the gradient of a vector field (giving a "tensor field") as we saw in the section of Vector Calculus because we can rewrite it now:
	
	 The reader will have certainly notice that through the examples above, the tensor product is non-commutative. That is:
	
	We can obviously build tensor products of order three (thus with $n^3$ terms) such as with the following tensor three times  contravariant vectors:
	
	etc.

	Let us study the properties of the basis changes of these components. Let us use for the basis changes relations of contravariant components of a vector, namely:
	
	Let us replace in the relation $u^{ij}=x^iy^j$ the components $x^i$ and $y^i$ by their basis change expression, we get:
	
	The quantities ${u'}^{kl}$ are the new components:
	
	The transformation formula of the $n^2$ quantities $u^{ij}$ on a change of basis change of $\mathcal{E}^n$ is finally (very similar to metric tensor):
	
	Such a change basis relation characterizes the contravariant components of a tensor of order two. Conversely, we get:
	
	So the $n^2$ quantities are the "\NewTerm{contravariant components of a tensor of order two}\index{contravariant components of a tensor of order two}".

	We can the build the same products by pairs for covariant components $x_i$ and $y_i$ of the vectors $\vec{x}$ and $\vec{y}$ thus:
	
	The formulas of basis change of the covariant components of the vectors are given by the following relations that we have already proved previously:
	
	Substituting the first relation in the product $u_{ij}=x_iy_i$, we get:
	
	This is the basis change relation of covariant components of a tensor of order two. We also easily check that we have:
	
	Identically we have of course ${u'}_{kl}={x'}_k{y'}_l$ since $u_{ij}=x_iy_i$.

	So the $n^2$ quantities are then the "\NewTerm{covariant components of a tensor of order two}\index{covariant components of a tensor of order two}".

	Let us now create $n^2$ quantities my multiplying two by two the covariant components of a vector $\vec{x}$ by contravariant components of a vector $\vec{y}$, we get:
	
	Let us perform a basis change in this last relation taking into account the expressions $x_i={A'}_i^k{x'}_k$ and $x^i={A}_k^i {x'}^k$, we get:
	
	This basis change relation characterizes the "\NewTerm{mixed components}\index{mixed components}" of an order two tensor. Conversely, we can verify that we have:
	
	These mixed components also constitutes the components of a tensor product of $\vec{x}$ by $\vec{y}$, according to a given basis.
	
	In general, a sequence of  $n^2$ quantities that transforms, during a basis of $\mathcal{E}^n$, just as previously established relation are therefore, by definition, "\NewTerm{mixed components of a tensor of order two}\index{mixed components of a tensor of order two}".
	
	\pagebreak
	\subsubsection{Tensor Spaces}
	In the previous study, we used as system of $n^2$ number, created from a vector space $\mathcal{E}^n$. When these numbers satisfy some basis change relations, we name these quantities, by definition, "\NewTerm{components of a tensor}\index{components of a tensor}".

	We have seen that any linear combination of these components constitutes the components of a new tensor. We can therefore add together the components of the tensor and multiply by a scalar, to get other components of a new tensor. These addition and multiplication properties mean that we can use these tensors quantities as vector components.
	
	To clarify how we define a tensor on a base, let us study the particular case of a tensor product of two vectors formed by triplets of numbers (that is to say in $\mathbb{R}^3$ typically). Consider therefore the Euclidean vector space $\mathcal{E}^3$ whose vectors are triplets of number of the form $\vec{x}=(x_1,x_2,x_3)$. The canonical orthonormal basis of $\mathcal{E}^3$ consists of three vectors that we know very well but written in tensor calculs as:
	
	with $i=1,2,3$ (nice way to write simple things isn't it...).
	Vectors of $\mathcal{E}^3$ gives the possibility to form the nine quantities that we have named the "\NewTerm{components of the tensor product}\index{components of a tensor product}" of the vectors $\vec{x}$ and $\vec{y}$.
	
	If we make all possible tensor products between vectors of $\mathcal{E}^3$, we get sequences of $9$ numbers that can be used to define the following vector:
		
	\begin{tcolorbox}[title=Remark,colframe=black,arc=10pt]
	We see immediately with the above relation and the previous relation that the tensor product is therefore not commutative.
	\end{tcolorbox}
	We are left then with the elements of a vector space $\mathcal{E}^9$ with nine-dimensional space, whose elements all combinations by pairs of three numbers.
	
	We then say that $\mathcal{E}^9$ has a "\NewTerm{tensor product structure}\index{tensor product structure}" which is denoted obviously and in standard calculus by $\mathcal{E}^9:\; \mathcal{E}^3\otimes \mathcal{E}^3$ or sometimes $\mathcal{E}_3^{(2)}$.

	These vectors can be decomposed, for example, on an orthonormal canonical basis:
	
	with $k=1,2,\ldots,9$.

	If we rewrite the quantities $x^iy^j$ according to their place in the expression of $\vec{U}$, ie:
	
	with $k=1,2,\ldots,9$ and $i,j=1,2,3$, the vectors $\vec{U}$ are then written:
	
	and is an example as we know of tensor of order $2$ (obviously we can generalize this approach).
	How do these tensor differ from ordinary vectors? Although they are identical to some vectors of $\mathcal{E}^9$ in our example but were formed by the vectors $\vec{x}$ and $\vec{y}$ of $\mathcal{E}^3$. To remember this fact, we write then as we already know:
	
	and they are named as we already know "tensor products of order two" of the vectors $\vec{x}$ and $\vec{y}$. The symbol $\otimes$ is defined in the way we have formed the quantities $x^iy^j=u^{ij}$ and in the order in which they were classified them to form the vector $\vec{U}$.

	To recall the dependence between a quantity $x^iy^j=u^{ij}$ and the basis vector $\vec{e}_i$ to which he is assigned, les us rewrite these vectors by putting in place of the index $k$ the two indices $i$ and $j$, relative to the components, namely:
	
	The latter can be written in the form:
	
	The vectors $\vec{e}_i\otimes\vec{j}$ generates a basis of $\mathcal{E}^9$ in the case of our example with is same the "\NewTerm{tensor associated basis}\index{tensor associated basis}".
	\begin{tcolorbox}[colframe=black,colback=white,sharp corners]
	\textbf{{\Large \ding{45}}Example:}\\\\
	Consider:
	
	We then have for example:
	
	That is to say:
	
	\end{tcolorbox}
	It follows that as element of a space $\mathcal{E}^n\otimes \mathcal{E}^n$, a tensor $\vec{U}$ is a vector of the general form:
	
	Let us study its properties vis-à-vis a base change of $\mathcal{E}^n$ such that:
	
	During such a base change, the base $(\vec{e}_i\otimes\vec{e}_j)$ associated to $\vec{e}_i$  becomes another base $(\vec{e}_k^{\prime}\otimes\vec{e}_l^{\prime})$ associated to $\vec{e}_k^{\prime}$, that is:
	
	It follows that the tensor product $\vec{U}$ has for components in the new basis:
	
	We have the following properties for the tensor product given:
	
	\begin{enumerate}
		\item[P1.] Right/Left distribituvity relatively to the addition of vectors:
		
		The proof of these relation is simple deduce from the definition of the tensor product. Indeed, we have for example:
		
		
		\item[P2.] Associativity with multiplication by a scalar:
		
		Indeed, we have:
		
	
		\item[P3.] When we choose a base in each of the vector spaces $(\vec{e}_i)$ for $\mathcal{E}^n$, $(\vec{f}_i)$ to for $\mathcal{F}^m$, the $n\cdot m$ elements of $G_{nm}$ that we denote by $\vec{e}_i \otimes\vec{f}_i$ also form a basis of $G_{nm}$.
		\begin{dem}
			Already made in the particular example we used earlier above.
		\begin{flushright}
			$\blacksquare$  Q.E.D.
		\end{flushright}
		\end{dem}
	\end{enumerate}
	\begin{tcolorbox}[title=Remark,colframe=black,arc=10pt]
	In practice, we often have to use tensor formed from vectors belonging to the same vector spaces $\mathcal{E}^n$.
	\end{tcolorbox}
	We can of course generalize the tensor product to any number of vectors. Gradually, given the property P1, we can consider $p$ vectors $\vec{x}_1,\vec{x}_2,\ldots,\vec{x}_p$ each belonging to different vector spaces $\mathcal{E}^{n_1},\mathcal{E}^{n_2},\ldots,\mathcal{E}^{n_p}$. If we have:
	
	we can form the tensor product:
	
with $i_1=\{1,\ldots,n_1\},i_2=\{1,\ldots,n_2\},\ldots,i_p=\{1,\ldots,n_p\}$.

	We build thus tensor products of order $p$ belonging to the vector space $\mathcal{E}^{n_1}\otimes\mathcal{E}^{n_2}\otimes\ldots \otimes\mathcal{E}^{n_p}$, space that has a product structure tensor. The elements of this space are by definition tensor of order $p$.

	In order to unify the classification, the elementary vector spaces, which can not be fitted with a tensor product structure can be regarded as having components of a tensor of order $1$. In general, we name these elements "\NewTerm{vectors}\index{vector}", reserving the term "\NewTerm{tensor}\index{tensor}" to elements of tensor spaces of order equal or greater than $2$!
	\begin{tcolorbox}[title=Remark,colframe=black,arc=10pt]
	It is of usage to name "\NewTerm{tensor of order zero}\index{tensor of order zero}" scalar quantities. It is also rare to meet tensor of order tensor greater than $2$.
	\end{tcolorbox}
	It is quite obvious and we will not do the proof  that we absolutely can redefine all the concepts (base, decomposition on a base, reciprocal basis, dot product, tensor product) that we have seen so far considering tensor of order $1$ as a vector (we should therefore rewrite everything that was already written above... which is useless in our point of view).

	It is also quite possible to repeat all these definitions for higher order tensor and thus generalize the concept of space tensor for all dimensions.

	From these considerations, we can state the "\NewTerm{tensoriality criterion}\index{tensoriality criterion}":
	
	So that the elements of a sequence of $n^p$, relative to a base of a vector space $\mathcal{E}_{(p)}^n$, can be considered as the components of a tensor, it is necessary and sufficient that these quantities to be linked together, in two different bases of $\mathcal{E}_{(p)}^n$, by the relations of transformation of the components.
	\begin{tcolorbox}[title=Remark,colframe=black,arc=10pt]
	A vector can be represented in any base by a sequence of $n$ components. However, we can not conclude that any sequence of $n$ numbers is a vector. Indeed, when we put ourselves in another base of space, the components must also change to represent the same object, then we say that the vector is an intrinsic object (whose existence does not depend on the choice of the base). It remains then to know that a vector is a tensor of order $1$.	
	\end{tcolorbox}

	\subsubsection{Linear combination of tensors}
	We can form other tensor by combining together the components of various tensor products defined using vectors of the same vector space. For example, let us consider the contravariant components of the tensor products of the vectors $\vec{x},\vec{y}$ and $\vec{w},\vec{z}$ :
	
	Let us form the following quantities:
	
	The $n^2$ quantities $t^{ij}$ also satisfied the general formulas for basis change. We have indeed by substituting the relations of transformation of the  contravariant components of a tensor product in the previous expression:
	
	The $n^2$ quantities $t^{ij}$, satisfying the relations of basis change also constitutes components of a tensor of order two.

	\subsubsection{Contraction of indices}
	Let us consider the mixed tensor product of two vectors $\vec{x}$ and $\vec{y}$ of respective contravariant $x^i$ and covariant $y_j$ components . The mixed components of the tensor product $\vec{V}$ of these two vectors are:
	
	Let us perform the addition of the various components of the tensor $\vec{V}$ such as $i=j$, ie:
	
	We thus get the expression of the dot product of vectors $\vec{x}$ and $\vec{y}$. The quantity $v$ is a scalar (tensor of order zero). Such an addition on different variance indices constitutes, by definition, the operation of "\NewTerm{contraction of indices}\index{contraction of indices}" of the tensor $\vec{V}$. This operation allowed us to move from a tensor of order two to a tensor of order zero. The tensor $\vec{V}$ has been amputated and of a covariance and a contravariance.
	
	Let us also take the example of a tensor $\vec{U}$ whose mixed components are one time covariant and one two times contravariant $u_k^{ij}$ (caution ... it is not a three-dimensional matrix but simply an indication that the components of this tensor are expressed from three other variables!!!). Let us consider some of its components such as $k=j$, that is the components $u_j^{ij}$ and let us perform the addition of the latter. We then get:
	
	These new quantities $v^i$ form the components of a tensor $\vec{V}$ of order one (thus a vector!) and constitute what we name then the "\NewTerm{contracted components}\index{contracted components}" of the tensor $\vec{U}$ and of course meet the basis relations change (on request we can prove it but you have to know that it is similar to the one we made for vectors). So we have indeed change form a tensor of order three to a tensor of order one!
	
	So we can see that the underlying idea of the contraction is to allow us to facilitate the resolution of a purely mathematical problem and depending on the situation it may be good  to raise or reduce the order of a tensor. This is often a choice that is made by trial and error based on a specific context or that naturally arises from a purely mathematical or mathematical-physical development (as we will see examples further below).

	\paragraph{Raising and lowering indices}\mbox{}\\\\
	If we start with a contravariant or covariant components tensor , we can lower / raise one or more indices by multiplication (if repeated) by $g_{ij}$ or $g^{ij}$ (unitary diagonal and positive signature: of canonical type) to get mixed components on which we can then perform contraction operations.
	
	Let us consider an Euclidean tensor $\vec{U}$ of contravariant components $u^{i_1i_2i_3\ldots i_p}$

	If we want to perform a contraction on this tensor, we will first need to transform it into a mixed tensor. This transformation we be done using a fundamental tensor.

	Let us write $\vec{U}$ in mixed components by lowering at the covariant position the index $i_1$ by example (it is therefore equivalent to express this in contravariant component in covariant component). So:
	
	We see well that in this case to take down a contravariant index in a tensor using a fundamental tensor, we must first go search in the covariants components of the fundamental tensor the one who is itself in contravariant position in the original tensor and replace its position (but this time in covariance) by the other index of the fundamental tensor (it is the same idea when it we desire to operate a contraction on a covariant tensor ).
	
	Indeed, let us recall that we have proved that:
	
	Also remember that raising and then lowering the same index (or conversely) are inverse operations, which is reflected in the covariant and contravariant metric tensors being inverse to each other:
	
	Now that we got a mixed tensor components, we can very well getting contract the indices. Let us choose for example the index $i_2$ and let us perform the contraction with the index $j_1$, let us put $i_2=j_1=k$ (we are then concerned only to some specific terms), then just writing the whole process from the beginning:
	
	So we get after lowering the index and one contraction, a tensor of order $p-2$.
	\begin{tcolorbox}[colframe=black,colback=white,sharp corners]
	\textbf{{\Large \ding{45}}Examples:}\\\\
	E1. Let us see a first example of raising down and contracting the covariant $4$-position first order tensor given by (\SeeChapter{see section Special Relativity page \pageref{four-vector displacement}}):
	
	in components:
	
	(where $x_j$ are the usual Cartesian coordinates) and the Minkowski metric tensor with signature $(-+++)$ given by (\SeeChapter{see section Special Relativity page \pageref{minkowski metric}}):
	
	in components:
	In components:
	
	To raise the index, multiply by the tensor and contract:
	
	Then for $\lambda = 0$:
	
	and for $\lambda = j = 1, 2, 3$:
	
	So the index-raised contravariant $4$-position is:
	
	\end{tcolorbox}
	
	\begin{tcolorbox}[colframe=black,colback=white,sharp corners]
	E2. For a second order tensor example let us consider the contravariant electromagnetic tensor in the $(+---)$ signature is given by (\SeeChapter{see section Electrodynamics page \pageref{electromagnetic tensor}}):
	
	in components:
	
	To obtain the covariant tensor $F_{\alpha\beta}$, we multiply by the metric tensor and contract:
	
	and since $F^{00} = 0$ and $F^{0i}=-F^{i0}$, this reduces to:
	
	Now for $\alpha = 0, \beta = k = 1, 2, 3$:
	
	and by antisymmetry, for $\alpha = k = 1, 2, 3$, $\beta = 0$:
	
	then finally for $\alpha = k = 1, 2, 3$, $\beta = \ell = 1, 2, 3$:
	
	The (covariant) lower indexed tensor is then:
	
	\end{tcolorbox}
	
	\begin{tcolorbox}[colframe=black,colback=white,sharp corners]
	E3. We will also see further below an example where we contract a tensor of order $1$ (one of the contravariant components of the vectors of the spherical base) having already a lowered index:
	
	\end{tcolorbox}
	\begin{tcolorbox}[title=Remarks,colframe=black,arc=10pt]
	\textbf{R1.} In the relation:
	
	the equality is an abusive notation that we can found in some books (because strictly speaking we should do the calculation in two steps).\\
	
	\textbf{R2.} As a result of the symmetry of the quantities $g_{ij}$ (the dot product is commutative), this latter tensor is identical to what we would get to the position by lowering to the covariant position the index $i_2$, and then by doing the contraction of the index $i_1$ with the index $j_2$.\\

	Let see this:\\

	The symmetry $g_{ij}=g_{ji}$ takes the form (this may seem confusing but let us remember that the number of the component $i$ indicates the place of this component):
	
	Therefore it comes:
	
	and putting $i_1=j_2=k$:
	
	\end{tcolorbox}
	In general, the contraction of a tensor allows to form a tensor of order $p-2$ from a tensor of order $p$. We can of course repeat the operation of contraction. Thus, an even tensor, $2p$, will become a scalar after $p$ contractions and an odd order tensor $2p+1$, will become a vector.

	We can extend after this definition of the contraction of indices, the tensoriality criterion. We have seen until now two ways to recognize the character of a tensor of a sequence of quantities:
	\begin{itemize}
		\item The first is to show that these quantities are formed by the tensor product of component vectors or by a sum of tensor products;

		\item The second is to study how these quantities are converted during a basis change and to check the conformity of the relations of transformation;

		\item The third and new one that brings to put that for a set of $n^{p+1}$ quantities, having $p$ upper and $q$ lower indexes to be a tensor, it is necessary and sufficient that their product fully contracted by the contravariant components of any $p$ vectors and the covariates components of any $q$ vectors, to be a quantity (the norm in fact...) that remains invariant under basis change.
	\end{itemize}
	
	\subsection{Special Tensors}
	We may face in theoretical physics and engineering with tensors that have interesting properties. To avoid redundant work in each case, we will list here and proved the various existing properties used in this book in other sections and discuss their possible implications briefly (the detailed analysis being reserved for their application in the other same sections of the book).
	
	\subsubsection{Symmetric Tensor}\label{symmetric tensor}
	Consider a tensor $\vec{T}$ of order two of contravariant components $T^{ij}$. Let us suppose that, following a base $(\vec{e}_i)$, all these components satisfy the relations:
	
	On another base $(\vec{e}_j^{\prime})$, related to the previous by the known transformation relations, the new components of ${T'}^{lm}$ satisfy the relation:
	
	We see that the property $T^{ij}=T^{ji}$ is therefore an intrinsic characteristic of the tensor $\vec{T}$, independent of the base! We then say that the tensor is a "\NewTerm{symmetric tensor}\index{symmetric tensor}" (we will come back again on this concept a little further below) also named "\NewTerm{totally invariant tensor}\index{totally invariant tensor}" (implicitly meaning: by base change).

	The symmetry property is also true for the covariant components of a symmetric tensor since we have:
	
	Conversely, the symmetry of the covariant components implies that of the contravariant components.

	For higher order tensor, the symmetry may be partial, being valid only for two covariant indices or two contravariant indices. Thus, a fourth order tensor, of mixed components $T_l^{ijk}$, may also be symmetrical in $i$ and $j$, for example, given:
	
	We check, as above, that such a property is intrinsic.
	
	A tensor is said to be "\NewTerm{completely symmetrical tensor}\index{completely symmetrical tensor}" if any transposition of two indices with the same variance, changes the corresponding component into itself. For example, for a tensor of order three $T^{ijk}$, completely symmetric, we have the following components that are equal:	
	
	There are many examples of symmetric tensors. Some include:
	\begin{itemize}
		\item the metric tensor $g_{\mu \nu }$ (\SeeChapter{see section General Relativity page \pageref{metric tensor}}) 
		\item the Einstein tensor $G_{\mu \nu }$ (see further below) 
		\item the Ricci tensor $R_{\mu \nu }$ (see also further below).
		\item the stress and strain tensor for fluids or solids  $\sigma_{ij}$ (\SeeChapter{see section Continuum Mechanics page \pageref{cauchy stress tensor}})
		\item  the Lorentz boost tensor $\Gamma_\nu^\mu$ (\SeeChapter{see section Special Relativity page \pageref{lorentz boost tensor}}) 
		\item ...
	\end{itemize}
	We can also (very interesting curiosity) obtain a geometric representation of the values of the components of a symmetric tensor of order two!! 
	
	For this let us, consider in the ordinary geometric  space coordinates $x^i$, the following equation:
	
	where, for recall, $x^ix^j$ can be seen as a tensor with $i,j=1,2,3$ and where the $a_{ij}$ are real given coefficients. Let us suppose that coefficients are such that:
	
	The above equation is then:
	
	Here we fall back on the equation of a surface of the second degree of a quadratic similar to of the plan that we saw in the section of Analytical Geometry. We know by extension in to three dimensions that these surfaces are ellipsoids or hyperboloids, depending to the values of the quantities $a_{ij}$.
	
	Let us study how the quantities $a_{ij}$ transform  when we make a change of coordinates such as:
	
	The equation of the quadric is written in this new coordinate system:
	
	Hence the expression of the coefficients in the new system of axes:
	
	The coefficients $a_{ij}$ therefore transform as the covariant components of a tensor of order two. Conversely, if the quantities $a_{ij}$ are the components of a symmetric tensor, these components define the coefficients of a quadric!! There is therefore a certain equivalence between a symmetric tensor and the coefficients of a quadratic...!!! We say then that the equation of the quadric is the "\NewTerm{quadric representation of a symmetric tensor}\index{quadric representation of a symmetric tensor}" or "\NewTerm{representation surface}".
	
	So the representation surface (or representation quadric) is a geometrical representation of a second rank symmetric tensor and is useful for giving us a visual image of the tensor as well as being useful for example in calculating magnitudes of material properties described by second rank symmetric tensors!!
	
	We know from our study of quadrics in the section of Analytical Geometry (by extending it to the three-dimensional case) that we can always find a coordinate system relative to which the equation of a quadratic takes a simpler form:
	
	In this case, the basis vectors are supported by the main axes of the quadric. In this coordinate system, the components of the tensor equation reduce to:
	
	and $a_{ij}=0$ for the other components. The quantity $b_i$ equation are named the "\NewTerm{principal values}\index{principal values}" of the tensor $a_{ij}$.

	If the quantities $b_1,b_2,b_3$ are positive, the surface is an ellipsoid, if two quantities are strictly positive and the third strictly negative, we have a one sheet hyperboloid, if two quantities are strictly negative and the third positive, we have two sheets hyperboloid (\SeeChapter{see section Analytical Geometry page \pageref{two sheets hyperboloid}}).
	
	Comparing the expression of the quadric obtained previously with the classic equation:
	
	where $a$, $b$, $c$ are the semi-axes of an ellipsoid shows that we have:
	
	Below we can see a screen shot of an interactive tool of Cambridge University to play with the ellipcity (only!) representation quadric of a rank two symmetric tensor:
	\begin{figure}[H]
		\centering
		\includegraphics[scale=1]{img/algebra/symmetric_tensor_represntation_quadric_cambridge.jpg}
		\caption[Visual link between a tensor and its representative surface]{Visual link between a tensor and its representative surface \\(source: \href{http://www.doitpoms.ac.uk/tlplib/tensors/representation.php}{http://www.doitpoms.ac.uk})}
	\end{figure}
	
	\subsubsection{Antisymmetric Tensor}
	When the contravariant or covariant components of a tensor of order two, satisfy the property:
	
	we the say that the tensor is a "\NewTerm{antisymmetric tensor}\index{antisymmetric tensor}\label{antisymmetric tensor}".  In other words a tensor is antisymmetric on (or with respect to) an index subset if it alternates sign ($+/-$) when any two indices of the subset are interchanged.
	
	It follows from this definition that an antisymmetric tensor must obviously satisfy the fact that its diagonal components are all zero, such as for example for a rank $2$ tensor:
	
	
	A well known antisymmetric tensor is the electromagnetic tensor $F_{\mu \nu }$ (\SeeChapter{see section Electrodynamics page \pageref{electromagnetic tensor}}).
	
	For example a covariant tensor of order three $T_{ijk}$ will be say to be symmetric in $i$ and $k$ if for all values that can take the indices, we have:
	
	Or the fourth order covariant tensor $T_{ijkl}$ will be said in antisymmetric on $i$ and $l$ if for all values that can take the indices, we have:
	 
	
	A tensor will "\NewTerm{totally antisymmetric}\index{totally antisymmetric tensor}" if any transposition of index of same variance (covariant/contravariant) changes the corresponding component into its opposite.

	\label{decomposition square matrix symmetric and antisymmetric} A tensor $T_{ij}$ can be put in the form of a sum of a symmetric tensor and an antisymmetric tensor. Indeed, we have:
	
	The first term of the sum above is a symmetric tensor and the second, an antisymmetric tensor.
	\begin{dem}
	Consider first that $T_{ij}$ is symmetric, then we have:
	
	So this proves that the left term is indeed symmetric for this special case.
	
	Consider secondly that $T_{ij}$ is symmetric, then we have:
	
	So this proves that the left term is indeed antisymmetric for this special case.

	Now if $T_{ij}$ is neither symmetric or antisymmetric we have:
	
	\begin{flushright}
		$\blacksquare$  Q.E.D.
	\end{flushright}
	\end{dem}
	A tensor $T_l^{ijk}$ will be "\NewTerm{partially antisymmetric}\index{partially antisymmetric tensor} if for example we have:
	
	That is to say antisymmetric only for a subset of indices.
	Let s now consider two vectors $\vec{x}=x^i\vec{e}_i$ and $\vec{y}=y^j\vec{e}_j$ of a vector space $\mathcal{E}^n$. Let us form the following antisymmetric quantities (we can see in it two tensor products):
	
	where we see immediately that the components $T^{ij}$ are those of an antisymmetric tensor $\vec{T}$ by construction as:
	
	The decomposition of the vector $\vec{T}$ in the base $\vec{e}_i\vec{e}_j$ is:
	
	The tensor $\vec{x}\otimes\vec{y}$ (written as it in analogy to the vectors cross product for $n=3$ ) is named the "\NewTerm{outer product}\index{outer product}" of the vectors $\vec{x}$ and $\vec{y}$. We say that this tensor is a "\NewTerm{bivector}\index{bivector}".
	
	The outer product is an antisymmetric tensor which satisfies the following properties:
	\begin{enumerate}
		\item[P1.] Anticommutativity:
		
		the result is:
		

		\item[P2.] Left and right distributivity for the vector addition:
		

		\item[P3.] Associativity for the multiplication by a scalar:
		

		\item[P4.] Outer products:
		
	\end{enumerate}
	constitute a base for all bivectors.
	\begin{dem}
	An antisymmetric tensor $\vec{T}$ of order two, element of $\mathcal{E}_{(2)}^2$ can, as we have proved it earlier, be written as:
	
	Exchanging, in the last sum of the above relation, the name of the indices and considering that $T^{ij}=-T^{ji}$, we get:
	
	The elements:
	
	are linearly independent vectors since the vectors $\vec{e}_i\otimes\vec{e}_j$ also are it. These elements constitute a base on which the antisymmetric tensor can be decomposed.
	\begin{flushright}
		$\blacksquare$  Q.E.D.
	\end{flushright}
	\end{dem}
	The number of distinguishable vectors $\vec{e}_i\otimes\vec{e}_j-\vec{e}_j\otimes\vec{e}_i$ is equal to the number of combinations of vectors taken two by two and distinguishable among $n$ such that (\SeeChapter{see section Probabilities page \pageref{choice function}}):
	
	Indeed among the $n^2$ components, $n$ components are equal to zero and $n(n-1)$ other components have opposed values to two by two. So we can consider that half of the latter is sufficient to characterize the tensor.

	In the context of the outer tensor product where we have:
	
	the number of distinguishable components is also equal to $n(n-1)/2$ and they are named "\NewTerm{strict components}\index{strict components}".
	
	We notice that for $n=3$, the strict number of components of the outer product of two vectors is also equal to three. This allows to form, with the components of the bivector, the components of a cross product $\vec{z}$.

	Thus, a cross product therefore exists only for a subspace of bivectors whose number of dimensions is equal to $3$ and the pre-images that are antisymmetric tensors.

	If all these conditions are satisfied, we say that the vector 
$\vec{z}$ is the "\NewTerm{adjoint tensor}\index{adjoint tensor}" of tensor $\vec{T}$.

	\subsubsection{Fundamental Tensor}
	We saw at the beginning of our study of Tensor Calculus the definition of the components of the fundamental covariant tensor $g_{ij}$, that is for recall:
	
	These quantities are involved, as we know it (see previous topics), in the expression of the dot product of two vectors $\vec{x}$ and $\vec{y}$ of contravariant components $x^i$ and $y^i$, given by the relation (for recall):
	
	Let us use the general test of tensoriality to highlight the tensor character of the $g_{ij}$. The previous expression is a product fully contracted of the quantities $g_{ij}$ with the contravariant components  $x^iy^i$ of an arbitrary tensor. As the dot product is an invariant quantity (in this case a scalar) with respect to the base changes, it follows that the $n^2$ quantities $g_{ij}$ are the covariant components of a tensor.

	This tensor is also symmetrical as a result of the symmetry property of the dot product of the basis vectors such that:
	
	We have the same for the contravariant components of the fundamental tensor:
	
	If we denote by $g_j^i$ the mixed components of a fundamental tensor to himself:
	
	obviously with the canonical basis:	
	
	
	\subsection{Curvilinear Coordinates}
	The conventional concepts of coordinate system can be generalized as we know to any specific $n$-dimensional punctual space (\SeeChapter{see section Principia page \pageref{point spaces}}). We name "\NewTerm{coordinate system}\index{coordinate system}" in $\mathcal{E}^n$, any mode of definition of a point $M$ in the considered system.
	
	\textbf{Definitions (\#\mydef):} 
	\begin{enumerate}
		\item[D1.] For a given coordinates system (Cartesian, spherical, cylindrical, polar ...) system we name "\NewTerm{coordinate line}\index{coordinate line}" the "place" of the points $M$ when a only single coordinate of $M$ varies, the other being keep as constant.

		\item[D2.] A "\NewTerm{curvilinear coordinates}\index{curvilinear coordinates}\label{curvilinear coordinates tensor calculus}" are a coordinate system for Euclidean space in which the coordinate lines may be curved. These coordinates may be derived from a set of Cartesian coordinates by using a transformation that is locally invertible (a one-to-one map) at each point (\SeeChapter{see section Vector Calculus page \pageref{system of coordinates}}).
	\end{enumerate}
	 This means that one can convert a point given in a Cartesian coordinate system to its curvilinear coordinates and back. The purpose as we already know is that depending on the application, a curvilinear coordinate system may be simpler to use than the Cartesian coordinate system. For instance, a physical problem with spherical symmetry defined (for example, motion of particles under the influence of central forces) is usually easier to solve in spherical polar coordinates than in Cartesian coordinates.
	 
	 Well-known examples of curvilinear coordinate systems are as we know in three-dimensional Euclidean space the Cartesian, cylindrical and spherical polar coordinates.
	 
	 Let us first study the generalization of a coordinate system relative to a fixed reference frame (we urge the reader to read first the subsection about Coordinate Systems in the section of Vector Calculus and the subsection of Analytical Mechanis in the section Principia).
	 
	 Let us consider a punctual space $\mathcal{E}^n$ and $(\vec{e}_i)$ a reference frame of that space. Given $x_i$ the rectilinear coordinates of a point $M$ of $\mathcal{E}^n$ relatively to this reference fame. Any curvilinear coordinate system $u^k$ (with $k=1,2,\ldots,n$ is obtained by giving $n$ arbitrary functions $f^i$ of parameters $u^k$, such that:
	 
	
	\begin{figure}[H]
		\centering
		\begin{tikzpicture}[x=(10:4cm),y=(90:4cm),z=(225:4cm),>=Triangle,scale=1.5]
		\coordinate (O) at (0,0,0); 
		\draw [->] (O) -- (1,0,0) node [at end, right] {$x^2$ axis};
		\draw [->] (O) -- (0,1,0) node [at end, above] {$x^3$ axis};
		\draw [->] (O) -- (0,0,1) node [at end, left]  {$x^1$ axis};
		
		\draw [draw=blue, -Circle] (O) to [bend left=8] 
		  coordinate [pos=7/8] (q2n) 
		  (1,-1/4,0) coordinate (q2) node [right] {$u^2$};
		\draw [draw=blue, -Circle] (O) to [bend right=8] 
		  coordinate [pos=7/8] (q3n) 
		  (0,1,1/2) coordinate (q3) node [left] {$u^3$};
		\draw [draw=blue, -Circle] (O) to [bend right=8] 
		  coordinate [pos=7/8] (q1n) 
		  (1/4,0,1) coordinate (q1) node [right] {$u^1$};
		
		\begin{pgfonlayer}{background}
		\begin{scope}
		\clip (O) to [bend left=8] (q2) -- (1,1,0) -- (q3n) to [bend right=8] (O);
		\shade [left color=green, right color=green!15!white, shading angle=135]
		  (O) to [bend left] (q3n) to [bend left=16] (3/4,1/2,0) to [bend left=16] (q2n) -- cycle;
		\end{scope}
		
		\begin{scope}
		\clip (O) to [bend left=8] (q2) -- (1,0,1) -- (q1) to [bend left=8] (O);
		\shade [left color=red, right color=red!15!white, shading angle=45]
		  (O) to [bend right] (q1n) to [bend left=16] (1,0,1) to [bend left=16] 
		  (q2n) to [bend right] (O);
		\end{scope}
		
		\begin{scope}
		\clip (O) to [bend right=8] (q1) -- (0,1,1) -- (q3) to [bend left=8] (O);
		\shade [left color=cyan, right color=cyan!15!white, shading angle=225] 
		  (O) -- (q1n) to [bend right=16] (0,1,1) to [bend left=16] (q3n) 
		to [bend left] (O);
		\end{scope}
		\end{pgfonlayer}
		
		\node at (1/3,1/3,0) {$q_1=\mbox{const}$};
		\node at (0,1/2,1/2) {$q_2=\mbox{const}$};
		\node at (1/2,0,1/3) {$q_3=\mbox{const}$};
		\end{tikzpicture}
		\caption{Coordinate surfaces, coordinate lines, and coordinate axes of general curvilinear coordinates}
	\end{figure}
	We will assume thereafter that these $n$ functions satisfy the following three properties:
	\begin{itemize}
		\item[P1.] They must be of class at least $\mathcal{C}^2$ (differentiable at least twice for the needs of physics: speed and acceleration). This assumption implies, that at any point where it is satisfied, that we have the permutation of derivations (with respect to the two partial derivaties as seen in the section of Differential and Integral Calculus):
		

		\item[P2.] These functions are such that we can solve the system of $n$ equations of coordinate system change relatively to the variables $u^k$ and express them in function of the parameters $x^i$, thus:
		
		still with $k=1,2, \ldots,n$.

		\item[P3.] When the variables $u^k$ vary in a domain $\Delta$, the variables $x^i$ vary in a domain $\Delta'$ (think to cartesian$\leftrightarrow$ spherical coordinates where in cartesian the variable range is infinite when in spherical both of them are limited a typical $2\pi$ width range ). 

		\item[P4.] The Jacobian of the functions $x^i=f^i(u^1,u^2,\ldots,u^n)$, defined by (\SeeChapter{see section Differential and Integral Calculus page \pageref{jacobian}}):
		
		will be assumed different from zero in the domain $\Delta$ (and also the Jacobian $\mathrm{D}(\partial_i u^k)$ of the functions $u^k=g^k(x^1,x^2,\ldots,x^n)$) and is the inverse of the Jacobian of $u^k=g^k(x^1,x^2,\ldots,x^n)$. If the jacobians exist, they are not zero as a result primarily of the second property above and implicitly the first.
	\end{itemize}
	As we have already mention if we fix $(n-1)$ parameters $u^k$ by varying only one parameter, $u^i$ for example, we get the coordinates $x_{(1)}^i$ of a set of points $M$ that constitute a "coordinate line line". In general, the coordinate lines are not straight but curved as we know. These coordinates $u^k$ are named for this reasons the "curvilinear coordinates". On a point $M$ of $\mathcal{E}^n$ intersect elsewhere $n$ coordinate lines (see figure above).
	
	We proved in the section of Analytical Mechanics, during our study of punctual spaces (see page \pageref{point spaces}), that partial derivatives of a vector $\overrightarrow{\text{O}M}$ are independent of the point O (origin) of a given reference frame. If $\mathcal{E}^n$ is assimilated to a system of curvilinear coordinate, we write:
	
	\begin{tcolorbox}[colframe=black,colback=white,sharp corners]
	\textbf{{\Large \ding{45}}Example:}\\\\
	A famous example of curvilinear coordinates, where each $u^k$ is a uniform function of the rectilinear coordinates $x^k$, the being moreover $u^k$  continuous functions at the current point $M$, is that of the spherical coordinates where we have (\SeeChapter{see section Vector Calculus page \pageref{spherical coordinates}}):
	
	and where for recall:
	
	Let us also recall that during our study of the system of spherical coordinates in the section of Vector Calculus we obtained after normalization of the basis vector:
	
	Therefore:
	
	with:
	
	\end{tcolorbox}
	In a non-Euclidean space, we can not define a unique valid basis over the whole space. Thus, we construct a base at each point separately and for this purpose we use the curvilinear coordinates such that at each point $M$, the base vectors $\vec{e}_k$ are tangent to the corresponding coordinate line equation via the relation given above:
	
	Given now $u^1, u^2, \ldots, u^n$ the curvilinear coordinates of the point $M$ with respect to a Cartesian basis $(\vec{e}_i^0)$. In this reference frame, we obviously have:
	
	where the Cartesian coordinates are functions $x^i=x^i(u^1,u^2,\ldots,u^n)$.
	
	The vector $\vec{e}_k$ therefore also be expressed by:
	
	The reader can check with the example of spherical coordinates (by looking to the explicit version of the corresponding vectors in the section of Vector Calculus) that this relation is correct but at the condition that we work with the non-normalized version of the orthogonal vectors $\vec{e}_r$, $\vec{e}_\theta$, $\vec{e}_\phi$!
	
	From the components $\partial_k x^i$ of the vector $\vec{e}_k$, we can form a determinant $\det(\partial_kx^i)$ which is precisely the Jacobian of the of the functions $x^i$ we have defined previously. Since this determinant is different from zero (at least imposed as such), it follows that the $n$ vectors $\vec{e}_k$ (as functions) are linearly independent (we have proved in the section of Differential and Integral Calculus that this Jacobian is in absolute value equal to $r^2\sin(\theta)$ for the spherical coordinates).
	\begin{tcolorbox}[title=Remark,colframe=black,arc=10pt]
	Let us recall that in the section of Differential and Integral Calculus we have proved that the determinant of the Jacobian appears when calculating the surface of a parallelogram in a non-euclidean space. Obviously is the determinant is zero, then the surface is zero as it means that the vector basis of the parallelogram are all collinear (linearly dependents). Hence the fact that if the determinant is not null then some of the vectors (or all) are independent!
	\end{tcolorbox}
	These $n$ vectors, defined by the relation:
	
	are named the "\NewTerm{natural basis}\index{natural basis}" at the point $M$ of the vector space $\mathcal{E}^n$. They are collinear to the tangents of the $n$ coordinated lines which intersect at the point $M$ where they are defined.
	
	We will not insist on the quite obvious fact that at each system of curvilinear coordinates there is an associated natural basis whose base is expressed by these same coordinates (\SeeChapter{see section Vector Calculus page \pageref{system of coordinates}}).
	\begin{tcolorbox}[colframe=black,colback=white,sharp corners]
	\textbf{{\Large \ding{45}}Example:}\\\\
	In spherical coordinates, the vectors of the natural basis are those that we obtained in our study of the spherical coordinate system in the section of Vector Calculus and that are orthogonal but not orthonormal.
	\end{tcolorbox}
	
	Let us associate at the point $M$ of $\mathcal{E}^n$ a reference frame formed by the point $M$ and by the vectors of the natural basis\footnote{For recall that natural basis in an Euclidean space is the set of unit vectors pointing in the direction of the axes of a Cartesian coordinate system}. This reference frame is named the "\NewTerm{natural reference frame}\index{natural reference frame}" on $M$ of the coordinate system $u^k$. It will be denoted by:
	
	The differential of the vector $\overrightarrow{OM}$ is then expressed as:
	
	The quantities $\mathrm{d}u^k$ are (obviously) the contravariant components of the vector $\mathrm{d}\vec{M}$ in the natural reference frame $(M,\vec{e}_k)$ of the coordinate system $u^k$.
	
	Let us now consider any two curvilinear coordinate systems $u^i$ and $u^k$ (thinks for example to the spherical and cylindrical coordinate systems), linked between them by the relations:
	
	where the functions $u^i=u^i({u'}^1,{u'}^2,\ldots,{u'}^n)$ are assumed as we already know to be several times continuously differentiable with respect to the ${u'}^k$ and same for the functions ${u'}^k={u'}^k(u^1,u^2,\ldots,u^n)$ with respect the coordinates $u^i$. When we move from one coordinate system to another, we say that we make a "\NewTerm{change of curvilinear coordinates}"\index{change of curvilinear coordinates}.
	
	We saw that in the sections of Differential Geometry and General Relativity that the squared distance $\mathrm{d}s$ between two points $M$ and $M'$ infinitely close was given by the relation:
	
	where the $\mathrm{d}x^i$ are the components of the vector $\mathrm{d}\vec{M}=\overrightarrow{MM}$, assimilated to a fixed reference frame of a punctual space $\mathcal{E}^n$. When this space is assimilated to a system of curvilinear coordinates $u^i$, we have seen that the relation:
	
	shows that the vector $\mathrm{d}\vec{M}$ has for curvilinear contravariant components the quantities $\mathrm{d}u^i$ with respect to the natural base $(M,\vec{e}_i$. The square of the distance $\mathrm{d}s^2$ (ie the "line-elements") is then written in the natural reference frame:
	
	Where the quantities $g_{ij}=\vec{e}_i\circ\vec{e}_j$ are the components of the "fundamental tensor" or of "metric tensor" defined using a natural base. The previous expression is named the  "\NewTerm{linear element of the point space}" $\mathcal{E}^n$ or sometimes the "\NewTerm{metric}" of this space.
	
	The vectors $\vec{e}_i$ of the natural reference frame generally vary from one point to another. This is the case, for example, of the spherical coordinates whose quantities $g_{ij}$ (we show it afterwards with a detailed example) are variable since depending on the parameters $r$, $\theta$ or $\phi$!
	
	A curve $\Gamma$ of $\mathcal{E}^n$ can be defined by the data of the curvilinear coordinates $u^i(\alpha)$ of the locus of the points $M(\alpha)$ as a function of a parameter $\alpha$ (\SeeChapter{see section Differential Geometry page \pageref{parametric curves}}). The elementary distance $\mathrm{d}s$ on this curve $\Gamma$ is then written:
	
	If we consider a tangent vector $\vec{v}$ of $\Gamma$ and denote its components $v^{i}$ on a basis $\vec{e}_{i}$. On another basis $\vec{e}_{i}^{\prime}$ we denoted the components $v^{\prime i}$, so:
	
	in which:
	
	If we express the new components in terms of the old ones, then:
	
	This is the explicit form of a transformation named the "\NewTerm{contravariant transformation}\index{contravariant transformation}\label{contravariant transformation}". In order to distinguish them such vectors from the covariant (tangent) vectors, the index is placed on top.
	
	A famous example of a contravariant transformation is given by a differential form $\mathrm{d}f$. For $f$ as a function of coordinates $x^{i}$, $\mathrm{d}f$ can be expressed in terms of $\mathrm{d}x^{i}$ (ie the total derivative). The differentials $\mathrm{d}x^{i}$ transform according to the contravariant rule since:
	
	
	\subsubsection{Natural basis in spherical coordinates (curvilinear basis in spherical coordinates)}
	Let us determine the natural basis of the vector space $E^3$ associated with the point space $\mathcal{E}^3$ of ordinary geometry, in spherical coordinates. Let us write the expression of the vectors $\overrightarrow{\text{O}M}$ in a fixed cartesian basis $(\vec{e}_i^{\,0})$ which is by definition (see the section Vector Calculus for more details):
	
	The vectors of the natural base are given by:
	
	Therefore we have:
	
	The derivative of $\overrightarrow{\text{O}M}$ with respect to $\theta$ gives the vector $\vec{e}_2$:
	
	The derivative with respect to $\varphi$ gives the vector $\vec{e}_3$:
	
	These three vectors are orthogonal to each other as we can easily verify it by performing the dot products $\vec{e}_i\circ\vec{e}_j$. When this is the case, we say that the coordinates are "\NewTerm{orthogonal curvilinear coordinates}\index{orthogonal curvilinear coordinates}" (\SeeChapter{see section Vector Caluculus page \pageref{orthogonal curvilinear coordinates}}).

	We thus find the same result as in the section of Vector Calculus! These vectors, however, are not all normalized (they're norm is not equal to $1$) since we have:
	
	The natural basis in spherical coordinates is thus formed by vectors that are variable in direction and in modulus at each point of $M$. The quantities $g_{ij}$ constitute an example of a metric tensor attached to each of the points $M$ of the space $\mathcal{E}^3$.
	\begin{figure}[H]
		\centering
		\includegraphics[scale=1]{img/algebra/spherical_natural_basis.jpg}
		\caption[Coordinate surfaces, coordinate lines, and coordinate axes of spherical coordinates]{Coordinate surfaces, coordinate lines, and coordinate axes of spherical coordinates (source: Wikipedia)}
	\end{figure}
	The linear element of the surface is then given by (the details of the calculations can be found in the section of General Relativity):
	
	
	\subsubsection{Natural basis in polar coordinates (curvilinear basis in polar coordinates)}
	Let us determine the natural basis of the vector space $E^2$ associated with the point space $\mathcal{E}^2$ of ordinary geometry, in polar coordinates. Let us write the expression of the vectors $\overrightarrow{\text{O}M}$ in a fixed cartesian basis $(\vec{e}_i^{\,0})$ which is by definition (see the section Vector Calculus for more details):
	
	The vectors of the natural basis are given by:
	
	We have:
	
	The derivative of $\overrightarrow{\text{O}M}$ with respect to $\phi$ gives the vector $\vec{e}_2$:
	
	These two vectors are orthogonal to each other as we can easily verify by performing the dot products $\vec{e}_1\circ\vec{e}_2$. We thus find the same result as in the section of Vector Calculus.
	
	We then have:
	
	The linear element of the plane is then given by (the details of the calculations can be found in the section of General Relativity):
	
	
	\subsubsection{Natural basis in cylindrical coordinates (cylindrical basis in polar coordinates)}
	Let us determine the natural basis of the vector space $E^3$ associated with the point space $\mathcal{E}^3$ of ordinary geometry, in cylindrical coordinates. Let us write the expression of the vectors $\overrightarrow{\text{O}M}$ in a fixed cartesian basis $(\vec{e}_i^{\,0})$ which is by definition (see the section Vector Calculus for more details):
	
	The vectors of the natural basis are given by:
	
	We have:
	
	The derivative of $\overrightarrow{\text{O}M}$ with respect to $\varphi$ gives the vector $\vec{e}_2$:
	
	and finally:
	
	These three vectors are orthogonal to each other as we can easily verify by performing the dot products $\vec{e}_i\circ\vec{e}_j$. We thus find the same result as in the section of Vector Calculus.
	
	The linear element of the surface is then given by (the details of the calculations can be found in the section of General Relativity):
	
	Notice now that:
	
	expresses something about the intrinsic metrical relations of the space, but it does so in terms of a specific coordinate system. If we considered the metrical relations at the same point in terms of a different system of coordinates (such as changing from Cartesian to polar coordinates), the coefficients $g_{\mu\nu}$ would be different. Fortunately there is a simple way of converting the $g_{\mu\nu}$ from one system of coordinates to another, based on the fact that they describe a purely localistic relation among differential quantities. Suppose we are given the metrical coefficients $g_{\mu\nu}$ for the coordinates $x^\alpha$, and we are also given another system of coordinates $y^\alpha$ that are defined in terms of the $x^\alpha$ by some arbitrary continuous functions:
	
	Assuming the Jacobian of this transformation isn't zero, we know that it's invertible, and so we can just as well express the original coordinates as continuous functions (at this point) of the new coordinates:
	
	Now we can evaluate the total derivatives of the original coordinates in terms of the new coordinates. For example, $\mathrm{d}x^0$ can be written as:
	
	and similarly for the $\mathrm{d}x^1, \mathrm{d}x^2$, and $\mathrm{d}x^3$. The product of any two of these differentials, $\mathrm{d}x^\mu$ and $\mathrm{d}x^\nu$, is of the form:
	
	(remembering the summation convention giving $16$ components). Substituting these expressions for the products of $x$ differentials in the metric formula:
	
	gives:
	
	The first three factors on the right hand side obviously represent the coefficient of $\mathrm{d}y^\alpha\mathrm{d}y^\beta$ in the metric formula with respect to the $y$ coordinates, so we've shown that the array of metric coefficients transforms from the $x$ to the $y$ coordinate system according to the relation:
	
	Also often denoted:
	
	and more rarely (sadly) denoted:
	
	Notice that each component of the new metric array is a linear combination of the old metric components, and the coefficients are the partials of the old coordinates with respect to the new. Arrays  that transform to each other in the way:
	
	 are named "\NewTerm{(second order) covariant\footnote{Keep in mind that when a tensor expressed in terms of Cartesian coordinates, there is no distinction between covariant and contravariant components} tensors}".
	 
	 When we speak of an array being transformed from one system of coordinates to another, it's clear that the array must have a definite meaning independent of the system of coordinates. We could, for example, have an array of scalar quantities, whose values are the same at a given point, regardless of the coordinate system. However, the components of the array might still be required to change for different systems.
	 
	 Now we define arrays that transform to each other in the way:
	
	 as "\NewTerm{(second order) contravariant tensors}".
	
	We will need this important result for our study of the Stress-Energy tensor in the section of General Relativity.
	
	Now the reader must keep in mind that in General Relativity, we don't equip space-time itself with a vector space structure, which means that the "four-position" is not a four-vector! As a result, the object $x_{\mu}$ whose components are given by $x_{\mu}=g_{\mu \nu} x^{\nu}$ as proved earlier is not a covector! One can show this by its transformation properties.
	
	If we change the coordinate systems from $x$ to $y$, then:
	
	and so:
	
	If this object was a covector, it would transform as proven earlier using the covariant transformation:
	
	Comparing this to what we found above, this is only true if:
	
	i.e. if the coordinate transformation is linear. Since coordinate transformations are, in general, not linear, then this equality does not hold, and $x_{\nu}$ are not the components of a covector.
	
	Finally another obvious identity that you can also sometimes found in some rare textbooks is (I never see a practical application of that latter):
	
	
	\pagebreak
	\subsection{Christoffel symbols}
	The study of tensor fields constitutes, for the physicist, the essential element of the tensorial analysis. The generic tensor $\vec{U}$ of this field is a function of the point $M$ and we will denote it simply by:
	
If the tensor $\vec{U}$ is a function only of $M$, the field considered is named a "\NewTerm{fixed field}". If $\vec{U}$ is moreover a function of one or more equation parameters other than the coordinates of $M$, then we say that this is a "\NewTerm{variable field}" and we note it:
	
	The different algebraic operations on the tensors $\vec{U}(M)$ associated with a same point $M$ do not generates any particular difficulty. The derivative of $\vec{U}(M)$ with respect to a parameter $\alpha$ leads to the use of the classical results relating to the derivation of the vectors.
	
	However, a difficulty appears when we try to calculate the derivative of a tensor $\vec{U}(M)$ with respect to the curvilinear coordinates. Indeed, the components of the tensor are defined at each point $M$ with respect to a natural coordinate system which varies from one point to another.

	Consequently, the calculation of the elementary variation, named "\NewTerm{elementary transport}":
	
	When we pass from a point $M$ to an infinitely neighboring point $M'$ this can be done in physics only if we use to the same basis. In order to compare the tensors $\vec{U}(M')$ and $\vec{U}(M)$ each other, we are led to study how a natural coordinate system for a given coordinate system varies when we pass from a point $M$ to an infinitely close point $M'$.
	
	For a system of curvilinear coordinates $u^i$ given a punctual space $\mathcal{E}^n$, a fundamental problem of tensor analysis consists in determining, with respect to the natural reference point $(M,\vec{e}_k)$ at the point $M$, the natural reference point $(M',\vec{e}_k^{\prime})$ at the infinitely close point $M'$. We then say that we are looking for an "\NewTerm{affine connection}\index{affine connection}".
	
	On the one hand, the point $M'$ will be perfectly defined with respect to $M$ if we determine the vector $\mathrm{d}\vec{M}$ such as $\overrightarrow{MM'}=\mathrm{d}\vec{M}$. For curvilinear coordinates $u^k$, the decomposition of an elementary vector $\mathrm{d}\vec{M}$ is given by the relation that we have previously proved:
	
	the quantities $\mathrm{d}u^k$ being for recall the contravariant components of the vector $\mathrm{d}\vec{M}$ on the natural basis $(\vec{e}_k)$.
	
	And now to make things physically comparable, we must guarantee that vectors of the both basis are also parallel! So the idea is that the derivative we are looking for allows one to transport vectors of a manifold (surface) along curves so that they stay parallel with respect to the connection (derivative). Such an idea is named in physics "\NewTerm{parallel transport}\index{parallel transport}".

	Therefore, the idea is to determine the vector $\vec{e}_k^{\prime}$ with elementary variations $\mathrm{d}\vec{e}_k$ of the vector $\vec{e}_k$ relatively to the natural reference frame $(M,\vec{e}_k)$, when we go from $M$ to $M'$. We then have:
	
	The computation of the vectors $\mathrm{d}\vec{e}_k$ then remains the main problem to solve. We will first consider an example of this type of calculation in spherical coordinates for pedagogical purposes as it can help to understand what will follows.

For this, let us return the expression of the vectors $\vec{e}_k$ of the natural base in spherical coordinates, that is:
	
	Since the basis vectors $\vec{i}$,$\vec{j}$,$\vec{k}$ of the fixed Cartesian reference being constant in modulus and in direction, the differential of the vector $\vec{e}_1$ is written:
	
	We notice that the terms in parentheses represent respectively the vectors $\vec{e}_2/r$ and $\vec{e}_3/r$, hence:
	
	We also compute, by differentiating the vector $\vec{e}_2$:
	
	with:
	
	we have:
	
	So finally:
	
	And:
	
	After a few elementary and very tricky algebraic operations (...), we get:
	
	The differentials $\mathrm{d}\vec{e}_k$ are thus decomposed on the natural basis $(\vec{e}_k)$. If we denote by $\omega_i^k$ the contravariant components of the vector $\mathrm{d}\vec{e}_1$, that latter is written:
	
	The components $\omega_i^k$ of the vector $\mathrm{d}\vec{e}_i$ are differential forms (linear combinations of differentials). We have, for example:
	
	If we denote by in a general way by $u^i$ the spherical parameters, we have:
	
	The coordinate differentials are then denoted:
	
	and the components $\omega_i^j$ are then written in a general way:
	
	Where the quantities $\Gamma_{ki}^j$ are functions of $r$,$\theta$,$\phi$ that will be explicitly obtained by identifying each component $\omega_i^j$. For example, the component $\omega_3^3$ is written with the notation of the previous relation:
	
	Identifying the coefficients of the differentials, it comes:
	
	By doing the same with the $9$ components $\omega_i^j$, we get the $27$ (...) terms for which the detailed calculations for the $3\cdot 9=27$ are given much further below in the text. For any curvilinear coordinate system, these quantities $\Gamma_{ki}^j$ are named "\NewTerm{Christoffel symbols of the second kind}\index{Christoffel symbols of the second kind}\label{Christoffel symbols of the second kind}" or "\NewTerm{Euclidean functions of affine connection}\index{Euclidean functions of affine connection}".
	
	Thus, for a punctual space $\mathcal{E}^3$ and a system of any curvilinear coordinates $u^j$, the differential $\mathrm{}\vec{e}_i=\omega_i^k\vec{e}_k$ of the vectors $\vec{e}_i$ of the natural basis is written on this basis
	 
	where $\Gamma_{ji}^k$ in this context is often named "\NewTerm{Levi-Civita connection}\index{Levi-Civita connection}".
	\begin{tcolorbox}[title=Remark,colframe=black,arc=10pt]
	Therefore this also means that given a vector $\vec{v}$ such that:
	
	Thus:
	
	But as:
	
	we have then:
	
	\end{tcolorbox}
	We have just seen, on the example of the spherical coordinates, that a direct calculation makes it possible, by identification, to obtain explicitly the quantities $\Gamma_{ki}^j$. We shall see that we can also obtain the expression of these quantities as a function of the components of $g_{ij}$.
	
	The calculation of the quantities $\Gamma_{ki}^j$ as a function of the $g_{ij}$ will lead us to introduce other Christoffel symbols. For this, let us write the covariant components, denoted $\omega_{ji}$, of the differentials $\mathrm{d}\vec{e}_i$, thus (it's kind a definition):
	
	\begin{tcolorbox}[colframe=black,colback=white,sharp corners]
	\textbf{{\Large \ding{45}}Example:}\\\\
	With our example above:
	and:
	
	We get (most dot products are orthogonal vectors, therefore they vanish):
	
	\end{tcolorbox}
	The covariant components are also a linear combinations of the  differentials $\mathrm{d}u^k$ that we can write in the form:
	
	The quantities $\Gamma_{kji}$ are named the "\NewTerm{Christoffel symbols of the first kind}\label{christoffel symbols of the first kind}".

	We see quite clearly by going through the definitions and examples of the Christoffel symbols above again that:
\begin{enumerate}
	\item For the Christoffel symbols of the second kind, they are symmetrical with respect to their lower indices and therefore if the metric is symmetric, we have:
	
	
	\item For the Christoffel symbols of the first kind, they are also symmetrical with respect to their extremal indices and therefore if the metric is symmetric, we have:
	
	\end{enumerate}
	Indeed (following the request of a reader), since we have:
	
	it then comes:
	
	and by swapping the indices $i$ and$ j$:
	
	The term-to-term identification of the development on a concrete case of the last two relations will give (necessarily) the equality:
	
	that we wanted to prove.

	Since the covariant components are related to the contravariant components by the relations (contraction of indices):
	
	We get the expression linking the Christoffel symbols of each kind:
	
	Conversely:
	
	\begin{tcolorbox}[title=Remark,colframe=black,arc=10pt]
	Various notations are used to represent the Christoffel symbols. The most common are:
	\begin{itemize}
		\item Christoffel symbol of the first kind:
		

		\item Christoffel symbol of the second kind:
		
	\end{itemize}
	\end{tcolorbox}
	Let us consider now a punctual space $\mathcal{E}^n$ and given a linear element $\mathrm{d}s^2$ of this space:
	
	Starting from:
	
	we get by differentiation:
	
	By injecting in it the expression of the differential $\mathrm{d}\vec{e}_j=\omega_j^l\vec{e}_l$ this gives us:
	
	where the term represents $\omega_j^l$ represents the mixed component of the vector $\mathrm{d}\vec{e}_j$. We can make this component covariant by multiplying it by the metric tensor $g_{il}$ so as to form an $\omega_{ij}$ quantity which can in turn be expressed by means of the Christoffel symbols as follows as we already know:
	
	Substituting the relation $\Gamma_{kji}=g_{jl}\Gamma_{ki}^l$ in the preceding expression (the indices used in this relation are not those of the expression in question, but mutatis mutandis it is equivalent), we then get:
	
	The differential $\mathrm{d}g_{ij}$ is then written:
	
	On the other hand, the differential of the function $g_{ij}$ is then written:
	
	hence by identifying the coefficients of the differentials $\mathrm{d}u^k$ in these two last expressions (much further below in this section there is a detailed example of all the relations which will follow with several coordinate systems):
	
	Relation that the reader can (if he doubt about its veracity) check with the detailed practical examples which are given much further below.

	As we have (in the case of a symmetric metric):
	
	where it is strongly recommended to the reader to remember that the permutation of the indices respecting this last relation generally only  works on the extreme indices!

	We can therefore write the prior-previous relation as:
	
	and by performing a circular permutation on the indices (hence it is not a permutation of the extreme indices!), we get:
	
	By making the sum:
	
	and by subtracting:
	
	Simplifying it comes:
	
	therefore:
	
	It is the expression of the Christoffel symbols of the first kind as a function of the partial derivatives of the components $g_{ij}$ of the fundamental tensor named "\NewTerm{first Christoffel identity}\index{first Christoffel identity}\label{first Christoffel identity}". We thus understand why in a locally inertial frame (of the Minkowski type) the Christoffel symbols are all zero (given that the metric is constant).
	
	We get those of the second kind from the following relation (by definition) named "\NewTerm{fundamental theorem of Riemannian geometry}\index{fundamental theorem of Riemannian geometry}\label{fundamental theorem of Riemannian geometry}" or "\NewTerm{Levi-Civita connection}\index{Levi-Civita connection}" or "\NewTerm{first Christoffel identity}\index{first Christoffel identity}":
	
	The last two expressions listed above allow the actual calculation of the Christoffel symbols for a given metric (hence an enormous gain in calculations). When the quantities $g_{ij}$ are given a priori, we can study the properties of the punctual space defined by the data of this metric, which is the case of the Riemann spaces we will see later.
	
	For our study of Cosmology we will need to calculate:
	
	For this purpose, notice that:
	
	Therefore:
	

	\pagebreak
	\begin{tcolorbox}[colframe=black,colback=white,sharp corners]
	\textbf{{\Large \ding{45}}Example:}\\\\
	We propose to calculate the Christoffel symbols of the second kind $\Gamma_{kj}^i$ corresponding to the polar coordinate system in the plane  (it will be sufficiently long...) that we will write this time (at the opposite of the section of Vector Calculus) in index notation:
	
	We will calculate the Christoffel symbols of the second kind from our last relation:
	
	Let us determine the components of the metric. By the way, they are the same as those we had calculated for the cylindrical coordinates above with the obvious difference that $g_{33}$ does not exist. Therefore, we have:
	
	Let us then calculate the $g_{ji}$. In this example it is rather trivial, it is enough to apply the relation demonstrated at the beginning of this section:
	
	Or by dealing as with standard matrices (\SeeChapter{see section Linear Algebra page \pageref{some matrix inverse}}):
	
	We then have immediately:
	
	Now let us develop the Christoffel symbols writing for these coordinates:
	\end{tcolorbox}
	
	
	\begin{tcolorbox}[colframe=black,colback=white,sharp corners]
	
	Hence due to the properties of symmetry:
	
	In the same way:
	
	\end{tcolorbox}
	
	
	\begin{tcolorbox}[colframe=black,colback=white,sharp corners]
	To sum up:
	
	\end{tcolorbox}
	
	\subsection{Ricci Theorem}
	\begin{tcolorbox}[colback=red!5,borderline={1mm}{2mm}{red!5},arc=0mm,boxrule=0pt]
	\bcbombe Before reading what follows ... We want to remind the reader that the writing of this section is not finished (as most chapters in the book)! Thus, we still have to illustrate the abstract notions that will follow with concrete practical examples but that are example outside of the special case of General Relativity!
	\end{tcolorbox}
	This being said, we have seen in the section of General Relativity that geodesics are the shortest distances between two points in any type of space. What will interest us now is to study the variations of a vector during such a displacement. Let us first recall that the geodesics equation\index{geodesic equation} for any curvilinear coordinate system $y^i$ of the punctual space $\mathcal{E}^n$ (\SeeChapter{see section Principles page \pageref{point spaces}}) is given by (\SeeChapter{see section General Relativity page \pageref{geodesic equation}}):
	
	Let us consider now a vector $v$ of $\mathcal{E}^n$ of covariant components $v_i$ and let us form the dot product of the vector $\vec{v}$ and $\vec{n}=\mathrm{d}y/\mathrm{d}s$ (the latter vector, denoted here abusively with the indices, gives the components tangent to the geodesic on which the first vector flows), we then have the following quantity:
	
	When moving along the geodesic from a point $M$ to an infinitely near point $M'$, the scalar undergoes the variation:
	
	and as:
	
	Hence:
	
	Let us replace in this last expression, on the one hand, the differential of $\mathrm{d}v_K$ by its exact total differential that we rewrite a bit:
	
	and on the other hand, the second derivative $\mathrm{d}^2y^i/\mathrm{d}s$ by its expression taken from the equation of geodesics. We are getting:
	
	which can also be written:
	
	Where we have put:
	
	which are by definition the absolute differentials of the covariant components of the vector $\vec{v}$. We also define the "\NewTerm{covariant derivative for $1$-forms}\index{covariant derivative}\label{covariant derivative}" (also named "\NewTerm{connection}\index{connection}" or "\NewTerm{affine connection}\index{affine connection}") by the relation:
	
	\begin{tcolorbox}[title=Remark,colframe=black,arc=10pt]
	In ancient or American textbooks this is often written in the form (which we will not use in this book):
	
	making use of the "$;$" to denote the covariant derivative and the "$,$" for the partial differential.
	\end{tcolorbox}
	Since the derivative of the product of two functions is the sum of the partial derivatives, then we also have:
	
	If we put $t_br_c\equiv \nabla_j v_i$ then we have (result which we will use after having proved the Ricci theorem to further determine the Einstein tensor necessary in the section of General Relativity):
	
	In curvilinear coordinates, in order for the differential of a vector to be a vector, the two vectors of which we take the difference must be in the same point of space. In other words, one of the two infinitely close vectors must be transported in one way or another to the point where the second is located, and only after making the difference between the two vectors which are now in one and only one point of the same punctual space. The "\NewTerm{parallel transport}\index{parallel transport}" operation must be defined in such a way that in cartesian coordinates (for the small example...), the difference of the components coincides with the ordinary difference $\mathrm{d}v_k$. 
	
	Thus, we have indeed in Cartesian coordinates:
	
	since in this system: $\Gamma_{kj}^i=0$.
	
	Thus, in curvilinear coordinates, the difference of the components of the two vectors after the transport of one of them to the point where the other is denoted $\delta v_k$ such that we have:
	
	This brings us by identification to write:
	
	But also to write the principle of least action (variational principle) in the tensorial form:
	
	Let us consider now a tensor of order two, product of two tensors of order one such that (as we have seen in our study of tensor compositions):
	
	Therefore:
	
	hence (we take out the last two equalities just for aesthetics!):
	
	A parallel transport is therefore an operation that takes a tangent vector and moves it along a path in space without turning it (relative to the space) or changing its length. In flat space we can say that the transported vector is parallel to the original vector at every point along the path. In curved space we cannot say such a thing. Let's use the spherical surface of Earth to show this! We start at the equator at longitude $0^\circ$ holding an arrow pointing north:
	\begin{figure}[H]
		\centering
		\includegraphics[scale=1]{img/algebra/parallel_curvature.jpg}
		\caption{Parallel transport illustration on a sphere}
	\end{figure}
	We go along longitude $0^\circ$ up to the North Pole, keeping our arrow parallel to the ground and pointing forward all the time. When we get to the North Pole our arrow is pointing in the direction of longitude $180^\circ$. Now we go south along longitude $90^\circ$ east keeping our arrow perpendicular to our path as it was at the North Pole. When we get to the equator our arrow is pointing to the east. Finally, we go west along the equator until we get back to our starting point. We keep the arrow pointing backwards all the way. Though we did not turn the arrow all the way, we are now at the starting point with our arrow turned $90^\circ$ relative to its original position. We surely can't say that it is now parallel to the original position. This means that the term "direction" cannot be defined globally in a curved space. We can only compare the direction of two vectors if they are at the same point.

	The fact that parallel transport along a closed loop changes the direction of a vector in a curved space but not in a flat space may lead to the idea of using it as a way to measure curvature. It turns out that if we choose a loop that is small enough around a point in a curved space, the amount of change in the direction of a vector that is parallel transported along it is proportional to the area enclosed by the loop. So, the ratio between the area of the loop and the amount of change in the direction of the vector (whatever way we chose to measure it) can be used as a measure to the curvature of the surface that includes the loop. Actually we define curvature by the value of this ratio.
	
	The previous relation leads us to be able to write the metric in its variational form named "\NewTerm{Ricci identity}\index{Ricci identity}":
	
	But we also have since $g_{ij}=\vec{e}_i\circ\vec{e}_j$:
	
	hence the identity:
	
	With the both relations:
	
	and the absolute differential (which is simply generalized for a tensor of order two):
	
	we get:
	
	Now, let us recall that we have by definition:
	
	Hence finally:
	
	The absolute differential on a geodesic in the approximation of an infinitesimal transport of the fundamental tensor is therefore (as we could expect) zero. This is the "\NewTerm{Ricci theorem}\index{Ricci theorem}". Some theoretical physicists then say that \og \textit{the covariant derivative kills the metric} \fg{} in the sense that the metric does not change on a space differential.

	Finally, we also see that for a tensor of order two (the metric in particular) we have:
	
	We can therefore write the absolute differential which in this particular case is zero:
	
	And therefore the "\NewTerm{covariant derivative of a $(0,2)$-form}\index{covariant derivative}" of the metric is null:
	
	\begin{tcolorbox}[title=Remark,colframe=black,arc=10pt]
	The reader will have to remember for the definition of the Einstein tensor that:
	
	and that this is another way of expressing that an infinitesimal variation on a geodesic according to the principle of least action kills the metric. We will therefore work from now on (as before) with nonlinear differential equations that must be integrated to find the behavior of matter in a given space.
	\end{tcolorbox}
	In general, if $T$ is an $(r,s)$ tensor field (we will need that during our study of General Relativity), then $\nabla T$ is an $(r,s+1)$-tensor field given by:
	
	where the latter expression is intrinsically defined by the equation:
	
	In particular, if $T$ is a $(1,1)$-tensor, then $(\nabla_X T)(\eta,Y)$ is intrinsically defined by the equation:
	
	and therefore the correct expression for $(\nabla_X T)(\eta,Y)$ is:
	
	This will help us to derive a special case of:
	
	for the case of a $(1,1)$-tensor (as we will have during our Study of General Relativity).

	So, by the definition above in the case of a $(1,1)$-tensor we have $T^{\mu}_{\phantom{\mu}\nu\,;\rho}:=\nabla_{e_\rho}T(f^\mu,e_\nu)$, where we are using $f^\mu$ and $e_v$ for $\mathrm{d}x^\mu$ and $\partial/\partial x^/nu$.

	Hence we have (it's very not obvious so maybe in the future we will detail the process more or found another approach):
	
	The first and third terms are easily evaluated:
	
	For the second term, since $f^\mu$ is a $(0,1)$-tensor field, $\nabla f^\mu$ is a $(0,2)$-tensor field and hence $\nabla_{e_\rho}f^\mu$, for fixed $e_\rho$, is a $(0,1)$-tensor field. We evaluate:
	
	Therefore $\nabla_{e_\rho}f^\mu =  -\Gamma_{\sigma\rho}^\mu f^\sigma$ and hence:
	
	Thus, we have:
	
	
	\begin{tcolorbox}[colframe=black,colback=white,sharp corners]
	\textbf{{\Large \ding{45}}Example:}\\\\
	Let's try to verify this by calculating one component of the covariant differentiation in the spherical coordinates.\\
	
	We recall from our article that the diagonal metric of the $2$-sphere surface $\mathcal{S}^2$ expression is (\SeeChapter{see section Differential Geometry \pageref{metric two sphere}}):
	
	If we were to calculate the component $g_{\phi\phi;\theta}$, we should then write:
	
	But $g_{ij}\neq 0$ only if $i=j$, based on the above expressions, so we can simplify this relation:
	
	As (\SeeChapter{see section Differential and Integral Calculus page \pageref{usual derivatives}}):
	
	 Further below (see page \pageref{Christoffel symbol 2-sphere}) we will prove that for $\mathcal{S}^2$ we have:
	
	So that:
	
	Finally, we confirm that this component of the covariant derivative with respect to $\theta$ equals also zero in a polar coordinates system, as expected:
	
	\end{tcolorbox}
	Let us now determine an expression which will be very useful in General Relativity when we will determine the Einstein field equations (another way of expressing that the covariant derivative of the metric is null):

	Let us perform the contracted multiplication of:
	
	by $g^{ij}$, it then comes by using the relation $g^{ij}g_{jl}=\delta_i^l$ (which we had proved much earlier above) that:
	
	hence the relation:
	
	The quantities $\Gamma_{jh}^j$ and $\Gamma_{ih}^h$ representing the same sums, we then have:
	
	\begin{theorem}
	Let us consider now the determinant $g$ of the quantities $g_{ij}$. The derivation of the determinant gives us:
	
	\end{theorem}
	\begin{dem}
	Given any variable that we choose here as being the time $t$ only to simplify the notations of the calculations that will follow. When the main part of the development is completed, the result can be adapted to any other variable! We will write for what will follow $g_j$ the elements of the $j$-th column of $g_{ij}$.

	For the following developments, we define the notations:
	
	The rule of derivation of a functional determinant is given for recall (\SeeChapter{see section Linear Algebra page \pageref{derivative of a determinant}}) by:
	
	By considering the first determinant, by using the minors (\SeeChapter{see section Linear Algebra page \pageref{minor}}) for the development of its first column:
	
	For the $j$-th determinant, it comes:
	
	Thus:
	
	Now, we have demonstrated much earlier that the metric tensor is its own inverse. Therefore:
	
	This allows us to write:
	
	and therefore:
	
	what is also written (following the conventions defined at the beginning of the proof):
	
	where the reader must therefore be careful not to read badly, typically thinking that the derivative $\mathrm{d}_t$ in the right-hand term derives everything ($g_{ij}g^{ij}$)... while it only derives $g_{ij}$.
	
	However, we can adopt another variable. Let $h$ be this other variable:
	
	Either by rearranging:
	
	This is what we wanted to prove.
	\begin{flushright}
		$\blacksquare$  Q.E.D.
	\end{flushright}
	\end{dem}
	Now by combining:
	
	Demonstrated earlier above and the result we have just proved:
	
	it comes:
	
	Therefore we have:
	
	Let us prove that it is possible to derive this last relation from the following important equality:
	
	Indeed:
	
	This relation does not mean much until we will make a more explicit use of it in our study of General Relativity (\SeeChapter{see section General Relativity page \pageref{general relativity}}).

	Let us now determine the second covariant derivative of the metric tensor. Let us remember before we go further (because it is important) that we had obtained:
	
	
	\subsection{Riemann-Christoffel symbols}\label{Riemann-Christoffel symbols}
	\begin{tcolorbox}[colback=red!5,borderline={1mm}{2mm}{red!5},arc=0mm,boxrule=0pt]
	\bcbombe Before reading what follows ... We want to remind the reader \underline{again} that the writing of this section is not finished (as most chapters in the book)! Thus, we still have to illustrate the abstract notions that will follow with concrete practical examples but that are example outside of the special case of General Relativity!
	\end{tcolorbox}
	
	Let us recall that we have demonstrated earlier above that:
	
	This relation means, exception of an interpretation error from the writer of those linear as sometimes interpreting Tensor Calculus result is a pain in the a.., that the covariant derivative of a tensor of order two - such as the metric - on a geodesic path in two perpendicular directions (the second covariant derivative making it possible to the "perpendicular geodesic" between the two geodesics infinitely close to the first covariant derivative). We know already that such a shift is "parallel transport".
	
	By substituting in it:
	
	We then have:
	
	Let us now switch the indices $j$ and $k$ in the previous expression to have a differential with respect to another path:
	
	Assuming that the components satisfy the classical properties $\partial_{kj}v_i=\partial_{jk}v_i$, we get by subtraction of the two previous expressions:
	
	And since we have proved that in the case of a symmetric metric we have:
	
	We have therefore:
	
	\begin{tcolorbox}[title=Remark,colframe=black,arc=10pt]
	The fact of having in the case of a symmetric metric $\Gamma_{jk}^r=\Gamma_{kj}^r$ and which vanishes in the preceding relation, leads many practitioners to define what we name the "\NewTerm{torsion tensor}\index{torsion tensor}\label{torsion tensor}" or "\NewTerm{torsor}" (but in reality it is a particular case of a more general relation which belongs to the domain of differential geometry). Thus, we define the torsion tensor as:
	
	and in the case of a symmetric metric (Euclidean space), the torsion is zero by extension as we have already seen it! In fact, the Einstein field equations which we shall prove later implicitly implies a symmetric metric with zero torsion. However, it is possible to proved that a non-symmetric tensor can always be decomposed into a symmetric and non-symmetric tensor (this is trivial because it is like separating a complete matrix in the sum of a matrix having a diagonal of non-zero component that are and another matrix having the diagonal zero).
	\end{tcolorbox}
	It then remains:
	
	Since parallel transport takes place on infinitely close geodesic paths, we take the limit:
	
	Which mainly underlies the fact that the velocity field is almost equal in two infinitely close parallel points.

	It then remains:
	
	This relation expresses the fact that, like gravity, the curvature of space-time causes a mutual acceleration between the geodesics! Moreover, it is easy to see that the mutual acceleration between the geodesics is zero if the Riemann-Christoffel tensors are null (typically in Cartesian coordinates, in extenso it means for a flat time-space). This is exactly what we expect of gravity: if we do not observe any acceleration, the curvature (we shall now define what it is) is zero and if the curvature is zero, we observe no acceleration. Morale of the story: the gravity is curvature and the curvature is gravity !!

	We see that the quantity in parentheses is a tensor of order four that we will write on this site as following (because there are several traditions in the way to write it...):
	
	and which summarizes itsel the parallel transport and the fact that gravity and geometry of space are linked together. Obviously, if the metric is of Minkowski type (or tends to a metric of Minkowski under certain conditions), then $R_{i,jk}^l$ is zero! Very rare authors write this last equality in the (unhappy ....) form:
	
	The tensor $R_{i,jk}^l$ is named the "\NewTerm{Riemann-Christoffel tensor}\index{Riemann-Christoffel tensor}" or "\NewTerm{Riemannian space tensor}\index{Riemannian space tensor}". The curvature of a Riemannian space can also be characterized using this tensor.

	If we multiply the tensor $R_{i,rs}^k$ by $g_{jk}$, then we have the covariant components of this tensor such that:
	
	and given the following relations that we proved earlier above:
	
	Therefore we get:
	
	and let us replace the quantities $g_{jk}\partial_r\Gamma_{is}^k$ by $\partial_r\left(g_{jk}\Gamma_{is}^k\right)-\Gamma_{is}^k\partial_r g_{jk}$. We then get:
	
	We also proved earlier before that:
	
	Hence:
	
	and as:
	
	we get:
	
	and we also proved that:
	
	Hence:
	
	And by reporting them into the prior-previous relation, we get:
	
	Finally, we get for the covariant expression of the Riemann-Christoffel tensor:
	
	It should be noticed that the Riemann-Christoffel tensor is therefore antisymmetric:
	
	and that in the parenthesis of the prior-previous relation we have only double partial derivatives, while outside of the parenthesis the Christoffel symbols contain only simple partial derivatives!

	Finally, the permutation of the indices $ij$ and $rs$ as a block gives us as a consequence of the symmetry of the $g_{ij}$ and by inverting their derivation order:
	
	Let us now perform a circular permutation on the indices $j$, $r$, $s$ in the expression (obtained just above)
	
	then we get:
	
	and we get (it is very simple to control by summing the three lines above):
	
	The previous identity is named the "\NewTerm{first Bianchi identity}\index{first Bianchi identity}\label{first bianchi identity}" or also "\NewTerm{Bianchi algebraic identity}\index{Bianchi algebraic identity}" and it highlights the cyclicity property of the  Riemann-Christoffeltensor. In reality, we should not use the word "identity" since it is verified only (at least to my knowledge) in the case of a symmetric metric tensor (otherwise, the torsion is not zero for recall!).

	The reader will observe that it is immediate that this last relation is satisfied in the case of the Minkowski metric, since if at all points the partial derivative of the metric is zero we have:
	
	And we will see in the sectin of General Relativity that this first identity will serve as a basis for the construction of the Schwarzschild metric.

	If the metric is of the Minkowski type (we change the notations of the indices to be more in conformity with the usual writings in general relativity) then it is immediate that we also have:
	
	But in the case where the metric is not of the Minkowski type, this latter relation can be satisfied and has an interest only if and only if the chosen metric is decomposable in Taylor series whose first partial derivatives are zero at $0$ (see the section of Differential Geometry for such Taylor series!).

	This relation is valid only in the case of a "\NewTerm{locally inertial frame (LIF)}\index{locally inertial frame}" in which all Christoffel symbols cancel each other but not their derivatives.

	By extension:
	
	Let us recall that implicitly, this relation, named "\NewTerm{Bianchi second identity}\index{Bianchi second identity}\label{Bianchi second identity}" or "\NewTerm{Bianchi differential identity}\index{Bianchi differential identity}", always expresses simply (if one may say ...) the fact that gravity and geometry of space are linked together.

	Following a reader request let us detail much more how to derivate this identity!

	The identity is easiest to derive at the origin of a locally inertial frame (LIF) as already mention, where the first derivatives of the metric tensor, and thus the Christoffel symbols, are all zero. At this point, we have
	
	If the Christoffel symbols are all zero, then the covariant derivative becomes the ordinary derivative
	
	Therefore, we get, at the origin of a LIF:
	
	By cyclically permuting the index of the derivative with the last two indices of the tensor, we get
	
	By adding up all linear with the covariant derivative and using the commutativity of partial derivatives, we see that the terms cancel in pairs, so we get
	
	As usual we can use the argument that since we can set up a LIF with its origin at any non-singular point in spacetime, this equation is true everywhere and since the covariant derivative is a tensor, this is a tensor equation and is thus valid in all coordinate systems.
	
	
	\subsection{Ricci curvature (Ricci tensor)}\label{Ricci tensor}
	Before we can see the consequences of second Bianchi's identity, we need to define the "\NewTerm{Ricci tensor}\index{Ricci tensor}":
	
	which is therefore simply the contraction of the first and third indices of the Riemann-Christoffel tensor which we have given above:
	
	in other words it is just a more condensed notation ... and then the letters for the upper or lower indices as well as the presence of the comma are at the free choice of the writer (according to the mood and especially if the context makes it possible to avoid any confusion).

	For example with the Riemann-Christoffel tensor we have just given, the Ricci tensor could be written in the following two ways (we keep the indices with latin letters):
	
	The Ricci tensor can be taken as the trace of the Riemann tensor, hence it is of lower rank, and has fewer components. If you have a small geodesic ball in free fall, then (ignoring shear and vorticity) the Ricci tensor tells you the rate at which the volume of that ball begins to change, whereas the Riemann tensor contains information not only about its volume, but also about its shape. 
	
	Other contractions of other indices may also be possible but because $R_{\alpha\beta,\mu\nu}$ is antisymmetric on $\alpha,\beta$ and $\mu,\nu$ then the contraction on these indices are equivalent to have $\pm R_{\alpha\beta}$.
	\begin{tcolorbox}[title=Remark,colframe=black,arc=10pt]
	There is no widely accepted convention for the sign of the Riemann
curvature tensor, or the Ricci tensor, so check the sign conventions of whatever book you are reading!
	\end{tcolorbox}
	Similarly, we define the "\NewTerm{Ricci scalar}\index{Ricci scalar}\label{ricci scalar}" (also sometimes named "\NewTerm{Riemann scalar}\index{Riemann scalar}") by the relation:
	
	which has the following properties:
	\begin{itemize}
		\item If the space is flat, the Ricci scalar is zero
	
		\item If space is curved like a sphere, the Ricci scalar is positive
	
		\item If the space is curved like a horse saddle, the Ricci scalar is negative
	\end{itemize}
	Either explicitly (by changing the notation for the indices in order to insist on the fact that it has no impact!):
	
	The Ricci scalar is the trace of the Ricci tensor, and it is a measure of scalar curvature. It can be taken as a way to quantify how the volume of a small geodesic ball (or alternatively its surface area) is different from that of a reference ball in flat space.

	We will have concrete practical examples in the section of General Relativity for the first two cases, but let us look at simplified examples for the first two cases (we will not, however, prove the reciprocal).
	
	\begin{tcolorbox}[colframe=black,colback=white,sharp corners]
	\textbf{{\Large \ding{45}}Examples:}\\\\
	E1. Let us start with the metric of flat space (without the temporal component). We have (\SeeChapter{see section General Relativity page \pageref{metric flat space}}):
	
	By taking the definition of Ricci's scalar in explicit form:
	
	It is immediate that R is zero since the partial derivatives will all be zero. So a flat space has a null Ricci scalar.\\
	
	E2. Let us now look at the metric of the plane expressed in spherical coordinates (without the temporal component). We have (\SeeChapter{see section General Relativity page \pageref{metric spherical space}}):
	
	and:
	
	with:
	
	We know that to compute the Ricci scalar (or Ricci curvature), we must therefore compute the contraction of the Riemann-Christoffel tensor (that is, the Ricci tensor) which itself depends on the Christoffel symbols of the second kind which themselves depend on the symbols of Christoffel of the first kind (argh!). We should therefore begin with the lowest level, that is to say by determining all the Christoffel symbols of the first kind given for recall by:
	
	
	\end{tcolorbox}
	\begin{tcolorbox}[colframe=black,colback=white,sharp corners]
	So we have an $3^3$, that is $27$ possible Christoffel symbols of the first kind! Even if some symbols are equal (we have proved it!), We will still calculate everything.\\

	Let's start with joy and good humor ...:
	
	\end{tcolorbox}
	
	\begin{tcolorbox}[colframe=black,colback=white,sharp corners]
	
	Let us now calculate all the symbols of Christoffel symbols of the second in the details:
	
	Again, as the metric tensor is diagonal, this will simplify the calculations!\\

	We have then:
	
	\end{tcolorbox}
	\begin{tcolorbox}[colframe=black,colback=white,sharp corners]
	Let us now calculate the $9$ components of the Ricci tensor in details according to:
	
	We then have (we calculate them all, even if we know that subsequently those which do not have $\alpha=\beta$ will be useless by the fact that the metric is diagonal):
	
	\end{tcolorbox}
	
	\begin{tcolorbox}[colframe=black,colback=white,sharp corners]
	
	\end{tcolorbox}
	
	\begin{tcolorbox}[colframe=black,colback=white,sharp corners]
	
	\end{tcolorbox}
	
	\begin{tcolorbox}[colframe=black,colback=white,sharp corners]
	
	\end{tcolorbox}
	
	\begin{tcolorbox}[colframe=black,colback=white,sharp corners]
	
	Let us now calculate the Ricci scalar:
	
	we then have:
	
	The Ricci scalar is therefore also zero. This result may be surprising, but in reality it is logical since we have only calculated the scalar curvature of a flat space expressed in spherical coordinates.
	\end{tcolorbox}
	
	\begin{tcolorbox}[colframe=black,colback=white,sharp corners]
	E3. \label{Christoffel symbol 2-sphere}Let us now impose ourselves the diagonal metric of the 2-sphere surface $\mathcal{S}^2$ (without the temporal component). We have then in accordance with what we have seen in the section of Differential Geometry page \pageref{metric two sphere}:
	
	and:
	
	Therefore (\SeeChapter{see section Differential geometry page \pageref{metric two sphere}}):
	
	often written as:
	
	where $r$ is a constant!\\

	We shall therefore begin with the lowest level, that is, by determining all the Christoffel symbols of the first kind given for recall by:
	
	We have therefore $2^3$, that is, $8$ possible Christoffel symbols of the first kind! Even if some symbols are equal (we have already proved it!), we will still calculate everything:
	
	\end{tcolorbox}
	
	\begin{tcolorbox}[colframe=black,colback=white,sharp corners]
	
	Let us now calculate all the Christoffel symbols of the second kind in the details:
	
	Again, as the metric tensor is diagonal, this will simplify the calculations!\\

	We have then:
	
	Let us now calculate the $4$ components of the Ricci tensor in the details according to:
	
	We then have (we calculate them all, even if we know that subsequently those which do not have $\alpha=\beta$ will be useless by the fact that the metric is diagonal):
	
	\end{tcolorbox}
	
	
	\begin{tcolorbox}[colframe=black,colback=white,sharp corners]
	
	Let us now calculate the Ricci scalar :
	
	\begin{tcolorbox}[title=Remark,colframe=black,arc=10pt]
	Notice that $R_{\alpha\beta}=1/r^2 g^{\alpha\beta}$ and for information we can also prove that (its aesthetics but so far it has no practical application in this book):
	
	\end{tcolorbox}
	We then have:
	
	We notice that:
	\begin{enumerate}
		\item The Ricci scalar is a constant. This means that the hypersurface has a constant curvature at all points of the surface (we know that the sphere by symmetry has a constant curvature at all points). It thus possesses a form of symmetry, with respect to its curvature. We are then dealing with a "\NewTerm{maximally symmetrical variety}".

		\item This scalar is positive which describes a domed space (ball, sphere) and if $r\rightarrow +\infty$ the curvature is zero!
	\end{enumerate}
	\end{tcolorbox}
	\begin{tcolorbox}[title=Remark,colframe=black,arc=10pt]
	Be careful not to confuse the value of the Ricci curvature with that of the Gauss curvature!!!
	\end{tcolorbox}
	
	\subsection{Einstein Tensor}\label{einstein tensor}
	Let us apply a contraction to the second Bianchi  identity (valid for recall with the "$+$" only if the metric is positive):
	
	Let us recall that $\nabla_\lambda g_{\alpha\mu}=0$ and similarly by extension that $\Delta_\lambda g^{\alpha\mu}=0$. So finally this leads us to write by the property of the derivatives (product in sum):
	
	and therefore to get:
	
	 Using the antisymmetry property of the Riemann-Christoffel tensor we write:
	
	What ultimately brings us to write from the definition of Ricci's tensor:
	
	This last relationship being named "\NewTerm{contracted Bianchi identity}".

	Let us contract this relation once more:
	 
	That which is identical to write using the properties of the Einstein summation (which allows to freely change the indices):
	
	Which is equivalent to:
	
	As $\nabla_\lambda=g_{\lambda}^\mu \nabla_\mu R$, we have:
	
	By raising the index $\lambda$ by multiplication with $g^{\nu\mu}$, we get the "\NewTerm{Einstein's identity}\index{Einstein's identity}":
	
	The "\NewTerm{Einstein tensor}\index{Enstein tensor}" (tensor of order two and contravariant in the present case) which is therefore a constant in a given Riemannian space is therefore defined by:
	
	And expresses in the shortest possible way, parallel transport under all assumptions seen so far.

	Identically, we can obtain the covariant form:
	
	The tensor is therefore constructed for a Riemannian metric only (which nevertheless makes a lot of possible spaces ...), and is automatically non-divergent:
	
	It must be remembered, however, that a large part of the latest developments consider a symmetrical metric. This is why some speak of "\NewTerm{symmetrical gravitational theory}" when we deal with General Relativity.

	We shall find this tensor naturally in the sectionof General Relativity when, by making use of the variational principle, we decompose the action into two terms:
	\begin{itemize}
		\item the action of mass in the gravitational field

		\item the action of the gravitational field in the absence of mass
	\end{itemize}
	By expressing the whole in a Riemannian space we will then get the no less famous Einstein field equations (without further explanations in this section):
	
	the details being given in the section of General Relativity.
	\begin{tcolorbox}[title=Remark,colframe=black,arc=10pt]
	As we see, we can very well add a constant term to the expression of Einstein's tensor, without changing the nullity of its divergence. This fact, used in Astrophysics, makes it possible to construct models of particular Universes that we will deal with in the section of Cosmology.
	\end{tcolorbox}
	\begin{tcolorbox}[colframe=black,colback=white,sharp corners]
	\textbf{{\Large \ding{45}}Example:}\\\\
	Let us calculate the order $2$ covariant Einstein tensor:
	
	based on the diagonal metricsurface of the 2-sphere $\mathcal{S}^2$ (without the temporal component):
	
	Since the metric is diagonal, we have proved earlier above and in detail by the example that:
	
	And as in the present case, we also have:
	
	It comes:
	
	So we have to focus only on two components:
	
	which confirms what we have said earlier above.
	\end{tcolorbox}
	The complexity of the Einstein Tensor expression can be shown using the formula for the Ricci tensor in terms of Christoffel symbols:
	
	where ${\displaystyle \delta _{\beta }^{\alpha }}$ is the Kronecker tensor and the Christoffel symbol $\Gamma ^{\alpha }{}_{\beta \gamma }$ is given for recall by:
	
	Before cancellations, this formula results in $2\times (6+6+9+9)=60$ individual terms. Cancellations bring this number down somewhat.
	
	\begin{flushright}
	\begin{tabular}{l c}
	\circled{95} & \pbox{20cm}{\score{4}{5} \\ {\tiny 40 votes,  82.5\%}} 
	\end{tabular} 
	\end{flushright}
		
	%to make section start on odd page
	\newpage
	\thispagestyle{empty}
	\mbox{}
	\section{Spinor Calculus}\label{spinors}
	\lettrine[lines=4]{\color{BrickRed}A}s we will see first in relativistic quantum physics, spinors play a major role in quantum theory, and therefore in all contemporary physics (quantum field theory, standard model, string theory, etc.). The actual purpose of this section on the spinors is only to give the tools to the reader that are necessary to a deep understanding of what will be done in the chapter about Atomistic.
	
	It was starting 1927 that the physicists Pauli, and after Dirac introduced spinors  for the representation of the wave functions (\SeeChapter{see section Relativistic Quantum Physics page \pageref{spinor relativistic quantum physics}}). However, in their mathematical form, spinors were discovered by Elie Cartan in 1913 during his research on the representations of the groups following the general theory of Clifford spaces (introduced by the mathematician W.K. Clifford in 1876). He showed, as we will see it, that in spinors provide in fact a linear representation of the group of rotations of a space with any number of dimensions. Thus, spinors are closely related to geometry but they are often introduce in an abstract way without intuitive geometric meaning. Thus, we will try (as always on this book) in this section to introduce this tool in the most simple and intuitive possible way with a maximum of details.
	
	The spinor formalism is not interest only of interest for quantum physics and its related developments, among others, Roger Penrose showed that the spinor theory was an extremely fruitful approach to the theory of General Relativity. Even if the most commonly used tool for the treatment of General Relativity is the tensor calculus, Penrose seems to have shown that in the specific case of the four-dimensional space and in the metric Lorentz the formalism of two components spinors was more appropriate.
	
	The theory of spinors named, "\NewTerm{spinor calculus}\index{spinor calculus}" or sometimes "\NewTerm{spin geometry}\index{spin geometry}" is extremely broad but as we know this book aims to address the physicists and engineers therefore we will limit ourselves to spinors properties useful in quantum physics (at least actually).
	
	\begin{tcolorbox}[title=Remark,colframe=black,arc=10pt]
We strongly recommend the reader to have previously read the subsubsection on quaternions (\SeeChapter{see section Numbers page \pageref{quaternions}}), the subsubsection on rotations in space (\SeeChapter{see section Euclidean Geometry page \pageref{rotation}}) and finally, if possible, for a physical practical example, the section on Relativistic Quantum Physics page \pageref{relativistic quantum physics}.
	\end{tcolorbox}
	
	\pagebreak
	\subsection{Unit Spinor}
	
	We will give here a first simplified and special definition (or example) of spinors. Thus, we will show that it is possible from such a tool to represent a vector equation of a space $e^3$ of three components using a two-component spinor. The method is extremely simple and the reader who has already read the part of the section on Quantum Wave Physics dealing with the Dirac equation and the section on Quantum Computing will see a rather beautiful analogy.
	
	Consider to start the sphere of radius of the following equation (\SeeChapter{see section Analytic Geometry page \pageref{sphere}}):
	
	And consider the following figure:
\begin{figure}[H]
\centering
\includegraphics[scale=0.75]{img/algebra/spinor_unit_sphere.jpg}
\caption[]{Unit sphere}
\end{figure}
Let us put on this sphere of center on O and unit radius a point $P$ of coordinates $(x,y,z)$  and denote by $N$ (north) and $S$ (south) the points of the sphere intersecting with the $Z$ axis.

The point $S$ will have by convention for coordinates:
	
We obtain a projection so-named "\NewTerm{stereographic projection} \index{stereographic projection}" $P'$ of the point $P$ by tracing the straight line $SP$ that passes through the complex equatorial plane $x\text{O}y$  (yes! we chose a $\mathbb{C}$ as plane!) at point $P'$ of coordinates $(x', y', z')$.

The similar triangles $SP'\text{O}$ and $SPQ$ (with $Q$ being the orthogonal projection onto the $Z$-axis of the point $P$) give us the following relations by simply applying Thales' theorem (\SeeChapter{see section Euclidean Geometry page \pageref{thales theorem}}):
	
	
	\begin{tcolorbox}[title=Remark,colframe=black,arc=10pt]
The last two relation (ratios) are obtained by simply applying Thales' theorem (see section Euclidean Geometry) in the complex equatorial plane.
	\end{tcolorbox}
	
	Let us write now to simplify the notations:
	
	We have from the prior-previous relation that:
	
	Taking the squared modulus (see the study of complex numbers in the section on Numbers):
	
	and from  the equation of the sphere it follows:
	
	we finally get:
	
	Let us write now the complex number $\xi$ under the form:
	
	
	 where $\phi,\psi$ are two complex numbers that we can always impose verify the following condition of unitarity (nothing prohibits to do that but for theoretical physics purpose this choice suit us well...):
	
	\begin{tcolorbox}[title=Remark,colframe=black,arc=10pt]
		The following complex numbers satisfy for example (hazard!!??)  the above condition:
		
	\end{tcolorbox}	
	Remember before continuing that we have proved in our study of complex numbers (\SeeChapter{see section Numbers page \pageref{module ration complex numbers}}) that:
	
	Therefore it comes by injecting the last two relations in the equation given above:
	
	So we get:
	
	Rearranged:
	
	Therefore:
	
	Finally we have a simple expression for the $z$ coordinate of the point $P$ because in the last equality above you must remember that $\psi\bar{\psi}+\phi\bar{\phi}=1$ then:
	
	As we have:
	
	Then by summing and respectively by substracting the two above relations and using previous results we get:
	
	To resume for the point $P$ we have with our choices:
	
	Thus, for any point $P$ on the sphere of radius unity, we can match a pair of complex numbers satisfying the imposed unitary identity!
	
	\pagebreak
	Therefore in complete and explicit form we finally have using what we know about complex numbers (\SeeChapter{see section Numbers page \pageref{complex numbers}}) and remarkable trigonometric functions (\SeeChapter{see section Trigonometry page \pageref{remarkable trigonometric identities}}):
	
	We notice easily that the norm of this vector is equal to $1$.
	
	This last relation also indicates us that $2\alpha$ is the angle between $\text{O}z$ and $\overrightarrow{OP}$ (since the hypotenuse of vector's angle has a unit norm) and therefore by deduction $\gamma-\beta$ represents the angle between $\text{O}x$ and the plane $(\text{O}z, \overrightarrow{OP})$:
	
	\begin{figure}[H]
		\centering
		\includegraphics{img/algebra/spinor_simple_rotation.jpg}
		\caption[]{Representation of the rotation}
	\end{figure}

	The pair of complex numbers:

	

	is by definition a "\NewTerm{unitary spinor}\index{unitary spinor}" (it contains also all the information about $z$). Thus, as we have seen, a unitary spinor can also be expressed in the form:
	
	As well any spinor can be written in the more general form:
	
	The spin is essentially measured from the oriented $z$-axis as we have seen it yet with the previous figure.
	
	The stereographic projection led us to represent certain vectors of the Euclidean space $\mathcal{E}^3$ with the elements of a complex two-dimensional vector space that is the "\NewTerm{space of spinors}\index{space of spinors}".
	
	\begin{tcolorbox}[title=Remark,colframe=black,arc=10pt]
This representation is not unique because the arguments of complex number are (in trigonometric form) determined at a given offset constant!
	\end{tcolorbox}	
	
	The reader who has already read a little bit the section on Wave Quantum Physics  will probably noticed the strange (not innocent) similarity of the following identity and relations:
	
	compared to the de Broglie normalization condition (the integral over the entire space of the sum of the products of the complex wave functions and its conjugate is equal to the unit) and to the developments determining the continuity equation also in the section of Wave Quantum Physics .
	
	Let us now see that for future needs, we can find two new vectors $\vec{v}_1,\vec{v}_2$ of Euclidean space $\mathcal{E}^3$ associated with a unitary spinor $(\psi,\phi) $ determined on the unit sphere. These vectors will have to be orthogonal to each other and with unit norm, each orthogonal to the vector $\overrightarrow{OP}$.	
	
	To simplify the notations let us write  $\vec{v}_3=\overrightarrow{OP}$ and $\vec{v}_i=(x_i,y_i,z_i),i\in \left\lbrace 1, 2 ,3\right\rbrace$.
	
	The $\vec{v}_1,\vec{v}_2,\vec{v}_3$ vectors are of course bounded by the cross product (\SeeChapter{see section Vector Calculus page \pageref{cross product}}):
	
	hence taking into account the expression of $\overrightarrow{OP}$ components based on its associated spinor, and the fact that $\psi\bar{\psi}+\phi\bar{\phi}=1$, we obtain:
	
	Writing the orthogonality of vectors we get them obviously six additional equations. However the orientation of vectors $\vec{v}_1,\vec{v}_2$ being not fixed, there is some uncertainty in the values of their components. Let us select values such that:
	
	Taking the complex conjugate quantities of previous relation and summing, to have only real parts we have to write:
	
	We can control the norm is equal to the unit. Just check with the squared norm:
	
	In the same way we get:
	
	We can easily check that these values restore well the relations of the vector cross product of $\vec{v}_2$. At any unitary spinor $(\psi,\phi)$ we can therefore associate three vectors $\vec{v}_1,\vec{v}_2,\vec{v}_3$. We can directly check that the vectors thus calculated are mutually orthogonal and with unit norm.
	
	A reader has make us the request to show in much details as possible this affirmation for $x_x$. So let's go:
	 
	
	\subsection{Geometric Properties}
	We will study the transformations of vectors associated to a spinor to derive the corresponding properties of spinor transformation. As we know some special (trivial) rotations in space can always be expressed as the product of two plane symmetries, therefore we begin by studying these latter.
	
	\subsubsection{Plane Symmetries}
	Let us consider first the plan symmetry of a vector:
	
	During a symmetry relative to a plane $P$, any vector $\overrightarrow{OM}$ is transformed into a vector $\overrightarrow{OM'}$. Let us determine a matrix $S$ representing this symmetry with respect to this plane! 
	
	Given $\overrightarrow{\text{O}A}$ a unit vector normal to the plane $\mathcal{P}$ and $H$  the root of the perpendicular projection from any given point $M$ of space on the plane $\mathcal{P}$:
	
	\begin{figure}[H]
		\centering
		\includegraphics{img/algebra/plane_symmetry.jpg}
		\caption[]{Plane symmetry of a vector relatively to a plane}
	\end{figure}
	Let $M'$ be the symmetric point $M$ with respect to the plane $\mathcal{P}$, we have:
	
	Given $a_1,a_2,a_3$ the cartesian components of $\overrightarrow{\text{O}A}$ and $(x,y,z),(x',y',z')$ the respective components of the vectors $\overrightarrow{\text{O}M},\overrightarrow{\text{O}M'}$, the above equation gives us the linear relations:
	
	The matrix $S$ that take us from vector $\vec{X}(x,y,z)$ to the vector $\vec{X}'(x,y,z)$ has therefore the following expression:
	
	We keep in mind this result and let us now consider two vectors $(\vec{X}_1,\vec{X}_2)$ orthogonal to each other and unitary, defining as we have above  seen a unitary spinor $(\psi,\phi)$ (we used the notation $\vec{v}_1,\vec{v}_2$ before). A symmetry with respect to a plane $\mathcal{P}$ transforms the vectors $\vec{X}_1,\vec{X}_2$ into vectors $\vec{X'}_1,\vec{X'}_2$ which are associated the spinor $(\psi',\phi')$. 
	
	\begin{theorem}
	We will now show that the following transformation of the spinor $(\psi,\phi)$ into spinor $(\psi',\phi')=(x'_3,y'_3,z'_3)$ is:
	
	\end{theorem}
	and transforms the vectors $\vec{X}_1,\vec{X}_2$ into vectors $\vec{X}_1^{\prime},\vec{X}_2^{\prime}$, these vectors being deduced respectively - as we will just show it - from each other by a single plane symmetry and the matrix $\mathcal{A}$ represents well the sought transformation.
	
	The previous relation gives us therefore:
	
	In all we have so far the previous set of relations and:
	
	Therefore we can deduce:
	
	After, using the fact that $\|\vec{A}\|=a_1^2+a_2^2+a_3^2=1$, we get:
	
	So we fall well back on the symmetry matrix:
	
	Thus, the matrix that we will see again in the section of Relativistic Quantum Physics:
	
	$\mathcal{A}$ therefore generates the transformation of a spinor $(\psi,\phi)$ into a spinor $(\psi',\phi')$ such that the associated vectors $(\vec{X}_1,\vec{X}_2)$ can be deduced respectively from $(\vec{X}_1^{\prime},\vec{X}_2^{\prime})$  by a planar symmetry.
	
	\subsubsection{Rotations}
	As we have saw it in the section of Euclidean Geometry, it is possible to rotate a vector in the plane or in space using matrices. Similarly, by extension, it is clear that the multiplication of two rotations is a rotation (that is the elementary linear algebra - at least we consider it as is).
	
	Consider therefore two planes $P, Q$ whose intersection generates a line $L$ and let us denote $\vec{A}(a_1,a_2,a_3)$ and $\vec{B}(b_1,b_2,b_2)$ the unit vectors carried by the respective normal vectors (\SeeChapter{see section Vector Calculus page \pageref{normal vector}}) to these two intersecting planes in $L$:
	\begin{figure}[H]
		\centering
		\includegraphics{img/algebra/spinor_intersecting_planes.jpg}
		\caption[]{Illustrated intersection of two planes}
	\end{figure}
	Let us denote by $\theta/2$ the angle of the vectors $\vec{A},\vec{B}$ between them (the reason for this notation comes from our study of quaternions (\SeeChapter{see section Numbers page \pageref{quaternions}})). Given $\vec{L}$ the unit direction vector carried by the line $L$ resulting from the intersection of planes $P, Q$ and such that:
	
	Explanations: $\vec{A},\vec{B}$ are unitary but not necessarily perpendicular and we still need to ensure that $\vec{L}$ is a unit vector (the norm equal to unity!). Therefore, the above relation ensures that:
	
	The previous vector product gives us for the components of $\vec{L}$:
	
	On the other hand, the scalar product can be written:	
	
	\begin{tcolorbox}[title=Remark,colframe=black,arc=10pt]
	We will use these two planes as symmetry planes for our rotations.
	\end{tcolorbox}	
	As we have noticed previously, a rotation in $\mathcal{E}^3$ can always be done with more than two plane symmetries. Thus, a rotation can be denoted by the application (multiplication) of two matrices of symmetry according to the results obtained previously:
	
	Developing the product of these two matrices and taking into account the relations arising from vector and dot product we get:
	
	Thus, we can write the transformation of a spinor $(\psi,\phi)$ and a spinor $(\psi',\phi')$ with a matrix of the form:
	
	whose parameters are named "\NewTerm{Cayley-Klein parameters}\index{Cayley-Klein parameters}".
	
	The matrix $R\left(\vec{L},\frac{\theta}{2}\right)$ can be written in another form if we do a limited development for infinitely small rotations $\theta/2=\varepsilon/2$ (that's where the physics comes back...):
	
	Using only the first order terms, the rotation matrix is finally written:
	
	This matrix is the limited development of the matrix of rotation in the neighborhood of the identity matrix, the latter obviously corresponding to the zero rotation. We note also the latter in the form:
	
	where the matrix $\sigma_0$ is the identity matrix of order two and $\chi(\vec{L})$ is named the "\NewTerm{infinitesimal rotation matrix}\index{infinitesimal rotation matrix}". Now, if we put $L_1=1,L_2=L_3=0$ in $\chi(\vec{L})$ we get:
	
	How to interpret this result? Well it's quite simple, choose $L_1=1,L_2=L_3=0$ gives us a collinear vector $\vec{L}$ to the axis $\text{O}x$. Therefore, we can very well imagine the planes generating the axis $\text{O}x$ that carries $\vec{L}$. As $\varepsilon/2$ (verbatim $\theta/2$) is generated by the vectors $\vec{A},\vec{B}$ perpendicular to $\vec{L}$ and thus to $\text{O}x$, then the angle $\varepsilon/2$ (or its variation) represents a variation of the direction of the normal planes to $\vec{A},\vec{B}$ which by symmetry are used to construct the rotation (recall that $\vec{A},\vec{B}$are not necessarily mutually orthogonal). So by extension, having  $L_1=1,L_2=L_3=0$ allows us then only to make rotations (symmetries) around the $x$-axis.
	
	Similarly, a rotation about the $y$-axis corresponds to $L_2=1,L_1=L_3=0$, which gives:
	
	and the same with $L_3=1,L_1=L_2=0$ we finally have:
	
	The three matrices:
	
	are rotation matrices in the space of "\NewTerm{two-dimensional spinors}\index{two-dimensional spinors}". Physicists and mathematicians say that these matrices are an irreducible representation of dimension two of the group "$\NewTerm{\text{SU} (2)}$" or named  "\NewTerm{special group of spatial rotations $\text{SU}(2)$}\index{special group of spatial rotations}" (\SeeChapter{see section Set Algebra page \pageref{special unitary group}}).
	
	The previous infinitesimal matrices therefore show us in a skillful way the following matrices:
	
	These matrices are named "\NewTerm{Pauli matrices}\index{Pauli matrices}\label{pauli matrices origin}" and we will find them again in the section of Relativistic Quantum Physics as part of the study of the Dirac equation and the determination of its explicit solutions (using spinors).
	
	Using Pauli matrices, the infinitesimal rotations matrix can finally be written:
	
	Let us define a vector $\vec{\sigma}$, named "\NewTerm{Pauli vector}\index{Pauli vector}", whose components are the Pauli matrices:
	
	The expression $L_1\sigma_1+L_2\sigma_2+L_3\sigma_3$ can be written as a sort of dot product which represents a sum of matrices\label{spinor dot product} (the arrow above the sigma is sometimes omitted if no confusion is possible):
	
	The limited development is then written:
	
	The rotations matrix:
	
	can using Pauli matrices be written in the remarkable form under the assumption of small angles:
	
	Expression that we will a lot in the section of Quantum Computing to express the $R$ matrices explicitly and also in the section of Set Algebra page \pageref{set algebra}.
	
	What is written sometimes:
	
	Which can also be written in an extensive form:
	
	which has the form of a quaternion of angle $\theta$ (don't remember that this is only for small angles!) and of axis $\vec{L}$. Hence the reason that  have from the beginning chosen the notation $\varepsilon/2$.
	
	It is clear, so that the analogy with quaternions to be stronger, that the $2\times 2$ Pauli matrices are a set of four linearly independent matrices! As the canonical basis for quaternions!
	
	If we denote by $\vec{L}(L_1,L_2,L_3)=\vec{X}(x,y,z)$  then the "\NewTerm{spinor product}\index{spinor product}" is finally defined by:
	
	This matrix constitutes as we have already mentioned, to the limited development of the rotation matrix in the neighborhood of the identity matrix, the components of $\vec{x}$ being associated with a spinor whose rotation is through the double symmetry defined by two planes whose intersection is defined by the vector $\vec{X}$.
	
	We can also notice the interesting consequence that a rotation of $2\pi$ ($360^\circ$) rotation does not restore the object to its original position!!!
	
	Indeed:
	
	Therefore we need a rotation of $4\pi$ ($720^\circ$) to make a full turn! This corresponds to the spin of ${}^1{\mskip -5mu/\mskip -3mu}_2$. It takes two turns to find that the object reappears equivalently (this is counter intuitive and can make thing we are dealing with object having higher dimensions that what we exepect!). We then say that the representation of rotations is "bivaluated".
	
	Schematically this can be represented as:
	\begin{figure}[H]
		\centering
		\includegraphics{img/algebra/full_rotation.jpg}
		\caption{Spinor full rotation (special example)}
	\end{figure}
	Or you can consider the analogy that is to hold a tea cup on the palm of your hand and you turn your hand by maintaining it flat to regain its (the hand, not the cup!) original position. You will see that your hand has to do two laps!
	
	\pagebreak
	\subsubsection{Properties of Pauli Matrices}
	The reader can easily check (if this is not the case he can always contact us and we will write the details) the following main properties of the Pauli matrices, some of which will be used in the section of Relativistic Quantum Physics:
	\begin{enumerate}
		\item[P1.] Unitarity:
		

		\item[P2.] Anticommutativity:
		
		or $i\neq j$ and $i,j=1,2,3$.

		The last two properties gives us:
		
		with $i,j=1,2,3$.

		\item[P3.] Cyclicity:
		

		\item[P4.] Commutativity:
		
		
		\item[P5.] Vector product:
		
		Given the square of the different $\sigma$ by noting abusively by "$1$" the unitary matrix (we change the indices to give you the habit to use other common notations):
		
		Leading us to write that (square norm of the Pauli vector):
		
		Let us consider now the following products:
		
		Let us consider now the following products:
		
		All these relation can be summarized into a unique one (!):
		
		where for recall (\SeeChapter{see section Tensor Calculus page \pageref{kronecker symbol}}) the Kronecker symbol is defined by:
		
		and the antisymmetric symbol by:
		In three dimensions, the Levi-Civita symbol is defined as follows:
		
		i.e.  $\varepsilon_{ijk}$  is $1$ if $(i, j, k)$ is an even permutation of $(1,2,3)$ or in the natural order $(1,2,3)$, $-1$ if it is an odd permutation, and $0$ if any index is repeated. In three dimensions only, the cyclic permutations of $(1,2,3)$ are all even permutations, similarly the anticyclic permutations are all odd permutations. This means in 3D it is sufficient to take cyclic or anticyclic permutations of $(1,2,3)$ and easily obtain all the even or odd permutations.
		
		We also have:
		
		We fall back here on the components of the vector product:
		
		Now let us develop an important spinor identity which will be useful to us in the section of Relativistic Quantum Physics:
		
		But we also have:
		
		So finally:	
		
	
		\item[P6.] We note that these matrices are also hermitian (let us recall that a Hermitian matrix is a transposed matrix followed by its complex conjugate according to what we saw in the section of Linear Algebra) such that:
		
		It is therefore in the language of quantum physics: Hermitian operators!!!
	\end{enumerate}
	Let us now see what are the eigenvectors and eigenvalues of the Pauli matrices because this result is very useful for the section of Relativistic Quantum Physics and of Quantum Computing!

	Let us recall that when a transformation (application of a matrix) act on a vector, it changes the direction of the vector except for specific matrices that have eigenvalues. In this case, the direction is conserved but not their length. This property is exploited in quantum mechanics (and not only as we will see in many other sections of this book).
	
	Let us determine in a first time, the associated eigenvectors and eigenvalues (\SeeChapter{see section Linear Algebra page \pageref{eigenvector}}) using the most common method:
	
	The eigenvalues equation (\SeeChapter{see section Linear Algebra page \pageref{eigenvalue equations}}) is thus written:
	
	Which gives us as characteristic equation:
	
	hence the eigenvalues $\lambda=\pm 1$. Which gives us the possibility to determine the eigenvectors as following:
	
	Therefore for $\lambda=1$:
	
	This impose that $y=x$. The eigenvector is therefore:
	
	whatever is the value of $x$.
	
	Conclusion: The proper direction of the vector is conserved but not its length (norm) because it depends on the value of $x$.
	
	For $\lambda=-1$:
	
	This impose that $y=-x$ and therefore that the eigenvector is  equal to:
	
	The previous eigenvectors written with the Dirac formalism (\SeeChapter{see section  Relativistic Quantum Physics page \pageref{dirac formalism}}) give for $\lambda=1$:
	
	with a norm ($1$ since we formalize to the unit):
	
	\begin{tcolorbox}[title=Remark,colframe=black,arc=10pt]
	In the formalism of Dirac $\langle v |$ is the is named a "Bra" and $|v \rangle$ a "Ket".
	\end{tcolorbox}
	This being only valid for components that are real numbers. The normalized eigenvector has therefore for expression:
	
	For $\lambda=-1$, we have:
	
	and:
	
	and the normalized eigenvector has thus for expression:
	
	Let us now determine the eigenvectors and eigenvalues associated to $\sigma_y$ by following the same procedure:
	
	So we have for the eigenvalues:
	
	The eigenvectors are determined as following:
	
	and therefore for $\lambda=1$:
	
	The eigenvector is therefore:
	and therefore for $\lambda=1$:
	
	The associated norm:
	
	The normalized vector is therefore expressed as:
	
	Let us now determine the eigenvectors and eigenvalues associated with $\sigma_z$ by doing the same again.
	
	We have therefore:
	
	The eigenvectors are then for $\lambda=1$:
	
	Which makes problem to us for say anything ... the only possibility is to choose $y=0$ and therefore:
	
	and the associated norm:
	
	The normalized vector has then for expression:
	
	and for $\lambda=-1$ we will have the same choice to do by choosing this time $x=0$ and therefore:
	
	hence the associated norm:
	
	The normalized eigenvector has then for expression:
	
	Therefore the normalized eigenvectors of $\sigma_z$ are on the directions of the Cartesian coordinate axes. It is for this particular reason that the eigenvectors of $\sigma_z$ are denoted in quantum computing by:
	
	and the reader should also know that we write also:
	
	
	
	\begin{flushright}
	\begin{tabular}{l c}
	\circled{100} & \pbox{20cm}{\score{4}{5} \\ {\tiny 18 votes,  84.44\%}} 
	\end{tabular} 
	\end{flushright}
	
		
 \chapter{Analyse}

	\textit{\textbf{The analysis is the rigorous formulation of calculus.}}(Wikipedia)
	\minitoc
	\pagebreak
	\input{Chapter_Analysis.tex}
		
\chapter{Géométrie}

	\textit{\textbf{Geometry is the mathematical discipline that focuses on the rigorous study of spaces and forms}}. (Larousse)
	\minitoc
	\pagebreak
	\input{Chapter_Geometry.tex}
	
   \chapter{Mécanique}

	\textit{\textbf{Mechanics is the branch of physics that relates to the study of forces and their actions in abstract form}}. (Larousse)
	\minitoc
	\input{Chapter_Mechanics.tex}
	
\chapter{Éléctromagnétisme}

	\textit{\textbf{Electrodynamic is the field of physics that study the dynamic action of electric currents and the propagation of electromagnetic waves.}}
	\minitoc
	\pagebreak
	\input{Chapter_Electromagnetism.tex}

\chapter{Atomistiques}

	\textit{\textbf{The atomic physics is the part of physics that deals with quantified energy states of corpuscular and wave particles and of exchange of energies within the atom}}. (Larousse)
	\minitoc
	\pagebreak
	\input{Chapter_Atomistic.tex}
	

\chapter{Cosmologie}

	\textit{\textbf{Cosmology is the science that studies the structure, evolution and the general laws of the universe as a whole}}. (Larousse)
	\minitoc
	\pagebreak
	\input{Chapter_Cosmology.tex}

	
\chapter{Chimie}

	\textit{\textbf{Chemistry is the science that studies the nature and properties of simple substances, the molecular action of these bodies on each other and combinations due to this action.}}(Larousse)
	\minitoc
	\pagebreak 
	\input{Chapter_Chemistry.tex}
	
	
\chapter{Informatique Théorique}
	\label{theoretical computing}
	\textit{\textbf{The theoretical computer science is the branch of science that deals with the development of algorithms and theoretical tools that can be apply to IT to solve formal problems related to the simulation of physical phenomena or data treatments and the exchange of information.}}(Larousse)
	\minitoc
	\pagebreak 
	\input{Chapter_Computing.tex}
	
	
\chapter{Sciences Sociales}

	\textit{\textbf{Social mathematics are the analysis and formal modeling tools of the behavior and management of a population and its trade in goods in all its activities, in interaction with its environment.}} (Sciences.ch)
	\minitoc
	\pagebreak
	\input{Chapter_SocialSciences.tex}
	
		
\chapter{Ingénierie}

	\textit{\textbf{Engineering is the set of practices consisting to apply the results of the exact sciences and basic research to practical industrial or daily problems.}} (Larousse)
	\minitoc
	\pagebreak
	\input{Chapter_Engineering.tex}

	
	
 	\chapter{Épilogue}
	Vous avez parcouru une long chemin avec nous à travers ce livre. Nous avons maintenant atteint l'épilogue, où par tradition, le rédacteur principal est autorisé à donner ses opinions personnelles.

	En effet, je veux vous laisser avec certaines de mes pensées sur la théorie par rapport à la pratique, le business et l'ingénierie, la recherche en mathématiques appliquées, et ce que j'espère vous avez appris après avoir lu ce livre.

	Par nature, l'ingénierie académique est très étroitement liée à sa pratique. La théorie et la pratique sont régies par les mêmes idées. En tant qu'enseignant universitaire depuis 2001 dans des sociétés du Fortune 500 et de nombreuses PME, je suis fier d'affirmer que la majorité des idées d'ingénierie ont été inventées ou développées dans le milieu universitaire avant d'être mises en pratique.
	
	Mais la recherche en ingénierie ne s'adresse pas seulement aux universitaires en herbe: comme je le sais très bien, les consultants en gestion et les consultants en économie sont essentiellement des chercheurs, même si la plupart d'entre eux ont un niveau analytique très bas. Des entreprises comme McKinsey, Ernst \& Young, KPMG et Accenture peuvent avoir des audiences, des vitesses d'analyse, des systèmes gestion d'équipe et des processus de publication et d'évaluation différents, mais ils sont souvent confrontés à des problèmes similaires à ceux des universitaires et à les résoudre avec les mêmes méthodes à la petite différence que le niveau de technicité est de savoir est significativement plus bas en général par rapport à ces derniers.

	Il y a aussi beaucoup de complémentarité: de nombreux professeurs travaillent régulièrement avec de grands cabinets de conseil aux Etats-Unis ou dans des sociétés d'investissement, et certains ont même carrément abandonné le monde universitaire pour multiplier leur rémunération (ou leur motivation...).

	Parce que l'ingénierie est par nature une discipline appliquée, après avoir lu ce livre, vous ne devriez pas avoir besoin d'autre chose pour comprendre la recherche en ingénierie aujourd'hui. C'est-à-dire que dans l'idéal, vous devriez être maintenant être tout à fait capable de lire les publications scientifiques à la pointe dans de nombreuses domaines et ce sans trop de difficultés.
	
	Alors, comprenons-nous vraiment l'ingénierie? Certainement pas complètement. Nous avons vu à travers cela que la finance est autant un art qu'une science. Compte tenu de nos déficiences, étant donné que toutes nos méthodes ont leurs erreurs, que devons-nous faire? Mon meilleur conseil est d'utiliser le bon sens (en évitant soigneusement les biais), d'employer un certain nombre de techniques différentes pour arriver à une gamme de réponses possibles, et ensuite de porter un jugement sur l'estimation la plus raisonnable à la lumière de différents modèles obtenus et validés par les pairs.
	
	Notre livre a couvert les principes de base de l'ingénierie dans une certaine profondeur et largeur. Vous devriez être très bien préparé maintenant pour les prochaines étapes de votre formation continue en ingénierie.
	
	J'ai pris beaucoup de plaisir co-écrire ce livre autant que mes supports de cours de formations en entreprise, et ce pour la même raison: Cela a été comme résoudre un puzzle intrigant que peut-être personne d'autre n'a compris et expliquer de la même manière. Bien sûr, la différence majeure étant que sa co-rédaction m'a pris pas mal de temps (15 ans en incluant pas les 5 années pour la traduction de la version anglophone!).
	
	Mais cela en aura valu la peine si vous en avez tiré des leçons! Si vous avez étudié le livre, vous devriez maintenant savoir environ $99\%$ de ce que je connais de l'ingénierie. Fait intéressant, il y avait un certain nombre de sujets que je pensais avoir compris, mais en réalité ce n'était visiblement pas le cas - et c'était seulement mon devoir de vous les expliquer qui les a clarifiés pour moi aussi. Et cela m'amène à un point clé que je veux vous laisser - ne jamais avoir peur de poser des questions, d'admettre ses biais, même sur les premiers principes. Le faire n'est pas un signe de bêtise au contraire, c'est souvent un signe d'approfondissement de la conscience et de la compréhension.

	Je ne me fais pas d'illusions: vous ne vous souviendrez très probablement pas de tous les détails de ce livre au fil du temps - je sais personnellement en tout cas que je ne le peux pas. Mais plus que les détails, j'espère que je vous aurai laissé une appréciation pour les grandes idées, un arsenal d'outils techniques et cognitifs, des méthodes pour aborder de nouveaux problèmes et des nouvelles perspectives. Vous pouvez maintenant penser comme un ingénieur ou mieux encore: comme un scientifique!
	\begin{figure}[H]
		\centering
		\includegraphics[scale=0.1]{img/knowledge_is_power.jpg}	
	\end{figure}
	
	\chapter{Biographies}
	Pour être informé des lauréats des prix Nobel (physique, chimie, économie), médaille de Fields... cliquez sur le lien suivant {\href{http://www.nobelprize.org/}{{\color{blue}Prix Nobel}}} ou sur celui-ci {\href{http://www.fields.utoronto.ca/aboutus/jcfields/fields_medal.html}{{\color{blue}Médaille de Fiels}}}.

Cette section présente une poignée d'individus qui postulent une étrange renommée. Selon les règles de l'histoire que l'on enseigne à l'école primaire, ils n'existent pas, ils n'ont commandé aucune armée, ils n'ont envoyé personne à la mort, ils n'ont dirigé aucun empire et ils n'ont eu qu'une part minime dans les grandes décisions historiques. Certains ont acquis quelque célébrité, mais aucun ne fut jamais un héros national. Pourtant, leur oeuvre a davantage influencé le cours de l'Histoire que bien des actes accomplis par des hommes d'État auréolés d'une gloire très supérieure. Leur oeuvre a aussi produit plus de bouleversements que le va-et-vient des armées en bataille par-dessus les frontières, elle a fait plus pour le bonheur ou le malheur que les édits des rois et des assemblées, car leur oeuvre, est d'avoir façonné l'esprit de l'homme!

Qui propage ses idées, manie un pouvoir bien supérieur à celui de l'épée ou du sceptre: aussi ont-ils façonné et dirigé le Monde. Pour la plupart, ils n'ont pas levé le moindre petit doigt pour agir physiquement; ils ont travaillé essentiellement en intellectuels, dans le silence et l'oubli, sans se soucier outre mesure du monde environnant. Mais, dans leur sillage, des empires se sont écroulés, des régimes politiques se sont soit renforcés, soit érodés, les classes se sont dressées les unes contre les autres, ainsi que les nations. Qui sont ces individus ?: des savants, économistes, chimistes, biologistes, mathématiciens, physiciens, informaticiens, ingénieurs,...

Les biographies ci-dessous des scientifiques les plus connus à travers le Monde et cités dans les différents chapitres du site sont triées par ordre alphabétique et presque tous les textes sont de simples copier/coller simplifiés de {\href{http://www.wikipedia.fr}{{\color{blue}Wikipédia}}}. Si vous souhaitez que nous rajoutions une entrée, il vous suffit de nous envoyer par courriel le nom et prénom de la personne concernée et la raison pour laquelle vous aimeriez la voir figurer dans la liste ci-dessous. Nous étudierons ensuite la proposition et prendrons la décision qui s'impose.

\textbf{Nous rendons aussi hommage aux millions de scientifiques\footnote{Il semblerait qu'en ce début de 3ème millénaire qu'il y ait pas loin de 7 millions de scientifiques de haut niveau à travers le Monde selon les statistiques de l'UNESCO.} (savants), ingénieurs, philosophes, artisans, mécènes, artistes, amateurs éclairés connus ou anonymes dont la collaboration a permis à travers les millénaires l'évolution de la science et de la condition humaine!}

Les tailles des biographies ci-dessous ne sont pas proportionnelles au nombre d'articles publiés ou découvertes effectuées, mais à la quantité d'informations trouvées sur ces personnages sur l'Internet ou dans la littérature. La liste n'est aussi pas exhaustive, mais son objectif est de rendre hommage et de se remémorer les individus qui ont fait des sciences pures et exactes ce qu'elles sont aujourd'hui et qui ont consacré une partie ou l'entier de leur vie à la science: l'art le plus contraint!

Attention! En physique (aussi bien qu'en mathématique) une théorie, une équation, une constante ou autre porte rarement le nom de son vrai inventeur. Ce fait est largement connu chez les scientifiques et est souvent source de moqueries de la communauté...

	\begin{figure}[H]
		\centering
		\includegraphics[scale=0.5]{img/shoulders_of_giants.jpg}	
	\end{figure}

\begin{center}
\hyperref[sec:A]{A} \hyperref[sec:B]{B} \hyperref[sec:C]{C} \hyperref[sec:D]{D} \hyperref[sec:E]{E} \hyperref[sec:F]{F} \hyperref[sec:G]{G} \hyperref[sec:H]{H} \hyperref[sec:I]{I} \hyperref[sec:J]{J} \hyperref[sec:K]{K} \hyperref[sec:L]{L} \hyperref[sec:M]{M} \hyperref[sec:N]{N} \hyperref[sec:O]{O} \hyperref[sec:P]{P} \hyperref[sec:Q]{Q} \hyperref[sec:R]{R} \hyperref[sec:S]{S} \hyperref[sec:T]{T} \hyperref[sec:U]{U} \hyperref[sec:V]{V} \hyperref[sec:W]{W} \hyperref[sec:X]{X} \hyperref[sec:Y]{Y} \hyperref[sec:Z]{Z}
\end{center}

\phantomsection
\addcontentsline{toc}{section}{A}
\label{sec:A}
		
\pichskip{15pt}% Horizontal gap between picture and text
\parpic[l][t]{
  \begin{minipage}{40mm}
    \fbox{\includegraphics[width=110px,height=140px]{img/medaillons/al.eps}}
  \end{minipage}
}		
\textbf{Al-Biruni, Muhammad Ibn Ahmad Abul-Rayhan} (973-1048) est un mathématicien, un astronome, un physicien, un érudit, un encyclopédiste, un philosophe, un astrologue, un voyageur, un historien, un pharmacologue et un précepteur, originaire de la Perse, qui contribua grandement aux domaines des mathématiques, philosophie, médecine et des sciences. Il est connu pour sa théorie sur la rotation de la Terre autour de son axe et autour du Soleil, et ceci bien avant Copernic. Il s'attacha notamment à calculer la marche du Soleil (apogée), corrigea certaines données de Ptolémée. Excellent mathématicien, Al-Biruni développa de nouvelles équations inconnues de ses prédécesseurs. Il calcula également le méridien local et les coordonnées des localités. Mais le tableau ne serait pas complet si l'on oubliait de mentionner que six siècles avant Galilée, Al Biruni mettait déjà en avant une Terre qui tournait autour de son axe. Avec l'aide d'un astrolabe, de la mer et d'une montagne avoisinante, il calcula la circonférence de la Terre en résolvant une équation complexe pour son époque. Le principal d'apport d'Al-Biruni aux mathématiques résiderait dans ses travaux en trigonométrie (calculs des valeurs des fonctions trigonométriques qui n'étaient pas encore définies en tant que tel à l'époque)..

\pichskip{15pt}% Horizontal gap between picture and text
\parpic[l][t]{%
  \begin{minipage}{40mm}
    \fbox{\includegraphics[width=110px,height=140px]{img/medaillons/alembert.eps}}
  \end{minipage}
}
\textbf{Alembert, Jean le Rond} (1717-1783), enfant naturel d'un commissaire d'artillerie, abandonné sur les marches de la chapelle parisienne de Saint-Jean-Le-Rond, le futur grand philosophe, mathématicien et physicien est recueilli par un vitrier qui recevra secrètement une pension pour subvenir à l'éducation du jeune garçon qui étudiera brillamment le droit, la médecine et la mathématique. Suite à la publication de divers mémoires (sur le calcul intégral, sur la réfraction des corps solides), d'Alembert entre à l'Académie des sciences (1741). On lui doit le célèbre principe de la quantité de mouvement, dit "principe de d'Alembert" dans son \textit{Traité de dynamique} (1743). En astronomie, il est l'auteur (1749) d'un traité sur la précession des équinoxes qu'il explique au moyen de la théorie de la gravitation universelle de Newton et d'une solution partielle au problème des trois corps. D'Alembert établit aussi une théorie mathématique des cordes vibrantes en étudiant la nature composite du son (harmoniques).

\pichskip{15pt}% Horizontal gap between picture and text
\parpic[l][t]{%
  \begin{minipage}{40mm}
    \fbox{\includegraphics[width=110px,height=140px]{img/medaillons/ampere.eps}}
  \end{minipage}
}
\textbf{Ampère, André Marie} (1775-1836) à 18 ans, il connaît la majeure partie des oeuvres mathématiques de son temps. Mathématicien de premier ordre, il montre comment l'on doit utiliser cette science, qu'il considérait comme une branche de la philosophie, à l'étude des découvertes des faits physiques pour en donner une relation définitive. En quelques semaines, Ampère établit les bases de toute une science à laquelle il donne le nom d'électromagnétisme. Il cherche à comprendre le magnétisme des aimants et en tire une hypothèse de "courants particulaires" (orbites électroniques et orientation du spin aujourd'hui). Il établit également l'égalité du nombre de molécules dans des volumes égaux de Gaz de natures différentes, mais mesurés dans des conditions identiques de température et de pression (observation expérimentale de Gay-Lussac).

\pichskip{15pt}% Horizontal gap between picture and text
\parpic[l][t]{%
  \begin{minipage}{40mm}
    \fbox{\includegraphics[width=110px,height=140px]{img/medaillons/archimede.eps}}
  \end{minipage}
}
\textbf{Archimedes of Syracuse} (287-212 BC.), mathématicien et ingénieur grec célèbre à la fois comme mécanicien théoricien et comme constructeur de machines. Archimède de Syracuse eut une production mathématique exceptionnelle, dont une partie nous est parvenue dans des traités comme \textit{Sur la sphère et le cylindre}; la \textit{Mesure du cercle}; la \textit{Quadrature de la parabole};\textit{ Des spirales};\textit{ Des conoïdes et sphéroïdes}; la \textit{Méthode}, \textit{Des corps flottants}... C'est à partir de ses travaux mécaniques que les principales anecdotes le mettant en scène, comme celle du levier ou du bain, vont se constituer. La célèbre maxime: «Donnez-moi une place où me tenir et je mettrai la Terre en mouvement» est un écho populaire de la contribution archimédienne à la statique, exposée dans le traité des \textit{Équilibres}. Archimède démontre la loi du levier, introduit la notion fondamentale de centre de gravité, et détermine ces barycentres pour les principales figures géométriques planes. Il en est de même pour l'anecdote d'Archimède, jaillissant nu de son bain, en criant «Eurêka», parce qu'il venait, dit-on, de trouver le moyen de résoudre le problème que lui avait posé le roi Hiéron. En fait, le récit est une mise en scène spectaculaire de la découverte du principe fondamental de l'hydrostatique (communément appelé depuis "principe d'Archimède"). En géométrie, l'oeuvre d'Archimède développe celle d'Eudoxe de Cnide telle que nous la connaissons par le livre XII des \textit{Éléments}  d'Euclide: il s'agit de comparer les mesures des figures planes et solides, en particulier des figures curvilignes. Ainsi, Archimède démontre que le volume du cylindre circonscrit à une sphère est égal à une fois et demi le volume de celle-ci et que la surface latérale du cylindre est égale à celle de la sphère ou quatre fois la surface d'un grand cercle. Donc, si l'on sait calculer la surface du cercle, on connaîtra celle de la sphère, du cylindre, son volume et celui de la sphère, etc. Son résultat le plus célèbre et le plus simple concerne le cercle. Archimède ramène sa quadrature à un autre problème: la rectification de sa circonférence, c'est-à-dire «trouver une ligne droite égale qui lui soit égale», problème qu'il résout à l'aide d'une courbe géométrique que l'on appelle désormais "spirale d'Archimède". En outre, il calcule des valeurs approchées du rapport circonférence/diamètre (ce que nous appelons le nombre "Pi" et noté $\pi$).

\pichskip{15pt}% Horizontal gap between picture and text
\parpic[l][t]{%
  \begin{minipage}{40mm}
    \fbox{\includegraphics[width=110px,height=140px]{img/medaillons/avogadro.eps}}
  \end{minipage}
}
\textbf{Avogadro, Amedeo} (1776-1856), fils d'un magistrat de Turin, Amadeo Avogadro commence par suivre la voie paternelle. Il passe une licence de droit en 1795 et s'inscrit au barreau de sa ville natale. Mais son goût pour la physique et la mathématique, auxquelles il n'a cessé de s'intéresser en solitaire, le pousse à entamer sur le tard des études scientifiques. En 1809, il fait une communication à l'Académie royale de Turin ; le succès qu'il remporte grâce à elle lui permet d'obtenir un poste de professeur au Collège royal de Verceil. En 1820, l'Université de Turin crée pour lui une chaire de physique qu'il gardera jusqu'à la fin de sa vie. C'est en étudiant les lois régissant la compression et la dilatation des gaz qu'Avogadro énonce, en 1811 l'hypothèse restée célèbre sous le nom de "loi d'Avogadro". Reposant sur la théorie atomique de Dalton et la loi de Gay-Lussac sur les rapports volumiques, la théorie d'Avogadro indique que deux volumes égaux de gaz différents, dans les mêmes conditions de température et de pression, contiennent le même nombre de molécules. Sous son apparente simplicité, cette loi comporte des implications importantes ; grâce à elle, il devient possible de déterminer la masse molaire d'un gaz à partir de celle d'un autre. Mais les chimistes de l'époque, plus intéressés par les expériences, boudent quelque peu les études théoriques d'Avogadro qui ne seront d'ailleurs reconnues que 50 ans plus tard. Le nom d'Avogadro reste également lié à celui du "nombre d'Avogadro" indiquant le nombre de molécules contenues dans une seule mole.

\phantomsection
\addcontentsline{toc}{section}{B}	
\label{sec:B}

\pichskip{15pt}% Horizontal gap between picture and text
\parpic[l][t]{%
  \begin{minipage}{40mm}
    \fbox{\includegraphics[width=110px,height=140px]{img/medaillons/bachelier.jpg}}
  \end{minipage}
}
\textbf{Bachelier, Louis} (1870-1946) est né au Havre dans une famille de négociants. Il apparaît à sa majorité sur les listes électorales du Havre en 1892 comme représentant de commerce à la même adresse professionnelle que son père. Après avoir effectué son service militaire, à l'âge de 22 ans, il reprend ses études à la faculté des sciences de Paris. Elles sont couronnées par une licence ès sciences en 1895 (mention passable) et par la soutenance en 1900 de sa non moins fameuse et méconnue thèse de doctorat en mathématiques. Bien que cette thèse soit considérée aujourd'hui comme un travail précurseur en théorie des probabilités et en théorie financière, elle ne vaut à l'époque à son auteur qu'une mention honorable. De 1913 à 1914 Bachelier dispensa un cours libre de théorie des probabilités appliquées à la mécanique, la balistique et la biométrie. Il fut également chargé de conférences supplémentaires sur la mathématique générale de 1913 à 1914.  Ce n'est qu'après la guerre de 1914-1918 qu'il obtient un premier poste de chargé de cours à la faculté des sciences de Besançon. Après divers remplacements à Dijon puis à Rennes, il revient à Besançon en 1927 comme professeur titulaire de la chaire de calcul différentiel et intégral, poste qu'il occupe jusqu'à sa retraite en 1937. Louis Bachelier a, parmi ses nombreux travaux, été le premier a avoir introduit la continuité dans les problèmes de probabilité en prenant le temps comme variable. En particulier, il a élaboré une théorie mathématique du mouvement brownien 5 ans avant Albert Einstein. Il est également bien avant Norbert Wiener, le premier à avoir défini la fonction du mouvement brownien et donné un grand nombre de ses propriétés.

\parpic[l][t]{%
  \begin{minipage}{40mm}
    \fbox{\includegraphics[width=110px,height=140px]{img/medaillons/balmer.eps}}
  \end{minipage}
}
\textbf{Balmer, Johann Jakob} (1825-1898) était un mathématicien et physicien suisse né à Lausen (Suisse) et mort à Bâle. Pendant sa scolarité, il a excellé en mathématiques et a donc décidé de se concentrer sur ce domaine lorsqu'il a fréquenté l'université. Il a étudié à l'Université de Karlsruhe et à l'Université de Berlin, puis a terminé son doctorat de l'Université de Bâle en 1849 avec une thèse sur la cycloïde. Balmer a ensuite passé toute sa vie à Bâle, où il a enseigné dans une école pour filles. Il a également enseigné à l'Université de Bâle. En dépit d'être un mathématicien, il personne ne souvient d'une quelconque recherche qu'il aurait fait dans ce domaine; au contraire, sa contribution majeure (faite à l'âge de soixante ans, en 1885) était une formule empirique pour les raies spectrales visibles de l'atome d'hydrogène, dont l'étude qu'il mena fit suite à la suggestion d'Eduard Hagenbach, également de Bâle. Une explication complète de l'explication de sa formule dû attendre jusqu'à la présentation du modèle Bohr de l'atome par Niels Bohr en 1913.

\parpic[l][t]{%
  \begin{minipage}{40mm}
    \fbox{\includegraphics[width=110px,height=140px]{img/medaillons/banach.eps}}
  \end{minipage}
}
\textbf{Banach, Stefan} (1892-1945) était un mathématicien polonais qui a posé les bases de l'analyse fonctionnelle. Né à Cracovie en 1892, en Autriche-Hongrie (actuellement ville polonaise). Banach fit ses études secondaires à Cracovie; il se révéla particulièrement brillant en mathématiques et en sciences naturelles, mais son désintérêt pour les autres matières l'empêcha d'obtenir les meilleures mentions.  La vie (au moins mathématique) de Banach va basculer au printemps 1916, quand il rencontre Steinhaus à Cracovie. Avec Otto Nikodym, ils décident de fonder une société mathématique. La recherche mathématique de Banach commence là. Son premier article est cosigné avec Steinhaus. Steinhaus lui avait parlé d'une propriété qu'il ne parvenait pas à démontrer, et après quelques jours de réflexion, Banach exhiba un contre-exemple. Il est difficile de dire ce qu'il serait advenu de l'activité mathématique de Banach sans la rencontre avec Steinhaus, mais toujours est-il qu'il entama à compter de celle-ci une recherche intense et fructueuse. Banach retourne à Lvov en 1920 où un poste d'assistant lui est proposé. Il soutient sa thèse en 1922, et c'est dans cette thèse qu'apparaît pour la première fois la notion d'espace de Banach, qu'y sont démontrés les théorèmes fondamentaux sur ces objets, où on y évoque la topologie faible... Bref, cette thèse marque la naissance de l'analyse fonctionnelle. En 1929, il fonde avec Steinhaus la revue mathématique "Studia Math", consacrée au développement de l'analyse fonctionnelle, et en 1939 il est élu président de la société mathématique de Pologne. En 1945, peu avant la fin de la Seconde Guerre Mondiale, il décède d'un long cancer. De nombreux théorèmes sont associés au nom de Banach, qu'il les ait démontrés lui-même, ou qu'ils fassent référence à ces idées. Citons entre autres: le théorème de Hahn-Banach de prolongement des formes linéaires continues, le théorème de Banach-Steinhaus, de Banach-Alaoglu, le théorème du point fixe de Banach, ainsi que le paradoxe de Banach-Tarski.

\parpic[l][t]{%
  \begin{minipage}{40mm}
    \fbox{\includegraphics[width=110px,height=140px]{img/medaillons/bell.eps}}
  \end{minipage}
}
\textbf{Bell, John} (1928-1990) fut dès la plus petite enfance attiré par les livres traitant des sciences. À cause de problèmes financiers familiaux, il ne put poursuivre immédiatement des études académiques. Il travailla donc pendant une année en tant que technicien au département de physique de l'Université de Queen's à Belfast avant de devenir étudiant en 1945 dans ce même département. Il sortit premier de sa promotion en mathématiques-physique. Bell trouva dans les années 1960 une nouvelle inspiration dans les bases de la théorie quantique, une matière supposée épuisée par les résultats de la discussion de Bohr-Einstein 30 ans plus tôt, et ignorée par pratiquement tous ceux qui ont employé la théorie quantique entre-temps. Effectivement, Bell était intrigué par les incertitudes quantiques de Heisenberg et voulait creuser le sujet en montrant que la discussion de tels concepts comme le "réalisme", le "déterminisme" et la "localité" pouvaient être affiliés dans un rapport mathématique rigoureux: "les inégalités de Bell" vérifiables expérimentalement. Bell poussa très loin les doutes qu'il avait sur les principes d'incertitudes au point qu'il en irrita même son professeur (Sloane) qui lui fit remarquer que maintenant il allait un peu trop loin! Bell attendit son travail de thèse pour développer ses idées. Malheureusement, à nouveau à cause de problèmes financiers, il dut repousser ses recherches à plus tard et joindre le centre anglais de recherche atomique à Harwell. Pendant sa carrière, il épousa une femme (Mary Bell) qui l'aida dans le développement de ses travaux sur les principes fondamentaux de la théorique quantique. C'est, en 1951, avec Rudolf Peierls que Bell développa sa célèbre théorie C.P.T. (Charge, Parité, Temps). Malheureusement, pour Bell, Gerhard Lüders et Wolfgang Pauli arrivèrent au même résultat dans la même période et c'est à eux que furent attribué les crédits de la découverte. Les développements théoriques de Bell sont à l'origine de la cryptographie et de la théorique de l'information quantique. L'attention à la théorie quantique de l'information a énormément augmenté au cours des dernières années, et le sujet semble sûr d'être l'un des secteurs scientifiques dont la croissance sera la plus importante au 21ème siècle. Un autre travail de première importance de Bell en 1969 fut la participation au développement de "l'anomalie A.B.J." (Adler-Bell-Jackiw) dans la théorie quantique des champs. Ces trois physiciens montrèrent que le modèle algébrique standard contentait une erreur. Effectivement, la quantification du modèle des champs brise une symétrie. Bell fut nommé pour le prix Nobel, qu'il aurait certainement obtenu s'il n'était pas décédé d'une attaque cérébrale en 1990.

\parpic[l][t]{%
  \begin{minipage}{40mm}
    \fbox{\includegraphics[width=110px,height=140px]{img/medaillons/berners_lee_timothy_john.jpg}}
  \end{minipage}
}
\textbf{Berners-Lee, Timothy John} (1955-) est un physicien anglais mieux connu comme l'inventeur du World Wide Web. Il est le directeur du World Wide Web Consortium (W3C), qui supervise le développement continu du Web. Il est également le fondateur de la World Wide Web Foundation, chercheur principal et titulaire de la chaire des fondateurs du Laboratoire d'informatique et d'intelligence artificielle du M.I.T. Il est directeur de la Web Science Research Initiative (WSRI) et membre du comité consultatif du M.I.T. Center for Collective Intelligence. Il a travaillé comme entrepreneur indépendant au CERN de juin à décembre 1980. Pendant son séjour à Genève, il a proposé un projet basé sur le concept d'hypertexte, pour faciliter le partage et la mise à jour des informations entre les chercheurs. L'adresse \url{info.cern.ch} était l'adresse du tout premier site web et serveur web au monde, fonctionnant sur un ordinateur NeXT au CERN. La première adresse de la page Web était \url{http://info.cern.ch/hypertext/WWW/TheProject.htm}.

\parpic[l][t]{%
  \begin{minipage}{40mm}
    \fbox{\includegraphics[width=110px,height=140px]{img/medaillons/bernoulli_daniel.eps}}
  \end{minipage}
}
\textbf{Bernoulli, Daniel} (1700-1782) est un savant suisse qui découvrit les principes de base du comportement d'un fluide (c'est le fils de Jean Bernoulli et le neveu de Jacques Bernoulli.). Il cultiva à la fois les sciences mathématiques et les sciences naturelles, enseigna les mathématiques, l'anatomie, la botanique et la physique. Ami de Leonhard Euler, il travailla avec lui dans plusieurs domaines des mathématiques et de la physique (il partagea avec lui dix fois le prix annuel de l'Académie des sciences de Paris à un point qu'il s'en fit une sorte de revenu...). Les différents problèmes qu'il tente de résoudre (théorie de l'élasticité, mécanisme des marées) le conduisent à s'intéresser et développer des outils mathématiques tels que les équations différentielles ou les séries. Il collabore également avec Jean le Rond d'Alembert dans l'étude des cordes vibrantes. Il étudia l'écoulement des fluides (1738) et formula le principe (le fameux théorème de Bernoulli) selon lequel la pression exercée par un fluide est inversement proportionnelle à sa vitesse d'écoulement. Il utilisa des concepts atomistes pour ébaucher la première théorie cinétique des gaz, en exprimant leur comportement en termes de probabilités sous des conditions particulières de pression et de température. On peut le considérer comme l'un des fondateurs de l'hydrodynamique.

\parpic[l][t]{%
  \begin{minipage}{40mm}
    \fbox{\includegraphics[width=110px,height=140px]{img/medaillons/bernoulli_jacques.eps}}
  \end{minipage}
}
\textbf{Bernoulli, Jacques} (1654-1705) était un mathématicien et physicien suisse, frère de Jean Bernoulli et oncle de Daniel Bernoulli et Nicolas Bernoulli. Né à Bâle en 1654, il rencontre Robert Boyle et Robert Hooke lors d'un voyage en Angleterre en 1676. Après cela, il se consacre à la physique et aux mathématiques. Il enseigne à l'Université de Bâle à partir de 1682, devenant professeur de mathématiques en 1687. Il mérita par ses travaux et ses découvertes d'être nommé associé de l'Académie des sciences de Paris (1699) et de celle de Berlin (1701). Sa correspondance avec Gottfried Wilhelm Leibniz le conduit à étudier le calcul infinitésimal en collaboration avec son frère Jean. Il fut un des premiers à comprendre et à appliquer le calcul différentiel et intégral, proposé par Leibniz, découvrit les propriétés des nombres dits depuis "nombres de Bernoulli" et donna la solution de problèmes regardés jusque-là comme insolubles. Il pose les principes du calcul des probabilités et introduit les nombres de Bernoulli dans un ouvrage publié après sa mort en 1713.

\parpic[l][t]{%
  \begin{minipage}{40mm}
    \fbox{\includegraphics[width=110px,height=140px]{img/medaillons/bernoulli_jean.eps}}
  \end{minipage}
}
\textbf{Bernoulli, Jean} (1667-1748) était un mathématicien et physicien suisse. Frère de Jacques Bernoulli et père de Daniel et Nicolas Bernoulli, il professa les mathématiques à Groningue (1695), puis à Bâle, après la mort de Jacques Bernoulli (1705), et devint associé des Académies de Paris, de Londres, de Berlin et de Saint-Pétersbourg. Formé par son frère Jacques Bernoulli, il avait longtemps travaillé de concert avec lui à développer les conséquences du nouveau calcul infinitésimal inventé par Gottfried Leibniz ; mais il s'établit ensuite entre eux, à l'occasion de la résolution de quelques problèmes, une rivalité qui dégénéra en inimitié. Il a aussi contribué dans beaucoup de secteurs aux mathématiques y compris le problème d'une particule se déplaçant dans un champ de gravité. Il trouva l'équation de la chaînette en 1690 et développa le calcul exponentiel en 1691. Il eut aussi la gloire de former Leonhard Euler. Il vint à Paris en 1690, et se lia avec les savants les plus distingués, particulièrement avec de L'Hôpital. Jean Bernoulli est devenu membre de la Royal Society en 1712.

\parpic[l][t]{%
  \begin{minipage}{40mm}
    \fbox{\includegraphics[width=110px,height=140px]{img/medaillons/bessel.eps}}
  \end{minipage}
}
\textbf{Bessel, Friedrich} (1784-1846) Né à Minden en Westphalie, Bessel commença à travailler très jeune comme commis. Attiré par la navigation maritime, il s'intéressa aux observations nautiques, construisant lui-même son sextant et étudiant l'astronomie à ses heures de liberté. Il calcula la trajectoire de la comète de Halley, résultat qui fut immédiatement publié et lui permit d'obtenir, en 1806, un emploi d'assistant à l'observatoire de Lilienthal. En 1810, il devint directeur du nouvel observatoire de Königsberg, tout en poursuivant des études mathématiques. Il dut enseigner la mathématique à ses étudiants en astronomie jusqu'en 1825 (date à laquelle Jacobi vint enseigner cette matière à Königsberg). Toute sa vie fut consacrée à l'astronomie (il écrivit plus de 350 articles) et, peu avant sa mort, il commença l'étude du mouvement d'Uranus, problème qui devait aboutir à la découverte de Neptune. En mathématiques, Bessel est connu pour avoir introduit les fonctions qui portent son nom, les utilisant pour la première fois, en 1817, lors de l'étude d'un problème de Kepler, et les employant plus complètement 7 ans plus tard pour étudier les perturbations planétaires.

\parpic[l][t]{%
  \begin{minipage}{40mm}
    \fbox{\includegraphics[width=110px,height=140px]{img/medaillons/biot.eps}}
  \end{minipage}
}
\textbf{Biot, Jean-Baptiste} (1774-1862) Né et décédé à Paris était un physicien, astronome et mathématicien. Jean-Baptiste fit des études secondaires (humanités) à Paris au collège Louis-le-Grand jusqu'en 1791. Il commence des études d'ingénieur à l'École des ponts et chaussées en janvier 1794, puis rejoint l'École centrale des travaux publics (future École polytechnique) à son ouverture en décembre 1794 au Palais Bourbon. Un an plus tard (1795) il rejoint à nouveau l'École des ponts et chaussées pour terminer sa formation d'ingénieur. C'est vers l'enseignement que Biot oriente sa carrière après ses études d'ingénieur. Il devient professeur de mathématiques à l'École centrale du département de l'Oise à Beauvais en 1797. Grâce à l'appui de Laplace, il est nommé en 1800, âgé de 26 ans, professeur de physique mathématique au Collège de France. Il est entre 1816 et 1826 chargé de la moitié du cours de physique pour l'acoustique, le magnétisme et l'optique, Gay-Lussac, titulaire de la chaire de physique, enseignant la chaleur, les gaz, l'hygrométrie, l'électricité et le galvanisme. Il formule avec Félix Savart, la loi de Biot-Savart, qui donne la valeur du champ magnétique produit en un point de l'espace par un courant électrique en fonction de la distance de ce point au conducteur.

\parpic[l][t]{%
  \begin{minipage}{40mm}
    \fbox{\includegraphics[width=110px,height=140px]{img/medaillons/bohr.eps}}
  \end{minipage}
}
\textbf{Bohr, Niels Henrik David} (1885-1962) était un physicien danois, prix Nobel en 1922, pour sa contribution à la physique nucléaire et à la compréhension de la structure atomique. La théorie de Bohr sur la structure atomique, pour laquelle il reçut le prix Nobel de physique en 1922, fut publiée entre 1913 et 1915. Son travail s'inspira du modèle nucléaire de l'atome de Rutherford, dans lequel l'atome est considéré comme un noyau compact entouré d'un essaim d'électrons. Le modèle pose en principe que l'atome n'émet de rayonnement électromagnétique que lorsqu'un électron se déplace d'un niveau quantique à un autre. Ce modèle contribua énormément aux développements ultérieurs de la physique atomique théorique.

\parpic[l][t]{%
  \begin{minipage}{40mm}
    \fbox{\includegraphics[width=110px,height=140px]{img/medaillons/boltzmann.eps}}
  \end{minipage}
}
\textbf{Boltzmann, Ludwig} (1844-1906) était un physicien autrichien qui contribua à établir les bases de la mécanique statistique. Ayant fait ses études à Vienne et à Oxford, il enseigna la physique dans différentes universités allemandes et autrichiennes pendant plus de 40 ans. Développant la théorie cinétique des gaz, notamment à partir des travaux de Maxwell, il établit que la seconde loi de la thermodynamique pouvait être obtenue sur la base de l'analyse statistique. Calculant le nombre de particules dotées d'une énergie donnée, il établit la statistique dite de Maxwell-Boltzmann. Il exprima l'entropie $S$ d'un système en fonction de la probabilité $W$ de son état (par l'intermédiaire de sa fameuse équation de transport à partir de laquelle démontra que l'entropie ne pouvait qu'augmenter au cours du temps... résultat qui était jusque là admis expérimentalement mais sans preuve théorique). Il put aussi établir de manière théorique la loi de Stefan relative au rayonnement d'un corps noir. Mais il lui fallut expliquer comment les principes mécaniques, où les phénomènes sont réversibles, pouvaient engendrer des lois thermodynamiques décrivant des phénomènes marqués par l'irréversibilité. Il avança l'idée que les évolutions irréversibles, quoiqu'elles ne soient que des possibilités parmi d'autres, sont si probables que ce sont pratiquement toujours elles qui se produisent.

\parpic[l][t]{%
  \begin{minipage}{40mm}
    \fbox{\includegraphics[width=110px,height=140px]{img/medaillons/boole.eps}}
  \end{minipage}
}
\textbf{Boole, George} (1815-1864) était un mathématicien et logicien anglais considéré comme le créateur de la logique symbolique. Né à Lincoln et fils d'un petit commerçant, il reçut ses premières leçons de mathématiques de son père, qui lui apprit aussi à fabriquer des instruments d'optique. En dehors des conseils de son père et de quelques années passées dans les écoles locales, Boole est un autodidacte. Quand les affaires de son père déclinèrent, il fut obligé de travailler pour aider sa famille et, dès 16 ans, il enseigna dans des écoles de village ; à 20 ans, il ouvrit sa propre école à Lincoln. Pendant ses loisirs, il étudiait la mathématique à l'Institut de mécanique, créé vers cette époque ; c'est là qu'il se familiarisa avec les Principia de Newton, la Mécanique céleste de Laplace et la Mécanique analytique de Lagrange et qu'il commença à résoudre des problèmes d'algèbre supérieure. Boole soumit au nouveau \textit{Cambridge Mathematical Journal} une série d'articles originaux dont le premier est \textit{Recherches sur la théorie des transformations analytiques} ; ces articles portaient sur les équations différentielles et sur les invariants par transformation linéaire. En 1844, il étudie les liens entre l'algèbre et le calcul infinitésimal dans un important mémoire publié dans les Transactions de la Royal Society, qui lui décerne une médaille cette même année pour sa contribution à l'analyse (c'est-à-dire l'utilisation de l'algèbre dans l'étude des infiniment petits et grands). Développant de nouvelles idées sur la méthode en logique et confiant dans le symbolisme qu'il avait élaboré à partir de ses recherches mathématiques, il publie, en 1847, un opuscule, \textit{Mathematical Analysis of Logic}, dans lequel il soutient que la logique doit être rattachée aux mathématiques et non à la philosophie. Bien qu'il n'eût aucun titre universitaire, Boole fut, sur la base de ses publications, nommé en 1849 professeur au Queen's College à Cork, en Irlande. Avec Boole, en 1847 et en 1854, commence l'algèbre de la logique, c'est-à-dire ce que l'on appelle de nos jours l'algèbre de Boole. Dans son ouvrage de 1854, Boole énonce complètement sa nouvelle méthode symbolique d'inférence logique, qui permet, étant donnés des propositions contenant un certain nombre de termes, d'en tirer, par traitement symbolique des prémisses, des conclusions qui étaient logiquement contenues dans les prémisses. Il rechercha aussi une méthode générale en calcul des probabilités, qui aurait permis, à partir des probabilités connues d'un système d'événements donnés, de déterminer la probabilité de tout autre événement relié logiquement aux événements donnés.

\parpic[l][t]{%
  \begin{minipage}{40mm}
    \fbox{\includegraphics[width=110px,height=140px]{img/medaillons/borel.eps}}
  \end{minipage}
}
\textbf{Borel, Emile} (1871-1956) Reçu major à l'X et à ULM, il choisit cette dernière et se consacre aux mathématiques. Il fonda l'institut Henri Poincaré et fut élu député de l'Aveyron et maire de Saint-Afrique. Il étudie les mesures d'ensembles et notamment, définit les ensembles de mesure nulle et l'ensemble des boréliens, sur lequel on peut définir une mesure. Il se tourne ensuite vers les probabilités et la physique-mathématique. Borel est également considéré comme un mathématicien constructiviste. Il fut à l'origine de la théorie des jeux stratégiques et de la cybernétique que développeront von Neumann et Morgenstern. Son élève Henri Lebesgue utilisera ses résultats en topologie et théorie de la mesure pour sa théorie de l'intégration.

\parpic[l][t]{%
  \begin{minipage}{40mm}
    \fbox{\includegraphics[width=110px,height=140px]{img/medaillons/born.eps}}
  \end{minipage}
}
\textbf{Born, Max} (1882-1970) Né à Breslau est décédé à Göttingen est un physicien allemand, puis britannique. Initialement il suit ses études au collège de König-Wilhelm, et poursuit à l'Université de Breslau suivie par celle d'Heidelberg et de Zürich. Pendant les études pour son doctorat il entra en contact avec des mathématiciens comme Klein, Hilbert, Minkowski, Runge, Schwarzschild. En 1921, il est nommé professeur de physique théorique à Göttingen. Il émigre en Écosse en 1933 et devient citoyen britannique en 1939. Physicien théoricien remarquable, il est principalement connu pour son importante contribution à la physique quantique: développement (1925) de la mécanique quantique matricielle introduite par Werner Heisenberg et, surtout, il sera le premier à donner au carré du module de la fonction d'onde la signification d'une densité de probabilité de présence. Il fut également un pionnier dans la théorie quantique des solides (conditions de Born-Von Karmann) et dans l'électrodynamique non linéaire de Born-Infeld. Il est lauréat de la moitié du prix Nobel de physique de 1954 (l'autre moitié a été remise à Walther Bothe) pour ses recherches fondamentales en mécanique quantique, particulièrement pour son interprétation statistique de la fonction d'onde. La Royal Society lui décerne la Médaille Hughes en 1950.

\parpic[l][t]{%
  \begin{minipage}{40mm}
    \fbox{\includegraphics[width=110px,height=140px]{img/medaillons/bose.eps}}
  \end{minipage}
}
\textbf{Bose, Satyendranath} (1894-1974)  était une mathématicien et physicien indien, connu pour ses contributions à la théorie quantique. Né à Calcutta, Bose a fait ses études au Presidency College de Calcutta. En 1924, il propose une description statistique des systèmes quantiques, reprise par Albert Einstein, et qui n'impose aucune restriction sur la distribution en énergie des particules du système. Cette description est connue sous le nom de "statistique de Bose-Einstein", par opposition à la "statistique de Fermi-Dirac". Appliquée à la théorie du rayonnement du corps noir, cette nouvelle statistique conduit à la distribution de Planck et permet de traiter ce rayonnement comme un gaz de photons. Dans le domaine de la physique des particules élémentaires, la statistique de Bose-Einstein impose à la fonction d'onde des particules (dans l'équation de Schrödinger) d'être parfaitement symétrique pour l'ensemble des variables d'espace et de spin. Les particules obéissant à cette statistique (photons, mésons $\pi$, etc.) sont appelées des bosons. Professeur de physique aux Universités de Calcutta et de Dacca, Satyendranath Bose a été nommé, en 1958, professeur national des Indes.

\parpic[l][t]{%
  \begin{minipage}{40mm}
    \fbox{\includegraphics[width=110px,height=140px]{img/medaillons/broglie.eps}}
  \end{minipage}
}
\textbf{de Broglie, Louis Victor} (1892-1987) était un physicien français et lauréat du prix Nobel, qui apporta une contribution essentielle à la théorie quantique avec ses études de la radiation électromagnétique. Né à Dieppe, Louis de Broglie fit ses études à Paris. Il essaya de cerner la nature dualiste de la matière et de l'énergie et proposa l'association d'une onde à toute particule. Il proposa ainsi directement comment il était possible d'obtenir les règles de quantification du modèle d'atome de Bohr et Sommerfeld en exigeant qu'un nombre entier d'ondes soit logé sur une orbite stationnaire. Sa découverte de la nature ondulatoire des électrons (1924) lui valut le prix Nobel de physique en 1929 sans qu'il proposa toutefois une équation d'onde décrivant les phénomènes quantiques (ce que fera Schrödinger). Il fut élu à l'Académie des sciences en 1933 et à l'Académie française en 1943. Il fut nommé professeur de physique théorique à l'Université de Paris (1928), secrétaire perpétuel de l'Académie des sciences (1942), et conseiller au Commissariat à l'énergie atomique (1945).

\parpic[l][t]{%
  \begin{minipage}{40mm}
    \fbox{\includegraphics[width=110px,height=140px]{img/medaillons/brouwer.eps}}
  \end{minipage}
}
\textbf{Brouwer, Luitzen Egbertus Jan} (1881-1966) fut un grand mathématicien néerlandais du début du 20e siècle. Né d'un père proviseur, il réalisa des études secondaires très brillantes, et très rapides. À l'Université d'Amsterdam, il fut formé par Korteweg, qui est connu pour des contributions en mathématiques appliquées. Il soutient son doctorat en 1904. De 1909 à 1913, Brouwer s'intéresse à la topologie, et découvre la majeure partie des théorèmes auxquels son nom est resté attaché, dont son fameux théorème du point fixe. Pour beaucoup, Brouwer est le père de la topologie moderne. En 1912, il obtient grâce aux recommandations de Hilbert, une chaire à l'Université d'Amsterdam. Il y enseigne la théorie des ensembles, celle des fonctions, et l'axiomatique. Plus tard, il refusera de rejoindre Hilbert à Göttingen. Pendant la première Guerre mondiale sa santé se fragilisa et il s'éloigna quelques temps des champs de la recherche scientifique. Quand il y revint, ce fut pour se consacrer à ses premières amours (sa thèse portait déjà sur ce sujet): les fondements des mathématiques.  Brouwer est le fer de lance avec Poincaré des mathématiques intuitionnistes, par opposition au logicisme de Russel et Frege, et au formalisme de Hilbert. En particulier, pour Brouwer, un théorème d'existence ne peut être vrai que si on peut exhiber un processus, même formel, de construction. Cela le conduit notamment à rejeter la loi du tiers-exclu, qui dit qu'une propriété est ou vraie, ou fausse! Les preuves ainsi obtenues sont souvent plus longues, mais Brouwer fut capable de réécrire des traités de théorie des ensembles, de théorie de la mesure, et de théorie des fonctions en se conformant aux règles de l'intuitionnisme. Bizarrement, Brouwer n'enseigna jamais la topologie. C'est probablement dû au fait que les théorèmes que lui-même avait prouvés ne rentraient plus dans le cadre qu'il s'était fixé. Selon les témoignages de quelques-uns de ses étudiants, il était un personnage vraiment étrange, fou amoureux de sa philosophie, et un professeur auquel il ne fallait surtout pas poser de questions!

\phantomsection
\addcontentsline{toc}{section}{C}
\label{sec:C}

\parpic[l][t]{%
  \begin{minipage}{40mm}
    \fbox{\includegraphics[width=110px,height=140px]{img/medaillons/cantor.eps}}
  \end{minipage}
}
\textbf{Cantor, Georg} (1845-1918) se révèle être un étudiant brillant, notamment dans les matières manuelles. Malgré les injonctions de son père, qui rêve d'en faire un ingénieur, il part en 1862 à Berlin étudier la mathématique, où ses maîtres sont Weierstrass et Kronecker. Il soutient son doctorat en 1867 (sur la théorie des nombres). Les premières recherches postdoctorales de Cantor sont consacrées à la décomposition des fonctions en sommes de séries trigonométriques (les célèbres séries de Fourier) et particulièrement à l'unicité de cette décomposition. Afin de résoudre complètement ce difficile problème, il est amené à introduire et à étudier des ensembles dits "ensembles exceptionnels". Cela le conduit à définir en 1872 très précisément ce qu'est un nombre réel, comme limite d'une suite de nombres rationnels; parallèlement, son ami Dedekind donne la même année une autre définition de la droite des réels, à partir des coupures. Cantor et Dedekind constatent à cette occasion qu'il y a beaucoup plus de réels que de rationnels, mais il n'y a pas jusque-là de définition mathématique à ce "beaucoup plus". En 1874, dans le prestigieux \textit{Journal de Crelle}, Cantor donne une définition du nombre d'éléments d'un ensemble infini qui prolonge naturellement celle du cardinal d'un ensemble infini, qui prolonge celle du cardinal d'un ensemble fini. Il en découle, jusqu'en 1897, une succession de découvertes étranges: il y autant d'entiers pairs que d'entiers tout court, autant de points sur un segment que dans un carré, beaucoup plus de nombres transcendants que de nombres rationnels. Cette hiérarchie dans les ensembles infinis conduit progressivement Cantor à définir des nouveaux nombres, les ordinaux transfinis, et à définir une arithmétique sur ces nombres. Les travaux de Cantor ont eu beaucoup d'influence au 20ème siècle. On citera d'abord, en 1903, un paradoxe soulevé par Russell dans la théorie naïve des ensembles: si $A$ est l'ensemble de tous les ensembles qui ne sont pas éléments d'eux-mêmes, $A$ est-il contenu dans $A$? Les logiciens surmonteront cette difficulté conceptuelle, sans rien changer des conclusions de Cantor. Citons aussi le problème de l'hypothèse du continu. Un des derniers axes de recherche de Cantor était d'estimer le nombre d'éléments de la droite réelle. Plus précisément, Cantor souhaitait prouver l'absence de tout ensemble dont le cardinal soit strictement compris entre le cardinal des entiers et celui des réels. C'est ce que l'on appelle "l'hypothèse du continu". Tous les travaux de Cantor et de ses successeurs pour confirmer ou infirmer l'hypothèse du continu furent vains, et pour cause: en 1963, le logicien Cohen prouva que, dans une théorie standard des ensembles, l'hypothèse du continu est indécidable. On peut très bien supposer qu'elle est vraie ou qu'elle est fausse sans obtenir de contradiction dans la théorie.

\parpic[l][t]{%
  \begin{minipage}{40mm}
    \fbox{\includegraphics[width=110px,height=140px]{img/medaillons/carnot.eps}}
  \end{minipage}
}
\textbf{Carnot, Nicolas Léonard Sadi} (1796-1832), physicien et ingénieur militaire français, considéré comme le créateur de la science thermodynamique. Fils aîné de Lazare Carnot, surnommé "le Grand Carnot", Sadi fit ses études à l'École Polytechnique. En 1824, il décrivit sa conception du moteur à chaleur idéal, appelé "moteur Carnot", dans lequel toute l'énergie disponible est utilisée. Il découvrit que la chaleur ne pouvait passer d'un corps froid à un corps plus chaud, et que le rendement d'un moteur dépendait de la quantité de chaleur qu'il était capable d'utiliser. Cette découverte, ou cycle de Carnot, est à la base de la seconde loi de la thermodynamique.\\

\parpic[l][t]{%
  \begin{minipage}{40mm}
    \fbox{\includegraphics[width=110px,height=140px]{img/medaillons/cartan.eps}}
  \end{minipage}
}
\textbf{Cartan, Élie} (1869-1951) fit ses études primaires à l'École de Dolomieu, puis au collège de Vienne et au lycée de Grenoble. Il suivit au lycée Jeanson-de-Sailly la préparation à l'École Normale Supérieure, où il entra en 1888. Il y suivit notamment les enseignements de Poincaré, Picard et de Hermite. Les premiers travaux d'Élie Cartan qui devaient déboucher sur sa thèse soutenue en 1894, portent sur les groupes de Lie simples complexes, où il reprend, corrige et développe les résultats de structure et de classification obtenus par W. Killing. Élie Cartan obtient un poste de lecteur à l'Université de Montpellier de 1894 à 1896, puis à la Faculté des sciences de Lyon de 1896 à 1903. La même année, il est nommé professeur à la Faculté des sciences de Nancy, où il restera jusqu'en 1909. Il donne en même temps des cours à l'École d'Électrotechnique et de Mécanique Appliquée. Il rédige deux grands articles sur une généralisation en dimension infinie des groupes de Lie simples. Il élabore la méthode du "repère mobile", et la théorie des formes extérieures qui devaient influencer le développement ultérieur de la géométrie différentielle. En 1909, il quitte Nancy pour venir enseigner à la Sorbonne, où il est nommé professeur en 1912. Il assure par ailleurs un enseignement à l'École de Physique et Chimie de Paris. En 1914, il résout le problème de la classification des groupes de Lie simples réels, et détermine les représentations de dimension finie de ces groupes. Pendant la guerre, il sert comme sergent dans l'hôpital aménagé dans les locaux de l'École Normale Supérieure, tout en continuant ses travaux mathématiques. Son oeuvre mathématique ultérieure est considérable, avec près de 200 publications et de nombreux ouvrages. Parmi les thèmes abordés, mentionnons l'étude des systèmes de Pfaff, la théorie de la déformation, l'étude des variétés à courbure constante négative, la théorie de la gravitation d'Einstein, la théorie des connexions affines, les groupes d'holonomie, les espaces riemanniens symétriques, les spineurs. Il est aussi l'auteur de plusieurs articles sur l'histoire de la géométrie. Il prit sa retraite en 1940.


\parpic[l][t]{%
  \begin{minipage}{40mm}
    \fbox{\includegraphics[width=110px,height=140px]{img/medaillons/cauchy.eps}}
  \end{minipage}
}
\textbf{Cauchy, Augustin-Louis} (1789-1857) C'est à Cherbourg que Cauchy commence ses recherches mathématiques sur les polyèdres, et ses premiers résultats sont prometteurs. Mais, fatigué par le cumul de la charge d'ingénieur et des longues veillées de recherche, Cauchy connaît un état dépressif qui s'éternise et le pousse à retourner vivre chez ses parents. À Paris, il cherche une situation en adéquation avec sa volonté de faire de la recherche en mathématique pure. En 1815, il achève un brillant mémoire où il démontre un célèbre théorème de Fermat sur les nombres polygonaux. Ceci fera beaucoup pour sa notoriété, et en 1816, il accède à l'Académie des Sciences, en remplacement de Carnot et Monge touchés par l'épuration. Le cours d'analyse que Cauchy professe à l'École Polytechnique est décrié tant par ses élèves que par ses collègues des autres matières. Pourtant, c'est ce cours publié en 1821 et 1823, qui devait devenir la référence de l'analyse au 19ème siècle, en mettant en avant la rigueur, et plus seulement l'intuition. C'est la première fois que de vraies définitions de limites, de continuité, de convergence de suites, de séries, sont énoncées. Cette rigueur reste toutefois encore relative, puisque Cauchy "prouve" que la limite d'une série de fonctions continues est continue, ce qui est faux. Il est vrai que Cauchy ne dispose pas encore d'une définition claire et précise des nombres réels. C'est l'époque aussi où Cauchy réalise des travaux profonds sur les fonctions d'une variable complexe (établissant par exemple l'expression des résidus), ainsi que des avancées dans la théorie des groupes finis. Cauchy ne fut jamais le chef d'une école de mathématiciens, et il se comporta parfois maladroitement avec de jeunes chercheurs comme Abel ou Galois, dont il sous-estime, ou même perd, des mémoires de première importance. Ses relations avec ses collègues ne sont en général pas très faciles.

\parpic[l][t]{%
  \begin{minipage}{40mm}
    \fbox{\includegraphics[width=110px,height=140px]{img/medaillons/cayley.eps}}
  \end{minipage}
}
\textbf{Cayley, Arthur} (1821-1895) Né à Richmond (Surrey), manifesta très tôt de vives dispositions pour la mathématique. Cependant, malgré le grand intérêt de ses premières publications, il ne put s'imposer comme mathématicien ; il décida de faire des études de droit et devint avocat en 1849. Pendant 14 ans, il exerça ce métier tout en s'adonnant à des recherches scientifiques. En 1863, Cayley est nommé professeur à Cambridge et peut enfin se consacrer entièrement aux mathématiques. Dans l'ensemble de l'oeuvre de Cayley, notamment dans ses travaux de jeunesse, est sensible l'influence des fondateurs de l'école algébrique anglaise qui avaient formulé le programme de l'algèbre moderne en accordant une priorité marquée à l'approche formelle des problèmes. Mathématicien lettré et créateur, Cayley, dans le sillage de l'école anglaise, sut élaborer de nouvelles et fructueuses théories. La richesse de l'approche de Cayley apparaît dès ses premiers travaux sur la théorie des groupes (1854). Cayley, abordant les travaux de Galois, Gauss et Cauchy avec les méthodes des algébristes anglais, donne une définition des groupes abstraits ce qui le conduisit à la notion d'isomorphisme. L'étude des systèmes d'équations linéaires conduisit Cayley à celle des déterminants. Dans ses premiers travaux, il établit de nombreuses règles de calcul sur les déterminants, y compris la relation de multiplication des déterminants qui figurait déjà dans les travaux de Cauchy, Binet et Jacobi. À côté d'études originales sur les déterminants, on y rencontre la notion de tableau rectangulaire représentant les coefficients d'un système d'équations linéaires ou les coefficients d'une transformation linéaire. Cayley étudie les matrices rectangulaires à coefficients réels ou complexes ; il introduit les opérations sur les matrices et décrit leurs propriétés, y compris le caractère non commutatif de la multiplication. Il s'agit là sans doute de la première apparition de l'algèbre linéaire. Quelques années plus tard, Cayley étudiera aussi les systèmes non associatifs et publiera des résultats d'algèbre multilinéaire. Cayley a consacré un grand nombre de ses publications aux problèmes de la géométrie et à l'étude des courbes et des surfaces algébriques. À 22 ans, il émettait l'idée de la géométrie à $n$ dimensions, idée qui fut formulée aussi, presque simultanément, mais sous une forme un peu différente, par Grassman. Cayley ne revint que beaucoup plus tard (en 1870) sur l'espace à $n$ dimensions, mais sa méthode algébrique contribua aux importantes découvertes qui eurent lieu dans les autres domaines de la géométrie. C'est ainsi que, dans le \textit{Sixth Memoir on Quantics} de 1859, il introduit la métrique projective, subordonnant ainsi la géométrie métrique à la géométrie projective ; il démontre alors que les notions fondamentales de la géométrie métrique (angles et distances) sont les invariants et les covariants de certaines transformations linéaires de la quadrique absolue.

\parpic[l][t]{%
  \begin{minipage}{40mm}
    \fbox{\includegraphics[width=110px,height=140px]{img/medaillons/chandrasekhar.eps}}
  \end{minipage}
}
\textbf{Chandrasekhar, Subrahmanyan} (1910-1995) obtint à l'âge de 23 ans son doctorat au Trinity College de l'Université de Cambridge. Spécialiste en astrophysique Chandrasekhar fit progresser de façon décisive la connaissance de l'évolution hydrodynamique et hydromagnétique des transferts d'énergie par rayonnement sans oublier les effets quantiques et relativistes dans les évolutions des étoiles. Sa contribution majeure dans ce domaine est la transformation des étoiles en naines blanches et au-delà d'un astre d'une masse supérieur à la limite de Chandrasekhar ($1.44$ celle du Soleil), l'effondrement de l'astre en une étoile à neutrons. Les objets plus massifs donnant eux des Trous Noirs.\\\\

\parpic[l][t]{%
  \begin{minipage}{40mm}
    \fbox{\includegraphics[width=110px,height=140px]{img/medaillons/clairaut.eps}}
  \end{minipage}
}
\textbf{Clairaut, Alexis-Claude} (1713-1765) était un membre de l'Académie française des sciences et fut l'un des mathématiciens et physiciens les plus renommés du 18ème siècle. À l'âge de 10 ans, il connaissait le calcul infinitésimal, à 12 ans, il soumettait sa première étude à l'académie des sciences et à 18 ans, il publia un livre contenant des extensions importantes à la géométrie qui lui ont valu l'admission à l'académie en 1731. Clairaut fut l'un des scientifiques qui accompagnaient Maupertuis en Laponie pour acquérir les dates nécessaires pour la détermination de la forme de la Terre. En 1743, il publia sa \textit{Théorie de la figure de la Terre}, qui calculait plus précisément que l'avait fait Newton, la forme qu'adopte un corps en rotation due à la gravitation naturelle de ses parties. En 1760, il publia sa\textit{ Théorie du mouvement des comètes}, qui prédit avec précision la date à laquelle la comète de Halley sera arrivée au point le plus proche du Soleil.

\parpic[l][t]{%
  \begin{minipage}{40mm}
    \fbox{\includegraphics[width=110px,height=140px]{img/medaillons/cohen.eps}}
  \end{minipage}
}
\textbf{Cohen, Paul Joseph} (1934-2007) était un mathématicien et logicien né au New Jersey en décédé à Stanford. En 1963, Cohen a découvert une nouvelle construction de modèles, appelée "forcing", qui joue désormais un rôle fondamental dans la théorie des ensembles et dans la théorie des modèles. Il a aussi construit des modèles de la théorie des ensembles (supposée consistante) dans lesquels l'axiome du choix et l'hypothèse du continu ne sont pas vérifiés, ce qui, compte tenu de l'oeuvre antérieure de Kurt Gödel, établit que l'axiome du choix et l'hypothèse du continu sont indépendants des systèmes usuels de la théorie des ensembles. Ce travail a valu à Cohen, en 1966, la médaille Fields de l'Union Mathématique Internationale. Il est également l'auteur de travaux intéressants en analyse classique.

\parpic[l][t]{%
  \begin{minipage}{40mm}
    \fbox{\includegraphics[width=110px,height=140px]{img/medaillons/connes.eps}}
  \end{minipage}
}
\textbf{Connes, Alain} (1947-) Né à Draguignan, ancien élève à l'École Normale Supérieure, il a reçu, en 1980, le prix Ampère, l'un des plus importants décernés par l'Académie des sciences. Il a été élu membre de cette académie, dont il a été le benjamin, en 1981. Les premiers travaux d'Alain Connes s'inscrivent directement dans la tradition de John von Neumann et de ses continuateurs immédiats. Le développement de la physique quantique vers les années vingt avait mis à l'ordre du jour l'étude d'espaces non plus à trois dimensions, comme celui où nous croyons vivre, ni à quatre, comme en relativité einsteinienne, mais à une infinité de dimensions (les espaces de Hilbert). L'un des outils essentiels de la physique quantique est la notion d'opérateur dans un tel espace, notion généralisant celle de rotation d'un espace euclidien. La théorie des algèbres d'opérateurs a débuté vers 1930 par les travaux de von Neumann, qui a montré l'importance d'un certain type d'algèbres d'opérateurs, appelées aujourd'hui "algèbres de von Neumann", et qui a établi pour ces algèbres un théorème de décomposition en facteurs premiers assez analogue au théorème de décomposition bien connu pour les nombres entiers usuels. Dès l'origine, les facteurs avaient été classés en trois types: facteurs de type I, II, III. On a eu assez tôt une bonne compréhension des facteurs de type I et pas mal d'informations sur  ceux de type II, mais les facteurs de type III sont restés pendant longtemps beaucoup plus mystérieux. Même les exemples étaient rares et von Neumann disait, à propos de ce cas: "C'est le plus réfractaire de tous, et les outils pour l'étudier nous font défaut, au moins pour l'instant". La première réussite de Connes, qui lui a d'emblée valu la renommée internationale, a été une percée spectaculaire vers l'élucidation de la structure des facteurs de type III ; on peut dire qu'il est le premier à avoir acquis une connaissance concrète de ces objets, jusque-là assez énigmatiques, pris dans leur ensemble. Très grosso modo, les résultats de Connes ramènent l'étude des facteurs de type III à celle des facteurs de type II et de leurs automorphismes. L'oeuvre d'Alain Connes est celle d'un mathématicien très complet, capable de résoudre des problèmes difficiles, légués par le passé, mais aussi de transformer entièrement une discipline par l'introduction d'idées nouvelles, d'une grande originalité.

\parpic[l][t]{%
  \begin{minipage}{40mm}
    \fbox{\includegraphics[width=110px,height=140px]{img/medaillons/copernic.eps}}
  \end{minipage}
}
\textbf{Copernic, Nicolas} (1473-1543) étudie à l'Université de Cracovie à partir de 1491, il se rend ensuite en Italie pour y suivre des cours de droit canon à l'Université de Bologne. Il suit également les cours d'astronomie de Domenico Maria Novara, un des premiers scientifiques à remettre en cause les enseignements de Ptolémée. En 1500, il enseigne la mathématique à Rome, avant de retourner pour un an à Frauenburg où son oncle l'a nommé chanoine en 1497. Ayant obtenu l'autorisation de poursuivre ses études en Italie, il s'inscrit aux facultés de droit et de médecine de Padoue et obtient son doctorat en droit canon à Ferrare en 1503. Enfin, il retourne à Frauenburg où il fait construire un observatoire et entame ses recherches en astronomie. Il y demeurera jusqu'à sa mort. La cosmologie de l'époque est alors basée sur le système géocentrique de Ptolémée. La Terre se trouve immobile au centre de plusieurs sphères concentriques qui portent la Lune, Mercure, Vénus, le Soleil, Mars, Jupiter, Saturne et enfin les étoiles. Mais ce système ne convient pas à Copernic, qu'il trouve compliqué et bancal. Il consulte alors les auteurs de l'Antiquité (Cicéron, Aristarque de Samos, etc.) et constate que certains d'entre eux envisagent la rotation des planètes, dont la Terre, autour du Soleil, considéré comme fixe. Copernic démontre alors que la combinaison des mouvements de la Terre et des planètes explique parfaitement le mouvement apparent des planètes (dans le sens direct et rétrograde). De plus, il établit que leurs changements de diamètre apparent apparaissent comme une conséquence de leur révolution autour du Soleil. Ses recherches se poursuivront pendant 36 ans et il démontrera que la Lune est un satellite de la Terre et que l'axe de la Terre n'est pas fixe. Son oeuvre maîtresse \textit{De Revolutionibus orbium coelestium} est publiée en 1543 à Nuremberg et Copernic n'en reçoit les premiers exemplaires que quelques heures avant sa mort. Dans la dédicace qu'il fait au Pape Paul III, il présente son système comme une pure hypothèse, évitant ainsi la vindicte de l'Église. Adopté un siècle après sa mort après avoir été violemment rejeté, le système copernicien apporta une profonde révolution dans la conception du monde et plus généralement dans la pensée scientifique.

\parpic[l][t]{%
  \begin{minipage}{40mm}
    \fbox{\includegraphics[width=110px,height=140px]{img/medaillons/coriolis.eps}}
  \end{minipage}
}
\textbf{Coriolis, Gaspard} (1792-1843) était un ingénieur et mathématicien français qui mit en évidence les forces centrifuges composées, dites "forces de Coriolis". Cet ingénieur des Ponts et Chaussées est l'auteur d'importants travaux en mécanique. En 1835, il démontra que l'accélération d'un mobile dans un référentiel en rotation est soumis à une complémentaire (force de Coriolis) perpendiculaire au sens de déplacement du mobile dans ce référentiel. Bien que de faible intensité à la surface de la Terre, cette force, produite par la rotation de la planète, influence la direction des courants marins et aériens. Elle produit une déviation vers l'est et explique, par exemple, le mouvement circulaire des ouragans.

\parpic[l][t]{%
  \begin{minipage}{40mm}
    \fbox{\includegraphics[width=110px,height=140px]{img/medaillons/coulomb.eps}}
  \end{minipage}
}
\textbf{Coulomb, Charles Augustin} (1736-1806) était un physicien français, pionnier de la théorie de l'électricité. Né à Angoulême, il servit comme ingénieur militaire pour la France aux Antilles, mais se retira à Blois à la révolution française, pour continuer ses recherches sur le magnétisme, le frottement et l'électricité. En 1777, il inventa la balance de torsion qui permet de mesurer la force de l'attraction magnétique et électrique. Grâce à cette invention, Coulomb fut capable de formuler le principe, maintenant connu sous le nom de "loi de Coulomb", qui gouverne l'interaction entre les charges électriques. En 1779, Coulomb publia le traité \textit{Théorie des machines simples}, une analyse du frottement dans les machines. Après la révolution, Coulomb quitta sa retraite et aida le nouveau gouvernement à concevoir un système métrique pour les poids et mesures. L'unité utilisée pour exprimer la quantité de charge électrique, le "Coulomb", tient son nom du physicien.

\parpic[l][t]{%
  \begin{minipage}{40mm}
    \fbox{\includegraphics[width=110px,height=140px]{img/medaillons/cournot.eps}}
  \end{minipage}
}
\textbf{Cournot, Antoine Augustin} (1801-1877) étudia au collège de Gray de 1809 à 1816. Il obtient des prix d'excellence de mathématiques. Il entre en 1820 au collège Royal de Besançon et obtient le prix d'honneur de mathématiques spéciales. Avec deux mémoires et deux traductions de traités divers de mathématiques, il se fait remarquer par Poisson, qui le fait nommer en 1834 professeur d'analyse et de mécanique à la faculté des sciences de Lyon. Augustin Cournot est un savant, c'est-à-dire un homme de savoir étendu à tous les domaines de la science, un savant philosophe mais, qui par sa modestie, n'a pas connu la célébrité. Cournot fut d'abord un professeur et un vulgarisateur d'une grande clarté. Trois ouvrages mathématiques le distingue:\textit{ Traité élémentaire de la théorie des fonctions et du calcul infinitésimal} (1841); \textit{Exposition de la théorie des chances et des probabilités} (1843) ; \textit{De l'origine et des limites de la correspondance entre l'algèbre et la géométrie} (1847). Mais le génie de Cournot se situe dans l'introduction des probabilités en économie. Il est le précurseur des théories modernes en économie, reprises ensuite par Léon Walras qui dans sa notice autobiographique achevée en 1904, ainsi que dans plusieurs lettres, a rappelé le rôle primordial qu'ont joué dans le développement de sa pensée, d'une part, l'oeuvre d'Antoine Augustin Cournot et d'autre part, celle de son père, l'économiste et philosophe Auguste Walras qui fut le condisciple d'Augustin Cournot à l'École Normale.

\parpic[l][t]{%
  \begin{minipage}{40mm}
    \fbox{\includegraphics[width=110px,height=140px]{img/medaillons/clausius.eps}}
  \end{minipage}
}
\textbf{Clausius, Rudolf} (1822-1888) était l'un des plus grands physiciens du 19ème siècle. Il est connu principalement pour sa contribution à l'étude de la thermodynamique. Le premier, ce savant allemand formula ce que l'on a coutume d'appeler le "deuxième principe" et proposa une définition claire de l'entropie. Il est aussi l'un des principaux créateurs de la théorie cinétique des gaz. Né à Köslin, en Poméranie, Clausius fréquenta les Universités de Berlin, puis de Halle dont il sortit diplômé en 1848. Professeur jusqu'à sa mort, il fut titulaire de la chaire de physique de l'École Royale d'Artillerie et du Génie à Berlin (1850-1855), puis, simultanément, à l'Université et à l'École Polytechnique de Zürich (1855-1867), ensuite à l'Université de Würzburg (1867-1869), enfin à celle de Bonn, de 1869 à sa mort. Sa première publication, en 1850 dans les \textit{Annalen der Physik} de Poggendorff, attira largement l'attention. Il cherchait à y concilier l'idée de l'équivalence entre le travail et la chaleur. Clausius fit remarquer que l'hypothèse de la conservation de la chaleur dans le processus de transfert n'était pas une partie essentielle de la théorie de Carnot. Il établit en fait que, dans une machine idéale, la quantité de chaleur prise à la chaudière doit toujours être supérieure à celle qui est cédée au condenseur, et ce d'une quantité exactement équivalente au travail fourni. Cette importante synthèse effectuée, Clausius, dans la même publication, énonça ce que nous appelons aujourd'hui le "deuxième principe de la thermodynamique". C'était la généralisation de la nécessité, déjà établie par Carnot, de la présence, non seulement d'un corps chaud (la chaudière), mais aussi d'un corps froid (le condenseur) pour qu'un travail soit fourni par une machine à vapeur. En 1854, Clausius, poussant plus avant les vues exprimées dès 1850, proposa le premier énoncé clair du concept de l'entropie. Il cherchait à mesurer l'aptitude de l'énergie calorifique de n'importe quel système réel non idéal à fournir du travail. Dans le cas de la conduction thermique le long d'un barreau solide, par exemple, la chaleur passe de l'extrémité chaude à l'extrémité froide sans fournir aucun travail, bien que ce transfert s'accompagne d'une diminution de l'aptitude de l'extrémité chaude à servir par la suite de source potentielle de travail. Cette diminution survient parce qu'à la fin du processus l'énergie calorifique est détenue par un corps situé à une température inférieure à celle de l'état initial. Elle n'a donc pas été perdue, mais seulement dégradée puisque, d'après le deuxième principe de la thermodynamique, on ne peut retrouver la température initiale qu'avec l'aide d'un travail extérieur. Les dernières contributions majeures de Clausius à la science datent de 1857 et 1858 et sont relatives à la théorie cinétique des gaz. Bien qu'il ne soit pas le premier à avoir conçu cette dernière, déjà proposée et discutée par Joule et Krönig notamment, il prend rang avec Maxwell parmi ses fondateurs. Il introduisit le concept du libre parcours moyen et établit l'importante distinction entre l'énergie de translation et l'énergie interne d'une particule de gaz. De plus, on lui reconnaît généralement le mérite d'avoir, par ses travaux théoriques, jeté un pont entre la théorie atomique et la thermodynamique.

\parpic[l][t]{%
  \begin{minipage}{40mm}
    \fbox{\includegraphics[width=110px,height=140px]{img/medaillons/curiepierre.eps}}
  \end{minipage}
}
\textbf{Curie, Pierre} (1859-1906) est considéré comme un des pionniers de la chimie/physique sur la radioactivité. C'est même lors d'une thèse publiée en 1898 que le terme "radioactivité" fut employé pour la première fois par sa femme Marie et lui. L'éducation de Pierre commença à un très jeune âge par son père, qui était médecin général. Les Curie avaient l'habitude de fréquenter la campagne et les environs de Paris les dimanches ; Pierre, lors de ses promenades, apprit rapidement tous les noms de plantes et d'animaux. Étant donné que l'école n'était pas obligatoire à cette époque (pas avant 1881 où la loi Ferry l'a rendue obligatoire), Pierre reçut son éducation à la maison, en compagnie de sa mère, ensuite avec son frère et par après, avec des précepteurs et finalement, seul. À l'âge de 14 ans, l'éducation de Pierre fut confiée à M. Bazille qui lui enseigna la mathématique élémentaire et spéciale, ceci développa énormément les capacités mentales de Pierre qui avait clairement un intérêt pour la mathématique. À l'âge de 16 ans, il fut reçu bachelier en sciences. En 1877, il obtint la licence en sciences physiques de l'école de pharmacie. Dans les années qui suivront, il étudiera les cristaux et le magnétisme, ce qui le mènera éventuellement à la découverte de la piézo-électricité. En 1877, il prit un poste comme préparateur où il fut payé la somme de 1200 francs par année. Il devint par après démonstrateur d'expériences de physique pour les laboratoires jusqu'en 1882 où il devint directeur de tous les travaux pratiques aux écoles de physique et de chimie industrielle. Pierre épousa sa femme Marie Sklodowska en 1895 et ils eurent ensemble deux enfants, Irène et Êve. Pierre Curie gagna en 1903, avec sa femme, le prix Nobel de physique pour leurs travaux sur les substances radioactives et leurs découvertes de deux nouveaux éléments: le radium et le polonium.

\parpic[l][t]{%
  \begin{minipage}{40mm}
    \fbox{\includegraphics[width=110px,height=140px]{img/medaillons/curiemarie.eps}}
  \end{minipage}
}
\textbf{Curie, Marie} (1867-1934) était une chimiste et physicienne née à Varsovie et décédée en Haute-Savoie. Fille d'un père professeur de mathématiques et de physique et d'une mère institutrice, elle est la benjamine d'une famille de 4 soeurs. Entre 1876 et 1878 elle perd une soeur et sa mère. Elle se réfugie alors dans les études où elle excelle dans toutes les matières, et où la note maximale lui est accordée. Elle obtient ainsi son diplôme de fin d'études secondaires avec la médaille d'or en 1883. Elle souhaite poursuivre des études supérieures et enseigner, mais ces études sont interdites aux femmes. Lorsque sa soeur aînée, Bronia, part faire des études de médecine à Paris, Marie s'engage comme gouvernante en province en espérant économiser pour la rejoindre, tout en ayant initialement pour objectif de revenir en Pologne pour enseigner. Au bout de trois ans, elle regagne Varsovie, où un cousin lui permet d'entrer dans un laboratoire. En 1891, elle part pour Paris, où elle est hébergée par sa sœur et son beau-frère. La même année, elle s'inscrit pour des études de physique à la faculté des sciences de Paris. Trois ans plus tard, elle obtient sa licence en sciences physiques, en étant première de sa promotion. Pendant l'été, une bourse d'études lui est accordée, qui lui permet de poursuivre ses études à Paris. Un an plus tard, elle obtient sa licence en sciences mathématiques, en étant seconde. Elle hésite alors à retourner en Pologne. Lors d'une soirée elle rencontre Pierre Curie (son future époux), qui est chef des travaux de physique à l'École municipale de physique et de chimie industrielles et étudie également le magnétisme, avec lequel elle va travailler. Marie reçoit (avec son époux Pierre Curie) une moitié du prix Nobel de physique de 1903 (l'autre moitié est remise à Henri Becquerel) pour leurs recherches sur les radiations. En 1911, elle obtient le prix Nobel de chimie pour ses travaux sur le polonium et le radium.

\phantomsection
\addcontentsline{toc}{section}{D}
\label{sec:D}

\parpic[l][t]{%
  \begin{minipage}{40mm}
    \fbox{\includegraphics[width=110px,height=140px]{img/medaillons/dalton.eps}}
  \end{minipage}
}
\textbf{Dalton, John} (1766-1844) était un chimiste et physicien britannique, qui développa la théorie atomique sur laquelle fut fondée la science physique moderne. Dalton commença en 1787 une série d'observations météorologiques qu'il poursuivit pendant 57 ans, accumulant quelque $200'000$ observations et mesures du temps dans la région de Manchester! L'intérêt de Dalton pour la météorologie le conduisit à étudier différents phénomènes ainsi que les instruments utilisés pour les mesurer. Il fut le premier à prouver la validité de l'idée selon laquelle la pluie est précipitée par une baisse de température, non par un changement de la pression atmosphérique. Dalton arriva à sa théorie atomique par une étude des propriétés physiques de l'air atmosphérique et des autres gaz. Au cours de ses recherches, il découvrit la loi des pressions partielles des gaz mélangés, souvent connue comme la "loi de Dalton", selon laquelle la pression totale exercée par un mélange de gaz est égale à la somme des pressions individuelles qu'exercerait chacun des gaz s'il occupait seul le volume entier.

\parpic[l][t]{%
  \begin{minipage}{40mm}
    \fbox{\includegraphics[width=110px,height=140px]{img/medaillons/davinci.eps}}
  \end{minipage}
}
\textbf{Da Vinci, Leonardo} (1452-1519) était un peintre, sculpteur, architecte et homme de sciences italien. Homme d'esprit universel, à la fois artiste, scientifique, inventeur et philosophe, Leonardo incarna l'esprit universaliste de la Renaissance et demeure l'un des grands hommes de cette époque. À 5 ans, son père ayant noté ses dons pour le dessin, le place comme apprenti dans l'atelier de Verrocchio, à Florence. Il entre à 20 ans à la Guilde des peintres, et débute sa carrière de peintre par des oeuvres immédiatement remarquables comme \textit{La vierge à l'oeillet}, ou \textit{L'Annonciation} (1473). Il améliore la technique du sfumato (impression de brume) à un point de raffinement jamais atteint avant lui. En 1481, le monastère de San Donato lui commande \textit{L'Adoration des Mages}, mais Leonardo, vexé de pas être choisi pour la décoration de la chapelle Sixtine à Rome, ne terminera jamais ce tableau et quitte Florence pour Milan. Après la réalisation de \textit{La Vierge aux rochers}, pour la chapelle San Francesco Grande, et celle de la statue équestre de Francesco Sforza, il trouve la gloire dans toute l'Italie. En 1495, les Dominicains de Sainte-Marie-des-Grâces lui commandent \textit{La Cène}. En 1498, il réalise le plafond du palais Sforza. De cette époque, datent aussi \textit{La Joconde} et \textit{La Bataille d'Anghiari}. Leonardo réalise aussi une grande quantité d'études sur la zoologie, la botanique, l'anatomie, la géologie. Il imagine de multiples appareils et machines, dont la première machine volante, qui resteront au stade de dessins. Plus qu'en tant que scientifique proprement dit, Leonardo da Vinci a impressionné ses contemporains et les générations suivantes par son approche méthodique du savoir, du savoir apprendre, du savoir observer, du savoir analyser. La démarche qu'il déploya dans l'ensemble des activités qu'il abordait, aussi bien en art qu'en technique (les deux ne se distinguant d'ailleurs pas dans son esprit), procédait d'une accumulation préalable d'observations détaillées, de savoirs disséminés ça et là, qui tendait vers un surpassement de ce qui existait déjà, avec la perfection pour objectif. Bon nombre des croquis, notes et traités de Leonardo da Vinci ne sont pas à proprement parler des trouvailles originales, mais sont le résultat de recherches effectuées dans un souci encyclopédique, avant l'heure. En 1516, il rejoint la cour de François Ier, où il participe à des projets d'urbanisme.De Léonard de Vinci, subsistent aujourd'hui 7'000 notes et dessins, et quarante oeuvres attestées, dont huit ont disparu.

\parpic[l][t]{%
  \begin{minipage}{40mm}
    \fbox{\includegraphics[width=110px,height=140px]{img/medaillons/dantzig.eps}}
  \end{minipage}
}
\textbf{Dantzig, George Bernard} (1914-2005) était un mathématicien né à Portland et décédé à Stanford, inventeur du fameux algorithme du simplexe en optimisation linéaire. Son père, Tobias, était un mathématicien russe qui avait étudié avec Poincaré à Paris. Il a épousé une collègue de la Sorbonne, Anja Ourisson, et le couple a émigré aux États-Unis. Il est l'acteur principal d'une histoire fameuse en mathématique. Dans l'un de ses cours de doctorat à l'Université de Berkeley, le professeur Jerzy Neyman a proposé deux problèmes dits "ouverts" en statistiques (pour rappel un problème ouvert est un problème qui bien qu'ayant été formulé, n'a pas encore été résolu). De tels problèmes sont d'une difficulté importante et demandent des recherches pouvant s'étaler sur plusieurs années. Dantzig était en retard et croyait qu'il s'agissait de devoirs. Sans prendre plusieurs années mais bien quelques jours, il les a résolus. Il a reçu son doctorat de Berkeley en 1946. Six ans plus tard, il était engagé pour faire de la recherche mathématique à la RAND Corporation, où il implante l'algorithme du simplexe dans les ordinateurs. En 1960, l'Université de Berkeley l'engage pour enseigner l'informatique, pour éventuellement devenir le responsable du centre de recherche opérationnelle. Six ans plus tard, il occupe un poste similaire à l'Université Stanford, poste qu'il occupe jusqu'à sa retraite pendant les années 1990. En plus de ses travaux sur l'algorithme du simplexe et l'optimisation linéaire, il a aussi travaillé sur les méthodes de décomposition des problèmes de grande taille, l'analyse de sensibilité, les méthodes de résolution matricielles avec pivot, l'optimisation non linéaire et l'optimisation stochastique.

\parpic[l][t]{%
  \begin{minipage}{40mm}
    \fbox{\includegraphics[width=110px,height=140px]{img/medaillons/debye.eps}}
  \end{minipage}
}
\textbf{Debye, Peter Joseph Wilhelm} (1884-1966) était un physicien et chimiste né à Maastricht et décédé à New-York. Debye s'inscrit en 1901 à l'Université d'Aix-la-Chapelle en Allemagne. Il y étudie les mathématiques et la physique classique et en sort en 1905 titulaire d'un diplôme d'électrotechnique. En 1907, il produit sa première publication scientifique, une solution mathématique élégante d'un problème mettant en jeu des courants de Foucault. Il étudie à Aix-la-Chapelle sous la direction d'Arnold Sommerfeld. En 1906, il accompagne Sommerfeld à Münich comme assistant. Il y obtient son doctorat en 1908 avec une thèse sur la pression de radiation. En 1910, il démontre la loi de Planck par une méthode dont Max Planck reconnut qu'elle était plus simple que la sienne. En 1911, Debye est nommé professeur à Zürich. Il se rend ensuite à Utrecht en 1912, à Göttingen en 1913, il est de retour à Zürich en 1920, se rend à Leipzig en 1927 et à Berlin en 1934 où il devient directeur de la \textit{Société Kaiser Wilhelm} qui prendra en 1938 le nom de \textit{Société Max Planck}. En 1912, il étend la théorie d'Albert Einstein de la chaleur spécifique aux basses températures en incluant des contributions des phonons de basses fréquences (modèle de Debye). En 1913, il étend la théorie de Niels Bohr de la structure atomique en introduisant des orbites elliptiques, un concept également proposé par Arnold Sommerfeld. Debye profite en 1938 d'une proposition de conférence à l'Université de Cornell à Ithaca pour se rendre aux États-Unis et ensuite rester à l'Université de Cornell, où il devient professeur, puis pendant 10 ans est directeur du département de chimie. Il demeure à Cornell le reste de sa carrière. Il prend sa retraite en 1952, mais continue ses recherches jusqu'à sa mort.

\parpic[l][t]{%
  \begin{minipage}{40mm}
    \fbox{\includegraphics[width=110px,height=140px]{img/medaillons/descartes.eps}}
  \end{minipage}
}
\textbf{Descartes, René} (1596-1650) était un philosophe, scientifique et mathématicien français, fondateur du rationalisme moderne. Né à La Haye, d'un père conseiller au parlement de Rennes, Descartes reçut, de 1607 à 1614, l'enseignement, décisif pour lui, des pères jésuites du Collège royal de La Flèche. Cette expérience le conduisit à proposer une refondation des sciences, critiquant l'absence de fondement de l'enseignement professé. Il reçut une formation de juriste en 1616 puis entra dans la carrière militaire en 1618, entreprit des voyages, mêla vie scientifique et vie mondaine, avant de se consacrer pleinement à la philosophie. Il passa sa vie entre la France et les Pays-Bas, fuyant les villes, fréquentant les bibliothèques et rencontrant les esprits les plus illustres de son temps, notamment Bérulle, Fermat, Gassendi, Hobbes et Pascal. Il mourut d'une pneumonie à Stockholm, léguant à la postérité une oeuvre entourée de légendes et imprégnée d'un esprit nouveau.

\parpic[l][t]{%
  \begin{minipage}{40mm}
    \fbox{\includegraphics[width=110px,height=140px]{img/medaillons/dirac.eps}}
  \end{minipage}
}
\textbf{Dirac, Paul Adrien Maurice} (1902-1984) Né à Bristol, Dirac fait ses études aux Universités de Bristol et de Cambridge. En 1926, pour son doctorat (la première thèse au monde ayant pour sujet la mécanique quantique!), il introduit un formalisme général pour la physique quantique peu après Heisenberg mais indépendamment (il y retrouve la non-commutativité des opérateurs position et quantité de mouvement). En 1928, il élabore une théorie relativiste pour décrire les propriétés de l'électron. Celle-ci le conduit à postuler l'existence d'une particule identique à l'électron dans tous ses aspects mais de charge opposée, c'est-à-dire positive et devant s'annihiler en même temps que l'électron négatif lors d'une collision avec celui-ci. La théorie de Dirac est confirmée en 1932 quand le physicien Carl Anderson découvre le positron. Dirac contribue aussi, avec Fermi, au développement de la statistique dite de Fermi-Dirac, décrivant le comportement collectif des particules de spin demi-entier. En 1933, Dirac partage le prix Nobel de physique avec le physicien autrichien Erwin Schrödinger. En 1939, il devient membre de la Société royale. Il est professeur de mathématiques à Cambridge de 1932 à 1968, professeur de physique à l'Université d'État de Floride de 1971 jusqu'à sa mort, et membre de l'Institute of Advanced Studies (IAS) périodiquement entre 1934 et 1959.

\parpic[l][t]{%
  \begin{minipage}{40mm}
    \fbox{\includegraphics[width=110px,height=140px]{img/medaillons/dirichlet.eps}}
  \end{minipage}
}
\textbf{Dirichlet (-Lejeune), Peter Gustav} (1805-1859) Né à Düren (Allemagne), Dirichlet est un élève brillant, qui achève ses études secondaires à 16 ans. Devant la faible qualité des formations universitaires allemandes à cette époque, Dirichlet décide de partir étudier à Paris, emportant avec lui les\textit{ Disquisitiones Arithmeticae} de Gauss comme une bible. Dans la capitale française, sa situation personnelle est facilitée par le général Foy, un ancien grand général des campagnes napoléoniennes, dont il devient le précepteur des enfants, et qui se montrera bienveillant avec lui. Dirichlet rencontre alors quelques-uns des plus grands mathématiciens, dont Legendre, Poisson, Laplace et Fourier. Ce dernier surtout impressionnera beaucoup Dirichlet, et sera à l'origine de l'intérêt qu'il portera aux séries trigonométriques et à la physique-mathématique. C'est à Paris que Dirichlet rédige sa première contribution d'importance aux mathématiques, étant à l'initiative en 1825 de la preuve du cas $n=5$ dans le grand théorème de Fermat, preuve achevée par Legendre dans la foulée. Fin 1825, le général Foy décède, et Dirichlet décide de retourner en Allemagne. Il enseigne d'abord à l'Université de Breslau, au lycée militaire de Berlin, puis à l'Université de Berlin à partir de 1829, où il restera 27 ans durant. Parmi ses élèves, on retiendra les noms de Kronecker et Riemann. Dirichlet est décrit comme un bon professeur, mais non exempt de défauts. Il donne l'apparence de quelqu'un de sale, toujours affublé d'un cigare et d'un café, visiblement peu préoccupé de l'image qu'il donne. On dit aussi de lui qu'il était très souvent en retard. En 1848, son maître et ami Karl Jacobi est diagnostiqué comme étant malade du diabète. Dirichlet l'accompagne dans un voyage de 18 mois en Italie. De retour en Allemagne, Dirichlet commence à être lassé des lourdes charges d'enseignement qu'il doit assumer. À la mort de Gauss, il prend sa succession à Göttingen. C'est malheureusement pour peu de temps, car lui-même s'éteint en 1859 des suites d'un malaise cardiaque. L'éventail des travaux de Dirichlet illustre la profondeur de la culture mathématique allemande au début de son âge d'or. On lui doit le premier énoncé d'une condition suffisante de convergence d'une série de Fourier (dans le cas des fonctions continues par morceaux), le théorème de la progression arithmétique, le prolongement des fonctions harmoniques définies sur la frontière d'un ouvert et toute une classe d'équations aux dérivées partielles porte le nom de "problème de Dirichlet".  Nous lui devons aussi de très nombreuses contributions en arithmétique, où il existe le théorème des unités de Dirichlet, les séries de Dirichlet, etc.

\parpic[l][t]{%
  \begin{minipage}{40mm}
    \fbox{\includegraphics[width=110px,height=140px]{img/medaillons/doppler.eps}}
  \end{minipage}
}
\textbf{Doppler, Christian} (1803-1853) était un mathématicien et physicien autrichien, célèbre pour sa découverte de l'effet Doppler. Après avoir étudié à l'Université de Vienne, Doppler devient assistant-professeur dans cet établissement en 1829. Ce poste n'étant pas renouvelé, il envisage un temps une émigration vers les États-Unis. Il renonce à quitter son pays après avoir été nommé à Prague en 1837, puis à l'école polytechnique de Vienne en 1849. En 1850, il fonde l'Institut de Physique de l'Université de Vienne dont il est seul professeur et le premier directeur. Atteint d'une affection pulmonaire, la tuberculose, il quitte ses fonctions en 1852. Son travail scientifique est varié : optique, astronomie, électricité... Sa publication la plus célèbre a été présentée en 1842 à l'académie royale des sciences de Bohème et a pour titre \textit{Sur la lumière colorée des étoiles doubles et d'autres étoiles du ciel}, utilisant l'effet Doppler. Ses calculs étaient erronés, le décalage réel de la fréquence lumineuse étant trop faible pour pouvoir être détecté à l'époque. En 1846, Doppler publie une correction de son travail initial où il tient compte des vitesses relatives de la source de lumière et de l'observateur.

\parpic[l][t]{%
  \begin{minipage}{40mm}
    \fbox{\includegraphics[width=110px,height=140px]{img/medaillons/drude.eps}}
  \end{minipage}
}
\textbf{Drude, Paul Karl Ludwig} (1863-1906) était un physicien né à Braunschweig et décédé à Berlin. Drude a commencé ses études en mathématiques à l'Université de Göttingen, mais s'est ensuite dirigé vers la physique. Il y termine son doctorat en 1887 et rédige un mémoire portant sur la réflexion et la diffraction de la lumière dans les cristaux. En 1894 il est nommé professeur à l'Université de Leipzig. En 1900 il obtient le poste d'éditeur de la revue scientifique \textit{Annalen der Physik}. La même année, il développé un modèle (le modèle de Drude) expliquant les propriétés thermiques, électriques et optiques de la matière qui sera repris en 1933 par Arnold Sommerfeld et Hans Bethe et deviendra alors le modèle Drude-Sommerfeld. Il enseigne à l'Université de Giessen de 1901 à 1905 et est promu directeur du département de physique de l'Université de Berlin. En 1906 il devient membre de l'académie de Berlin.

\phantomsection
\addcontentsline{toc}{section}{E}
\label{sec:E}

\parpic[l][t]{%
  \begin{minipage}{40mm}
    \fbox{\includegraphics[width=110px,height=140px]{img/medaillons/einstein.eps}}
  \end{minipage}
}
\textbf{Einstein, Albert} (1879-1955) Né à Ulm et mort à Princeton c'était un physicien théoricien qui fut successivement allemand, puis apatride (1896), suisse (1901), et enfin sous la double nationalité helvético-américaine (1940). Il publie sa théorie de la Relativité Restreinte en 1905, et une théorie de la gravitation dite Relativité Générale en 1915. Il contribue largement au développement de la mécanique quantique et de la cosmologie, et reçoit le prix Nobel de physique de 1921 pour son explication de l'effet photoélectrique. Son travail est notamment connu pour l'équation d'équivalence qui établit une équivalence entre la matière et l'énergie d'un système. Il est aussi connu pour son hypothèse audacieuse sur la nature corpusculaire de la lumière. Mais il a également contribué au développement de nombre d'autres théories (physique quantique y comprise). En 1905, Einstein obtint son doctorat de l'Université de Zürich pour une thèse théorique sur les dimensions des molécules. Il publia également trois articles théoriques d'une importance capitale pour le développement de la physique du 20ème siècle. Dans le premier de ces articles, sur le mouvement brownien, il fit des prédictions importantes sur le mouvement des particules distribuées aléatoirement dans un fluide.  Pendant le reste de sa vie, Einstein consacra énormément de temps à généraliser encore plus sa théorie de la Relativité Générale. Il visait une théorie de champ unifié, qui ne fut pas complètement couronnée de succès, et fit de nombreuses tentatives pour décrire l'interaction électromagnétique et l'interaction gravitationnelle dans un modèle commun.

\parpic[l][t]{%
  \begin{minipage}{40mm}
    \fbox{\includegraphics[width=110px,height=140px]{img/medaillons/erdos.eps}}
  \end{minipage}
}
\textbf{Erdös, Paul} (1913-1996) était le plus prolifique des mathématiciens du 20ème siècle, avec environ 1'500 articles publiés (il faut remonter à Euler pour obtenir un tel volume). Plus que quelqu'un qui bâtissait des théories, il résolvait des problèmes, le plus souvent avec élégance et simplicité. Erdös est né à Budapest. Ses deux parents étaient professeurs de mathématiques dans le secondaire. Alors que Erdös était âgé d'à peine un an, son père fut fait prisonnier par les Russes et déporté en Sibérie. Ces événements ont contribué au développement d'une relation très forte mère/fils, qui influera beaucoup sur le cours de la vie de Paul Erdös. C'est à l'âge de 19 ans, alors qu'il commence ses études à l'université et qu'il se fait connaître des milieux mathématiques. Il publie en effet une nouvelle démonstration du postulat de Bertrand, qui affirme qu'il existe un nombre premier entre $n$ et $2n$, pour tout $n$. Deux ans plus tard, il obtient son doctorat (à 21 ans), puis s'en va faire un post-doc à Manchester. Comme Erdös est d'origine juive, il ne peut retourner en Hongrie à la fin des années 30, et il émigre aux États-Unis. Après quelques visites en Europe aux rescapés de sa famille après l'Holocauste, il a des problèmes aux États-Unis avec le MacCarthysme, et il se voit interdit de séjour sur le territoire américain. Erdös est donc contraint de poser ses valises en Israël. Avec ses 1'500 articles, les contributions de Erdös aux mathématiques sont nombreuses: en théorie des nombres, en combinatoire, en mathématiques discrètes, il fut un maître. Erdös avait une exceptionnelle aptitude à s'entourer des mathématiciens les plus compétents pour résoudre ses conjectures. Il en résulte que Erdös a eu beaucoup de collaborateurs: 500 mathématiciens environ ont écrit un article en commun avec lui. Les mathématiciens se sont amusés à définir un nombre de Erdös: tout mathématicien qui a publié un papier en commun avec Erdös a un nombre de Erdös égal à 1. Toute personne qui a publié un article en commun avec une personne qui a un nombre de Erdös égal à 1 a un nombre de Erdös égal à 2. Et ainsi de suite... Albert Einstein est l'un d'entre eux: son nombre de Erdös est 2.  Pourtant, parmi toutes ces collaborations, une au moins a mal tourné, et c'est d'autant plus regrettable qu'elle concerne le plus grand succès d'Erdös. À la fin du 19ème siècle, Hadamard et de La Vallée Poussin avaient démontré le théorème des nombres premiers, à savoir que le nombre de nombres premiers inférieurs ou égaux à $n$ est équivalent, quand $n$ est grand, à $n/\ln(n)$. Leur démonstration est particulièrement rude! En 1949, Atle Selberg trouve une inégalité qu'il pense pouvoir être une étape importante vers une démonstration élémentaire du théorème des nombres premiers. Elle est présentée à Erdös, qui trouve la clef manquante pour boucler la preuve. Un article coécrit de plus aurait sans doute été la solution la plus appropriée pour mesurer les apports de chacun. Mais, à la suite d'un malentendu lié à l'envoi de cartes postales triomphales d'Erdös, Selberg craint qu'Erdös ne tire la couverture à lui. Il publie seul une preuve complète. Il recevra la médaille Fields en 1950, alors qu'Erdös devra se contenter du prix Wolf en 1984. La vie d'Erdös fut vraiment étrange. Il n'avait pas de maison, pas d'épouse, les contingences matérielles étaient pénibles pour lui. Il voyageait en solitaire, accompagné de deux valises qui portaient toutes ses affaires, allant d'université en université, habitant à l'hôtel ou chez un ami mathématicien... Il est par ailleurs l'auteur de nombreux "erdosismes", comme cette phrase célèbre: "un mathématicien est une machine à transformer le café en théorème". Lui-même dopé à toutes sortes d'amphétamines! Jusqu'à la fin de sa vie, Erdös ne ralentira pas son activité mathématique. Mourir signifiait pour lui arrêter de faire des mathématiques. Il décède à Varsovie, en plein congrès.

\parpic[l][t]{%
  \begin{minipage}{40mm}
    \fbox{\includegraphics[width=110px,height=140px]{img/medaillons/erlang.eps}}
  \end{minipage}
}
\textbf{Erlang, Agner Krarup} (1878-1979) était un mathématicien danois ayant beaucoup travaillé sur la théorie des files d'attente, et la gestion des réseaux téléphoniques. Erlang s'est attelé, sur la base notamment des travaux de Poisson dont la loi des événements rares a trouvé toute sa dimension appliquée aux réseaux de télécommunications, à l'élaboration d'un modèle mathématique pour le dimensionnement des réseaux télécoms sur une approche statistique afin de parvenir à des coûts d'exploitation de nature à permettre un marché de masse.\\\\\\

\parpic[l][t]{%
  \begin{minipage}{40mm}
    \fbox{\includegraphics[width=110px,height=140px]{img/medaillons/euclide.eps}}
  \end{minipage}
}
\textbf{Euclide} (3ème siècle av. J.-C.) On ne sait que très peu de choses sur la vie d'Euclide. Il semble qu'il ait enseigné la mathématique à Alexandrie à la demande de Ptolémée Ier. Il apparaîtrait donc comme le fondateur de la célèbre École d'Alexandrie qui influença les travaux d'Archimède. En revanche, les théories d'Euclide sont connues et constituent une référence dans l'histoire des mathématiques. L'oeuvre maîtresse d'Euclide est incontestablement les \textit{Éléments}. Cet ouvrage représente une synthèse remarquable de résultats mathématiques et a marqué de son empreinte la discipline tout entière. Il est composé de treize livres. Les quatre premiers traitent de géométrie dans le plan avec les définitions du point, de la droite et de la surface. Ils exposent également le calcul d'aires de différents polygones. Le livre V contient les premières notions d'analyse. Le sixième aborde la similitude des figures et donne la résolution des équations du second degré à l'aide de constructions géométriques. Les livres VII, VIII, et IX portent sur l'arithmétique. Le X étudie les nombres irrationnels et enfin les trois derniers (XI, XII, XIII) abordent la géométrie dans l'espace. Euclide a, en outre, rédigé des ouvrages sur l'analyse géométrique, l'optique et l'astronomie. Représentation parfaite de l'exposé scientifique, les \textit{Éléments} sont composés de différentes propositions classées en deux groupes: les hypothèses et les axiomes. Parmi les 5 axiomes, on trouve le célèbre postulat d'Euclide: "par tout point du plan passe une et une seule droite parallèle à une autre droite." Cet axiome constitue le fondement de la géométrie euclidienne, en opposition aux géométries non-euclidiennes apparues quelque 2000 ans plus tard.

\parpic[l][t]{%
  \begin{minipage}{40mm}
    \fbox{\includegraphics[width=110px,height=140px]{img/medaillons/euler.eps}}
  \end{minipage}
}
\textbf{Euler, Leonhard} (1707-1783) était un mathématicien suisse, physicien, ingénieur et philosophe, et l'un des fondateurs des méthodes de calcul différentiel et intégral. Leonhard Euler naquit à Bâle, de Paul Euler, pasteur des Églises réformées et de Marguerite Brucker, fille de pasteur. Il eut deux jeunes sœurs du nom d'Anna Maria et de Maria Magdalena. Peu de temps après la naissance de Leonhard, la famille Euler déménagea de Bâle pour rejoindre la ville de Riehen, où Euler passa la plupart de son enfance. Paul Euler était un ami de la famille Bernoulli. Jean Bernoulli, alors considéré comme le principal mathématicien européen, pourrait être celui ayant eu la plus grande influence sur le jeune Leonhard. L'éducation officielle d'Euler commença tôt à Bâle, où il fut envoyé vivre avec sa grand-mère maternelle. À l'âge de 13 ans, il s'inscrivit à l'Université de Bâle, et en 1723, obtint son Master of Philosophy grâce à une dissertation qui comparait la philosophie de Descartes à celle de Newton. À cette époque, il recevait tous les samedis après-midi des leçons de Jean Bernoulli, qui découvrit rapidement chez son nouvel élève un incroyable talent pour les mathématiques. Euler commença alors à étudier la théologie, le grec et l'hébreu à la demande de son père, afin de devenir un pasteur, mais Jean Bernoulli convainquit Paul Euler que Leonhard était destiné à devenir un grand mathématicien. Euler fut le premier à traiter de manière analytique et complète l'algèbre, la théorie des équations, la trigonométrie et la géométrie analytique. Dans ce travail, il traita le sujet du développement des séries de fonctions et formula la règle selon laquelle seules les séries infinies convergentes pouvaient être correctement évaluées. Il discuta aussi des surfaces à trois dimensions et prouva que les sections coniques sont représentées par l'équation générale du second degré à deux dimensions. D'autres travaux traitent du calcul, dont le calcul des variations, la théorie des nombres, les nombres imaginaires et transcendants, l'algèbre déterminée et indéterminée, la théorie des graphes. Euler apporta ses contributions dans les domaines de l'astronomie, de la mécanique analytique (calcul variationnel), l'hydrodynamique, l'optique et l'acoustique. Euler est considéré comme un éminent mathématicien du 18ème siècle et l'un des plus grands et des plus prolifiques de tous les temps et a introduit (ou a contribué à introduire ou à rendre d'usage) une grande partie des notations encore utilisées en ce début de 21ème siècle (symboles pour la somme, fonction, logarithme, exponentiel, etc.).

\phantomsection
\addcontentsline{toc}{section}{F}
\label{sec:F}

\parpic[l][t]{%
  \begin{minipage}{40mm}
    \fbox{\includegraphics[width=110px,height=140px]{img/medaillons/faraday.eps}}
  \end{minipage}
}
\textbf{Faraday, Michael} (1791-1867) était un scientifique anglais qui a contribué aux domaines de l'électromagnétisme et l'électrochimie. Le jeune Faraday, qui était le troisième des 4 enfants, n'ayant que l'éducation scolaire la plus élémentaire, a dû se former en autodidacte. À 14 ans, il devient apprenti chez un relieur local. Au cours de ses 7 années d'apprentissage il a pu lire beaucoup de livres. A cette époque, il a également développé un intérêt pour la science, en particulier pour l'électricité. Faraday a été particulièrement inspiré par les livre \textit{Conversations on Chemistry} de Jane Marcet. Ses principales découvertes comprennent l'induction électromagnétique, le diamagnétisme et l'électrolyse. C'est par ses recherches sur le champ magnétique autour d'un conducteur parcouru par un courant continu que Faraday a établi la base pour le concept de champ électromagnétique en physique. Faraday a également établi que le magnétisme pourrait affecter rayons de lumière et qu'il y avait une relation sous-jacente entre les deux phénomènes. Il a découvert le principe même de l'induction électromagnétique, en même temps que Joseph Henry, diamagnétisme, et les lois de l'électrolyse. Ses inventions de dispositifs électromagnétiques rotatifs ont formé la base de la technologie des moteurs électriques, et c'est en grande partie grâce à ses efforts que l'électricité est devenue pratique pour une utilisation dans la technologie.

\parpic[l][t]{%
  \begin{minipage}{40mm}
    \fbox{\includegraphics[width=110px,height=140px]{img/medaillons/feigenbaum.eps}}
  \end{minipage}
}
\textbf{Feigenbaum, Mitchell} (1994-) est né à New York, d'immigrants polonais et ukrainiens. Il a fréquenté l'école secondaire Samuel J. Tilden, à Brooklyn, New York, et le City College de New York. En 1964, il a commencé ses études supérieures au Massachusetts Institute of Technology (M.I.T.). En s'inscrivant à des études supérieures en génie électrique, il a changé son domaine à la physique. Il a terminé son doctorat en 1970 pour une thèse sur les relations de dispersion. Après des courtes positions à l'Université de Cornell et au Virginia Polytechnic Institute et à l'université d'État, il s'est vu offrir un poste à plus long terme au Laboratoire National de Los Alamos au Nouveau-Mexique pour étudier la turbulence dans les fluides. Bien que ce groupe de chercheurs ait finalement été incapable de démêler la théorie actuellement intraitable des fluides turbulents, ses recherches l'ont amené à étudier des mappages chaotiques. Certains mappages mathématiques impliquant un seul paramètre linéaire présentent le comportement apparemment aléatoire appelé "chaos" lorsque le paramètre se situe dans certaines plages. Lorsque le paramètre est augmenté vers cette région, le mappage subit des bifurcations à des valeurs précises du paramètre. Au début, il y a un point stable, puis bifurquant vers une oscillation entre deux valeurs, puis bifurquant à nouveau pour osciller entre quatre valeurs et ainsi de suite. En 1975, le Dr Feigenbaum, utilisant la petite calculatrice HP-65 qui lui avait été délivrée, a découvert que le rapport de la différence entre les valeurs auxquelles se produisent de telles bifurcations successives de doublement de période tend vers une constante d'environ $4.66692$. Il a été capable de fournir une preuve mathématique de ce fait, et il a ensuite montré que le même comportement, avec la même constante mathématique, se produirait dans une large classe de fonctions mathématiques, avant le début du chaos. Pour la première fois, ce résultat universel a permis aux mathématiciens de faire leurs premiers pas pour démêler le comportement aléatoire apparemment insoluble des systèmes chaotiques. Ce "rapport de convergence" est maintenant connu comme la première "constante de Feigenbaum".

\parpic[l][t]{%
  \begin{minipage}{40mm}
    \fbox{\includegraphics[width=110px,height=140px]{img/medaillons/fermat.eps}}
  \end{minipage}
}
\textbf{Fermat, Pierre de} (1601-1665) était un mathématicien français, surnommé le "prince des mathématiques", auteur d'un célèbre théorème sans démonstration en arithmétique (grand théorème de Fermat). Il fut à l'origine du "principe de Fermat" (optique) et avec son ami Blaise Pascal de celui du calcul des probabilités. Il créa également la théorie des nombres et fit dans ce domaine différentes découvertes. Ainsi, certains le considèrent comme le père de la théorie moderne. Il devança le calcul différentiel par ses travaux sur le calcul infinitésimal. Il laissa à la postérité le soin de démontrer un théorème (le fameux "grand théorème de Fermat" déjà mentionné précédemment) sur lequel les mathématiciens se sont acharnés pendant plus de trois siècles. Ce n'est qu'en 1993 que le chercheur britannique Andrew Wiles en proposa une démonstration.

\parpic[l][t]{%
  \begin{minipage}{40mm}
    \fbox{\includegraphics[width=110px,height=140px]{img/medaillons/fermi.eps}}
  \end{minipage}
}
\textbf{Fermi, Enrico} (1901-1954) était un physicien italien connu pour la réalisation de la première réaction nucléaire contrôlée. Très jeune, Enrico Fermi fait preuve d'une mémoire exceptionnelle et d'une grande intelligence, qui lui permettent d'exceller dans les études. Enrico, profondément marqué par le décès d'un de ses très jeune frère, se jette alors dans l'étude de la physique pour surmonter sa douleur. Bon élève, il se passionne très vite pour la physique et les mathématiques et commence à étudier divers ouvrages qu'il achète et qui traitent de mécanique, d'optique, d'astronomie et d'acoustique. Un ami de son père, l'ingénieur Adolfo Amidei, qui prend conscience des qualités hors du commun du jeune Fermi, lui prête divers ouvrages traitant de mathématiques. Ainsi, à 17 ans, Enrico Fermi maîtrise la géométrie analytique, la géométrie projective, le calcul infinitésimal, le calcul intégral et la mécanique rationnelle. À partir de 1918, Fermi étudie à l'Université de Pise au sein de l'École normale supérieure de Pise. Comme à son habitude, il étudie seul divers problèmes de physique mathématique et consulte des ouvrages de Poincaré, de Poisson ou d'Appell. À partir de 1919, il s'intéresse aux nouvelles théories comme la relativité ou la physique atomique, ainsi il acquiert une grande connaissance de théories telles que la relativité restreinte, la théorie du corps noir ou encore le modèle de l'hydrogène de Bohr. Ainsi Enrico Fermi, le seul à l'université au fait de ces théories, en arrive, sur l'insistance de ses professeurs, à donner des conférences où il expose aux professeurs et aux assistants les dernières découvertes de physique atomique. En 1922, après 4 ans passés à l'université, Enrico Fermi publie son premier article qui traite de la Relativité Générale. Dans une communauté scientifique italienne hostile aux travaux d'Einstein, il est l'un des rares avec Levi-Civita à défendre la théorie de la relativité. En 1922, Fermi obtient son diplôme de fin d'études après avoir présenté un mémoire sur la diffraction des rayons X. Il fréquente ensuite divers physiciens de haut rang dans l'Italie de l'époque, avant de devenir, pendant 2 ans, conférencier à l'Université de Florence. En 1926, il devient professeur de physique théorique à l'Université La Sapienza de Rome. C'est durant cette période qu'il développe la théorie statistique quantique que l'on appellera plus tard la "statistique de Fermi-Dirac". À partir de 1932, il se tourne plus précisément vers la physique nucléaire, et c'est cette même année qu'il rédige un article sur la radioactivité bêta. En 1934, il développe sa théorie sur l'émission de rayonnement bêta en y incluant le neutron postulé en 1930 par Wolfgang Pauli, qu'il rebaptise neutrino (le nom neutron étant déjà utilisé pour une autre particule), et s'oriente vers la création d'isotopes radioactifs artificiels par bombardement de neutrons lents (raison pour laquelle il reçut le prix Nobel en 1938).

\parpic[l][t]{%
  \begin{minipage}{40mm}
    \fbox{\includegraphics[width=110px,height=140px]{img/medaillons/feynman.eps}}
  \end{minipage}
}
\textbf{Feynman, Richard Phillips} (1918-1988) était né à Far Rockaway dans le Queens, quartier de New York (États-Unis) de parents d'origine polonaise et russe. Son père, qui l'encourageait à poser des questions et à remettre en cause les choses communément admises, l'a durablement influencé. De sa mère, il tient un solide sens de l'humour qui ne l'a jamais quitté. Feynman est l'un des physiciens les plus influents de la seconde moitié du 20ème siècle, en raison notamment de ses travaux sur l'électrodynamique quantique relativiste, les quarks et l'hélium superfluide. Durant sa dernière année à l'école secondaire de Far Rockaway, Feynman remporta le championnat de mathématiques de l'Université de New York. Il reçut donc une bourse pour étudier au Massachusetts Institute of Technology (M.I.T.) où il reçut son baccalauréat en 1939 après s'être orienté d'abord en électronique, puis en mathématiques, et enfin avoir assisté à tous les cours de physique offerts y compris pendant sa seconde année un cours de physique théorique réservé aux étudiants de maîtrise. Feynman obtient un score remarquable aux examens d'entrée de l'Université de Princeton en mathématiques et en physique, mais il eut une note très faible dans la partie littéraire de l'examen. Durant ses études à l'Institute for Advanced Study de Princeton (IAS) (créé depuis peu et dirigé par Albert Einstein), Feynman travailla sous la direction de John Wheeler sur le principe de moindre action appliqué à la mécanique quantique. Il établit ici les bases des diagrammes de Feynman et de l'approche de la mécanique quantique par les intégrales de chemin. Il obtint son doctorat en 1942. Il reformula entièrement la mécanique quantique à l'aide de son intégrale de chemin qui généralise le principe de moindre action de la mécanique classique et inventa les diagrammes qui portent son nom et qui sont désormais largement utilisés en théorie quantique des champs (dont l'électrodynamique quantique fait partie). Musicien, pédagogue remarquable, rédacteur de nombreux ouvrages de vulgarisation, il a aussi été impliqué dans le développement de la bombe atomique américaine. Après la seconde guerre mondiale, il enseigna à l'Université de Cornell puis au Caltech où il effectua des travaux fondamentaux notamment dans la théorie de la superfluidité et des quarks. Sin-Itiro Tomonaga, Julian Schwinger et lui sont colauréats du prix Nobel de physique de 1965 pour leurs travaux en électrodynamique quantique.

\parpic[l][t]{%
  \begin{minipage}{40mm}
    \fbox{\includegraphics[width=110px,height=140px]{img/medaillons/fisher.eps}}
  \end{minipage}
}
\textbf{Fisher, Ronald Aymler} (1890-1962) Né à Londres était un biologiste et statisticien britannique, qui a énormément contribué à fonder les statistiques modernes. Ses travaux sur les statistiques lui valurent la médaille Darwin en 1948, la médaille Copley en 1955 et la médaille d'argent Darwin-Wallace en 1958. Dans le domaine des statistiques, il a introduit de nombreux concepts clés tels que le maximum de vraisemblance, l'information de Fisher et l'analyse de la variance (ANOVA). Il est considéré comme un grand précurseur de Shannon. Il est également un des fondateurs de la génétique moderne et un grand continuateur de Darwin, en particulier grâce à son utilisation des méthodes statistiques, incontournables dans la génétique des populations. Il a ainsi contribué à la formalisation mathématique du principe de sélection naturelle. Il est d'abord attiré par la physique et obtient en 1912 une licence d'astronomie à l'Université de Cambridge. De 1915 à 1919, il enseigne les mathématiques à Londres dans des écoles privées. En 1919, il est engagé à la station expérimentale de Rothamsted pour analyser l'effet des précipitations sur le rendement du blé où il travaille jusqu'en 1933. Dans son article de 1922 \textit{On the mathematical foundations of theoritical statistics}, il définit une quinzaine de notions fondamentales en statistiques dont la notion de convergence, d'efficacité, de vraisemblance et de statistique suffisante. Il propose l'estimateur du maximum de vraisemblance en 1922 après avoir présenté une première version en 1912. Il introduit aussi en 1924 l'analyse de la variance. En 1925 il publie des innovations en séries temporelles et en analyse des corrélations multiples.

\parpic[l][t]{%
  \begin{minipage}{40mm}
    \fbox{\includegraphics[width=110px,height=140px]{img/medaillons/foucault.eps}}
  \end{minipage}
}
\textbf{Foucault, Leon} (1819-1868) était un physicien français célèbre pour sa démonstration du mouvement de la Terre par la rotation du plan d'oscillation du pendule. Né à Paris, il travailla avec le physicien français Armand Fizeau sur la détermination de la vitesse de la lumière. Foucault prouva, de façon indépendante, que la vitesse de la lumière dans l'air était plus élevée que dans l'eau. En 1851, il fit une démonstration spectaculaire de la rotation de la Terre en suspendant un pendule à un long câble attaché à la coupole du Panthéon à Paris. Le mouvement du pendule mis en évidence la rotation de la Terre sur son axe. En 1855 il découvre que la force nécessaire à la rotation d'un disque de cuivre augmente quand il doit tourner avec sa jante entre les pôles d'un aimant, le disque chauffant dans le même temps du fait des "courants de Foucault" induits dans le métal. Il conçut également une méthode de mesure de la courbure des miroirs de télescopes. Il développa d'autres instruments dont un prisme polarisateur et une forme de gyroscope qui est à la base du gyrocompas moderne.

\parpic[l][t]{%
  \begin{minipage}{40mm}
    \fbox{\includegraphics[width=110px,height=140px]{img/medaillons/fourier.eps}}
  \end{minipage}
}
\textbf{Fourier, Joseph} (1768-1830) était un physicien et mathématicien français connu pour la découverte des séries trigonométriques et des transformées qui portent son nom. Fourier est orphelin de père et de mère à 10 ans. L'organiste d'Auxerre, Joseph Pallais, le fait entrer dans le pensionnat qu'il dirige. Recommandé par l'évêque d'Auxerre, il fait ses études à l'école militaire d'Auxerre tenue alors par les Bénédictins de la Congrégation de Saint-Maur. Destiné à l'état monastique, il préfère s'adonner aux sciences pour lesquelles il remporte la plupart des premiers prix. Élève brillant, il y est promu professeur dès l'âge de 16 ans et peut dès lors commencer ses recherches personnelles. Il intègre l'école normale supérieure à 26 ans, où il a entre autres comme professeurs Joseph-Louis Lagrange, Gaspard Monge et Pierre-Simon de Laplace, auquel il succède à la chaire à Polytechnique en 1797. Fourier a contribué à la résolution numérique des équations et à la diffusion de la chaleur dont une des lois porte son nom. Ses travaux ont une implication directe dans la convergence des séries et leur somme infinie. Il participa, avec Monge, à la campagne d'Égypte en tant qu'observateur scientifique. Anobli sous Napoléon, il fut professeur à l'école polytechnique, secrétaire de l'institut d'Égypte et préfet de l'Isère. Il fut aussi élu à l'académie des sciences et à l'académie française. On le considère comme l'un des fondateurs, avec le français Poisson et le suisse Daniel Bernoulli, de ce que l'on appelle aujourd'hui la "physique-mathématique".

\parpic[l][t]{%
  \begin{minipage}{40mm}
    \fbox{\includegraphics[width=110px,height=140px]{img/medaillons/fraunhofer.eps}}
  \end{minipage}
}
\textbf{Fraunhofer, Joseph von} (1787-1826) était un opticien et physicien allemand, né à Straubing. Fraunhofer apporta de nombreuses améliorations à la fabrication du verre optique, au meulage et au polissage des lentilles et à la construction des télescopes et d'autres instruments d'optique. Joseph Fraunhofer était le onzième enfant d'un souffleur de verre. Il avait 11 ans à la mort de ses parents, aussi son tuteur l'envoya-t-il à Münich en apprentissage pour 6 ans afin qu'il apprenne la miroiterie. C'est là, qu'en 1801, il faillit trouver la mort dans l'effondrement de l'atelier. À la fin de son apprentissage en 1806, il eut la possibilité de poursuivre une formation d'opticien dans l'institut de mécanique Reichenbach. Les ateliers furent transférés en 1807 à Benediktbeuern, et Fraunhofer y fut nommé contremaître. Là, il mit au point de nouvelles machines à polir les miroirs et de nouveaux types de verres optiques (le verre flint achromatique), qui apportèrent une amélioration décisive à la qualité des lentilles. Fraunhofer inventa aussi de nombreux instruments scientifiques. Son nom est associé à des lignes fixes et noires dans le spectre solaire, appelées les "lignes Fraunhofer", qu'il fut le premier à décrire en détail. Ses recherches dans le domaine de la réfraction et de la dispersion de la lumière aboutirent à l'invention du spectroscope et au développement de la spectroscopie.

\parpic[l][t]{%
  \begin{minipage}{40mm}
    \fbox{\includegraphics[width=110px,height=140px]{img/medaillons/fresnel.eps}}
  \end{minipage}
}
\textbf{Fresnel, Augustin Jean} (1788-1827) était un physicien français, fondateur de l'optique moderne. Il proposa une explication de tous les phénomènes optiques dans le cadre de la théorie ondulatoire de la lumière. Il commença par réaliser de nombreuses expériences sur les interférences lumineuses, pour lesquelles il forgea la notion de longueur d'onde, et calcula les intégrales dites "Intégrales de Fresnel". Il fut le premier à prouver que deux faisceaux de lumière polarisés dans des plans différents n'ont aucun effet d'interférence. Il déduisit très justement de cette expérience que le mouvement ondulatoire de la lumière polarisée est transversal et non longitudinal (comme celui du son) ainsi qu'on le croyait avant lui. En outre, il fut le premier à produire une lumière polarisée circulaire. Pour expliquer la propagation des ondes lumineuses, Fresnel eut recours à la notion d'éther, malheureusement contradictoire avec d'autres expériences. Cette théorie sera abandonnée avec la relativité, mais les relations dites "relations de Fresnel" sur la réfraction sont toujours utilisées. Dans le domaine de l'optique appliquée, Fresnel conçut la lentille à échelons utilisée pour accroître le pouvoir éclairant des phares. De son vivant, les travaux scientifiques de Fresnel n'étaient connus que d'un petit groupe de scientifiques et certains de ses articles ne furent publiés qu'après sa mort.

\phantomsection
\addcontentsline{toc}{section}{G}
\label{sec:G}

\parpic[l][t]{%
  \begin{minipage}{40mm}
    \fbox{\includegraphics[width=110px,height=140px]{img/medaillons/galilee.eps}}
  \end{minipage}
}
\textbf{Galileo, Galilei} (1564-1642) était un physicien et astronome italien né à Pise à l'origine de la révolution scientifique du 17ème siècle. Ses théories ainsi que celles de l'astronome allemand Johannes Kepler servirent de fondement aux travaux du physicien britannique sir Isaac Newton sur la loi de l'attraction universelle. Sa principale contribution à l'astronomie fut l'amélioration considérable (quand la technique marchait...) de la lunette astronomique (ce qui lui a permis de procéder à des observations qui ont bouleversé la discipline) et la découverte des taches solaires, des montagnes et des vallées lunaires, des quatre plus grands satellites de Jupiter et des phases de Vénus. En physique, il découvrit la loi de la chute des corps et les mouvements paraboliques des projectiles. Ses études sur les oscillations du pendule pesant l'ont amené à inventer le pulsomètre. Cet appareil permettait d'aider à la mesure du pouls et fournissait un étalon de temps, qui n'existait pas à l'époque. Il débute aussi ses études sur la chute des corps. Dans l'histoire de la culture, Galilée est le symbole de la bataille livrée contre les autorités religieuses pour la liberté de la recherche (il avait cependant très bonne réputation et de très bonnes relations auprès des instances religieuses ce qui a aidé...). Dans le domaine des mathématiques et de la physique, il a contribué à faire avancer les connaissances relatives à la cinématique et la dynamique, jetant ainsi les fondements des sciences mécaniques. Il est de ce fait considéré comme le fondateur de la physique moderne.

\parpic[l][t]{%
  \begin{minipage}{40mm}
    \fbox{\includegraphics[width=110px,height=140px]{img/medaillons/galois.eps}}
  \end{minipage}
}
\textbf{Galois, Evariste} (1811-1832) était un mathématicien français, qui a donné son nom à une branche des mathématiques: la "théorie de Galois". Sa vie est tellement mythique qu'il est parfois difficile de démêler le mythe et la réalité. Dès 1827-1828, la fureur des mathématiques domine. Galois lit Legendre, Lagrange, Euler, Gauss, Jacobi. Le professeur, Louis-Paul-Émile Richard, admire le génie mathématique de son élève et garde les copies qu'il confiera à un autre de ses élèves: Charles Hermite. C'est l'époque où il publie son premier article dans les\textit{Annales mathématiques} de Joseph Gergonne (il démontre un théorème sur les fractions continues périodiques). Il rédige aussi un premier mémoire sur la théorie des équations, envoyé à l'académie des Sciences, perdu par Cauchy. Il échoue au concours d'entrée à Polytechnique. On raconte qu'il a jeté le chiffon à effacer la craie à la tête de son examinateur devant la stupidité des questions posées. Sur les conseils de son professeur, Galois entre à l'école préparatoire (future école Normale). Il rédige le résultat de ses recherches dans un mémoire - \textit{Conditions pour qu'une équation soit résoluble par radicaux} - afin de concourir au grand prix de mathématiques de l'académie des Sciences. Fourier emporte le manuscrit chez lui et meurt peu après: le manuscrit est perdu, et le grand prix est décerné à Abel (mort l'année précédente), et à Jacobi. Pour des raisons politiques, Galois se retrouve en prison, où il y continue ses travaux. Libéré en 1832, il s'éprend en 1832 d'une femme, avec qui il rompt la même année. On ne sait trop pourquoi, mais un duel semble en résulter quelques jours plus tard. La nuit précédente, le 29 mai, Galois rassemble ses dernières découvertes dans une splendide lettre adressée à son ami Auguste Chevalier. De cette lettre naquit la légende selon laquelle Galois fit ses découvertes majeures en une seule nuit, pris par la fièvre de la mort. La matinée du 30 mai, Galois, abandonné, grièvement blessé, est relevé par un paysan et conduit à l'hôpital Cochin. Il meurt le 31 mai 1832 dans les bras de son jeune frère est est enterré dans la fosse commune du cimetière de Montparnasse. Les travaux de Galois sont redécouverts une dizaine d'années plus tard par Liouville, qui en 1843 annonce à l'académie des Sciences qu'il vient de trouver dans les papiers de Galois une solution aussi exacte que profonde au problème de la résolubilité par radicaux. Ce n'est qu'en 1846 qu'il publie les textes sans y joindre de commentaires. À partir de 1850, les écrits de Galois sont enfin accessibles par les meilleurs mathématiciens.

\parpic[l][t]{%
  \begin{minipage}{40mm}
    \fbox{\includegraphics[width=110px,height=140px]{img/medaillons/gamow.eps}}
  \end{minipage}
}
\textbf{Gamow, George} (1904-1968) était un physicien théorique, astronome, cosmologiste et auteur/vulgarisateur scientifique américano-russe né à Odessa en Ukraine. Gamow vient en 1928 à Göttingen, où il utilise la physique quantique pour faire une théorie quantique de la radioactivité alpha. Deux mois plus tard, il rejoint Niels Bohr à Copenhague. Il émet l'idée d'un noyau atomique se comportant comme un fluide nucléaire, modèle repris presque une décennie plus tard par Bohr. En 1929, il obtient une nouvelle bourse et il rejoint Ernest Rutherford à l'Université de Cambridge. Il développe l'idée de l'effet tunnel afin de faire interagir des protons pour obtenir des noyaux de numéro atomique plus élevé. Il y rencontre John Cockcroft, qui construit peu après le premier accélérateur de particules, parvenant ainsi à valider le modèle de Gamow en réussissant une transmutation du lithium. Professeur à Washington en 1934, Gamow collabore avec Edward Teller pour formuler la théorie de l'émission bêta (1936). S'intéressant ensuite à l'astrophysique, Gamow et Teller donnent un modèle de la structure interne des étoiles géantes rouges (1942). En 1954, c'est vers la biochimie qu'il se tourne, proposant le concept de code génétique déterminé par l'ordre des composants de l'ADN. En 1956, il est nommé professeur de physique à Boulder (Colorado).

\parpic[l][t]{%
  \begin{minipage}{40mm}
    \fbox{\includegraphics[width=110px,height=140px]{img/medaillons/gauss.eps}}
  \end{minipage}
}
\textbf{Gauss, Carl Friedrich} (1777-1855) était un mathématicien allemand, qui a apporté des contributions essentielles à la plupart des branches des sciences exactes et appliquées. À l'âge de 17 ans, il essaya de trouver une solution au problème classique de construction d'un polygone à sept côtés, à la règle et au compas. Il réussit à prouver l'impossibilité de cette construction et poursuivit sa démarche en donnant des méthodes de construction de polygones à 17, 257, et 65'537 côtés. Plus généralement, il prouva que la construction, à la règle et au compas, d'un polygone régulier à nombre impair de côtés n'était possible que si le nombre de côtés est un des nombres premiers 3, 5, 17, 257, et 65'537, ou un produit de ces nombres. Pour sa thèse de doctorat, il démontra que toute équation algébrique a au moins une racine. Ce théorème, dont la démonstration avait résisté aux mathématiciens les plus célèbres, est encore appelé le "théorème fondamental de l'algèbre" ou "théorème de d'Alembert-Gauss". Gauss tourna ensuite son attention vers le domaine de l'astronomie pour laquel il élabora également une nouvelle méthode de calcul des orbites des corps célestes, en développant une théorie des erreurs d'observation connue sous le nom de "méthode des moindres carrés". En probabilités, son nom est attaché à la loi Normale (dite aussi "loi de Laplace-Gauss"), dont la répartition est décrite par la fameuse courbe en cloche ou courbe de Gauss. On lui doit aussi des travaux en géodésie. Avec le physicien allemand Wilhelm Eduard Weber, Gauss fit, à partir de 1831, des recherches approfondies dans le domaine du magnétisme et de l'électricité. Il fit aussi des recherches en optique, en particulier sur les systèmes de lentilles. Pour revenir aux mathématiques, il fut le premier, en étudiant la série hypergéométrique, à donner des conditions rigoureuses de convergence d'une série. Il étudia des généralisations fructueuses de la loi de réciprocité quadratique et dégagea leurs liens avec la théorie des fonctions elliptiques. Son mémoire de 1828 sur la théorie intrinsèque des surfaces fut le point de départ d'une théorie générale des espaces courbes (travaux de Riemann et de ses successeurs). Signalons aussi l'étude arithmétique des entiers de Gauss (de la forme $a+\mathrm{i}b$) qui repose sur une présentation géométrique des nombres complexes comme points du plan.

\parpic[l][t]{%
  \begin{minipage}{40mm}
    \fbox{\includegraphics[width=110px,height=140px]{img/medaillons/gibbs.eps}}
  \end{minipage}
}
\textbf{Gibbs, Josiah Willard} (1839-1903) était un physicien et mathématicien né et décédé à New Haven dans le Connecticut (après y avoir passé presque toute son existence en célibataire). Issu d'une famille de lettrés, il poursuit des études de latin et de physique, puis il entreprend une carrière de professeur de physique-mathématique au Yale College. Il séjourne successivement à Paris, à Berlin où il suit les leçons de Heinrich Gustav Magnus et à Heidelberg où il rencontre Gustav Kirchhoff et Herman Ludwig Helmholtz. Il laisse le souvenir d'un savant d'une modestie proverbiale et d'une extraordinaire puissance d'investigation scientifique. Son oeuvre remarquablement compacte fut d'abord peu connue. Aujourd'hui, elle est considérée comme un monument au sein des contributions scientifiques du 19ème siècle. Les deux principales publications datent de 1877 et de 1902. La première s'intitule \textit{On the Equilibrium of Heterogeneous Substances} et est comparée, en importance, à la chimie pondérale créée par Antoine Laurent Lavoisier. La seconde, jugée plus originale encore, est intitulée \textit{Elementary Principles in Statistical Mechanics}, et est comparée, pour son génie, à la mécanique analytique de Joseph Louis Lagrange. Bien que les exposés de Gibbs se distinguent par une exceptionnelle clarté, et la façon dont l'idée essentielle y est toujours soigneusement dégagée, le premier des deux mémoires n'a guère retenu tout d'abord l'attention des chimistes de son époque, peu accoutumés au langage rigoureux des sciences exactes. La richesse des méthodes thermodynamiques sur lesquelles il s'appuie en a fait cependant une base unifiée de la théorie physico-chimique des états d'équilibre et de leur stabilité. La plupart des lois qui se rapportent à cette discipline, et qui portèrent d'abord d'autres noms, furent redécouvertes ultérieurement au sein de ce premier mémoire. Il en est ainsi, par exemple, de la loi des phases donnant la variance des systèmes en équilibre, longtemps attribuée à Bakkuis Roozeboom (également des lois dites "loi de Van't Hoff" et aussi "loi de Le Chatelier"), relatives aux déplacements d'équilibre à température constante et à pression constante. Il en est encore de même, des critères de stabilité de l'équilibre, dont le théorème de modération dit "théorème de Braun et Le Chatelier". En bref, la plupart des propriétés qui relèvent à présent de la thermodynamique chimique des états d'équilibre, telles que la pression osmotique, l'influence de la tension superficielle, celle des déformations élastiques, la loi relative à l'entropie des mélanges gazeux et le paradoxe de Gibbs associé, ont ce même mémoire pour origine. Gibbs conduit à développer, dans deux communications antérieures à la précédente, un exposé complet des diagrammes et des surfaces thermodynamiques qui contribua largement à la diffusion de leur emploi auprès des praticiens. La théorie de Gibbs utilise pour la première fois la notion d'ensemble ainsi que la distinction entre un ensemble canonique et un ensemble microcanonique de même qu'entre un grand et un petit ensemble. Elle introduit aussi le concept d'espace des phases, caractérisé par les coordonnées et les quantités de mouvement de chaque élément. Elle établit, à partir de l'équation de Liouville, la loi de conservation de l'élément d'extension en phase, ainsi que celle de densité et de probabilité de l'état statistique. Il réalise finalement un accord formel mais remarquable avec les lois macroscopiques de la thermodynamique, régissant le comportement des milieux matériels en équilibre. Les développements actuels de la mécanique statistique constituent encore, sur plus d'un point, des prolongements de la méthode de Gibbs. Il définit pour les réactions chimiques deux quantités très utiles, à savoir l'enthalpie qui représente la chaleur d'une réaction à pression constante, et l'enthalpie libre qui détermine si oui ou non une réaction peut procéder de façon spontanée à température et pression constante. Cette dernière quantité est maintenant nommée "énergie de Gibbs" en son honneur (ou comme anglicisme énergie libre de Gibbs). L'emploi du point pour désigner un produit scalaire, celui de la croix de Saint-André pour un produit vectoriel et l'adoption des opérateurs vectoriels différentiels del ($\nabla()$) et nabla ($\vec{\nabla}\times$) proviendraient de Gibbs.

\parpic[l][t]{
  \begin{minipage}{40mm}
    \fbox{\includegraphics[width=110px,height=140px]{img/medaillons/godel.eps}}
  \end{minipage}
}
\textbf{Gödel, Kurt} (1906-1978) Né à Brünn et décédé à Stanford, était le mathématicien et logicien austro-américain, qui de tout le 20ème siècle, a le plus révolutionné les fondements logiques des mathématiques. Il était un homme tellement obsédé par la logique qu'on raconte que, alors qu'il cherchait à obtenir sa naturalisation américaine, il osa démontrer devant le juge la contradiction de certains articles de la constitution des États-Unis. Sa thèse, et surtout un article publié en 1931 sous le titre \textit{Über formal unentscheidbare Sätze der Principia Mathematica und verwandter Systeme} (sur l'indécidabilité formelle des \textit{Principia Mathematica} et de systèmes équivalents), donneront à Gödel une réputation internationale. Gödel met fin aux espoirs de Hilbert d'axiomatiser totalement la mathématique, et de n'en faire qu'une suite de déductions mécaniques ne laissant aucune place à l'intuition. Ainsi, Gödel montre qu'il existe des propositions vraies sur les nombres entiers, mais que l'on ne sait pas démontrer. Il montre même que, si on ajoute d'autres axiomes, on trouvera toujours des propositions vraies indécidables (qu'on ne sait pas démontrer). Il prouve notamment que l'hypothèse du continu et l'axiome du choix ne sont pas en contradiction avec les autres axiomes de la théorie des ensembles. Puis il s'oriente vers la relativité, étant en relation directe à Princeton avec son ami Einstein. Il est notamment connu des physiciens pour avoir démontré que le voyage vers le passé est possible dans le cadre des équations de la Relativité Générale.

\parpic[l][t]{%
  \begin{minipage}{40mm}
    \fbox{\includegraphics[width=110px,height=140px]{img/medaillons/goeppertmayer.eps}}
  \end{minipage}
}
\textbf{Göpper-Meyer, Maria} (1906-1972) était une physicienne américaine d'origine allemande, prix Nobel en 1963, pour son étude de la structure nucléaire. Elle était mariée à un physicien, le spécialiste de la physique du solide Joseph Mayer (1904-1983). Mais, dans ce couple, chacun travaillait de son côté et dans sa spécialité. Goeppert-Mayer obtint son doctorat à l'Université de Göttingen, en Allemagne. Elle enseigna dans de nombreuses institutions avant de rentrer à l'Université de Californie à San Diego, en 1960. En 1963, elle partagea avec H.D.Jensen et E.Wigner le prix Nobel de physique, et fut citée par le comité Nobel pour son oeuvre indépendante à la fin des années 1940. Elle démontra que le noyau atomique possède un nombre de neutrons et de protons bien définis: elle introduisit un modèle structural du noyau atomique en couches. Ce modèle développé en détail à partir de 1948 supposait que la forte interaction entre le mouvement de rotation intrinsèque (quantifié par le spin) des nucléons et leur mouvement orbital était responsable de la structure des niveaux d'énergie des noyaux. De nombreuses conséquences déduites de cette hypothèse se révélèrent vérifiées par les mesures expérimentales. Quelques années plus tard, James Rainwater, Aage Bohr et Ben R. Mottelson (tous trois prix Nobel de physique 1975) complétaient la théorie en tenant compte du couplage entre les mouvements des nucléons de la couche externe et le mouvement collectif du coeur nucléaire.

\parpic[l][t]{%
  \begin{minipage}{40mm}
    \fbox{\includegraphics[width=110px,height=140px]{img/medaillons/gosset.eps}}
  \end{minipage}
}
\textbf{Gosset, William Sealy} (1876-1937) connu sous le pseudonyme de "Student" c'était un statisticien anglais. Employé de la fameuse brasserie Guinness pour stabiliser le goût de la bière, il a ainsi inventé le "test de Student" utilisé de manière standard dans de très nombreux domaines de l'industrie ou de l'économie. Il a aussi déterminé en 1908 l'origine de la distribution  expérimentale qu'il obtenait dans le cadre de son travail et après avoir suivi un cours de statistique avec Karl Pearson, il obtint son fameux résultat qu'il publia sous le pseudonyme de Student avec la loi qui porte son nom et son test.\\

\parpic[l][t]{%
  \begin{minipage}{40mm}
    \fbox{\includegraphics[width=110px,height=140px]{img/medaillons/gottlob.eps}}
  \end{minipage}
}
\textbf{Gottlob, Frege Friedrich Ludwig} (1848-1925) était un mathématicien et philosophe allemand, initiateur de la logique moderne. Frege est né à Wismar en 1848, et fit ses études aux Universités de Iéna et de Göttingen, où il obtint son doctorat de philosophie en 1873. De 1879 à 1917, il fut professeur à la faculté de philosophie d'Iéna. Ses travaux concernent notamment la logique mathématique et ses applications. Confronté à l'ambiguïté du langage ordinaire et à l'imperfection des systèmes logiques disponibles, il inventa de nombreuses notations symboliques, comme les quantificateurs et les variables, posant alors les bases de la logique mathématique moderne. Il est ainsi le premier à avoir présenté une théorie cohérente du calcul des prédicats et du calcul des propositions. Il fut aussi le premier à faire dériver l'arithmétique de la logique. Il définit ainsi notamment la suite des nombres entiers à partir de l'ensemble vide, en appliquant quelques règles simples.

\parpic[l][t]{%
  \begin{minipage}{40mm}
    \fbox{\includegraphics[width=110px,height=140px]{img/medaillons/grothendieck.eps}}
  \end{minipage}
}
\textbf{Grothendieck, Alexander} (1928-2014) était né à Berlin d'un père anarchiste russe, tué par les nazis, et d'une mère femme de lettres, réfugiée en France. Il passe sa licence à la faculté des sciences de Montpellier, puis passe une année en 1948-1949 à l'École Normale Supérieure à Paris, avant de migrer en 1949 à l'Université de Nancy. Il y devient l'élève, en analyse fonctionnelle, de Schwartz et Dieudonné. Ce dernier le trouve un peu prétentieux, et lui propose de travailler sur des questions que ni Schwartz, ni lui n'ont su résoudre. Voilà ce qu'en dit Schwartz dans son autobiographie: "Dieudonné, avec l'agressivité (toujours passagère), dont il était capable, lui passa un savon mémorable, arguant qu'on ne devait pas travailler de cette manière, en généralisant pour le plaisir de généraliser. [...] L'article s'achevait sur 14 questions, des problèmes que nous n'avions pas su résoudre, Dieudonné et moi. Dieudonné lui [Grothendieck] proposa de réfléchir à certains d'entre eux qu'il choisirait. Nous ne le revîmes plus pendant quelques semaines. Lorsqu'il avait réapparu, il avait trouvé la solution de la moitié d'entre eux!".  Rapidement, Grothendieck rédige sa thèse intitulée \textit{Produits tensoriels topologiques et espaces nucléaires}, et devient le spécialiste mondial de la théorie des espaces vectoriels topologiques. Il devient aussi membre du célèbre groupe Bourbaki auprès de ses aînés.  Au début des années 1960, il obtient une charge au tout récent Institut des Hautes Études Scientifiques (IHES), et son centre d'intérêt s'oriente vers la géométrie algébrique. Il y réalise des travaux gigantesques, qui lui valent la médaille Fields en 1966. Toutefois, Grothendieck refuse de se rendre en URSS pour la recevoir, afin de protester contre la répression de l'insurrection hongroise en 1956. On la lui remet plus tard, mais il l'offre au Viêtnam, afin qu'il utilise son or. Il y enseigne d'ailleurs plusieurs semaines sous les bombardements américains. Vers la fin des années 60, Grothendieck, qui a perdu l'habitude de rédiger (Dieudonné a rédigé des années durant son séminaire), devient de moins en moins clair. Il ne pardonnera jamais aux autres mathématiciens de ne pas le comprendre et de dénaturer ainsi ses idées. Si ses relations avec la communauté mathématique n'avaient jamais été faciles (il travaillait énormément en solitaire, ses journées faisaient 27 ou 28 heures, de sorte que parfois il lui arrivait de se décaler - Il méprisait légèrement Dieudonné, séquelle du premier coup de gueule de ce dernier - ses prises de becs avec Weil causèrent son départ de Bourbaki...), elles sont plus tendues que jamais... Il abandonne peu à peu la mathématique et quitte l'IHES après une dispute interne sur des financements militaires, pour se retirer dans sa maison de l'Hérault, où il se consacre à la méditation et à l'écologie. Il écrit vers 1985 une sorte d'autobiographie, \textit{Récoltes et semailles}, qui ne trouve pas d'éditeur. Ceux qui ont pu la lire sont unanimes pour dire qu'elle contenait de nombreuses attaques contre la communauté des mathématiciens.

\phantomsection
\addcontentsline{toc}{section}{H}
\label{sec:H}

\parpic[l][t]{%
  \begin{minipage}{40mm}
    \fbox{\includegraphics[width=110px,height=140px]{img/medaillons/hall.eps}}
  \end{minipage}
}
\textbf{Hall, Edwin Herbert} (1855-1938) était un physicien né dans le Main et décédé à Cambridge (U.S.A.). Hall a fait ses études de premier cycle au Bowdoin College, obtenant son diplôme en 1875. Il fait ses études supérieures et de recherche, et obtint son doctorat (1880), à l'Université Johns Hopkins, où ses expériences ont été effectuées. L'effet Hall a été découvert par Hall en 1879, alors qu'il travaillait sur sa thèse de doctorat en physique. Hall a été nommé professeur de physique à Harvard en 1895. Il était aussi anecdotiquement connu pour donner des conférences sans chaussures et a écrit de nombreux livres sur la physique.\\

\parpic[l][t]{%
  \begin{minipage}{40mm}
    \fbox{\includegraphics[width=110px,height=140px]{img/medaillons/hamilton.eps}}
  \end{minipage}
}
\textbf{Hamilton, William Rowan} (1805-1865) était un mathématicien, physicien et astronome irlandais (né et mort à Dublin) qui fut l'objet de son vivant des plus grands honneurs, on l'appelait le "Lagrange irlandais", et même le "Newton irlandais", et pourtant son oeuvre était peu connue et rarement étudiée. Il est connu pour sa découverte des quaternions, mais il contribua aussi au développement de l'optique, de la dynamique et de l'algèbre. Ses recherches se révélèrent importantes pour le développement de la mécanique quantique. Les travaux mathématiques de Hamilton incluent l'étude de l'optique géométrique, l'adaptation des méthodes dynamiques aux systèmes optiques, l'application des quaternions et des vecteurs aux problèmes de mécanique et géométriques, les possibilités de résolution des équations polynomiales, notamment l'équation générale du cinquième degré, les opérateurs linéaires, dont il prouve un résultat concernant ces opérateurs dans l'espace des quaternions et qui est un cas spécial du théorème de Cayley-Hamilton. Sa carrière scientifique fut prédestinée par des études à Trinity College, à Dublin, où, à l'âge de 19 ans, il terminait un travail remarquable sur l'optique. À 23 ans, il devint professeur d'astronomie à Dublin et astronome royal à l'observatoire de Dunsink. Il restera toute sa vie fidèle à Dublin et à son observatoire. Hamilton s'efforce de donner aux principes fondamentaux de la mécanique une forme simple permettant d'édifier toute une théorie déductive. Pour cela, il modifie les principes de variations antérieurs, notamment le principe de moindre action, et introduit ce qu'on appelle de nos jours le "principe de Hamilton". Indiquons enfin qu'on lui doit la forme dite "canonique" des équations de la dynamique qui n'apporte rien de nouveau à la physique mais fournit une méthode plus puissante pour résoudre les équations du mouvement. Dans ses travaux des années 1832 à 1835 Hamilton attache une grande importance à l'interprétation géométrique des nombres complexes, et c'est à partir de là qu'il cherche un calcul algébrique qui s'interpréterait dans l'espace à trois dimensions. Il n'arrive à ce but qu'en 1843, en construisant les quaternions. Dans les années qui suivent cette découverte, il se consacre à son développement et à sa diffusion, en lui trouvant des applications à divers domaines des mathématiques et de la physique. Les quaternions de Hamilton constituent un des premiers systèmes de vecteurs et ont, par leurs conséquences théoriques, beaucoup contribué à l'élaboration de l'algèbre et de la physique quantique du 20ème siècle.

\parpic[l][t]{%
  \begin{minipage}{40mm}
    \fbox{\includegraphics[width=110px,height=140px]{img/medaillons/hawking.eps}}
  \end{minipage}
}
\textbf{Hawking, Stephen} (1942-2018) Né à Oxford, était un physicien théoricien et cosmologiste britannique. Au même titre qu'Albert Einstein, Hawking n'aurait pas été particulièrement brillant à la petite école, mais son goût pour les sciences physiques le mène à l'Université d'Oxford, un lieu d'après lui d'ennui relatif d'où il sort avec les honneurs. Après avoir obtenu son diplôme B.A. à Oxford en 1962, il est resté pour étudier l'astronomie. Il a décidé d'arrêter quand il trouva que l'étude des taches solaires ne l'attirait pas et qu'il était plus intéressé par la théorie que par l'observation. Il a quitté Oxford, avec les honneurs, pour Trinity Hall où il a participé à l'étude de l'astronomie théorique et la cosmologie théorique. L'Université de Cambridge est un tout autre monde: d'un côté, Hawking y débute son passionnant doctorat sur la Relativité Générale, de l'autre, sa maladie se déclare. Malgré cette difficulté, l'étude des singularités, concept physique et astronomique récent, permet au chercheur de développer différentes théories, qui le mèneront du Big Bang aux Trous Noirs. En premier lieu, Roger Penrose et lui construisent la structure mathématique répondant à la question d'une singularité comme origine de l'Univers. Ensuite, à partir des années 1970, Hawking approfondit ses recherches sur les densités infinies locales, et ses études sur les Trous Noirs ont fait progresser bien d'autres domaines. Enfin, la "théorie du tout", visant à unifier les quatre forces physiques, est au centre des dernières recherches de Hawking. Le but est de démontrer que l'Univers peut être décrit par un modèle mathématique stable, déterminé par les lois physiques connues, en vertu du principe de croissance finie mais non bornée, modèle auquel Hawking a donné beaucoup de crédit. Son handicap lourd ne saurait expliquer à lui seul le grand succès de ses recherches ; Hawking a cherché à vulgariser son travail, et son livre \textit{Une brève histoire du temps} est l'un des plus grands succès de littérature scientifique. En 2001, paraît son deuxième ouvrage, \textit{L'univers dans une coquille de noix} qui vulgarise le dernier état de ses réflexions, en abordant la supergravité et la supersymétrie, la théorie quantique et théorie-M, l'holographie et la dualité, la théorie des supercordes et des $p$-branes... Il s'interroge également sur la possibilité de voyager dans le temps et sur l'existence d'univers multiples.

\parpic[l][t]{%
  \begin{minipage}{40mm}
    \fbox{\includegraphics[width=110px,height=140px]{img/medaillons/hausdorff.eps}}
  \end{minipage}
}
\textbf{Hausdorff, Felix} (1868-1942) La renommée du mathématicien allemand Felix Hausdorff repose surtout sur son ouvrage \textit{Grundzüge der Mengenlehre} (1914), qui en fit le fondateur de la topologie et de la théorie des espaces métriques. Né à Breslau dans une famille de marchands aisés, Hausdorff fit ses études secondaires à Leipzig, puis étudia la mathématique et l'astronomie à Leipzig, Fribourg-en-Brisgau et Berlin. En 1891, il obtint son doctorat à Leipzig et y enseigna de 1896 à 1902. Durant toute cette époque, Hausdorff, tout en publiant plusieurs mémoires d'astronomie, d'optique et de mathématiques, s'intéressa surtout à la philosophie, la littérature et l'art. De 1910 à 1935, il était professeur de mathématiques à l'Université de Bonn, à l'exception des années 1913-1921, où il enseignait à Greifswald. Depuis sa retraite forcée, en 1935, les travaux de Hausdorff ne furent plus publiés en Allemagne. Juif, Hausdorff risqua le camp de concentration et, lorsqu'en 1942 l'internement devint imminent, il se suicida à Bonn, avec sa femme et sa belle-soeur. Les contributions de Hausdorff au développement des mathématiques se situent dans plusieurs domaines. Son étude approfondie des séries déboucha sur la démonstration de théorèmes sur les méthodes de sommation et les coefficients de Fourier (1921). Considérant les propriétés d'ensembles numériques, il introduisit une classe importante de mesures. Il a étudié, en théorie générale des ensembles, les ensembles partiellement ordonnés et a obtenu plusieurs théorèmes sur les ensembles ordonnés (1906-1909). En théorie descriptive des ensembles, il a démontré le théorème sur la cardinalité des ensembles boréliens (1916). Outre des résultats isolés mais profonds en topologie et en théorie des ensembles, Hausdorff a surtout, par ses \textit{Grundzüge der Mengenlehre}, posé les fondements d'une discipline. Hausdorff développe une théorie des espaces topologiques et métriques englobant parfaitement les résultats antérieurs. Il choisit de construire sa théorie des espaces abstraits sur la notion de voisinage. Il ajouta bon nombre de résultats nouveaux à la théorie des espaces métriques, dont le plus profond est le théorème affirmant que chaque espace métrique peut être étendu d'une manière unique à un espace métrique complet. Hausdorff était un professeur méthodique, mais ses cours, au contenu riche et rigoureusement structuré, passèrent au-dessus du niveau de ses auditeurs.

\parpic[l][t]{%
  \begin{minipage}{40mm}
    \fbox{\includegraphics[width=110px,height=140px]{img/medaillons/heaviside.eps}}
  \end{minipage}
}
\textbf{Heaviside, Oliver} (1850-1925) est né à Camden Town en Angleterre et est mort à Torquay dans le Devon (situé aussi en Angleterre). C'est là qu'il a vécu les 25 dernières années de sa vie. Il est issu d'une famille assez pauvre. Il a attrapé la scarlatine quand il était un enfant en bas âge, ce qui a affecté son audition, il est resté partiellement sourd. Ce qui a eu un impact sur sa vie rendant son enfance difficile surtout au niveau des relations avec les autres enfants. Il a compensé par la timidité et le sarcasme. Cependant, malgré tout, son rendement académique était plutôt élevé. On peut même dire qu'à 16 ans, c'était un étudiant supérieur, mais il a échoué dans la géométrie d'Euclide. Il a détesté devoir déduire un fait d'autres. Le primat de la preuve rigoureuse en arithmétique, idée fortement détestée par Heaviside en fit le sujet où il était le plus faible. Bien qu'ayant interrompu ces études à 16 ans, il a continué à s'instruire par lui-même. Il a appris le code Morse, étudié l'électricité et d'autres langues en particulier le Danois et l'Allemand. Il était autodidacte. En 1868, après avoir quitté ses études, Heaviside est allé au Danemark et il est devenu opérateur de télégraphe. Il a progressé rapidement dans sa profession et il est revenu en Angleterre en 1871. C'est son travail qui l'a incité à étudier l'électricité. Il a donc lu le nouveau traité de Maxwell sur l'électricité et le magnétisme. Après avoir lu ce traité, il a apporté des changements à sa vie.  Il a arrêté de travailler et il s'est enfermé dans une chambre de la maison familiale pour travailler sur la théorie de Maxwell. Heaviside a réduit la théorie de Maxwell et c'est à partir de ce moment que la théorie électrique a pris sa forme moderne. Maxwell avait écrit 20 équations à 20 variables. Heaviside réduit ces 20 équations en les remplaçant par 4 équations à 2 variables. Aujourd'hui, nous appelons ces équations: "Les 4 équations de Maxwell", oubliant qu'elles sont en fait les équations de Heaviside! Cependant, c'est Hertz qui a obtenu le crédit pour cela, mais il admet que ses idées lui sont venues de Heaviside.

\parpic[l][t]{%
  \begin{minipage}{40mm}
    \fbox{\includegraphics[width=110px,height=140px]{img/medaillons/heisenberg.eps}}
  \end{minipage}
}
\textbf{Heisenberg, Werner Karl} (1901-1976) Né à Wurtzbourg et décédé à Münich était un physicien allemand. Il fut le fondateur des concepts théoriques rigoureux de la mécanique quantique. Il est lauréat du prix Nobel de physique de 1932. Il fréquente le prestigieux Maximiliangymnasieum où Max Planck avait étudié 40 ans plus tôt. A l'âge de 12 ans, il se mit à apprendre le calcul intégral et plus tard, passionné par les mathématiques, il suivit en auditeur libre plusieurs cours de l'Université de Münich, notamment sur les méthodes mathématiques de la physique moderne. Il accomplit ses études de physique dans le délai record de 3 ans, et soutint sa thèse (qu'il faillit louper à cause de lacunes en physique expérimentale élémentaire) sous la direction d'Arnold Sommerfeld avec lequel il élabora une théorie expliquant l'effet Zeeman anormal à l'âge de 20 ans qui attira l'attention des grands physiciens européens (il était considéré aussi brillant que Pauli qui lui-même était déjà considéré comme plus génial qu'Einstein). Dès 1924 il devenait l'assistant de Max Born à Göttingen puis il travailla avec Niels Bohr à Copenhague. C'est au cours des années suivantes qu'avec Max Born et Pascual Jordan, il jeta les bases théoriques de la mécanique quantique. Heisenberg fut recruté en 1927 comme professeur à l'Université de Leipzig âgé seulement de 26 ans. Il fit de cet établissement l'un des hauts-lieux de la physique théorique (et en particulier de la physique nucléaire) en Europe. Il développa la première formalisation de la mécanique quantique, en 1925, en même temps qu'Erwin Schrödinger. Toutefois le formalisme mathématique était différent. Heisenberg adopta une formalisation matricielle complexe (alors qu'il ne savait pas ce qu'était une matrice comme la majorité des physiciens de son temps...) qui faisait naturellement émerger la non commutativité alors que Schrödinger utilisa une approche par les équations différentielles (simple équation d'ondre). Pour cette raison, on crut d'abord que les deux théories étaient distinctes, mais l'année suivante, Schrödinger établit l'équivalence mathématique des deux formulations. Son principe d'incertitude, découvert en 1927, affirme que la détermination de certains couples de valeurs, par exemple la position et la quantité de mouvement, ne peut se faire avec une précision infinie. À partir de 1929, il travailla avec Wolfgang Pauli à l'élaboration de la théorie quantique des champs. Après la découverte du neutron par James Chadwick en 1932, Heisenberg proposa le modèle proton-neutron du noyau atomique, et s'en servit pour expliquer le spin nucléaire des isotopes.

\parpic[l][t]{%
  \begin{minipage}{40mm}
    \fbox{\includegraphics[width=110px,height=140px]{img/medaillons/hemlholtz.eps}}
  \end{minipage}
}
\textbf{Helmholtz, Hermann Ludwig Ferdinand von} (1821-1894) Né à Potsdam et mort à Berlin. Il n'est guère de domaines des sciences de la nature auxquels Helmholtz n'ait consacré quelques recherches. On pourrait répéter à son endroit, ce qu'il disait lui-même de Friedrich von Humboldt dans sa célèbre conférence inaugurale du colloque scientifique d'Innsbruck (sur le but et les progrès de la science de la Nature, 1869): "Il avait réussi à dominer toutes les sciences de la Nature à son époque et à pénétrer jusqu'en chacune de leurs spécialités." Même si Helmholtz ajoute que dans la seconde moitié du 19ème siècle ce savoir encyclopédique est désormais impossible, et qu'il faut se résigner à besogner dans un secteur étroitement délimité, il suffit de jeter un regard sur l'ensemble de ses travaux pour constater qu'il s'est préoccupé de matières aussi différentes que la thermodynamique, l'hydrodynamique, l'électrodynamique et la théorie de l'électricité, la physique météorologique, la physiologie, et plus particulièrement la théorie de l'acoustique et l'optique physiologique. Pourvu de dons remarquables pour la vulgarisation des résultats scientifiques les plus récents, il écrivit de nombreux articles et prononça maintes conférences où les exposés scientifiques populaires voisinent avec des préoccupations esthétiques ou philosophiques. Son nom reste surtout attaché à la formulation du principe de la conservation de l'énergie, qui fait de lui l'un des pères de l'énergétique, même si certaines de ses assertions peuvent sembler d'un mécanisme intransigeant et ont pu le faire considérer par certains comme le dernier tenant de la physique galiléenne. Son nom est aussi lié également à quelques inventions notoires comme celle de l'ophtalmoscope ou des résonateurs sphériques. Sur la fin de sa vie, Helmholtz reconnaîtra l'importance et l'universalité d'un autre principe physique, le principe de moindre action, qu'il appliquera, en particulier, à l'électrodynamique.

\parpic[l][t]{%
  \begin{minipage}{40mm}
    \fbox{\includegraphics[width=110px,height=140px]{img/medaillons/henry.eps}}
  \end{minipage}
}
\textbf{Henry, Joseph} (1797-1878) était un scientifique américain né à New-York et décédé à Washington. Ses parents étaient pauvres, et le père d'Henry est mort alors qu'il était encore jeune. Pour le reste de son enfance, Henry a vécu avec sa grand-mère à New York. Il a fréquenté une école qui allait plus tard être nommée Joseph Henry Elementary School, à son honneur. Après l'école, il a travaillé dans un magasin général, et à l'âge de 13 ans, il est devenu apprenti horloger et orfèvre. Son intérêt pour la science s'est éveillé à l'âge de 16 ans par un livre de conférences sur des sujets scientifiques intitulé \textit{Popular Lectures on Experimental Philosophy}. En 1819, il entra à l'académie Albany, où il a suivi des cours gratuits. Il était si pauvre qu'il devait subvenir à ses besoins avec des postes d'enseignement et de tutorat privé. Il avait l'intention d'étudier la médecine, mais en 1824 il a été nommé assistant ingénieur de l'inspection nationale des routes en cours de construction entre la rivière Hudson et le lac Érié. Dès lors, il a été inspiré d'une carrière soit dans le génie civil ou mécanique. Henry a tellement excellé dans ses études (à tel point, qu'il aidait souvent ses professeurs à enseigner la science) que, en 1826, il est nommé professeur de mathématiques et de philosophie naturelle à l'académie Albany. Certains de ses travaux de recherche les plus importants ont été réalisé dans pendant qu'il était à ce poste. Sa curiosité sur le magnétisme terrestre l'a amené à faire des expériences avec le magnétisme en général. Il a été le premier à enrouler un fil isolé étroitement autour d'un noyau de fer dans le but de faire un électro-aimant puissant. Pendant la construction des électro-aimants, Henry a découvert le phénomène électromagnétique d'auto-induction. Il a également découvert l'inductance mutuelle indépendamment de Michael Faraday, mais puisque Faraday a publié ses résultats en premier, il est devenu le découvreur officiellement reconnu du phénomène. En utilisant son principe électromagnétique nouvellement développé, Henry en 1831 a créé l'une des premières machines à utiliser l'électromagnétisme pour le mouvement. Ce fut le premier ancêtre du moteur à courant continu. L'unité SI de l'inductance, la Henry, est nommé en son honneur.

\parpic[l][t]{%
  \begin{minipage}{40mm}
    \fbox{\includegraphics[width=110px,height=140px]{img/medaillons/hermite.eps}}
  \end{minipage}
}
\textbf{Hermite, Charles} (1822-1901) Né à Dieuze, il publia ses premiers travaux alors qu'il était encore élève à l'École polytechnique, et à 30 ans, il était déjà considéré comme un des meilleurs mathématiciens de son temps. Il fut successivement professeur à l'école Polytechnique, au Collège de France et enfin à la Sorbonne à partir de 1869 où son enseignement et sa volumineuse correspondance eurent une influence considérable. Il vécut à Paris jusqu'à sa mort. Il avait été élu membre de l'académie des Sciences à 34 ans. En algèbre, Hermite prit une part active aux premiers développements de la théorie des invariants, inaugurée par Arthur Cayley et James Joseph Sylvester, il acheva, entre autres, la détermination des invariants des formes binaires du 5 degré, commencée par Sylvester, et découvrit la loi de réciprocité entre covariants de formes binaires de degrés différents. On lui doit aussi un procédé d'interpolation améliorant la méthode de Lagrange en tenant compte des valeurs des dérivées premières, et la découverte de la famille de polynômes orthogonaux qui portent son nom. Les travaux d'analyse d'Hermite portent la marque de son tempérament d'algébriste. Son sujet de prédilection pendant toute sa vie a été la théorie des fonctions elliptiques et des fonctions abéliennes, dont il aimait particulièrement explorer les liens cachés avec l'algèbre et la théorie des nombres. Un de ses résultats qui frappa le plus ses contemporains est la résolution de l'équation du cinquème degré à l'aide des fonctions elliptiques. Sa virtuosité dans les calculs des fonctions lui permit d'obtenir directement les remarquables relations sur les nombres de classes d'idéaux des corps quadratiques, que Kronecker avait déduites de la multiplication complexe. Il fut un des pionniers dans l'étude des fonctions abéliennes, où il développa la théorie de la transformation et rencontra à cette occasion pour la première fois le groupe symplectique. Enfin, le plus célèbre des mémoires d'Hermite est celui où, en 1872, il démontra la transcendance du nombre $e$ ; il y avait été conduit par ses recherches sur les fractions continuées algébriques, et sa méthode est restée presque la seule dont on dispose encore aujourd'hui pour aborder les problèmes de transcendance.

\parpic[l][t]{%
  \begin{minipage}{40mm}
    \fbox{\includegraphics[width=110px,height=140px]{img/medaillons/hertz.eps}}
  \end{minipage}
}
\textbf{Hertz, Heinrich Rudolf} (1857-1894) était un physicien allemand né à Hambourg et décédé à Bonn. Il fit ses études à l'Université de Berlin. En 1879, il est l'élève de Gustav Kirchhoff et de Hermann von Helmholtz à l'institut de physique de Berlin. Il devient maître de conférence à l'Université de Kiel en 1883 où il effectue des recherches sur l'électromagnétisme. De 1885 à 1889, à l'origine de la télégraphie sans fil, il fut professeur de physique à l'école technique de Karlsruhe, et, à partir de 1889, enseigna la physique à l'Université de Bonn. Hertz clarifia et étendit la théorie électromagnétique de la lumière proposée par le physicien anglais James Maxwell, en 1884. Il prouva que l'électricité pouvait être transmise par des ondes électromagnétiques qui se déplacent à la vitesse de la lumière et possèdent de nombreuses autres propriétés de la lumière. Ses expérimentations avec ces ondes aboutirent au développement du télégraphe sans fil et de la radio. L'unité de fréquence, une période par seconde, fut dénommée le "Hertz".

\parpic[l][t]{%
  \begin{minipage}{40mm}
    \fbox{\includegraphics[width=110px,height=140px]{img/medaillons/hilbert.eps}}
  \end{minipage}
}
\textbf{Hilbert, David} (1862-1943) Né à Königsberg et décédé à Göttingen fut un étudiant de Lindemann sous la supervision duquel il obtint sa thèse en 1885 et eut pour camarade Herman Minkowski, avec qui il resta lié par une profonde amitié. Bien que les intérêts mathématiques de Hilbert furent vastes, il préféra travailler à un sujet à la fois. Ses principaux domaines d'intérêts furent: jusqu'en 1892, la théorie algébrique des invariants; de 1892 à 1899 la théorie algébrique des nombres; de 1899 à 1905, le calcul des variations; de 1901 à 1912, les équations intégrales; de 1912 à 1917, les fondements mathématiques de la physique. Vers 1910, Hilbert soutient les efforts d'Emmy Noether, mathématicienne de premier ordre, qui souhaite enseigner à l'Université de Göttingen. Pour déjouer le système établi contre les femmes, Hilbert prête son nom à Noether qui peut ainsi annoncer l'horaire de ses cours sans entacher la réputation de l'université. De 1917 jusqu'à la fin de sa vie, il s'occupa de la logique mathématique. Il donna une impulsion décisive à l'essor des recherches sur les fondements des mathématiques. Au congrès international des mathématiques de 1900, Hilbert présenta une liste de $23$ problèmes dont plusieurs ne sont pas encore résolus aujourd'hui en ce début de 21ème siècle. Il a adopté et défendu avec vigueur les idées de Georg Cantor en théorie des ensembles et sur les nombres transfinis. Il est aussi connu comme l'un des fondateurs de la théorie de la démonstration, de la logique mathématique et a clairement distingué les mathématiques des métamathématiques. Il est considéré par plusieurs comme le plus grand mathématicien du 20ème siècle.

\parpic[l][t]{%
  \begin{minipage}{40mm}
    \fbox{\includegraphics[width=110px,height=140px]{img/medaillons/hotelling.eps}}
  \end{minipage}
}
\textbf{Hotelling, Harold} (1895-1973) était un statisticien mathématicien et un théoricien économique influent, connu pour la distribution $T^2$ de Hotelling dans les statistiques. Il a été professeur agrégé de mathématiques à l'Université de Stanford de 1927 à 1931, membre de la faculté de l'Université de Columbia de 1931 à 1946 et professeur de statistiques mathématiques à l'Université de Caroline du Nord à Chapel Hill de 1946 jusqu'à sa mort. Une rue à Chapel Hill porte d'ailleurs son nom. En 1972, il a reçu le prix de la Caroline du Nord pour ses contributions à la science. Hotelling est connu des statisticiens en raison de la distribution $T^2$ de Hotelling qui est une généralisation de la distribution $T$ de Student dans un contexte multivarié, et son utilisation dans les tests d'hypothèses statistiques et les régions de confiance. Il a également introduit l'analyse de corrélation canonique. Aux Etats-Unis d'Amérique (USA), Harold Hotelling est connu pour son leadership dans le métier de statisticien, notamment pour sa vision d'un département de statistiques à l'université, qui a convaincu de nombreuses universités de créer des départements statistiques.

\parpic[l][t]{%
  \begin{minipage}{40mm}
    \fbox{\includegraphics[width=110px,height=140px]{img/medaillons/hoyle.eps}}
  \end{minipage}
}
\textbf{Hoyle, Fred} (1915-2001) Né à Bingley, dans le Yorkshire et décédé à Bournemouth, il atudie la mathématique et la physique théorique à Cambridge de 1933 à 1939. Lorsque les hostilités de la deuxième guerre mondiale éclatent, il s'engage dans la Royal Navy pour travailler au développement du radar au centre de recherche de Witley. Il y rencontre les deux physiciens Hermann Bondi et Thomas Gold. Tous trois passionnés de cosmologie, ils considèrent avec scepticisme le modèle standard de l'Univers de l'époque (d'un point de vue philosophique il leur est inacceptable). À l'époque, le modèle standard achoppait de plus à une difficulté sérieuse: d'après les estimations de Hubble, l'âge de l'Univers devait être d'environ 2 milliards d'années, or les données géologiques conduisaient à un âge de la Terre d'au moins 4 milliards d'années. Pendant la guerre, et dans les quelques années qui suivent la fin des hostilités, Hoyle publie plusieurs études sur la théorie de l'accrétion et sur la théorie de la structure stellaire, en particulier pour les étoiles géantes et les naines blanches. La guerre terminée, les trois hommes retournent à Cambridge, où Hoyle obtient une chaire de mathématiques. En 1948, ils exposent leur théorie dans deux articles, l'un de Bondi et Gold, l'autre de Hoyle. En 1963, le premier quasar est découvert. Sa luminosité intrinsèque est très supérieure à celle de tout autre objet céleste connu: il est cent fois plus lumineux que n'importe quelle galaxie! En 1962, Hoyle et William A. Fowler avaient proposé une théorie qui pouvait rendre compte de la luminosité énorme des quasars ; il s'agissait de la théorie des étoiles supermassives. Des considérations théoriques permettent de démontrer que des étoiles normales  de masses supérieures à environ 60 masses solaires seraient le siège d'instabilités violentes dues à la pression de radiation et à la génération de l'énergie nucléaire. Cette hypothèse est corroborée par le fait que l'on n'observe pas d'étoiles normales au-delà de la limite d'instabilité. En dépit de cet argument, Hoyle et Fowler proposaient le concept d'étoile supermassive, étoile qui serait supportée presque entièrement par la pression de radiation. Ainsi, pour atteindre la luminosité caractéristique d'un quasar, l'étoile supermassive doit avoir une masse de l'ordre de 100 millions de masses solaires. Lorsque la densité devient suffisamment élevée, une étoile supermassive de moins de 1 million de masses solaires explose, tandis qu'une étoile plus massive subit un effondrement cataclysmique et forme des trous noirs supermassifs. Ces deux possibilités sont très importantes pour comprendre les quasars, et elles ont été étudiées par de nombreux chercheurs. Une autre explication du phénomène quasar, suggérée pour la première fois par Donald Lynden-Bell, suppose l'accrétion de matière dans un Trou Noir supermassif situé au centre d'une galaxie (hypothèse actuellement adoptée par consensus de la communauté scientifique).

\parpic[l][t]{%
  \begin{minipage}{40mm}
    \fbox{\includegraphics[width=110px,height=140px]{img/medaillons/huygens.eps}}
  \end{minipage}
}
\textbf{Huygens, Christian} (1629-1695) était un astronome, mathématicien et physicien néerlandais. Ses découvertes scientifiques, nombreuses et originales, lui valurent une large reconnaissance et les honneurs parmi les personnalités scientifiques du 17ème siècle. Avec son \textit{Traité de la lumière} (1690), il est à l'origine de la théorie ondulatoire de la lumière (qui plus tard prit son nom): chaque point d'ondes en mouvement est lui-même source de nouvelles ondes. Il se penche très vite, dès 1652 sur les règles exposées par Descartes dans les \textit{Principes de la philosophie}. Prenant appui sur la conservation cartésienne de la quantité de mouvement $p=mv$, il a l'idée de résoudre algébriquement le problème du choc en comparant les quantités $mv^2$ qui ne sont introduites que pour le bien du calcul, sans signification physique particulière. Découvrant alors que ces quantités se conservent avant et après le choc, il peut écrire les règles dans le cas général, ce que Descartes n'avait pu faire incluant donc conservation de la quantité de mouvement et de l'énergie cinétique. En 1655, il inventa une méthode de meulage et de polissage des lentilles d'optique. La définition plus fine ainsi obtenue lui permit de découvrir un satellite de Saturne et de fournir la première description précise des anneaux de Saturne. La nécessité de disposer d'une mesure exacte du temps pour l'observation du ciel l'amena à appliquer les lois du pendule composé pour régler les mouvements des horloges et montres. En 1656, il conçut une lunette de télescope qui porte son nom. Entre 1658 et 1659, Huygens travaille à la théorie du pendule oscillant. Il a en effet l'idée de réguler des horloges au moyen d'un pendule, afin de rendre la mesure du temps plus précise. Il découvre la formule de l'isochronisme rigoureux en 1659: lorsque l'extrémité du pendule parcourt un arc de cycloïde, la période d'oscillation est constante quelle que soit l'amplitude. Dans Horologium oscillatorium (1673), il détermina la véritable relation existant entre la longueur d'un pendule et la durée d'oscillation, et présenta ses théories sur la force centrifuge des mouvements circulaires, qui aidèrent le physicien anglais Isaac Newton à formuler les lois de la gravité. En 1673, Huygens et son jeune assistant Denis Papin, mettent en évidence le principe des moteurs à combustion interne, qui conduiront au xixe siècle à l'invention de l'automobile. En 1678, il découvrit la polarisation de la lumière par double réfraction sur la calcite.

\phantomsection
\addcontentsline{toc}{section}{I}
\label{sec:I}

\parpic[l][t]{%
  \begin{minipage}{40mm}
    \fbox{\includegraphics[width=110px,height=140px]{img/medaillons/ibn.eps}}
  \end{minipage}
}
\textbf{Ibn Al Haytham} (965-1039) était un mathématicien, un philosophe et un physicien Arabe. Il est l'un des pères de la physique quantitative et de l'optique moderne, le pionnier de la méthode scientifique moderne et le fondateur de la physique expérimentale et certains, pour ces raisons, l'ont décrit comme le premier scientifique. Al Haytham commença sa carrière de scientifique dans sa ville natale de Bassorah (Irak). Il fut cependant convoqué par le calife Hakim qui voulait maîtriser les inondations du Nil qui frappaient l'Égypte année après année. Après avoir mené une expédition en plein désert pour remonter jusqu'à la source du fameux fleuve, Alhazen se rendit compte que ce projet était pratiquement impossible. De retour au Caire, il craignait que le calife qui était furieux de son échec ne se vengeât et décida donc de feindre la folie. Le calife se borna à l'assigner à résidence. Alhazen profita de ce loisir forcé pour écrire plusieurs livres sur des sujets variés comme l'astronomie, la médecine, la mathématique, la méthode scientifique et l'optique. Le nombre exact de ses écrits n'est pas connu avec certitude mais on parle d'un nombre entre 80 et 200. Peu de ces ouvrages, en effet, ont survécu jusqu'à nos jours. Quelques-uns d'entre eux, ceux sur la cosmologie et ses traités sur l'optique notamment, n'ont survécu que grâce à leur traduction latine. La plupart de ses recherches concernaient l'optique géométrique et physiologique. Contrairement à une croyance populaire, il a été le premier à expliquer pourquoi le Soleil et la Lune semblent plus gros (on a cru longtemps que c'était Ptolémée), il établit aussi que la lumière de la Lune vient du Soleil. C'est aussi lui qui a contredit Ptolémée sur le fait que l'oeil émettrait de la lumière. Selon lui, si l'oeil était conçu de cette façon on pourrait voir la nuit. Il a compris que la lumière du Soleil était diffusée par les objets et ensuite entrait dans l'oeil. En astronomie il a tenté de mesurer la hauteur de l'atmosphère et a trouvé que le phénomène du crépuscule est dû à un phénomène de réfraction. Il parla également de l'attraction des masses et on croit qu'il connaissait l'accélération gravitationnelle. Alhazen a devancé de quelques siècles plusieurs découvertes faites par des scientifiques occidentaux pendant la Renaissance. Il fut un des premiers à se servir d'une méthode d'analyse scientifique et influença grandement des scientifiques comme Roger Bacon et Johannes Kepler.

\phantomsection
\addcontentsline{toc}{section}{J}
\label{sec:J}

\parpic[l][t]{%
  \begin{minipage}{40mm}
    \fbox{\includegraphics[width=110px,height=140px]{img/medaillons/jacobi.eps}}
  \end{minipage}
}
\textbf{Jacobi, Carl} (1804-1851) Né à Potsdam et décédé à Berlin fut, avec N. H. Abel, le fondateur de la théorie des fonctions elliptiques dont il donna de nombreuses applications aux branches les plus diverses des mathématiques. On lui doit également des exposés de mécanique théorique où il reprend les résultats de W. R. Hamilton, et des applications de la théorie des équations différentielles à la dynamique. À son entrée au gymnase, en 1816, Jacobi avait déjà achevé le cycle des études secondaires et, assez réfractaire à l'enseignement traditionnel, il étudia directement les oeuvres des grands mathématiciens, particulièrement celles d'Euler et de Lagrange. Inscrit en 1821 à l'Université de Berlin, il y apprit la philologie et la mathématique, à laquelle il se consacra bientôt uniquement. En 1825 il était docteur en philosophie avec une thèse où il démontrait ou généralisait certaines formules de Lagrange. Il enseigna à Berlin pendant une année environ, puis à Koenigsberg où il fut transféré par décision ministérielle. Fin 1827, il fut nommé professeur extraordinaire à l'université de cette ville où il entra en contact avec l'astronome Friedrich Wilhelm Bessel. Pensionné par le gouvernement de Prusse, il fut, après un voyage en Italie, en 1843, nommé académicien à Berlin, dispensé de tout enseignement mais autorisé à traiter, à l'université, tout sujet qui lui conviendrait. Présenté comme candidat aux élections de mai 1848, il fut persécuté un temps pour ses opinions libérales. Jacobi consacra de nombreux travaux à la transformation des intégrales et apporta une contribution essentielle à la théorie des équations différentielles et des équations aux dérivées partielles. C'est à cela que se rattachent ses apports au calcul des variations, à la dynamique des solides et à la mécanique céleste – problème des trois corps, perturbations des mouvements planétaires. L'algèbre lui doit d'importantes recherches sur les formes quadratiques et une exposition devenue classique de la théorie des déterminants, prélude au mémoire sur les déterminants fonctionnels appelés de nos jours "jacobiens". Il perfectionne la théorie de l'élimination et enseigne à représenter les racines d'une équation algébrique par des intégrales définies ou par des séries. Il étudie les points communs aux courbes et aux surfaces algébriques, et trouve directement le nombre des tangentes doubles d'une courbe plane, établi déjà par J. Plücker en utilisant la dualité.

\parpic[l][t]{%
  \begin{minipage}{40mm}
    \fbox{\includegraphics[width=110px,height=140px]{img/medaillons/jordan_camille.jpg}}
  \end{minipage}
}
\textbf{Jordan, Camille} (1838-1921) Né à Lyon et décédé à Paris fut le spécialiste indiscuté de la théorie des groupes pendant toute la fin du 19ème siècle et on lui doit de très nombreux résultats, tant sur les groupes finis que sur les groupes dits classiques, dont il fut le premier à mesurer toute l'importance. Ses cours d'analyse contribuèrent au développement de la théorie des fonctions de variable réelle. En 1855, à 17 ans, il est reçu premier à l'École Polytechnique et sort de l'École des Mines en 1861. Il sera, du moins en titre, ingénieur chargé de la surveillance des carrières de Paris jusqu'en 1885, ce qui n'empêchera pas une intense activité de recherche mathématique. Nommé examinateur à l'École Polytechnique en 1873, puis professeur en 1876, il entre à l'Académie des Sciences en 1881 puis succède à Joseph Liouville au Collège de France deux années plus tard. De 1885 à 1921, il assume la direction du \textit{Journal de mathématiques pures et appliquées} fondé par Liouville. Malgré les efforts de Liouville, l'oeuvre d'Évariste Galois était restée à peu près totalement inconnue du monde des mathématiques (seul Leopold Kronecker avait utilisé certains de ses résultats), et c'est à Jordan, avec son \textit{Traité des substitutions et des équations algébriques}, publié à Paris en 1870, que l'on doit le premier exposé systématique de théorie des groupes, enrichi de dix années de recherches personnelles. Il s'y limite aux groupes finis, plus précisément aux groupes de permutations, et introduit de nombreux concepts nouveaux. Dans des mémoires ultérieurs, Jordan étudie en détail, essentiellement du point de vue des facteurs de composition, le groupe linéaire et les groupes orthogonaux et symplectiques sur un corps premier. Les études de Jordan sur le groupe linéaire font intervenir des considérations sur la réduction des matrices, et, en particulier, la forme dite "forme de Jordan". Indiquons enfin les efforts de Jordan pour déterminer tous les groupes résolubles finis en réponse au problème, posé par Niels Henrik Abel, de rechercher toutes les équations de degré donné résolubles par radicaux. En plus des résultats donnés ci-dessus relatifs au groupe linéaire, on doit à Jordan un exposé complet de la géométrie euclidienne réelle à $n$ dimensions par des méthodes entièrement analytiques. L'enseignement de Jordan à l'École Polytechnique, puis au Collège de France, l'amène à préciser de nombreuses notions de la théorie des fonctions de variable réelle et son \textit{Cours d'analyse de l'École polytechnique}, dont la première édition date de 1880, contribuera à former des générations de mathématiciens. On lui doit aussi la notion de fonction à variation bornée, qui lui permet de donner une définition correcte de la longueur d'une courbe et d'obtenir sous sa forme générale le théorème de convergence des séries de Fourier ; mais le résultat le plus célèbre est celui qui affirme qu'une courbe fermée simple (dite, de nos jours, "courbe de Jordan") sépare le plan en deux régions. Signalons enfin, pour terminer, que Jordan, précurseur de Poincaré, a écrit plusieurs mémoires d'Analysis situs, c'est-à-dire sur la topologie combinatoire. On lui doit une démonstration, devenue classique, du théorème d'Euler sur les polyèdres et le fait que deux surfaces de même genre sont applicables l'une sur l'autre (ce qui, comme l'a montré Poincaré, n'est pas vrai en général pour les hypersurfaces).

\parpic[l][t]{%
  \begin{minipage}{40mm}
    \fbox{\includegraphics[width=110px,height=140px]{img/medaillons/jordanp.eps}}
  \end{minipage}
}
\textbf{Jordan, Pascual} (1902-1980) Né à Hanovre et décédé à Hamburg il était un physicien théoricien allemand, professeur à l'Université de Göttingen. Jordan passa dès 1921 une partie de ses études à l'Université Technique de Hanovre où il étudia un mélange de zoologie, de mathématique et physique. En 1923 il se spécialisa lors de son entrée à l'Université de Göttingen, qui était alors à son zénith tant du point de vue de la mathématique que de la physique. À Göttingen, Jordan devint assistant de Richard Courant et surtout de Max Born qui l'influença grandement. Il contribua ainsi de façon décisive à la fondation de la mécanique quantique et de la théorie quantique des champs. En raison de son affiliation au parti nazi, il fut cependant isolé de la communauté physicienne. En 1925, avec Max Born, Jordan écrit la relation de commutation canonique entre la quantité de mouvement et la position. Dans ce même article, il propose également l'idée qu'il faut aussi quantifier le champ électromagnétique, ouvrant la voie à la théorie quantique des champs. En 1925 également, avec Max Born et Werner Heisenberg, Jordan développe la formulation matricielle de Heisenberg de la mécanique quantique. Ils introduisent les transformations canoniques, la théorie des perturbations, le traitement des systèmes dégénérés, et enfin la fameuse relation de commutation canonique des composantes du moment cinétique.

\parpic[l][t]{%
  \begin{minipage}{40mm}
    \fbox{\includegraphics[width=110px,height=140px]{img/medaillons/joule.eps}}
  \end{minipage}
}
\textbf{Joule, James Prescott} (1818-1889) était un physicien britannique, né à Salford, dans le Lancashire et décédé à Sale (Angleterre). Il fut l'un des plus grands physiciens de son époque. Joule est célèbre pour ses travaux de recherche en électricité et en thermodynamique. Au cours de ses recherches sur la chaleur émise dans un circuit électrique, il formula la loi, connue sous le nom de "loi de Joule", sur la chaleur électrique, qui indique que la quantité de chaleur produite chaque seconde dans un conducteur par le passage du courant électrique est proportionnelle à la résistance électrique du conducteur et au carré du courant électrique. Joule a vérifié expérimentalement la loi de la conservation de l'énergie dans son étude sur la transformation de l'énergie mécanique en énergie thermique (relation entre joules et calories: il faut $1$ calorie, soit $4.18$ joules pour augmenter $1$ gramme d'eau de $1$ degré).

\phantomsection
\addcontentsline{toc}{section}{K}
\label{sec:K}

\parpic[l][t]{%
  \begin{minipage}{40mm}
    \fbox{\includegraphics[width=110px,height=140px]{img/medaillons/kepler.eps}}
  \end{minipage}
}
\textbf{Kepler, Johannes} (1571-1630) était astronome et physicien allemand, célèbre pour sa formulation et sa vérification des trois lois du mouvement planétaire. Ces lois sont maintenant connues sous le nom de "lois de Kepler". Son principal traité contient les formulations de deux des lois du mouvement planétaire. La première stipule que les planètes se déplacent selon des orbites elliptiques avec le Soleil comme foyer et la seconde, ou "loi des aires", énonce que la ligne imaginaire que l'on tracerait entre le Soleil et une planète balaie des aires identiques d'une ellipse pendant des intervalles de temps égaux; en d'autres termes, plus la planète se rapproche du Soleil, plus elle se déplace rapidement. Un autre traité contient une autre découverte sur le mouvement planétaire: le cube de la distance entre une planète et le Soleil divisé par la période orbitale de cette planète au carré est une constante et est la même pour toutes les planètes. Le mathématicien et physicien anglais Isaac Newton se reposa fortement sur les théories et les observations de Kepler pour formuler sa théorie de la force gravitationnelle. Kepler apporta également sa contribution dans le domaine de l'optique et développa en mathématiques un système infinitésimal qui fut le précurseur du calcul infinitésimal.

\parpic[l][t]{%
  \begin{minipage}{40mm}
    \fbox{\includegraphics[width=110px,height=140px]{img/medaillons/keynes.eps}}
  \end{minipage}
}
\textbf{Keynes, John Maynard} (1883-1946) était un économiste britannique. Il est le fondateur du "keynésianisme", doctrine économique qui encourage l'intervention de l'État au sein de l'économie, pour assurer le plein-emploi. Keynes est né dans une famille d'universitaires. À 7 ans, il entra à Perse School. Deux ans plus tard, il entrait en classe préparatoire à St Faith's. Avec les années, il se montra très prometteur et en 1894, il termina premier de sa classe et reçut un prix pour la première fois en mathématiques. Un an plus tard, il intègre l'Eton College où il brille et gagne, en 1899 et en 1900, le prix de mathématiques. En 1901, il finit premier en mathématiques, histoire et anglais. En 1902, il gagne sa place pour le King's college de Cambridge. Keynes est sans aucun doute une importante figure de l'histoire de la science économique qu'il révolutionna avec son oeuvre principale, la \textit{Théorie générale de l'emploi, de l'intérêt et de la monnaie} parue en 1936. L'ouvrage est considéré comme le traité de science sociale le plus influent du 20ème siècle dans la mesure où il a rapidement et continuellement modifié la façon dont le Monde a considéré l'économie et le rôle du pouvoir politique dans la société. Certains estiment qu'aucun autre ouvrage n'a eu une telle importance depuis en Europe, bien que l'ouvrage de Friedrich Hayek qui lui valut son Prix Nobel, \textit{The Road to Serfdom}, fasse la démonstration fulgurante des limites de la théorie keynésienne. Avec la \textit{Théorie générale}, Keynes a développé une théorie qui pouvait expliquer le niveau de la production et par conséquent de l'emploi ; le facteur déterminant étant la demande. Parmi les concepts révolutionnaires apportés par Keynes, on retiendra surtout: ceux de l'équilibre de sous-emploi où le chômage est possible pour un niveau donné de la demande effective, l'absence de mécanisme de régulation par les prix afin de résorber le chômage, une théorie de la monnaie fondée sur la préférence pour la liquidité, l'introduction de l'incertain et des prévisions, la notion d'efficacité marginale de l'investissement brisant la loi de Say (et renversant donc le lien de causalité épargne-investissement). Ces concepts accréditent la possibilité de politiques interventionnistes pour éliminer les récessions et freiner les surchauffes économiques. L'ensemble de ces concepts constitue ce qu'on appelle aujourd'hui la "macroéconomie".

\parpic[l][t]{%
  \begin{minipage}{40mm}
    \fbox{\includegraphics[width=110px,height=140px]{img/medaillons/kirchhoff.eps}}
  \end{minipage}
}
\textbf{Kirchhoff, Gustav Robert} (1824-1887) Né à Könisberg (aujourd'hui Kaliningrad en Russie) et décédé à Berlin, Kirchhoff étudie la physique-mathématique auprès de Franz Neumann. Après un doctorat en 1847, il devient conférencier à l'Université de Berlin avant d'obtenir, en 1850, le poste de professeur de physique extraordinaire à l'Université de Breslau. C'est là qu'il fait la connaissance du chimiste Robert Wilhelm Bunsen, avec qui il sera amené à travailler de nombreuses années. Leur collaboration se poursuivra en effet au-delà de 1854, date à laquelle Kirchhoff est nommé à professeur de physique à l'Université de Heidelberg. Élu vice-recteur de cette même université en 1865, il finit par accepter une chaire de physique théorique à Berlin en 1875. Kirchhoff est encore étudiant lorsqu'il commence à s'intéresser aux problèmes liés à l'électricité. En 1845, il établit la notion de potentiel électrique et énonce les lois de réseaux qui portent son nom (loi des noeuds, loi des mailles). Il généralise la loi d'Ohm sur le courant électrique à des conducteurs à trois dimensions et, plus tard, montre que le passage du courant à travers un conducteur se fait à la vitesse de la lumière. Sa rencontre avec Bunsen aboutit à la naissance de la spectroscopie. Ensemble, les deux chercheurs découvrent le caractère spécifique du spectre de la lumière émise par chaque corps chimique. Grâce à ce nouvel outil d'analyse, ils dépistent deux éléments encore inconnus: le césium (1860) et le rubidium (1861). La mise au point du spectroscope à prisme, pour analyser la lumière de substances en combustion, permet également à Kirchhoff d'établir la loi du rayonnement: le rapport des pouvoirs d'émission et d'absorption d'un corps, indépendant des propriétés de ce corps, est fonction de la température et de la longueur d'onde. Le pouvoir d'émission est ainsi proportionnel à celui du "corps noir", défini par Kirchhoff (1862) comme le corps parfaitement absorbant. Cette loi, qui explique notamment la présence des raies sombres d'absorption (dites "raies de Fraunhofer") dans le spectre de rayonnement solaire, marque le début d'une nouvelle ère en astrophysique et annonce l'avènement de la théorie des quanta de Planck.

\parpic[l][t]{%
  \begin{minipage}{40mm}
    \fbox{\includegraphics[width=110px,height=140px]{img/medaillons/klein.eps}}
  \end{minipage}
}
\textbf{Klein, Felix} (1849-1925)  fit ses études à Bonn, à Göttingen et à Berlin. En 1872, il devint professeur de mathématiques à l'Université d'Erlangen, où son cours inaugural fut l'énoncé des grandes lignes de son fameux programme d'Erlangen. Il enseigna ensuite à Münich (1875-1880), puis à l'université de Leipzig (1880-1886) et enfin à Göttingen (1886-1913). À partir de 1872, il édita les \textit{Mathematische Annalen} de Göttingen et fonda, en 1895, la grande \textit{Encyclopédie mathématique}, dont il supervisa la rédaction jusqu'à sa mort, à Göttingen. Il fut le chef incontesté de l'école mathématique allemande, et son influence fut très grande (il donna de nombreuses conférences à l'étranger, dont les États-Unis), notamment sur le développement de la géométrie, grâce à son programme d'Erlangen. Avec le texte, publié dans son ouvrage \textit{Gesammelte mathematische Abhandlungen} (1921-1923), Klein donne une définition de la géométrie englobant aussi bien la géométrie classique (c'est-à-dire euclidienne) que la géométrie projective, les géométries non euclidiennes, etc., mettant fin aux controverses stériles entre partisans de la géométrie synthétique et ceux de la géométrie analytique. Pour lui, une géométrie est l'étude des propriétés invariantes par un groupe donné de transformations: ainsi les théorèmes de géométrie classique sont l'expression de relations entre invariants du groupe des similitudes ; ceux de la géométrie projective entre covariants du groupe projectif. On doit aussi à Klein d'importants travaux sur l'équation différentielle hypergéométrique, sur les fonctions abéliennes, sur le groupe de l'icosaèdre régulier (\textit{Lectures on the Icosahedron}, 1914), sur les fonctions elliptiques, à partir desquelles il dégage la notion de fonction modulaire (\textit{Vorlesungen über die Theorie der automorphen Funktionen}, 1897-1902).

\parpic[l][t]{%
  \begin{minipage}{40mm}
    \fbox{\includegraphics[width=110px,height=140px]{img/medaillons/kolmogorov.eps}}
  \end{minipage}
}
\textbf{Kolmogorov, Andrei} (1903-1987) était un mathématicien russe dont les apports en mathématiques sont considérables. Kolmogorov est né à Tambov. Sa mère célibataire mourut à sa naissance et il fut élevé par sa tante avec les économies de son grand-père. On pense que son père fut tué lors de la guerre civile russe. Kolmogorov fut scolarisé à l'école du village de sa tante, et ses premiers efforts littéraires et articles mathématiques furent imprimés dans le journal de l'école. Adolescent, il conçut des machines à mouvement perpétuel, cachant tellement bien leurs défauts intrinsèques que ses professeurs d'enseignement secondaire n'arrivaient pas à les découvrir. En 1910, sa tante l'adopta et ils déménagèrent à Moscou, où il intégra un Gymnasium et y obtint son diplôme en 1920. Après avoir terminé ses études secondaires, il suit les cours à l'Université de Moscou et à l'institut Mendeleïev. Il étudie non seulement les mathématiques, mais aussi l'histoire russe et la métallurgie. En 1922, Kolmogorov publie ses premiers résultats concernant la théorie des ensembles et, en 1923 il publie ses travaux sur la théorie de l'intégration, sur l'analyse de Fourier et pour la première fois sur la théorie des probabilités et commence à devenir connu à l'étranger. Après la fin de ses études supérieures en 1925, il commence son doctorat auprès de Nikolaï Louzine, qu'il termine en 1929. En 1931, il reçoit une chaire de professeur à l'Université de Moscou. En 1933, paraît en allemand son manuel des \textit{Fondements de la théorie des probabilités} dans lequel il présente son axiomatisation du calcul des probabilités ainsi qu'une manière adaptée à traiter les processus stochastiques. La même année, il devient directeur de l'Institut de mathématiques de l'Université de Moscou. En 1934, il publie son travail sur la cohomologie et obtient, grâce à cette thèse, le titre de docteur en mathématique et en physique. Il obtient des récompenses des autorités soviétiques, comme l'Ordre de la science socialiste (1940), le prix Staline (1941) et plusieurs fois le prix Lénine. En 1941, il élabore une théorie fameuse de la turbulence des fluides. En 1953 et 1954, il décrit la théorie KAM (Kolmogorov-Arnold-Moser) de stabilité des systèmes dynamiques (un système mécanique complexe exactement soluble est stable si on le perturbe seulement un tout petit peu). Il introduit également la notion d'entropie métrique pour les systèmes dynamiques mesurés. En 1955, il devient docteur honoris causa de la Sorbonne. En 1962, il reçoit le Prix Balzan pour les mathématiques.

\parpic[l][t]{%
  \begin{minipage}{40mm}
    \fbox{\includegraphics[width=110px,height=140px]{img/medaillons/kronecker.eps}}
  \end{minipage}
}
\textbf{Kronecker, Leopold} (1823-1891) était un mathématicien allemand qui nous apparaît comme l'un des plus grands arithméticiens du 19ème siècle et l'un des fondateurs de la grande théorie des nombres algébriques. Ses travaux sur le corps de classes dans un cas particulier ont préparé ceux de Hilbert. Né à Liegnitz, dans une famille de riches commerçants, Kronecker suivit au gymnase les cours d'Ernst Kummer, qu'il devait retrouver plus tard comme professeur à l'Université de Breslau, puis comme collègue à Berlin, et qui, avec Peter Gustav Lejeune-Dirichlet, devait avoir l'influence la plus profonde sur le développement de sa pensée. Après avoir soutenu, en 1845, une thèse très originale sur les unités des corps cyclotomiques, il s'occupa pendant plusieurs années des affaires familiales, et ne put se livrer entièrement de nouveau aux recherches mathématiques qu'à partir de 1853. Élu, en 1860, membre de l'Académie des sciences de Berlin, il donna, à partir de cette époque, des cours libres à cette université, où il fut nommé professeur titulaire en 1883 et où il acheva sa vie. Bien que maniant avec virtuosité toutes les ressources de l'analyse (comme le montrent ses travaux sur les fonctions elliptiques, les séries de Dirichlet ou encore sa formule intégrale donnant le nombre des racines d'un système d'équations dans un espace à n dimensions), Kronecker est avant tout un algébriste et un arithméticien. Vers la fin de sa vie, il professait une doctrine tendant à rejeter l'infini actuel des mathématiques en ne gardant comme valable que ce qui pouvait être uniquement fondé sur les nombres entiers (ses polémiques avec Cantor sont restées célèbres). En algèbre, Kronecker fut l'un des animateurs les plus actifs du groupe de mathématiciens qui, dans les années 1860-1890, achevèrent de mettre sur pied l'algèbre linéaire et multilinéaire inaugurée par Arthur Cayley et Hermann Grassmann aux alentours de 1845. C'est ainsi qu'il reprit et compléta les travaux de Karl Weierstrass et fut l'un des premiers à comprendre et à utiliser les travaux d'Évariste Galois (publiés en 1846).

\phantomsection
\addcontentsline{toc}{section}{L}
\label{sec:L}

\parpic[l][t]{%
  \begin{minipage}{40mm}
    \fbox{\includegraphics[width=110px,height=140px]{img/medaillons/lagrange.eps}}
  \end{minipage}
}
\textbf{Lagrange, Joseph Louis} (1736-1813) Né à Turin et décédé à Paris, était comme l'un des plus grands mathématicien et astronome du 18ème siècle. Élève brillant issu d'un milieu aisé, il étudie au collège de Turin. Il prend goût pour les mathématiques par hasard à l'âge de 17 ans après la lecture d'un mémoire de Edmund Halley portant sur les applications de l'algèbre en optique. Le sujet l'intéresse au plus haut point. Dès lors, il se passionne pour les mathématiques qu'il étudie seul et assidûment. Il devient rapidement un mathématicien confirmé et ses premiers résultats ne se font pas attendre. Dans une lettre adressée à Leonhard Euler il jette les bases du calcul variationnel. Cet échange est le début d'une longue correspondance entre les deux hommes. Lagrange a alors 19 ans et enseigne à l'école d'artillerie de Turin où il fut nommé en 1755. Il fonde en 1758 l'Académie des Sciences de Turin qui publiera ses premiers résultats sur l'application du calcul variationnel à des problèmes de mécanique (propagation du son, corde vibrante...). En 1764, ses travaux sur les librations de la Lune (petites variations de son orbite) sont récompensés par le Grand Prix de l'Académie des Sciences de Paris. Il introduisit de nouvelles méthodes pour le calcul des variations et pour l'étude des équations différentielles, qui lui permirent de donner un exposé systématique de la mécanique dans son célèbre ouvrage Mécanique analytique (1788). Il travailla sur la théorie additive des nombres. On lui doit le théorème sur la décomposition d'un entier en quatre carrés. Dans l'étude des équations algébriques, il introduisit des concepts qui conduiront à la théorie des groupes développée plus tard par Abel et Galois. En physique, en précisant le principe de moindre action, avec le calcul des variations, vers 1756, il invente la fonction de Lagrange, qui vérifie les équations de Lagrange, puis développe la mécanique analytique, vers 1788, pour laquelle il introduit les multiplicateurs de Lagrange. Il entreprend aussi des recherches importantes sur le problème des trois corps en astronomie, un de ses résultats étant la mise en évidence des points de libration (dits points de Lagrange) (1772).

\parpic[l][t]{%
  \begin{minipage}{40mm}
    \fbox{\includegraphics[width=110px,height=140px]{img/medaillons/langevin.eps}}
  \end{minipage}
}
\textbf{Langevin, Paul} (1872-1946) était un physicien français né et décédé à Paris. Très jeune, Langevin manifeste des dons exceptionnels. Encouragé par ses instituteurs, il parcourt rapidement les divers échelons de l'enseignement obligatoire avant d'entrer à 16 ans à l'École Supérieure de Physique et de Chimie Industrielle de la Ville de Paris. Langevin y suit les cours et l'enseignement de laboratoire de Pierre Curie, avec lequel il se lie d'amitié. À sa sortie de l'école, il renonce à la carrière d'ingénieur et décide, sur les conseils de Pierre Curie, de se consacrer à la recherche et à l'enseignement. Aussi, se présente-t-il à l'École Normale Supérieure où il est reçu premier en 1894. En 1897, il bénéficie d'une bourse pour aller travailler un an au Cavendish Laboratory de Cambridge, haut lieu de la science européenne où se trouvent alors E. Rutherford et J. J. Thomson. De retour en France, il soutient sa thèse en 1902, est nommé professeur suppléant, puis professeur au Collège de France. En 1904, il succède à Pierre Curie à l'École de Physique et de Chimie, dont il devient directeur en 1925. L'oeuvre de Langevin se situe dans cette longue période de transition qui, de 1900 à 1930, mène de la physique classique à la physique moderne, dominée par la théorie de la relativité et la théorie quantique. Ses premiers travaux (sur l'ionisation des gaz) l'amènent à élaborer en 1905 son modèle théorique majeur, qui devrait par la suite servir de base à de nombreuses autres explications des propriétés macroscopiques de la matière, dans lequel les électrons à l'intérieur des atomes décrivent des orbites fermées, conférant ainsi aux atomes des propriétés analogues à celles de petits aimants. En 1906 il aboutit au résultat étonnant selon lequel l'inertie serait une propriété de l'énergie..., du moins dans le cas de l'électron. Ce n'est que quelques mois plus tard qu'il lira le mémoire d'Einstein sur la théorie de la relativité restreinte à laquelle il va consacrer son enseignement dans ses cours au Collège de France. Langevin est aussi à l'origine des fameux Congrès Solvay qui, dès 1911, réunirent périodiquement tous les grands noms de la physique, et où furent largement discutés les concepts de la théorie quantique. C'est d'ailleurs grâce à Langevin que les travaux de son élève Louis de Broglie sur la mécanique ondulatoire connurent la diffusion qu'ils méritaient: d'abord étonné, Langevin fut très vite convaincu de la justesse des idées de De Broglie et inscrivit immédiatement la nouvelle mécanique ondulatoire au programme de son cours au Collège de France. Fidèle à l'idéal de clarté pédagogique qui fut toujours le sien, Langevin a par ailleurs effectué, sur les concepts encore en gestation de la théorie quantique, un travail d'analyse et de refonte épistémologiques dont on mesure aujourd'hui l'importance.

\parpic[l][t]{%
  \begin{minipage}{40mm}
    \fbox{\includegraphics[width=110px,height=140px]{img/medaillons/langevin.eps}}
  \end{minipage}
}
\textbf{Laplace, Pierre Simon} (1749-1827) Né à Beaumont-en-Auge et décédé à Paris, fils de cultivateur, Laplace s'initia aux mathématiques à l'École Militaire de cette petite ville. Il y commença son enseignement. Il doit cette éducation à ses voisins aisés qui avaient détecté son intelligence exceptionnelle. À 18 ans, il arrive à Paris avec une lettre de recommandation pour rencontrer le mathématicien d'Alembert, mais ce dernier refuse de rencontrer l'inconnu. Mais Laplace insiste: il envoie à d'Alembert un article qu'il a écrit sur la mécanique classique. D'Alembert en est si impressionné qu'il est tout heureux de patronner Laplace. Il lui obtient un poste d'enseignement en mathématiques. L'oeuvre la plus importante de Laplace concerne le calcul des probabilités, les équations différentielles (laplacien) et la mécanique céleste. Il établit aussi, grâce à ses travaux avec Lavoisier entre 1782 et 1784 la relation des transformations adiabatiques d'un gaz, ainsi que deux lois fondamentales de l'électromagnétisme. En mécanique, c'est avec le mathématicien Joseph-Louis de Lagrange, que Laplace résume ses travaux et réunit ceux de Newton, Halley, Clairaut, d'Alembert et Euler, concernant la gravitation universelle (particulièrement le problème de stabilité du système solaire), dans les 5 volumes de sa \textit{Mécanique céleste} (1798-1825). On rapporte (mais c'est très probablement une légende) que, feuilletant la \textit{Mécanique céleste}, Napoléon fit remarquer à Laplace qu'il n'y était nulle part fait mention de Dieu. "Je n'ai pas eu besoin de cette hypothèse", rétorqua le savant qui n'était par ailleurs pas modeste (se considérant - probablement à juste titre - comme le meilleur mathématicien de sa génération). Il aussi un des premiers scientifiques à concevoir l'existence des Trous Noirs et la notion de "collapsus gravitationnel" (effondrement gravitationnel).

\parpic[l][t]{%
  \begin{minipage}{40mm}
    \fbox{\includegraphics[width=110px,height=140px]{img/medaillons/anonymous.eps}}
  \end{minipage}
}
\textbf{Laurent, Pierre Alphonse} (1813-1854) était un mathématicien français né à Paris et qui s'est rendu célèbre pour la découverte de la série de Laurent dans le domaine de l'analyse complexe et qui a un grand impact dans le calcul de certaines intégrales en physique. Il est entré à l'École Polytechnique de Paris en 1830. Laurent a été diplômé en 1832 comme un des meilleurs élèves de l'année et entra dans le corps d'ingénierie comme lieutenant. Pendant la gestion de ses projets de développement du port du Havre, Laurent écrivait sa première publication mathématique sur les séries de Laurent. Cette recherche était contenue dans un mémoire soumis au Grand prix de l'Académie des sciences en 1843, mais sa candidature étant trop tardive, l'article n'a jamais été inscrit au prix. Cependant, Cauchy fit une référence dans ses travaux au papier de Laurent 3 mois plus tard. Le même problème se réitéra pour une autre publication importante de Laurent quelques mois plus tard. Après ces événements, Laurent, déçu changea de domaine de recherche pour se concentrer sur la physique (mathématique appliquée). Cauchy lui propose un poste vacant à l'Académie des Sciences en 1846, mais sa candidature ne fut pas retenue. Laurent est décédé à Paris, à l'âge de 41 ans. Ses écrits n'ont été publiés qu'après son décès.

\parpic[l][t]{%
  \begin{minipage}{40mm}
    \fbox{\includegraphics[width=110px,height=140px]{img/medaillons/lavoisier.eps}}
  \end{minipage}
}
\textbf{Lavoisier, Antoine Laurent} (1743-1794) était un chimiste français, considéré comme le fondateur de la chimie moderne. Lavoisier naquit à Paris et fit ses études au collège Mazarin. Il fut élu membre de l'Académie des Sciences en 1768. Il occupa de nombreux postes, y compris celui de directeur des Poudreries Nationales en 1776, de membre de la Commission pour l'Établissement du Nouveau Système de Poids et Mesures en 1790 et de secrétaire de la Trésorerie en 1791. Il tenta d'introduire des réformes dans le système monétaire et fiscal français, ainsi que dans le système agricole. Lavoisier fut l'un des premiers à réaliser des expériences chimiques réellement quantitatives. Il montra qu'en dépit du changement d'état de la matière au cours d'une réaction chimique, la quantité de matière restait constante entre le début et la fin de chaque réaction. Ces expérimentations ont fourni des preuves en faveur de la loi de la conservation de la matière. Lavoisier fit également des recherches sur la composition de l'eau, dont il appela les composants "oxygène" et "hydrogène". L'une des plus importantes expériences de Lavoisier concerna la nature de la combustion (ou brûlage). Il démontra ainsi que le processus de combustion implique la présence d'oxygène. Il démontra également le rôle de l'oxygène dans la respiration chez les animaux et chez les végétaux. Les explications de Lavoisier sur la combustion remplacèrent la doctrine du phlogistique. Celle-ci postulait en effet qu'une substance se dégageait, le "phlogiston", lorsque la matière se consumait. Étant l'un des vingt-huit fermiers généraux, Lavoisier est bêtement stigmatisé comme traître par les révolutionnaires en 1794 et guillotiné lors de la terreur à Paris le 8 mai 1794, à l'âge de 50 ans, en même temps que l'ensemble de ses collègues.

\parpic[l][t]{%
  \begin{minipage}{40mm}
    \fbox{\includegraphics[width=110px,height=140px]{img/medaillons/lebesgue.eps}}
  \end{minipage}
}
\textbf{Lebesgue, Henri Léon} (1875-1941) Né à Beauvais est décédé à Paris est un ancien élève de l'École Normale Supérieur, il eut Émile Borel comme professeur (à qui l'on doit les premiers travaux importants en théorie de la mesure). Après quelques années au lycée de Nancy, Lebesgue enseignera à Rennes. C'est pendant cette période qu'il se fera connaître par son élégante théorie de la mesure. Professeur à la Sorbonne puis au collège de France, il sera élu à l'Académie des Sciences en 1922. Par sa théorie des fonctions mesurables (1901) s'appuyant sur les tribus boréliennes  (du nom du mathématicien Émile Borel), Lebesgue a profondément remanié et généralisé le calcul intégral. Sa théorie de l'intégration (1902-1904) répond aux besoins des physiciens en permettant la recherche et l'existence de primitives pour des fonctions "irrégulières". On lui doit aussi la transformée de Fourier établie dans la fin des années 1930. Il est nommé professeur à la Sorbonne en 1910, puis au Collège de France en 1921. Il donne également des cours à l'École Supérieure de Physique et de Chimie Industrielles de la ville de Paris de 1927 à 1937 et à l'École Normale Supérieure de Sèvres.

\parpic[l][t]{%
  \begin{minipage}{40mm}
    \fbox{\includegraphics[width=110px,height=140px]{img/medaillons/lee.eps}}
  \end{minipage}
}
\textbf{Lee, Tsung-Dao} (1926-) Né à Shanghai (Chine), Lee Tsung-Dao est le fils d'un homme d'affaires. La guerre sino-japonaise de 1937-1945 lui fit quitter l'Université Kweichow dans la province du Zhejiang, pour rejoindre celle de Kunming, dans le Yunnan, où il rencontra Yang Chen-Ning, dont il sera longtemps l'ami et le collaborateur. Une bourse du gouvernement chinois lui permit de terminer ses études à l'Université de Chicago (États-Unis), où il soutint sa thèse sur le contenu en hydrogène des naines blanches, en 1950. Membre de l'Institute for Advanced Study de Princeton (New Jersey) de 1951 à 1953, il devint bientôt, à 29 ans, le plus jeune professeur de l'Université Columbia, à New York.  En 1956, les physiciens étaient en butte à une énigme surgie du dépouillement des données fournies par l'accélérateur de particules du laboratoire national de Brookhaven, près de New York: deux particules, appelées "tau" et "thêta", semblaient avoir même masse et mêmes interactions nucléaires, mais différaient par leurs produits de désintégration. Lee et Yang proposèrent qu'elles n'étaient qu'une seule particule, maintenant notée "K0", et que l'interaction faible responsable de leur désintégration ne respectait pas la symétrie de parité. Ils en conclurent qu'il était indispensable de soumettre à vérification expérimentale le fait que l'interaction faible distingue la droite de la gauche. Six mois suffirent à l'équipe du National Bureau of Standards de Washington, mobilisée par la physicienne chinoise Wu Chien-Shiung, pour montrer que des noyaux radioactifs de Cobalt 60 polarisés émettaient plus d'électrons dans une direction que dans la direction opposée. Confirmée rapidement par plusieurs autres groupes expérimentaux, cette violation de la symétrie miroir valut à Lee Tsung-Dao et à Yang Chen-Ning de se partager le prix Nobel de physique 1957.

\parpic[l][t]{%
  \begin{minipage}{40mm}
    \fbox{\includegraphics[width=110px,height=140px]{img/medaillons/anonymous.eps}}
  \end{minipage}
}
\textbf{Legendre, Adrien Marie} (1752-1833) était un mathématicien français né à Paris et décédé à Auteil. Il occupe la chaire de Mathématiques de l'École Militaire de Paris de 1775 à 1780. En 1783, il devient membre de l'Académie des Sciences. En 1787, il est nommé commissaire chargé des opérations géodésiques. Les centres d'intérêts de Legendre étaient variés: analyse, théorie des nombres, géométrie, statistiques (méthodes des moindres carrés) et mécanique (transformation de Legendre en mécanique analytique et thermodynamique). Environ un siècle avant que l'on en obtienne les preuves, il conjectura le théorème des nombres premiers (distribution asymptotique des nombres premiers) ainsi que la loi de réciprocité quadratique (expression d'un nombre premier comme un carré modulo un autre nombre premier). Toute sa vie, il s'intéressa aux intégrales elliptiques, dont les travaux allaient finalement donner naissance aux courbes elliptiques, sujet très étudié par les mathématiciens contemporains. Il laisse en héritage à la communauté mathématique du 19ème siècle un traité de géométrie élémentaire, qui s'avère très précieux dans le monde de l'enseignement..

\parpic[l][t]{%
  \begin{minipage}{40mm}
    \fbox{\includegraphics[width=110px,height=140px]{img/medaillons/leibniz.eps}}
  \end{minipage}
}
\textbf{Leibniz, Gottfried Wilhelm} (1646-1716) Né à Leipzig et décédé à Hanovre était philosophe, juriste et mathématicien considéré comme un des plus brillants esprits du 17ème siècle. Fils d'un jurisconsulte il obtient son baccalauréat en philosophie ancienne en 1663 et écrira un peu plus tard une théorie des probabilités en Droit. Il entre ensuite à l'université de Leipzig et en 1666 obtient son doctorat en Droit... En 1669, il devient conseiller à la Chancellerie de l'Électorat de Mayence. Il est envoyé en 1672 à Paris, en mission diplomatique dit-on, pour convaincre Louis XIV de porter ses conquêtes vers l'Égypte plutôt que l'Allemagne. Il y reste jusqu'en 1676 et y rencontre les grands savants de l'époque. C'est pendant cette période que Leibniz travail sur ses travaux scientifiques. En 1676 il est nommé bibliothécaire du Brunswick-Lunebourg et s'y occupe aussi de mathématique, de physique, de religion et de diplomatie. Leibniz contribua aux mathématiques en découvrant, en 1675, les principes fondamentaux du calcul infinitésimal. Cette découverte fut réalisée indépendamment des découvertes de Newton, qui inventa son système de calcul en 1666. Le système de Leibniz fut publié en 1684, celui de Newton en 1687, date à laquelle la méthode de notation imaginée par Leibniz fut adoptée et on le considère aussi comme un pionnier du développement de la logique mathématique.

\parpic[l][t]{%
  \begin{minipage}{40mm}
    \fbox{\includegraphics[width=110px,height=140px]{img/medaillons/landau.eps}}
  \end{minipage}
}
\textbf{Landau, Lev Davidovich} (1908-1968) Né en Azerbaïdjan et décédé à Moscou, il était le fils d'un ingénieur et médecin. Après avoir achevé ses études au Département de Physique de l'Université de Léningrad à l'âge 19 ans, il commence sa carrière scientifique à l'Institut Physico-technique de Léningrad. De 1932 à 1937 il est le chef du Département Théorique de l'Institut Physico-technique Ukrainien à Kharkov et dès 1937 il est nommé chef du Département Théorique de l'Institut pour les Problèmes Physiques de l'Académie des Sciences de l'URSS à Moscou. Le travail de Landau couvre toutes les branches de physique théorique, aux limites de la mécanique liquide à la théorie des champs quantiques. Une grande partie de ses papiers se réfère à la théorie de l'état condensé. Ils ont commencé en 1936 par une formulation d'une théorie générale des transitions de phase du deuxième ordre. Après la découverte de Kapitsa, en 1938, de la superfluidité de l'hélium liquide, Landau a engagé la vaste recherche qui l'a mené à la construction de la théorie complète des liquides quantiques aux températures très basses. Parmi ses écrits, couvrant une vaste gamme de thèmes liés aux phénomènes physiques, on relève plus de cent articles et de nombreux livres, dont le célèbre \textit{Cours de physique théorique}, publié en 1943 avec E.M.Lifchitz. Landau a dominé toute la physique théorique de 1930 à 1965. Il avait créé un ensemble d'examens de physique théorique, appelé le "Minimum théorique" que les étudiants ou chercheurs confirmés devaient  passer pour entrer dans son groupe de recherche, examen qui incluait des problèmes dans toutes les branches des mathématiques.

\parpic[l][t]{%
  \begin{minipage}{40mm}
    \fbox{\includegraphics[width=110px,height=140px]{img/medaillons/levicivita.eps}}
  \end{minipage}
}
\textbf{Levi-Civita, Tullio} (1873-1941) Né à Padoue et décédé à Rome il fut diplômé en 1892 de la faculté de mathématiques de l'Université de Padoue. En 1894, il obtint un diplôme d'enseignement au Collège d'enseignement de la faculté des sciences de Pavie. En 1898, il fut nommé à la tête de la chaire de mécanique analytique et céleste de Padoue et il y rencontra Libera Trevisani, une de ses élèves, avec qui il se maria en 1914. Il resta à Padoue jusqu'en 1918, puis fut nommé à la chaire d'analyse supérieure à l'Université de Rome, où il prit 2 ans plus tard la chaire de mécanique. Avant tout physicien, ses travaux s'orientent principalement vers l'électromagnétisme et les théories de Lorentz et de Maxwell. En 1900, Ricci et lui publièrent \textit{La Théorie des tenseurs dans les méthodes de calcul différentiel et leurs applications} qu'Einstein utilisa afin de mieux maîtriser le calcul tensoriel, un outil-clef pour Einstein dans le développement de sa théorie de la Relativité Générale. Levi-Civita discuta aussi d'une série de problèmes à propos du champ gravitationnel statique dans sa correspondance avec Einstein entre les années 1915-1917. Leur correspondance tournait autour de la formulation variationnelle des équations de champs gravitationnelles et leurs propriétés covariantes, et la définition de l'énergie gravitationnelle et de l'existence d'ondes gravitationnelles. En 1933 Levi-Civita contribua aussi aux équations de la mécanique quantique de Dirac.

\parpic[l][t]{%
  \begin{minipage}{40mm}
    \fbox{\includegraphics[width=110px,height=140px]{img/medaillons/lie.eps}}
  \end{minipage}
}
\textbf{Lie, Sophus} (1842-1899) était un mathématicien norvégien qui fit ses études à l'Université de Christiana. Il donna des leçons particulières pour gagner sa vie, et passa avec Klein l'hiver 1869-1870 à Berlin, l'été 1870 à Paris. En 1872, une chaire de mathématiques fut créée pour lui à Christiana, et en 1886, il succéda à Klein à Leipzig. Outre des travaux en géométrie projective de l'espace, on retient surtout de Lie l'étude de structures algébriques nouvelles qu'il applique à la géométrie, jusqu'à la création de toutes pièces de la théorie des groupes et algèbres qui portent son nom. Dans la notion de groupe et d'algèbre de Lie, interviennent des propriétés de continuité (groupe topologique), annonçant la nouvelle branche importante des mathématiques que sera la topologie. Les travaux de Lie, dans ce domaine, seront principalement poursuivis par Élie Cartan.

\parpic[l][t]{%
  \begin{minipage}{40mm}
    \fbox{\includegraphics[width=110px,height=140px]{img/medaillons/lindemann.eps}}
  \end{minipage}
}
\textbf{Lindemann, Ferdinand} (1852-1939) Né à Hanovre et décédé à Münich Lindemann a été le premier mathématicien à démontrer la transcendance de $\pi$. Quand Ferdinand était âgé de 2 ans, il se déplaça à Schwerin où il passa ses années d'enfance et sa scolarité primaire. Comme il était de pratique à cette époque en Allemagne pendant la seconde moitié du 19ème, Lindemann se déplaça fréquemment d'une université à l'autre. Il commença ses études à Göttingen en 1870 et y fut grandement influence par Clebsch. Plus tard, Lindemann qui avait établi des très bonnes relations avec Clebsch rédigea à nouveau les notes de géométrie de ce dernier après son décès pour leur publication en 1876. Ensuite, Lindemann étudia à Erlangen à Münich où il effectua son travail de doctorat sous la direction de Klein sur les géométries non-euclidiennes et ses applications à la physique. Après avoir obtenu son doctorat, Lindemann fit des visites importantes à des centres de mathématiques anglais et français. En Angleterre, il visita Oxford, Cambridge et Londres, alors qu'en France, il passa la majeure partie de son temps à Paris où il fut grandement influencé par Chasles, Bertrand, Jordan et Hermite. Lorsqu'il retourna en Allemagne, Lindemann travailla sur des publications permettant sa réintégration et sa reconnaissance dans le domaine scientifique Allemand. Ce fut enfin en 1877 qu'il fut nominé professeur extraordinaire à l'Université de Würzburg et professeur ordinaire à l'université de Freiburg en 1879. Le principal travail de Lindemann porta sur la géométrie et l'analyse. En 1873, alors que Lindemann venait d'avoir obtenu son doctorat, Hermite démontra la transcendance du nombre d'Euler $e$. Peu de temps après, Lindemann rencontra Hermite à Paris et discuta des méthodes utilisées pour la démonstration. Ainsi, utilisant un raisonnement similaire, Lindemann démontra en 1882 la transcendance de $\pi$ (sur la base que le nombre d'Euler est lui-même transcendant).

\parpic[l][t]{%
  \begin{minipage}{40mm}
    \fbox{\includegraphics[width=110px,height=140px]{img/medaillons/liouville.eps}}
  \end{minipage}
}
\textbf{Liouville, Joseph} (1809-1882) Né à St-Omer et décédé à Paris il fut un artisan des mathématiques déployant une activité considérable dans l'enseignement et la diffusion des idées mathématiques de son temps. Il est le fondateur du \textit{Journal de mathématiques pures et appliquées} appelé traditionnellement "Journal de Liouville". Ses principaux travaux portent sur l'analyse et on lui doit un important théorème sur l'approximation des irrationnels algébriques. L'élection de Joseph Liouville à l'Assemblée constituante de 1848 est seule à rompre l'unité d'une carrière toute scientifique: sorti de l'École Polytechnique en 1827, il y revenait en 1833 comme répétiteur puis professeur d'analyse. Dès sa 31ème année, il était élu à l'Académie des Sciences, dans la section d'astronomie. Il fut un des meilleurs professeurs de son temps, et ses cours, à Polytechnique et au Collège de France, prirent une grande part de son activité. Liouville fonda le \textit{Journal de mathématiques pures et appliquées} en 1836 et le géra pendant 39 ans. Ses tâches d'académicien et d'éditeur lui ôtèrent la liberté d'esprit nécessaire à une recherche approfondie ce pour quoi il se plaint. Mais il mit à profit l'une et l'autre tâche pour aider plusieurs jeunes mathématiciens de grand avenir, par exemple C. Hermite et C. Jordan, par des rapports élogieux devant l'Académie, ou par la publication de leurs travaux dans son journal. Quant à lui, il y publia surtout de courtes notes sur un grand nombre de questions: analyse, arithmétique, géométrie, mécanique, astronomie. Il partage avec A. Cauchy le mérite d'avoir soumis l'analyse à une règle de rigueur souvent transgressée au 18ème siècle, et ce mérite est d'autant plus grand que le langage mathématique de son temps n'aidait guère à la rigueur.

\parpic[l][t]{%
  \begin{minipage}{40mm}
    \fbox{\includegraphics[width=110px,height=140px]{img/medaillons/lobatchevski.eps}}
  \end{minipage}
}
\textbf{Lobachevsky, Nikolai Ivanovich} (1792-1856) était un mathématicien russe né à Nijni-Novgorod et décédé à Kazan. Lobatchevski étudia à l'Université de Kazan, où il enseigna à partir de 1812 et occupa la chaire de mathématiques pures de 1822 à 1846. Sous l'influence de Gauss et de Laplace, ses premiers travaux sont: \textit{Théorie du mouvement elliptique des corps célestes} et \textit{De la solution de l'équation algébrique complexe simple}. Mais ses principales recherches concernent la géométrie. Son premier ouvrage, \textit{Géométrie} (1823), jugé trop révolutionnaire (il utilisait le système métrique), ne pourra être publié de son vivant. En 1826, Lobatchevski expose devant ses collègues de l'université un mémoire qui montre qu'il fut l'un des premiers mathématiciens à être convaincu de la possibilité d'une géométrie différente de celle d'Euclide. Malgré le scepticisme de ses collègues, il continue l'étude de cette nouvelle géométrie (où le postulat d'Euclide est remplacé par le postulat suivant, dit "postulat de Lobatchevski": par tout point extérieur à une droite, il passe une infinité de parallèles à cette droite) et consacre sa vie de mathématicien à essayer de convaincre le monde scientifique. Il publie successivement \textit{Éléments de géométrie} (1829), \textit{Nouveaux Éléments de géométrie avec la théorie complète des parallèles} (1838) et \textit{Pangéométrie} (1855). Mais la pleine reconnaissance de la valeur de ses travaux ne viendra qu'après sa mort (lorsque Eugenio Beltrami, en 1868, construira un modèle de la géométrie de Lobatchevski: la pseudo-sphère). En plus de ses recherches mathématiques, Lobatchevski fut l'animateur de l'Université de Kazan: recteur de 1827 à 1846, il eut la charge de la bibliothèque de l'université, mit en place son observatoire, organisa son muséum et dirigea la construction de nouveaux locaux universitaires.

\parpic[l][t]{%
  \begin{minipage}{40mm}
    \fbox{\includegraphics[width=110px,height=140px]{img/medaillons/lorentz.eps}}
  \end{minipage}
}
\textbf{Lorentz, Hendrik} (1853-1928) Né à Arnhem et décédé à Haarlem (Pays-Bas) il avait amélioré la théorie électromagnétique de Maxwell dans sa thèse doctorale sur la théorie de la réflexion et la réfraction de la lumière qu'il présenta en 1875. Il fut nommé professeur de physique-mathématique à l'Université de Leyde en 1878. Il est resté dans cet établissement jusqu'en 1912 où Ehrenfest a été nommé à sa place. Lorentz est ensuite nommé directeur de recherche à l'Institut de Teyler, Haarlem. Il a maintenu une position honorifique à Leyde, où il a continué à donner quelques cours. Avant que l'existence des électrons ait été confirmée, Lorentz a proposé que les vagues de lumière étaient dues aux oscillations d'une charge électrique dans l'atome. Lorentz a développé sa théorie mathématique de l'électron pour lequel il a reçu le prix Nobel en 1902. Le prix Nobel a été attribué conjointement à Lorentz et à Pieter Zeeman, un étudiant de Lorentz. Zeeman avait vérifié expérimentalement le travail théorique de Lorentz sur la structure atomique, démontrant l'effet d'un champ magnétique fort sur les oscillations en mesurant le changement de la longueur d'onde de la lumière produite. Lorentz est également célèbre pour son travail sur la contraction de Fitzgerald-Lorentz, qui est une contraction dans la longueur d'un objet aux vitesses relativistes. Les transformations de Lorentz, qu'il a présentées en 1904, forment la base de la théorie de Relativité Restreinte d'Einstein qui a l'époque était appelée "théorie de la relativité d'Einstein-Lorentz". Elles décrivent l'augmentation de la masse, du rapetissement de la longueur, et de la dilatation de temps d'un corps se déplaçant aux vitesses proches de celle de la lumière. Lorentz était président de la première conférence de Solvay tenue à Bruxelles en automne de 1911. Cette conférence avait pour sujet les deux approches de la théorie atomique, à savoir la théorie classique et la physique quantique. Cependant, Lorentz n'a jamais entièrement accepté la théorie quantique et a toujours espéré qu'il serait possible de l'incorporer de nouveau dans l'approche classique.

\parpic[l][t]{%
  \begin{minipage}{40mm}
    \fbox{\includegraphics[width=110px,height=140px]{img/medaillons/lucas.eps}}
  \end{minipage}
}
\textbf{Lucas, Edward} (1842-1891) Né à Guildford et décédé à Bath est un pasteur anglican, qui s'inquiéta de la croissance trop importante de la population en Angleterre au début de la révolution industrielle (de 1750 à 1900). Sa crainte tournait autour de l'idée que la progression démographique est plus rapide que l'augmentation des ressources, d'où une paupérisation de la population. Les anciens régulateurs démographiques (les guerres et les épidémies) ne jouant plus leurs rôles, il imagine de nouveaux obstacles, comme la limitation de la taille des familles et le recul de l'âge du mariage. Ces propositions ne sont appliquées à ce jour, toutes les deux, qu'en Chine, qui est en effet obligée de limiter sévèrement sa démographie. Les prévisions sinistres de Malthus sont dans la réalité mises à mal, car il n'imaginait pas une si grande augmentation des ressources et des rendements agricoles (révolution verte: chimie appliquée à l'agronomie ce qui n'est pas pour autant bénéfique...); les nouveaux moyens d'échanges internationaux de biens de subsistance (contribuant à la pollution des océans, au déforestage et au passage à la paupérisation des régions productrices...); le fait que le trop plein d'individus émigrerait vers les États-Unis ou les colonies. En revanche, si les prévisions de Malthus ne sont pas au rendez-vous, sa théorie garde tous ses droits. Il est exact que la population est en croissance dans certains pays (Arabie Saoudite: six enfants par femme) il est aussi exact (et heureux) que les progrès de l'hygiène et de la médecine augmentent la taille de la population, il est exact que les ressources renouvelables sur Terre sont limitées, in fine par l'énergie solaire que reçoit celle-ci, qui elle-même détermine la biomasse, sauf découverte scientifique majeure... et dans ces conditions, la mathématique est formelle: il ne sera pas possible à la population Terrestre d'augmenter indéfiniment, et la régulation devra intervenir à un moment ou à un autre, et d'une manière ou d'une autre (ne serait que par la pollution non gérée par certaines populations dont le sens des responsabilités et des priorités est très relatif...)!

\phantomsection
\addcontentsline{toc}{section}{M}
\label{sec:M}

\parpic[l][t]{%
  \begin{minipage}{40mm}
    \fbox{\includegraphics[width=110px,height=140px]{img/medaillons/malthus.eps}}
  \end{minipage}
}
\textbf{Malthus, Thomas Robert} (1766-1834) Né à Guildford et décédé à Bath est un pasteur anglican, qui s'inquiéta de la croissance trop importante de la population en Angleterre, au début de la révolution industrielle (de 1750 à 1900). Sa crainte tournait autour de l'idée que la progression démographique est plus rapide que l'augmentation des ressources, d'où une paupérisation de la population. Les anciens régulateurs démographiques (les guerres et les épidémies) ne jouant plus leurs rôles, il imagine de nouveaux obstacles, comme la limitation de la taille des familles et le recul de l'âge du mariage. Ces propositions ne sont appliquées à ce jour, toutes les deux, qu'en Chine, qui est en effet obligée de limiter sévèrement sa démographie. Les prévisions sinistres de Malthus sont dans la réalité mises à mal, car il n'imaginait pas une si grande augmentation des ressources et des rendements agricoles (révolution verte: chimie appliqué à l'agronomie ce qui n'est pas pour autant bénéfique...); les nouveaux moyens d'échanges internationaux de biens de subsistance (contribuant à la pollution des océans au passage...); le fait que le trop plein d'individus émigrerait vers les États-Unis ou les colonies. En revanche, si les prévisions de Malthus ne sont pas au rendez-vous, sa théorie garde tous ses droits. Il est exact que la population est en croissance dans certains pays (Arabie Saoudite: 6 enfants par femme) il est aussi exact (et heureux) que les progrès de l'hygiène et de la médecine augmentent la taille de la population, il est exact que les ressources renouvelables sur Terre sont limitées, in fine par l'énergie solaire que reçoit celle-ci, qui elle-même détermine la biomasse, sauf découverte scientifique majeure... et dans ces conditions, la mathématique est formelle: il ne sera pas possible à la population Terrestre d'augmenter indéfiniment, et la régulation devra intervenir à un moment ou à un autre, et d'une manière ou d'une autre!

\parpic[l][t]{%
  \begin{minipage}{40mm}
    \fbox{\includegraphics[width=110px,height=140px]{img/medaillons/marconi.eps}}
  \end{minipage}
}
\textbf{Marconi, Guglielmo} (1874-1937) Né et décédé à Rome, c'était un physicien, inventeur et homme d'affaires italien. Il est, avec Karl Ferdinand Braun, colauréat du prix Nobel de physique de 1909 en reconnaissance de leurs contributions au développement de la télégraphie sans fil (on peut considérer qu'il est l'origine des appareils de transmission/réception d'ondes électromagnétiques et donc de la radio et télévision hertzienne). Marconi est né dans une famille aisée, second fils de Giuseppe Marconi, un propriétaire italien, et d'une mère irlandaise, Annie Jameson, petite-fille du fondateur de la Distillerie Jameson Whiskey. Il a fait ses études à Bologne dans le laboratoire d'Augusto Righi, à Florence, à l'Institut Cavallero et, plus tard, à Livourne. Il fait en 1985 des expériences sur les ondes découvertes par Heinrich Rudolf Hertz sept ans auparavant. Il reproduit le matériel utilisé par Hertz en l'améliorant avec un cohéreur de Branly pour augmenter la sensibilité et l'antenne de Alexandre Popov. Après ses toutes premières expériences en Italie, il réalise dans les Alpes suisses à Salvan une liaison de 1.5 [km] durant l'été 1895. L'année d'après, faute d'être suivi par ses compatriotes, il part pour l'Angleterre, poursuit ses expériences et dépose un brevet. En 1897 il effectue la première communication en morse à plus de 13 [km] entre Lavernock (Pays de Galles) et Brean (Angleterre) par-dessus le Canal de Bristol. L'année d'après, il ouvre la première usine de radios au monde, à Chelmsford, Angleterre. Au début du 20ème siècle le nom Marconi est (malheureusement) plutôt connu comme étant le propriétaire du groupe de cinéma Pathé (qui s'appelle en réalité Pathé-Marconi).

\parpic[l][t]{%
  \begin{minipage}{40mm}
    \fbox{\includegraphics[width=110px,height=140px]{img/medaillons/mandelbrot.eps}}
  \end{minipage}
}
\textbf{Mandelbrot, Benoit} (1924-2010) est né à Varsovie et décédé à Cambridge. Sa famille a quitté la Pologne pour Paris afin de fuir la menace hitlérienne. C'est à Paris qu'il fut initié aux mathématiques par deux oncles dont un était professeur au Collège de France. L'invasion allemande force la famille à se réfugier ensuite à Brive-la-Gaillarde. Après avoir fréquenté le lycée Edmond-Perrier de Tulle, il poursuit ses études au lycée du Parc, à Lyon. Après avoir quitté l'École polytechnique (promotion 1944), où il a suivi les cours d'un spécialiste du calcul des probabilités (Paul Lévy), il s'intéresse aux phénomènes d'information, les idées de Claude Shannon étant alors en plein essor. Mandelbrot fit ses études en France et aux États-Unis et obtint son doctorat de mathématiques à l'Université de Paris en 1952. Il enseigna l'économie à l'Université Harvard, l'ingénierie à Yale, la physiologie à la faculté de médecine et la mathématique à Paris et à Genève. À partir de 1958, il travailla pour IBM au centre de recherche Thomas B. Watson à New York sur la transmission optimale dans les milieux bruités, tout en poursuivant son travail sur des objets étranges jusque-là assez négligés par les mathématiciens : les objets à complexité récursivement définie, comme la courbe de Von Koch, auxquels il pressent une unité: la géométrie fractale. La géométrie fractale se distingue par son approche plus abstraite de la dimension qu'elle ne l'est dans la géométrie traditionnelle. Elle trouve de plus en plus d'applications dans différents domaines de la science et de la technologie.



\parpic[l][t]{%
  \begin{minipage}{40mm}
    \fbox{\includegraphics[width=110px,height=140px]{img/medaillons/markov.eps}}
  \end{minipage}
}
\textbf{Markov, Andrei Andreyevich} (1856-1922) était un mathématicien russe spécialiste de la théorie des nombres, de la théorie des probabilités et de l'analyse mathématique né à Riazan et décédé à Petrograd. Issu d'une la famille d'un petit fonctionnaire du gouvernement, il fait ses études à l'Université de Saint-Pétersbourg et reçoit une médaille d'or pour son mémoire \textit{De l'intégration des équations différentielles par la méthode des fractions continues} (1878). Professeur à l'Université de Saint-Pétersbourg en 1886, il devient membre de l'Académie des Sciences en 1896. Les recherches de Markov continuent l'oeuvre de ses devanciers de l'école mathématique pétersbourgeoise: P. L. Tchebychev, E. I. Zolotarev et A. N. Korkin. Sa thèse \textit{Des formes quadratiques bilinéaires de déterminant positif} (1880) inaugure ses travaux dans le domaine de la théorie des nombres. En analyse, ses recherches concernent les fractions continues, les limites d'intégrales, la convergence des séries et la théorie de l'approximation. On lui doit une solution simple de la détermination de la limite supérieure de la dérivée d'un polynôme (inégalité de Markov). Après 1910, se tournant vers la théorie des probabilités, il démontre de façon rigoureuse, sous des conditions assez générales, le théorème central limite relatif à une somme de variables aléatoires indépendantes identiquement distribuées. Cherchant à généraliser ce théorème aux variables aléatoires dépendantes, il est amené à considérer la notion importante d'événements en chaînes, appelés depuis chaînes de Markov, et il établit une série de lois, fondement de la théorie des processus de Markov. Il étend plusieurs résultats classiques concernant des événements indépendants à certains types de chaînes. Ses travaux sont à l'origine de la théorie moderne des processus stochastiques. Markov s'intéressait aussi aux applications de la théorie des probabilités, et il a justifié de façon probabiliste la méthode des moindres carrés.

\parpic[l][t]{%
  \begin{minipage}{40mm}
    \fbox{\includegraphics[width=110px,height=140px]{img/medaillons/markowitz.eps}}
  \end{minipage}
}
\textbf{Markowitz, Harry Maurice} (1927 -) Né à Chicago, professeur à la City University de New York est connu pour avoir développé la théorie dite du "choix des portefeuilles pour le placement des fortunes". Markowitz ne se doutait pas que son article de jeunesse publié en 1952 dans le \textit{Journal of Finance}, puis développé dans un livre paru en 1959, \textit{Portofolio Selection: Efficient diversification}, jetterait les bases de la théorie moderne du portefeuille et de son utilisation par un grand nombre de praticiens. Plus précisément, Markowitz a montré que l'investisseur cherche à optimiser ses choix en tenant compte non seulement de la rentabilité attendue de ses placements, mais aussi du risque de son portefeuille qu'il définit mathématiquement par la variance de sa rentabilité. Appliquant des théorèmes classiques du calcul statistique et des techniques probabilistes, il a ainsi démontré qu'un portefeuille composé de plusieurs titres est toujours moins risqué qu'un portefeuille composé d'un seul titre, quand bien même il s'agirait du moins risqué d'entre eux. La mise en oeuvre du modèle de Markowitz a très vite posé des problèmes d'ordre pratique. Alors que le volume des statistiques nécessaires au calcul augmentait rapidement avec le nombre de titres retenus (avec $100$ titres, le nombre de statistiques nécessaires était de $3'150$, mais il passait à $20'300$ pour $200$ titres et à $125'750$ pour $300$ titres !), la collecte des informations et leur traitement devenaient presque impossibles avec les ordinateurs disponibles dans les années 1960, entraînant de surcroît des coûts de traitement prohibitifs. C'est la raison pour laquelle William F. Sharpe cherchera une méthode de sélection des portefeuilles efficients plus simple. Markowitz et Sharpe seront alors reconnus comme les pères fondateurs de la gestion de portefeuilles et du corps doctrinal sur lequel elle se fonde. Le prix Nobel de sciences économiques leur sera décerné ainsi qu'à Merton Miller en 1990.

\parpic[l][t]{%
  \begin{minipage}{40mm}
    \fbox{\includegraphics[width=110px,height=140px]{img/medaillons/markx.eps}}
  \end{minipage}
}
\textbf{Marx, Karl} (1818-1883) Né à Trèves et décédé à Londres il entra à l'Université de Bonn puis à celle de Berlin, après avoir terminé le Lycée de Trèves. Il étudia à Berlin le droit, mais surtout l'histoire et la philosophie. Marx a ensuite contribué à parachever les trois principaux courants d'idées du 19ème siècle: la philosophie classique allemande, l'économie politique classique anglaise et le socialisme français. La théorie sociale de Marx a pour objectif de dévoiler la loi économique de la société capitaliste où ce qui domine est la production des marchandises en recherchant l'origine de la forme monétaire de la valeur. Ainsi pour Marx, l'argent (en tant que produit suprême du développement de l'échange et de la production marchande) estompe et dissimule le caractère et le lien social du travail individuel. À un certain degré du développement de la production des marchandises, l'argent se transforme aussi en capital. Ainsi, la séquence de la circulation des marchandises était: M (marchandise) - A (argent) - M (marchandise), c'est-à-dire vente d'une marchandise pour l'achat d'une autre. La séquence générale du capital est par contre A-M-A, c'est-à-dire l'achat pour la vente (avec un profit). C'est cet accroissement de la valeur primitive de l'argent, donc sa transformation en capital, que Marx appelle "plus-value" et qui ne peut provenir de la circulation des marchandises, car celle-ci ne connaît que l'échange d'équivalents; elle ne peut provenir non plus d'une majoration des prix, étant donné que les pertes et les profits réciproques des acheteurs et des vendeurs s'équilibrent à grande échelle. Pour obtenir de la plus-value il faut selon Marx une marchandise dont le processus de consommation fût en même temps un processus de création de valeur. Or, cette marchandise est la force de travail humaine. Le possesseur d'argent achète la force de travail à sa valeur, déterminée, comme celle de toute autre marchandise, par le temps de travail socialement nécessaire à sa production. Ayant acheté la force de travail, le possesseur d'argent est en droit de la consommer, c'est-à-dire de l'obliger à travailler toute la journée, disons, $8$ heures. Or, en $5$ heures (temps de travail nécessaire), l'ouvrier crée un produit qui couvre les frais de son entretien, et, pendant les $3$ autres heures (temps de travail supplémentaire), il crée un produit supplémentaire, non rétribué par le capitaliste, et qui est la plus-value. Aussi, pour exprimer le degré d'exploitation de la force de travail par le capital, faut-il comparer la plus-value non pas au coût total de production, mais uniquement au coût variable de la main d'oeuvre humaine.

\parpic[l][t]{%
  \begin{minipage}{40mm}
    \fbox{\includegraphics[width=110px,height=140px]{img/medaillons/maxwell.eps}}
  \end{minipage}
}
\textbf{Maxwell, James Clerk} (1831-1879) est né à Edimbourg et décédé à Glenlair (Écosse). Brillant élève au collège, James Clerk Maxwell poursuit des études de mathématiques à l'Université de Cambridge. Il obtient une chaire de philosophie naturelle à Aberdeen à l'âge de 25 ans. Puis, de 1860 à 1865, il occupe le poste de professeur au King's College de Londres. À la suite de ces 5 années d'enseignement, il décide de se retirer dans sa propriété de Glenair, en Écosse. Il y restera 5 autres années qu'il emploiera à étudier. En 1871, Maxwell est nommé directeur du laboratoire Cavendish que vient de fonder le duc du Devonshire. Il n'aura alors de cesse de le développer afin qu'il devienne le centre de formation scientifique le plus illustre. Dès le début de sa carrière, Maxwell s'intéresse à la dynamique des gaz. Après avoir prouvé mathématiquement que les anneaux de Saturne sont constitués de particules distinctes, il étudie la répartition des vitesses des molécules gazeuses (conforme à loi de Gauss). En 1860, il montre que l'énergie cinétique de ces molécules ne dépend que de leur nature. Mais ce sont ses recherches en électromagnétisme qui font de Maxwell un des savants les plus célèbres du 19ème siècle. En se basant sur les travaux de Faraday, il introduit dès 1862 la notion de champ. Puis, il montre qu'un champ magnétique peut être créé par la variation d'un champ électrique (Faraday avait alors découvert l'induction, phénomène par lequel la variation d'un champ électrique crée un champ magnétique). Son enseignement purement mathématique va alors lui permettre d'élaborer les célèbres équations différentielles décrivant la nature des champs électromagnétiques dans l'espace et le temps. Il les expose dans son \textit{Traité d'électricité et de magnétisme} publié en 1873. Maxwell, en élaborant les théories de l'électromagnétisme, a également défini la lumière en tant qu'onde électromagnétique, ouvrant ainsi la voie aux recherches d'autres physiciens comme Heinrich Rudolph Hertz.

\parpic[l][t]{%
  \begin{minipage}{40mm}
    \fbox{\includegraphics[width=110px,height=140px]{img/medaillons/mcfadden.eps}}
  \end{minipage}
}
\textbf{McFadden, Daniel} (1937 -) Né à Raleigh (Caroline du Nord) est un économétricien ayant reçu en 2000, avec James Heckman, le prix Nobel d'économie pour son apport aux théories et méthodes de l'analyse des choix discrets. Il obtient un Bachelor de Science en physique à l'âge de 19 ans à l'Université du Minnesota puis un doctorat de philosophie en sciences du comportement (économie) 5 ans plus tard en 1962. En 1964, il intègre l'Université de Berkeley et focalise ses recherches sur les comportements de choix, et sur les liens entre la théorie économique et les mesures économiques. En 1975, il est récompensé par la médaille John Bates Clark. En 1977, il se rend au Massachusetts Institute of Technology (M.I.T.), mais retourne à Berkeley en 1991, car le M.I.T. n'avait pas de département de statistiques. Après son retour, il fonde le laboratoire d'économétrie, qui est dévoué à l'informatique statistique appliquée à l'économie. McFadden a développé en microéconométrie des théories et des méthodes d'analyse des comportements par choix discrets (par exemple, les données sur les métiers et les lieux de résidence des individus) et est connu pour être à l'origine du coefficient pseudo-R de la régression logistique probit. À partir de sa théorie économique sur les choix discrets, McFadden a développé de nouvelles méthodes statistiques qui ont eu une influence décisive sur la recherche théorique, mais qui sont aussi largement utilisées par le marketing.

\parpic[l][t]{%
  \begin{minipage}{40mm}
    \fbox{\includegraphics[width=110px,height=140px]{img/medaillons/meitner.eps}}
  \end{minipage}
}
\textbf{Meitner, Lise} (1878-1960) était une physicienne née à Vienne et décédée à Cambridge. En 1899, Lise commença une préparation accélérée de 2 ans pour son entrée à l'Université, afin de se présenter à cet examen. Elle fut reçue et entra à l'Université de Vienne en 1901, à l'âge de 22 ans. Après la première année, au cours de laquelle Lise suivit de nombreux cours en physique, chimie, mathématiques et botanique, elle se concentra sur la physique. Dès la seconde année, elle choisit de suivre tous les cours donnés par Ludwig Boltzmann ; cela témoigne de la fascination que ce grand physicien théoricien exerçait sur ses étudiants, avec qui il développait des liens intellectuels mais aussi personnels. Elle obtint son doctorat en 1905. Lise demeura à Vienne durant l'année qui suivit son doctorat. En tant que femme, elle ne pouvait espérer une carrière académique, mais continua malgré tout la recherche. Elle rencontra Paul Ehrenfest, ancien étudiant de Boltzmann, qui attira son attention sur les articles publiés par Lord Rayleigh. L'un d'eux décrivait un effet d'optique que Rayleigh ne parvenait pas à expliquer. Lise trouva l'explication théorique et en dériva de nouvelles observations. Lise partit pour Berlin en 1907 afin de suivre les cours de Max Planck. Lise et Otto Hahn étudièrent la radioactivité et ils devinrent réputés pour leurs travaux, notamment pour la découverte du protactinium en 1918. Indépendamment de ses travaux avec Hahn, Lise mena des recherches pionnières en physique nucléaire. Elle se consacra d'abord à l'étude des spectres de rayonnements bêta et gamma. En 1923, elle découvrit ainsi la transition non-radiative connue comme l'effet Auger, appelé en l'honneur de Pierre Auger, un scientifique français qui le découvre indépendamment 2 ans plus tard. Elle découvrit également l'émission de paires électron-positron lors de la désintégration bêta plus. Elle effectua différentes mesures de la masse du neutron. En 1939 elle joue un rôle majeur dans la découverte de la fission nucléaire, dont elle fournit avec son neveu Otto Frisch la première explication théorique en 1939 en employant le modèle de la goutte liquide de Niels Bohr. Raison pour laquelle elle est considérée comme la "mère de la bombe nucléaire" par les médias de l'époque.

\parpic[l][t]{%
  \begin{minipage}{40mm}
    \fbox{\includegraphics[width=110px,height=140px]{img/medaillons/mendel.eps}}
  \end{minipage}
}
\textbf{Mendel, Gregor Johann} (1822-1884) Né à Heinzendorf et décédé à Brno, a été moine dans le monastère de Brno (en Moravie). Mendel est communément reconnu comme un botaniste père fondateur de la génétique. Il est à l'origine de ce qui est aujourd'hui appelé les "lois de Mendel", qui définissent la manière dont les gènes se transmettent de génération en génération. Mendel naît dans une famille de paysans. Doué pour les études, mais de tendance dépressive qui lui vaudra de multiples indispositions dans la suite de sa carrière, le jeune garçon est très vite remarqué par le curé du village qui décide de l'envoyer poursuivre ses études loin de chez lui. Mendel part en 1851 pour suivre les cours, en tant qu'auditeur libre, de l'Institut de Physique de Christian Doppler. Il y étudie, en plus des matières obligatoires: la botanique, la physiologie végétale, l'entomologie et la paléontologie. Durant 2 années, il acquiert les bases méthodologiques qui lui permettront de réaliser plus tard ses expériences. Au cours de son séjour à Vienne, Mendel est amené à s'intéresser aux théories de Franz Unger, professeur de physiologie végétale. Celui-ci préconise l'étude expérimentale pour comprendre l'apparition des caractères nouveaux chez les végétaux au cours de générations successives. Il espère ainsi résoudre le problème que pose l'hybridation chez les végétaux. De retour au monastère, Mendel installe un jardin expérimental dans la cour et dans la serre, en accord avec son abbé, et met sur pied un plan d'expériences visant à comprendre les lois de l'origine et de la formation des hybrides. Il choisit pour cela le pois qui a l'avantage d'être facilement cultivable avec de nombreuses variétés décrites. En 1865, il expose à la Société des Sciences Naturelles de Brünn et publie en 1866 les résultats de ses études. Après 10 années de travaux minutieux, Mendel a ainsi posé les bases théoriques de la génétique et de l'hérédité moderne. Son travail ne va pas susciter d'enthousiasme auprès de ses contemporains qui ont du mal à comprendre la formalisation mathématique de ses expériences. Très peu de scientifiques de son temps vont citer son travail et Mendel ne reçoit guère de réponses auprès des différents correspondants à qui il envoie un tiré-à-part. Parmi ces derniers, seul Karl Wilhelm von Nägeli, professeur de botanique à Münich, lui écrit, doutant d'ailleurs de certaines de ses conclusions. En 1868, Mendel est élu supérieur de son couvent à la mort de l'abbé.

\parpic[l][t]{%
  \begin{minipage}{40mm}
    \fbox{\includegraphics[width=110px,height=140px]{img/medaillons/mendeleiev.eps}}
  \end{minipage}
}
\textbf{Mendeleev, Dmitri Ivanovich} (1834-1907) Né à Tobolsk et décédé à Saint-Pétersbourg, c'était chimiste russe surtout connu pour sa classification périodique des éléments publiée en 1869. Il montra en effet que les propriétés chimiques des éléments dépendaient directement de leur poids atomique et qu'elles étaient des fonctions périodiques de ce poids. Il entre à l'âge de 14 ans au lycée de Tobolsk, après la mort de son père. En 1849, la famille devenue pauvre s'installe à Saint-Pétersbourg et Mendeleïev entre à l'Université en 1850. Après avoir reçu son diplôme, il contracte la tuberculose ce qui l'oblige à se déplacer dans la péninsule criméenne près de la Mer Noire en 1855, où il devient responsable des sciences du lycée local. Il revient complètement guéri à Saint-Pétersbourg en 1856 où il étudie la chimie et fut diplômé en 1856. À 25 ans, il vient travailler à Heidelberg avec des savants comme Robert Bunsen et Gustav Kirchhoff. À Heidelberg, il rencontra le chimiste italien Stanislao Cannizzaro, dont les idées sur le poids atomique influencèrent sa réflexion. Mendeleïev retourna à Saint-Pétersbourg et enseigna la chimie à l'Institut Technique en 1863. En 1864, il soutient sa thèse de doctorat intitulée \textit{Considérations sur la combinaison de l'alcool et de l'eau}. En 1867, il est nommé professeur de chimie minérale (toujours à l'Université de Saint-Pétersbourg) et y fut enfin nommé professeur de chimie générale en 1866.

\parpic[l][t]{%
  \begin{minipage}{40mm}
    \fbox{\includegraphics[width=110px,height=140px]{img/medaillons/merton.eps}}
  \end{minipage}
}
\textbf{Merton, Robert Cox} (1944-) a reçu le prix Nobel d'Économie en 1997, en même temps que son compatriote Myron Scholes, pour avoir élaboré la méthode d'évaluation des instruments financiers dérivés. Cette méthode d'évaluation a certainement accéléré la croissance rapide des marchés des instruments financiers dérivés depuis les années 1980 et permis l'amélioration de la gestion des risques attachés à ces nouveaux produits financiers. Merton a sans conteste contribué à ouvrir une voie nouvelle dans le champ des sciences économiques et fortement influencé les deux autres lauréats. Né en 1944 à New York, il quitte le California Institute of Technology avec un Master en mathématiques appliquées. Il obtient par la suite un doctorat en sciences économiques au Massachusetts Institute of Technology (M.I.T.) de Cambridge, sous la direction de Paul Samuelson (Prix Nobel d'Économie 1970) et se spécialise dans les problèmes d'application des méthodes probabilistes à l'évolution aléatoire des cours des actifs financiers. En 1988, il occupe la chaire George Fischer Backer de professeur en Business Administration à la Harvard Business School de Cambridge. Le travail novateur de Merton date du début des années 1970, période pendant laquelle il élabore une méthode originale de calcul de la valeur des instruments dérivés. L'échec de sa méthode appliquée à la gestion d'un fonds de placement à risques américain (Long-Term Capital Management) en 1998, a quelque peu terni sa réputation de spécialiste de la finance internationale. Mais Merton avait lui-même déclaré à une chaîne de télévision américaine, au lendemain de l'attribution de sa récompense que c'est une mauvaise interprétation de penser que l'on peut éliminer les risques simplement parce qu'on les comprend et qu'on les mesure.

\parpic[l][t]{%
  \begin{minipage}{40mm}
    \fbox{\includegraphics[width=110px,height=140px]{img/medaillons/minkowski.eps}}
  \end{minipage}
}
\textbf{Minkowski, Hermann} (1864-1909) Né à Alexotas et décédé à Göttingen, c'était un physicien-mathématicien qui a étudié aux Universités de Berlin et de Königsberg. Il fit des études secondaires au lycée de Königsberg où il se fait remarquer par ses résultats en mathématiques et reçoit son doctorat en 1885 dans la même ville. Il a ensuite enseigné dans plusieurs universités, à Bonn, Königsberg et à Zürich. À Zürich, Einstein était un des étudiants dans plusieurs des cours qu'il a donnés. Minkowski a accepté une chaire en 1902 à l'Université de Göttingen, où il est resté pour le reste de sa vie. À Göttingen, il apprit la physique-mathématique de Hilbert, il a participé à une conférence sur la théorie de l'électron en 1905 et appris les derniers résultats dans la théorie dans l'électrodynamique. En 1907, Minkowski s'est rendu compte que le travail de Lorentz et d'Einstein pourrait mieux être compris dans un espace non-euclidien. Il a considéré l'espace et le temps, qui a été autrefois pensé pour être indépendant, d'être couplé ensemble dans un continuum d'espace-temps quadridimensionnel. Minkowski a établi un traitement quadridimensionnel de l'électrodynamique. Ce continuum d'espace-temps a fourni un cadre pour tout le travail mathématique postérieur dans la relativité. Ces idées ont été employées par Albert Einstein en développant la théorie Générale de Relativité. Minkowski était principalement intéressé par la mathématique pure et a passé beaucoup de son temps en étudiant les formes quadratiques et les fractions continues. Son travail le plus original était cependant sa \textit{Géométrie des nombres}. Cette étude a mené à des travaux sur les corps convexes et aux questions au sujet des problèmes d'emballage (les manières dans lesquelles des figures d'une forme donnée peuvent être placées dans une autre figure donnée).

\parpic[l][t]{%
  \begin{minipage}{40mm}
    \fbox{\includegraphics[width=110px,height=140px]{img/medaillons/mobius.eps}}
  \end{minipage}
}
\textbf{Möbius, August Ferdinand} (1790-1868) était un mathématicien et astronome allemand né à Schulpforta et mort à Leipzig (Allmagne). Möbius fit ses études à Leipzig, à Göttingen (sous la direction de Gauss) et à Halle. En 1815, il devint professeur d'astronomie à Leipzig, puis directeur de l'observatoire de cette ville, après en avoir dirigé la construction. On lui doit plusieurs ouvrages d'astronomie théorique, notamment \textit{De computandis occultationibus fixarum per planetas} (1815). Ses travaux mathématiques concernent principalement la géométrie et furent, pour la plupart, publiés dans le \textit{Journal des mathématiques pures et appliquées} de Crelle, de 1828 à 1858, comme compléments à son ouvrage fondamental \textit{Der barycentrische Calcul} (1827). En introduisant un nouveau système de coordonnées, Möbius y étudie les transformations géométriques, principalement la transformation projective. Son ouvrage eut une très grande importance dans le développement de la géométrie projective. Étudiant la statique sous l'angle de la géométrie, Möbius développa également la théorie des complexes linéaires de droites (Lehrbuch der Statik, 1837). On peut considérer Möbius comme un des pionniers de la topologie, avec la découverte, publiée dans un mémoire à l'Académie des sciences française, du fameux "ruban de Möbius", surface n'ayant qu'un seul côté.

\parpic[l][t]{%
  \begin{minipage}{40mm}
    \fbox{\includegraphics[width=110px,height=140px]{img/medaillons/monge.eps}}
  \end{minipage}
}
\textbf{Monge, Gaspard} (1746-1818) Né à Beaune et décédé à Paris Monge est fils d'un marchand forain. Il suit d'abord le collège de Beaune, puis va ensuite collège de Lyon, où il enseigne dès l'âge de 16 ans les sciences physiques. Un officier du génie, qui avait vu un plan de la ville de Beaune fait par Monge à l'aide de nouvelles méthodes d'observation et de construction graphique, le recommande au commandant de l'École Militaire de Mézières. Mais il ne peut y être admis à cause de son origine roturière et n'est accepté que dans une annexe technique de l'école. Ses talents scientifiques sont reconnus lorsqu'un jour il dresse le plan de fortifications à l'aide d'une méthode bien plus rapide que les méthodes connues jusque-là. Il est alors admis à l'École Militaire comme professeur de mathématiques et continue ses recherches, arrivant à la méthode générale de représentation géométrique connue depuis lors sous le nom de géométrie descriptive. Mais ses découvertes, considérées comme secret militaire de grande valeur, ne peuvent être publiées. En 1780, il vient à Paris enseigner l'hydrodynamique. Il entre aussitôt à l'Académie des Sciences, où il fait une communication sur les lignes de courbure tracées sur une surface (problème déjà étudié par Euler en 1760). En 1786, il publie son célèbre \textit{Traité élémentaire de la statique} et fonde peu après l'École Polytechnique, où il aura l'occasion de donner des leçons de géométrie descriptive et de publier ses travaux jusque-là inconnus. Chargé de mission en Italie, Monge rencontre Bonaparte et se charge du recrutement des savants pour l'expédition d'Égypte. Revenu en France, il reprend son enseignement à l'École Polytechnique, devient sénateur et est anobli . Mais la Restauration le privera de tous ses titres, le rayera de la liste des membres de l'Institut et lui enlèvera son poste d'enseignant. En 1989, ses cendres ont été transférées au Panthéon. Toutes ses recherches mêlent étroitement la géométrie pure, l'analyse infinitésimale et la géométrie analytique, lui permettant, par exemple, de lier chaque famille de surfaces à une équation aux dérivées partielles et, par là, de trouver les solutions d'équations différentielles à l'aide de sa théorie des surfaces. L'influence de Monge s'exerça par son enseignement oral, et la plupart des mathématiciens français du 19ème siècle ont été ses élèves.

\phantomsection
\addcontentsline{toc}{section}{N}
\label{sec:N}

\parpic[l][t]{%
  \begin{minipage}{40mm}
    \fbox{\includegraphics[width=110px,height=140px]{img/medaillons/napier.eps}}
  \end{minipage}
}
\textbf{Napier, John} (1550-1617) Né et décédé à Merchiston, il était  théologien, physicien, astronome et mathématicien. Comme c'était la pratique courante pour les membres de la noblesse à l'époque, John Napier n'est pas entré à l'école avant l'âge de 13 ans. Il n'est pas resté à l'école très longtemps, cependant. On croit qu'il a quitté l'école en Écosse et a peut-être voyagé en Europe continentale afin de mieux poursuivre ses études. En 1571, Napier, âgé de 21 ans, est retourné en Écosse, et a acheté un château à Gartness en 1574. A la mort de son père en 1608, Napier et sa famille ont emménagé dans Merchiston Castle à Edimbourg, où il résida le reste de sa vie. La mathématique n'était pas son activité principale mais il ne manquait pas d'idées pour simplifier les calculs. Il établit quelques formules de trigonométrie sphérique, popularisa l'usage du point pour la notation anglo-saxonne des nombres décimaux mais surtout inventa les logarithmes. Son objectif était de simplifier les calculs trigonométriques nécessaires en astronomie. Il s'attacha à définir le logarithme d'un sinus en s'appuyant sur des considérations mécaniques de points mobiles et sur le lien entre les progressions arithmétique et géométrique.

\parpic[l][t]{%
  \begin{minipage}{40mm}
    \fbox{\includegraphics[width=110px,height=140px]{img/medaillons/navier.eps}}
  \end{minipage}
}
\textbf{Navier, Henri} (1785-1836) Né à Dijon et décédé à Paris, c'était un ingénieur, mathématicien et économiste surtout connu pour ses travaux sur l'hydrodynamique. Henri devient orphelin à 9 ans, après la mort de son père, avocat réputé et ancien député durant la Révolution. Son oncle, ingénieur du Corps des Ponts et Chaussées s'occupe de son éducation à Paris, le considère comme son fils avant de l'adopter avec sa femme, également proche parente du jeune Henri. L'ingénieur cantonnier le pousse à se présenter à l'École Polytechnique. Bien qu'étant parmi les derniers reçus en 1802, il y réussit sa scolarité et son classement lui permet d'intégrer le corps des Ponts et Chaussées. Il est nommé ingénieur ordinaire des Ponts et Chaussées en 1808. Plus tard, il deviendra inspecteur divisionnaire de ce corps et, semble-t-il quelque temps, inspecteur général à l'instar de son oncle. De 1819 à 1835, il assure le cours de mécanique appliquée de l'École Nationale des Ponts et Chaussées (il y est titularisé en 1830 à la suite de la retraite d'Eisenmann). Au début des années 1820, il explore avec Augustin-Louis Cauchy les facettes de la théorie mathématique de l'élasticité, ce qui lui permet de proposer des équations sur le mouvement des fluides newtoniens.

\parpic[l][t]{%
  \begin{minipage}{40mm}
    \fbox{\includegraphics[width=110px,height=140px]{img/medaillons/nash.eps}}
  \end{minipage}
}
\textbf{Nash, John} (1928-2015) Né en Virginie Occidentale, fils de John Nash Sr., ingénieur, et Virginia Martin, enseignante. Jeune, il passait beaucoup de temps à lire et à faire des expériences dans sa chambre qu'il avait convertie en laboratoire. De 1945 à 1948, Nash a étudié au Carnegie Institute of Technology à Pittsburgh, dans l'intention de devenir ingénieur comme son père. À la place, il y a développé une passion durable pour les mathématiques, et en particulier pour la théorie des nombres, les équations diophantiennes, la mécanique quantique et la théorie de la relativité. Il fut admis en troisième cycle, à 20 ans, dans toutes les universités qu'il avait sollicitées: Harvard, Princeton... Il choisit d'aller à Princeton. Ayant un intérêt pour l'économie, Nash se mit à étudier la théorie des jeux, domaine qu'avait défriché John von Neumann, un des grands noms de Princeton, un peu plus d'une décennie auparavant. C'est sur ce sujet qu'il décida de faire sa thèse et qu'il obtint le prix Nobel d'Économie en 1994. Durant l'été 1950, Nash fut employé comme consultant à la RAND, institut top-secret qui employait de la matière grise pour mettre au point diverses stratégies de statu quo soit de victoire, en cas de conflit faisant appel à l'arme nucléaire. À la suite, Nash se mit à étudier les variétés lisses compactes, ce qui fit l'objet d'un papier. Il devint ensuite assistant au M.I.T à la rentrée 1951-52, âgé de 23 ans seulement. Il avait vraiment un tempérament de problem-solver et releva ainsi le pari de résoudre une question de Waren Ambros: est-il possible de plonger une variété riemannienne quelconque dans un espace euclidien? Nash trouve une méthode fondamentale originale pour y arriver. Nash devint malade après quelques problèmes privés et professionnels, mais il attribua sa maladie à sa tentative de résoudre les contradictions de la physique quantique. D'autant plus que peu de temps auparavant, il avait réalisé des travaux sur les équations différentielles partielles elliptiques non linéaires qui lui valurent beaucoup d'admiration autour de lui, mais dont il dut finalement partager la paternité avec un jeune italien qui avait énoncé, indépendamment et quelques semaines avant lui, des résultats similaires: ceci leur valu de ne pas obtenir la médaille Fields en 1958...

\parpic[l][t]{%
  \begin{minipage}{40mm}
    \fbox{\includegraphics[width=110px,height=140px]{img/medaillons/newton.eps}}
  \end{minipage}
}
\textbf{Newton, Isaac} (1642-1727) était un mathématicien et physicien anglais, considéré comme l'un des plus grands scientifiques de l'histoire. Newton nait dans le Lincolnshire (Angleterre), de parents paysans et décédera à Londres. À 5 ans, il fréquente l'école primaire de Skillington, puis à 12 ans celle de Grantham. Il y reste quatre années jusqu'à ce que sa mère le rappelle à Woolsthorpe pour qu'il devienne fermier et qu'il apprenne à administrer son domaine. Pourtant, sa mère, s'apercevant que son fils était plus doué pour la mécanique que pour le bétail, l'autorisa à retourner à l'école pour peut-être pouvoir entrer un jour à l'université. À 17 ans, Newton tombe amoureux d'une camarade de classe, mademoiselle Clara Storey. On l'autorise à la fréquenter et même à se fiancer avec elle, mais il doit terminer ses études avant de se marier. Finalement, le mariage ne se fit pas et Newton restera alors célibataire toute sa vie. À 18 ans, il entre alors au Trinity College de Cambridge (il y restera 7 ans), où il se fait remarquer par son maître, Isaac Barrow. Il a également comme professeur Henry More qui l'influencera dans sa conception de l'espace absolu. À Cambridge, il étudie l'arithmétique, la géométrie dans les \textit{Éléments} d'Euclide et la trigonométrie, mais s'intéresse particulièrement à l'astronomie, à l'alchimie et à la théologie. Il devient à 25 ans bachelier des arts, mais est contraint de suspendre ses études pendant deux années suite à l'apparition de la peste qui s'est abattue sur la ville en 1665 ; il retourne alors dans sa région natale. C'est à cette période que Newton progresse fortement en mathématiques, physique et surtout en optique. Il laissa d'importantes contributions à de nombreuses branches de la science. Ses découvertes et théories furent à la base d'une grande partie des progrès scientifiques réalisés après lui. Newton fut l'un des inventeurs de la branche des mathématiques appelée "calcul infinitésimal" (l'autre inventeur fut le mathématicien allemand Gottfried Wilhelm Leibniz). Il éclaircit également les mystères de la lumière et de l'optique, formula trois lois sur le mouvement et en déduisit la loi de la gravitation universelle en sa basant sur les lois de Kepler. Il parvint au raisonnement selon lequel la lumière est un mélange de différents rayons de couleurs différentes, et qu'en raison des phénomènes de réflexion et de réfraction, ses couleurs apparaissent en composants séparés. Newton mit en évidence sa théorie des couleurs en faisant passer de la lumière au travers d'un prisme, qui scinde le faisceau lumineux en couleurs séparées. En 1696, il quitte Cambridge pour devenir d'abord gardien de la Royal Mint puis maître de la monnaie dès l'année suivante. En 1699, il est nommé membre du conseil de la Royal Society et y est élu président en 1703. Il garde cette place jusqu'à son décès.

\parpic[l][t]{%
  \begin{minipage}{40mm}
    \fbox{\includegraphics[width=110px,height=140px]{img/medaillons/neumann.eps}}
  \end{minipage}
}
\textbf{Neumann Von, John} (1903-1957) était un mathématicien né à Budapest (Hongrie) et décédé à Washington. Neumann est un enfant prodige: à 6 ans, il converse avec son père en grec ancien et peut mentalement faire la division d'un nombre à huit chiffres. Une anecdote rapporte qu'à 8 ans, il a déjà lu les 44 volumes de l'\textit{Histoire universelle} de la bibliothèque familiale et qu'il les a entièrement mémorisés : doté d'une mémoire absolue, il sera capable de citer de mémoire des pages entières de livres lus des années auparavant. Il entre au lycée à Budapest en 1911. C'est âgé d'à peine 23 ans qu'il reçoit son doctorat en mathématiques (avec des mineures en physique expérimentale et en chimie) de l'Université de Budapest. En parallèle, il obtient un diplôme en génie chimique de l'École Polytechnique Fédérale de Zürich (à la demande de son père, désireux que son fils s'investisse dans un secteur plus rémunérateur que les mathématiques). Il est intéressant de noter que von Neumann n'a mis les pieds dans ces deux universités que pour les examens. Entre 1926 et 1930, il est privatdozent à Berlin et à Hambourg. Il travaille également à Göttingen avec Robert Oppenheimer sous la direction de David Hilbert. Durant cette période, l'une des plus fécondes de sa vie, il côtoie également Werner Heisenberg et Kurt Gödel. En 1930, von Neumann est professeur-invité à l'université de Princeton. Puis, de 1933 à sa mort en 1957, il est professeur de mathématiques à la faculté de l'Institute for Advanced Study qui vient d'être créée (Neumann émigra aux États-Unis en 1933). Il y rejoint donc Albert Einstein et Kurt Gödel. Il rédigea un important ouvrage sur la mathématique appliquée et effectua un travail majeur dans l'axiomatisation de la physique quantique (c'est lui qui réalisa système quantique peut être considéré comme un point dans un espace de Hilbert et qui introduisit les opérateurs linéaires). Il participa durant la deuxième guerre mondiale au développement théorique de la bombe atomique et à l'étude des ondes de choc. Ces travaux mathématiques sur les calculs ultra-rapides pour les simulations de la bombe H, contribua de façon non négligeable au développement de l'informatique (il est aussi à l'origine des méthodes de Monte-Carlo). Il contribua également à la théorie des jeux où certains de ces résultats eurent une grande influence sur l'économie.

\parpic[l][t]{%
  \begin{minipage}{40mm}
    \fbox{\includegraphics[width=110px,height=140px]{img/medaillons/abel.eps}}
  \end{minipage}
}
\textbf{Niels, Abel} (1802-1829) était un mathématicien norvégien né à Frindoë et décédé à Froland. Son père était un éminent homme politique norvégien, mais à la fin de sa vie, il tomba en disgrâce, et quand il mourut en 1820, c'est Abel qui dut supporter toute la charge de la famille. Son père, éduqua lui-même Abel jusqu'en 1815, puis l'envoya au collège paroissial d'Oslo. Dans ce lycée, le latin, le grec et la religion étaient enseignés à l'ancienne, avec punitions et châtiments corporels. La situation évolua en 1817 à la suite du renvoi d'un professeur consécutif au décès d'un élève: le lycée recruta un jeune enseignant ouvert aux idées nouvelles et instruit de mathématiques qui en découvrant l'intérêt de Niels pour les mathématiques, lui trouva une bourse pour l'université. Grâce à l'aide financière de ses professeurs, il parvient donc toutefois à poursuivre ses études et à faire ses premières découvertes. Mais ses mémoires sont perdus par Cauchy et mésestimés par Gauss. Après son doctorat, Abel ne parvint pas à trouver un poste et ses conditions de vie devinrent de plus en plus précaires et sa santé se fragilisa: il fut ainsi atteint de la tuberculose. Malgré des déplacements à Paris et à Berlin, ses travaux ne sont toujours pas perçus à leur juste valeur. Dans ses dernières semaines, il n'a plus assez de force pour quitter son lit. Il décède à même pas 27 ans, alors qu'un ami venait juste de lui trouver un poste à Berlin.  C'est Jacobi qui comprendra tout le génie de ce jeune mathématicien. Abel avait notamment démontré, à l'âge de 19 ans dans un article en français intitulé \textit{Mémoire sur les équations algébriques, où l'on démontre l'impossibilité de la résolution de l'équation générale du cinquième degré}, l'impossibilité de résoudre par radicaux les équations algébriques de degré 5, ce que son contemporain Galois généralisera à tout degré. À titre posthume, Abel recevra en 1830 le grand prix de Mathématiques de l'Institut de France.

\parpic[l][t]{%
  \begin{minipage}{40mm}
    \fbox{\includegraphics[width=110px,height=140px]{img/medaillons/noether.eps}}
  \end{minipage}
}
\textbf{Nöther, Emmy} (1882 -1935) Née à Erlangen et décédée à Princeton, Emmy envisage d'abord d'enseigner le français et l'anglais après avoir passé les examens requis, mais étudie finalement les mathématiques à l'Université d'Erlangen où son père donne des conférences. Durant le semestre d'hiver 1903-1904, elle étudie à l'Université de Göttingen et assiste aux cours de l'astronome Karl Schwarzschild et des mathématiciens Hermann Minkowski, Felix Klein et David Hilbert. Après avoir achevé sa thèse en 1907 elle travaille bénévolement à l'Institut de Mathématiques d'Erlangen pendant 7 ans. En 1915, elle est invitée par David Hilbert et Felix Klein à rejoindre le très renommé département de mathématiques de l'université de Göttingen jusqu'en 1933. En 1935, elle est opérée en raison d'un kyste ovarien et, malgré des signes de rétablissement, meurt 4 jours plus tard à l'âge de 53 ans. Elle reste dans l'histoire des mathématiques comme la fondatrice principale de l'Algèbre abstraite, ou algèbre moderne, qui est une des branches essentielles des mathématiques contemporaines. Cette algèbre abstraite prend de la hauteur par rapport aux calculs menés dans divers ensembles, munis de diverses opérations, et montre ce que ces calculs ont en commun. En physique, le théorème de Noether explique le lien fondamental entre la symétrie et les lois de conservation. Ses idées dans  ont contribué aussi au progrès de la physique, en particulier dans la théorie de la Relativité. Malgré toutes ses qualités, elle eut des difficultés à mener une carrière normale de professeur d'université, car elle était une femme, dans un milieu exclusivement masculin. Elle bénéficia cependant de l'estime et de l'appui de David Hilbert, d'Albert Einstein et de Felix Klein.

\phantomsection
\addcontentsline{toc}{section}{O}
\label{sec:O}

\parpic[l][t]{%
  \begin{minipage}{40mm}
    \fbox{\includegraphics[width=110px,height=140px]{img/medaillons/ohm.eps}}
  \end{minipage}
}
\textbf{Ohm, Georg Simon} (1789-1854) était un physicien né à Erlangen et décédé à Münich. Bien que ses parents n'aient pas faits d'études supérieures, le père de Ohm était un homme respecté et un autodidacte qui a lui-même donné à son fils une excellente éducation. Depuis sa plus jeune enfance Georg reçoit de son père des enseignements de très bon niveau en physique, mathématiques, chimie et philosophie. Georg fréquente le lycée d'Erlangen de 11 à 15 ans et il y reçoit une éducation scientifique très restreinte, contrastant avec les enseignements de son père. En 1805, à l'âge de 15 ans, Ohm entre à l'Université d'Erlangen. Comme Ohm est dissipé, son père en colère devant le gâchis de ses possibilités, l'envoie en Suisse où, en 1806, il prend un poste de professeur de mathématiques dans une école de Gottstadt bei Nydau. Ohm quitte son poste d'enseignant à Gottstadt bei Nydau en 1809 pour devenir précepteur à Neuchâtel pendant 2 ans. Puis, en 1811 il retourne à l'Université d'Erlangen. Ses études lui furent utiles pour obtenir son doctorat de l'Université d'Erlangen la même année et rejoindre immédiatement l'équipe enseignante comme maître de conférence en mathématiques. Le roi Frédéric-Guillaume III de Prusse lui offre un poste au lycée jésuite de Cologne en 1817. Grâce à la réputation de cette école dans l'enseignement des sciences, Ohm se retrouve à enseigner aussi bien les mathématiques que la physique. Le laboratoire de physique étant bien équipé, il se consacre à des expérimentations. Ce qui est actuellement connu sous le nom de "loi d'Ohm" est apparu en 1827 dans le livre \textit{Die galvanische Kette, mathematisch bearbeitet} dans lequel il fournit une théorie complète de l'électricité. Il entre à l'École Polytechnique de Nuremberg en 1833 et en 1852 devient professeur de physique expérimentale à l'Université de Münich, où il meurt un peu plus tard.

\parpic[l][t]{%
  \begin{minipage}{40mm}
    \fbox{\includegraphics[width=110px,height=140px]{img/medaillons/oppenheimer.eps}}
  \end{minipage}
}
\textbf{Oppenheimer, J. Robert} (1904-1967) était un physicien né à New-York et décédé à Princeton. Il a été le directeur scientifique du projet Manhattan et y dirigea donc la mise au point des premières bombes atomiques. Entré à Harvard avec une année de retard à cause d'une attaque de colite ulcéreuse, il profite de cette période pour se rendre, avec son ancien professeur d'anglais, au Nouveau-Mexique. Il y devint amateur de promenades à cheval ainsi que des montagnes et plateaux de cette région. À son retour, il obtient son diplôme de chimie en 3 ans. Percy Bridgman lui fait découvrir la physique expérimentale. C'est durant ses études au laboratoire Cavendish d'Ernest Rutherford à Cambridge, qu'il réalise qu'il maîtrise mieux la théorie que la pratique en raison de sa maladresse. En 1926, il poursuit ses études sous la direction de Max Born à l'Université de Göttingen, et obtient son doctorat à l'âge de 22 ans. À Göttingen, il publie des articles sur la théorie quantique. En 1927, il retourne à Harvard puis l'année suivante à l'Institut de Technologie de Californie. Il est aussi connu pour sa contribution à la théorie quantique et à la théorie de la relativité, et pour ses études sur les rayons cosmiques, les positrons et les étoiles à neutrons. Il fait des recherches importantes en astrophysique, en physique nucléaire, et en spectroscopie. Il découvre alors l'approximation de Born-Oppenheimer.

\parpic[l][t]{%
  \begin{minipage}{40mm}
    \fbox{\includegraphics[width=110px,height=140px]{img/medaillons/ostrogradsky.eps}}
  \end{minipage}
}
\textbf{Ostrogradsky, Mikhail Vasilyevich} (1801-1862) était un physicien et mathématicien ukrainien. Il commença ses études de mathématiques à l'Université de Kharkov, et les continua ensuite à Paris où il fut en contact étroit avec les célèbres mathématiciens français Cauchy, Binet, Fourier et Poisson. De retour dans sa patrie, il enseigna à l'École des Cadets de la Marine, à l'Académie du Génie Nicolas et à l'École d'Artillerie de Saint-Pétersbourg. Il est célèbre en particulier pour avoir établi le théorème de flux-divergence, qui permet d'exprimer l'intégrale sur un volume (ou intégrale triple) de la divergence d'un champ vectoriel comme l'intégrale de surface (intégrale double étendue à la superficie qui entoure ce volume) du flux défini par ce champ. Il fut élu à l'Académie Américaine des Arts et des Aciences en 1834, à l'Académie des Sciences de Turin en 1841, et à l'Académie des Sciences de Rome en 1853. Enfin il fut élu membre correspondant de l'Académie des Sciences de Paris en 1856. Les travaux scientifiques d'Ostrogradski sont dans le droit fil des principes professés à cette époque à l'École Polytechnique dans les domaines de l'analyse, et des mathématiques appliquées. En physique mathématique, il imagina une synthèse grandiose qui embrasserait l'hydromécanique, la théorie de l'élasticité, la théorie de la chaleur, et la théorie de l'électricité dans le cadre d'une seule méthode homogène.

\phantomsection
\addcontentsline{toc}{section}{P}
\label{sec:P}

\parpic[l][t]{%
  \begin{minipage}{40mm}
    \fbox{\includegraphics[width=110px,height=140px]{img/medaillons/pareto.eps}}
  \end{minipage}
}
\textbf{Pareto, Vilfredo} (1848-1923) était un économiste et sociologue italien, dont la contribution la plus célèbre à la théorie économique est la définition du concept d'optimum économique. Né à Paris d'un père italien en exil et d'une mère française, il retourna en Italie à l'âge de 10 ans. Il fit ses études à l'Université de Turin et devint ingénieur. En 1893, il fut nommé à la chaire d'Économie Politique de l'Université de Lausanne (il décédera à Céligny en Suisse), où il succédait à Léon Walras. Parmi ses travaux figure l'analyse des anticipations des agents économiques. Celles-ci, n'étant pas indépendantes les unes des autres, peuvent susciter des mouvements d'opinion pessimistes qui génèrent des crises. Pareto est également le père de la notion d'optimum. L'économie est à un optimum lorsqu'on ne peut améliorer la situation d'un agent sans détériorer celle d'au moins un autre agent. Ce concept est très utilisé en économie, car il permet de prendre en compte la non-additivité des utilités des différents agents. La concurrence permet d'atteindre l'optimum au sens de Pareto. Pareto a également intégré les courbes d'indifférence (formalisées par Francis Edgeworth) à la logique walrassienne d'équilibre général. Le travail sociologique de Pareto fut plus discuté. Dans le Traité de sociologie générale, paru en 1916, il présenta sa théorie des élites, selon laquelle le pouvoir d'État est dans toutes les sociétés l'objet d'un combat entre les seules élites. Cette thèse discréditait les démocraties, et contribua implicitement au développement du fascisme alors montant en Italie.

\parpic[l][t]{%
  \begin{minipage}{40mm}
    \fbox{\includegraphics[width=110px,height=140px]{img/medaillons/pascal.eps}}
  \end{minipage}
}
\textbf{Pascal, Blaise} (1623-1662) était un mathématicien, physicien, théologien, mystique, philosophe, moraliste et polémiste né à Clermont et décédé à Paris. Enfant précoce (à 11 ans, il compose un court\textit{ Traité des sons des corps vibrants} et aurait démontré la 32e proposition du Ier livre d’Euclide, à 16 ans il écrit un traité sur les \textit{Coniques}), il est éduqué par son père qui était mathématicien. Les tout premiers travaux de Pascal concernent les sciences naturelles et appliquées. Il contribua de manière importante à l’étude des fluides. Il a clarifié les concepts de pression et de vide, en étendant le travail de Torricelli. L'étendue des domaines d'intérêt et du génie de Pascal est impressionnante: inventeur de la machine à calculer, concepteur des premiers transports en commun en France, artisan de l'assèchement des marais poitevins il fut également l'un des plus brillants prosateurs de la langue française et l'une des plus grandes figures du 17ème siècle français.

\parpic[l][t]{%
  \begin{minipage}{40mm}
    \fbox{\includegraphics[width=110px,height=140px]{img/medaillons/pauli.eps}}
  \end{minipage}
}
\textbf{Pauli, Wolfgang} (1900-1958) était un physicien autrichien, né à Vienne et décédé à Zürich, connu pour sa définition du principe d'exclusion en mécanique quantique, ce qui lui valut le prix Nobel de Physique de 1945. Pauli est né d'un père professeur des universités et d'une mère journaliste et juriste. Au lycée à Vienne, Pauli était considéré comme un enfant prodige en mathématiques. À partir de 1919, il commence ses études de physique à l'Université de Mïnich avec pour professeur Arnold Sommerfeld. Depuis 1898, Sommerfeld était en charge d'écrire le cinquième volume de la \textit{Enzyklopädie der mathematischen Wissenschaften} (20'000 pages), consacré à la physique. Il requiert dans un premier temps la collaboration d'Albert Einstein pour rédiger l'article sur la relativité, mais ce dernier refuse. Sommerfeld fait alors appel à Pauli, dont la relativité était la spécialité lors de son inscription aux cours de Sommerfeld. C'est ainsi qu'à 21 ans, Pauli publie son article de synthèse des théories de la relativité restreinte et de la relativité générale. En 1921, il obtient son doctorat avec pour sujet l'atome d'hydrogène qui montra clairement la limite du modèle de l'atome de Bohr, auquel il travaillera en tant qu'assistant de Max Born à Göttingen entre 1921 et 1922. Pendant les années 1922 et 1923, il travailla aux côtés de Niels Bohr à Copenhague. Entre 1923 et 1928, il enseigna à Hambourg avant de partir à l'ETH de Zürich, où il obtint un poste de professeur de physique théorique. À partir de 1935, il est parti pour les États-Unis, où il occupe des postes de professeur invité, notamment à l'Institute for Advanced Study à Princeton durant les années 1935-1936, mais aussi à l'Université du Michigan, en 1931 et 1941, et l'Université Purdue, en 1942. En 1946, il obtient la citoyenneté américaine, mais revient la même année à l'ETH de Zürich, où une place de professeur lui avait été gardée. En 1949, il devient citoyen suisse. Dans les années 1950, il retourne régulièrement à Princeton afin de donner des cours en tant que professeur invité. Dans les dernières années de sa vie, il participa à la fondation du CERN. Il meurt d'un ulcère gastro-duodénal.

\parpic[l][t]{%
  \begin{minipage}{40mm}
    \fbox{\includegraphics[width=110px,height=140px]{img/medaillons/pearson.eps}}
  \end{minipage}
}
\textbf{Pearson, Karl} (1857-1936) était un mathématicien britannique né à Londres et décédé à Surrey, considéré comme un des fondateurs des statistiques modernes. L'analyse statistique connaît un grand développement à la fin du 19ème siècle au Royaume-Uni et Karl Pearson domine ses contemporains par l'étendue et la variété de ses contributions bien que s'étant intéressés aux statistiques seulement à partir de l'âge de 33 ans. Il développe des méthodes d'analyse pour l'étude de la sélection naturelle et de l'eugénisme dont il est un ardent promoteur. Ses principales contributions sont la création du test du d'indépendance du Khi-deux destiné à estimer si les écarts observés dans un ensemble de variables par rapport aux valeurs théoriques peuvent être attribués ou non à un échantillonnage au hasard et la définition du coefficient de corrélation. Il reçoit la médaille Darwin (biologie) en 1898. Pearson était aussi consultant dans les entreprises. Il a entre autres donné des cours à William S. Gosset qui introduisit la loi de Student en 1910. Il est l'un des fondateurs de la revue \textit{Biometrika} dont il a été l'éditeur pendant 36 ans et qu'il a hissé au rang de meilleure revue de statistiques mathématiques.

\parpic[l][t]{%
  \begin{minipage}{40mm}
    \fbox{\includegraphics[width=110px,height=140px]{img/medaillons/penrose.eps}}
  \end{minipage}
}
\textbf{Penrose, Roger} (1931-) est un physicien et mathématicien né à Colchester (Angleterre). Penrose obtient son diplôme en mathématiques de l'Université du Collège de Londres et son doctorat à l'Université de Cambridge avec une thèse sur les méthodes tensorielles en géométrie algébrique. Entre 1964 et 1973, il enseigne la mathématique au Birkbeck College de Londres et rencontre le célèbre physicien Stephen W. Hawking avec lequel il travailla sur une théorie de l'origine de l'Univers en apportant sa contribution mathématique à la théorie de la Relativité Générale appliquée à la cosmologie et à l'étude des Trous Noirs. En 1965, à Cambridge, il prouve que des singularités gravitationnelles peuvent être formées à partir de l'effondrement gravitationnel d'étoiles massives en fin de vie. En 1971, Prenrose découvre les réseaux de spin qui devaient plus tard former la géométrie de l'espace-temps dans la théorie quantique à boucles. Professeur à Oxford, il reçut, avec Hawking, le prix Wolf 1988 pour la physique

\parpic[l][t]{%
  \begin{minipage}{40mm}
    \fbox{\includegraphics[width=110px,height=140px]{img/medaillons/picard.eps}}
  \end{minipage}
}
\textbf{Picard, Charles-Emile} (1856-1941) Né et décédé à Paris il fait ses études classiques au lycée de Vanves dès 1864, puis au lycée Napoléon (futur lycée Henry IV) de 1868 à 1874 où il se révèle excellent élève, mais peu attiré par la mathématique. Il obtient cette année-là le baccalauréat ès lettres puis l'année suivante le baccalauréat ès sciences. Il est reçu second à l'École Polytechnique, et premier à l'École Normale Supérieure. Finalement, passionné par les sciences, il opte pour cette dernière, où il prépare l'Agrégation qu'il réussit en 1877. Après divers postes d'assistant à Paris et Toulouse, il devient en 1881 Maître de Conférences à l'École Normale Supérieure. Son nom est déjà célèbre dans le cercle des mathématiciens, car il a démontré un théorème important sur les singularités des fonctions holomorphe qui lui vaudra une nomination pour devenir membre de l'Académie des Sciences. Il est cependant trop jeune, et son élection est reportée en 1889. En 1885, Picard devient professeur à la Sorbonne, où il occupe la chaire de Calcul Différentiel. Là encore, son jeune âge est une gêne (il faut avoir au minimum 30 ans pour occuper un tel poste) et il faut utiliser une procédure astucieuse pour contourner la législation. Plus tard, Picard occupera la chaire d'analyse et d'algèbre, et il exercera aussi à l'École Centrale des Arts et Manufacture (1894-1937): il y forme à la mécanique plus de 10'000 ingénieurs, et est, selon Hadamard, un excellent professeur. Les travaux de Picard sont ardus, et ouvrent la voie à de nouvelles recherches. Il est le premier à utiliser le théorème du point fixe dans une méthode d'approximations successives qui permet de résoudre des équations aux dérivées partielles. On lui doit également des travaux en géométrie algébrique, comme des recherches plus appliquées sur l'élasticité ou la chaleur. Il est aussi l'un des premiers défenseurs des théories d'Albert Einstein. Son \textit{Traité d'Analyse} constitua longtemps une référence, et Picard fut aussi philosophe et historien des sciences. Parmi les distinctions que Picard a reçues, citons qu'il présida le Congrès International des Mathématiciens, qu'il fut élu membre de l'Académie Française en 1924, et qu'il reçut la médaille d'or Mittag-Leffler en 1937.

\parpic[l][t]{%
  \begin{minipage}{40mm}
    \fbox{\includegraphics[width=110px,height=140px]{img/medaillons/planck.eps}}
  \end{minipage}
}
\textbf{Planck, Max} (1858-1947) était un physicien allemand né à Kiel et décédé à Göttingen considéré comme le fondateur de la physique quantique. Après avoir obtenu son baccalauréat à 17 ans à Münich où son père enseigne, Max Planck poursuit ses études de physique à Berlin. Passionné par la thermodynamique, il soutient une thèse de doctorat sur le second principe de la thermodynamique et la notion d'entropie 1879, notion qui restera le moteur explicatif de la majorité de ses recherches. L'année suivante, il devient maître de conférence à l'Université de Münich puis obtient la chaire de physique de l'Université de Kiel en 1885. Quatre ans plus tard, il est professeur de physique à l'Université de Berlin, poste qu'il occupera pendant près de 40 ans. En 1930 il prend la direction de l'Institut Kaiser Wilhem pour la recherche scientifique qui portera bientôt son nom. Amorcées par sa thèse de doctorat, les recherches de Planck en thermodynamique se portent rapidement sur le corps noir. Entité purement théorique, le corps noir absorbe toutes les radiations qu'il reçoit (le noir de carbone, en absorbant $97\%$ du rayonnement, se rapproche de cet idéal). Pour tenter d'expliquer ce phénomène, Planck élabore une nouvelle théorie. Il émet l'hypothèse que l'énergie d'un rayonnement ne peut être émise ou absorbée par la matière que par quantités finies, les quanta. Il montre alors que ces "paquets d'énergie" ont pour valeur $h \nu$, où $\nu$ est la fréquence du rayonnement et $h$ une constante universelle (la "constante de Planck"). En exposant sa théorie à la Société Allemande dePhysique en 1900, à Berlin, Planck ne sait pas encore qu'il vient d'inventer une nouvelle branche de la physique: la physique quantique. Sa découverte entraînera alors la création du modèle de l'atome par Niels Bohr, l'élaboration de la mécanique ondulatoire par Louis de Broglie , l'explication du phénomène photoélectrique par Albert Einstein ou encore la découverte du principe d'incertitude par Werner Heisenberg. Considéré comme l'un des plus célèbres physiciens, Planck recevra le prix Nobel en 1918.

\parpic[l][t]{%
  \begin{minipage}{40mm}
    \fbox{\includegraphics[width=110px,height=140px]{img/medaillons/poincare.eps}}
  \end{minipage}
}
\textbf{Poincaré, Henri} (1854-1912) était un mathématicien et physicien français né à Nancy et décédé à Paris dont on a dit qu'il était le dernier savant susceptible de connaître la totalité des mathématiques de son temps. Élève d'exception au Lycée Impérial de Nancy, il obtient en 1871, le baccalauréat ès lettres, mention Bien, et la même année son baccalauréat ès sciences. Il se classe premier au concours d'entrée à l'École Polytechnique en 1873, puis à l'École des Mines de Paris, comme ingénieur du Corps des Mines, en 1875. Il est licencié ès sciences en 1876. Nommé ingénieur des mines de troisième classe en 1879 à Vesoul, il obtient, la même année le doctorat ès sciences Mathématiques à la Faculté des Sciences de Paris, et devient chargé de cours d'Analyse à la faculté des sciences de Caen. Les premiers travaux de Poincaré portent sur les fonctions automorphes ou fuchsiennes, la théorie qualitative des équations différentielles et la théorie des fonctions. Dans une série de six articles publiés à partir de 1894, il est le créateur de la topologie algébrique, science en pleine expansion au 20ème siècle et dans laquelle plusieurs conjectures dues à Poincaré restent ouvertes. Il s'est vivement intéressé à la mécanique céleste: \textit{Les Méthodes nouvelles de la mécanique céleste}, trois volumes parus entre 1892 et 1899, annoncent les recherches modernes sur les systèmes dynamiques et le chaos. En physique-mathématique, il dégagea les propriétés du groupe de Poincaré-Lorenz, qui allaient quelques mois plus tard conduire à l'article fondamental d'Albert Einstein sur la Relativité Restreinte.

\parpic[l][t]{%
  \begin{minipage}{40mm}
    \fbox{\includegraphics[width=110px,height=140px]{img/medaillons/poisson.eps}}
  \end{minipage}
}
\textbf{Poisson, Siméon Denis} (1781-1840) était un mathématicien français dont les travaux ont porté sur les intégrales définies, la théorie électromagnétique et le calcul des probabilités. Sa famille le força à faire des études de médecine qu'il abandonna, en 1798, pour aller étudier la mathématique à l'École Polytechnique, où il fut l'élève de Laplace et  Lagrange, qui devinrent l'un et l'autre ses amis. Il enseigna à l'École Polytechnique à partir de 1802 et en 1808, il fut nommé astronome du Bureau des Longitudes et, à sa création, en 1809, professeur à la Faculté des Sciences. Les travaux les plus importants de Poisson portent sur les applications des mathématiques à la physique et à la mécanique. Son \textit{Traité de mécanique} a été l'ouvrage de référence en mécanique pendant de nombreuses années. Un mémoire, publié en 1812, contient les lois les plus usuelles de l'électrostatique et la théorie selon laquelle l'électricité est constituée de deux fluides dont les éléments semblables se repoussent, tandis que les éléments différents s'attirent. En mathématiques pures, il a publié une série d'articles sur les intégrales définies, et ses recherches sur les séries de Fourier ont annoncé celles de Dirichlet et de Riemann sur ce sujet. C'est dans l'ouvrage \textit{Recherches sur la probabilité des jugements}... (1837), qui est un livre important sur le calcul des probabilités, qu'apparaît pour la première fois la distribution de Poisson (ou "loi de Poisson"). Obtenue initialement comme une approximation de la loi binomiale de Bernoulli, elle est devenue fondamentale dans de très nombreux problèmes. Les autres publications de Poisson comprennent\textit{ Théorie nouvelle de l'action} (1831) et \textit{Théorie mathématique de la chaleur} (1835). Le nom de Poisson est attaché à de nombreuses notions mathématiques et physiques (intégrale et équation de Poisson en théorie du potentiel, crochets de Poisson dans la théorie des équations différentielles, rapport de Poisson en élasticité et constante de Poisson en électricité).

\parpic[l][t]{%
  \begin{minipage}{40mm}
    \fbox{\includegraphics[width=110px,height=140px]{img/medaillons/poynting.eps}}
  \end{minipage}
}
\textbf{Poynting, John Henry} (1852-1912) était un physicien né dans le Lancashire et décédé à Birmingham qui a travaillé, en autres, sur les ondes électromagnétiques. Il a défini ce que l'on appelle le "vecteur de Poynting" qui représente la puissance par unité de surface que transporte une onde électromagnétique et la direction de ce flux d'énergie. Poynting suivi l'école élémentaire dans un école dirigée par son père. De 1867 à 1872 il suivit les cours du collège d'Owen (aujourd'hui Université de Manchester) où il eut comme professeur Osborne Reynolds. De 1872 à 1875 il est étudiant à l'Université de Cambridge où il obtint les honneurs en mathématiques. À la fin des années 1870 il travailla au laboratoire Cavendish sous les ordres de James Clerk Maxwell. En 1903 il fut le premier à réaliser que la radiation solaire pouvait attirer les petites particules vers le Soleil, effet reconnu plus tard sous le nom d'effet Poynting-Robertson. Pendant l'année 1884, il analysa les prix des bourses de commerce, notamment ceux du blé, de la soie, et du coton, à l'aide de méthodes statistiques. Il fut professeur de physique au Mason Science College (qui devint plus tard l'Université de Birmingham) jusqu'à sa mort

\phantomsection
\addcontentsline{toc}{section}{R}
\label{sec:R}

\parpic[l][t]{%
  \begin{minipage}{40mm}
    \fbox{\includegraphics[width=110px,height=140px]{img/medaillons/ramanujan.eps}}
  \end{minipage}
}
\textbf{Ramanujan, Srivanasa} (1887-1920) était né à Erode, un petit village situé $400$ [km] au sud de Madras, dans une famille pauvre de la caste des Brahmanes. Il passe son enfance dans la ville de Kumbakonam, où son père exerce le métier de comptable chez un drapier. À partir de l'âge de 5 ans, il fréquente différentes écoles primaires avant de pouvoir intégrer la Town High School en 1898. En 1900, il commence à développer ses propres mathématiques en se basant sur son premier livre de mathématiques, \textit{La Trigonométrie plane}. Il définit seul une méthode pour résoudre les équations du 3e degré, puis du 4e, puis il tente aussi de résoudre celles du 5e degré, ignorant qu'elles ne peuvent être résolues par les radicaux. On est alors en 1902 et c'est à cette époque que Ramanujan se procure le second (et dernier!) livre dans lequel il puisera ses connaissances mathématiques de bases, \textit{Synopsis of elementary results in pure mathematics}, compilation d'environ $6'000$ théorèmes et autres formules par G.S. Carr. Ce livre étant essentiellement un livre de résultats, la plupart sans démonstrations, influencera le style futur de Ramanujan, qui n'a laissé que très peu de preuves de ses propres résultats. À 17 ans, sa démarche est déjà celle d'un chercheur en mathématiques. Comme ses résultats scolaires sont bons, il reçoit une bourse lui permettant d'entrer au Government College de Kumbakonam en 1904. Cependant, il consacre trop de temps à ses recherches en mathématiques et néglige les autres matières, ce qui lui vaut la suppression de cette bourse l'année suivante. Sans argent, il part, à l'insu de ses parents, pour la ville de Vizagapatnam où il poursuit ses travaux sur les séries hypergéométriques et les relations entre intégrales et séries. En 1906, il retourne à nouveau au lycée, à Madras cette fois-ci, avec l'idée de passer un examen lui permettant d'entrer à l'université. Il assiste quelques mois aux cours puis tombe malade. Au cours de l'examen, il réussit seulement en mathématiques et échoue partout ailleurs, ce qui lui interdit l'entrée à l'Université de Madras. Dans les années qui suivent, il continue alors de développer seul ses idées, sans aucune aide extérieure et sans connaissance des thèmes de recherche possibles, en dehors de ceux découlant des notions abordées dans le livre de Carr. Ramanujan étudie ainsi les fractions continues et les séries divergentes en 1908. Il tombe alors de nouveau très malade et doit subir, en 1909, une opération dont il aura du mal à se remettre. Il commence alors de poser et de résoudre des problèmes mathématiques dans le journal de la \textit{Société Indienne de Mathématiques} (SIM). En 1910, il développe des relations sur les équations modulaires elliptiques. Un an plus tard, la publication d'un article brillant sur les nombres de Bernoulli dans ce même journal lui vaut la reconnaissance de son travail par ses pairs. Bien qu'il ne possède aucun diplôme universitaire, il acquiert la réputation de génie des mathématiques dans la région de Madras. La même année, il rencontre le fondateur de la SIM, qui lui permet d'obtenir un emploi temporaire chez un comptable de Madras et lui conseille de contacter Ramachandra Rao, un mécène membre de la SIM. Grâce à cette lettre, Ramanujan obtient le poste et commence son travail en 1912. Il a alors la chance d'être entouré de personnes ayant une formation en mathématiques et qui s'intéressent à son travail. Le chef comptable du port de Madras est un mathématicien qui publie un article sur le travail de Ramanujan en 1913, \textit{On the distribution of primes}. D'autre part, un professeur du Madras Engineering College est intéressé par les capacités de Ramanujan. Ayant lui-même fait ses études à Londres, il écrit à un de ses professeurs de mathématiques, à qui il envoie de quelques résultats de Ramanujan. L'Université de Madras allouera plus tard une bourse à Ramanujan en 1913 et en 1914, Hardy le fait venir au Trinity College, à Cambridge. C'est le début d'une extraordinaire collaboration entre les deux hommes. En 1916, il obtient le titre de docteur de l'Université de Cambridge, malgré qu'il ne possède pas les diplômes requis pour préparer une thèse. En 1918, Ramanujan est élu membre de la Cambridge Philosophical Society. Trois jours plus tard, probablement le plus grand honneur de toute sa carrière, son nom apparaît sur la liste des élections des membres de la "Royal Society of London". Il a été proposé par une liste impressionnante d'éminents mathématiciens. Son élection a effectivement lieu en 1918 et il est également élu membre du Trinity College pour 6 ans. Ramanujan repart pour l'Inde en 1919. Cependant, son état de santé déjà très mauvais ne cesse de se dégrader. Il meurt l'année suivante probablement à cause de graves carences alimentaires. Ramanujan a laissé derrière lui un grand nombre de cahiers non-publiés (les fameux Carnets de Ramanujan), remplis de théorèmes que les mathématiciens continuent d'étudier. Aujourd'hui, ses travaux ont bien sûr des applications en physique théorique.

\parpic[l][t]{%
  \begin{minipage}{40mm}
    \fbox{\includegraphics[width=110px,height=140px]{img/medaillons/riccicurbastro.eps}}
  \end{minipage}
}
\textbf{Ricci-Curbastro, Gregorio} (1853-1925) Né à Lugo et décédé à Boulogne il était un mathématicien spécialiste de la géométrie différentielle et l'un des pères du calcul tensoriel. Après des études de philosophie et de mathématiques, Ricci soutient sa thèse de doctorat à l'Université de Pise. En 1880, il sera nommé professeur de physique-mathématique à l'Université de Padoue. Levi-Civita fut son élève et contribua avec Ricci à l'élaboration de son calcul différentiel absolu (1900) visant à expliciter en mécanique, dans des espaces abstraits (variétés différentiables), des relations indépendantes du système de coordonnées utilisé, inhérentes au phénomène étudié (invariants différentiels). Associée à la géométrie différentielle de Gauss et de Riemann, le célèbre physicien Albert Einstein trouva, dans cette nouvelle approche de la mécanique qu'il nomma "calcul tensoriel" (1916), les outils mathématiques nécessaires à sa théorie de la Relativité Générale.

\parpic[l][t]{%
  \begin{minipage}{40mm}
    \fbox{\includegraphics[width=110px,height=140px]{img/medaillons/riemann.eps}}
  \end{minipage}
}
\textbf{Riemann, Georg Friedrich Bernhard} (1826-1866) était un mathématicien allemand. Au lycée, Riemann étudie la Bible intensivement, mais il est distrait par les mathématiques. Il essaie même de prouver, mathématiquement, l'exactitude de la Genèse. Ses professeurs sont surpris par ses capacités à résoudre des problèmes complexes en mathématique. En 1846, grâce à l'argent de sa famille, il commence à étudier la philosophie et la théologie pour devenir prêtre afin de financer sa famille. En 1847, son père l'autorise à étudier les mathématiques. Il étudie d'abord à l'Université de Göttingen où il rencontre Carl Friedrich Gauss, puis à l'université de Berlin, où il a entre autres comme professeurs Jacobi, Steiner et Dirichlet. Dans sa thèse, présentée en 1851 sous la direction de Gauss, Riemann met au point la théorie des fonctions d'une variable complexe. En 1854 il donne un exposé qui jette les bases de la géométrie différentielle. Il y introduit la bonne façon d'étendre à $n$ dimensions les résultats de Gauss lui-même sur les surfaces. Cette présentation a profondément changé la conception de la notion de géométrie, notamment en ouvrant la voie aux géométries non euclidiennes et à la théorie de la Relativité Générale. On lui doit également d'importants travaux sur les intégrales, poursuivant ceux de Cauchy, qui ont donné entre autres ce qu'on appelle aujourd'hui les "intégrales de Riemann". Intéressé par la dynamique des gaz, il jette les bases de l'analyse des équations aux dérivées partielles de type hyperbolique. Il succédera à Dirichlet sur la chaire de Gauss en 1859. À 39 ans, il fut emporté par la tuberculose.

\phantomsection
\addcontentsline{toc}{section}{S}
\label{sec:S}

\parpic[l][t]{%
  \begin{minipage}{40mm}
    \fbox{\includegraphics[width=110px,height=140px]{img/medaillons/salam.eps}}
  \end{minipage}
}
\textbf{Salam, Abdus} (1926-1996) était un physicien pakistanais, ayant reçu le prix Nobel de physique en 1979 pour ses travaux sur l'interaction électrofaible, synthèse de l'électromagnétisme et de l'interaction faible. Né à Jhang Sadar, il étudie au Government College à Lahore. À l'âge de 14 ans, Salam obtint les meilleures notes jamais enregistrées pour l'examen d'entrée à l'Université du Punjab. Persécuté par la majorité musulmane de son pays pour son appartenance religieuse (ahmadiste), il doit fuir son pays. Réfugié en Grande-Bretagne, il obtient en 1952, un doctorat en mathématiques et en physique de l'Université de Cambridge. Sa thèse doctorale fut une étude fondamentale en électrodynamique quantique. Ses travaux le rendirent célèbre internationalement. Il retourna au Government College de Lahore en tant que professeur de mathématiques, garda cette fonction de 1951 à 1954, puis retourna ensuite à Cambridge en tant conférencier en Mathématiques. Il enseigne dans ces établissements, puis en 1957, est nommé professeur de physique théorique à l'Imperial College de Londres. Il y demeura jusqu'à sa retraite. En 1959, il devient le plus jeune membre de la Royal Society à l'âge de 33 ans. Durant les années 1960, Salam joua un rôle important dans l'établissement de l'agence de recherche nucléaire du Pakistan, et de l'agence de recherche spatiale du Pakistan, de laquelle il fut le directeur fondateur. En 1964, il devient directeur du Centre international de Physique Théorique de Trieste, nouvellement créé. Cette même année, il est lauréat de la Médaille Hughes. En 1967, avec le physicien Steven Weinberg, Salam propose une théorie permettant d'unifier les interactions électromagnétiques et faibles entre particules élémentaires, théorie qui sera confirmée par l'expérience. Salam sera ainsi le premier musulman à obtenir le prix Nobel de physique en 1979, conjointement aux physiciens Sheldon Lee Glashow et Weinberg.

\parpic[l][t]{%
  \begin{minipage}{40mm}
    \fbox{\includegraphics[width=110px,height=140px]{img/medaillons/samuelson.eps}}
  \end{minipage}
}
\textbf{Samuelson, Paul} (1915-2009) était un économiste américain, prix Nobel d'économie en 1970 et chef de file de l'école qu'il appela la "synthèse néo-classique", qui entendait reprendre à son compte à la fois les théories de Keynes en macroéconomie et les enseignements néoclassiques en microéconomie. Samuelson est considéré comme un des père de la microéconomie traditionnelle actuelle et serait l'un des pionniers économistes à généraliser, dans un cadre économique, l'usage des modèles mathématiques mis en place pour l'analyse thermodynamique. Ainsi, il aurait aidé à faire passer l'économique de discipline principalement littéraire en un domaine du savoir hautement mathématique, formalisé et axiomatisé.

\parpic[l][t]{%
  \begin{minipage}{40mm}
    \fbox{\includegraphics[width=110px,height=140px]{img/medaillons/savart.eps}}
  \end{minipage}
}
\textbf{Savart, Felix} (1791-1841) était un médecin chirurgien et physicien né en Ardennes et décédé à Paris. Inventeur du sonomètre, d'une roue dentée qui porte son nom et du polariscope. Il jeta les bases de la physique moléculaire. Avec le physicien Jean-Baptiste Biot, il mesura le champ magnétique créé par un courant et formula la "loi de Biot-Savart". Il étudia également les propriétés des cordes vibrantes. Il fut membre de l'Académie des Sciences, élu en 1827, et titulaire de la Chaire de Physique Générale et Expérimentale du Collège de France, nommé en 1836, succédant à André-Marie Ampère. Il est élu membre étranger de la Royal Society en 1839. Son nom a été donné à une unité de mesure des intervalles musicaux : le savart.

\parpic[l][t]{%
  \begin{minipage}{40mm}
    \fbox{\includegraphics[width=110px,height=140px]{img/medaillons/say.eps}}
  \end{minipage}
}
\textbf{Say, Jean-Baptiste} (1767-1832)  était un économiste, journaliste et industriel français né à Lyon et décédé à Paris. Il est issu d'une famille de négociants nîmois ayant émigré à Amsterdam puis à Genève. C'est au cours d'un voyage en Grande-Bretagne, où la révolution industrielle est en cours, qu'il adoptera les idées libérales et en particulier les théories d'Adam Smith dont il sera un ardent défenseur de retour en France. En 1789, il publie la brochure: la \textit{Liberté de la presse}. En 1792, il participe aux campagnes militaires de la révolution française en Champagne. D'abord employé dans une banque, il dirigea ensuite une filature de coton à Auchy-lès-Hesdin, dans le Pas-de-Calais. Ses nombreux ouvrages d'économie politique firent qu'il fut nommé professeur au Conservatoire National des Arts et Métiers en 1821, puis au Collège de France en 1830. La "loi de Say", ou "loi des débouchés", stipule que plus les producteurs sont nombreux et les productions multiples, plus les débouchés sont faciles, variés et vastes. Dans une économie où la concurrence est libre et parfaite, les crises de surproduction sont impossibles. Il ne peut y avoir de déséquilibre global dans les économies de marché et de libre entreprise, il y a un équilibrage spontané des flux économiques (production, consommation, épargne, investissement). Cette loi est parfois réduite à tort à la formule: toute offre crée sa propre demande. Un meilleur résumé de cette approche serait: on dépense que l'argent qu'on a gagné. L'économie de l'offre, dans la tradition de Say, s'oppose à l'économie de la demande, qui est celle de Malthus et plus tard de Keynes.

\parpic[l][t]{%
  \begin{minipage}{40mm}
    \fbox{\includegraphics[width=110px,height=140px]{img/medaillons/schaefer.eps}}
  \end{minipage}
}
\textbf{Schaefer, Milner Baily} (1912-1970) Né au Wyoming et décédé à San Diego a étudié à l'Université de Washington où il avait obtenu un baccalauréat en science en 1935. Dès l'obtention du baccalauréat, il a travaillé au Département des Pêches de l'état de Washington à Seattle. De 1937 à 1942, il a travaillé à la Comission des Pêches du Saumon du Pacifique à Westminster, Colombie-Britannique. Il a servi dans la marine durant la guerre et par la suite, il a occupé divers postes en tant que biologiste spécialiste des pêches. Après avoir terminé son doctorat à l'Université de Washington, en 1950, Schaefer est devenu directeur des enquêtes de l'IATTC (Inter-American Tropical Tuna Commission), une commission internationale des pêches. Pendant les dix années qui ont suivi, il a travaillé sur la théorique de la dynamique de la pêche et mis au point un modèle de population des espèces marines qui est connu sous le nom de "modèle de Schaefer". Au courant des années 1950, Schaefer est devenu de plus en plus impliqué dans plusieurs comités, groupes et organisations concernés par les ressources marines, en particulier la pêche et tous les aspects de l'océanographie. Durant cette période, il a donné des cours sur la dynamique et l'exploitation des populations de poissons. En 1962, il démissionna de son poste de directeur des enquêtes à la IATTC pour occuper le poste de directeur de l'Institut des Ressources marines de l'Université de Californie tout en agissant comme conseiller scientifique pour l'IATTC.


\parpic[l][t]{%
  \begin{minipage}{40mm}
    \fbox{\includegraphics[width=110px,height=140px]{img/medaillons/scholes.eps}}
  \end{minipage}
}
\textbf{Scholes, Myron} (1941-) Né à Ontaria, il présente son doctorat en 1969 à l'Université de Chicago. Il occupe en 1988 la chaire Frank E. Buck de professeur de finance au Graduate School of Business de l'Université Stanford (Californie) où il dirige également des recherches pour l'Institution Hoover. Il a reçu le prix Nobel d'économie en 1997 pour avoir élaboré, avec Fischer Black, une méthode d'évaluation des instruments financiers dérivés (résultat mathématique novateur pour estimer les risques liés aux options sur actions) ayant ouvert de nouveaux horizons au champ des évaluations économiques. Le colauréat de Myron Scholes, Robert Merton, a joué un rôle très important dans l'élaboration de cette méthode d'évaluation ainsi que dans les applications qu'elle a permises pour améliorer la gestion des risques attachés aux nouveaux produits financiers. Déjà en 1900, Louis Bachelier, présentait à la Sorbonne une thèse de doctorat au titre visionnaire: \textit{Théorie de la spéculation}. Dans les années 1960, des auteurs tels James Boness et Paul Samuelson (Prix Nobel d'économie en 1970) proposaient des modèles pour déterminer les prix d'équilibre des options. Leurs hypothèses ne se sont pas révélées suffisamment réalistes pour entraîner des applications, mais des améliorations apportées à ces modèles au début des années 1970 ont permis d'obtenir des résultats plus satisfaisants. C'est en 1973 que Scholes et Black mettent leurs compétences en commun et proposent la première version de la formule de calcul du prix des options qui leur vaudra le prix Nobel. Si Myron Scholes et Fischer Black ont eu l'intuition fondamentale de la démonstration, ils ont pris pour base de recherche le modèle d'équilibre des actifs financiers (ou Capital Asset Pricing Model, dit C.A.P.M.) de leur compatriote William Sharpe récompensé à ce titre par le jury du Nobel en 1990 (les deux autres lauréats étaient Harry Markowitz et Merton Miller).

\parpic[l][t]{%
  \begin{minipage}{40mm}
    \fbox{\includegraphics[width=110px,height=140px]{img/medaillons/schrodinger.eps}}
  \end{minipage}
}
\textbf{Schrödinger, Erwin} (1887-1961) Né et décédé à Vienne il entre au gymnase de cette même ville en 1898. Presque depuis son premier jour de classe jusqu'à son départ du lycée huit ans plus tard, Schrödinger fut un excellent élève. Il était toujours premier de sa classe en travaillant dur chez lui entre les quatre murs de son bureau personnel.. Il poursuit ses études à l'Université d'Iéna (Allemagne). En 1920, il est nommé professeur à la Haute École Technique de Stuttgart puis à l'Université de Breslau l'année suivante. En 1927, il succède à Max Planck à l'Université de Berlin. Israélite, il quitte le pays à l'avènement du national-socialisme pour se rendre à Oxford où il obtient une chaire en 1933. Sept ans plus tard, il devient professeur de physique théorique à Dublin à l'Institut des Hautes Études de l'État Libre d'Irlande. Il ne rentrera en Autriche qu'en 1956. Schrödinger comme son contemporain Albert Einstein avait horreur d'apprendre par coeur et d'être forcé de retenir des faits inutiles. Les premiers travaux de Schrödinger portent sur l'étude des couleurs et la théorie des quanta. Mais il est avant tout reconnu pour ses recherches en mécanique ondulatoire, discipline développée par le Français Louis de Broglie. L'équation de Schrödinger, élaborée en 1926, permet de calculer la fonction d'onde d'une particule se déplaçant dans un champ. En établissant cette équation de propagation, il donne à la mécanique quantique un outil intuitif aujourd'hui indispensable (au contraire de l'approche matricielle et abstraite de Heisenberg) qu'Einstein qualifia d'idée de génie! Avec celle de Werner Heisenberg, la théorie de Schrödinger constitue ainsi la base de la mécanique quantique. En 1933, Schrödinger partage le prix Nobel de physique avec Paul Dirac pour leur contribution au développement de cette nouvelle discipline. Schrödinger essaiera également d'appliquer sa théorie à la biologie et à la génétique dans ses ouvrages \textit{What is life} (1944) et \textit{Science and Humanism} (1951).

\parpic[l][t]{%
  \begin{minipage}{40mm}
    \fbox{\includegraphics[width=110px,height=140px]{img/medaillons/schwartz.eps}}
  \end{minipage}
}
\textbf{Schwartz, Lawrence} (1915-2002) était un mathématicien français né et décédé à Paris. Ses travaux sont principalement relatifs à l'analyse. Ancien élève de l'École Normale Supérieure, Laurent Schwartz a enseigné de 1959 à 1960 et de 1963 à 1983 à l'École Polytechnique. En 1975, il est élu membre de l'Académie des Sciences. Sa thèse (1943) porte sur l'approximation et l'étude des sommes d'exponentielles. La théorie des distributions, dont l'idée initiale remonte à 1945, lui a valu la médaille Fields en 1950. Le langage et les notations de Schwartz pour les distributions ont été adoptées par les mathématiciens et constituent le cadre naturel de la théorie des équations aux dérivées partielles. De 1959 à 1962, Schwartz se consacre à la physique théorique: l'emploi des distributions lui permet une formulation mathématique correcte de la théorie des particules élémentaires. Il a aussi effectué des recherches sur les mesures de Radon et sur les espaces topologiques quelconques ; il a écrit diverses publications sur les probabilités cylindriques et les désintégrations de mesures.

\parpic[l][t]{%
  \begin{minipage}{40mm}
    \fbox{\includegraphics[width=110px,height=140px]{img/medaillons/schwarzschild.eps}}
  \end{minipage}
}
\textbf{Schwarzschild, Karl} (1873-1916) était un astronome mathématicien et physicien né à Francfort et décédé à Postdam qui prédit l'existence des Trous Noirs. Sa curiosité pour les étoiles se manifesta dès ses premières années scolaires, lorsqu'il construisit un petit télescope. Témoin de cet intérêt, son père le présenta à un ami mathématicien qui avait un observatoire privé. Schwarszchild apprit à utiliser un télescope et étudia des mathématiques plus avancées qu'à l'école. Il devint célèbre avec ses deux premiers articles sur la théorie des orbites publiés à l'âge de 16 ans alors qu'il était encore au collège. Il étudia à l'Université de Strasbourg, puis de Münich, et obtint son doctorat à l'âge de 23 ans pour des travaux sur les théories de Henri Poincaré. Il fut alors engagé en tant qu'assistant à l'Observatoire Kuffner à Ottakring. Il se consacra principalement à la photométrie: il accomplit un travail de pionnier pour améliorer les plaques photographiques et implanter leur utilisation en astronomie, ainsi que dans l'étude spectrale des étoiles. De 1901 à 1909, il officia comme professeur au prestigieux institut de Göttingen, où il eut l'occasion de travailler avec des personnalités telles que David Hilbert et Hermann Minkowski. Il occupa ensuite un poste à l'Observatoire d'astrophysique de Potsdam en 1909. Schwarzschild est surtout connu pour ses contributions théoriques, tant en physique du Soleil qu'en Relativité Générale, ou en cinématique stellaire, ainsi que dans divers domaines de l'astrophysique. En 1916, il détermina une grandeur, dite "rayon de Schwarzschild", dans le cadre de la théorie de la relativité, énoncée peu de temps avant par Albert Einstein. Lorsqu'une étoile suffisamment massive explose en supernova, la contraction gravitationnelle produit ce que l'on appelle un "Trou Noir": rien, pas même la lumière, ne peut sortir de ce champ de gravitation intense. Lorsque le rayon d'une masse gazeuse devient inférieur au "rayon de Schwarzschild" pour cette masse, elle s'effondre en Trou Noir.

\parpic[l][t]{%
  \begin{minipage}{40mm}
    \fbox{\includegraphics[width=110px,height=140px]{img/medaillons/shannon.eps}}
  \end{minipage}
}
\textbf{Shannon, Claude Elwood} (1916-2001) Né au Migichan et décédé au Massachusetts c'était un mathématicien spécialiste en mathématiques appliquées et un ingénieur électricien, qui développa la théorie de la communication, aujourd'hui connue sous le nom de la "théorie de l'information". Shannon suivit les cours de l'Université du Michigan et obtint en 1940 son doctorat de l'Institut de Technologie du Massachusetts (M.I.T.), de la faculté duquel il devint un membre, en 1956, après avoir travaillé aux laboratoires de téléphone Bell. En 1949, Shannon publia la \textit{Théorie mathématique de la communication}, un article dans lequel il présenta son concept initial pour une théorie unificatrice de la transmission et du traitement des informations. Les informations, selon cette théorie, incluent toutes formes de messages transmis, y compris ceux envoyés le long des canaux nerveux des organismes vivants. La théorie de l'information est aujourd'hui importante dans de nombreux domaines.


\parpic[l][t]{%
  \begin{minipage}{40mm}
    \fbox{\includegraphics[width=110px,height=140px]{img/medaillons/sharpe.eps}}
  \end{minipage}
}
\textbf{Sharpe, William Forsyth} (1934-) est un économiste né à Boston. L'Académie Royale des Sciences de Suède a décerné en 1990 le prix Nobel de sciences économiques à 3 professeurs américains: Harry Markowitz, Merton Miller et William Sharpe. Même si les travaux récompensés étaient déjà anciens et se situent pour l'essentiel entre 1950 et 1970, l'Académie a jugé que les lauréats étaient des novateurs dans le domaine de la théorie de l'économie financière et du financement des entreprises. Ils ont en effet tous contribué à faire sortir de l'ombre de quelques universités américaines, une nouvelle discipline: la finance. C'était la première fois que l'Académie Royale de Suède récompensait des travaux traitant des marchés boursiers et de la gestion de portefeuilles plutôt que des grands équilibres économiques. William Sharpe, de l'Université Stanford, fut récompensé pour son modèle d'équilibre des actifs financiers et pour ses travaux sur la théorie de la formation des prix des avoirs financiers. Il s'est aussi engagé dans ses recherches dans la voie ouverte par Harry Markowitz. Ce dernier avait en effet élaboré une procédure complexe de sélection des titres boursiers afin d'optimiser un portefeuille de placements. Mais la mise en oeuvre de ce modèle a très vite posé des problèmes d'ordre pratique, au point que la collecte des informations nécessaires et leur traitement devenaient presque impossibles avec les ordinateurs disponibles dans les années 1960. C'est la raison pour laquelle William Sharpe se mit à chercher une méthode de sélection des portefeuilles efficients plus simple. Il découvre que les variations de la rentabilité de chaque titre sont liées, linéairement, à la variation du marché dans son ensemble, mesurée par l'indice du marché concerné (par exemple l'indice Standard \& Poor 500 aux États-Unis, ou le C.A.C. 40 en France). Le nombre de statistiques nécessaires s'en est trouvé fortement réduit: $302$ statistiques au lieu de $3'150$ dans le modèle Markowitz pour $100$ titres, $602$ au lieu de $20'300$ pour $200$ titres et $10'002$ au lieu de $125'750$ pour $300$ titres, le calcul fut aussitôt facilité. C'est à partir de ce concept, simple en apparence, que Sharpe découvre ensuite le fameux coefficient Bêta reliant la rentabilité d'un titre à celle de l'indice du marché et constituant une mesure du risque associé à la volatilité du marché. Au-delà de leur apport pratique, les travaux de Sharpe ont contribué de façon décisive à la formulation d'une théorie de la formation des cours des actifs financiers plus connue sous le nom de "modèle C.A.P." (Capital Asset Pricing) ou, en français, de "Modèle d'équilibre des actifs financiers" (MEDAF).

\parpic[l][t]{%
  \begin{minipage}{40mm}
    \fbox{\includegraphics[width=110px,height=140px]{img/medaillons/smith.eps}}
  \end{minipage}
}
\textbf{Smith, Adam} (1723-1790) Né à Kirkcaldy et décédé à Edimbourg, en Écosse, c'était un économiste et philosophe. Il étudia aux Universités de Glasgow et Oxford. De 1748 à 1751, il enseigna la rhétorique et les belles-lettres à Édimbourg. Durant cette période, il se lia avec le philosophe David Hume, dont la pensée exerça une grande influence sur les conceptions de Smith en matière d'éthique et d'économie. Smith fut nommé professeur de logique en 1751 puis professeur de philosophie morale en 1752 à l'Université de Glasgow. Plus tard, il rassembla les cours d'éthique qu'il dispensait et les publia dans sa première oeuvre maîtresse intitulée \textit{Theory of Moral Sentiments}, en 1759. En 1763, il démissionna de son poste de professeur pour accompagner le duc de Buccleuch dans un voyage de 18 mois en France et en Suisse, en qualité de précepteur. De 1766 à 1776, il vécut à Kirkcaldy où il travailla à son ouvrage fondamental, la \textit{The Wealth of Nations}. Smith fut ensuite nommé commissaire des douanes à Édimbourg en 1778, poste qu'il occupa jusqu'à sa mort. En 1787, il fut également nommé recteur de l'université de Glasgow. Son célèbre traité \textit{An Inquiry into the Nature and Causes of the Wealth of Nations} (1776), première étude tentant de décrire la nature du capital et le développement historique de l'industrie et des échanges entre les pays européens, lui valut d'être considéré comme le père de la science économique moderne. La \textit{The Wealth of Nations} constitue le premier essai traitant de l'histoire de la science économique qui considère l'économie politique comme une discipline autonome, distincte de la science politique, de l'éthique et de la jurisprudence. Smith y propose une analyse du processus de production et de répartition de la richesse, et démontre que les sources principales de tout revenu, c'est-à-dire les formes fondamentales dans lesquelles la richesse est distribuée, sont les rentes, les salaires et les profits. \textit{The Wealth of Nations} affirme contre les physiocrates le principe selon lequel le travail est la source de toute richesse, et présente le développement de l'industrie comme une source d'accroissement de la production. Pour Smith, théoricien du capitalisme libéral, le progrès économique et moral procède de la concurrence, la production et les échanges de biens ne pouvant être stimulés, et en conséquence le niveau de vie général amélioré, que lorsque les gouvernements régulent et contrôlent au minimum les activités industrielles et commerciales individuelles. Pour décrire cette situation, il parle d'un ordre naturel réglé par la "main invisible", qui fait naturellement converger la somme des intérêts individuels vers l'intérêt général. En conséquence, trop d'intervention de l'État dans ce contexte de libre concurrence ne pourrait être que néfaste.

\parpic[l][t]{%
  \begin{minipage}{40mm}
    \fbox{\includegraphics[width=110px,height=140px]{img/medaillons/sommerfeld.eps}}
  \end{minipage}
}
\textbf{Sommerfeld, Arnold} (1868-1951) était un physicien allemand né à Königsberg et décédé à Münch. Il étudia les mathématiques et les sciences naturelles à l'Université de Königsberg où il reçut son doctorat en 1891. Il occupa successivement les chaires de mathématiques à Clausthal (1897), de mathématiques appliquées à Aix-la-Chapelle (1900) et de physique théorique à MÜnich (1906-1931). En 1897, il commença, avec C. F. Klein, un traité en 4 volumes sur le gyroscope, qu'il mit 13 ans à terminer et, à la même époque, fit également des recherches dans d'autres domaines de physique appliquée et d'ingénierie, comme la friction, la lubrification et la radio. On lui doit une amélioration du modèle de Bohr (1916) introduisant des orbites elliptiques et des corrections relativistes. Ce nouveau modèle, qui implique une dépendance de l'énergie vis-à-vis du deuxième nombre quantique, permet d'expliquer la structure fine des raies spectrales émises par les atomes. Sommerfeld introduisit d'ailleurs la fameuse "constante de structure fine". Il s'intéressa également après Drude et Lorenz au modèle des électrons libres qui explique certaines propriétés des métaux, en particulier la conduction, en considérant un comportement quantique des électrons. Il participa ainsi aux développements de la théorie des bandes en physique du solide, formulant en 1928 l'idée selon laquelle les électrons occupent des états quantifiés dans la matière.

\parpic[l][t]{%
  \begin{minipage}{40mm}
    \fbox{\includegraphics[width=110px,height=140px]{img/medaillons/stokes.eps}}
  \end{minipage}
}
\textbf{Stokes, George Gabriel} (1819-1903) était un mathématicien et physicien né en Irlande et décédé à Cambridge. En 1841, il reçoit son diplôme avec mention d'honneur de l'Université de Cambridge et entame une carrière de chercheur. Influencé par son ancien professeur, il se consacre à l'étude des fluides visqueux. Il publie en 1845 le résultat de ses travaux sur les mouvements des fluides dans sa thèse \textit{On the theories of the internal friction of fluids in motion}. Son approche mathématique décrivant l'écoulement d'un fluide newtonien imcompressible dans un espace tridimensionel, en ajoutant une force de viscosité à partir des équations d'Euler (Principes généraux du mouvement des fluides, 1755), est à l'origine des équations de Navier-Stokes. L'ensemble de ses recherches est synthétisé par son traité \textit{Report on recent research in Hydrodynamics}, paru en 1846, texte fondateur de l'hydrodynamique. Il devient dès 1849 professeur à la chaire de mathématique de cette même université. Élu en 1851 à la Royal Society, il en sera le président de 1885 à 1890. Les trois derniers postes cités avaient été occupés par Isaac Newton. Il est lauréat du prix Smith en 1841, de la médaille Rumford en 1852 et de la médaille Copley en 1893.

\parpic[l][t]{%
  \begin{minipage}{40mm}
    \fbox{\includegraphics[width=110px,height=140px]{img/medaillons/stefan.eps}}
  \end{minipage}
}
\textbf{Stefan, Josef} (1835-1893) était un physicien autrichien né à Sankt Peter près de Klagenfurt et décédé à Vienne. Les travaux originaux de Stefan comprennent la théorie cinétique des gaz, l'hydrodynamique et surtout la théorie du rayonnement. Après des études à l'Université de Vienne où il obtient son doctorat en 1858, nommé Privatdozent de physique-mathématique, il devient professeur de physique en 1863, puis directeur de l'Institut de Physique (1866). Membre de l'Académie des Sciences de Vienne, il en est le secrétaire à partir de 1875. Avant les travaux de Stefan, G. R. Kirchhoff avait déjà décrit les propriétés du "corps parfaitement noir", susceptible d'absorber la totalité du rayonnement incident et d'émettre un spectre étendu de longueurs d'ondes. Stefan démontre empiriquement en 1879 que l'intensité du rayonnement du corps noir est proportionnelle à la quatrième puissance de sa température absolue, relation connue depuis sous le nom de "loi de Stefan-Boltzmann", Boltzmann l'ayant déduite en 1884 de considérations thermodynamiques. Cette loi constitue l'une des premières étapes importantes qui ont conduit à l'interprétation du rayonnement du corps noir et à la théorie quantique du rayonnement.

\parpic[l][t]{%
  \begin{minipage}{40mm}
    \fbox{\includegraphics[width=110px,height=140px]{img/medaillons/sturm.eps}}
  \end{minipage}
}
\textbf{Sturm, Charles François} (1803-1855) Après avoir été étudiant à l'Université de Genève (sa ville natale), Sturm se rend, pour être précepteur dans la famille De Broglie, à Paris, où il fréquente les plus grands savants de l'époque et où il se fixe définitivement à partir de 1825. Il détermine en 1826 la vitesse de propagation du son dans l'eau, ce qui lui vaut, l'année suivante, le grand prix de mathématiques proposé pour le meilleur mémoire sur la compressibilité des liquides. En 1829, il énonce le célèbre théorème qui porte son nom, essentiel pour l'étude des propriétés des racines d'une équation algébrique et qui précise le nombre de racines réelles d'une équation numérique comprises entre deux limites données. Il publie la démonstration de ce théorème en 1835. À partir de 1830, en liaison avec son ami Liouville, il aborde le problème de la théorie générale des oscillations et étudie des équations différentielles du second ordre (problème de Sturm-Liouville) dans plusieurs articles, dont \textit{Sur les équations différentielles linéaires du second ordre} (1836) et\textit{ Sur une classe d'équations à différences partielles} (1836). Les méthodes employées seront à l'origine de nombreux travaux et découvertes mathématiques. Il est élu en 1836 à l'Académie des Sciences et travaille à l'École Polytechnique. Succédant à Poisson, il enseigne, à partir de 1840, à la faculté des sciences de Paris (chaire de mécanique). Ses \textit{Cours d'analyse de l'École polytechnique} (1857-1863) et ses \textit{Cours de mécanique de l'École polytechnique} (1861) seront publiés après son décès à Paris

\phantomsection
\addcontentsline{toc}{section}{T}
\label{sec:T}

\parpic[l][t]{%
  \begin{minipage}{40mm}
    \fbox{\includegraphics[width=110px,height=140px]{img/medaillons/taylor.eps}}
  \end{minipage}
}
\textbf{Taylor, Brook} (1685-1731) était un mathématicien anglais né à Edmonton et décédé à Londres, célèbre pour ses contributions au développement du calcul infinitésimal. Taylor fit ses études au collège Saint John, à Cambridge. Il obtint, en 1708, une remarquable solution du problème du centre d'oscillation, qui pourtant demeura inédite jusqu'en 1714 lorsque son droit de priorité lui fut contesté par Jean Bernoulli. L'ouvrage de Taylor, \textit{Methodus incrementorum directa et inversa} (1715), ajoute aux mathématiques supérieures un nouveau chapitre, que l'on appelle de nos jours le "calcul des différences finies". Entre autres applications ingénieuses, il s'en sert pour déterminer la forme du mouvement d'une corde vibrante en le réduisant avec succès aux principes de la mécanique. Le même ouvrage contient la célèbre formule connue sous le nom de "théorème de Taylor", dont l'importance n'apparut qu'en 1772, quand Louis de Lagrange réalisa sa puissance et en fit le principe fondamental du calcul différentiel. Dans son essai \textit{Linear Perspective}, Taylor pose les principes de l'art sous une forme originale et plus générale qu'aucun de ses prédécesseurs. Mais l'ouvrage souffrit de la confusion et du manque de clarté qui affectaient la plupart de ses écrits. Taylor fut élu membre de la Royal Society en 1712. Il siégea la même année au comité chargé de régler les querelles de priorité entre Newton et Leibniz et fut secrétaire de la société de 1714 à 1718. À partir de 1715, ses recherches prirent une orientation philosophique et religieuse.

\parpic[l][t]{%
  \begin{minipage}{40mm}
    \fbox{\includegraphics[width=110px,height=140px]{img/medaillons/teller.eps}}
  \end{minipage}
}
\textbf{Teller, Edward} (1908-2003) était un physicien nucléaire né à Budapest et décédé à Stanford. Il quitte Budapest en 1926 pour aller à Karlsruhe (Allemagne), afin d'étudier la chimie, mais très vite une affinité se créera avec la nouvelle théorie de la physique quantique ce qui l'amènera à étudier à l'Université de Leipzig où il obtiendra son doctorat à l'âge de 22 ans. Teller obtint ce titre sous la direction de Werner Heisenberg qui participa plus tard activement dans le camp des nationalistes allemands lors de la seconde guerre mondiale. En 1935, Teller s'expatria aux États-Unis et ses compétences dans la physique de pointe l'amenèrent à se faire beaucoup de relations et une très bonne réputation dans la communauté scientifique. Il fut ainsi nommé professeur dans de nombreuses universités américaines et travailla en 1942 au projet Manhattan où il mena des travaux très importants qui permirent de créer la première bombe nucléaire à fission. Le travail effectué, Teller soutint la continuité du travail pour la recherche d'une bombe thermonucléaire par peur de l'avancée des Russes dans ce domaine (Teller était anticommuniste et très bon ami de Landau qui se fit arrêter par la police communiste). Teller réussit à convaincre l'administration américaine à financer les recherches pour une bombe à hydrogène et mena les travaux avec succès qui fait qu'on le considère aujourd'hui comme le père de la bombe H.

\parpic[l][t]{%
  \begin{minipage}{40mm}
    \fbox{\includegraphics[width=110px,height=140px]{img/medaillons/tesla.eps}}
  \end{minipage}
}
\textbf{Tesla, Nikola} (1856-1943) était un inventeur et ingénieur serbe de génie dans le domaine de l'électricité décédé à New York. Il est souvent considéré comme l'un des plus grands scientifiques dans l'histoire de la technologie, pour avoir déposé plus de 900 brevets (qui sont pour la plupart repris au compte de Thomas Edison) traitant de nouvelles méthodes pour aborder la conversion de l'énergie. En 1875, il entre à l'École Polytechnique de Graz, en Autriche, où il étudie la mathématique, la physique et la mécanique. Une bourse lui est attribuée par l'administration des Confins Militaires (Vojna Krajina), le mettant à l'abri des problèmes d'argent. Ceci ne l'empêche cependant pas de travailler avec acharnement pour assimiler le programme des deux premières années d'études en un an. L'année suivante, la suppression des Confins Militaires retire toute aide financière à Tesla, hormis celle, très maigre, que peut lui apporter son père, ce qui ne lui permet pas d'achever sa seconde année d'études. On lui doit le moteur électrique asynchrone, l'alternateur polyphasé, le montage triphasé en étoile, la commutatrice. Tesla découvre le principe de la réflexion des ondes sur les objets en 1900, il étudie et publie, malgré des problèmes financiers, les bases de ce qui deviendra presque trois décennies plus tard le radar.

\parpic[l][t]{%
  \begin{minipage}{40mm}
    \fbox{\includegraphics[width=110px,height=140px]{img/medaillons/thom.eps}}
  \end{minipage}
}
\textbf{Thom, René} (1923-2002) était un mathématicien français auteur d'importants travaux en topologie différentielle. Né à Montbéliard et décédé à Bures-sur-Yvette, Thom fut élève de l'École Normale Supérieure. En 1958, il a reçu la médaille Fields pour sa théorie du cobordisme (relation d'équivalence entre variétés différentielles compactes). Dans une communication au colloque de Strasbourg (1951), Thom établit que, si les zéros d'un idéal polynomial forment une variété, c'est une variété bordante, et sa thèse, \textit{Espaces fibrés en sphères et carrés de Steenrod} (1951), contient déjà en germe les principales méthodes cobordistes. C'est dans le dernier chapitre d'un mémoire de 1954 (\textit{Quelques Propriétés globales des variétés différentiables}) que la théorie du cobordisme est exposée pour la première fois. Après 1955, Thom a surtout étudié les espaces feuilletés et les ensembles et morphismes stratifiés. On lui doit des résultats sur les approximations des transformations différentiables et leurs singularités, les comparaisons de structures différentiables sur une variété triangulée et une théorie de Morse pour les variétés feuilletées. Il est également l'un des premiers à avoir utilisé les techniques de "chirurgie" des variétés. Depuis 1969, Thom s'est consacré aux applications de la topologie aux phénomènes de la vie. Pour décrire la naissance et l'évolution des formes, il a élaboré une mathématique spécifique: sa théorie des catastrophes est une théorie des singularités de certaines équations différentielles. Concrètement, elle permet, à partir de phénomènes observés, de remonter à leurs causes inconnues, au moins partiellement. Thom a donné un exposé de ses travaux dans l'ouvrage Stabilité structurelle et morphogenèse (1973).

\parpic[l][t]{%
  \begin{minipage}{40mm}
    \fbox{\includegraphics[width=110px,height=140px]{img/medaillons/thales.eps}}
  \end{minipage}
}
\textbf{Thalès de Milet} ($\sim$624 av. J.-C. - $\sim$524 av. J.-C.) était le premier mathématicien dont l'histoire ait retenu le nom. Il est né à Milet, en Asie mineure, sur les côtes méditerranéennes de l'actuelle Turquie. Plus qu'un simple mathématicien, Thalès était un savant universel, curieux de tout, astronome et philosophe, très observateur. On ne démontrait pas ce qu'on avançait à l'époque de Thalès, on ne faisait que remarquer certaines propriétés. Mais la façon qu'avait Thalès de réfléchir, d'analyser des situations, d'en rechercher les causes font de lui le précurseur des scientifiques (il s'en tenait à l'observation et à l'expérimentation). Une de ses grandes interrogations était l'eau, et les causes de la pluie. Il avait remarqué que l'air se transformait en pluie, et il en cherchait désespérément les réponses. Thalès a formulé plusieurs propriétés géométriques qu'il tenait peut-être des Égyptiens et dont les premières traces de démonstration connues sont bien ultérieures mais, ce faisant, il pose les premiers jalons du raisonnement sur des figures géométriques idéales grâce auquel il obtint plusieurs résultats connus sous le nom de "théorèmes de Thalès". Mais le fait d'armes de Thalès est sans conteste la prévision d'une éclipse du Soleil, probablement celle du 8 mai 585 avant notre ère. On lui doit notamment la première connaissance de l'électricité, grâce à deux expériences. Il remarqua d'abord que l'ambre avait la propriété d'attirer les matériaux légers. Une autre expérience réalisée en Magnésie..., vers -600, lui permet de mettre en évidence les propriétés d'aimantation de l'oxyde de Fer.

\parpic[l][t]{%
  \begin{minipage}{40mm}
    \fbox{\includegraphics[width=110px,height=140px]{img/medaillons/turing.eps}}
  \end{minipage}
}
\textbf{Turing, Alan} (1912-1954) Par ses travaux théoriques dans les domaines de la logique et des probabilités, Turing est considéré, sinon comme le fondateur des ordinateurs, en tout cas, comme l'un des pères spirituels de l'intelligence artificielle. Né à Paddington (Londres) Turing connaît une scolarité sans éclat malgré un esprit brillant et de nettes dispositions pour les sciences. En 1928, à la Sherborne School où il est entré deux ans plus tôt, il fait une rencontre qui provoque en lui un déclic et l'amène à s'intéresser réellement à la science et plus exactement aux mathématiques. De 1931 à 1934, Alan Turing est étudiant en mathématiques au King's College de l'Université de Cambridge. Au cours de cette période, il prend connaissance des travaux de John von Neumann sur la mécanique quantique. Stimulé par ces recherches, il se lance dans l'étude de problèmes de probabilités et de logique. C'est aussi au King's College qu'il rencontre des théoriciens de l'économie comme John Keynes. Diplôme en poche, il apprend à l'été 1936 les avancées de Max Newman concernant l'élaboration d'une théorie mathématique sur l'incomplétude de Gödel et la question de la décidabilité de Hilbert. Si pour beaucoup de propositions, il est facile de trouver un algorithme, qu'en est-il de celles pour lesquelles l'algorithme, pas assez rigoureux, est insuffisant à valider la proposition? Doit-on en déduire qu'elles ne peuvent être validée? C'est désormais dans ce sens que vont s'orienter les recherches de Turing. En 1936 il reçoit le prix Smith pour ses travaux sur les probabilités et le concept de la "Machine de Turing". Ce concept constitue la base de toutes les théories sur les automates et plus généralement celle de la théorie de la calculabilité. Il s'agit en fait de formaliser le principe d'algorithme, représenté par une succession d'instructions – agissant en séquence sur des données d'entrée – susceptible de fournir un résultat. Cette formalisation oblige Turing à développer la notion de calculabilité et à déterminer des classes de problèmes "décidables". Cela le conduit à introduire une nouvelle classe de fonctions: les "fonctions calculables au sens de Turing". Au cours de son doctorat à l'Université de Princeton, de 1936 à 1938, Turing conçoit l'idée de la construction d'un ordinateur. De retour à Cambridge, il poursuit ses études mathématiques et s'intéresse à la fonction zêta de Riemann. La seconde Guerre Mondiale lui offre bientôt l'opportunité de mettre en pratique ses théories. C'est au département des communications du Ministère des affaires étrangères britannique qu'il se retrouve confronté au secret d'Enigma, nom de code de la machine utilisée par la marine allemande pour communiquer avec leurs sous-marins. Le cryptage utilisé par les Nazis échappait toujours aux modes d'investigation classiques. Mais avec la collaboration de W. G. Welchman, Turing réussit à percer le code en appliquant sa nouvelle méthode et, de façon indirecte, contribue ainsi à la victoire de la bataille de l'Atlantique. La guerre achevée, Turing intègre le National Physical Laboratory de Grande-Bretagne où il entreprend, en concurrence avec les projets américains, de créer le premier ordinateur. Les avancées technologiques lui laissent entrevoir la réalisation de cet objectif dans un avenir proche. En 1948, grâce à Newman, il obtient un poste de chargé de cours en mathématiques à l'Université de Manchester qu'il occupera jusqu'à la fin de sa vie. Deux ans plus tard, il participe avec Frederic Williams et Tom Kilburn à la réalisation d'un calculateur électronique, le Mark I, et conçoit à cette occasion un manuel de programmation. Dans la foulée, il publie \textit{Can a machine think ?} dans lequel il fait la synthèse des bases mathématiques et conceptuelles de l'ordinateur électronique programmable et résume sa philosophie de la "machine intelligente". Il énonce également le célèbre "Test de Turing" qui se résume à une expérience dans laquelle un homme tient une conversation avec une machine. Comment dans ce cas, un observateur, par l'unique analyse des messages échangés, pourra-t-il distinguer l'homme de la machine? Turing était convaincu que tout n'était qu'un problème d'information et que le développement des technologies permettrait d'ici 50 ans aux machines de tenir en échec l'être humain au moins cinq minutes. Turing se suicida par empoisonnement au cyanure à cause des pressions homophobes qu'il subissait au Royaume-Uni.

\phantomsection
\addcontentsline{toc}{section}{V}
\label{sec:V}

\parpic[l][t]{%
  \begin{minipage}{40mm}
    \fbox{\includegraphics[width=110px,height=140px]{img/medaillons/vanderwaals.eps}}
  \end{minipage}
}
\textbf{Van Der Waals, Johannes Diderik} (1837-1923) était un physicien néerlandais né à Leyde et décédé à Amsterdam. Van Der Waals fut tout d'abord instituteur dès l'âge de 20 ans avant de devenir, à la suite d'efforts solitaires, professeur dans l'enseignement moyen (1863). Il fréquenta les cours de l'Université de Leyde de 1862 à 1865 et enseigna la physique à Deventer et à La Haye (1866). En 1873, il fut reçu docteur par l'Université de Leyde, après la défense d'une dissertation intitulée: \textit{Over de continuiteit van den gas en vloeistoftoestand} qui contient la présentation de l'équation d'état qui porte son nom et conduit à des résultats beaucoup plus satisfaisants que l'équation classique des gaz parfaits au voisinage de la zone de liquéfaction. Cette étude contribua d'une façon décisive à accréditer l'idée de l'existence de forces intermoléculaires d'attraction et à déterminer le rôle du volume d'encombrement moléculaire dans le comportement des gaz à haute pression, deux concepts encore mal assurés à l'époque. Le succès rapide de la nouvelle théorie est illustré par les multiples traductions de la dissertation originale qui suivirent sa présentation. On sait à présent que l'équation de Van der Waals est encore imparfaite et qu'il serait téméraire de vouloir lui conserver le nom "d'équation des gaz réels" qui lui fut naguère attribué. En effet, des équations d'état encore mieux appropriées permettent d'atteindre aujourd'hui une approximation plus complète qui sont en général déduites de considérations de cinétique moléculaire fondées sur le théorème du viriel des forces. De 1877 à 1907, date de sa retraite, Van der Waals occupa la chaire de physique à l'Université d'Amsterdam. C'est pendant cette période qu'il fit connaître sa loi dite "loi des états correspondants" (1880). Cette équation d'état unique pour tous les corps purs contribua largement, elle aussi, à sa renommée, car elle servit par la suite de guide aux essais préalables à la liquéfaction de l'hydrogène et de l'hélium. D'un autre point de vue, cette contribution de Van der Waals est également considérée comme l'une des premières tentatives pour exprimer des lois de la physique en fonction de variables réduites. Parmi les autres travaux de Van der Waals, citons une contribution à la théorie moléculaire des mélanges binaires et l'étude de la capillarité. Le prix Nobel de physique lui a été décerné en 1910 pour ses travaux concernant l'équation de l'état d'agrégation des gaz et des liquides.

\parpic[l][t]{
  \begin{minipage}{40mm}
    \fbox{\includegraphics[width=110px,height=140px]{img/medaillons/viete.eps}}
  \end{minipage}
}
\textbf{Viète, François} (1540-1603) Né à Fontenay-le-Comte et décédé à Paris. Viète est célèbre aujourd'hui en tant qu'inventeur de l'algèbre moderne. Or, à son époque, il était plus connu comme maître des requêtes et conseiller privé d'Henri IV que comme mathématicien. Toute sa vie est en effet marquée par cette dualité d'une carrière politique brillante et d'un ardent travail de cabinet sur les plus hauts problèmes posés par la mathématique de son siècle. Son oeuvre scientifique a beaucoup souffert de ses nombreuses occupations politiques et du peu de temps qu'elles lui laissaient. Il reste néanmoins que la contribution de Viète au développement des mathématiques à la fin du 16ème siècle est fort importante. Elle se caractérise par l'introduction systématique de la représentation littérale dans les problèmes algébriques, tant pour les inconnues que pour les quantités connues, ce qui présente le principal avantage de traiter le cas général et non les cas particuliers et de s'intéresser à la structure des problèmes plutôt qu'à leur expression. Dans sa jeunesse Viète est l'élève des franciscains, au collège des Cordeliers. Il poursuit ses études de droit à la faculté de Poitiers et entra dans la vie active comme avocat. Il est nommé conseiller au parlement de Bretagne en 1573, il y séjourne en fait assez peu, occupé qu'il est par ses travaux mathématiques et les missions confidentielles que lui confie le roi. On retrouve ensuite sa trace à Paris en 1579 où il publie le Canon mathematicus, accompagné du Liber singularis. Nommé maître des requêtes de l'hôtel du roi en 1580, il est démis de sa fonction en 1585, à la suite de conflits de personnes. En 1589, il est à Tours et prépare la publication de son oeuvre scientifique. Il s'occupe également de cryptographie statistique pour le compte du roi. Il regagne Paris avec ce dernier et est nommé conseiller privé. Viète décédera après une assez longue période de déclin du à la maladie.


\phantomsection
\addcontentsline{toc}{section}{W}
\label{sec:W}

\parpic[l][t]{
  \begin{minipage}{40mm}
    \fbox{\includegraphics[width=110px,height=140px]{img/medaillons/walras.eps}}
  \end{minipage}
}
\textbf{Walras, Leon} (1834-1910) était un économiste français né à Évreux (France) et décédé à Clarens (Suisse). Il est le fils d'Auguste Walras, un économiste français dont la pensée influencera beaucoup celle de son fils, dans le domaine de la réforme sociale en général et foncière en particulier. Il étudie au collège de Caen en 1844, puis au lycée de Douai en 1850. Il est diplômé bachelier-ès-lettres en 1851 et bachelier-ès-sciences en 1853. La même année, il n'est pas déclaré admissible à l'École polytechnique et ce aussi lors d'un second essai. En 1854, il est reçu élève externe à l'École des Mines de Paris, mais il n'a pas d’intérêt pour la formation d'ingénieur et il abandonne cette école. Nommé professeur à l'université de Lausanne, Walras dénonça à partir des années 1870, les théories économiques libérales alors enseignées dans les universités, qu'il jugeait incapables de rendre compte des problèmes économiques de son temps. Dans ses Éléments d'économie politique pure (1874), sa critique vise en particulier les théories de la valeur travail et de la rente foncière mais à travers lui c'est tout l'héritage classique qu'il remet en cause (notamment celui d'Adam Smith). Influencé par le mathématicien Antoine Cournot, il est l'un des premiers à introduire de manière systématique le calcul mathématique en économie. Walras place l'entreprise au coeur de l'économie et s'intéresse à son action dans le cadre d'une concurrence entre agents, ainsi que dans celui d'une interdépendance de tous les marchés économiques: les marchés des produits (biens et services) et ceux des facteurs de production (notamment la terre, le travail et les capitaux). Il se demande comment se fixent les prix et les quantités de façon simultanée, et pose le problème de l'équilibre général, c'est-à-dire de la stabilité des équilibres sur tous les marchés. L'attention portée à cette question caractérise les membres de l'École de Lausanne, en particulier le successeur de Walras, Vilfredo Pareto. Avec l'Autrichien Carl Menger et le Britannique Stanley Jevons, qu'il ne connaissait pas au moment où il s'engageait sur cette voie, il est considéré comme l'un des fondateurs du courant néoclassique et du marginalisme.

\parpic[l][t]{
  \begin{minipage}{40mm}
    \fbox{\includegraphics[width=110px,height=140px]{img/medaillons/weber.eps}}
  \end{minipage}
}
\textbf{Weber, Wilhelm} (1804-1891) était un physicien allemand né à Wittenber et décédé à Göttingen qui se spécialisa en électrodynamique. Weber écrivit, en 1824, un traité sur le mouvement ondulatoire avec son frère aîné, Ernst Heinrich Weber, anatomiste réputé, et étudia, avec son frère cadet Eduard Friedrich Weber le mécanisme de la marche (1836). À Göttingen, il collabora avec Carl Friedrich Gauss pour l'étude du géomagnétisme, et il relia leurs laboratoires par un télégraphe électrique: ce fut l'une des premières transmissions par télégraphe que l'on connaisse. Sa réalisation majeure fut celle qu'il mena à Leipzig, avec F.W.G. Kohlrausch: il détermina le rapport des unités de charge électrostatiques et électrodynamiques (la constante de Weber) qui se révéla être l'équivalent d'une vitesse, et fut utilisé plus tard par James Clerk Maxwell pour renforcer sa théorie sur l'électromagnétisme.

\parpic[l][t]{
  \begin{minipage}{40mm}
    \fbox{\includegraphics[width=110px,height=140px]{img/medaillons/weierstrass.eps}}
  \end{minipage}
}
\textbf{Weierstrass, Karl Theodor Wilhelm} (1815-1897) était un mathématicien allemand, qui donna à la théorie des fonctions sa forme moderne en précisant en particulier le formalisme des limites et est considéré à ce titre comme le père de l'analyse moderne. Né à Ostenfelde, il fit ses études à Bonn et à Münster où il fut instituteur. C'est là qu'il s'intéressa aux mathématiques, et plus particulièrement à l'étude des fonctions elliptiques. Pendant de nombreuses années, Weierstrass travailla dans l'ombre pour établir sa théorie des fonctions de variable complexe, qui repose sur les développements en série entière. En 1854, il publia un mémoire sur les intégrales abéliennes et sur l'inversion des intégrales hyperelliptiques, qui établit sa réputation comme mathématicien et lui valut un doctorat honoraire de l'Université de Königsberg. Nommé professeur à l'Université de Berlin, il enseigna de 1864 à sa mort. Il a peu publié de son vivant et sa réputation est venue principalement de l'influence de ses cours à Berlin. Ceux-ci furent suivis par de nombreux mathématiciens et établirent la théorie des fonctions sur des bases de rigueur auxquelles son nom reste attaché, la "rigueur weierstrassienne". Il est aussi connu pour avoir rendu public un exemple de fonction continue nulle part dérivable (fonction de Weierstrass).

\parpic[l][t]{
  \begin{minipage}{40mm}
    \fbox{\includegraphics[width=110px,height=140px]{img/medaillons/weyl.eps}}
  \end{minipage}
}
\textbf{Weyl, Hermann} (1885-1955)  est un des mathématiciens les plus influents du 20ème siècle, l'un des premiers à combiner la Relativité Générale avec les lois de l'électromagnétisme. Ses recherches en mathématiques portèrent essentiellement sur la topologie et la géométrie. Il effectua des recherches en mécanique quantique et en théorie des nombres. Né à Elmshorn à proximité de Hambourg en Allemagne, Weyl étudia de 1904 à 1908 à Göttingen et à Münich, principalement intéressé par la mathématique et la physique. Son doctorat fut soutenu à Göttingen sous la direction de Hilbert et Minkowski. En 1910, il obtint un poste d'enseignant comme lecteur privé à Göttingen. Il enseigna la mathématique à l'École Polytechnique Fédérale de Zürich en Suisse en 1913. C'est à Princeton qu'il travailla avec Einstein. Weyl rechercha une unification de la gravitation et de l'électromagnétisme. Cette recherche donna des explications de la violation de la non-conservation de la parité, une caractéristique des interactions faibles. Weyl continua à travailler à l'IAS (Institude of Advanced Studies) jusqu'à sa retraite en 1952 ; il mourut à Zürich. En 1918, il introduit la notion de jauge, première étape de ce qui deviendra la théorie de jauge. En réalité, sa vision était une tentative non réussie de modéliser les champs électromagnétiques et gravitationnels comme des propriétés géométriques de l'espace-temps. Ces travaux se révélèrent fondamentaux pour comprendre la symétrie des lois de la mécanique quantique. Il en posa les bases, donnant naissance aux spineurs, devenus relativement familiers autour des années 1930.

\parpic[l][t]{
  \begin{minipage}{40mm}
    \fbox{\includegraphics[width=110px,height=140px]{img/medaillons/weinberg.eps}}
  \end{minipage}
}
\textbf{Weinberg, Steven} (1933-) Né à New York il débuta ses études à New York même puis à l'Université Cornell (dans l'État de New York) et soutint, en 1957 à Princeton, sa thèse sur les effets de l'interaction forte dans les processus dominés par l'interaction faible. Chercheur à l'Université de Californie à Berkeley de 1959 à 1966, il s'intéressa à de multiples problèmes en théorie quantique des champs, en physique des particules et en astrophysique. Professeur à Harvard à partir de 1973, il contribua de façon décisive à la compréhension moderne des interactions fondamentales. Il rejoignit l'Université du Texas à Austin en 1982. L'unification des forces fondamentales a sous-tendu les efforts des physiciens modernes depuis Newton, Maxwell et Einstein qui, après avoir uni l'espace et le temps, tenta, mais en vain, d'englober en une seule théorie gravitation et électromagnétisme. La découverte, au début du 20ème siècle, des deux forces nucléaires, les interactions faible et forte, donna un nouvel élan à ces tentatives. En 1967, Weinberg et le physicien pakistanais Abdus Salam proposèrent, indépendamment, que l'électromagnétisme et l'interaction nucléaire faible soient issus d'une même interaction électrofaible, dont la symétrie de jauge est spontanément brisée et dont le vecteur est un triplet de bosons massifs et le photon. Quelques années plus tard des expériences au CERN de Genève apportaient les premières confirmations du modèle de Weinberg-Salam. Le prix Nobel de physique 1979 (partagé avec l'Américain Sheldon Lee Glashow, pour l'importance de ses travaux de précurseur) récompensa les deux auteurs de ce qu'on appelle maintenant le "modèle standard" des interactions électrofaibles. Pédagogue, Weinberg est l'auteur de plusieurs cours de physique de haut niveau, tant sur la gravitation que sur la théorie des champs. Vulgarisateur de talent, son livre\textit{ Les Trois Premières Minutes de l'Univers} fut un succès mondial.

\parpic[l][t]{
  \begin{minipage}{40mm}
    \fbox{\includegraphics[width=110px,height=140px]{img/medaillons/wilcoxon.eps}}
  \end{minipage}
}
\textbf{Wilcoxon, Frank} (1892-1965) était un chimiste et statisticien connu pour le développement de tests statistiques très répandus! Frank Wilcoxon est né de parents américains à County Cork en Irelande. Il a grandi a Catskill, New York mais a suivi une partie de sa scolarité en Angleterre. En 1917, il est diplômé du collège militaire de Pennsilvanie avec une licence. Après la première guerre mondiale, il débute des cours de maîtrise à l'Université de Rugters où il obtint sa maîtrise en chimie en 1922 et ensuite à l'Université de Cornell où il obtint son doctorat en chimie physique en 1924. Wilcoxon commença sa carrière de chercheur à l'Institut Boyce Thompson en 1925 et y resta jusqu'en 1941. Ensuite, il prit un poste dans la compagnie Atlas Powder où mis en place et dirigea le laboratoire de contrôle avant de joindre la compagnie chimique American Cyanimid en 1943. Pendant cette période il développa un intérêt pour la statistique inférentielle à travers les lectures des textes de R.A. Fisher de 1925. Il prit sa retraite en 1957. Pendant sa carrière, Wilcoxon publia 70 articles, le plus connu étant celui contenant les deux tests statistiques qui portent ce nom: le test de la somme des rangs de Wilcoxon et le test de la somme des rangs signés de Wilcoxon. Il s'agit d'alternatives non paramétriques aux tests-$T$ de Student. Wilcoxon mourut après une brève maladie.

\parpic[l][t]{
  \begin{minipage}{40mm}
    \fbox{\includegraphics[width=110px,height=140px]{img/medaillons/witten.eps}}
  \end{minipage}
}
\textbf{Witten, Edward} (1951-) est un mathématicien et physicien, lauréat de la médaille Fields en 1990. Né à Baltimore (Maryland), Witten fait ses études supérieures à l'Université Brandeis à Waltham (Massachusetts), puis à l'Université de Princeton (New Jersey), où il soutient sa thèse de doctorat en physique en 1974. Chercheur à l'Université Harvard de 1976 à 1980, il enseigne ensuite à l'Université de Princeton, puis devient membre de l'Institute for Advanced Study de Princeton en 1987. Après des travaux en physique théorique des particules élémentaires, Witten axe ses recherches sur la physique-mathématique et contribue en particulier de façon déterminante au développement des théories des supercordes dans l'espoir que celles-ci pourraient émerger vers une compréhension de l'interaction gravitationnelle au niveau quantique. En mathématiques, il a contribué à l'étude de la théorie de Morse, démontrant les inégalités classiques de Morse en reliant les points critiques à l'homologie. En 1987, il démontre une suite infinie de théorèmes de rigidité sur l'espace des solutions d'équations différentielles, telles que l'équation de Rarita-Schwinger, rencontrées en physique. En théorie des noeuds, il a montré en 1989 qu'on peut interpréter les invariants de noeuds de Vaughan Jones comme des intégrales de Feynman pour une théorie de jauge tridimensionnelle. Il a, de plus, exploré les relations entre la théorie quantique des champs et la topologie différentielle des variétés bi- ou tridimensionnelles. Les progrès récents dans la compréhension des modèles bidimensionnels de la gravitation sont largement dus à l'influence des idées originales de Witten.

\phantomsection
\addcontentsline{toc}{section}{Y}
\label{sec:Y}

\parpic[l][t]{
  \begin{minipage}{40mm}
    \fbox{\includegraphics[width=110px,height=140px]{img/medaillons/yang.eps}}
  \end{minipage}
}
\textbf{Yang, Chen-Ning} (1922-) Professeur à l'Université Chinoise de Hong Kong et à l'Université de Tsinghua à Pékin, professeur émérite de l'Université de New York à Stony Brook, Yang est l'un des plus grands physiciens théoriciens de la seconde moitié du 20ème siècle. Il obtient son Master of Science à l'Université de Tsinghua en 1944. Il s'inscrit en 1946 à l'Université de Chicago que Fermi venait de rejoindre. Plus tard, il décide de se consacrer à la physique théorique et, en 1949, il soutient sa thèse avec un travail sur la phénoménologie des réactions nucléaires. Sa carrière débute à l'Institute for Advanced Studies à Princeton en 1949. En 1965, il refuse de succéder à Oppenheimer comme directeur, mais il décide en 1966 de sortir de sa tour d'ivoire et finit par accepter la chaire Einstein et le poste de directeur de l'Institut de Physique Théorique de la toute nouvelle Université de New York à Stony Brook. À partir de 1971 il s'engage très activement dans le rétablissement des relations scientifiques entre la Chine et les États-Unis et s'implique dans la création de nouveaux instituts de recherche, en particulier à Nankin. Les contributions de Yang se caractérisent par leur profondeur, par l'ampleur et la variété de leur spectre, de la phénoménologie des particules à la théorie quantique des champs, en passant par la mécanique statistique ainsi que par différentes incursions en physique de la matière condensée. Ses travaux sur la brisure de la symétrie par réflexion d'espace (ou violation de la parité) dans les interactions faibles constituent un exemple parfait d'analyse phénoménologique d'une expérience en contradiction avec les idées reçues, à savoir l'absence d'une orientation privilégiée de l'espace dans les lois de la physique. Son grand mérite porte sur deux points: d'une part, il met en évidence le fait que l'hypothèse en question n'avait pas été testée pour les interactions faibles et, d'autre part, il a imaginé tout un ensemble de tests nouveaux pour l'invariance par réflexion d'espace. Ce bond en avant de la théorie des interactions faibles a permis d'aboutir, avec l'introduction des champs de Yang-Mills, au modèle standard électrofaible. L'idée de Yang fut de généraliser l'invariance de jauge aux groupes des rotations dans un espace abstrait à 3 dimensions censé décrire les degrés de liberté interne des champs de matière. Les champs de Yang-Mills s'imposèrent comme outil fondamental pour la construction d'une théorie prédictive de l'ensemble des interactions faibles, fortes et électromagnétiques, événement décisif qui engagea la révolution de la physique des années 1970. L'ensemble de ses travaux ont eu un impact considérable en physique théorique. Près de 20 ans après la publication de son article avec Mills, Yang a donné une reformulation précise de la théorie des champs de Yang-Mills dans le cadre rigoureux des espaces fibrés. L'analogie avec la théorie de la gravitation devient ainsi apparente et les notions de courbure et de transport parallèle s'introduisent naturellement. Des solutions particulières des équations de Yang-Mills, comme celle découverte par Gerard't Hooft, sont utilisées par les mathématiciens pour explorer les propriétés des variétés différentielles à quatre dimensions. Yang a reçu de nombreux prix scientifiques, dont le prix Nobel de physique en 1957 qu'il a partagé avec Tsung-Dao Lee. Ce prix prestigieux leur a été accordé pour leurs travaux sur les lois de la parité dans le domaine des particules élémentaires. Ces travaux fondamentaux sont particulièrement importants parce qu'ils ont montré que la symétrie droite-gauche des particules élémentaires, universellement admise à l'époque, était tout simplement incorrecte, ce qui fut ensuite prouvé expérimentalement.

\parpic[l][t]{
  \begin{minipage}{40mm}
    \fbox{\includegraphics[width=110px,height=140px]{img/medaillons/yukawa.eps}}
  \end{minipage}
}
\textbf{Yukawa Hideki} (1907-1981) était un physicien japonais, né et décédé à Tokyo, il était le cinquième de sept enfants qui devinrent tous des universitaires renommés. Il fut très vite porté vers la mathématique et la philosophie. Admis au département de physique de l'Université de Kyoto en 1926, grand lecteur, Yukawa se passionna vite pour les nouvelles conceptions philosophiques accompagnant la relativité et la théorie des quanta, conceptions qu'il avait découvertes en particulier dans les ouvrages de Max Planck. En marge de ses études, il eut connaissance des développements contemporains de la physique quantique qui aboutirent à sa formulation bien établie vers la fin des années 1920. Il obtint son diplôme à l'Université de Kyoto en 1929 et commença, dès lors, des recherches personnelles dans la double direction de la physique quantique relativiste et de la physique nucléaire qui se dessinait alors. Il s'attacha tout d'abord au problème de la liaison nucléaire électron-proton, le neutron étant une particule encore inconnue, puis à la théorie quantique des champs. Tout en enseignant la physique quantique, Yukawa poursuivait ses recherches sur les problèmes de la physique des noyaux. En 1934, il s'attaqua au problème de la force nucléaire, que la théorie de Fermi était impuissante à résoudre. Il reprit une idée qu'il avait déjà considérée lors de ses premiers travaux, celle d'une force d'échange, transmise entre le neutron et le proton par une particule nouvelle associée à un champ nouveau, dont il se proposait de déduire les propriétés à partir de la force nucléaire. C'est en octobre 1934 qu'il découvrit la solution, en obtenant une relation entre la masse de cette particule d'échange hypothétique et la portée de l'action des forces nucléaires. La particule de Yukawa, le méson, devait avoir une masse valant $200$ fois celle de l'électron. Il fallait supposer que ces mésons étaient de spin entier ou nul, qu'ils obéissaient à la statistique de Bose-Einstein et qu'ils étaient pourvus de charges positive et négative. Ce travail n'attira pas l'attention jusqu'au jour où d'autres chercheurs annoncèrent la découverte d'une particule nouvelle dans le rayonnement cosmique, ayant la masse prévue par Yukawa. Il apparut toutefois que l'interaction de ce méson avec la matière était trop peu intense pour qu'il puisse être la particule d'échange des forces nucléaires. La théorie des deux mésons pallia la difficulté. Il avait découvert entre-temps le mécanisme de désintégration du noyau par capture d'un électron orbital, en appliquant la théorie de Fermi. Il fut le premier Japonais à recevoir le prix Nobel de Physique, en 1949, pour sa théorie mésique des forces nucléaires. Yukawa fonda l'Institut de Recherches de Physique Fondamentale de l'Université de Kyoto et le dirigea jusqu'à sa retraite, en 1970. Il ne se cantonna pas dans une activité de physicien: il écrivit des essais sur la créativité scientifique et milita en faveur de la paix, signant l'appel d'Albert Einstein et de Bertrand Russell contre l'utilisation des armes atomiques.

\parpic[l][t]{
  \begin{minipage}{40mm}
    \fbox{\includegraphics[width=110px,height=140px]{img/medaillons/young.eps}}
  \end{minipage}
}
\textbf{Young, Thomas} (1773-1829) était un physicien, médecin et égyptologue britannique né à Milverton et décédé à Londres, surtout connu pour ses découvertes en optique (phénomènes d'interférence), en élasticité des matériaux et en médecine (explication de la vision colorée).À l'âge de 14 ans il se débrouille déjà dans plus d'une dizaine de langues étrangères. Young commence à étudier la médecine en 1792 à Londres, part en 1794 pour Édimbourg, puis un an plus tard pour Göttingen, où il obtient le titre de docteur en physique en 1796. En 1799, il commença à pratiquer la médecine à Londres. À partir de 1802, et jusqu'à sa mort, il occupa le poste de secrétaire de la Royal Society. En 1811, Young fut nommé à l'hôpital Saint-George de Londres. Il fit partie de plusieurs commissions scientifiques officielles et, à partir de 1818, il fut nommé secrétaire du Bureau des longitudes et éditeur de l'Almanach nautique. En optique, Young découvrit le phénomène des interférences, et contribua ainsi à établir le caractère ondulatoire de la lumière. Il fut le premier à décrire et à mesurer l'astigmatisme et à trouver une explication physiologique à la sensation de couleur. Young est également connu pour ses travaux sur les théories de la capillarité et de l'élasticité. Il contribua également au déchiffrage des hiéroglyphes inscrits sur la pierre de Rosette. Ses écrits comportent d'importants travaux en médecine, en égyptologie et en physique.

\phantomsection
\addcontentsline{toc}{section}{Z}
\label{sec:Z}

\parpic[l][t]{
  \begin{minipage}{40mm}
    \fbox{\includegraphics[width=110px,height=140px]{img/medaillons/zeeman.eps}}
  \end{minipage}
}
\textbf{Zeeman, Pieter} (1865-1943) était un physicien né Zonnemaire et décédé à Amsterdam. Il commença à s'intéresser très jeune à la science. En 1883 lors d'aurores boréales visibles aux Pays-Bas, Zeeman, alors étudiant au collège, fit une description et un dessin détaillé du phénomène qui fut publié dans la revue \textit{Nature}. Après avoir passé ses examens d'entrée en 1885, il étudia la physique à l'Université de Leiden sous la direction de Hendrik Lorentz. En 1890, avant même de terminer sa thèse, il devint l'assistant de Lorentz. Celà lui permit de participer à un programme de recherche sur l'effet Kerr. En 1893 il soumit sa thèse sur l'effet Kerr, la réflexion de la lumière polarisée sur une surface magnétisée. Après avoir obtenu son doctorat il partit pour un semestre à l'institut F. Kohlrausch à Strasbourg. En 1895, après son retour de Strasbourg, Zeeman devint Privatdozent en mathématiques et physique à Leiden. En 1896, trois ans après avoir soumis sa thèse sur l'effet Kerr, il désobéit aux ordres directs de ses supérieurs et utilisa l'équipement du laboratoire pour mesure la séparation des lignes du spectres sous un champ magnétique intense. Il a été licencié pour ses efforts... mais il fut récompensé plus tard: il obtint le prix Nobel de Physique en 1902 pour sa découvert ce qui est connu aujourd'hui sous le nom d'effet Zeeman. En plus de son travail de thèse, dans l'étude de l'effet d'un champ magnétique sur une source de lumière. Grâce à sa découverte, Zeeman si vit offrir un poste d'assistant professeur à Amsterdam en 1897. En 1900 s'ensuivit la place de professeur à l'Université d'Amsterdam. En 1902 avec son mentor Lorentz il se vit attribuer le Prix Nobel de physique pour l'effet Zeeman. Cinq années plus tard, en 1908, il succéda à Van der Waals comme professeur à temps plein et directeur à l'Institut de Physique à Asterdam. Il se retira encore en tant que professeur en 1935.

	\chapter{Chronologie}
	En arrivant à la trois-millième page A4 de l'écriture de ce livre et à l'occasion de la 3ème édition, il nous a semblé approprié d'essayer de donner une chronologie approximative de la plupart des sujets cités dans ce livre \footnote{Même si pour certaines dates il n'est pas toujours possible de vérifier s'il s'agit de légendes ou de faits réels...}. Cela peut donner une meilleure perspective sur les outils utilisés et aussi pour rendre hommage à nos illustres prédécesseurs à qui nous devons notre qualité de vie, notre longévité, notre maîtrise de l'environnement (pas nécessairement son respect...) et de sa compréhension.

Si des dates importantes manquent (mais seulement sur des sujets proches de ceux présentés dans les différentes sections de ce livre!), ou que vous identifiez des erreurs, n'hésitez pas à nous le faire savoir, c'est une première ébauche, et donc la chronologie ne peut qu'être améliorée.

Pour plus d'informations, le lecteur peut se référer à \href{http://www.wikipedia.com}{{\color{blue} Wikipédia}}, qui a atteint le plus haut niveau dans le nombre de dates historiques disponibles à ce jour sur Internet (du moins à notre connaissance) et ce avec une qualité respectable (vérification des sources!).

\begin{center}
\textit{C'est l'histoire de la façon dont l'histoire a fait la science et comment la science est entrée dans l'histoire, et comment les idées qui ont émergé ont fait le Monde moderne.}
\end{center}

\includegraphics[width=\textwidth]{img/raphael_school_of_athens.jpg}

\textbf{+2016}\footnote{Les années sont données sur la base du calendrier Grégorien.}\\
La première observation d'ondes gravitationnelles a été faite le 14 septembre 2015 et a été annoncée par les collaborations LIGO et Virgo le 11 février 2016.

\textbf{+2013}\\
Le 14 mars 2013, le CERN a confirmé que CMS et ATLAS ont comparé un certain nombre d'options pour la parité de spin d'une particule, et que cette dernière privilégiaient toutes l'absence de spin et même de parité. Ceci, couplé avec les interactions mesurées de la nouvelle particule avec d'autres particules, indique fortement qu'il s'agit d'un boson de Higgs. Cela fait également de cette particule la première particule scalaire élémentaire à être découverte dans la nature. En juillet 2017, le CERN a confirmé que toutes les mesures étaient toujours en accord avec les prédictions du modèle standard.

\textbf{+2006}\\
Le psychologue cognitif et informaticien Geoffrey Everest Hinton publie \textit{Un algorithme d'apprentissage rapide pour les réseaux d'apprentissages profondes} qui relance l'intérêt pour l'apprentissage en profondeur (deep learning).

\textbf{+2001}\\
Première édition du livre d'Opera Magistris mais sous le nom de "Sciences.ch". Un compendium sur les mathématiques appliquées qui a pour but de compilier un maximum de connaissances modernes sur les STIM (science, technologie, ingénierie et mathématiques) et au-delà au niveau Licence/Maîtrise avec un maximum de détails dans les développements mathématiques.

\textbf{+1995}\\
Michel Mayor et Didier Queloz observent la première planète extrasolaire autour d'une étoile de la séquence principale. La même année, le premier condensat gazeux de Bose-Einstein est produit par les physiciens Eric Cornell et Carl Wieman de l'Université du Colorado au laboratoire Boulder NIST-JILA dans un gaz d'atomes de rubidium refroidi à $170$ nanokelvins.

\textbf{+1994}\\
Les travaux du mathématicien Andrew Wiles (plus de 10 ans de recherche!) donnent une solution au dernier théorème de Fermat. La même année, le premier algorithme utilisant un ordinateur quantique pour la factorisation de nombre premiers est publié par Peter Shor.

\textbf{+1992}\\
Les machines vectorielles de support (SVM) sont inventées.

\textbf{+1986}\\
David E. Rumelhart, Geoffrey E. Hinton et Ronald J. Williams inventent la rétropropagation, une nouvelle procédure d'apprentissage pour les réseaux neurones. Les ingénieurs Bill Smith et Mikel J. Harry qui travaillent chez Motorola présentent la méthodologie Six Sigma ($ 6\sigma $), un ensemble de techniques et d'outils pour l'amélioration des processus scientifiques.

\textbf{+1983}\\
Les physiciens Carlo Rubbia, Simon van der Meer et l'équipe UA-1 du CERN  découvrent les bosons W et Z qui confirment l'unification des forces nucléaires faibles et électromagnétiques.

\textbf{+1982}\\
L'astrophysicien Werner Becker découvre le premier pulsar milliseconde. La même année, l'équipe du physicien Alain Aspect observe la violation des inégalités de Bell.

\textbf{+1978}\\
Les mathématiciens et cryptologues Ronald Rivest, Adi Shamir et Leonard Adelman proposent une procédure de cryptage à clé publique appelée "RSA" basée sur la difficulté de la factorisation en nombres nombres.

\textbf{+1976}\\
Développement d'une méthode par les physiciens Peter Mansfield et Andrew Maudsley pour rendre possible des scanners à résonance magnétique nucléaire (RMN).

\textbf{+1975}\\
L'informaticien John Holl invente les algorithmes génétiques.

\textbf{+1973}\\
Le mathématicien Fischer Black et l'économiste Myron S. Scholes publient un modèle de valorisation des actifs financiers.

\textbf{+1969}\\
Dans son \textit{Cartes de contrôle pour les mesures avec des tailles d'échantillons variables}, l'ingénieur qualité Irving Wingate Burr introduit les fameuses constantes non biaisées $c_4$ et $d_2$ qui portent son nom.

\textbf{+1968}\\
Les modèles de Markov (MMC) cachés sont inventés.

\textbf{+1967}\\
L'astrophysicienne Jocelyn Bell Burnell et l'astronome Antony Hewish découvrent le premier pulsar.

\textbf{+1965}\\
Détection par les astrophysiciens Arno Penzias et Robert Wilson fond diffus cosmologique due au rayonnement micro-onde de fond de ciel prédit par la théorie du physicien Robert Dicke. Développement de l'algorithme de Transformée de Fourier Rapide par le mathématicien James W. Cooley et le statisticien John W. Tukey qui a de nombreuses applications en sciences. Le physicien John Stewart Bell découvre les inégalités de Bell.

\textbf{+1964}\\
L'astrophysicien Irwin Shapiro prédit un retard gravitationnel du voyage de rayonnement comme un test de la Relativité Générale. La même année, le physicien John Stewart Bell montre que toutes les théories de variables cachées locales doivent satisfaire l'inégalité de Bell.

\textbf{+1963}\\
Le mathématicien et météorologue Edward Lorenz touve ce qui est probablement le premier attracteur étrange et ouvre la voie à la théorie du chaos.

\textbf{+1962}\\
Le mathématicien Benoît Mandelbrot découvre les fractales par hasard dans l'analyse des signaux situés aux Bell Laboratories aux Etats-Unis d'Amérique où il utilisera sans cesse les ordinateurs pour répéter des motifs graphiques et dont le principe est à la base de la théorie des fractales. La même année, l'économiste William Forsyth Sharpe publie le CAPM (Capital Asset Pricing Model).

\textbf{+1960}\\
Le physicien Abdus Salam postule l'existence des bosons W et Z pour expliquer la désintégration bêta et l'émergence d'un nouveau boson Z, qui n'avait jamais été vu auparavant. La même année, les pysiciens Ali Javan et Theodore Maiman inventen chacun un type particulier de LASER. Les mathématiciens et ingénieurs Irving Reed et Gustave Solomon présentent le code correcteur d'erreur de Reed-Solomon.

\textbf{+1959}\\
Les physiciens Yakir Aharonov et David Bohm prédisent l'effet Aharonov-Bohm (particule tournant autour d'un champ magnétique mais dans une région où le champ magnétique est nul est sensible au champ à travers le potentiel vectoriel) et l'année suivante le physicien Robert G. Chambers confirme l'effet expérimentalement.

\textbf{+1958}\\
Le perceptron est développé à l'Université Cornell par le psychologue  Frank Rosenblatt. C'est le premier programme capable d'apprendre par essais et erreurs.

\textbf{+1957}\\
Les physiciens John Bardeen, Leon Neil Cooper et John Robert Schrieffer proposent et la théorie de la supraconductivité.

\textbf{+1956}\\
Les physiciens Clyde Cowan et Fred Reines observent le neutrino hypothétisé il y a 25 ans par le physicien Wolfgang Pauli. La même année, le linguiste, philosophe, cognitiviste, historien, critique social et activiste politique Avram Noam Chomsky publie trois modèles pour la description du langage où il introduit la classification des grammaires formelles appelée aujourd'hui la "hiérarchie Chomsky", qui contient la classe des grammaires hors contexte, jouant un rôle important en informatique pour créer des langages de programmation. Dans \textit{The Logic Theorist} par l'informaticien cognitif Allen Newell et le politologue, économiste, sociologue, psychologue et informaticien Herbert Simon, le premier programme informatique d'intelligence artificielle est proposé, qui a produit des preuves dans le système \textit{Principia Mathematica} de Bertrand Russell et Alfred North Whitehead. Pour l'un des théorèmes, le programme produit une preuve plus simple que celle présentée dans les \textit{Principia}!

\textbf{+1955}\\
Les physiciens Owen Chamberlain, Emilio Gino Segrè, Clyde Wiegand et Thomas Ypsilantis découvrent l'antiproton.

\textbf{+1954}\\
L'économiste Harry Markowitz publie sa thèse sur le modèle de diversification efficace des portefeuilles d'actifs financiers. La même année, les physiciens John Bell et le duo Wolfgang Pauli et Gerhart Lüders développent la théorie CPT analysant la symétrie des lois physiques pour les transformations impliquant simultanément la charge, la parité et le temps. Le physicien Charles Hard Townes développe le MASER. Le biologiste Milner Baily Schaefer publie son modèle de population d'équilibre.

\textbf{+1953}\\
Les Méthodes de Monte-Carlo par chaînes de Markov (MMCM) sont inventées. L'inférence bayésienne devient traitable sur de vrais problèmes.


\textbf{+1952}\\
Le mathématicien George Bernard Dantzig développe l'algorithme du simplex pour la recherche opérationnelle.

\textbf{+1951}\\
Le théorème CPT apparaît pour la première fois implicitement dans le travail du physicien Julian Schwinger pour prouver la corrélation entre spin et statistiques.

\textbf{+1950}\\
Les physiciens Johannes Hans Daniel Jensen et Maria Goeppert-Mayer développent le modèle de goutte liquide du noyau nucléaire. La même année, l'économiste et mathématicien John Forbes Nash développe le concept de jeux non-coopératifs et généralise la notion de minimax pour les jeux à somme nulle; l'ingénieur David Huffman trouve l'algorithme utilisé pour compresser tout type de symboles de séries. Dans ses \textit{codes de détection d'erreur et de correction d'erreur}, le mathématicien Richard Wesley Hamming présente la famille de codes Hamming, codes linéaires utilisés pour détecter et corriger les erreurs de transmission. Une recherche basée sur le test en double aveugle est publiée pour la première fois, par Greiner et al.

\textbf{+1949}\\
Le mathématicien et ingénieur électricien Claude Shannon publie un article contenant la théorie de l'information qui deviendra le fondement d'un certain nombre de théories physiques, de statistiques et de méthodes numériques. Le physicien Richeard Feynman propose l'interprétation du positron comme un électron remontant dans le temps dans son article \textit{The Theory of Positrons}.

\textbf{+1948}\\
La physicienne Maria Göpper-Meyer développe avec succès un modèle théorique pour la structure du noyau atomique et l'ingénieur textile et statisticien Genichi Taguchi développe les plans d'expérience(DOE) qui portent son nom (plans de Taguchi). Le physicien Richard Feynman présente les diagrammes qui portent son nom ainsi que la formulation intégrale de chemin  en physique quantique. Le physicien Polykarp Kusch mesure le moment magnétique anormal de l'électron (déviation de la prédiction théorique de la théorie de Dirac) conduisant ainsi à reconsidérer et à innover en électrodynamique quantique.

\textbf{+1947}\\
Les physiciens Cecil Powell, Cesare Mansueto Giulio Latte et Giuseppe Occhialini découvrent le pion dans l'étude des rayons cosmiques. La même année, les physiciens John Bardeen, Walter H. Brattain et William Schockley inventent des transistors semi-conducteurs dans les laboratoires de la compagnie de téléphone Bell aux Etats-Unis d'Amériques qui vont provoquer la révolution informatique. Mesure du déplacement de Lamb par Willis Eugene Lamb (décalage du spectre d'énergie non-prédit par la théorie de Dirac mais expliquée dans le cadre de l'électrodynamique quantique).

\textbf{+1946}\\
Le physicien et chimiste Willard Frank Libby développe et découvre la possibilité de datation au carbone 14. La même année, les physiciens Walter Houser Brattain, John Bardeen et William Bradford Shockley découvrent l'effet des transistors.

\textbf{+1945}\\
Le Trinity Test, la première détonation réussie d'une arme nucléaire par le physicien Robert Oppenheimer et son équipe au Nouveau-Mexique.

\textbf{+1944}\\
Le mathématicien John von Neumann développe les bases de la théorie mathématique des jeux.

\textbf{+1943}\\
Le physicien Tomonaga Sin-Itiro publie un article posant la base de de l'électrodynamique quantique.

\textbf{+1942}\\
Le physicien Enrico Fermi et son équipe ont mené la première réaction en chaîne contrôlée dans le but de construire la première bombe atomique.

\textbf{+1941}\\
Le physicien Ernst Stueckelberg interprète les positrons comme des électrons à énergie positive remontant dans le temps. Le physicien Lev Davidovich Landau publie une théorie de la superfluidité.

\textbf{+1940}\\
Le physicien Edward Teller voit la possibilité d'utiliser l'énorme quantité de chaleur générée par l'explosion d'une bombe à fission pour déclencher le processus de fusion nucléaire. C'est l'approche considérée comme la découverte de la fusion nucléaire. La même année, le physicien William Donald Kerst développa le premier bétatron. Le physicien John Wheeler, dans un appel téléphonique à Richard Feynman, émet l'hypothèse que tous les électrons et positons sont en fait des manifestations d'une seule entité qui va et vient dans le temps (le «postulat d'un électron»).

Alan Turing et son équipe développent la première machine électromécanique ("The Bombe") installée à Bletchley Park. Cette machine complexe se composait d'environ 100 tambours rotatifs, 16 kilomètres de fil et environ 1 million de connexions soudées. Par ordinateur à la fin du 20e siècle, nous entendons généralement quelque chose qui peut faire beaucoup de choses. Donc, dans ce sens, La Bombe n'était pas un ordinateur. Car il ne pouvait résoudre qu'un seul problème: casser les clés Enigma. Bletchley Park a construit le premier ordinateur comme nous l'appellerions. Il s'agissait d'une machine appelée "Colossus" et qui a été développée entre 1943-1945 par Tommy Flowers, assisté de Sidney Broadhurst et William Chandler.

\textbf{+1939}\\
Les chimistes Otto Hahn et Fritz Strassmann bombardent l'Uranium avec des neutrons et découvrent que du Baryum est produit par l'expérience (découverte de la fission nucléaire). La même année, les physiciens Lise Meitner et Otto Robert Frisch déterminent que la fission nucléaire s'est produite pendant l'expérience de Hahn-Strassman. Les physiciens Wolfgang Pauli, Markus Fierz et Frederik Jozef Belinfante prouvent que les propriétés de permutation de particules, bosons ou fermions identiques sont contrôlées par leur spin.

\textbf{+1938}\\
Le chimiste et physicien Isidor Isaac Rabi et ses collègues étudient les effets de la mise en place de faisceaux de molécules dans des champs magnétiques externes puissants, conduisant au développement de la résonance magnétique nucléaire (RMN). La même année, les physiciens Hans Bethe et Carl von Weizsäcker proposent une théorie nucléaire des étoiles et le mathématicien et ingénieur électricien Claude Shannon publie ce qui est probablement la thèse de maîtrise la plus célèbre du XXe siècle (\textit{Une analyse symbolique des relais et de la commutation circuits}), et prouvent qu'il est possible de simplifier la conception des circuits logiques en utilisant l'algèbre booléenne. Ce mémoire de maîtrise a joué un rôle important dans la conception d'ordinateurs électroniques. Le physicien Piotr Leonidovich Kapitza observe le phénomène de la superfluidité.

\textbf{+1937}\\
Les physiciens Seth Neddermeyer, Carl Anderson, Jabez Curry Street et E.C. Stevenson découvrent des muons dans les traces laissées par les rayons cosmiques dans une chambre à bulles. La même année, le mathématicien John von Neumann développa les méthodes de Monte Carlo pour différentes méthodes numériques et le physicien Niels Bohr développa le modèle de goutte liquide du noyau.

\textbf{+1936}\\
Les physiciens George Gamow et Edward Teller travaillent ensemble pour formuler la théorie des émissions radioactives bêta. La même année, dans son \textit{Sur les nombres calculables}, l'informaticien, mathématicien, logicien, cryptanalyste, philosophe et biologiste théorique Alan Mathison Turing analyse le concept de calculabilité en utilisant le concept de machine de Turing, l'un des fondements de l'informatique théorique. L'ingénieur, le statisticien, le professeur, l'auteur, le conférencier et le consultant en gestion Walter A. Shewhart publie ses travaux sur les cartes de contrôle CUSUM, UWMA et EWMA.

\textbf{+1935}\\
Le physicien Hideki Yukawa présente la théorie de l'interaction forte et prédit l'existence de mésons. La même année, l'astrophysicien et mathématicien Subrahmanyan Chandrasekhar rapporte les résultats de ses recherches sur l'effondrement des étoiles en naines blanches et au-delà de $1.44$ masses solaires en étoiles à neutrons. L'article des physiciensAlbert Einstein, Boris Podolsky et Nathan Rosen sur le paradoxe de l'EPR est publié dans Physical Review et remet en question la non-localité de l'interprétation de Copenhague.

\textbf{+1934}\\
Le physicien Pavel Cherenkov Alekseyevich étudie l'émission de lumière lorsque des particules relativistes traversent un milieu amorphe. La même année, le physicien Enrico Fermi a suggéré de bombarder des atomes d'Uranium avec des neutrons pour obtenir un élément avec 93 protons, formule la théorie de la désintégration bêta et le physicien Leó Szilárd réalise qu'une réaction en chaîne nucléaire est possible. Les physiciens Irène Joliot-Curie et Frédéric Joliot bombardent des atomes d'Aluminium avec des particules alpha et créent artificiellement du Phosphore-30 radioactif. La falsifiabilité comme critère d'évaluation de nouvelles hypothèses est popularisée par Karl Popper.

\textbf{+1933}\\
Le mathématicien Andrei Nikolaevich Kolmogorov a publié un livre contenant une base solide d'axiomes de probabilité. Le physicien Ernst August Friedrich Ruska réalise le premier microscope électronique par transmission en utilisant des électrons au lieu de photons.

\textbf{+1932}\\
Le physicien Carl David Anderson découvre le positron. La même année, le physicien Werner Heisenberg présente le modèle théorique du noyau nucléaire proton-neutron et l'utilise pour expliquer les isotopes. Le physicien James Chadwick découvrit le neutron et les physiciens John Cockcroft et Ernest Walton brisent le noyau nucléaire du Lithium et du Bore par bombardement de protons.

\textbf{+1931}\\
Le physicien Wolfgang Pauli avance l'hypothèse du neutrino pour expliquer la violation apparente du principe de conservation de l'énergie dans la désintégration bêta. La même année, le mathématicien et logicien Kurt Gödel monte qu'un système peut être à la fois cohérent et complet (théorème d'incomplétude) et que si le système est cohérent, la cohérence des axiomes ne peut être prouvée dans le système. Le physicien Ernest Lawrence invente le premier cyclotron et le physicien, ingénieur et statisticien Walter Andrew Shewhart publie son livre \textit{Contrôle économique de la qualité du produit manufacturé} où il présente les principales cartes de contrôle.

\textbf{+1930}\\
Le physicien Fritz London explique que les forces de Van der Waals sont dues à l'interaction des moments dipolaires des molécules. La même année, le physicien Paul Dirac présente sa théorie des électrons-trous et l'économiste John Maynard Keynes publie son \textit{Traité sur la monnaie}.

\textbf{+1929}\\
L'astronome Edwin Hubble en étudiant le redshift (décalage vers le rouge) émet l'hypothèse que l'Univers n'est pas statique. La même année, le physicien Robert Van de Graaff invente le premier accélérateur de particules, connu aujourd'hui sous le nom "d'accélérateur Van de Graaff".

\textbf{+1928}\\
Le physicien Paul Dirac établi son équation d'onde relativiste pour l'électron, qui généralise et améliore l'équation relativiste sans spin de Klein-Gordon. La même année, les physiciens Friedrich Hund et Robert S. Mulliken introduisent le concept d'orbitale moléculaire et le physicien et cosmologiste George Gamow développe le modèle théorique quantique de la décroissance alpha  par effet tunnel. Le physicien Félix Bloch étudie le problème d'une particule soumise à un potentiel périodique et développe la théorie des bandes qui deviendra une base fondamentale de la théorie des solides, en particulier pour comprendre leurs propriétés de transport ou leurs propriétés optiques.

\textbf{+1927}\\
Le physicien Werner Heisenberg établit le principe d'incertitude, par lequel la position et la quantité de mouvement d'une particule ne peuvent pas être connus simultanément avec précision, indirectement en développant une nouvelle base théorique pour la mécanique quantique. La même année, les physiciens Walter Heitler et Fritz London présentent la théorie quantique de la liaison chimique établie à partir de la molécule d'hydrogène et le physicien Max Born interprète la fonction d'onde de Schrödinger comme probabilités et avec l'aide du physicien Robert Oppenheimer présente l'approximation de Born-Oppenheimer. Les physiciens Clinton Joseph Davisson, Lester Germer et George Paget Thomson confirment la nature ondulatoire des électrons par diffraction. Le physicien Paul Ehrenfest prouve le fameux théorème de la physique quantique qui porte son nom. Le physicien Paul Dirac introduit la quantification du champ électromagnétique.

\textbf{+1926}\\
Le physicien Erwin Schrödinger établit son équation d'onde qui définit la mécanique quantique sous une forme analytique en développant les idées de De Broglie sur la théorie de la mécanique ondulatoire et prouve que les formulations d'onde et de matrice de la théorie quantique sont mathématiquement équivalentes. La même année, les physiciens Oskar Klein et Walter Gordon établissent l'équation de la mécanique quantique relativiste pour les particules sans spin et Paul Dirac définit les statistiques de Fermi-Dirac. Dans le domaine de la dynamique des populations, le physicien et mathématicien Vito Volterra puublie l'équation différentielle non linéaire modélisant l'équilibre prédateur-proie. Les plans d'expériences aléatoires sont popularisés et analysés par le statisticien britannique Ronald Fisher.

\textbf{+1925}\\
Le physicien Pierre Auger découvre l'effet Auger (deux ans après Lise Meitner) et la même année les physiciens George Uhlenbeck et Samuel Goudsmit postulent et révèlent l'existence du spin électronique. Aussi la même année, le physicien Wolfgang Pauli établi par nécessité le principe de l'exclusion quantique. Les physiciens Werner Heisenberg, Max Born et Pascual Jordan formulent la version matricielle de la mécanique  quantique.

\textbf{+1924}\\
Le physicien John Lennard-Jones propose une description semi-empirique des forces d'interaction interatomiques et la même année, les physiciens Satyendranath Bose et Albert Einstein définissent les statistiques de Bose-Einstein. Dans le domaine des statistiques, le statisticien Ronald Fisher définit les grands concepts modernes de la statistique.

\textbf{+1923}\\
L'astronome Edwin Hubble estime la distance entre la Terre et les galaxies spirales, montrant qu'elles sont loin de la Voie Lactée et la même année le physicien Louis de Broglie suggère la dualité onde-particule de la théorie quantique et de l'équivalence masse-énergie et la physicienne Lise Meitner découvre l'effet Auger. Le mathématicien Norbert Wiener introduit la théorie du mouvement brownien.

\textbf{+1922}\\
Le physicien Arthur Compton étudie la diffusion des photons X par les électrons et l'astrophysicien Alexander Friedmann développe des modèles d'univers non statiques.

\textbf{+1921}\\
Le physicien Alfred Landé définit le rapport gyromagnétique et introduit aussi des nombres quantiques semi-entiers. La même année, les physiciens Otto Stern et Walter Gerlach montrent expérimentalement que le moment intrinsèque de l'électron est quantifié. Le mathématicien et physicien Theodor Franz Eduard Kaluza prouve qu'une version en cinq dimensions des équations d'Einstein unifie la gravitation et l'électromagnétisme.

\textbf{+1920}\\
L'astronome Vesto Melvin Slipher met en évidence le phénomène du décalage vers le rouge dans le spectre des galaxies. La même année, le physicien Arnold Sommerfeld introduit un quatrième nombre quantique au modèle original de l'atome de Bohr. Le physicien Niels Bohr introduit le principe de la correspondance assurant la transition de la physique quantique $\mapsto$ à la physique classique lorsque $\hbar \rightarrow 0$. L'astronome et astrophysicien Ernst Julius Öpik confirme que la "nébuleuse d'Andromède" est située en dehors de la Voie Lactée, ce qui confirme que notre Univers est beaucoup plus vaste que nous le pensions car il n'est pas limité à la Voie Lactée.

\textbf{+1919}\\
Le physicien Ernest Rutherford a effectué la première désintégration artificielle d'un atome en bombardant l'Azote avec des particules alpha. La même année, la physicienne et mathématicienne Amali Emmy Noether développe son théorème sur les invariants différentiels dans le calcul des variations, l'un des théorèmes mathématiques les plus importants jamais prouvés pour guider le développement de la physique moderne.

\textbf{+1918}\\
L'astronome Harlow Shapley fait la première estimation précise de la taille de notre galaxie et de la position du Soleil. La même année, le physicien Hermann Weyl introduit la notion de jauge, première étape de ce qui deviendra la théorie de la jauge.

\textbf{+1917}\\
Le physicien Albert Einstein introduit l'idée de l'émission stimulée de rayonnement utilisée dans la base de fabrication de LASER. La même année, le physicien Arnold Sommerfeld introduit un troisième nombre quantique au modèle original de l'atome de Bohr.

\textbf{+1916}\\
Le physicien Albert Einstein développe sa théorie de la Relativité Générale et comment la matière joue sur l'espace-temps pour produire des effets gravitationnels. C'est la première théorie nommée "indépendante de fond". La même année, les physiciens Gilbert Lewis et Irving Langmuir présentent le modèle de coquille électronique pour expliquer les liaisons chimiques et le physicien Arnold Sommerfeld introduit la relativité dans son modèle de 1915 et cette correction relativiste explique les valeurs observées par les spectrographes haute résolution et donc la division spectrale des lignes appelées "structure fine" et il introduit en même temps un deuxième nombre quantique décrivant des orbites elliptiques. Le physicien Karl Schwarzschild trouve une solution mathématique aux équations d'Einstein, qu'il applique aux étoiles à neutrons et aux Trous Noirs.

\textbf{+1915}\\
Le physicien Arnold Sommerfeld affine le modèle atomique du physicien Niels Bohr en introduisant des orbites elliptiques pour expliquer les fines lignes spectroscopiques de structure de l'atome d'hydrogène. Ce nouveau modèle n'explique cependant pas la gamme des spectres observés de l'atome d'hydrogène. La même année, le physicien Albert Einstein calcule la trajectoire de Mercure avec la Relativité Générale. Il utilise sa théorie pour calculer l'avance de Mercure de la périhélie avec une grande précision et la déviations des rayons lumineux dans le champ gravitationnel du Soleil. La fin de cette même année il soumet l'article qui décrit les équations des champs de la gravitation qui seront à la base de la théorie de la Relativité Générale.

\textbf{+1914}\\
Le physicien Ernest Rutherford montre que les noyaux atomiques chargés positivement contiennent des protons. La même année, le physicien Albert Einstein et le mathématicien Marcel Grossmann publient un article sur le calcul tensoriel, et plus particulièrement sur le tenseur de Riemann-Christoffel et de Ricci (plus généralement sur l'analyse tensorielle et la géométrie différentielle) et le physicien Peter Debye développe un modèle de comportement de la capacité thermique des solides en fonction de la température. Le mathématicien Felix Hausdorff introduit les concepts de distance de Hausdorff et de dimension de Hausdorff.

\textbf{+1913}\\
Le physicien Niels Bohr présente le modèle quantique en couches circulaires de l'atome et la même année le physicien Robert Millikan mesure la charge électrique fondamentale. La même année, les physiciens William Henry Bragg et William Lawrence Bragg trouvent la condition de Bragg pour les forces de réflexion des rayons X et le physicien Henry Moseley montre que le nombre atomique est le vrai critère de discrimination des éléments chimiques fondamentaux. En mathématiques, le mathématicien Elie Cartan annonce sa découverte des spineurs. Le physicien Johannes Stark démontre que les champs électriques forts vont diviser la série de raies spectrales Balmer de l'hydrogène.

\textbf{+1912}\\
Le physicien Max von Laue propose l'utilisation de réseaux cristallins pour diffracter les rayons X et la même année, les physiciens Walter Friedrich et Paul Knipping diffractent les rayons X en utilisant du sulfure de Zinc. La même année, le physicien Ernest Rutherford propose l'utilisation de la radioactivité comme moyen de datation. Les chimistes Otto Sackur et le physicien Hugo Tetrode mettent au point une formule pour calculer l'entropie d'un gaz mono-atomique (première apparition de la constante de Planck en thermodynamique).

\textbf{+1911}\\
Le physicien Ernest Rutherford découvre le noyau atomique en bombardant une fine feuille d'or avec des particules alpha. Certaines particules rebondissent sur le noyau des atomes d'or. La même année, le physicien et chimiste Jean Perrin prouve l'existence d'atomes et de molécules et le physicien Heike Kammerlingh Onnes découvre la supraconductivité.

\textbf{+1911}\\
Le physicien Robert Andrews Millikan détermine la charge électrique portée par un seul électron avec sa fameuse expérience de goutte d'huile.

\textbf{+1909}\\
Les physiciens Hans Geiger et Ernest Marsden découvrent que les particules alpha peuvent être fortement déviées par de minces feuilles de métal et la même année, les physiciens Ernest Rutherford et Thomas Royds ont démontré que les particules alpha sont des atomes d'Hélium ionisés deux fois. Dans le domaine des mathématiques appliquées, le mathématicien Agner Krarup Erlang puble le premier article sur la théorie des files d'attente.

\textbf{+1908}\\
Le statisticien William Sealy Gosset publie un article proposant une nouvelle distribution statistique et un nouveau test statistique nommés respectivement "Loi de student" et "test $T$ de Student". La même année, le mathématicien Ernst Friedrich Ferdinand Zermelo propose une amélioration des axiomes de la théorie des ensembles. Dans son \textit{Proportions Mendéliennes dans une population mixte}, le mathématicien Godfrey Harold Hardy expose ce qui est maintenant connu comme le "principe de Hardy-Weinberg" en génétique qui établit comment les traits génétiques dominants et récessifs se propagent dans une grande population. Hardy est connu comme mathématicien fondamental, spécialiste de la théorie des nombres, mais a contribué de manière significative à l'étude de la génétique des populations par les résultats présentés dans cet article, trouvé indépendamment par l'obstétricien-gynécologue Wilhelm Weinberg.

\textbf{+1907}\\
Le physicien Albert Einstein déduit l'expression de la fameuse équivalence entre masse et énergie, et la même année il établit l'expression de la capacité calorifique des solides cristallins et calcule le redshift gravitationnel. Le mathématicien et physicien Hermann Minkowski unifie l'espace-temps dans une structure mathématique unifiée. Le mathématicien Guido Fubini prouve le théorème intégral multiple qui porte son nom.

\textbf{+1906}\\
Le physicien Walther Nernst présente une formulation de la troisième loi de la thermodynamique. La même année, le mathématicien Andreï Markov publie le premier ouvrage sur les chaînes d'événements qui portera plus tard son nom (chaînes de Markov) et qui ont occupé une place importante dans la physique quantique de son temps.

\textbf{+1905}\\
Le physicien Albert Einstein explique l'effet photoélectrique par l'existence de quantums et la même année il explique mathématiquement le mouvement brownien à la suite du mouvement aléatoire des molécules et publié sa recherche sur sa théorie de la relativité restreinte qui prouve l'équivalence de masse et d'énergie. Le physicien Paul Langevin publie sa théorie sur la susceptibilité des matériaux paramagnétiques.

\textbf{+1904}\\
Le physicien Antoon Lorentz découvre la contraction du temps dans le sens du mouvement du corps par rapport à la vitesse constante de la lumière et propose les équations de transformation des forces électromagnétiques. La même année, le physicien Hantaro Nagaoka propose un modèle théorique saturnien de l'atome où les électrons tournent autour d'un noyau positif massif comme les anneaux de Saturne.

\textbf{+1903}\\
La radioactivité est expliquée en termes de fission des atomes par le physicien Ernest Rutherford et par le radiochimiste Frederick Soddy.

\textbf{+1902}\\
Le physicien Philipp Lenard observe que l'effet photoélectrique ne dépend pas de la puissance du faisceau lumineux mais de sa fréquence. La même année, le chimiste Theodor Svedberg suggère que les fluctuations du bombardement moléculaire créent un mouvement brownien et le logicien Bertrand Russell propose son paradoxe «ultime» sapant la théorie naïve des ensembles. Le physicien James Jeans décrit le phénomène d'effondrement gravitationnel qui peut se produire par exemple dans un nuage de matériau gazeux basé sur une masse ou un rayon critique. Le mathématicien Henri Léon Lebesgue pose la base de la théorie de la mesure et introduit l'intégrale de Lebesgue.

\textbf{+1901}\\
Les mathématiciens Gregorio Ricci-Curbastro et son assistant Tullio Levi-Civita développent le calcul tensoriel.

\textbf{+1900}\\
Le physicien Max Planck suggère que la lumière peut être émise à fréquences discrètes en généralisant la loi du rayonnement du corps noir. La même année, le physicien Johannes Rydberg affine l'expression mathématique des longueurs d'onde des raies de l'hydrogène de Balmer et le physicien Paul Villard découvre les rayons gamma en étudiant la désintégration de l'Uranium. Dans le domaine des mathématiques appliquées, le mathématicien Louis Bachelier développe le mouvement modèle brownien appliqué à la théorie du jeu et de la spéculation qui sera le pilier des outils financiers quantitatifs du XXe siècle. La même année, le mathématicien Karl Pearson définit la distribution statistique du Khi-2 et explore les propriétés importantes de cette distribution pour l'inférence statistique. Le physicien Paul Drude adapte la théorie cinétique des gaz aux électrons dans les métaux et obtient un modèle qui porte encore son nom.

\textbf{+1899}\\
Le physicien Ernest Rutherford découvre que les radiations émises par les composés d'Uranium sont des particules alpha chargées positivement et des particules bêta chargées négativement. La même année, le mathématicien David Hilbert remplace les $5$ axiomes habituels de la géométrie euclidienne par $21$ axiomes pour éliminer les faiblesses de la géométrie euclidienne.

\textbf{+1898}\\
Le mathématicien David Hilbert donne une première approche des corps commutatifs. La même année, les physiciens Marie et Pierre Curie isolent et étudient le Radium et le Polonium et le physicien Wilhelm Wien Carl Werner identifie une nouvelle particule avec une charge positive à peu près égale à la masse d'hydrogène qu'il nommera le "proton". L'ingénieur Alfred-Marie Liénard calcule le champ électromagnétique produit par une charge ponctuelle en mouvement (expressions mathématiques qui ont été établies indépendamment, mais deux ans plus tard par le physicien Emil Wiechert).

\textbf{+1897}\\
Le physicien Joseph John Thomson mesure le rapport charge/masse de certaines particules négatives créées par des rayons cathodiques. Il mesure leur charge, et il en conclut que leur masse est environ $2'000$ fois inférieure à celle de l'hydrogène. Ces particules sont plus tard appelées "électrons", expression suggérée par le physicien George Johnstone Stoney. Les téléviseurs et autres écrans cathodiques sont des versions améliorées du dispositif de Thomson.

\textbf{+1896}\\
Le physicien Henri Becquerel découvre la radioactivité de l'uranium et la même année le physicien Pieter Zeeman étudie la décomposition des raies D du sodium lorsqu'il est chauffé dans un fort champ magnétique et il découvre que les raies spectrales d'une source lumineuse soumise à un champ magnétique a de nombreuses composants, chacun ayant une certaine polarisation. Pour expliquer ce phénomène, il faut ajouter un nombre quantique additionnel nommé "nombre quantique magnétique". La même année, le physicien Wilhelm Wien Carl Werner établit la loi qui porte son nom pour l'énergie émise par le corps noir.

\textbf{+1895}\\
Le physicien Wilhelm Röntgen découvre les rayons X et la même année, le physicien et inventeur Guglielmo Marconi réalise dans les Alpes suisses à Salvan la première transmission "longue distance" (pour l'époque...) sans fil de 1.5 kilomètre.

\textbf{+1893}\\
Le mathématicien, physicien, ingénieur et philosophe Henri Poincaré publie ses études sur le problème des trois corps et introduit l'étude qualitative des équations différentielles et de la théorie du chaos. La même année, le mathématicien Georg Cantor développe la théorie des ensembles transfinis et la proposition de l'ingénieur Nikola Tesla d'utiliser le courant alternatif au lieu du courant continu est adopté par le premier état américain. Dans son \textit{Uniplanar Algebra}, le mathématicien Irving Stringham utilise le symbole $\ln$ pour le logarithme naturel au lieu du $\log_e$ traditionnel.

\textbf{+1892}\\
Le physicien autodidacte Oliver Heaviside réduit les $8$ équations de l'électrodynamique de Maxwell à $4$ équations différentielles.

\textbf{+1891}\\
Dans son \textit{Arithmetices Principia, ova methodo exposita}, le mathématicien Giuseppe Peano introduit les axiomes pour construire l'ensemble des nombres naturels $\mathbb{N}$ et le symbole d'appartenance $\in$ et une première version des quantificateurs symboliques. Leur forme finale sera donnée par le mathématicien David Hilbert. Il fournit plus de $40'000$ définitions dans une langue qu'il veut universelle.

\textbf{+1890}\\
L'étude systématique des groupes se développe avec les mathématiciens Sophus Lie, Issai Schur et Élie Cartan. Ce dernier introduit la notion de groupe algébrique et de groupes continus.

\textbf{+1889}\\
Le mathématicien Giuseppe Peano postule $5$ propriétés d'entiers comme axiomes dans l'idée de faire avec des entiers ce qu'Euclide a fait pour la géométrie. Il définit aussi l'axiomatique de l'espace vectoriel dans $\mathbb{R}$ et introduit le concept d'application linéaire. Il introduit aussi la notation $\cup$ et $\cap$ pour l'union et l'intersection des ensembles.

\textbf{+1888}\\
L'anthropologue, explorateur, géographe, inventeur, météorologue, proto-généticien, psychométricien et statisticien ... Francis Galton définit le concept de coefficient de corrélation statistique. La même année, le mathématicien Richard Dedekind propose la définition d'un ensemble fini.

\textbf{+1887}\\
Les physiciens Albert Michelson et Edward Morley mesurent la vitesse de la lumière pour tester l'hypothèse de l'éther que leurs résultats expérimentaux rejettent et la même année le physicien Heinrich Hertz découvre l'effet photoélectrique et mène des expériences sur les ondes électromagnétiques (production et réception).

\textbf{+1886}\\
Le mathématicien et physicien Oliver Heaviside introduit les opérateurs différentiels en tant qu'entités algébriques qui vont plus tard amener aux transformées de Laplace.

\textbf{+1885}\\
Le chimiste Johan Balmer trouve l'expression mathématique qui donne la longueur d'onde des différentes raies du spectre de l'hydrogène.

\textbf{+1884}\\
Le physicien et chimiste Willard Gibbs définit la notation encore en usage au début du 21ème siècle pour les produits scalaires et vectoriels ainsi que les opérateurs différentiels vectoriels dans ses livres sur le calcul vectoriel. La même année, le physicien Ludwig Boltzmann tire la loi de  Stefan-Boltzmann du flux lumineux du corps noir uniquement avec des considérations thermodynamiques et le physicien John Henry Poynting introduit le vecteur qui porte encore aujourd'hui son nom. Le mathématicien Karl Hermann Amandus Schwarz prouve que la sphère est le solide avec la surface minimale pour un volume donné (ce résultat explique de nombreuses formes visibles dans l'Univers).

\textbf{+1882}\\
Le mathématicien Ferdinand von Lindemann pourve la transcendance de $\pi $. La même année, l'astronome, mathématicien, économiste et statisticien Simon Newcomb observe une précession excessive de $43''$ par siècle de l'orbite de Mercure. Là où la mécanique d'Isaac Newton explique correctement la période de précession des autres planètes, elle échoue à expliquer celle de Mercure.

\textbf{+1880}\\
Le mathématicien et physicien Oliver Heaviside introduit la fonction "escalier" qui porte toujours son nom aujourd'hui (fonction de Heaviside).

\textbf{+1879}\\
Le mathématicien et physicien Joseph Stefan publie la loi de Stefan qui stipule que la puissance transmise à travers toute la gamme spectrale est proportionnelle à la quatrième puissance de la température absolue d'une étoile à sa surface. Pour une même température de surface, une étoile est aussi plus brillante, plus elle est grande. La même année, le physicien Edwin Herbert Hall découvre qu'un courant électrique à travers un matériau immergé dans un champ magnétique génère une tension perpendiculaire à la direction initiale du courant électrique. Le philosophe, logicien et mathématicien Gottlob Frege publie son \textit{Begriffsschrift, eine der arithmetischen nachgebildete Formelsprache des reinen denkens} où il introduit la logique de prédicat axiomatique, les quantificateurs (pour tout $\forall$, il existe $\exists$) , la théorie de la variable quantifiée, le concept rigoureux de formule / fonction et de variable.

\textbf{+1878}\\
Le mathématicien et philosophe William Kingdon Clifford introduit l'opérateur de divergence.

\textbf{+1877}\\
Le physicien et chimiste Willard Gibbs définit pour les réactions chimiques deux quantités utiles, à savoir l'enthalpie qui représente la chaleur de réaction à pression constante et l'énergie libre qui détermine si une réaction peut se dérouler spontanément à température et pression constantes. Le physicien John William Strutt (3e Baron Rayleigh) introduit les fondements de la théorie moderne du son. 

\textbf{+1876}\\
Le mathématicien et philosophe William Kingdon Clifford suggère que le mouvement de la matière peut être dû à des changements dans la géométrie de l'espace.

\textbf{+1874}\\
Le physicien Lord Kelvin énonce formellement la deuxième loi de la thermodynamique. La même année, l'économiste mathématique Léon Walras publie ses \textit{Elements of Pure Economics}.

\textbf{+1873}\\
Le mathématicien Georg Cantor pose les bases de la théorie des ensembles et des cardinaux et montre que les nombres algébriques sont en fait dénombrables et définit rigoureusement les nombres réels $\mathbb{R}$, et introduit la fameuse méthode diagonale. La même année, le physicien James Clerk Maxwell montre que la lumière est un phénomène électromagnétique et réduit les équations de l'électrodynamique à $8$ équations au lieu de $20$ (en même temps il définit l'opérateur rotationnel) et le physicien Johannes van der Waals introduit idée qu'il existe de faibles forces d'attraction entre les molécules. Dans son \textit{Sur la fonction exponentielle} le mathématicien Charles Hermite prouve la transcendance de $e$ (deux preuves, une de $11$ pages et l'autre de $20$ pages).

\textbf{+1872}\\
Le mathématicien Karl Weierstrass présente à l'Académie royale des sciences de Berlin un exemple d'une fonction continue partout mais différentiable nulle part.

\textbf{+1871}\\
Le chimiste Dmitri Mendeleev examine son tableau périodique et prédit l'existence du Gallium, du Scandium et du Germanium. La même année, le physicien James Clerk Maxwell établit les relations thermodynamiques de Maxwell.

\textbf{+1870}\\
Le physicien Rudolph Clausius prouve le théorème du viriel (scalaire).

\textbf{+1869}\\
Le chimiste Dmitri Mendeleev propose le tableau périodique des éléments qui porte encore son nom aujourd'hui.

\textbf{+1867}\\
L'historien, journaliste, philosophe, économiste, sociologue Karl Marx publie \textit{Das Kapital}.

\textbf{+1866}\\
Le physicien James Clerk Maxwell élabore, indépendamment du physicien Ludwig Boltzmann, la théorie cinétique des gaz de Maxwell-Boltzmann. La même année, le moine et botaniste Gregor Johann Mendel formule les lois de l'hybridation statistique (expérience sur $29'000$ petits pois ...) et le mathématicien Léopold Kronecker utilise pour la première fois le symbole qui porte toujours son nom aujourd'hui.

\textbf{+1865}\\
Le physicien James Clerk Maxwell publie pour la première fois les équations de l'électrodynamique sous la forme de $20$ équations avec $20$ inconnues utilisant des quaternions.

\textbf{+1862}\\
Le physicien Gustav Kirchhoff développe le concept du corps noir qui peut absorber et émettre des radiations à toutes les fréquences et que l'énergie émise dépend seulement de la fréquence du rayonnement émis et de la température du corps noir lui-même.

\textbf{+1859}\\
Le physicien James Clerk Maxwell découvre la loi de la distribution des vitesses moléculaires. La même année l'astronome Urbain Le Verrier rapporte une anomalie dans le mouvement de Mercure non prévisible par la loi de Newton et le physicien Gustav Kirchhoff avec le chimiste Robert Wilhelm Bunsen développent la spectroscopie prismatique.

\textbf{+1858}\\
L'avocat et mathématicien Arthur Cayley fait émerger la notion d'espace vectoriel, la notion de matrice et expose l'utilité en utilisant la multiplication des matrices et des déterminants; il réécrit le système des équations linéaires sous forme matricielle. Ses travaux sont souvent considérés comme l'émergence de l'algèbre linéaire.

\textbf{+1855}\\
L'astronome et physicien Léon Foucault découvre que la force nécessaire à la rotation d'un disque de cuivre augmente quand on doit le tourner avec sa jante entre les pôles d'un aimant, le disque chauffant en même temps à cause des "courants de Foucault" induits dans le métal.

\textbf{+1854}\\
Le mathématicien George Boole publie son système de logique symbolique, maintenant connu sous le nom d'algèbre de Boole. La même année, le mathématicien Arthur Cayley montre que les quaternions peuvent être utilisés pour représenter les rotations dans l'espace à quatre dimensions et le mathématicien Georg Friedrich Bernhard Riemann donne une nouvelle définition de l'intégrale et pose les bases de la géométrie différentielle. Le mathématicien Charles Hermite définit le concept des matrices orthogonales et prouve que leurs valeurs propres sont des nombres réels.

\textbf{+1852}\\
Les physiciens James Joule et William Thomson Kelvin montre que le gaz en expansion se refroidit rapidement.

\textbf{+1851}\\
L'astronome et physicien Léon Foucault fait une preuve spectaculaire de la rotation de la Terre en suspendant un pendule avec un long câble attaché au dôme du Panthéon à Paris. La même année, le mathématicien Georg Friedrich Bernhard Riemann publie le premier travail sur les fonctions avec une variable complexe. Dans son \textit{Paradoxien des Unendlichen}, le mathématicien, logicien, philosophe et théologien Bernardus Placidus Johann Nepomuk Bolzano discute des questions liées à la manipulation des infinis en mathématiques.

\textbf{+1850}\\
Les mathématiciens Arthur Cayley et James Joseph Sylvester introduisent le terme de matrices et la même année, le mathématicien Richard Dedekind introduit le terme de corps. Le physicien Rudolf Clausius développe la théorie mécanique de la chaleur et formule le second principe de la thermodynamique. Le phyisicien George Stokes prouve le théorème de Stokes.

\textbf{+1849}\\
Le mathématicien et astronome Edouard Roche trouve le rayon limite de destruction de création par forces de marées et d'un corps qui tient par sa seule gravité utilise ce résultat pour expliquer pourquoi les anneaux de Saturne ne se condensent en un satellite.

\textbf{+1848}\\
Le physicien William Thomson Kelvin découvre le point absolu $0$ de température et définit sa propre unité de mesure. La même année le physicien et astronome Hyppolite Fizeau transpose les résultats de Christian Doppler à la lumière qui, comme le son, possède un caractère ondulatoire (effet Doppler-Fizeau) et met en évidence le décalage vers le rouge et vers le bleu.

\textbf{+1847}\\
Le physicien James Joule trouve expérimentalement l'équivalent mécanique de la chaleur et la même année, le physiologiste et physicien Hermann Helmholtz formalise le concept de conservation de l'énergie.

\textbf{+1845}\\
Le physicien Gustav Kirchoff définit le concept de potentiel électrique et énonce les lois de réseaux qui portent son nom (loi des noeuds, loi des mailles). La même année, le physicien George Stokes publie ce qui sera les fondements des équations de Navier-Stokes en mécanique des fluides et le physicien Michael Faraday découvre que la propagation de la lumière dans un matériau peut être influencé par des champs magnétiques externes.

\textbf{+1844}\\
Le mathématicien Joseph Liouville montre l'existence d'une infinité de nombres transcendants.

\textbf{+1843}\\
Le physicien, mathématicien et astronome William Rowan Hamilton définit des espaces de vecteurs complexes (quaternions). La notion d'espace vectoriel sera clairement définie par le mathématicien et astronome August Ferdinand Möbius et par le mathématicien et linguiste Giuseppe Peano $40$ ans plus tard. La même année, le mathématicien Laurent Pierre Alphonse publie son mémoire sur ce qui deviendra plus tard les séries qui portent son nom en analyse complexe.

\textbf{+1842}\\
Le principe de la conservation de l'énergie est formulé par le physicien Julius Von Mayer qui a calculé la quantité de travail pouvant être obtenue par transformation de la chaleur en énergie, soit l'équivalent mécanique de la calorie. La même année, le physicien Christian Doppler découvre l'effet acoustique qui porte son nom (variation de la fréquence avec le mouvement relatif).

\textbf{+1841}\\
Le mathématicien Karl Weierstrass découvre mais ne publie pas ce que nous appelons aujourd'hui les séries de Laurent. La même année, le matématicien Carl Gustav Jacob Jacobi introduit les matrices jacobiennes et réintroduit la notation de la dérivée partielle proposée initialement par le mathématicien André-Marie Legendre.

\textbf{+1838}\\
L'astronome et mathématicien Friedrich Bessel mesure que la distance qui nous sépare de l'étoile 61 Cygni est de 96 billions de kilomètres.

\textbf{+1835}\\
Le mathématicien Carl Friedrich Gauss donne une construction rigoureuse des nombres complexes et le mathématicien Augustin Louis Cauchy établit une première théorie des déterminants. La même année, le mathématicien et ingénieur Gaspard Coriolis démontre que l'accélération d'un mobile dans un référentiel en rotation est soumise à une force complémentaire perpendiculaire au sens de déplacement du mobile dans ce référentiel.

\textbf{+1834}\\
L'ingénieur et physicien Émile Clapeyron présente une formulation de la seconde loi de la thermodynamique. La même année, le physicien Heinrich Lenz établit la loi d'induction électromagnétique.

\textbf{+1832}\\
Le physicien Michael Faraday établit la théorie fondamentale de l'électrolyse.

\textbf{+1831}\\
Le physicien Michael Faraday découvre l'induction électromagnétique, à savoir l'obtention d'un courant électrique à partir de la variation d'un champ magnétique (principe de la dynamo: le contraire de l'expérience d'Ørsted). La même année le mathématicien, astronome et physicien Carl Friedrich Gauss énonce deux des quatre équations de Maxwell.

\textbf{+1830}\\
Le mathématicien et inventeur anglais Charles Babbage est crédité pour avoir conçu théoriquement le premier ordinateur numérique automatique. Au milieu des années 1830, Babbage élabora des plans pour une machine analytique. Bien qu'elle n'ait jamais été achevée, la machine analytique aurait eu la plupart des éléments de base de l'ordinateur actuel. Ainsi Babbage ne peut être crédité pour la création du premier ordinateur fonctionnel.

\textbf{+1829}\\
Le mathématicien Évariste Galois présente la première ébauche de son travail sur les équations résolubles qui sera à l'origine de l'approche ensembliste de la résolution d'équation algébrique par radicaux. Le mathématicien Augustin Louis Cauchy prouve que les valeurs propres d'une matrice symétrique sont toutes réelles. Le mathématicien Johann Peter Gustav Lejeune Dirichlet étudie la convergence des séries de Fourier.

\textbf{+1828}\\
Le physicien George Green prouve le théorème de Green.

\textbf{+1827}\\
Le botaniste Robert Brown découvre le mouvement brownien des particules de pollen et de colorant dans l'eau ; la même année le physicien Georg Ohm établit la loi de la résistance électrique et le physicien et mathématicien André Ampère découvre les lois qui lient les forces magnétiques au courant électrique. Le physicien, mathématicien et astronome William Rowan Hamilton présente la théorie d'une unique fonction qui unifie la mécanique, optique et mathématiques et qui aida à établir la théorie ondulatoire de la lumière.

\textbf{+1826}\\
Le mathématicien Niels Henrik Abel prouve qu'il est impossible de résoudre l'équation quintique générale (polynômes d'ordre $5$) par radicaux. Dans son \textit{Sur une méthode d'expression par l'action des machines}, le mathématicien, philosophe, inventeur et ingénieur mécanique Charles Babbage décrit un langage symbolique qui l'aidera à concevoir sa machine analytique, la première machine mécanique universelle. La "machine de Babbage" est le premier ordinateur programmable complet (ayant les mêmes capacités de calcul qu'une machine de Turing ou un ordinateur moderne) qui a été conçu.

\textbf{+1825}\\
Le mathématicien Augustin-Louis Cauchy présente le théorème de Cauchy et l'intégration curviligne générale et introduit le théorème des résidus. Le scientifique et inventeur William Sturgeon invente le premier électroaimant.

\textbf{+1824}\\
Le mathématicien Augustin-Louis Cauchy découvre le polynôme caractéristique d'une matrice et prouve qu'il est invariant par transformation linéaire et calcule pour la première fois des valeurs propres et des vecteurs propres. La même année, le physicien et ingénieur Sadi Carnot analyse scientifiquement l'efficacité des machines à vapeur (cycle de Carnot), montrant que leurs performances sont limitées et définit également le second principe de la thermodynamique.

\textbf{+1823}\\
Le physicien et chimiste Michael Faraday présente une série de papiers sur la liquéfaction des gaz.  Le mathématicien Pierre Frédéric Sarrus introduit le symbole de la barre verticale pour l'intégrale: $\int_a^b f(x)\mathrm{d}x= F(x)|_a^b$.

\textbf{+1822}\\
Le mathématicien Jean-Victor Poncelet fonde la géométrie projective. La même année, le physicien et mathématicien Joseph Fourier présente formellement l'utilisation des dimensions (unités) pour des quantités physiques et introduit la notation $\int_a^bf(x)\mathrm{d}x$.

\textbf{+1821}\\
Le principe de la dynamo est décrit pour la première fois par le physicien et chimiste Michael Faraday. La même année le physicien John Herapath propose que la chaleur n'est en réalité que l'effet d'agitation et donc de mouvement de corps élémentaires.

\textbf{+1820}\\
Le physicien et chimiste Hans Christian Ørsted découvre et prouve les effets magnétiques du courant électrique. La même année, les physiciens Jean-Baptiste Biot et Félix Savart déterminent dans le domaine du magnétisme la fameuse loi qui porte leur nom.

\textbf{+1819}\\
Le physicien et chimiste Hans Christian Ørsted montre que le courant électrique dévie une aiguille aimantée, démontrant ainsi l'électromagnétisme et annonçant une révolution industrielle.

\textbf{+1818}\\
Le mathématicien, géomètre et physicien Simeon Poisson calcule le point lumineux de Poisson au centre de l'ombre d'un obstacle circulaire opaque.

\textbf{+1817}\\
En étudiant la polarisation de la lumière, le physicien Augustin Fresnel montre que cette dernière est un mouvement ondulatoire transversal et non longitudinal et montre aussi que la diffraction et l'interférence peuvent être expliquées si l'on considère la lumière comme une onde. La même année, l'astronome Friedrich Bessel publie des travaux faisant usage des fameuses fonctions qui portent son nom.

\textbf{+1816}\\
Le mathématicien Joseph Diaz Gergonne introduit le symbole marquant l'inclusion dans la théorie des ensembles.

\textbf{+1814}\\
Le physicien et opticien Joseph Von Fraunhofer étudie pour la première fois les raies d'absorption du spectre solaire et ce au moyen du spectroscope dont il fût l'inventeur. Le mathématicien, astronome et physicien Pierre-Simon Laplace fait l'hypothèse qu'une parfaite connaissance de l'état actuel de l'Univers permettrait de déterminer parfaitement tous ses états futurs.

\textbf{+1812}\\
Le mathématicien, astronome et physicien Pierre-Simon de Laplace publie un ouvrage majeur sur la théorie des probabilités (incluant aussi la méthode des moindres carrés) dont il est considéré comme l'un des fondateurs.

\textbf{+1811}\\
Le chimiste Amaedo Avogadro avance l'hypothèse selon laquelle des volumes égaux de gaz différents contiennent le même nombre de molécules, dans des conditions identiques de température et de pression.

\textbf{+1810}\\
Le mathématicien, astronome et physicien Carl Friedrich Gauss découvre les concepts de base de la géométrie non-euclidienne mais ne publiera jamais ses travaux à ce sujet. La même année, le physicien et mathématicien Joseph Fourier modélise l'évolution de la température au travers de séries trigonométriques.

\textbf{+1809}\\
Le mathématicien, astronome et physicien Carl Friedrich Gauss développe la méthode des moindres carrés indépendamment du mathématicien André-Marie Legendre . La même année, le mathématicien, astronome et physicien Pierre-Simon de Laplace démontre la forme générale du théorème central limite. L'ingénieur, physicien et mathématicien Etienne Malus publie la loi de Malus.

\textbf{+1808}\\
Le physicien et chimiste John Dalton propose ce qui est considéré comme la première théorie de l'atome. Le mathématicien Christian Kramp introduit dans son \textit{Éléments d'arithmétique universelle} la notation $n!$ pour la factorielle.

\textbf{+1806}\\
Le mathématicien Jean Robert Argand publie la première représentation plane des nombres complexes et utilise des mesures algébriques. Le mathématicien Johann Carl Friedrich Gauss développe l'idée d'ajouter des vecteurs sous une forme géométrique et introduit la notation $\overrightarrow{ab}$ pour un vecteur.

\textbf{+1805}\\
Le mathématicien André-Marie Legendre développe la méthode des moindres carrés.

\textbf{+1803}\\
Le physicien et chimiste John Dalton a l'idée originale de considérer que chaque élément chimique est constitué d'atomes différents. Une combinaison chimique s'explique alors par l'union de ces atomes en proportions fixes et les masses atomiques relatives devenaient calculables à partir de faits expérimentaux. La même année, l'économiste, journaliste et industriel Jean-Baptiste Say publie son \textit{Traité d'économie politique}.

\textbf{+1802}\\
Le physicien Thomas Young prouve la nature ondulatoire de la lumière par une expérience importante qui montre l'interférence des ondes. La même année, le chimiste et physicien Louis Joseph Gay-Lussac découvre la fameuse loi sur les gaz qui relie volume et température d'un gaz réel, loi qui porte son nom (loi de Gay-Lussac).

\textbf{+1801}\\
Le chimiste et physicien John Dalton, découvre la loi de la somme des pressions partielles des gaz qui porte encore aujourd'hui son nom.

\textbf{+1800}\\
Le chimiste William Nicholson et le chirurgien Anthony Carlisle utilisent l'électrolyse pour séparer l'eau en hydrogène et oxygène. La même année, l'astronome Willian Herschel découvre le rayonnement infrarouge et le physicien Allessandro Volta invente la première pile électrique.

\textbf{+1799}\\
Le mathématicien Gaspard Monge publie son ouvrage de géométrie descriptive. Il en est considéré comme l'inventeur. Premières preuves satisfaisantes mais incomplètes du théorème fondamental de l'algèbre par le mathématicien Johann Carl Friedrich Gauss.

\textbf{+1798}\\
Le mathématicien Carl Friedrich Gauss donne une démonstration rigoureuse du théorème de d'Alembert (théorème fondamental de l'algèbre). La même année, le physicien Benjamin Thompson a l'idée que la chaleur est une forme d'énergie et le physicien et chimiste Henry Cavendish mesure la valeur de la constante gravitationnelle. L'économiste Thomas Malthus énonçe sa loi de population (le modèle exponentiel).

\textbf{+1797}\\
Le mathématicien Caspar Wessel associe des vecteurs aux nombres complexes et étudie l'interprétation géométrique des opérations sur les nombres complexes.

\textbf{+1793}\\
L'assemblée nationale de la République Française instaure le système métrique.

\textbf{+1789}\\
Le physicien et chimiste Antoine Lavoisier énoncé le principe de conservation de la masse.

\textbf{+1788}\\
Dans son \textit{Méchanique Analitique}, le mathématicien et phyisicien Joseph-Louis Lagrange formule une nouvelle façon d'étudier la mécanique classique (par Isaac Newton pour le rappel) en utilisant le principe de moindre action. La même année, l'Académie des sciences approuve la création d'un système de mesure universel, le futur système métrique. Ce projet sera également approuvé par l'Assemblée nationale française en 1790, qui donnera la première définition du mètre.

\textbf{+1787}\\
Le physicien, chimiste et inventeur Jacques Alexandre César Charles détermine expérimentalement que le volume d'une masse fixe d'un gaz à pression constante est proportionnel à la température.

\textbf{+1786}\\
L'astronome William Herschel fait une description précise de notre galaxie.

\textbf{+1785}\\
Le physicien Charles-Augustin Coulomb prouve que les forces entre charges électriques et entre aimants s'exercent en raison inverse du carré de la distance.

\textbf{+1783}\\
L'ecclésiastique et philosophe naturaliste John Michell, dans un article des \textit{Philosophical Transactions} de la Royal Society de Londres, lu le 27 novembre 1783, proposa d'abord l'idée qu'il existait des Trous Noirs, qu'il appela des «étoiles sombres». Quelques années après que Michell eut inventé le concept des trous noirs, le mathématicien français Pierre-Simon Laplace suggéra essentiellement la même idée dans son livre de 1796, \textit{Exposition du Système du Monde}. 

\textbf{+1782}\\
Le mathématicien, physicien et astronome Pierre-Simon de Laplace introduit la "transformée de Laplace", une transformation qui permet la résolution d'équations différentielles en physique (ou facilite leur résolution).

\textbf{+1781}\\
Le chimiste et physicien Joseph Priestley crée de l'eau par combustion d'hydrogène et d'oxygène ce qui démontre que l'eau n'est pas un élément fondamental comme on le pensait depuis Aristote.

\textbf{+1778}\\
Les physiciens et chimistes Carl Scheele et Antoine Lavoisier découvrent que l'air est composé essentiellement d'azote et d'oxygène.

\textbf{+1777}\\
Le physicien et mathématicien Leonhard Euler introduit la lettre $\mathrm{i}$ pour la partie imaginaire des nombres complexes.

\textbf{+1776}\\
Le philosophe moraliste et pionnier de l'économie politique Adam Smith publie son \textit{Une enquête sur la nature et les causes de la richesse des nations}.

\textbf{+1774}\\
Le mathématicien, astronome et physicien Pierre-Simon de Laplace explicite l'intégrale d'Euler. La même année, le théologien, pasteur dissident, philosophe naturel, pédagogue et théoricien de la politique Joseph Priestley fit sa principale découverte, celle de l'oxygène.

\textbf{+1772}\\
Le mathématicien, mécanicien et astronome Joseph-Louis Lagrange étudie le problème des trois corps et découvre les points de libration appelés aujourd'hui "points de Lagrange"..

\textbf{+1770}\\
Dans son \textit{Mémoire sur les équations aux différences partielles}, le philosophe, mathématicien et premier politologue, Jean-Marc Antoine de Caritat, marquis de Condorcet, introduit le symbole des dérivées partielles $\partial$.

\textbf{+1769}\\
Dans son \textit{Institutiones calculi integralis}, le mathématicien Leonhard Euler étudie pour la première fois les doubles intégrales, les calcule par intégration successive et changements de variable. Ces méthodes seront généralisées aux intégrales triples par le mathématicien, ingénieur et astronome Joseph-Louis Lagrange, qui donne aussi la formule générale pour le changement de variables (déterminant du jacobien).

\textbf{+1766}\\
Le physicien et chimiste Henry Cavendish découvre et étudie l'hydrogène.

\textbf{+1764}\\
Les chaleurs latente et spécifique sont décrites pour la première fois par le physicien et chimiste Joseph Black. Il est également le premier à distinguer nettement température et quantité de mouvement. La même année, le physicien et mathématicien Leonhard Euler examine l'équation aux dérivées partielles pour la vibration d'un tambour circulaire et trouve l'une des fonctions de Bessel en tant que solution.

\textbf{+1763}\\
Un article posthume de mathématicien et pasteur Thomas Bayes met en évidence qu'il a découvert ce qui est appelée encore aujourd'hui le "théorème de Bayes".

\textbf{+1757}\\
Le physicien et mathématicien Leonhard Euler fonde l'hydrodynamique moderne.

\textbf{+1756}\\
Le mathématicien, ingénieur et astronome Joseph-Louis Lagrange développe la mécanique analytique basée sur son invention du calcul des variations indépendamment de Leonhard Euler.

\textbf{+1755}\\
Le physicien et mathématicien Leonhard Euler introduit la lettre grecque sigma majuscule ($\Sigma$) pour le symbole de la somme.

\textbf{+1753}\\
Dans son étude de 1749 sur les mouvements de la Terre, Leonhard Euler obtient des équations différentielles pour les éléments orbitaux et, en 1753, il a appliqué la méthode de variation des constantes à son étude des mouvements de la lune.

\textbf{+1750}\\
Le mathématicien Gabriel Cramer établit la règle de Cramer pour la résolution de systèmes linéaires.

\textbf{+1749}\\
L'astronome et physicien Jean le Rond D'Alembert développe le premier modèle de précession des équinoxes basé sur la théorie de la gravitation de Newton et donne une piste de solution pour le problème des trois corps.

\textbf{+1748}\\
Dans son \textit{Introductio in Analysin Infinitorum}, le mathématicien Leonhard Euler introduit le concept de fonction (défini comme toute composition d'expression algébrique et analytique), définit les concepts de fonctions paires et impaires, popularise l'utilisation des symboles $e$ et $\pi$, prouve l'identité d'Euler et définit la fonction $\Gamma$ généralisant la factorielle.

\textbf{+1746}\\
L'encyclopédiste Jean le Rond D'Alembert donne la première preuve (acceptable mais qui sera corrigée plus tard) du théorème fondamental de l'algèbre. L'année d'après (1747) il publie l'équation des cordes vibrantes qui a été le premier exemple de l'équation des ondes. Cela fait de D'Alembert, l'un des fondateurs de la physique mathématique.

\textbf{+1744}\\
Le philosophe, mathématicien, physicien, astronome et naturaliste Pierre Louis Moreau de Maupertuis énonce le principe de moindre action qui sera formalisé mathématiquement 22 ans plus tard par le mathématicien, mécanicien et astronome Joseph-Louis Lagrange. La même année, le physicien et mathématicien Leonhard Euler montre l'existence des nombres transcendants.

\textbf{+1742}\\
L'astronome Anders Celsius définit sa propre unité de mesure de la température.

\textbf{+1739}\\
Le physicien et mathématicien Leonhard Euler résout les équations différentielles linéaires homogènes à coefficients constants.

\textbf{+1738}\\
Le médecin, physicien et mathématicien Daniel Bernoulli publie un livre sur l'hydrodynamique introduisant la théorie cinétique des gaz et le fameux théorème de Bernoulli (balance de pression). La même année dans son \textit{Doctrine du hasard}, le mathématicien Abraham De Moivre introduit la distribution gaussienne comme un moyen d'approximation de la loi binomiale pour un grand nombre d'expériences et démontre une version partielle du théorème de la limite centrale.

\textbf{+1737}\\
Le physicien et mathématicien Leonhard Euler résout le problème de théorie des graphes relatif aux ponts de Königsberg. La résolution de ce problème est considérée comme le premier théorème de la théorie des graphes. Il établit par la même occasion la "formule d'Euler" liant le nombre de sommets, d'arêtes et de faces d'un polyèdre convexe, et donc d'un graphe planaire.

\textbf{+1736}\\
L'inventeur Jonathan Hulls dépose le premier brevet d'un bateau propulsé par une machine à vapeur.

\textbf{+1734}\\
Le physicien et mathématicien Leonhard Euler introduit la notation $f(x)$ pour une fonction appliquée à l'argument $x$.

\textbf{+1733}\\
Le mathématicien Geralamo Saccheri étuide ce que serait la géométrie si les 5ème postulat d'Euclide était faux.

\textbf{+1729}\\
Le teinturier et amateur de physique et d'astronomie Stephen Gray est le premier à découvrir la transmission de l'électricité dans des matériaux qu'il appellera des "conducteurs".

\textbf{+1727}\\
Le physicien et mathématicien Leonhard Euler introduit la notation moderne des fonctions trigonométriques et la lettre $e$ pour la base du logarithme naturel (également connue occasionnellement sous le nom de "nombre d'Euler").

\textbf{+1724}\\
Le mathématicienAbraham De Moivre étudie les statistiques de mortalité et fonde la théorie des annuités sur la vie.

\textbf{+1715}\\
Le mathématicien Brook Taylor publie les outils permettant de faire des intégrations par parties ainsi que des développements en série de fonctions (les fameuses séries de Taylor).

\textbf{+1714}\\
Le mathématicien Brook Taylor dérive la fréquence fondamentale de vibration d'une corde tendue en fonction de sa tension et de sa densité linéique en résolvant une équation différentielle ordinaire.

\textbf{+1713}\\
Le mathématicien et physicien Jacques Bernoulli publie les principes rigoureux de base du calcul des probabilités et statistique.

\textbf{+1705}\\
L'astronome Edmund (ou Edmond) Halley prédit avec une erreur quasi négligeable à l'aide du calcul que la comète qui est passée près de la Terre en 1682 repassera en 1758.

\textbf{+1704}\\
Le physicien et mathématicien Isaac Newton constate expérimentalement que la lumière blanche se compose de multiples couleurs. Il suppose également qu'un rayon lumineux est formé de corpuscules.

\textbf{+1701}\\
Dans son \textit{Explication de l'Arithmétique Binaire}, Gottfried Wilhelm Leibniz introduit l'arithmétique binaire (Leibniz a peut-être été le premier informaticien et théoricien de l'information). Il a anticipé l'interpolation lagrangienne et la théorie de l'information algorithmique. Son \textit{Calculus ratiocinator} a anticipé les aspects de la machine universelle de Turing. En 1961, le mathématicien et philosophe Norbert Wiener a suggéré que Leibniz soit considéré comme le saint patron de la cybernétique.

\textbf{+1698}\\
Le mathématicien et physicien Jacques Bernoulli pose clairement le problème de la courbe de brachistochrone (qui appartient à la famille des courbes cycloïdes) et propose une solution. La même année, le mathématicien Guillaume de L'Hopital énonce sa règle pour l'examen des formes indéterminées. Gottfried Leibniz propose l'utilisation du point $\cdot$ pour désigner la multiplication, au lieu de la croix $\times$ qui est trop facilement confondue avec la variable $x$ dans les équations.

\textbf{+1693}\\
L'astronome et ingénieur Edmund Halley découvre la relation qui lie la focale d'une lentille avec la distance de l'image à son axe et de l'objet réel à son axe. La même année, il construit la première table statistique de mortailté liant le taux de décès à l'âge.

\textbf{+1691}\\
Le philosophe et mathématicien Gottfried Leibniz découvre une technique de séparation de variables pour les équations différentielles ordinaires.

\textbf{+1690}\\
La théorie ondulatoire de la lumière est avancée par le physicien et astronome Christian Huygens ; la même année le physicien et mathématicien Jean Bernoulli développe le calcul exponentiel et trouve l'équation de la chaînette. La même année, le mathématicien et physicien Jacques Bernoulli (frère de Jean Bernoulli) développe le calcul intégral.

\textbf{+1687}\\
Le physicien et mathématicien Isaac Newton publie un ouvrage où il explique la force de gravité et les orbites des planètes. Il énonce également les trois lois de la dynamique. Il s'agit de la première révolution scientifique (avant la relativité restreinte/générale et la physique quantique).

\textbf{+1685}\\
Le philosophe et mathématicien Gottfried Leibniz résout les systèmes linéaires en usant sans justification théorique de matrices et de déterminants.

\textbf{+1682}\\
Le physicien et mathématicien Isaac Newton établit la loi de la gravitation qui porte aujourd'hui son nom.

\textbf{+1679}\\
Le philosophe et mathématicien Gottfried Leibniz introduit l'arithmétique binaire et met au point une machine à calculer qui effectue les 4 opérations. La même année, le physicien, mathématicien et inventeur Denis Papin montre expérimentalement l'influence de la pression atmosphérique sur le point d'ébullition de l'eau.

\textbf{+1678}\\
Le mathématicien, astronome et physicien Christian Huygens postule le son principe du front d'onde.

\textbf{+1676}\\
Le physicien Robert Hooke énonce que l'étirement d'un ressort est proportionnel à la tension.

\textbf{+1675}\\
L'astronome Olaus Roemer fait des mesures précises de la vitesse de la lumière. La même année, le chimiste apothicaire Nicolas Lemery écrit \textit{Cours de chymie} qui est considéré comme le premier grand traité de chimie où les mélanges sont définis, la première théorie des bases et des acides, etc. et le physicien et astronome Isaac Newton invente un algorithme pour calculer les racines de fonctions.

\textbf{+1673}\\
Le philosophe et mathématicien Gottfried Leibniz invente son calcul différentiel, introduit le symbole $\int $ et utilise le terme "test de convergence" pour les séries alternées et utilise pour la première fois la définition de la convergence. Leibniz utilise aussi la notation $\mathrm{d}y / \mathrm{d}x$ et $\mathrm{d}x$, $\mathrm{d}x$.

\textbf{+1671}\\
Première tentative de calcul des rentes viagères (comparable à l'assurance-vie) par le politicien Johan de Witt en collaboration avec le mathématicien Christian Huygens et premiers calculs de l'espérance de vie.

\textbf{+1670}\\
Le mathématicien John Wallis introduit les symboles $\le$ et $\ge$.

\textbf{+1669}\\
Le mathématicien, astronome et physicien Christian Huygens publie ses résultats sur l'observation de la conservation de l'énergie cinétique devenant textuellement le découvreur du concept d'énergie cinétique. La même année, dans son manuscrit, le physicien et astronome Isaac Newton donne la première description de la méthode de Newton qui permet de trouver des approximations de racines fonctionnelles par processus itéré. La description de Newton ne s'applique qu'aux polynômes et n'utilise pas la notion de dérivée (le manuscrit sera publié en 1711).

\textbf{+1668}\\
Le physicien et astronome Isaac Newton réalise le premier télescope à réflexion et la même année le mathématicien John Wallis suggère la loi de conservation de la quantité de mouvement.

\textbf{+1667}\\
Dans son \textit{Vera Circuli et Hyperbolae Quadratura} le mathématicien et astronome James Gregory donne la première preuve du théorème fondamental de l'Algèbre et découvre indépendamment les séries de Taylor. Ébauche du concept de nombre transcendantal élaboré par rapport au problème de la quadrature du cercle.

\textbf{+1665}\\
Le physicien Isaac Newton formule les trois lois de la mécanique. Il jette les bases du calcul différentiel, ces techniques lui permettant à partir de l'expression d'une force inverse au carré de la distance, de retrouver la forme générale des lois de Kepler.

\textbf{+1664}\\
Le physicien Isaac Newton commence à travailler sur le calcul différentiel et intégral.

\textbf{+1661}\\
Le fondateur de la démographie statistique John Graunt publie la première table de mortalité ; la même année le physicien et chimiste Robert Boyle détermine les lois de compressibilité des gaz portant aujourd'hui son nom attaché parfois à celui du physicien Edme Mariotte  qui redécouvrit quelques années après les mêmes lois.

\textbf{+1659}\\
Le mathématicien, astronome et physicien Christian Huygens découvre la formule de l'isochronisme rigoureux (lorsque l'extrémité du pendule parcourt un arc de cycloïde, la période d'oscillation est constante quelle que soit l'amplitude).

\textbf{+1658}\\
Le mathématicien, astronome et physicien Christian Huygens découvre expérimentalement que les balles placées n'importe où sur une cycloïde renversée atteignent le point le plus bas de la cycloïde dans le même temps et ainsi montre expérimentalement que l'isochronisme de la cycloïde.

\textbf{+1657}\\
Le juriste et mathématicien Pierre de Fermat énonce son "principe de Fermat" en optique comme quoi la lumière se propage d'un point à un autre sur des trajectoires telles que la durée du parcours soit localement minimale.

\textbf{+1655}\\
Le mathématicien, astronome et physicien Christian Huygens est le premier à utiliser le concept d'espérance en probabilités. Le mathématicien John Wallis introduit le symbole $\infty$ dans son \textit{Mathesis Universalis}.

\textbf{+1654}\\
Le mathématicien, physicien, inventeur, philosophe, moraliste et théologien Blaise Pascal et le juriste et mathématicien Pierre de Fermat créent la théorie des probabilités.

\textbf{+1650}\\
La plus ancienne institution scientifique nationale du monde, la Royal Society, est fondée à Londres. Il établit la preuve expérimentale comme l'arbitre de la vérité.

\textbf{+1644}\\
Le physicien et mathématicien Evangelista Torricelli a l'idée de substituer du mercure à de l'eau dans l'expérience dite de Torricelli de mise en évidence du "grosso-vido"; suivront plus tard les travaux du mathématicien, physicien, inventeur, philosophe, moraliste et théologien Pascal Blaise (expérience du Puy de Dôme).

\textbf{+1638}\\
Le mathématicien, géomètre, physicien et astronome Galileo Galilée publie la relation mathématique définissant la période du pendule simple.

\textbf{+1637}\\
Le philosophe et mathématicien René Descartes renomme les inconnues $x$, $y$, $z$ et les paramètres $a$, $b$, $c$ et étend l'usage de l'algèbre aux longueurs et au plan, créant avec le juriste et mathématicien Pierre de Fermat la géométrie analytique. La même année, toujours René Descartes, détermine quantitativement les angles primaires et secondaires des arc-en-ciels respectivement à l'inclinaison du Soleil.

\textbf{+1631}\\
Le mathématicien Thomas Harriot introduit, dans une publication posthume, les symboles $>$ et $<$. La même année le mathématicien et théologien William Oughtred donne pour la première fois le symbole multiplié $\times$ et le symbole $\pm$.

\textbf{+1629}\\
L'avocat et mathématicien Pierre de Fermat développe un calcul différentiel rudimentaire. La même année, dans son \textit{Invention nouvelle en algèbre}, Albert Girard déclare, sans preuves, pour la première fois le théorème fondamental de l'algèbre (un polynôme de degré $ n $ a $ n $ racines complexes distinctes ou non distinctes) en utilisant des nombres complexes.

\textbf{+1626}\\
Publication de tables de sinus, tangente et sécante par l'ingénieur Albert Girard et utilisation des abréviations $\sin$, $\cos$ et $\tan$ par ce dernier.

\textbf{+1624}\\
Invention du premier thermomètre (dont les graduations ne sont bien évidemment pas normalisées...) par le médecin Santorio Santorio.

\textbf{+1621}\\
L'astronome et physicien Willebrord Snell découvre que l'angle de réfraction de la lumière est déterminé par le sinus de l'angle formé par la lumière incidente avec la normale au dioptre.

\textbf{+1620}\\
L'ingénieur Francis Thomas Bacon défend et documente la méthode expérimentale (ie méthode scientifique) et mène de nombreuses observations sur la chaleur. Il suggère que la chaleur est reliée au mouvement.

\textbf{+1619}\\
L'astronome et mathématicien Johannes Kepler a fini de publier les trois lois relatives au mouvement des planètes.

\textbf{+1614}\\
Le mathématicien John Napier invente les logarithmes, qui apportent les opérations de multiplication et de division à de simples additions ou soustractions.

\textbf{+1611}\\
L'astronome et mathématicien Johannes Kepler découvre la réflexion interne totale, la loi de réfraction aux petits angles de réfraction et l'optique des lentilles minces.

\textbf{+1610}\\ 
Dans son \textit{Sidereus Nuncius}, Galileo Galilée rapporte les premières observations au télescope: la découverte des satellites de Jupiter, la confirmation que la voie lactée est constituée d'étoiles, la découverte des anneaux de Saturne. Ces observations auront un effet important car elles contredisent certaines des idées des modèles de l'Univers de l'époque et des écrits bibliques.

\textbf{+1609}\\
Dans son \textit{Astronomia Nova}, l'astronome et mathématicien Johannes Kepler explique les deux premières lois de Kepler sur le mouvement des planètes.

\textbf{+1608}\\
L'opticien Hans Lippershey invente le télescope qui sera utilisé et amélioré (avec une qualité aléatoire) l'année suivante par le mathématicien, géomètre, physicien et astronome Galileo Galilée pour confirmer les théories de Copernic.

\textbf{+1604}\\
Dans une lettre à Paolo Sarpi, Galilée énonce la loi de la chute des corps: la distance parcourue est proportionnelle au carré du temps de chute.

\textbf{+1603}\\
Le mathématicien et astronome Thomas Harriot détermine la façon de calculer qualitativement la surface d'un triangle sphérique.

\textbf{+1596}\\
L'histoire quantitative des étoiles variables commence avec les observations de l'apparition et de la disparition de l'omicron Ceti (Mira) par David Fabricius 1596. L'étoile variable à éclipses Algol a été notée pour la première fois par Gernian Montanari en 1667, mais sa période de 2.867 jours n'a été mesurée qu'à partir des travaux de 1783 de Nathaniel Pigott, John Goodricke et Johann Georg Palitzsh. Cependant, l'idée de variabilité d'Algol de Miras était connue dans l'Antiquité par les Babyloniens entre 1895 et 539 avant JC (voir Schaumberger, J. 1935 \textit{Sternkunde und Sterndienst in Babel}, vol. 3, Verlag der Aschendorffschen Verlagsbuchhandlung) et les Chinois (250 avant JC à 25 après JC).

\textbf{+1591}\\
Le mathématicien François Viète ouvre une nouvelle période de l'algèbre en faisant opérer les calculs sur des lettres, utilisant les voyelles pour désigner les inconnues et les consonnes pour les paramètres. Par ailleurs, il donne le développement du binôme de Newton.

\textbf{+1590}\\
L'astronome Galileo Galilei démontre expérimentalement que tous les corps en chute libre ont une accélération identique. La même année, les opticiens Hans et Zacheraius Janssen créent le premier microscope en associant plusieurs lentilles ce qui définira les débuts de la biologie et la médecine scientifique.

\textbf{+1586}\\
L'ingénieur et physicien Simon Stevin démontra la méthode du parallélogramme des forces et découvre que la pression d'un liquide sur le fond d'un récipient est indépendante de sa forme, et aussi de la surface du fond; elle dépend seulement de la hauteur d'eau dans le récipient. Il donna aussi la mesure de la pression sur n'importe quelle portion du côté d’un récipient.

\textbf{+1576}\\
L'astronome Tycho Brahé observe une nouvelle étoile dans la constellation de Cassiopée et construit un observatoire dans l'île de Hveen.

\textbf{+1572}\\
Le mathématicien Rafaelle Bombelli donne une formulation des nombres complexes et les règles de calculs effectifs.  Il introduisit les termes più di meno (pdm) et meno di meno (mdm) pour représenter $+\mathrm{i}$ et $-\mathrm{i}$.

\textbf{+1548}\\
Le mathématicien et physicien Simon Stevin écrit les puissances du dixième cernées d'un exposant. Il donne la première écriture des vecteurs. 

\textbf{+1545}\\
Le mathématicien Ludovico Ferrari donne la solution des équations de degré $4$ (connue sous le nom de "formule de Cardan").

\textbf{+1543}\\
L'ouvrage de l'astronome Nicolas Copernic résumant 26 ans de recherches et d'observations est publié et met clairement en avant que le système héliocentrique de Ptolémée n'est pas valide.

\textbf{+1536}\\
Le mathématicien Niccolò Fontana développe la science de la balistique.

\textbf{+1530}\\
Le mathématicien et physicien Robert Recorde introduit le signe $=$ et le mathématicien Michael Stifel développe une première forme de notation algébrique.

\textbf{+1525}\\
Le mathématicien Christoff Rudolff introduit la notation des racines carrées $\sqrt{\phantom{a}}$.

\textbf{+1515}\\
Publication des travaux de Scipione del Ferro où il trouve une formule donnant la solution générale des équations polynomiales de degré $ 3 $. Ces solutions impliquent la manipulation implicite de nombres imaginaires

\textbf{+1510}\\
Le peintre, graveur et mathématicien Albert Dürer développe les bases de la géométrie descriptive et de la perspective.

\textbf{+1500}\\
Le mathématicien italien Scipione del Ferro parvient pour la première fois à une résolution algébrique d'un grand type d'équations du troisième degré.

\textbf{+1490}\\
Le peintre, sculpteur, architecte, musicien, mathématicien, inventeur, anatomiste, géologiste, cartographe, botaniste et écrivain... Leonardo da Vinci décrit le phénomène de capillarité.

\textbf{+1489}\\
Dans son \textit{Behende und hupsche Rechnung auf allen kauffmanschafft} la mathématicien Johannes Widmannwe introduit les symboles "$ + $" et "$ - $" sont introduits pour la première fois (avant plus était noté "P" et le moins "M").

\textbf{+1464}\\
Dans le \textit{Triparty en la science des nombres} par la mathématicien Nicolas Chuquet nous trouvons la première utilisation des puissances négatives et de la puisse nulle ainsi que l'énoncé de la propriété des exposants que nous utilisons encore $x^{n + m} = x^n x^m$.

\textbf{+1420}\\
Le mathématicien et astronome Jamshīd Al- Kāshī calcule et observe les éclipses solaires de 1406, 1407 et 1408. Il serait également le premier à utiliser la notation décimale en arithmétique et dans les chiffres arabes.

\textbf{+1400}\\
Le mathématicien et astronome Jamshīd Al-Kāshī développe une première forme de la méthode Regula falsi de Newton.

\textbf{+1200-1400}\\
Madhava et l'Ecole du Kerala (Inde) découvrent plusieurs séries infinies pour des nombres comme $\pi$ et des valeurs spécifiques de fonctions trigonométriques - celles-ci devancent celles des Européens sur le calcul différentiel et intégral et la série de puissances.

\textbf{+1350}\\
\textit{Tractatus de configurationibus qualitatum et motuum} d'Oresme est un brouillon de géométrie utilisant des coordonnées, et utilise des axes pour différentes tailles, ce qui est une étape importante dans la transition de la science qualitative basée principalement sur Aristote à la science quantitative. Preuve du théorème de vitesse moyenne, qui anticipe les résultats de Gallilée sur le mouvement rectiligne uniforme et les corps en chute libre en liant la zone sous la courbe de la vitesse à la position dans un graphique.

\textbf{+1303}\\
\textit{Siyuan Yujian} (traduction: \textit{Précieux miroir des éléments}) de Zhu Sjijie décrit la méthode d'élimination pour résoudre des systèmes d'équations contenant jusqu'à quatre inconnues et jusqu'au degré $14$ pour une certaine forme d'équations. On y trouve aussi la définition du triangle de Pascal et les formules de sommations pour certaines séries.

\textbf{+1300}\\
Le philosophe, poète, théologien, missionnaire, apologiste chrétien et romancier Raymon Lulle développe une machine géométrique (inutile) pour automatiser la logique théiste. Cette idée influencera le  philosophe, scientifique, mathématicien, logicien, diplomate, juriste, bibliothécaire Gottfried Wilhelm Leibniz dans sa recherche d'un langage universel pour le raisonnement, recherche qui l'amènera à s'intéresser à l'écriture chinoise et à l'arithmétique binaire. Les idées de Lulle anticipent les idées modernes des systèmes de déduction formelle.

\textbf{+1269}\\
Le savant Pierre de Maricourt invente les expressions de la magnétique "pôle nord" et "pôle sud" et il a été le premier qui a écrit que les pôles opposés d'un aimant s'attirent.

\textbf{+1268}\\
Le philosophe, savant et alchimiste Roger Bacon publie des propositions pour réformer l'école, argumentant que pour étudier la nature, l'utilisation des observations des mesures est la seule base rigoureuse de l'expérimentation et de la vérification tout en affirmant ainsi la nécessité des mathématiques.

\textbf{+1200}\\
Le mathématicien Jordan de Nemore introduit la notation des inconnues par des symboles.

\textbf{+1150}\\
Création de la notation moderne pour les fractions (barre horizontale) par Al-Hassãr. A la même époque, nous avons la traduction latine du traité de 820 par Al-Khwarizmi sur le calcul indien qui permet au système décimal et à l'utilisation du zéro de se propager en Europe et aussi l'écrivain et traducteur Gérard de Crémone publie une traduction en latin de la version arabe de l'Almagest de Ptolemée, le nom "sinus" vient de cette traduction ...

\textbf{+1121}\\
L'astronome, physicien, biologiste, chimiste, mathématicien et philosophe Abu al-Fath Khāzini publie un livre dans lequel il propose que la gravité et l'énergie potentielle gravitationnelle varient selon la distance du centre de la Terre. Il fait aussi une distinction entre la force, la masse et le poids. Il invente aussi plusieurs instruments scientifiques, y compris une balance romaine et une balance hydrostatique. Il introduit aussi des méthodes scientifiques expérimentales à la statique et la dynamique, les unifie dans la science de la mécanique et combine l'hydrostatique avec la dynamique pour créer l'hydrodynamique.

\textbf{+1114}\\
Le mathématicien Bhaskara offre un résumé complet de la mathématique hindoue, telle qu'elle s'est développée du 5ème au 7ème siècle de notre ère. Ainsi il  reconnaît les racines carrées négatives, résout des équations quadratiques à plusieurs inconnues, des équations d'ordre supérieur comme celles de Fermat ainsi que les équations du second degré générales. Il a également été un pionner dans le principe du calcul différentiel de près de $500$ ans relativement à Isaac Newton et Gottfried Wilhelm Leibniz.

\textbf{+1100}\\
Le philosophe et physicien Hibat Allah Abu'l-Barakat al-Baghdaadi est le premier à nier l'idée d'Aristote selon laquelle une force constante produit un mouvement uniforme ce qui préfigure la seconde loi de Newton sur le mouvement. Comme Newton, il a décrit l'accélération comme étant la variation de vitesse.

\textbf{+1037}\\
Le mathématicien, physicien et philosophe Ibn al-Haytham est conscient de la magnitude de l'accélération due a la gravité. Il découvre la loi de l'inertie, connue aujourd'hui comme la première loi du mouvement de Newton.

\textbf{+1030}\\
Le philosophe, écrivain, médecin et scientifique Abu 'Ali al-Husayn Ibn Abd Allah Ibn Sina (connu en Occident sous le nom de Avicenne) note que si la perception de la lumière est due à l'émission d'une sorte de particules par une source lumineuse, la vitesse de la lumière doit être finie. Il a également fourni une explication sophistiquée pour le phénomène de l'arc en ciel.  Le mathématicien, astronome, physicien, érudit, encyclopédiste, philosophe, astrologue, voyageur, historien, pharmacologue Abū al-Rayhān Muhammad ibn Ahmad  Al-Biruni, et plus tard l'astronome Abu al'Fath Khāzini, ont été les premiers à appliquer des méthodes scientifiques expérimentales dans la mécanique, en particulier les domaines de la statique et la dynamique, pour déterminer le poids spécifique, tels que ceux basés sur la théorie des équilibres et de pesage.

\textbf{+1021}\\
Le philosophe, mathématicien et physicien Ibn al-Haytham considéré comme le père de l'optique et un pionnier de la méthode scientifique explique correctement la lumière et la vision, et introduit la méthode scientifique expérimentale, jetant les bases de la physique expérimentale. Il discute aussi de la psychologie expérimentale et décrit les divers instruments d'optique comme la chambre noire. Il a été capable d'estimer la largeur de l'atmosphère avec une précision de $1$ [km], il a défini les principes d'inertie (première loi de Newton), de moment linéaire et il a calculé $\sum_ {k = 1}^n k^4=\frac{n(2n+1)(n+1)(3n^2+2n-1)}{30}$.

\textbf{+1019}\\
Le mathématicien, astronome et physicien Abū Rayhān Al-Biruni a observé et décrit l'éclipse solaire du 8 avril 1019, ainsi que l'éclipse lunaire du 17 Septembre 1019, en détail ; il a donné la localisation exacte des étoiles lors de l'éclipse lunaire. Il invente l'astrolabe orthographique et le planisphère.

\textbf{+1010}\\
Le mathématicien Al-Sijistani invente le Zuraqi, un astrolabe unique conçu pour un modèle planétaire héliocentrique dans lequel la Terre est en mouvement plutôt que le ciel.

\textbf{+1000}\\
Le mathématicien, physicien et astronome Abu Sahl al-Qouhi découvre que la lourdeur des corps varie en fonction de leur distance du centre de la Terre, et résout des équations plus élevées que le deuxième degré. Durant le même décennie, le mathématicien et ingénieur Al'Karkhi écrit un livre contenant les premières preuves connues par induction mathématique. Il l'utilise pour démontrer le théorème du binôme, le triangle de Pascal, et la somme des cubes intégrales.

\textbf{+996}\\
L'astrolabe mécanique orienté, comportant 8 roues dentées est inventé par le mathématicien, astronome et physicien Abū Rayhān Al-Biruni qui est aussi l'auteur de travaux sur la sommation de séries et la combinatoire.

\textbf{+980}\\
Abitu al-Wafaa fait le premier calcul des valeurs des fonctions trigonométriques et publie la loi des sinus adaptée pour les triangles sur la sphère. Il a aussi prouvé par induction que $\sum_{k=0}^n k^3=\frac{n^2(n+1)^2}{4}$.

\textbf{+964}\\
Le mathématicien, philosophe et physicien Abd al-Rahman al-Soufi explique le pouvoir grossissant des lentilles et fut un des premiers à se servir d'une méthode d'analyse scientifique qui influencera grandement de futurs scientifiques.

\textbf{+953}\\
Le mathématicien et ingénieur Al-Karkhi définit différents monômes et donne des règles pour les produits de n'importe quels deux d'entre eux. Il a aussi découvert le théorème du binôme pour des exposants entiers.

\textbf{+952}\\
Le mathématicien Abu'l-Hasan al-Uqlidisi modifie les méthodes de calcul pour le système numérique indien pour le rendre possible aux plumes et à l'utilisation du papier. Jusque-là, faire des calculs avec les chiffres indiens nécessitait l'emploi d'une planche.

\textbf{+900}\\
La première référence à un tube d'observation se trouve dans l'oeuvre de l'astronome et mathématicien Al-Battani, et la première description exacte du tube d'observation a été donnée par le mathématicien, astronome, physicien, érudit, encyclopédiste, philosophe, astrologue, voyageur, historien, pharmacologue Al-Biruni, dans une section de son travail  dédiée à vérifier la présence du nouveau croissant de Lune à l'horizon. Bien que ces tubes d'observations préliminaires n'aient pas de lentilles, ils ont permis à un observateur de se concentrer sur une partie du ciel en éliminant les interférences lumineuses. Ces tubes d'observation ont été adoptés ultérieurement en Europe latine, où ils ont influencé le développement du télescope.

\textbf{+880}\\
L'astronome et mathématicien Al-Battani découvre le mouvement de l'apogée du Soleil, calcule les valeurs de la précession des équinoxes et l'inclinaison de l'axe terrestre. Il est à l'origine de la définition de la fonction trigonométrique tangente et cotangente.

\textbf{+820}\\
Le mot "algèbre" naît. Le mathématicien, géographe, astrologue et astronome Muhammad ibn Musa Al'Khwarizmi est souvent considéré comme le père de l'algèbre médiévale, car il dégage celle-ci de l'emprise géométrique. On lui doit aussi le quadrant, l'instrument mural, le quadrant des sinus qui a été utilisé pour résoudre les problèmes trigonométriques et faire des observations astronomiques.

\textbf{+800}\\
Des astronomes inventent le cadran solaire universel et le cadran horaire universel à Bagdad.

\textbf{+780}\\
L'alchimiste Jabir Ibn Hayyan introduit la méthode scientifique expérimentale pour la chimie, ainsi que les appareils de laboratoire tels que l'alambic et des processus tels que la distillation pure, la liquéfaction, la cristallisation, et la filtration. Il à également inventé plus de vingt types d'appareils de laboratoire, ce qui a entraîné la découverte de plusieurs substances chimiques. Il a également développé des recettes pour le verre coloré.

\textbf{+773}\\
Les chiffres arabes (adaptés de l'Inde) font leur première apparition en Europe.

\textbf{+628}\\
Le mathématicien Brahmagupta donne des règles pour résoudre les équations linéaires et quadratiques. Il découvre que les équations quadratiques ont deux racines: la négative et l'irrationnelle et donne la forme moderne de la solution que nous connaissons aujourd'hui. Il donne aussi les règles à calculer avec des signes négatifs (arithmétique des nombres négatifs).

\textbf{+550}\\
Les mathématiciens hindous donnent à zéro une représentation numérique dans un système de notation positionnelle. \textit{Brahmasphutasiddhanta de Brahmagupta} est le premier livre qui fournit des règles pour les manipulations arithmétiques qui s'appliquent à zéro et aux nombres négatifs. Le \textit{Brahmasphutasiddhanta} est le premier texte connu à traiter le zéro comme un nombre à part entière, plutôt que comme un simple chiffre de substitution pour représenter un autre nombre comme cela a été fait par les Babyloniens.

\textbf{+499}\\
Le mathématicien Âryabhat obtient le nombre total de solutions d'un système d'équations linéaires par des méthodes équivalentes aux méthodes modernes, et décrit la solution générale de telles équations. Il fournit également des solutions d'équations différentielles. Il affirme également que la Lune et les objets célestes autres que les étoiles reflètent la lumière du Soleil, il explique correctement les causes des éclipses lunaires et solaires, donne la durée de l'année sidérale à quelques minutes, approxime $\pi$ par $62832/20000$ , calcule le diamètre de la Terre avec plus de précision que Erathostene, décrit le calcul avec le système de numérotation indien, donne la plus ancienne table des sinus pour $24$ angles.

\textbf{+275}\\
Le mathématicien Diophante d'Alexandrie considéré comme le père de l'algèbre étudie les équations à variables rationnelles (incluant donc les équations du second degré) et les équations diophantiennes.

\textbf{+195}\\
\textit{Suàn shù shu} (traduction: \textit{livre sur les nombres et calculs}) est l'un des plus anciens textes mathématiques chinois connus qui contient entre autres, des calculs de sommes de progression géométrique pour l'intérêt.

\textbf{+130}\\
\textit{Compositions mathématiques} (aussi connu sous le nom de \textit{Almagest's}) de Ptolémée d'Alexandrie présente un modèle géométrique du système solaire qui tente de décrire le mouvement des planètes. Le modèle est inspiré par une idée géométrique d'Apollonius et utilise des cercles dont le centre se déplace dans des orbites circulaires. Ce modèle met la Terre au centre du système solaire mais donne une assez bonne description des mouvements observés des différentes étoiles. Ce sera le modèle dominant jusqu'à Copernic. Quelques idées y préfigurent les séries de Fourier qui seront introduites au 19ème siècle.

\textbf{+125}\\
Le \textit{Yale Papyrus Musique} et le \textit{Michigan Papyrus Instrumental} semblent contenir les plus anciens exemples connus de notation musicale.

\textbf{+121}\\
Année correspondant au plus ancien document faisant mention de la pierre magnétique.

\textbf{+120}\\
Le catalogue d'étoiles de Zhang Heng contient $2'500$ étoiles. Zhang Heng décrit correctement la cause des éclipses et démontre que la lune est sphérique.

\textbf{+100}\\
L'ingénieur, mécanicien et mathématicien Héron d'Alexandrie redécouvre (après les chinois) le concept de force. Il invente aussi un système d'engrenages pour soulever des poids utilisant la puissance de la vapeur. Il fait la première description du sextant (mais ne l'a toutefois pas inventé). Son contemporain astronome Claude Ptolémée invente le sextant et décrit l'astrolabe (peut-être inventé par l'astronome, géographe et mathématicien Hipparque) et étudie la réfraction et la réflexion. Pendant le même siècle le mathématicien et philosophe Nicomaque de Gérase définit ainsi les nombres pairs et impairs, les nombres premiers, composés et nombres parfaits.  Toujours au cours du même siècle, la version finale de \textit{Neuf chapitres sur l'art mathématique} (presque $1'180$ pages), écrite sur dix ans par plusieurs auteurs chinois anonymes, contient la première utilisation de nombres négatifs, le chapitre $9$ utilise le théorème de Pythagore, le chapitre $8$ utilise des matrices et l'élimination de Gauss pour résoudre des systèmes d'équations (au moins 1700 ans avant Gauss !!!).

\textbf{+80}\\
L'érudit Wang Ch'ung réalise la première boussole aimantée sur un plateau de laiton.

\textbf{-87}\\
Année correspondant à la datation de la machine d'Anticythère, considérée comme le premier calculateur analogique et la première machine à engrenages (une trentaine!) antique permettant de calculer des positions astronomiques complexes. Le soin et l'adresse avec lesquels cette machine fut réalisée, ainsi que les capacités nécessaires en mécanique et en astronomie remettent en question les connaissances historiques sur les sciences grecques avant sa découverte. En effet, aucun objet de même âge et de même complexité n'était connu dans le monde et il faut attendre près d'un millénaire pour voir apparaître des mécanismes comparables! Le physicien, mathématicien et ingénieur Archimède de Syracuse en est l'hypothétique créateur.

\textbf{-100}\\
Le texte indien \textit{Anuyoga Dwara Sutra} contient plusieurs identités impliquant des racines carrées et des carrés qui semblent impliquer une certaine connaissance des lois des exposants ou des logarithmes. Une identité tirée de ce texte en notation moderne est: $ \sqrt{a}\sqrt{\sqrt {a}} = (\sqrt{\sqrt{a}})^3$.

\textbf{-134}\\
L'astronome, géographe et mathématicien Hipparque de Nicéedécouvre la précession des équinoes.

\textbf{-150}\\
L'astronome, géographe et mathématicien Hipparque est souvent désigné comme le créateur de la trigonométrie et des tables numériques correspondantes. Il calcule le premier la période de révolution du Soleil autour de la Terre (mais les résultats numériques sont en réalité ceux de la rotation de la Terre autour du Soleil) et élabore la théorie des excentriques et des épicycles.

\textbf{-200}\\
Dans le siècle, les chinois auraient inventé l'écluse, le gouvernail, le principe de la machine à vapeur plusieurs centaines d'années avant les occidentaux! Pendant ce siècle, le scientifique et ingénieur Philon de Byzance rédige des traités sur les leviers, la pneumatique, les automates, la traction et les clepsydres.

\textbf{-225}\\
Le géomètre et astronome Apollonius de Perge publie la première étude sur les coniques donnant à l'ellipse, à la parabole et à l'hyperbole les noms que nous leur connaissons. On lui attribue en outre l'hypothèse des orbites excentriques pour expliquer le mouvement apparent des planètes et la variation de vitesse de la Lune.

\textbf{-250}\\
Le mathématicien, physicien et ingénieur Archimède de Syracuse étudie des machines simples tel que le levier, la fameuse vis permettant entre autre de pomper l'eau ("vis d'Archimède") et découvre le principe d'Archimède qui explique la flottaison. Dans la même décennie, l'astronome, géographe, philosophe et mathématicien Ératosthène de Cyrène calcule le diamètre de la Terre à l'aide d'un gnomon et de son ombre et démontra l'inclinaison de l'écliptique.

\textbf{-260}\\
Le mathématicien, physicien et ingénieur Archimedes of Syracuse calcule  avec une précision de deux décimales utilisant des polygones inscrits et circonscrits et calcule la surface sous un segement de parabole.

\textbf{-281}\\
L'astronome et mathématicien Aristarque de Samos fait l'hypothèse que le Soleil occupe le centre du système solaire et utilise la trigonométrie pour estimer le rayon de la Lune et sa distance à la Terre en utilisant l'ombre de la Terre pendant une éclipse Lunaire.

\textbf{-300}\\
Le mathématicien et géomètre Euclide publie ses \textit{Éléments}, où il réorganise toute la connaissance de la géométrie incluant des démonstrations logiques, la construction des 5 solides platoniciens. Dans son \textit{Optica}, il nota que la lumière va en ligne droite et décrit la loi de réflexion. Nous avons aussi que les concepts de conique, ellipse, parabole et hyperbole qui apparaissent dans les travaux de deux mathématiciens de la Grèce antique, à savoir Menaech-mus et Appolonius de Pergé, qui introduisent aussi le concept de tangente. La même période Bhagabati Sutra calcule les permutations et les combinaisons d'ordre $1$, $2$ et $3$.

Dans la cosmologie hindoue, les Manusmriti (1.67–80) et les Puranas décrivent le temps comme cyclique, avec un nouvel univers (planètes et vie) créé par Brahma tous les 8.64 milliards d'années. L'univers est créé, maintenu et détruit au cours d'une période de kalpa (jour de Brahma) d'une durée de 4.32 milliards d'années, et est suivie d'une période de pralaya (nuit) de dissolution partielle d'une durée égale. Dans certains Puranas (par exemple Bhagavata Purana), un cycle de temps plus large est décrit où la matière (mahat-tattva ou utérus universel) est créée à partir de la matière primitive (prakriti) et de la matière racine (pradhana) tous les 622.08 milliards d'années, à partir de laquelle Brahma est né. Les éléments de l'univers sont créés, utilisés par Brahma et complètement dissous dans une période de maha-kalpa (vie de Brahma; 100 de ses 360 jours) d'une durée de 311.04 milliards d'années contenant 36000 kalpas (jours) et pralayas (nuits), et est suivie d'une période maha-pralaya de dissolution complète d'une durée égale. Les textes parlent également d'innombrables mondes ou univers.

\textbf{-310}\\
Le savant Autolycus de Pitane définit le mouvement uniforme tel qu'un objet parcoure une quantité égale de distance en un temps égal.

\textbf{-370}\\
Le philosophe Aristote développe la logique avec une théorie naïve des propositions, des quantités et du raisonnement par inférence. Il a également écrit un traité de météorologie nommé \textit{Meteorologica}. Il ressort clairement de ce traité qu'il comprenait déjà le cycle de l'eau et savait que les nuages étaient constitués d'eau évaporée et sont donc des structures très lourdes prises dans leur globalité.

\textbf{-388}\\
Le philosophe et astronome Héraclide du Pont fait l'hypothèse de la rotation de la Terre sur elle-même afin d'expliquer le mouvement apparent des étoiles au cours de la nuit (mais toujours dans un cadre géocentrique) et suggère que chaque planète est un corps comme la Terre. Il savait également (d'une manière naïve certes!) et a testé expérimentalement comment l'eau salée et l'eau normale étaient séparées et mélangées dans les mers.

\textbf{-400}\\
L'École Stoïcienne développe les propositions composées et les connecteurs logiques: "implication", "et", "ou" et les inférences "Modus ponens" et "Modus tollens".

\textbf{-430}\\
Le philosophe Démocrite d'Abdère avance l'idée selon laquelle la matière se compose de particules identiques et minuscules qu'il appelle des "atomes". En réalité il s'agit plutôt d'une vulgarisation des idées de son maître, le philosophe Leucippe de Milet élaborées dix ans avant. Hippasus, un disciple de Pythagore, aurait donné ce qui est probablement la première preuve rigoureuse de l'irrationalité de $\sqrt{2}$. La preuve utilise un raisonnement de Reductio ad absurdum pour montrer que les côtés d'un carré sont incommensurables avec sa diagonale.

La plus ancienne trace écrite connue de la Camera Obscura est une description du philosophe chinois Han Mozi (environ 470 à environ 391 avant JC). Mozi a correctement affirmé que l'image de la caméra obscura est inversée parce que la lumière se déplace en lignes droites depuis sa source. Au 11ème siècle, le physicien arabe Ibn al-Haytham (Alhazen) a écrit des livres très influents sur l'optique, y compris des expériences avec la lumière à travers une petite ouverture dans une pièce sombre.

\textbf{-500}\\
Les philosophes Leucippe et Démocrite fondent l'atomisme.

\textbf{-540}\\
Le philosope et mathématicien Pythagore étudie la géométrie propositionnelle et la vibration de la corde de la lyre.

\textbf{-600}\\
Le philosophe Empédocle plaide pour une décomposition du monde en quatre éléments fondamentaux: l'eau, la terre, l'air et le feu. Le même siècle, le mathématicien Thalès de Milet met en évidence l'électrostatique en frottant un morceau d'ambre, prédit une éclipse et développe la géométrie du triangle.

Les philosophes grecs, dès Anaximandre, introduisent l'idée d'univers multiples voire infinis. Démocrite a précisé en outre que ces mondes variaient en distance, en taille; la présence, le nombre et la taille de leurs soleils et lunes; et qu'ils sont sujets à des collisions destructrices. Aussi pendant cette période, les Grecs ont établi que la Terre est sphérique plutôt que plate.

\textbf{-750}\\
Manava Sulbar Sutras de Manava trouve l'irrationalité de $\sqrt{2}$ et $\sqrt{61}$ et accepte d'utiliser des nombres irrationnels dans ses calculs. La première université connue (en utilisant strictement la définition de ce mot qui est dérivé du latin \textit{universitas magistrorum et studentium}, qui signifie approximativement \og communauté d'enseignants et d'érudits \fg{}) a été créée à Milet (où Thales a étudié).

\textbf{-800}\\
Les Assyriens utilisent des clepsydres et les Chinois tracent les mouvements planétaires pour leur calendrier.

\textbf{-1295}\\
Vers 1295 av.J.-C., un nouveau mot hiéroglyphique apparaît. Bia-n-pet se traduit littéralement par «fer du ciel». Le nouveau mot a été appliqué à tout le fer métallique à partir de cette époque. Une explication évidente de l'émergence soudaine du nouveau mot serait un événement majeur, tel qu'un grand impact ou une pluie de météorites, observé par la population égyptienne antique. Cela aurait laissé peu d'incertitude quant à l'origine exacte du fer mystérieux et aurait créé une association suffisamment forte entre le fer et le ciel pour que le nouveau mot devienne synonyme de toutes les formes de fer. En 2008, un grand cratère d'impact formé par une météorite de fer a été découvert dans le sud de l'Égypte. Bien que son âge exact soit inconnu, l'archéologie locale suggère qu'il s'est formé au cours des 5000 dernières années, ce qui en fait un événement candidat possible.

\textbf{-1400}\\
Les peuples néolithiques d'Écosse construisent des modèles en pierre des cinq solides de Platon (polyèdres réguliers).

\textbf{-1500}\\
Les indiens développent la théorie des 4 éléments (eau, air, terre feu).

\textbf{-1700}\\
Le mathématicien Apastamba résout les équations linéaires générales et utilise les systèmes d'équations diophantiennes comportant jusqu'à cinq inconnues. Le même siècle, les mathématiciens égyptiens utilisent des fractions simples.

\textbf{-1800}\\
Les origines de l'algèbre peuvent être attribuées aux anciens Babyloniens qui ont développé un système de nombre positionnel qui les a grandement aidés à résoudre leurs équations algébriques rhétoriques. Les Babyloniens n'étaient pas intéressés par des solutions exactes, mais plutôt par des approximations, et ils utilisaient donc couramment l'interpolation linéaire pour approximer les valeurs intermédiaires.  L'une des tablettes les plus célèbres est la tablette Plimpton 322, créée vers 1900–1600 avant JC, qui donne un tableau des triplets de Pythagore et représente certaines des mathématiques les plus avancées avant les mathématiques grecques.

\textbf{-1800}\\\
On savait à l'époque védique et postérieure (\textit{Rigveda}, I, 19.7, I 23.17, I, 32.9,VIII, 6.19 ; VIII, 6.20 ; et VIII, 12.3) (Sarasvati, 2009) que l'eau n'est pas perdue dans les différents processus du cycle hydrologique, à savoir l'évaporation, la condensation, les précipitations, l'écoulement, etc., mais se transforme d'une forme en une autre (transfert d'eau de la Terre à l'atmosphère par le Soleil et le vent).

\textbf{-2000}\\
Les prêtres babyloniens font les premiers relevés des observations célestes.

\textbf{-2300}\\
Les astronomes chinois font les premières observations du ciel.

\textbf{-2500}\\
Les Mésopotamiens imaginent un système de numérotation de position composé de symboles dont la valeur est fonction de leur rang à l'intérieur d'un nombre.

\textbf{-2600}\\
La plus ancienne table mathématique connue de multiplication est gravée pour calculer les surfaces.

\textbf{-3000}\\
Les Chinois et les Babyloniens inventent l'abaque, première machine à additionner. Des concepts géométriques sont développés pour l'arpentage (hypoténuse). C'est aussi la période correspondante au plus ancien outil connu utilisé par les Incas pour enregistrer des nombres grâce aux noeuds sur une corde et aussi à la représentation murale des roues.

\textbf{-3500}\\
Le plus vieux rapport météo est trouvé sur une pierre en Égypte. Les conditions météorologiques inhabituelles décrites sur la dalle étaient le résultat d'une explosion volcanique massive à Thera, l'île actuelle de Santorin en Méditerranée.

\textbf{-3700}\\
Une tablette babylonienne semble contenir les premiers angles trigonométriques remarquables et la preuve que le théorème de Pythagore était déjà connu par les Babyloniens.

\textbf{-4900}\\
Le cercle de Goseck (allemand: Sonnenobservatorium Goseck) pourrait être l'un des plus anciens observatoires solaires du monde.

\textbf{-5000}\\
Le système décimal est utilisé dans l'Egypte ancienne (il semble que le consensus scientifique de datation se situe entre $-6000$ et $-3000$).

\textbf{-5200}\\
Datation au radiocarbone des roues les plus anciennes (roue de Ljubljana Marshes).

\textbf{-8000}\\
Warren Field est l'emplacement d'un calendrier mésolithique. Il comprend des $12$ fosses supposées correspondre aux phases de la Lune et utilisées comme calendrier lunaire. Il est considéré comme le plus ancien calendrier lunaire trouvé. On associe au même  millénaire le fait que les marques d'un os ou de bois sont lentement remplacées par des jetons de différentes formes à compter.

\textbf{-20000}\\
L'os d'Ishango est un outil en os, daté de l'ère du Paléolithique supérieur. C'est une long morceau d'os brun foncé, le péroné d'un babouin, avec un morceau de quartz pointu fixé à une extrémité, peut-être pour la gravure. Il a d'abord été considéré comme un bâton de pointage, car il a une série de ce qui a été interprété comme des marques de pointage sculptées dans trois colonnes exécutant la longueur de l'outil. Il a également été suggéré que les rayures pourraient avoir été créées dans le but d'avoir une meilleure prise sur le manche ou pour une autre raison non-mathématique.

\textbf{-200000}\\
Les revendications pour les premières preuves définitives de la maîtrise du feu par un membre de l'Homo vont de $0.2$ à $1.7$ million d'années. Les preuves de l'utilisation contrôlée du feu par Homo Erectus, qui remontent à $400'000$ ans, bénéficient d'un large soutien scientifique.

	\chapter{Humour}
	\minitoc
	\pagebreak
	Si vous avez des histoires humoristiques dans le genre "scientifique" n'hésitez pas à nous les transmettre. Dans tous les cas, nous vous souhaitons du bon temps (certaines histoires sont en anglais car elles perdent leurs sens en français ou nous ne possédons pas les originaux des images pour modifier les textes).
	
\begin{center}
\textbf{Cette page est transmise avec des électrons 100\% recyclés}
\end{center}

	\section{Situations}
	
	Lors d'un grand jeu télévisé, les trois concurrents se trouvent être un ingénieur, un physicien et un mathématicien. Ils ont une épreuve à réaliser. Cette épreuve consiste à construire une clôture tout autour d'un troupeau de moutons en utilisant aussi peu de matériel que possible.

\begin{itemize}	 
	\item[$-$] L'ingénieur: Regroupe le troupeau dans un cercle, puis décide de construire une barrière tout autour.

	\item[$-$] Le physicien: Construit une clôture d'un diamètre infini et tente de relier les bouts de la clôture entre eux jusqu'au moment où tout le troupeau peut tenir dans le cercle.

	\item[$-$] Le mathématicien: Voyant ceci, construit une clôture autour de lui-même et se définit comme y étant à l'extérieur.
\end{itemize}

\begin{center}\underline{\hspace{5 cm}}\end{center}

Deux hommes se déplaçant en ballon sont perdus dans le désert. Ils aperçoivent un individu en train de méditer à l'ombre d'un arbre.

\begin{itemize}	 
	\item[$-$] "Où sommes-nous, s'il vous plaît ?" lui demandent-ils.
\end{itemize}

Après un long moment de réflexion, l'homme leur répond:

\begin{itemize}	 
	\item[$-$] "Dans un ballon."

	\item[$-$] "Merci, monsieur le mathématicien."
\end{itemize}

L'homme demande étonné:

\begin{itemize}	 
	\item[$-$] "Comment avez-vous su que j'étais mathématicien?"

	\item[$-$] "Pour trois raisons (répond l'aéronaute). Premièrement, vous avez beaucoup réfléchi avant de nous répondre. Deuxièmement, votre réponse est très exacte. Troisièmement, elle ne sert absolument à rien."
\end{itemize}

\begin{center}\underline{\hspace{5 cm}}\end{center}
	
	\begin{center}
		\includegraphics[scale=0.7]{img/humour/research_in_peace.jpg}	
	\end{center}
	
\begin{center}\underline{\hspace{5 cm}}\end{center}

Un ingénieur, un mathématicien et un physicien séjournent une nuit dans un hôtel. Heureusement pour ce gag, un début d'incendie s'enclenche dans chacune de leur chambre.

\begin{itemize}	 
	\item[$-$] Le physiciens se réveille, voit le feu, fait quelques observations méticuleuses et de nombreux calculs sur la couverture de la carte des vins de l'hôtel. Une fois ceci fait, il s'empare de l'extincteur et éteind de le feu de manière très précise en un seul et unique coup et retourne se coucher.

	\item[$-$] L'ingénieur se réveille, voit le feu, fait quelques observations méticuleuses et sur la couverture de la carte du restaurant fait quelques calculs. Une fois ceci fait et après avoir ajouté un facteur de sécurité de $5$, il s'empare de l'extincteur et asperge l'ensemble de la chambre plusieurs fois de suite et retourne se coucher.

	\item[$-$] Le mathématicien se réveille à son tour, fait quelques observation méticuleuses et sur un tableau noir se trouvant dans la chambre fait de nombreux calculs. Soudainement il s'exclame: "une solution existe!". Une fois la solution trouvée, il retourne dans son lit...
\end{itemize}	
\begin{center}\underline{\hspace{5 cm}}\end{center}

Un médecin, un légiste et un mathématicien discutent des mérites comparés d'une épouse et d'une maîtresse.

\begin{itemize}	 
	\item[$-$] Le légiste: "Il vaut mieux avoir une maîtresse. En cas de divorce, une épouse pose de nombreux problèmes légaux."

	\item[$-$] Le médecin: "Il vaut mieux avoir une épouse, car le sentiment de sécurité réduit le stress, et c'est bon pour la santé."

	\item[$-$] Le mathématicien: "Vous avez tous les deux torts. Le mieux est d'avoir les deux. Quand votre femme vous croît chez votre maîtresse, et votre maîtresse chez votre femme, vous pouvez faire des maths."
\end{itemize}	

\begin{center}\underline{\hspace{5 cm}}\end{center}

Un grand homme d'affaires engage un mathématicien, un informaticien et un physicien afin de pouvoir gagner à tous les Tiercés.

\begin{itemize}	 
	\item[$-$] Le mathématicien le premier s'attaque à la tâche, il calcule des matrices à n'en plus finir, pose des axiomes à tout bout de champs et après de longues semaines de Lemmes, théorèmes et conjectures, il conclut que le problème est formellement irrésolvable.

	\item[$-$] Ensuite, l'informaticien heureux d'avoir vu le mathématicien en échec, s'approche de son Cray III et après avoir écrit quantités d'algorithmes en C++ et introduit tous les paramètres et conditions initiales annonce joyeusement qu'il faudra juste quelques centaines d'années pour calculer le résultat de chaque Tiercé...

	\item[$-$] Le physicien, le sourire aux lèvres, informe ses éminents collègues qu'il a la solution. Il s'approche d'un tableau noir et tout en dessinant une sphère commence par dire: "Approximons le cheval par une sphère parfaite..."
\end{itemize}

\begin{center}\underline{\hspace{5 cm}}\end{center}

Lors d'un entretien d'embauche, un chef d'entreprise reçoit quatre ingénieurs: un ayant fait l'École polytechnique, le second HEC, le troisième informaticien, et le dernier sortant de l'université. Celui-ci explique aux quatre candidats qu'en définitive, pour faire marcher une entreprise, il suffit de savoir compter.

Il s'adresse donc au premier d'entre eux, le polytechnicien, et lui dit: "allez-y, comptez..."

\begin{itemize}	 
	\item[$-$] Le polytechnicien: "un... deux... un... deux..." two..."
\end{itemize}


L'homme étonné s'adresse ensuite à l'ingénieur sortant d'HEC: "À vous! Comptez..."

\begin{itemize}	 
	\item[$-$] L'ingénieur sortant d'HEC: "un KiloFranc... deux KiloFrancs, trois KF..."
\end{itemize} 


Il se retourne ensuite, inquiet, vers l'informaticien: "À vous! Comptez..."

\begin{itemize}	 
	\item[$-$] L'informaticien: "0... 1... 0... 1..." 
\end{itemize}

Désespéré, le chef d'entreprise s'adresse au dernier candidat sortant de faculté: "Allez-y, comptez..."

\begin{itemize}	 
	\item[$-$] Le jeune homme commence: "1... 2... 3... 4... 5... 6... 7..." 
\end{itemize}

Le chef d'entreprise rassuré: "continuez, continuez..."

\begin{itemize}	 
	\item[$-$] "8... 9... 10... valet... dame.. roi... " !!
\end{itemize}

	\begin{center}\underline{\hspace{5 cm}}\end{center}

	\begin{center}
		\includegraphics[scale=0.7]{img/humour/work_science_vs_doesnt_work.jpg}	
	\end{center}
	\pagebreak

Plusieurs personnes ont été invitées à résoudre le problème suivant: "Montrer que tous les entiers impairs sont premiers."

\begin{itemize}	 
	\item[$-$] Mathématicien: 3 est un nombre premier, 5 est un nombre premier, 7 est un nombre premier, 9 n'est pas un nombre premier - contre-exemple - la proposition est fausse!

	\item[$-$] Physicien: 3 est premier, 5 est premier, 7 est  premier, 9 est une erreur expérimentale, 11 est premier ...

	\item[$-$] Ingénieur: 3 est premier, 5 est premier, 7 est  premier, 9 est premier, 11 est premier ...

	\item[$-$] Computer Scientist: 3 est premier, 5 est premier, 7 est  premier ... erreur de segmentation

	\item[$-$] Avocats: un est premier, trois est premier, cinq est premier, sept est premier, bien qu'il semble y avoir une preuve prima facie que neuf n'est pas premier, il existe un précédent substantiel pour indiquer que neuf devrait être considéré comme premier. La lette  suivant présente le cas de la primauté de neuf ...

	\item[$-$] Libéraux: Le fait que neuf n'est pas premier indique un environnement culturel défavorisé qui ne peut être corrigé que par un programme d'enrichissement culturel financé par le gouvernement fédéral.

	\item[$-$] Programmeurs informatiques: un est premier, trois est premier, cinq est premier, cinq est premier, cinq est premier, cinq est premier cinq est premier, cinq est premier, cinq est premier ...

	\item[$-$] Professeur: 3 est premier, 5 est premier, 7 est premier, et le reste est laissé comme un exercice pour l'étudiant.

	\item[$-$] Linguiste: 3 est un premier impair, 5 est un premier impair, 7 est un premier impair, 9 est un premier très ...

	\item[$-$] Chimiste: 1 est premier, 3  est premier, 5  est premier ... hey, publions!

	\item[$-$] New Yorker: 3 est premier, 5 est premier, 7 est premier, 9 est ... CE N'EST PAS VOS AFFAIRES!

	\item[$-$] Programmeur: 3 est premier, 5 est premier, 7 est premier, 9 sera corrigé dans la prochaine version!

	\item[$-$] Commercial: 3 est un premier, 5 est un premier, 7 est un premier, 9 - laissez-moi vous faire faire une affaire ...

	\item[$-$] Publicitaire: 3 est un nombre premier, 5 est un nombre premier, 7 est un nombre premier, 11 est un nombre premier, ...

	\item[$-$] Comptable: 3 est premier, 5 est premier, 7 est premier, 9 est premier, déduction de 10\% d'impôt et 5\% d'autres obligations.

	\item[$-$] Statisticien: Essayons plusieurs nombres choisis au hasard: 17 est un nombre premier, 23 est un nombre premier, 11 est un nombre premier ... Ça me va bien!

	\item[$-$] Psychologue: 3 est un nombre premier, 5 est un nombre premier, 7 est un nombre premier, 9 est un nombre premier mais tente de le supprimer ...
	
	\item[$-$] RH: C'est quoi un nombre premier et un nombre impair?
\end{itemize}

	\begin{center}\underline{\hspace{5 cm}}\end{center}
	
Un mathématicien, un ingénieur et un physicien se voient remettre une balle en gomme rouge afin d'en déterminer le volume.

\begin{itemize}	 
	\item[$-$] Le mathématicien: Mesure le diamètre et évalue le résultat de la triple intégrale.

	\item[$-$] Le physicien: Remplis un tonneau d'eau, dépose la balle dans l'eau et mesure le déplacement total de volume.

	\item[$-$] L'ingénieur: Observe le modèle et le numéro de série dans sa table des "balles en gomme rouge".
\end{itemize}		
	\begin{center}\underline{\hspace{5 cm}}\end{center}

Cette histoire est une légende urbaine fort sympathique:

J'ai reçu un coup de fil d'un collègue à propos d'un étudiant. Il estimait qu'il devait lui donner un zéro à une question de physique, alors que l'étudiant réclamait un 20. Le professeur et l'étudiant se mirent d'accord pour choisir un arbitre impartial et je fus choisi. Je lus la question de l'examen: "Montrez comment il est possible de déterminer la hauteur d'un building a l'aide d'un baromètre". L'étudiant avait répondu: "On prend le baromètre en haut du building, on lui attache une corde, on le fait glisser jusqu'au sol, ensuite on le remonte et on calcule la longueur de la corde. La longueur de la corde donne la hauteur du building."

L'étudiant avait raison vu qu'il avait répondu juste et complètement à la question. D'un autre côté, je ne pouvais pas lui mettre ses points: dans ce cas, il aurait reçu son grade de physique alors qu'il ne m'avait pas montré de connaissances en physique. J'ai proposé de donner une autre chance à l'étudiant en lui donnant six minutes pour répondre à la question avec l'avertissement que pour la réponse, il devait utiliser ses connaissances en physique. Après cinq minutes, il n'avait encore rien écrit. Je lui ai demandé s'il voulait abandonner, mais il répondit qu'il avait beaucoup de réponses pour ce problème et qu'il cherchait la meilleure d'entre elles. Je me suis excusé de l'avoir interrompu et lui ai demandé de continuer. Dans la minute qui suivit, il se hâta pour me répondre: "On place le baromètre à la hauteur du toit. On le laisse tomber en calculant son temps de chute avec un chronomètre. Ensuite en utilisant la bonne formule connue par tous, on trouve la hauteur du building".  À ce moment, j'ai demandé à mon collègue s'il voulait abandonner. Il me répondit par l'affirmative et donna presque 20 à l'étudiant. En quittant son bureau, j'ai rappelé l'étudiant, car il avait dit qu'il avait plusieurs solutions à ce problème. "Hé bien, dit-il, il y a plusieurs façons de calculer la hauteur d'un building avec un baromètre. Par exemple, on le place dehors lorsqu'il y a du soleil. On calcule la hauteur du baromètre, la longueur de son ombre et la longueur de l'ombre du building. Ensuite, avec un simple calcul de proportion, on trouve la hauteur du building."

Bien, lui répondis-je, et les autres? À quoi l'élève répondit: "Il y a une méthode assez basique que vous allez apprécier. On monte les  étages avec un baromètre et en même temps on marque la longueur du baromètre sur le mur. En comptant le nombre de traits, on a la hauteur du building en longueur de baromètre. C'est une méthode très directe. Bien sûr, si vous voulez une méthode plus sophistiquée, vous pouvez pendre le baromètre à une corde, le faire balancer comme un pendule et déterminer la valeur de g au niveau de la rue et au niveau du toit. À partir de la différence de g la hauteur de building peut être calculée. De la même façon, on l'attache à une grande corde et en étant sur le toit, on le laisse descendre jusqu'à peu près le niveau de la rue. On le fait balancer comme un pendule et on calcule la hauteur du building à partir de sa période de balancement."

Finalement, l'élève conclut: "Il y a encore d'autres façons de résoudre ce problème. Probablement la meilleure est d'aller au sous-sol, frapper à la porte du concierge et lui dire: "J'ai pour vous un superbe baromètre si vous me dites quelle est la hauteur du building."

J'ai ensuite demandé à l'étudiant s'il connaissait la réponse que j'attendais. Il a admis que oui, mais qu'il en avait marre du collège et des professeurs qui essayaient de lui apprendre comment il devait penser.

Pour l'anecdote, l'étudiant était Niels Bohr (prix Nobel de Physique en 1923) et l'arbitre Rutherford (prix Nobel de Chimie en 1908).

\begin{center}\underline{\hspace{5 cm}}\end{center}

\begin{center}
	\includegraphics[scale=0.4]{img/humour/evidence_based.jpg}	
\end{center}
\pagebreak
Dans un cours de Russell portant sur le fait que d'une proposition fausse, toute proposition peut être déduite, un étudiant lui posa la question suivante:

\begin{itemize}	 
	\item[$-$] "Prétendez-vous que de $2 + 2 = 5$, il s'ensuit que vous êtes le pape ? "

	\item[$-$] "Oui", répondit Russell

	\item[$-$] "Et pourriez-vous le prouver !", demanda l'étudiant sceptique

	\item[$-$] "Certainement", réplique Russell, qui proposa sur le champ la démonstration suivante:

	\begin{enumerate}
		\item Supposons que $2 + 2 = 5$

		\item Soustrayons $2$ de chaque membre de l'égalité, nous obtenons $2 = 3$

		\item Par symétrie, $3 = 2$

		\item Soustrayant $1$ de chaque côté, il vient $2 =1$
	\end{enumerate}

	\item[$-$] Maintenant le pape et moi sommes deux. Puisque $2 = 1$, le pape et moi sommes un. Par suite, je suis le pape.
\end{itemize}
Sur ce ...

	\begin{center}\underline{\hspace{5 cm}}\end{center}
	
What is "pi"?

\begin{itemize}	 
	\item[$-$] Mathematician: "Pi is the ratio of the circumference of a circle to its diameter."

	\item[$-$] Engineer: "Pi is about 22/7."

	\item[$-$] Physicist: "Pi is 3.14159 plus or minus 0.000005"

	\item[$-$] Computer Programmer: "Pi is 3.141592653589 in double precision."

	\item[$-$] Nutritionist: "You're one track math-minded fellows, Pie is a healthy and delicious dessert!"
\end{itemize}

	\begin{center}\underline{\hspace{5 cm}}\end{center}
	
Un astronome, un physicien et un mathématicien et un informaticien prennent des vacances ensemble en Écosse. En regardant à l'exétrieur d'une fenêtre de train pendant un déplacement, ils observent tous un mouton noir au milieu d'un champ.

\begin{itemize}	 
	\item[$-$] "Comme c'est intéressant" s'exclame l'astronome. "Tous les moutons écossais sont noirs!". 

	\item[$-$] Ce à quoi le physicien répond: "Non, non! Quelques moutons écossais sont noirs!". 

	\item[$-$] Le mathématicien soupirant... dit: "En Écosse, il y a à dans un champ au moins un mouton dont au moins un des côtés est noir".

	\item[$-$] L'informaticien s'exclame: "Oh non! Un bug!".	
\end{itemize}

	\begin{center}\underline{\hspace{5 cm}}\end{center}	
	
Un mathématicien, un biologiste et un physicien sont assis à une terrasse d'un café observant les gens entrant et sortant d'un immeuble de l'autre côté de la rue.

D'abord ils ont observé deux personnes entrer dans l'immeuble. Après un certain temps, ils observent que trois personnes sortent de l'immeuble.

\begin{itemize}	 
	\item[$-$]  Le physicien: "Le moyen de mesure n'était pas adapté".

	\item[$-$] Le biologiste: "Ils se sont reproduits".

	\item[$-$] Le mathématicien: "Si maintenant il y a une personne qui entre, l'immeuble sera à nouveau vide"
\end{itemize}

	\begin{center}\underline{\hspace{5 cm}}\end{center}

Un mathématicien et un ingénieur suivent un cours dispensé par un physicien. Un des sujets concerne les théories de Kaluza-Klein impliquant des développements à $9$, $12$ et un nombre encore supérieur de dimensions. Le mathématicien se pose et écoute tranquillement le physicien alors que l'ingénieur est confus et perdu. À la fin de l'exposé, l'ingénieur a un terrible mal de tête alors que le mathématicien commente l'exposé du physicien.

\begin{itemize}	 
	\item[$-$] L'ingénieur dit: "Comment faites-vous pour comprendre tout cela?"

	\item[$-$] Le mathématicien répond: "Je visualise le processus dans ma tête!".

	\item[$-$] L'ingénieur rétorque: "Comment pouvez-vous visualiser quelque chose qui a lieu dans un espace à neuf dimensions?".

	\item[$-$] Le mathématicien répond: "C'est très simple! Je le visualise d'abord en $n$ dimensions et après je réduis $n$ à $9$."
\end{itemize}

	\begin{center}\underline{\hspace{5 cm}}\end{center}

Deux mathématiciens sont trouven dans un bar. Le premier explique au second que la connaissance des gens en mathématique est vraiment très élémentaire. Le second en désaccord insiste sur le fait que le niveau moyen est bien meilleur que ce que l'on peut soupconner.

Le premier mathématicien s'en va un instant aux toilettes et pendant son absence, le second interpelle une serveuse. Il lui raconte que d'ici quelques minutes un ami va revenir, l'interpeller et lui poser une question. Tout ce qu'elle aura à faire est de répondre "un tiers de x cube".

Elle répète "un tier--- dx cub"?

Il lui répète "un tiers de x cube".

Elle demande encore une fois, "un tiers dex cube?"

"C'est exact" réponde-t-il.

La serveur s'éloigne en marmonnant "un tiers dex cube...".

Le premier mathématicien revient des toilettes. Le second lui propose alors pour clore leur débat de demander un calcul de math à une des serveuses. Alors le premier rétorque qu'il suffit de demander à la serveuse blonde le calcul d'une intégrale élémentaire. Son ami rigolant approuve la démarche. Le premier mathématicien appelle donc l'unique serveur blonde et lui demande: "Quelle est l'intégrale de x au carré?".

La serveuse répond alors"un tiers de x cube" et pendant qu'elle s'éloigne, elle se retourne rapidement en ajouter "plus une constante!".

	\begin{center}\underline{\hspace{5 cm}}\end{center}

Paroles de profs:

\begin{itemize}	 
	\item[$-$] Il reste des lambeaux de l'argument
	\item[$-$] Le signe '-' devant le potentiel vous interpelle ? Et si je mets un '+' est-ce que cela vous convient mieux? Oui ! Alors mettons un '+'...
	\item[$-$] C'est une courbe de phase ça ? Non, c'est une jambe de chien !
	\item[$-$] On va le tuer et on fera croire à un suicide
	\item[$-$] Vous pouvez dire que c'est prévisible puisque c'est imprévisible !
	\item[$-$] Je compte sur vous pour comprendre un petit peu à fond la suite
	\item[$-$] Nous allons étudier cela maintenant un peu plus tard
	\item[$-$] Je vais vous présenter des résultats dépendant des équations de Maxwell que vous n'avez pas encore vues... de toute façon, au point où on en est...
	\item[$-$] Si vous ne comprenez pas, c'est normal... le contraire serait d'ailleurs étonnant
	\item[$-$] La relativité générale ne sert à rien ! C'est pas avec ça qu'on met des satellites en orbite!
	\item[$-$] Question à cinquante centimes
	\item[$-$] De temps en temps, il faut être absurde
	\item[$-$] Quand on n'a rien à se mettre sous la dent, on prend le théorème de Gauss
	\item[$-$] On ne voit pas par quelle vertu du Saint-Esprit il deviendrait neutre ?!
	\item[$-$] Vous comprendrez quand vous serez grand...
	\item[$-$] Prenons l'exemple d'une banque: monsieur y fait un dépôt et madame y fait un retrait... enfin comme d'habitude quoi !
	\item[$-$] ... plus le parallélisme sera parallèle...
	\item[$-$] Effacez-moi ces gris-gris résidus de calculs antérieurs
	\item[$-$] Effacez le membre de gauche. J'ai dit DE GAUCHE ! Où est votre droite ? C'est bien... eh bien la gauche, c'est de l'autre côté !
	\item[$-$] Voyons, qui sera la prochaine victime ?
	\item[$-$] Le '+' se reconnaît, et les électrons ne s'y trompent pas, à sa belle couleur rouge !
	\item[$-$] Le but du jeu est de vérifier que le critère ne dit pas de conneries
	\item[$-$] Si vous n'arrivez pas à faire cela, je vous rassure: c'est foutu pour l'examen
	\item[$-$] Si vous n'arrivez pas à faire cela, faites-le !
	\item[$-$] Vous être très forts pour trouver des choses fausses
	\item[$-$] Ce genre de démonstration, ça me plaît. Vous, ça vous fait faire des cauchemars
	\item[$-$] Je ne résoudrai pas ceci, car cela risquerait de heurter votre sensibilité !
	\item[$-$] Tout ce que les profs adorent mérite votre méfiance !
	\item[$-$] Vous ajoutez des patates et des cochons, c'est sans dimension...
	\item[$-$] On peut chercher la dérivée de l'impulsion de Dirac, c'est pas ça qui va nous sauter à la tête
	\item[$-$] C'est bête mais Riemann, c'est comme ça
	\item[$-$] Regardez l'équation que je viens d'effacer
	\item[$-$] Juridiquement parlant, le courant est dans ce sens
	\item[$-$] Il est hors de question que je perde mon temps à vous résoudre cette plaisanterie de basse volée !
	\item[$-$] Si la fille a compris alors vous devez avoir tous compris
	\item[$-$] Ah oui, vous vous moquez de moi. En fait, vous avez raison de vous moquer des profs car nous, on n'hésite pas à se moquer de vous
	\item[$-$] Attention, très très important: je vous propose d'en rêver la nuit
	\item[$-$] J'ai l'impression du jouer du stradivarius devant des vaches
	\item[$-$] La main de dieu répartit le champ perpendiculairement à la surface
	\item[$-$] Le miracle de la disparition de l'harmonique n'aura pas lieu aujourd'hui
	\item[$-$] Si vous vous réveillez la nuit, dites-vous que échantilloner dans le domaine des temps, c'est périodiser dans le domaine des fréquences... puis rendormez-vous
	\item[$-$] Les courbes gauches ne sont pas droites
	\item[$-$] Vous voyez, quelques fois mes résultats sont corrects
	\item[$-$] On note "$\mathbb{Q}$" la sortie, c'est évident...
	\item[$-$] Il va falloir une entourloupette pour réussir à récupérer cette variable
	\item[$-$] Je suis un 68000; M. le directeur passe dans le couloir, frappe à la porte: on ne fait pas mieux comme interruption !
	\item[$-$] Attention: un, deux, trois, à vos cerveaux !
	\item[$-$] Dans la steppe de l'automatique, nous arrivons à la partie aride des mathématiques
	\item[$-$] Quel intérêt ? Eh bien aucun. C'est une figure de rhétorique pédagogique
	\item[$-$] Si vous avez quelques souvenirs sur les graphes de fluence que jadis nous traçâmes...
	\item[$-$] Les électrons dans du métal très chaud, c'est la place de la Concorde à 5h du soir, donc ça ne conduit pas non plus
	\item[$-$] Les cercles, c'est ce qui a le moins de coins !
	\item[$-$] Si vous mettez vos doigts dans une prise, ce n'est pas un nombre complexe qui sort
	\item[$-$] Voir la démonstration dans vos lectures journalières... donc je m'en dispense
	\item[$-$] Et puis vient ensuite le père Carnot: il ramène sa fraise pour ne pas dire grand chose
	\item[$-$] Quelquefois, les gens qui ont un peu de culture connaissent ça... c'est raté pour aujourd'hui !
	\item[$-$] On n'appellera pas cela "rendement", car on peut obtenir des résultats supérieurs à $1$, ce qui pourrait troubler quelques esprits faibles
	\item[$-$] Moi, si on m'avait envoyé au tableau, j'aurais pas écrit cela
	\item[$-$] Vous trouverez la réponse sur 36 15 Archimède
	\item[$-$] J'utilise une méthode qui date de quelques siècles... comme moi d'ailleurs !
	\item[$-$] Posez-moi des questions... j'aimerais qu'on me pose des questions! ... d'autres questions ? Je vais être obligé de prendre la liste et de dire: "toi, tu as une question à poser"
	\item[$-$] L'ordonnancement implémenté dépend de l'instanciation du garbage collector s'il est synchronisé sur le multithreading préemptif de l'OS.
	\item[$-$] Vous n'êtes que des boîtes qui reçoivent des entrées et crachent des sorties
\end{itemize}

	\begin{center}\underline{\hspace{5 cm}}\end{center}

Sherlock Holmes et le Dr Watson sont au camping. Après un bon repas et une bonne bouteille de vin, ils gagnent leur sac de couchage dans la tente et s'endorment.

Quelques heures plus tard, Holmes se réveille et aussitôt secoue son compagnon:

\begin{itemize}
	\item[$-$] "Watson, regardez le ciel et dites-moi ce que vous voyez."
	
	\item[$-$] "Je vois des millions et des millions d'étoiles." répond Watson.
	
	\item[$-$] "Et qu'est-ce que vous en concluez ?" demande Holmes.
	
	\item[$-$] "Du point de vue astronomique, répond Watson, cela me dit qu'il y a des millions de galaxies et potentiellement des milliards de planètes. Du point de vue astrologique, j'observe que Saturne est en Lion. Du point de vue chronologique, j'en déduis qu'il est environ 3 heures 15. Du point de vue théologique, je vois que Dieu est tout-puissant et que nous sommes petits et insignifiants. Du point de vue météorologique, je pense que nous aurons une belle journée demain. Et vous, Holmes ?"
\end{itemize}

Sherlock Holmes reste pensif une minute puis déclare:

\begin{itemize}
	\item[$-$] "Watson, vous êtes un âne. On nous a volé la tente."
\end{itemize}

	\begin{center}\underline{\hspace{5 cm}}\end{center}

Un ingénieur, un mathématicien et un physicien séjournent une nuit dans un hôtel. Malheureusement un début d'incendi s'enclenche dans chacune de leur chambre.

Le physiciens se réveille, voit le feu, fait quelques observations méticuleuses et de nombreux calculs sur la couverture de la carte des vins de l'hôtel. Une fois ceci fait, il s'empare de l'extincteur et éteind de le feu de manière très précise en un seul et unique coup et retourne se coucher.

L'ingénieur se réveille, voit le feu, fait quelques observations méticuleuses et sur la couverture de la carte du restaurante fait quelques calculs. Une fois ceci fait et après avoir ajouté un facteur de sécurité de 5, il s'empare de l'extincteur et asperge l'ensemble de la chambre plusieurs fois de suite et retourne se coucher.

Le mathématicien se réveille à son tour, fait quelques observation méticuleuses et sur un tableau noir se trouvant dans la chambre fait de nombreux calculs. Soudainement il s'exclame: "une solution existe!". Une fois la solution trouvée, il retourne dans son lit...

	\begin{center}\underline{\hspace{5 cm}}\end{center}

Comment pouvez-vous deviner que la personne conduisant le véhicule face au vôtre est un physicien?

C'est simple: Il a un autocollant rouge à l'arrière où il est écrit: "Si vous voyez cet autocollant en bleu c'est que vous roulez trop vite."

	\begin{center}\underline{\hspace{5 cm}}\end{center}

Un physicien des plasma de Princeton se trouve à la plage et découvre une vieille lampe à huile dans le sable. En essuyant la lampe, un génie surgit. Pour le remercier de l'avoir libéré, le génie lui offre un voeux. Le physicien sort alors un carte du monde et entoure d'un cercle la région du moyen-orient et dit au génie "je veux que tu apportes la paix dans cette région".

Après 10 minutes de silence, le génie répond, "Eh bien dites... il y en a des problèmes là-bas avec le Liban, l'Iraq, Israel et toutes les autres lieux. C'est terriblement embarasssant, je n'ai jamais eu une tâche d'une telle difficutlé. Je suis dans l'obligation de vous demander d'exaucer un autre voeu. Celui-là étant trop ardu pour moi.".

Sur le coup, le physicien réfléchit et demande: "Je souhaite que le tokamak de l'université de Princeton réussisse la fusion stable".

Après 10 minutes de silence, le génie répond: "Puis-je voir la carte du moyen-orient à nouveau?"

	\begin{center}\underline{\hspace{5 cm}}\end{center}

Vous êtes un chasseur qui se trouve dans la jungle. Vous avez avec vous 2 cartouches et un fusil. Tout à coup vous tombez nez à nez avec une panthère. Vous avez soudain envie de fumer une pipe. Comment fais-tu?

Tu tires d'abord sur la panthère, mais tu la loupes. Tu as donc une loupe. Avec, la seconde cartouche, tu tues la panthère. Ensuite tu la prends par la queue et la fait tourner autour de ta tête. Le périmètre du cercle formé vaut 2Pi panthère. Tu possèdes alors 2 pipes en terre. Tu prends une pipe que tu casses. Ensuite tu gardes la moitié en main et dépose l'autre moitié par terre. Il y a donc un tas haut, et un tas bas. Tu as donc du tabac. C'est fini tu déposes le tabac dans la pipe en terre que tu allumes avec la loupe. Ouf!!!!!

	\pagebreak
	\section{Mathématiques}

Il y a $5$ personnes sur $4$ qui n'y connaissent rien en fraction....

	\begin{center}\underline{\hspace{5 cm}}\end{center}

L'amour c'est comme $\pi$, naturel, irrationnel, transcendant mais très réel.

	\begin{center}\underline{\hspace{5 cm}}\end{center}

Un professeur de maths explique à une blonde les limites. Il fait avec elle l'exercice suivant:
	\begin{gather*}
	\lim_{x \rightarrow 8} \dfrac{1}{x-8}=+\infty
	\end{gather*}
	À la fin de l'exercice, il demande à la blonde si elle a compris. "Oh oui monsieur j'ai tout compris!". N'y croyant qu'à moitié, il lui pose l'exercice suivant:

	Soit à calculer:
	\begin{gather*}
	\lim_{x \rightarrow 5} \dfrac{1}{x-5}
	\end{gather*}
	Et la blonde de résoudre:
	\begin{gather*}
	\lim_{x \rightarrow 8} \dfrac{1}{x-8}=+\infty \quad  \text{alors} \quad \lim_{x \rightarrow 5} \dfrac{1}{x-5}= \rotatebox[origin=c]{90}{5}  
	\end{gather*}
	
	\begin{center}\underline{\hspace{5 cm}}\end{center}
	
	\begin{center}
		\includegraphics{img/humour/self_complementary_graph.jpg}	
	\end{center}
	
	\pagebreak
Évolution de l'enseignement des Mathématiques (...):

\begin{itemize}	 
	\item[$-$] Enseignement 1960: Un paysan vend un sac de pommes de terre pour $100$ Frs. Ses frais de production s'élèvent au $4/5$ du prix de vente. Quel est son bénéfice ?

	\item[$-$] Enseignement traditionnel 1970: Un paysan vend un sac de pommes de terre pour $100$ Frs. Ses frais de production s'élèvent au $4/5$ du prix de vente, c'est-à-dire à $80$ Frs. Quel est son bénéfice ?

	\item[$-$] Enseignement moderne 1970: Un paysan échange un ensemble $P$ de pommes de terre contre un ensemble M de monnaie. Le cardinal de l'ensemble $M $est égal à $100$, et chaque élément PFM vaut $1$ Fr. Dessinez $100$ gros points représentant les éléments de l'ensemble $M$. L'ensemble $F$ des frais de production comprend $20$ gros points de moins que l'ensemble $M$. Représentez l'ensemble $F$ comme sous-ensemble de l'ensemble $M$ et donnez la réponse à la question suivante: Quel est le cardinal de l'ensemble $B$ des bénéfices ? (à dessiner en rouge)

	\item[$-$] Enseignement rénové 1980: Un agriculteur vend un sac de pommes de terre pour $100$ Frs. Les frais de production s'élèvent à $80$ Frs et le bénéfice est de $20$ Frs. Devoir: souligne les mots "Pommes de terre" et discutes-en avec ton voisin.

	\item[$-$] Enseignement réformé 1990: Un peizan kapitalist privilégié sanrichi injustement de $20$ Frs sur un sac de patat, analiz le tekst et recherche les fotes de contenu, de gramère, d'orthografe, de ponctuassion et ensuite dit ce que tu panse de set manière de s'enrichir.

	\item[$-$] Enseignement start-up 1999: Un producteur de l'espace agricole câblé consulte une data bank qui display le day-rate de la patate. Il load son progiciel de computation fiable et détermine le cash-flow sur écran bit-map (sous WMil avec config floppy et DD 40Go). Dessine avec ta souris le contour intégré 3D du sac de pommes de terre. Puis logues-toi au network par le www.blue-potatoe.com et suis les indications du menu.
	
	\item Enseignement 2010: Qu'est-ce qu'un paysan ?

\end{itemize}
	\begin{center}\underline{\hspace{5 cm}}\end{center}
	\begin{center}
		\includegraphics[scale=0.9]{img/humour/homework.jpg}	
	\end{center}

	\begin{table}[H]
	\begin{center}
		\definecolor{gris}{gray}{0.85}
			\begin{tabular}{|p{7.5cm}|p{7.5cm}|}
				\hline
				\multicolumn{1}{c}{\cellcolor{black!30}\textbf{
Lorsque vous lisez ou entendez}} & 
  \multicolumn{1}{c}{\cellcolor{black!30}\textbf{il faut comprendre...}} \\ \hline
				c'est trivial (ou évident) & je n'arrive pas à dire pourquoi c'est vrai \\ \hline
				automatiquement on a & idem \\ \hline
				un calcul montre que & un calcul que je n'ai pas fait montrerait certainement que\\ \hline
				le lecteur montrera facilement que & ça m'ennuie de montrer que\\ \hline
				Nous conseillons vivement au lecteur de faire les exercices indiqués & comme je ne les ai pas faits, vous pourriez me les corriger\\ \hline
				j'ai montré ce résultat dans un papier antérieur & je ne sais plus diable comment on fait pour prouver ce truc là
				\\ \hline
				on généralise facilement à & la généralisation dépasse mon niveau			
				\\ \hline
				d'après une propriété bien connue & par 10 personnes au monde
				\\ \hline
				la preuve tient en deux lignes & 	oui, mais moyennant cinq lemmes
				\\ \hline
				c'est de l'algèbre & ce n'est pas intéressant (de la bouche d'un analyste)
				\\ \hline
				c'est de l'analyse & ce n'est pas intéressant (de la bouche d'un algébriste)
				\\ \hline
				c'est élémentaire (ou classique) & dans la théorie des espaces bornitziens de deuxième espèce
				\\ \hline
				je n'ai pas bien compris ce pas dans votre démonstration & tu t'es planté dans ta démo
				\\ \hline
				Cette conférence était très intéressante & 	je n'y ai rien compris
				\\ \hline
		\end{tabular}
	\end{center}
	\end{table}	
	
	\begin{center}\underline{\hspace{5 cm}}\end{center}
	
	Le numéro que vous avez demandé est imaginaire ; veuillez tourner votre téléphone d'un quart de tour à droite et renuméroter…
	
	\begin{center}\underline{\hspace{5 cm}}\end{center}
	
	\begin{center}
		\includegraphics[scale=0.6]{img/humour/pizza.eps}	
	\end{center}
	\begin{center}\underline{\hspace{5 cm}}\end{center}
	
	Un mathématicien à son ami:

\begin{itemize}	 
	\item[$-$] "Es-tu fidèle?"

	\item[$-$] "Oui, à un isomorphisme près"
\end{itemize}

	\begin{center}\underline{\hspace{5 cm}}\end{center}
	
Comment les mathématiciens le font:

\begin{itemize}	 
	\item[$-$] Les théoriciens des nombres l'ont fait en premier

	\item[$-$] Nous savons que les analystes réels le font continûment, mais pour les spécialistes de théorie des ensembles, ce n'est qu'une hypothèse

	\item[$-$] Les analystes complexes le font entièrement mais avec conformisme

	\item[$-$] Les algébristes le font avec détermination et sans discrimination

	\item[$-$] Les topologistes le font ouvertement, mais compactement

	\item[$-$] Les topologistes différentiels et algébriques le font avec variété

	\item[$-$] Les spécialistes de combinatoire le font discrètement

	\item[$-$] Les statisticiens le font soit presque toujours, soit presque jamais

	\item[$-$] Les théoriciens de la mesure le font presque partout

	\item[$-$] Les logiciens le font avec consistance

	\item[$-$] Les géomètres le font au foyer mais avec courbures et torsions

	\item[$-$] Les théoriciens des groupes le font simplement et fidèlement

	\item[$-$] Les théoriciens des anneaux le font avec intégrité

	\item[$-$] Les théoriciens des corps le font en inversé

	\item[$-$] Les spécialistes de programmation linéaire maximisent la performance et minimisent les efforts

	\item[$-$] Markov avait besoin de chaînes pour le faire, et Noether d'anneaux

	\item[$-$] Euler le faisait en cercle, tandis que Bernoulli le faisait en spirale ou en huit

	\item[$-$] Möbius le faisait toujours du même côté

	\item[$-$] Gauss le faisait normalement, Lebesgue, avec mesure, et Cauchy le faisait complètement, au contraire de Gödel

	\item[$-$] Fermat a essayé de le faire dans la marge, mais il n'y avait pas assez de place

	\item[$-$] On pense que Riemann et Goldbach l'on fait, mais on n'est encore jamais arrivé à le prouver
 \end{itemize}
 
	\begin{center}\underline{\hspace{5 cm}}\end{center}
	 
Qu'est-ce qu'un homme complexe dit à une femme réelle?

Réponse: "viens danser!" (il faut lire "dans C", c'est-à-dire l'ensemble des complexes $\mathbb{C}$)

	\begin{center}\underline{\hspace{5 cm}}\end{center}
	
	\begin{center}
		\includegraphics{img/humour/rotation_matrix.jpg}	
	\end{center}
	
	\begin{center}\underline{\hspace{5 cm}}\end{center}

Phrases à double sens:

\begin{itemize}	 
	\item[$-$] NOus allons maintenant résoudre ce problème sans complexes

	\item[$-$] Un repère d'origine O (un repaire d'originaux...) 

	\item[$-$] Une partie de $\mathbb{Q}$ (a fucking party...)

	\item[$-$] Ne confondez pas un $\rho$ avec un $p$... (don't confuse between a fart and a burp)

	\item[$-$] Une variété de Poisson... (a variety of fish)
\end{itemize}

	\begin{center}\underline{\hspace{5 cm}}\end{center}

Les fonctions logarithme et exponentielle sont au restaurant. Quand viendra l'addition, qui payera?

Réponse: Exponentielle, car logarithme né paie rien.

Plus tard dans la nuit, Logarithme et Exponentielle rentrent chez eux un peu bourrés. Logarithme demande: Est-ce que je prends le volant?

Exponentielle répond: Je préfère que ce soit moi qui conduise. Au cas où on dérive...

	\begin{center}\underline{\hspace{5 cm}}\end{center}

Deux suites de Cauchy veulent sortir en boite. Elles arrivent devant une boite où se déroule la soirée "No Limit". Elles décident de rentrer, mais le vigile les arrête en leur disant: "Désolé, c'est complet!". (dans un espace complet, une suite de Cauchy est convergente par définition, donc elle a une limite.)

	\begin{center}\underline{\hspace{5 cm}}\end{center}

	\begin{center}
		\includegraphics{img/humour/socks.eps}	
	\end{center}
	\begin{center}\underline{\hspace{5 cm}}\end{center}	

Un mathématicien devint fou en pensait tout le temps qu'il était l'opérateur de différentation. Ses amis le placèrente dans un hôpital psychiatrique en attendant son rétablissement. Toutes les journées il les passait à dire aux autres patients de l'hôpital: "Je te dérive!".

Un jour, il rencontra un nouveau patient et il lui dit immédiatement: "Je te dérive!", mais pour une fois, sa victime ne réagissa nullement à son attaque verbale. Surpris, le mathématicien cria et avec toute son énergie: "Je te dérive!", mais l'autre patient continua à n'avoir aucune réaction. Finalement, après un grand ressentiment de frustration, le mathématicien cria une dernière fois: "JE TE DÉRIVE!".

Le nouveau patient se retourna calmement et lui dit: "Tu peux me différencier autant de fois que tu veux. Je suis l'exponentielle de $x$".

	\begin{center}\underline{\hspace{5 cm}}\end{center}	

Ce que les mathématiciens disent et ce qu'il faut comprendre:

\begin{itemize}	 
	\item[$-$] Trivial: Si je dois vous montrer ceci, vous êtes dans la mauvaise classe

	\item[$-$]  On peut trivialement montrer: On a pas besoin de plus de 4 heures pour le démontrer

	\item[$-$] Contrôlez vous-même: C'est la partie difficile de la démonstration donc vous pouvez le faire sur votre temps libre

	\item[$-$]  Similairement: Au moins une ligne de la démonstration est identique à la précédente

	\item[$-$]  Procédons formellement: Nous allons manipuler des symboles avec des règles bien prédéfinies sans rien comprendre au sens réel du résultat.

	\item[$-$]  Nous nous dispenserons de la démonstration: Faites-moi confiance, c'est vrai!

	\item[$-$]  Le lecteur montrera facilement: Ça m'ennuie de montrer que...

	\item[$-$]  Nous conseillons vivement au lecteur de faire les exercices indiqués: Comme je les ai pas faits, vous pourriez me les corriger

	\item[$-$]  J'ai montré ce résultat dans un papier antérieur: Je ne sais plus diable comment on fait pour prouver ce truc-là

	\item[$-$]  On généralise facilement: La généralisation dépasse mon niveau

	\item[$-$]  D'après une propriété bien connue: Par 10 personnes au monde..
\end{itemize}

\begin{center}\underline{\hspace{5 cm}}\end{center}
	
	\begin{center}
		\includegraphics[scale=0.9]{img/humour/fresh_men.jpg}	
	\end{center}
	
	\begin{center}\underline{\hspace{5 cm}}\end{center}	

Il y a 3 types de personnes: ceux qui peuvent compter et ceux qui ne peuvent pas compter ...

	\begin{center}\underline{\hspace{5 cm}}\end{center}	
	
Tout le monde connaît le "Théorème du Salaire" qui établit que les ingénieurs et les scientifiques ne peuvent JAMAIS gagner autant que les hommes d'affaires et les commerciaux. Ce théorème peut enfin se démontrer par la résolution d'une équation mathématique simple.

Notre équation s'appuie sur deux postulats très connus:
\begin{itemize}
	\item[P1.] La Connaissance c'est la Puissance
	\item[P2.] Le Temps c'est de l'Argent
\end{itemize}

Tout ingénieur sait ensuite que:

\begin{center}
$\text{Puissance}=\dfrac{\text{Travail}}{\text{Temps}}$
\end{center}

Puisque:

\begin{center}
$\text{Connaissance}=\text{Puissance}$
\end{center}

et que:

\begin{center}
$\text{Temps}=\text{Argent}$
\end{center}

Nous avons donc: 

\begin{center}
$\text{Connaissance}=\dfrac{\text{Travail}}{\text{Argent}}$
\end{center}

Nous obtenons alors facilement: 

\begin{center}
$\text{Argent}=\dfrac{\text{Travail}}{\text{Connaissance}}$
\end{center}

Ainsi quand la Connaissance tend vers zéro, l'Argent tend vers l'infini quelle que soit la valeur attribuée à Travail, cette valeur peut être très faible. À l'inverse quand la Connaissance tend vers l'infini, l'Argent tend alors vers zéro, même si la valeur Travail est élevée.

D'où la conclusion évidente suivante: Moins vous en connaissez, plus vous gagnez d'argent.

PS: Ceux d'entre vous qui ont eu quelques difficultés de compréhension doivent être les mieux rémunérés... 

	\begin{flushright}
		$\blacksquare$  Q.E.D.
	\end{flushright}

	\begin{center}\underline{\hspace{5 cm}}\end{center}
	
	\begin{center}
		\includegraphics{img/humour/proof_trivial.jpg}	
	\end{center}
	
	\begin{center}\underline{\hspace{5 cm}}\end{center}	
	
Dans le même genre voici le "Théorème du misogyne":

D'abord, nous déclarons que les filles sont des variables factorisables en quantités de temps et d'argent telle que:

\begin{center}
$\text{Filles}=\text{Temps}\times\text{Argent}$
\end{center}

Comme nous le savons tous "le temps c'est de l'argent!". Donc:

\begin{center}
$\text{Te}mps=\text{Argent}$
\end{center}

et parce que "l'argent est la racine du mal…" et que le mal est un synonyme du ce qui est mauvais:

\begin{center}
$\text{Argent}=\sqrt{\text{Mal}}$
\end{center}

Donc nous avons par substitution:

\begin{center}
$\text{Filles}=\left(\sqrt{\text{Mal}}\right)^2$
\end{center}

Nous sommes donc forcés de conclure:

\begin{center}
$\text{Filles}=\text{Mal}$
\end{center}

	\begin{flushright}
		$\blacksquare$  Q.E.D.
	\end{flushright}
	
	\begin{center}\underline{\hspace{5 cm}}\end{center}
	\begin{center}
		\includegraphics{img/humour/professor_xi.jpg}	
	\end{center}
	
	Les dix meilleures excuses pour ne pas faire vos devoirs de maths:
	
	\begin{itemize}
	
		\item[$\text{\#}10.$] Galilée ne connaissait pas l'algèbre; Pourquoi alors en aurais-je besoin?
	
		\item[$\text{\#}09.$]  Un drogué des maths m'a volé mes devoirs.
	
		\item[$\text{\#}08.$] Jai pris l'option physique et les devoirs semblent impliquer des maths, donc j'ai pensé que je pourrais juste faire cela à la place.
	
		\item[$\text{\#}07.$] J'ai la preuve, mais il n'y a pas de place pour l'écrire dans la marge.
	
		\item[$\text{\#}06.$] J'ai une calculatrice à énergie solaire et c'était nuageux.
	
		\item[$\text{\#}05.$] Je regardais les World Series et je me suis ligoté en essayant de prouver que ça convergeait.
	
		\item[$\text{\#}04.$] Je ne pouvais qu'être arbitrairement proche de mon manuel. (J'ai atteint la moitié du chemin, puis la moitié, et puis ...)
	
		\item[$\text{\#}03.$] I couldn't figure out whether i am the square root of negative one or i is the square root of negative one.
	
		\item[$\text{\#}02.$] C'était le jour de l'anniversaire d'Einstein et de pi et nous avons eu cette célébration! (Cela ne fonctionne que pour le 14 mars)
	
		\item[$\text{\#}01.$] J'ai accidentellement divisé par zéro et mon papier a pris feu.	
	\end{itemize}

	\begin{center}\underline{\hspace{5 cm}}\end{center}
	\begin{center}
		\includegraphics{img/humour/close.jpg}	
	\end{center}
	\begin{center}\underline{\hspace{5 cm}}\end{center}
	
	\pagebreak
	Quel est le résultat de:
	\begin{center}
		 $\dfrac{2ab}{2Fr.16}$\\
		(lire "2 abbés sur 2 françaises")
	\end{center} 

	Réponse: 
	\begin{center}
		$2bb \dfrac{a}{e}$\\
		(lire "2 bébés assurés")
	\end{center} 
	\begin{center}\underline{\hspace{5 cm}}\end{center}
	
	Quel est le résultat de: 
	\begin{center}
	 $\dfrac{\text{cheval}}{\text{oiseau}}$\\
	(lire "cheval sur oiseau")
	 \end{center} 
	
	Comme nous avons: 
	\begin{center}
	 $\dfrac{\text{cheval}}{\text{oiseau}}=\dfrac{\text{vache} \cdot \text{l}}{\beta \cdot \text{l}}$\\
	(lire "vache + l" (contient toutes les lettres de "cheval") divisé par "bête à ailes") 
	 \end{center} 
	
	Mais: 
	\begin{center}
	 $\dfrac{\text{vache} \cdot \text{l}}{\beta \cdot \text{l}}=\dfrac{\beta \cdot \pi \cdot \text{l}}{\beta \cdot \text{l}}$\\
	(lire "bête à pie + l" divisé par "bête à ailes")  
	 \end{center}
	
	Nous simplifions pour obtenir: 
	\begin{center}
	 $\dfrac{\cancel{\beta} \cdot \pi \cdot \cancel{\text{l}}}{\cancel{\beta} \cdot \cancel{\text{l}}}=\pi$  
	 \end{center}
	
	Ce qui prouve que $\pi$ est irrational parce qu'il n'y pas de comparaison rationelle entre un "cheval" et un "oiseau"... 
	
		\begin{center}\underline{\hspace{5 cm}}\end{center}

	\pagebreak
	À démontrer: 
	\begin{center}
	$\dfrac{\text{ROSSINI}}{\text{SOLSIDO}}=1$  
	\end{center} 
	
	Nous pouvons écrire:
	\begin{center}
	$\dfrac{\text{ROS SI NI}}{\text{SOL SI DO}}=\dfrac{\text{ROS NI}}{\text{SOL DO}}$  
	\end{center}
	
	mais NI$=$DO ("NI vaut DO" c'est-à-dire "niveau d'eau") donc: 
	\begin{center}
	$\dfrac{\text{ROS}}{\text{SOL}}$  
	\end{center}
	
	Mais SOL fait RINO (Solferino: c'est lors de cette bataille qu'Henri Dunant en voyant l'horreur a décidé de créer la Croix-Rouge) donc:
	\begin{center}
	$\dfrac{\text{ROS}}{\text{RI NO}}$  
	\end{center}
	
	et comme RINO c'est ROS (rhinocéros) donc RINO $=$ ROS alors:
	\begin{center}
	$\dfrac{\text{ROS}}{\text{ROS}}=1$  
	\end{center}

	\begin{flushright}
		$\blacksquare$  Q.E.D.
	\end{flushright}	

	\begin{center}\underline{\hspace{5 cm}}\end{center}
	\begin{center}
		\includegraphics[scale=0.6]{img/humour/day_of_an_eigenvector.jpg}	
	\end{center}
	Chaque futur ingénieur apprend à inscrire la somme de deux chiffres rationnels, par exemple:
	\begin{center}
	$1+1=2$  
	\end{center}
	
	Cette forme est cependant assez banale et indique des lacunes dans votre éducation.

	En premier semestre, on apprend que:
	\begin{center}
	$1=\ln(e)$  
	\end{center}
	
	et:
	\begin{center}
	$1=\sin^2(p)+\cos^2(q)$  
	\end{center}
	
	Tout le monde sait aussi que:
	\begin{gather*}
	2=\sum_{n=0}^{+\infty} \left( \dfrac{1}{2} \right)^n
	\end{gather*}
	
	et que donc l'équation:
	\begin{gather*}
	1+1=2
	\end{gather*}
	
	peut être écrite plus simplement:
	\begin{gather*}
	\ln(e)+\sin^2(p)+\cos^2(q)=\sum_{n=0}^{+\infty} \left( \dfrac{1}{2} \right)^n
	\end{gather*}
	
	(il faut admettre que l'aspect est bien plus clair et plus scientifique)

	D'autre part, il est évident que:
	\begin{gather*}
	1=\cosh(q)\sqrt{1-\tanh^2(q)}
	\end{gather*}
	
	et aussi:
	\begin{gather*}
	e=\lim_{z \rightarrow +\infty}\left(1+\dfrac{1}{z} \right) 
	\end{gather*}
	
	il en résulte que:
	\begin{gather*}
	\ln(e)+\sin^2(p)+\cos^2(q)=\sum_{n=0}^{+\infty} \left( \dfrac{1}{2} \right)^n
	\end{gather*}
	
	peut être réécrite comme suit:
	\begin{gather*}
	\ln\left( \lim_{z \rightarrow +\infty}\left(1+\dfrac{1}{z} \right)\right)+\sin^2(p)+\cos^2(q)=\sum_{n=0}^{\infty} \left( \dfrac{\cosh(q)\sqrt{1-\tanh^2(q)}}{2} \right)^n
	\end{gather*}
	
	Il faut également se rappeler que:
	\begin{gather*}
	0!=1
	\end{gather*}
	et que l'exposant inverse de l'exposant opposé est égal à l'exposant opposé de l'exposant inverse. En supposant un espace à $n$ dimension, on sait que:
	\begin{gather*}
	\left( X^T\right) ^{-1}-\left( X^{-1}\right) ^{-T}=0
	\end{gather*}
	Si l'on prend encore la matrice comme la métrique d'un espace canonique orthogonal et orienté:
	\begin{gather*}
	\left( g_{ij}^T\right) ^{-1}-\left( g_{ij}^{-1}\right) ^{-T}=0
	\end{gather*}
	
	Nous obtenons logiquement:
	\begin{gather*}
	\left(\left( g_{ij}^T\right) ^{-1}-\left( g_{ij}^{-1}\right) ^{-T}\right)!=1
	\end{gather*}
	Nous obtenons ainsi une expression simple et claire pour tout le monde de $1+1=2$:
	\begin{gather*}
	\ln\left( \lim_{z \rightarrow +\infty}\left(\left(\left( g_{ij}^T\right) ^{-1}-\left( g_{ij}^{-1}\right) ^{-T}\right)!+\dfrac{1}{z} \right)\right)+\sin^2(p)+\cos^2(q)=\sum_{n=0}^{+\infty} \left( \dfrac{\cosh(q)\sqrt{1-\tanh^2(q)}}{2} \right)^n
	\end{gather*}
	
	Il est donc évident que cette égalité est bien plus compréhensible que:
	\begin{gather*}
	1+1=2
	\end{gather*}
	Il serait possible de montrer plusieurs autres développements de cette simple expression et nous le ferons à partir du moment où vous commencerez à comprendre les principes simples de la méthode précédente.
	
	\begin{center}\underline{\hspace{5 cm}}\end{center}

	\begin{center}
		\includegraphics{img/humour/coloring_problem.jpg}	
	\end{center}

	Un catcheur, un physicien et un mathématicien sont sujet à une expérience: on les enferme dans une pièce avec chacun une boite d'épinards, fermée, et sans ouvre-boîte. Au bout de 24 heures, on va voir ce qu'ils sont devenus. 
	
	\begin{itemize}	 
		\item[$-$] Le catcheur a réussi à ouvrir sa boîte: « Et bien, j'ai simplement violemment projeté la boîte contre le mur. L'impact a été tel qu'elle s'est ouverte », explique t-il. 
	
		\item[$-$] Le physicien a également réussi à ouvrir sa boîte: « J'ai observé le solide, et distingué ses points de rupture. J'ai alors effectué une pression de manière à exercer une force maximale sur ceux-ci, et la boîte s'est tout naturellement ouverte. » 
	
		\item[$-$] Le mathématicien, enfin, est retrouvé prostré dans un coin de la pièce, la sueur ruisselant sur son visage, et sa boîte de conserve, fermée, entre les pieds: « Admettons que la boîte est ouverte... Admettons que...» 
	\end{itemize}
	\begin{center}\underline{\hspace{5 cm}}\end{center}
	\begin{center}
		\includegraphics[scale=0.9]{img/humour/math_useful.jpg}	
	\end{center}

	\begin{center}\underline{\hspace{5 cm}}\end{center}
	Mathématique de la vie:
	\begin{gather}
		\setlength{\tabcolsep}{1pt}
		\begin{tabular}{cccccc}
		& & \text{Vie} & + & \text{Amour}= & \text{Bonheur} \\
		$+$& & \text{Vie} & - & \text{Amour}= & \text{Malheur} \\ \hline
		&2\text{Vie}& & & = & \text{Bonheur}+\text{Malheur} \\
		\end{tabular}
	\end{gather}
	Ainsi:
	\begin{gather}
		\text{Vie}=\dfrac{\text{Bonheur}+\text{Malheur}}{2}
	\end{gather}
	En développant:
	\begin{gather}
		\text{Vie}=\dfrac{1}{2}\text{Bonheur}+\dfrac{1}{2}\text{Malheur}
	\end{gather}
	C'est ça la vraie vie. Profitez-en!
	
	\begin{center}\underline{\hspace{5 cm}}\end{center}
	
	An opinion without $3.14$ is an onion. You'll understand!
	
	\begin{center}\underline{\hspace{5 cm}}\end{center}

	\begin{center}
		\includegraphics[scale=0.4]{img/humour/math_man_sex.jpg}	
	\end{center}
	
	\begin{center}\underline{\hspace{5 cm}}\end{center}
	
	L'épouse d'un logicien accouche d'un bébé. Le médecin remet immédiatement le nouveau-né au père.

	Sa femme demande avec impatience: "Alors, est-ce un garçon ou une fille"?

	Le logicien répond: "Oui".

	\begin{center}\underline{\hspace{5 cm}}\end{center}
	
	Question: Que représente le "B" dans Benoit B. Mandelbrot?

	Réponse: Benoit B. Mandelbrot
	
	\begin{center}\underline{\hspace{5 cm}}\end{center}
	
	Ce n'est pas ainsi que nous faisons cette dérivée ...:
	\begin{gather}
		\dfrac{\mathrm{d}}{\mathrm{d}x}\dfrac{1}{x}=\dfrac{\mathrm{d}}{\mathrm{d}}\dfrac{1}{x^2}=\dfrac{\cancel{\mathrm{d}}}{\cancel{\mathrm{d}}}\dfrac{1}{x^2}=-\dfrac{1}{x^2}
	\end{gather}
	
	\begin{center}\underline{\hspace{5 cm}}\end{center}

	\begin{center}
		\includegraphics[scale=0.6]{img/humour/asymptote.jpg}	
	\end{center}
	
	\begin{center}\underline{\hspace{5 cm}}\end{center}
	
	\begin{center}
		\includegraphics[scale=0.6]{img/humour/pourcentages.jpg}	
	\end{center}

	\pagebreak
	\section{Physique}
	
	\begin{center}
	\includegraphics{img/humour/heisenberg.eps}
	\end{center}
	
	\begin{center}\underline{\hspace{5 cm}}\end{center}	
	
	\begin{itemize}	 
		\item[$-$] En théorie, il n'y a pas de différences entre la théorie et la pratique. En pratique, si!
	
		\item[$-$] La théorie, c'est quand on sait tout, mais que rien ne marche. La pratique, c'est quand tout marche, mais qu'on ne sait pas pourquoi. En informatique, la théorie et la pratique sont réunies: rien ne marche et on ne sait pas pourquoi!
	\end{itemize}

	\begin{center}\underline{\hspace{5 cm}}\end{center}
	
	\begin{itemize}	 
		\item[$-$]  La matière est fondamentalement fainéante: Elle prend toujours le chemin de moinde action.
	
		\item[$-$] La matière est fondamentalement stupide: Elle essaie tous les chemins d'abord.
		
		\item[$-$] Ceci est le coeur de la physique, le reste n'étant que détails!
	\end{itemize}
	
	\begin{center}\underline{\hspace{5 cm}}\end{center}
	
	Ce sont deux atomes qui se rencontrent.  

	L'un dit à l'autre: "Merde, j'ai perdu un électron!"

	L'autre: "T'es sûr?"

	Et le premier répond: "POSITIVEMENT !!"
	
	\begin{center}\underline{\hspace{5 cm}}\end{center}

	\begin{center}
	\includegraphics{img/humour/einstein.eps}
	\end{center}
	
	\begin{center}\underline{\hspace{5 cm}}\end{center}	
	
	Vous entrez dans un laboratoire et vous voyez une expérience. Comment allez-vous deviner de quelle section il s'agit?
	
	\begin{itemize}	 
		\item[$-$] Si c'est vert et visqueux, c'est la biologie.
	
		\item[$-$] Si ça pue, c'est la chimie.
	
		\item[$-$] Si cela ne marche pas, c'est la physique.
	\end{itemize}
	
	\begin{center}\underline{\hspace{5 cm}}\end{center}	

	Théorème: Un chat à neuf vies.

	Démonstration: Comme aucun chat à huit vies, tout chat à une vie de plus que pas de chat. Donc par extension un chat à neuf vies.
	
	\begin{center}\underline{\hspace{5 cm}}\end{center}

	\begin{center}
	\includegraphics[scale=0.75]{img/humour/schrodinger_cat.eps}
	\end{center}
	
	\begin{center}\underline{\hspace{5 cm}}\end{center}

	Un physicien étudiant la physique quantique, c'est quelqu'un qui ne voyant pas très bien, cherche dans une chambre obscure, un chat noir, qui probablement n'existe pas.
	
	\begin{center}\underline{\hspace{5 cm}}\end{center}

	Un ingénieur, un physicien, un mathématicien et un mystique sont interrogés sur la plus grande inventions de tous les temps.

\begin{itemize}	 
	\item[$-$]  L'ingénieur choisit le feu, qui donna à l'humanité le contrôle de la matière.

	\item[$-$] Le physicien choisit la roue, qui donne à l'humanité le contrôle de l'espace.

	\item[$-$] Le mathématicien choisit l'alphabet, qui donna à l'humanité le contrôle sur les symboles.

	\item[$-$] Le mystique choisit la bouteille thermos.

"Pourquoi la bouteille thermos?" lui demandèrent les trois autres?

	\item[$-$] Question à laquelle le mystique répondit: "Ben oui! Le bouteille thermos elle garde les liquides chauds quand il fait froid, et froids quand il fait chaud."

	\item[$-$] "Et... alors?" lui demandèrent les autres?

	\item[$-$] "Réfléchissez!" dit le mystique. "La petite bouteille, peut-elle savoir cela?".
	\end{itemize}
	
\begin{center}\underline{\hspace{5 cm}}\end{center}

	\begin{center}
	\includegraphics[scale=0.4]{img/humour/milkiway.jpg}
	\end{center}
\begin{center}\underline{\hspace{5 cm}}\end{center}

Un professeur de physique a effectué une expérience et a déterminé une équation empirique pour expliquer les données obtenues. Il demande alors à un professeur de maths d'y jeter un coup d'oeil.

Une semainee plus tard, le professeur de math lui dit que son équation n'est pas valide. Ce à quoi le physicien rétorque que pourtant cette dernière lui a permis de prévoir les autres expériences qu'il a fait entre temps et qu'il a obtenu des résultats excellents, alors il demande au mathématicien de vérifier à nouveau.

Une autre semaine passe, et les deux professeurs se rencontrent à nouveau. Le mathématicien admet alors que l'équation marche effectivement, mais "Seulement dans le cas trivial où les nombres sont réels et positifs".

\begin{center}\underline{\hspace{5 cm}}\end{center}
	
	Les centrales de fusion nucléaire utilisables ne le seront pas avant 30 ans et il en sera toujours ainsi!
	
	\pagebreak
	\begin{center}
	\includegraphics{img/humour/howscientistseeworld.eps}
	\end{center}
	
	\pagebreak
	
	Heisenberg conduisait une voiture sur l'autoroute lorsqu'un policier l'interpella.

	Le policier lui demanda:	
	\begin{itemize}	 
		\item[$-$] ""Savez-vous à quelle vitesse vous rouliez au moins?"
	\end{itemize}
	
	Ce à quoi Heisenberg répondit:
	
	\begin{itemize}	 
		\item[$-$] "Non mais je sais où je me trouve".
	\end{itemize}
	
	\begin{center}\underline{\hspace{5 cm}}\end{center}
	
	Quelle est la différence entre la mécanique automobile et la mécanique quantique?

	Réponse: La mécanique quantique peut parfois ranger la voiture dans le garage sans ouvrir le porte.
	
	\begin{center}\underline{\hspace{5 cm}}\end{center}
	
	\begin{center}
	Ce n'est pas le:
	
	\includegraphics{img/humour/kill_fall.jpg}
	\end{center}
	\begin{gather*}
		v_f=v_0+at
	\end{gather*}
	\begin{center}
	qui vous tue, c'est le:
	\end{center}
	\begin{gather*}
		F=m\dfrac{\Delta v}{\Delta t}
	\end{gather*}
	
	\begin{center}
	\includegraphics{img/humour/kill_final.jpg}
	\end{center}

	\begin{center}
	\includegraphics{img/humour/superstring.eps}
	\end{center}
	
\begin{center}\underline{\hspace{5 cm}}\end{center}
	
Pourquoi Dieu n'a jamais reçu un docteur honoris causa d'une quelconque Université:
\begin{enumerate}
	\item Il a seulement une unique publication majeure,

	\item Cette publication est en hébreu.

	\item Elle n'a aucune référence bibliographique.

	\item Elle n'a jamais été publiée dans une revue de référence.

	\item Il y a des doutes qu'il l'ait rédigée lui-même.

	\item  Il est peut-être vrai qu'il a créé le monde, mais qu'a-t-il fait depuis?

	\item Sa participation et sa coopération ont été relativement limitées.

	\item La communauté scientifique a une grande difficulté à reproduire ses résultats.

	\item Il n'a jamais appliquée la convention éthtique sur la manipulation d'êtres humains.

	\item Quand une personne tenta de tester ses résultats, il noya tout les lieux et sujet d'expérimentation.

	\item Quand ses résultats n'étaient pas conformes à ses attentes, il les supprima de l'échantillon.

	\item  Il a rarement suivi des classes et mais par contre n'a pas hésité à conseiller de lire un ouvrage.

	\item Certains disent que son fils et dans les classes.

	\item  Il a dispensé ses deux premiers élèves de tout enseignement.

	\item  Dans ses examens, il y avait toujours 10 critères et la majorité des ses étudiants y ont échoué.

	\item  Ses horaires de disponibilité sont irréguliers et il n'est jamais là quand on a besoin de lui.
\end{enumerate}

\begin{center}\underline{\hspace{5 cm}}\end{center}

	\begin{center}
		\includegraphics{img/humour/physics_gang_sign.jpg}
	\end{center}
	\pagebreak

Au tout début, il y avait Aristote:
\begin{itemize}
	\item Et les objets au repos tendaient à rester au repos
	\item Et les obets en mouvement tendaient à revenir au repos
	\item Dieu vit que c'était ennuyant et peu utile.
\end{itemize}

Alors Dieu créa Newton:
\begin{itemize}
	\item Et les objets au reps tendaient à rester au repos
	\item Et les objets au mouvement tendaient à rester en mouvement
	\item Et l'énergie ainsi que le moment étaient conservés
	\item Et la matière était conservée
	\item Dieu vit que cela était très conservatif.
\end{itemize}

Et Dieu créa Einstein:
\begin{itemize}
	\item Et tout était relatif
	\item Et les choses rapides devinrent petites
	\item Et les choses plates devinrent courbes
	\item Et l'Univers était rempli de référentiel inertiels
	\item AEt Dieu vit que cela était la relativité générale, mais une partie était particulièrement relative.
\end{itemize}

Alors Dieu créa Bohr:
\begin{itemize}
	\item Et il y eu LE principe
	\item Et le principe était quantifié
	\item Et tout était quantifié
	\item Mais certains choses restaient relatives
	\item Et Dieu dit que cela était confus.
\end{itemize}

Alors Dieu allait créer Furgeson:
\begin{itemize}
	\item Et Furgeson allait tout unifier
	\item Et il allait mette les théories sous forme de champs
	\item Et tout ne serait plus qu'un
	\item Mais c'était le 7ème jour
	\item Et Dieu se reposa
	\item Et ce qui est au repos tend à rester au repos...
\end{itemize}

	\pagebreak

	\begin{center}
		\includegraphics[scale=0.6]{img/humour/schrodinger_survey.jpg}	
	\end{center}

\begin{center}\underline{\hspace{5 cm}}\end{center}
\begin{center}
\textbf{La table des lois des physiciens}
\end{center}

Nous supposerons triviaux les postulats suivants comme quoi tous les physiciens sont nés égaux, en première approximation, et qu'ils ont été dotés par la créateur de quelques privilèges, avec une espérance de vie suffisante et $n$ degrés de liberté, et les lois suivantes qui sont invariantes pour toute transformation linéaire:
\begin{enumerate}
	\item Approximer tous les problèmes à des cas idéaux.

	\item Utiliser des ordres de grandeurs de calculs particuliers uniquement quand cela est nécessaire.

	\item Utiliser des simplifications dès que le problème étudié implique autre chose que des nombres entiers.

	\item Négliger ou considéréer comme non physique toutes les fonctions qui divergent.

	\item  Invoquer le principe d'incertitude quand confronté à des mathématiciens, chimistes, ingénieurs ou autres scientifiques confus.

	\item  Marmonner quand un non-physicien pose une question trivial impliquant des mathématiques élémentaires.

	\item Égaliser les deux côtés d'une équation qui sont dimensionnellement inconsistentes, avec une remarque du type "De toute façon, nous sommes seulement intéressés à un ordre de grandeur"....

	\item Utiliser des notations non traditionnelles où les conventions mathématiques d'usage ne fonctionnement pas.

	\item Inventer des forces fictives ou virtuelles pour brouillet le public. 

	\item Utiliser des raisonnements incorrects sur la base que cela donne le bon résultat.

	\item Utiliser toujours des conditions initiales utilisant le principe général de trivialité.

	\item Utiliser des arguments plausibles en lieu et place de preuves et de se référer à ses arguments comme étant des preuves.

	\item Prendre pour vrai tout principe qui semble correct mais que ne peut pas être démontré.
\end{enumerate}

	\begin{center}\underline{\hspace{5 cm}}\end{center}
		\begin{center}
		\includegraphics[scale=0.6]{img/humour/travelling_light.jpg}	
	\end{center}

\pagebreak
Vous pouvez utiliser ce petit quiz pour vérifier si vous êtes un vrai scientifique ou non.

\begin{enumerate}
	\item À noël vous: 
	\begin{enumerate}
		\item[a.] Prenez quelques jours pour vous passer du temps avec votre famille.
		\item[b.] Vous partez un peu plus tôt du travail pour prendre le temps d'achter des cadeaux à votre famille. 
		\item[c.] Vous travaillez seulement une demi journée au bureau, passant le reste de la journée à travailler à la maison.
	\end{enumerate}

	\item Votre épouse souhaite discuter des plans pour les prochaines vacances. Vous:
	\begin{enumerate}
		\item[a.] Proposez du camping ce qui vous permettre d'expliquer la beauté de la nature aux enfants.
		\item[b.] Vous proposez d'aller dans une autre ville comme cela vous pouvez allez voir un autre collègue dans son laboratoire pendant que votre épouse surveille les enfants.
		\item[c.] Vous demandez à votre femme: "Nous avons des enfants?"
	\end{enumerate}

	\item À un colloque scientifique dans une île du pacifique sud, aucune réunion n'est prévue pendant un après-midi. Pendant ce temps libre, vous: 
	\begin{enumerate}
		\item[a.] Suiviez la tradition locale et prenez un bain nu sur la plage.
		\item[b.] Prenez place sur la plage complétement habillé, désintéressé de la vue des baigneuses nues et parlez sciences avec vos collègues.
		\item[c.] Allez dans votre chambre d'hôtel avec les rideaux fermées pour travailler sur votre prochain article.
	\end{enumerate}

	\item La maîtresse d'école vous appelle en disant que votre enfant à la varicielle. Vous:
	\begin{enumerate}
		\item[a.]  Stoppez immédiatement tout ce que vous êtes en train de faire et partez immédiatement pour l'école afin de récuperer votre enfant. 
		\item[b.] Stoppez immédiatement tout ce que vous êtes en train de faire et cherchez un traitement contre la varicelle.
		\item[c.] Vous demandez à la maîtresse la route pour l'école et le nom de vos enfants.
	\end{enumerate}

	\item Des êtres d'une autre planète visitent la Terre et vous êtes le premier humain qu'ils rencontrent. Pour vous montrer leur pacificité, ils vous offrent un présent capable de prolonger la vie, arrêter la suffrance humaine mondiale et éradiquer la faim dans le monde. Vous: 
	\begin{enumerate}
		\item[a.] Présentez le cadeau aux Nations Unies.
		\item[b.] Vous posez un brevet.
		\item[c.] Vous le cassez pour voir comment il fonctionne à l'intérieur.
	\end{enumerate}

	\item Quelle est la plus grande période de temps pendant laquelle vous n'avez pas pris de vacances (en excluant les colloques scientifiques...)? 
	\begin{enumerate}
		\item[a.] Six mois.
		\item[b.] Deux ans.
		\item[c.] Vous avez pris un week-end il y a environ 10 ans.
	\end{enumerate}

	\item Quels sont vos loisirs? 
	\begin{enumerate}
		\item[a.] Sport, musique, danse parce qu'ils permettent à la partie analytique de mon cerveau de se reposer.
		\item[b.] Cuisiner, parce que c'est un peu comme une science.
		\item[c.] Relire les anomalies dans les revues scientifiques, numéro par numéro. 
	\end{enumerate}

	\item Votre meilleur ami est: 
	\begin{enumerate}
		\item[a.] Un membre de votre entourage.
		\item[b.] Un membre de votre famille.
		\item[c.] Un membre de la même race que la votre.
	\end{enumerate}
\end{enumerate}

Score:

Donnez-vous $1$ un point pour chaque question répondu par un "a", $5$ points pour chaque "b" et pour chaque "c". Si vous avez refait le test 3 fois et que vous avez effectué une moyenne de vos points, rajoutez $100$ points en extra. Si vous avez calculé l'erreur standard de la moyenne, rajoutez-vous $500$ points en extra.

Si vous avez un score inférieur à $10$, vous êtes normal. Un score entre $11$ et $50$ indique une obsession scientifique et vous devriez penser à joindre les \textit{Scientifiques Anonymes}. Si votre score est supérieur à $500$, oubliez les \textit{Scientifiques Anonymes} et retournez au travail car il est impossible de faire quoi que ce soit pour vous, de plus vous allez très probablement réussir en tant que scientifique.

\begin{center}\underline{\hspace{5 cm}}\end{center}

	\begin{center}
	\includegraphics{img/humour/cow.jpg}
	\end{center}
	
\pagebreak

Comment il faut comprendre certaines phrases dans les publications des physiciens:

\begin{itemize}
	\item Il est bien établi que...: Je ne me suis pas donné la peine de lire les références, mais...

	\item Ceci est de grande importance théorique: Ceci est important pour moi.

	\item Quoiqu'il n'ait pas été possible de donner une réponse définitive: L'expérience échoua, mais il me semble tout de même pouvoir en tirer une publication.

	\item La technique utilisée fut particulièrement adéquate...: Le copain du labo d'à côté avait déjà mis la technique au point.

	\item  3 échantillons furent choisis pour une étude exhaustive: Les résultats obtenus à partir des autres échantillons n'ont rien donné de cohérent .

	\item  Manipulé avec la plus grande précaution durant toute l'expérimentation: Ne fut pas jeté à l'égout.

	\item  La concordance avec la théorie est excellente: elle est passable.

	\item  La concordance avec la théorie est bonne: elle est faible.

	\item  La concordance avec la théorie est satisfaisante: elle est douteuse.

	\item  La concordance avec la théorie est passable: elle est totalement imaginaire.

	\item  Il est généralement admis que...: 2 collègues pensent comme moi

	\item  Il est admis que: Je crois que.

	\item  Il est évident que des travaux complémentaires seront utiles: Je n'ai rien compris.

	\item  Voici quelques résultats typiques: Voici les meilleurs résultats.

	\item  Significatif dans un intervalle de confiance de...: non significatif.

	\item  Les réactifs utilisés furent synthétisés au laboratoire selon des techniques standardisées: Les réactifs furent achetés chez...

	\item  Malheureusement, les bases quantitatives permettant de tirer profit des résultats n'ont pas encore été formulées: Personne n'est arrivé à comprendre quoi que ce soit à ce qui a été observé.

	\item  Nous remercions X pour sa précieuse collaboration et Y pour les discussions fructueuses: X a fait le travail et Y m'a expliqué ce que signifiaient les résultats.
\end{itemize}

	\begin{center}\underline{\hspace{5 cm}}\end{center}

	\begin{center}
	\includegraphics{img/humour/duality.eps}
	\end{center}
	
	\begin{center}\underline{\hspace{5 cm}}\end{center}
	An experimental physicist performs an experiment involving tow cats, and an inclined tin roof.
	
	The two cats are very nearly identical; same sex, age, weight, breed, eye and hair colour.
	
	The physicist places both cats on the roof at the same height and lets them both go at the same time.
	
	One of the cats fall off the roof first so obviously there is some difference between the two cats.
	
	What is the difference?
	
	One cat has a greater mew!
	
	\begin{center}\underline{\hspace{5 cm}}\end{center}
	
	\begin{center}
	\includegraphics{img/humour/three_body_problem.jpg}
	\end{center}
	
	\begin{center}\underline{\hspace{5 cm}}\end{center}
	
	\begin{center}
	\includegraphics[scale=0.85]{img/humour/bus_stop_physicists.jpg}
	\end{center}
	
	\begin{center}\underline{\hspace{5 cm}}\end{center}
	
	\begin{center}
		This is how physicists see the Pokemon:
		\includegraphics[scale=0.8]{img/humour/pokemon.jpg}
	\end{center}
	
	\begin{center}
	\includegraphics[scale=0.55]{img/humour/feynman_diagrams.jpg}
	\end{center}
	
	\begin{center}\underline{\hspace{5 cm}}\end{center}
	
	\begin{center}
	\includegraphics[scale=0.55]{img/humour/cat_physics.jpg}
	\end{center}
	
	\begin{center}\underline{\hspace{5 cm}}\end{center}
	
	\begin{center}
	\includegraphics[scale=0.7]{img/humour/sun_sohn.jpg}
	\end{center}
	
	\begin{center}\underline{\hspace{5 cm}}\end{center}
	
	\begin{center}
	\includegraphics[scale=1]{img/humour/bad_good_news_schrodinger_cat.jpg}
	\end{center}
	

	\pagebreak
	\section{Statistiques}

Trois statisticiens sortent faire du tir sur cible statique. Le premier statisticien fait feu et tire trop à gauche, le deuxième fait feu, mais tire symétriquement trop à droite. Enfin, le dernier ne tire pas, mais s'exprime triomphalement: "En moyenne nous l'avons eu!"

\begin{center}\underline{\hspace{5 cm}}\end{center}

	\begin{center}
	\includegraphics[scale=0.7]{img/humour/correlation.jpg}
	\end{center}

\begin{center}\underline{\hspace{5 cm}}\end{center}

Patient: "Vais-je survivre à cette opération délicate?"

Chirurgien: "Oui, je suis absolument sûr que vous y survivrez."

Patient: "Comment pouvez-vous être si sûr?"

Chirurgien: "9 patients sur 10 meurent lors de cette opération mon neuvième patient est mort hier!"

\begin{center}\underline{\hspace{5 cm}}\end{center}

	\begin{center}
	\includegraphics[scale=0.6]{img/humour/gauss.eps}
	\end{center}

\pagebreak
$10$ bonnes raisons pour s'orienter dans le domaine de la statistique:
\begin{enumerate}
	\item Estimer des paramètres est plus simple que de se battre dans la vraie vie

	\item Les statisticiens sont des gens reconnus

	\item Vous apprendrez l'alphabet grec en entier

	\item La probabilité que vous obteniez un job dans ce domaine est $> 0.9999$

	\item Si vous êtes virés, vous pourrez toujours vous reconvertir à l'ingénierie

	\item Vous faites ce travail dans la confidence, la régularité et la variabilité

	\item Vous êtes normal et le reste du monde est faux

	\item La ligne de régression paraît meilleure que la ligne du chômage

	\item Vous n'avez jamais besoin d'être exact - seulement approximatif

	\item Personne ne comprend ce que vous faites, alors vous avez toujours raison
\end{enumerate} 

	\begin{center}\underline{\hspace{5 cm}}\end{center}
	\begin{center}
	\includegraphics{img/humour/statistician.eps}
	\end{center}
	
	\begin{center}\underline{\hspace{5 cm}}\end{center}		
	\begin{center}
	\includegraphics[scale=0.9]{img/humour/bayesian_inference.jpg}
	\end{center}
	
	\begin{center}\underline{\hspace{5 cm}}\end{center}
	\begin{center}
	\includegraphics[scale=0.5]{img/humour/normal_distribution.jpg}
	\end{center}
	
	\pagebreak
	Un statisticien et un biologiste sont condamnés à mort. On leur accorde uen dernière faveur.
	\begin{itemize}
		\item Je voudrais donner une grande conférence sur la statistique devant tout le monde, dit le statisticien
		\item Accordé! Répond le juge. Et pour vous?
	\end{itemize}
	Le biologiste n'exprime aucune hésitation:
	\begin{itemize}
		\item Je souhaiterais être exécuté le premier!
	\end{itemize}
	\begin{center}\underline{\hspace{5 cm}}\end{center}
	\begin{center}
	\includegraphics{img/humour/bedtime_stories.jpg}
	\end{center}
		
	\pagebreak
	\section{Chimie}

On vient de découvrir un nouvel élément chimique:

\begin{itemize}
	\item[$\bullet$] ÉLÉMENT NUMÉRO: $115$

	\item[$\bullet$] NOM: Femme

	\item[$\bullet$] SYMBOLE: Fm

	\item[$\bullet$] MASSE ATOMIQUE: Acceptable 60 kg mais des isotopes connus de $40$ à $250$ kg

	\item[$\bullet$] OCCURRENCE: Très abondant de par le monde

	\item[$\bullet$] PROPRIÉTÉS PHYSIQUES:

- Entre en ébullition pour un rien et gèle sans raison

- Conductivité thermique: faible surtout aux extrémités inférieures

- Coefficient de dilatation: augmente avec les années

- Cède aux pressions appliquées aux points sensibles

	\item[$\bullet$] STRUCTURE MOLÉCULAIRE:

Parfaite? 90/60/90, existe aux USA sous forme croissante 60/90/120 et dans les pays nordiques sous forme dite plate 50/50/50

	\item[$\bullet$] PROPRIÉTÉS CHIMIQUES:

- Très grande affinité pour l'or, l'argent, le platine et tous les métaux nobles. Absorbe de grandes quantités de substances onéreuses

- Peut exploser spontanément sans avertissement

- Insoluble dans les liquides mais présente une activité grandement augmentée par saturation dans l'alcool

- Réactivité très variable selon les périodes de la journée

- Grande aptitude aux changements d'humeur et à la jalousie.

- Sensible à certaines contraintes qui lui transmettent parfois la migraine

	\item[$\bullet$] UTILISATIONS COURANTES:

- Hautement décorative surtout dans les voitures de sport

- Puissant agent nettoyant

- Aide efficace pour la relaxation et la détente

	\item[$\bullet$] TEST: 

- L'élément pur passe parfois au rose quand heureux.

- Tourne au vert si placée à côté d'un spécimen de meilleure qualité

	\item[$\bullet$] PRÉCAUTIONS D'EMPLOI:

- Hautement dangereuse si placée entre des mains non expertes

- Il est illégal d'en posséder plus d'un spécimen, mais il est possible d'en entretenir plusieurs à des endroits différents tant que les différents spécimens n'entrent pas en contact (risque d'explosion)

\end{itemize}

ATTENTION: 

Certains chercheurs d'Amérique du Sud ont découvert le moyen d'en fabriquer artificiellement, présentées généralement sous les marques "Travelo" ou "Dragqueen". Ne consommer que le produit générique.

	\begin{center}\underline{\hspace{5 cm}}\end{center}

	\begin{center}
	\includegraphics[scale=0.5]{img/humour/thorium.jpg}
	\end{center}
	
	\begin{center}\underline{\hspace{5 cm}}\end{center}

On vient de découvrir un nouvel élément chimique:

\begin{itemize}
	\item[$\bullet$] ÉLÉMENT NUMÉRO: 116

	\item[$\bullet$] NOM: Homme

	\item[$\bullet$] SYMBOLE: Hm

	\item[$\bullet$] ANALYSE QUANTITATIVE:	

Mesuré à 17 [cm], bien que quelques isotopes existent en 25, 20, 13 et même 10 [cm].

	\item[$\bullet$] DÉCOUVREUR:

Éve (découvert par accident un jour où elle avait envie de côtelettes)

	\item[$\bullet$] LIEU D'EXTRACTION:	
	
Se trouve en grandes quantités en présence d'un gisement de Fm très pur

	\item[$\bullet$] PROPRIÉTÉS PHYSIQUES:

- Surface recouverte de poils, raides par endroits, doux dans d'autres

- Bout quand on l'agite, se glace quand on le met en présence de la logique et du bon sens, se liquéfie quand on le traite comme un dieu

- Devient exécrable lorsqu'on le mélange à n'importe quel alcool

- Peut être la cause de maux de tête (ou des maux d'autres parties du corps); à manipuler avec précaution

- Diminue son entropie directement après sa réaction avec l'élément Fm (état se manifestant par des ronflements... zzzzz)

- Augmente sa masse considérablement en vieillissant, perd de ses capacités réactionnelles

- Se déshydrate rapidement par temps sec

- Rarement trouvé à l'état pur après 14 ans

- Possède souvent un attachement inexplicable à sa roche mère, rendant l'extraction difficile

- Si on le met sous pression, devient trop dur et improductif; n'est productif que si l'on utilise la subtilité, les subterfuges, et la flatterie

	\item[$\bullet$] PROPRIÉTÉS CHIMIQUES:

- Tendance très forte à réagir avec l'élément Fm, même si la réaction est parfois endothermique

- Réputé être le meilleur catalyseur pour les réactions de transformation de l'élément Fm

- Possède la faculté d'entrer en réaction avec à peu près n'importe quoi

- En cas de réaction importante, l'aspect de l'élément change pour virer au rouge cramoisi.

- S'il est saturé en alcool, il devient inerte et repoussant pour la plupart des éléments

- Ne convient pas pour les tâches ménagères et les opérations de nettoyage

- Ne convient pas non plus pour les tâches familiales

- Est neutre en ce qui concerne la courtoisie et l'impartialité

	\item[$\bullet$] USAGES COURANTS:

- Transport de choses lourdes, chauffeur, dîners gratuits au restaurant...

- Usage possible pour les activités sexuelles

	\item[$\bullet$] TESTS:

Les spécimens les plus purs ne sont pas synonymes de pureté, et ceux qui ont déjà servi, encore moins

	\item[$\bullet$] DANGERS:

La réaction avec un autre élément Hm est extrêmement violente si l'élément Fm est le catalyseur
\end{itemize}

\begin{center}\underline{\hspace{5 cm}}\end{center}

La question suivante a réellement été posée en ces termes à l'université de chimie de Washington:

L'Enfer est-il exothermique (dégage-t-il de la chaleur) ou endothermique (absorbe-t-il de la chaleur)? Appuyez votre réponse avec une preuve.

La plupart des étudiants écrivirent comme preuve de leurs théories la loi de Boyle (les gaz se réchauffent quand ils sont comprimés et se refroidissent quand ils se décompriment) ou une variante.

Un étudiant, toutefois, a écrit ce qui suit: 

Premièrement, nous avons besoin de savoir comment la masse de l'Enfer évolue dans le temps. Ce qui signifie aussi que nous avons besoin de connaître le rythme auquel les âmes vont en Enfer et le rythme auquel elles en sortent. Je pense que nous pouvons sans crainte affirmer qu'une fois qu'une âme est en Enfer, elle n'en sortira plus. Par conséquent, aucune âme ne sort des enfers.

Pour ce qui est des nombreuses âmes qui vont en Enfer, examinons les différentes religions qui existent de par le monde aujourd'hui. Certaines d'entre elles décrètent que si vous n'êtes pas membre de leur religion, vous irez en Enfer. Depuis qu'il y a plus d'une religion de cette sorte et depuis que les gens ne pratiquent qu'une seule religion, nous pouvons en déduire que presque tout le monde et toutes les âmes vont en Enfer.

Avec le rythme des naissances et des morts qui sont ce qu'ils sont, nous pouvons nous attendre à ce que le nombre des âmes en Enfer augmente de façon exponentielle.

Maintenant occupons-nous du rythme d'évolution du volume de l'Enfer, parce que la loi de Boyle prédit que pour que la température et la pression restent les mêmes, le volume de l'Enfer doit s'agrandir proportionnellement aux âmes qui s'ajoutent.

Ceci nous donne deux possibilités:

\begin{enumerate}
	\item Si l'Enfer croît a un rythme plus lent que celui des âmes qui arrivent en Enfer, alors la température et la pression s'accroissent jusqu'à ce que l'Enfer craque de partout.

	\item Bien sûr, si l'Enfer s'agrandit à un rythme plus rapide que le nombre d'âmes en Enfer s'accroît, alors la pression et la température baissent jusqu'a ce que l'Enfer gèle tout entier.
\end{enumerate}

\begin{center}\underline{\hspace{5 cm}}\end{center}

Un chimiste entre dans un pharmacie et demande à son pharmacien "Avez-vous de l'acide acetylsalicylic?"


\begin{itemize}
	\item[$-$] "Vous voulez parler d'aspirine?" Demanda le pharmacien.

	\item[$-$] "Oui c'est cela! Je n'arrive jamais à m'en rappeler le nom."
\end{itemize}

\begin{center}\underline{\hspace{5 cm}}\end{center}

Un physicien, un biologiste et un chimiste se rendent à l'océan pour la première fois. 

\begin{itemize}
	\item Le physicien en voyant les vagues de l'océan fut fasciné par les vagues. Il désira faire quelques recherche sur la dynamique des fluides et marcha dans l'océan. Malheureusement, il se noya et ne revint jamais. 

	\item Le biologiste désira faire quelques recherches sur la faune et la flore marine et marcha lui aussi dans l'océan. Il ne revint jamais non plus.

	\item Le chimiste attenda lui pendant un très long moment, notant ses observations: "Les physiciens et les chimistes sont solubles dans l'eau".
\end{itemize}

\begin{center}\underline{\hspace{5 cm}}\end{center}

CLASSIFICATION DE LA CHIMIE

\begin{itemize}
	\item \textit{Chimie physique}: L'art d'appliquer $y=mx+b$ à tout et n'importe quoi dans l'univers.

	\item \textit{Chimie organique}: L'art de transmuter des substances en publications.

	\item \textit{Chimie inorganique}: C'est ce qui reste après le chimie organique, analytique et physique une fois que l'on a utilisé tout le tableau périodique des éléments.

	\item \textit{Ingénierie chimique}:  L'art de tirer un profit de ce que le chimiste organique fait seulement pour le fun.
\end{itemize}
\begin{center}\underline{\hspace{5 cm}}\end{center}

\begin{center}
\includegraphics[scale=0.7]{img/humour/cute.jpg}
\end{center}

\begin{center}\underline{\hspace{5 cm}}\end{center}

Les radicaux libres ont révolutionné la chimie!

\begin{center}\underline{\hspace{5 cm}}\end{center}

Les derniers mots des chimistes: 

\begin{itemize}
	\item Et maintenant la phase de goûtage...

	\item  Et maintenant secouons un petit peu

	\item  Dans quelle bouteille était déjà mon eau minérale?

	\item  Pourquoi ce truc fait une flamme verte?!?

	\item  Et maintenant le problème de détonation du gaz.

	\item  Ceci est une expérience complétement sûre.

	\item  Maintenant vous pouvez retirer la fenêtre de protection...

	\item  D'où viennent tous ces petits trous sur ma blouse de travail?

	\item  Et maintenant un cigarette...
\end{itemize}

	\pagebreak
	\section{Ingénierie}

	Des scientifiques de la NASA ont développé un fusil spécialement conçu pour projeter des poulets morts dans les pare-brises des avions de ligne, des jets militaires et des navettes spatiales. Le but étant de vérifier les conséquences de possibles collisions avec des volatiles et d'adapter les matériaux des pare-brises en conséquence.

	Les ingénieurs britanniques ayant entendus parler de ce genre de fusil ont de suite développés un identique pour effectuer le même type de tests sur leurs trains à très grande vitesse. Lorsque le premier test fut effectué, le poulet traversa le pare-brise, la console de commande, la motorisation pour finir sa course dans le mur opposé. 

	Horrifiés et étonnés par ces résultats, les ingénieurs britanniques les communiquèrent  à leurs confrères américains afin d'obtenir des suggestions de leur part. La NASA leur répondit en une seule phrase: "Décongelez le poulet avant!"

	\begin{center}\underline{\hspace{5 cm}}\end{center}
	
	\begin{center}
	\includegraphics[scale=0.3]{img/humour/great_power_great_bills.jpg}
	\end{center}

	\begin{center}\underline{\hspace{5 cm}}\end{center}

Deux ingénieurs et un ami non-ingénieurs se rencontrent à un bar un vendredi soir pour raconter leur semaine de travail.

	\begin{itemize}
		\item Le premier ingénieur: "J'ai passé un semaine horrible à faire des plans un à la fois chaque jour."
	
		\item Le deuxième ingénieur: "J'ai fait un peu moins pire. J'ai au moins pu faire des plans complets plusieurs fois par jour."
	
		\item Le troisième ami non-ingénieur: "Ben les gars vous en avez de la chance! Moi je me limite à des plans culs qu'une fois par mois".
	\end{itemize}
	\begin{center}\underline{\hspace{5 cm}}\end{center}

	\begin{center}
	\includegraphics{img/humour/acdc.jpg}
	\end{center}

	\begin{center}\underline{\hspace{5 cm}}\end{center}

Tentatives pour comprendre les ingénieurs:

\begin{itemize}

	\item Tentative N\degree 1

	Deux élèves ingénieurs marchent le long de leur campus lorsque l'un des deux dit à l'autre, admiratif: "Où est-ce que tu as trouvé ce vélo ?" Le second lui répond: "Ben en fait, alors que je marchais, hier, et que j'étais dans mes pensées, je croise une super nana en vélo qui s'arrête devant moi, pose son vélo par terre, se déshabille entièrement et me dit: "Prends ce que tu veux... ". J'ai donc choisi son vélo. 
	
	Le premier lui répond: "Tu as raison, les vêtements auraient certainement été trop serrés.

	\item Tentative N\degree 2 

	Pour une personne optimiste, le verre est à moitié plein.
	Pour une personne pessimiste, il est à moitié vide.
	Pour l'ingénieur, il est deux fois plus grand que nécessaire.

	\item Tentative N\degree 3 

	Un pasteur, un médecin et un ingénieur jouent au golf. Ils attendent après un groupe de golfeurs particulièrement lents. Au bout d'un moment, l'ingénieur explose et dit: "Mais qu'est-ce qu'ils fichent? ça fait bien un quart d'heure qu'on attend là !" Le docteur intervient, exaspéré lui aussi: "Je ne sais pas, mais je n'ai jamais vu des gens s'y prendre aussi mal !" Le pasteur dit alors: "Attendez, voilà quelqu'un du golf. On n'a qu'à le lui demander. Dites-moi, il y a un problème avec le groupe de devant. Ils sont plutôt lents, non ?" L'autre répond: "Ah oui, c'est un groupe de pompiers aveugles. Ils ont perdu la vue en tentant de sauver le golf des flammes l'année dernière, alors depuis, on les laisse jouer gratuitement". Le groupe reste silencieux un moment, et le pasteur dit: "C'est si triste. Je vais faire une prière spécialement pour eux ce soir". Le médecin ajoute: "Bonne idée. Et moi, je vais contacter un copain chercheur ophtalmologiste pour voir ce qu'il peut faire". Et l'ingénieur: "Mais putain ! Pourquoi ils jouent pas la nuit ?"

	\item Tentative N\degree 4 

	Un ingénieur traversait la rue lorsqu'une grenouille l'appela et lui dit: "Si tu m'embrasses, je me transformerai en une magnifique princesse". Il se baissa, ramassa la grenouille et la mit dans sa poche. La grenouille lui dit alors: "Si tu m'embrasses, je me transformerai en une magnifique princesse et je resterai à tes côtés pendant une semaine". L'ingénieur sortit la grenouille de sa poche, lui fit un sourire et la replaça dans sa poche. La grenouille se mit alors à crier: "Si tu m'embrasses, je me transformerai en une magnifique princesse, je resterai à tes côtés pendant une semaine et je ferai TOUT ce que tu veux". Encore une fois, l'ingénieur sortit la grenouille de sa poche, lui sourit et la remit dans sa poche. La grenouille lui demanda alors: " Quoi, qu'est-ce qu'il y a ? Je te dis que je suis une magnifique princesse, que je resterai à tes côtés pendant une semaine et que je ferai tout ce que tu veux. Alors pourquoi tu ne m'embrasses pas ?" L'ingénieur répondit: "Regarde-moi, je suis un ingénieur. J'ai pas le temps d'avoir une petite amie. Par contre, une grenouille qui parle, ça, c'est cool!"

	\item Tentative N\degree 5 

	Un journaliste interviewe un paysan corse: "Dites-moi, comment faites-vous pour tracer les routes ici? ". Le paysan répond: "beh, on lâche un âne et on regarde par où il passe dans la montagne....et c'est là qu'on fait passer la route". Le journaliste alors rétorque: "et si vous n'avez pas d'âne?". Ce a quoi le paysan répond: "ah....beh on prend un ingénieur.... ".
\end{itemize}

	\begin{center}\underline{\hspace{5 cm}}\end{center}
	
	\begin{center}
	\includegraphics[scale=0.25]{img/humour/weather_forecast.jpg}
	\end{center}
		
	Lors de la course à la conquête spatiale dans les années 1960, la NASA décida pour le besoin de ses astronautes de développer un stylo à bille fonctionnant dans un système sans gravité.

	Après un temps considérable de recherche et de développements, le stylo "Astronaute" fut développé pour un coût de 1 million de dollars. Le stylo fonctionna très bien mais trouva un faible intérêt de la part du public sur Terre.

	L'Union Sovétique, se trouvant confronté au même problème, utilisa elle un crayon...

	\begin{center}\underline{\hspace{5 cm}}\end{center}
	
	Le grand mathématicien John Von Neumann fut consulté par un groupe qui fabriquait une fusée destinée à être envoyée dans l'espace. Quand il vit la structure actuelle, il demanda: "Où avez-vous obtenu les plans pour cette fusée?".

	On lui donna comme réponse: "Nous avons notre équipe d'ingénieur"

	Il répliqua alors désolé: "Des ingénieurs! Mais pourquoi? J'ai déjà publié toute la théorie mathématique sur la science des fusées. Voyez ma pubication de 1952.".

	Alors, le groupe consultat son document de 1952 et investirent 10 millions de dollar dans la structure et reconstruisirent toute la fusée conformément aux plans de Von Neumann. À la minute du lancement, la structure entière se désagréga. En colère, ils contactères alors Von Neumann et lui dirent: "Nous avons suivi vos instructions et quand nous avons démarré l'engin, il a explosé! Pourquoi?".

	Ce à quoi Von Neumann répliqua: "Ah oui! C'est un phénomène connu sous le nom technique du problème d'explosion - Je l'ai traité dans mon article de 1954."

	\begin{center}\underline{\hspace{5 cm}}\end{center}
	
	Dans un laboratoire d'électronique:
	
	\begin{itemize}
		\item Dis-donc, c'est quoi le semi-remorque sur le parking devant?
	
		\item Le semi-remorque?
	
		\item Ben oui, le chauffeur dit que tu es au courant...
	
		\item Ah oui!! J'ai commandé un condensateur de 1 Farad.
	\end{itemize}

	\begin{center}\underline{\hspace{5 cm}}\end{center}
	
	\begin{figure}[H]
		\begin{center}
		\includegraphics[scale=0.2]{img/humour/iso.jpg}
		\end{center}	
	\end{figure}
	
	\begin{center}\underline{\hspace{5 cm}}\end{center}

	Ce que les ingénieurs disent et ce qu'ils pensent en réalité:

	\begin{itemize} 
		\item Avancée technologique majeure: Retour au tableau noir

		\item Développé après des années de recherche intensives: Découvert par accident

		\item Le design est compris dans les limites imposées: Nous venons de le faire en retouchant un peu

		\item Les résultats ont été extrêmement satisfaisants: Par surprise cela cela marché!

		\item La satisfaction du client sera certainement assurée: Nous somme si loin dans des délais que le client était content de ne rien avoir du tout.

		\item Nous avons une forte coordination dans le projet: Nous aurions dû demander à quelqu'un d'autre

		\item Le projet est un peu hors délai à caus d'imprévus: Nous travaillons sur autre chose

		\item Le design sera finalisé dans la prochaine période de rapport: Nous n'avons pas encore commencé le travail mais nous devions dire quelque chose

		\item Un certain nombre d'approches sont tentées: Nous ne savons pas où nous allons, mais nous avancons...

		\item Des efforts intensifs sont mis en oeuvre pour avoir une nouvelle perspective sur le problème: Nous venons de licencié trois gars

		\item Les tests opérationnels préliminaires ne sont pas concluants: Le truc a explosé quand nous l'avons testé

		\item Le concept entier a été abandonné: Le seul gars qui comprenait le concept est parti ailleurs

		\item Des modifications sont en cours pour corriger quelques difficultés mineures: Nous avons jeté le tout et recommancé depuis le début.
		
		\item Nous sommes presque au bout: Nous avons fait la moitié

		\item Nous avons prévu: Nous croyons en Dieu!

		\item Les dessins techniques sont un peu en retard: Nous n'avons pas le moindre dessin

		\item Le risk est élevé mais acceptable: Nous n'avons aucune chance mais avec 10 fois le budget actuel et 10 fois plus de temps et de ressources, nous avons une chance sur deux de réussir.

		\item C'est un problème sérieux mais pas insurmontable: C'est un miracle si nous trouvons une solution. Il faudrait avoir Dieu comme chef de projet.

		\item Le cahier des charges n'est pas très bien défini: Personne n'a pensé à faire un cahier des charges

		\item Cela nécessitera des analysises et un suivi ultérieur: C'est totallement hors de contrôle

		\item Le projet a été concu pour une haute disponibilité: Les disfonctionnements seront mis sur le dos des opérateurs.

		\item Ce projet nécessite peu d'opérations de maintenance: Nous ne laisserions pas un technicien changer une ampoule, autant jeter l'eau du bain avec le bébé.

		\item Le logiciel est en cours de développement sans manquement au cahier des charges: La documentation sera rédigée en chinois et en chine faute de temps

		\item La livraison est prévue pour le dernier trimestre de l'année prochaine: Cela nous laisse du temps pour décider qui blâmer d'être actuellement en retard.
	\end{itemize}
	
	\begin{center}\underline{\hspace{5 cm}}\end{center}

\begin{itemize} 
	\item Combien d'étudiants ingénieurs de première année sont nécessaires pour changer une ampoule?: Aucun. C'est un sujet de deuxième année.

	\item  Combien d'étudiants ingénieurs de deuxième année sont nécessaires pour changer une ampoule?: Aucun. Un, mais le reste de la classe copie le rapport.

	\item  Combien d'étudiants ingénieurs de troisième année sont nécessaires pour changer une ampoule?: Est-ce que cette question sera dans l'examen final?

	\item  Combien d'ingénieurs civils sont nécessaires pour changer une ampoule?: Deux. Le premier change l'ampoule et le second tient la bougie.

	\item  Combien d'ingénieurs électricient sont nécessaires pour changer une ampoule?: Aucun. Ils redéfinissent simplement l'obscurité comme nouveau standard industriel.

	\item  Combien d'ingénieurs informaticiens sont nécessaires pour changer une ampoule?: Pourquoi changer? De toute façon le connecteur ne sera plus valable d'ici six mois.

	\item  Combien d'ingénieurs mécaniciens sont nécessaires pour changer une ampoule?: Cinq! Un décide la manière de changer l'ampoule, le deuxième calcule la force nécessaire, le troisième concoit un outil pour tourner l'ampoule, le quatrième pour concevoir un outi confortable mais fonctionnel pour attraper l'ampoule et le cinquième qui utilie les résultats des quatre premiers equipment. 

	\item  Combien d'ingénieurs nucléaires sont nécessaires pour changer une ampoule?: Sept! Un pour installer la nouvelle ampoule et 6 pour trouver quoi faire avec l'ancienne pour les 10'000 prochaines années. 
\end{itemize}

\begin{center}\underline{\hspace{5 cm}}\end{center}

	\begin{figure}[H]
		\centering
		\includegraphics[scale=1]{img/humour/2bornot2b.jpg}
	\end{figure}
	
\begin{center}\underline{\hspace{5 cm}}\end{center}

Cela se passe à Moscou: un couple de touristes demande son chemin, sur un pont à un russe qui se trouve être ingénieur.

Le gars leur dit: «Vous traversez, et, d'ici 50 mètres, vous tournez à droite...»

Remerciements de la part des touristes... , puis ils partent. Alors le gars leur court derrière:

«Attendez, attendez ! Je viens de me souvenir que le pont fait 70 mètre. Si vous tournez à droite au bout de 50 mètres, comme je vous l'ai dit, vous tomberez à l'eau.»

\begin{center}\underline{\hspace{5 cm}}\end{center}

Une équipe d'ingénieur qualité travaille sur l'AMDEC d'un nouvelle usine chimique. Après plusieurs semaines, pendant la réunion de debriefing:

Ingénieurs Qualité: «Notre conclusion est: Il y a 1 chance sur 10'000 que l'usine explose, tuant beaucoup de gens et induisant un terrible impact écologique, c'est éthiquement inacceptable!».

Manager: «Les normes parlent de risque acceptable si le taux est à 1/7'000 pourtant?!»

L'équipe d'ingénieurs se regroupe et prend alors la parole:

Ingénieurs Qualité: «Notre conclusion est: Vous avez un problème de surqualité, c'est éthiquement inacceptable!»

\begin{center}\underline{\hspace{5 cm}}\end{center}

	\begin{figure}[H]
		\centering
		\includegraphics[scale=1]{img/humour/airplane_magic.jpg}
	\end{figure}

\begin{center}\underline{\hspace{5 cm}}\end{center}

Un homme était assis à côté d'une fille de 10 ans dans un avion. S'ennuyant, il se tourna vers la fille et dit: «Parlons! J'ai entendu dire que les vols vont plus vite si vous entamez une conversation avec votre compagnon de route.»

La fille qui lisait un livre la ferma lentement et dit au gars: «De quoi voudriez-vous parler?»

«Oh, je ne sais pas» dit l'homme. «De physique nucléaire?»

«OK» disa-t-elle. «Cela pourrait être un sujet intéressant. Mais laissez-moi d'abord vous poser une question! Un cheval, une vache et un cerf mangent tous la même chose ... de l'herbe. Pourquoi un cerf excrète de petites boulettes, une vache une flaque et un cheval produit des touffes d'herbes sèches?»

L'homme a réfléchi et a dit: «Hmmm, je n'en ai aucune idée.»

Alors, sarcastique, la petite fille lui dit : «Comment voulez vous que je vous explique ce qu'est la physique nucléaire alors que vous ne maîtrisez même pas un petit problème de merde ?»

\begin{center}\underline{\hspace{5 cm}}\end{center}

	\begin{center}
	\includegraphics[scale=3]{img/humour/fourier_transform.jpg}
	\end{center}
	
\begin{center}\underline{\hspace{5 cm}}\end{center}

Newton asked: How write $4$ in between $5$?

\begin{enumerate}
	\item Medicine students said: Joke!
	
	\item Science students said: Impossible!
	
	\item Management students said: Not found on the internet!
	
	\item Engineering student said: "F(IV)E"
\end{enumerate}

	\pagebreak
	\section{Informatique}

	\begin{center}
	\includegraphics{img/humour/meaning_life.jpg}
	\end{center}
	
\begin{center}\underline{\hspace{5 cm}}\end{center}	

Comment devenir un bon Hacker ?
	
	Bon, si tu veux être un vrai hacker, il va te falloir Linux.
	
	Là, tu as 2 solutions:
	\begin{enumerate}
		\item Tu es un sale bourgeois capitaliste et tu l'achètes 150 balles à la FNAC.
		
		\item Tu es un vrai trou du cul, et là tu le downloades par le Net.
	\end{enumerate}
	
	Évidemment, tu es un vrai trou du cul donc tu ouvres ton tit client FTP et tu te tapes tranquillement les 20 ou 25 heures de download pour une Slack ou une Debian. Evite la Red Hat, ça fait trop grand public, toi t'es un mec uNdERgrOuNd maintenant, c'est normal, t'es Hacker.
	
	Bon, tu as ton Linux, maintenant c'est bon oublie-le. Pas la peine de se casser le cul à apprendre un nouvel OS dont tu ne te serviras jamais parce que "XWing vs Tie Fighter" tourne pas dessus. La meilleure solution consiste carrément à niquer lilo, comme ça tu es sûr que tu ne booteras que sous Windows 95. C'est une solution élégante que de nombreux trous du cul semblent avoir choisie. Pour ça, ouvre une session DOS par Windows et tape fdisk /mbr. Ca va effacer lilo qui était installé sur le MBR de ton disque dur, comme ça tu n'auras plus à te soucier de Linux.
	
	L'essentiel est de l'avoir, pas de savoir s'en servir.
	
	"Ouais mais comment je peux prouver aux gens que j'ai Linux et passer pour un gros rebelle ?"
	
	C'est une question bien naturelle. J'ai pensé à toi petit looser et voici une série de phrases qu'il faut balancer à propos de Linux:
	
	\begin{itemize}
		\item "Linux c'est trop puissant, t'es complètement libre par rapport à ces OS de fachos genre Windaube. De toute façon, MS c'est trop ripoux."
	
		\item "Bah si t'es un débutant, va pas sous Linux, c'est fait pour les eLiTeS ce truc, toi reste sous winfuck."
	
		\item "Dis, tu sais pas ou je pourrais trouver la libc5.4.36 ? Parce que chez moi la 5.4.35 est incompatible avec les modifs que j'ai faites au kernel."
	
		\item "Ça pue netscape, moi ça me dumpe des cores de 10 mégas dès que le lance, je préfère Lynx au moins c'est pas prise de gueule c'est mieux le mode texte."
	
		\item "Ralala le bouffon que c'est lui ! Il s'est installé une Red Hat !! Tain c'est de la daube les Red Hat c'est nul y'a que la Debian qui est bien, au moins tu sais ce que tu fais t'es le master de ton system nan vraiment c'est ripoux Red Hat."
	\end{itemize}
	
	Avec ce genre de petites phrases, tu te retrouveras très vite classé dans la catégorie "OK c'est un trou du cul, mais un trou du cul sous Linux", ce qui est la première étape pour être un vrai hacker. Maintenant que tout le monde sait que tu as ton Linux, il faut passer au stade suivant, celui du pro des réseaux, genre le mec qui maîtrise ICMP à mort. C'est la deuxième étape de ton long périple.
	
	Ici, il faut mettre la main au porte-monnaie. Direction la FNAC, tu achètes n'importe quel bouquin sur Unix et sur les réseaux. L'essentiel est que le titre soit compliqué. Un petit "Protocole rlogin sur réseau Ethernet en sous-adressage" sera du meilleur effet. N'hésite pas, dès que tu ne comprends même pas le titre, il faut acheter le bouquin: c'est pas pour lire, c'est pour impressionner tes autres potes trous du cul.
	
	Une bonne méthode consiste à acheter un bouquin genre "TCP/IP volume 43" et de prendre des mots au hasard à apprendre par coeur: rai socket, sur-adressage, FDDI, telnet par exemple. Ensuite, tu les ressors dans une phrase, même hors contexte c'est pas grave personne n'ira vérifier ce que ça veut dire. Par exemple, il ne faut pas hésiter à balancer un "Le Telnet, ça prends combien de rai sockets en sur-adressage sur un FDDI" sur un bon gros channel de cowboyz, ça impressionne toujours, et personne n'ira te dire que ça n'a aucun sens, ne t'inquiète pas.
	
	Dispose ensuite ces bouquins dans ta chambre, avec les titres les plus compliqués aux endroits les plus visibles. Corne quelques pages pour faire plus vrai. Prends aussi quelques feuilles et dessine des schémas bidons de réseau, ou met des trucs genre 123.44.5.34 root/lydia pour faire croire que tu te chopes de password comme un ouf. Faut se la jouer à mort, ne jamais hésiter à en rajouter, scanne-toi une photo de Mitnick et accroche la au-dessus de ton lit, ou met des autocollants à tête de mort sur ton UC pour bien dire que maintenant, t'es un voyou, un mec dangereux.
	
	Pour compléter le tout et vraiment passer pour un hacker, il ne faut pas hésiter à dire des conneries du genre "Je ne suis qu'un assoiffé de connaissances". OK, tu quadruples ta seconde, mais bon c'est pas grave, tu aimes quand même apprendre, c'est ta grande passion et tu as beaucoup de volonté. Précise bien que jamais tu ne causes de dégâts aux très nombreuses machines que tu pénètres, dis que tu fais juste ça "pour le challenge intellectuel". Oui, là, il faudra te forcer pour ne pas exploser de rire, mais entraîne-toi devant ta glace avant.
	
	Quand on est un mec dangereux comme toi, on doit se réunir avec d'autres bandits pour mettre en péril la sûreté de l'État. Pour ça, il existe LE rendez-vous de toute la racaille, c'est le "Meet 2600". Tous les mois, tu iras dans un MacDo de Paris, Place d'Italie, et là tu rencontreras des grands monsieurs, des mecs qui ont rebooté tout Internet avec un prog en Visual Basic et qui ont des coupes de cheveux de rebelles de la société.
	
	Bon, tu n'y apprendras pas grand chose, les loosers qui viennent là-bas se branlent entre eux en se disant "Ouais, on est des hAcKeRz, on est sans pitié, on est des vrais durs, oh zut il est déjà 18 heures faut que je rentre ma mère va me cogner sinon". Tu pourras quand même avoir un vrai frisson en t'imaginant que le MacDo est truffé de caméras et de micros, et que tous les employés sont des agents de la DST qui écoutent des conversations aussi dangereuses que:
	
	\begin{itemize}
		\item[$-$] Trouduku1: il est à combien le Whooper ?
		\item[$-$] Trouduku2: euh MacDo fait des Whoopers maintenant ?
		\item[$-$] Trouduku1: bah ouais ils en ont toujours fait nan ? 
	\end{itemize}
	
	La communauté des hAcKeRz aime bien aussi les raves. Ça fait partie du trip "rebel no future fuck da society, on gobe des extas on écoute de la musique de daube mais on s'en fout c'est super parce que c'est interdit ". N'hésite pas à te rendre là-bas, ça fait incontestablement partie de la culture du paumé que d'aller jouer les chauds dans ces soirées.
	
	Toi, t'es un vrai trou du cul qui hacke, et tu entends bien répandre ton savoir pour former d'autres minables comme toi. Pour ça, il existe les e-zine. On peut citer les plus connus comme NoWay ou NoRoute ou le pire cotoie le meilleur (et c'est dommage pour le meilleur...) mais aussi des vraies merdes qui mériteraient d'être plus connues, comme l'excellent Core-Dump qui est une véritable farandole de guignolos expliquant des trucs archi connus dans un français que mon chat comprends mieux que moi.
	
	Évidemment, tu n'as pas lu les bouquins sur Unix, tu n'as jamais hacké la moindre machine de ta vie donc tu ne sais pas quoi écrire. Rassure-toi, tu n'es pas le seul dans ce cas. La meilleure méthode est de pondre un article sur le rap, à raconter sa dernière rave ou à pomper Phrack sans rien comprendre. Là encore, si tu pompes Phrack, n'hésite pas à carrément corriger le mec ou à rajouter des trucs pour faire plus compliqué, personne n'ira vérifier, donc vas-y lâche-toi t'es un assoiffé de connaissances, oublies pas.
	
	Maintenant, c'est clair, tu es un vrai hacker, une racaille de l'IRC, un loubard d'Internet, tu fais peur à toutes les agences gouvernementales et IBM veut t'embaucher pour sécuriser leur réseau parce que cette pédale d'Henri leur a encore collé un virus d'Internet. Il va donc falloir, au quotidien, se comporter comme un hacker, un vrai, un dur, c'est à dire avec un esprit hacker et un langage de hacker.
	
	Un hacker, ça vit avant tout sur IRC. Une fois que tes amis et ta famille auront bien vu que tu as changé, que tu n'es plus le même homme, il va falloir répandre aussi la nouvelle sur IRC et te faire des nouveaux amis qui seront comme toi des trous du cul. Fini les \#coquelicots ou les \#amitié\_fr, maintenant tu devras aller dans les bas fond de l'IRC, le cyber-Bronx, nuke-city, là où seuls les vrais cogneurs réussissent à se faire une place dans cet univers de violence. Pour ça, tu vas devoir passer du stade hacker trou du cul à celui de trou du cul sur IRC qui se la pète Mitnick, à savoir le c0wb0y.

\begin{center}\underline{\hspace{5 cm}}\end{center}	

	\begin{center}
	\includegraphics[scale=0.8]{img/humour/quantum_computing.eps}
	\end{center}
	
\begin{center}\underline{\hspace{5 cm}}\end{center}

Les 20 meilleures réponses des programmeurs lorsque leurs programmes ne fonctionnent pas:
	\begin{enumerate}[nolistsep]
		\item[20.] "C'est bizarre ..."
		\item[19.] "Cela n'a jamais fait cela auparavant."
		\item[18.] "Cela a fonctionné hier."
		\item[17.] "Comment est-ce possible?"
		\item[16.] "Ce doit être un problème de matériel."
		\item[15.] "Qu'est-ce que vous avez mal saisi pour le faire bugger?"
		\item[14.] "Il y a quelque chose de bizarre dans vos données."
		\item[13.] "Je n'ai pas touché à ce module depuis des semaines!"
		\item[12.] "Vous devez avoir la mauvaise version."
		\item[11.] "C'est juste une coïncidence malchanceuse."
		\item[10.] "Je ne peux pas tout tester!"
		\item[9.] "CECI ne peut pas être la source de CELA."
		\item[8.] "Ca marche, mais ça n'a pas été testé."
		\item[7.] "Quelqu'un doit avoir changé mon code."
		\item[6.] "Avez-vous détecté un virus sur votre système?"
		\item[5.] "Même si ça ne marche pas, comment ça se passe?
		\item[4.] "Vous ne pouvez pas utiliser cette version sur votre système."
		\item[3.] "Pourquoi voulez-vous le faire comme ça?"
		\item[2.] "Où étiez-vous quand le programme a explosé?"
	\end{enumerate}

	Et la réponse Numéro Un par les programmeurs lorsque leurs programmes ne fonctionnent pas:
	\begin{enumerate}
		\item "Cela fonction pourtant sur ma machine!"		
	\end{enumerate}


	\begin{center}\underline{\hspace{5 cm}}\end{center}	
	
	\begin{center}
	\includegraphics[scale=0.45]{img/humour/programing_languages.jpg}
	\end{center}
	
Cherchez votre niveau en programmation dans l'échelle suivante, l'objectif étant d'écrire un programme affichant "hello world" à l'écran:

\begin{itemize}
	\item Collège:

 \texttt{10 PRINT "HELLO WORLD"\\
20 END}

	\item 1ère année du Lycée

\texttt{program Hello(input, output)\\
begin\\
writeln('Hello World')\\
end.}

	\item Terminale du Lycée:
	
\texttt{(defun hello\\
(print\\
(cons 'Hello (list 'World))))}

	\item Nouveau sur le marché de l'emploi

\texttt{\#include <stdio.h>\\
void main(void)\\
\{\\
char *message[] = \{"Hello ", "World"\};\\
int i;\\
for(i = 0; i < 2; ++i)\\
printf("\%s", message[i]);\\
printf("\\n");\\
\}}

	\item Professionnel chevronné
	
\texttt{\#include <iostream.h>\\
\#include <string.h>\\
class string\{\\
private:\\
int size;\\
char *ptr;\\
public:\\
string() : size(0), ptr(new char('\textbackslash 0')) \{\}\\
string(const string \&s) : size(s.size)\\
\{\\
ptr = new char[size + 1];\\
strcpy(ptr, s.ptr);\\
\}\\
~string()\\
\{\\
delete [] ptr;\\
\}\\
friend ostream \& operator <<(ostream \& , const string \& );\\
string \& operator=(const char *);\\
\};\\
ostream \& operator<<(ostream \& stream, const string \& s)\\
\{\\
return(stream << s.ptr);\\
\}\\
string \&string::operator=(const char *chrs)\\
\{\\
if (this != \& chrs)\\
\{\\
delete [] ptr;\\
size = strlen(chrs);\\
ptr = new char[size + 1];\\
strcpy(ptr, chrs);\\
\}\\
return(*this);\\
\}\\
int main()\\
\{\\
string str;\\
str = "Hello World";\\
cout << str << endl;\\
return(0);
\}
}

	\item Administrateur système

\texttt{\#include <stdio.h>\\
\#include <stdlib.h>\\
main()\\
\{\\
char *tmp;\\
int i=0;\\
tmp=(char *)malloc(1024*sizeof(char));\\
while (tmp[i]="Hello Wolrd"[i++]);\\
i=(int)tmp[8];\\
tmp[8]=tmp[9];\\
tmp[9]=(char)i;\\
printf("\%s \textbackslash n",tmp);\\
\}\\
}

	\item Apprenti Hacker

\texttt{\#!/usr/local/bin/perl\\
\$msg="Hello, world.\textbackslash n";\\
if (\$\#ARGV >= 0) \{
while(defined(\$arg=shift(@ARGV))) \{\\
\$outfilename = \$arg;\\
open(FILE, ">" . \$outfilename) || die "Can't write \$arg: \$! \textbackslash n";\\
print (FILE \$msg);\\
close(FILE) || die "Can't close \$arg: \$!\textbackslash n";\\
\}
\} else \{\\
print (\$msg);\\
\}\\
1;}

	\item Hacker expérimenté

\texttt{\#include <stdio.h>\\
\#include <string.h>\\
\#define S "Hello, World\\n"\\
main()\{exit(printf(S) == strlen(S) ? 0 : 1);\}}

	\item Hacker chevronné

\texttt{\%cc -o a.out ~/src/misc/hw/hw.c\\
\%a.out\\
Hello, world.\\
Guru Hacker\\
\%cat
Hello, world.}

	\item Manager junior

\texttt{10 PRINT "HELLO WORLD"\\
20 END}

	\item Manager

\texttt{mail -s "Hello, world." bob@b12\\
Bob, could you please write me a program that prints "Hello, world."?\\
I need it by tomorrow.\\
\^D\\}

	\item Manager senior

\texttt{\%zmail jim\\
I need a "Hello, world." program by this afternoon.}

	\item Directeur

\texttt{\%letter\\
letter: Command not found.
\%mail\\
To: \^X \^F \^C\\
\%help mail\\
help: Command not found.\\
\%damn!\\
!:Event unrecognized\\
\% logout\\}

	\item Chercheur

\texttt{PROGRAM HELLO\\
PRINT *, 'Hello World'\\
END}

	\item Chercheur senior

\texttt{WRITE (6, 100)\\
100 FORMAT (1H ,11HHELLO WORLD)\\
CALL EXIT\\
END}

\end{itemize}

\begin{center}\underline{\hspace{5 cm}}\end{center}	

	\begin{center}
	\includegraphics[scale=0.8]{img/humour/punition.jpg}
	\end{center}
	
\begin{center}\underline{\hspace{5 cm}}\end{center}

Ce que disent les ingénieurs en informatique... et ce qu'il faut comprendre:
\begin{itemize}
	\item "Nous allons inscrire ce projet au planning": On s'en occupera si on a rien d'autre à faire

	\item "C'est un programme complètement nouveau!": C'est pas du tout compatible avec l'ancienne version

	\item "Ce programme ne nécessite aucune maintenance": C'est impossible à déboguer

	\item "Ce programme ne nécessite que peu de maintenance": C'est quasiment impossible à déboguer

	\item "Nous respecterons les standards": On a toujours fait comme ça et ce n'est pas aujourd'hui qu'on va changer

	\item "Nous tenons à respecter les standards": Vous n'allez pas remettre en cause tout ce qu'on vient de faire

	\item "La nouvelle version de ce programme est 100% compatible avec la précédente": On n'a touché à rien

	\item "Différentes approches ont été tentées": On essaie encore de deviner ce qui se passe.

	\item "On approche d'une solution": On s'est réunis pour prendre un café...

	\item "Les tests préliminaires n'ont pas été franchement concluants": Ce satané programme a planté dès qu'on a lancé

	\item "Il va falloir abandonner le concept en son entier": La seule personne qui comprenait quelque chose vient de démissionner

	\item "On prépare un rapport complet, selon une approche entièrement nouvelle": On vient juste d'engager trois bleus sortis de l'école

	\item "C'est une avancée technologique majeure": On n'arrive toujours pas à comprendre pourquoi ça ne marche pas

	\item "C'est le résultat d'années de développement": On a enfin réussi à faire fonctionner un bout du programme...

	\item "C'est en cours": On est tellement dans le pétrin que c'est sans espoir

	\item "Faites-nous part de vos réflexions": On écoutera ce que vous avez à dire tant que ça ne remet pas en cause ce qui est déjà fait, ou ce que nous avons décidé de faire

	\item "Nous allons y jeter un coup d'oeil": Laissez tomber! On a déjà assez de problèmes comme ça...

	\item "Je n'ai pas reçu votre e-mail": Ça fait des lustres que je n'ai pas vérifié ma messagerie...
\end{itemize}

\begin{center}\underline{\hspace{5 cm}}\end{center}	

	\begin{center}
	\includegraphics[scale=0.3]{img/humour/tesla_ipaddress.jpg}
	\end{center}

	\pagebreak
	\section{Sciences Sociales}
	
Pour aider à comprendre le jargon spécifique du marketing..... et éviter de paraître ridicule dans une soirée (pot de départ par exemple...):
\begin{enumerate}
	\item Michel est à une soirée et voit une nana très attirante. Il s'approche d'elle et lui dit: "Je suis un très bon coup". C'est ce qu'on appelle du "marketing direct"

	\item Michel est à une soirée avec un groupe de copains et il voit une nana très attirante. Un de ses amis s'approche d'elle et lui dit: "Tu vois ce garçon là-bas, c'est un très bon coup". C'est ce qu'on appelle de la "publicité."

	\item Michel est à une soirée et il voit une nana très attirante. Il lui demande son numéro de téléphone. Le lendemain, il l'appelle et lui dit: "Je suis un très bon coup". C'est ce qu'on appelle du "télémarketing."

	\item Michel est à une soirée et il voit une nana très attirante. Il la reconnaît, s'approche d'elle, lui rafraîchit la mémoire et lui dit: "Tu te souviens que je suis un très bon coup ?". C'est ce qu'on appelle du "Customer Relationship Management (CRM)"

	\item Michel est à une soirée et il voit une nana très attirante. Il se lève, s'arrange un peu, s'approche d'elle et lui sert un verre. Il lui ouvre la porte lorsqu'elle part, ramasse son sac lorsqu'il tombe, lui offre une cigarette et lui dit: "Je suis un très bon coup". C'est ce qu'on appelle des "relations publiques" ou "public relations" (PR)

	\item Michel est à une soirée et il voit une nana très attirante. Il invite à danser toutes ses copines, leur offre à boire et les fait rire ostensiblement par ses plaisanteries très spirituelles. La belle nana l'aborde et lui dit: "J'ai l'impression que tu es un très bon coup". C'est ce qu'on appelle du "lobbying".

	\item Michel est à une soirée et il voit une nana très attirante. Elle s'approche de lui et lui dit: "J'ai entendu dire que tu es un très bon coup". C'est ce qu'on appelle le "pouvoir de la marque".

	\item Michel est à une soirée et voit une super belle nana. Il la mate avec ses potes, fait des réflexions très fines, se bourre la gueule, ne fait rien du tout et rentre bredouille. C'est ce qu'on appelle la "réalité du marché"...
\end{enumerate}

\begin{center}\underline{\hspace{5 cm}}\end{center}
	\begin{center}
	\includegraphics{img/humour/fluid_dynamics.jpg}
	\end{center}

	\begin{center}
	\includegraphics{img/humour/stupid.jpg}
	\end{center}
\begin{center}\underline{\hspace{5 cm}}\end{center}

Un homme, dans la nacelle d'une montgolfière ne sait plus où il se trouve. Il descend et aperçoit une femme au sol. Il descend encore plus bas et l'interpelle:
\begin{itemize}
	\item[$-$] Excusez-moi! Pouvez-vous m'aider? J'avais promis à un ami de le rencontrer et j'ai déjà une heure de retard, car je ne sais plus où je me trouve…

	\item[$-$] La femme au sol répond: Vous êtes dans la nacelle d'un ballon à air chaud à environ 10 m du sol. Vous vous trouvez exactement à 49°,28' et 11'' Nord et 8°,25' et 58'' Est 

	\item[$-$] "Vous devez être ingénieur" dit l'aérostier. 

	\item[$-$] "Effectivement", répond la femme, "Comment avez-vous deviné"?

	\item[$-$] "Eh bien!", dit l'aérostier, "Tout ce que vous m'avez dit à l'air techniquement parfaitement correct, mais je n'ai pas la moindre idée de ce que je peux faire de vos informations et en fait je ne sais toujours pas où je me trouve. Pour parler ouvertement, vous ne m'avez été d'aucune aide. Pire, vous avez encore retardé mon voyage!"

	\item[$-$] La femme lui répond: "Vous devez être un manager!"

	\item[$-$] "C'est exact!", répond l'homme avec fierté, "Mais comment avez-vous deviné???"
	
	\item[$-$] "Eh bien!", dit la femme, "Vous ne savez ni où vous êtes, ni où vous allez. Vous avez atteint votre position actuelle en chauffant et en brassant une énorme quantité d'air. Vous avez fait une promesse sans avoir la moindre idée comment vous pourriez la tenir et vous comptez maintenant sur les gens situés en dessous de vous pour qu'ils résolvent votre problème. Votre situation avant et après notre rencontre n'a pas changé, mais comme par hasard, c'est moi maintenant qui à vos yeux en suis responsable..."
\end{itemize}

\begin{center}\underline{\hspace{5 cm}}\end{center}

Un homme va un samedi à un mariage en Corse dans un petit village. Il est en retard et il conduit aussi vite que possible sur des routes sinueuses. Soudain, après un virage, il doit s'arrêter net, un troupeau d'ovins occupe toute la route. Le berger est là et fait lentement avancer son troupeau. L'automobiliste use du klaxon de son véhicule plusieurs fois sans le moindre effet. Au bout de quelques minutes, le conducteur apostrophe le berger et lui dit:
\begin{itemize}
	\item[$-$] "Je suis en retard, je vais à un mariage qui sera suivi d'un méchoui, si je vous dis combien de moutons vous avez, m'en céderez-vous un ?"

	\item[$-$] "Volontiers" dit le berger, "je ne suis pas à un près".  

	\item[$-$] Le conducteur prend sa calculatrice et au bout d'une minute annonce: "1233".

	\item[$-$]  "Vous avez gagné" dit le berger, "Choisissez votre animal".
\end{itemize}
Le conducteur en désigne alors un. Le berger dit alors:
\begin{itemize}
	\item[$-$] "Si je trouve quelle est votre profession, me rendrez-vous ma bête ?"

	\item[$-$] "Bien sûr" dit le conducteur, "Je vous écoute".

	\item[$-$] "Vous êtes haut fonctionnaire et vous avez fait l'ENA ou une autre grande école de ce genre?"

	\item[$-$] "Vous avez raison" dit le conducteur, "Mais comment avez-vous deviné ?".

	\item[$-$] Le berger: "Je vous prie de me rendre mon chien!"
\end{itemize}

\begin{center}\underline{\hspace{5 cm}}\end{center}

	\begin{center}
	\includegraphics[scale=0.7]{img/humour/meeting_girls.jpg}
	\end{center}

	\begin{center}
	\includegraphics[scale=0.7]{img/humour/fibonaughty_sexquence.jpg}
	\end{center}
	
	\begin{center}\underline{\hspace{5 cm}}\end{center}
	
	\begin{center}
	\includegraphics[scale=0.7]{img/humour/made_of.jpg}
	\end{center}

	\begin{table}[H]
		\centering
			\begin{tabular}{c m{0.1cm} c m{0.1cm} c}
		    \begin{minipage}{.3\textwidth}
    		\center \includegraphics{img/humour/worker.eps}\\
		    \center Ouvrier
		    \end{minipage}
	    	&
			+
			& 
		    \begin{minipage}{.3\textwidth}
    		\center \includegraphics{img/humour/process.eps}\\
		    \center Processus
		    \end{minipage}
		    &
		    =
		    &
		   	\begin{minipage}{.3\textwidth}
    		\center \includegraphics{img/humour/engineer.eps}\\
		    \center Ingénieur
		    \end{minipage}
	    \\
		    \begin{minipage}{.3\textwidth}
    		\center \includegraphics{img/humour/engineer.eps}\\
		    \center Ingénieur
		    \end{minipage}
	    	&
			+
			& 
		    \begin{minipage}{.3\textwidth}
    		\center \includegraphics{img/humour/sociability.eps}\\
		    \center Sociabilité
		    \end{minipage}
		    &
		    =
		    &
		   	\begin{minipage}{.3\textwidth}
    		\center \includegraphics{img/humour/marketing.eps}\\
		    \center Marketeur
		    \end{minipage}
	    \\
		    \begin{minipage}{.3\textwidth}
    		\center \includegraphics{img/humour/marketing.eps}\\
		    \center Marketeur
		    \end{minipage}
	    	&
			-
			& 
		    \begin{minipage}{.3\textwidth}
    		\center \includegraphics{img/humour/truth.eps}\\
		    \center Vérité
		    \end{minipage}
		    &
		    =
		    &
		   	\begin{minipage}{.3\textwidth}
    		\center \includegraphics{img/humour/commercial.eps}\\
		    \center Commercial
		    \end{minipage}
	    \\
		    \begin{minipage}{.3\textwidth}
    		\center \includegraphics{img/humour/commercial.eps}\\
		    \center Commercial
		    \end{minipage}
	    	&
			-
			& 
		    \begin{minipage}{.3\textwidth}
    		\center \includegraphics{img/humour/brain.eps}\\
		    \center Cerveau
		    \end{minipage}
		    &
		    =
		    &
		   	\begin{minipage}{.3\textwidth}
    		\center \includegraphics{img/humour/manager.eps}\\
		    \center Manager
		    \end{minipage}
	    \\
		    \begin{minipage}{.3\textwidth}
    		\center \includegraphics{img/humour/manager.eps}\\
		    \center Manager
		    \end{minipage}
	    	&
			+
			& 
		    \begin{minipage}{.3\textwidth}
    		\center \includegraphics{img/humour/ego.eps}\\
		    \center Ego
		    \end{minipage}
		    &
		    =
		    &
		   	\begin{minipage}{.3\textwidth}
    		\center \includegraphics{img/humour/project_manager.eps}\\
		    \center Chef de projet
		    \end{minipage}
	    \\
	   		\begin{minipage}{.3\textwidth}
    		\center \includegraphics{img/humour/project_manager.eps}\\
		    \center Chef de projet
		    \end{minipage}
	    	&
			-
			& 
		    \begin{minipage}{.3\textwidth}
    		\center \includegraphics{img/humour/humour.eps}\\
		    \center Humour
		    \end{minipage}
		    &
		    =
		    &
		   	\begin{minipage}{.3\textwidth}
    		\center \includegraphics{img/humour/hr.eps}\\
		    \center Responsable RH
		    \end{minipage}
	    \\	    
		\end{tabular}
	\end{table}

	\begin{center}
	\includegraphics[scale=0.7]{img/humour/xmas.jpg}
	\end{center}
	
	\begin{center}\underline{\hspace{5 cm}}\end{center}	
	\begin{center}
		\includegraphics[scale=0.8]{img/humour/evolution.jpg}
	\end{center}

 	\chapter{Liens}
		Cet annuaire contient des liens répartis dans $7$ catégories tous ayant trait aux sciences et que nous trouvons intéressants. Nous tenons à préciser qu'en aucun cas nous avons été rétribué sous quelque forme que ce soit pour l'ajout de liens dans la liste ci-dessous! Vous pouvez aussi trouver des apps iPad/iPhone "scientifiques" en bas de la liste que nous recommandons fortement (nous n'avons pas contrôlé s'il existe un équivalent de chacune de ces applications pour Android ou Microsoft Windows Surface/Phone).
	
	\begin{itemize}	 
		\item[$-$] {\Large \ding{52}} Site de qualité tant du point de vue contenu que contenant
		\item[$-$] {\Large \ding{45}} Contenu avec développements et démonstrations détaillées
		\item[$-$] {\Large \ding{41}} Site avec forum de discussion
		\item[$-$] {\Large \ding{36}} Softwares, sharewares, freewares (logiciels) scientifiques à télécharger
		\item[$-$] {\Large \ding{229}} Ouvrages, publications, magazines, journaux scientifiques (à visualiser ou télécharger)
		\item[$-$] {\Large \ding{44}} Site sympathique
		\item[$-$] {\Large \ding{170}} Coup de coeur
		\item[$-$] {\Large \ding{73}} À voir absolument!
	\end{itemize}	
	S'il y avait par ailleurs, trois liens à mettre en évidence parmi tous ce serait \href{http://www.google.com}{\color{blue} Google}, \href{http://www.wikipedia.com}{\color{blue} Wikipedia} et \href{http://www.youtube.com}{\color{blue} YouTube} dont nous sommes redevables de nombreux emprunts!
		
	\begin{tcolorbox}[title=Remark,colframe=black,arc=10pt]
	Nous ne sommes pas responsables de la persistance ou de l'exactitude des URL des sites Internet externes ou tiers mentionnés dans ce livre, et ne garantissons pas que le contenu de ces sites est ou restera exact ou approprié.
	\end{tcolorbox}
	
	\pagebreak

	\section{Sciences Exactes}

	{\Large \ding{52}}{\Large \ding{45}}{\Large \ding{36}}{\Large \ding{44}}{\Large \ding{170}}{\Large \ding{73}}\bcdfrance{} ChronoMath, "petite" chronologie des mathématiques, est un site Internet pédagogique en perpétuel renouvellement que Serge Mehl fit connaître dès 1988 en tant que conseiller pédagogique de mathématiques en Afrique francophone où il fut coopérant pendant de nombreuses années. Plus de 450 mathématiciens (et leurs travaux) sont passés en revue...!!! À voir!\\
	\href{http://www.chronomath.com}{\color{blue} http://www.chronomath.com}
	
	{\Large \ding{41}}{\Large \ding{36}}{\Large \ding{229}}\bcdfrance{} Ce site propose une actualité des mathématiques, une encyclopédie avec parties dictionnaire, des biographies et formulaires, ainsi que différents dossiers sur des sujets mathématiques divers et un forum. Ce site est particulièrement impressionnant relativement à la quantité d'informations proposée en téléchargement. \\
	\href{http://www.bibmath.net}{\color{blue} http://www.bibmath.net} 
	
	{\Large \ding{52}}{\Large \ding{45}}\bcdfrance{} Un bon site traitant de quelques sujets pertinents en physique (synophysique). Le design pourrait être revu en ce qui concerne la navigation utilisant des cadres... mais il faut tout de même privilégier la qualité du contenu et celle-ci prédomine largement.\\
	\href{http://www.sciences.univ-nantes.fr/physique/perso/blanquet/frame3.htm}{\color{blue} http://www.sciences.univ-nantes.fr/physique/perso/blanquet/frame3.htm}
	
	{\Large \ding{52}}{\Large \ding{45}}{\Large \ding{36}}\bcdfrance{} Nombreux contenus PDFs sur l'algèbre, l'analyse et la géométrie.\\
	\href{http://c.caignaert.free.fr}{\color{blue} http://c.caignaert.free.fr}
	
	{\Large \ding{52}}{\Large \ding{45}}{\Large \ding{36}} arXiv est une archive de prépublications électroniques d'articles scientifiques. Le laps de temps s'écoulant entre le moment où un chercheur termine un projet et le moment où son travail est publié dans un journal peut être de l'ordre d'un an. À l'échelle du temps de la recherche c'est une durée longue. La mise en place de l'arXiv a donc été un moyen de pallier à ce problème temporel et de coût.\\
	\href{http://arxiv.org}{\color{blue} http://arxiv.org}
	
	{\Large \ding{52}}{\Large \ding{45}}{\Large \ding{36}}\bcdfrance{} L'archive ouverte pluridisciplinaire HAL est destinée au dépôt et à la diffusion gratuite d'articles scientifiques de niveau recherche, publiés ou non, et de thèses, émanant des établissements d'enseignement et de recherche français ou étrangers, des laboratoires publics ou privés. On y trouve même des livres scientifiques publiés récemment par des maisons d'édition françaises préstigieuses.\\
	\href{http://hal.archives-ouvertes.fr/}{\color{blue} http://hal.archives-ouvertes.fr/}
	
	\pagebreak
	\section{Éditions/Magazines}

	{\Large \ding{52}}{\Large \ding{41}}{\Large \ding{229}}{\Large \ding{170}}{\Large \ding{73}}\bcdfrance{} On peut considérer le site proposé ici comme l'équivalent du précédent, mais pour les ingénieurs francophones. Il est cependant qualitativement meilleur et on y trouve une quantité de dossiers toute aussi grande mais plus homogène. On regrettera juste peut-être l'accès à certains éléments qui n'est pas toujours aisé du premier coup.\\
	\href{http://www.techniques-ingenieur.fr}{\color{blue} http://www.techniques-ingenieur.fr}
	
	{\Large \ding{52}}{\Large \ding{229}}{\Large \ding{170}}{\Large \ding{73}}\bcdfrance{} Absolument excellent et à voir! De nombreux cours complets de l'École Polytechnique (France) sont publiées et disponibles gratuitement au téléchargement au format PDF (est-ce que cela va durer?). Environ $1'000$ ressources éducatives sont disponibles au total, dont la qualité est hautement variable, mais dont la pertinence et la rareté sont toujours égales.\\
	\href{http://catalogue.polytechnique.fr}{\color{blue} http://catalogue.polytechnique.fr}
	
	{\Large \ding{52}}{\Large \ding{229}}{\Large \ding{170}}{\Large \ding{73}}\bcdfrance{} Le site Internet en soi n'est pas excellent (ce qui est fort dommage) mais le magazine pour lequel il vous propose de vous abonner (contre payement) vaut largement le détour pour les passionnées de mathématiques et de leur actualité.\\
	\href{http://tangente.poleditions.com/}{\color{blue}http://tangente.poleditions.com/}
	
	{\Large \ding{52}}{\Large \ding{229}}{\Large \ding{170}}{\Large \ding{73}}\bcdfrance{} Un magazine de la même trempe que Tangente mais à un niveau technique et académique bien supérieur selon mes critères personnels. Le contenu est particulièrement orienté mathématiques pures et très rarement avec des applications directes et explicites à la physique, ingénierie ou à l'économie.\\
	\href{http://www.quadrature.info}{\color{blue}http://www.quadrature.info}
	
	{\Large \ding{52}}{\Large \ding{229}}{\Large \ding{170}}{\Large \ding{73}} Ce site en soi n'est pas excellent non plus (ce qui est fort dommage) mais certains des ouvrages proposés sont simplement historiques!!!\\
	\href{http://urss.ru}{\color{blue}http://urss.ru}
	
	{\Large \ding{52}}{\Large \ding{229}}{\Large \ding{170}}{\Large \ding{73}}\bcdfrance{} Excellent site avec une ressource considérable de documentation électronique traitant des mathématiques uniquement. Le site contient des liens vers les pages des auteurs des documents (sinon quoi il faudrait un serveur considérable...).\\
	\href{http://mathslinker.chez-alice.fr}{\color{blue}http://mathslinker.chez-alice.fr}
	
	{\Large \ding{229}}\bcdfrance{} Ce site contient une bibliothèque en ligne de publications de grands mathématiciens du 20ème et 19ème siècle. À voir absolument!\\
	\href{http://matwbn.icm.edu.pl/wyszukiwarka.php}{\color{blue}http://matwbn.icm.edu.pl/wyszukiwarka.php}
	
	{\Large \ding{52}}{\Large \ding{229}}{\Large \ding{170}}{\Large \ding{73}}\bcdfrance{} Reprints d'oeuvres fondamentales concernant la mathématique, la physique, l'histoire et la philosophie des Sciences.\\
	\href{http://www.gabay.com}{\color{blue}http://www.gabay.com}
	
	\pagebreak
	{\Large \ding{52}}{\Large \ding{229}}{\Large \ding{170}}{\Large \ding{73}}\bcdfrance{} Éditions Eyrolles - Excellent site contenant de nombreux ouvrages francophones et anglophones de qualité. Propose au fait des ouvrages de plusieurs maisons d'éditions dont Dunod, Springer, etc. Il vaut la peine d'aller jeter un coup d'oeil dans les sections "Mathématique" et "Physique" y'a du bon... \\
	\href{http://www.eyrolles.com}{\color{blue}http://www.eyrolles.com}
	
	{\Large \ding{52}}{\Large \ding{229}}{\Large \ding{170}}{\Large \ding{73}}\bcdfrance{} Éditions Dunod - Proposent des ouvrages scientifiques contemporains (niveau 1-3ème cycle et +). Les ouvrages de cette maison d'édition sont pour la plupart excellents. Personnellement ils sont techniquement (mais pas pédagogiquement) mes préférés, car souvent les développements y sont souvent très détaillés.\\
	\href{http://www.dunod.com}{\color{blue}http://www.dunod.com}
	
	{\Large \ding{229}}{\Large \ding{170}} Éditions Springer - Proposent des ouvrages scientifiques de niveau postdoc. La navigation sur le site n'est pas aisée car les choix sont disposés un peu n'importe comment mais sinon les ouvrages proposés sont très techniques et de haut niveau... très haut niveau...\\
	\href{http://www.springer.de}{\color{blue}http://www.springer.de}
	
	{\Large \ding{229}}{\Large \ding{170}}{\Large \ding{73}}\bcdfrance{} Le programme NUMDAM, piloté par la Cellule MathDoc (UMS 5638 CNRS - UJF) pour le compte du CNRS, propose la numérisation rétrospective des fonds mathématiques publiés en France.\\
	\href{http://www.numdam.org}{\color{blue}http://www.numdam.org}
	
	{\Large \ding{52}}{\Large \ding{229}}{\Large \ding{170}}{\Large \ding{73}} Éditions Wrox - Proposent des ouvrages développement informatique (niveau expert). D'après ce que je sais (et j'en pense de même), cette maison d'édition est la référence mondiale dans le domaine des ouvrages traitant des langages de programmation.\\
	\href{http://www.wrox.com}{\color{blue}http://www.wrox.com}
	
	{\Large \ding{52}}{\Large \ding{170}}{\Large \ding{73}} Éditions Cambridge University Press - Proposent des uvrages scientifiques niveau postdoc aussi. Un peu l'équivalent des éditions Springer mais la structure du site Internet est un peu mieux faite. On appréciera particulièrement la possibilité de s'abonner gratuitement pour recevoir régulièrement leur catalogue.\\
	\href{http://www.cambridge.org}{\color{blue}http://www.cambridge.org}
	
	
	{\Large \ding{45}}{\Large \ding{229}}{\Large \ding{170}}{\Large \ding{73}} La Bibliothèque Numérique de France (BNF) propose gratuitement en téléchargement des ouvrages scientifiques (entre autres) dont les droits d'auteur sont tombés après 75 années. Le système de téléchargement est peu convivial, mais on peut cependant tomber sur des ouvrages très pertinents. Les PDF font souvent plus de 10 MB donc attention à ceux qui ont du petit débit.\\
	\href{http://gallica.bnf.fr}{\color{blue} http://gallica.bnf.fr}
	
	{\Large \ding{45}}{\Large \ding{229}}{\Large \ding{170}}{\Large \ding{73}} Le Physical Review Letters (PRL) search machine (articles scientifiques) propose contre inscription et paiement (sic!) l'accès à des articles scientifiques de recherche en physique théorique et expérimentale publisée depuis plus de 100 ans. Il faut admettre que le niveau de détail des articles proposés n'est souvent pas très élevé mais là n'est pas l'objectif pour les spécialistes.\\
	\href{http://prola.aps.org/search}{\color{blue} http://prola.aps.org/search}
	
	\pagebreak
	{\Large \ding{45}}{\Large \ding{229}}{\Large \ding{73}} Éditions Lavoisier - Très bonne maison d'édition proposant des ouvrages très spécialisés dans le domaine de l'ingénierie électronique, électrotechnique, civile, informatique, etc. On appréciera aussi particulièrement la possibilité de s'abonner gratuitement pour recevoir régulièrement leur excellent catalogue. Voir plus particulièrement la partie "Hermès Sciences" et " Tech \& Doc".\\
	\href{http://www.lavoisier.fr}{\color{blue} http://www.lavoisier.fr} (\href{http://www.editions-hermes.fr}{\color{blue} http://www.editions-hermes.fr} + \href{http://www.tec-et-doc.com}{\color{blue} http://www.tec-et-doc.com})
	
	{\Large \ding{170}}{\Large \ding{45}}{\Large \ding{229}} Éditions World Scientific - Éditeur américain proposant de très nombreux ouvrages de haut niveau pour les sciences physiques et mathématiques (voir la large palette de sections scientifiques correspondantes dans leur site).\\
	\href{http://www.worldscibooks.com}{\color{blue}http://www.worldscibooks.com}
	
	\section{Associations}

		{\Large \ding{44}} myScience.ch donne une vue d'ensemble des sciences, de la recherche, des universités, des entreprises et d'autres centres de recherche en Suisse. Le site fournit des informations pratiques sur l'emploi, le financement et la vie en Suisse ainsi que des nouvelles scientifiques aux chercheurs, scientifiques, universitaires et à toute personne intéressée par les sciences. Il a une proportion plus grande d'articles en allemand... même dans la version francophone...\\
		\href{http://www.myscience.ch}{\color{blue}http://www.myscience.ch}
		
		{\Large \ding{229}}{\Large \ding{44}}{\Large \ding{73}}\bcdfrance{} L'Union Rationaliste a pour but de promouvoir le rôle de la raison dans le débat intellectuel comme dans le débat public, face à toutes les dérives irrationnelles. Elle s'emploie à mettre à la disposition de chacun la possibilité d'accéder à une conception intelligible du monde et de la vie. Elle lutte pour que l'État demeure laïque, assume sa fonction de protection contre toute forme d'endoctrinement.\\
		\href{http://www.union-rationaliste.org}{\color{blue}http://www.union-rationaliste.org}
		
		{\Large \ding{45}}{\Large \ding{229}} La Société Américaine de Mathématiques (en anglais: American Mathematical Society, AMS) est une association de mathématiciens professionnels, dédiée aux intérêts de la recherche et de l'enseignement des mathématiques, ce qu'elle fait sous forme de différentes publications gratuites et conférences, et de prix décernés à des mathématiciens.\\
		\href{http://www.ams.org}{\color{blue}http://www.ams.org}
		
		{\Large \ding{45}}{\Large \ding{229}} La Société Américaine de Physique (en anglais: American Physical Society, APS) est une société savante fondée le 20 mai 1899, basée aux États-Unis, très active dans le domaine de la recherche scientifique en physique.\\
		\href{http://www.aps.org}{\color{blue}http://www.aps.org}
		
		{\Large \ding{45}}{\Large \ding{229}}{\Large \ding{170}} L'Association Européenne de Physique (en anglais: Physical Society, EPS) est une organisation à but non lucratif dont le but est de promouvoir la physique et les physiciens en Europe. Cette société propose un abonnement à son magazine \textit{Europhysicsnews} et à bien d’autres choses..\\
		\href{http://www.eps.org}{\color{blue}http://www.eps.org}

	\pagebreak
	\section{Emploi}
	
	{\Large \ding{73}} Voici une liste non-exhaustive de sites d'offres d'emplois déstinée particulièrement aux physiciens et mathématiciens suisses... (cependant un lien est destiné uniquement au territoire U.S.!). Il en découle que souvent les descriptifs des offres sont en allemand (...) et que ces dernières sont souvent relatives au domaine bancaire (qui reprèsente 10-15\% de l'emploi en Suisse d'après ce que je sais...). Certaines offres sont reliées à des sites plus généraux comme le fameux jobup.ch bien connu en Suisse.
	
	\begin{itemize}	 
		\item[$\bullet$] \href{http://www.telejob.ch}{\color{blue}http://www.telejob.ch} 
	
		\item[$\bullet$] \href{http://www.jobs.myscience.ch}{\color{blue}http://www.jobs.myscience.ch}
	
		\item[$\bullet$] \href{http://www.math-jobs.ch}{\color{blue}http://www.math-jobs.ch}
	
		\item[$\bullet$] \href{http://www.analyticrecruiting.com}{\color{blue}http://www.analyticrecruiting.com}
	\end{itemize}
	
	\section{Télévision/Radio}

	{\Large \ding{73}} Excellent site expliquant avec des animations 3D de nombreux appareillages ultra-classiques de la vie de tous les jours conçus et développés par les méthodes de l'ingénieur (moteur, réfrigérateur, pompe, etc.).\\
	\href{http://www.learnengineering.org}{\color{blue}http://www.learnengineering.org}
	
	\bcdfrance{} C'est pas Sorcier: La célèbre émission de télévision française (version moderne de "Bill Nye the Science guy") sur toutes les sciences que presque tous les jeunes francophones connaissent ou découvrent encore!\\
	\href{https://www.youtube.com/user/cestpassorcierftv/}{\color{blue}https://www.youtube.com/user/cestpassorcierftv/}
	
	\bcdfrance{} Cité-Sciences: visionnement de vidéo-conférences, ou écoute d'enregistrements audio sur des sujets de physique et astronomie (+ d'autres) vulgarisés (requière RealOne Player au jour où nous écrivons ces lignes).\\
	\href{http://www.cite-sciences.fr}{\color{blue}http://www.cite-sciences.fr}
	
	\bcdfrance{} Sur Canal Académie, les académiciens, spécialistes en physique, mathématique, sciences humaines, philosophie, sociologie, droit et jurisprudence, économie, politique et finances, histoire, géographie et démographie et sciences politiques vous font partager leurs réflexions sur l'actualité et les évolutions de la société.\\
	\href{http://www.canalacademie.com}{\color{blue}http://www.canalacademie.com}
	
	\pagebreak
	Et voici une liste non-exhaustive de chaînes scientifiques YouTube produites par des particuliers passionnés qui sont scientifiquement relativement \underline{fiables} et dont la qualité de post-production est très bonne et qui peuvent s'avérer très très utiles pour mieux comprendre certains théorèmes complexes et sujets traités dans ce livre (la plupart sont de niveau scolaire secondaire):
	\begin{itemize}
		 \item[$\bullet$] \href{https://www.youtube.com/user/epenser1}{\color{blue}https://www.youtube.com/user/epenser1} (fr, Bruce Benamran)
		 
		 \item[$\bullet$] \href{https://www.youtube.com/user/ScienceEtonnante}{\color{blue}https://www.youtube.com/user/ScienceEtonnante} (en, David Louapre)
		 
		 \item[$\bullet$] \href{https://www.youtube.com/user/Micmaths}{\color{blue}https://www.youtube.com/user/Micmaths} (fr, Mickaël Launay)
		 
		 \item[$\bullet$]  \href{https://www.youtube.com/user/physicswoman}{\color{blue}https://www.youtube.com/user/physicswoman} (en, Dianna Cowern)
		 
		 \item[$\bullet$] \href{https://www.youtube.com/user/1veritasium}{\color{blue}https://www.youtube.com/user/1veritasium} (en, Derek Muller)
		 
		 \item[$\bullet$] \href{https://www.youtube.com/channel/UCYO_jab_esuFRV4b17AJtAw}{\color{blue}https://www.youtube.com/channel/UCYO\_jab\_esuFRV4b17AJtAw} (en, Grant Sanderson)
		 
		 \item[$\bullet$] \href{https://www.youtube.com/user/fauxsceptique}{\color{blue}https://www.youtube.com/user/fauxsceptique}(fr, Christophe Michel)
		 
		 \item[$\bullet$] \href{https://www.youtube.com/channel/UC7_gcs09iThXybpVgjHZ_7g}{\color{blue}https://www.youtube.com/channel/UC7\_gcs09iThXybpVgjHZ\_7g} (en, Matt O'Dowd)
		 
		 \item[$\bullet$] \href{https://www.youtube.com/user/mathdude2012}{\color{blue}https://www.youtube.com/user/mathdude2012} (en, Andrey Kopot)
		 
		 \item[$\bullet$] \href{https://www.youtube.com/user/LearnEngineeringTeam}{\color{blue}https://www.youtube.com/user/LearnEngineeringTeam} (en, Sabin Mathew)
	\end{itemize}

	\pagebreak
	\section{Divers Sciences}

	{\Large \ding{52}}{\Large \ding{45}}{\Large \ding{41}}{\Large \ding{44}}\bcdfrance{} Site comportant de nombreuses fiches de cours et surtout quantités d'exercices résolus intéressants. Malheureusement, le site est devenu payant avec accès pendant une période limitée... le prix étant peu élevé il peut être quand même pertinent de débourser la somme ad hoc pour la qualité.\\
	\href{http://www.web-sciences.com}{\color{blue}http://www.web-sciences.com}
	
	{\Large \ding{52}}{\Large \ding{41}}{\Large \ding{229}}{\Large \ding{44}}{\Large \ding{73}}\bcdfrance{} Futura-Sciences est un site à vocation de vulgarisation sur les sciences pures et exactes avec des informations quotidiennes sur les sciences et technologies, de nombreux dossiers dans toutes les thématiques et des forums de sciences de débats et de discussions ... (le forum de physique y est souvent très bien fréquenté).\\
	\href{http://www.futura-sciences.com}{\color{blue}http://www.futura-sciences.com}
	
	{\Large \ding{52}}{\Large \ding{41}}{\Large \ding{229}}{\Large \ding{44}}{\Large \ding{73}}\bcdfrance{} Physics Forum est considéré comme la plus grande communauté de physique au monde. Il y a beaucoup de discussions sur de nombreux sujets auxquels la communauté a répondu avec qualité. Ce forum est également considéré comme le forum partenaire de du présent livre car je n'ai plus le temps de répondre gratuitement aux questions qu'on m'envoie par courriel.\\
	\href{http://www.physicsforums.com}{\color{blue}http://www.physicsforums.com}
	
	{\Large \ding{52}}{\Large \ding{45}}{\Large \ding{41}}{\Large \ding{36}}{\Large \ding{229}}{\Large \ding{44}}{\Large \ding{170}}{\Large \ding{73}}\bcdfrance{} Astrosurf est un portail de liens pour les astronomes amateurs francophones. Il y est aussi proposé des forums d'astronomie, des petites annonces, l'hébergement gratuit de sites d'astronomie, des éphémérides et tout sur les clubs et associations d'astronomie francophones. On y trouve en particulier le site fameux de Thierry Lombry (\href{http://astrosurf.com/luxorion/}{\color{blue}ici}).\\
	\href{http://www.astrosurf.com}{\color{blue}http://www.astrosurf.com}
	
	{\Large \ding{52}}{\Large \ding{229}}\bcdfrance{} Le C.E.A. est le Commissariat Français à l'Énergie Atomique. Le site propose des dossiers et nouvelles intéressantes sur certains domaines particuliers de la physique de pointe. On peut également trouver des ressources pédagogiques gratuites pour les professeurs (diaporamas et posters).\\
	\href{http://www.cea.fr}{\color{blue}http://www.cea.fr}
	
	{\Large \ding{44}}{\Large \ding{73}} Voici un site qui, si j'étais encore enfant, aurait fait mon bonheur... et le malheur de mes parents. Ce n'est pas tant le contenu qui est intéressant mais surtout la boutique en ligne (eStore) qui propose une bonne centaine de gadgets et jeux ludo-éducatifs pour les passionnées de sciences. Attention au porte-monnaie quand même!\\
	\href{http://www.xump.com}{\color{blue}http://www.xump.com}
	
	{\Large \ding{44}}{\Large \ding{73}} Pendant des années, les dessins de S. Harris ont ajouté de l'humour à d'innombrables magazines, livres, bulletins d'information, publicités et sites Internet dans le domaine des sciences.\\
	\href{http://www.sciencecartoonsplus.com}{\color{blue}http://www.sciencecartoonsplus.com}
	
	\pagebreak
	\section{Logiciels/Applications}
	{\Large \ding{52}}{\Large \ding{170}}{\Large \ding{73}} Rosetta Code est un site de programmation chrestomathy. L'idée est de présenter des solutions à la même tâche dans autant de langues de programmation différents que possible, de montrer comment les languages sont similaires et différentes, et d'aider une personne ayant une approche unique à résoudre un problème pour en apprendre une autre. Le code Rosetta comporte actuellement $861$ tâches, $208$ tâches à l'état de brouillon et $671$ languages.\\
	\href{https://rosettacode.org/wiki/Category:Programming_Tasks}{\color{blue}https://rosettacode.org}
	
	{\Large \ding{52}}{\Large \ding{36}}{\Large \ding{170}}{\Large \ding{73}} Site Internet du logiciel TeXMaker utilisé pour écrire ce livre en \LaTeX{} (logiciel fonctionnant sur plusieurs systèmes d'exploitation).\\
	\href{http://www.xm1math.net/texmaker/index.html}{\color{blue}http://www.xm1math.net}
	
	{\Large \ding{52}}{\Large \ding{36}}{\Large \ding{170}}{\Large \ding{73}} Minitab est un des logiciels (payant) de référence pour les ingénieurs travaillant dans l'industrie et appliquant la maîtrise statistique des procédés (voir la section de Génie Industriel de ce livre) dans le cadre de leur travail ou faisant tout autre étude statistique dans le cadre de la R\&D.\\
	\href{http://www.minitab.com}{\color{blue}http://www.minitab.com}
	
	{\Large \ding{52}}{\Large \ding{36}}{\Large \ding{170}}{\Large \ding{73}} Isograph est une excellente suite de logiciels pour les ingénieurs travaillant dans l'industrie et appliquant les techniques de la maintenance préventive (fiabilité, AMDEC) et d'assistance à la décision (voir chapitre de Théorie Des Jeux Et De La Décision).\\
	\href{http://www.isograph-software.com}{\color{blue}http://www.isograph-software.com}
	
	{\Large \ding{52}}{\Large \ding{36}}{\Large \ding{170}}{\Large \ding{73}} L'entreprise ReliaSoft développe probablement les meilleurs logiciels du marché pour les ingénieurs en fiabilité et en assurance. Leurs logiciels sont utilisés par les entreprises étant à la pointe dans l'ingénierie mondiale. Globalement leurs logiciels sont de vrais petits bijoux (en particulier Weibull++).\\
	\href{http://ww.reliasoft.com}{\color{blue}http://ww.reliasoft.com}
	
	{\Large \ding{52}}{\Large \ding{36}}{\Large \ding{73}} JMP de la société SAS est probablement le meilleur logiciel (dans le sens de sa complétude) pour les plans d'expérience et l'analyse de la capabilité à ce jour. Son approche est pédagogique, structurée et il existe de très bonnes documentations avec les démonstrations mathématiques de la démarche utilisée par le logiciel.\\
	\href{http://www.jmp.com}{\color{blue}http://www.jmp.com}
	
	{\Large \ding{52}}{\Large \ding{36}}{\Large \ding{170}}{\Large \ding{73}} Site internet officiel de l'excellent logiciel de calcul formel Maple utilisé pour de nombreux exemples dans les différentes chapitres du présent livre. Le site propose aussi beaucoup de documentation et de compléments à télécharger.\\
	\href{http://www.mapleapps.com}{\color{blue}http://www.mapleapps.com}
	
	{\Large \ding{52}}{\Large \ding{36}}{\Large \ding{73}} Site internet officiel du logiciel de calcul et simulation numérique MATLAB™. Je ne suis pas un fan mais c'est un outil incontournable dans certaines entreprises et particulièrement pour la partie SimuLink. Il s'agit en quelque sorte d'un "must have" pour les ingénieurs au même titre que LabView.\\
	\href{http://www.mathworks.com}{\color{blue}http://www.mathworks.com}
	
	{\Large \ding{52}}{\Large \ding{36}}{\Large \ding{73}} Site officiel de la société Statistica. Une référence mondiale dans le domaine de l'analyse statistique, de l'exploration de données et du contrôle statistique des processus qui se situe à priori dans un mouchoir de poche avec IBM SPSS.\\
	\href{http://www.statsoft.com}{\color{blue}http://www.statsoft.com}
	
	{\Large \ding{52}}{\Large \ding{36}}{\Large \ding{170}}{\Large \ding{73}} COMSOL Multiphysics (anciennement FEMLAB) est un excellent environnement interactif pour la modélisation d'applications industrielles et scientifiques basées sur les équations aux dérivées partielles (EDP) utilisant les méthodes par éléments finis (plus simple à utiliser que ANSYS). À acheter pour les ingénieurs et chercheurs (pour ceux qui ont les moyens d'acheter la licence bien sûr..)!\\
	\href{http://www.comsol.fr}{\color{blue}http://www.comsol.fr}
	
	{\Large \ding{52}}{\Large \ding{36}}{\Large \ding{170}}{\Large \ding{73}} Palisade @Risk est une suite de compléments pour Microsoft Excel et Microsoft Project pour la simulation probabiliste (de Monte Carlo et Latin Hypercube). Son intégration conjointe avec Microsoft Project et Microsoft Excel (ie convivialité) ainsi que ses modules de calculs basés sur les algorithmes génétiques et réseaux de neurones et son module de théorie de la décision en fait un outil très convoité pas les hauts dirigeants et ingénieurs des grandes entreprises dans le domaine de la finance, de la qualité, la gestion de projets, l'audit et de la production.\\
	\href{http://www.palisade-europe.com}{\color{blue}http://www.palisade-europe.com}
	
	{\Large \ding{52}}{\Large \ding{36}}{\Large \ding{170}}{\Large \ding{73}} Le logiciel Decision Analysis de TreeAge (DATA) permet à l'utilisateur de construire, d'analyser et de distribuer des arbres d'analyse, des modèles de Markov et des diagrammes d'influence. Les modèles DATA intègrent visuellement les aspects quantitatifs et qualitatifs reliés aux décisions d'entreprise afin de fournir un outil d'assistance dans les projets lors d'analyse de risques complexes.\\
	\href{http://www.treeage.com}{\color{blue}http://www.treeage.com}
	
	{\Large \ding{52}}{\Large \ding{36}}{\Large \ding{170}}{\Large \ding{73}} MathType est un éditeur d'équations puissant (logiciel utilisé pour le site Internet companion au présent livre) pour importer / exporter du MathML ou TeX que vous pouvez l'exécuter dans une application autonome. Avec une barre d'outils, ce logiciel s'intègre également parfaitement aux programmes de la suite Microsoft Office 2003-2013 (Word, Excel, PowerPoint), mais aussi à 500 autres logiciels (en particulier des logiciels de mathématiques). À la différence de \ \LaTeX 2 $\varepsilon$, vous ne perdrez pas d’heures pour résoudre les problèmes de compatibilité et de compilation.\\ 
	\href{http://www.dessci.com/en/}{\color{blue}http://www.dessci.com/en/}
	
	{\Large \ding{52}}{\Large \ding{36}}{\Large \ding{73}} Scilab (contraction de Scientific Laboratory) est un logiciel libre, développé à l'INRIA Rocquencourt (France). C'est un environnement de calcul numérique qui permet d'effectuer rapidement de nombreuses résolutions et représentations graphiques couramment rencontrées en mathématiques appliquées pour les gens ou petites entreprises n'ayant pas les moyens financiers d'acquérir MATLAB™.\\
	\href{http://www.scilab.org}{\color{blue}http://www.scilab.org}
	
	{\Large \ding{36}}{\Large \ding{73}} Site officiel du fameux logiciel de géométrie dynamique Cabri (référence dans le domaine). destiné principalement à l'apprentissage de la géométrie en milieu scolaire. Il permet d'animer des figures géométriques, au contraire de celles dessinées au tableau. Il se décline pour la géométrie plane ou pour la géométrie en 3D. C'est l'ancêtre de tous les logiciels de géométrie dynamique.\\
	\href{http://www.cabri.com}{\color{blue}http://www.cabri.com}
	
	{\Large \ding{36}}{\Large \ding{170}}{\Large \ding{73}}\bcdfrance{}  	Site Internet d'un prof de math ayant développé un très bon petit logiciel gratuit pour faire de nombreux plots, calculs d'intégrales, simulations de statistiques et probabilités pertinents très simplement et ludiquement en classe (niveau BAC/Matu).\\
	\href{http://www.patrice-rabiller.fr}{\color{blue}http://www.patrice-rabiller.fr}
	
	{\Large \ding{36}}{\Large \ding{73}} ACDLabs développe l'excellent logiciel Chemsketch pour la modélisation et la conception de molécules (2D, 3D) avec accès à certaines propriétés physiques et chimiques. C'est un outil très utile également pour préparer des documents ou des articles scientifiques (publications) dans le domaine de la chimie.\\
	\href{http://www.acdlabs.com}{\color{blue}http://www.acdlabs.com}
	
	{\Large \ding{36}}{\Large \ding{73}}{\Large \ding{170}} MolView est une application web intuitive, open-source, pour rendre la science de la chimie moléculaire plus intuitive!\\
	\href{http://molview.org}{\color{blue}http://molview.org}
	
	{\Large \ding{36}}{\Large \ding{73}} Excellent logiciel de conception de structures de génie civil de haut niveau basé sur les méthodes par éléments finis (pas mal utilisé en Suisse pour des projets simples ou complexes et pendant les études d'ingénieurs).\\
	\href{http://www.scia-online.com}{\color{blue}http://www.scia-online.com}
	
	{\Large \ding{36}}{\Large \ding{73}} Maxima est un logiciel libre de calcul formel, descendant sous licence GNU GPL du package Macsyma, le logiciel de calcul symbolique développé à l'origine pour les besoins du Département de l'Énergie américain. Maxima permet de faire du calcul sur les polynômes, les matrices, de l'intégration, de la dérivation, du calcul de séries, de limites, résolutions de systèmes, d'équations différentielles, etc.\\
	\href{http://maxima.sourceforge.net}{\color{blue}http://maxima.sourceforge.net}
	
	{\Large \ding{36}}{\Large \ding{73}} SPSS (Statistical Package for the Social Sciences) est a priori le logiciel le plus utilisé dans les entreprises en Suisse pour l'analyse statistique de données car il comporte un nombre impressionnant de tests ainsi que de nombreux packages métier. De par son prix et son propriétaire (IBM) on peut le considérer comme la solution haut de gamme de la statistique.\\
	\href{http://www.ibm.com/software/fr/analytics/spss}{\color{blue}http://www.ibm.com/software/fr/analytics/spss}
	
	{\Large \ding{36}}{\Large \ding{170}}{\Large \ding{73}} R est un langage puissant et un environnement pour le calcul statistique et les graphiques. R fournit une très grande variété de statistiques (modélisation linéaire et non linéaire, tests statistiques classiques, analyse de séries chronologiques, la classification, clustering, ...) et de techniques graphiques qui par ailleurs sont très extensible (accès au code source de toutes les fonctions est disponible!). Une des forces de R est la facilité avec laquelle on peut rédiger correctement des publications scientifiques de qualité.\\
	\href{http://www.r-project.org}{\color{blue}http://www.r-project.org}
	
	{\Large \ding{36}}{\Large \ding{73}} LabVIEW (Laboratory Virtual Instrument Engineering Workbench) est un puissant outil visuel ( langage de programmation graphique) de contrôle d'instruments de mesures ou de robots depuis un PC utilisé par énormément d'entreprises à traver à le monde (particulièrement par les ingénieurs) et développé par National Instruments.\\
	\href{http://www.ni.com/labview}{\color{blue}http://www.ni.com/labview}
	
	Et enfin une petite liste de quelques applications "scientifiques" pour iPhone / iPad (nous ne donnons pas de liens puisqu'il suffit de les trouver via le iTunes Store en écrivant leur nom dans l'outil de recherche):
	\begin{itemize}
		\item[$\bullet$] \textbf{Microsoft OneNote}: Pour prendre facilement des notes (au clavier ou stylet) et écrire des équations en pseudo LaTeX et les synchroniser avec votre ordinateur et le cloud.
		
		\item[$\bullet$] \textbf{iAnnote}: Super lecteur PDF avec de nombreuses fonctions pour annoter et modifier vos manuels de cours et publications scientifiques et gérer des livres volumineux.
		
		\item[$\bullet$] \textbf{WeatherProHD}: Pour les personnes qui aiment la météorologie et les prévisions météorologiques présentées de manière belle et professionnelle avec beaucoup d'options.
		
		\item[$\bullet$] \textbf{Analyser}: Pour ceux qui veulent lancer R ou Python sur leur iPad et leur iPhone. Application incroyable à la science des données en voyage!
		
		\item[$\bullet$] \textbf{TeX Writer}: Jusqu'à présent, la meilleure application \LaTeX{} que nous connaissons vous permettant d'écrire des équations et des documents de qualité sur vos appareils mobiles!
		
		\item[$\bullet$] \textbf{Telesat}: Pour identifier ou anticiper dans le ciel nocturne le passage de certains satellites bien connus.
		
		\item[$\bullet$] \textbf{ISS Dectector}: Même idée que pour Telesat mais pour la Station Spatiale Internationale (ISS).
		
		\item[$\bullet$] \textbf{Fligthradar}: Mêmes idées que pour Telesat et ISS Detector mais pour presque tous les avions civils (mais la version gratuite montre les avions avec un petit décalage pour des raisons de sécurité évidentes...).
		
		\item[$\bullet$]  \textbf{Sun Info}: Une application qui vous aide à anticiper la position du Soleil et de la Lune pendant de nombreuses années, ainsi que le solstice été / hiver, les prochaines éclipses, etc.
		
		\item[$\bullet$] \textbf{Stellarium}: Comme son nom l'indique, c'est un magnifique stellarium en temps réel et complet pour regarder les étoiles, les galaxies, les constellations et aussi certains satellites.
		
		 \item[$\bullet$] \textbf{C. Anatomy 19}: Une application (assez chère avec un abonnement mensuel / annuel) pour découvrir en haute résolution 3D l'intérieur du corps humain.
		 
		 \item[$\bullet$] \textbf{EarthViewer}: Une application pour découvrir et jouer avec les dérives des continents de la Terre à travers le temps.
		 
		 \item[$\bullet$] \textbf{MathStudio}: Une calculatrice arithmétique et algébrique très puissante avec des options de calcul similaires à celle de Maple.
		 
		 \item[$\bullet$] \textbf{PCalc Lite}: Un calculateur arithmétique complet très intuitif (mais évidemment moins puissant que MathStudio!).
		 
		 \item[$\bullet$] \textbf{Molecules}: Un visualiseur de molécules 3D où vous pouvez télécharger des molécules à partir d'une base de données en ligne.
		 
		 \item[$\bullet$] \textbf{E Numbers}: E Numbers vous donnent les codes pour les substances pouvant être utilisées comme additifs alimentaires.
		 
		 \item[$\bullet$] \textbf{iCircuit}: Une belle application pour créer des circuits électroniques simples et les simuler! Très bon pour apprendre les bases de l'électronique pendant les vacances sans avoir besoin de composants physiques réels.
		 
		 \item[$\bullet$] \textbf{Electronic TB}: Une application qui fournit une liste très importante (exhaustive) et des calculatrices pour de nombreux composants électroniques, y compris plusieurs de leurs propriétés techniques!
	\end{itemize}

			
	\chapter{Citations}
	\begin{figure}[H]
		\centering
		\includegraphics{img/be_greater_than_average.jpg}	
	\end{figure}
	\begin{fquote}[John Aubrey]Sans la foi, Dieu n'est rien. Sans science, l'homme n'est rien!
 	\end{fquote}
 	
	\begin{fquote}[Leonardo da Vinci]Celui qui aime la pratique sans la théorie est comme le marin qui monte à bord sans gouvernail ni compas et ne sait jamais où il peut acoster.
 	\end{fquote}
 	
	\begin{fquote}[Richard Feynman]La physique est comme le sexe. Bien sûr, cela peut avoir des résultats pratiques, mais ce n’est pas pour cela que nous le faisons.
 	\end{fquote}
 	
	\begin{fquote}[Martin Gardner]Les mathématiques ne sont pas seulement réelles, mais c'est la seule réalité. C'est que tout l'univers est fait de matière, évidemment. Et la matière est faite de particules. Il est composé d'électrons, de neutrons et de protons. Donc, l'univers entier est composé de particules. Maintenant, de quoi sont faites les particules? Elles ne sont pas faits de rien. La seule chose que vous pouvez dire de la réalité d'un électron est de citer ses propriétés mathématiques. Il y a donc un sens dans lequel la matière s'est complètement dissoute et ce qui reste n'est qu'une structure mathématique.
 	\end{fquote}
 	
 	\begin{fquote}[Owen Chamberlain]Chaque génération de scientifiques se tient sur les épaules de ceux qui ont précédé.
 	\end{fquote}
 	
 	\begin{fquote}[Kiyoshi Ito]Dans les structures mathématiques construites avec précision, les mathématiciens trouvent le même genre de beauté que les autres trouvent dans des pièces musicales enchanteresses ou dans des architectures magnifiques. Il y a cependant une grande différence entre la beauté des structures mathématiques et celle du grand art. La musique de Mozart, par exemple, impressionne même ceux qui ne connaissent pas la théorie musicale; la cathédrale de Cologne accable les spectateurs même s'ils ne connaissent rien du christianisme. La beauté des structures mathématiques, cependant, ne peut être appréciée sans comprendre un groupe de relations numériques qui expriment des lois de la logique. Seuls les mathématiciens peuvent lire des "partitions musicales" contenant de nombreuses relation numériques et jouer cette "musique" dans leurs coeurs.
 	\end{fquote}
 	
 	\begin{fquote}[Ray Bradbury]Vous n'avez pas besoin de brûler des livres pour détruire une culture. Empêchez juste les gens de lire.
 	\end{fquote}
 	
 	 \begin{fquote}[Edward Teller]La science d'aujourd'hui est la technologie de demain.
 	\end{fquote}
 	
 	 \begin{fquote}[Brian Cox]La science est différente de tous les autres systèmes de pensée ... parce que vous n'avez pas besoin d'avoir foi en elle, vous pouvez vérifier par vous-même qu'elle fonctionne!
 	\end{fquote}
 	
 	\begin{fquote}[George Bernard Shaw]Le fait qu'un croyant soit plus heureux qu'un sceptique n'est pas plus frappant que le fait qu'un homme ivre soit plus heureux que celui qui est sobre.
 	\end{fquote}
 	
 	\begin{fquote}[Carl Sagan]Je ne veux pas croire, je veux savoir!
 	\end{fquote}
 	
 	 \begin{fquote}[Richard Feynman]La science n'est que de l'imagination dans dans une camisole de force.
 	\end{fquote}
 	
 	 \begin{fquote}[Stephen Hawking]Le plus grand ennemi de la connaissance n'est pas l'ignorance, c'est l'illusion de la connaissance.
 	\end{fquote}
 	
 	 \begin{fquote}[?]Sans données, vous êtes juste une autre personne sans opinions, mais sans les mathématiques, vous n'êtes qu'un collectionneur de données.
 	\end{fquote}
 	
 	\begin{fquote}[Albert Einstein]Je n'ai pas de talents particuliers, je ne suis que passionnément curieux.
 	\end{fquote}

	\begin{fquote}[Dara Briain]Les gens continuent à dire "la science ne sait pas tout!". Eh bien, la science sait qu'elle ne sait pas tout; sinon elle s'arrêterait.
 	\end{fquote}
 

 	 \begin{fquote}[Stephen Hawking]On ne peut pas prouver que Dieu n'existe pas, mais la Science rend Dieu inutile.
 	\end{fquote}

 	 \begin{fquote}[Voltaire]Le pouvoir des nombres était beaucoup plus respecté quand nous ne savions rien d'eux.
 	\end{fquote}

	\begin{fquote}[Jean Rostand]Réfléchir, c'est déranger ses pensées!
 	\end{fquote} 
 	
 	\begin{fquote}[John Cleese]Si vous êtes stupide, comment pouvez-vous vous rendre compte que vous êtes stupide?
 	\end{fquote} 	
 	
 	\begin{fquote}[Frank Wilczek]Si vous ne faites pas d'erreurs, c'est que vous ne travaillez pas sur des problèmes assez dur. Et c'est une grosse erreur.
 	\end{fquote}

	\begin{fquote}[?]Le succès repose sur le succès de ceux qui nous ont précédés.
 	\end{fquote}
 	
 	\begin{fquote}[Brigitte Le Roux, Henry Rouanet]Entre quantité et qualité, il y a la géométrie.
 	\end{fquote}

	\begin{fquote}[Leonardo da Vinci]La pratique devrait toujours être basée sur une bonne connaissance de la théorie.
 	\end{fquote}
 	
 	\begin{fquote}[Richard Feynman]Peu importe la beauté de votre théorie, quelle que soit votre intelligence! Si ce n'est pas en accord avec l'expérience, c'est que c'est faux!
 	\end{fquote}
 	
 	\begin{fquote}[Ronald Fisher]Appeler le statisticien après que l'expérience soit faite n'est peut-être rien de plus que de lui demander d'exécuter un constant post-mortem - il pourra peut-être dire de quoi l'expérience est morte...
 	\end{fquote}
 	
 	\begin{fquote}[Charles Babbage]Les erreurs utilisant des données inadéquates sont beaucoup moins nombreuses que celles n'utilisant aucune donnée!
 	\end{fquote}
 	
 	\begin{fquote}[Leonardo da Vinci]Les détails font la perfection et la perfection n'est pas un détail.
 	\end{fquote}
 	
 	\begin{fquote}[Neil Degrasse Tyson]L'ignorance est un virus. Une fois qu'il commence à se répandre, il ne peut être guéri que par la raison. Pour l'amour de l'humanité, nous devons être ce remède!
 	\end{fquote}
 	
 	\begin{fquote}[David Hilbert]L'art de faire des mathématiques consiste à trouver ce cas particulier qui contient tous les germes de la généralité.
 	\end{fquote}

 	\begin{fquote}[Derek Bok]Si vous pensez que l'éducation coûte cher, essayez l'ignorance.
 	\end{fquote}
 	
 	 \begin{fquote}[Benjamin Franklin]Dites-moi et j'oublie, enseignez-moi et je me souviens, impliquez-moi et j'apprends.
 	\end{fquote}
 

 	 \begin{fquote}[Niels Bohr]Un expert est une personne qui a découvert par sa propre expérience toutes les erreurs que l'on peut commettre dans un domaine très restreint.
 	\end{fquote}
 	
 	 \begin{fquote}[Daniel C. Denett]Ce que vous pouvez imaginer dépend de ce que vous savez.
 	\end{fquote}
 	
 	\begin{fquote}[Arthur C. Clarke]Toute technologie suffisamment avancée ne se distingue pas de la divinité.
 	\end{fquote}
 	
 	\begin{fquote}[Nikola Tesla] Ce qu'un homme appelle Dieu, un autre l'appelle les lois de la Physique.
 	\end{fquote}
 	
 	\begin{fquote}[?]La seule personne dont vous avez besoin d'être meilleur que est la personne que vous étiez hier!
 	\end{fquote}
 	
 	\begin{fquote}[Eleanor Roosevelt]Les grands esprits discutent des idées. Les esprits moyens discutent des événements. Les petits esprits discutent des gens.
 	\end{fquote}
 	
 	\begin{fquote}[Marie Curie]Soyez moins curieux des gens et plus curieux des idées.
 	\end{fquote}
 	
 	\begin{fquote}[Albert Einstein]La créativité est de l'intelligence qui s'amuse.
 	\end{fquote}
 	
 	\begin{fquote}[Aaron Swartz]Avec suffisamment d'entre nous, partout dans le monde, nous ne ferons pas que lancer un message fort contre la privatisation de la connaissance - nous en ferons un passé!
 	\end{fquote}
 	
 	\begin{fquote}[Buddha]Lorsque vous passez votre focalisation de la concurrence à la contribution, la vie devient une célébration! N'essayez jamais de vaincre les gens, gagnez simplement leurs coeurs.
 	\end{fquote}
 	
 	\begin{fquote}[Bertrand Russell]Le problème dans le Monde est que les gends stupides sont sûrs d'eux et que les intelligents sont pleins de doutes.
 	\end{fquote}
 	
 	\begin{fquote}[Gerge Bernard Show]Certains hommes voient les choses telles qu'elles sont et demandent pourquoi? D'autres rêvent de choses qui n'ont jamais existé et se demandent pourquoi pas?
 	\end{fquote}
 	
 	\begin{fquote}[Albert Einstein]Si vous ne pouvez pas l'expliquer à un enfant de six ans, vous ne le comprenez pas assez bien.
 	\end{fquote}
 	
 	\begin{fquote}[Hubert Reeves]L'homme est l'espèce la plus folle. Il vénère un Dieu invisible et détruit une nature visible. Ignorant que la nature qu'il détruit est ce Dieu qu'il adore.
 	\end{fquote}
 	
 	\begin{fquote}[Willima Thomson (Lord Kelvin)]Vous ne pouvez pas mesurer, vous ne pouvez l'améliorer!
 	\end{fquote}
 	
 	\begin{fquote}[Albert Einstein]Dès que vous cessez d'apprendre, vous commencez à mourir.
 	\end{fquote}
 	
 	\begin{fquote}[Socrates]Le seul bien est la CONNAISSANCE, le seul mal est l'IGNORANCE.
 	\end{fquote}
 	
 	\begin{fquote}[?]Le but de l'argumentation ne devrait pas être la victoire, mais le progrès.
 	\end{fquote}
 	
 	\begin{fquote}[Richard Dawkins]La science est la poésie de la réalité!
 	\end{fquote}
 	
 	\begin{fquote}[Sthepen Hawking]La vraie science peut être beaucoup plus étrange que la science-fiction et beaucoup plus satisfaisante.
 	\end{fquote}
 	
 	\begin{fquote}[Alan Watts]Vous ne comprendrez pas les hypothèses de base de votre propre culture si votre propre culture est la seule culture que vous connaissez.
 	\end{fquote}
 	
 	\begin{fquote}[Elon Musk]Ceux qui sont assez fous pour penser qu'ils peuvent changer le monde sont ceux qui le font!
 	\end{fquote}
 	
 	\begin{fquote}[Leonardo da Vinci]Le plus grand plaisir est la joie de la compréhension!
 	\end{fquote}
 	
 	\begin{fquote}[Galileo Galilei]Le doute est la source de toute invention!
 	\end{fquote}
 	
 	\begin{fquote}[Fred Mosteller]Il est facile de mentir avec des statistiques, mais il est plus facile de mentir sans elles.
 	\end{fquote}

\chapter{Journal des changements}

Ceci est un journal de changement (log) détaillé du présent livre pour les personnes intéressées à voir comment ce dernier a évolué (pour rester informé des nouvelles versions, envoyez-nous un e-mail pour vous abonner à la newsletter):
	\begin{itemize}
		\item \textbf{Mai 2002}
		\begin{itemize}[noitemsep]
			\item Définitions (sciences, loi, théorème, postulat, axiome, corollaire,...)
			\item Opérations d'addition, soustraction, multiplication, division (puissance)
			\item Les nombres (entiers, relatifs, réels, fractionnaires, complexes, algébriques, abstraits,...)
			\item Domaine de définition d'une variable
			\item Polynômes arithmétiques
			\item Valeur absolue
			\item Relations binaires, relations d'ordre
			\item Fonction Gamma d'Euler, constante d'Euler
			\item Equation d'onde électromagnétique, vitesse de l'onde, vitesse de la lumière, énergie véhiculée
			\item Introduction à l'optique
			\item Règle de trois, pourcentages
			\item Quantités nettes, prix de revient d'achat, indices, prix de ventes, prix de ventre brut, bénéfice net, nombre clé, actifs
			\item Intérêts simples et composés, versements tardifs et précoces
			\item Gestion de portefeuilles (critères de décision, goodwill, return on investisment)
			\item Modèle de Markowitz (fonction d'utilité, élaboration du critère de choix)
			\item Bourse
			\item Fréquence Plasma
			\item Probabilités (univers, événements, axiomes)
			\item Analyse combinatoire 
			\item Statistiques (variables discrètes, variables continues, écart-type, variance et covariance)
			\item Fonctions de distribution (fonction discrète uniforme, de Bernoulli, binomiale, hypergéométrique, multinomiale, de Poisson, de Gauss-Laplace, de Cauchy, bêta, gamma, Khi-deux, loi de Student)
			\item Estimateurs, la corrélation
			\item Matrice des covariances
			\item Tests statistiques d'adéquation 
		\end{itemize}
		\item \textbf{Juin 2002}
			\begin{itemize}[noitemsep]
				\item Ensembles (théorie Zermelo-Fraenkel, l'inclusion, la complémentarité, l'intersection, la réunion, la différence, le produit, l'ensemble vide
				\item Moyennes arithmétique, harmonique, géométrique, quadratique
				\item Univocités
				\item Fonctions (logarithmes et exponentielles)
				\item Nombre d'or
				\item Introduction aux séries de Fourier
				\item Loi de Coulomb, champ électrostatique, potentiel électrostatique
				\item Théorème d'Ampère
				\item Loi de Biot et Savart, champ magnétique, induction magnétique
				\item Equations de Maxwell
				\item Dimensions
				\item Formes géométriques unidimensionelles, bidimensionelles, tridimensionelles
				\item Théorème de Pythagore
				\item Systèmes d'unités en physique
				\item Principe de moindre action et de conservation de l'énergie
				\item Positions, vitesse et accélération
				\item Equation de continuité
				\item Equation de Bernoulli
				\item L'effet Doppler
				\item Relativité Restreinte, principe d'invariance, transformations de Lorentz (temps, longueurs, addition des vitesses, augmentation de la masse, espace-temps de Minkowski
				\item Relativité générale (ligne d'univers, événement ponctuel, particules inertielles, cônes nuls, vecteurs d'espace-temps, espaces plans et courbes, tenseur métrique)
				\item Principes d'incertitudes de Heisenberg, équation de Schrödinger, densité de probabilité
				\item Introduction aux supercordes
			\end{itemize}
		\item \textbf{Juillet 2002}
			\begin{itemize}[noitemsep]
				\item Création d'une table des matières
				\item Représentation des fonctions
				\item Polynômes et zéros de polynômes du deuxième degré
				\item Opérateur de champs vectoriels et scalaires (gradient, nabla, divergence, rotationnel, Laplacien)
				\item Analyse vectorielle (notion de flèche, ensemble des vecteurs, multiplication par un scalaire, espace vectoriel, combinaisons linéaires, familles génératrices, bases d'un espace vectoriel)
				\item Calcul Tensoriel (convention d'Einstein, symbole de Kronecker, symbole d'antisymétrie)
				\item Notations de sommation (Sigma), notation de multiplication (Pi majuscule)
				\item Axiomes pour l'ensemble des réels
				\item Inégalités
				\item Mouvements circulaires et relatifs
				\item Forces d'inertie (Force de coriolis)
				\item Equation de Drake
			\end{itemize}
		\item \textbf{Août 2002}
			\begin{itemize}[noitemsep]
				\item Nouveaux personnages (Hilbert, Riemann, Legendre)
				\item Nouveau chapitre "Humour"
				\item Introduction à la Topologie
				\item Postulats d'Euclide
				\item Plan de Gauss
				\item Formule de Moivre
				\item Transformations dans le plan complexe
				\item Changement de base et produit scalaire tensoriel
				\item Composantes covariantes et contravariantes
				\item Transformations relativiste de la quantité de mouvement
				\item Préfixe des multiples et sous-multiples des unités
				\item Contrôles de qualité (probabilités), courbe d'efficacité, valeur de niveau de qualité acceptable (NQA)
				\item Lois de Kepler
				\item Démonstration de la force gravitationelle classique à partir des lois de Kepler
				\item Démonstration de la loi des Aires (deuxième loi de Kepler)
				\item Moment de force
				\item Moment cinétique
				\item Théorème du moment cinétique
				\item Equation de Newton-Poisson
				\item Modèle de l'atome de Thomson, de Bohr, de Bohr-Sommerfeld
				\item Spectre de l'hydrogène
				\item Hypothèse du Neutron
				\item Nombres quantiques
				\item Principes d'exclusion de Pauli
				\item Analyse classique de l'équation de Schrödinger pour le puits de potentiel rectangulaire idéal
				\item Lois de Newton
				\item Démonstration de la troisième loi de Newton à partir du principe de moindre action
				\item Vitesse de libération
				\item Définition du "Scientisme"
				\item Variation de l'accélération gravifique dans et à l'extérieur d'un astre homogène sphérique
				\item Principe de moindre action et physique quantique (limite semi-classique)
				\item Balistique (portée maximale, parabole de sûreté)
				\item Introduction à la physique nucléaire
				\item Nombre atomique, nombre de masse
				\item Radioactivité, activité, filiation, isotopes, isotones, nucléides, datation
				\item Système de masse atomique (UMA)
				\item Défaut de masse
				\item Fusion et fission nucléaire
				\item Désintégration alpha, beta (moins et plus), capture électronique, émission gamma
			\end{itemize}
		\item \textbf{Septembre 2002}	
			\begin{itemize}[noitemsep]
				\item Biographies de Dalton, Boltzmann et Broglie
				\item Principe du bon ordre
				\item Propriété Archimédienne
				\item Principe d'induction
				\item Divisibilité (division euclidienne)
				\item Nombres congrus
				\item Preuve par neuf
				\item Bases de nombres
				\item Définition de l'utilitié des séparateurs de milliers
				\item Priorité des parenthèses, crochets et accolades
				\item Priorité des opérandes
				\item Théorie de la démonstration (intro)
				\item Notions de termes, formules et démonstrations
				\item Définition des langages, symboles, relations et fonctions
				\item Définition des fonctions périodiques, composées, élémentaires, rationnelles entières, fractionnelles, irrationnelles, algébriques et transcendantes
				\item Généralisation de l'algèbre élémentaire
				\item Dimensions d'un espace vectoriel, pronlogement d'une famille libre, rang d'une famille finie
				\item Hyperplan vectoriel, sommes directes
				\item Définition rigoureuse des notions de ligne, surface (plan) et volume
				\item Droites sécantes, demi-droites, segments, partie aliquote
				\item Axiome de continuité de la droite
				\item Déplacement et retournement de plan
				\item Angles, unités, mesures, côtés de l'angle, angles saillants, angles plats, angles égaux
				\item Axiome de continuité du plan
				\item Angles droites, aigus, obtus, supplémentaires, complémentaires
				\item Droites perpendiculaires, bissectrice d'un angle
				\item Définition du travail et de l'énergie: Thèroème de l'énergie cinétique et potentielle (travail moteur et résistant)
				\item Notion de champ conservatif
				\item Conservation de l'énergie et de la quantité de mouvement
				\item Théorème du centre de masse
				\item Transformation relativiste de la force
				\item Transformées relativistes des champs électrique et magnétique
				\item Masse limite de Chandreskhar (limite d'effondrement de naines blanches)
				\item Défintions de l'optique, généralisation de la loi de la réfraction
				\item Condition de normalisation de de Broglie, état liés et non liés
				\item Oscillateur harmonique
				\item Chimie quantique, vibrations moléculaire
			\end{itemize}
		\item \textbf{Octobre 2002}
			\begin{itemize}[noitemsep]
				\item Ajout de biographies sur Cauchy, Neumann, Bessel et Archimède
				\item Nouvelle section sur les "Méthodes Numériques"
				\item Plus grand commun diviseur, plus petit commun multiple
				\item Règle des signes (...)
				\item Démonstration de l'irrationalité d'un nombre
				\item Introduction aux progressions arithmétiques, harmoniques et géométriques
				\item Limite et continuité des fonctions
				\item Définition des espaces affines
				\item Introduction aux tenseurs euclidiens et leurs propriétés
				\item Triangles et propriétés des triangles
				\item Définition des systèmes thermodynamiques
				\item Définition de la masse réduite
				\item Fonctions de Bessel
				\item Définition du moment d'inertie
				\item Introduction à la théorie quantique des champs
				\item Introduction à la radioprotection, formule de Bethe-Bloch
				\item Effet tunnel
				\item Introduction au formalisme de Dirac
				\item Algorithme d'Heron et d'Archimède
				\item Introduction aux ensembles fractals
				\item Introduction à la théorie des jeux (jeux coopératifs, gains, matrice des gains, formes extensives, optimums de Pareto, équilibre de Nash, jeux évolutionnaires)
			\end{itemize}
			\item \textbf{Novembre 2002}
				\begin{itemize}[noitemsep]
				\item Théorème fondamental de l'arithmétique
				\item Introduction à la crypthographie (RSA, DES, MD5, SHA-1)
				\item Fonction phi d'Euler
				\item Petit théorème de Fermat
				\item Introduction aux dimensions topologiques et d'homothétie
				\item Cosinus directeurs
			\end{itemize}
			\item \textbf{Décembre 2002}
				\begin{itemize}[noitemsep]
				\item Ajout de biographies sur Nash, Cartan, Lucas, et Lie
				\item Développement détaillé de la fonction "Entière"
				\item Superposition linéaire des états quantiques (cohérence quantique)
				\item Définition des fonctions Lipschitziennes et des fonctions contractantes
				\item Définition d'une suite convergente de Cauchy
				\item Théorème du point fixe (utilisé dans les fractals, méthodes de Newton et bcp d'autres)
				\item Définitions de la réalité, du problème, de la théorie et essai sur la réalité
				\item Définition de de l'espace euclidien et de l'espace affine euclidien
				\item Définition des notions de propriété dans le domaine de la chimie
			\end{itemize}
			\item \textbf{Janvier 2003}
			\begin{itemize}[noitemsep]
				\item Théorème de Pascal
				\item Poussée d'Archimède
				\item Introduction simple aux différentes symétries en physique (temporelle, spatiale)
				\item Introduction simple aux différentes transformation dans le plan (translation, homothétie, réflexion, isométrie, rotation)
				\item Définition d'une application réciproque et composée
				\item Trigonométrie (introduction, relations remarquables, trigonométrie sphérique)
				\item Signature d'un espace vectoriel
				\item Méthodes d'orthogonalisation de Schmidt, changements de bases, espaces associés de Fourier
				\item Tout (ou presque) sur la trigonométrie plane et sphérique
				\item Trajectoires d'orbitales képleriennes
				\item Introduction au modèle monétaire néo-classique (loi de say, postulat d'homogénéité)
				\item Algèbre de Boole (propriétés et théorèmes simples)
				\item Refonte du chapitre de physique quantique (ordre de présentation des sujets)
				\item Démonstration de l'équation d'évolution de Schrödinger
				\item Démonstration de l'équation d'évolution relativiste de Schrödinger
				\item Introduction à la théorie de l'anti-matière
			\end{itemize}
		\item \textbf{Février 2003}
				\begin{itemize}[noitemsep]
				\item Démonstration gradient, divergence, rotationnel et Laplacien en coordonnées cartésiennes, polaires, cylindriques et sphériques
				\item Développements mathématiques complets des expressions de la vitesse et de l'accélération en coordonnées cartésiennes, polaires, cylindriques et sphériques
				\item Démonstration de l'invariance relativiste de la charge électrique (équation de conservation de la charge électrique... je pense)
				\item Démonstration de l'existence d'anti-particule de charge opposées
				\item Introduction à la théorie de Jauges (quadripotentiel, jauge de Lorenz, jauge de Coulomb, d'Alembertien)
				\item Introduction au formalisme Lagrangien et Hamiltonien (coordonnées généralisées, espaces des configurations, équation d'Euler-Lagrange, formalisme canonique, transformation de Legendre, crochet de Poisson)
				\item Définition rigoureuse du principe de moindre action
				\item Espaces tensoriels et définition
				\end{itemize}
		\item \textbf{Mars 2003}
			\begin{itemize}[noitemsep]
				\item Ajout de biographies sur Lorentz, Hermann, Ricci-Curbastro et Levi-Civita
				\item Définition du produit cartésien et extension du cadre d'application des cardinaux
				\item Démonstration de l'inégalité de Cauchy-Schwarz
				\item Démonstration de l'inégalité triangulaire
				\item Définitions du produit vectoriel et du produit mixte
				\item Démonstration de la forme condensée la somme des n premiers nombres entiers
				\item Démonstration de la validité de l'intégration par partie
				\item Définition de la structure algébrique vectorielle et d'un algèbre
				\item Définition d'un homorphismes, isomorphisme, endomorphisme, automorphisme
			\end{itemize}
		\item \textbf{Avril 2003}
			\begin{itemize}[noitemsep]
				\item Théorie de l'équilibre de Cournot (concurrence)
				\item Modèle de Wilson (gestion de stocks)
				\item Mathématique des phaseurs
				\item Modèle relativiste de l'atome de Sommerfeld
				\item Résolution analytique de l'équation de Schrödinger
				\item Principes d'incertitudes quantiques de Heisenberg
				\item Formalisme Lagrangien de la physique quantique des champs
				\item Ajout de biographies sur Göpper-Meyer, Hideki, Nöther et Cournot 
				\item 10 nouveaux liens vers des pages web intéressantes (associations + mathématiques)
			\end{itemize}
		\item \textbf{Mai 2003}
			\begin{itemize}[noitemsep]
				\item Ajout de biographies sur Bell, Ramanujan et Landau 
				\item Démonstration de la précession du périhélie des orbites d'astres couplés et de charges couplées
				\item Définition et développements relatifs au théorème du Viriel
				\item Calcul de l'énergie potentielle d'un sphère de matière (température interne des étoiles)
				\item Définition des nombres premiers et démonstration qu'ils sont en nombre infini 
				\item Définition d'un anneau intégralement clos
				\item Démonstration qu'un nombre rationnel est un nombre algébrique si et seulement si c'est un entier relatif. 
				\item Définition d'une application multi-linéaire (ou morphisme d'espace vectoriel) 
				\item Définition d'une partition d'un ensemble et d'une classe d'équivalence. 
				\item Descriptions des opérations ensemblistes d'absorption et d'idempotence. 
				\item Définitions et exemples de diagrammes sagittals.
				\item Définition d'un magma et d'un monoïde
				\item Pseudo-démonstrations des structures aglébriques des ensembles fondamentaux de l'arithmétique
				\item Développement de la théorie du moment cinétique en physique quantique ondulatoire
				\item Définition d'une équation diophantienne et enoncé du grand théorème de Fermat
			\end{itemize}
		\item \textbf{Octobre 2003}
			\begin{itemize}[noitemsep]
				\item Ajout de biographies sur Abel, Banach, Boole, Bose, Luitzen Egbertus, Clausius, Cayley, Curie, Connes, Dirichlet, Frege, Gibbs, Picard, Erdös, Grothendieck,Hamilton, Hausdorf, Heaviside, Helmholtz, Hermite, Hoyle, Jacobi, Klein, Kronecker, Langevin, Lee, Lobatchevski, Möbius, Monge, Poisson, Schwartz, Shannon, Thom, Van Der Waals, Viète, Weinberg, Witten, Gamow, Sturm, Liouville, Clairaut, Teller
				\item Définition des espaces ponctuels en mécanique classique, des conventions d'écriture et des changements de repères.
				\item Théorie mathématique de la perspective projective à points de fuite et définition des perspectives techniques isométriques et cavalières.
				\item Définition simpliste du concept de "dérivée" en analyse fonctionnelle. Démonstration de quelques dérivées courantes (polynômes, fonctions composées, fonctions réciproques, cosinus, sinus, arcsinus, arcosinus, quotient de deux fonctions).
				\item Méthodes numériques: définition mathématique de la complexité d'un algorithme et recherches élémentaires d'optimisation
				\item Méthode de calcul du nombre e
				\item Méthode de résolution des systèmes de N équations linéaires à N inconues par la méthode du pivot 
				\item Recherche des racines de fonction par la méthodes des parties proportionnelles, de la bissection, de la sécante (regula falsi) et de la méthode de Newton - Calculs des aires et intégrales par la méthode des sommes de riemann 
				\item Exposition de la méthode de Monte-Carlo pour le calcul d'intégrales, de $\pi$ et de recherche de racines (dichotomie) 
				\item Développement mathématique de l'algorithme du simplexe utilisé dans le cadre des recherches opérationnelles (programmation linéaire) 
				\item Calcul tensoriel: la contraction des indices - de la définition et des propriétés de quelquens tenseurs particuliers (tenseurs symétriques, antisymétriques, tenseur fondamental) 
				\item Coordoonnées curvilignes (détermination de la métrique et élément linéaire d'un espace sphérique et cartésien et du plan en coordonnées polaires) 
				\item Symboles de christoffel (de première et de deuxième espèce).
				\item Démonstration du théorème de Cantor-Bernstein
				\item Détermination du lagrangien libre généralisé en Relativité Générale
			\end{itemize}
		\item \textbf{Décembre 2003}
			\begin{itemize}[noitemsep]
				\item Ajout des biographes sur Kirchhoff, Markowitz, Cox Merton, Sholes, Sharpe, Lindemann, Bachelier et Stefan 
				\item Démonstration d'une des formules de Stirling
				\item Calcul de la température de surface d'une étoile
				\item Calculs des propriétés temporelles des titres de valeurs
				\item Démonstration des séries de Taylor et Maclaurin (limité et non limité)
				\item Définition du reste de Lagrange et des critères de divergences d'Alembert, de Cauchy, Test de l'intégrale, Convergence absolue
				\item Développements relatifs aux définitions de la luminosité, éclat, magnitude apparente et absolue des étoiles et calcul de la distance des céphéides
				\item Définitions de l'angle solide, l'angle solide de révolution et l'angle solide élémentaire
				\item Introduction à la photométrie: définition des grandeurs photométriques, photoniques et S.I., définitions et développements relatifs à l'intensité lumineuse, le flux énergétique (avec démonstration de la loi de Beer-Lambert), l'émittance énergétique, luminance énergétique (avec démonstration de la loi de Lambert), loi de Kirchhoff (vois le chapitre d'optique dans la section d'électromagnétisme)
				\item Démonstration de la loi de Stefan et de la loi de Stefan-Boltzmann
				\item Résolutions des équations du troisième degré par radicaux (méthode de Cardan) et développements de la résolution des équations du second degré dans l'ensemble des complexes
				\item Définition du concept d'équations et inéquations
				\item Détermination de l'équation cartésienne du plan, d'un droite (dans l'espace), d'un cône, d'une sphère
				\item Détermination de l'équation de Black \& Sholes (définition du postulat d'efficience du marché) et présentation des processus de Wiener et du lemme d'Ito ainsi que du mouvement Brownien (marche au hasard)
			\end{itemize}
		\item \textbf{January 2004}
			\begin{itemize}[noitemsep]
				\item Définition de l'intégrale indéfinie
				\item Modèle cosmologique newtonien (sans la constante cosmologique)
				\item Enoncé des axiomes de Peano
				\item Introduction à l'algèbre linéaire (méthode de réduction de Gauss, opérations élémentaires entre matrices)
				\item Enoncé des 5 axiomes d'Euclide et des 5 groupes d'axiomes en géométrie
				\item Démonstration des relations de calcul du perimètre, de la surface et du barycentre du carré, du rectangle, et du triangle. Démonstration des relations du calcul des volumes et des surfaces du tore, de la sphère, de l'ellipsoïde, du cylindre et du cône
				\item Définition du barycentre (centre de gravité) et démonstration de quatre propriétés y relatives
				\item Démonstration de la décomposition en fonction impaire et paire de toute fonction 
				\item Définition des fonctions trigonométriques hyperboliques et énumération des relations et propriétés remarquables y relatives
				\item Introduction à la géométrie différentielle (définition d'une géométrie riemanienne, trièdre de Frenet, nappe paramétrée, ...)
				\item Introduction à la théorie des graphes (ponts de Königsberg)
				\item Définition d'un espace topologique et de Hausdorff, définition d'un espace métrique/ultra-métrique ainsi que des distances associées (hölderiennes, discrète, équivalentes, ...), définition d'une fonction lipschitzienne (et des isométries associées), définition des ensemble ouverts et fermés (boules ouvertes, fermées, sphères, adhérence, diamètre, excès de Hausdorff), définition des diamètres définition des distances ensemblistes (gap), définition d'une variété/carte/atlas et homéomorphisme différentiel.
			\end{itemize}
		\item \textbf{Avril 2004}
			\begin{itemize}[noitemsep]
				\item Démonstration du théorème de Guldin
				\item Démonstration du théorème de König de l'énergie cinétique et du moment cinétique
				\item Présentation et démonstration de techniques de calcul pour les moments d'inertie: théorème d'Huygens-Steiner, moment d'inertie polaire, tenseur d'inertie, théorème d'Huygens-Steiner généralisé
				\item Définition de la "Puissance" et du "Rendement" et démonstration du calcul de la puissance d'une machine tournante
				\item Démonstration de loi d'entropie thermodynamique de Boltzmann ainsi que des distributions statistique suivantes: vitesses de Maxwell, Maxwell-Boltzmann, Fermi-Dirac, Bose-Einstein
				\item Démonstration de quelques moments d'inertie principaux des corps suivants: tore, boule, boule creuse, cône, plaque rectangulaire, tube
				\item Introduction à l'optique ondulatoire: éconcé du principe de Huygens, démonstration de la loi de Malus, développement du modèle de la diffraction de Fraunhofer dans le cas d'un fente rectangulaire. Définition et détermination du pouvoir de résolution d'une fente rectangulaire simple.
				\item Démonstration de la provenance et de la solution de la non moins fameuse équation différentielle de Bessel d'ordre n
				\item Démonstration de la fonction d'onde d'une corde tendue et d'une membrane circulaire tendue
				\item Démonstration de la loi de Planck et démonstration des approximations connues: première loi de Wien, loi de Rayleigh-Jeans.
				\item Démonstration de la loi de déplacement (deuxième loi de Wien) et de la loi de Stefan-Boltzmann en passant par la loi de Planck et détermination de la constante de Stefan-Boltzmann
				\item Démonstration des dimensions de Planck: longueur de Planck, masse de Planck, densité de Planck, temps de Planck, énergie de Planck
			\end{itemize}
		\item \textbf{Juillet 2004}
			\begin{itemize}[noitemsep]
			\item Démonstration de l'origine physique de la chaleur
				\item Démonstration du théorème de Toricelli, de l'effet Venturi et de la loi de Poiseuille
				\item Démonstration de l'effet Compton relativiste
				\item Démonstration de l'existence du rayonnement fossile de l'Univers
				\item Introductions aux ensembles quotients (en l'occurrence $\mathbb{Z}/n\mathbb{Z}$)
				\item Démonstration des transformations de Lorentz des vitesses et des accélérations
				\item Détermination du lagrangien relativiste d'un système libre
				\item Démonstration et définition du théorème de Ricci, de la dérivée covariante, de l'identité de Ricci, du tenseur de Riemann-Christoffel, du tenseur de Ricci, du scalaire de Ricci, des deux identités de Bianchi et enfin du tenseur d'Einstein
				\item Définition mathématique de la capacité et détermination de l'expression d'une capacité plane parallèle 
				\item Détermination de l'énergie potentielle électrostatique
				\item Démonstration de la valeur du champ et potentiel électrique d'un fil rectiligne infini.
				\item Détermination des propriétés fondamentales des dipôles électriques tel que le moment dipolaire rigide, la présentation du moment dipolaire induite, les liaisons hydrogènes, les forces de Van der Waals, etc.
				\item Définition du principe de symétrie de Curie et énoncé des 6 propriétés en découlant
				\item Définition des pseudo-vecteurs
				\item Démonstration du calcul du volume de la pyramide, du paraboloïde, du tetraèdre et de l'ocataèdre ainsi que écritue du volume du cube et de parallélépipéde
				\item Détermination du champ magnétique produit par un solénoïde toroïdal, d'un solénoïde rectiligne infini et d'une boucle de courant
				\item Détermination du champ magnétique produit par un dipôle magnétique et définitions fondamentales des propriétés des matériaux magnétiques 
				\item Calcul du rayon de Larmor et de la pulsation gyromagnétique dans un cadre non relativiste
				\item Détermination du lagrangien du champ électromagnétique et par extension dans l'approximation non relativiste du tenseur du champ électromagnétique
				\item Introduction aux calculs du rayonnement émis par une charge accélérée (rayonnement synchrotron, potentiels retardés de Liénard-Wiechert)
				\item Calcul des valeurs des résistances et capacités en série. 
				\item Différence entre le potentiel électrique et le potentiel électromoteur
				\item Démonstration de la loi de Faraday et définition de la "self"
				\item Démonstrations des formules de Descartes pour les surfaces sphériques concaves et convexes réfringent et non réfrignent ainsi que pour les lentilles réfringentes. 
				\item Définition du stigmatisme et démonstration que la parabole est rigoureusement stigmatique
				\item Démonstration des formules de Descartes pour les lentilles minces et détermination de loi de conjugaison
				\item Définition de la dioptrie et explication des différentes handicaps visuels
			\end{itemize}
		\item \textbf{Septembre 2004}
			\begin{itemize}[noitemsep]
				\item Énonce du principe de Mach
				\item Présentation de l'effet photoélectrique et détermination de la loi physique le régissant. Démonstration par l'exemple que la lumière peut être vue à la fois comme un corpuscule ou comme une onde
				\item Démonstration du théorème de la classe monotone
				\item Détermination détaillée de l'équation de Klein-Gordon généralisée (particule relativiste dans un champ magnétique) ainsi que de l'équation de Dirac libre avec laquelle sont exposées les solutions explicites de Pauli (particules, antiparticules). 
				\item Détermination du rayon de l'atome en utilisant la diffusion de Rutherford-Coulomb (in extenso: détermination de la section efficace de Rutherford)
				\item Présentation des interactions macroscopiques et microscopiques du rayonnement X et gamma avec la matière (étude qui comporte les détails de la matérialisation du photon en une paire électron-positron). 
				\item Introduction aux spineurs en définissant.
				\item Définition des propriétés opératoires des matrices, des matrices remarquables, des déterminants, des vecteurs et valeurs propres dans le chapitre d'algèbre linéaire. 
				\item Enoncés des postulats de la physique quantique ondulatoire
- Détermination des orbitales de l'atome hydrogénoïde
			\end{itemize}
		\item \textbf{Novembre 2004}
			\begin{itemize}[noitemsep]
				\item Exposé et démonstration du théorème de Noether
- Enumération de quelques constantes physiques, chimiques, et astronomiques
				\item Ajout de biographies sur Smith, Say, Malthus, Keynes, Walras et Pareto
				\item Introduction à la théorie de la spéculation: calcul de l'espérance mathématique prévisionnelle d'un actif financier
				\item Introduction à la théorie de préférence (modèle d'Arrow-Debreu). En d'autres termes: étude de l'utilité des paniers des agents économiques et des courbes d'indifférences
				\item Exposé des solutions de l'équation de Black \& Scholes et remarques sur le delta
				\item Démonstration de l'équation de parité Call-Put
				\item Modèle de détermination du stock intial (optimal) dans la cadre des technique de gestion de production 
			\end{itemize}
		\item \textbf{Janvier 2005}
			\begin{itemize}[noitemsep]
				\item Ajout de biographies sur Penrose, Hawking, Turing et Marx
				\item Développement de la version "duaire" des équations de Maxwell et exposé (démonstration) de la provenance de l'expression (hypothèse) des monopôles magnétiques
				\item Définition des algorithmes de classe P, NP et NPC.
				\item Démonstration du théorème fondamental de l'analyse ou également appelé "théorème fondamental du calcul intégral et différentiel"
				\item Présentation du cube $cGh$ et de l'interprétation de Copenhague
				\item Introduction aux concepts des réseaux de neurones formels
				\item Introduction aux concepts des algorithmes génétiques
				\item Introduction rigoureuse aux espaces fractales
				\item Introduction à l'informatique quantique
				\item Introduction à la logique floue
				\item Démonstration du théorème de Shannon, des théorèmes de Morgan, théorèmes d'expansion, tables de Karnaugh, addionneur complet, soustracteur complet
			\end{itemize}
		\item \textbf{Avril 2005}
			\begin{itemize}[noitemsep]
				\item Définitions de base sur les codes en blocs, les codes linéaires, les codes systématiques
				\item Démonstration de la relation de la variation relativiste de la masse
				\item Introduction aux codes et codes préfixes
				\item Calcul du nombre de jours entre deux dates données
				\item Techniques de calculs d'arrondis
				\item Définition des rentes post et praenumerando rigides ou non à taux constant (avenir certain) ou variables
				\item Définition et étude des propriétés des emprunts à remboursement ou annuité constant. 
			\end{itemize}
		\item \textbf{Avril 2006}
			\begin{itemize}[noitemsep]
				\item résentation du critère de Schild par l'intermédiaire de l'effet Einstein (redshift gravitationnel)
				\item Développement de la l'approximation newtonienne de l'équation des géodésiques.
				\item Définitions et démonstrations des formes développés des quadrivecteurs déplacement, vitesse, courant, accélération et énergie-impulsion
				\item Démonstration de la provenance du tenseur du champ électromagnétique et calculs de transformation de référentiels.
				\item Définition d'une tautologie, principe de non-tautologie
				\item Descriptions, définitions et démonstrations nombreuses sur les quaternions + démonstration de l'irrationalité du nombre d'euler
				\item Définition de la loi log-normal et triangulaire et démonstration de leur espérance et écart-type
				\item Introduction au calcul d'erreurs (incertitudes absolues et relatives, propagation des erreurs, chiffres significatifs, etc.)
				\item Définition de la loi de Weibull, démonstration de son espérance et écart-type
				\item Démonstration de la déviation de la lumière au bord d'un astre avec le modèle newtonien
				\item Définition d'une matrice de rotation (et développements y relatifs)
				\item Démonstration de l'existence de la division euclidienne dans l'anneau des polynômes
				\item Définitions du MWRR (Money Weighted Time of Return) et du TWRR (Time Weighted Rate of Return)
				\item Démonstration du théorème de Gauss-Ostrogradsky 
			\end{itemize}
		\item \textbf{Juillet 2006}
			\begin{itemize}[noitemsep]
				\item Définition du concept d'espace dual
				\item Définition de la loi de Pareto et démonstration de l'espérance et de la variance de la loi
				\item Définition des quantiles (quartile, centile)
				\item Démonstration que le mode est la valeur qui minimise la dispersion absolue
				\item Démonstration de l'inégalité de Minkowski et Bienaymé-Tchebychev
				\item Démonstration de la loi faible des grands nombres
				\item Démonstration de la formule d'Euler pour les graphes planaires
				\item Introduction mathématique à la méthode de contrôles de processus Six Sigma
				\item Démonstration du théorème du calcul variationnel
				\item Démonstration du calcul de la vitesse superluminique apparente d'un astre à Redshift élevé
				\item Méthode de résolution de jeux à somme nul utilisant la recherche opérationnelle
				\item Définition des fonds de placement
				\item Démonstration de l'expression du bêta selon les modèle de régression linéaire simple (Sharpe)		
			\end{itemize}
		\item \textbf{Octobre 2006}
			\begin{itemize}[noitemsep]
				\item Démonstration des estimateurs et vraisemblances des lois de Poisson et Binomiale
				\item Présentation du concept de "couleur" et de la synthèse additive et soustractive
				\item Démonstration de l'équation d'Einstein des champs (approche par l'approximation des champs faibles)
				\item Différenciation du principe d'équivalence, principe d'équivalence faible et principe d'équivalence d'Einstein
				\item Démonstration de la durée de l'arc Diurne des planètes dans l'approximation de la précession et nutation nulle
				\item Étude numérique et formelle (approximative) des points de Lagrange d'un système binaire
				\item Méthode de résolution des équations polynomiales du 4èmes degré (méthode de Ferrari)
				\item Présentation du déterminant de Gram via le volume euclidien représenté par le produit mixte des vecteurs d'une base canonique
				\item Définition des fonctions monotones, strictement monotones, etc.. sans approche formelle pure
				\item Détermination approximative via le potentiel de Yukawa (champs massiques) de la masse des mésons de l'interactions faible et de l'interaction nucléaire forte.
				\item Introduction aux équation différentielles linéaires et du premier ordre
			\end{itemize}
		\item \textbf{Décembre 2007}
			\begin{itemize}[noitemsep]
				\item Nombres et polynômes de Bernoulli
				\item Limite de Roche
				\item Aplattisement des corps célestes
				\item Pression et température cinétique
				\item Force aimant ou électroaimant
				\item Espaces vectoriels hermitiens et de Hilbert
				\item Solution de Schwarzschild en relativité générale
				\item Précession du périhélie de Mercure en relativité générale
				\item Déflexion de la lumière en relativité générale
				\item Effet Shapiro en relativité générale
				\item Modèle d'évaluation des actifs financiers
				\item L'Univers Trou Noir
				\item Constantes de couplage des interactions fondamentales
				\item Hamiltonien de l'équation de Schrödinger pour une particule chargée dans un champ électromagnétique
				\item Chaînes de Markov
				\item Théorie des files d'attentes
				\item Introduction aux mathématiques du génie météo et marin
				\item Exemples pratiques dans MS Excel du modèle d'efficience de Markowitz
				\item Exemples pratiques dans MS Excel du modèle d'efficience de Sharp
				\item Démonstration (simpliste) du théorème de Green(-Riemann) et de Stokes
				\item Introduction à l'analyse par composantes principales
				\item Démonstration du théorème spectral/cas réel
				\item Ajout des biographies sur Heckman, McFadden et Tesla
				\item Introduction à la régression logistique
				\item Démonstration du théorème de Rolle et des accroissements finis
				\item Démonstration du théorème de l'Hopital et des accroissements finis généralisés
				\item Introduction et démonstration en statistique de la fonction géométrique, sa variance et son espérance
				\item Démonstration du calcul de la surface et du volume des 5 polyèdres réguliers platoniciens
				\item Introduction à l'algèbre et géométrie corporelle
				\item Masse critique (masse Jeans) et Rayon critique (Rayons de Jeans) d'effondrement d'un nuage interstellaire/pépinières d'étoiles
				\item Temps d'effondrements d'un nuage interstellaire
				\item Durée de vie nucléaire d'une étoile
				\item Introduction aux transformées de Fourier
				\item Résolution détaillée des E.D.L. homogènes à coefficients constants
				\item Spirale de Cornu
				\item Mathématique des fonctions biométriques
				\item Détermination et résolution simple de l'équation de Pauli
				\item Solution générale (transformée de Fourier) de l'équation d'onde électromagnétique
				\item Introduction à la resistance des matériaux
				\item Horizon visuel
				\item Taux de croissance d'une population en fonction de la température
				\item Tube de Pitot et Perte de charge
				\item Théories de jauge $U(1)$ en physique quantique des champs
				\item Courbes de Bézier
				\item Système d'équation différentielles avec exponentiation de matrices
			\end{itemize}
		\item \textbf{Septembre 2008}
			\begin{itemize}[noitemsep]
				\item Nouvelles primitives usuelles importantes
				\item Démonstration détaillée des fonctions arcsinh et arccosh
				\item Démonstration de l'équation de Laplace et de la relation de Mayer
				\item Démonstration de la propagation d'ondes de pression
				\item Introduction historique pour la physique nucléaire
				\item Démonstration des relations de Maxwell en thermodynamique et introduction à l'énergie et enthalpie libre
				\item Modèle d'atmosphère adiabatique
				\item Démonstration des équations de Lorenz et de l'effet papillon
				\item Prisme
				\item Conditions de cohérence/interférence d'ondes électromagnétique
				\item Sphère de Bloch
				\item Primitives supplémentaires pour le Génie Civil et la Mécanique analytique
				\item Démonstration du volume de révolution de surface minimale
				\item Traitement de la particule libre
				\item Traitement du qubit polarisé et de qubit de spin 1/2
				\item Fonction caractéristique et théorème central limite
				\item Quelques démonstrations sur les inégalités dans les triangles
				\item Démonstration du volume d'un tonneau à section circulaire
				\item Démonstration de l'origine de la variance et de l'espérance de la loi de Student et Fisher-Snedecor
				\item Introduction au coût marginal
				\item Test statistique de l'ANOVA à un facteur
				\item Test statistique d'ajustement du Khi-deux de Pearson
				\item Ajout des biographies sur Pearson, Gosset et Fisher
				\item Introduction à l'analyse de la variance de la régression
				\item Analyse Factorielle des Correspondances
				\item Développement des circuits linéaires RC, RL, RLC série libres et forcés
			\end{itemize}
		\item \textbf{Septembre 2009}
			\begin{itemize}[noitemsep]
				\item Analyse de systèmes à topologie simple ou complexe pour la maintenance préventive
				\item Introduction aux fractions continues finies et infinies
				\item Solutions détaillées de l'effet tunnel d'une barrière rectangulaire
				\item Modèle mathématique de la désintégration alpha via effet tunnel
				\item Introduction au mouvement brownien selon modèle de Langevin
				\item Introduction à la tribologie/frottement
				\item Introduction à l'analyse des séries temporelles
				\item Démonstrations supplémentaires sur les indices de capabilité et procédé long terme et court terme ainsi que des appareils de mesure en statistique des procédés, des PPM et démonstration de la relation de Taguchi
				\item Démonstration de l'expression des potentiels électrique et magnétique de Lienard-Wiechert
				\item Introduction à l'analyse complexe
				\item Démonstration de la deuxième équation de Friedmann en cosmologie
				\item Démonstration du "ralentissement" de la lumière aux abords d'un Trou Noir
				\item Démonstration de l'expression du développement de Taylor d'une fonction de deux variables réelles
				\item Introduction aux plans d'expérience
				\item Démonstration du théorème d'Ehrenfest
				\item Démonstration des estimateurs de la loi de Weibull à deux paramètres
				\item Ajout d'un exemple d'application de la théorie de la décision
				\item Théorie des bandes (approximation parabolique et semi-classique) dans le cadre des semi-conducteur
				\item Théorème des résidus et séries de Laurent
				\item Démonstration des valeurs du Lean Six Sigma pour les processus
			\end{itemize}
		\item \textbf{Octobre 2011}
			\begin{itemize}[noitemsep]
				\item Calcul de l'orbite géostationnaire
				\item Calculs sur les ballons sonde PVC en météorologie
				\item Développements sur le gyroscope symétrique pesant et de la toupie
				\item Indice de Gini
				\item Équilibre séculaire, transitoire et non-équilibre en filiation radioactive
				\item Détermination approximative du rayon d'étoiles en rotation rapide
				\item Présentation de la Value At Risk delta-normale historique et en variance-covariance
				\item Deux nouvelles histoires drôles dans la section Humour...
				\item Ajout d'une biographie sur Agner Krarup Erlang
				\item Modèle statistique empirique de contrôle des salaires
				\item Démonstration simplifiée de l'absence d'opportunité d'arbitrage en finance
				\item Présentation du concept de portefeuille autofinancé sur sous-jacent risqué
				\item  Introduction aux techniques mathématiques des assurances
				\item Développements mathématiques sur la puissance et l'intensité d'une onde sonore longitudinale
				\item  Test statistique $Z$ bilatéral sur la différence de deux moyennes
				\item Test statistique de Student sur deux moyennes d'échantillons appariés
				\item Démonstration de la relation de l'intervalle de confiance statistique de proportions d'échantillons de grande taille
				\item Test statistique de l'égalité de deux proportions d'échantillons de grande taille
				\item Application du théorème de Shannon au calcul d'un indice de diversité en statistiques
				\item Démonstration de la détermination des coefficients d'une régression linéaire multiple
				\item Démonstration de la détermination des coefficients d'une régression linéaire simple passant par l'origine
				\item Introduction à l'analyse de la sensibilité
				\item Introduction aux statistiques de rangs/ordres
				\item Démonstration de la provenance, de l'espérance et de la variance de la loi binomiale négative
				\item Introduction aux cartes de contrôles avec démonstrations mathématiques détaillées
				\item Présentation de l'approche mathématique du Google Page Rank à ses débuts
				\item Démonstration de l'identité de Beltrami pour la simplification de l'équation d'Euler-Lagrange
				\item Test statistique binomial exact pour l'équilibre d'une population ayant deux caractéristiques
				\item Développements et études des vagues de gravité dans un fluide
				\item Quelques développements simples sur les engrenages/arbres d'engrenages
				\item Démonstration mathématique de l'effet de peau
				\item Théorie de l'arc-en-ciel
				\item Théorie du pendule double
				\item Loi de distribution de Boltzmann
				\item Loi de Dalton et d'Amagat
				\item Écoulement de la chaleur
				\item Puissance moyenne en courant alternatif
				\item Présentation de quelques calculs sur le betatron
			\end{itemize}
		\item \textbf{Mai 2013}
			\begin{itemize}[noitemsep]
				\item Exemple détaillé de construction d'un réseau de neurones particulier avec MS Office Excel
				\item Résolution des équations différentielles linéaires homogènes d'ordre 1 à coefficients non constants
				\item Ajout de l'exemple d'une transformée de Fourier d'une fonction Gaussienne et propriété de dérivation de la transformée de Fourier d'une dérivée
				\item Introduction aux interactions dans les ANOVA à deux facteurs
				\item Intervalle de confiance et intervalle de prédiction d'une régression linéaire
				\item Test des signes (médiane) en statistiques
				\item Introduction aux probabilités conditionnelles et l'espérance conditionnelle avec la loi de Pareto
				\item Détermination des estimateurs de la loi Gamma selon les méthodes des moments
				\item Ajout d'une deuxième approche pour la mise en évidence des marées
				\item Test statistique d'ajustement de Kolmogorov-Smirnov avec approche de Lilliefors
				\item Démonstration du modèle de gestion des quotas d'exploitation de Scheafer
				\item Démonstration du calcul de la période synodique des planètes et du temps de rétrogradation
				\item Démonstration de la construction de ponts browniens
				\item Démonstration de la provenance de la carte de contrôle des événements rares
				\item Calcul du facteur d'actualisation d'une assurance retraite basé sur l'inflation et l'espérance de vie
				\item Démonstration de l'ANOVA à deux facteurs sans répétition et à deux facteurs avec répétition
				\item Démonstration mathématique et physique élémentaire du fonctionnement du LASER
				\item Ajout de petites biographies sur Neper, Wilcoxon, Born, Heisenberg, Jordan, Kolmogorov, Stokes, Ostrogradsky, Zeeman, Henry, Faraday,  Meitner, Biot, Debye, Drude et Ohm
				\item Théorème de la série de Taylor avec reste intégral
				\item Stabilité de la loi de Poisson
				\item Test statistique de Poisson à 1 et 2 échantillons
				\item Tests statistiques non paramétriques de Kruskal-Wallis et Friedman
				\item Test statistique de normalité de Ryan-Joiner
				\item Test statistique C de Cochran
				\item Séries de Taylor-Maclaurin usuelles
				\item Régression polynomiale par la méthode des moindres carrés
				\item M.D.F. spatio-temporelle avec équations de Maxwell
				\item Analyse du seuil de rentabilité
				\item Oscillateur harmonique mécanique
				\item Effet Doppler acoustique
				\item Superposition d'ondes périodiques
				\item Test statistique de l'étendue de Tukey
				\item Modèle prévisionnel par moyenne mobile, coefficients saisonniers, lissage simple, double de Brown, double de Holt (additif), double de Holt et Winter (multiplicatif)
				\item Introduction élémentaire aux processus autorégressifs AR, autorégressifs AM, ARMA et ARIMA
				\item Démonstration de l'expression du facteur de correction sur population finie
				\item Test statistique exact de Fisher
				\item Lissage exponentiel de Laplace
				\item Kappa d'agrément de Cohen et test statistique de McNemar
				\item Modèle d'analyse de survie de Kaplan-Meier
				\item V de Cramer
				\item Clustering avec algorithme des K-Means
				\item Clustering avec algorithme pour dendrogrammes
				\item Test statistique de la somme des rangs signés de Wilcoxon à 1 échantillon et à deux échantillons appariés
				\item Étude quantitative de l'énergie potentielle effective (modèle harmonique de la liaison atomique) de l'atome hydrogénoïde
				\item Démonstration de l'équation des poutres (équation d'Euler-Bernoulli)
				\item Calcul du taux de défaillance d'un système en utilisant la technique du maximum de vraisemblance
				\item Ajout d'une chronologie des sciences
				\item Démonstration du modèle d'Einstein (loi de Dulong-Petit) de la capacité calorifique des solides cristallins et dérivation du modèle de Debye
				\item Modèle de Langenvin du diamagnétisme et paramagnétisme
				\item Introduction au calcul d'intégrales curvilignes
				\item Détermination naïve de l'énergie d'un dipôle magnétique
				\item Modèle nucléaire en "goutte liquide"
				\item Modèle de la résonance magnétique de spin
				\item Méthode du facteur d'intégrant de résolution d'équations différentielles
				\item Méthode de variation de la constante pour la résolution d'équations différentielles
				\item Loi de Mendel
				\item Rente viagère temporaire et différée
				\item Cycle de Carnot
				\item Modèle d'évaluation des actions de Durand et Gordon-Shapiro
				\item Test statistique d'Anderson-Darling
				\item Optimisation non-linéaire par la méthode Newton-Quadratique et de Gauss-Newton
				\item Méthode d'interpolation polynomiale de Lagrange
				\item Test statistique de Cochran-Mantel-Heanzel
			\end{itemize}
		\item \textbf{Novembre 2016}
			\begin{itemize}[noitemsep]
				\item Décomposition en valeurs singulières SVD
			\end{itemize}
		\item \textbf{Décembre 2016}
			\begin{itemize}[noitemsep]
				\item Test de Fieller (rapport de deux moyennes)
			\end{itemize}
		\item \textbf{Janvier 2017}
			\begin{itemize}[noitemsep]
				\item Problème de la cheminée en chute
				\item Cartes de contrôle de Levey-Jennings
				\item Plans d'expérience de mélange en réseau avec variables de processus
				\item Taux de panne moyen (fiabilité)
				\item Modèle de fiabilité de chaîne de Markov
				\item Conception de tests de fiabilité (temps de test du khi-carré, taille de l'échantillon binomial, taille de l'échantillon bêta-binomial)
				\item Linéarisation de la distribution de Weibull
				\item Pendule inversé
				\item Test d'autocorrélation de Durbin-Watson
				\item Méthode de Fisher pour $p$-valeurs multiples
				\item Loupe
				\item Cartes de contrôle de Laney
				\item Classification des coniques par le déterminant	
				\item Classification des équations aux dérivées partielles
			\end{itemize}
		\item \textbf{Février 2017}
			\begin{itemize}[noitemsep]
				\item Bases de la loi Normale repliée (fonction de densité et cumulée)
				\item Fonction de densité et cumulée de la loi demi-Normal et variance, espérance et médiane correspondante
				\item Séries téléscopiques et de Gandi
				\item Somme de Césaro
				\item Différenciation implicite
				\item Dérivation composée bivariée
			\end{itemize}
		\item \textbf{Mars 2017}
			\begin{itemize}[noitemsep]
				\item Méthode d'intégration de Laplace
				\item Pseudo Erreur Standard (PSE) de Lenth pour les erreur margianles des plans factoriels non-répliqués
				\item Erreurs marginales de Pareto pour les plans d'expériences factoriels avec réplications
				\item Désirabilité des plans d'expérience
			\end{itemize}
		\item \textbf{Avril 2017}
			\begin{itemize}[noitemsep]
				\item Métrique de Friedmann–Lemaître–Robertson–Walker
				\item Inégalité de Jensen
			\end{itemize}
		\item \textbf{Mai 2017}
			\begin{itemize}[noitemsep]
				\item Introduction aux ondes gravitationnelles en champs faibles
				\item Une approche mathématique du "diviser pour mieux régner" en gouvernance
				\item Trois nouveaux gags dans la section d'humour
			\end{itemize}
		\item \textbf{Juillet 2017} (v3.7 $\rightarrow$ v3.8)
			\begin{itemize}[noitemsep]
				\item Photos d'un arc-en-ciel quasi-vertical et photo d'un appereil fonctionnement sur la base de la résonance magnétique nucléaire
				\item Détalis sur la méthode de rétro-propagation pour les réseaux de neurones
				\item Nouveaux détails mathématiques sur l'effet de levier en régression linéaire simple
				\item Applications numériques de quelques tests expérimentaux de la Relativité Générale
				\item Détermination de la géodésique de la sphère (comme exemple dans le section de Mécanique Analytique)
				\item Méthode de classification ZeroR
				\item Méthode des $K$ plus proches voisins
				\item Définition d'une fonction concave/convexe (pour la démonstration de l'inégalité de Jensen)
				\item Démonstration de l'orthogonalité des polynômes de Hermite
				\item Démonstration du modèle d'évaluation des options de Bachelier
				\item Démonstration de l'égalité de Cox-Ross-Ingersoll pour les Forward/Future
				\item Démonstration du tenseur énergie-impulsion pour un fluide non-relativiste
				\item Définition de la distance orthodromique
				\item Démonstration de l'aire (surface) s'une section d'ellipse
				\item Démonstration de l'orbite stable profonde de Schwarzschild
				\item Expérience Hafele–Keating avec traitement via le formalisme de la Relativité Générale
				\item Introduction des hyperparamètres de Machine Learning (apprentissage machine)
				\item Lissage par densité de noyaux (Kernel smoothing)
				\item Risque de défaut de crédit
				\item Ajout de nombreuses dates dans la section de chronologie
			\end{itemize}
		\item \textbf{Novembre 2017} (v3.8 $\rightarrow$ v3.9)
			\begin{itemize}[noitemsep]
				\item Création des références croisées dans le texte (mais à améliorer!)
				\item Ajout d'un nouveau point au règles de publiction scientifiques (citer les études équivalents, si existantes, pour méta-analyses ultérieures)
				\item Démonstration de la dérivée de $f(g)^{g(x)}$
				\item Arbre de décision OneR (One Rule) pour la Data Science avec matrix de confusion y relative
				\item Règles d'association pour le machine learning (apprentissage machine)
				\item Kurtosis (coefficient d'aplatissement) et Skewness (coefficient d'asymétrie)
			\end{itemize}
		\item \textbf{Janvier 2018} (v3.9 $\rightarrow$ v3.10)
			\begin{itemize}[noitemsep]
				\item Équation de continuité de Dirac
				\item Modèle CCR (Charnes, Cooper et Rhodes) du modèle Data Envelopment Analysis (DEA)
				\item Divergence de Kullback-Leibler
				\item Convolution discrète et linéaire
				\item Intégrale de Dirichlet
				\item Tests de permutation
				\item Ajout de références croisées supplémentaires
			\end{itemize}
			\item \textbf{June 2018} (v3.10 $\rightarrow$ v3.11)
			\begin{itemize}[noitemsep]
				\item Il y avait une erreur de frappe concernant le signe de l'angle retourné par la fonction $\mathrm{atan()}$ ($-$ au lieu d'un $+$)
				\item Ajout de quelques tables de Taguchi supplémentaires
				\item Démonstration de la loi du khi-deux non-central à un degré de liberté pour l'étude des intervalles de tolérance
				\item Ajout de détails sur la régression forcée à l'origine (particulièrement sur le problème du calcul du $R^2$ dans ce cas là!)
				\item Ajout de nouvelles images (sur la tromperie des données, un gag sur le chat de Schrödinger, comparaison des tailles des planètes, une illustration de la différence en le bagging et boosting en apprentissage machine)
				\item Ajout dans la section d'Astronomie de tables de données intéressantes sur les planètes du système solaire (incluant Pluton!)
				\item Ajout dans la section d'Astrophysique d'une table de données intéressante sur quelques étoiles fameuses
				\item Modèle de survie de Cox (modèle à hasard proportionnel de Cox)
				\item Nous avons rendu plus explicite le lien entre la méthode de Gauss-Newton et la méthode des gradients descendants
				\item Démonstration des intervalles de confiance pour l'estimateur de Kaplan-Meier et le modèle de hasard proportionnel de Cox (ration de hasard)
		\end{itemize}
		\item \textbf{June 2019} (v3.11 $\rightarrow$ v3.12)
			\begin{itemize}[noitemsep]
				\item Améliorations et corrections relativement au décalage vers le rouge
				\item Ajout de petites biographies sur Yates, Friedmann, Lemaître et Cramer
				\item Améliorations sur la partie concernant les modèles d'Univers de FRLW
				\item Ajout d'une dizaine d'harmoniques sphériques supplémentaires dans la section de Chimie Quantique (grâce a Wikipédia!)
				\item Brève introduction à l'analyse $XYZ$
				\item Introduction au concept d'avantage mécanique idéal et d'avantage mécanique actuel
		\end{itemize}
	\end{itemize}

\chapter{Nomenclature}

This chapter contains a summary with simple description of all symbols used in this book.

	\begin{table}[H]
	\centering
	\begin{tabular}{*8l}
	$\alpha$ \verb?alpha? &$\theta$ \verb?theta? & o o &$\tau$ \verb?tau? \\
	$\beta$ \verb?beta? &$\vartheta$ \verb?vartheta? &$\pi$ \verb?pi?         &$\upsilon$ \verb?upsilon? \\
	$\gamma$ \verb?upsilon? &$\xi$ \verb?xi?  &$\varpi$ \verb?varpi? &$\phi$ \verb?phi?  \\
	$\delta$ \verb?delta? &$\kappa$ \verb?kappa? &$\rho$ \verb?rho? &$\varphi$ \verb?varphi? \\
	$\varepsilon$ \verb?epsilon? &$\lambda$ \verb?lambda? &$\varrho$ \verb?varrho? &$\chi$ \verb?chi?  \\
	$\varepsilon$ \verb?varepsilon? &$\mu$ \verb?mu? &$\sigma$ \verb?sigma? &$\psi$ \verb?psi? \\
	$\zeta$ \verb?zeta? &$\nu$ \verb?nu? &$\varsigma$ \verb?varsigma? &$\omega$ \verb?omega? \\
	$\eta$ \verb?eta?\\
    \\
	$\Gamma$ \verb?Gamma? &$\Lambda$ \verb?Lambda? &$\Sigma$ \verb?Sigma? &$\Psi$ \verb?Psi? \\
	$\Delta$ \verb?Delta? &$\Xi$ \verb?Xi? &$\Upsilon$ \verb?Upsilon? &$\Omega$ \verb?Omega?\\
	$\Theta$ \verb?Theta? &$\Pi$ \verb?Pi? &$\Phi$ \verb?Phi?
	\end{tabular}
	\caption{Greek letters}\label{greek}
	\end{table}

	and the mathematical operators and objects used in the book:
	\begin{itemize}[label={},leftmargin=0.5cm]
		\setlength{\itemsep}{1pt}
  		\item $($ Open parenthesis
  		\item $)$ Close parenthesis
  		\item $[$ Open bracket
  		\item $]$ Close bracket
  		\item $\therefore$ Therefore
  		\item $\because$ Because
	 	\item $\varnothing$ Empty Set
	 	\item $=$ Equality symbol
	 	\item $:=$, $\triangleq$ By definition symbols
	 	\item $>$ greater than
	 	\item $<$ less than
	 	\item $\gg$ much greater than
	 	\item $\ll$ much smaller than
	 	\item $\leq$ less than or equal to
	 	\item $\geq$ bigger or equal to
	 	\item $\succ$ preferred to (for utility in econometry)
	 	\item $\prec$ not preferred to (for utility in econometry)
	 	\item $\succeq$ preferred or equal to (for utility in econometry)
	 	\item $\preceq$ not preferred or equal to (for utility in econometry)
	 	\item $\sim$ equivalent to (for utility in econometry)
	 	\item $\mathbb{N}$ Natural Numbers set (positive integers)
	 	\item $\mathbb{Z}$ Relative Numbers set (all integers)
	 	\item $\mathbb{Q}$ Rational Numbers set (ratio of relative numbers
	 	\item $\mathbb{R}$ Real Numbers set
	 	\item $\mathbb{C}$ Complex Numbers set
	 	\item $\Re$ Real part of a complex number
	 	\item $\Im$ Imaginary par of a complex number
	 	\item $\aleph$ Transfinite Cardinal symbol
	 	\item $\wedge$ AND operator, noted \& in computer science and corresponding to multiplication in maths
	 	\item $\equiv$ Identity symbol (left term is assumed to be equal to the right one and vice-versa)
	 	\item $\cong$ Approximately equal symbol
	 	\item $\propto$ Linear proportional symbol 
	 	\item $\in$ Symbol that means left term belongs to the right term
	 	\item $\not\in$ Symbol that means left term does not belong to the right term
	 	\item $\subset$ Symbol that means left term (that is a ) is a subset of the set on the right 
	 	\item $\not\subset$ Symbol that means left term (that is a set) is not a subset of the set on the right
	 	\item $\subseteq$ Symbol that means left term (that is a set) is a subset or a set equal to the set on the right
	 	\item $\not\subseteq$ Symbol that means left term (that is a set) is not a subset or even a set equal to the set on the right
	 	\item $\cup$ Symbol that means left term (that is a set) is merged (union) with right term that is also a set
	 	\item $\sqcup$ Symbol that means left term (that is an interval) is merged (union) with right term that is also an interval
	 	\item $\cap$ Symbol that means we take only intersection (equal) items of left and right terms that are sets
	 	\item $\displaystyle \bigcup_{i=1}^n$ Union of multiple sets
	 	\item $\displaystyle \bigcap_{i=1}^n$ Intersection of multiple sets
	 	\item $\mid$ Such that...
	 	\item $\forall$ For all...
	 	\item $\exists$ It exists...
	 	\item $+$ Addition symbol of two terms
	 	\item $\displaystyle \sum_{i=1}^n$ Summation of multiple terms
	 	\item $-$ Subtraction symbol of two terms
	 	\item $\times, \cdot$ Multiplication (product) symbol of two terms
	 	\item $\times$ If left and right terms are vectors, this is the cross product (vectorial product)
	 	\item $\circ$ dot product, also named "inner product" a scalar product
	 	\item $\otimes$ tensor product
	 	\item $\displaystyle \prod_{i=1}^n$ Multiplication (product) symbol of multiple terms
	 	\item $\displaystyle\int$ Riemann primitive
	 	\item $\displaystyle\int\limits_a^b$ Riemann integral in range $[a,b]$
	 	\item $\displaystyle\oint$ Closed non-oriented curvilinear integral (line integral)
	 	\item $\displaystyle\ointclockwise$ Clockwise path integral
	 	\item $\displaystyle\ointctrclockwise$ Counterclockwise path integral
	 	\item $\div, \backslash$, $:$ Symbols for division for two terms depending on school level
	 	\item $:$ When the left term is a matrix at the right one a vectros, this is "Frobenius (matrix) dot product" or "Frobenius inner product"
	 	\item $P$ Depending on the context this is a Probability, Cumulated Probability or Part of a set
	 	\item $\text{E}$ In statistics the expected mean
	 	\item $\text{V},\sigma^2$ In statistics the variance
	 	\item $\hat{x}$ In physics the amplitude of $x$, in statistics an estimator of $x$
	 	\item $C_k^n,\begin{pmatrix}n\\k\end{pmatrix}$ is the binomial coefficient define by $n!/(k!(n-k)!)$
	 	\item $\mathds{1}$ Unity matrix (diagonal with $1$, $0$ everywhere else)
	 	\item $\ln$ Natural logarithm (base $e$) of a number
	 	\item $\log$ Base $10$ (if not indicated) logarithm of a number
	 	\item $\earth$ Symbol in astronomy and astrophysics to refer to the Earth
	 	\item $\odot$ Symbol in astronomy and astrophysics to refer to the Sun
	\end{itemize}


	%\chapter{About the Redactor}
	%\input{Chapter_AboutRedactor.tex}


	\cleardoublepage
	\phantomsection
	\addcontentsline{toc}{chapter}{Liste des Figures}
	\listoffigures

	\newpage\null\thispagestyle{empty}\newpage %création d'une nouvelle page en forcant la disparition du numéro de page	
	\phantomsection
	\addcontentsline{toc}{chapter}{Liste des Tables}
	\listoftables
	
	\newpage\null\thispagestyle{empty}\newpage %création d'une nouvelle page en forcant la disparition du numéro de page	
	\phantomsection
	\addcontentsline{toc}{chapter}{Liste des Algorithmes}
	\listofalgorithms

	\newpage\null\thispagestyle{empty}\newpage %création d'une nouvelle page en forcant la disparition du numéro de page	
	\phantomsection
	\addcontentsline{toc}{chapter}{Bibliographie}
	\defbibnote{myprenote}{Puisque cet ouvrage se veut être une référence, nous avons choisi d'exclure quand cela était possible toute référence externe dans les textes. Nous avons donc simplement rassemblé ici une liste bibliographique finale en fonction d'un classement d'utilisation et sans prétendre à son exhaustivité.

	Les livres et documents listés ci-dessous sont ce que nous considérons comme les meilleures références (excluant ainsi les PDF gratuits) qui ont été consultées pour la préparation de ce livre et à qui nous devons de nombreux emprunts de haute qualité. Leur lecture peut être aussi profitable pour tous ceux qui souhaitent améliorer, approfondir et élargir leur spectre de connaissances.

	De nombreuses références, cependant, ne sont plus disponibles sur le marché et il est également nécessaire que le lecteur se souvienne que chaque entrée dans la bibliographie ci-dessous fait elle-même référence à des dizaines d'autres références, il est ainsi impossible d'avoir une liste exhaustive et objective (sans tomber sur un cercle vicieux...) de tous les livres de haute qualité existant dans les domaines couverts par \textit{Opera Magistris}.

	Et donc voici la bibliographie:}

	\nocite{*}
	\printbibliography[prenote=myprenote]
	
	+ un millier de publications scientifiques de mauvaise qualité et incomplets qui ne mértient pas d'être cités.
	
	\phantomsection
	\cleardoublepage
	\addcontentsline{toc}{chapter}{Index}
	\printindex  
	
	\chapter{Dons}
	La rédaction de ce livre a pris jusqu'à maintenant $15$ ans et d'énormes efforts et sacrifices, donc si vous trouvez ce livre utile, tous les dons ou cadeux sont grandement appréciés et à cette fin, vous pouvez utiliser les liens ci-dessous en fonction de vos préférences:
	
	\begin{center}
	\href{http://www.sciences.ch/htmlfr/donate.php}{\includegraphics[scale=0.14]{img/paypal.jpg}} $\qquad$ \href{https://www.patreon.com/sciences}{\includegraphics[scale=0.20]{img/patreon.jpg}} $\qquad$ \href{https://www.tipeee.com/elements-of-applied-mathematics}{\includegraphics[scale=0.023]{img/tipeee.jpg}} $\qquad$ \href{http://ko-fi.com/operamagistris}{\includegraphics[scale=0.34]{img/kofi.jpg}} $\quad$ \href{http://a.co/cqLIx5V}{\includegraphics[scale=0.55]{img/amazon_whish_list.jpg}}
	\end{center}
	\begin{center}
		{\large \faBitcoin} 4248d58b-90f0-493e-8114-8dc9f8e5b492
	\end{center}
	Les dons serviront principalement aux fins suivantes:
	\begin{itemize}
		\item Payer des professionnels pour la relecture
		\item Mandater des graphistes pour dessiner des illustrations vectorielles copyleft haute définition
		\item Mandater des professionnels de  \LaTeX{} pour améliorer la conception de tous les tableaux et en-têtes
		\item Mandater de photographes pour faire des photos d'installations techniques
		\item Dégager du temps pour écrire les $2'400$ pages restantes et faire des améliorations continues
		\item Pour garder le livre librement accessible dans le monde entier et le traduire dans d'autres langues
		\item Donner les sources \LaTeX{} (et donc aussi les illustrations) gratuitement
		\item Redistribuer $1\%$ à OpenStax, $5\%$ à Wikipedia, et $10\%$ au développeur TeXMaker et MiKTeX
	\end{itemize}
	Il existe également d'autres façons de soutenir ce livre pour s'assurer qu'il soit maintenu dans le futur! Passez le mot et soumettez des du contenu (avec preuves mathématiques détaillées) aidera aussi grandement!
	
	Merci pour votre aimable attention et votre soutien.
	
	\newpage\null\thispagestyle{empty}\newpage %Creation of a new empty page and force removal of page number
	\pagestyle{empty}
	\pagecolor{gray}
	\newgeometry{margin=2.5cm}
	{\Huge \textbf{Opera Magistris}}
	
	{\Large Éléments de Mathématiques Appliquées pour Ingénieurs}

	{\color{white}Le but de ce livre et de son PDF associé est de présenter aux personnes qui découvrent l'étude des Mathématiques Appliquées, les concepts de base et de le faire avec un certain niveau de rigueur (les démonstrations sont complètes ou au moins poussées au point où elles peuvent jugées l'être), de détails et de cohérence en ce qui concerne les conventions d'écriture et avec un maximum de pédagogie. Ce livre est également né du souhait de présenter quelques idées scientifiques et de comprendre comment ils affectent modestement notre style de vie, notre façon de penser, de travailler et leur impact sur notre écosystème et l'efficacité déraisonnable des mathématiques appliquées. En effet, les enjeux liés à la science prennent de plus en plus d'importance dans notre culture contemporaine et représentent un enjeu majeur en termes d'éthique, de citoyenneté et de développement.

	Ce livre n'est pas destiné à être un roman à lire du début à la fin. Il est conçu comme un ouvrage de référence (après correction de toutes les erreurs et textes complétés ...) qui, lorsque de simples questions se posent, permet de trouver rapidement des réponses en utilisant la technologie informatique gratuitement. Ce livre ne peut pas (et ne prétend pas) remplacer une éducation scolaire formelle par un enseignant et de nombreux exercices pratiques. Il peut cependant être vu comme une référence de formules (avec des preuves) ou un complément théorique relativement bon à la préparation de divers examens.

	Le point de vue qui est adopté dans ce livre est celui de l'ingénieur pragmatique, désireux d'étudier les mathématiques, la physique classique, l'économétrie, l'analyse numérique, la statistique, la mécanique relativiste, la physique quantique, les mathématiques sociales, l'informatique, la chimie, etc. sans perdre du temps dans un vocabulaire formel et extravagant et inutilisable dans l'industrie moderne. De ce point de vue, les concepts et méthodes présentés ne sont que quelques-uns des outils mathématiquo-physique typiques (minimum minimorum dans le domaine). L'expert spécialiste n'y trouvera probablement rien de nouveau et l'étudiant qui serait intéressé par une théorie particulière devrait savoir que chaque sujet est malheureusement beaucoup plus vaste que tout ce qui a été discuté ici jusqu'à présent... L'ambition n'est pas celle que nous pouvons avoir pour les étudiants d'un cours de mathématiques pour lequel l'acquisition d'outils est importante. Cela rend donc possible un style un peu formel où l'on peut donner des preuves moins complètes et se focaliser sur la compréhension intuitive des sujets présentés, afin de les visualiser et de les maîtriser.

	Nous dédions ce livre à tous ceux pour qui chaque réponse est une question!


\begin{flushright}
\includegraphics{img/ISBN.eps} 
\end{flushright}

\end{document}

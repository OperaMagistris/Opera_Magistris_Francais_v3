En arrivant à la trois-millième page A4 de l'écriture de ce livre et à l'occasion de la 3ème édition, il nous a semblé approprié d'essayer de donner une chronologie approximative de la plupart des sujets cités dans ce livre \footnote{Même si pour certaines dates il n'est pas toujours possible de vérifier s'il s'agit de légendes ou de faits réels...}. Cela peut donner une meilleure perspective sur les outils utilisés et aussi pour rendre hommage à nos illustres prédécesseurs à qui nous devons notre qualité de vie, notre longévité, notre maîtrise de l'environnement (pas nécessairement son respect...) et de sa compréhension.

Si des dates importantes manquent (mais seulement sur des sujets proches de ceux présentés dans les différentes sections de ce livre!), ou que vous identifiez des erreurs, n'hésitez pas à nous le faire savoir, c'est une première ébauche, et donc la chronologie ne peut qu'être améliorée.

Pour plus d'informations, le lecteur peut se référer à \href{http://www.wikipedia.com}{{\color{blue} Wikipédia}}, qui a atteint le plus haut niveau dans le nombre de dates historiques disponibles à ce jour sur Internet (du moins à notre connaissance) et ce avec une qualité respectable (vérification des sources!).

\begin{center}
\textit{C'est l'histoire de la façon dont l'histoire a fait la science et comment la science est entrée dans l'histoire, et comment les idées qui ont émergé ont fait le Monde moderne.}
\end{center}

\includegraphics[width=\textwidth]{img/raphael_school_of_athens.jpg}

\textbf{+2016}\footnote{Les années sont données sur la base du calendrier Grégorien.}\\
La première observation d'ondes gravitationnelles a été faite le 14 septembre 2015 et a été annoncée par les collaborations LIGO et Virgo le 11 février 2016.

\textbf{+2013}\\
Le 14 mars 2013, le CERN a confirmé que CMS et ATLAS ont comparé un certain nombre d'options pour la parité de spin d'une particule, et que cette dernière privilégiaient toutes l'absence de spin et même de parité. Ceci, couplé avec les interactions mesurées de la nouvelle particule avec d'autres particules, indique fortement qu'il s'agit d'un boson de Higgs. Cela fait également de cette particule la première particule scalaire élémentaire à être découverte dans la nature. En juillet 2017, le CERN a confirmé que toutes les mesures étaient toujours en accord avec les prédictions du modèle standard.

\textbf{+2006}\\
Le psychologue cognitif et informaticien Geoffrey Everest Hinton publie \textit{Un algorithme d'apprentissage rapide pour les réseaux d'apprentissages profondes} qui relance l'intérêt pour l'apprentissage en profondeur (deep learning).

\textbf{+2001}\\
Première édition du livre d'Opera Magistris mais sous le nom de "Sciences.ch". Un compendium sur les mathématiques appliquées qui a pour but de compilier un maximum de connaissances modernes sur les STIM (science, technologie, ingénierie et mathématiques) et au-delà au niveau Licence/Maîtrise avec un maximum de détails dans les développements mathématiques.

\textbf{+1995}\\
Michel Mayor et Didier Queloz observent la première planète extrasolaire autour d'une étoile de la séquence principale. La même année, le premier condensat gazeux de Bose-Einstein est produit par les physiciens Eric Cornell et Carl Wieman de l'Université du Colorado au laboratoire Boulder NIST-JILA dans un gaz d'atomes de rubidium refroidi à $170$ nanokelvins.

\textbf{+1994}\\
Les travaux du mathématicien Andrew Wiles (plus de 10 ans de recherche!) donnent une solution au dernier théorème de Fermat. La même année, le premier algorithme utilisant un ordinateur quantique pour la factorisation de nombre premiers est publié par Peter Shor.

\textbf{+1992}\\
Les machines vectorielles de support (SVM) sont inventées.

\textbf{+1986}\\
David E. Rumelhart, Geoffrey E. Hinton et Ronald J. Williams inventent la rétropropagation, une nouvelle procédure d'apprentissage pour les réseaux neurones. Les ingénieurs Bill Smith et Mikel J. Harry qui travaillent chez Motorola présentent la méthodologie Six Sigma ($ 6\sigma $), un ensemble de techniques et d'outils pour l'amélioration des processus scientifiques.

\textbf{+1983}\\
Les physiciens Carlo Rubbia, Simon van der Meer et l'équipe UA-1 du CERN  découvrent les bosons W et Z qui confirment l'unification des forces nucléaires faibles et électromagnétiques.

\textbf{+1982}\\
L'astrophysicien Werner Becker découvre le premier pulsar milliseconde. La même année, l'équipe du physicien Alain Aspect observe la violation des inégalités de Bell.

\textbf{+1978}\\
Les mathématiciens et cryptologues Ronald Rivest, Adi Shamir et Leonard Adelman proposent une procédure de cryptage à clé publique appelée "RSA" basée sur la difficulté de la factorisation en nombres nombres.

\textbf{+1976}\\
Développement d'une méthode par les physiciens Peter Mansfield et Andrew Maudsley pour rendre possible des scanners à résonance magnétique nucléaire (RMN).

\textbf{+1975}\\
L'informaticien John Holl invente les algorithmes génétiques.

\textbf{+1973}\\
Le mathématicien Fischer Black et l'économiste Myron S. Scholes publient un modèle de valorisation des actifs financiers.

\textbf{+1969}\\
Dans son \textit{Cartes de contrôle pour les mesures avec des tailles d'échantillons variables}, l'ingénieur qualité Irving Wingate Burr introduit les fameuses constantes non biaisées $c_4$ et $d_2$ qui portent son nom.

\textbf{+1968}\\
Les modèles de Markov (MMC) cachés sont inventés.

\textbf{+1967}\\
L'astrophysicienne Jocelyn Bell Burnell et l'astronome Antony Hewish découvrent le premier pulsar.

\textbf{+1965}\\
Détection par les astrophysiciens Arno Penzias et Robert Wilson fond diffus cosmologique due au rayonnement micro-onde de fond de ciel prédit par la théorie du physicien Robert Dicke. Développement de l'algorithme de Transformée de Fourier Rapide par le mathématicien James W. Cooley et le statisticien John W. Tukey qui a de nombreuses applications en sciences. Le physicien John Stewart Bell découvre les inégalités de Bell.

\textbf{+1964}\\
L'astrophysicien Irwin Shapiro prédit un retard gravitationnel du voyage de rayonnement comme un test de la Relativité Générale. La même année, le physicien John Stewart Bell montre que toutes les théories de variables cachées locales doivent satisfaire l'inégalité de Bell.

\textbf{+1963}\\
Le mathématicien et météorologue Edward Lorenz touve ce qui est probablement le premier attracteur étrange et ouvre la voie à la théorie du chaos.

\textbf{+1962}\\
Le mathématicien Benoît Mandelbrot découvre les fractales par hasard dans l'analyse des signaux situés aux Bell Laboratories aux Etats-Unis d'Amérique où il utilisera sans cesse les ordinateurs pour répéter des motifs graphiques et dont le principe est à la base de la théorie des fractales. La même année, l'économiste William Forsyth Sharpe publie le CAPM (Capital Asset Pricing Model).

\textbf{+1960}\\
Le physicien Abdus Salam postule l'existence des bosons W et Z pour expliquer la désintégration bêta et l'émergence d'un nouveau boson Z, qui n'avait jamais été vu auparavant. La même année, les pysiciens Ali Javan et Theodore Maiman inventen chacun un type particulier de LASER. Les mathématiciens et ingénieurs Irving Reed et Gustave Solomon présentent le code correcteur d'erreur de Reed-Solomon.

\textbf{+1959}\\
Les physiciens Yakir Aharonov et David Bohm prédisent l'effet Aharonov-Bohm (particule tournant autour d'un champ magnétique mais dans une région où le champ magnétique est nul est sensible au champ à travers le potentiel vectoriel) et l'année suivante le physicien Robert G. Chambers confirme l'effet expérimentalement.

\textbf{+1958}\\
Le perceptron est développé à l'Université Cornell par le psychologue  Frank Rosenblatt. C'est le premier programme capable d'apprendre par essais et erreurs.

\textbf{+1957}\\
Les physiciens John Bardeen, Leon Neil Cooper et John Robert Schrieffer proposent et la théorie de la supraconductivité.

\textbf{+1956}\\
Les physiciens Clyde Cowan et Fred Reines observent le neutrino hypothétisé il y a 25 ans par le physicien Wolfgang Pauli. La même année, le linguiste, philosophe, cognitiviste, historien, critique social et activiste politique Avram Noam Chomsky publie trois modèles pour la description du langage où il introduit la classification des grammaires formelles appelée aujourd'hui la "hiérarchie Chomsky", qui contient la classe des grammaires hors contexte, jouant un rôle important en informatique pour créer des langages de programmation. Dans \textit{The Logic Theorist} par l'informaticien cognitif Allen Newell et le politologue, économiste, sociologue, psychologue et informaticien Herbert Simon, le premier programme informatique d'intelligence artificielle est proposé, qui a produit des preuves dans le système \textit{Principia Mathematica} de Bertrand Russell et Alfred North Whitehead. Pour l'un des théorèmes, le programme produit une preuve plus simple que celle présentée dans les \textit{Principia}!

\textbf{+1955}\\
Les physiciens Owen Chamberlain, Emilio Gino Segrè, Clyde Wiegand et Thomas Ypsilantis découvrent l'antiproton.

\textbf{+1954}\\
L'économiste Harry Markowitz publie sa thèse sur le modèle de diversification efficace des portefeuilles d'actifs financiers. La même année, les physiciens John Bell et le duo Wolfgang Pauli et Gerhart Lüders développent la théorie CPT analysant la symétrie des lois physiques pour les transformations impliquant simultanément la charge, la parité et le temps. Le physicien Charles Hard Townes développe le MASER. Le biologiste Milner Baily Schaefer publie son modèle de population d'équilibre.

\textbf{+1953}\\
Les Méthodes de Monte-Carlo par chaînes de Markov (MMCM) sont inventées. L'inférence bayésienne devient traitable sur de vrais problèmes.


\textbf{+1952}\\
Le mathématicien George Bernard Dantzig développe l'algorithme du simplex pour la recherche opérationnelle.

\textbf{+1951}\\
Le théorème CPT apparaît pour la première fois implicitement dans le travail du physicien Julian Schwinger pour prouver la corrélation entre spin et statistiques.

\textbf{+1950}\\
Les physiciens Johannes Hans Daniel Jensen et Maria Goeppert-Mayer développent le modèle de goutte liquide du noyau nucléaire. La même année, l'économiste et mathématicien John Forbes Nash développe le concept de jeux non-coopératifs et généralise la notion de minimax pour les jeux à somme nulle; l'ingénieur David Huffman trouve l'algorithme utilisé pour compresser tout type de symboles de séries. Dans ses \textit{codes de détection d'erreur et de correction d'erreur}, le mathématicien Richard Wesley Hamming présente la famille de codes Hamming, codes linéaires utilisés pour détecter et corriger les erreurs de transmission. Une recherche basée sur le test en double aveugle est publiée pour la première fois, par Greiner et al.

\textbf{+1949}\\
Le mathématicien et ingénieur électricien Claude Shannon publie un article contenant la théorie de l'information qui deviendra le fondement d'un certain nombre de théories physiques, de statistiques et de méthodes numériques. Le physicien Richeard Feynman propose l'interprétation du positron comme un électron remontant dans le temps dans son article \textit{The Theory of Positrons}.

\textbf{+1948}\\
La physicienne Maria Göpper-Meyer développe avec succès un modèle théorique pour la structure du noyau atomique et l'ingénieur textile et statisticien Genichi Taguchi développe les plans d'expérience(DOE) qui portent son nom (plans de Taguchi). Le physicien Richard Feynman présente les diagrammes qui portent son nom ainsi que la formulation intégrale de chemin  en physique quantique. Le physicien Polykarp Kusch mesure le moment magnétique anormal de l'électron (déviation de la prédiction théorique de la théorie de Dirac) conduisant ainsi à reconsidérer et à innover en électrodynamique quantique.

\textbf{+1947}\\
Les physiciens Cecil Powell, Cesare Mansueto Giulio Latte et Giuseppe Occhialini découvrent le pion dans l'étude des rayons cosmiques. La même année, les physiciens John Bardeen, Walter H. Brattain et William Schockley inventent des transistors semi-conducteurs dans les laboratoires de la compagnie de téléphone Bell aux Etats-Unis d'Amériques qui vont provoquer la révolution informatique. Mesure du déplacement de Lamb par Willis Eugene Lamb (décalage du spectre d'énergie non-prédit par la théorie de Dirac mais expliquée dans le cadre de l'électrodynamique quantique).

\textbf{+1946}\\
Le physicien et chimiste Willard Frank Libby développe et découvre la possibilité de datation au carbone 14. La même année, les physiciens Walter Houser Brattain, John Bardeen et William Bradford Shockley découvrent l'effet des transistors.

\textbf{+1945}\\
Le Trinity Test, la première détonation réussie d'une arme nucléaire par le physicien Robert Oppenheimer et son équipe au Nouveau-Mexique.

\textbf{+1944}\\
Le mathématicien John von Neumann développe les bases de la théorie mathématique des jeux.

\textbf{+1943}\\
Le physicien Tomonaga Sin-Itiro publie un article posant la base de de l'électrodynamique quantique.

\textbf{+1942}\\
Le physicien Enrico Fermi et son équipe ont mené la première réaction en chaîne contrôlée dans le but de construire la première bombe atomique.

\textbf{+1941}\\
Le physicien Ernst Stueckelberg interprète les positrons comme des électrons à énergie positive remontant dans le temps. Le physicien Lev Davidovich Landau publie une théorie de la superfluidité.

\textbf{+1940}\\
Le physicien Edward Teller voit la possibilité d'utiliser l'énorme quantité de chaleur générée par l'explosion d'une bombe à fission pour déclencher le processus de fusion nucléaire. C'est l'approche considérée comme la découverte de la fusion nucléaire. La même année, le physicien William Donald Kerst développa le premier bétatron. Le physicien John Wheeler, dans un appel téléphonique à Richard Feynman, émet l'hypothèse que tous les électrons et positons sont en fait des manifestations d'une seule entité qui va et vient dans le temps (le «postulat d'un électron»).

Alan Turing et son équipe développent la première machine électromécanique ("The Bombe") installée à Bletchley Park. Cette machine complexe se composait d'environ 100 tambours rotatifs, 16 kilomètres de fil et environ 1 million de connexions soudées. Par ordinateur à la fin du 20e siècle, nous entendons généralement quelque chose qui peut faire beaucoup de choses. Donc, dans ce sens, La Bombe n'était pas un ordinateur. Car il ne pouvait résoudre qu'un seul problème: casser les clés Enigma. Bletchley Park a construit le premier ordinateur comme nous l'appellerions. Il s'agissait d'une machine appelée "Colossus" et qui a été développée entre 1943-1945 par Tommy Flowers, assisté de Sidney Broadhurst et William Chandler.

\textbf{+1939}\\
Les chimistes Otto Hahn et Fritz Strassmann bombardent l'Uranium avec des neutrons et découvrent que du Baryum est produit par l'expérience (découverte de la fission nucléaire). La même année, les physiciens Lise Meitner et Otto Robert Frisch déterminent que la fission nucléaire s'est produite pendant l'expérience de Hahn-Strassman. Les physiciens Wolfgang Pauli, Markus Fierz et Frederik Jozef Belinfante prouvent que les propriétés de permutation de particules, bosons ou fermions identiques sont contrôlées par leur spin.

\textbf{+1938}\\
Le chimiste et physicien Isidor Isaac Rabi et ses collègues étudient les effets de la mise en place de faisceaux de molécules dans des champs magnétiques externes puissants, conduisant au développement de la résonance magnétique nucléaire (RMN). La même année, les physiciens Hans Bethe et Carl von Weizsäcker proposent une théorie nucléaire des étoiles et le mathématicien et ingénieur électricien Claude Shannon publie ce qui est probablement la thèse de maîtrise la plus célèbre du XXe siècle (\textit{Une analyse symbolique des relais et de la commutation circuits}), et prouvent qu'il est possible de simplifier la conception des circuits logiques en utilisant l'algèbre booléenne. Ce mémoire de maîtrise a joué un rôle important dans la conception d'ordinateurs électroniques. Le physicien Piotr Leonidovich Kapitza observe le phénomène de la superfluidité.

\textbf{+1937}\\
Les physiciens Seth Neddermeyer, Carl Anderson, Jabez Curry Street et E.C. Stevenson découvrent des muons dans les traces laissées par les rayons cosmiques dans une chambre à bulles. La même année, le mathématicien John von Neumann développa les méthodes de Monte Carlo pour différentes méthodes numériques et le physicien Niels Bohr développa le modèle de goutte liquide du noyau.

\textbf{+1936}\\
Les physiciens George Gamow et Edward Teller travaillent ensemble pour formuler la théorie des émissions radioactives bêta. La même année, dans son \textit{Sur les nombres calculables}, l'informaticien, mathématicien, logicien, cryptanalyste, philosophe et biologiste théorique Alan Mathison Turing analyse le concept de calculabilité en utilisant le concept de machine de Turing, l'un des fondements de l'informatique théorique. L'ingénieur, le statisticien, le professeur, l'auteur, le conférencier et le consultant en gestion Walter A. Shewhart publie ses travaux sur les cartes de contrôle CUSUM, UWMA et EWMA.

\textbf{+1935}\\
Le physicien Hideki Yukawa présente la théorie de l'interaction forte et prédit l'existence de mésons. La même année, l'astrophysicien et mathématicien Subrahmanyan Chandrasekhar rapporte les résultats de ses recherches sur l'effondrement des étoiles en naines blanches et au-delà de $1.44$ masses solaires en étoiles à neutrons. L'article des physiciensAlbert Einstein, Boris Podolsky et Nathan Rosen sur le paradoxe de l'EPR est publié dans Physical Review et remet en question la non-localité de l'interprétation de Copenhague.

\textbf{+1934}\\
Le physicien Pavel Cherenkov Alekseyevich étudie l'émission de lumière lorsque des particules relativistes traversent un milieu amorphe. La même année, le physicien Enrico Fermi a suggéré de bombarder des atomes d'Uranium avec des neutrons pour obtenir un élément avec 93 protons, formule la théorie de la désintégration bêta et le physicien Leó Szilárd réalise qu'une réaction en chaîne nucléaire est possible. Les physiciens Irène Joliot-Curie et Frédéric Joliot bombardent des atomes d'Aluminium avec des particules alpha et créent artificiellement du Phosphore-30 radioactif. La falsifiabilité comme critère d'évaluation de nouvelles hypothèses est popularisée par Karl Popper.

\textbf{+1933}\\
Le mathématicien Andrei Nikolaevich Kolmogorov a publié un livre contenant une base solide d'axiomes de probabilité. Le physicien Ernst August Friedrich Ruska réalise le premier microscope électronique par transmission en utilisant des électrons au lieu de photons.

\textbf{+1932}\\
Le physicien Carl David Anderson découvre le positron. La même année, le physicien Werner Heisenberg présente le modèle théorique du noyau nucléaire proton-neutron et l'utilise pour expliquer les isotopes. Le physicien James Chadwick découvrit le neutron et les physiciens John Cockcroft et Ernest Walton brisent le noyau nucléaire du Lithium et du Bore par bombardement de protons.

\textbf{+1931}\\
Le physicien Wolfgang Pauli avance l'hypothèse du neutrino pour expliquer la violation apparente du principe de conservation de l'énergie dans la désintégration bêta. La même année, le mathématicien et logicien Kurt Gödel monte qu'un système peut être à la fois cohérent et complet (théorème d'incomplétude) et que si le système est cohérent, la cohérence des axiomes ne peut être prouvée dans le système. Le physicien Ernest Lawrence invente le premier cyclotron et le physicien, ingénieur et statisticien Walter Andrew Shewhart publie son livre \textit{Contrôle économique de la qualité du produit manufacturé} où il présente les principales cartes de contrôle.

\textbf{+1930}\\
Le physicien Fritz London explique que les forces de Van der Waals sont dues à l'interaction des moments dipolaires des molécules. La même année, le physicien Paul Dirac présente sa théorie des électrons-trous et l'économiste John Maynard Keynes publie son \textit{Traité sur la monnaie}.

\textbf{+1929}\\
L'astronome Edwin Hubble en étudiant le redshift (décalage vers le rouge) émet l'hypothèse que l'Univers n'est pas statique. La même année, le physicien Robert Van de Graaff invente le premier accélérateur de particules, connu aujourd'hui sous le nom "d'accélérateur Van de Graaff".

\textbf{+1928}\\
Le physicien Paul Dirac établi son équation d'onde relativiste pour l'électron, qui généralise et améliore l'équation relativiste sans spin de Klein-Gordon. La même année, les physiciens Friedrich Hund et Robert S. Mulliken introduisent le concept d'orbitale moléculaire et le physicien et cosmologiste George Gamow développe le modèle théorique quantique de la décroissance alpha  par effet tunnel. Le physicien Félix Bloch étudie le problème d'une particule soumise à un potentiel périodique et développe la théorie des bandes qui deviendra une base fondamentale de la théorie des solides, en particulier pour comprendre leurs propriétés de transport ou leurs propriétés optiques.

\textbf{+1927}\\
Le physicien Werner Heisenberg établit le principe d'incertitude, par lequel la position et la quantité de mouvement d'une particule ne peuvent pas être connus simultanément avec précision, indirectement en développant une nouvelle base théorique pour la mécanique quantique. La même année, les physiciens Walter Heitler et Fritz London présentent la théorie quantique de la liaison chimique établie à partir de la molécule d'hydrogène et le physicien Max Born interprète la fonction d'onde de Schrödinger comme probabilités et avec l'aide du physicien Robert Oppenheimer présente l'approximation de Born-Oppenheimer. Les physiciens Clinton Joseph Davisson, Lester Germer et George Paget Thomson confirment la nature ondulatoire des électrons par diffraction. Le physicien Paul Ehrenfest prouve le fameux théorème de la physique quantique qui porte son nom. Le physicien Paul Dirac introduit la quantification du champ électromagnétique.

\textbf{+1926}\\
Le physicien Erwin Schrödinger établit son équation d'onde qui définit la mécanique quantique sous une forme analytique en développant les idées de De Broglie sur la théorie de la mécanique ondulatoire et prouve que les formulations d'onde et de matrice de la théorie quantique sont mathématiquement équivalentes. La même année, les physiciens Oskar Klein et Walter Gordon établissent l'équation de la mécanique quantique relativiste pour les particules sans spin et Paul Dirac définit les statistiques de Fermi-Dirac. Dans le domaine de la dynamique des populations, le physicien et mathématicien Vito Volterra puublie l'équation différentielle non linéaire modélisant l'équilibre prédateur-proie. Les plans d'expériences aléatoires sont popularisés et analysés par le statisticien britannique Ronald Fisher.

\textbf{+1925}\\
Le physicien Pierre Auger découvre l'effet Auger (deux ans après Lise Meitner) et la même année les physiciens George Uhlenbeck et Samuel Goudsmit postulent et révèlent l'existence du spin électronique. Aussi la même année, le physicien Wolfgang Pauli établi par nécessité le principe de l'exclusion quantique. Les physiciens Werner Heisenberg, Max Born et Pascual Jordan formulent la version matricielle de la mécanique  quantique.

\textbf{+1924}\\
Le physicien John Lennard-Jones propose une description semi-empirique des forces d'interaction interatomiques et la même année, les physiciens Satyendranath Bose et Albert Einstein définissent les statistiques de Bose-Einstein. Dans le domaine des statistiques, le statisticien Ronald Fisher définit les grands concepts modernes de la statistique.

\textbf{+1923}\\
L'astronome Edwin Hubble estime la distance entre la Terre et les galaxies spirales, montrant qu'elles sont loin de la Voie Lactée et la même année le physicien Louis de Broglie suggère la dualité onde-particule de la théorie quantique et de l'équivalence masse-énergie et la physicienne Lise Meitner découvre l'effet Auger. Le mathématicien Norbert Wiener introduit la théorie du mouvement brownien.

\textbf{+1922}\\
Le physicien Arthur Compton étudie la diffusion des photons X par les électrons et l'astrophysicien Alexander Friedmann développe des modèles d'univers non statiques.

\textbf{+1921}\\
Le physicien Alfred Landé définit le rapport gyromagnétique et introduit aussi des nombres quantiques semi-entiers. La même année, les physiciens Otto Stern et Walter Gerlach montrent expérimentalement que le moment intrinsèque de l'électron est quantifié. Le mathématicien et physicien Theodor Franz Eduard Kaluza prouve qu'une version en cinq dimensions des équations d'Einstein unifie la gravitation et l'électromagnétisme.

\textbf{+1920}\\
L'astronome Vesto Melvin Slipher met en évidence le phénomène du décalage vers le rouge dans le spectre des galaxies. La même année, le physicien Arnold Sommerfeld introduit un quatrième nombre quantique au modèle original de l'atome de Bohr. Le physicien Niels Bohr introduit le principe de la correspondance assurant la transition de la physique quantique $\mapsto$ à la physique classique lorsque $\hbar \rightarrow 0$. L'astronome et astrophysicien Ernst Julius Öpik confirme que la "nébuleuse d'Andromède" est située en dehors de la Voie Lactée, ce qui confirme que notre Univers est beaucoup plus vaste que nous le pensions car il n'est pas limité à la Voie Lactée.

\textbf{+1919}\\
Le physicien Ernest Rutherford a effectué la première désintégration artificielle d'un atome en bombardant l'Azote avec des particules alpha. La même année, la physicienne et mathématicienne Amali Emmy Noether développe son théorème sur les invariants différentiels dans le calcul des variations, l'un des théorèmes mathématiques les plus importants jamais prouvés pour guider le développement de la physique moderne.

\textbf{+1918}\\
L'astronome Harlow Shapley fait la première estimation précise de la taille de notre galaxie et de la position du Soleil. La même année, le physicien Hermann Weyl introduit la notion de jauge, première étape de ce qui deviendra la théorie de la jauge.

\textbf{+1917}\\
Le physicien Albert Einstein introduit l'idée de l'émission stimulée de rayonnement utilisée dans la base de fabrication de LASER. La même année, le physicien Arnold Sommerfeld introduit un troisième nombre quantique au modèle original de l'atome de Bohr.

\textbf{+1916}\\
Le physicien Albert Einstein développe sa théorie de la Relativité Générale et comment la matière joue sur l'espace-temps pour produire des effets gravitationnels. C'est la première théorie nommée "indépendante de fond". La même année, les physiciens Gilbert Lewis et Irving Langmuir présentent le modèle de coquille électronique pour expliquer les liaisons chimiques et le physicien Arnold Sommerfeld introduit la relativité dans son modèle de 1915 et cette correction relativiste explique les valeurs observées par les spectrographes haute résolution et donc la division spectrale des lignes appelées "structure fine" et il introduit en même temps un deuxième nombre quantique décrivant des orbites elliptiques. Le physicien Karl Schwarzschild trouve une solution mathématique aux équations d'Einstein, qu'il applique aux étoiles à neutrons et aux Trous Noirs.

\textbf{+1915}\\
Le physicien Arnold Sommerfeld affine le modèle atomique du physicien Niels Bohr en introduisant des orbites elliptiques pour expliquer les fines lignes spectroscopiques de structure de l'atome d'hydrogène. Ce nouveau modèle n'explique cependant pas la gamme des spectres observés de l'atome d'hydrogène. La même année, le physicien Albert Einstein calcule la trajectoire de Mercure avec la Relativité Générale. Il utilise sa théorie pour calculer l'avance de Mercure de la périhélie avec une grande précision et la déviations des rayons lumineux dans le champ gravitationnel du Soleil. La fin de cette même année il soumet l'article qui décrit les équations des champs de la gravitation qui seront à la base de la théorie de la Relativité Générale.

\textbf{+1914}\\
Le physicien Ernest Rutherford montre que les noyaux atomiques chargés positivement contiennent des protons. La même année, le physicien Albert Einstein et le mathématicien Marcel Grossmann publient un article sur le calcul tensoriel, et plus particulièrement sur le tenseur de Riemann-Christoffel et de Ricci (plus généralement sur l'analyse tensorielle et la géométrie différentielle) et le physicien Peter Debye développe un modèle de comportement de la capacité thermique des solides en fonction de la température. Le mathématicien Felix Hausdorff introduit les concepts de distance de Hausdorff et de dimension de Hausdorff.

\textbf{+1913}\\
Le physicien Niels Bohr présente le modèle quantique en couches circulaires de l'atome et la même année le physicien Robert Millikan mesure la charge électrique fondamentale. La même année, les physiciens William Henry Bragg et William Lawrence Bragg trouvent la condition de Bragg pour les forces de réflexion des rayons X et le physicien Henry Moseley montre que le nombre atomique est le vrai critère de discrimination des éléments chimiques fondamentaux. En mathématiques, le mathématicien Elie Cartan annonce sa découverte des spineurs. Le physicien Johannes Stark démontre que les champs électriques forts vont diviser la série de raies spectrales Balmer de l'hydrogène.

\textbf{+1912}\\
Le physicien Max von Laue propose l'utilisation de réseaux cristallins pour diffracter les rayons X et la même année, les physiciens Walter Friedrich et Paul Knipping diffractent les rayons X en utilisant du sulfure de Zinc. La même année, le physicien Ernest Rutherford propose l'utilisation de la radioactivité comme moyen de datation. Les chimistes Otto Sackur et le physicien Hugo Tetrode mettent au point une formule pour calculer l'entropie d'un gaz mono-atomique (première apparition de la constante de Planck en thermodynamique).

\textbf{+1911}\\
Le physicien Ernest Rutherford découvre le noyau atomique en bombardant une fine feuille d'or avec des particules alpha. Certaines particules rebondissent sur le noyau des atomes d'or. La même année, le physicien et chimiste Jean Perrin prouve l'existence d'atomes et de molécules et le physicien Heike Kammerlingh Onnes découvre la supraconductivité.

\textbf{+1911}\\
Le physicien Robert Andrews Millikan détermine la charge électrique portée par un seul électron avec sa fameuse expérience de goutte d'huile.

\textbf{+1909}\\
Les physiciens Hans Geiger et Ernest Marsden découvrent que les particules alpha peuvent être fortement déviées par de minces feuilles de métal et la même année, les physiciens Ernest Rutherford et Thomas Royds ont démontré que les particules alpha sont des atomes d'Hélium ionisés deux fois. Dans le domaine des mathématiques appliquées, le mathématicien Agner Krarup Erlang puble le premier article sur la théorie des files d'attente.

\textbf{+1908}\\
Le statisticien William Sealy Gosset publie un article proposant une nouvelle distribution statistique et un nouveau test statistique nommés respectivement "Loi de student" et "test $T$ de Student". La même année, le mathématicien Ernst Friedrich Ferdinand Zermelo propose une amélioration des axiomes de la théorie des ensembles. Dans son \textit{Proportions Mendéliennes dans une population mixte}, le mathématicien Godfrey Harold Hardy expose ce qui est maintenant connu comme le "principe de Hardy-Weinberg" en génétique qui établit comment les traits génétiques dominants et récessifs se propagent dans une grande population. Hardy est connu comme mathématicien fondamental, spécialiste de la théorie des nombres, mais a contribué de manière significative à l'étude de la génétique des populations par les résultats présentés dans cet article, trouvé indépendamment par l'obstétricien-gynécologue Wilhelm Weinberg.

\textbf{+1907}\\
Le physicien Albert Einstein déduit l'expression de la fameuse équivalence entre masse et énergie, et la même année il établit l'expression de la capacité calorifique des solides cristallins et calcule le redshift gravitationnel. Le mathématicien et physicien Hermann Minkowski unifie l'espace-temps dans une structure mathématique unifiée. Le mathématicien Guido Fubini prouve le théorème intégral multiple qui porte son nom.

\textbf{+1906}\\
Le physicien Walther Nernst présente une formulation de la troisième loi de la thermodynamique. La même année, le mathématicien Andreï Markov publie le premier ouvrage sur les chaînes d'événements qui portera plus tard son nom (chaînes de Markov) et qui ont occupé une place importante dans la physique quantique de son temps.

\textbf{+1905}\\
Le physicien Albert Einstein explique l'effet photoélectrique par l'existence de quantums et la même année il explique mathématiquement le mouvement brownien à la suite du mouvement aléatoire des molécules et publié sa recherche sur sa théorie de la relativité restreinte qui prouve l'équivalence de masse et d'énergie. Le physicien Paul Langevin publie sa théorie sur la susceptibilité des matériaux paramagnétiques.

\textbf{+1904}\\
Le physicien Antoon Lorentz découvre la contraction du temps dans le sens du mouvement du corps par rapport à la vitesse constante de la lumière et propose les équations de transformation des forces électromagnétiques. La même année, le physicien Hantaro Nagaoka propose un modèle théorique saturnien de l'atome où les électrons tournent autour d'un noyau positif massif comme les anneaux de Saturne.

\textbf{+1903}\\
La radioactivité est expliquée en termes de fission des atomes par le physicien Ernest Rutherford et par le radiochimiste Frederick Soddy.

\textbf{+1902}\\
Le physicien Philipp Lenard observe que l'effet photoélectrique ne dépend pas de la puissance du faisceau lumineux mais de sa fréquence. La même année, le chimiste Theodor Svedberg suggère que les fluctuations du bombardement moléculaire créent un mouvement brownien et le logicien Bertrand Russell propose son paradoxe «ultime» sapant la théorie naïve des ensembles. Le physicien James Jeans décrit le phénomène d'effondrement gravitationnel qui peut se produire par exemple dans un nuage de matériau gazeux basé sur une masse ou un rayon critique. Le mathématicien Henri Léon Lebesgue pose la base de la théorie de la mesure et introduit l'intégrale de Lebesgue.

\textbf{+1901}\\
Les mathématiciens Gregorio Ricci-Curbastro et son assistant Tullio Levi-Civita développent le calcul tensoriel.

\textbf{+1900}\\
Le physicien Max Planck suggère que la lumière peut être émise à fréquences discrètes en généralisant la loi du rayonnement du corps noir. La même année, le physicien Johannes Rydberg affine l'expression mathématique des longueurs d'onde des raies de l'hydrogène de Balmer et le physicien Paul Villard découvre les rayons gamma en étudiant la désintégration de l'Uranium. Dans le domaine des mathématiques appliquées, le mathématicien Louis Bachelier développe le mouvement modèle brownien appliqué à la théorie du jeu et de la spéculation qui sera le pilier des outils financiers quantitatifs du XXe siècle. La même année, le mathématicien Karl Pearson définit la distribution statistique du Khi-2 et explore les propriétés importantes de cette distribution pour l'inférence statistique. Le physicien Paul Drude adapte la théorie cinétique des gaz aux électrons dans les métaux et obtient un modèle qui porte encore son nom.

\textbf{+1899}\\
Le physicien Ernest Rutherford découvre que les radiations émises par les composés d'Uranium sont des particules alpha chargées positivement et des particules bêta chargées négativement. La même année, le mathématicien David Hilbert remplace les $5$ axiomes habituels de la géométrie euclidienne par $21$ axiomes pour éliminer les faiblesses de la géométrie euclidienne.

\textbf{+1898}\\
Le mathématicien David Hilbert donne une première approche des corps commutatifs. La même année, les physiciens Marie et Pierre Curie isolent et étudient le Radium et le Polonium et le physicien Wilhelm Wien Carl Werner identifie une nouvelle particule avec une charge positive à peu près égale à la masse d'hydrogène qu'il nommera le "proton". L'ingénieur Alfred-Marie Liénard calcule le champ électromagnétique produit par une charge ponctuelle en mouvement (expressions mathématiques qui ont été établies indépendamment, mais deux ans plus tard par le physicien Emil Wiechert).

\textbf{+1897}\\
Le physicien Joseph John Thomson mesure le rapport charge/masse de certaines particules négatives créées par des rayons cathodiques. Il mesure leur charge, et il en conclut que leur masse est environ $2'000$ fois inférieure à celle de l'hydrogène. Ces particules sont plus tard appelées "électrons", expression suggérée par le physicien George Johnstone Stoney. Les téléviseurs et autres écrans cathodiques sont des versions améliorées du dispositif de Thomson.

\textbf{+1896}\\
Le physicien Henri Becquerel découvre la radioactivité de l'uranium et la même année le physicien Pieter Zeeman étudie la décomposition des raies D du sodium lorsqu'il est chauffé dans un fort champ magnétique et il découvre que les raies spectrales d'une source lumineuse soumise à un champ magnétique a de nombreuses composants, chacun ayant une certaine polarisation. Pour expliquer ce phénomène, il faut ajouter un nombre quantique additionnel nommé "nombre quantique magnétique". La même année, le physicien Wilhelm Wien Carl Werner établit la loi qui porte son nom pour l'énergie émise par le corps noir.

\textbf{+1895}\\
Le physicien Wilhelm Röntgen découvre les rayons X et la même année, le physicien et inventeur Guglielmo Marconi réalise dans les Alpes suisses à Salvan la première transmission "longue distance" (pour l'époque...) sans fil de 1.5 kilomètre.

\textbf{+1893}\\
Le mathématicien, physicien, ingénieur et philosophe Henri Poincaré publie ses études sur le problème des trois corps et introduit l'étude qualitative des équations différentielles et de la théorie du chaos. La même année, le mathématicien Georg Cantor développe la théorie des ensembles transfinis et la proposition de l'ingénieur Nikola Tesla d'utiliser le courant alternatif au lieu du courant continu est adopté par le premier état américain. Dans son \textit{Uniplanar Algebra}, le mathématicien Irving Stringham utilise le symbole $\ln$ pour le logarithme naturel au lieu du $\log_e$ traditionnel.

\textbf{+1892}\\
Le physicien autodidacte Oliver Heaviside réduit les $8$ équations de l'électrodynamique de Maxwell à $4$ équations différentielles.

\textbf{+1891}\\
Dans son \textit{Arithmetices Principia, ova methodo exposita}, le mathématicien Giuseppe Peano introduit les axiomes pour construire l'ensemble des nombres naturels $\mathbb{N}$ et le symbole d'appartenance $\in$ et une première version des quantificateurs symboliques. Leur forme finale sera donnée par le mathématicien David Hilbert. Il fournit plus de $40'000$ définitions dans une langue qu'il veut universelle.

\textbf{+1890}\\
L'étude systématique des groupes se développe avec les mathématiciens Sophus Lie, Issai Schur et Élie Cartan. Ce dernier introduit la notion de groupe algébrique et de groupes continus.

\textbf{+1889}\\
Le mathématicien Giuseppe Peano postule $5$ propriétés d'entiers comme axiomes dans l'idée de faire avec des entiers ce qu'Euclide a fait pour la géométrie. Il définit aussi l'axiomatique de l'espace vectoriel dans $\mathbb{R}$ et introduit le concept d'application linéaire. Il introduit aussi la notation $\cup$ et $\cap$ pour l'union et l'intersection des ensembles.

\textbf{+1888}\\
L'anthropologue, explorateur, géographe, inventeur, météorologue, proto-généticien, psychométricien et statisticien ... Francis Galton définit le concept de coefficient de corrélation statistique. La même année, le mathématicien Richard Dedekind propose la définition d'un ensemble fini.

\textbf{+1887}\\
Les physiciens Albert Michelson et Edward Morley mesurent la vitesse de la lumière pour tester l'hypothèse de l'éther que leurs résultats expérimentaux rejettent et la même année le physicien Heinrich Hertz découvre l'effet photoélectrique et mène des expériences sur les ondes électromagnétiques (production et réception).

\textbf{+1886}\\
Le mathématicien et physicien Oliver Heaviside introduit les opérateurs différentiels en tant qu'entités algébriques qui vont plus tard amener aux transformées de Laplace.

\textbf{+1885}\\
Le chimiste Johan Balmer trouve l'expression mathématique qui donne la longueur d'onde des différentes raies du spectre de l'hydrogène.

\textbf{+1884}\\
Le physicien et chimiste Willard Gibbs définit la notation encore en usage au début du 21ème siècle pour les produits scalaires et vectoriels ainsi que les opérateurs différentiels vectoriels dans ses livres sur le calcul vectoriel. La même année, le physicien Ludwig Boltzmann tire la loi de  Stefan-Boltzmann du flux lumineux du corps noir uniquement avec des considérations thermodynamiques et le physicien John Henry Poynting introduit le vecteur qui porte encore aujourd'hui son nom. Le mathématicien Karl Hermann Amandus Schwarz prouve que la sphère est le solide avec la surface minimale pour un volume donné (ce résultat explique de nombreuses formes visibles dans l'Univers).

\textbf{+1882}\\
Le mathématicien Ferdinand von Lindemann pourve la transcendance de $\pi $. La même année, l'astronome, mathématicien, économiste et statisticien Simon Newcomb observe une précession excessive de $43''$ par siècle de l'orbite de Mercure. Là où la mécanique d'Isaac Newton explique correctement la période de précession des autres planètes, elle échoue à expliquer celle de Mercure.

\textbf{+1880}\\
Le mathématicien et physicien Oliver Heaviside introduit la fonction "escalier" qui porte toujours son nom aujourd'hui (fonction de Heaviside).

\textbf{+1879}\\
Le mathématicien et physicien Joseph Stefan publie la loi de Stefan qui stipule que la puissance transmise à travers toute la gamme spectrale est proportionnelle à la quatrième puissance de la température absolue d'une étoile à sa surface. Pour une même température de surface, une étoile est aussi plus brillante, plus elle est grande. La même année, le physicien Edwin Herbert Hall découvre qu'un courant électrique à travers un matériau immergé dans un champ magnétique génère une tension perpendiculaire à la direction initiale du courant électrique. Le philosophe, logicien et mathématicien Gottlob Frege publie son \textit{Begriffsschrift, eine der arithmetischen nachgebildete Formelsprache des reinen denkens} où il introduit la logique de prédicat axiomatique, les quantificateurs (pour tout $\forall$, il existe $\exists$) , la théorie de la variable quantifiée, le concept rigoureux de formule / fonction et de variable.

\textbf{+1878}\\
Le mathématicien et philosophe William Kingdon Clifford introduit l'opérateur de divergence.

\textbf{+1877}\\
Le physicien et chimiste Willard Gibbs définit pour les réactions chimiques deux quantités utiles, à savoir l'enthalpie qui représente la chaleur de réaction à pression constante et l'énergie libre qui détermine si une réaction peut se dérouler spontanément à température et pression constantes. Le physicien John William Strutt (3e Baron Rayleigh) introduit les fondements de la théorie moderne du son. 

\textbf{+1876}\\
Le mathématicien et philosophe William Kingdon Clifford suggère que le mouvement de la matière peut être dû à des changements dans la géométrie de l'espace.

\textbf{+1874}\\
Le physicien Lord Kelvin énonce formellement la deuxième loi de la thermodynamique. La même année, l'économiste mathématique Léon Walras publie ses \textit{Elements of Pure Economics}.

\textbf{+1873}\\
Le mathématicien Georg Cantor pose les bases de la théorie des ensembles et des cardinaux et montre que les nombres algébriques sont en fait dénombrables et définit rigoureusement les nombres réels $\mathbb{R}$, et introduit la fameuse méthode diagonale. La même année, le physicien James Clerk Maxwell montre que la lumière est un phénomène électromagnétique et réduit les équations de l'électrodynamique à $8$ équations au lieu de $20$ (en même temps il définit l'opérateur rotationnel) et le physicien Johannes van der Waals introduit idée qu'il existe de faibles forces d'attraction entre les molécules. Dans son \textit{Sur la fonction exponentielle} le mathématicien Charles Hermite prouve la transcendance de $e$ (deux preuves, une de $11$ pages et l'autre de $20$ pages).

\textbf{+1872}\\
Le mathématicien Karl Weierstrass présente à l'Académie royale des sciences de Berlin un exemple d'une fonction continue partout mais différentiable nulle part.

\textbf{+1871}\\
Le chimiste Dmitri Mendeleev examine son tableau périodique et prédit l'existence du Gallium, du Scandium et du Germanium. La même année, le physicien James Clerk Maxwell établit les relations thermodynamiques de Maxwell.

\textbf{+1870}\\
Le physicien Rudolph Clausius prouve le théorème du viriel (scalaire).

\textbf{+1869}\\
Le chimiste Dmitri Mendeleev propose le tableau périodique des éléments qui porte encore son nom aujourd'hui.

\textbf{+1867}\\
L'historien, journaliste, philosophe, économiste, sociologue Karl Marx publie \textit{Das Kapital}.

\textbf{+1866}\\
Le physicien James Clerk Maxwell élabore, indépendamment du physicien Ludwig Boltzmann, la théorie cinétique des gaz de Maxwell-Boltzmann. La même année, le moine et botaniste Gregor Johann Mendel formule les lois de l'hybridation statistique (expérience sur $29'000$ petits pois ...) et le mathématicien Léopold Kronecker utilise pour la première fois le symbole qui porte toujours son nom aujourd'hui.

\textbf{+1865}\\
Le physicien James Clerk Maxwell publie pour la première fois les équations de l'électrodynamique sous la forme de $20$ équations avec $20$ inconnues utilisant des quaternions.

\textbf{+1862}\\
Le physicien Gustav Kirchhoff développe le concept du corps noir qui peut absorber et émettre des radiations à toutes les fréquences et que l'énergie émise dépend seulement de la fréquence du rayonnement émis et de la température du corps noir lui-même.

\textbf{+1859}\\
Le physicien James Clerk Maxwell découvre la loi de la distribution des vitesses moléculaires. La même année l'astronome Urbain Le Verrier rapporte une anomalie dans le mouvement de Mercure non prévisible par la loi de Newton et le physicien Gustav Kirchhoff avec le chimiste Robert Wilhelm Bunsen développent la spectroscopie prismatique.

\textbf{+1858}\\
L'avocat et mathématicien Arthur Cayley fait émerger la notion d'espace vectoriel, la notion de matrice et expose l'utilité en utilisant la multiplication des matrices et des déterminants; il réécrit le système des équations linéaires sous forme matricielle. Ses travaux sont souvent considérés comme l'émergence de l'algèbre linéaire.

\textbf{+1855}\\
L'astronome et physicien Léon Foucault découvre que la force nécessaire à la rotation d'un disque de cuivre augmente quand on doit le tourner avec sa jante entre les pôles d'un aimant, le disque chauffant en même temps à cause des "courants de Foucault" induits dans le métal.

\textbf{+1854}\\
Le mathématicien George Boole publie son système de logique symbolique, maintenant connu sous le nom d'algèbre de Boole. La même année, le mathématicien Arthur Cayley montre que les quaternions peuvent être utilisés pour représenter les rotations dans l'espace à quatre dimensions et le mathématicien Georg Friedrich Bernhard Riemann donne une nouvelle définition de l'intégrale et pose les bases de la géométrie différentielle. Le mathématicien Charles Hermite définit le concept des matrices orthogonales et prouve que leurs valeurs propres sont des nombres réels.

\textbf{+1852}\\
Les physiciens James Joule et William Thomson Kelvin montre que le gaz en expansion se refroidit rapidement.

\textbf{+1851}\\
L'astronome et physicien Léon Foucault fait une preuve spectaculaire de la rotation de la Terre en suspendant un pendule avec un long câble attaché au dôme du Panthéon à Paris. La même année, le mathématicien Georg Friedrich Bernhard Riemann publie le premier travail sur les fonctions avec une variable complexe. Dans son \textit{Paradoxien des Unendlichen}, le mathématicien, logicien, philosophe et théologien Bernardus Placidus Johann Nepomuk Bolzano discute des questions liées à la manipulation des infinis en mathématiques.

\textbf{+1850}\\
Les mathématiciens Arthur Cayley et James Joseph Sylvester introduisent le terme de matrices et la même année, le mathématicien Richard Dedekind introduit le terme de corps. Le physicien Rudolf Clausius développe la théorie mécanique de la chaleur et formule le second principe de la thermodynamique. Le phyisicien George Stokes prouve le théorème de Stokes.

\textbf{+1849}\\
Le mathématicien et astronome Edouard Roche trouve le rayon limite de destruction de création par forces de marées et d'un corps qui tient par sa seule gravité utilise ce résultat pour expliquer pourquoi les anneaux de Saturne ne se condensent en un satellite.

\textbf{+1848}\\
Le physicien William Thomson Kelvin découvre le point absolu $0$ de température et définit sa propre unité de mesure. La même année le physicien et astronome Hyppolite Fizeau transpose les résultats de Christian Doppler à la lumière qui, comme le son, possède un caractère ondulatoire (effet Doppler-Fizeau) et met en évidence le décalage vers le rouge et vers le bleu.

\textbf{+1847}\\
Le physicien James Joule trouve expérimentalement l'équivalent mécanique de la chaleur et la même année, le physiologiste et physicien Hermann Helmholtz formalise le concept de conservation de l'énergie.

\textbf{+1845}\\
Le physicien Gustav Kirchoff définit le concept de potentiel électrique et énonce les lois de réseaux qui portent son nom (loi des noeuds, loi des mailles). La même année, le physicien George Stokes publie ce qui sera les fondements des équations de Navier-Stokes en mécanique des fluides et le physicien Michael Faraday découvre que la propagation de la lumière dans un matériau peut être influencé par des champs magnétiques externes.

\textbf{+1844}\\
Le mathématicien Joseph Liouville montre l'existence d'une infinité de nombres transcendants.

\textbf{+1843}\\
Le physicien, mathématicien et astronome William Rowan Hamilton définit des espaces de vecteurs complexes (quaternions). La notion d'espace vectoriel sera clairement définie par le mathématicien et astronome August Ferdinand Möbius et par le mathématicien et linguiste Giuseppe Peano $40$ ans plus tard. La même année, le mathématicien Laurent Pierre Alphonse publie son mémoire sur ce qui deviendra plus tard les séries qui portent son nom en analyse complexe.

\textbf{+1842}\\
Le principe de la conservation de l'énergie est formulé par le physicien Julius Von Mayer qui a calculé la quantité de travail pouvant être obtenue par transformation de la chaleur en énergie, soit l'équivalent mécanique de la calorie. La même année, le physicien Christian Doppler découvre l'effet acoustique qui porte son nom (variation de la fréquence avec le mouvement relatif).

\textbf{+1841}\\
Le mathématicien Karl Weierstrass découvre mais ne publie pas ce que nous appelons aujourd'hui les séries de Laurent. La même année, le matématicien Carl Gustav Jacob Jacobi introduit les matrices jacobiennes et réintroduit la notation de la dérivée partielle proposée initialement par le mathématicien André-Marie Legendre.

\textbf{+1838}\\
L'astronome et mathématicien Friedrich Bessel mesure que la distance qui nous sépare de l'étoile 61 Cygni est de 96 billions de kilomètres.

\textbf{+1835}\\
Le mathématicien Carl Friedrich Gauss donne une construction rigoureuse des nombres complexes et le mathématicien Augustin Louis Cauchy établit une première théorie des déterminants. La même année, le mathématicien et ingénieur Gaspard Coriolis démontre que l'accélération d'un mobile dans un référentiel en rotation est soumise à une force complémentaire perpendiculaire au sens de déplacement du mobile dans ce référentiel.

\textbf{+1834}\\
L'ingénieur et physicien Émile Clapeyron présente une formulation de la seconde loi de la thermodynamique. La même année, le physicien Heinrich Lenz établit la loi d'induction électromagnétique.

\textbf{+1832}\\
Le physicien Michael Faraday établit la théorie fondamentale de l'électrolyse.

\textbf{+1831}\\
Le physicien Michael Faraday découvre l'induction électromagnétique, à savoir l'obtention d'un courant électrique à partir de la variation d'un champ magnétique (principe de la dynamo: le contraire de l'expérience d'Ørsted). La même année le mathématicien, astronome et physicien Carl Friedrich Gauss énonce deux des quatre équations de Maxwell.

\textbf{+1830}\\
Le mathématicien et inventeur anglais Charles Babbage est crédité pour avoir conçu théoriquement le premier ordinateur numérique automatique. Au milieu des années 1830, Babbage élabora des plans pour une machine analytique. Bien qu'elle n'ait jamais été achevée, la machine analytique aurait eu la plupart des éléments de base de l'ordinateur actuel. Ainsi Babbage ne peut être crédité pour la création du premier ordinateur fonctionnel.

\textbf{+1829}\\
Le mathématicien Évariste Galois présente la première ébauche de son travail sur les équations résolubles qui sera à l'origine de l'approche ensembliste de la résolution d'équation algébrique par radicaux. Le mathématicien Augustin Louis Cauchy prouve que les valeurs propres d'une matrice symétrique sont toutes réelles. Le mathématicien Johann Peter Gustav Lejeune Dirichlet étudie la convergence des séries de Fourier.

\textbf{+1828}\\
Le physicien George Green prouve le théorème de Green.

\textbf{+1827}\\
Le botaniste Robert Brown découvre le mouvement brownien des particules de pollen et de colorant dans l'eau ; la même année le physicien Georg Ohm établit la loi de la résistance électrique et le physicien et mathématicien André Ampère découvre les lois qui lient les forces magnétiques au courant électrique. Le physicien, mathématicien et astronome William Rowan Hamilton présente la théorie d'une unique fonction qui unifie la mécanique, optique et mathématiques et qui aida à établir la théorie ondulatoire de la lumière.

\textbf{+1826}\\
Le mathématicien Niels Henrik Abel prouve qu'il est impossible de résoudre l'équation quintique générale (polynômes d'ordre $5$) par radicaux. Dans son \textit{Sur une méthode d'expression par l'action des machines}, le mathématicien, philosophe, inventeur et ingénieur mécanique Charles Babbage décrit un langage symbolique qui l'aidera à concevoir sa machine analytique, la première machine mécanique universelle. La "machine de Babbage" est le premier ordinateur programmable complet (ayant les mêmes capacités de calcul qu'une machine de Turing ou un ordinateur moderne) qui a été conçu.

\textbf{+1825}\\
Le mathématicien Augustin-Louis Cauchy présente le théorème de Cauchy et l'intégration curviligne générale et introduit le théorème des résidus. Le scientifique et inventeur William Sturgeon invente le premier électroaimant.

\textbf{+1824}\\
Le mathématicien Augustin-Louis Cauchy découvre le polynôme caractéristique d'une matrice et prouve qu'il est invariant par transformation linéaire et calcule pour la première fois des valeurs propres et des vecteurs propres. La même année, le physicien et ingénieur Sadi Carnot analyse scientifiquement l'efficacité des machines à vapeur (cycle de Carnot), montrant que leurs performances sont limitées et définit également le second principe de la thermodynamique.

\textbf{+1823}\\
Le physicien et chimiste Michael Faraday présente une série de papiers sur la liquéfaction des gaz.  Le mathématicien Pierre Frédéric Sarrus introduit le symbole de la barre verticale pour l'intégrale: $\int_a^b f(x)\mathrm{d}x= F(x)|_a^b$.

\textbf{+1822}\\
Le mathématicien Jean-Victor Poncelet fonde la géométrie projective. La même année, le physicien et mathématicien Joseph Fourier présente formellement l'utilisation des dimensions (unités) pour des quantités physiques et introduit la notation $\int_a^bf(x)\mathrm{d}x$.

\textbf{+1821}\\
Le principe de la dynamo est décrit pour la première fois par le physicien et chimiste Michael Faraday. La même année le physicien John Herapath propose que la chaleur n'est en réalité que l'effet d'agitation et donc de mouvement de corps élémentaires.

\textbf{+1820}\\
Le physicien et chimiste Hans Christian Ørsted découvre et prouve les effets magnétiques du courant électrique. La même année, les physiciens Jean-Baptiste Biot et Félix Savart déterminent dans le domaine du magnétisme la fameuse loi qui porte leur nom.

\textbf{+1819}\\
Le physicien et chimiste Hans Christian Ørsted montre que le courant électrique dévie une aiguille aimantée, démontrant ainsi l'électromagnétisme et annonçant une révolution industrielle.

\textbf{+1818}\\
Le mathématicien, géomètre et physicien Simeon Poisson calcule le point lumineux de Poisson au centre de l'ombre d'un obstacle circulaire opaque.

\textbf{+1817}\\
En étudiant la polarisation de la lumière, le physicien Augustin Fresnel montre que cette dernière est un mouvement ondulatoire transversal et non longitudinal et montre aussi que la diffraction et l'interférence peuvent être expliquées si l'on considère la lumière comme une onde. La même année, l'astronome Friedrich Bessel publie des travaux faisant usage des fameuses fonctions qui portent son nom.

\textbf{+1816}\\
Le mathématicien Joseph Diaz Gergonne introduit le symbole marquant l'inclusion dans la théorie des ensembles.

\textbf{+1814}\\
Le physicien et opticien Joseph Von Fraunhofer étudie pour la première fois les raies d'absorption du spectre solaire et ce au moyen du spectroscope dont il fût l'inventeur. Le mathématicien, astronome et physicien Pierre-Simon Laplace fait l'hypothèse qu'une parfaite connaissance de l'état actuel de l'Univers permettrait de déterminer parfaitement tous ses états futurs.

\textbf{+1812}\\
Le mathématicien, astronome et physicien Pierre-Simon de Laplace publie un ouvrage majeur sur la théorie des probabilités (incluant aussi la méthode des moindres carrés) dont il est considéré comme l'un des fondateurs.

\textbf{+1811}\\
Le chimiste Amaedo Avogadro avance l'hypothèse selon laquelle des volumes égaux de gaz différents contiennent le même nombre de molécules, dans des conditions identiques de température et de pression.

\textbf{+1810}\\
Le mathématicien, astronome et physicien Carl Friedrich Gauss découvre les concepts de base de la géométrie non-euclidienne mais ne publiera jamais ses travaux à ce sujet. La même année, le physicien et mathématicien Joseph Fourier modélise l'évolution de la température au travers de séries trigonométriques.

\textbf{+1809}\\
Le mathématicien, astronome et physicien Carl Friedrich Gauss développe la méthode des moindres carrés indépendamment du mathématicien André-Marie Legendre . La même année, le mathématicien, astronome et physicien Pierre-Simon de Laplace démontre la forme générale du théorème central limite. L'ingénieur, physicien et mathématicien Etienne Malus publie la loi de Malus.

\textbf{+1808}\\
Le physicien et chimiste John Dalton propose ce qui est considéré comme la première théorie de l'atome. Le mathématicien Christian Kramp introduit dans son \textit{Éléments d'arithmétique universelle} la notation $n!$ pour la factorielle.

\textbf{+1806}\\
Le mathématicien Jean Robert Argand publie la première représentation plane des nombres complexes et utilise des mesures algébriques. Le mathématicien Johann Carl Friedrich Gauss développe l'idée d'ajouter des vecteurs sous une forme géométrique et introduit la notation $\overrightarrow{ab}$ pour un vecteur.

\textbf{+1805}\\
Le mathématicien André-Marie Legendre développe la méthode des moindres carrés.

\textbf{+1803}\\
Le physicien et chimiste John Dalton a l'idée originale de considérer que chaque élément chimique est constitué d'atomes différents. Une combinaison chimique s'explique alors par l'union de ces atomes en proportions fixes et les masses atomiques relatives devenaient calculables à partir de faits expérimentaux. La même année, l'économiste, journaliste et industriel Jean-Baptiste Say publie son \textit{Traité d'économie politique}.

\textbf{+1802}\\
Le physicien Thomas Young prouve la nature ondulatoire de la lumière par une expérience importante qui montre l'interférence des ondes. La même année, le chimiste et physicien Louis Joseph Gay-Lussac découvre la fameuse loi sur les gaz qui relie volume et température d'un gaz réel, loi qui porte son nom (loi de Gay-Lussac).

\textbf{+1801}\\
Le chimiste et physicien John Dalton, découvre la loi de la somme des pressions partielles des gaz qui porte encore aujourd'hui son nom.

\textbf{+1800}\\
Le chimiste William Nicholson et le chirurgien Anthony Carlisle utilisent l'électrolyse pour séparer l'eau en hydrogène et oxygène. La même année, l'astronome Willian Herschel découvre le rayonnement infrarouge et le physicien Allessandro Volta invente la première pile électrique.

\textbf{+1799}\\
Le mathématicien Gaspard Monge publie son ouvrage de géométrie descriptive. Il en est considéré comme l'inventeur. Premières preuves satisfaisantes mais incomplètes du théorème fondamental de l'algèbre par le mathématicien Johann Carl Friedrich Gauss.

\textbf{+1798}\\
Le mathématicien Carl Friedrich Gauss donne une démonstration rigoureuse du théorème de d'Alembert (théorème fondamental de l'algèbre). La même année, le physicien Benjamin Thompson a l'idée que la chaleur est une forme d'énergie et le physicien et chimiste Henry Cavendish mesure la valeur de la constante gravitationnelle. L'économiste Thomas Malthus énonçe sa loi de population (le modèle exponentiel).

\textbf{+1797}\\
Le mathématicien Caspar Wessel associe des vecteurs aux nombres complexes et étudie l'interprétation géométrique des opérations sur les nombres complexes.

\textbf{+1793}\\
L'assemblée nationale de la République Française instaure le système métrique.

\textbf{+1789}\\
Le physicien et chimiste Antoine Lavoisier énoncé le principe de conservation de la masse.

\textbf{+1788}\\
Dans son \textit{Méchanique Analitique}, le mathématicien et phyisicien Joseph-Louis Lagrange formule une nouvelle façon d'étudier la mécanique classique (par Isaac Newton pour le rappel) en utilisant le principe de moindre action. La même année, l'Académie des sciences approuve la création d'un système de mesure universel, le futur système métrique. Ce projet sera également approuvé par l'Assemblée nationale française en 1790, qui donnera la première définition du mètre.

\textbf{+1787}\\
Le physicien, chimiste et inventeur Jacques Alexandre César Charles détermine expérimentalement que le volume d'une masse fixe d'un gaz à pression constante est proportionnel à la température.

\textbf{+1786}\\
L'astronome William Herschel fait une description précise de notre galaxie.

\textbf{+1785}\\
Le physicien Charles-Augustin Coulomb prouve que les forces entre charges électriques et entre aimants s'exercent en raison inverse du carré de la distance.

\textbf{+1783}\\
L'ecclésiastique et philosophe naturaliste John Michell, dans un article des \textit{Philosophical Transactions} de la Royal Society de Londres, lu le 27 novembre 1783, proposa d'abord l'idée qu'il existait des Trous Noirs, qu'il appela des «étoiles sombres». Quelques années après que Michell eut inventé le concept des trous noirs, le mathématicien français Pierre-Simon Laplace suggéra essentiellement la même idée dans son livre de 1796, \textit{Exposition du Système du Monde}. 

\textbf{+1782}\\
Le mathématicien, physicien et astronome Pierre-Simon de Laplace introduit la "transformée de Laplace", une transformation qui permet la résolution d'équations différentielles en physique (ou facilite leur résolution).

\textbf{+1781}\\
Le chimiste et physicien Joseph Priestley crée de l'eau par combustion d'hydrogène et d'oxygène ce qui démontre que l'eau n'est pas un élément fondamental comme on le pensait depuis Aristote.

\textbf{+1778}\\
Les physiciens et chimistes Carl Scheele et Antoine Lavoisier découvrent que l'air est composé essentiellement d'azote et d'oxygène.

\textbf{+1777}\\
Le physicien et mathématicien Leonhard Euler introduit la lettre $\mathrm{i}$ pour la partie imaginaire des nombres complexes.

\textbf{+1776}\\
Le philosophe moraliste et pionnier de l'économie politique Adam Smith publie son \textit{Une enquête sur la nature et les causes de la richesse des nations}.

\textbf{+1774}\\
Le mathématicien, astronome et physicien Pierre-Simon de Laplace explicite l'intégrale d'Euler. La même année, le théologien, pasteur dissident, philosophe naturel, pédagogue et théoricien de la politique Joseph Priestley fit sa principale découverte, celle de l'oxygène.

\textbf{+1772}\\
Le mathématicien, mécanicien et astronome Joseph-Louis Lagrange étudie le problème des trois corps et découvre les points de libration appelés aujourd'hui "points de Lagrange"..

\textbf{+1770}\\
Dans son \textit{Mémoire sur les équations aux différences partielles}, le philosophe, mathématicien et premier politologue, Jean-Marc Antoine de Caritat, marquis de Condorcet, introduit le symbole des dérivées partielles $\partial$.

\textbf{+1769}\\
Dans son \textit{Institutiones calculi integralis}, le mathématicien Leonhard Euler étudie pour la première fois les doubles intégrales, les calcule par intégration successive et changements de variable. Ces méthodes seront généralisées aux intégrales triples par le mathématicien, ingénieur et astronome Joseph-Louis Lagrange, qui donne aussi la formule générale pour le changement de variables (déterminant du jacobien).

\textbf{+1766}\\
Le physicien et chimiste Henry Cavendish découvre et étudie l'hydrogène.

\textbf{+1764}\\
Les chaleurs latente et spécifique sont décrites pour la première fois par le physicien et chimiste Joseph Black. Il est également le premier à distinguer nettement température et quantité de mouvement. La même année, le physicien et mathématicien Leonhard Euler examine l'équation aux dérivées partielles pour la vibration d'un tambour circulaire et trouve l'une des fonctions de Bessel en tant que solution.

\textbf{+1763}\\
Un article posthume de mathématicien et pasteur Thomas Bayes met en évidence qu'il a découvert ce qui est appelée encore aujourd'hui le "théorème de Bayes".

\textbf{+1757}\\
Le physicien et mathématicien Leonhard Euler fonde l'hydrodynamique moderne.

\textbf{+1756}\\
Le mathématicien, ingénieur et astronome Joseph-Louis Lagrange développe la mécanique analytique basée sur son invention du calcul des variations indépendamment de Leonhard Euler.

\textbf{+1755}\\
Le physicien et mathématicien Leonhard Euler introduit la lettre grecque sigma majuscule ($\Sigma$) pour le symbole de la somme.

\textbf{+1753}\\
Dans son étude de 1749 sur les mouvements de la Terre, Leonhard Euler obtient des équations différentielles pour les éléments orbitaux et, en 1753, il a appliqué la méthode de variation des constantes à son étude des mouvements de la lune.

\textbf{+1750}\\
Le mathématicien Gabriel Cramer établit la règle de Cramer pour la résolution de systèmes linéaires.

\textbf{+1749}\\
L'astronome et physicien Jean le Rond D'Alembert développe le premier modèle de précession des équinoxes basé sur la théorie de la gravitation de Newton et donne une piste de solution pour le problème des trois corps.

\textbf{+1748}\\
Dans son \textit{Introductio in Analysin Infinitorum}, le mathématicien Leonhard Euler introduit le concept de fonction (défini comme toute composition d'expression algébrique et analytique), définit les concepts de fonctions paires et impaires, popularise l'utilisation des symboles $e$ et $\pi$, prouve l'identité d'Euler et définit la fonction $\Gamma$ généralisant la factorielle.

\textbf{+1746}\\
L'encyclopédiste Jean le Rond D'Alembert donne la première preuve (acceptable mais qui sera corrigée plus tard) du théorème fondamental de l'algèbre. L'année d'après (1747) il publie l'équation des cordes vibrantes qui a été le premier exemple de l'équation des ondes. Cela fait de D'Alembert, l'un des fondateurs de la physique mathématique.

\textbf{+1744}\\
Le philosophe, mathématicien, physicien, astronome et naturaliste Pierre Louis Moreau de Maupertuis énonce le principe de moindre action qui sera formalisé mathématiquement 22 ans plus tard par le mathématicien, mécanicien et astronome Joseph-Louis Lagrange. La même année, le physicien et mathématicien Leonhard Euler montre l'existence des nombres transcendants.

\textbf{+1742}\\
L'astronome Anders Celsius définit sa propre unité de mesure de la température.

\textbf{+1739}\\
Le physicien et mathématicien Leonhard Euler résout les équations différentielles linéaires homogènes à coefficients constants.

\textbf{+1738}\\
Le médecin, physicien et mathématicien Daniel Bernoulli publie un livre sur l'hydrodynamique introduisant la théorie cinétique des gaz et le fameux théorème de Bernoulli (balance de pression). La même année dans son \textit{Doctrine du hasard}, le mathématicien Abraham De Moivre introduit la distribution gaussienne comme un moyen d'approximation de la loi binomiale pour un grand nombre d'expériences et démontre une version partielle du théorème de la limite centrale.

\textbf{+1737}\\
Le physicien et mathématicien Leonhard Euler résout le problème de théorie des graphes relatif aux ponts de Königsberg. La résolution de ce problème est considérée comme le premier théorème de la théorie des graphes. Il établit par la même occasion la "formule d'Euler" liant le nombre de sommets, d'arêtes et de faces d'un polyèdre convexe, et donc d'un graphe planaire.

\textbf{+1736}\\
L'inventeur Jonathan Hulls dépose le premier brevet d'un bateau propulsé par une machine à vapeur.

\textbf{+1734}\\
Le physicien et mathématicien Leonhard Euler introduit la notation $f(x)$ pour une fonction appliquée à l'argument $x$.

\textbf{+1733}\\
Le mathématicien Geralamo Saccheri étuide ce que serait la géométrie si les 5ème postulat d'Euclide était faux.

\textbf{+1729}\\
Le teinturier et amateur de physique et d'astronomie Stephen Gray est le premier à découvrir la transmission de l'électricité dans des matériaux qu'il appellera des "conducteurs".

\textbf{+1727}\\
Le physicien et mathématicien Leonhard Euler introduit la notation moderne des fonctions trigonométriques et la lettre $e$ pour la base du logarithme naturel (également connue occasionnellement sous le nom de "nombre d'Euler").

\textbf{+1724}\\
Le mathématicienAbraham De Moivre étudie les statistiques de mortalité et fonde la théorie des annuités sur la vie.

\textbf{+1715}\\
Le mathématicien Brook Taylor publie les outils permettant de faire des intégrations par parties ainsi que des développements en série de fonctions (les fameuses séries de Taylor).

\textbf{+1714}\\
Le mathématicien Brook Taylor dérive la fréquence fondamentale de vibration d'une corde tendue en fonction de sa tension et de sa densité linéique en résolvant une équation différentielle ordinaire.

\textbf{+1713}\\
Le mathématicien et physicien Jacques Bernoulli publie les principes rigoureux de base du calcul des probabilités et statistique.

\textbf{+1705}\\
L'astronome Edmund (ou Edmond) Halley prédit avec une erreur quasi négligeable à l'aide du calcul que la comète qui est passée près de la Terre en 1682 repassera en 1758.

\textbf{+1704}\\
Le physicien et mathématicien Isaac Newton constate expérimentalement que la lumière blanche se compose de multiples couleurs. Il suppose également qu'un rayon lumineux est formé de corpuscules.

\textbf{+1701}\\
Dans son \textit{Explication de l'Arithmétique Binaire}, Gottfried Wilhelm Leibniz introduit l'arithmétique binaire (Leibniz a peut-être été le premier informaticien et théoricien de l'information). Il a anticipé l'interpolation lagrangienne et la théorie de l'information algorithmique. Son \textit{Calculus ratiocinator} a anticipé les aspects de la machine universelle de Turing. En 1961, le mathématicien et philosophe Norbert Wiener a suggéré que Leibniz soit considéré comme le saint patron de la cybernétique.

\textbf{+1698}\\
Le mathématicien et physicien Jacques Bernoulli pose clairement le problème de la courbe de brachistochrone (qui appartient à la famille des courbes cycloïdes) et propose une solution. La même année, le mathématicien Guillaume de L'Hopital énonce sa règle pour l'examen des formes indéterminées. Gottfried Leibniz propose l'utilisation du point $\cdot$ pour désigner la multiplication, au lieu de la croix $\times$ qui est trop facilement confondue avec la variable $x$ dans les équations.

\textbf{+1693}\\
L'astronome et ingénieur Edmund Halley découvre la relation qui lie la focale d'une lentille avec la distance de l'image à son axe et de l'objet réel à son axe. La même année, il construit la première table statistique de mortailté liant le taux de décès à l'âge.

\textbf{+1691}\\
Le philosophe et mathématicien Gottfried Leibniz découvre une technique de séparation de variables pour les équations différentielles ordinaires.

\textbf{+1690}\\
La théorie ondulatoire de la lumière est avancée par le physicien et astronome Christian Huygens ; la même année le physicien et mathématicien Jean Bernoulli développe le calcul exponentiel et trouve l'équation de la chaînette. La même année, le mathématicien et physicien Jacques Bernoulli (frère de Jean Bernoulli) développe le calcul intégral.

\textbf{+1687}\\
Le physicien et mathématicien Isaac Newton publie un ouvrage où il explique la force de gravité et les orbites des planètes. Il énonce également les trois lois de la dynamique. Il s'agit de la première révolution scientifique (avant la relativité restreinte/générale et la physique quantique).

\textbf{+1685}\\
Le philosophe et mathématicien Gottfried Leibniz résout les systèmes linéaires en usant sans justification théorique de matrices et de déterminants.

\textbf{+1682}\\
Le physicien et mathématicien Isaac Newton établit la loi de la gravitation qui porte aujourd'hui son nom.

\textbf{+1679}\\
Le philosophe et mathématicien Gottfried Leibniz introduit l'arithmétique binaire et met au point une machine à calculer qui effectue les 4 opérations. La même année, le physicien, mathématicien et inventeur Denis Papin montre expérimentalement l'influence de la pression atmosphérique sur le point d'ébullition de l'eau.

\textbf{+1678}\\
Le mathématicien, astronome et physicien Christian Huygens postule le son principe du front d'onde.

\textbf{+1676}\\
Le physicien Robert Hooke énonce que l'étirement d'un ressort est proportionnel à la tension.

\textbf{+1675}\\
L'astronome Olaus Roemer fait des mesures précises de la vitesse de la lumière. La même année, le chimiste apothicaire Nicolas Lemery écrit \textit{Cours de chymie} qui est considéré comme le premier grand traité de chimie où les mélanges sont définis, la première théorie des bases et des acides, etc. et le physicien et astronome Isaac Newton invente un algorithme pour calculer les racines de fonctions.

\textbf{+1673}\\
Le philosophe et mathématicien Gottfried Leibniz invente son calcul différentiel, introduit le symbole $\int $ et utilise le terme "test de convergence" pour les séries alternées et utilise pour la première fois la définition de la convergence. Leibniz utilise aussi la notation $\mathrm{d}y / \mathrm{d}x$ et $\mathrm{d}x$, $\mathrm{d}x$.

\textbf{+1671}\\
Première tentative de calcul des rentes viagères (comparable à l'assurance-vie) par le politicien Johan de Witt en collaboration avec le mathématicien Christian Huygens et premiers calculs de l'espérance de vie.

\textbf{+1670}\\
Le mathématicien John Wallis introduit les symboles $\le$ et $\ge$.

\textbf{+1669}\\
Le mathématicien, astronome et physicien Christian Huygens publie ses résultats sur l'observation de la conservation de l'énergie cinétique devenant textuellement le découvreur du concept d'énergie cinétique. La même année, dans son manuscrit, le physicien et astronome Isaac Newton donne la première description de la méthode de Newton qui permet de trouver des approximations de racines fonctionnelles par processus itéré. La description de Newton ne s'applique qu'aux polynômes et n'utilise pas la notion de dérivée (le manuscrit sera publié en 1711).

\textbf{+1668}\\
Le physicien et astronome Isaac Newton réalise le premier télescope à réflexion et la même année le mathématicien John Wallis suggère la loi de conservation de la quantité de mouvement.

\textbf{+1667}\\
Dans son \textit{Vera Circuli et Hyperbolae Quadratura} le mathématicien et astronome James Gregory donne la première preuve du théorème fondamental de l'Algèbre et découvre indépendamment les séries de Taylor. Ébauche du concept de nombre transcendantal élaboré par rapport au problème de la quadrature du cercle.

\textbf{+1665}\\
Le physicien Isaac Newton formule les trois lois de la mécanique. Il jette les bases du calcul différentiel, ces techniques lui permettant à partir de l'expression d'une force inverse au carré de la distance, de retrouver la forme générale des lois de Kepler.

\textbf{+1664}\\
Le physicien Isaac Newton commence à travailler sur le calcul différentiel et intégral.

\textbf{+1661}\\
Le fondateur de la démographie statistique John Graunt publie la première table de mortalité ; la même année le physicien et chimiste Robert Boyle détermine les lois de compressibilité des gaz portant aujourd'hui son nom attaché parfois à celui du physicien Edme Mariotte  qui redécouvrit quelques années après les mêmes lois.

\textbf{+1659}\\
Le mathématicien, astronome et physicien Christian Huygens découvre la formule de l'isochronisme rigoureux (lorsque l'extrémité du pendule parcourt un arc de cycloïde, la période d'oscillation est constante quelle que soit l'amplitude).

\textbf{+1658}\\
Le mathématicien, astronome et physicien Christian Huygens découvre expérimentalement que les balles placées n'importe où sur une cycloïde renversée atteignent le point le plus bas de la cycloïde dans le même temps et ainsi montre expérimentalement que l'isochronisme de la cycloïde.

\textbf{+1657}\\
Le juriste et mathématicien Pierre de Fermat énonce son "principe de Fermat" en optique comme quoi la lumière se propage d'un point à un autre sur des trajectoires telles que la durée du parcours soit localement minimale.

\textbf{+1655}\\
Le mathématicien, astronome et physicien Christian Huygens est le premier à utiliser le concept d'espérance en probabilités. Le mathématicien John Wallis introduit le symbole $\infty$ dans son \textit{Mathesis Universalis}.

\textbf{+1654}\\
Le mathématicien, physicien, inventeur, philosophe, moraliste et théologien Blaise Pascal et le juriste et mathématicien Pierre de Fermat créent la théorie des probabilités.

\textbf{+1650}\\
La plus ancienne institution scientifique nationale du monde, la Royal Society, est fondée à Londres. Il établit la preuve expérimentale comme l'arbitre de la vérité.

\textbf{+1644}\\
Le physicien et mathématicien Evangelista Torricelli a l'idée de substituer du mercure à de l'eau dans l'expérience dite de Torricelli de mise en évidence du "grosso-vido"; suivront plus tard les travaux du mathématicien, physicien, inventeur, philosophe, moraliste et théologien Pascal Blaise (expérience du Puy de Dôme).

\textbf{+1638}\\
Le mathématicien, géomètre, physicien et astronome Galileo Galilée publie la relation mathématique définissant la période du pendule simple.

\textbf{+1637}\\
Le philosophe et mathématicien René Descartes renomme les inconnues $x$, $y$, $z$ et les paramètres $a$, $b$, $c$ et étend l'usage de l'algèbre aux longueurs et au plan, créant avec le juriste et mathématicien Pierre de Fermat la géométrie analytique. La même année, toujours René Descartes, détermine quantitativement les angles primaires et secondaires des arc-en-ciels respectivement à l'inclinaison du Soleil.

\textbf{+1631}\\
Le mathématicien Thomas Harriot introduit, dans une publication posthume, les symboles $>$ et $<$. La même année le mathématicien et théologien William Oughtred donne pour la première fois le symbole multiplié $\times$ et le symbole $\pm$.

\textbf{+1629}\\
L'avocat et mathématicien Pierre de Fermat développe un calcul différentiel rudimentaire. La même année, dans son \textit{Invention nouvelle en algèbre}, Albert Girard déclare, sans preuves, pour la première fois le théorème fondamental de l'algèbre (un polynôme de degré $ n $ a $ n $ racines complexes distinctes ou non distinctes) en utilisant des nombres complexes.

\textbf{+1626}\\
Publication de tables de sinus, tangente et sécante par l'ingénieur Albert Girard et utilisation des abréviations $\sin$, $\cos$ et $\tan$ par ce dernier.

\textbf{+1624}\\
Invention du premier thermomètre (dont les graduations ne sont bien évidemment pas normalisées...) par le médecin Santorio Santorio.

\textbf{+1621}\\
L'astronome et physicien Willebrord Snell découvre que l'angle de réfraction de la lumière est déterminé par le sinus de l'angle formé par la lumière incidente avec la normale au dioptre.

\textbf{+1620}\\
L'ingénieur Francis Thomas Bacon défend et documente la méthode expérimentale (ie méthode scientifique) et mène de nombreuses observations sur la chaleur. Il suggère que la chaleur est reliée au mouvement.

\textbf{+1619}\\
L'astronome et mathématicien Johannes Kepler a fini de publier les trois lois relatives au mouvement des planètes.

\textbf{+1614}\\
Le mathématicien John Napier invente les logarithmes, qui apportent les opérations de multiplication et de division à de simples additions ou soustractions.

\textbf{+1611}\\
L'astronome et mathématicien Johannes Kepler découvre la réflexion interne totale, la loi de réfraction aux petits angles de réfraction et l'optique des lentilles minces.

\textbf{+1610}\\ 
Dans son \textit{Sidereus Nuncius}, Galileo Galilée rapporte les premières observations au télescope: la découverte des satellites de Jupiter, la confirmation que la voie lactée est constituée d'étoiles, la découverte des anneaux de Saturne. Ces observations auront un effet important car elles contredisent certaines des idées des modèles de l'Univers de l'époque et des écrits bibliques.

\textbf{+1609}\\
Dans son \textit{Astronomia Nova}, l'astronome et mathématicien Johannes Kepler explique les deux premières lois de Kepler sur le mouvement des planètes.

\textbf{+1608}\\
L'opticien Hans Lippershey invente le télescope qui sera utilisé et amélioré (avec une qualité aléatoire) l'année suivante par le mathématicien, géomètre, physicien et astronome Galileo Galilée pour confirmer les théories de Copernic.

\textbf{+1604}\\
Dans une lettre à Paolo Sarpi, Galilée énonce la loi de la chute des corps: la distance parcourue est proportionnelle au carré du temps de chute.

\textbf{+1603}\\
Le mathématicien et astronome Thomas Harriot détermine la façon de calculer qualitativement la surface d'un triangle sphérique.

\textbf{+1596}\\
L'histoire quantitative des étoiles variables commence avec les observations de l'apparition et de la disparition de l'omicron Ceti (Mira) par David Fabricius 1596. L'étoile variable à éclipses Algol a été notée pour la première fois par Gernian Montanari en 1667, mais sa période de 2.867 jours n'a été mesurée qu'à partir des travaux de 1783 de Nathaniel Pigott, John Goodricke et Johann Georg Palitzsh. Cependant, l'idée de variabilité d'Algol de Miras était connue dans l'Antiquité par les Babyloniens entre 1895 et 539 avant JC (voir Schaumberger, J. 1935 \textit{Sternkunde und Sterndienst in Babel}, vol. 3, Verlag der Aschendorffschen Verlagsbuchhandlung) et les Chinois (250 avant JC à 25 après JC).

\textbf{+1591}\\
Le mathématicien François Viète ouvre une nouvelle période de l'algèbre en faisant opérer les calculs sur des lettres, utilisant les voyelles pour désigner les inconnues et les consonnes pour les paramètres. Par ailleurs, il donne le développement du binôme de Newton.

\textbf{+1590}\\
L'astronome Galileo Galilei démontre expérimentalement que tous les corps en chute libre ont une accélération identique. La même année, les opticiens Hans et Zacheraius Janssen créent le premier microscope en associant plusieurs lentilles ce qui définira les débuts de la biologie et la médecine scientifique.

\textbf{+1586}\\
L'ingénieur et physicien Simon Stevin démontra la méthode du parallélogramme des forces et découvre que la pression d'un liquide sur le fond d'un récipient est indépendante de sa forme, et aussi de la surface du fond; elle dépend seulement de la hauteur d'eau dans le récipient. Il donna aussi la mesure de la pression sur n'importe quelle portion du côté d’un récipient.

\textbf{+1576}\\
L'astronome Tycho Brahé observe une nouvelle étoile dans la constellation de Cassiopée et construit un observatoire dans l'île de Hveen.

\textbf{+1572}\\
Le mathématicien Rafaelle Bombelli donne une formulation des nombres complexes et les règles de calculs effectifs.  Il introduisit les termes più di meno (pdm) et meno di meno (mdm) pour représenter $+\mathrm{i}$ et $-\mathrm{i}$.

\textbf{+1548}\\
Le mathématicien et physicien Simon Stevin écrit les puissances du dixième cernées d'un exposant. Il donne la première écriture des vecteurs. 

\textbf{+1545}\\
Le mathématicien Ludovico Ferrari donne la solution des équations de degré $4$ (connue sous le nom de "formule de Cardan").

\textbf{+1543}\\
L'ouvrage de l'astronome Nicolas Copernic résumant 26 ans de recherches et d'observations est publié et met clairement en avant que le système héliocentrique de Ptolémée n'est pas valide.

\textbf{+1536}\\
Le mathématicien Niccolò Fontana développe la science de la balistique.

\textbf{+1530}\\
Le mathématicien et physicien Robert Recorde introduit le signe $=$ et le mathématicien Michael Stifel développe une première forme de notation algébrique.

\textbf{+1525}\\
Le mathématicien Christoff Rudolff introduit la notation des racines carrées $\sqrt{\phantom{a}}$.

\textbf{+1515}\\
Publication des travaux de Scipione del Ferro où il trouve une formule donnant la solution générale des équations polynomiales de degré $ 3 $. Ces solutions impliquent la manipulation implicite de nombres imaginaires

\textbf{+1510}\\
Le peintre, graveur et mathématicien Albert Dürer développe les bases de la géométrie descriptive et de la perspective.

\textbf{+1500}\\
Le mathématicien italien Scipione del Ferro parvient pour la première fois à une résolution algébrique d'un grand type d'équations du troisième degré.

\textbf{+1490}\\
Le peintre, sculpteur, architecte, musicien, mathématicien, inventeur, anatomiste, géologiste, cartographe, botaniste et écrivain... Leonardo da Vinci décrit le phénomène de capillarité.

\textbf{+1489}\\
Dans son \textit{Behende und hupsche Rechnung auf allen kauffmanschafft} la mathématicien Johannes Widmannwe introduit les symboles "$ + $" et "$ - $" sont introduits pour la première fois (avant plus était noté "P" et le moins "M").

\textbf{+1464}\\
Dans le \textit{Triparty en la science des nombres} par la mathématicien Nicolas Chuquet nous trouvons la première utilisation des puissances négatives et de la puisse nulle ainsi que l'énoncé de la propriété des exposants que nous utilisons encore $x^{n + m} = x^n x^m$.

\textbf{+1420}\\
Le mathématicien et astronome Jamshīd Al- Kāshī calcule et observe les éclipses solaires de 1406, 1407 et 1408. Il serait également le premier à utiliser la notation décimale en arithmétique et dans les chiffres arabes.

\textbf{+1400}\\
Le mathématicien et astronome Jamshīd Al-Kāshī développe une première forme de la méthode Regula falsi de Newton.

\textbf{+1200-1400}\\
Madhava et l'Ecole du Kerala (Inde) découvrent plusieurs séries infinies pour des nombres comme $\pi$ et des valeurs spécifiques de fonctions trigonométriques - celles-ci devancent celles des Européens sur le calcul différentiel et intégral et la série de puissances.

\textbf{+1350}\\
\textit{Tractatus de configurationibus qualitatum et motuum} d'Oresme est un brouillon de géométrie utilisant des coordonnées, et utilise des axes pour différentes tailles, ce qui est une étape importante dans la transition de la science qualitative basée principalement sur Aristote à la science quantitative. Preuve du théorème de vitesse moyenne, qui anticipe les résultats de Gallilée sur le mouvement rectiligne uniforme et les corps en chute libre en liant la zone sous la courbe de la vitesse à la position dans un graphique.

\textbf{+1303}\\
\textit{Siyuan Yujian} (traduction: \textit{Précieux miroir des éléments}) de Zhu Sjijie décrit la méthode d'élimination pour résoudre des systèmes d'équations contenant jusqu'à quatre inconnues et jusqu'au degré $14$ pour une certaine forme d'équations. On y trouve aussi la définition du triangle de Pascal et les formules de sommations pour certaines séries.

\textbf{+1300}\\
Le philosophe, poète, théologien, missionnaire, apologiste chrétien et romancier Raymon Lulle développe une machine géométrique (inutile) pour automatiser la logique théiste. Cette idée influencera le  philosophe, scientifique, mathématicien, logicien, diplomate, juriste, bibliothécaire Gottfried Wilhelm Leibniz dans sa recherche d'un langage universel pour le raisonnement, recherche qui l'amènera à s'intéresser à l'écriture chinoise et à l'arithmétique binaire. Les idées de Lulle anticipent les idées modernes des systèmes de déduction formelle.

\textbf{+1269}\\
Le savant Pierre de Maricourt invente les expressions de la magnétique "pôle nord" et "pôle sud" et il a été le premier qui a écrit que les pôles opposés d'un aimant s'attirent.

\textbf{+1268}\\
Le philosophe, savant et alchimiste Roger Bacon publie des propositions pour réformer l'école, argumentant que pour étudier la nature, l'utilisation des observations des mesures est la seule base rigoureuse de l'expérimentation et de la vérification tout en affirmant ainsi la nécessité des mathématiques.

\textbf{+1200}\\
Le mathématicien Jordan de Nemore introduit la notation des inconnues par des symboles.

\textbf{+1150}\\
Création de la notation moderne pour les fractions (barre horizontale) par Al-Hassãr. A la même époque, nous avons la traduction latine du traité de 820 par Al-Khwarizmi sur le calcul indien qui permet au système décimal et à l'utilisation du zéro de se propager en Europe et aussi l'écrivain et traducteur Gérard de Crémone publie une traduction en latin de la version arabe de l'Almagest de Ptolemée, le nom "sinus" vient de cette traduction ...

\textbf{+1121}\\
L'astronome, physicien, biologiste, chimiste, mathématicien et philosophe Abu al-Fath Khāzini publie un livre dans lequel il propose que la gravité et l'énergie potentielle gravitationnelle varient selon la distance du centre de la Terre. Il fait aussi une distinction entre la force, la masse et le poids. Il invente aussi plusieurs instruments scientifiques, y compris une balance romaine et une balance hydrostatique. Il introduit aussi des méthodes scientifiques expérimentales à la statique et la dynamique, les unifie dans la science de la mécanique et combine l'hydrostatique avec la dynamique pour créer l'hydrodynamique.

\textbf{+1114}\\
Le mathématicien Bhaskara offre un résumé complet de la mathématique hindoue, telle qu'elle s'est développée du 5ème au 7ème siècle de notre ère. Ainsi il  reconnaît les racines carrées négatives, résout des équations quadratiques à plusieurs inconnues, des équations d'ordre supérieur comme celles de Fermat ainsi que les équations du second degré générales. Il a également été un pionner dans le principe du calcul différentiel de près de $500$ ans relativement à Isaac Newton et Gottfried Wilhelm Leibniz.

\textbf{+1100}\\
Le philosophe et physicien Hibat Allah Abu'l-Barakat al-Baghdaadi est le premier à nier l'idée d'Aristote selon laquelle une force constante produit un mouvement uniforme ce qui préfigure la seconde loi de Newton sur le mouvement. Comme Newton, il a décrit l'accélération comme étant la variation de vitesse.

\textbf{+1037}\\
Le mathématicien, physicien et philosophe Ibn al-Haytham est conscient de la magnitude de l'accélération due a la gravité. Il découvre la loi de l'inertie, connue aujourd'hui comme la première loi du mouvement de Newton.

\textbf{+1030}\\
Le philosophe, écrivain, médecin et scientifique Abu 'Ali al-Husayn Ibn Abd Allah Ibn Sina (connu en Occident sous le nom de Avicenne) note que si la perception de la lumière est due à l'émission d'une sorte de particules par une source lumineuse, la vitesse de la lumière doit être finie. Il a également fourni une explication sophistiquée pour le phénomène de l'arc en ciel.  Le mathématicien, astronome, physicien, érudit, encyclopédiste, philosophe, astrologue, voyageur, historien, pharmacologue Abū al-Rayhān Muhammad ibn Ahmad  Al-Biruni, et plus tard l'astronome Abu al'Fath Khāzini, ont été les premiers à appliquer des méthodes scientifiques expérimentales dans la mécanique, en particulier les domaines de la statique et la dynamique, pour déterminer le poids spécifique, tels que ceux basés sur la théorie des équilibres et de pesage.

\textbf{+1021}\\
Le philosophe, mathématicien et physicien Ibn al-Haytham considéré comme le père de l'optique et un pionnier de la méthode scientifique explique correctement la lumière et la vision, et introduit la méthode scientifique expérimentale, jetant les bases de la physique expérimentale. Il discute aussi de la psychologie expérimentale et décrit les divers instruments d'optique comme la chambre noire. Il a été capable d'estimer la largeur de l'atmosphère avec une précision de $1$ [km], il a défini les principes d'inertie (première loi de Newton), de moment linéaire et il a calculé $\sum_ {k = 1}^n k^4=\frac{n(2n+1)(n+1)(3n^2+2n-1)}{30}$.

\textbf{+1019}\\
Le mathématicien, astronome et physicien Abū Rayhān Al-Biruni a observé et décrit l'éclipse solaire du 8 avril 1019, ainsi que l'éclipse lunaire du 17 Septembre 1019, en détail ; il a donné la localisation exacte des étoiles lors de l'éclipse lunaire. Il invente l'astrolabe orthographique et le planisphère.

\textbf{+1010}\\
Le mathématicien Al-Sijistani invente le Zuraqi, un astrolabe unique conçu pour un modèle planétaire héliocentrique dans lequel la Terre est en mouvement plutôt que le ciel.

\textbf{+1000}\\
Le mathématicien, physicien et astronome Abu Sahl al-Qouhi découvre que la lourdeur des corps varie en fonction de leur distance du centre de la Terre, et résout des équations plus élevées que le deuxième degré. Durant le même décennie, le mathématicien et ingénieur Al'Karkhi écrit un livre contenant les premières preuves connues par induction mathématique. Il l'utilise pour démontrer le théorème du binôme, le triangle de Pascal, et la somme des cubes intégrales.

\textbf{+996}\\
L'astrolabe mécanique orienté, comportant 8 roues dentées est inventé par le mathématicien, astronome et physicien Abū Rayhān Al-Biruni qui est aussi l'auteur de travaux sur la sommation de séries et la combinatoire.

\textbf{+980}\\
Abitu al-Wafaa fait le premier calcul des valeurs des fonctions trigonométriques et publie la loi des sinus adaptée pour les triangles sur la sphère. Il a aussi prouvé par induction que $\sum_{k=0}^n k^3=\frac{n^2(n+1)^2}{4}$.

\textbf{+964}\\
Le mathématicien, philosophe et physicien Abd al-Rahman al-Soufi explique le pouvoir grossissant des lentilles et fut un des premiers à se servir d'une méthode d'analyse scientifique qui influencera grandement de futurs scientifiques.

\textbf{+953}\\
Le mathématicien et ingénieur Al-Karkhi définit différents monômes et donne des règles pour les produits de n'importe quels deux d'entre eux. Il a aussi découvert le théorème du binôme pour des exposants entiers.

\textbf{+952}\\
Le mathématicien Abu'l-Hasan al-Uqlidisi modifie les méthodes de calcul pour le système numérique indien pour le rendre possible aux plumes et à l'utilisation du papier. Jusque-là, faire des calculs avec les chiffres indiens nécessitait l'emploi d'une planche.

\textbf{+900}\\
La première référence à un tube d'observation se trouve dans l'oeuvre de l'astronome et mathématicien Al-Battani, et la première description exacte du tube d'observation a été donnée par le mathématicien, astronome, physicien, érudit, encyclopédiste, philosophe, astrologue, voyageur, historien, pharmacologue Al-Biruni, dans une section de son travail  dédiée à vérifier la présence du nouveau croissant de Lune à l'horizon. Bien que ces tubes d'observations préliminaires n'aient pas de lentilles, ils ont permis à un observateur de se concentrer sur une partie du ciel en éliminant les interférences lumineuses. Ces tubes d'observation ont été adoptés ultérieurement en Europe latine, où ils ont influencé le développement du télescope.

\textbf{+880}\\
L'astronome et mathématicien Al-Battani découvre le mouvement de l'apogée du Soleil, calcule les valeurs de la précession des équinoxes et l'inclinaison de l'axe terrestre. Il est à l'origine de la définition de la fonction trigonométrique tangente et cotangente.

\textbf{+820}\\
Le mot "algèbre" naît. Le mathématicien, géographe, astrologue et astronome Muhammad ibn Musa Al'Khwarizmi est souvent considéré comme le père de l'algèbre médiévale, car il dégage celle-ci de l'emprise géométrique. On lui doit aussi le quadrant, l'instrument mural, le quadrant des sinus qui a été utilisé pour résoudre les problèmes trigonométriques et faire des observations astronomiques.

\textbf{+800}\\
Des astronomes inventent le cadran solaire universel et le cadran horaire universel à Bagdad.

\textbf{+780}\\
L'alchimiste Jabir Ibn Hayyan introduit la méthode scientifique expérimentale pour la chimie, ainsi que les appareils de laboratoire tels que l'alambic et des processus tels que la distillation pure, la liquéfaction, la cristallisation, et la filtration. Il à également inventé plus de vingt types d'appareils de laboratoire, ce qui a entraîné la découverte de plusieurs substances chimiques. Il a également développé des recettes pour le verre coloré.

\textbf{+773}\\
Les chiffres arabes (adaptés de l'Inde) font leur première apparition en Europe.

\textbf{+628}\\
Le mathématicien Brahmagupta donne des règles pour résoudre les équations linéaires et quadratiques. Il découvre que les équations quadratiques ont deux racines: la négative et l'irrationnelle et donne la forme moderne de la solution que nous connaissons aujourd'hui. Il donne aussi les règles à calculer avec des signes négatifs (arithmétique des nombres négatifs).

\textbf{+550}\\
Les mathématiciens hindous donnent à zéro une représentation numérique dans un système de notation positionnelle. \textit{Brahmasphutasiddhanta de Brahmagupta} est le premier livre qui fournit des règles pour les manipulations arithmétiques qui s'appliquent à zéro et aux nombres négatifs. Le \textit{Brahmasphutasiddhanta} est le premier texte connu à traiter le zéro comme un nombre à part entière, plutôt que comme un simple chiffre de substitution pour représenter un autre nombre comme cela a été fait par les Babyloniens.

\textbf{+499}\\
Le mathématicien Âryabhat obtient le nombre total de solutions d'un système d'équations linéaires par des méthodes équivalentes aux méthodes modernes, et décrit la solution générale de telles équations. Il fournit également des solutions d'équations différentielles. Il affirme également que la Lune et les objets célestes autres que les étoiles reflètent la lumière du Soleil, il explique correctement les causes des éclipses lunaires et solaires, donne la durée de l'année sidérale à quelques minutes, approxime $\pi$ par $62832/20000$ , calcule le diamètre de la Terre avec plus de précision que Erathostene, décrit le calcul avec le système de numérotation indien, donne la plus ancienne table des sinus pour $24$ angles.

\textbf{+275}\\
Le mathématicien Diophante d'Alexandrie considéré comme le père de l'algèbre étudie les équations à variables rationnelles (incluant donc les équations du second degré) et les équations diophantiennes.

\textbf{+195}\\
\textit{Suàn shù shu} (traduction: \textit{livre sur les nombres et calculs}) est l'un des plus anciens textes mathématiques chinois connus qui contient entre autres, des calculs de sommes de progression géométrique pour l'intérêt.

\textbf{+130}\\
\textit{Compositions mathématiques} (aussi connu sous le nom de \textit{Almagest's}) de Ptolémée d'Alexandrie présente un modèle géométrique du système solaire qui tente de décrire le mouvement des planètes. Le modèle est inspiré par une idée géométrique d'Apollonius et utilise des cercles dont le centre se déplace dans des orbites circulaires. Ce modèle met la Terre au centre du système solaire mais donne une assez bonne description des mouvements observés des différentes étoiles. Ce sera le modèle dominant jusqu'à Copernic. Quelques idées y préfigurent les séries de Fourier qui seront introduites au 19ème siècle.

\textbf{+125}\\
Le \textit{Yale Papyrus Musique} et le \textit{Michigan Papyrus Instrumental} semblent contenir les plus anciens exemples connus de notation musicale.

\textbf{+121}\\
Année correspondant au plus ancien document faisant mention de la pierre magnétique.

\textbf{+120}\\
Le catalogue d'étoiles de Zhang Heng contient $2'500$ étoiles. Zhang Heng décrit correctement la cause des éclipses et démontre que la lune est sphérique.

\textbf{+100}\\
L'ingénieur, mécanicien et mathématicien Héron d'Alexandrie redécouvre (après les chinois) le concept de force. Il invente aussi un système d'engrenages pour soulever des poids utilisant la puissance de la vapeur. Il fait la première description du sextant (mais ne l'a toutefois pas inventé). Son contemporain astronome Claude Ptolémée invente le sextant et décrit l'astrolabe (peut-être inventé par l'astronome, géographe et mathématicien Hipparque) et étudie la réfraction et la réflexion. Pendant le même siècle le mathématicien et philosophe Nicomaque de Gérase définit ainsi les nombres pairs et impairs, les nombres premiers, composés et nombres parfaits.  Toujours au cours du même siècle, la version finale de \textit{Neuf chapitres sur l'art mathématique} (presque $1'180$ pages), écrite sur dix ans par plusieurs auteurs chinois anonymes, contient la première utilisation de nombres négatifs, le chapitre $9$ utilise le théorème de Pythagore, le chapitre $8$ utilise des matrices et l'élimination de Gauss pour résoudre des systèmes d'équations (au moins 1700 ans avant Gauss !!!).

\textbf{+80}\\
L'érudit Wang Ch'ung réalise la première boussole aimantée sur un plateau de laiton.

\textbf{-87}\\
Année correspondant à la datation de la machine d'Anticythère, considérée comme le premier calculateur analogique et la première machine à engrenages (une trentaine!) antique permettant de calculer des positions astronomiques complexes. Le soin et l'adresse avec lesquels cette machine fut réalisée, ainsi que les capacités nécessaires en mécanique et en astronomie remettent en question les connaissances historiques sur les sciences grecques avant sa découverte. En effet, aucun objet de même âge et de même complexité n'était connu dans le monde et il faut attendre près d'un millénaire pour voir apparaître des mécanismes comparables! Le physicien, mathématicien et ingénieur Archimède de Syracuse en est l'hypothétique créateur.

\textbf{-100}\\
Le texte indien \textit{Anuyoga Dwara Sutra} contient plusieurs identités impliquant des racines carrées et des carrés qui semblent impliquer une certaine connaissance des lois des exposants ou des logarithmes. Une identité tirée de ce texte en notation moderne est: $ \sqrt{a}\sqrt{\sqrt {a}} = (\sqrt{\sqrt{a}})^3$.

\textbf{-134}\\
L'astronome, géographe et mathématicien Hipparque de Nicéedécouvre la précession des équinoes.

\textbf{-150}\\
L'astronome, géographe et mathématicien Hipparque est souvent désigné comme le créateur de la trigonométrie et des tables numériques correspondantes. Il calcule le premier la période de révolution du Soleil autour de la Terre (mais les résultats numériques sont en réalité ceux de la rotation de la Terre autour du Soleil) et élabore la théorie des excentriques et des épicycles.

\textbf{-200}\\
Dans le siècle, les chinois auraient inventé l'écluse, le gouvernail, le principe de la machine à vapeur plusieurs centaines d'années avant les occidentaux! Pendant ce siècle, le scientifique et ingénieur Philon de Byzance rédige des traités sur les leviers, la pneumatique, les automates, la traction et les clepsydres.

\textbf{-225}\\
Le géomètre et astronome Apollonius de Perge publie la première étude sur les coniques donnant à l'ellipse, à la parabole et à l'hyperbole les noms que nous leur connaissons. On lui attribue en outre l'hypothèse des orbites excentriques pour expliquer le mouvement apparent des planètes et la variation de vitesse de la Lune.

\textbf{-250}\\
Le mathématicien, physicien et ingénieur Archimède de Syracuse étudie des machines simples tel que le levier, la fameuse vis permettant entre autre de pomper l'eau ("vis d'Archimède") et découvre le principe d'Archimède qui explique la flottaison. Dans la même décennie, l'astronome, géographe, philosophe et mathématicien Ératosthène de Cyrène calcule le diamètre de la Terre à l'aide d'un gnomon et de son ombre et démontra l'inclinaison de l'écliptique.

\textbf{-260}\\
Le mathématicien, physicien et ingénieur Archimedes of Syracuse calcule  avec une précision de deux décimales utilisant des polygones inscrits et circonscrits et calcule la surface sous un segement de parabole.

\textbf{-281}\\
L'astronome et mathématicien Aristarque de Samos fait l'hypothèse que le Soleil occupe le centre du système solaire et utilise la trigonométrie pour estimer le rayon de la Lune et sa distance à la Terre en utilisant l'ombre de la Terre pendant une éclipse Lunaire.

\textbf{-300}\\
Le mathématicien et géomètre Euclide publie ses \textit{Éléments}, où il réorganise toute la connaissance de la géométrie incluant des démonstrations logiques, la construction des 5 solides platoniciens. Dans son \textit{Optica}, il nota que la lumière va en ligne droite et décrit la loi de réflexion. Nous avons aussi que les concepts de conique, ellipse, parabole et hyperbole qui apparaissent dans les travaux de deux mathématiciens de la Grèce antique, à savoir Menaech-mus et Appolonius de Pergé, qui introduisent aussi le concept de tangente. La même période Bhagabati Sutra calcule les permutations et les combinaisons d'ordre $1$, $2$ et $3$.

Dans la cosmologie hindoue, les Manusmriti (1.67–80) et les Puranas décrivent le temps comme cyclique, avec un nouvel univers (planètes et vie) créé par Brahma tous les 8.64 milliards d'années. L'univers est créé, maintenu et détruit au cours d'une période de kalpa (jour de Brahma) d'une durée de 4.32 milliards d'années, et est suivie d'une période de pralaya (nuit) de dissolution partielle d'une durée égale. Dans certains Puranas (par exemple Bhagavata Purana), un cycle de temps plus large est décrit où la matière (mahat-tattva ou utérus universel) est créée à partir de la matière primitive (prakriti) et de la matière racine (pradhana) tous les 622.08 milliards d'années, à partir de laquelle Brahma est né. Les éléments de l'univers sont créés, utilisés par Brahma et complètement dissous dans une période de maha-kalpa (vie de Brahma; 100 de ses 360 jours) d'une durée de 311.04 milliards d'années contenant 36000 kalpas (jours) et pralayas (nuits), et est suivie d'une période maha-pralaya de dissolution complète d'une durée égale. Les textes parlent également d'innombrables mondes ou univers.

\textbf{-310}\\
Le savant Autolycus de Pitane définit le mouvement uniforme tel qu'un objet parcoure une quantité égale de distance en un temps égal.

\textbf{-370}\\
Le philosophe Aristote développe la logique avec une théorie naïve des propositions, des quantités et du raisonnement par inférence. Il a également écrit un traité de météorologie nommé \textit{Meteorologica}. Il ressort clairement de ce traité qu'il comprenait déjà le cycle de l'eau et savait que les nuages étaient constitués d'eau évaporée et sont donc des structures très lourdes prises dans leur globalité.

\textbf{-388}\\
Le philosophe et astronome Héraclide du Pont fait l'hypothèse de la rotation de la Terre sur elle-même afin d'expliquer le mouvement apparent des étoiles au cours de la nuit (mais toujours dans un cadre géocentrique) et suggère que chaque planète est un corps comme la Terre. Il savait également (d'une manière naïve certes!) et a testé expérimentalement comment l'eau salée et l'eau normale étaient séparées et mélangées dans les mers.

\textbf{-400}\\
L'École Stoïcienne développe les propositions composées et les connecteurs logiques: "implication", "et", "ou" et les inférences "Modus ponens" et "Modus tollens".

\textbf{-430}\\
Le philosophe Démocrite d'Abdère avance l'idée selon laquelle la matière se compose de particules identiques et minuscules qu'il appelle des "atomes". En réalité il s'agit plutôt d'une vulgarisation des idées de son maître, le philosophe Leucippe de Milet élaborées dix ans avant. Hippasus, un disciple de Pythagore, aurait donné ce qui est probablement la première preuve rigoureuse de l'irrationalité de $\sqrt{2}$. La preuve utilise un raisonnement de Reductio ad absurdum pour montrer que les côtés d'un carré sont incommensurables avec sa diagonale.

La plus ancienne trace écrite connue de la Camera Obscura est une description du philosophe chinois Han Mozi (environ 470 à environ 391 avant JC). Mozi a correctement affirmé que l'image de la caméra obscura est inversée parce que la lumière se déplace en lignes droites depuis sa source. Au 11ème siècle, le physicien arabe Ibn al-Haytham (Alhazen) a écrit des livres très influents sur l'optique, y compris des expériences avec la lumière à travers une petite ouverture dans une pièce sombre.

\textbf{-500}\\
Les philosophes Leucippe et Démocrite fondent l'atomisme.

\textbf{-540}\\
Le philosope et mathématicien Pythagore étudie la géométrie propositionnelle et la vibration de la corde de la lyre.

\textbf{-600}\\
Le philosophe Empédocle plaide pour une décomposition du monde en quatre éléments fondamentaux: l'eau, la terre, l'air et le feu. Le même siècle, le mathématicien Thalès de Milet met en évidence l'électrostatique en frottant un morceau d'ambre, prédit une éclipse et développe la géométrie du triangle.

Les philosophes grecs, dès Anaximandre, introduisent l'idée d'univers multiples voire infinis. Démocrite a précisé en outre que ces mondes variaient en distance, en taille; la présence, le nombre et la taille de leurs soleils et lunes; et qu'ils sont sujets à des collisions destructrices. Aussi pendant cette période, les Grecs ont établi que la Terre est sphérique plutôt que plate.

\textbf{-750}\\
Manava Sulbar Sutras de Manava trouve l'irrationalité de $\sqrt{2}$ et $\sqrt{61}$ et accepte d'utiliser des nombres irrationnels dans ses calculs. La première université connue (en utilisant strictement la définition de ce mot qui est dérivé du latin \textit{universitas magistrorum et studentium}, qui signifie approximativement \og communauté d'enseignants et d'érudits \fg{}) a été créée à Milet (où Thales a étudié).

\textbf{-800}\\
Les Assyriens utilisent des clepsydres et les Chinois tracent les mouvements planétaires pour leur calendrier.

\textbf{-1295}\\
Vers 1295 av.J.-C., un nouveau mot hiéroglyphique apparaît. Bia-n-pet se traduit littéralement par «fer du ciel». Le nouveau mot a été appliqué à tout le fer métallique à partir de cette époque. Une explication évidente de l'émergence soudaine du nouveau mot serait un événement majeur, tel qu'un grand impact ou une pluie de météorites, observé par la population égyptienne antique. Cela aurait laissé peu d'incertitude quant à l'origine exacte du fer mystérieux et aurait créé une association suffisamment forte entre le fer et le ciel pour que le nouveau mot devienne synonyme de toutes les formes de fer. En 2008, un grand cratère d'impact formé par une météorite de fer a été découvert dans le sud de l'Égypte. Bien que son âge exact soit inconnu, l'archéologie locale suggère qu'il s'est formé au cours des 5000 dernières années, ce qui en fait un événement candidat possible.

\textbf{-1400}\\
Les peuples néolithiques d'Écosse construisent des modèles en pierre des cinq solides de Platon (polyèdres réguliers).

\textbf{-1500}\\
Les indiens développent la théorie des 4 éléments (eau, air, terre feu).

\textbf{-1700}\\
Le mathématicien Apastamba résout les équations linéaires générales et utilise les systèmes d'équations diophantiennes comportant jusqu'à cinq inconnues. Le même siècle, les mathématiciens égyptiens utilisent des fractions simples.

\textbf{-1800}\\
Les origines de l'algèbre peuvent être attribuées aux anciens Babyloniens qui ont développé un système de nombre positionnel qui les a grandement aidés à résoudre leurs équations algébriques rhétoriques. Les Babyloniens n'étaient pas intéressés par des solutions exactes, mais plutôt par des approximations, et ils utilisaient donc couramment l'interpolation linéaire pour approximer les valeurs intermédiaires.  L'une des tablettes les plus célèbres est la tablette Plimpton 322, créée vers 1900–1600 avant JC, qui donne un tableau des triplets de Pythagore et représente certaines des mathématiques les plus avancées avant les mathématiques grecques.

\textbf{-1800}\\\
On savait à l'époque védique et postérieure (\textit{Rigveda}, I, 19.7, I 23.17, I, 32.9,VIII, 6.19 ; VIII, 6.20 ; et VIII, 12.3) (Sarasvati, 2009) que l'eau n'est pas perdue dans les différents processus du cycle hydrologique, à savoir l'évaporation, la condensation, les précipitations, l'écoulement, etc., mais se transforme d'une forme en une autre (transfert d'eau de la Terre à l'atmosphère par le Soleil et le vent).

\textbf{-2000}\\
Les prêtres babyloniens font les premiers relevés des observations célestes.

\textbf{-2300}\\
Les astronomes chinois font les premières observations du ciel.

\textbf{-2500}\\
Les Mésopotamiens imaginent un système de numérotation de position composé de symboles dont la valeur est fonction de leur rang à l'intérieur d'un nombre.

\textbf{-2600}\\
La plus ancienne table mathématique connue de multiplication est gravée pour calculer les surfaces.

\textbf{-3000}\\
Les Chinois et les Babyloniens inventent l'abaque, première machine à additionner. Des concepts géométriques sont développés pour l'arpentage (hypoténuse). C'est aussi la période correspondante au plus ancien outil connu utilisé par les Incas pour enregistrer des nombres grâce aux noeuds sur une corde et aussi à la représentation murale des roues.

\textbf{-3500}\\
Le plus vieux rapport météo est trouvé sur une pierre en Égypte. Les conditions météorologiques inhabituelles décrites sur la dalle étaient le résultat d'une explosion volcanique massive à Thera, l'île actuelle de Santorin en Méditerranée.

\textbf{-3700}\\
Une tablette babylonienne semble contenir les premiers angles trigonométriques remarquables et la preuve que le théorème de Pythagore était déjà connu par les Babyloniens.

\textbf{-4900}\\
Le cercle de Goseck (allemand: Sonnenobservatorium Goseck) pourrait être l'un des plus anciens observatoires solaires du monde.

\textbf{-5000}\\
Le système décimal est utilisé dans l'Egypte ancienne (il semble que le consensus scientifique de datation se situe entre $-6000$ et $-3000$).

\textbf{-5200}\\
Datation au radiocarbone des roues les plus anciennes (roue de Ljubljana Marshes).

\textbf{-8000}\\
Warren Field est l'emplacement d'un calendrier mésolithique. Il comprend des $12$ fosses supposées correspondre aux phases de la Lune et utilisées comme calendrier lunaire. Il est considéré comme le plus ancien calendrier lunaire trouvé. On associe au même  millénaire le fait que les marques d'un os ou de bois sont lentement remplacées par des jetons de différentes formes à compter.

\textbf{-20000}\\
L'os d'Ishango est un outil en os, daté de l'ère du Paléolithique supérieur. C'est une long morceau d'os brun foncé, le péroné d'un babouin, avec un morceau de quartz pointu fixé à une extrémité, peut-être pour la gravure. Il a d'abord été considéré comme un bâton de pointage, car il a une série de ce qui a été interprété comme des marques de pointage sculptées dans trois colonnes exécutant la longueur de l'outil. Il a également été suggéré que les rayures pourraient avoir été créées dans le but d'avoir une meilleure prise sur le manche ou pour une autre raison non-mathématique.

\textbf{-200000}\\
Les revendications pour les premières preuves définitives de la maîtrise du feu par un membre de l'Homo vont de $0.2$ à $1.7$ million d'années. Les preuves de l'utilisation contrôlée du feu par Homo Erectus, qui remontent à $400'000$ ans, bénéficient d'un large soutien scientifique.
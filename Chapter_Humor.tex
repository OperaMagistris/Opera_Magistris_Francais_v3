Si vous avez des histoires humoristiques dans le genre "scientifique" n'hésitez pas à nous les transmettre. Dans tous les cas, nous vous souhaitons du bon temps (certaines histoires sont en anglais car elles perdent leurs sens en français ou nous ne possédons pas les originaux des images pour modifier les textes).
	
\begin{center}
\textbf{Cette page est transmise avec des électrons 100\% recyclés}
\end{center}

	\section{Situations}
	
	Lors d'un grand jeu télévisé, les trois concurrents se trouvent être un ingénieur, un physicien et un mathématicien. Ils ont une épreuve à réaliser. Cette épreuve consiste à construire une clôture tout autour d'un troupeau de moutons en utilisant aussi peu de matériel que possible.

\begin{itemize}	 
	\item[$-$] L'ingénieur: Regroupe le troupeau dans un cercle, puis décide de construire une barrière tout autour.

	\item[$-$] Le physicien: Construit une clôture d'un diamètre infini et tente de relier les bouts de la clôture entre eux jusqu'au moment où tout le troupeau peut tenir dans le cercle.

	\item[$-$] Le mathématicien: Voyant ceci, construit une clôture autour de lui-même et se définit comme y étant à l'extérieur.
\end{itemize}

\begin{center}\underline{\hspace{5 cm}}\end{center}

Deux hommes se déplaçant en ballon sont perdus dans le désert. Ils aperçoivent un individu en train de méditer à l'ombre d'un arbre.

\begin{itemize}	 
	\item[$-$] "Où sommes-nous, s'il vous plaît ?" lui demandent-ils.
\end{itemize}

Après un long moment de réflexion, l'homme leur répond:

\begin{itemize}	 
	\item[$-$] "Dans un ballon."

	\item[$-$] "Merci, monsieur le mathématicien."
\end{itemize}

L'homme demande étonné:

\begin{itemize}	 
	\item[$-$] "Comment avez-vous su que j'étais mathématicien?"

	\item[$-$] "Pour trois raisons (répond l'aéronaute). Premièrement, vous avez beaucoup réfléchi avant de nous répondre. Deuxièmement, votre réponse est très exacte. Troisièmement, elle ne sert absolument à rien."
\end{itemize}

\begin{center}\underline{\hspace{5 cm}}\end{center}
	
	\begin{center}
		\includegraphics[scale=0.7]{img/humour/research_in_peace.jpg}	
	\end{center}
	
\begin{center}\underline{\hspace{5 cm}}\end{center}

Un ingénieur, un mathématicien et un physicien séjournent une nuit dans un hôtel. Heureusement pour ce gag, un début d'incendie s'enclenche dans chacune de leur chambre.

\begin{itemize}	 
	\item[$-$] Le physiciens se réveille, voit le feu, fait quelques observations méticuleuses et de nombreux calculs sur la couverture de la carte des vins de l'hôtel. Une fois ceci fait, il s'empare de l'extincteur et éteind de le feu de manière très précise en un seul et unique coup et retourne se coucher.

	\item[$-$] L'ingénieur se réveille, voit le feu, fait quelques observations méticuleuses et sur la couverture de la carte du restaurant fait quelques calculs. Une fois ceci fait et après avoir ajouté un facteur de sécurité de $5$, il s'empare de l'extincteur et asperge l'ensemble de la chambre plusieurs fois de suite et retourne se coucher.

	\item[$-$] Le mathématicien se réveille à son tour, fait quelques observation méticuleuses et sur un tableau noir se trouvant dans la chambre fait de nombreux calculs. Soudainement il s'exclame: "une solution existe!". Une fois la solution trouvée, il retourne dans son lit...
\end{itemize}	
\begin{center}\underline{\hspace{5 cm}}\end{center}

Un médecin, un légiste et un mathématicien discutent des mérites comparés d'une épouse et d'une maîtresse.

\begin{itemize}	 
	\item[$-$] Le légiste: "Il vaut mieux avoir une maîtresse. En cas de divorce, une épouse pose de nombreux problèmes légaux."

	\item[$-$] Le médecin: "Il vaut mieux avoir une épouse, car le sentiment de sécurité réduit le stress, et c'est bon pour la santé."

	\item[$-$] Le mathématicien: "Vous avez tous les deux torts. Le mieux est d'avoir les deux. Quand votre femme vous croît chez votre maîtresse, et votre maîtresse chez votre femme, vous pouvez faire des maths."
\end{itemize}	

\begin{center}\underline{\hspace{5 cm}}\end{center}

Un grand homme d'affaires engage un mathématicien, un informaticien et un physicien afin de pouvoir gagner à tous les Tiercés.

\begin{itemize}	 
	\item[$-$] Le mathématicien le premier s'attaque à la tâche, il calcule des matrices à n'en plus finir, pose des axiomes à tout bout de champs et après de longues semaines de Lemmes, théorèmes et conjectures, il conclut que le problème est formellement irrésolvable.

	\item[$-$] Ensuite, l'informaticien heureux d'avoir vu le mathématicien en échec, s'approche de son Cray III et après avoir écrit quantités d'algorithmes en C++ et introduit tous les paramètres et conditions initiales annonce joyeusement qu'il faudra juste quelques centaines d'années pour calculer le résultat de chaque Tiercé...

	\item[$-$] Le physicien, le sourire aux lèvres, informe ses éminents collègues qu'il a la solution. Il s'approche d'un tableau noir et tout en dessinant une sphère commence par dire: "Approximons le cheval par une sphère parfaite..."
\end{itemize}

\begin{center}\underline{\hspace{5 cm}}\end{center}

Lors d'un entretien d'embauche, un chef d'entreprise reçoit quatre ingénieurs: un ayant fait l'École polytechnique, le second HEC, le troisième informaticien, et le dernier sortant de l'université. Celui-ci explique aux quatre candidats qu'en définitive, pour faire marcher une entreprise, il suffit de savoir compter.

Il s'adresse donc au premier d'entre eux, le polytechnicien, et lui dit: "allez-y, comptez..."

\begin{itemize}	 
	\item[$-$] Le polytechnicien: "un... deux... un... deux..." two..."
\end{itemize}


L'homme étonné s'adresse ensuite à l'ingénieur sortant d'HEC: "À vous! Comptez..."

\begin{itemize}	 
	\item[$-$] L'ingénieur sortant d'HEC: "un KiloFranc... deux KiloFrancs, trois KF..."
\end{itemize} 


Il se retourne ensuite, inquiet, vers l'informaticien: "À vous! Comptez..."

\begin{itemize}	 
	\item[$-$] L'informaticien: "0... 1... 0... 1..." 
\end{itemize}

Désespéré, le chef d'entreprise s'adresse au dernier candidat sortant de faculté: "Allez-y, comptez..."

\begin{itemize}	 
	\item[$-$] Le jeune homme commence: "1... 2... 3... 4... 5... 6... 7..." 
\end{itemize}

Le chef d'entreprise rassuré: "continuez, continuez..."

\begin{itemize}	 
	\item[$-$] "8... 9... 10... valet... dame.. roi... " !!
\end{itemize}

	\begin{center}\underline{\hspace{5 cm}}\end{center}

	\begin{center}
		\includegraphics[scale=0.7]{img/humour/work_science_vs_doesnt_work.jpg}	
	\end{center}
	\pagebreak

Plusieurs personnes ont été invitées à résoudre le problème suivant: "Montrer que tous les entiers impairs sont premiers."

\begin{itemize}	 
	\item[$-$] Mathématicien: 3 est un nombre premier, 5 est un nombre premier, 7 est un nombre premier, 9 n'est pas un nombre premier - contre-exemple - la proposition est fausse!

	\item[$-$] Physicien: 3 est premier, 5 est premier, 7 est  premier, 9 est une erreur expérimentale, 11 est premier ...

	\item[$-$] Ingénieur: 3 est premier, 5 est premier, 7 est  premier, 9 est premier, 11 est premier ...

	\item[$-$] Computer Scientist: 3 est premier, 5 est premier, 7 est  premier ... erreur de segmentation

	\item[$-$] Avocats: un est premier, trois est premier, cinq est premier, sept est premier, bien qu'il semble y avoir une preuve prima facie que neuf n'est pas premier, il existe un précédent substantiel pour indiquer que neuf devrait être considéré comme premier. La lette  suivant présente le cas de la primauté de neuf ...

	\item[$-$] Libéraux: Le fait que neuf n'est pas premier indique un environnement culturel défavorisé qui ne peut être corrigé que par un programme d'enrichissement culturel financé par le gouvernement fédéral.

	\item[$-$] Programmeurs informatiques: un est premier, trois est premier, cinq est premier, cinq est premier, cinq est premier, cinq est premier cinq est premier, cinq est premier, cinq est premier ...

	\item[$-$] Professeur: 3 est premier, 5 est premier, 7 est premier, et le reste est laissé comme un exercice pour l'étudiant.

	\item[$-$] Linguiste: 3 est un premier impair, 5 est un premier impair, 7 est un premier impair, 9 est un premier très ...

	\item[$-$] Chimiste: 1 est premier, 3  est premier, 5  est premier ... hey, publions!

	\item[$-$] New Yorker: 3 est premier, 5 est premier, 7 est premier, 9 est ... CE N'EST PAS VOS AFFAIRES!

	\item[$-$] Programmeur: 3 est premier, 5 est premier, 7 est premier, 9 sera corrigé dans la prochaine version!

	\item[$-$] Commercial: 3 est un premier, 5 est un premier, 7 est un premier, 9 - laissez-moi vous faire faire une affaire ...

	\item[$-$] Publicitaire: 3 est un nombre premier, 5 est un nombre premier, 7 est un nombre premier, 11 est un nombre premier, ...

	\item[$-$] Comptable: 3 est premier, 5 est premier, 7 est premier, 9 est premier, déduction de 10\% d'impôt et 5\% d'autres obligations.

	\item[$-$] Statisticien: Essayons plusieurs nombres choisis au hasard: 17 est un nombre premier, 23 est un nombre premier, 11 est un nombre premier ... Ça me va bien!

	\item[$-$] Psychologue: 3 est un nombre premier, 5 est un nombre premier, 7 est un nombre premier, 9 est un nombre premier mais tente de le supprimer ...
	
	\item[$-$] RH: C'est quoi un nombre premier et un nombre impair?
\end{itemize}

	\begin{center}\underline{\hspace{5 cm}}\end{center}
	
Un mathématicien, un ingénieur et un physicien se voient remettre une balle en gomme rouge afin d'en déterminer le volume.

\begin{itemize}	 
	\item[$-$] Le mathématicien: Mesure le diamètre et évalue le résultat de la triple intégrale.

	\item[$-$] Le physicien: Remplis un tonneau d'eau, dépose la balle dans l'eau et mesure le déplacement total de volume.

	\item[$-$] L'ingénieur: Observe le modèle et le numéro de série dans sa table des "balles en gomme rouge".
\end{itemize}		
	\begin{center}\underline{\hspace{5 cm}}\end{center}

Cette histoire est une légende urbaine fort sympathique:

J'ai reçu un coup de fil d'un collègue à propos d'un étudiant. Il estimait qu'il devait lui donner un zéro à une question de physique, alors que l'étudiant réclamait un 20. Le professeur et l'étudiant se mirent d'accord pour choisir un arbitre impartial et je fus choisi. Je lus la question de l'examen: "Montrez comment il est possible de déterminer la hauteur d'un building a l'aide d'un baromètre". L'étudiant avait répondu: "On prend le baromètre en haut du building, on lui attache une corde, on le fait glisser jusqu'au sol, ensuite on le remonte et on calcule la longueur de la corde. La longueur de la corde donne la hauteur du building."

L'étudiant avait raison vu qu'il avait répondu juste et complètement à la question. D'un autre côté, je ne pouvais pas lui mettre ses points: dans ce cas, il aurait reçu son grade de physique alors qu'il ne m'avait pas montré de connaissances en physique. J'ai proposé de donner une autre chance à l'étudiant en lui donnant six minutes pour répondre à la question avec l'avertissement que pour la réponse, il devait utiliser ses connaissances en physique. Après cinq minutes, il n'avait encore rien écrit. Je lui ai demandé s'il voulait abandonner, mais il répondit qu'il avait beaucoup de réponses pour ce problème et qu'il cherchait la meilleure d'entre elles. Je me suis excusé de l'avoir interrompu et lui ai demandé de continuer. Dans la minute qui suivit, il se hâta pour me répondre: "On place le baromètre à la hauteur du toit. On le laisse tomber en calculant son temps de chute avec un chronomètre. Ensuite en utilisant la bonne formule connue par tous, on trouve la hauteur du building".  À ce moment, j'ai demandé à mon collègue s'il voulait abandonner. Il me répondit par l'affirmative et donna presque 20 à l'étudiant. En quittant son bureau, j'ai rappelé l'étudiant, car il avait dit qu'il avait plusieurs solutions à ce problème. "Hé bien, dit-il, il y a plusieurs façons de calculer la hauteur d'un building avec un baromètre. Par exemple, on le place dehors lorsqu'il y a du soleil. On calcule la hauteur du baromètre, la longueur de son ombre et la longueur de l'ombre du building. Ensuite, avec un simple calcul de proportion, on trouve la hauteur du building."

Bien, lui répondis-je, et les autres? À quoi l'élève répondit: "Il y a une méthode assez basique que vous allez apprécier. On monte les  étages avec un baromètre et en même temps on marque la longueur du baromètre sur le mur. En comptant le nombre de traits, on a la hauteur du building en longueur de baromètre. C'est une méthode très directe. Bien sûr, si vous voulez une méthode plus sophistiquée, vous pouvez pendre le baromètre à une corde, le faire balancer comme un pendule et déterminer la valeur de g au niveau de la rue et au niveau du toit. À partir de la différence de g la hauteur de building peut être calculée. De la même façon, on l'attache à une grande corde et en étant sur le toit, on le laisse descendre jusqu'à peu près le niveau de la rue. On le fait balancer comme un pendule et on calcule la hauteur du building à partir de sa période de balancement."

Finalement, l'élève conclut: "Il y a encore d'autres façons de résoudre ce problème. Probablement la meilleure est d'aller au sous-sol, frapper à la porte du concierge et lui dire: "J'ai pour vous un superbe baromètre si vous me dites quelle est la hauteur du building."

J'ai ensuite demandé à l'étudiant s'il connaissait la réponse que j'attendais. Il a admis que oui, mais qu'il en avait marre du collège et des professeurs qui essayaient de lui apprendre comment il devait penser.

Pour l'anecdote, l'étudiant était Niels Bohr (prix Nobel de Physique en 1923) et l'arbitre Rutherford (prix Nobel de Chimie en 1908).

\begin{center}\underline{\hspace{5 cm}}\end{center}

\begin{center}
	\includegraphics[scale=0.4]{img/humour/evidence_based.jpg}	
\end{center}
\pagebreak
Dans un cours de Russell portant sur le fait que d'une proposition fausse, toute proposition peut être déduite, un étudiant lui posa la question suivante:

\begin{itemize}	 
	\item[$-$] "Prétendez-vous que de $2 + 2 = 5$, il s'ensuit que vous êtes le pape ? "

	\item[$-$] "Oui", répondit Russell

	\item[$-$] "Et pourriez-vous le prouver !", demanda l'étudiant sceptique

	\item[$-$] "Certainement", réplique Russell, qui proposa sur le champ la démonstration suivante:

	\begin{enumerate}
		\item Supposons que $2 + 2 = 5$

		\item Soustrayons $2$ de chaque membre de l'égalité, nous obtenons $2 = 3$

		\item Par symétrie, $3 = 2$

		\item Soustrayant $1$ de chaque côté, il vient $2 =1$
	\end{enumerate}

	\item[$-$] Maintenant le pape et moi sommes deux. Puisque $2 = 1$, le pape et moi sommes un. Par suite, je suis le pape.
\end{itemize}
Sur ce ...

	\begin{center}\underline{\hspace{5 cm}}\end{center}
	
What is "pi"?

\begin{itemize}	 
	\item[$-$] Mathematician: "Pi is the ratio of the circumference of a circle to its diameter."

	\item[$-$] Engineer: "Pi is about 22/7."

	\item[$-$] Physicist: "Pi is 3.14159 plus or minus 0.000005"

	\item[$-$] Computer Programmer: "Pi is 3.141592653589 in double precision."

	\item[$-$] Nutritionist: "You're one track math-minded fellows, Pie is a healthy and delicious dessert!"
\end{itemize}

	\begin{center}\underline{\hspace{5 cm}}\end{center}
	
Un astronome, un physicien et un mathématicien et un informaticien prennent des vacances ensemble en Écosse. En regardant à l'exétrieur d'une fenêtre de train pendant un déplacement, ils observent tous un mouton noir au milieu d'un champ.

\begin{itemize}	 
	\item[$-$] "Comme c'est intéressant" s'exclame l'astronome. "Tous les moutons écossais sont noirs!". 

	\item[$-$] Ce à quoi le physicien répond: "Non, non! Quelques moutons écossais sont noirs!". 

	\item[$-$] Le mathématicien soupirant... dit: "En Écosse, il y a à dans un champ au moins un mouton dont au moins un des côtés est noir".

	\item[$-$] L'informaticien s'exclame: "Oh non! Un bug!".	
\end{itemize}

	\begin{center}\underline{\hspace{5 cm}}\end{center}	
	
Un mathématicien, un biologiste et un physicien sont assis à une terrasse d'un café observant les gens entrant et sortant d'un immeuble de l'autre côté de la rue.

D'abord ils ont observé deux personnes entrer dans l'immeuble. Après un certain temps, ils observent que trois personnes sortent de l'immeuble.

\begin{itemize}	 
	\item[$-$]  Le physicien: "Le moyen de mesure n'était pas adapté".

	\item[$-$] Le biologiste: "Ils se sont reproduits".

	\item[$-$] Le mathématicien: "Si maintenant il y a une personne qui entre, l'immeuble sera à nouveau vide"
\end{itemize}

	\begin{center}\underline{\hspace{5 cm}}\end{center}

Un mathématicien et un ingénieur suivent un cours dispensé par un physicien. Un des sujets concerne les théories de Kaluza-Klein impliquant des développements à $9$, $12$ et un nombre encore supérieur de dimensions. Le mathématicien se pose et écoute tranquillement le physicien alors que l'ingénieur est confus et perdu. À la fin de l'exposé, l'ingénieur a un terrible mal de tête alors que le mathématicien commente l'exposé du physicien.

\begin{itemize}	 
	\item[$-$] L'ingénieur dit: "Comment faites-vous pour comprendre tout cela?"

	\item[$-$] Le mathématicien répond: "Je visualise le processus dans ma tête!".

	\item[$-$] L'ingénieur rétorque: "Comment pouvez-vous visualiser quelque chose qui a lieu dans un espace à neuf dimensions?".

	\item[$-$] Le mathématicien répond: "C'est très simple! Je le visualise d'abord en $n$ dimensions et après je réduis $n$ à $9$."
\end{itemize}

	\begin{center}\underline{\hspace{5 cm}}\end{center}

Deux mathématiciens sont trouven dans un bar. Le premier explique au second que la connaissance des gens en mathématique est vraiment très élémentaire. Le second en désaccord insiste sur le fait que le niveau moyen est bien meilleur que ce que l'on peut soupconner.

Le premier mathématicien s'en va un instant aux toilettes et pendant son absence, le second interpelle une serveuse. Il lui raconte que d'ici quelques minutes un ami va revenir, l'interpeller et lui poser une question. Tout ce qu'elle aura à faire est de répondre "un tiers de x cube".

Elle répète "un tier--- dx cub"?

Il lui répète "un tiers de x cube".

Elle demande encore une fois, "un tiers dex cube?"

"C'est exact" réponde-t-il.

La serveur s'éloigne en marmonnant "un tiers dex cube...".

Le premier mathématicien revient des toilettes. Le second lui propose alors pour clore leur débat de demander un calcul de math à une des serveuses. Alors le premier rétorque qu'il suffit de demander à la serveuse blonde le calcul d'une intégrale élémentaire. Son ami rigolant approuve la démarche. Le premier mathématicien appelle donc l'unique serveur blonde et lui demande: "Quelle est l'intégrale de x au carré?".

La serveuse répond alors"un tiers de x cube" et pendant qu'elle s'éloigne, elle se retourne rapidement en ajouter "plus une constante!".

	\begin{center}\underline{\hspace{5 cm}}\end{center}

Paroles de profs:

\begin{itemize}	 
	\item[$-$] Il reste des lambeaux de l'argument
	\item[$-$] Le signe '-' devant le potentiel vous interpelle ? Et si je mets un '+' est-ce que cela vous convient mieux? Oui ! Alors mettons un '+'...
	\item[$-$] C'est une courbe de phase ça ? Non, c'est une jambe de chien !
	\item[$-$] On va le tuer et on fera croire à un suicide
	\item[$-$] Vous pouvez dire que c'est prévisible puisque c'est imprévisible !
	\item[$-$] Je compte sur vous pour comprendre un petit peu à fond la suite
	\item[$-$] Nous allons étudier cela maintenant un peu plus tard
	\item[$-$] Je vais vous présenter des résultats dépendant des équations de Maxwell que vous n'avez pas encore vues... de toute façon, au point où on en est...
	\item[$-$] Si vous ne comprenez pas, c'est normal... le contraire serait d'ailleurs étonnant
	\item[$-$] La relativité générale ne sert à rien ! C'est pas avec ça qu'on met des satellites en orbite!
	\item[$-$] Question à cinquante centimes
	\item[$-$] De temps en temps, il faut être absurde
	\item[$-$] Quand on n'a rien à se mettre sous la dent, on prend le théorème de Gauss
	\item[$-$] On ne voit pas par quelle vertu du Saint-Esprit il deviendrait neutre ?!
	\item[$-$] Vous comprendrez quand vous serez grand...
	\item[$-$] Prenons l'exemple d'une banque: monsieur y fait un dépôt et madame y fait un retrait... enfin comme d'habitude quoi !
	\item[$-$] ... plus le parallélisme sera parallèle...
	\item[$-$] Effacez-moi ces gris-gris résidus de calculs antérieurs
	\item[$-$] Effacez le membre de gauche. J'ai dit DE GAUCHE ! Où est votre droite ? C'est bien... eh bien la gauche, c'est de l'autre côté !
	\item[$-$] Voyons, qui sera la prochaine victime ?
	\item[$-$] Le '+' se reconnaît, et les électrons ne s'y trompent pas, à sa belle couleur rouge !
	\item[$-$] Le but du jeu est de vérifier que le critère ne dit pas de conneries
	\item[$-$] Si vous n'arrivez pas à faire cela, je vous rassure: c'est foutu pour l'examen
	\item[$-$] Si vous n'arrivez pas à faire cela, faites-le !
	\item[$-$] Vous être très forts pour trouver des choses fausses
	\item[$-$] Ce genre de démonstration, ça me plaît. Vous, ça vous fait faire des cauchemars
	\item[$-$] Je ne résoudrai pas ceci, car cela risquerait de heurter votre sensibilité !
	\item[$-$] Tout ce que les profs adorent mérite votre méfiance !
	\item[$-$] Vous ajoutez des patates et des cochons, c'est sans dimension...
	\item[$-$] On peut chercher la dérivée de l'impulsion de Dirac, c'est pas ça qui va nous sauter à la tête
	\item[$-$] C'est bête mais Riemann, c'est comme ça
	\item[$-$] Regardez l'équation que je viens d'effacer
	\item[$-$] Juridiquement parlant, le courant est dans ce sens
	\item[$-$] Il est hors de question que je perde mon temps à vous résoudre cette plaisanterie de basse volée !
	\item[$-$] Si la fille a compris alors vous devez avoir tous compris
	\item[$-$] Ah oui, vous vous moquez de moi. En fait, vous avez raison de vous moquer des profs car nous, on n'hésite pas à se moquer de vous
	\item[$-$] Attention, très très important: je vous propose d'en rêver la nuit
	\item[$-$] J'ai l'impression du jouer du stradivarius devant des vaches
	\item[$-$] La main de dieu répartit le champ perpendiculairement à la surface
	\item[$-$] Le miracle de la disparition de l'harmonique n'aura pas lieu aujourd'hui
	\item[$-$] Si vous vous réveillez la nuit, dites-vous que échantilloner dans le domaine des temps, c'est périodiser dans le domaine des fréquences... puis rendormez-vous
	\item[$-$] Les courbes gauches ne sont pas droites
	\item[$-$] Vous voyez, quelques fois mes résultats sont corrects
	\item[$-$] On note "$\mathbb{Q}$" la sortie, c'est évident...
	\item[$-$] Il va falloir une entourloupette pour réussir à récupérer cette variable
	\item[$-$] Je suis un 68000; M. le directeur passe dans le couloir, frappe à la porte: on ne fait pas mieux comme interruption !
	\item[$-$] Attention: un, deux, trois, à vos cerveaux !
	\item[$-$] Dans la steppe de l'automatique, nous arrivons à la partie aride des mathématiques
	\item[$-$] Quel intérêt ? Eh bien aucun. C'est une figure de rhétorique pédagogique
	\item[$-$] Si vous avez quelques souvenirs sur les graphes de fluence que jadis nous traçâmes...
	\item[$-$] Les électrons dans du métal très chaud, c'est la place de la Concorde à 5h du soir, donc ça ne conduit pas non plus
	\item[$-$] Les cercles, c'est ce qui a le moins de coins !
	\item[$-$] Si vous mettez vos doigts dans une prise, ce n'est pas un nombre complexe qui sort
	\item[$-$] Voir la démonstration dans vos lectures journalières... donc je m'en dispense
	\item[$-$] Et puis vient ensuite le père Carnot: il ramène sa fraise pour ne pas dire grand chose
	\item[$-$] Quelquefois, les gens qui ont un peu de culture connaissent ça... c'est raté pour aujourd'hui !
	\item[$-$] On n'appellera pas cela "rendement", car on peut obtenir des résultats supérieurs à $1$, ce qui pourrait troubler quelques esprits faibles
	\item[$-$] Moi, si on m'avait envoyé au tableau, j'aurais pas écrit cela
	\item[$-$] Vous trouverez la réponse sur 36 15 Archimède
	\item[$-$] J'utilise une méthode qui date de quelques siècles... comme moi d'ailleurs !
	\item[$-$] Posez-moi des questions... j'aimerais qu'on me pose des questions! ... d'autres questions ? Je vais être obligé de prendre la liste et de dire: "toi, tu as une question à poser"
	\item[$-$] L'ordonnancement implémenté dépend de l'instanciation du garbage collector s'il est synchronisé sur le multithreading préemptif de l'OS.
	\item[$-$] Vous n'êtes que des boîtes qui reçoivent des entrées et crachent des sorties
\end{itemize}

	\begin{center}\underline{\hspace{5 cm}}\end{center}

Sherlock Holmes et le Dr Watson sont au camping. Après un bon repas et une bonne bouteille de vin, ils gagnent leur sac de couchage dans la tente et s'endorment.

Quelques heures plus tard, Holmes se réveille et aussitôt secoue son compagnon:

\begin{itemize}
	\item[$-$] "Watson, regardez le ciel et dites-moi ce que vous voyez."
	
	\item[$-$] "Je vois des millions et des millions d'étoiles." répond Watson.
	
	\item[$-$] "Et qu'est-ce que vous en concluez ?" demande Holmes.
	
	\item[$-$] "Du point de vue astronomique, répond Watson, cela me dit qu'il y a des millions de galaxies et potentiellement des milliards de planètes. Du point de vue astrologique, j'observe que Saturne est en Lion. Du point de vue chronologique, j'en déduis qu'il est environ 3 heures 15. Du point de vue théologique, je vois que Dieu est tout-puissant et que nous sommes petits et insignifiants. Du point de vue météorologique, je pense que nous aurons une belle journée demain. Et vous, Holmes ?"
\end{itemize}

Sherlock Holmes reste pensif une minute puis déclare:

\begin{itemize}
	\item[$-$] "Watson, vous êtes un âne. On nous a volé la tente."
\end{itemize}

	\begin{center}\underline{\hspace{5 cm}}\end{center}

Un ingénieur, un mathématicien et un physicien séjournent une nuit dans un hôtel. Malheureusement un début d'incendi s'enclenche dans chacune de leur chambre.

Le physiciens se réveille, voit le feu, fait quelques observations méticuleuses et de nombreux calculs sur la couverture de la carte des vins de l'hôtel. Une fois ceci fait, il s'empare de l'extincteur et éteind de le feu de manière très précise en un seul et unique coup et retourne se coucher.

L'ingénieur se réveille, voit le feu, fait quelques observations méticuleuses et sur la couverture de la carte du restaurante fait quelques calculs. Une fois ceci fait et après avoir ajouté un facteur de sécurité de 5, il s'empare de l'extincteur et asperge l'ensemble de la chambre plusieurs fois de suite et retourne se coucher.

Le mathématicien se réveille à son tour, fait quelques observation méticuleuses et sur un tableau noir se trouvant dans la chambre fait de nombreux calculs. Soudainement il s'exclame: "une solution existe!". Une fois la solution trouvée, il retourne dans son lit...

	\begin{center}\underline{\hspace{5 cm}}\end{center}

Comment pouvez-vous deviner que la personne conduisant le véhicule face au vôtre est un physicien?

C'est simple: Il a un autocollant rouge à l'arrière où il est écrit: "Si vous voyez cet autocollant en bleu c'est que vous roulez trop vite."

	\begin{center}\underline{\hspace{5 cm}}\end{center}

Un physicien des plasma de Princeton se trouve à la plage et découvre une vieille lampe à huile dans le sable. En essuyant la lampe, un génie surgit. Pour le remercier de l'avoir libéré, le génie lui offre un voeux. Le physicien sort alors un carte du monde et entoure d'un cercle la région du moyen-orient et dit au génie "je veux que tu apportes la paix dans cette région".

Après 10 minutes de silence, le génie répond, "Eh bien dites... il y en a des problèmes là-bas avec le Liban, l'Iraq, Israel et toutes les autres lieux. C'est terriblement embarasssant, je n'ai jamais eu une tâche d'une telle difficutlé. Je suis dans l'obligation de vous demander d'exaucer un autre voeu. Celui-là étant trop ardu pour moi.".

Sur le coup, le physicien réfléchit et demande: "Je souhaite que le tokamak de l'université de Princeton réussisse la fusion stable".

Après 10 minutes de silence, le génie répond: "Puis-je voir la carte du moyen-orient à nouveau?"

	\begin{center}\underline{\hspace{5 cm}}\end{center}

Vous êtes un chasseur qui se trouve dans la jungle. Vous avez avec vous 2 cartouches et un fusil. Tout à coup vous tombez nez à nez avec une panthère. Vous avez soudain envie de fumer une pipe. Comment fais-tu?

Tu tires d'abord sur la panthère, mais tu la loupes. Tu as donc une loupe. Avec, la seconde cartouche, tu tues la panthère. Ensuite tu la prends par la queue et la fait tourner autour de ta tête. Le périmètre du cercle formé vaut 2Pi panthère. Tu possèdes alors 2 pipes en terre. Tu prends une pipe que tu casses. Ensuite tu gardes la moitié en main et dépose l'autre moitié par terre. Il y a donc un tas haut, et un tas bas. Tu as donc du tabac. C'est fini tu déposes le tabac dans la pipe en terre que tu allumes avec la loupe. Ouf!!!!!

	\pagebreak
	\section{Mathématiques}

Il y a $5$ personnes sur $4$ qui n'y connaissent rien en fraction....

	\begin{center}\underline{\hspace{5 cm}}\end{center}

L'amour c'est comme $\pi$, naturel, irrationnel, transcendant mais très réel.

	\begin{center}\underline{\hspace{5 cm}}\end{center}

Un professeur de maths explique à une blonde les limites. Il fait avec elle l'exercice suivant:
	\begin{gather*}
	\lim_{x \rightarrow 8} \dfrac{1}{x-8}=+\infty
	\end{gather*}
	À la fin de l'exercice, il demande à la blonde si elle a compris. "Oh oui monsieur j'ai tout compris!". N'y croyant qu'à moitié, il lui pose l'exercice suivant:

	Soit à calculer:
	\begin{gather*}
	\lim_{x \rightarrow 5} \dfrac{1}{x-5}
	\end{gather*}
	Et la blonde de résoudre:
	\begin{gather*}
	\lim_{x \rightarrow 8} \dfrac{1}{x-8}=+\infty \quad  \text{alors} \quad \lim_{x \rightarrow 5} \dfrac{1}{x-5}= \rotatebox[origin=c]{90}{5}  
	\end{gather*}
	
	\begin{center}\underline{\hspace{5 cm}}\end{center}
	
	\begin{center}
		\includegraphics{img/humour/self_complementary_graph.jpg}	
	\end{center}
	
	\pagebreak
Évolution de l'enseignement des Mathématiques (...):

\begin{itemize}	 
	\item[$-$] Enseignement 1960: Un paysan vend un sac de pommes de terre pour $100$ Frs. Ses frais de production s'élèvent au $4/5$ du prix de vente. Quel est son bénéfice ?

	\item[$-$] Enseignement traditionnel 1970: Un paysan vend un sac de pommes de terre pour $100$ Frs. Ses frais de production s'élèvent au $4/5$ du prix de vente, c'est-à-dire à $80$ Frs. Quel est son bénéfice ?

	\item[$-$] Enseignement moderne 1970: Un paysan échange un ensemble $P$ de pommes de terre contre un ensemble M de monnaie. Le cardinal de l'ensemble $M $est égal à $100$, et chaque élément PFM vaut $1$ Fr. Dessinez $100$ gros points représentant les éléments de l'ensemble $M$. L'ensemble $F$ des frais de production comprend $20$ gros points de moins que l'ensemble $M$. Représentez l'ensemble $F$ comme sous-ensemble de l'ensemble $M$ et donnez la réponse à la question suivante: Quel est le cardinal de l'ensemble $B$ des bénéfices ? (à dessiner en rouge)

	\item[$-$] Enseignement rénové 1980: Un agriculteur vend un sac de pommes de terre pour $100$ Frs. Les frais de production s'élèvent à $80$ Frs et le bénéfice est de $20$ Frs. Devoir: souligne les mots "Pommes de terre" et discutes-en avec ton voisin.

	\item[$-$] Enseignement réformé 1990: Un peizan kapitalist privilégié sanrichi injustement de $20$ Frs sur un sac de patat, analiz le tekst et recherche les fotes de contenu, de gramère, d'orthografe, de ponctuassion et ensuite dit ce que tu panse de set manière de s'enrichir.

	\item[$-$] Enseignement start-up 1999: Un producteur de l'espace agricole câblé consulte une data bank qui display le day-rate de la patate. Il load son progiciel de computation fiable et détermine le cash-flow sur écran bit-map (sous WMil avec config floppy et DD 40Go). Dessine avec ta souris le contour intégré 3D du sac de pommes de terre. Puis logues-toi au network par le www.blue-potatoe.com et suis les indications du menu.
	
	\item Enseignement 2010: Qu'est-ce qu'un paysan ?

\end{itemize}
	\begin{center}\underline{\hspace{5 cm}}\end{center}
	\begin{center}
		\includegraphics[scale=0.9]{img/humour/homework.jpg}	
	\end{center}

	\begin{table}[H]
	\begin{center}
		\definecolor{gris}{gray}{0.85}
			\begin{tabular}{|p{7.5cm}|p{7.5cm}|}
				\hline
				\multicolumn{1}{c}{\cellcolor{black!30}\textbf{
Lorsque vous lisez ou entendez}} & 
  \multicolumn{1}{c}{\cellcolor{black!30}\textbf{il faut comprendre...}} \\ \hline
				c'est trivial (ou évident) & je n'arrive pas à dire pourquoi c'est vrai \\ \hline
				automatiquement on a & idem \\ \hline
				un calcul montre que & un calcul que je n'ai pas fait montrerait certainement que\\ \hline
				le lecteur montrera facilement que & ça m'ennuie de montrer que\\ \hline
				Nous conseillons vivement au lecteur de faire les exercices indiqués & comme je ne les ai pas faits, vous pourriez me les corriger\\ \hline
				j'ai montré ce résultat dans un papier antérieur & je ne sais plus diable comment on fait pour prouver ce truc là
				\\ \hline
				on généralise facilement à & la généralisation dépasse mon niveau			
				\\ \hline
				d'après une propriété bien connue & par 10 personnes au monde
				\\ \hline
				la preuve tient en deux lignes & 	oui, mais moyennant cinq lemmes
				\\ \hline
				c'est de l'algèbre & ce n'est pas intéressant (de la bouche d'un analyste)
				\\ \hline
				c'est de l'analyse & ce n'est pas intéressant (de la bouche d'un algébriste)
				\\ \hline
				c'est élémentaire (ou classique) & dans la théorie des espaces bornitziens de deuxième espèce
				\\ \hline
				je n'ai pas bien compris ce pas dans votre démonstration & tu t'es planté dans ta démo
				\\ \hline
				Cette conférence était très intéressante & 	je n'y ai rien compris
				\\ \hline
		\end{tabular}
	\end{center}
	\end{table}	
	
	\begin{center}\underline{\hspace{5 cm}}\end{center}
	
	Le numéro que vous avez demandé est imaginaire ; veuillez tourner votre téléphone d'un quart de tour à droite et renuméroter…
	
	\begin{center}\underline{\hspace{5 cm}}\end{center}
	
	\begin{center}
		\includegraphics[scale=0.6]{img/humour/pizza.eps}	
	\end{center}
	\begin{center}\underline{\hspace{5 cm}}\end{center}
	
	Un mathématicien à son ami:

\begin{itemize}	 
	\item[$-$] "Es-tu fidèle?"

	\item[$-$] "Oui, à un isomorphisme près"
\end{itemize}

	\begin{center}\underline{\hspace{5 cm}}\end{center}
	
Comment les mathématiciens le font:

\begin{itemize}	 
	\item[$-$] Les théoriciens des nombres l'ont fait en premier

	\item[$-$] Nous savons que les analystes réels le font continûment, mais pour les spécialistes de théorie des ensembles, ce n'est qu'une hypothèse

	\item[$-$] Les analystes complexes le font entièrement mais avec conformisme

	\item[$-$] Les algébristes le font avec détermination et sans discrimination

	\item[$-$] Les topologistes le font ouvertement, mais compactement

	\item[$-$] Les topologistes différentiels et algébriques le font avec variété

	\item[$-$] Les spécialistes de combinatoire le font discrètement

	\item[$-$] Les statisticiens le font soit presque toujours, soit presque jamais

	\item[$-$] Les théoriciens de la mesure le font presque partout

	\item[$-$] Les logiciens le font avec consistance

	\item[$-$] Les géomètres le font au foyer mais avec courbures et torsions

	\item[$-$] Les théoriciens des groupes le font simplement et fidèlement

	\item[$-$] Les théoriciens des anneaux le font avec intégrité

	\item[$-$] Les théoriciens des corps le font en inversé

	\item[$-$] Les spécialistes de programmation linéaire maximisent la performance et minimisent les efforts

	\item[$-$] Markov avait besoin de chaînes pour le faire, et Noether d'anneaux

	\item[$-$] Euler le faisait en cercle, tandis que Bernoulli le faisait en spirale ou en huit

	\item[$-$] Möbius le faisait toujours du même côté

	\item[$-$] Gauss le faisait normalement, Lebesgue, avec mesure, et Cauchy le faisait complètement, au contraire de Gödel

	\item[$-$] Fermat a essayé de le faire dans la marge, mais il n'y avait pas assez de place

	\item[$-$] On pense que Riemann et Goldbach l'on fait, mais on n'est encore jamais arrivé à le prouver
 \end{itemize}
 
	\begin{center}\underline{\hspace{5 cm}}\end{center}
	 
Qu'est-ce qu'un homme complexe dit à une femme réelle?

Réponse: "viens danser!" (il faut lire "dans C", c'est-à-dire l'ensemble des complexes $\mathbb{C}$)

	\begin{center}\underline{\hspace{5 cm}}\end{center}
	
	\begin{center}
		\includegraphics{img/humour/rotation_matrix.jpg}	
	\end{center}
	
	\begin{center}\underline{\hspace{5 cm}}\end{center}

Phrases à double sens:

\begin{itemize}	 
	\item[$-$] NOus allons maintenant résoudre ce problème sans complexes

	\item[$-$] Un repère d'origine O (un repaire d'originaux...) 

	\item[$-$] Une partie de $\mathbb{Q}$ (a fucking party...)

	\item[$-$] Ne confondez pas un $\rho$ avec un $p$... (don't confuse between a fart and a burp)

	\item[$-$] Une variété de Poisson... (a variety of fish)
\end{itemize}

	\begin{center}\underline{\hspace{5 cm}}\end{center}

Les fonctions logarithme et exponentielle sont au restaurant. Quand viendra l'addition, qui payera?

Réponse: Exponentielle, car logarithme né paie rien.

Plus tard dans la nuit, Logarithme et Exponentielle rentrent chez eux un peu bourrés. Logarithme demande: Est-ce que je prends le volant?

Exponentielle répond: Je préfère que ce soit moi qui conduise. Au cas où on dérive...

	\begin{center}\underline{\hspace{5 cm}}\end{center}

Deux suites de Cauchy veulent sortir en boite. Elles arrivent devant une boite où se déroule la soirée "No Limit". Elles décident de rentrer, mais le vigile les arrête en leur disant: "Désolé, c'est complet!". (dans un espace complet, une suite de Cauchy est convergente par définition, donc elle a une limite.)

	\begin{center}\underline{\hspace{5 cm}}\end{center}

	\begin{center}
		\includegraphics{img/humour/socks.eps}	
	\end{center}
	\begin{center}\underline{\hspace{5 cm}}\end{center}	

Un mathématicien devint fou en pensait tout le temps qu'il était l'opérateur de différentation. Ses amis le placèrente dans un hôpital psychiatrique en attendant son rétablissement. Toutes les journées il les passait à dire aux autres patients de l'hôpital: "Je te dérive!".

Un jour, il rencontra un nouveau patient et il lui dit immédiatement: "Je te dérive!", mais pour une fois, sa victime ne réagissa nullement à son attaque verbale. Surpris, le mathématicien cria et avec toute son énergie: "Je te dérive!", mais l'autre patient continua à n'avoir aucune réaction. Finalement, après un grand ressentiment de frustration, le mathématicien cria une dernière fois: "JE TE DÉRIVE!".

Le nouveau patient se retourna calmement et lui dit: "Tu peux me différencier autant de fois que tu veux. Je suis l'exponentielle de $x$".

	\begin{center}\underline{\hspace{5 cm}}\end{center}	

Ce que les mathématiciens disent et ce qu'il faut comprendre:

\begin{itemize}	 
	\item[$-$] Trivial: Si je dois vous montrer ceci, vous êtes dans la mauvaise classe

	\item[$-$]  On peut trivialement montrer: On a pas besoin de plus de 4 heures pour le démontrer

	\item[$-$] Contrôlez vous-même: C'est la partie difficile de la démonstration donc vous pouvez le faire sur votre temps libre

	\item[$-$]  Similairement: Au moins une ligne de la démonstration est identique à la précédente

	\item[$-$]  Procédons formellement: Nous allons manipuler des symboles avec des règles bien prédéfinies sans rien comprendre au sens réel du résultat.

	\item[$-$]  Nous nous dispenserons de la démonstration: Faites-moi confiance, c'est vrai!

	\item[$-$]  Le lecteur montrera facilement: Ça m'ennuie de montrer que...

	\item[$-$]  Nous conseillons vivement au lecteur de faire les exercices indiqués: Comme je les ai pas faits, vous pourriez me les corriger

	\item[$-$]  J'ai montré ce résultat dans un papier antérieur: Je ne sais plus diable comment on fait pour prouver ce truc-là

	\item[$-$]  On généralise facilement: La généralisation dépasse mon niveau

	\item[$-$]  D'après une propriété bien connue: Par 10 personnes au monde..
\end{itemize}

\begin{center}\underline{\hspace{5 cm}}\end{center}
	
	\begin{center}
		\includegraphics[scale=0.9]{img/humour/fresh_men.jpg}	
	\end{center}
	
	\begin{center}\underline{\hspace{5 cm}}\end{center}	

Il y a 3 types de personnes: ceux qui peuvent compter et ceux qui ne peuvent pas compter ...

	\begin{center}\underline{\hspace{5 cm}}\end{center}	
	
Tout le monde connaît le "Théorème du Salaire" qui établit que les ingénieurs et les scientifiques ne peuvent JAMAIS gagner autant que les hommes d'affaires et les commerciaux. Ce théorème peut enfin se démontrer par la résolution d'une équation mathématique simple.

Notre équation s'appuie sur deux postulats très connus:
\begin{itemize}
	\item[P1.] La Connaissance c'est la Puissance
	\item[P2.] Le Temps c'est de l'Argent
\end{itemize}

Tout ingénieur sait ensuite que:

\begin{center}
$\text{Puissance}=\dfrac{\text{Travail}}{\text{Temps}}$
\end{center}

Puisque:

\begin{center}
$\text{Connaissance}=\text{Puissance}$
\end{center}

et que:

\begin{center}
$\text{Temps}=\text{Argent}$
\end{center}

Nous avons donc: 

\begin{center}
$\text{Connaissance}=\dfrac{\text{Travail}}{\text{Argent}}$
\end{center}

Nous obtenons alors facilement: 

\begin{center}
$\text{Argent}=\dfrac{\text{Travail}}{\text{Connaissance}}$
\end{center}

Ainsi quand la Connaissance tend vers zéro, l'Argent tend vers l'infini quelle que soit la valeur attribuée à Travail, cette valeur peut être très faible. À l'inverse quand la Connaissance tend vers l'infini, l'Argent tend alors vers zéro, même si la valeur Travail est élevée.

D'où la conclusion évidente suivante: Moins vous en connaissez, plus vous gagnez d'argent.

PS: Ceux d'entre vous qui ont eu quelques difficultés de compréhension doivent être les mieux rémunérés... 

	\begin{flushright}
		$\blacksquare$  Q.E.D.
	\end{flushright}

	\begin{center}\underline{\hspace{5 cm}}\end{center}
	
	\begin{center}
		\includegraphics{img/humour/proof_trivial.jpg}	
	\end{center}
	
	\begin{center}\underline{\hspace{5 cm}}\end{center}	
	
Dans le même genre voici le "Théorème du misogyne":

D'abord, nous déclarons que les filles sont des variables factorisables en quantités de temps et d'argent telle que:

\begin{center}
$\text{Filles}=\text{Temps}\times\text{Argent}$
\end{center}

Comme nous le savons tous "le temps c'est de l'argent!". Donc:

\begin{center}
$\text{Te}mps=\text{Argent}$
\end{center}

et parce que "l'argent est la racine du mal…" et que le mal est un synonyme du ce qui est mauvais:

\begin{center}
$\text{Argent}=\sqrt{\text{Mal}}$
\end{center}

Donc nous avons par substitution:

\begin{center}
$\text{Filles}=\left(\sqrt{\text{Mal}}\right)^2$
\end{center}

Nous sommes donc forcés de conclure:

\begin{center}
$\text{Filles}=\text{Mal}$
\end{center}

	\begin{flushright}
		$\blacksquare$  Q.E.D.
	\end{flushright}
	
	\begin{center}\underline{\hspace{5 cm}}\end{center}
	\begin{center}
		\includegraphics{img/humour/professor_xi.jpg}	
	\end{center}
	
	Les dix meilleures excuses pour ne pas faire vos devoirs de maths:
	
	\begin{itemize}
	
		\item[$\text{\#}10.$] Galilée ne connaissait pas l'algèbre; Pourquoi alors en aurais-je besoin?
	
		\item[$\text{\#}09.$]  Un drogué des maths m'a volé mes devoirs.
	
		\item[$\text{\#}08.$] Jai pris l'option physique et les devoirs semblent impliquer des maths, donc j'ai pensé que je pourrais juste faire cela à la place.
	
		\item[$\text{\#}07.$] J'ai la preuve, mais il n'y a pas de place pour l'écrire dans la marge.
	
		\item[$\text{\#}06.$] J'ai une calculatrice à énergie solaire et c'était nuageux.
	
		\item[$\text{\#}05.$] Je regardais les World Series et je me suis ligoté en essayant de prouver que ça convergeait.
	
		\item[$\text{\#}04.$] Je ne pouvais qu'être arbitrairement proche de mon manuel. (J'ai atteint la moitié du chemin, puis la moitié, et puis ...)
	
		\item[$\text{\#}03.$] I couldn't figure out whether i am the square root of negative one or i is the square root of negative one.
	
		\item[$\text{\#}02.$] C'était le jour de l'anniversaire d'Einstein et de pi et nous avons eu cette célébration! (Cela ne fonctionne que pour le 14 mars)
	
		\item[$\text{\#}01.$] J'ai accidentellement divisé par zéro et mon papier a pris feu.	
	\end{itemize}

	\begin{center}\underline{\hspace{5 cm}}\end{center}
	\begin{center}
		\includegraphics{img/humour/close.jpg}	
	\end{center}
	\begin{center}\underline{\hspace{5 cm}}\end{center}
	
	\pagebreak
	Quel est le résultat de:
	\begin{center}
		 $\dfrac{2ab}{2Fr.16}$\\
		(lire "2 abbés sur 2 françaises")
	\end{center} 

	Réponse: 
	\begin{center}
		$2bb \dfrac{a}{e}$\\
		(lire "2 bébés assurés")
	\end{center} 
	\begin{center}\underline{\hspace{5 cm}}\end{center}
	
	Quel est le résultat de: 
	\begin{center}
	 $\dfrac{\text{cheval}}{\text{oiseau}}$\\
	(lire "cheval sur oiseau")
	 \end{center} 
	
	Comme nous avons: 
	\begin{center}
	 $\dfrac{\text{cheval}}{\text{oiseau}}=\dfrac{\text{vache} \cdot \text{l}}{\beta \cdot \text{l}}$\\
	(lire "vache + l" (contient toutes les lettres de "cheval") divisé par "bête à ailes") 
	 \end{center} 
	
	Mais: 
	\begin{center}
	 $\dfrac{\text{vache} \cdot \text{l}}{\beta \cdot \text{l}}=\dfrac{\beta \cdot \pi \cdot \text{l}}{\beta \cdot \text{l}}$\\
	(lire "bête à pie + l" divisé par "bête à ailes")  
	 \end{center}
	
	Nous simplifions pour obtenir: 
	\begin{center}
	 $\dfrac{\cancel{\beta} \cdot \pi \cdot \cancel{\text{l}}}{\cancel{\beta} \cdot \cancel{\text{l}}}=\pi$  
	 \end{center}
	
	Ce qui prouve que $\pi$ est irrational parce qu'il n'y pas de comparaison rationelle entre un "cheval" et un "oiseau"... 
	
		\begin{center}\underline{\hspace{5 cm}}\end{center}

	\pagebreak
	À démontrer: 
	\begin{center}
	$\dfrac{\text{ROSSINI}}{\text{SOLSIDO}}=1$  
	\end{center} 
	
	Nous pouvons écrire:
	\begin{center}
	$\dfrac{\text{ROS SI NI}}{\text{SOL SI DO}}=\dfrac{\text{ROS NI}}{\text{SOL DO}}$  
	\end{center}
	
	mais NI$=$DO ("NI vaut DO" c'est-à-dire "niveau d'eau") donc: 
	\begin{center}
	$\dfrac{\text{ROS}}{\text{SOL}}$  
	\end{center}
	
	Mais SOL fait RINO (Solferino: c'est lors de cette bataille qu'Henri Dunant en voyant l'horreur a décidé de créer la Croix-Rouge) donc:
	\begin{center}
	$\dfrac{\text{ROS}}{\text{RI NO}}$  
	\end{center}
	
	et comme RINO c'est ROS (rhinocéros) donc RINO $=$ ROS alors:
	\begin{center}
	$\dfrac{\text{ROS}}{\text{ROS}}=1$  
	\end{center}

	\begin{flushright}
		$\blacksquare$  Q.E.D.
	\end{flushright}	

	\begin{center}\underline{\hspace{5 cm}}\end{center}
	\begin{center}
		\includegraphics[scale=0.6]{img/humour/day_of_an_eigenvector.jpg}	
	\end{center}
	Chaque futur ingénieur apprend à inscrire la somme de deux chiffres rationnels, par exemple:
	\begin{center}
	$1+1=2$  
	\end{center}
	
	Cette forme est cependant assez banale et indique des lacunes dans votre éducation.

	En premier semestre, on apprend que:
	\begin{center}
	$1=\ln(e)$  
	\end{center}
	
	et:
	\begin{center}
	$1=\sin^2(p)+\cos^2(q)$  
	\end{center}
	
	Tout le monde sait aussi que:
	\begin{gather*}
	2=\sum_{n=0}^{+\infty} \left( \dfrac{1}{2} \right)^n
	\end{gather*}
	
	et que donc l'équation:
	\begin{gather*}
	1+1=2
	\end{gather*}
	
	peut être écrite plus simplement:
	\begin{gather*}
	\ln(e)+\sin^2(p)+\cos^2(q)=\sum_{n=0}^{+\infty} \left( \dfrac{1}{2} \right)^n
	\end{gather*}
	
	(il faut admettre que l'aspect est bien plus clair et plus scientifique)

	D'autre part, il est évident que:
	\begin{gather*}
	1=\cosh(q)\sqrt{1-\tanh^2(q)}
	\end{gather*}
	
	et aussi:
	\begin{gather*}
	e=\lim_{z \rightarrow +\infty}\left(1+\dfrac{1}{z} \right) 
	\end{gather*}
	
	il en résulte que:
	\begin{gather*}
	\ln(e)+\sin^2(p)+\cos^2(q)=\sum_{n=0}^{+\infty} \left( \dfrac{1}{2} \right)^n
	\end{gather*}
	
	peut être réécrite comme suit:
	\begin{gather*}
	\ln\left( \lim_{z \rightarrow +\infty}\left(1+\dfrac{1}{z} \right)\right)+\sin^2(p)+\cos^2(q)=\sum_{n=0}^{\infty} \left( \dfrac{\cosh(q)\sqrt{1-\tanh^2(q)}}{2} \right)^n
	\end{gather*}
	
	Il faut également se rappeler que:
	\begin{gather*}
	0!=1
	\end{gather*}
	et que l'exposant inverse de l'exposant opposé est égal à l'exposant opposé de l'exposant inverse. En supposant un espace à $n$ dimension, on sait que:
	\begin{gather*}
	\left( X^T\right) ^{-1}-\left( X^{-1}\right) ^{-T}=0
	\end{gather*}
	Si l'on prend encore la matrice comme la métrique d'un espace canonique orthogonal et orienté:
	\begin{gather*}
	\left( g_{ij}^T\right) ^{-1}-\left( g_{ij}^{-1}\right) ^{-T}=0
	\end{gather*}
	
	Nous obtenons logiquement:
	\begin{gather*}
	\left(\left( g_{ij}^T\right) ^{-1}-\left( g_{ij}^{-1}\right) ^{-T}\right)!=1
	\end{gather*}
	Nous obtenons ainsi une expression simple et claire pour tout le monde de $1+1=2$:
	\begin{gather*}
	\ln\left( \lim_{z \rightarrow +\infty}\left(\left(\left( g_{ij}^T\right) ^{-1}-\left( g_{ij}^{-1}\right) ^{-T}\right)!+\dfrac{1}{z} \right)\right)+\sin^2(p)+\cos^2(q)=\sum_{n=0}^{+\infty} \left( \dfrac{\cosh(q)\sqrt{1-\tanh^2(q)}}{2} \right)^n
	\end{gather*}
	
	Il est donc évident que cette égalité est bien plus compréhensible que:
	\begin{gather*}
	1+1=2
	\end{gather*}
	Il serait possible de montrer plusieurs autres développements de cette simple expression et nous le ferons à partir du moment où vous commencerez à comprendre les principes simples de la méthode précédente.
	
	\begin{center}\underline{\hspace{5 cm}}\end{center}

	\begin{center}
		\includegraphics{img/humour/coloring_problem.jpg}	
	\end{center}

	Un catcheur, un physicien et un mathématicien sont sujet à une expérience: on les enferme dans une pièce avec chacun une boite d'épinards, fermée, et sans ouvre-boîte. Au bout de 24 heures, on va voir ce qu'ils sont devenus. 
	
	\begin{itemize}	 
		\item[$-$] Le catcheur a réussi à ouvrir sa boîte: « Et bien, j'ai simplement violemment projeté la boîte contre le mur. L'impact a été tel qu'elle s'est ouverte », explique t-il. 
	
		\item[$-$] Le physicien a également réussi à ouvrir sa boîte: « J'ai observé le solide, et distingué ses points de rupture. J'ai alors effectué une pression de manière à exercer une force maximale sur ceux-ci, et la boîte s'est tout naturellement ouverte. » 
	
		\item[$-$] Le mathématicien, enfin, est retrouvé prostré dans un coin de la pièce, la sueur ruisselant sur son visage, et sa boîte de conserve, fermée, entre les pieds: « Admettons que la boîte est ouverte... Admettons que...» 
	\end{itemize}
	\begin{center}\underline{\hspace{5 cm}}\end{center}
	\begin{center}
		\includegraphics[scale=0.9]{img/humour/math_useful.jpg}	
	\end{center}

	\begin{center}\underline{\hspace{5 cm}}\end{center}
	Mathématique de la vie:
	\begin{gather}
		\setlength{\tabcolsep}{1pt}
		\begin{tabular}{cccccc}
		& & \text{Vie} & + & \text{Amour}= & \text{Bonheur} \\
		$+$& & \text{Vie} & - & \text{Amour}= & \text{Malheur} \\ \hline
		&2\text{Vie}& & & = & \text{Bonheur}+\text{Malheur} \\
		\end{tabular}
	\end{gather}
	Ainsi:
	\begin{gather}
		\text{Vie}=\dfrac{\text{Bonheur}+\text{Malheur}}{2}
	\end{gather}
	En développant:
	\begin{gather}
		\text{Vie}=\dfrac{1}{2}\text{Bonheur}+\dfrac{1}{2}\text{Malheur}
	\end{gather}
	C'est ça la vraie vie. Profitez-en!
	
	\begin{center}\underline{\hspace{5 cm}}\end{center}
	
	An opinion without $3.14$ is an onion. You'll understand!
	
	\begin{center}\underline{\hspace{5 cm}}\end{center}

	\begin{center}
		\includegraphics[scale=0.4]{img/humour/math_man_sex.jpg}	
	\end{center}
	
	\begin{center}\underline{\hspace{5 cm}}\end{center}
	
	L'épouse d'un logicien accouche d'un bébé. Le médecin remet immédiatement le nouveau-né au père.

	Sa femme demande avec impatience: "Alors, est-ce un garçon ou une fille"?

	Le logicien répond: "Oui".

	\begin{center}\underline{\hspace{5 cm}}\end{center}
	
	Question: Que représente le "B" dans Benoit B. Mandelbrot?

	Réponse: Benoit B. Mandelbrot
	
	\begin{center}\underline{\hspace{5 cm}}\end{center}
	
	Ce n'est pas ainsi que nous faisons cette dérivée ...:
	\begin{gather}
		\dfrac{\mathrm{d}}{\mathrm{d}x}\dfrac{1}{x}=\dfrac{\mathrm{d}}{\mathrm{d}}\dfrac{1}{x^2}=\dfrac{\cancel{\mathrm{d}}}{\cancel{\mathrm{d}}}\dfrac{1}{x^2}=-\dfrac{1}{x^2}
	\end{gather}
	
	\begin{center}\underline{\hspace{5 cm}}\end{center}

	\begin{center}
		\includegraphics[scale=0.6]{img/humour/asymptote.jpg}	
	\end{center}
	
	\begin{center}\underline{\hspace{5 cm}}\end{center}
	
	\begin{center}
		\includegraphics[scale=0.6]{img/humour/pourcentages.jpg}	
	\end{center}

	\pagebreak
	\section{Physique}
	
	\begin{center}
	\includegraphics{img/humour/heisenberg.eps}
	\end{center}
	
	\begin{center}\underline{\hspace{5 cm}}\end{center}	
	
	\begin{itemize}	 
		\item[$-$] En théorie, il n'y a pas de différences entre la théorie et la pratique. En pratique, si!
	
		\item[$-$] La théorie, c'est quand on sait tout, mais que rien ne marche. La pratique, c'est quand tout marche, mais qu'on ne sait pas pourquoi. En informatique, la théorie et la pratique sont réunies: rien ne marche et on ne sait pas pourquoi!
	\end{itemize}

	\begin{center}\underline{\hspace{5 cm}}\end{center}
	
	\begin{itemize}	 
		\item[$-$]  La matière est fondamentalement fainéante: Elle prend toujours le chemin de moinde action.
	
		\item[$-$] La matière est fondamentalement stupide: Elle essaie tous les chemins d'abord.
		
		\item[$-$] Ceci est le coeur de la physique, le reste n'étant que détails!
	\end{itemize}
	
	\begin{center}\underline{\hspace{5 cm}}\end{center}
	
	Ce sont deux atomes qui se rencontrent.  

	L'un dit à l'autre: "Merde, j'ai perdu un électron!"

	L'autre: "T'es sûr?"

	Et le premier répond: "POSITIVEMENT !!"
	
	\begin{center}\underline{\hspace{5 cm}}\end{center}

	\begin{center}
	\includegraphics{img/humour/einstein.eps}
	\end{center}
	
	\begin{center}\underline{\hspace{5 cm}}\end{center}	
	
	Vous entrez dans un laboratoire et vous voyez une expérience. Comment allez-vous deviner de quelle section il s'agit?
	
	\begin{itemize}	 
		\item[$-$] Si c'est vert et visqueux, c'est la biologie.
	
		\item[$-$] Si ça pue, c'est la chimie.
	
		\item[$-$] Si cela ne marche pas, c'est la physique.
	\end{itemize}
	
	\begin{center}\underline{\hspace{5 cm}}\end{center}	

	Théorème: Un chat à neuf vies.

	Démonstration: Comme aucun chat à huit vies, tout chat à une vie de plus que pas de chat. Donc par extension un chat à neuf vies.
	
	\begin{center}\underline{\hspace{5 cm}}\end{center}

	\begin{center}
	\includegraphics[scale=0.75]{img/humour/schrodinger_cat.eps}
	\end{center}
	
	\begin{center}\underline{\hspace{5 cm}}\end{center}

	Un physicien étudiant la physique quantique, c'est quelqu'un qui ne voyant pas très bien, cherche dans une chambre obscure, un chat noir, qui probablement n'existe pas.
	
	\begin{center}\underline{\hspace{5 cm}}\end{center}

	Un ingénieur, un physicien, un mathématicien et un mystique sont interrogés sur la plus grande inventions de tous les temps.

\begin{itemize}	 
	\item[$-$]  L'ingénieur choisit le feu, qui donna à l'humanité le contrôle de la matière.

	\item[$-$] Le physicien choisit la roue, qui donne à l'humanité le contrôle de l'espace.

	\item[$-$] Le mathématicien choisit l'alphabet, qui donna à l'humanité le contrôle sur les symboles.

	\item[$-$] Le mystique choisit la bouteille thermos.

"Pourquoi la bouteille thermos?" lui demandèrent les trois autres?

	\item[$-$] Question à laquelle le mystique répondit: "Ben oui! Le bouteille thermos elle garde les liquides chauds quand il fait froid, et froids quand il fait chaud."

	\item[$-$] "Et... alors?" lui demandèrent les autres?

	\item[$-$] "Réfléchissez!" dit le mystique. "La petite bouteille, peut-elle savoir cela?".
	\end{itemize}
	
\begin{center}\underline{\hspace{5 cm}}\end{center}

	\begin{center}
	\includegraphics[scale=0.4]{img/humour/milkiway.jpg}
	\end{center}
\begin{center}\underline{\hspace{5 cm}}\end{center}

Un professeur de physique a effectué une expérience et a déterminé une équation empirique pour expliquer les données obtenues. Il demande alors à un professeur de maths d'y jeter un coup d'oeil.

Une semainee plus tard, le professeur de math lui dit que son équation n'est pas valide. Ce à quoi le physicien rétorque que pourtant cette dernière lui a permis de prévoir les autres expériences qu'il a fait entre temps et qu'il a obtenu des résultats excellents, alors il demande au mathématicien de vérifier à nouveau.

Une autre semaine passe, et les deux professeurs se rencontrent à nouveau. Le mathématicien admet alors que l'équation marche effectivement, mais "Seulement dans le cas trivial où les nombres sont réels et positifs".

\begin{center}\underline{\hspace{5 cm}}\end{center}
	
	Les centrales de fusion nucléaire utilisables ne le seront pas avant 30 ans et il en sera toujours ainsi!
	
	\pagebreak
	\begin{center}
	\includegraphics{img/humour/howscientistseeworld.eps}
	\end{center}
	
	\pagebreak
	
	Heisenberg conduisait une voiture sur l'autoroute lorsqu'un policier l'interpella.

	Le policier lui demanda:	
	\begin{itemize}	 
		\item[$-$] ""Savez-vous à quelle vitesse vous rouliez au moins?"
	\end{itemize}
	
	Ce à quoi Heisenberg répondit:
	
	\begin{itemize}	 
		\item[$-$] "Non mais je sais où je me trouve".
	\end{itemize}
	
	\begin{center}\underline{\hspace{5 cm}}\end{center}
	
	Quelle est la différence entre la mécanique automobile et la mécanique quantique?

	Réponse: La mécanique quantique peut parfois ranger la voiture dans le garage sans ouvrir le porte.
	
	\begin{center}\underline{\hspace{5 cm}}\end{center}
	
	\begin{center}
	Ce n'est pas le:
	
	\includegraphics{img/humour/kill_fall.jpg}
	\end{center}
	\begin{gather*}
		v_f=v_0+at
	\end{gather*}
	\begin{center}
	qui vous tue, c'est le:
	\end{center}
	\begin{gather*}
		F=m\dfrac{\Delta v}{\Delta t}
	\end{gather*}
	
	\begin{center}
	\includegraphics{img/humour/kill_final.jpg}
	\end{center}

	\begin{center}
	\includegraphics{img/humour/superstring.eps}
	\end{center}
	
\begin{center}\underline{\hspace{5 cm}}\end{center}
	
Pourquoi Dieu n'a jamais reçu un docteur honoris causa d'une quelconque Université:
\begin{enumerate}
	\item Il a seulement une unique publication majeure,

	\item Cette publication est en hébreu.

	\item Elle n'a aucune référence bibliographique.

	\item Elle n'a jamais été publiée dans une revue de référence.

	\item Il y a des doutes qu'il l'ait rédigée lui-même.

	\item  Il est peut-être vrai qu'il a créé le monde, mais qu'a-t-il fait depuis?

	\item Sa participation et sa coopération ont été relativement limitées.

	\item La communauté scientifique a une grande difficulté à reproduire ses résultats.

	\item Il n'a jamais appliquée la convention éthtique sur la manipulation d'êtres humains.

	\item Quand une personne tenta de tester ses résultats, il noya tout les lieux et sujet d'expérimentation.

	\item Quand ses résultats n'étaient pas conformes à ses attentes, il les supprima de l'échantillon.

	\item  Il a rarement suivi des classes et mais par contre n'a pas hésité à conseiller de lire un ouvrage.

	\item Certains disent que son fils et dans les classes.

	\item  Il a dispensé ses deux premiers élèves de tout enseignement.

	\item  Dans ses examens, il y avait toujours 10 critères et la majorité des ses étudiants y ont échoué.

	\item  Ses horaires de disponibilité sont irréguliers et il n'est jamais là quand on a besoin de lui.
\end{enumerate}

\begin{center}\underline{\hspace{5 cm}}\end{center}

	\begin{center}
		\includegraphics{img/humour/physics_gang_sign.jpg}
	\end{center}
	\pagebreak

Au tout début, il y avait Aristote:
\begin{itemize}
	\item Et les objets au repos tendaient à rester au repos
	\item Et les obets en mouvement tendaient à revenir au repos
	\item Dieu vit que c'était ennuyant et peu utile.
\end{itemize}

Alors Dieu créa Newton:
\begin{itemize}
	\item Et les objets au reps tendaient à rester au repos
	\item Et les objets au mouvement tendaient à rester en mouvement
	\item Et l'énergie ainsi que le moment étaient conservés
	\item Et la matière était conservée
	\item Dieu vit que cela était très conservatif.
\end{itemize}

Et Dieu créa Einstein:
\begin{itemize}
	\item Et tout était relatif
	\item Et les choses rapides devinrent petites
	\item Et les choses plates devinrent courbes
	\item Et l'Univers était rempli de référentiel inertiels
	\item AEt Dieu vit que cela était la relativité générale, mais une partie était particulièrement relative.
\end{itemize}

Alors Dieu créa Bohr:
\begin{itemize}
	\item Et il y eu LE principe
	\item Et le principe était quantifié
	\item Et tout était quantifié
	\item Mais certains choses restaient relatives
	\item Et Dieu dit que cela était confus.
\end{itemize}

Alors Dieu allait créer Furgeson:
\begin{itemize}
	\item Et Furgeson allait tout unifier
	\item Et il allait mette les théories sous forme de champs
	\item Et tout ne serait plus qu'un
	\item Mais c'était le 7ème jour
	\item Et Dieu se reposa
	\item Et ce qui est au repos tend à rester au repos...
\end{itemize}

	\pagebreak

	\begin{center}
		\includegraphics[scale=0.6]{img/humour/schrodinger_survey.jpg}	
	\end{center}

\begin{center}\underline{\hspace{5 cm}}\end{center}
\begin{center}
\textbf{La table des lois des physiciens}
\end{center}

Nous supposerons triviaux les postulats suivants comme quoi tous les physiciens sont nés égaux, en première approximation, et qu'ils ont été dotés par la créateur de quelques privilèges, avec une espérance de vie suffisante et $n$ degrés de liberté, et les lois suivantes qui sont invariantes pour toute transformation linéaire:
\begin{enumerate}
	\item Approximer tous les problèmes à des cas idéaux.

	\item Utiliser des ordres de grandeurs de calculs particuliers uniquement quand cela est nécessaire.

	\item Utiliser des simplifications dès que le problème étudié implique autre chose que des nombres entiers.

	\item Négliger ou considéréer comme non physique toutes les fonctions qui divergent.

	\item  Invoquer le principe d'incertitude quand confronté à des mathématiciens, chimistes, ingénieurs ou autres scientifiques confus.

	\item  Marmonner quand un non-physicien pose une question trivial impliquant des mathématiques élémentaires.

	\item Égaliser les deux côtés d'une équation qui sont dimensionnellement inconsistentes, avec une remarque du type "De toute façon, nous sommes seulement intéressés à un ordre de grandeur"....

	\item Utiliser des notations non traditionnelles où les conventions mathématiques d'usage ne fonctionnement pas.

	\item Inventer des forces fictives ou virtuelles pour brouillet le public. 

	\item Utiliser des raisonnements incorrects sur la base que cela donne le bon résultat.

	\item Utiliser toujours des conditions initiales utilisant le principe général de trivialité.

	\item Utiliser des arguments plausibles en lieu et place de preuves et de se référer à ses arguments comme étant des preuves.

	\item Prendre pour vrai tout principe qui semble correct mais que ne peut pas être démontré.
\end{enumerate}

	\begin{center}\underline{\hspace{5 cm}}\end{center}
		\begin{center}
		\includegraphics[scale=0.6]{img/humour/travelling_light.jpg}	
	\end{center}

\pagebreak
Vous pouvez utiliser ce petit quiz pour vérifier si vous êtes un vrai scientifique ou non.

\begin{enumerate}
	\item À noël vous: 
	\begin{enumerate}
		\item[a.] Prenez quelques jours pour vous passer du temps avec votre famille.
		\item[b.] Vous partez un peu plus tôt du travail pour prendre le temps d'achter des cadeaux à votre famille. 
		\item[c.] Vous travaillez seulement une demi journée au bureau, passant le reste de la journée à travailler à la maison.
	\end{enumerate}

	\item Votre épouse souhaite discuter des plans pour les prochaines vacances. Vous:
	\begin{enumerate}
		\item[a.] Proposez du camping ce qui vous permettre d'expliquer la beauté de la nature aux enfants.
		\item[b.] Vous proposez d'aller dans une autre ville comme cela vous pouvez allez voir un autre collègue dans son laboratoire pendant que votre épouse surveille les enfants.
		\item[c.] Vous demandez à votre femme: "Nous avons des enfants?"
	\end{enumerate}

	\item À un colloque scientifique dans une île du pacifique sud, aucune réunion n'est prévue pendant un après-midi. Pendant ce temps libre, vous: 
	\begin{enumerate}
		\item[a.] Suiviez la tradition locale et prenez un bain nu sur la plage.
		\item[b.] Prenez place sur la plage complétement habillé, désintéressé de la vue des baigneuses nues et parlez sciences avec vos collègues.
		\item[c.] Allez dans votre chambre d'hôtel avec les rideaux fermées pour travailler sur votre prochain article.
	\end{enumerate}

	\item La maîtresse d'école vous appelle en disant que votre enfant à la varicielle. Vous:
	\begin{enumerate}
		\item[a.]  Stoppez immédiatement tout ce que vous êtes en train de faire et partez immédiatement pour l'école afin de récuperer votre enfant. 
		\item[b.] Stoppez immédiatement tout ce que vous êtes en train de faire et cherchez un traitement contre la varicelle.
		\item[c.] Vous demandez à la maîtresse la route pour l'école et le nom de vos enfants.
	\end{enumerate}

	\item Des êtres d'une autre planète visitent la Terre et vous êtes le premier humain qu'ils rencontrent. Pour vous montrer leur pacificité, ils vous offrent un présent capable de prolonger la vie, arrêter la suffrance humaine mondiale et éradiquer la faim dans le monde. Vous: 
	\begin{enumerate}
		\item[a.] Présentez le cadeau aux Nations Unies.
		\item[b.] Vous posez un brevet.
		\item[c.] Vous le cassez pour voir comment il fonctionne à l'intérieur.
	\end{enumerate}

	\item Quelle est la plus grande période de temps pendant laquelle vous n'avez pas pris de vacances (en excluant les colloques scientifiques...)? 
	\begin{enumerate}
		\item[a.] Six mois.
		\item[b.] Deux ans.
		\item[c.] Vous avez pris un week-end il y a environ 10 ans.
	\end{enumerate}

	\item Quels sont vos loisirs? 
	\begin{enumerate}
		\item[a.] Sport, musique, danse parce qu'ils permettent à la partie analytique de mon cerveau de se reposer.
		\item[b.] Cuisiner, parce que c'est un peu comme une science.
		\item[c.] Relire les anomalies dans les revues scientifiques, numéro par numéro. 
	\end{enumerate}

	\item Votre meilleur ami est: 
	\begin{enumerate}
		\item[a.] Un membre de votre entourage.
		\item[b.] Un membre de votre famille.
		\item[c.] Un membre de la même race que la votre.
	\end{enumerate}
\end{enumerate}

Score:

Donnez-vous $1$ un point pour chaque question répondu par un "a", $5$ points pour chaque "b" et pour chaque "c". Si vous avez refait le test 3 fois et que vous avez effectué une moyenne de vos points, rajoutez $100$ points en extra. Si vous avez calculé l'erreur standard de la moyenne, rajoutez-vous $500$ points en extra.

Si vous avez un score inférieur à $10$, vous êtes normal. Un score entre $11$ et $50$ indique une obsession scientifique et vous devriez penser à joindre les \textit{Scientifiques Anonymes}. Si votre score est supérieur à $500$, oubliez les \textit{Scientifiques Anonymes} et retournez au travail car il est impossible de faire quoi que ce soit pour vous, de plus vous allez très probablement réussir en tant que scientifique.

\begin{center}\underline{\hspace{5 cm}}\end{center}

	\begin{center}
	\includegraphics{img/humour/cow.jpg}
	\end{center}
	
\pagebreak

Comment il faut comprendre certaines phrases dans les publications des physiciens:

\begin{itemize}
	\item Il est bien établi que...: Je ne me suis pas donné la peine de lire les références, mais...

	\item Ceci est de grande importance théorique: Ceci est important pour moi.

	\item Quoiqu'il n'ait pas été possible de donner une réponse définitive: L'expérience échoua, mais il me semble tout de même pouvoir en tirer une publication.

	\item La technique utilisée fut particulièrement adéquate...: Le copain du labo d'à côté avait déjà mis la technique au point.

	\item  3 échantillons furent choisis pour une étude exhaustive: Les résultats obtenus à partir des autres échantillons n'ont rien donné de cohérent .

	\item  Manipulé avec la plus grande précaution durant toute l'expérimentation: Ne fut pas jeté à l'égout.

	\item  La concordance avec la théorie est excellente: elle est passable.

	\item  La concordance avec la théorie est bonne: elle est faible.

	\item  La concordance avec la théorie est satisfaisante: elle est douteuse.

	\item  La concordance avec la théorie est passable: elle est totalement imaginaire.

	\item  Il est généralement admis que...: 2 collègues pensent comme moi

	\item  Il est admis que: Je crois que.

	\item  Il est évident que des travaux complémentaires seront utiles: Je n'ai rien compris.

	\item  Voici quelques résultats typiques: Voici les meilleurs résultats.

	\item  Significatif dans un intervalle de confiance de...: non significatif.

	\item  Les réactifs utilisés furent synthétisés au laboratoire selon des techniques standardisées: Les réactifs furent achetés chez...

	\item  Malheureusement, les bases quantitatives permettant de tirer profit des résultats n'ont pas encore été formulées: Personne n'est arrivé à comprendre quoi que ce soit à ce qui a été observé.

	\item  Nous remercions X pour sa précieuse collaboration et Y pour les discussions fructueuses: X a fait le travail et Y m'a expliqué ce que signifiaient les résultats.
\end{itemize}

	\begin{center}\underline{\hspace{5 cm}}\end{center}

	\begin{center}
	\includegraphics{img/humour/duality.eps}
	\end{center}
	
	\begin{center}\underline{\hspace{5 cm}}\end{center}
	An experimental physicist performs an experiment involving tow cats, and an inclined tin roof.
	
	The two cats are very nearly identical; same sex, age, weight, breed, eye and hair colour.
	
	The physicist places both cats on the roof at the same height and lets them both go at the same time.
	
	One of the cats fall off the roof first so obviously there is some difference between the two cats.
	
	What is the difference?
	
	One cat has a greater mew!
	
	\begin{center}\underline{\hspace{5 cm}}\end{center}
	
	\begin{center}
	\includegraphics{img/humour/three_body_problem.jpg}
	\end{center}
	
	\begin{center}\underline{\hspace{5 cm}}\end{center}
	
	\begin{center}
	\includegraphics[scale=0.85]{img/humour/bus_stop_physicists.jpg}
	\end{center}
	
	\begin{center}\underline{\hspace{5 cm}}\end{center}
	
	\begin{center}
		This is how physicists see the Pokemon:
		\includegraphics[scale=0.8]{img/humour/pokemon.jpg}
	\end{center}
	
	\begin{center}
	\includegraphics[scale=0.55]{img/humour/feynman_diagrams.jpg}
	\end{center}
	
	\begin{center}\underline{\hspace{5 cm}}\end{center}
	
	\begin{center}
	\includegraphics[scale=0.55]{img/humour/cat_physics.jpg}
	\end{center}
	
	\begin{center}\underline{\hspace{5 cm}}\end{center}
	
	\begin{center}
	\includegraphics[scale=0.7]{img/humour/sun_sohn.jpg}
	\end{center}
	
	\begin{center}\underline{\hspace{5 cm}}\end{center}
	
	\begin{center}
	\includegraphics[scale=1]{img/humour/bad_good_news_schrodinger_cat.jpg}
	\end{center}
	

	\pagebreak
	\section{Statistiques}

Trois statisticiens sortent faire du tir sur cible statique. Le premier statisticien fait feu et tire trop à gauche, le deuxième fait feu, mais tire symétriquement trop à droite. Enfin, le dernier ne tire pas, mais s'exprime triomphalement: "En moyenne nous l'avons eu!"

\begin{center}\underline{\hspace{5 cm}}\end{center}

	\begin{center}
	\includegraphics[scale=0.7]{img/humour/correlation.jpg}
	\end{center}

\begin{center}\underline{\hspace{5 cm}}\end{center}

Patient: "Vais-je survivre à cette opération délicate?"

Chirurgien: "Oui, je suis absolument sûr que vous y survivrez."

Patient: "Comment pouvez-vous être si sûr?"

Chirurgien: "9 patients sur 10 meurent lors de cette opération mon neuvième patient est mort hier!"

\begin{center}\underline{\hspace{5 cm}}\end{center}

	\begin{center}
	\includegraphics[scale=0.6]{img/humour/gauss.eps}
	\end{center}

\pagebreak
$10$ bonnes raisons pour s'orienter dans le domaine de la statistique:
\begin{enumerate}
	\item Estimer des paramètres est plus simple que de se battre dans la vraie vie

	\item Les statisticiens sont des gens reconnus

	\item Vous apprendrez l'alphabet grec en entier

	\item La probabilité que vous obteniez un job dans ce domaine est $> 0.9999$

	\item Si vous êtes virés, vous pourrez toujours vous reconvertir à l'ingénierie

	\item Vous faites ce travail dans la confidence, la régularité et la variabilité

	\item Vous êtes normal et le reste du monde est faux

	\item La ligne de régression paraît meilleure que la ligne du chômage

	\item Vous n'avez jamais besoin d'être exact - seulement approximatif

	\item Personne ne comprend ce que vous faites, alors vous avez toujours raison
\end{enumerate} 

	\begin{center}\underline{\hspace{5 cm}}\end{center}
	\begin{center}
	\includegraphics{img/humour/statistician.eps}
	\end{center}
	
	\begin{center}\underline{\hspace{5 cm}}\end{center}		
	\begin{center}
	\includegraphics[scale=0.9]{img/humour/bayesian_inference.jpg}
	\end{center}
	
	\begin{center}\underline{\hspace{5 cm}}\end{center}
	\begin{center}
	\includegraphics[scale=0.5]{img/humour/normal_distribution.jpg}
	\end{center}
	
	\pagebreak
	Un statisticien et un biologiste sont condamnés à mort. On leur accorde uen dernière faveur.
	\begin{itemize}
		\item Je voudrais donner une grande conférence sur la statistique devant tout le monde, dit le statisticien
		\item Accordé! Répond le juge. Et pour vous?
	\end{itemize}
	Le biologiste n'exprime aucune hésitation:
	\begin{itemize}
		\item Je souhaiterais être exécuté le premier!
	\end{itemize}
	\begin{center}\underline{\hspace{5 cm}}\end{center}
	\begin{center}
	\includegraphics{img/humour/bedtime_stories.jpg}
	\end{center}
		
	\pagebreak
	\section{Chimie}

On vient de découvrir un nouvel élément chimique:

\begin{itemize}
	\item[$\bullet$] ÉLÉMENT NUMÉRO: $115$

	\item[$\bullet$] NOM: Femme

	\item[$\bullet$] SYMBOLE: Fm

	\item[$\bullet$] MASSE ATOMIQUE: Acceptable 60 kg mais des isotopes connus de $40$ à $250$ kg

	\item[$\bullet$] OCCURRENCE: Très abondant de par le monde

	\item[$\bullet$] PROPRIÉTÉS PHYSIQUES:

- Entre en ébullition pour un rien et gèle sans raison

- Conductivité thermique: faible surtout aux extrémités inférieures

- Coefficient de dilatation: augmente avec les années

- Cède aux pressions appliquées aux points sensibles

	\item[$\bullet$] STRUCTURE MOLÉCULAIRE:

Parfaite? 90/60/90, existe aux USA sous forme croissante 60/90/120 et dans les pays nordiques sous forme dite plate 50/50/50

	\item[$\bullet$] PROPRIÉTÉS CHIMIQUES:

- Très grande affinité pour l'or, l'argent, le platine et tous les métaux nobles. Absorbe de grandes quantités de substances onéreuses

- Peut exploser spontanément sans avertissement

- Insoluble dans les liquides mais présente une activité grandement augmentée par saturation dans l'alcool

- Réactivité très variable selon les périodes de la journée

- Grande aptitude aux changements d'humeur et à la jalousie.

- Sensible à certaines contraintes qui lui transmettent parfois la migraine

	\item[$\bullet$] UTILISATIONS COURANTES:

- Hautement décorative surtout dans les voitures de sport

- Puissant agent nettoyant

- Aide efficace pour la relaxation et la détente

	\item[$\bullet$] TEST: 

- L'élément pur passe parfois au rose quand heureux.

- Tourne au vert si placée à côté d'un spécimen de meilleure qualité

	\item[$\bullet$] PRÉCAUTIONS D'EMPLOI:

- Hautement dangereuse si placée entre des mains non expertes

- Il est illégal d'en posséder plus d'un spécimen, mais il est possible d'en entretenir plusieurs à des endroits différents tant que les différents spécimens n'entrent pas en contact (risque d'explosion)

\end{itemize}

ATTENTION: 

Certains chercheurs d'Amérique du Sud ont découvert le moyen d'en fabriquer artificiellement, présentées généralement sous les marques "Travelo" ou "Dragqueen". Ne consommer que le produit générique.

	\begin{center}\underline{\hspace{5 cm}}\end{center}

	\begin{center}
	\includegraphics[scale=0.5]{img/humour/thorium.jpg}
	\end{center}
	
	\begin{center}\underline{\hspace{5 cm}}\end{center}

On vient de découvrir un nouvel élément chimique:

\begin{itemize}
	\item[$\bullet$] ÉLÉMENT NUMÉRO: 116

	\item[$\bullet$] NOM: Homme

	\item[$\bullet$] SYMBOLE: Hm

	\item[$\bullet$] ANALYSE QUANTITATIVE:	

Mesuré à 17 [cm], bien que quelques isotopes existent en 25, 20, 13 et même 10 [cm].

	\item[$\bullet$] DÉCOUVREUR:

Éve (découvert par accident un jour où elle avait envie de côtelettes)

	\item[$\bullet$] LIEU D'EXTRACTION:	
	
Se trouve en grandes quantités en présence d'un gisement de Fm très pur

	\item[$\bullet$] PROPRIÉTÉS PHYSIQUES:

- Surface recouverte de poils, raides par endroits, doux dans d'autres

- Bout quand on l'agite, se glace quand on le met en présence de la logique et du bon sens, se liquéfie quand on le traite comme un dieu

- Devient exécrable lorsqu'on le mélange à n'importe quel alcool

- Peut être la cause de maux de tête (ou des maux d'autres parties du corps); à manipuler avec précaution

- Diminue son entropie directement après sa réaction avec l'élément Fm (état se manifestant par des ronflements... zzzzz)

- Augmente sa masse considérablement en vieillissant, perd de ses capacités réactionnelles

- Se déshydrate rapidement par temps sec

- Rarement trouvé à l'état pur après 14 ans

- Possède souvent un attachement inexplicable à sa roche mère, rendant l'extraction difficile

- Si on le met sous pression, devient trop dur et improductif; n'est productif que si l'on utilise la subtilité, les subterfuges, et la flatterie

	\item[$\bullet$] PROPRIÉTÉS CHIMIQUES:

- Tendance très forte à réagir avec l'élément Fm, même si la réaction est parfois endothermique

- Réputé être le meilleur catalyseur pour les réactions de transformation de l'élément Fm

- Possède la faculté d'entrer en réaction avec à peu près n'importe quoi

- En cas de réaction importante, l'aspect de l'élément change pour virer au rouge cramoisi.

- S'il est saturé en alcool, il devient inerte et repoussant pour la plupart des éléments

- Ne convient pas pour les tâches ménagères et les opérations de nettoyage

- Ne convient pas non plus pour les tâches familiales

- Est neutre en ce qui concerne la courtoisie et l'impartialité

	\item[$\bullet$] USAGES COURANTS:

- Transport de choses lourdes, chauffeur, dîners gratuits au restaurant...

- Usage possible pour les activités sexuelles

	\item[$\bullet$] TESTS:

Les spécimens les plus purs ne sont pas synonymes de pureté, et ceux qui ont déjà servi, encore moins

	\item[$\bullet$] DANGERS:

La réaction avec un autre élément Hm est extrêmement violente si l'élément Fm est le catalyseur
\end{itemize}

\begin{center}\underline{\hspace{5 cm}}\end{center}

La question suivante a réellement été posée en ces termes à l'université de chimie de Washington:

L'Enfer est-il exothermique (dégage-t-il de la chaleur) ou endothermique (absorbe-t-il de la chaleur)? Appuyez votre réponse avec une preuve.

La plupart des étudiants écrivirent comme preuve de leurs théories la loi de Boyle (les gaz se réchauffent quand ils sont comprimés et se refroidissent quand ils se décompriment) ou une variante.

Un étudiant, toutefois, a écrit ce qui suit: 

Premièrement, nous avons besoin de savoir comment la masse de l'Enfer évolue dans le temps. Ce qui signifie aussi que nous avons besoin de connaître le rythme auquel les âmes vont en Enfer et le rythme auquel elles en sortent. Je pense que nous pouvons sans crainte affirmer qu'une fois qu'une âme est en Enfer, elle n'en sortira plus. Par conséquent, aucune âme ne sort des enfers.

Pour ce qui est des nombreuses âmes qui vont en Enfer, examinons les différentes religions qui existent de par le monde aujourd'hui. Certaines d'entre elles décrètent que si vous n'êtes pas membre de leur religion, vous irez en Enfer. Depuis qu'il y a plus d'une religion de cette sorte et depuis que les gens ne pratiquent qu'une seule religion, nous pouvons en déduire que presque tout le monde et toutes les âmes vont en Enfer.

Avec le rythme des naissances et des morts qui sont ce qu'ils sont, nous pouvons nous attendre à ce que le nombre des âmes en Enfer augmente de façon exponentielle.

Maintenant occupons-nous du rythme d'évolution du volume de l'Enfer, parce que la loi de Boyle prédit que pour que la température et la pression restent les mêmes, le volume de l'Enfer doit s'agrandir proportionnellement aux âmes qui s'ajoutent.

Ceci nous donne deux possibilités:

\begin{enumerate}
	\item Si l'Enfer croît a un rythme plus lent que celui des âmes qui arrivent en Enfer, alors la température et la pression s'accroissent jusqu'à ce que l'Enfer craque de partout.

	\item Bien sûr, si l'Enfer s'agrandit à un rythme plus rapide que le nombre d'âmes en Enfer s'accroît, alors la pression et la température baissent jusqu'a ce que l'Enfer gèle tout entier.
\end{enumerate}

\begin{center}\underline{\hspace{5 cm}}\end{center}

Un chimiste entre dans un pharmacie et demande à son pharmacien "Avez-vous de l'acide acetylsalicylic?"


\begin{itemize}
	\item[$-$] "Vous voulez parler d'aspirine?" Demanda le pharmacien.

	\item[$-$] "Oui c'est cela! Je n'arrive jamais à m'en rappeler le nom."
\end{itemize}

\begin{center}\underline{\hspace{5 cm}}\end{center}

Un physicien, un biologiste et un chimiste se rendent à l'océan pour la première fois. 

\begin{itemize}
	\item Le physicien en voyant les vagues de l'océan fut fasciné par les vagues. Il désira faire quelques recherche sur la dynamique des fluides et marcha dans l'océan. Malheureusement, il se noya et ne revint jamais. 

	\item Le biologiste désira faire quelques recherches sur la faune et la flore marine et marcha lui aussi dans l'océan. Il ne revint jamais non plus.

	\item Le chimiste attenda lui pendant un très long moment, notant ses observations: "Les physiciens et les chimistes sont solubles dans l'eau".
\end{itemize}

\begin{center}\underline{\hspace{5 cm}}\end{center}

CLASSIFICATION DE LA CHIMIE

\begin{itemize}
	\item \textit{Chimie physique}: L'art d'appliquer $y=mx+b$ à tout et n'importe quoi dans l'univers.

	\item \textit{Chimie organique}: L'art de transmuter des substances en publications.

	\item \textit{Chimie inorganique}: C'est ce qui reste après le chimie organique, analytique et physique une fois que l'on a utilisé tout le tableau périodique des éléments.

	\item \textit{Ingénierie chimique}:  L'art de tirer un profit de ce que le chimiste organique fait seulement pour le fun.
\end{itemize}
\begin{center}\underline{\hspace{5 cm}}\end{center}

\begin{center}
\includegraphics[scale=0.7]{img/humour/cute.jpg}
\end{center}

\begin{center}\underline{\hspace{5 cm}}\end{center}

Les radicaux libres ont révolutionné la chimie!

\begin{center}\underline{\hspace{5 cm}}\end{center}

Les derniers mots des chimistes: 

\begin{itemize}
	\item Et maintenant la phase de goûtage...

	\item  Et maintenant secouons un petit peu

	\item  Dans quelle bouteille était déjà mon eau minérale?

	\item  Pourquoi ce truc fait une flamme verte?!?

	\item  Et maintenant le problème de détonation du gaz.

	\item  Ceci est une expérience complétement sûre.

	\item  Maintenant vous pouvez retirer la fenêtre de protection...

	\item  D'où viennent tous ces petits trous sur ma blouse de travail?

	\item  Et maintenant un cigarette...
\end{itemize}

	\pagebreak
	\section{Ingénierie}

	Des scientifiques de la NASA ont développé un fusil spécialement conçu pour projeter des poulets morts dans les pare-brises des avions de ligne, des jets militaires et des navettes spatiales. Le but étant de vérifier les conséquences de possibles collisions avec des volatiles et d'adapter les matériaux des pare-brises en conséquence.

	Les ingénieurs britanniques ayant entendus parler de ce genre de fusil ont de suite développés un identique pour effectuer le même type de tests sur leurs trains à très grande vitesse. Lorsque le premier test fut effectué, le poulet traversa le pare-brise, la console de commande, la motorisation pour finir sa course dans le mur opposé. 

	Horrifiés et étonnés par ces résultats, les ingénieurs britanniques les communiquèrent  à leurs confrères américains afin d'obtenir des suggestions de leur part. La NASA leur répondit en une seule phrase: "Décongelez le poulet avant!"

	\begin{center}\underline{\hspace{5 cm}}\end{center}
	
	\begin{center}
	\includegraphics[scale=0.3]{img/humour/great_power_great_bills.jpg}
	\end{center}

	\begin{center}\underline{\hspace{5 cm}}\end{center}

Deux ingénieurs et un ami non-ingénieurs se rencontrent à un bar un vendredi soir pour raconter leur semaine de travail.

	\begin{itemize}
		\item Le premier ingénieur: "J'ai passé un semaine horrible à faire des plans un à la fois chaque jour."
	
		\item Le deuxième ingénieur: "J'ai fait un peu moins pire. J'ai au moins pu faire des plans complets plusieurs fois par jour."
	
		\item Le troisième ami non-ingénieur: "Ben les gars vous en avez de la chance! Moi je me limite à des plans culs qu'une fois par mois".
	\end{itemize}
	\begin{center}\underline{\hspace{5 cm}}\end{center}

	\begin{center}
	\includegraphics{img/humour/acdc.jpg}
	\end{center}

	\begin{center}\underline{\hspace{5 cm}}\end{center}

Tentatives pour comprendre les ingénieurs:

\begin{itemize}

	\item Tentative N\degree 1

	Deux élèves ingénieurs marchent le long de leur campus lorsque l'un des deux dit à l'autre, admiratif: "Où est-ce que tu as trouvé ce vélo ?" Le second lui répond: "Ben en fait, alors que je marchais, hier, et que j'étais dans mes pensées, je croise une super nana en vélo qui s'arrête devant moi, pose son vélo par terre, se déshabille entièrement et me dit: "Prends ce que tu veux... ". J'ai donc choisi son vélo. 
	
	Le premier lui répond: "Tu as raison, les vêtements auraient certainement été trop serrés.

	\item Tentative N\degree 2 

	Pour une personne optimiste, le verre est à moitié plein.
	Pour une personne pessimiste, il est à moitié vide.
	Pour l'ingénieur, il est deux fois plus grand que nécessaire.

	\item Tentative N\degree 3 

	Un pasteur, un médecin et un ingénieur jouent au golf. Ils attendent après un groupe de golfeurs particulièrement lents. Au bout d'un moment, l'ingénieur explose et dit: "Mais qu'est-ce qu'ils fichent? ça fait bien un quart d'heure qu'on attend là !" Le docteur intervient, exaspéré lui aussi: "Je ne sais pas, mais je n'ai jamais vu des gens s'y prendre aussi mal !" Le pasteur dit alors: "Attendez, voilà quelqu'un du golf. On n'a qu'à le lui demander. Dites-moi, il y a un problème avec le groupe de devant. Ils sont plutôt lents, non ?" L'autre répond: "Ah oui, c'est un groupe de pompiers aveugles. Ils ont perdu la vue en tentant de sauver le golf des flammes l'année dernière, alors depuis, on les laisse jouer gratuitement". Le groupe reste silencieux un moment, et le pasteur dit: "C'est si triste. Je vais faire une prière spécialement pour eux ce soir". Le médecin ajoute: "Bonne idée. Et moi, je vais contacter un copain chercheur ophtalmologiste pour voir ce qu'il peut faire". Et l'ingénieur: "Mais putain ! Pourquoi ils jouent pas la nuit ?"

	\item Tentative N\degree 4 

	Un ingénieur traversait la rue lorsqu'une grenouille l'appela et lui dit: "Si tu m'embrasses, je me transformerai en une magnifique princesse". Il se baissa, ramassa la grenouille et la mit dans sa poche. La grenouille lui dit alors: "Si tu m'embrasses, je me transformerai en une magnifique princesse et je resterai à tes côtés pendant une semaine". L'ingénieur sortit la grenouille de sa poche, lui fit un sourire et la replaça dans sa poche. La grenouille se mit alors à crier: "Si tu m'embrasses, je me transformerai en une magnifique princesse, je resterai à tes côtés pendant une semaine et je ferai TOUT ce que tu veux". Encore une fois, l'ingénieur sortit la grenouille de sa poche, lui sourit et la remit dans sa poche. La grenouille lui demanda alors: " Quoi, qu'est-ce qu'il y a ? Je te dis que je suis une magnifique princesse, que je resterai à tes côtés pendant une semaine et que je ferai tout ce que tu veux. Alors pourquoi tu ne m'embrasses pas ?" L'ingénieur répondit: "Regarde-moi, je suis un ingénieur. J'ai pas le temps d'avoir une petite amie. Par contre, une grenouille qui parle, ça, c'est cool!"

	\item Tentative N\degree 5 

	Un journaliste interviewe un paysan corse: "Dites-moi, comment faites-vous pour tracer les routes ici? ". Le paysan répond: "beh, on lâche un âne et on regarde par où il passe dans la montagne....et c'est là qu'on fait passer la route". Le journaliste alors rétorque: "et si vous n'avez pas d'âne?". Ce a quoi le paysan répond: "ah....beh on prend un ingénieur.... ".
\end{itemize}

	\begin{center}\underline{\hspace{5 cm}}\end{center}
	
	\begin{center}
	\includegraphics[scale=0.25]{img/humour/weather_forecast.jpg}
	\end{center}
		
	Lors de la course à la conquête spatiale dans les années 1960, la NASA décida pour le besoin de ses astronautes de développer un stylo à bille fonctionnant dans un système sans gravité.

	Après un temps considérable de recherche et de développements, le stylo "Astronaute" fut développé pour un coût de 1 million de dollars. Le stylo fonctionna très bien mais trouva un faible intérêt de la part du public sur Terre.

	L'Union Sovétique, se trouvant confronté au même problème, utilisa elle un crayon...

	\begin{center}\underline{\hspace{5 cm}}\end{center}
	
	Le grand mathématicien John Von Neumann fut consulté par un groupe qui fabriquait une fusée destinée à être envoyée dans l'espace. Quand il vit la structure actuelle, il demanda: "Où avez-vous obtenu les plans pour cette fusée?".

	On lui donna comme réponse: "Nous avons notre équipe d'ingénieur"

	Il répliqua alors désolé: "Des ingénieurs! Mais pourquoi? J'ai déjà publié toute la théorie mathématique sur la science des fusées. Voyez ma pubication de 1952.".

	Alors, le groupe consultat son document de 1952 et investirent 10 millions de dollar dans la structure et reconstruisirent toute la fusée conformément aux plans de Von Neumann. À la minute du lancement, la structure entière se désagréga. En colère, ils contactères alors Von Neumann et lui dirent: "Nous avons suivi vos instructions et quand nous avons démarré l'engin, il a explosé! Pourquoi?".

	Ce à quoi Von Neumann répliqua: "Ah oui! C'est un phénomène connu sous le nom technique du problème d'explosion - Je l'ai traité dans mon article de 1954."

	\begin{center}\underline{\hspace{5 cm}}\end{center}
	
	Dans un laboratoire d'électronique:
	
	\begin{itemize}
		\item Dis-donc, c'est quoi le semi-remorque sur le parking devant?
	
		\item Le semi-remorque?
	
		\item Ben oui, le chauffeur dit que tu es au courant...
	
		\item Ah oui!! J'ai commandé un condensateur de 1 Farad.
	\end{itemize}

	\begin{center}\underline{\hspace{5 cm}}\end{center}
	
	\begin{figure}[H]
		\begin{center}
		\includegraphics[scale=0.2]{img/humour/iso.jpg}
		\end{center}	
	\end{figure}
	
	\begin{center}\underline{\hspace{5 cm}}\end{center}

	Ce que les ingénieurs disent et ce qu'ils pensent en réalité:

	\begin{itemize} 
		\item Avancée technologique majeure: Retour au tableau noir

		\item Développé après des années de recherche intensives: Découvert par accident

		\item Le design est compris dans les limites imposées: Nous venons de le faire en retouchant un peu

		\item Les résultats ont été extrêmement satisfaisants: Par surprise cela cela marché!

		\item La satisfaction du client sera certainement assurée: Nous somme si loin dans des délais que le client était content de ne rien avoir du tout.

		\item Nous avons une forte coordination dans le projet: Nous aurions dû demander à quelqu'un d'autre

		\item Le projet est un peu hors délai à caus d'imprévus: Nous travaillons sur autre chose

		\item Le design sera finalisé dans la prochaine période de rapport: Nous n'avons pas encore commencé le travail mais nous devions dire quelque chose

		\item Un certain nombre d'approches sont tentées: Nous ne savons pas où nous allons, mais nous avancons...

		\item Des efforts intensifs sont mis en oeuvre pour avoir une nouvelle perspective sur le problème: Nous venons de licencié trois gars

		\item Les tests opérationnels préliminaires ne sont pas concluants: Le truc a explosé quand nous l'avons testé

		\item Le concept entier a été abandonné: Le seul gars qui comprenait le concept est parti ailleurs

		\item Des modifications sont en cours pour corriger quelques difficultés mineures: Nous avons jeté le tout et recommancé depuis le début.
		
		\item Nous sommes presque au bout: Nous avons fait la moitié

		\item Nous avons prévu: Nous croyons en Dieu!

		\item Les dessins techniques sont un peu en retard: Nous n'avons pas le moindre dessin

		\item Le risk est élevé mais acceptable: Nous n'avons aucune chance mais avec 10 fois le budget actuel et 10 fois plus de temps et de ressources, nous avons une chance sur deux de réussir.

		\item C'est un problème sérieux mais pas insurmontable: C'est un miracle si nous trouvons une solution. Il faudrait avoir Dieu comme chef de projet.

		\item Le cahier des charges n'est pas très bien défini: Personne n'a pensé à faire un cahier des charges

		\item Cela nécessitera des analysises et un suivi ultérieur: C'est totallement hors de contrôle

		\item Le projet a été concu pour une haute disponibilité: Les disfonctionnements seront mis sur le dos des opérateurs.

		\item Ce projet nécessite peu d'opérations de maintenance: Nous ne laisserions pas un technicien changer une ampoule, autant jeter l'eau du bain avec le bébé.

		\item Le logiciel est en cours de développement sans manquement au cahier des charges: La documentation sera rédigée en chinois et en chine faute de temps

		\item La livraison est prévue pour le dernier trimestre de l'année prochaine: Cela nous laisse du temps pour décider qui blâmer d'être actuellement en retard.
	\end{itemize}
	
	\begin{center}\underline{\hspace{5 cm}}\end{center}

\begin{itemize} 
	\item Combien d'étudiants ingénieurs de première année sont nécessaires pour changer une ampoule?: Aucun. C'est un sujet de deuxième année.

	\item  Combien d'étudiants ingénieurs de deuxième année sont nécessaires pour changer une ampoule?: Aucun. Un, mais le reste de la classe copie le rapport.

	\item  Combien d'étudiants ingénieurs de troisième année sont nécessaires pour changer une ampoule?: Est-ce que cette question sera dans l'examen final?

	\item  Combien d'ingénieurs civils sont nécessaires pour changer une ampoule?: Deux. Le premier change l'ampoule et le second tient la bougie.

	\item  Combien d'ingénieurs électricient sont nécessaires pour changer une ampoule?: Aucun. Ils redéfinissent simplement l'obscurité comme nouveau standard industriel.

	\item  Combien d'ingénieurs informaticiens sont nécessaires pour changer une ampoule?: Pourquoi changer? De toute façon le connecteur ne sera plus valable d'ici six mois.

	\item  Combien d'ingénieurs mécaniciens sont nécessaires pour changer une ampoule?: Cinq! Un décide la manière de changer l'ampoule, le deuxième calcule la force nécessaire, le troisième concoit un outil pour tourner l'ampoule, le quatrième pour concevoir un outi confortable mais fonctionnel pour attraper l'ampoule et le cinquième qui utilie les résultats des quatre premiers equipment. 

	\item  Combien d'ingénieurs nucléaires sont nécessaires pour changer une ampoule?: Sept! Un pour installer la nouvelle ampoule et 6 pour trouver quoi faire avec l'ancienne pour les 10'000 prochaines années. 
\end{itemize}

\begin{center}\underline{\hspace{5 cm}}\end{center}

	\begin{figure}[H]
		\centering
		\includegraphics[scale=1]{img/humour/2bornot2b.jpg}
	\end{figure}
	
\begin{center}\underline{\hspace{5 cm}}\end{center}

Cela se passe à Moscou: un couple de touristes demande son chemin, sur un pont à un russe qui se trouve être ingénieur.

Le gars leur dit: «Vous traversez, et, d'ici 50 mètres, vous tournez à droite...»

Remerciements de la part des touristes... , puis ils partent. Alors le gars leur court derrière:

«Attendez, attendez ! Je viens de me souvenir que le pont fait 70 mètre. Si vous tournez à droite au bout de 50 mètres, comme je vous l'ai dit, vous tomberez à l'eau.»

\begin{center}\underline{\hspace{5 cm}}\end{center}

Une équipe d'ingénieur qualité travaille sur l'AMDEC d'un nouvelle usine chimique. Après plusieurs semaines, pendant la réunion de debriefing:

Ingénieurs Qualité: «Notre conclusion est: Il y a 1 chance sur 10'000 que l'usine explose, tuant beaucoup de gens et induisant un terrible impact écologique, c'est éthiquement inacceptable!».

Manager: «Les normes parlent de risque acceptable si le taux est à 1/7'000 pourtant?!»

L'équipe d'ingénieurs se regroupe et prend alors la parole:

Ingénieurs Qualité: «Notre conclusion est: Vous avez un problème de surqualité, c'est éthiquement inacceptable!»

\begin{center}\underline{\hspace{5 cm}}\end{center}

	\begin{figure}[H]
		\centering
		\includegraphics[scale=1]{img/humour/airplane_magic.jpg}
	\end{figure}

\begin{center}\underline{\hspace{5 cm}}\end{center}

Un homme était assis à côté d'une fille de 10 ans dans un avion. S'ennuyant, il se tourna vers la fille et dit: «Parlons! J'ai entendu dire que les vols vont plus vite si vous entamez une conversation avec votre compagnon de route.»

La fille qui lisait un livre la ferma lentement et dit au gars: «De quoi voudriez-vous parler?»

«Oh, je ne sais pas» dit l'homme. «De physique nucléaire?»

«OK» disa-t-elle. «Cela pourrait être un sujet intéressant. Mais laissez-moi d'abord vous poser une question! Un cheval, une vache et un cerf mangent tous la même chose ... de l'herbe. Pourquoi un cerf excrète de petites boulettes, une vache une flaque et un cheval produit des touffes d'herbes sèches?»

L'homme a réfléchi et a dit: «Hmmm, je n'en ai aucune idée.»

Alors, sarcastique, la petite fille lui dit : «Comment voulez vous que je vous explique ce qu'est la physique nucléaire alors que vous ne maîtrisez même pas un petit problème de merde ?»

\begin{center}\underline{\hspace{5 cm}}\end{center}

	\begin{center}
	\includegraphics[scale=3]{img/humour/fourier_transform.jpg}
	\end{center}
	
\begin{center}\underline{\hspace{5 cm}}\end{center}

Newton asked: How write $4$ in between $5$?

\begin{enumerate}
	\item Medicine students said: Joke!
	
	\item Science students said: Impossible!
	
	\item Management students said: Not found on the internet!
	
	\item Engineering student said: "F(IV)E"
\end{enumerate}

	\pagebreak
	\section{Informatique}

	\begin{center}
	\includegraphics{img/humour/meaning_life.jpg}
	\end{center}
	
\begin{center}\underline{\hspace{5 cm}}\end{center}	

Comment devenir un bon Hacker ?
	
	Bon, si tu veux être un vrai hacker, il va te falloir Linux.
	
	Là, tu as 2 solutions:
	\begin{enumerate}
		\item Tu es un sale bourgeois capitaliste et tu l'achètes 150 balles à la FNAC.
		
		\item Tu es un vrai trou du cul, et là tu le downloades par le Net.
	\end{enumerate}
	
	Évidemment, tu es un vrai trou du cul donc tu ouvres ton tit client FTP et tu te tapes tranquillement les 20 ou 25 heures de download pour une Slack ou une Debian. Evite la Red Hat, ça fait trop grand public, toi t'es un mec uNdERgrOuNd maintenant, c'est normal, t'es Hacker.
	
	Bon, tu as ton Linux, maintenant c'est bon oublie-le. Pas la peine de se casser le cul à apprendre un nouvel OS dont tu ne te serviras jamais parce que "XWing vs Tie Fighter" tourne pas dessus. La meilleure solution consiste carrément à niquer lilo, comme ça tu es sûr que tu ne booteras que sous Windows 95. C'est une solution élégante que de nombreux trous du cul semblent avoir choisie. Pour ça, ouvre une session DOS par Windows et tape fdisk /mbr. Ca va effacer lilo qui était installé sur le MBR de ton disque dur, comme ça tu n'auras plus à te soucier de Linux.
	
	L'essentiel est de l'avoir, pas de savoir s'en servir.
	
	"Ouais mais comment je peux prouver aux gens que j'ai Linux et passer pour un gros rebelle ?"
	
	C'est une question bien naturelle. J'ai pensé à toi petit looser et voici une série de phrases qu'il faut balancer à propos de Linux:
	
	\begin{itemize}
		\item "Linux c'est trop puissant, t'es complètement libre par rapport à ces OS de fachos genre Windaube. De toute façon, MS c'est trop ripoux."
	
		\item "Bah si t'es un débutant, va pas sous Linux, c'est fait pour les eLiTeS ce truc, toi reste sous winfuck."
	
		\item "Dis, tu sais pas ou je pourrais trouver la libc5.4.36 ? Parce que chez moi la 5.4.35 est incompatible avec les modifs que j'ai faites au kernel."
	
		\item "Ça pue netscape, moi ça me dumpe des cores de 10 mégas dès que le lance, je préfère Lynx au moins c'est pas prise de gueule c'est mieux le mode texte."
	
		\item "Ralala le bouffon que c'est lui ! Il s'est installé une Red Hat !! Tain c'est de la daube les Red Hat c'est nul y'a que la Debian qui est bien, au moins tu sais ce que tu fais t'es le master de ton system nan vraiment c'est ripoux Red Hat."
	\end{itemize}
	
	Avec ce genre de petites phrases, tu te retrouveras très vite classé dans la catégorie "OK c'est un trou du cul, mais un trou du cul sous Linux", ce qui est la première étape pour être un vrai hacker. Maintenant que tout le monde sait que tu as ton Linux, il faut passer au stade suivant, celui du pro des réseaux, genre le mec qui maîtrise ICMP à mort. C'est la deuxième étape de ton long périple.
	
	Ici, il faut mettre la main au porte-monnaie. Direction la FNAC, tu achètes n'importe quel bouquin sur Unix et sur les réseaux. L'essentiel est que le titre soit compliqué. Un petit "Protocole rlogin sur réseau Ethernet en sous-adressage" sera du meilleur effet. N'hésite pas, dès que tu ne comprends même pas le titre, il faut acheter le bouquin: c'est pas pour lire, c'est pour impressionner tes autres potes trous du cul.
	
	Une bonne méthode consiste à acheter un bouquin genre "TCP/IP volume 43" et de prendre des mots au hasard à apprendre par coeur: rai socket, sur-adressage, FDDI, telnet par exemple. Ensuite, tu les ressors dans une phrase, même hors contexte c'est pas grave personne n'ira vérifier ce que ça veut dire. Par exemple, il ne faut pas hésiter à balancer un "Le Telnet, ça prends combien de rai sockets en sur-adressage sur un FDDI" sur un bon gros channel de cowboyz, ça impressionne toujours, et personne n'ira te dire que ça n'a aucun sens, ne t'inquiète pas.
	
	Dispose ensuite ces bouquins dans ta chambre, avec les titres les plus compliqués aux endroits les plus visibles. Corne quelques pages pour faire plus vrai. Prends aussi quelques feuilles et dessine des schémas bidons de réseau, ou met des trucs genre 123.44.5.34 root/lydia pour faire croire que tu te chopes de password comme un ouf. Faut se la jouer à mort, ne jamais hésiter à en rajouter, scanne-toi une photo de Mitnick et accroche la au-dessus de ton lit, ou met des autocollants à tête de mort sur ton UC pour bien dire que maintenant, t'es un voyou, un mec dangereux.
	
	Pour compléter le tout et vraiment passer pour un hacker, il ne faut pas hésiter à dire des conneries du genre "Je ne suis qu'un assoiffé de connaissances". OK, tu quadruples ta seconde, mais bon c'est pas grave, tu aimes quand même apprendre, c'est ta grande passion et tu as beaucoup de volonté. Précise bien que jamais tu ne causes de dégâts aux très nombreuses machines que tu pénètres, dis que tu fais juste ça "pour le challenge intellectuel". Oui, là, il faudra te forcer pour ne pas exploser de rire, mais entraîne-toi devant ta glace avant.
	
	Quand on est un mec dangereux comme toi, on doit se réunir avec d'autres bandits pour mettre en péril la sûreté de l'État. Pour ça, il existe LE rendez-vous de toute la racaille, c'est le "Meet 2600". Tous les mois, tu iras dans un MacDo de Paris, Place d'Italie, et là tu rencontreras des grands monsieurs, des mecs qui ont rebooté tout Internet avec un prog en Visual Basic et qui ont des coupes de cheveux de rebelles de la société.
	
	Bon, tu n'y apprendras pas grand chose, les loosers qui viennent là-bas se branlent entre eux en se disant "Ouais, on est des hAcKeRz, on est sans pitié, on est des vrais durs, oh zut il est déjà 18 heures faut que je rentre ma mère va me cogner sinon". Tu pourras quand même avoir un vrai frisson en t'imaginant que le MacDo est truffé de caméras et de micros, et que tous les employés sont des agents de la DST qui écoutent des conversations aussi dangereuses que:
	
	\begin{itemize}
		\item[$-$] Trouduku1: il est à combien le Whooper ?
		\item[$-$] Trouduku2: euh MacDo fait des Whoopers maintenant ?
		\item[$-$] Trouduku1: bah ouais ils en ont toujours fait nan ? 
	\end{itemize}
	
	La communauté des hAcKeRz aime bien aussi les raves. Ça fait partie du trip "rebel no future fuck da society, on gobe des extas on écoute de la musique de daube mais on s'en fout c'est super parce que c'est interdit ". N'hésite pas à te rendre là-bas, ça fait incontestablement partie de la culture du paumé que d'aller jouer les chauds dans ces soirées.
	
	Toi, t'es un vrai trou du cul qui hacke, et tu entends bien répandre ton savoir pour former d'autres minables comme toi. Pour ça, il existe les e-zine. On peut citer les plus connus comme NoWay ou NoRoute ou le pire cotoie le meilleur (et c'est dommage pour le meilleur...) mais aussi des vraies merdes qui mériteraient d'être plus connues, comme l'excellent Core-Dump qui est une véritable farandole de guignolos expliquant des trucs archi connus dans un français que mon chat comprends mieux que moi.
	
	Évidemment, tu n'as pas lu les bouquins sur Unix, tu n'as jamais hacké la moindre machine de ta vie donc tu ne sais pas quoi écrire. Rassure-toi, tu n'es pas le seul dans ce cas. La meilleure méthode est de pondre un article sur le rap, à raconter sa dernière rave ou à pomper Phrack sans rien comprendre. Là encore, si tu pompes Phrack, n'hésite pas à carrément corriger le mec ou à rajouter des trucs pour faire plus compliqué, personne n'ira vérifier, donc vas-y lâche-toi t'es un assoiffé de connaissances, oublies pas.
	
	Maintenant, c'est clair, tu es un vrai hacker, une racaille de l'IRC, un loubard d'Internet, tu fais peur à toutes les agences gouvernementales et IBM veut t'embaucher pour sécuriser leur réseau parce que cette pédale d'Henri leur a encore collé un virus d'Internet. Il va donc falloir, au quotidien, se comporter comme un hacker, un vrai, un dur, c'est à dire avec un esprit hacker et un langage de hacker.
	
	Un hacker, ça vit avant tout sur IRC. Une fois que tes amis et ta famille auront bien vu que tu as changé, que tu n'es plus le même homme, il va falloir répandre aussi la nouvelle sur IRC et te faire des nouveaux amis qui seront comme toi des trous du cul. Fini les \#coquelicots ou les \#amitié\_fr, maintenant tu devras aller dans les bas fond de l'IRC, le cyber-Bronx, nuke-city, là où seuls les vrais cogneurs réussissent à se faire une place dans cet univers de violence. Pour ça, tu vas devoir passer du stade hacker trou du cul à celui de trou du cul sur IRC qui se la pète Mitnick, à savoir le c0wb0y.

\begin{center}\underline{\hspace{5 cm}}\end{center}	

	\begin{center}
	\includegraphics[scale=0.8]{img/humour/quantum_computing.eps}
	\end{center}
	
\begin{center}\underline{\hspace{5 cm}}\end{center}

Les 20 meilleures réponses des programmeurs lorsque leurs programmes ne fonctionnent pas:
	\begin{enumerate}[nolistsep]
		\item[20.] "C'est bizarre ..."
		\item[19.] "Cela n'a jamais fait cela auparavant."
		\item[18.] "Cela a fonctionné hier."
		\item[17.] "Comment est-ce possible?"
		\item[16.] "Ce doit être un problème de matériel."
		\item[15.] "Qu'est-ce que vous avez mal saisi pour le faire bugger?"
		\item[14.] "Il y a quelque chose de bizarre dans vos données."
		\item[13.] "Je n'ai pas touché à ce module depuis des semaines!"
		\item[12.] "Vous devez avoir la mauvaise version."
		\item[11.] "C'est juste une coïncidence malchanceuse."
		\item[10.] "Je ne peux pas tout tester!"
		\item[9.] "CECI ne peut pas être la source de CELA."
		\item[8.] "Ca marche, mais ça n'a pas été testé."
		\item[7.] "Quelqu'un doit avoir changé mon code."
		\item[6.] "Avez-vous détecté un virus sur votre système?"
		\item[5.] "Même si ça ne marche pas, comment ça se passe?
		\item[4.] "Vous ne pouvez pas utiliser cette version sur votre système."
		\item[3.] "Pourquoi voulez-vous le faire comme ça?"
		\item[2.] "Où étiez-vous quand le programme a explosé?"
	\end{enumerate}

	Et la réponse Numéro Un par les programmeurs lorsque leurs programmes ne fonctionnent pas:
	\begin{enumerate}
		\item "Cela fonction pourtant sur ma machine!"		
	\end{enumerate}


	\begin{center}\underline{\hspace{5 cm}}\end{center}	
	
	\begin{center}
	\includegraphics[scale=0.45]{img/humour/programing_languages.jpg}
	\end{center}
	
Cherchez votre niveau en programmation dans l'échelle suivante, l'objectif étant d'écrire un programme affichant "hello world" à l'écran:

\begin{itemize}
	\item Collège:

 \texttt{10 PRINT "HELLO WORLD"\\
20 END}

	\item 1ère année du Lycée

\texttt{program Hello(input, output)\\
begin\\
writeln('Hello World')\\
end.}

	\item Terminale du Lycée:
	
\texttt{(defun hello\\
(print\\
(cons 'Hello (list 'World))))}

	\item Nouveau sur le marché de l'emploi

\texttt{\#include <stdio.h>\\
void main(void)\\
\{\\
char *message[] = \{"Hello ", "World"\};\\
int i;\\
for(i = 0; i < 2; ++i)\\
printf("\%s", message[i]);\\
printf("\\n");\\
\}}

	\item Professionnel chevronné
	
\texttt{\#include <iostream.h>\\
\#include <string.h>\\
class string\{\\
private:\\
int size;\\
char *ptr;\\
public:\\
string() : size(0), ptr(new char('\textbackslash 0')) \{\}\\
string(const string \&s) : size(s.size)\\
\{\\
ptr = new char[size + 1];\\
strcpy(ptr, s.ptr);\\
\}\\
~string()\\
\{\\
delete [] ptr;\\
\}\\
friend ostream \& operator <<(ostream \& , const string \& );\\
string \& operator=(const char *);\\
\};\\
ostream \& operator<<(ostream \& stream, const string \& s)\\
\{\\
return(stream << s.ptr);\\
\}\\
string \&string::operator=(const char *chrs)\\
\{\\
if (this != \& chrs)\\
\{\\
delete [] ptr;\\
size = strlen(chrs);\\
ptr = new char[size + 1];\\
strcpy(ptr, chrs);\\
\}\\
return(*this);\\
\}\\
int main()\\
\{\\
string str;\\
str = "Hello World";\\
cout << str << endl;\\
return(0);
\}
}

	\item Administrateur système

\texttt{\#include <stdio.h>\\
\#include <stdlib.h>\\
main()\\
\{\\
char *tmp;\\
int i=0;\\
tmp=(char *)malloc(1024*sizeof(char));\\
while (tmp[i]="Hello Wolrd"[i++]);\\
i=(int)tmp[8];\\
tmp[8]=tmp[9];\\
tmp[9]=(char)i;\\
printf("\%s \textbackslash n",tmp);\\
\}\\
}

	\item Apprenti Hacker

\texttt{\#!/usr/local/bin/perl\\
\$msg="Hello, world.\textbackslash n";\\
if (\$\#ARGV >= 0) \{
while(defined(\$arg=shift(@ARGV))) \{\\
\$outfilename = \$arg;\\
open(FILE, ">" . \$outfilename) || die "Can't write \$arg: \$! \textbackslash n";\\
print (FILE \$msg);\\
close(FILE) || die "Can't close \$arg: \$!\textbackslash n";\\
\}
\} else \{\\
print (\$msg);\\
\}\\
1;}

	\item Hacker expérimenté

\texttt{\#include <stdio.h>\\
\#include <string.h>\\
\#define S "Hello, World\\n"\\
main()\{exit(printf(S) == strlen(S) ? 0 : 1);\}}

	\item Hacker chevronné

\texttt{\%cc -o a.out ~/src/misc/hw/hw.c\\
\%a.out\\
Hello, world.\\
Guru Hacker\\
\%cat
Hello, world.}

	\item Manager junior

\texttt{10 PRINT "HELLO WORLD"\\
20 END}

	\item Manager

\texttt{mail -s "Hello, world." bob@b12\\
Bob, could you please write me a program that prints "Hello, world."?\\
I need it by tomorrow.\\
\^D\\}

	\item Manager senior

\texttt{\%zmail jim\\
I need a "Hello, world." program by this afternoon.}

	\item Directeur

\texttt{\%letter\\
letter: Command not found.
\%mail\\
To: \^X \^F \^C\\
\%help mail\\
help: Command not found.\\
\%damn!\\
!:Event unrecognized\\
\% logout\\}

	\item Chercheur

\texttt{PROGRAM HELLO\\
PRINT *, 'Hello World'\\
END}

	\item Chercheur senior

\texttt{WRITE (6, 100)\\
100 FORMAT (1H ,11HHELLO WORLD)\\
CALL EXIT\\
END}

\end{itemize}

\begin{center}\underline{\hspace{5 cm}}\end{center}	

	\begin{center}
	\includegraphics[scale=0.8]{img/humour/punition.jpg}
	\end{center}
	
\begin{center}\underline{\hspace{5 cm}}\end{center}

Ce que disent les ingénieurs en informatique... et ce qu'il faut comprendre:
\begin{itemize}
	\item "Nous allons inscrire ce projet au planning": On s'en occupera si on a rien d'autre à faire

	\item "C'est un programme complètement nouveau!": C'est pas du tout compatible avec l'ancienne version

	\item "Ce programme ne nécessite aucune maintenance": C'est impossible à déboguer

	\item "Ce programme ne nécessite que peu de maintenance": C'est quasiment impossible à déboguer

	\item "Nous respecterons les standards": On a toujours fait comme ça et ce n'est pas aujourd'hui qu'on va changer

	\item "Nous tenons à respecter les standards": Vous n'allez pas remettre en cause tout ce qu'on vient de faire

	\item "La nouvelle version de ce programme est 100% compatible avec la précédente": On n'a touché à rien

	\item "Différentes approches ont été tentées": On essaie encore de deviner ce qui se passe.

	\item "On approche d'une solution": On s'est réunis pour prendre un café...

	\item "Les tests préliminaires n'ont pas été franchement concluants": Ce satané programme a planté dès qu'on a lancé

	\item "Il va falloir abandonner le concept en son entier": La seule personne qui comprenait quelque chose vient de démissionner

	\item "On prépare un rapport complet, selon une approche entièrement nouvelle": On vient juste d'engager trois bleus sortis de l'école

	\item "C'est une avancée technologique majeure": On n'arrive toujours pas à comprendre pourquoi ça ne marche pas

	\item "C'est le résultat d'années de développement": On a enfin réussi à faire fonctionner un bout du programme...

	\item "C'est en cours": On est tellement dans le pétrin que c'est sans espoir

	\item "Faites-nous part de vos réflexions": On écoutera ce que vous avez à dire tant que ça ne remet pas en cause ce qui est déjà fait, ou ce que nous avons décidé de faire

	\item "Nous allons y jeter un coup d'oeil": Laissez tomber! On a déjà assez de problèmes comme ça...

	\item "Je n'ai pas reçu votre e-mail": Ça fait des lustres que je n'ai pas vérifié ma messagerie...
\end{itemize}

\begin{center}\underline{\hspace{5 cm}}\end{center}	

	\begin{center}
	\includegraphics[scale=0.3]{img/humour/tesla_ipaddress.jpg}
	\end{center}

	\pagebreak
	\section{Sciences Sociales}
	
Pour aider à comprendre le jargon spécifique du marketing..... et éviter de paraître ridicule dans une soirée (pot de départ par exemple...):
\begin{enumerate}
	\item Michel est à une soirée et voit une nana très attirante. Il s'approche d'elle et lui dit: "Je suis un très bon coup". C'est ce qu'on appelle du "marketing direct"

	\item Michel est à une soirée avec un groupe de copains et il voit une nana très attirante. Un de ses amis s'approche d'elle et lui dit: "Tu vois ce garçon là-bas, c'est un très bon coup". C'est ce qu'on appelle de la "publicité."

	\item Michel est à une soirée et il voit une nana très attirante. Il lui demande son numéro de téléphone. Le lendemain, il l'appelle et lui dit: "Je suis un très bon coup". C'est ce qu'on appelle du "télémarketing."

	\item Michel est à une soirée et il voit une nana très attirante. Il la reconnaît, s'approche d'elle, lui rafraîchit la mémoire et lui dit: "Tu te souviens que je suis un très bon coup ?". C'est ce qu'on appelle du "Customer Relationship Management (CRM)"

	\item Michel est à une soirée et il voit une nana très attirante. Il se lève, s'arrange un peu, s'approche d'elle et lui sert un verre. Il lui ouvre la porte lorsqu'elle part, ramasse son sac lorsqu'il tombe, lui offre une cigarette et lui dit: "Je suis un très bon coup". C'est ce qu'on appelle des "relations publiques" ou "public relations" (PR)

	\item Michel est à une soirée et il voit une nana très attirante. Il invite à danser toutes ses copines, leur offre à boire et les fait rire ostensiblement par ses plaisanteries très spirituelles. La belle nana l'aborde et lui dit: "J'ai l'impression que tu es un très bon coup". C'est ce qu'on appelle du "lobbying".

	\item Michel est à une soirée et il voit une nana très attirante. Elle s'approche de lui et lui dit: "J'ai entendu dire que tu es un très bon coup". C'est ce qu'on appelle le "pouvoir de la marque".

	\item Michel est à une soirée et voit une super belle nana. Il la mate avec ses potes, fait des réflexions très fines, se bourre la gueule, ne fait rien du tout et rentre bredouille. C'est ce qu'on appelle la "réalité du marché"...
\end{enumerate}

\begin{center}\underline{\hspace{5 cm}}\end{center}
	\begin{center}
	\includegraphics{img/humour/fluid_dynamics.jpg}
	\end{center}

	\begin{center}
	\includegraphics{img/humour/stupid.jpg}
	\end{center}
\begin{center}\underline{\hspace{5 cm}}\end{center}

Un homme, dans la nacelle d'une montgolfière ne sait plus où il se trouve. Il descend et aperçoit une femme au sol. Il descend encore plus bas et l'interpelle:
\begin{itemize}
	\item[$-$] Excusez-moi! Pouvez-vous m'aider? J'avais promis à un ami de le rencontrer et j'ai déjà une heure de retard, car je ne sais plus où je me trouve…

	\item[$-$] La femme au sol répond: Vous êtes dans la nacelle d'un ballon à air chaud à environ 10 m du sol. Vous vous trouvez exactement à 49°,28' et 11'' Nord et 8°,25' et 58'' Est 

	\item[$-$] "Vous devez être ingénieur" dit l'aérostier. 

	\item[$-$] "Effectivement", répond la femme, "Comment avez-vous deviné"?

	\item[$-$] "Eh bien!", dit l'aérostier, "Tout ce que vous m'avez dit à l'air techniquement parfaitement correct, mais je n'ai pas la moindre idée de ce que je peux faire de vos informations et en fait je ne sais toujours pas où je me trouve. Pour parler ouvertement, vous ne m'avez été d'aucune aide. Pire, vous avez encore retardé mon voyage!"

	\item[$-$] La femme lui répond: "Vous devez être un manager!"

	\item[$-$] "C'est exact!", répond l'homme avec fierté, "Mais comment avez-vous deviné???"
	
	\item[$-$] "Eh bien!", dit la femme, "Vous ne savez ni où vous êtes, ni où vous allez. Vous avez atteint votre position actuelle en chauffant et en brassant une énorme quantité d'air. Vous avez fait une promesse sans avoir la moindre idée comment vous pourriez la tenir et vous comptez maintenant sur les gens situés en dessous de vous pour qu'ils résolvent votre problème. Votre situation avant et après notre rencontre n'a pas changé, mais comme par hasard, c'est moi maintenant qui à vos yeux en suis responsable..."
\end{itemize}

\begin{center}\underline{\hspace{5 cm}}\end{center}

Un homme va un samedi à un mariage en Corse dans un petit village. Il est en retard et il conduit aussi vite que possible sur des routes sinueuses. Soudain, après un virage, il doit s'arrêter net, un troupeau d'ovins occupe toute la route. Le berger est là et fait lentement avancer son troupeau. L'automobiliste use du klaxon de son véhicule plusieurs fois sans le moindre effet. Au bout de quelques minutes, le conducteur apostrophe le berger et lui dit:
\begin{itemize}
	\item[$-$] "Je suis en retard, je vais à un mariage qui sera suivi d'un méchoui, si je vous dis combien de moutons vous avez, m'en céderez-vous un ?"

	\item[$-$] "Volontiers" dit le berger, "je ne suis pas à un près".  

	\item[$-$] Le conducteur prend sa calculatrice et au bout d'une minute annonce: "1233".

	\item[$-$]  "Vous avez gagné" dit le berger, "Choisissez votre animal".
\end{itemize}
Le conducteur en désigne alors un. Le berger dit alors:
\begin{itemize}
	\item[$-$] "Si je trouve quelle est votre profession, me rendrez-vous ma bête ?"

	\item[$-$] "Bien sûr" dit le conducteur, "Je vous écoute".

	\item[$-$] "Vous êtes haut fonctionnaire et vous avez fait l'ENA ou une autre grande école de ce genre?"

	\item[$-$] "Vous avez raison" dit le conducteur, "Mais comment avez-vous deviné ?".

	\item[$-$] Le berger: "Je vous prie de me rendre mon chien!"
\end{itemize}

\begin{center}\underline{\hspace{5 cm}}\end{center}

	\begin{center}
	\includegraphics[scale=0.7]{img/humour/meeting_girls.jpg}
	\end{center}

	\begin{center}
	\includegraphics[scale=0.7]{img/humour/fibonaughty_sexquence.jpg}
	\end{center}
	
	\begin{center}\underline{\hspace{5 cm}}\end{center}
	
	\begin{center}
	\includegraphics[scale=0.7]{img/humour/made_of.jpg}
	\end{center}

	\begin{table}[H]
		\centering
			\begin{tabular}{c m{0.1cm} c m{0.1cm} c}
		    \begin{minipage}{.3\textwidth}
    		\center \includegraphics{img/humour/worker.eps}\\
		    \center Ouvrier
		    \end{minipage}
	    	&
			+
			& 
		    \begin{minipage}{.3\textwidth}
    		\center \includegraphics{img/humour/process.eps}\\
		    \center Processus
		    \end{minipage}
		    &
		    =
		    &
		   	\begin{minipage}{.3\textwidth}
    		\center \includegraphics{img/humour/engineer.eps}\\
		    \center Ingénieur
		    \end{minipage}
	    \\
		    \begin{minipage}{.3\textwidth}
    		\center \includegraphics{img/humour/engineer.eps}\\
		    \center Ingénieur
		    \end{minipage}
	    	&
			+
			& 
		    \begin{minipage}{.3\textwidth}
    		\center \includegraphics{img/humour/sociability.eps}\\
		    \center Sociabilité
		    \end{minipage}
		    &
		    =
		    &
		   	\begin{minipage}{.3\textwidth}
    		\center \includegraphics{img/humour/marketing.eps}\\
		    \center Marketeur
		    \end{minipage}
	    \\
		    \begin{minipage}{.3\textwidth}
    		\center \includegraphics{img/humour/marketing.eps}\\
		    \center Marketeur
		    \end{minipage}
	    	&
			-
			& 
		    \begin{minipage}{.3\textwidth}
    		\center \includegraphics{img/humour/truth.eps}\\
		    \center Vérité
		    \end{minipage}
		    &
		    =
		    &
		   	\begin{minipage}{.3\textwidth}
    		\center \includegraphics{img/humour/commercial.eps}\\
		    \center Commercial
		    \end{minipage}
	    \\
		    \begin{minipage}{.3\textwidth}
    		\center \includegraphics{img/humour/commercial.eps}\\
		    \center Commercial
		    \end{minipage}
	    	&
			-
			& 
		    \begin{minipage}{.3\textwidth}
    		\center \includegraphics{img/humour/brain.eps}\\
		    \center Cerveau
		    \end{minipage}
		    &
		    =
		    &
		   	\begin{minipage}{.3\textwidth}
    		\center \includegraphics{img/humour/manager.eps}\\
		    \center Manager
		    \end{minipage}
	    \\
		    \begin{minipage}{.3\textwidth}
    		\center \includegraphics{img/humour/manager.eps}\\
		    \center Manager
		    \end{minipage}
	    	&
			+
			& 
		    \begin{minipage}{.3\textwidth}
    		\center \includegraphics{img/humour/ego.eps}\\
		    \center Ego
		    \end{minipage}
		    &
		    =
		    &
		   	\begin{minipage}{.3\textwidth}
    		\center \includegraphics{img/humour/project_manager.eps}\\
		    \center Chef de projet
		    \end{minipage}
	    \\
	   		\begin{minipage}{.3\textwidth}
    		\center \includegraphics{img/humour/project_manager.eps}\\
		    \center Chef de projet
		    \end{minipage}
	    	&
			-
			& 
		    \begin{minipage}{.3\textwidth}
    		\center \includegraphics{img/humour/humour.eps}\\
		    \center Humour
		    \end{minipage}
		    &
		    =
		    &
		   	\begin{minipage}{.3\textwidth}
    		\center \includegraphics{img/humour/hr.eps}\\
		    \center Responsable RH
		    \end{minipage}
	    \\	    
		\end{tabular}
	\end{table}

	\begin{center}
	\includegraphics[scale=0.7]{img/humour/xmas.jpg}
	\end{center}
	
	\begin{center}\underline{\hspace{5 cm}}\end{center}	
	\begin{center}
		\includegraphics[scale=0.8]{img/humour/evolution.jpg}
	\end{center}
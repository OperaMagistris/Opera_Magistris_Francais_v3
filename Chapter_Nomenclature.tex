Ce chapitre contient un résumé avec une description simple de tous les symboles utilisés dans ce livre.

	\begin{table}[H]
	\centering
	\begin{tabular}{*8l}
	$\alpha$ \verb?alpha? &$\theta$ \verb?theta? & o o &$\tau$ \verb?tau? \\
	$\beta$ \verb?beta? &$\vartheta$ \verb?vartheta? &$\pi$ \verb?pi?         &$\upsilon$ \verb?upsilon? \\
	$\gamma$ \verb?upsilon? &$\xi$ \verb?xi?  &$\varpi$ \verb?varpi? &$\phi$ \verb?phi?  \\
	$\delta$ \verb?delta? &$\kappa$ \verb?kappa? &$\rho$ \verb?rho? &$\varphi$ \verb?varphi? \\
	$\varepsilon$ \verb?epsilon? &$\lambda$ \verb?lambda? &$\varrho$ \verb?varrho? &$\chi$ \verb?chi?  \\
	$\varepsilon$ \verb?varepsilon? &$\mu$ \verb?mu? &$\sigma$ \verb?sigma? &$\psi$ \verb?psi? \\
	$\zeta$ \verb?zeta? &$\nu$ \verb?nu? &$\varsigma$ \verb?varsigma? &$\omega$ \verb?omega? \\
	$\eta$ \verb?eta?\\
    \\
	$\Gamma$ \verb?Gamma? &$\Lambda$ \verb?Lambda? &$\Sigma$ \verb?Sigma? &$\Psi$ \verb?Psi? \\
	$\Delta$ \verb?Delta? &$\Xi$ \verb?Xi? &$\Upsilon$ \verb?Upsilon? &$\Omega$ \verb?Omega?\\
	$\Theta$ \verb?Theta? &$\Pi$ \verb?Pi? &$\Phi$ \verb?Phi?
	\end{tabular}
	\caption{Lettres grecques}\label{greek}
	\end{table}

	et les opérateurs mathématiques et objets utilisés dans le livre:
	\begin{itemize}[label={},leftmargin=0.5cm]
		\setlength{\itemsep}{1pt}
  		\item $($ Parenthèse ouverte
  		\item $)$ Parenthèse fermée
  		\item $[$ Crochet ouvert
  		\item $]$ Crochet fermé
  		\item $\therefore$ Donc
  		\item $\because$ Parce que
	 	\item $\varnothing$ Ensemble vide
	 	\item $=$ Symbole d'égalité
	 	\item $:=$, $\triangleq$ Symbolse de "par définition"
	 	\item $>$ plus grand que
	 	\item $<$ plus petit que
	 	\item $\gg$ beaucoup plus grand que
	 	\item $\ll$ beaucoup plus petit que
	 	\item $\leq$ plus petit ou égal à
	 	\item $\geq$ plus grand ou égal à
	 	\item $\succ$ préféré à (pour utilité en économétrie)
	 	\item $\prec$ not preferred to (for utility in econometry)
	 	\item $\succeq$ non préféré à (pour utilité en économétrie)
	 	\item $\preceq$ pas préféré ou égal à (pour utilité en économétrie)
	 	\item $\sim$ équivalent à (pour utilité en économétrie)
	 	\item $\mathbb{N}$ Ensemble de nombres naturels (entiers positifs)
	 	\item $\mathbb{Z}$ Ensemble de nombres relatifs (tous les entiers)
	 	\item $\mathbb{Q}$ Ensemble de nombres rationnels (rapport des nombres relatifs
	 	\item $\mathbb{R}$ Ensemble des nombres réels
	 	\item $\mathbb{C}$ Ensemble de nombres complexes
	 	\item $\Re$ Partie réelle d'un nombre complexe
	 	\item $\Im$ Partie imaginaire d'un nombre complexe
	 	\item $\aleph$ Symbole cardinal transfini
	 	\item $\wedge$ Opérateur ET, noté \& en informatique et correspondant à la multiplication en maths
	 	\item $\vee$ Opérateur OU, noté souvent || en informatique
	 	\item $\equiv$ Symbole d'identité (le terme de gauche est supposé identique à celui de droite et inversement)
	 	\item $\cong$ Symbole approximativement égal
	 	\item $\propto$ Symbole proportionnel linéaire
	 	\item $\in$ Symbole qui signifie que le terme gauche appartient au terme de droite
	 	\item $\not\in$ Symbole qui signifie que le terme gauche n’appartient pas au terme de droite
	 	\item $\subset$ Symbole qui signifie que le terme à gauche (un ensemble) est un sous-ensemble de l’ensemble situé à droite
	 	\item $\not\subset$ Symbole qui signifie terme que le terme à gauche (qui est un ensemble) n’est pas un sous-ensemble de l’ensemble situé à droite
	 	\item $\subseteq$ Symbole qui signifie que le terme à gauche (qui est un ensemble) est un sous-ensemble ou un ensemble égal à l'ensemble de droite
	 	\item $\not\subseteq$Symbole qui signifie terme à gauche (c'est-à-dire un ensemble) n'est pas un sous-ensemble ni même un ensemble égal à l'ensemble de droite
	 	\item $\cup$ Symbole qui signifie que le terme de gauche (qui est un ensemble) est fusionné (union) avec le terme de droite qui est également un ensemble
	 	\item $\sqcup$ Symbole qui signifie que le terme gauche (qui est un intervalle) est fusionné (union) avec le terme droit qui est également un intervalle
	 	\item $\cap$ Symbole qui signifie que nous prenons uniquement les éléments d'intersection (égaux) des termes de gauche et de droite qui sont des ensembles
	 	\item $\displaystyle \bigcup_{i=1}^n$ Union de multiples ensembles
	 	\item $\displaystyle \bigcap_{i=1}^n$ Intersection de plusieurs ensembles
	 	\item $\mid$ Tel que...
	 	\item $\forall$ Pout tout...
	 	\item $\exists$ Il existe...
	 	\item $+$ Symbole d'addition de deux termes
	 	\item $\displaystyle \sum_{i=1}^n$ Sommation de plusieurs termes indexés
	 	\item $-$ Symbole de soustraction de deux termes
	 	\item $\times, \cdot$ Symbole de multiplication (produit) de deux termes
	 	\item $\times$ Si les termes gauche et droit sont des vecteurs, il s'agit du produit vectoriel
	 	\item $\circ$ Si les termes gauche et droit sont des vecteurs, il s'agit du produit scalaire, sinon c'est aussi un symbole utilisé pour la composition de fonctions
	 	\item $\otimes$ produit tensoriel
	 	\item $\displaystyle \prod_{i=1}^n$ Symbole de multiplication (produit) de plusieurs termes indexés
	 	\item $\displaystyle\int$ Primitive de Riemann
	 	\item $\displaystyle\int\limits_a^b$ Intégrale de Riemann dans l'intervalle $[a,b]$
	 	\item $\displaystyle\oint$Intégrale curviligne fermée non orientée (intégrale de chemin)
	 	\item $\displaystyle\ointclockwise$ Intégrale de chemin le sens horaire
	 	\item $\displaystyle\ointctrclockwise$ Intégrale de chemin le sens anti-horaire
	 	\item $\div, /$, $:$ Symboles pour la division de deux termes en fonction du niveau scolaire
	 	\item $/, |$ dans le cas de l'inférence bayésienne, il s'agit de "étant donnée"
	 	\item $:$ Lorsque le terme de gauche est une matrice et celui à droite un vecteur, il s'agit du "produit scalaire (matriciel) de Frobenius"
	 	\item $P$ En fonction du contexte, il s'agit d'une Probabilité, d'une Probabilité cumulée ou d'une Partie d'un ensemble
	 	\item $\text{E}$ En statistique la moyenne espérée (espérance)
	 	\item $\text{V},\sigma^2$ En statistique la variance ($\sigma$ étant l'écart-type)
	 	\item $\hat{x}$ En physique, l'amplitude de $x$, en statistique, un estimateur de $x$
	 	\item $C_k^n,\begin{pmatrix}n\\k\end{pmatrix}$ est la notation du coefficient binomial\footnote{Ce livre, pour rappel, ne respecte pas la notation ISO de ce coefficient!} défini par$n!/(k!(n-k)!)$
	 	\item $\mathcal{F}(x(t)), \mathcal{F}(x(t))(\omega), X(\mathrm{i}\omega), X[k]$ sont les différentes notations pour la transformée de Fourier de $x(t)$
	 	\item $\mathcal{F}^{-1}(F(\omega))(t), \mathcal{F}^{-1}(X[k])$ sont les différentes notations pour la transformée de Fourier inverse
	 	\item $\mathcal{F}[x[m]]$ est la transformée de Fourier discrète de $x[m]$
	 	\item $\mathcal{F}^{-1}[X[n]]$ est la transformée de Fourier discrète inverse
	 	\item $\mathcal{L}(x(t))$ est la transformation de Laplace
	 	\item $\mathcal{L}(X(s))$ est la transformation inverse de Laplace
	 	\item ${\cal UL}(x(t))$ est la transformation unilatérale de Laplace
	 	\item $\mathcal{Z}[x[n]]$ est la transformation bilatérale en $\mathcal{Z}$
	 	\item $\mathcal{Z}^{-1}[X[z]]$ est la transformée inverse en $\mathcal{Z}$
	 	\item ${\cal UZ}[x[n]]$ est la transformée en $\mathcal{Z}$ unilatérale
	 	\item $\mathds{1}_n$ Matrice carrée unitaire (identité) (diagonale avec $1$, $0$ partout ailleurs) de dimension $n$
	 	\item $\earth$ Symbole en astronomie et astrophysique pour désigner la Terre
	 	\item $\odot$ Symbole en astronomie et astrophysique pour désigner le Soleil
	\end{itemize}
	%to make section start on odd page
	\newpage
	\thispagestyle{empty}
	\mbox{}
	\parpic[l][t]{%
	  \begin{minipage}{30mm}
	    \fbox{\includegraphics[width=80px,height=100px]{img/einstein.eps}}
	  \end{minipage}
	}		
	Ce livre dont la première \'edition a \'et\'e publi\'ee en 2001 est conçu pour que les connaissances requises pour le lire soient aussi simples que possible. Il n'est pas n\'ecessaire d'avoir un doctorat pour le consulter, il suffit de savoir raisonner, penser de façon critique, observer et avoir le temps...
	\begin{flushright}
	\textit{"La simplicit\'e est le sceau de la v\'erit\'e et elle rayonne de beaut\'e"} \\
	 Albert EINSTEIN
	\end{flushright}
	
	\section{Avant-Propos}
	\lettrine[lines=4]{\color{BrickRed}A}ucune entreprise humaine n'a eu plus d'impact que la Science\footnote{Du latin \textit{scientia} "connaissance, savoir, expertise". Lui-même venant de \textit{sciens} (g\'enitif scientis) qui signifie "intelligent, talentueux", participe pr\'esent de \textit{scire} qui signifie "connaître" vient probablement de "s\'eparer une chose d'une autre, distinguer" li\'e à \textit{scindere} "couper, diviser".} sur nos vies et notre conception de notre Monde et de nous-mêmes. Ses th\'eories, conquêtes et r\'esultats sont tout autour de nous dans le quotidien de la majorit\'e des habitants de cette planète.

	Omnipr\'esents dans l'industrie (a\'erospatiale, imagerie, cryptographie, transport, chimie, algorithmique, etc.) ou dans les services (banque, fintech, assurance, ressources humaines, projets, logistique, architecture, communication, etc.), les math\'ematiques appliqu\'ees apparaîssent \'egalement dans d'autres domaines: enquêtes, mod\'elisation des risques, protection des donn\'ees, politique, etc. Les math\'ematiques appliqu\'ees (parfois aussi appel\'ees "ing\'enierie math\'ematique") influencent nos vies (t\'el\'ecommunications, transport, m\'edecine, m\'et\'eorologie, musique, gestion de projet) et contribuent à la r\'esolution des enjeux d'actualit\'e: \'energie, sant\'e, environnement, climat, optimisation, d\'eveloppement durable, etc. bien plus que toute technique ou m\'ethodologie de soft skills! Leur grand succès est leur fabuleuse dispersion dans le monde r\'eel et leur int\'egration croissante dans toutes les activit\'es d'intelligence humaine et artificielle qui n\'ecessitent transparence et l'\'evitement de biais cognitifs. Nous allons donc à une situation où les math\'ematiciens et les ing\'enieurs n'auront plus le monopole des math\'ematiques, mais où presque tous les postes de "cols blancs" devront faire des math\'ematiques avanc\'ees.

	En tant qu'ancien \'etudiant dans le domaine de l'ing\'enierie, j'ai souvent regrett\'e l'absence d'un seul livre assez complet, d\'etaill\'e (sans aller à l'extrême ...) et \'educatif si possible gratuit (!) et portable (êtant personnellement fan des eBooks ...) contenant au moins une id\'ee non exhaustive du programme g\'en\'eral de math\'ematiques appliqu\'ees dans les \'ecoles d'ing\'enieurs avec un aperçu de ce qui est utilis\'e dans les entreprises avec des preuves plus intuitives que rigoureuses, mais avec suffisamment de d\'etails pour \'eviter tout effort inutile au lecteur. Aussi un livre qui n'exige pas que le lecteur adopte à chaque fois une nouvelle notation ou une terminologie sp\'ecifique à l'auteur - quand il n'est pas carr\'ement n\'ecessaire parfois de changer la lecture dans une langue \'etrangère ... - et où n'importe qui peut sugg\'erer des am\'eliorations ou des ajouts (à travers un forum, uen livre d'or ou un simple courriel).

	J'ai aussi \'et\'e frustr\'e pendant mes \'etudes d'avoir souvent à avaler des «formules» ou des «lois» cens\'ees être (et à tort) non prouvables ou trop compliqu\'ees comme le disaient mes professeurs ou même d'être d\'eçu par des livres d'auteurs de renom (où les d\'eveloppements sont laiss\'es au lecteur ou comme exercice et où aucune application r\'eelle est mentionn\'e ...). Dans ce pr\'esent livre pr\'edomine la volont\'e de ne jamais confondre le lecteur avec des phrases vides comme "il est \'evident que ...", "il est facile de prouver que ...", "nous laissons cela au lecteur comme un exercice ... ", puisque tous les d\'eveloppements sont pr\'esent\'es en d\'etail. Mais je ne suis pas un puriste des maths! Je n'ai qu'une ambition: expliquer le plus facilement possible.
	
	Bien que je doive admettre que certaines relations math\'ematiques pr\'esent\'ees dans le cursus des \'ecoles d'ing\'enieurs ne peuvent être r\'ealis\'ees faute de temps dans le programme officiel ou à cause de la taille limite d'un livre papier, je ne peux accepter qu'un enseignant ou un auteur en racontent à ses \'elèves (respectivement, ses lecteurs) que certaines lois et relations ne sont pas prouvables (parce que la plupart du temps ce n'est pas vrai!) ou que telle ou telle preuve est trop compliqu\'ee sans donner de r\'ef\'erence (où l'\'etudiant pourrait trouver l'information n\'ecessaire pour satisfaire sa curiosit\'e) ou au moins une preuve simplifi\'ee mais satisfaisante.
	
	De plus, je pense qu'il est totalement archaïque aujourd'hui que certains enseignants continuent à demander à leurs \'elèves de prendre une quantit\'e massive de notes pendant les cours. Il serait beaucoup plus favorable et optimal de distribuer un document de cours contenant tous les d\'etails afin de pouvoir se concentrer sur les points essentiels avec les \'elèves, c'est-à-dire les explications orales, les interpr\'etations, la compr\'ehension, le raisonnement et la pratique. \'evidemment en fournissant un support de cours complet, certains \'elèves seront brillants par leur absence mais ... c'est probablement mieux ainsi! Ainsi, ceux qui sont passionn\'es peuvent approfondir des sujets à la maison ou à la bibliothèque universitaire, les non-int\'eress\'es et je m'en foutistes feront ce qu'ils ont à faire (...) et le reste (\'etudiants en difficult\'e mais travailleurs) suivra le cours dispens\'e par l'enseignant pour profiter de poser des questions plutôt que de suivre un cours sans r\'efl\'echir à copier un tableau noir dans une salle avec un effectif d'\'elève surnum\'eaire.
	
	Inspir\'e d'un modèle d'apprentissage d'un \'erudit am\'ericain, dont j'ai oubli\'e le nom (...), ce livre propose et impose au lecteur les caract\'eristiques suivantes: d\'ecouvrir, m\'emoriser, citer, int\'egrer, expliquer, reformuler, d\'eduire, s\'electionner, utiliser, d\'ecomposer, comparer, interpr\'eter, juger, argumenter, mod\'eliser, d\'evelopper, cr\'eer, rechercher, raisonner, d\'evelopper dans un  enseignement progressif limpide pour d\'evelopper les comp\'etences analytiques et l'ouverture d'esprit.

	Donc, dans mon esprit, ce livre non exhaustif (et ses PDF compagnons associ\'es) doit être un substitut, gratuit pour tous les \'etudiants et employ\'es à travers le monde, à de nombreuses r\'ef\'erences et lacunes du système scolaire, permettant à tout \'etudiant curieux de ne pas être frustr\'e pendant de nombreuses ann\'ees pendant son cursus acad\'emique. Sinon, la science de l'ing\'enieur pourrait avoir l'aspect d'une science fig\'ee, mis à part les d\'eveloppements scientifiques et techniques, une accumulation h\'et\'eroclite de connaissances et surtout de formules qui la ferait consid\'erer comme un sous-produit insipide des math\'ematiques et qui amène les entreprises et les gouvernements à beaucoup de faux r\'esultats et de mauvaises d\'ecisions ...
	
	Ce livre a \'egalement \'et\'e conçu pour r\'epondre aux besoins des dirigeants, aussi bien financiers que non financiers. Tout dirigeant qui veut approfondir et comprendre les fondamentaux de la finance strat\'egique, du marketing strat\'egique ou de l'ing\'enierie de gestion de projet, et de l'ing\'enierie de la gestion d'entreprise/administrations et des questions li\'ees à la chaîne d'approvisionnement b\'en\'eficiera de la lecture de ce livre.
	
	Ce livre a \'egalement pour but de d\'ecrire et d'expliquer comment notre Univers et notre Monde (\'egalement probablement d'autres "mondes" de notre Univers) fonctionne d'une manière beaucoup plus pr\'ecise, plus complète et plus d\'etaill\'ee que n'importe quel livre "Saint". Il donne des modèles et des m\'ethodes de quantification pour l'origine des espèces, des galaxies, des planètes, des ph\'enomènes quantiques, des mouvements physiques, de la physique stellaire, des \'ev\'enements extrêmes observables et aussi des \'ev\'enements extrêmement rares et explique les strat\'egies sociales et technologiques de manière prouvable que tout tout à chacun peut v\'erifier par lui-même et ce en exposant chaque fois les hypothèses que toute entit\'e raisonnable devrait prendre a priori en charge dans l'\'etat actuel de nos connaissances!
	
	De toute \'evidence, les math\'ematiques appliqu\'ees sont un sujet si vast qu'un livre de cette envergure ne peut qu'aborder la base. Les lecteurs sont certainement encourag\'es à aller au-delà (voir la bibliographie à la fin du livre).

	Maintenant, ceux qui voient les Math\'ematiques Appliqu\'ees seulement comme un outil (ce qu'elles sont aussi), ou comme l'ennemi des croyances religieuses, ou comme une \'ecole de campagne ennuyeuse, sont l\'egion. Cependant, il est peut-être utile de rappeler que, comme l'a dit Galil\'ee, «\textit{le livre de la nature est \'ecrit dans le langage des math\'ematiques}» (sans vouloir faire de scientisme!). Quand vous allez en Chine, vous apprenez le chinois. Quand vous voulez aller à l'Univers, vous apprenez la math\'ematique parce que cette dernière est la langue de l'Univers et de la Nature perceptible. C'est pourquoi la math\'ematique est fondamentale et si incroyable car elle s'applique à tout l'Univers et ce à l'\'etat de nos connaissances actuelles, aussi à travers le temps. C'est dans cet esprit que ce livre traite des math\'ematiques appliqu\'ees pour les \'etudiants en sciences naturelles, terrestres et de la vie, ainsi que pour tous ceux qui ont une occupation li\'ee aux diff\'erents sujets, y compris la philosophie ou pour toute personne int\'eress\'ee par la science dans la vie de tous les jours.

	Le choix d'\'etudier l'ing\'enierie dans ce livre comme une branche des Math\'ematiques Appliqu\'ees vient du fait que les diff\'erences entre tous les domaines de la physique (anciennement connu comme «philosophie naturelle») et les math\'ematiques sont si difficilement distinguable que la m\'edaille Fields (la plus haute distinction aujourd'hui dans le domaine des math\'ematiques) a \'et\'e d\'ecern\'e en 1990 au physicien Edward Witten, qui a utilis\'e des id\'ees de la physique pour prouver un th\'eorème math\'ematique. Cette tendance n'est certainement pas fortuite, car on peut observer que toute la science, puisqu'elle cherche à approfondir la compr\'ehension du sujet qu'elle \'etudie, finit toujours ses essais et erreurs dans le domaines math\'ematiques pures (le chemin absolu par excellence!). Ainsi, nous pouvons pr\'edire dans un futur lointain, la convergence de toutes les sciences (pures, exactes ou sociales) vers les math\'ematiques pour les techniques de mod\'elisation (voir par exemple le PDF français "\textit{L'explosion des math\'ematiques}" disponible sur la page de t\'el\'echargement du site Web compagnon).

	Il peut parfois nous sembler difficile (en raison de la peur irrationnelle et injustifi\'ee des sciences pures d'une partie significative de la population) de transmettre le sentiment de la beaut\'e math\'ematique de la Nature, son harmonie la plus profonde et la m\'ecanique bien huil\'ee de l'Univers à ceux qui ne connaissent que les bases de l'algèbre. Le physicien Richard Feynman a parl\'e d'une une fois de «deux cultures»: les gens qui ont et ceux qui n'ont pas une compr\'ehension suffisante des math\'ematiques pour appr\'ecier la structure scientifique de la Nature. Il est dommage que les math\'ematiques soient n\'ecessaires pour comprendre profond\'ement la Nature et qu'elles aient \'egalement une mauvaise r\'eputation. Pour l'anecdote, on pr\'etend qu'un roi qui a demand\'e à Euclide de lui enseigner la g\'eom\'etrie se plaignait de la difficult\'e de cette dernière. Euclide r\'epondit: "Il n'y a pas de voie royale". Les physiciens et les math\'ematiciens ne peuvent pas se convertir à une langue diff\'erente. Si vous voulez en apprendre davantage sur la Nature, pour en appr\'ecier la vraie valeur, vous devez comprendre sa langue. La Nature n'est r\'ev\'el\'ee que sous cette forme et nous ne pouvons être pr\'etentieux au point de lui demander de changer ce fait.
	
	De même, aucune discussion intellectuelle ne vous permettra de communiquer avec une personne sourde ce que vous ressentez en \'ecoutant de la musique. De même, toute discussion sur le monde reste vaine pour  transmettre une compr\'ehension intime de la Nature de ceux de «l'autre culture». Les philosophes et les th\'eologiens peuvent essayer de vous donner des id\'ees qualitatives sur l'Univers. Le fait que la m\'ethode scientifique (au sens propre du terme) ne puisse pas convaincre le Monde de ses forces à travers son processus it\'eratif (le fait que la science est r\'e\'ecrite, r\'e\'ecrite et r\'e\'ecrite par am\'eliorations incr\'ementales comme dans la m\'ethodologie DMAIC Six Sigma\footnote{DMAIC (acronyme anglais pour D\'efinir, Mesurer, Analyser, Am\'eliorer et Contrôler) se r\'efère à un cycle d'am\'elioration pilot\'e par les donn\'ees utilis\'e pour am\'eliorer, optimiser et stabiliser les processus et la conception. Le cycle d'am\'elioration DMAIC est l'outil de base utilis\'e pour conduire les projets Six Sigma que nous \'etudierons en profondeur dans la section de G\'enie Industriel. Cependant, DMAIC n'est pas exclusif à Six Sigma et peut être utilis\'e comme cadre pour d'autres applications d'am\'elioration.}) est peut-être le fait de l'horizon limit\'e de certaines personnes qui imaginent que l'humain, ou un autre concept intuitif, sentimental ou arbitraire est le centre de l'Univers (principe anthropocentrique).
	\begin{figure}[H]
		\centering
		\includegraphics[width=1.0\textwidth]{img/intro/scientific_method.jpg}
		\caption[Processus cyclique de la méthode scientifique]{Processus cyclique de la méthode scientifique (source: ?)}
	\end{figure}
	
	\begin{tcolorbox}[title=Remarque,colframe=black,arc=10pt]
	Si vous êtes un scientifique honnête, la grande majorité de vos idées, même de bonnes idées, seront exclues, non pas par de nouvelles expériences mais déjà par incohérence avec d'anciennes expériences. C'est ce qui rend vraiment la science très différente, et ce qui nous donne une notion interne du bien et du mal avant de nouvelles expériences. Donc, contrairement à ce que certains profanes sceptiques pensent, l'idée que vous pouvez simplement inventer de la merde est fausse!
	\end{tcolorbox}
	
	\subsection{Motivations et objectifs}
	Bien sûr, afin de partager cette connaissance math\'ematique, il peut sembler paradoxal d'augmenter, avec notre ouvrage, la longue liste de livres d\'ejà disponibles dans les bibliothèques, dans le commerce et sur Internet. N\'eanmoins, je dois être capable de pr\'esenter des arguments qui justifient la cr\'eation d'un tel ouvrage (et de son site Internet compagnon) par rapport à des livres tels que les Feynman, Landau ou Bourbaki et Wikipedia / Wolfram eux-mêmes ou Khan Academy ou OpenStax. Alors qu'est-ce que je pense que je peux ajouter à une telle richesse de mat\'eriels?
	\begin{enumerate}
		\item Le grand plaisir que nous prenons à \'ecrire ce livre ("garder la main" et am\'eliorer nos comp\'etences) et à avoir un compendium d\'etaill\'e de haute qualit\'e des outils math\'ematiques pour nos clients et nos \'etudiants (et aussi tous ceux du Monde entier) et ce gratuitement!

		\item La passion pour le partage de connaissances gratuitement (bataille contre la "folie des droits d'auteur" (RIP Aaron Swartz!)) et sans frontières avec un outil de haute qualit\'e comme \LaTeX{} (à l'oppos\'e de Wikipedia qui m\'elange \LaTeX{} et le contenu horrible et honteux de Khan Academy \footnote{OpenStax a de bons PDF de premier cycle - en particulier les exemples dans leurs livres - mais il y a entre $40$-$60\%$ de d\'emonstrations math\'ematiques manquantes dans leurs PDF et la table des matières et l'index de leur PDF ne sont à ce jour pas interactifs ... et problème majeur ...: le contenu est limit\'e uniquement aux sujets de premier cycle}).
		
		\item Soutenir l'\'education scientifique gratuite, la pens\'ee critique et la compr\'ehension fond\'ee sur les preuves. En outre, il est clair qu'il existe un app\'etit insatiable pour certains gens à comprendre les choses (même si cela semble être actuellement une minorit\'e) et ce livre a \'et\'e \'ecrit à cette fin.
		
		\item Nous voulons pr\'esenter les Math\'ematiques Appliqu\'ees d'une manière agr\'eable et facile à apprendre (typiquement à l'oppos\'e des livres des $9$ livres de Landau), parce que nous pensons que la maîtrise des Math\'ematiques Appliqu\'ees changent la façon dont nous comprenons l'Univers et am\'eliore la compr\'ehension et tol\'erance mutuelle entre humains et nos interactions avec la Nature.
		
		\item Cet ouvrage a \'et\'e \'ecrit avant (ann\'ee 2001) que la version française de Wikip\'edia ait un contenu math\'ematique satisfaisant et longtemps avant que Khan Academy ou OpenStax n'existent.

		\item Les possibilit\'e d'effectuer rapidement des mises à jour ou corrections critiques  (à l'oppos\'e des vid\'eos de Khan Academy) ainsi que de  collaboration que permettent un e-book gratuit (avec en plus les opportunit\'es qu'offrent les outils de recherche et d'annotation des lecteurs de PDF).

		\item Le contenu peut facilement être adapt\'e en fonction des demandes / commentaires des lecteurs et de nos int\'erêts (à l'oppos\'e des vid\'eos de Khan Academy ou des livres de OpenStax ou Landau)!
		
		\item A l'oppos\'e des publications scientifiques (PRL ou autre) qui sont discutables car ne donnent pas de preuves d\'etaill\'ees et tournent parfois dans une boucle infinie de r\'ef\'erences bibliographiques, nous fournissons toujours les d\'emonstration les plus d\'etaill\'ees possible.
		
		\item L'accès aux sources \LaTeX{} est disponible au Monde entier gratuit, donc personne (\'elève ou prof) n'a besoin de recr\'eer la roue et perdre des centaines ou des milliers d'heures de r\'edaction au lieu de faire de l'innovation (à l'oppos\'e des livres de Landau)!

		\item Une pr\'esentation rigoureuse avec des preuves d\'etaill\'ees simplifi\'ees de tous les concepts pr\'esent\'es (à l'oppos\'e de Wikipedia, Khan Academy et OpenStax qui se concentrent uniquement sur les d\'emonstrations math\'ematiques des concepts de premier cycle et souvent avec des d\'emonstrations très lacunaires).

		\item La pr\'esentation de nombreux outils math\'ematiques avanc\'es et d\'etaill\'es utilis\'es dans les affaires et la R\&D en gardant à l'esprit que le langage math\'ematique semble \'eternel et d'être l'un des seuls d\'enominateurs culturel commun entre tous les pays de notre Planète.

		\item L'occasion pour les \'etudiants et les enseignants de pouvoir r\'eutiliser un contenu par copier/coller (à l'oppos\'e de Khan Academy ou des livres de Landau Books).

		\item Proposer une notation constante (à l'oppos\'e de Wikip\'edia, Khan Academy et OpenStax) tout au long du livre pour les op\'erateurs math\'ematiques, un langage clair sur tous les sujets (critère 3.C.: clair, complet et concis) et se concentrer sur les bases pour faire un important travail p\'edagogique sur les sujets (à l'oppos\'e des livres de Landau).

		\item Compiler autant d'informations que possible sur les sciences pures et exactes dans un unique livre \'electronique (portable), homogène et rigoureux et de haute qualit\'e visuelle (mais en allant pas aussi loin que les livres de Landau).

		\item Distinguer de toutes les pseudo-v\'erit\'es, seulement les v\'erit\'es qui peuvent être prouv\'ees par la d\'emarche math\'ematique.

		\item B\'en\'eficier du d\'eveloppement des m\'ethodes d'enseignement qui utilisent Internet pour rechercher la solution de problèmes math\'ematiques.

		\item L'am\'elioration spectaculaire des logiciels de traduction automatique et de la puissance de calcul qui fera de ce livre, du moins on l'espère, une r\'ef\'erence internationale dans le domaine des sciences.
		
		\item Un PDF est meilleur qu'un site Internet car d'abord tous ceux qui utilisent Internet depuis 1990 savent que la grande majorit\'e des sites disparaissent après environ $10$ ans et deuxièmement il est bien connu que certains pays bloquent Wikip\'edia et autres sites de connaissances pour garder leur population dans l'ignorance (et bloquer un PDF qui peut être partag\'e dans un e-mail est beaucoup plus difficile).
		
		\item et ... parce que les Math\'ematiques Appliqu\'ees sont belles et surtout lorsqu'elles sont \'ecrites en \LaTeX{} et illustr\'ees (à l'oppos\'e des livres de Landau dont les illustrations sont assez anciennes et pauvres en couleurs).
\end{enumerate}

	Et aussi ... Je tends à penser que les r\'esultats de la recherche sont la propri\'et\'e de l'humanit\'e et devraient être accessibles gratuitement à tous ceux qui explorent les ph\'enomènes de la Nature. De cette façon, le travail de chacun b\'en\'eficie à tous, et c'est pour toute l'humanit\'e que nos connaissances se cumulent et c'est la tendance que permet Internet.

	Je ne cache pas que ma contribution est largement limit\'ee à ce jour à celle d'un collectionneur qui glane ses informations dans les oeuvres de maîtres ou de publications ou de pages web anonymes et qui complète et argumente les d\'eveloppements math\'ematiques et les am\'eliore quand c'est possible. Par cons\'equent, certains \'el\'ements de ce livre sont originaux et certains proviennent de la litt\'erature de r\'ef\'erence. Cependant la grande majorit\'e de ce que nous avons \'ecrit est une reformulation des r\'esultats pr\'esent\'es dans la vaste bibliothèque d'existences de quelques livres fantastiques (et rares). Pour ceux qui m'accuseraient de plagiat, ils devraient penser que les th\'eorèmes pr\'esent\'es dans la plupart des livres non libres et disponibles dans le commerce ont \'et\'e d\'ecouverts et \'ecrits par leurs pr\'ed\'ecesseurs des auteurs de ces mêmes livres et que leur contribution personnelle a \'et\'e, comme la mienne, de mettre toutes ces informations sous une forme claire et moderne quelques centaines d'ann\'ees plus tard. En outre, on peut douter que nous demandions à payer pour l'accès à une culture qui est certainement la seule vraiment valable et juste dans le Monde et où il n'y a pas de brevets ou de droits de propri\'et\'e intellectuelle.
	
	\begin{fquote}[Wilson Mizner]Si vous volez d'un auteur, c'est du plagiat; si vous volez de beaucoup d'auteurs, c'est de la recherche!
 	\end{fquote}
	
	Ce livre reflète \'egalement en grande partie mes propres limites intellectuelles relativement aux sciences dures et pures. Bien que j'essaie d'\'etudier autant de domaines scientifiques et math\'ematiques que possible, il est impossible de les maîtriser tous. Ce livre ne montre clairement que mes propres int\'erêts et exp\'eriences en tant que consultant et professeur, mais aussi mes forces et mes faiblesses. Je suis responsable de la s\'election des intrants et, bien sûr, en grande partie des erreurs et des imperfections possibles.

	Après avoir tent\'e un ordre de pr\'esentation strictement lin\'eaire du sujet de la Math\'ematique Appliqu\'ee, j'ai d\'ecid\'e d'organiser ce livre d'une manière plus p\'edagogique (th\'ematique) et toujours avec des exemples pratiques d'applications. Il est à mon avis très difficile de parler d'un sujet aussi vaste dans un ordre lin\'eaire dans une seule vie humaine, c'est-à-dire lorsque les concepts sont introduits un à un, parmi ceux d\'ejà connus (où chaque th\'eorie, op\'erateur, etc. n'apparaîtrait pas avant sa d\'efinition). Ainsi, avec ce choix th\'ematique, le lecteur pourra probablement se rendre compte de l'extrême complexit\'e du sujet.

	Les cons\'equences de ce choix sont les suivantes:
	\begin{enumerate}
		\item Parfois il faudra admettre certains concepts, quitte à les comprendre plus tard.
	
		\item Il sera probablement n\'ecessaire au lecteur de parcourir au moins deux fois dans le livre. En première lecture, il appr\'ehendra l'essentiel et à la deuxième lecture, il comprendrea les d\'etails (je f\'elicite dans tous les cas le lecteur d\'ebutant qui comprend toutes les subtilit\'es la première fois!).
	
		\item Vous devez accepter le fait que certains sujets (et très faible nombre) sont parfois r\'ep\'et\'es pour le confort de lecture et l'assimilation du cerveau et qu'il existe de nombreuses r\'ef\'erences crois\'ees et des remarques compl\'ementaires en pied de page.
	\end{enumerate}
	
	Certains savent que pour chaque th\'eorème et modèle math\'ematique, il existe presque toujours plusieurs approches pour la d\'emonstration math\'ematique. J'ai toujours essay\'e de choisir l'approche qui semblait lla plus simple (par exemple, en relativit\'e et en physique quantique, il y a le formalisme alg\'ebrique et matriciel). L'objectif \'etant d'arriver au même r\'esultat de toute façon. Dans certains cas int\'eressants, nous pr\'esentons même plusieurs approches d'une d\'emonstration car nous consid\'erons les diff\'erentes approches alors comme \'etant très "formatrices".
	
	Ce livre \'etant dans sa version brouillon, il manque forc\'ement des contrôles de convergence, de continuit\'e, de grammaire et autres ... (qui vont horrifier certains lecteurs et math\'ematiciens ...)! Cependant, j'ai \'evit\'e (ou, sinon, je l'indique) les approximations habituelles de la physique et l'utilisation de l'analyse dimensionnelle, en l'utilisant le moins possible. J'essaie aussi d'\'eviter autant que possible les sujets avec des outils math\'ematiques qui n'ont pas \'et\'e pr\'esent\'es auparavant et d\'emontr\'es rigoureusement (selon les critères de l'ing\'enieur!).
	
	Enfin, cette pr\'esentation des Math\'ematiques Appliqu\'ees, qui peut encore être am\'elior\'ee, n'est pas une r\'ef\'erence absolue et contient des erreurs probablement par endroits. Tout commentaire est donc la bienvenue par e-mail ou via le forum du site compagnon d\'ejà indiqu\'es plus haut. Je m'efforcerai, dans la mesure du possible, de corriger les faiblesses et d'apporter les changements n\'ecessaires le plus rapidement possible (cela prend quelques mois en g\'en\'eral).
	
	Cependant, alors que les math\'ematiques sont exactes et indiscutables, la physique th\'eorique (ses modèles) est toujours interpr\'et\'ee dans le vocabulaire commun (mais pas dans le vocabulaire math\'ematique) et ses conclusions sont toutes relatives d'un individu à l'autre. Je ne peux que conseiller, en lisant ce livre, de lire par vous-même et de ne pas subir d'influences ext\'erieures dans un premier temps. Vous devez avoir un esprit très (très) critique, ne rien prendre pour acquis et tout remettre en question sans h\'esitation. En outre, le mot cl\'e du bon scientifique devrait être: "Doute, doute, doute ... doute encore, et v\'erifie toujours.". Nous rappelons aussi que «rien de ce que nous pouvons voir, entendre, sentir, toucher ou goûter, n'est ce qu'il semble être», ne comptez donc pas sur votre exp\'erience quotidienne pour tirer des conclusions hâtives, soyez critique, cart\'esien, rationnel et rigoureux dans vos d\'eveloppements, raisonnements et conclusions et confrontez-les avec d'autres en admettant vos erreurs et remises en question avec sagesse!
	
	\begin{tcolorbox}[title=Remarque,colframe=black,arc=10pt]
	L'une des causes du syndrôme de l'anti-intellectualisme r\'eside probablement dans l'incapacit\'e des \'etablissements d'enseignement à inculquer la pens\'ee critique au lyc\'ee. En particulier les universit\'es, incapables de d\'evelopper la pens\'ee critique de leurs \'etudiants. Internet et son amalgame de vrai et de faux ont aussi leur part de responsabilit\'e ainsi que les mass-m\'edias qui manquent de rigueur dans la manière de citer leurs sources et leurs m\'ethodes d'investigations. En outre, certains experts amplifient le problème en parlant s'exprimant publiquement et de manière non quantifi\'ee et non sourc\'ee sur des questions en dehors de leur domaine d'expertise !!!!
	\end{tcolorbox}
	Je veux dire à ceux qui essaieraient de retrouver par eux-mêmes les r\'esultats de certains d\'eveloppements de ce livre, de ne pas s'inqui\'eter s'ils ne r\'eussissent pas ou s'ils doutent de leurs comp\'etences à cause du temps pass\'e à r\'esoudre une \'equation ou un problème donn\'e. Effectivement, quelques th\'eories qui semblent parfois \'evidents ou faciles aujourd'hui, ont parfois n\'ecessit\'e plusieurs semaines, mois, voire ann\'ees, pour être d\'evelopp\'es par des math\'ematiciens ou des physiciens de premier plan dans le pass\'e!
	
	J'ai aussi essay\'e de faire en sorte que ce livre soit agr\'eable à consulter et lire en y ajoutant de nombreuses illustrations en couleurs et \'egalement en choisissant un style d'\'ecriture peu formel.
	
	Enfin, j'ai choisi d'\'ecrire ce travail à la première personne du pluriel: «nous». En effet, la physique math\'ematique n'est pas une science qui a \'et\'e faite ou qui a \'evolu\'e à travers un travail individuel, mais avec une collaboration intense entre des personnes li\'ees par la même passion et d\'esir de la connaissance. Ainsi, en utilisant le «nous», je voudrais rendre hommage aux scientifiques d\'ec\'ed\'es, bien \'evidemment aux autres co-auteurs de ce livre mais aussi aux chercheurs contemporains et futurs pour le travail qu'ils vont effectuer afin d'approcher la v\'erit\'e et la sagesse.
	
	\begin{center}
	\includegraphics[scale=1]{img/humour/pure_math_vs_applied_math.jpg}
	\end{center}

	%to make section start on odd page
	\newpage
	\thispagestyle{empty}
	\mbox{}
	\section{M\'ethodes}	
	\lettrine[lines=4]{\color{BrickRed}L}a Science est l'ensemble de tous les efforts syst\'ematiques (observations scrupuleuses et hypothèses plausibles jusqu'à la preuve du contraire) pour acqu\'erir des connaissances sur notre environnement, pour les organiser et les synth\'etiser en lois et th\'eories testables, dont le but principal est d'expliquer les choses (et PAS le pourquoi!) souvent par une approche en quatre \'etapes:
	
	\begin{itemize}
		\item[$-$] Qu'est-ce que l'on a?
		\item[$-$] Où est-ce que l'on ira?
		\item[$-$] Quel est notre objectif?	
		\item[$-$] Est-ce que c'est conforme aux donn\'ees exp\'erimentales?
	\end{itemize}
	
	Les scientifiques doivent soumettre leurs id\'ees et r\'esultats à une v\'erification ind\'ependante et à la r\'eplication par leurs pairs ("\NewTerm{peer-review}\index{peer-review}"). Ils doivent abandonner ou modifier leurs conclusions lorsqu'ils sont confront\'es à des preuves exp\'erimentales et calculatoires plus complètes ou diff\'erentes. La cr\'edibilit\'e de la Science repose donc sur ce m\'ecanisme d'autocorrection et c'est ce qui fait encore qu'au XXIe siècle, la Science n'est peut-être pas LE meilleur outil (car nous ne savons pas ce qui existera dans le futur ...) mais elle a montré qu'elle était la meilleure m\'ethode d'investigation des mécanismes de la Nature et de l'Univers par rapport à toutes les autres m\'ethodes ou croyances existantes à ce jour. L'histoire de la science montre que ce système fonctionne depuis très longtemps et très bien par rapport à tous les autres puisqu'il permet d'\'eviter de nombreux biais cognitifs et culturels. Grâce à cela, dans chaque domaine, les progrès ont \'et\'e spectaculaires. Cependant, le système a parfois \'echou\'e (et la d\'etection de ses \'echecs est elle aussi une r\'eussite!) et doit \'egalement être corrig\'e avant que de petites d\'erives ne s'accumulent.

	Le problème majeur est et reste que les scientifiques sont des humains. Ils ont les imperfections de tous les humains, et surtout, la vanit\'e, la fiert\'e, la colère, la vanit\'e et les biais cognitifs. De nos jours, il arrive que beaucoup de personnes travaillant sur le même sujet pendant un temps donn\'e d\'eveloppent une foi commune et croient qu'elles d\'etiennent la v\'erit\'e ou qu'elles d\'eveloppent des biais à cause de leur environnement quotidien. Le Chef de la foi est alors le Pape et distille son opinion. Le Pape qui joue le jeu prend sa mitre et son bâton de pèlerin pour \'evang\'eliser ses compagnons h\'er\'etiques. Jusque-là, cela fait sourire. Mais, comme dans les religions r\'eelles, ils sont parfois agaçants de vouloir \'etendre leur opinion à ceux qui ne croient pas et qui basent leurs opinions uniquement sur les donn\'ees exp\'erimentales. Certaines de ces "\'eglises" n'h\'esitent pas à se comporter comme l'Inquisition. Ceux qui osent exprimer une opinion diff\'erente sont brûl\'es à chaque occasion, lors de conf\'erences ou sur leur lieu de travail ou sur les r\'eseaux sociaux. Certains jeunes chercheurs, peu inspir\'es, pr\'efèrent se convertir à la religion dominante, devenir des clercs ce qui est bien \'evidemment une voie plus rapide et plus simple que celle des chercheurs innovants ou même des iconoclastes. Le grand Pape \'ecrit sa Bible pour diffuser ses id\'ees, l'impose à lire aux \'etudiants et aux nouveaux venus. Il met alors en forme la pens\'ee des jeunes g\'en\'erations et assure son trône. C'est une attitude m\'edi\'evale qui peut bloquer le progrès. Certains Papes vont si loin qu'ils croient être le Pape dans leur domaine de sp\'ecialisation leur donne automatiquement le même droit d'avoir le même trône dans tous les autres domaines (bias cognitif à nouveau)...
	

	Ce dernier avertissement, et les rappels qui vont suivre, doivent servir le scientifique ou tout lecteur en faisant bon usage de ce que nous consid\'erons aujourd'hui comme les bonnes pratiques de travail / raisonnement (nous discuterons en d\'etail des principes de la m\'ethode Descartes plus tard) à r\'esoudre des problèmes ou d\'evelopper des modèles th\'eoriques de façon rigoureuse et non biais\'ee. 

	À cette fin, voici un tableau r\'ecapitulatif qui fournit les \'etapes qui devraient être suivies par un scientifique qui travaille en math\'ematiques ou en physique th\'eorique (pour les d\'efinitions, voir ci-dessous) ou toute personne souhaitent avoir une d\'emarche intellectuelle un tant soit peu honnnête et rigoureuse:

	\begin{table}[H]
		\centering
		\definecolor{gris}{gray}{0.85}
			\begin{tabular}{|p{7.5cm}|p{7.5cm}|}
				\hline
				\multicolumn{1}{c}{\cellcolor{black!30}\textbf{Math\'ematiques}} & 
  \multicolumn{1}{c}{\cellcolor{black!30}\textbf{Physique}} \\ \hline
				\textbf{1.} Exposer formellement ou en langage commun le ou les "hypothèses", "conjectures", "propri\'et\'es" à prouver (les hypothèses sont not\'ees H1., H2., etc. les conjectures CJ1., CJ2., etc. et les propri\'et\'es P1., P2 ., etc.). & \textbf {1.} Exposer correctement dans un langage formel ou commun tous les d\'etails des "problèmes" à r\'esoudre (les problèmes sont not\'es P1., P2., etc.). \\ \hline
				\textbf{2.} D\'efinir les "axiomes" (non d\'emontrables, ind\'ependants et non contradictoires) qui donneront les points de d\'epart et \'etabliront des restrictions aux d\'eveloppementx (les axiomes sont not\'es A1., A2, etc.)\footnotemark. \newline\newline
Dans la même veine, les math\'ematiciens d\'efinissent le vocabulaire sp\'ecialis\'e li\'e aux op\'erateurs math\'ematiques qui sera not\'e D1., D2., etc. & \textbf{2.} D\'efinir (ou \'enoncer) les "postulats" ou les "principes" ou les "hypothèses" (suppos\'ees improuvables ...) qui donneront le point de d\'epart et \'etabliront des restrictions sur les d\'eveloppements et raisonnements (typiquement, les hypothèses et les principes sont not\'es P1 ., P2., etc., les hypothèses H1., H2., etc. en essayant d'\'eviter la confusion entre les postulats et les principes)\footnotemark. \\ \hline
				\textbf{3.} Une fois les axiomes pos\'es, en extraire directement les "lemmes" ou "propri\'et\'es" dont la validit\'e en d\'ecoule et pr\'eparer le d\'eveloppement du th\'eorème suppos\'e valider l'hypothèse  ou les conjectures de d\'epart (les lemmes \'etant not\'es L1., L2., etc. et les  propri\'et\'es P1., P2. , etc.). & \textbf{3.} Une fois le "modèle th\'eorique" d\'evelopp\'e, v\'erifier les unit\'es dimensionnelles des \'equations pour identifier des erreurs possibles triviales dans les d\'eveloppements (ces contrôles \'etant marqu\'es VA1., VA2., etc.).\\ \hline
				\textbf{4.} Une fois les "th\'eorèmes" (not\'es T1., T2., etc.) d\'emontr\'es on peut conclure sur les "cons\'equences" (not\'ees C1., C2., Etc.) et même les propri\'et\'es (not\'ees P1., P2., Etc.). & \textbf{4.} Recherchez les cas limites (y compris les "singularit\'es") du modèle pour v\'erifier intuitivement la validit\'e (ces contrôles borderline sont not\'es CL1., CL2., Etc.).\\ \hline
				\textbf{5.} Tester la robustesse ou l'utilit\'e des conjectures ou des hypothèses en prouvant l'inverse (r\'eciproque) du th\'eorème ou en les comparant avec d'autres exemples de th\'eories math\'ematiques bien connues pour voir si l'ensemble forume une structure coh\'erente (les exemples \'etant not\'es E1., E2. , etc.). & \textbf{5.} Tester le modèle th\'eorique obtenu exp\'erimentalement  et soumettre le travail à comparer avec d'autres \'equipes de recherche ind\'ependantes. Le nouveau modèle devrait fournir des r\'esultats exp\'erimentaux et jamais observ\'es (pr\'edictions à falsifier). Si le modèle est valid\'e, alors il obtient le statut officiel de "th\'eorie" jusqu'à ce qu'il soit mis en d\'efaut. \\ \hline
				\textbf{6.} De possibles remarques peuvent être ajout\'ee dans un ordre hi\'erarchiquement structur\'e et not\'ees R1., R2., etc. & \textbf{6.} De possibles remarques peuvent être ajout\'ee dans un ordre hi\'erarchiquement structur\'e et not\'ees R1., R2., etc.			
				\\ \hline
		\end{tabular}
		\caption{M\'ethodologie pour les d\'evelopements en Math\'ematique \& Physique}
	\end{table}	
	\footnotetext[4]{Parfois, les «propriétés», les «conditions» et les «axiomes» sont confondus alors que le concept d'axiome est beaucoup plus précis et profond.}
	\footnotetext[5]{Il ne faut cependant pas oublier que la validité d'un modèle ne dépend pas du réalisme de ses hypothèses mais de la conformité de ses implications avec la réalité.}
	
	\begin{tcolorbox}[title=Remark,colframe=black,arc=10pt]
	Le fait qu'une question puisse être formulée dans une phrase dans un français grammaticalement correcte ne la rend pas significative, ni ne lui donne droit à une quelconque attention sérieuse. Ni, le fait qu'un mot existe dans le dictionnaire (comme le mot "âme") ne rend réel le concept sous-jacent...
	\end{tcolorbox}
	
	Proc\'eder comme dans le tableau ci-dessus est une base de travail possible pour les personnes actives dans le domaine des math\'ematiques ou de la physique, ou toute personne int\'eress\'ee à avoir une d\'emarche intelectuelle rigoureuse\footnote{Pour vous éduquer et vous exercer ou vos élèves sur le sujet de la pensée critique, nous recommandons fortement la lecture de \cite{parker2016looseleaf} et des tests de pensée critique comme le test Ennis-Weir (test d'écriture libre dans lequel le candidat évalue, paragraphe par paragraphe, un cas raisonné) ou le California Critical Thinking Disposition Inventory (CCTDI) qui mesure les dispositions à utiliser les compétences de pensée critique ainsi que le Watson-Glaser Critical Thinking Appraisal (WGCTA).}. \'evidemment, proc\'eder proprement et traditionnellement comme ci-dessus prend un peu plus de temps que de faire les choses n'importe comment (c'est pourquoi la plupart des enseignants ne suivent pas ces règles, ils n'ont pas assez de temps pour couvrir tout le programme), c'est aussi la raison pour laquelle la science amène une majorit\'e de personnes à l'ext\'erieur de leur zone de confort intelectuelle (la plupart des gens cherchant à r\'esoudre des problèmes et à trouver des r\'eponses à leurs questions en moins d'une heure...).

\begin{center}
\includegraphics[scale=0.9]{img/intro/hypothesis_definitions.jpg}
\end{center}
Le lecteur doit aussi savoir que nous insistons sur le fait que les vrais scientifiques ne devraient pas avoir d'\'emotions derrière les sujets qu'ils \'etudient ou dont ils parlent, ce afin simplement d'\'eviter les biais cognitifs. Ils doivent uniquement utiliser des preuves (faits bas\'es sur des donn\'ees, \'evaluation par des pairs, exp\'eriences reproductibles, consensus de la communaut\'e scientifique) plutôt que des analyses individuelles \'emotionnelles, biais\'ees\footnote{Parmi tous les principaux biais que nous présenterons plus tard, le Biais de Confirmation - la tendance à rechercher, interpréter, favoriser et rappeler des informations qui confirment ou soutiennent ses croyances ou valeurs personnelles a priori - est très probablement le plus courant chez les personnes illétrées scientifiquement.}, subjectives et \'educatives qui ne sont pas bas\'ees sur des donn\'ees sourc\'ees, reproductibles et r\'efutables.

Signalons aussi une forme amusante scientifique des $10$ commandements:
\begin{enumerate}
\item Les ph\'enomènes tu observeras\\
Et jamais mesure tu ne falsifieras \\
(attention à l'erreur de confirmation: \'etudier que des ph\'enomènes qui valident ses convictions)

\item Des hypothèses tu formuleras\\
Que par l'exp\'erimentation ou la simulation tu testeras

\item L'exp\'erience pr\'ecis\'ement tu d\'ecriras, tes donn\'ees et algorithmes tu fourniras\\
Car ton collègue la reproduira\\
(attention au piège de la discipline narrative: coller les faits aux r\'esultats d\'esir\'es)

\item Fort de tes r\'esultats\\
Une th\'eorie tu bâtiras

\item De parcimonie tu useras\\
Et l'hypothèse la plus simple tu retiendras

\item Jamais v\'erit\'e d\'efinitive ne sera (humilit\'e \'epist\'emique)\\
Et toujours tu chercheras

\item D'une thèse non r\'efutable tu t'abstiendras\\
Car hors de la science elle restera

\item Tout \'echec sera pris comme une r\'eussite\\
Car la science doit confirmer mais aussi infirmer

\item Mon autorit\'e je n'utiliserai pas (argument d'autorit\'e)\\
Pour biaiser les opinions des gens dans des domaines où je n'ai pas d'expertise prouv\'ee

\item Je respecterai le serment d'Archimède et les Règles de Publication Scientifique\\
Car la science doit être transparente et responsable
\end{enumerate}

	
	\begin{tcolorbox}[colback=red!5,borderline={1mm}{2mm}{red!5},arc=0mm,boxrule=0pt]
	\bcbombe Attention! Il est très facile de faire de nouvelles th\'eories physiques en alignant simplement des mots. Ce type de d\'emarche est nomm\'e "\NewTerm{philosophie}\index{philosophie}" et les Grecs ont d\'eduit l'existance des atomes avec cette m\'ethode d'investigation. Cela peut donc conduire avec beaucoup de chance à une vraie th\'eorie. Par contre il est beaucoup plus difficile de faire une "\NewTerm{th\'eorie pr\'edictive}"\index{th\'eorie pr\'edictive}, c'est-à-dire avec des \'equations qui pr\'edisent le r\'esultat d'une exp\'erience.\\
	
	De plus, de nombreux philosophes réinventent des arguments que les physiciens savent depuis longtemps être faux. Nous avons entendu des philosophes s'inquiéter de paradoxes résolus il y a des siècles par les physiciens, et nous avons entendu des philosophes déduire comment les lois naturelles devraient être tout en ignorant comment les lois naturelles sont. Bref, il y a malheureusement beaucoup de philosophes qui ne remarquent pas quand ils sont hors de leur domaine d'expertise. On peut en dire autant des physiciens. Certains physiciens s'appuient plus fréquemment sur des arguments philosophiques qu'ils n'aiment l'admettre. Il est cependant assez facile pour nous de rejeter la philosophie comme utile -
parce que c'est inutile.
	\end{tcolorbox}

	\begin{tcolorbox}[title=Remarques,colframe=black,arc=10pt]
	Ce qui s\'epare les math\'ematiques et la physique, c'est qu'en math\'ematiques, l'hypothèse est toujours vraie. Le discours math\'ematique n'est pas une preuve d'une v\'erit\'e de recherche externe, mais une cible de coh\'erence. Ce qui devrait être correct est juste le raisonnement.
	\end{tcolorbox}
	
	Lorsque ces règles ne sont pas respect\'ees, on parle de "\NewTerm {fraude scientifique}"\index{fraude scientifique}" ou de "\NewTerm {fraude intellectuelle}" (qui conduit souvent à être renvoy\'e de son travail mais malheureusement nous ne retirons toujours pas les diplômes quand cela arrive). En g\'en\'eral, la fraude scientifique elle-même se pr\'esente sous quatre formes principales: le plagiat, la fabrication de donn\'ees et l'alt\'eration des r\'esultats d\'efavorables à l'hypothèse, l'omission d'hypothèses de travail claires et de donn\'ees recolt\'ees, l'usage d'arguments fallacieux et biais\'es. A ces fraudes on peut aussi ajouter des comportements qui posent des problèmes de qualit\'e de travail ou plus sp\'ecifiquement d'\'ethique, tels que ceux visant à soumettre par exemple plusieurs fois la même publication avec seulement quelques modifications, l'omission de conflit d'int\'erêt, les exp\'eriences dangereuses, la non-conservation des donn\'ees primaires, etc.
	\begin{figure}[H]
		\centering
		\includegraphics[scale=0.7]{img/intro/peer_review.jpg}
		\caption[]{Source: \url{http://cartoonsbyjosh.co.uk}}
	\end{figure}	

	\subsection{M\'ethode de Descartes}
	Pr\'esentons maintenant les quatre principes de la m\'ethode de Descartes qui, rappelons-le, est consid\'er\'e comme le premier scientifique de l'histoire de par sa m\'ethode d'analyse:
	\begin{itemize}
		\item[P1.] Ne recevoir jamais aucune chose pour vraie que je ne la connusse \'evidemment être telle. C'est-à-dire, d'\'eviter soigneusement la pr\'ecipitation et de ne comprendre rien de plus en mes jugements que ce qui se pr\'esenterait si clairement et si distinctement à mon esprit, que je n'eusse aucune occasion de le mettre en doute.
		
		\item[P2.] De diviser chacune des difficult\'es que j'examinerais, en autant de parcelles qu'il se pourrait (observations scrupuleuses et hypothèses vraisemblables jusqu'à preuve du contraire), et qu'il serait requis pour les mieux r\'esoudre.
		
		\item[P3.] De conduire par ordre mes pens\'ees, en commençant par les objets les plus simples et les plus ais\'es à connaître, pour monter peu à peu comme par degr\'es jusqu'à la connaissance des plus compos\'es, et supposant même de l'ordre entre ceux qui ne se pr\'ecèdent point naturellement les uns les autres.
		
		\item[P4.] Faire partout des d\'enombrements si entiers et des revues si g\'en\'erales, que je fusse assur\'e de ne rien omettre.
	\end{itemize}	
	
	\begin{figure}[H]
		\centering
		\includegraphics[scale=0.3]{img/intro/nullius_in_verba.jpg}
	\end{figure}
	\textit{Nullius in verba} (en latin "ne croire personne sur parole") est la devise de la Royal Society. C'est une expression de la détermination des membres de la Royal Society à résister à la domination de l'autorité et à vérifier toutes les déclarations par un appel à des faits déterminés par des expériences reproductibles et par un examen scientifique précautionneux par les pairs\footnote{Le processus d'examen par les pairs s'applique généralement aux articles de revues scientifiques, mais il est possible qu'un livre soit également évalué par des pairs. Bien que de nombreux livres passent par une sorte de processus d'édition ou de révision, il n'y a pas de méthode facile pour déterminer si un livre est évalué par des pairs. Une méthode pour trouver des livres évalués par des pairs consiste à jeter un oeil aux publications de livres des presses universitaires. Les livres publiés par les presses universitaires sont presque toujours soumis à un processus d'examen par les pairs. Les livres des presses universitaires sont généralement écrits par des membres du corps professoral qui sont soumis à une immense pression pour produire une littérature savante faisant autorité. Le processus d'examen par les pairs pour les presses universitaires implique généralement deux ou trois arbitres indépendants qui examineront initialement le manuscrit. Si le manuscrit reçoit une évaluation positive, la presse universitaire l'enverra à son comité de rédaction, qui sont tous membres du corps professoral, pour examen final.} de toute information.

	\subsubsection{\'Etudes en aveugle}
	Les exp\'eriences scientifiques\footnote{Ce texte est un copier/coller d'un article \'ecrit par Manuel Gnida à \url{http://www.symmetrymagazine.org/article/the-facts-and-nothing-but-the-facts}} sont conçues pour d\'eterminer des faits sur notre monde en utilisant soit des "\NewTerm {\'etudes r\'etrospectives}\index{\'etudes r\'etrospectives}" bas\'ees sur la recherche de corr\'elations en exploitant des bases de donn\'ees existantes ou sur des "\NewTerm{\'etudes prospectives}\index{\'etudes prospectives}" bas\'ees sur la recherche de causalit\'es en utilisant des exp\'eriences contrôl\'ees, randomis\'ees et en double aveugle. Mais dans des analyses compliqu\'ees, il y a un risque que les chercheurs faussent involontairement les r\'esultats pour qu'ils correspondent à ce qu'ils s'attendaient à trouver. Pour r\'eduire ou \'eliminer ce biais potentiel, les scientifiques appliquent une m\'ethode appel\'ee "\NewTerm{l'analyse aveugle} \index{analyse aveugle}".
	
	Les \'etudes à l'aveugle sont probablement mieux connues par leur utilisation dans les essais cliniques de m\'edicaments (le terme «triple aveugle se r\'efère parfois à ces derniers), dans lequel les patients sont tenus dans l'ignorance - ou dans l'aveugle - quand à savoir s'ils reçoivent un m\'edicament ou un placebo (pour faire simple car dans la r\'ealit\'e c'est plus subtile!). Cette approche aide les chercheurs à juger si leurs r\'esultats proviennent du traitement lui-même ou de la croyance des patients qu'ils le reçoivent. Mais la m\'ethode est \'egalement utilis\'ee dans la d\'egustation gastronomique ou dans les laboratoires m\'edico-l\'egaux.
	
	Les physiciens des particules et les astrophysiciens font \'egalement des \'etudes "en aveugle". L'approche est particulièrement utile lorsque les scientifiques recherchent des effets extrêmement faibles cach\'es dans le bruit de fond qui indiquent l'existence de quelque chose de nouveau, non pris en compte dans le modèle actuel. Parmi les exemples, citons les d\'ecouvertes largement diffus\'ees du boson de Higgs par des exp\'eriences men\'ees au Large Hadron Collider du CERN (Centre Europ\'een de la Recherche Nucl\'eaire) et des ondes gravitationnelles par le d\'etecteur Advanced LIGO (Laser Interferometer Gravitational-Wave Observatory).
	\begin{figure}[H]
		\centering
		\includegraphics[scale=0.7]{img/intro/scientific_evidence_02.jpg}
		\caption{\'Evidence Scientifique Hi\'erarchis\'ee (pyramide des données probantes)}
	\end{figure}
	
	La figure ci-dessus illustre au plus bas niveau d'évidence le fait trivial que nous ne devrions pas faire confiance aux scientifiques (leurs opinions et croyances) et encore moins aux médecins, chirurgiens ou ingénieurs (qui ne sont pas des scientifiques !) - la science n'est pas dirigée par les croyances et opinions des humains (!) - mais uniquement par les évidence fournies par les données d'expériences indépendantes reproductibles et l'analyse statistique à un niveau de méta-analyse dans des journaux soumis à la revue par les pairs à haut niveau d'indice H!
	
	\begin{tcolorbox}[title=Remarque,colframe=black,arc=10pt]
	Les témoignages ou anecdotes personnelles n'ont presque aucune valeur si la taille de l'échantillon est petite et biaisée et s'il n'y a pas de moyen direct de mesurer directement l'événement associé ! C'est pourquoi les évidences historiques basées sur seulement quelques témoignages écrits dans un livre unique vieux de plusieurs milliers d'années n'ont aucune valeur scientifique (même s'il existe des dizaines de tels livres avec des témoignages concordants).
	\end{tcolorbox}
	
	Ceux qui citent Nietzche :
	\begin{fquote}[Friedrich Nietzsche]Il n'y a pas de faits, que des interprétations !
 	\end{fquote}
 	ne comprennent pas non plus qu'ils sont au niveau EL01 parce que cette affirmation est elle elle-même un fait... ????

	"\textit{Les analyses scientifiques sont des processus it\'eratifs, dans lesquels nous effectuons une s\'erie de petits ajustements aux modèles th\'eoriques jusqu'à ce que les modèles d\'ecrivent avec pr\'ecision les donn\'ees exp\'erimentales}", explique Elisabeth Krause, postdoc à l'Institut Kavli d'astrophysique des particules et de cosmologie, institut qui est conjointement exploit\'e conjointement par l'Universit\'e Stanford et le D\'epartement de l'\'energie SLAC National Accelerator Laboratory. "\textit{À chaque \'etape d'une analyse, il y a le danger que les connaissances ant\'erieures nous guident dans la façon dont nous proc\'edons aux ajustements, tandis que les analyses aveugles nous aident à prendre des d\'ecisions ind\'ependantes et meilleures}".
	
	Le Retour sur EXp\'erience (REX) montre comme attendu que les analyses en aveugle doivent être conçues individuellement pour chaque exp\'erience. Effectivement, la façon dont "l'aveuglement" est fait doit laisser aux chercheurs suffisamment d'informations pour permettre une analyse significative, et cela d\'epend du type de donn\'ees provenant d'une exp\'erience sp\'ecifique.

	Une approche commune consiste à baser l'analyse uniquement sur certaines donn\'ees, à l'exclusion de la partie dans laquelle une anomalie est cens\'ee se cacher. Les donn\'ees exclues sont dites être dans une "boîte noire" ou une "boîte de signalisation cach\'ee".

	Prenons le cas de la recherche du boson de Higgs. En utilisant les donn\'ees recueillies avec le Large Hadron Collider (LHC) jusqu'à la fin de 2011, les chercheurs ont vu des signes d'une bosse dans les statistiques comme un signe potentiel d'une nouvelle particule avec une masse d'environ $ 125 $ gigaelectronvolts. Donc, quand ils ont regard\'e de nouvelles donn\'ees, ils ont d\'elib\'er\'ement mis en quarantaine la plage de masse autour de cette bosse et se sont concentr\'es sur les donn\'ees restantes à la place.
	
	Ils ont utilis\'e ces donn\'ees pour s'assurer qu'ils travaillaient avec un modèle suffisamment pr\'ecis. Puis ils ont "ouvert la boîte" et appliqu\'e ce même modèle à la r\'egion pr\'ealabement mise en quarantaine. La bosse s'est r\'ev\'el\'ee finalement être la particule de Higgs tant recherch\'ee.

	Cela a bien fonctionn\'e pour les chercheurs du boson de Higgs. Cependant, comme les scientifiques impliqu\'es dans l'exp\'erience du Large Underground Xenon (LUX) ont rapport\'e après cette approche, la m\'ethode de la "boîte noire" de l'analyse aveugle peut bien \'evidemment poser des problèmes si les donn\'ees que nous mettons dans la boîte noire contiennent des \'ev\'enements cruciaux.
	
	LUX a r\'ecemment effectu\'e l'une des recherches les plus sensibles au monde sur les WIMP (Weakly Interacting Massive Particles) - des particules hypoth\'etiques de matière noire, une forme invisible de la matière qui est cinq fois plus r\'epandue que la matière ordinaire. Les scientifiques du LUX ont fait beaucoup de travail pour prot\'eger le LUX contre les particules du bruit de fond: construction du d\'etecteur dans une salle blanche, le remplir de liquide complètement purifi\'e, l'entourer de blindage et l'installer sous $1,600$ mètres de roche. Mais quelques particules errantes le traversent n\'eanmoins, et les scientifiques ont besoin de regarder toutes leurs donn\'ees pour les trouver et les \'eliminer.

	Pour cette raison, les chercheurs du LUX ont choisi une approche diff\'erente pour leurs analyses. Au lieu d'utiliser une "boîte noire", ils utilisent un processus appel\'e "salage".

	Les scientifiques du LUX qui n'\'etaient pas impliqu\'es dans l'analyse LUX la plus r\'ecente ont ajout\'e de faux \'ev\'enements aux signaux simul\'es par des donn\'ees qui ressemblent à des signaux r\'eels. Tout comme les patients dans un essai de m\'edicament à l'aveugle, les scientifiques du LUX ne savaient pas s'ils analysaient des donn\'ees r\'eelles ou placebo. Une fois qu'ils ont termin\'e leur analyse, les scientifiques qui ont fait le "salage" ont r\'ev\'el\'e quels \'ev\'enements \'etaient faux.

	Une technique similaire a \'et\'e utilis\'ee par les scientifiques de LIGO (Laser Interferometer Gravitational-Wave Observatory ), qui ont finalement fait la première d\'etection de très petites ondulations dans l'espace-temps appel\'ees "ondes gravitationnelles".

	Pas tout le monde dans la communaut\'e scientifique est convaincu que les analyses en aveugle soient n\'ecessaires. Les analyses en aveugle sont bien \'evidemment plus compliqu\'ees à concevoir que les analyses non-aveugles et prennent plus de temps à compl\'eter et sont donc plus on\'ereuses. Certains scientifiques participant à des analyses en aveugle passent in\'evitablement du temps à regarder de fausses donn\'ees, ce qui peut donner l'impression d'un certain gaspillage.
	
	Typiquement certains médecins et ingénieurs assez célèbres (tous détestant les mathématiques car ils étaient très mauvais dans ce domaine pendant leurs études et ne comprennent pas comment appliquer des statistiques avancées ni comment lire les résultats analytiques correspondants!) fustigent la "religion du tout randomisé en double aveugle" et une médecine/ingénierie qui est passée des mains humanistes des soignants/inventeurs à celles froides des statisticiens et des "méthodologues". Certes (...) nous savons effectivement depuis longtemps que la règle du pouce, les sentiments et les croyances fonctionnent bien mieux que la méthode statistique...
	\begin{center}
		\includegraphics[scale=0.5]{img/intro/evidence_truth.jpg}
	\end{center}
	Le lecteur doit également garder à l'esprit que nous n'avons jamais prétendu dans ce livre que les méta-analyses ou les ECR (essais contrôlés randomisés) sont la règle d'or de la science fondée basée sur l'évidence. Nous prétendons simplement que ce sont les outils qui semblent factuellement - au moment où nous écrivons ces lignes - donner les meilleurs résultats. De toute évidence, ils ne sont pas parfaits (car les humains qui mènent les expériences ne sont pas non plus des êtres parfaits...) et parfois ils ont échoué, mais quiconque critique les méta-analyses et les ECR devrait fournir des preuves quantitatives qu'il existe d'autres méthodes (en mentionnant laquelle) qui fonctionnent statistiquement significativement mieux !
	
	\begin{fquote}[L. Aron Nelson]La plupart des gens ne veulent pas vraiment la vérité ; ils veulent juste être rassurés que ce qu'ils croient déjà, est la vérité.
 	\end{fquote}
	
	\pagebreak
	\subsection{Serment d'Archimède}
	Sur le modèle du serment d'Hippocrate, un groupe d'\'etudiants de l'\'ecole Polytechnique F\'ed\'erale de Lausanne (E.P.F.L.) a \'elabor\'e en 1990 un serment d'Archimède exprimant les responsabilit\'es et les devoirs de l'ing\'enieur et du technicien. Il a \'et\'e repris sous diverses versions par d'autres \'ecoles d'ing\'enieurs europ\'eennes et pourrait servir d'inspiration de base comme serment pour les chercheurs en science (il manque cependant quelques points comme, dans la m\'edecine\footnote{Rappelons au passage que contrairement à une idée faussement répandue que les médecins et les chirurgiens ou tout personnel médical ne sont bien évidemment pas des scientifiques (nous laissons le soin au lecteur de faire des recherches approfondies si cela l'étonne)!}, d'être radi\'e de l'ordre des scientifiques en cas de tromperie grave).

	« Consid\'erant la vie d'Archimède de Syracuse qui illustra dès l'Antiquit\'e le potentiel ambivalent de la technique, consid\'erant la responsabilit\'e croissante des ing\'enieurs et des scientifiques à l'\'egard des hommes et de la nature, consid\'erant l'importance des problèmes \'ethiques que soulèvent la technique et ses applications, aujourd'hui, je prends les engagements suivants et m'efforcerai de tendre vers l'id\'eal qu'ils repr\'esentent:\\
	\begin{enumerate}[label=\protect\circledbullet{\arabic*},leftmargin=15mm]
		\item Je pratiquerai ma profession pour le bien des personnes, dans le respect des Droits de l'Homme et de l'environnement.

		\item Je reconnaîtrai, m'\'etant inform\'e au mieux, la responsabilit\'e de mes actes et ne m'en d\'echargerai en aucun cas sur autrui.

		\item Je comprends que mon travail peut avoir des impacts consid\'erables sur la soci\'et\'e et l'\'economie et ce bien au delà de ma compr\'ehension.

		\item Je m'appliquerai à parfaire mes comp\'etences professionnelles.

		\item Dans le choix et la r\'ealisation de mes projets, je resterai attentif à leur contexte et à leurs cons\'equences, notamment des points de vue technique, \'economique, social, \'ecologique...

		\item Je contribuerai, dans la mesure de mes moyens, à promouvoir des rapports \'equitables entre les hommes et à soutenir le d\'eveloppement des pays \'economiquement faibles.

		\item Je transmettrai, avec rigueur et honnêtet\'e, à des interlocuteurs choisis avec discernement, toute information importante, si elle repr\'esente un acquis pour la soci\'et\'e ou si sa r\'etention constitue un danger pour autrui. Dans ce dernier cas, je veillerai à ce que l'information d\'ebouche sur des dispositions concrètes.

		\item Je ne me laisserai pas dominer par la d\'efense de mes int\'erêts ou ceux de ma profession.

		\item Je m'efforcerai, dans la mesure de mes moyens, d'amener mon entreprise à prendre en compte les pr\'eoccupations du pr\'esent Serment.

		\item Je pratiquerai ma profession en toute honnêtet\'e intellectuelle, avec conscience et dignit\'e.
		
		\item Je le promets solennellement, librement et sur mon honneur. »\\ 
\end{enumerate}
	Malheureusement, ce serment devrait être compl\'et\'e par la "\NewTerm{D\'eclaration de Munich sur les droits et devoirs des journalistes (1971)}\index{D\'eclaration de Münich sur les devoirs et les droits des journalistes}". C'est-à-dire les tâches essentielles du scientifique dans la collecte, la communication et le commentaire des donn\'ees consistent en:
\begin{itemize}
	\item Respecter la v\'erit\'e, quelles qu’en puissent être les cons\'equences pour lui-même, et ce, en raison du droit que le public a de connaître la v\'erit\'e.

	\item D\'efendre la libert\'e de l’information, du commentaire et de la critique.

	\item Publier seulement les informations dont l’origine est connue ou les accompagner, si c’est n\'ecessaire, des r\'eserves qui s’imposent ; ne pas supprimer les informations essentielles et ne pas alt\'erer les textes et les documents.

	\item Ne pas user de m\'ethodes d\'eloyales pour obtenir des informations, des photographies et des documents.

	\item S'obliger à respecter la vie priv\'ee des personnes.

	\item Rectifier toute information publi\'ee qui se r\'evèle inexacte.

	\item Garder le secret professionnel et ne pas divulguer la source des informations obtenues confidentiellement.

	\item S'interdire le plagiat, la calomnie, la diffamation, les accusations sans fondement ainsi que de recevoir un quelconque avantage en raison de la publication ou de la suppression d'une information.

	\item Ne jamais confondre le m\'etier de journaliste avec celui du publicitaire ou du propagandiste ; n’accepter aucune consigne, directe ou indirecte, des annonceurs.

	\item Refuser toute pression et n’accepter de directives r\'edactionnelles que des responsables de la r\'edaction.
\end{itemize}

	\begin{center}
		\includegraphics[scale=0.7]{img/intro/serment_archimede.jpg}
	\end{center}

	\pagebreak
	\subsection{Règles de Publication Scientifique (RPS)}
	Il est impossible d'avoir un d\'ebat ou une analyse constructive si le mat\'eriel de base est inutilisable ou indisponible. Malheureusement, au XXIe siècle, il est assez facile de trouver des publications de Prix Nobel qui ont fait l'objet d'un examen par les pairs\footnote{Certaines \'etudes sont publi\'ees sans aucun examen par les pairs, même des \'etudes de Prix Nobel (...), ces "\'editeurs pr\'edateurs" inondent la litt\'erature scientifique avec des journaux qui sont essentiellement fallacieux, tout auteur peut y être publi\'e, il suffit qu'il paie pour cela!} et qui sont scientifiquement inutilisables (sans compter le fait qu'une proportion importante de revues scientifiques priv\'ees r\'echignent à publier les r\'eplications exp\'erimentales car trop ennuyantes selon elles, c'est-à-dire ne rapportant pas assez d'argent...). C'est pourquoi nous rappelons ici les règles de publication scientifique fondamentales pour qu'une publication soit accept\'ee par un v\'eritable comit\'e d'\'evaluation scientifique:
	\begin{enumerate}[label=\protect\circledbullet{\arabic*},leftmargin=15mm]
		\item Utilisation de \LaTeX{} pour la r\'edaction de la publication
		
		\item Tous les fichiers de r\'edaction (*.tex) et les fichiers de donn\'ees brutes doivent avoir des noms conformes à la norme ISO 9660
		
		\item La publication doit avoir un GUID (un code unique semblable à l'identificateur d'objet num\'erique DOI)
		
		\item Mettre les dates de publication et d'\'evaluation par les pairs (format de date/heure ISO 8601)
		
		\item Mettre la version majeure et mineure de la publication (ex: v3.6 r58) et les mots-clés relatifs à la toxonomie du domaine
		
		\item Mettre la date de la p\'eriode d'exp\'erimentation/d\'eveloppement (format de date/heure ISO) 
		
		\item R\'ediger un r\'esum\'e ou "abstract" (bref r\'ecaptiluatif des objectifs, de l'exp\'erience, des hypothèses, du protocole et des conclusions)
		
		\item \'Ecrire une introduction
		
		\item Toutes les unit\'es de mesure\footnote{Mesures qui au passage doivent être enregistrables, reproductibles et r\'epondre à une ou plusieurs causes identifi\'ees.} et notations math\'ematiques doivent respecter les normes ISO 80000
		
		\item Utiliser le "principe de pr\'ecaution\footnote{L'usage des expression "...nous pensons", "nous croyons" et "...nous avons la foi" ou toute expression similaire est bien \'evidemment interdite.}" (utilisation de conditionnel)
		
		\item Utiliser des "r\'eponses r\'eactives", c'est-à-dire faire les confrontations entre hypothèses / donn\'ees, hypothèses / faits, hypothèses / observations
		
		\item Utiliser, si possible, des "facteurs de levier" pour donner du contenu et du cr\'edit à la publication en faisant r\'ef\'erence à une autre publication correspondante sur le même sujet \footnote{Ceci est aussi l'\'etape très importante de la "revue personnelles", c'est-à-dire de plusieurs dizaines / centaines de publications scientifiques dont vous avez fait une analyse critique que vous utilisez pour construire votre propre argumentation.}
		
		\item Le mat\'eriel et les m\'ethodes doivent être d\'ecrits en d\'etails. Pour les articles th\'eoriques, ils doivent fournir un lien (URL) ou une r\'ef\'erence où la preuve d\'etaill\'ee peut être trouv\'ee (si la preuve d\'etaill\'ee est omise dans la publication originale!). Pour les exp\'eriences, le protocole d\'etaill\'e randomis\'e en double aveugle doit être fourni\footnote{Pour \'eviter typiquement un scandale, exemple parmi d'autres connus, comme l'affaire Jacques Benveniste...}
		
		\item Inclure des captures d'\'ecran (ou exports) haute r\'esolution de graphiques (incluant obligatoirement les erreurs de mesures et les intervalles de confiance/pr\'ediction visibles sur les graphiques avec le code source pour reproduire ces derniers!) ou de photos
		
		\item \'Ecrire les r\'esultats et pour les donn\'ees exp\'erimentales toujours fournir une analyse statistique pour montrer si l'effet semble significatif ou non (tailles d'effets, intervalles de fluctuations, moyennes, m\'edianes, \'ecarts-types, erreurs standards, tailles d'\'echantillons, kurtosis, skewness et si la $p$-value est communiqu\'ee \underline{toujours} communiquer avec la puissance du test!)
		
		\item Calculer la propagation des erreurs des instruments de mesure
		
		\item \'Ecrire la conclusion pr\'eliminaire pour éviter le HARKing\index{HARKing} (Hypothesizing After Results Are Known)\footnote{La conclusion pour les r\'esultats exp\'erimentaux (rejeter l'hypothèse nulle ou non) doit être \'ecrite avant (!) que l'exp\'erience soit ex\'ecut\'ee et non modifi\'ee par la suite pour \'eviter les biais cognitifs humains.}
		
		\item Donner accès aux donn\'ees brutes dans un format non propri\'etaire à la communaut\'e scientifique
		
		\item Donner accès aux scripts / codes utilis\'es pour l'analyse des donn\'ees et la reproductabilit\'e des calculs par la communaut\'e scientifique\footnote{Pour éviter des situations comme la fameuse (et honteuse) erreur de Reinhart-Rogoff...}
		
		\item Donner accès aux sources \LaTeX{} de la publication à la communaut\'e scientifique
		
		\item Fournir la version exacte (avec version mineure!) des logiciels utilis\'es pour l'expérimentation (calculs, r\'edaction et autres)
		
		\item Inclure la bibliographie avec les r\'ef\'erences à la fin du document aux normes ISO 690 (numérique) et rendre disponible le fichier BibTeX correspondant
		
		\item Citer des \'etudes \'equivalentes pour la m\'eta-analyse \footnote{S'il n'y a pas d'\'etudes \'equivalentes, alors aucune m\'eta-analyse n'est possible et les r\'esultats ainsi que les conclusions ne peuvent amener à aucun consensus scientifique pour rappel!}
		
		\item Mettre le \% de soutien financier de chaque sponsor de l'\'etude (conflits d'int\'erêts\footnote{Pas uniquement industriels et \'economiques, mais aussi religieux comme le fait de travailler pour une universit\'e non-laïque!}, sources de financement)
		
		\item Soumettre la publication au comit\'e à l'examen par les pairs (en simple ou en double aveugle\footnote{Le "simple aveugle" consiste à ce que les pairs ne connaissent pas le nom des auteurs de la publication, le"double aveugle" est que ni les auteurs ni les pairs connaissent l'identit\'e des uns et des autres.})
		
		\item Dresser la liste de tous les intervenants (avec fonctions, diplôme de quel universit\'e et e-mail\footnote{Et si possible avec le genre, le pays d'origine et l'ann\'ee de naissance pour des objectifs statistiques. Par exemple: Albert Einstein (Chercheur, Dr ès sc. Physique ETHZ, a.einstein@ethz.ch, M, CHE, 1879)}) et des pairs (seulement le noms de famille pour ces derniers) de la publication
	\end{enumerate}
	Toute publication ne respectant pas au moins une de ces règles ne peut être consid\'er\'ee comme une publication "scientifique"! Un grand nombre des points ci-dessus s'appliquent \'egalement aux contenus vid\'eos (vid\'eos TEDx ou vid\'eos YouTube où l'intervenant ne cite pas les sources et les m\'eta-analyses lorsqu'il argumente ou expose son "exp\'erience personnelle", ses "opinions" ou son "expertise").
	
	Le fait d’être publi\'e dans une revue à comit\'e de lecture ne garantit cependant pas la qualit\'e d’un article. C’est effectivement une donn\'ee int\'eressante qui m\'erite d’être à nouveau soulign\'ee et \'etudi\'ee. Est-elle surprenante pour autant ? \'evidemment pas! Tous les « publiants » savent que les « rapporteurs » jugeant de la qualit\'e des articles soumis aux revues sont eux-mêmes des chercheurs, pris dans la spirale de leurs propres travaux et enseignements, qui n’ont ni le d\'esir ni la possibilit\'e mat\'erielle de v\'erifier point par point tous les calculs d’un article de physique th\'eorique ou toutes les r\'ef\'erences d’un article de sciences humaines. Ce n’est d’ailleurs pas leur mission. L’immense majorit\'e des articles clairement erron\'es sont rejet\'es par le système (les auteurs des canulars rat\'es ne s’en ventent pas et personne ne saura combien ont \'et\'e d\'ejou\'es). Quelques-uns passent pourtant à travers les mailles du filet. C’est \'evidemment regrettable mais parfaitement connu de chaque communaut\'e concern\'ee. Celle des sciences dures n’est pas \'epargn\'ee et de notoires charlatans sont parvenus à publier dans des revues respectables et reconnues. Les sciences dures ne s’en sont naturellement pas trouv\'ee globalement disqualifi\'ees ! Ces articles ont tout simplement \'et\'e ignor\'es pour la grande majorit\'e (donc malheureusement pas tous et en particulier ceux qui ne d\'etaillent pas tous les d\'eveloppements math\'ematiques!): ni lus, ni cit\'es.
	
	\begin{tcolorbox}[title=Remarque,colframe=black,arc=10pt]
	Même s'il existe un consensus entre les scientifiques, une \'etude orient\'ee unique (qui peut être très importante) peut être utilis\'ee pour influencer l'opinion des principaux m\'edias, gouvernements et individus. C'est pourquoi une \'etude doit toujours être r\'ep\'et\'ee, \'evalu\'ee par des pairs et m\'eta-analys\'ee par des \'equipes et des laboratoires ind\'ependants.
	\end{tcolorbox}
	
	\begin{tcolorbox}[colback=red!5,borderline={1mm}{2mm}{red!5},arc=0mm,boxrule=0pt]
	\bcbombe Attention! Un certain nombre de gens pensent qu'un "\NewTerm{consensus scientifique}" fait r\'ef\'erence à un grand groupe de scientifiques qui sont tous d'accord sur le fait que quelque chose est vrai. En r\'ealit\'e, un consensus scientifique est un vaste corpus d'\'etudes scientifiques qui s'accordent et se soutiennent mutuellement ("consensus de donn\'ees"). L'accord entre les scientifiques eux-mêmes est simplement un sous-produit de la preuve coh\'erente.
	\end{tcolorbox}
	
	Un exemple bien connu de consensus inexistant est celui des religions. En effet, si quelqu'un pr\'etend que comme les statistiques ne mentent pas, alors le Dieu chr\'etien doit exister car c'est la religion la plus suivie au monde avec $2$ milliards de chr\'etiens et comme $2$ milliards de personnes ne peuvent pas avoir tort, vous pouvez rappeler à cette même personne que comme il y a $7$ milliards d'individus dans le Monde, les autres $5$ milliards qui ne croient pas au Dieu chr\'etien ne peuvent pas se tromper car... justement les statistiques ne mentent pas... Le même raisonnement s'applique si vous fusionnez les musulmans et les chr\'etiens, alors seulement $55\%$ des personnes dans le monde croient en un Dieu unique et $55\% $ ce n'est statistiquement pas assez pour atteindre le consensus scientifique qui lui est à un seuil de $95\%$...
	
	\begin{fquote}Si une religion n’avait ne serait-ce qu’une seule vraie évidence au-delà de tout doute raisonnable, il n’y aurait pas d’autres religions!
 	\end{fquote}
	
	\begin{tcolorbox}[title=Remarque,colframe=black,arc=10pt]
	Il est possible de réfuter logiquement l'existence de dieux ayant certains attributs, en montrant une incohérence entre ces attributs et la définition du dieu ou d'autres faits établis. Pour de nombreux exemples de cela, voir  \cite{martin2003impossibility}.
	\end{tcolorbox}
	
	\begin{center}
		\includegraphics[scale=2.5]{img/intro/scientific_papers.jpg}
	\end{center}
	Il est alors facile de comprendre pourquoi les pages Internet et les vid\'eos YouTube (ou toute autre plateforme similaire) ne sont pas des sources scientifiques fiables selon le protocole ci-dessus puisque:
	\begin{enumerate}
	   \item Les noms des pairs sont la grande majorit\'e du temps non indiqu\'e (à ce jour du moins!)
	   
	   \item Les contributeurs / \'editeurs sont anonymes en grande majroit\'e ne peuvent donc pas être identifi\'es (typiquement un problème de Wikip\'edia)
	   
	   \item Les d\'etails math\'ematiques ne sont pas fournis (ou même pire, il n'y a pas du tout d'\'equations!). Donc il est difficile, voire impossible de v\'erifier par vous-même si le raisonnement pr\'esent\'e est exact
	   
	   \item Le protocole exact de l'exp\'erience n'est pas communique\'e, il est donc impossible de savoir si les r\'esultats sont faux ou r\'eels ou même de les reproduire
	   
	   \item Aucune source ou r\'ef\'erence crois\'ee donn\'ee
	   
	   \item Le contenu est dans un format non fiable (une vid\'eo ou une page web ne sont pas des sources p\'erennes et prot\'eg\'ees \footnote{Au 21ème siècle un PDF par exemple devrait être prot\'eg\'e contre l'\'edition et sign\'e \'electroniquement})
	   
	   \item Les nouveaux modèles th\'eoriques pr\'esent\'es pr\'edisent bien ce que fait le pr\'ec\'edent, mais ne pr\'edisent rien de nouveau et ne sont donc pas falsifiables (r\'efutables)
	   
	   \item L'orateur sur la vid\'eo fait des hypothèses qui ne sont pas falsifiables (r\'ef\'erence à des dieux divers et vari\'es ou à des th\'eories dont les d\'etails math\'ematiques ne sont pas fournis)
	   
	   \item etc.
	\end{enumerate}
	
	\begin{center}
		\includegraphics[scale=0.5]{img/intro/fake_science.jpg}
	\end{center}
	
	La solidit\'e des preuves produites par de diff\'erents types d’\'etudes (par exemple revues syst\'ematiques, m\'eta-analyses, essais contrôl\'es randomis\'es, recherche par observation, \'etudes sur animaux, \'etudes sur cellules et avis d’experts) peut varier. Cette infographie vous aidera à comprendre les avantages et les restrictions de diff\'erents types de preuves (n'oubliez pas ce terme est abusif et que en toute rigueur on doit parler "d'évidences"!) scientifiques:
	\begin{figure}[H]
		\centering
		\includegraphics[width=1\textwidth]{img/intro/how_strong_is_the_scientific_evidence.pdf}
		\caption[Diff\'erents types de preuves scientifiques]{Diff\'erents types de preuves scientifiques (source: EUFIC)}
	\end{figure}
	
	\begin{tcolorbox}[title=Remarque,colframe=black,arc=10pt]
	Le monde n’est pas systématiquement en adéquation avec nos désirs, nos espoirs, nos préjugés, nos aspirations philosophiques, ni même avec nos appréhensions ou nos angoisses. Ce n’est pas parce qu’il serait supercoolde pouvoir déplacer des objets par la seule pensée que cela est nécessairement possible. De même, ce n’est pas parce qu’il serait heureux qu’un traitement contre une maladie rare existe qu’une allégation thérapeutique relative à cette pathologie est nécessairement vraie. Nos rêves, nos espoirs et notre imagination sont des attributs précieux de notre humanité, qui peuvent nous mettre en grand danger si nous oublions que les productions intellectuelles associées ne nous sont pas dues par la réalité. Dans le même ordre d’idée, une chose n’est pas nécessairement vraie parce qu’il est envisageable qu’elle le soit.
	\end{tcolorbox}

	\pagebreak
	\subsection{Communication des M\'edias de Masse en Sciences}
	Le lecteur de m\'edias grand public ou de r\'eseaux sociaux ne doit jamais faire confiance à une \'etude scientifique si le document de r\'ef\'erence et revu par les pairs n'est pas donn\'e en lien (et tout en gardant à l'esprit qu'en plus le document en question se doit respecter les règles de publication scientifique que nous avons \'enum\'er\'ees plus bas!). L'\'etude ne doit pas non plus être consid\'er\'ee comme "v\'erit\'e absolue" par le lecteur s'il y a un consensus de la communaut\'e scientifique seulement sur ... UNE SEULE ET UNIQUE ... publication/article\footnote{Gardez à l'esprit que même une pendule cass\'ee affiche l'heure juste deux fois par jour...}. La seule façon d'être \underline{presque} sûr est alors est de lire l'\'etude elle-même et v\'erifier si elle respecte les règles pr\'ec\'edemment \'enum\'er\'ees.

	Un premier exemple typique est une nouvelle qui a \'et\'e faussement et mal reprise par de nombreux m\'edias grand public à travers le monde sur la borr\'eliose de Lyme dont voici une capture d'\'ecran:
	\begin{figure}[H]
		\centering
		\includegraphics[scale=0.25]{img/intro/lyme_borreliose.jpg}
		\caption[Publication de la T\'el\'evision Suisse sur le traîtement de la borr\'eliose de Lyme]{Publication de la TV Suisse à propos du traîtement de la\\ borr\'eliose de Lyme le 2017-01-08 (source: App RTS/ATS)}
	\end{figure}
	En r\'esum\'e ce que le "journaliste scientifique" d'une des principales chaînes nationales suisses (donc une t\'el\'evision qui a assez d'argent pour enquêter correctement sur toute information avant de la relayer ... au moins en th\'eorie ... dans un pays qui estime être le num\'ero un dans presque tout...), a publi\'e est une très mauvaise interpr\'etation de la publication scientifique originale. L'article ci-dessus rapporte que: "\textit{... un traitement appliqu\'e pendant $3$ jours au plus tard $72$ heures après après la morsure de la tique a r\'ev\'el\'e être efficacit\'e de $100\%$...}. La chose est que cet article est fourni par l'Agence T\'el\'egraphique Suisse (et relay\'e par la suite par la t\'el\'evision suisse) qui se targue bêtement d'être $100 \%$ fiable (nous d\'etectons donc un manque de pr\'ecaution scientifique de la part de leurs journalistes ou d'une formation journalistique lacunaire...).
	
	En r\'ealit\'e (si les m\'edias avaient pris soin de lire la publication originale jusqu'à la fin ...) l'\'etude a \'et\'e arrêt\'ee après $8$ semaines et il a \'et\'e d\'emontr\'e que le traitement n'a pas de meilleur effet qu'un placebo...
	
	Une deuxième erreur typique et r\'ecurrente des m\'edias grand public est le biais de confirmation (nous verrons l'\'etude des biais plus tard) dont voici un exemple lassant et honteux tellement il se r\'epète (à croire qu'ils font exprès...):
	\begin{figure}[H]
		\centering
		\includegraphics[scale=0.25]{img/intro/miracle_lourdes.jpg}
		\caption[Publication de la T\'el\'evision Suisse sur un miracle à Lourdes]{Publication de la TV Suisse à propos d'un miracle à\\ Lourdes le 2018-02-12 (source: App RTS/AFP)}
	\end{figure}
	Bien \'evidemment n'importe quelle personne ayant un minimum de culture g\'en\'erale peut v\'erifier assez simplement via des m\'eta-analyses existantes que des "miracles\footnote{Un "miracle" est le terme utilis\'e par de nombreuses personnes lorsqu'elles ne savent pas expliquer un ph\'enomène observ\'e. Les \'eclipses \'etaient un exemple de "miracle" il n'y a pas si longtemps... Si quelque chose peut arriver et se produit, ce n'est guère miraculeux. Donc, un miracle doit être quelque chose qui ne peut pas arriver. Mais, si un miracle se produit, alors clairement, cela peut arriver. Et donc ce n'était pas un miracle au départ. Donc, si la vie est pleine de miracles, alors vous ne comprenez pas la Nature et les probabilités! Les miracles ont toujours été l'écueil des ignorants et l'asile des ambitieux...}" ont aussi lieu dans les hôpitaux et qu'en termes de "taux" de r\'emission, Lourdes ne fait pas mieux que le simple hasard en comparaison aux hôpitaux dispers\'es sur tout le planète relativement à ce type d'observation.
	
	Il est donc à nouveau honteux qu'une des principales chaînes nationales suisses (donc une t\'el\'evision qui a assez d'argent pour enquêter correctement sur toute information avant de la relayer ... au moins en th\'eorie ... dans un pays qui estime être le num\'ero un dans presque tout...) ait publi\'e une information biais\'ee et c'est d'autant plus grave que ce type d'information vient de l'AFP (Agence France Presse).
	
	\begin{center}
		\includegraphics[width=1.0\textwidth]{img/intro/deformation_medias.jpg}
	\end{center}

	\subsubsection{R\'eseaux sociaux}		
	À propos des r\'eseaux sociaux et de la communication scientifique... A priori on pourrait constater aussi bien sur Facebook, YouTube, Twitter, Instagram et TikTok que lors d'\'echanges:
	\begin{itemize}
		\item Si on simplifie le discours scientifique (en pensant bien faire...) sur certains sujets d\'elicats\footnote{Sujets où typiquement les gens vous diront que: "les faits sont d\'ementis par leur opinion"...} on peut se faire assez vite accuser de d\'eformer la r\'ealit\'e (c'est le problème effectivement de la simplification...!). Et si vous pr\'ecisez que vous avez simplifi\'e pour la compr\'ehension des illettr\'es scientifiques, vous serez probablement accus\'e d’attaque ad hominem. C'est une situation par la suite où il est difficile voire impossible de r\'etablir la confiance.
	
		\item Si on ne simplifie pas le discours scientifique (en utilisant le vocabulaire et les m\'ethodes quantitative du domaine), ou qu'on communique les liens vers les \'etudes ou th\'eories scientifiques elles-mêmes\footnote{A noter qu'il semble arriver r\'egulièrement que lorsque des liens vers les \'etudes ou m\'eta-analyses sont fournies, il y a presque toujours des personnes pour dire que soient elles sont financ\'ees par des lobbistes, soient elles ont \'et\'e choisies dans le sens des arguments d\'efendus, soient il n'y pas tous les liens vers toutes les \'etudes du monde et que pour le coup elles ne sont pas repr\'esentatives...}, on se fait accuser de cacher la v\'erit\'e sous un vocabulaire abscont et des termes et outils techniques n'ayant ni queue ni tête ou de faire dire aux statistiques ce qu'on l'on veut. Le r\'esultat est encore pire lorsque l'accès aux \'etudes est payant!
	
		\item Si on parle d'un sujet sur lequel on pas d'expertise ou de diplôme, ou qui n'est pas notre domaine d'activit\'e, on se fait remballer rapidement (à juste titre!)
	
		\item Sur certaines groupes et comptes de r\'eseaux sociaux, certains messages sont supprim\'es par les administrateurs, ce qui ne donne plus la possibilit\'e d'\'etayer des arguments ou contre-arguments ou biaise compl\'etement les \'echanges car certaines informations disparaîssent  ou n'apparaîssent jamais (certains intervenants ou messages sont typiquement masqu\'es ou supprim\'es par l'administrateur).
		
		\item Un petit pourcentage de personnes sont très bien éduquées mais peuvent être biaisées car elles sont endoctrinées depuis l'enfance et même pire (car assez difficile à détecter) ... certains ne sont que des trolls (avec un niveau scolaire de troisième cycle) qui aiment provoquer les gens sur les réseaux sociaux juste pour le plaisir de lire les réponses ou ... pour le plaisir d'analyser par curiosité les réponses des gens à leurs trollages.
		
		\item Statistiquement, $2.2\%$ des utilisateurs de réseaux sociaux ouverts comme Facebook, TikTok, Snapchat, Instagram (cela exclut automatiquement les réseaux sociaux comme ResearchGate évidemment!) ont un QI inférieur à $70$ points (et même certains avec un QI plus élevé souffrent d'une sorte de pédomorphose psychologique ou d'endoctrinement infantile). Dans un pays comme la France qui compte en 2021 quarante millions d'utilisateurs actifs quotidiens, cela fait $880'000$ utilisateurs... et donc pas mal de gens sans instruction, illétrés scientifiquement et non rationnels (sans oublier qu'il faut prendre en compte le fait que ce type de personnes est la majorité du temps sans emploi et a donc beaucoup plus à consacrer aux réseaux sociaux que les personnes en possession d'un Doctorat).
	
		\item Gardez finalement en-tête que la totalit\'e des r\'eseaux sociaux ne supportent par \LaTeX{}, il est donc impossible d'y avoir des \'echanges scientifiques (c'est-à-dire faisant usage d'\'equations math\'ematiques ou formules chimiques).
	\end{itemize}
	Le but ici n'est pas de donner une solution scientifique à ces problèmes (ce serait toutefois pertinent que des \'etudes soient men\'ees sur le sujet...!). Toutefois... des pistes qui semblent assez bien fonctionner sont de bloquer les commentaires sur les contenus publi\'es par les scientifiques ou par la communaut\'e scientifique. D'intervenir dans des \'echanges que si et seulement si on y est invit\'e à y communiquer (et non pas d'y intervenir de son propre chef).
	\begin{center}
		\includegraphics[scale=0.18]{img/intro/opinions.jpg}
	\end{center}
	Citons enfin pour clore les techniques fallacieuses d'argumentation courantes particulièrement flagrantes sur les r\'eseaux sociaux (et dans les autres m\'edias en g\'en\'eral aussi...) de la part de ceux qui sont imperm\'eables à la m\'ethode scientifique et aux analyses et simulations statistiques et qui s'inspirent volontairement (ou pas?) des principes de la propagande de guerre que l'historienne Anne Morelle a \'enonc\'es:
	\begin{enumerate}
		\item Nous ne voulons pas la guerre (ie nous ne voulons pas le "changement")
		
		\item Le camp adverse est le seul responsable de la guerre (c'est lui qui force le "changement non-naturel")
		
		\item Le chef du camp adverse a le visage du diable (ou "l'affreux de service")
		\item C'est une cause noble que nous d\'efendons (par exemple un "service public") et non des int\'erêts particuliers
		
		\item Le camp adverse provoque sciemment des atrocit\'es (meurtres, licenciements, d\'ecès,...) et si nous nous commettons des erreurs c'est involontairement
		
		\item Le camp adverse  utilise des armes non autoris\'ees (ie nous ne comprenons pas les arguments de l'autre)
		
		\item Nous subissons très peu de pertes, les pertes du camp adverse sont \'enormes (le système actuel est bon il ne faut pas le changer car il sera pire)
		
		\item Les artistes et intellectuels soutiennent notre cause
		
		\item Notre cause a un caractère sacr\'e (chercheur de v\'erit\'e, service public, etc.)
		
		\item Ceux (et celles) qui mettent en doute notre propagande sont des traîtres (ie ils mettent en pu\'eril la coh\'esion sociale, la coh\'esion ou nationale, cherchent à faire du profit sur les plus pauvres, etc.)
	\end{enumerate}
	Il s'agit généralement de circonstances dites de "post-vérité" telles que définies par le dictionnaire Oxford\footnote{Post-vérité: Fait référence à des circonstances dans lesquelles des faits objectifs influencent moins l'opinion publique que les appels à l'émotion et aux croyances personnelles.}.
	
	On peut également ajouter à cette liste le fameux: \og \textit{J'ai fait mes propres recherches} \fg{}. Les individus utilisant cet argument on souvent fait 200 heures de recherche sur Twitter, YouTube et des blogs amateurs pour trouver des preuves scientifiques en éliminant soigneusement les évidences qui allaient à l'encontre de leurs croyances, et ce en ignorant en plus les méta-analyses et ce sans les compétences techniques pour comprendre les protocoles scientifiques et les indicateurs statistiques (littéralement, ils récupèrent les premiers résultats sur Google en pensant que n'importe quel texte sur Internet ou dans un livre de fiction sacré a une valeur scientifique...).
	\begin{center}
		\includegraphics[width=0.7\textwidth]{img/intro/mes_recherches.jpg}
	\end{center}
	Il ne faut pas s’étonner, si le monde est rempli d’opinions vaines et ridicules, rien n’étant plus capable de leur donner cours, que l’ignorance!
	
	De plus, gardons à l'esprit "\NewTerm{l'effet Dunning-Kruger}\index{effet Dunning-Kruger}\label{Dunning-Kruger effect}" qui est un type de biais cognitif dans lequel les gens croient qu'ils sont plus intelligents et plus capables qu'ils ne le sont vraiment. Essentiellement, les personnes à faibles capacités ne possèdent pas les compétences nécessaires pour reconnaître leur propre incompétence.
	\begin{center}
		\includegraphics[width=0.8\textwidth]{img/intro/dunning_kruger_effect.jpg}
	\end{center}
	\begin{fquote}[Charles Darwin]L'ignorance engendre plus souvent la confiance que la connaissance : ce sont ceux qui savent peu, non ceux qui savent beaucoup, qui affirment si positivement que tel ou tel problème ne sera jamais résolu par la science.
 	\end{fquote}
	
	\pagebreak
	\pagebreak
	\subsubsection{Opinions d'Experts}
	Nous devons également être prudents avec la méthode choisie par les médias traditionnels pour interviewer des "experts".

	En effet, en science, cela n'a pas vraiment de sens d'inviter un expert à parler surtout si ce dernier:
	\begin{itemize}
		\item Utilise des arguments sans fournir d'évidences détaillées (nom de la méta-analyse évaluée par les pairs)
		
		\item Ne fait pas la différence entre  "opinions" et "évidences scientifiques"
		
		\item Parle de sujets en dehors de son domaine de spécialisation
		
		\item Ne travaille plus depuis de nombreuses années dans les laboratoires
		
		\item Utilise ses récompenses et ses livres pour se donner une position d'expert légitime
		
		\item Utiliser le travail de son équipe de laboratoire pour se promouvoir
		
		\item Est seul à faire un monologue\footnote{Les gens ne doivent pas faire confiance aux monologues - que ce soit à la radio, à la télévision ou sur les réseaux sociaux - car il n'y a pas d'experts pour contrer les éventuels mauvais arguments ou concepts mal définis qui peuvent conduire une partie de l'auditoire à de fausses interprétations spéculatives! De plus, les humains sous le stress de savoir qu'ils sont enregistrés sont naturellement sujets aux erreurs de vocabulaire et c'est sans compter les plus de 200 biais cognitifs du cerveau qui conduisent parfois à la simplification erronée de pensées complexes ...!} (typique de TEDx comme déjà mentionné)
		
		\item Est interviewé par un journaliste illétré scientifiquement
		
		\item Est plein de certitudes (les idiots sont pleins de certitudes, les gens intelligents pleins de doutes)
		
		\item À une maladie mentale (à ne pas confondre avec une maladie physique)
	\end{itemize}
	Il y a beaucoup d'exemples célèbres (comme certains prix Nobel qui sont devenus obsédés par des sujets irrationnels en dehors de leur domaine de compétence). Cependant, donnons deux exemples que nous rencontrerons plus tard dans ce livre dans différentes sections.
	
	Le premier exemple est celui de l'équipe de Nosek qui a invité des chercheurs à participer à un projet d'analyse de données. La configuration était simple. Les participants ont tous reçu le même ensemble de données et la même question: les arbitres de football donnent-ils plus de cartons rouges aux joueurs à la peau foncée qu'aux joueurs à la peau claire? Ils ont ensuite été invités à soumettre leur approche analytique pour obtenir les commentaires des autres équipes avant de se plonger dans l'analyse.
	
	Vingt-neuf équipes avec un total de $61$ analystes y ont participé. Les chercheurs ont utilisé une grande variété de méthodes, allant - pour ceux d'entre vous intéressés par le gore méthodologique - des techniques de régression linéaire simples aux régressions multiniveaux complexes et aux approches bayésiennes. Ils ont également pris différentes décisions concernant les variables secondaires à utiliser dans leurs analyses.

	Malgré l'analyse des mêmes données, les chercheurs ont obtenu une variété de résultats. Vingt équipes ont conclu que les arbitres de football donnaient plus de cartons rouges aux joueurs à la peau foncée, et neuf équipes n'ont trouvé aucune relation significative entre la couleur de la peau et les cartons rouges.
	\begin{figure}[H]
		\centering
		\includegraphics[width=0.8\textwidth]{img/arithmetics/repetability.jpg}
		\caption{Mêmes données, conclusions différentes (objectifs des méta-analyses)}
	\end{figure}
	La variabilité des résultats visible ci-dessus n'est pas due à une quelconque fraude ou à un travail bâclé. Ce sont des analystes très compétents qui étaient motivés pour trouver la vérité, a déclaré Eric Luis Uhlmann, psychologue à l'école de commerce Insead à Singapour et l'un des chefs de projet. Même les chercheurs les plus qualifiés doivent faire parfois des choix subjectifs qui ont un impact énorme sur le résultat qu'ils trouvent. C'est pourquoi, encore une fois, il est tout à fait insensé de ne faire parler qu'un seul expert dans les médias grand public dans un monologue ...
	
	Le deuxième exemple récent (évidemment c'est un cas particulier qui ne doit pas être généralisé!) a été la propagation du COVID-19 au premier trimestre de l'année 2020 aux USA. Il a été demandé à quelques experts américains quel est leur pronostic pour le nombre de cas positifs de COVID-19 au 29 mars aux USA. Cela a été résumé dans la figure suivante:
	\begin{figure}[H]
		\centering
		\includegraphics[width=1\textwidth]{img/intro/covid19_experts.jpg}
		\caption{Problèmes d'opinions/estimations d'experts}
	\end{figure}
	Pour information le nombre de cas positifs connus au 29 mars aux USA aura été en fait un peu supérieur à $100'000$ . C'est là encore un bon exemple pour lequel une opinion ou une estimation d'un experts unique - quel que soit le domaine - n'est pas forcément toujours très fiable.
	
	Évidemment, le lecteur doit garder à l'esprit que la variabilité existe dans les deux exemples ci-dessus car les scientifiques n'étaient pas autorisés à utiliser la "\NewTerm{méthode Delphes}\index{méthode de Delphes}". Fondamentalement, la méthode Delphes est un processus utilisé pour parvenir à une opinion ou à une décision de groupe en interrogeant un panel d'experts. Les experts répondent à plusieurs séries de questionnaires, et les réponses sont agrégées et partagées avec le groupe après chaque série à l'aide de techniques statistiques (nous reviendrons sur cette technique dans la section Théorie des Jeux et de la Décision page \pageref{Delphi method}).
	
	%to make section start on odd page
	\newpage
	\thispagestyle{empty}
	\mbox{}
	\section{Vocabulaire}
	\lettrine[lines=4]{\color{BrickRed}L}a physique-math\'ematique, comme tout domaine de sp\'ecialisation, a son vocabulaire propre. Afin que le lecteur ne soit pas perdu dans la compr\'ehension de certains textes qu'il pourra lire sur ce site (et son PDF associ\'e), nous avons choisi d'exposer ici les quelques termes, abr\'eviations et d\'efinitions fondamentaux à connaître. 

	Ainsi, le math\'ematicien aime bien terminer ses d\'emonstrations (quand il pense qu'elles sont justes) par l'abr\'eviation "C.Q.F.D" qui signifie "Ce Qu'il Fallait D\'emontrer" ou encore dans les hautes \'ecoles par souci d'esth\'etisme et de traditions certains professeurs (et mêmes \'elèves) notent cela en latin "Q.E.D" qui signifie "Quod Erat Demonstrandum" (cela en jette...).

	Et lors de d\'efinitions (elles sont nombreuses en math\'ematique et physique...) le scientifique fait souvent usage des terminologies suivantes:
	
	\begin{itemize}
	\item ... il suffit que  ...
	
	\item ... si et seulement si ...
	
	\item ... n\'ecessaire et suffisant ...
	
	\item ... signifie que ...
	
	\item ... prouve que ...
	\end{itemize}
	Les cinq ne sont pas \'equivalentes (identiques au sens strict). Car "il suffit que" correspond à une condition suffisante, mais pas à une condition n\'ecessaire. Il faut aussi noter que ces cinq terminologies doivent être planc\'es dans le contexte de l'analyse des donn\'ees, de l'exactitude des donn\'ees, de la reproduction et de l'\'evaluation par les pairs et non sur une croyance personnelle ou commune ou même \'emotionnelle d'un groupe de personnes (même si ce groupe à une taille de plusieurs milliards d'individus...)!
	
	De plus, il est peut être pertinent de noter que de nombreuses discussions ou d\'ebats dans la vie en g\'en\'eral (en priv\'e ou en public dans les m\'edias) sont souvent st\'eriles juste par le fait que le vocabulaire de base utilis\'e, les hypothèses de travail ou de raisonnement, ou l'objectif du d\'ebat (et des questions y relatives) n'ont pas \'et\'e correctement d\'efinis dès le d\'ebut. Même si cela est acceptable pour le citoyen lambda, ce type de situations n'est pas acceptable en science!
	\begin{center}
		\includegraphics[scale=0.30]{img/intro/an_old_age_argument.jpg}
	\end{center}

	\subsection{Sur les "sciences"}	
	Il est important que nous d\'efinissions rigoureusement les diff\'erents types de sciences auxquelles l'être humain fait souvent r\'ef\'erence. Effectivement, il semble qu'au 21ème siècle un abus de langage malsain s'instaure et qu'il ne devienne plus possible pour la population de distinguer la "qualit\'e intrinsèque" d'une science d'une autre.

	\begin{tcolorbox}[title=Remarque,colframe=black,arc=10pt]
	Etymologiquement le mot "science" vient du latin "scienta" (connaissance) dont la racine est le verbe "scire" qui veut dire "savoir".
	\end{tcolorbox}
	
	Cet abus de langage vient probablement du fait que les sciences pures et exactes perdent leurs illusions d'universalit\'e et d'objectivit\'e, dans le sens où elles s'auto-corrigent. Ceci ayant pour cons\'equence que certaines sciences sont rel\'egu\'ees au second plan et tentent d'en emprunter les m\'ethodes, les principes et les origines pour cr\'eer une confusion. Il faut ainsi être très prudent au sujet des pr\'etentions de scientificit\'e en sciences humaines, et cela vaut \'egalement (ou surtout) pour les courants dominants en \'economie, en sociologie et en psychologie. Tout simplement, les problèmes trait\'es par les sciences humaines sont extrêmement complexes, peu reproductibles, et les arguments empiriques \'etayant leurs th\'eories sont souvent assez faibles.

	\marginnote{\textcolor{NavyBlue}{{\footnotesize \textbf{~\thechapter:\myparagraph}}}}En soi, la science cependant ne produit pas de v\'erit\'e absolue. Par principe, une th\'eorie scientifique est valable tant qu'elle permet de pr\'edire des r\'esultats mesurables statistiquement et reproductibles\footnote{Donc, ce que nous pouvons lire dans tous les différents livres religieux existant dans le monde, ce ne sont pas des "théories scientifiques" mais des "théories spéculatives"!}. Mais les problèmes inhérents aux interpr\'etations des données que certaines veulent faire de ces r\'esultats statistiques font partie de la philosophie naturelle.
	
	\begin{center}
		\NewTerm{\textbf{Aucune th\'eorie scientifique n'est prouv\'ee ou prouvable. Elle n'est simplement pas r\'efut\'ee tant qu'une exp\'erience ne vient pas dire le contraire.}}
	\end{center}
	Cependant, la m\'ethodologie scientifique est suffisamment fiable pour que le pouvoir juridique ne soit pas l\'egitime à prendre position sur des \'evidences scientifiques.

	\'etant donn\'e la diversit\'e des ph\'enomènes à \'etudier, au cours des siècles s'est constitu\'e un nombre grandissant de disciplines comme la chimie, la biologie, la thermodynamique, etc. Toutes ces disciplines a priori h\'et\'eroclites ont pour socle commun la physique, pour langage la math\'ematique et comme principe \'el\'ementaire la m\'ethode scientifique.
	\begin{tcolorbox}[title=Remarque,colframe=black,arc=10pt]
	Par cons\'equent, gardez à l'esprit qu'un t\'emoignage, ou une simple phrase dans un livre ou dans un s\'eminaire même dite par un scientifique, n'a AUCUNE VALEUR SCIENTIFIQUE ou n'a aucune valeur pour tirer des conclusions, s'il n'est pas accompagn\'e de donn\'ees exp\'erimentales et d'un modèle math\'ematique d\'etaill\'e! Cependant, les t\'emoignages restent utiles pour construire des hypothèses sp\'eculatives et concevoir des exp\'eriences et des \'etudes. Si les hypothèses sp\'eculatives ne sont cependant pas accompagn\'ees d'une m\'ethode exp\'erimentale pour la v\'erifier ou la r\'efuter, alors ce n'est toujours pas de le science mais juste de la pseudo-science de comptoir!!!
	\end{tcolorbox}
	Ainsi, un petit rafraîchissement de m\'emoire peut être utile:

	\textbf{D\'efinitions (\#\mydef):}
	
	\begin{itemize}
		\item[D1.] Nous d\'efinissons par "\NewTerm{science pure}"\index{science pure}, tout ensemble de connaissances fond\'ees sur un raisonnement rigoureux valable quel que soit le facteur (arbitraire) \'el\'ementaire choisi (nous disons alors "ind\'ependant de la r\'ealit\'e sensible") et restreint au minimum n\'ecessaire. Il n'y a que la math\'ematique (appel\'ee souvent "reine des sciences") qui peut être classifi\'ee dans cette cat\'egorie.
	
		\item[D2.] Nous d\'efinissons par "\NewTerm{science exacte}"\index{science exacte} ou "\NewTerm{science dure}"\index{science dure}, tout ensemble de connaissances fond\'ees sur l'\'etude d'une observation, observation qui aura \'et\'e transcrite sous forme symbolique (physique th\'eorique par exemple). Principalement, le but des sciences exactes est non d'expliquer le "pourquoi" mais le "comment".
		
		Et n'oubliez jamais ... La science (en particulier la physique) n'a pas à "faire sens", elle doit juste faire toutes les bonnes pr\'edictions testables (instrumentalisme)! Selon le philosophe Karl Popper, une th\'eorie est scientifiquement acceptable si, comme pr\'esent\'ee, elle peut être "\NewTerm{falsifiable}\index{falsiable}\footnote{C'est pourquoi la philosophie et la logique humaine naïve elles-mêmes ne peuvent rien "prouver". Parce qu'elles dépendent fortement et varient d'un contexte socioculturel où les gens sont nés à un autre (les gens de New-York ont une logique et une philosophie assez différentes de la tribu des Sentinelles...). Sans la falsification, la Science serait une anarchie de modèles logiquement cohérents mais toutefois inutiles et qui conviendraient simplement à la fantaisie de quelqu'un.}" (les synonymes sont "\NewTerm{r\'efutable}\index{r\'efutable}" ou "\NewTerm{testable}\index{testable}"), c'est-à-dire qu'elle peut être soumises à des tests exp\'erimentaux (ou s'il est possible de concevoir une observation ou un argument qui nie l'\'enonc\'e en question). La "connaissance scientifique" est alors par d\'efinition l'ensemble des th\'eories qui ont r\'esist\'e à la falsification (r\'efutation). La science est donc de par sa nature sujette à un questionnement continu.
		
		Attention! Il n'y a aucun doute que les sciences exactes jouissent pour l'instant d'un prestige \'enorme, y compris parmi leur d\'etracteurs, à cause de leurs succès th\'eoriques et pratiques. Il est certain que certains scientifiques abusent parfois de ce prestige en exhibant un sentiment de sup\'eriorit\'e non n\'ecessairement justifi\'e. De plus, il arrive assez souvent que des scientifiques exposent, dans la litt\'erature de vulgarisation, des id\'ees fort sp\'eculatives comme si elles \'etaient bien \'etablies, ou extrapolent leurs r\'esultats en dehors du contexte où ils ont \'et\'e v\'erifi\'es (et encore... à condition qu'elles aient \'et\'e v\'erifi\'ees un jour...).
	
		\begin{tcolorbox}[title=Remarque,colframe=black,arc=10pt]
		Les deux d\'efinitions pr\'ec\'edentes sont souvent incluses dans la d\'efinition de "\NewTerm{sciences d\'eductives}"\index{sciences d\'eductives} ou encore de "\NewTerm{sciences ph\'enom\'enologiques}"\index{sciences ph\'enom\'enologiques}.
		\end{tcolorbox}
		
		\item[D3.] Nous d\'efinissons par "\NewTerm{science de l'ing\'enieur}"\index{science de l'ing\'enieur}, tout ensemble de connaissances th\'eoriques ou pratiques appliqu\'ees aux besoins de la soci\'et\'e humaine tels que: l'\'electronique, la chimie, l'informatique, les t\'el\'ecommunications, la robotique, l'a\'erospatiale, biotechnologies...
	
		\item[D4.] Nous d\'efinissons par "\NewTerm{science}"\index{science} tout ensemble de connaissances fond\'ees sur des \'etudes ou observations de faits dont l'interpr\'etation n'a pas encore \'et\'e retranscrite ni v\'erifi\'ee avec la rigueur math\'ematique, caract\'eristique des sciences qui pr\'ecèdent, mais qui applique des raisonnements comparatifs statistiques. Nous incluons dans cette d\'efinition: la m\'edecine (il faut cependant prendre garde au fait que certaines parties de la m\'edecine \'etudient des ph\'enomènes descriptifs sous forme math\'ematique tels que les r\'eseaux de neurones ou autres ph\'enomènes associ\'es à des causes physiques connues), la sociologie, la psychologie, l'histoire, la biologie...
		
		\item[D5.] Nous définissons un "\NewTerm {scientifique}"\index{scientifique} (dans le sens moderne du terme) comme un professionnel (un actif et donc non un retraité!) qui détient un Doctorat et qui recueille et utilise systématiquement des recherches et des preuves expérimentales reproductibles, pour faire des hypothèses et les tester à l'aide de méthodes scientifiques de pointe (des méthodes statistiques de niveau PhD ou des simulations numériques reproductibles avancées), pour acquérir et partager une compréhension et des connaissances évaluées par les pairs dans des revues scientifiques de référence dans un domaine d'expertise très spécialisé.

		La définition exclut donc l'immense majorité des ingénieurs (qui ne publient pas des articles évalués par les pairs et ne savent pas analyser les données à l'aide de méthodes statistiques de niveau Doctorat), cela exclut les assistants de laboratoire qui ne font que des manipulations expérimentales mais ne font pas d'hypothèses ni analysent les mesures résultantes de leurs expériences mais aussi les médecins (voir \cite{smith2004doctors} et \cite{freed2004doctors}) qui n'appliquent que les résultats de la recherche scientifique mais ne font pas de recherche comme spécifié ci-dessus. Les mathématiciens ne sont pas exclus de cette définition car leurs théories mathématiques (preuves) sont publiées dans des articles évalués par des pairs et sont reproduites (vérifiées) indépendamment par d'autres mathématiciens.
	
		\item[D6.] Nous d\'efinissons par "\NewTerm{science molle}"\index{science molle}, "\NewTerm{para-science}"\index{para-science} ou "\NewTerm{pseudo-science}"\index{pseudo-science} tout ensemble de connaissances ou de pratiques qui sont actuellement bas\'ees sur des faits non v\'erifiables et non r\'efutables (non reproductibles scientifiquement) par l'exp\'erience ou par les math\'ematiques. Nous incluons typiquement dans cette d\'efinition: l'astrologie, la th\'eologie, le paranormal (qui a \'et\'e d\'emoli par la science z\'et\'etique), la graphologie, la justice \footnote{En effet, en Suisse, par exemple, le juge cantonal et le juge f\'ed\'eral ne donnent pas le même jugement puisque ce dernier est non scientifique mais plutôt bas\'e sur l'exp\'erience subjective de la vie du juge et de ses biais cognitifs}, etc.
		
		Comme le disent certains scientifiques: «\textit{Cela ressemble à de la science, cela utilise le vocabulaire de la science ... mais ce n'est pas du tout de la science.}»
		
		Les pseudo-sciences sont particulièrement caract\'eris\'ees par:
		\begin{itemize}
			\item Elles commencent par une conclusion (croire), puis travaillent en arrière pour essayer de confirmer les croyances
	
			\item Elles sont hostiles à la critique et la remise en question
	
			\item Elles utilisent des raisonnements / arguments circulaires
	
			\item Elles utilisent un jargon vague pour cr\'eer de la confusion
	
			\item Elles utilisent des strat\'egies subtiles pour changer l'esprit des gens (en particulier les enfants)
	
			\item Elles font de la "cueillette des cerises" sur des preuves favorables
	
			\item Elles utilisent des m\'ethodes non reproductibles / non r\'efutables avec des r\'esultats non r\'ep\'etables
	
			\item Elles utilisent un langage al\'eatoire pseudo-scientifique pour impressionner le public
	
			\item Elles utilisent une logique incoh\'erente (\'emotionnelle) et invalide
	
			\item Les gens qui travaillent dans le domaine sont dogmatiques et inflexibles
		\end{itemize}
	
		\item[D7.] Nous d\'efinissons par "\NewTerm{sciences ph\'enom\'enologiques}" ou "\NewTerm{sciences naturelles}", toute science qui n'est pas inclue dans les d\'efinitions pr\'ec\'edentes (histoire, sociologie, psychologie, zoologie, biologie,...)
	
		\item[D8.] Le "\NewTerm{scientisme}"\index{scientisme} est une id\'eologie selon laquelle la science exp\'erimentale est le seul mode de connaissance valable, ou, du moins, sup\'erieur à toutes les autres formes d'interpr\'etation du monde. Dans cette perspective, il n'existe pas de v\'erit\'es philosophiques, religieuses ou morales sup\'erieures aux th\'eories scientifiques. Seul compte ce qui est scientifiquement d\'emontr\'e.
	
		\item[D9.] Le "\NewTerm{positivisme}"\index{positivisme} d\'esigne un ensemble de courants qui considère que seules l'analyse et la connaissance des faits r\'eels v\'erifi\'es par l'exp\'erience peuvent expliquer les ph\'enomènes du monde sensible. La certitude en est fournie exclusivement par l'exp\'erience scientifique. Il rejette l'introspection, l'intuition et toute approche m\'etaphysique pour expliquer la connaissance des ph\'enomènes.\\
		
		Ce qui est int\'eressant dans cette doctrine, c'est que c'est certainement une des seules qui demande aux gens de devoir r\'efl\'echir par eux-mêmes et de comprendre l'environnement qui les entoure en remettant continuellement tout en question et sans ne jamais rien accepter comme acquis (...). De plus, les vraies sciences ont ceci d'extraordinaire qu'elles permettent de comprendre au-delà de ce que nous pouvons voir.
	\end{itemize}

Mais enfin, la science, c'est la science, et rien de plus: une certaine mise en ordre, pas trop mal r\'eussie, des choses qui ne conduisent plus à la m\'etaphysique comme du temps d'Aristote, mais qui n'a pas le pr\'etention de nous livrer toute la r\'ealit\'e ni même le fond des choses visibles et cela même si c'est la meilleure m\'ethode d'investigation intellectuelle existante actuellement à notre \'epoque et aussi depuis plusieurs mill\'enaires.

	\pagebreak
	\subsection{Terminologie}

Le tableau m\'ethodique que nous avons pr\'esent\'e plus haut contient des termes qui peuvent peut-être vous sembler inconnus ou barbares. C'est la raison pour laquelle il nous semble fondamental de pr\'esenter les d\'efinitions de ces derniers, ainsi que de quelques autres tout aussi importants qui peuvent \'eviter des confusions malheureuses.

\textbf{D\'efinitions (\#\mydef):}

\begin{itemize}
	\item[D1.] Au-delà de son sens n\'egatif, l'id\'ee de "\NewTerm{problème}"\index{problème} renvoie à la première \'etape de la d\'emarche scientifique. Formuler un problème est ainsi essentiel à sa r\'esolution et permet de comprendre correctement ce qui fait problème et de voir ce qui doit être r\'esolu.\\
	
	Le concept de problème est intimement reli\'e au concept "d'hypothèse" dont nous allons voir la d\'efinition ci-dessous.

	\item[D2.] Une "\NewTerm{hypothèse}"\index{hypothèse} est toujours, dans le cadre d'une th\'eorie d\'ejà constitu\'ee ou sous-jacente, une supposition en attente de confirmation ou d'infirmation qui tente d'expliquer un groupe de faits ou de pr\'evoir l'apparition de faits nouveaux.\\
	
	Ainsi, une hypothèse peut être à l'origine d'un problème th\'eorique qu'il faudra formellement r\'esoudre. 

	\item[D3.] Le "\NewTerm{postulat}"\index{postulat} en physique correspond fr\'equemment à un principe (voir d\'efinition ci-dessous) dont l'admission est n\'ecessaire pour \'etablir une d\'emonstration (nous sous-entendons que cela est une proposition non-d\'emontrable).\\
	
	Ainsi, une hypothèse peut être à l'origine d'un problème th\'eorique qu'il faudra formellement r\'esoudre. 

	\item[D4.] Un "\NewTerm{principe}"\index{principe} (parent proche du "postulat") est donc une proposition admise comme base d'un raisonnement ou une règle g\'en\'erale th\'eorique qui guide la conduite des raisonnements qu'il faudra effectuer. En physique, il s'agit \'egalement d'une loi g\'en\'erale r\'egissant un ensemble de ph\'enomènes et v\'erifi\'ee par l'exactitude de ses cons\'equences.\\
	
	Le mot "principe" est utilis\'e avec abus dans les petites classes ou \'ecoles d'ing\'enieurs par les professeurs ne sachant (ce qui est très rare), ou ne voulant (plutôt fr\'equent), ou ne pouvant faute de temps (quasi exclusivement) d\'emontrer une relation.\\

	L'\'equivalent du postulat ou du principe en math\'ematiques est "l'axiome" que nous d\'efinissons ainsi:

	\item[D5.] Un "\NewTerm{axiome}"\index{axiome} est une v\'erit\'e ou proposition \'evidente par elle-même dont l'admission est n\'ecessaire pour \'etablir une d\'emonstration.\label{axiom} 
\end{itemize}

	\begin{tcolorbox}[title=Remarques,colframe=black,arc=10pt]
	\textbf{R1.} Nous pourrions dire que c'est quelque chose que nous posons comme une v\'erit\'e pour le discours que nous nous proposons de tenir, comme une règle du jeu, et qu'elle n'a pas forc\'ement par ailleurs une valeur de v\'erit\'e universelle dans le monde sensible qui nous entoure.\\

	\textbf{R2.} Les axiomes doivent toujours être ind\'ependants (on ne doit pas pouvoir d\'emontrer l'un à partir de l'autre) et non contradictoires (nous disons \'egalement parfois qu'ils doivent être "consistants").
	\end{tcolorbox}	
	
\begin{itemize}
	\item[D6.]  Le "\NewTerm{corollaire}"\index{corollaire} est un terme malheureusement quasi inexistant en physique (à tort !) et qui est en fait une proposition r\'esultant d'une v\'erit\'e d\'ejà d\'emontr\'ee. Nous pouvons \'egalement dire qu'un corollaire est une cons\'equence n\'ecessaire et \'evidente d'un th\'eorème (ou parfois d'un postulat en ce qui concerne la physique).

	\item[D7.] Un "\NewTerm{lemma}"\index{lemma} constitue une proposition d\'eduite d'un ou de plusieurs postulats ou axiomes et dont la d\'emonstration pr\'epare celle d'un th\'eorème.
\end{itemize}

	\begin{tcolorbox}[title=Remarque,colframe=black,arc=10pt]
	Le concept de "lemme" est lui aussi (et c'est malheureux) quasi r\'eserv\'e aux math\'ematiques.
	\end{tcolorbox}	

\begin{itemize}
	\item[D8.] Une "\NewTerm{conjecture}"\index{conjecture} constitue une supposition ou opinion fond\'ee sur la vraisemblance d'un r\'esultat math\'ematique.
	
	Beaucoup de conjectures jouent un rôle un peu comparable à des lemmes, car elles sont des passages oblig\'es pour obtenir d'importants r\'esultats.
	
	\item[D9.] Par-delà son sens faible de conjecture, une "\NewTerm{th\'eorie}"\index{th\'eorie} ou "\NewTerm{th\'eorème}"\index{th\'eorème} est un ensemble articul\'e autour d'une hypothèse et \'etay\'e par un ensemble de faits ou d\'eveloppements qui lui confèrent un contenu positif et rendent l'hypothèse bien fond\'ee (ou tout au moins plausible dans le cas de la physique th\'eorique).

	\item[D10.]  Une "\NewTerm{singularit\'e}"\index{singularit\'e} est une ind\'etermination d'un calcul qui intervient par l'apparition d'une division par le nombre z\'ero. Ce terme est aussi bien utilis\'e en math\'ematique qu'en physique.  

	\item[D11.] Une "\NewTerm{d\'emonstration}"\index{d\'emonstration} constitue un ensemble de proc\'edures math\'ematiques à suivre pour d\'emontrer le r\'esultat d\'ejà connu ou non d'un th\'eorème.

	\item[D12.] Si le mot "\NewTerm{paradoxe}"\index{paradoxe} signifie \'etymologiquement: contraire à l'opinion commune, ce n'est cependant pas par pur goût de la provocation, mais bel et bien pour des raisons solides. Le "\NewTerm{sophisms}"\index{sophisms} quant à lui, est un \'enonc\'e volontairement provocateur, une proposition fausse reposant sur un raisonnement apparemment valide. Ainsi parle-t-on du fameux "paradoxe de Z\'enon", alors qu'il ne s'agit que d'un sophisme. Le paradoxe ne se r\'eduit pas à de la fausset\'e, mais implique la coexistence de la v\'erit\'e et de la fausset\'e, au point qu'on ne parvient plus à discriminer le vrai et le faux. Le paradoxe apparaît alors problème insoluble ou "\NewTerm{aporia}"\index{aporia}. 
	
\end{itemize}

	\begin{tcolorbox}[title=Remarque,colframe=black,arc=10pt]
	Ajoutons que les grands paradoxes, par les interrogations qu'ils ont suscit\'ees, ont fait progresser la science et amen\'e des r\'evolutions conceptuelles de grande ampleur, en math\'ematique comme en physique th\'eorique (les paradoxes sur les ensembles et sur l'infini en math\'ematique, ceux à la base de la relativit\'e et de la physique quantique).
	\end{tcolorbox}	

	%to make section start on odd page
	\newpage
	\thispagestyle{empty}
	\mbox{}
	\section{Science et Foi}
	\lettrine[lines=4]{\color{BrickRed}N}ous verrons qu'en science, une th\'eorie est normalement incomplète, car elle ne peut d\'ecrire exhaustivement la complexit\'e du monde r\'eel (except\'e pour la Physique Quantique ou la Relativit\'e G\'en\'erale). Il en est ainsi de toutes les th\'eories, comme celle du Big Bang ((\SeeChapter{voir section d'Astrophyque page \pageref{astrophysics}}) ou de l'\'evolution des espèces (\SeeChapter{voir sections de Dynamique des Populations page \pageref{population dynamics} ou de Th\'eorie des Jeux et de la D\'ecision page \pageref{game and decision theory}}) ne serait-ce que parce qu'elles ne sont pas reproductibles dans des conditions identiques.
	
	\begin{center}
		\includegraphics[scale=0.3]{img/intro/science_we_trust.jpg}
	\end{center}
	
	Il convient de distinguer diff\'erents courants scientifiques majeurs: 
	\begin{itemize}
		\item Le "\NewTerm{r\'ealisme}"\index{r\'ealisme} est une doctrine où les th\'eories physiques ont pour objectif de d\'ecrire la r\'ealit\'e telle qu'elle est en soi, dans ses composantes inobservables.
	
		\item L'\NewTerm{Instrumentalisme}"\index{instrumentalisme} est une doctrine où les th\'eories sont des outils servant à pr\'edire des observations mais qui ne d\'ecrivent pas la r\'ealit\'e en soi.
	
		\item Le "\NewTerm{fictionalisme}"\index{fictionalisme} est le courant où le contenu r\'ef\'erentiel (principes et postulats) des th\'eories est un leurre, utile seulement pour assurer l'articulation linguistique des \'equations fondamentales.
	\end{itemize}

	Même si aujourd'hui les th\'eories scientifiques ont le soutien de beaucoup de sp\'ecialistes, les th\'eories alternatives ont des arguments valables et nous ne pouvons totalement les \'ecarter. Pour autant, la cr\'eation du monde en 7 jours d\'ecrite par la Bible ne peut plus être perçue comme un possible, et bien des croyants reconnaissent qu'une lecture litt\'erale est peu compatible avec l'\'etat actuel de nos connaissances et qu'il est plus sage de l'interpr\'eter comme une parabole (même s'il est écrit dans leurs propres livre que ce type de d'exercice intellectuel est interdit...). Si la science ne fournit jamais de r\'eponse d\'efinitive, il n'est plus possible de ne pas en tenir compte.
	
	La foi (qu'elle soit religieuse, superstitieuse, pseudo-scientifique ou autre non pilot\'ee par les donn\'ees) a au contraire pour objectif de donner des v\'erit\'es absolues d'une toute autre nature puisqu'elle relève d'une conviction personnelle inv\'erifiable et non reproductibles (par exemple, la science n\'ecessite des preuves ou des \'evidences exp\'erimentales pour être valable alors que les religions n\'ecessitent que la foi pour être consid\'er\'ees comme valable). C'est pourquoi un certain nombre de gens disent que \textit{la Science s'ajuste en fonction des observations alors que la foi est le rejet de l'observation afin que les croyances puissent être conserv\'ees}... En fait, l'une des fonctions des religions est de fournir du sens à des ph\'enomènes qui ne sont pas explicables rationnellement\footnote{Ce fut avec la pluie, le tonnerre, les maladies, les \'etoiles, les comètes, les tremblements de terre, les \'eruptions volcaniques, etc. il y a quelques centaines d'ann\'ees et est souvent d\'esign\'e par les scientifiques sous le nom d'argument d'ignorance. \index{argument d'ignorane}}. Les progrès de la connaissance entraînent donc parois une remise en cause des dogmes religieux par la science\footnote{Quand les théistes disent qu'un dieu doit exister parce que la Terre est "parfaitement" placée pour la vie, ils supposent que leur dieu est limité à placer la vie là où elle pourrait se produire de toute façon naturellement...}. 
	\begin{fquote}La SCIENCE n'est pas définie par ce que nous croyons. Elle N'EST PAS un système de croyance. Elle n'a pas de doctrine religieuse! Il s'agit simplement d'un système intellectuel qui rejette une affirmation qui n'a aucune évidence reproductible expérimentale à l'appui.
 	\end{fquote}
 	\begin{center}
		\includegraphics[scale=0.6]{img/intro/science_like_religion.jpg}
	\end{center}
	A contrario, sauf à pr\'etendre imposer sa foi (qui n'est autre qu'une conviction intimement personnelle et subjective) aux autres, il faut se d\'efier de la tentation naturelle de qualifier de fait scientifiquement prouv\'e les extrapolations des modèles scientifiques au-delà de leur champ d'application.
	
	Le mot "Science" est, comme nous l'avons d\'ejà mentionn\'e plus haut, de plus en plus utilis\'e pour soutenir qu'il existe des preuves scientifiques là où il n'y a que croyance (certaines pages web de ce genre prolifèrent de plus en plus). Selon ses d\'etracteurs c'est le cas, par exemple, du mouvement de scientologie (mais il y en a beaucoup d'autres). Selon ces derniers, nous devrions plutôt parler de "\NewTerm{sciences occultes}"\index{sciences occultes}.

	Les sciences occultes et sciences traditionnelles existent depuis l'Antiquit\'e; elles consistent en un ensemble de connaissances et de pratiques myst\'erieuses ayant pour but de p\'en\'etrer et dominer les secrets de la nature. Au cours des derniers siècles, elles ont \'et\'e progressivement exclues du champ de la science. Le philosophe Karl Popper s'est longuement interrog\'e sur la nature de la d\'emarcation entre science et pseudoscience. Après avoir remarqu\'e qu'il est possible de trouver des observations pour confirmer à peu près n'importe quelle th\'eorie, il propose une m\'ethodologie fond\'ee sur la r\'efutabilit\'e. Une th\'eorie doit selon lui, pour m\'eriter le qualificatif de "scientifique", pouvoir garantir l'impossibilit\'e de certains \'ev\'enements. Elle devient dès lors r\'efutable, donc (et alors seulement) apte à int\'egrer la science. Il suffirait en effet d'observer un de ces \'ev\'enements pour invalider la th\'eorie, et s'orienter par cons\'equent sur une am\'elioration de celle-ci.
	
	Et notons aussi que la diff\'erence majeure entre les livres de sciences et les livres de religions est que si vous d\'etruisiez ces derniers, dans un millier d'ann\'ees, la probabilit\'e qu'ils soient r\'e\'ecrits naturellement à l'identique est très faible. Alors que si nous prenions chaque livre de science et chaque note de chaque observation exp\'erimentale et les d\'etruisions tous, dans mille ans ils seraient refaits de manière identiques et ce pour la simple raison que la majorit\'e des tests exp\'erimentaux redonneraient les mêmes r\'esultats (observations) et que la math\'ematique est ind\'ependante de l'endroit où nous sommes n\'es sur Terre et du choix religieux que nous impose la famille dans la majorit\'e des cas.
	\begin{fquote}La raison pour laquelle il y a un conflit entre la science et les religions est que la science ne cesse de réfuter les choses que les religions prétendent être vraies ou exactes!
	\end{fquote}
	\begin{tcolorbox}[title=Remarque,colframe=black,arc=10pt]
	Si la science et, par-dessus tout, les MATHÉMATIQUES sont le langage d'un quelconque Dieu, alors pourquoi tous les livres saints sont-ils écrits sans mathématiques? Oh, euh ... nous savons pourquoi!
	\end{tcolorbox}
	
	\begin{center}
		\includegraphics[scale=0.55]{img/intro/methode_scientifique_vs_methode_religieuse.jpg}
	\end{center}
	
	Ceci est important car chacune des séries interminables de pseudo-preuves de l'existence de divers dieux qui a été proposée, de l'Antiquité à nos jours, est automatiquement un échec car, comme déjà mentionné, une déduction logique ne nous dit rien qui soit pas déjà intégré dans ses locaux. Tout ce que la logique peut faire pour nous est de tester l'auto-cohérence de ces prémisses. Il n'y a qu'un seul moyen fiable que les humains ont découvert jusqu'à présent pour obtenir des connaissances qu'ils ne possèdent pas déjà: l'observation! Et la science est la collecte méthodique d'observations et la construction et le test de modèles pour décrire ces observations. Sans la falsification, la science serait une anarchie de modèles logiquement cohérents mais toujours inutiles qui conviennent tout simplement à la fantaisie de quelqu'un !
	
	\begin{fquote}Un scientifique lira des centaines de livres au cours de sa vie, mais sera toujours persuadé qu'il lui reste beaucoup à apprendre. Un fanatique religieux n'en lira qu'un et sera persuadé d'avoir tout compris.
 	\end{fquote}
	
	Notons que les religions (et particulièrement les scientifiques croyants) ont eu des milliers d'années pour prouver que n'importe quel dieu existe. Pourtant, les croyants ne peuvent pas faire mieux à ce jour que: \og \textit{VOUS VERREZ QUAND VOUS SEREZ MORT} \fg{}. Cela en dit long ... d'autant plus que si les dieux étaient réels, ils n'auraient pas besoin de mortels pour défendre leur existence! Et si les croyants accordent vraiment plus d'importance à leur foi qu'en la science, ils peuvent le prouver facilement: la prochaine fois qu'ils tombent malades, ils peuvent aller dans leur lieu culte (temple, mosquée ou église) au lieu d'aller à l'hôpital!
	
	\begin{fquote}La religion est l'art d'utiliser des arguments absurdes pour expliquer l'ignorance.
 	\end{fquote}
	
	Le lecteur doit également savoir qu'à l'opposé d'un mythe commun auprès des personnes scientifiquement illétrées, tout phénomène ou hypothèse (même surnaturel!) s'il est bien défini peut être testé par la méthode scientifique. C'est pourquoi nous avons (parmi beaucoup d'autres) des évidences scientifiques que les prières n'ont pas d'effet thérapeutique meilleur que celui d'un placebo. Voir par exemple sur un sujet considéré comme surnaturel l'étude suivante \textit{Study of the Therapeutic Effects of Intercessory Prayer (STEP) in cardiac bypass patients: a multicenter randomized trial of uncertainty and certainty of receiving intercessory prayer} \cite{benson2006study} pour un assez bon protocole et analyse scientifique ou l'article suivant \textit{Positive therapeutic effects of intercessory prayer in a coronary care unit population} \cite{byrd1988positive} pour un mauvais protocole et analyse scientifique. Il existe également quelques méta-analyses sur ce sujet surnaturel...
	
	\begin{fquote} Peu importe si un grand scientifique était religieux. Ce qui compte, c'est qu'aucun d'eux n'a jamais prouvé que son dieu existe! \end{fquote}
	
	Les faits exaspèrent les croyants, qui les poussent au solipsisme. Leur seul moyen de défendre alors leurs croyances qui ne sont pas concordantes avec les données scientifiques est de remettre en cause la réalité elle-même (c'est pourquoi avant de débattre avec eux les scientifiques doivent se mettre d'accord avec eux sur une définition de la "réalité" sinon tout débat est inutile) ! Donc à chaque fois que vous vous disputez avec les croyants, parce qu'ils ne peuvent pas attaquer les faits alors ils attaquent l'épistomologie. Comme c'est comme si leur position était si faible que leur dieu ne pourrait pas être réel à moins que la réalité ne le soit pas...
	
	Enfin citons un passage de \cite{2014traité}:
	
	\og Le peuple (j'entends par ce mot le vulgaire ramassé, la tourbe et lie populaire, gens, sous quelque couvert que ce soit de basse, servile et mécanique condition) est une bête à plusieurs têtes, vagabonde, errante, folle, étourdie, sans conduite, sans esprit, ni jugement. 
	
	Que Postel lui persuade que Jésus-Christ n'a sauvé que les hommes et que la mère Jeanne doit sauver les femmes, il le croira soudain. Que David George se dise fils de Dieu, il l'adorera, qu'un tailleur enthousiaste et fanatique contrefasse le roi dans Munster et dise que Dieu l'a destiné pour châtier toutes les puissances de la Terre, il lui obéira et le respectera comme le plus grand monarque du monde. Que le Père Domptius lui annonce la venue de l'Antéchrist, qu'il est àgé de dix ans [et] qu'il a des cornes, il témoignera de s'en effrayer. Que des imposteurs et charlatans se qualifient Frères de la Rose-Croix, il courra après eux. Qu'on lui rapporte que Paris doit bientôt s'abîmer, il s'enfuira. Que tout le monde doit être submergé, il bâtira des arches et des bateaux de bonne heure pour n'être pas surpris. Que la mer doit sécher et que des chariots pourront aller de Gênes à Jérusalem, il se préparera pour faire le voyage.
	
	Qu'on lui conte les fables de Mélusine, du Sabbat des Sorciers, des Loups-garous, des Lutins, des Fées, des Parèdres, il les admirera. Que la matrice tourmente quelque pauvre fille, il dira qu'elle était possédée, ou croira à quelque prêtre ignorant ou méchant, qui la fait passer pour telle. Que quelque alchimiste, magicien, astrologue, lulliste, cabaliste, commencent un peu à le cajoler, il les prendra pour les plus savants et pour plus honnêtes gens du monde. Qu'un Pierre l'hermite vienne prêcher la croisade, il fera des reliques du poil de son mulet.
Qu'on lui dise en riant qu'une cane ou un oiseau sont inspirés du Saint-Esprit, il le croira sérieusement. Que la peste ou la tempête ruine une province, il en accusera soudain des graisseurs ou magiciens. Bref, si on le trompe aujourd'hui, il se laissera encore surprendre demain, ne faisant jamais profit des rencontres passées, pour se gouverner dans les présentes ou futures; et en ces choses consistent les principaux signes de sa grande faiblesse et imbécibilité.\fg{}
	
	\pagebreak
	\subsection{Kit de D\'etection de Balivernes }
	Grâce à leur formation, la majorit\'e des scientifiques sont \'equip\'es de ce que Carl Sagan appelait "\NewTerm{kit de d\'etection de balivernes }\index{kit de d\'etection de balivernes}" ou "\NewTerm{kit de d\'etection de conneries}\index{kit de d\'etection de conneries}" qui sont des outils cognitifs et des techniques qui fortifient l'esprit contre les arguments fallacieux et biais\'es et qui permettent de d\'efinir des limites entre la science et la pseudoscience ou plus simplement entre le rationnel et l'\'emotionnel. Ce n'est pas seulement un outil de la science, il contient des outils inestimables de scepticisme sain qui s'appliquent tout aussi \'el\'egamment, et tout aussi n\'ecessairement, à la vie quotidienne. En adoptant le kit, nous pouvons tous nous prot\'eger contre la ruse et la manipulation d\'elib\'er\'ee.
	
	Il existe de nombreuses versions de ces outils de d\'etection mais en voici une assez complète (mais encore incomplète par construction) propos\'ee par Michael Shermer (\'editeur fondateur de \href{http://www.skeptic.com}{Skeptic Magazine} et auteur de \textit{The Borderlands of Science}):
	
	\begin{enumerate}[label=\protect\circledbullet{\arabic*},leftmargin=15mm]
		\item \textit{\textbf{Quelle est la fiabilit\'e de la source de l'information?}}

		Les pseudoscientifiques semblent souvent assez fiables, mais lorsqu'on les examine de près, les faits et les chiffres qu'ils citent sont d\'eform\'es, non sourc\'es, pris hors contexte ou parfois même fabriqu\'es. Bien sûr, tout le monde fait des erreurs. Et en tant qu'historien de la science, Daniel Kevles a montr\'e assez efficacement dans son livre \textit{The Baltimore Affair}, qu'il peut être difficile de d\'etecter un signal frauduleux dans un bruit de fond et que la n\'egligence peut alors être parfois consid\'er\'ee comme une partie normale du processus scientifique. La question est: Est-ce que les donn\'ees et les interpr\'etations montrent des signes de distorsion intentionnelle? Lorsqu'un comit\'e ind\'ependant \'etabli pour enquêter sur une fraude potentielle a examin\'e un ensemble de notes de recherche dans le laboratoire du laur\'eat du prix Nobel David Baltimore, il a r\'ev\'el\'e un nombre surprenant d'erreurs. Baltimore a \'et\'e exon\'er\'e parce que les erreurs de son laboratoire \'etaient al\'eatoires et non directionnelles ... Donc, en science, il n'y a pas d'autorit\'es. Tout au plus, il y a des experts!

		\item \textit{\textbf{Est-ce que cette source fait souvent des affirmations similaires?}}

		Les pseudoscientifiques ont l'habitude d'aller bien au-delà des faits. Les g\'eologues des inondations (les cr\'eationnistes qui croient que les inondations de No\'e peuvent repr\'esenter de nombreuses formations g\'eologiques de la Terre) font constamment des affirmations qui n'ont aucun rapport avec la science g\'eologique. Bien sûr, certains grands penseurs vont souvent au-delà des donn\'ees dans leurs sp\'eculations cr\'eatives. Thomas Gold de l'Universit\'e Cornell est connu pour ses id\'ees radicales, mais il a souvent eu raison de dire que d'autres scientifiques \'ecoutaient ce qu'il avait à dire. Gold propose, par exemple, que le p\'etrole n'est pas du tout un combustible fossile mais le sous-produit d'une biosphère chaude et profonde (micro-organismes vivant à des profondeurs inattendues dans la croûte). Presque tous les scientifiques de la terre avec qui j'ai parl\'e pensent que Gold a raison, et pourtant ils ne le considèrent pas comme une r\'ef\'erence intelectuelle dans le domaine de la g\'eologie. M\'efiez-vous donc des tendances marginales qui ignorent ou d\'eforment constamment les donn\'ees exp\'erimentales!

		\item \textit{\textbf{Est-ce que la source a \'et\'e v\'erifi\'ee par par d'autres experts ind\'ependents?}}
		
		Typiquement, les pseudoscientifiques font des d\'eclarations qui ne sont v\'erifi\'ees ou v\'erifiables que par une source dans leur propre cercle de croyance. Nous devons alors nous demander qui v\'erifie la source et même qui v\'erifie les v\'erificateurs? Le plus gros problème avec la d\'ebâcle de la fusion froide, par exemple, n'\'etait pas que Stanley Pons et Martin Fleischman avaient tort. C'est qu'ils ont annonc\'e leur d\'ecouverte spectaculaire lors d'une conf\'erence de presse avant que d'autres laboratoires ne le v\'erifient. Pire, lorsque la fusion à froid n'a pas \'et\'e reproduite, ils ont continu\'e à s'accrocher à leurs arguments. La v\'erification ext\'erieure est essentielle à la bonne science et à un esprit sain.

		\item \textit{\textbf{Comment la source correspond-elle à ce que nous savons de la façon scientifique dont le monde fonctionne?}}

		Une revendication extraordinaire par une source pseudoscientifique doit être plac\'ee dans un contexte plus large pour voir comment elle s'intègre. Quand les gens pr\'etendent que les pyramides \'egyptiennes et le Sphinx ont \'et\'e construits il y a plus de 10'000 ans par une race inconnue et avanc\'ee, ils ne pr\'esentent aucun contexte pour cette civilisation ant\'erieure. Où sont les autres artefacts de ces autres races? Où sont leurs oeuvres d'art, leurs armes, leurs vêtements, leurs outils, leurs d\'echets? L'arch\'eologie ne fonctionne tout simplement pas de cette façon!

		\item \textit{\textbf{Est-ce qu'il y a des experts qui r\'efutent la source, ou a-t-on seulement des experts qui la supportent?}}

		Cette situation correspond au "biais de confirmation" (nous reviendrons sur les biais cognitif dans la section sur la Th\'eorie de la D\'ecision \pageref{cognitive bias}), ou la tendance à rechercher des preuves confirmatives et à rejeter ou ignorer les preuves allant dans le sens contraire. Le biais de confirmation est puissant, omnipr\'esent et presque impossible à \'eviter pour chacun d'entre nous. C'est pourquoi les m\'ethodes scientifiques qui mettent l'accent sur la v\'erification et la rev\'erification, la r\'eplication à l'identique ou par d'autres approces alternatives, et en particulier les tentatives de falsification d'une revendication, sont si essentielles!

		\item \textit{\textbf{La pr\'epond\'erance de la preuve va-t-elle dans le sens de la source ou mène-t-elle à une conclusion diff\'erente?}}

		La th\'eorie de l'\'evolution, par exemple, est appuy\'ee à travers une convergence d'\'evidences provenant d'un certain nombre d'exp\'eriences ind\'ependantes sur des sujets diff\'erents. Aucun fossile, aucune pièce biologique ou pal\'eontologique ne porte \'ecrit sur elle le mot "\'evolution"; Au lieu de cela, des dizaines de milliers d'\'el\'ements probants s'ajoutent à l'histoire de l'\'evolution de la vie. Les cr\'eationnistes ignorent commod\'ement cette confluence, se concentrant plutôt sur des anomalies triviales ou des ph\'enomènes actuellement inexpliqu\'es dans l'histoire de la vie et de la Terre en g\'en\'eral.

		\item \textit{\textbf{La source utilise-t-elles les règles de la m\'ethode scientifique, ou les a-t-on abandonn\'ees au profit d'autres (typiquement \'emotionnelles) qui mènent à la conclusion souhait\'ee?}} 

		Une distinction claire peut être faite entre les scientifiques SETI (Search for Extraterrestrial Intelligence) et les ufologues. Les scientifiques du SETI partent de l'hypothèse nulle que les ETI n'existent pas et qu'elles doivent fournir des preuves concrètes avant de faire l'affirmation extraordinaire que nous ne sommes pas seuls dans l'Univers. Les ufologues commencent par l'hypothèse positive que les ETI existent et nous ont visit\'e, puis utilisent des techniques de recherche douteuses pour soutenir cette croyance, comme la r\'egression hypnotique (r\'ev\'elations d'exp\'eriences d'abduction), le raisonnement anecdotique (innombrables histoires d'observations d'OVNIS), la pens\'ee conspiratrice (le gouvernement nous ment sur les rencontres extraterrestres), des preuves visuelles de mauvaise qualit\'e (photographies floues et vid\'eos granuleuses), et la pens\'ee anomaliste (anomalies atmosph\'eriques et perceptions erron\'ees visuelles par des t\'emoins oculaires). Ce type de d\'emarche intellectuelle est aussi celle des religions.

		\item \textit{\textbf{Le source fournit-elle une explication pour les ph\'enomènes observ\'es ou simplement nie-t-il l'explication existante?}}
	
		C'est une strat\'egie de d\'ebat classique: critiquez votre adversaire et n'affirmez jamais ce que vous croyez pour \'eviter les critiques. Il est presque impossible d'amener les cr\'eationnistes à offrir une explication à la vie (autre que "Dieu l'a fait"). Les cr\'eationnistes de la conception intelligente (ID) n'ont rien fait de mieux, se d\'ebarrassant des faiblesses des explications scientifiques pour les problèmes difficiles et l'offre à leur place. "IL l'a fait." Ce stratagème est inacceptable en science mais pourtant tellement courant dans les d\'ebats qui ont lieu même en dehors de la sphère des sciences (politique, religions, etc.).

		\item \textit{\textbf{Si la source pr\'esente une nouvelle explication, est-ce que cela tient compte d'autant de ph\'enomènes que l'ancienne explication?}}
	
		Beaucoup de sceptiques du VIH / SIDA affirment que le mode de vie cause le SIDA. Pourtant, leur th\'eorie alternative n'explique pas autant de donn\'ees que la th\'eorie du VIH. Pour faire valoir leur argument, ils doivent ignorer les diverses preuves à l'appui du VIH comme vecteur causal du sida tout en ignorant la corr\'elation significative entre l'augmentation du SIDA chez les h\'emophiles peu de temps après l'introduction du VIH dans l'approvisionnement en sang par inadvertance.

		\item \textit{\textbf{Les convictions personnelles et les pr\'ejug\'es (biais) de la source conduisent-elles aux conclusions, ou vice versa?}}

		Tous les scientifiques (et plus particulièrement les non-scientifiques qui ne sont pas form\'es pendant de nombreuses ann\'ees) ont des croyances sociales, politiques et id\'eologiques qui pourraient biaiser leurs interpr\'etations des donn\'ees et de la situation (c'est un "biais de confirmation" \'egalement appel\'e "biais de s\'election de cerises" qui est aussi la principale cause de rejet des r\'esultats et des outils scientifiques par les non-scientifiques), mais comment ces pr\'ejug\'es et ces croyances affectent-ils leur recherche dans la pratique? Habituellement, pendant le système d'\'evaluation par les pairs, ces pr\'ejug\'es et croyances sont extirp\'es, ou le papier ou le livre est rejet\'e. 
	\end{enumerate}
	
	Un certain nombre d'humains et particulièrement les croyants (pas que dans les religions mais sur tout sujet pour lequel il n'y a aucune évidence scientifique) sont doués pour ce que l'on appelle le "concordisme". L'idée sous-jacente du concordisme est d'interpréter des textes ou trouver des analogies aléatoires dans la Nature (en ayant évidemment zéro connaissances en combinatoire et probabilités) pour justifier que toutes les inventions ou découvertes des sciences modernes ont été annoncées dans leurs livres "saints" ou créés par des gens appartenant au même groupe de croyance qu'eux des centaines ou milliers d'années avant les autres. Évidemmment tout cela masqué sous un rhétorique fumeuse, des analogies qui font zéro sens (comparer les études statistiques avec des textes simples...), aucune méthodologie statistique, en omettant de citer des faits ou conclusions contradictoires, et avec une absence totale de croisement de sources et de vérifications par les pairs! C’est le triomphe de l’obscurantisme!!!
	
	On note majoritairement que cette démarche concordiste est l'apanage des gens peu éduqués ou endoctrinés depuis l'enfance (plus ceux souffrant de troubles mentaux pour diverses raisons) qui ne sont pas capables d'identifier leurs propres biais,  qui sont fermés à toute contradiction et qui ne sont pas capables d'itendifier la complexité réelle (multifactorialité) et les nuances d'un sujet même simple (ils ont tendance à tout binariser en Vrai/Faux).
	
	\begin{fquote}Vous pouvez gagner un débat contre 100'000 savants avec un unique fait, mais vous ne battrez très probablement jamais un idiot même avec 100 faits.
 	\end{fquote}
	
	\begin{figure}[H]
		\centering
		\includegraphics[width=1.0\textwidth]{img/intro/baloney_detection_toolkit.jpg}
	\end{figure}
	En ajustant, nous pouvons aller plus loin sur les sophismes de raisonnement. Voici une liste plus exhaustive avec des cas ultra ultra classiques (qui \'enervent \'enorm\'ement par ailleurs dans les d\'ebats les gens auxquels on indique clairement qu'ils tombent dans un ou plusieurs de ces sophismes):
	\begin{enumerate}
		\item Ad hominem: Un argument ad hominem attaque le messager, pas le message lui-même.

		\item Argument d'autorit\'e: Argument qui repose sur l'identit\'e d'une autorit\'e plutôt que sur les composants de l'argument lui-même.

		\item Argument des cons\'equences n\'egatives: Dire que parce que les implications d'une d\'eclaration \'etant vraie cr\'eeraient des r\'esultats n\'egatifs, cela ne doit pas être vrai.

		\item Appel à l'ignorance: Si quelque chose n'est pas connu pour être faux, cela doit être vrai.

		\item Plaidoyer sp\'ecial: \'enoncer un principe universel, en insistant sur le fait qu'il ne s'applique pas à vos affirmations pour une raison quelconque.

		\item Exposer la question / supposer la r\'eponse: Cela se produit quand une d\'eclaration a une pr\'emisse non prouv\'ee. Ce sophisme est \'egalement appel\'e "raisonnement circulaire" ou "logique circulaire".

		\item S\'election observationnelle: En regardant seulement des preuves positives tout en ignorant les n\'egatifs et vice versa.

		\item Statistiques de petits nombres: Utilisation de petits nombres pour signaler de fortes augmentations en pourcentage.

		\item M\'econnaissance de la nature des statistiques: L'ignorance à propos des hypothèses statistiques centrales et la d\'efinition des paramètres (la confusion entre la corr\'elation et la causalit\'e, la taille de l'\'echantillon et la haine du biais math\'ematique sont des exemples bien connus).

		\item Post hoc, ergo propter hoc: Fonder un effet sur une cause uniquement sur la base de la chronologie.

		\item Fausse dichotomie: Repr\'esenter un problème ou un argument comme n'ayant que deux options et aucun spectre entre les deux (biais d'estimation ponctuelle).

		\item Court terme vs Long terme: En supposant qu'une tendance actuelle demeurera constante tout au long de son histoire et continuera de le faire à l'avenir, même si rien ne laisse croire qu'une telle extrapolation soit justifi\'ee.

		\item Pente glissante, li\'ee à la fausse dichotomie: Dire que quelque chose ne va pas parce que c'est à côt\'e ou vaguement li\'e à quelque chose de mal.

		\item Preuves r\'eprim\'ees et demi-v\'erit\'es: tirer une conclusion injustifi\'ee de pr\'emisses qui sont au moins en partie correctes.

		\item Paroles en l'air (manque de franchise): L'utilisation de r\'ef\'erences vagues et non sp\'ecifiques.
	\end{enumerate}
	
	En plus de nous enseigner ce qu'il faut faire au minimum lors de l'\'evaluation d'une nouvelle source de connaissances, tout bon kit de d\'etection de balivernes doit \'egalement nous apprendre ce qu'il ne faut pas faire. Cela nous aide à reconnaître les erreurs les plus communes et les plus p\'erilleuses de la logique et de la rh\'etorique. Beaucoup de bons exemples peuvent être trouv\'es dans la religion et la politique, parce que leurs praticiens sont souvent oblig\'es de justifier deux propositions contradictoires (sans parler des d\'ebats sur les r\'eseaux sociaux...).

	Enfin, citons Lavoisier: «\textit{Le physicien peut aussi, dans le silence de son laboratoire et de son cabinet, exercer des fonctions patriotiques; il peut esp\'erer par ses travaux diminuer la masse des maux qui affligent bonheur et, n'eût-il contribu\'e, par les routes nouvelles qu'il s'est ouvertes, qu'à prolonger de quelques ann\'ees, de quelques jours, la vie moyenne des hommes, il pourrait aspirer aussi au titre glorieux de bienfaiteur de l'humanit\'e.}»
	\begin{center}
		\includegraphics[scale=0.30]{img/humour/evidence_based.jpg}	
	\end{center}
	
	\pagebreak
	\section{L'effet retour de flamme en Sciences}
	\lettrine[lines=4]{\color{BrickRed}U}n autre point important qu'il est important de souligner au sujet de la communication scientifique: Scientifiques, arrêtez de penser qu'expliquer la science va r\'esoudre les problèmes et \'eviter les pr\'ejug\'es, surtout si vous vous trouvez dans un \'etat d'incr\'edulit\'e ou d'\'evidence qui vous rend fou relativement aux nombreux complots comme celui de la Terre plate. les vaccins, le changement climatique, etc., car les m\'edias traditionnels (ou "pseudo-journalistes") ne savent pas comment communiquer sur les sujets scientifiques!!!

	Les raisons sont majoritairement les suivantes et s'appliquent en dehors du cas où les gens viennent vous \'ecouter ou \'ecouter d'autres scientifiques dans le cadre d'une conf\'erence ou d'un s\'eminaire:
	\begin{enumerate}
		\item La plupart des gens ne veulent pas \'ecouter quoi que ce soit à propos de la m\'ethode scientifique surtout quand ils ne vous ont jamais demand\'e «est-ce vrai?», «Est-ce la meilleure m\'ethode?», «N'est-ce pas un biais?». Si vous utilisez "votre science" juste pour souligner qu'ils ont tort sur ce qu'ils disent ou en argumentant vous allez juste les sortir de leur zone de confort et les faire en plus hair la science et les scientifiques (anti-intellectualisme!).
		
		\begin{figure}[H]
			\centering
			\includegraphics[scale=0.35]{img/intro/scientists_arent_arrogant.jpg}	
		\end{figure}
		
		\item La plupart des humains sont pleins de pr\'ejug\'es, de biais et opportunistes, ils n'aiment alors pas admettre que cela est vrai car ils supposent que l'humain est au sommet de l'\'evolution et ne peut donc pas avoir de tels d\'efauts. Donc, quand vous leur expliquez qu'ils ont des biais, vous faites simplement remarquer qu'ils ne sont pas fiables. Parlez donc de partialit\'e seulement si les gens vous demandent de le faire.
		
		\item La grande majorit\'e des humains croient que leur exp\'erience personnelle est plus fiable et significative que les centaines d'ann\'ees de revue par les pairs, de tests exp\'erimentaux, de contrôles de la "m\'ethode scientifique" qui semble jusqu'ici, sinon LA meilleure, au moins être la meilleure m\'ethode d'investigation intellectuelle que l'on connaisse à ce jour.
	\end{enumerate}
	
	\begin{fquote}Aucune quantité d'évidences revues par les pairs ne persuadera jamais un idiot, d'autant plus si ce dernier est scientifiquement analphabète et ne maîtrise pas les calculs statistiques de haut niveau et les subtilités de la méthode scientifique.
 	\end{fquote}
	
	Maintenant, citons quelques paragraphes d'un excellent \href{http://www.slate.com/articles/health_and_science/science/2017/04/explaining_science_won_t_fix_information_illiteracy.html}{{\color{blue} article}} de Tim Requarth puisque ceux-ci sont quasiment parfaits pour illustrer nos propos:
	
	«La th\'eorie que de nombreux scientifiques semblent supporter est techniquement connue sous le nom de "modèle de d\'eficit", qui stipule que les opinions des gens diffèrent du consensus scientifique parce qu'ils manquent de connaissances scientifiques. En 2010, Dan Kahan, un psychologue de Yale, semble avoir montr\'e que cette th\'eorie \'etait significativement fausse. Il a \href{http://www.nature.com/nclimate/journal/v2/n10/full/nclimate1547.html}{{\color{blue} sond\'e}} plus de 1'500 Am\'ericains, classant chaque personne dans une «vision du monde culturelle» d'une \'echelle qui correspond grossièrement avec une tendance politique lib\'erale ou conservatrice. Il a ensuite \'evalu\'e la culture scientifique de chaque personne avec des questions telles que «Vrai ou Faux: les \'electrons sont plus petits que les atomes». Enfin, il les a interrog\'es sur le changement climatique. Si le modèle de d\'eficit \'etait correct alors les gens avec une culture scientifique sup\'erieure, ind\'ependamment de leur vision du monde, devraient être d'accord avec les consensus scientifique que le changement climatique repr\'esente un risque s\'erieux pour l'humanit\'e.
  
	Ce n'est pas ce Dan Kahan a observ\'e. Au lieu de cela, les donn\'ees mettent en \'evidence que l'augmentation des connaissances scientifiques avait en r\'ealit\'e un petit effet n\'egatif: les r\'epondants conservateurs qui en savaient le plus sur la science semblent penser que le changement climatique pose le moins de risques. La culture scientifique, semble-t-il, augmente la polarisation. Dans une \'etude ult\'erieure, Dan Kahan a ajout\'e une vice dans le questionnaire: Il a demand\'e aux r\'epondants ce que selon eux les climatologues croyaient (\'eviement l'usage du verbe "croire" est vicieux dans ce contexte!). Les r\'epondants qui en savaient plus sur la science en g\'en\'eral, quelle que soit leur orientation politique, \'etaient plus à même d'identifier le consensus scientifique - en d'autres termes, la polarisation avait disparu. Pourtant, quand on a demand\'e aux mêmes personnes leurs opinions sur le changement climatique, la polarisation est revenue. Il a donc \'et\'e montr\'e sur cette unique exp\'erience que même lorsque les gens comprennent le consensus scientifique, ils peuvent souvent ne pas l'accepter.
	
	\begin{fquote}[Utilisateur lambda des réseaux sociaux]N'ayant aucune formation postdoctorale sur ce sujet, ni aucun biais, et n'ayant fait aucune recherche dans un laboratoire à un niveau professionnel que ce soit, je ne le comprends pas. Par conséquent, cela n'a pas de sens et c'est faux!
 	\end{fquote}

	Le point à retenir est clair: l'augmentation de la culture scientifique seule ne changera pas les esprits et le biais cognitifs. En fait, les tentatives bien intentionn\'ees des scientifiques pour informer le public pourraient même se retourner contre eux. Pr\'esenter des faits qui entrent en conflit avec la vision du Monde d'un individu peut, en fait, inciter les gens à aller plus loin. Les psychologues, à juste titre, ont surnomm\'e cela "l'effet de retour de flamme"\index{effet retour de flamme}.
	\begin{figure}[H]
		\centering
		\includegraphics[scale=0.4]{img/intro/explain_science.jpg}
		\caption[]{Source: Dr. Jones, https://www.ratbotcomics.com}
	\end{figure}
	Si les scientifiques veulent simplement expliquer la science à un public curieux, diffuser plus largement leur recherche ou \'ecrire pour s'amuser, cela n'a pas beaucoup d'importance. Mais si les scientifiques sont motiv\'es à faire changer les opionons - et nombreux sont les scientifiques inscrits à des ateliers de communication scientifique qui semblent avoir cet objectif - ils seront a priori très d\'eçus.

	Cela ne veut pas dire que les scientifiques devraient retourner à la laboratoire et rester muets. Ils devraient juste se rendre compte que la combler le "manque d'information (ou de culture)" n'est pas l'objectif r\'eel. Au lieu de cela, les scientifiques devraient apprendre à communiquer la science de façon strat\'egique!

	Il y a des raisons \'evidentes pour lesquelles la communication scientifique est une entreprise n\'ecessaire et utile, mais une particulièrement importante est: qu'il y a dans certains pays un politique organis\'ee faite d\'ecridibiliser la science et att\'enuer son efficacit\'e. Lors d'une conf\'erence au Heartland Institute en Mars 2017, Lamar Smith, le pr\'esident r\'epublicain du comit\'e scientifique a d\'eclar\'e aux participants qu'il qualifierait maintenant la «science du climat» de «science politiquement correcte» et ce afin de minimiser les critiques. Cela cat\'egorse donc implicitement les scientifiques comme faisant partie de la "gauche" politique et, comme Daniel Engber l'a soulign\'e dans le magazine Slate à propos de la prochaine Marche Pour La Science, à red\'efinir l'autorit\'e scientifique comme une forme d'\'elitisme.

	Est-il surprenant, alors que des conf\'erences de scientifiques construites sur le principe qu'ils en savent simplement plus (même si c'est vrai) n'arrivent pas à convaincre leur public? Plutôt que de combler le d\'eficit d'information en construisant un arsenal de faits et d'\'evidences exp\'erimentales, les scientifiques devraient plutôt envisager comment d\'eployer leurs connaissances. Ils peuvent avoir plus de chance de communiquer si, en plus de pr\'esenter des faits et des chiffres, ils font appel aux \'emotions (biais \'emotionnels). Cela pourrait signifier non seulement expliquer la science de la façon dont quelque chose fonctionne mais passer du temps sur les raisons pour lesquelles cela compte pour l'auditeur et pourquoi cela devrait avoir de l'importance pour le conf\'erencier. La recherche montre \'egalement que les communicateurs scientifiques peuvent être plus efficaces après avoir gagn\'e la confiance du public. Dans cette optique, il serait peut-être plus utile de comprendre comment parler de la science avec des gens qui la connaissent d\'ejà, par le biais d'interactions locales et communautaires, que d'essayer de publier des explications sur des sites d'information nationaux. Et les scientifiques pourraient envisager de r\'ediger des \'editoriaux pour leurs journaux locaux, en se concentrant sur les raisons pour lesquelles la science compte pour leurs communaut\'es respectives.

	Les scientifiques peuvent \'egalement apprendre à \'eviter certains pièges. J'ai parl\'e avec Gretchen Goldman, directrice de l'Union des Centres Scientifiques Concern\'es pour la Science et la D\'emocratie, qui propose des ateliers de communication et de plaidoyer. Une leçon contre-intuitive qu'elle a apprise est que r\'efuter des histoires qui nient le changement climatique en abordant chaque revendication et en expliquant pourquoi c'est faux n'est pas très productif. En fait, cela pourrait être contre-productif: «Si vous r\'ep\'etez le mythe, c'est la partie que les gens se souviennent même si vous la d\'emystifiez imm\'ediatement», dit-elle. Selon elle, une meilleure approche consiste à recadrer le problème. Ne continuez pas à expliquer pourquoi le changement climatique est r\'eel, expliquez comment le changement climatique va nuire à la sant\'e publique ou à l'\'economie locale. La communication qui fait appel à des valeurs, pas seulement à l'intellect, montre la recherche, peut être beaucoup plus efficace (biais \'emotionnel).

	[...] Mais les obstacles rencontr\'es par les communicateurs scientifiques ne sont pas \'epist\'emologiques mais culturels. Les comp\'etences requises ne sont pas celles d'un professeur d'universit\'e mais d'un rh\'eteur.

	C'est donc un but probablement admirable de communiquer sur la science, mais presque certainement destin\'e à \'echouer. C'est parce que la façon dont la plupart des scientifiques pensent à la communication scientifique - qui simplement qu'en expliquant la vraie science aidera - est tout à fait fausse. En fait, c'est tellement faux que c'est souvent l'effet inverse de ce qu'ils essaient d'accomplir qui est r\'ecolt\'e. [...]»
	
	\begin{fquote}Le silence est parfois la meilleure réponse aux imbéciles.
 	\end{fquote}
	
	\begin{tcolorbox}[title=Remarque,colframe=black,arc=10pt]
	Dans le cadre d’un échange cordial, adopter une position de juge est une mauvaise stratégie. Il y a déjà de fortes chances pour que nos arguments soient perçus comme une agression, ce n’est donc pas la peine d’en rajouter. Il semble plus sage d’adopter la stratégie de l’entretien épistémique. Comparable à la maïeutique de Socrate, elle consiste à aider notre interlocuteur à présenter sa pensée, à la synthétiser pour pouvoir mieux, avec lui, la scruter et mettre en évidence ses failles éventuelles.
	\end{tcolorbox}
	
	\pagebreak
	\section{La Science est-elle dogmatique?}
	Nous allons ici répéter principalement des choses que nous avons déjà mentionnées plus haut. Cependant, comme certaines personnes scientifiquement illetrées pensent encore en ce début du 21e siècle que les vidéos YouTube contenant un monologue rhétorique\footnote{Les gens ne doivent pas faire confiance aux monologues - que ce soit à la radio, à la télévision ou sur les réseaux sociaux - car il n'y a pas d'experts pour contrer les éventuels mauvais arguments ou concepts mal définis qui peuvent conduire une partie de l'auditoire à de fausses interprétations spéculatives! De plus, les humains sous le stress de savoir qu'ils sont enregistrés sont naturellement sujets aux erreurs de vocabulaire et c'est sans compter les plus de 200 biais cognitifs du cerveau qui conduisent parfois à la simplification erronée de pensées complexes ...!} ou des livres sans preuves et sans analyse statistiques de données  constituent une sorte de "d'évidence", il est peut-être nécessaire de revenir sur quelques sujets mais avec une perspective différente\footnote {Et n'oubliez pas que le but de la science n'est pas la "vérité" mais c'est seulement un outil qui explique assez bien les choses que nous voyons ou ressentons en suivant les meilleurs modèles actuels. Et n'oubliez pas non plus que citer un livre, un scientifique célèbre, un blog ou une vidéo YouTube n'est au mieux qu'une évidence de niveau 2 ou 3!}!
	
	Pour la Science, si quelque chose existe avec des évidences au-delà de tout doute raisonnable (ADDTDR), dans son état actuel de connaissance, cela signifie qu'elle peut être mesurée! Si elle ne peut pas être mesurée ou est mal définie, eh bien, alors la Science ne peut pas fournir l'évidence qu'elle n'existe pas (n'oubliez pas que le but de la Science est de réfuter les modèles et si elle échoue à le faire pendant assez longtemps alors un modèle peut prendre le statut de théorie!). Dans une métaphore simpliste et légèrement hors contexte, cela équivaut à dire que pour les aveugles (en leur enlevant aussi tous leurs autres sens), le monde n’existe pas avec des preuves solides parce qu’ils ne peuvent ni le voir ni le sentir. Certains disent alors que la science est matérialiste! Cependant, cela ne signifie pas du tout que la Science a échoué en tant que méthode d'investigation de la Nature, mais cela signifie simplement que la Science sait qu'elle ne sait pas tout sinon elle s'arrêterait et que la méthode scientifique doit être constamment corrigée en fonction des nouvelles évidences disponibles.
	
	Typiquement, la science ne rejette pas complètement la parapsychologie ou l'existence de l'une des mille divinités créées dans le monde par les humains, même si en réalité tous les tests expérimentaux menés jusqu'à ce jour les ont rejetées avec des évidences solides (mais pas définitivement!). Les partisans de la parapsychologie ou des religions (ou de certaines médecines alternatives) peuvent faire valoir que la science ne peut ni prouver ni réfuter ce qu'elle ne peut pas mesurer directement ou indirectement. Et ils ont très vraisemblablement absolument raison! La science ne peut \underline{échouer à rejeter avec un niveau d'évidence donné} que si quelque chose existe ou non \ \underline{à son état actuel de connaissances}. Donc, si quelqu'un soutient que les licornes volantes existent sans fournir de protocole afin que des milliers d'autres personnes puissent vérifier ce fait de manière reproductible et indiscutable... alors la science (scientifiques) nous dit humblement que: \textit{nous n'avons aucune évidence au-delà doute raisonnable à notre état actuel de connaissance que les licornes existent ou non}!
	
	Ce n'est donc pas la Science qui est dogmatique, ni l'immense majorité de la communauté scientifique. Mais seuls quelques scientifiques mal éduqués et biaisés (personne n'est parfait...).
	
	\begin{figure}[H]
		\centering
		\includegraphics[width=1\textwidth]{img/intro/science_dogmatic.jpg}
	\end{figure}
	
	Certains humains n'aiment pas que les scientifiques ne sachent pas tout sur quelque chose, ou que les scientifiques font des erreurs, et que la Science prend du temps et que pire encore, ces mêmes personnes n'auront peut-être jamais de réponse à leurs principales questions avant leur mort. Mais c'est ainsi que fonctionne la Science!!! Si quelqu'un touve un moyen plus fiable et plus rapide d'étudier l'Univers et ses phénomènes que l'état actuel de la Méthode Scientifique, alors la majorité de la communauté scientifique - en particulier la communauté académique - le vérifiera et si vraiment cela fonctionne mieux alors l'adoptera avec plaisir!
	
	\begin{fquote}[Carl Sagan]Les affirmations extraordinaires nécessitent des preuves extraordinaires.
 	\end{fquote}
 	
 	Ainsi, certaines personnes peuvent demander à la Science d'être vraiment scientifique, c'est-à-dire qu'elle devrait s'interroger sur la certitude de ses propres postulats et instruments. Mais comme nous l'avons déjà dit à plusieurs reprises dans les sections précédentes ci-dessus, c'est justement ce que font les chercheurs dans leur travail quotidien!!! Ils essaient de trouver de nouvelles évidences pour rejeter les théories, les modèles, les méthodes, les postulats ou les mauvais instruments actuels... sinon la science s'arrêterait et les scientifiques perdraient leur travail! Cependant, le processus est lent et prend des jours, des semaines, des mois, des années, des décennies, des siècles et parfois même des millénaires.
	 
	\begin{tcolorbox}[title=Remarque,colframe=black,arc=10pt]
	Certaines personnes soulignent que la version de l'Univers dont disposent actuellement les scientifiques n'est que ce que leurs instruments, et surtout leur imagination leur permet de comprendre. Et ils aussi très vraisemblablement raison! Nous - tous les chercheurs académiques professionnels - savons ce fait depuis des siècles en science et c'est pourquoi nous essayons chaque jour de repousser les limites de la connaissance (donc aussi de l'imagination) et de développer de nouveaux instruments et méthodes pour mesurer des choses qui n'étaient même pas connues quelques décennies auparavant! Le lecteur doit garder à l'esprit que c'est pour cela que nous sommes payés! S'il n'y a rien de nouveau à découvrir, nous perdrons tous nos emplois!\\
	
	Alors oui dans son état actuel, la Science ne peut pas rendre compte des phénomènes de conscience, de la synchronicité, ou même des expériences de mort imminente ou des épiphanies spontanées, etc. Mais les scientifiques professionnels correctement éduqués ne prétendront pas qu'ils existent ou n'existent pas parce que nous n'avons pas encore actuellement d'évidences expérimentales reproductible pour soutenir ou rejeter l'une de ces positions! Les scientifiques attendent juste de ceux qui prétendent que de telles choses existent - en faisant même des affaires très lucratives avec - de leur fournir des évidences expérimentales reproductibles. Malheureusement, jusqu'à présent, toutes les personnes qui ont fait de telles affirmations et qui ont des activités lucratives avec n'ont fourni aucune évidence expérimentale et même pire ils ont été systématiquement identifiés comme charlatans.\\
	
	N'importe qui devrait se demander pourquoi les pasteurs évangélistes ou tout type de guérisseur faiseur de miracles (...) ne pratiquent pas leur art dans l'environnement scientifique contrôlé des hôpitaux plutôt que dans leur église (devant un public en grande majorité biaisé, sans instruction et scientifiquement illetrés) ou avec un petit groupe de personnes inconnues et suspectes derrière une caméra..........
	\end{tcolorbox}
	 
	Des gens comme Rupert Sheldrake, titulaire d'un doctorat (de Cambridge) en biochimie et chercheur à la retraite dans le domaine de la parapsychologie qui a proposé le concept de résonance morphique ... et également auteur de nombreux livres enrichissants (au point figuré du terme!) ...... explore dix dogmes de la science qui devraient être reconsidérés en fonction de son opinion personnelle et subjective...
	
	Présentons ci-dessous ces dix dogmes et ... nous les commenterons car le Dr Sheldrake n'a pas de doctorat en physique, ni en cosmologie mais a un bon sens de la rhétorique surtout quand il fait un monologue devant un public de néophytes (le lecteur et le Dr Sheldrake pourront trouver les preuves mathématiques de nos réponses dans les pages à $7000$ de ce livre s'ils sont curieux - contrairement au Dr Sheldrake, nous aimons plus les preuves et les évidences expérimentales que les monologues rhétoriques...):
	\begin{enumerate}
		\item \og La nature est mécanique: toutes les créatures et systèmes de la nature sont des robots faits pour suivre un programme génétique donné. \fg{}
		
		$\vartriangleright$ Notre commentaire: Nous ne sommes pas sûrs de la définition du mot «Mécanique» dans cette phrase (les bases d'un débat sont de s'entendre sur le sens des mots ...) mais évidemment un système biologique ne peut être comparé à un système mécanique (un système mécanique n'évolue pas). Cependant, comme nous le prouverons dans ce livre, la nature est basée sur l'Information à tous les niveaux, basée sur les Probabilités au niveau microscoptique, et se comporte Statistiquement et Mécaniquement au niveau macroscopique. Donc comme la plupart du temps en Science ..., ce n'est pas aussi facile qu'il y paraît (il semble que le Dr Sheldrake ait une vision binaire du Monde et de l'Univers assez surprenante compte tenu de son niveau d'éducation supposé).
		
		\item \og La matière est inconsciente: Les plantes, les étoiles, les animaux et les éléments sont des choses matérielles qui sont et ne peuvent pas avoir une conscience d'elles-mêmes.\fg{}
		
		$\vartriangleright$ Notre commentaire: Nous devons d'abord nous demander comment la «conscience» est définie par le Dr Sheldrake. Deuxièmement, qui a affirmé cela dans la communauté scientifique? Existe-t-il un consensus scientifique écrit noir sur blanc sur ce sujet ou cela est-t-il issu de l'imagination du Dr Sheldrake?

		\item \og Les lois de la nature sont fixes: au moment du Big Bang, toutes les constantes nécessaires jusqu'à la fin des temps ont été établies. Les habitudes de la nature n'évoluent pas.\fg{}
		
		$\vartriangleright$ Notre commentaire: Quel consensus scientifique a déclaré cela? C'est très probablement faux car nous avons en fait quelques modèles en physique théorique qui prouvent que les constantes de l'Univers peuvent tout à fait avoir changé et nous avons aussi des évidences expérimentales que les lois de la Nature ont aussi peut-être changé sur de longues périodes de temps. Mais une chose est presque sûre: il n'y a pas encore de consensus scientifique écrit noir sur blanc sur ce sujet au jour où nous écrivons ces lignes!

		\item \og La quantité de nature et d'énergie dans l'Univers est toujours la même. \fg{}
		
		$\vartriangleright$ Notre commentaire: La communauté scientifique a de solides évidences au-delà de tout doute raisonnable de cette affirmation pour l'Univers \underline{observable}. Cependant, la dynamique de l'Univers réfute la conservation de l'énergie à grande échelle (cela est dérivé des théorèmes de Noether!). Notez que nous n'avons cependant aucune preuve de cette déclaration de conservation d'énergie pour l'Univers dans son entier (l'observable et non observable).

		\item \og La Nature n'a pas de buts: il n'y a pas de conception dans la Nature en termes d'intention et le processus d'évolution est mécanique.\fg{}
		
		$\vartriangleright$ Notre commentaire: Nous avons des évidences solides au-delà de tout doute raisonnable en effet qu'il n'y a pas de conception intelligente car la conception observable actuelle est défectueuse à bien des égards. Et le processus d'évolution tel que prouvé mathématiquement dans ce livre et expérimentalement (au-delà de tout doute raisonnable) en laboratoire n'est pas mécanique mais stochastique.
		
		\item \og Patrimoine biologique: Les plans pour produire un être vivant sont composés dans la matière physique logée dans leurs gènes. \fg{}
		
		$\vartriangleright$ Notre commentaire: Ce n'est pas tout à fait exact. Les observations expérimentales nous montrent que certaines bases des plans sont aléatoires et influencées par des modifications externes et internes. Encore une fois, le Dr Sheldrake donne une vue binaire d'un phénomène bien plus complexe. Mais c'est quelque chose que nous savons comme typique des scientifiques qui parlent de sujets qu'ils ont mal étudiés: ils réduisent quelque chose de complexe à quelque chose de simple parce qu'ils ne peuvent pas saisir la complexité du Monde et de l'Univers.

		\item \og La mémoire est conservée dans le cerveau sous forme d'empreintes matérielles: la mémoire est constituée de protéines et de terminaisons nerveuses organisées comme des tiroirs.\fg{}
		
		$\vartriangleright$ Notre commentaire: Si c'était le cas, nous n'oublierions pas des choses ... C'est parce que le cerveau est beaucoup plus complexe et implique des probabilités, des processus bayésiens et stochastiques que nous savons pourquoi le cerveau humain oublie des choses et a parfois des problèmes de construction...

		\item \og L'esprit est dans la tête: l'esprit a une connexion physique avec la tête et le cerveau, reléguant la subordination intellectuelle sur le reste du corps.\fg{}
		
		$\vartriangleright$ Notre commentaire: Qu'est-ce que «l'esprit» pour le Dr Sheldrake? À notre connaissance, il n'y a pas de consensus scientifique sur sa définition. De plus, l'esprit est-il ce que nous observons dans les scanners à RMN?

		\item \og Les phénomènes comme la télépathie sont impossibles: les pensées n'ont aucun effet sur le monde à cause du numéro 8 de la liste (l'esprit est dans la tête). \fg{}
		
		$\vartriangleright$ Notre commentaire: Quel consensus scientifique ou communauté a déclaré cela? En fait, la science n'a aucune évidence que la télépathie fonctionne ou existe, oui (!) - mais aucun scientifique bien éduqué ne dirait que c'est "impossible" (d'ailleurs tout scientifique bien éduqué sait qu'il vaut mieux éviter d'utiliser le mot "impossible" à propos d'un futur inconnu ou de phénomènes non mesurables).

		\item \og Seule la médecine mécanique fonctionne: c’est simplement par hasard ou par effet placebo que les pratiques de guérison traditionnelles ou les remèdes naturels ont un effet sur la santé des gens. \fg{}
		
		$\vartriangleright$ Notre commentaire: Encore une fois ... quel consensus scientifique ou quelle communauté a déclaré cela? Ce n'est pas exact. Peut-être que les scientifiques sont d'accord sur le fait qu'aucune autre méthode que la médecine scientifique n'a fourni un meilleur rapport de cotes que d'autres médicaments. Mais si un jour certaines personnes fournissent des évidences solides que les pratiques de guérison traditionnelles ou les remèdes naturels fonctionnent, alors presque sûrement ils seront promus par la communauté académique scientifique.
	\end{enumerate}
	
	Si ce sont là les meilleures évidences que le Dr Sheldrake a pour sa défendre sa position, alors cela signifie qu'il n'y a pas d'évidence du tout (au mieux du niveau 1). Ses arguments sont basés essentiellement sur de la rhétorique pseudo-scientifique et la spéculation, usant d'idées populaires fausses et d'opinions plutôt que des évidences et des exemples de "dogmes" en science.

	Nous pouvons comprendre pourquoi quelqu'un qui lit ou écoute ce genre rhétorique pseudo-scientifique pourrait penser qu'il existe une base d'évidence. Ainsi, quand quelqu'un prétend que «l'establishment» est dogmatique, immoral ou quoi que ce soit, nous pouvons sincèrement espérer que n'importe quel lecteur ou auditeur évaluera de manière critique les affirmations de cette personne.
	
	\begin{figure}[H]
		\centering
		\includegraphics[width=0.7\textwidth]{img/intro/fausses_equivalences.jpg}
	\end{figure}
	
	\begin{flushright}
	Note de qualit\'e de la section: \score{4}{5} 151 votes, 75.23\%
	\end{flushright}
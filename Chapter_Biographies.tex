Pour être informé des lauréats des prix Nobel (physique, chimie, économie), médaille de Fields... cliquez sur le lien suivant {\href{http://www.nobelprize.org/}{{\color{blue}Prix Nobel}}} ou sur celui-ci {\href{http://www.fields.utoronto.ca/aboutus/jcfields/fields_medal.html}{{\color{blue}Médaille de Fiels}}}.

Cette section présente une poignée d'individus qui postulent une étrange renommée. Selon les règles de l'histoire que l'on enseigne à l'école primaire, ils n'existent pas, ils n'ont commandé aucune armée, ils n'ont envoyé personne à la mort, ils n'ont dirigé aucun empire et ils n'ont eu qu'une part minime dans les grandes décisions historiques. Certains ont acquis quelque célébrité, mais aucun ne fut jamais un héros national. Pourtant, leur oeuvre a davantage influencé le cours de l'Histoire que bien des actes accomplis par des hommes d'État auréolés d'une gloire très supérieure. Leur oeuvre a aussi produit plus de bouleversements que le va-et-vient des armées en bataille par-dessus les frontières, elle a fait plus pour le bonheur ou le malheur que les édits des rois et des assemblées, car leur oeuvre, est d'avoir façonné l'esprit de l'homme!

Qui propage ses idées, manie un pouvoir bien supérieur à celui de l'épée ou du sceptre: aussi ont-ils façonné et dirigé le Monde. Pour la plupart, ils n'ont pas levé le moindre petit doigt pour agir physiquement; ils ont travaillé essentiellement en intellectuels, dans le silence et l'oubli, sans se soucier outre mesure du monde environnant. Mais, dans leur sillage, des empires se sont écroulés, des régimes politiques se sont soit renforcés, soit érodés, les classes se sont dressées les unes contre les autres, ainsi que les nations. Qui sont ces individus ?: des savants, économistes, chimistes, biologistes, mathématiciens, physiciens, informaticiens, ingénieurs,...

Les biographies ci-dessous des scientifiques les plus connus à travers le Monde et cités dans les différents chapitres du site sont triées par ordre alphabétique et presque tous les textes sont de simples copier/coller simplifiés de {\href{http://www.wikipedia.fr}{{\color{blue}Wikipédia}}}. Si vous souhaitez que nous rajoutions une entrée, il vous suffit de nous envoyer par courriel le nom et prénom de la personne concernée et la raison pour laquelle vous aimeriez la voir figurer dans la liste ci-dessous. Nous étudierons ensuite la proposition et prendrons la décision qui s'impose.

\textbf{Nous rendons aussi hommage aux centaines de milliers de scientifiques (savants), ingénieurs, philosophes, artisans, mécènes, artistes, amateurs éclairés connus ou anonymes dont la collaboration a permis à travers les millénaires l'évolution de la science et de la condition humaine!}

Les tailles des biographies ci-dessous ne sont pas proportionnelles au nombre d'articles publiés ou découvertes effectuées, mais à la quantité d'informations trouvées sur ces personnages sur l'Internet ou dans la littérature. La liste n'est aussi pas exhaustive, mais son objectif est de rendre hommage et de se remémorer les individus qui ont fait des sciences pures et exactes ce qu'elles sont aujourd'hui et qui ont consacré une partie ou l'entier de leur vie à la science: l'art le plus contraint!

Attention! En physique (aussi bien qu'en mathématique) une théorie, une équation, une constante ou autre porte rarement le nom de son vrai inventeur. Ce fait est largement connu chez les scientifiques et est souvent source de moqueries de la communauté...

	\begin{figure}[H]
		\centering
		\includegraphics[scale=0.5]{img/shoulders_of_giants.jpg}	
	\end{figure}

\begin{center}
\hyperref[sec:A]{A} \hyperref[sec:B]{B} \hyperref[sec:C]{C} \hyperref[sec:D]{D} \hyperref[sec:E]{E} \hyperref[sec:F]{F} \hyperref[sec:G]{G} \hyperref[sec:H]{H} \hyperref[sec:I]{I} \hyperref[sec:J]{J} \hyperref[sec:K]{K} \hyperref[sec:L]{L} \hyperref[sec:M]{M} \hyperref[sec:N]{N} \hyperref[sec:O]{O} \hyperref[sec:P]{P} \hyperref[sec:Q]{Q} \hyperref[sec:R]{R} \hyperref[sec:S]{S} \hyperref[sec:T]{T} \hyperref[sec:U]{U} \hyperref[sec:V]{V} \hyperref[sec:W]{W} \hyperref[sec:X]{X} \hyperref[sec:Y]{Y} \hyperref[sec:Z]{Z}
\end{center}

\phantomsection
\addcontentsline{toc}{section}{A}
\label{sec:A}
		
\pichskip{15pt}% Horizontal gap between picture and text
\parpic[l][t]{
  \begin{minipage}{40mm}
    \fbox{\includegraphics[width=110px,height=140px]{img/medaillons/al.eps}}
  \end{minipage}
}		
\textbf{Al-Biruni, Muhammad Ibn Ahmad Abul-Rayhan} (973-1048) est un mathématicien, un astronome, un physicien, un érudit, un encyclopédiste, un philosophe, un astrologue, un voyageur, un historien, un pharmacologue et un précepteur, originaire de la Perse, qui contribua grandement aux domaines des mathématiques, philosophie, médecine et des sciences. Il est connu pour sa théorie sur la rotation de la Terre autour de son axe et autour du Soleil, et ceci bien avant Copernic. Il s'attacha notamment à calculer la marche du Soleil (apogée), corrigea certaines données de Ptolémée. Excellent mathématicien, Al-Biruni développa de nouvelles équations inconnues de ses prédécesseurs. Il calcula également le méridien local et les coordonnées des localités. Mais le tableau ne serait pas complet si l'on oubliait de mentionner que six siècles avant Galilée, Al Biruni mettait déjà en avant une Terre qui tournait autour de son axe. Avec l'aide d'un astrolabe, de la mer et d'une montagne avoisinante, il calcula la circonférence de la Terre en résolvant une équation complexe pour son époque. Le principal d'apport d'Al-Biruni aux mathématiques résiderait dans ses travaux en trigonométrie (calculs des valeurs des fonctions trigonométriques qui n'étaient pas encore définies en tant que tel à l'époque)..

\pichskip{15pt}% Horizontal gap between picture and text
\parpic[l][t]{%
  \begin{minipage}{40mm}
    \fbox{\includegraphics[width=110px,height=140px]{img/medaillons/alembert.eps}}
  \end{minipage}
}
\textbf{Alembert, Jean le Rond} (1717-1783), enfant naturel d'un commissaire d'artillerie, abandonné sur les marches de la chapelle parisienne de Saint-Jean-Le-Rond, le futur grand philosophe, mathématicien et physicien est recueilli par un vitrier qui recevra secrètement une pension pour subvenir à l'éducation du jeune garçon qui étudiera brillamment le droit, la médecine et la mathématique. Suite à la publication de divers mémoires (sur le calcul intégral, sur la réfraction des corps solides), d'Alembert entre à l'Académie des sciences (1741). On lui doit le célèbre principe de la quantité de mouvement, dit "principe de d'Alembert" dans son \textit{Traité de dynamique} (1743). En astronomie, il est l'auteur (1749) d'un traité sur la précession des équinoxes qu'il explique au moyen de la théorie de la gravitation universelle de Newton et d'une solution partielle au problème des trois corps. D'Alembert établit aussi une théorie mathématique des cordes vibrantes en étudiant la nature composite du son (harmoniques).

\pichskip{15pt}% Horizontal gap between picture and text
\parpic[l][t]{%
  \begin{minipage}{40mm}
    \fbox{\includegraphics[width=110px,height=140px]{img/medaillons/ampere.eps}}
  \end{minipage}
}
\textbf{Ampère, André Marie} (1775-1836) à 18 ans, il connaît la majeure partie des oeuvres mathématiques de son temps. Mathématicien de premier ordre, il montre comment l'on doit utiliser cette science, qu'il considérait comme une branche de la philosophie, à l'étude des découvertes des faits physiques pour en donner une relation définitive. En quelques semaines, Ampère établit les bases de toute une science à laquelle il donne le nom d'électromagnétisme. Il cherche à comprendre le magnétisme des aimants et en tire une hypothèse de "courants particulaires" (orbites électroniques et orientation du spin aujourd'hui). Il établit également l'égalité du nombre de molécules dans des volumes égaux de Gaz de natures différentes, mais mesurés dans des conditions identiques de température et de pression (observation expérimentale de Gay-Lussac).

\pichskip{15pt}% Horizontal gap between picture and text
\parpic[l][t]{%
  \begin{minipage}{40mm}
    \fbox{\includegraphics[width=110px,height=140px]{img/medaillons/archimede.eps}}
  \end{minipage}
}
\textbf{Archimedes of Syracuse} (287-212 BC.), mathématicien et ingénieur grec célèbre à la fois comme mécanicien théoricien et comme constructeur de machines. Archimède de Syracuse eut une production mathématique exceptionnelle, dont une partie nous est parvenue dans des traités comme \textit{Sur la sphère et le cylindre}; la \textit{Mesure du cercle}; la \textit{Quadrature de la parabole};\textit{ Des spirales};\textit{ Des conoïdes et sphéroïdes}; la \textit{Méthode}, \textit{Des corps flottants}... C'est à partir de ses travaux mécaniques que les principales anecdotes le mettant en scène, comme celle du levier ou du bain, vont se constituer. La célèbre maxime: «Donnez-moi une place où me tenir et je mettrai la Terre en mouvement» est un écho populaire de la contribution archimédienne à la statique, exposée dans le traité des \textit{Équilibres}. Archimède démontre la loi du levier, introduit la notion fondamentale de centre de gravité, et détermine ces barycentres pour les principales figures géométriques planes. Il en est de même pour l'anecdote d'Archimède, jaillissant nu de son bain, en criant «Eurêka», parce qu'il venait, dit-on, de trouver le moyen de résoudre le problème que lui avait posé le roi Hiéron. En fait, le récit est une mise en scène spectaculaire de la découverte du principe fondamental de l'hydrostatique (communément appelé depuis "principe d'Archimède"). En géométrie, l'oeuvre d'Archimède développe celle d'Eudoxe de Cnide telle que nous la connaissons par le livre XII des \textit{Éléments}  d'Euclide: il s'agit de comparer les mesures des figures planes et solides, en particulier des figures curvilignes. Ainsi, Archimède démontre que le volume du cylindre circonscrit à une sphère est égal à une fois et demi le volume de celle-ci et que la surface latérale du cylindre est égale à celle de la sphère ou quatre fois la surface d'un grand cercle. Donc, si l'on sait calculer la surface du cercle, on connaîtra celle de la sphère, du cylindre, son volume et celui de la sphère, etc. Son résultat le plus célèbre et le plus simple concerne le cercle. Archimède ramène sa quadrature à un autre problème: la rectification de sa circonférence, c'est-à-dire «trouver une ligne droite égale qui lui soit égale», problème qu'il résout à l'aide d'une courbe géométrique que l'on appelle désormais "spirale d'Archimède". En outre, il calcule des valeurs approchées du rapport circonférence/diamètre (ce que nous appelons le nombre "Pi" et noté $\pi$).

\pichskip{15pt}% Horizontal gap between picture and text
\parpic[l][t]{%
  \begin{minipage}{40mm}
    \fbox{\includegraphics[width=110px,height=140px]{img/medaillons/avogadro.eps}}
  \end{minipage}
}
\textbf{Avogadro, Amedeo} (1776-1856), fils d'un magistrat de Turin, Amadeo Avogadro commence par suivre la voie paternelle. Il passe une licence de droit en 1795 et s'inscrit au barreau de sa ville natale. Mais son goût pour la physique et la mathématique, auxquelles il n'a cessé de s'intéresser en solitaire, le pousse à entamer sur le tard des études scientifiques. En 1809, il fait une communication à l'Académie royale de Turin ; le succès qu'il remporte grâce à elle lui permet d'obtenir un poste de professeur au Collège royal de Verceil. En 1820, l'Université de Turin crée pour lui une chaire de physique qu'il gardera jusqu'à la fin de sa vie. C'est en étudiant les lois régissant la compression et la dilatation des gaz qu'Avogadro énonce, en 1811 l'hypothèse restée célèbre sous le nom de "loi d'Avogadro". Reposant sur la théorie atomique de Dalton et la loi de Gay-Lussac sur les rapports volumiques, la théorie d'Avogadro indique que deux volumes égaux de gaz différents, dans les mêmes conditions de température et de pression, contiennent le même nombre de molécules. Sous son apparente simplicité, cette loi comporte des implications importantes ; grâce à elle, il devient possible de déterminer la masse molaire d'un gaz à partir de celle d'un autre. Mais les chimistes de l'époque, plus intéressés par les expériences, boudent quelque peu les études théoriques d'Avogadro qui ne seront d'ailleurs reconnues que 50 ans plus tard. Le nom d'Avogadro reste également lié à celui du "nombre d'Avogadro" indiquant le nombre de molécules contenues dans une seule mole.

\phantomsection
\addcontentsline{toc}{section}{B}	
\label{sec:B}

\pichskip{15pt}% Horizontal gap between picture and text
\parpic[l][t]{%
  \begin{minipage}{40mm}
    \fbox{\includegraphics[width=110px,height=140px]{img/medaillons/bachelier.jpg}}
  \end{minipage}
}
\textbf{Bachelier, Louis} (1870-1946) est né au Havre dans une famille de négociants. Il apparaît à sa majorité sur les listes électorales du Havre en 1892 comme représentant de commerce à la même adresse professionnelle que son père. Après avoir effectué son service militaire, à l'âge de 22 ans, il reprend ses études à la faculté des sciences de Paris. Elles sont couronnées par une licence ès sciences en 1895 (mention passable) et par la soutenance en 1900 de sa non moins fameuse et méconnue thèse de doctorat en mathématiques. Bien que cette thèse soit considérée aujourd'hui comme un travail précurseur en théorie des probabilités et en théorie financière, elle ne vaut à l'époque à son auteur qu'une mention honorable. De 1913 à 1914 Bachelier dispensa un cours libre de théorie des probabilités appliquées à la mécanique, la balistique et la biométrie. Il fut également chargé de conférences supplémentaires sur la mathématique générale de 1913 à 1914.  Ce n'est qu'après la guerre de 1914-1918 qu'il obtient un premier poste de chargé de cours à la faculté des sciences de Besançon. Après divers remplacements à Dijon puis à Rennes, il revient à Besançon en 1927 comme professeur titulaire de la chaire de calcul différentiel et intégral, poste qu'il occupe jusqu'à sa retraite en 1937. Louis Bachelier a, parmi ses nombreux travaux, été le premier a avoir introduit la continuité dans les problèmes de probabilité en prenant le temps comme variable. En particulier, il a élaboré une théorie mathématique du mouvement brownien 5 ans avant Albert Einstein. Il est également bien avant Norbert Wiener, le premier à avoir défini la fonction du mouvement brownien et donné un grand nombre de ses propriétés.

\parpic[l][t]{%
  \begin{minipage}{40mm}
    \fbox{\includegraphics[width=110px,height=140px]{img/medaillons/balmer.eps}}
  \end{minipage}
}
\textbf{Balmer, Johann Jakob} (1825-1898) était un mathématicien et physicien suisse né à Lausen (Suisse) et mort à Bâle. Pendant sa scolarité, il a excellé en mathématiques et a donc décidé de se concentrer sur ce domaine lorsqu'il a fréquenté l'université. Il a étudié à l'Université de Karlsruhe et à l'Université de Berlin, puis a terminé son doctorat de l'Université de Bâle en 1849 avec une thèse sur la cycloïde. Balmer a ensuite passé toute sa vie à Bâle, où il a enseigné dans une école pour filles. Il a également enseigné à l'Université de Bâle. En dépit d'être un mathématicien, il personne ne souvient d'une quelconque recherche qu'il aurait fait dans ce domaine; au contraire, sa contribution majeure (faite à l'âge de soixante ans, en 1885) était une formule empirique pour les raies spectrales visibles de l'atome d'hydrogène, dont l'étude qu'il mena fit suite à la suggestion d'Eduard Hagenbach, également de Bâle. Une explication complète de l'explication de sa formule dû attendre jusqu'à la présentation du modèle Bohr de l'atome par Niels Bohr en 1913.

\parpic[l][t]{%
  \begin{minipage}{40mm}
    \fbox{\includegraphics[width=110px,height=140px]{img/medaillons/banach.eps}}
  \end{minipage}
}
\textbf{Banach, Stefan} (1892-1945) était un mathématicien polonais qui a posé les bases de l'analyse fonctionnelle. Né à Cracovie en 1892, en Autriche-Hongrie (actuellement ville polonaise). Banach fit ses études secondaires à Cracovie; il se révéla particulièrement brillant en mathématiques et en sciences naturelles, mais son désintérêt pour les autres matières l'empêcha d'obtenir les meilleures mentions.  La vie (au moins mathématique) de Banach va basculer au printemps 1916, quand il rencontre Steinhaus à Cracovie. Avec Otto Nikodym, ils décident de fonder une société mathématique. La recherche mathématique de Banach commence là. Son premier article est cosigné avec Steinhaus. Steinhaus lui avait parlé d'une propriété qu'il ne parvenait pas à démontrer, et après quelques jours de réflexion, Banach exhiba un contre-exemple. Il est difficile de dire ce qu'il serait advenu de l'activité mathématique de Banach sans la rencontre avec Steinhaus, mais toujours est-il qu'il entama à compter de celle-ci une recherche intense et fructueuse. Banach retourne à Lvov en 1920 où un poste d'assistant lui est proposé. Il soutient sa thèse en 1922, et c'est dans cette thèse qu'apparaît pour la première fois la notion d'espace de Banach, qu'y sont démontrés les théorèmes fondamentaux sur ces objets, où on y évoque la topologie faible... Bref, cette thèse marque la naissance de l'analyse fonctionnelle. En 1929, il fonde avec Steinhaus la revue mathématique "Studia Math", consacrée au développement de l'analyse fonctionnelle, et en 1939 il est élu président de la société mathématique de Pologne. En 1945, peu avant la fin de la Seconde Guerre Mondiale, il décède d'un long cancer. De nombreux théorèmes sont associés au nom de Banach, qu'il les ait démontrés lui-même, ou qu'ils fassent référence à ces idées. Citons entre autres: le théorème de Hahn-Banach de prolongement des formes linéaires continues, le théorème de Banach-Steinhaus, de Banach-Alaoglu, le théorème du point fixe de Banach, ainsi que le paradoxe de Banach-Tarski.

\parpic[l][t]{%
  \begin{minipage}{40mm}
    \fbox{\includegraphics[width=110px,height=140px]{img/medaillons/bell.eps}}
  \end{minipage}
}
\textbf{Bell, John} (1928-1990) fut dès la plus petite enfance attiré par les livres traitant des sciences. À cause de problèmes financiers familiaux, il ne put poursuivre immédiatement des études académiques. Il travailla donc pendant une année en tant que technicien au département de physique de l'Université de Queen's à Belfast avant de devenir étudiant en 1945 dans ce même département. Il sortit premier de sa promotion en mathématiques-physique. Bell trouva dans les années 1960 une nouvelle inspiration dans les bases de la théorie quantique, une matière supposée épuisée par les résultats de la discussion de Bohr-Einstein 30 ans plus tôt, et ignorée par pratiquement tous ceux qui ont employé la théorie quantique entre-temps. Effectivement, Bell était intrigué par les incertitudes quantiques de Heisenberg et voulait creuser le sujet en montrant que la discussion de tels concepts comme le "réalisme", le "déterminisme" et la "localité" pouvaient être affiliés dans un rapport mathématique rigoureux: "les inégalités de Bell" vérifiables expérimentalement. Bell poussa très loin les doutes qu'il avait sur les principes d'incertitudes au point qu'il en irrita même son professeur (Sloane) qui lui fit remarquer que maintenant il allait un peu trop loin! Bell attendit son travail de thèse pour développer ses idées. Malheureusement, à nouveau à cause de problèmes financiers, il dut repousser ses recherches à plus tard et joindre le centre anglais de recherche atomique à Harwell. Pendant sa carrière, il épousa une femme (Mary Bell) qui l'aida dans le développement de ses travaux sur les principes fondamentaux de la théorique quantique. C'est, en 1951, avec Rudolf Peierls que Bell développa sa célèbre théorie C.P.T. (Charge, Parité, Temps). Malheureusement, pour Bell, Gerhard Lüders et Wolfgang Pauli arrivèrent au même résultat dans la même période et c'est à eux que furent attribué les crédits de la découverte. Les développements théoriques de Bell sont à l'origine de la cryptographie et de la théorique de l'information quantique. L'attention à la théorie quantique de l'information a énormément augmenté au cours des dernières années, et le sujet semble sûr d'être l'un des secteurs scientifiques dont la croissance sera la plus importante au 21ème siècle. Un autre travail de première importance de Bell en 1969 fut la participation au développement de "l'anomalie A.B.J." (Adler-Bell-Jackiw) dans la théorie quantique des champs. Ces trois physiciens montrèrent que le modèle algébrique standard contentait une erreur. Effectivement, la quantification du modèle des champs brise une symétrie. Bell fut nommé pour le prix Nobel, qu'il aurait certainement obtenu s'il n'était pas décédé d'une attaque cérébrale en 1990.

\parpic[l][t]{%
  \begin{minipage}{40mm}
    \fbox{\includegraphics[width=110px,height=140px]{img/medaillons/berners_lee_timothy_john.jpg}}
  \end{minipage}
}
\textbf{Berners-Lee, Timothy John} (1955-) est un physicien anglais mieux connu comme l'inventeur du World Wide Web. Il est le directeur du World Wide Web Consortium (W3C), qui supervise le développement continu du Web. Il est également le fondateur de la World Wide Web Foundation, chercheur principal et titulaire de la chaire des fondateurs du Laboratoire d'informatique et d'intelligence artificielle du M.I.T. Il est directeur de la Web Science Research Initiative (WSRI) et membre du comité consultatif du M.I.T. Center for Collective Intelligence. Il a travaillé comme entrepreneur indépendant au CERN de juin à décembre 1980. Pendant son séjour à Genève, il a proposé un projet basé sur le concept d'hypertexte, pour faciliter le partage et la mise à jour des informations entre les chercheurs. L'adresse \url{info.cern.ch} était l'adresse du tout premier site web et serveur web au monde, fonctionnant sur un ordinateur NeXT au CERN. La première adresse de la page Web était \url{http://info.cern.ch/hypertext/WWW/TheProject.htm}.

\parpic[l][t]{%
  \begin{minipage}{40mm}
    \fbox{\includegraphics[width=110px,height=140px]{img/medaillons/bernoulli_daniel.eps}}
  \end{minipage}
}
\textbf{Bernoulli, Daniel} (1700-1782) est un savant suisse qui découvrit les principes de base du comportement d'un fluide (c'est le fils de Jean Bernoulli et le neveu de Jacques Bernoulli.). Il cultiva à la fois les sciences mathématiques et les sciences naturelles, enseigna les mathématiques, l'anatomie, la botanique et la physique. Ami de Leonhard Euler, il travailla avec lui dans plusieurs domaines des mathématiques et de la physique (il partagea avec lui dix fois le prix annuel de l'Académie des sciences de Paris à un point qu'il s'en fit une sorte de revenu...). Les différents problèmes qu'il tente de résoudre (théorie de l'élasticité, mécanisme des marées) le conduisent à s'intéresser et développer des outils mathématiques tels que les équations différentielles ou les séries. Il collabore également avec Jean le Rond d'Alembert dans l'étude des cordes vibrantes. Il étudia l'écoulement des fluides (1738) et formula le principe (le fameux théorème de Bernoulli) selon lequel la pression exercée par un fluide est inversement proportionnelle à sa vitesse d'écoulement. Il utilisa des concepts atomistes pour ébaucher la première théorie cinétique des gaz, en exprimant leur comportement en termes de probabilités sous des conditions particulières de pression et de température. On peut le considérer comme l'un des fondateurs de l'hydrodynamique.

\parpic[l][t]{%
  \begin{minipage}{40mm}
    \fbox{\includegraphics[width=110px,height=140px]{img/medaillons/bernoulli_jacques.eps}}
  \end{minipage}
}
\textbf{Bernoulli, Jacques} (1654-1705) était un mathématicien et physicien suisse, frère de Jean Bernoulli et oncle de Daniel Bernoulli et Nicolas Bernoulli. Né à Bâle en 1654, il rencontre Robert Boyle et Robert Hooke lors d'un voyage en Angleterre en 1676. Après cela, il se consacre à la physique et aux mathématiques. Il enseigne à l'Université de Bâle à partir de 1682, devenant professeur de mathématiques en 1687. Il mérita par ses travaux et ses découvertes d'être nommé associé de l'Académie des sciences de Paris (1699) et de celle de Berlin (1701). Sa correspondance avec Gottfried Wilhelm Leibniz le conduit à étudier le calcul infinitésimal en collaboration avec son frère Jean. Il fut un des premiers à comprendre et à appliquer le calcul différentiel et intégral, proposé par Leibniz, découvrit les propriétés des nombres dits depuis "nombres de Bernoulli" et donna la solution de problèmes regardés jusque-là comme insolubles. Il pose les principes du calcul des probabilités et introduit les nombres de Bernoulli dans un ouvrage publié après sa mort en 1713.

\parpic[l][t]{%
  \begin{minipage}{40mm}
    \fbox{\includegraphics[width=110px,height=140px]{img/medaillons/bernoulli_jean.eps}}
  \end{minipage}
}
\textbf{Bernoulli, Jean} (1667-1748) était un mathématicien et physicien suisse. Frère de Jacques Bernoulli et père de Daniel et Nicolas Bernoulli, il professa les mathématiques à Groningue (1695), puis à Bâle, après la mort de Jacques Bernoulli (1705), et devint associé des Académies de Paris, de Londres, de Berlin et de Saint-Pétersbourg. Formé par son frère Jacques Bernoulli, il avait longtemps travaillé de concert avec lui à développer les conséquences du nouveau calcul infinitésimal inventé par Gottfried Leibniz ; mais il s'établit ensuite entre eux, à l'occasion de la résolution de quelques problèmes, une rivalité qui dégénéra en inimitié. Il a aussi contribué dans beaucoup de secteurs aux mathématiques y compris le problème d'une particule se déplaçant dans un champ de gravité. Il trouva l'équation de la chaînette en 1690 et développa le calcul exponentiel en 1691. Il eut aussi la gloire de former Leonhard Euler. Il vint à Paris en 1690, et se lia avec les savants les plus distingués, particulièrement avec de L'Hôpital. Jean Bernoulli est devenu membre de la Royal Society en 1712.

\parpic[l][t]{%
  \begin{minipage}{40mm}
    \fbox{\includegraphics[width=110px,height=140px]{img/medaillons/bessel.eps}}
  \end{minipage}
}
\textbf{Bessel, Friedrich} (1784-1846) Né à Minden en Westphalie, Bessel commença à travailler très jeune comme commis. Attiré par la navigation maritime, il s'intéressa aux observations nautiques, construisant lui-même son sextant et étudiant l'astronomie à ses heures de liberté. Il calcula la trajectoire de la comète de Halley, résultat qui fut immédiatement publié et lui permit d'obtenir, en 1806, un emploi d'assistant à l'observatoire de Lilienthal. En 1810, il devint directeur du nouvel observatoire de Königsberg, tout en poursuivant des études mathématiques. Il dut enseigner la mathématique à ses étudiants en astronomie jusqu'en 1825 (date à laquelle Jacobi vint enseigner cette matière à Königsberg). Toute sa vie fut consacrée à l'astronomie (il écrivit plus de 350 articles) et, peu avant sa mort, il commença l'étude du mouvement d'Uranus, problème qui devait aboutir à la découverte de Neptune. En mathématiques, Bessel est connu pour avoir introduit les fonctions qui portent son nom, les utilisant pour la première fois, en 1817, lors de l'étude d'un problème de Kepler, et les employant plus complètement 7 ans plus tard pour étudier les perturbations planétaires.

\parpic[l][t]{%
  \begin{minipage}{40mm}
    \fbox{\includegraphics[width=110px,height=140px]{img/medaillons/biot.eps}}
  \end{minipage}
}
\textbf{Biot, Jean-Baptiste} (1774-1862) Né et décédé à Paris était un physicien, astronome et mathématicien. Jean-Baptiste fit des études secondaires (humanités) à Paris au collège Louis-le-Grand jusqu'en 1791. Il commence des études d'ingénieur à l'École des ponts et chaussées en janvier 1794, puis rejoint l'École centrale des travaux publics (future École polytechnique) à son ouverture en décembre 1794 au Palais Bourbon. Un an plus tard (1795) il rejoint à nouveau l'École des ponts et chaussées pour terminer sa formation d'ingénieur. C'est vers l'enseignement que Biot oriente sa carrière après ses études d'ingénieur. Il devient professeur de mathématiques à l'École centrale du département de l'Oise à Beauvais en 1797. Grâce à l'appui de Laplace, il est nommé en 1800, âgé de 26 ans, professeur de physique mathématique au Collège de France. Il est entre 1816 et 1826 chargé de la moitié du cours de physique pour l'acoustique, le magnétisme et l'optique, Gay-Lussac, titulaire de la chaire de physique, enseignant la chaleur, les gaz, l'hygrométrie, l'électricité et le galvanisme. Il formule avec Félix Savart, la loi de Biot-Savart, qui donne la valeur du champ magnétique produit en un point de l'espace par un courant électrique en fonction de la distance de ce point au conducteur.

\parpic[l][t]{%
  \begin{minipage}{40mm}
    \fbox{\includegraphics[width=110px,height=140px]{img/medaillons/bohr.eps}}
  \end{minipage}
}
\textbf{Bohr, Niels Henrik David} (1885-1962) était un physicien danois, prix Nobel en 1922, pour sa contribution à la physique nucléaire et à la compréhension de la structure atomique. La théorie de Bohr sur la structure atomique, pour laquelle il reçut le prix Nobel de physique en 1922, fut publiée entre 1913 et 1915. Son travail s'inspira du modèle nucléaire de l'atome de Rutherford, dans lequel l'atome est considéré comme un noyau compact entouré d'un essaim d'électrons. Le modèle pose en principe que l'atome n'émet de rayonnement électromagnétique que lorsqu'un électron se déplace d'un niveau quantique à un autre. Ce modèle contribua énormément aux développements ultérieurs de la physique atomique théorique.

\parpic[l][t]{%
  \begin{minipage}{40mm}
    \fbox{\includegraphics[width=110px,height=140px]{img/medaillons/boltzmann.eps}}
  \end{minipage}
}
\textbf{Boltzmann, Ludwig} (1844-1906) était un physicien autrichien qui contribua à établir les bases de la mécanique statistique. Ayant fait ses études à Vienne et à Oxford, il enseigna la physique dans différentes universités allemandes et autrichiennes pendant plus de 40 ans. Développant la théorie cinétique des gaz, notamment à partir des travaux de Maxwell, il établit que la seconde loi de la thermodynamique pouvait être obtenue sur la base de l'analyse statistique. Calculant le nombre de particules dotées d'une énergie donnée, il établit la statistique dite de Maxwell-Boltzmann. Il exprima l'entropie $S$ d'un système en fonction de la probabilité $W$ de son état (par l'intermédiaire de sa fameuse équation de transport à partir de laquelle démontra que l'entropie ne pouvait qu'augmenter au cours du temps... résultat qui était jusque là admis expérimentalement mais sans preuve théorique). Il put aussi établir de manière théorique la loi de Stefan relative au rayonnement d'un corps noir. Mais il lui fallut expliquer comment les principes mécaniques, où les phénomènes sont réversibles, pouvaient engendrer des lois thermodynamiques décrivant des phénomènes marqués par l'irréversibilité. Il avança l'idée que les évolutions irréversibles, quoiqu'elles ne soient que des possibilités parmi d'autres, sont si probables que ce sont pratiquement toujours elles qui se produisent.

\parpic[l][t]{%
  \begin{minipage}{40mm}
    \fbox{\includegraphics[width=110px,height=140px]{img/medaillons/boole.eps}}
  \end{minipage}
}
\textbf{Boole, George} (1815-1864) était un mathématicien et logicien anglais considéré comme le créateur de la logique symbolique. Né à Lincoln et fils d'un petit commerçant, il reçut ses premières leçons de mathématiques de son père, qui lui apprit aussi à fabriquer des instruments d'optique. En dehors des conseils de son père et de quelques années passées dans les écoles locales, Boole est un autodidacte. Quand les affaires de son père déclinèrent, il fut obligé de travailler pour aider sa famille et, dès 16 ans, il enseigna dans des écoles de village ; à 20 ans, il ouvrit sa propre école à Lincoln. Pendant ses loisirs, il étudiait la mathématique à l'Institut de mécanique, créé vers cette époque ; c'est là qu'il se familiarisa avec les Principia de Newton, la Mécanique céleste de Laplace et la Mécanique analytique de Lagrange et qu'il commença à résoudre des problèmes d'algèbre supérieure. Boole soumit au nouveau \textit{Cambridge Mathematical Journal} une série d'articles originaux dont le premier est \textit{Recherches sur la théorie des transformations analytiques} ; ces articles portaient sur les équations différentielles et sur les invariants par transformation linéaire. En 1844, il étudie les liens entre l'algèbre et le calcul infinitésimal dans un important mémoire publié dans les Transactions de la Royal Society, qui lui décerne une médaille cette même année pour sa contribution à l'analyse (c'est-à-dire l'utilisation de l'algèbre dans l'étude des infiniment petits et grands). Développant de nouvelles idées sur la méthode en logique et confiant dans le symbolisme qu'il avait élaboré à partir de ses recherches mathématiques, il publie, en 1847, un opuscule, \textit{Mathematical Analysis of Logic}, dans lequel il soutient que la logique doit être rattachée aux mathématiques et non à la philosophie. Bien qu'il n'eût aucun titre universitaire, Boole fut, sur la base de ses publications, nommé en 1849 professeur au Queen's College à Cork, en Irlande. Avec Boole, en 1847 et en 1854, commence l'algèbre de la logique, c'est-à-dire ce que l'on appelle de nos jours l'algèbre de Boole. Dans son ouvrage de 1854, Boole énonce complètement sa nouvelle méthode symbolique d'inférence logique, qui permet, étant donnés des propositions contenant un certain nombre de termes, d'en tirer, par traitement symbolique des prémisses, des conclusions qui étaient logiquement contenues dans les prémisses. Il rechercha aussi une méthode générale en calcul des probabilités, qui aurait permis, à partir des probabilités connues d'un système d'événements donnés, de déterminer la probabilité de tout autre événement relié logiquement aux événements donnés.

\parpic[l][t]{%
  \begin{minipage}{40mm}
    \fbox{\includegraphics[width=110px,height=140px]{img/medaillons/borel.eps}}
  \end{minipage}
}
\textbf{Borel, Emile} (1871-1956) Reçu major à l'X et à ULM, il choisit cette dernière et se consacre aux mathématiques. Il fonda l'institut Henri Poincaré et fut élu député de l'Aveyron et maire de Saint-Afrique. Il étudie les mesures d'ensembles et notamment, définit les ensembles de mesure nulle et l'ensemble des boréliens, sur lequel on peut définir une mesure. Il se tourne ensuite vers les probabilités et la physique-mathématique. Borel est également considéré comme un mathématicien constructiviste. Il fut à l'origine de la théorie des jeux stratégiques et de la cybernétique que développeront von Neumann et Morgenstern. Son élève Henri Lebesgue utilisera ses résultats en topologie et théorie de la mesure pour sa théorie de l'intégration.

\parpic[l][t]{%
  \begin{minipage}{40mm}
    \fbox{\includegraphics[width=110px,height=140px]{img/medaillons/born.eps}}
  \end{minipage}
}
\textbf{Born, Max} (1882-1970) Né à Breslau est décédé à Göttingen est un physicien allemand, puis britannique. Initialement il suit ses études au collège de König-Wilhelm, et poursuit à l'Université de Breslau suivie par celle d'Heidelberg et de Zürich. Pendant les études pour son doctorat il entra en contact avec des mathématiciens comme Klein, Hilbert, Minkowski, Runge, Schwarzschild. En 1921, il est nommé professeur de physique théorique à Göttingen. Il émigre en Écosse en 1933 et devient citoyen britannique en 1939. Physicien théoricien remarquable, il est principalement connu pour son importante contribution à la physique quantique: développement (1925) de la mécanique quantique matricielle introduite par Werner Heisenberg et, surtout, il sera le premier à donner au carré du module de la fonction d'onde la signification d'une densité de probabilité de présence. Il fut également un pionnier dans la théorie quantique des solides (conditions de Born-Von Karmann) et dans l'électrodynamique non linéaire de Born-Infeld. Il est lauréat de la moitié du prix Nobel de physique de 1954 (l'autre moitié a été remise à Walther Bothe) pour ses recherches fondamentales en mécanique quantique, particulièrement pour son interprétation statistique de la fonction d'onde. La Royal Society lui décerne la Médaille Hughes en 1950.

\parpic[l][t]{%
  \begin{minipage}{40mm}
    \fbox{\includegraphics[width=110px,height=140px]{img/medaillons/bose.eps}}
  \end{minipage}
}
\textbf{Bose, Satyendranath} (1894-1974)  était une mathématicien et physicien indien, connu pour ses contributions à la théorie quantique. Né à Calcutta, Bose a fait ses études au Presidency College de Calcutta. En 1924, il propose une description statistique des systèmes quantiques, reprise par Albert Einstein, et qui n'impose aucune restriction sur la distribution en énergie des particules du système. Cette description est connue sous le nom de "statistique de Bose-Einstein", par opposition à la "statistique de Fermi-Dirac". Appliquée à la théorie du rayonnement du corps noir, cette nouvelle statistique conduit à la distribution de Planck et permet de traiter ce rayonnement comme un gaz de photons. Dans le domaine de la physique des particules élémentaires, la statistique de Bose-Einstein impose à la fonction d'onde des particules (dans l'équation de Schrödinger) d'être parfaitement symétrique pour l'ensemble des variables d'espace et de spin. Les particules obéissant à cette statistique (photons, mésons $\pi$, etc.) sont appelées des bosons. Professeur de physique aux Universités de Calcutta et de Dacca, Satyendranath Bose a été nommé, en 1958, professeur national des Indes.

\parpic[l][t]{%
  \begin{minipage}{40mm}
    \fbox{\includegraphics[width=110px,height=140px]{img/medaillons/broglie.eps}}
  \end{minipage}
}
\textbf{de Broglie, Louis Victor} (1892-1987) était un physicien français et lauréat du prix Nobel, qui apporta une contribution essentielle à la théorie quantique avec ses études de la radiation électromagnétique. Né à Dieppe, Louis de Broglie fit ses études à Paris. Il essaya de cerner la nature dualiste de la matière et de l'énergie et proposa l'association d'une onde à toute particule. Il proposa ainsi directement comment il était possible d'obtenir les règles de quantification du modèle d'atome de Bohr et Sommerfeld en exigeant qu'un nombre entier d'ondes soit logé sur une orbite stationnaire. Sa découverte de la nature ondulatoire des électrons (1924) lui valut le prix Nobel de physique en 1929 sans qu'il proposa toutefois une équation d'onde décrivant les phénomènes quantiques (ce que fera Schrödinger). Il fut élu à l'Académie des sciences en 1933 et à l'Académie française en 1943. Il fut nommé professeur de physique théorique à l'Université de Paris (1928), secrétaire perpétuel de l'Académie des sciences (1942), et conseiller au Commissariat à l'énergie atomique (1945).

\parpic[l][t]{%
  \begin{minipage}{40mm}
    \fbox{\includegraphics[width=110px,height=140px]{img/medaillons/brouwer.eps}}
  \end{minipage}
}
\textbf{Brouwer, Luitzen Egbertus Jan} (1881-1966) fut un grand mathématicien néerlandais du début du 20e siècle. Né d'un père proviseur, il réalisa des études secondaires très brillantes, et très rapides. À l'Université d'Amsterdam, il fut formé par Korteweg, qui est connu pour des contributions en mathématiques appliquées. Il soutient son doctorat en 1904. De 1909 à 1913, Brouwer s'intéresse à la topologie, et découvre la majeure partie des théorèmes auxquels son nom est resté attaché, dont son fameux théorème du point fixe. Pour beaucoup, Brouwer est le père de la topologie moderne. En 1912, il obtient grâce aux recommandations de Hilbert, une chaire à l'Université d'Amsterdam. Il y enseigne la théorie des ensembles, celle des fonctions, et l'axiomatique. Plus tard, il refusera de rejoindre Hilbert à Göttingen. Pendant la première Guerre mondiale sa santé se fragilisa et il s'éloigna quelques temps des champs de la recherche scientifique. Quand il y revint, ce fut pour se consacrer à ses premières amours (sa thèse portait déjà sur ce sujet): les fondements des mathématiques.  Brouwer est le fer de lance avec Poincaré des mathématiques intuitionnistes, par opposition au logicisme de Russel et Frege, et au formalisme de Hilbert. En particulier, pour Brouwer, un théorème d'existence ne peut être vrai que si on peut exhiber un processus, même formel, de construction. Cela le conduit notamment à rejeter la loi du tiers-exclu, qui dit qu'une propriété est ou vraie, ou fausse! Les preuves ainsi obtenues sont souvent plus longues, mais Brouwer fut capable de réécrire des traités de théorie des ensembles, de théorie de la mesure, et de théorie des fonctions en se conformant aux règles de l'intuitionnisme. Bizarrement, Brouwer n'enseigna jamais la topologie. C'est probablement dû au fait que les théorèmes que lui-même avait prouvés ne rentraient plus dans le cadre qu'il s'était fixé. Selon les témoignages de quelques-uns de ses étudiants, il était un personnage vraiment étrange, fou amoureux de sa philosophie, et un professeur auquel il ne fallait surtout pas poser de questions!

\phantomsection
\addcontentsline{toc}{section}{C}
\label{sec:C}

\parpic[l][t]{%
  \begin{minipage}{40mm}
    \fbox{\includegraphics[width=110px,height=140px]{img/medaillons/cantor.eps}}
  \end{minipage}
}
\textbf{Cantor, Georg} (1845-1918) se révèle être un étudiant brillant, notamment dans les matières manuelles. Malgré les injonctions de son père, qui rêve d'en faire un ingénieur, il part en 1862 à Berlin étudier la mathématique, où ses maîtres sont Weierstrass et Kronecker. Il soutient son doctorat en 1867 (sur la théorie des nombres). Les premières recherches postdoctorales de Cantor sont consacrées à la décomposition des fonctions en sommes de séries trigonométriques (les célèbres séries de Fourier) et particulièrement à l'unicité de cette décomposition. Afin de résoudre complètement ce difficile problème, il est amené à introduire et à étudier des ensembles dits "ensembles exceptionnels". Cela le conduit à définir en 1872 très précisément ce qu'est un nombre réel, comme limite d'une suite de nombres rationnels; parallèlement, son ami Dedekind donne la même année une autre définition de la droite des réels, à partir des coupures. Cantor et Dedekind constatent à cette occasion qu'il y a beaucoup plus de réels que de rationnels, mais il n'y a pas jusque-là de définition mathématique à ce "beaucoup plus". En 1874, dans le prestigieux \textit{Journal de Crelle}, Cantor donne une définition du nombre d'éléments d'un ensemble infini qui prolonge naturellement celle du cardinal d'un ensemble infini, qui prolonge celle du cardinal d'un ensemble fini. Il en découle, jusqu'en 1897, une succession de découvertes étranges: il y autant d'entiers pairs que d'entiers tout court, autant de points sur un segment que dans un carré, beaucoup plus de nombres transcendants que de nombres rationnels. Cette hiérarchie dans les ensembles infinis conduit progressivement Cantor à définir des nouveaux nombres, les ordinaux transfinis, et à définir une arithmétique sur ces nombres. Les travaux de Cantor ont eu beaucoup d'influence au 20ème siècle. On citera d'abord, en 1903, un paradoxe soulevé par Russell dans la théorie naïve des ensembles: si $A$ est l'ensemble de tous les ensembles qui ne sont pas éléments d'eux-mêmes, $A$ est-il contenu dans $A$? Les logiciens surmonteront cette difficulté conceptuelle, sans rien changer des conclusions de Cantor. Citons aussi le problème de l'hypothèse du continu. Un des derniers axes de recherche de Cantor était d'estimer le nombre d'éléments de la droite réelle. Plus précisément, Cantor souhaitait prouver l'absence de tout ensemble dont le cardinal soit strictement compris entre le cardinal des entiers et celui des réels. C'est ce que l'on appelle "l'hypothèse du continu". Tous les travaux de Cantor et de ses successeurs pour confirmer ou infirmer l'hypothèse du continu furent vains, et pour cause: en 1963, le logicien Cohen prouva que, dans une théorie standard des ensembles, l'hypothèse du continu est indécidable. On peut très bien supposer qu'elle est vraie ou qu'elle est fausse sans obtenir de contradiction dans la théorie.

\parpic[l][t]{%
  \begin{minipage}{40mm}
    \fbox{\includegraphics[width=110px,height=140px]{img/medaillons/carnot.eps}}
  \end{minipage}
}
\textbf{Carnot, Nicolas Léonard Sadi} (1796-1832), physicien et ingénieur militaire français, considéré comme le créateur de la science thermodynamique. Fils aîné de Lazare Carnot, surnommé "le Grand Carnot", Sadi fit ses études à l'École Polytechnique. En 1824, il décrivit sa conception du moteur à chaleur idéal, appelé "moteur Carnot", dans lequel toute l'énergie disponible est utilisée. Il découvrit que la chaleur ne pouvait passer d'un corps froid à un corps plus chaud, et que le rendement d'un moteur dépendait de la quantité de chaleur qu'il était capable d'utiliser. Cette découverte, ou cycle de Carnot, est à la base de la seconde loi de la thermodynamique.\\

\parpic[l][t]{%
  \begin{minipage}{40mm}
    \fbox{\includegraphics[width=110px,height=140px]{img/medaillons/cartan.eps}}
  \end{minipage}
}
\textbf{Cartan, Élie} (1869-1951) fit ses études primaires à l'École de Dolomieu, puis au collège de Vienne et au lycée de Grenoble. Il suivit au lycée Jeanson-de-Sailly la préparation à l'École Normale Supérieure, où il entra en 1888. Il y suivit notamment les enseignements de Poincaré, Picard et de Hermite. Les premiers travaux d'Élie Cartan qui devaient déboucher sur sa thèse soutenue en 1894, portent sur les groupes de Lie simples complexes, où il reprend, corrige et développe les résultats de structure et de classification obtenus par W. Killing. Élie Cartan obtient un poste de lecteur à l'Université de Montpellier de 1894 à 1896, puis à la Faculté des sciences de Lyon de 1896 à 1903. La même année, il est nommé professeur à la Faculté des sciences de Nancy, où il restera jusqu'en 1909. Il donne en même temps des cours à l'École d'Électrotechnique et de Mécanique Appliquée. Il rédige deux grands articles sur une généralisation en dimension infinie des groupes de Lie simples. Il élabore la méthode du "repère mobile", et la théorie des formes extérieures qui devaient influencer le développement ultérieur de la géométrie différentielle. En 1909, il quitte Nancy pour venir enseigner à la Sorbonne, où il est nommé professeur en 1912. Il assure par ailleurs un enseignement à l'École de Physique et Chimie de Paris. En 1914, il résout le problème de la classification des groupes de Lie simples réels, et détermine les représentations de dimension finie de ces groupes. Pendant la guerre, il sert comme sergent dans l'hôpital aménagé dans les locaux de l'École Normale Supérieure, tout en continuant ses travaux mathématiques. Son oeuvre mathématique ultérieure est considérable, avec près de 200 publications et de nombreux ouvrages. Parmi les thèmes abordés, mentionnons l'étude des systèmes de Pfaff, la théorie de la déformation, l'étude des variétés à courbure constante négative, la théorie de la gravitation d'Einstein, la théorie des connexions affines, les groupes d'holonomie, les espaces riemanniens symétriques, les spineurs. Il est aussi l'auteur de plusieurs articles sur l'histoire de la géométrie. Il prit sa retraite en 1940.


\parpic[l][t]{%
  \begin{minipage}{40mm}
    \fbox{\includegraphics[width=110px,height=140px]{img/medaillons/cauchy.eps}}
  \end{minipage}
}
\textbf{Cauchy, Augustin-Louis} (1789-1857) C'est à Cherbourg que Cauchy commence ses recherches mathématiques sur les polyèdres, et ses premiers résultats sont prometteurs. Mais, fatigué par le cumul de la charge d'ingénieur et des longues veillées de recherche, Cauchy connaît un état dépressif qui s'éternise et le pousse à retourner vivre chez ses parents. À Paris, il cherche une situation en adéquation avec sa volonté de faire de la recherche en mathématique pure. En 1815, il achève un brillant mémoire où il démontre un célèbre théorème de Fermat sur les nombres polygonaux. Ceci fera beaucoup pour sa notoriété, et en 1816, il accède à l'Académie des Sciences, en remplacement de Carnot et Monge touchés par l'épuration. Le cours d'analyse que Cauchy professe à l'École Polytechnique est décrié tant par ses élèves que par ses collègues des autres matières. Pourtant, c'est ce cours publié en 1821 et 1823, qui devait devenir la référence de l'analyse au 19ème siècle, en mettant en avant la rigueur, et plus seulement l'intuition. C'est la première fois que de vraies définitions de limites, de continuité, de convergence de suites, de séries, sont énoncées. Cette rigueur reste toutefois encore relative, puisque Cauchy "prouve" que la limite d'une série de fonctions continues est continue, ce qui est faux. Il est vrai que Cauchy ne dispose pas encore d'une définition claire et précise des nombres réels. C'est l'époque aussi où Cauchy réalise des travaux profonds sur les fonctions d'une variable complexe (établissant par exemple l'expression des résidus), ainsi que des avancées dans la théorie des groupes finis. Cauchy ne fut jamais le chef d'une école de mathématiciens, et il se comporta parfois maladroitement avec de jeunes chercheurs comme Abel ou Galois, dont il sous-estime, ou même perd, des mémoires de première importance. Ses relations avec ses collègues ne sont en général pas très faciles.

\parpic[l][t]{%
  \begin{minipage}{40mm}
    \fbox{\includegraphics[width=110px,height=140px]{img/medaillons/cayley.eps}}
  \end{minipage}
}
\textbf{Cayley, Arthur} (1821-1895) Né à Richmond (Surrey), manifesta très tôt de vives dispositions pour la mathématique. Cependant, malgré le grand intérêt de ses premières publications, il ne put s'imposer comme mathématicien ; il décida de faire des études de droit et devint avocat en 1849. Pendant 14 ans, il exerça ce métier tout en s'adonnant à des recherches scientifiques. En 1863, Cayley est nommé professeur à Cambridge et peut enfin se consacrer entièrement aux mathématiques. Dans l'ensemble de l'oeuvre de Cayley, notamment dans ses travaux de jeunesse, est sensible l'influence des fondateurs de l'école algébrique anglaise qui avaient formulé le programme de l'algèbre moderne en accordant une priorité marquée à l'approche formelle des problèmes. Mathématicien lettré et créateur, Cayley, dans le sillage de l'école anglaise, sut élaborer de nouvelles et fructueuses théories. La richesse de l'approche de Cayley apparaît dès ses premiers travaux sur la théorie des groupes (1854). Cayley, abordant les travaux de Galois, Gauss et Cauchy avec les méthodes des algébristes anglais, donne une définition des groupes abstraits ce qui le conduisit à la notion d'isomorphisme. L'étude des systèmes d'équations linéaires conduisit Cayley à celle des déterminants. Dans ses premiers travaux, il établit de nombreuses règles de calcul sur les déterminants, y compris la relation de multiplication des déterminants qui figurait déjà dans les travaux de Cauchy, Binet et Jacobi. À côté d'études originales sur les déterminants, on y rencontre la notion de tableau rectangulaire représentant les coefficients d'un système d'équations linéaires ou les coefficients d'une transformation linéaire. Cayley étudie les matrices rectangulaires à coefficients réels ou complexes ; il introduit les opérations sur les matrices et décrit leurs propriétés, y compris le caractère non commutatif de la multiplication. Il s'agit là sans doute de la première apparition de l'algèbre linéaire. Quelques années plus tard, Cayley étudiera aussi les systèmes non associatifs et publiera des résultats d'algèbre multilinéaire. Cayley a consacré un grand nombre de ses publications aux problèmes de la géométrie et à l'étude des courbes et des surfaces algébriques. À 22 ans, il émettait l'idée de la géométrie à $n$ dimensions, idée qui fut formulée aussi, presque simultanément, mais sous une forme un peu différente, par Grassman. Cayley ne revint que beaucoup plus tard (en 1870) sur l'espace à $n$ dimensions, mais sa méthode algébrique contribua aux importantes découvertes qui eurent lieu dans les autres domaines de la géométrie. C'est ainsi que, dans le \textit{Sixth Memoir on Quantics} de 1859, il introduit la métrique projective, subordonnant ainsi la géométrie métrique à la géométrie projective ; il démontre alors que les notions fondamentales de la géométrie métrique (angles et distances) sont les invariants et les covariants de certaines transformations linéaires de la quadrique absolue.

\parpic[l][t]{%
  \begin{minipage}{40mm}
    \fbox{\includegraphics[width=110px,height=140px]{img/medaillons/chandrasekhar.eps}}
  \end{minipage}
}
\textbf{Chandrasekhar, Subrahmanyan} (1910-1995) obtint à l'âge de 23 ans son doctorat au Trinity College de l'Université de Cambridge. Spécialiste en astrophysique Chandrasekhar fit progresser de façon décisive la connaissance de l'évolution hydrodynamique et hydromagnétique des transferts d'énergie par rayonnement sans oublier les effets quantiques et relativistes dans les évolutions des étoiles. Sa contribution majeure dans ce domaine est la transformation des étoiles en naines blanches et au-delà d'un astre d'une masse supérieur à la limite de Chandrasekhar ($1.44$ celle du Soleil), l'effondrement de l'astre en une étoile à neutrons. Les objets plus massifs donnant eux des Trous Noirs.\\\\

\parpic[l][t]{%
  \begin{minipage}{40mm}
    \fbox{\includegraphics[width=110px,height=140px]{img/medaillons/clairaut.eps}}
  \end{minipage}
}
\textbf{Clairaut, Alexis-Claude} (1713-1765) était un membre de l'Académie française des sciences et fut l'un des mathématiciens et physiciens les plus renommés du 18ème siècle. À l'âge de 10 ans, il connaissait le calcul infinitésimal, à 12 ans, il soumettait sa première étude à l'académie des sciences et à 18 ans, il publia un livre contenant des extensions importantes à la géométrie qui lui ont valu l'admission à l'académie en 1731. Clairaut fut l'un des scientifiques qui accompagnaient Maupertuis en Laponie pour acquérir les dates nécessaires pour la détermination de la forme de la Terre. En 1743, il publia sa \textit{Théorie de la figure de la Terre}, qui calculait plus précisément que l'avait fait Newton, la forme qu'adopte un corps en rotation due à la gravitation naturelle de ses parties. En 1760, il publia sa\textit{ Théorie du mouvement des comètes}, qui prédit avec précision la date à laquelle la comète de Halley sera arrivée au point le plus proche du Soleil.

\parpic[l][t]{%
  \begin{minipage}{40mm}
    \fbox{\includegraphics[width=110px,height=140px]{img/medaillons/cohen.eps}}
  \end{minipage}
}
\textbf{Cohen, Paul Joseph} (1934-2007) était un mathématicien et logicien né au New Jersey en décédé à Stanford. En 1963, Cohen a découvert une nouvelle construction de modèles, appelée "forcing", qui joue désormais un rôle fondamental dans la théorie des ensembles et dans la théorie des modèles. Il a aussi construit des modèles de la théorie des ensembles (supposée consistante) dans lesquels l'axiome du choix et l'hypothèse du continu ne sont pas vérifiés, ce qui, compte tenu de l'oeuvre antérieure de Kurt Gödel, établit que l'axiome du choix et l'hypothèse du continu sont indépendants des systèmes usuels de la théorie des ensembles. Ce travail a valu à Cohen, en 1966, la médaille Fields de l'Union Mathématique Internationale. Il est également l'auteur de travaux intéressants en analyse classique.

\parpic[l][t]{%
  \begin{minipage}{40mm}
    \fbox{\includegraphics[width=110px,height=140px]{img/medaillons/connes.eps}}
  \end{minipage}
}
\textbf{Connes, Alain} (1947-) Né à Draguignan, ancien élève à l'École Normale Supérieure, il a reçu, en 1980, le prix Ampère, l'un des plus importants décernés par l'Académie des sciences. Il a été élu membre de cette académie, dont il a été le benjamin, en 1981. Les premiers travaux d'Alain Connes s'inscrivent directement dans la tradition de John von Neumann et de ses continuateurs immédiats. Le développement de la physique quantique vers les années vingt avait mis à l'ordre du jour l'étude d'espaces non plus à trois dimensions, comme celui où nous croyons vivre, ni à quatre, comme en relativité einsteinienne, mais à une infinité de dimensions (les espaces de Hilbert). L'un des outils essentiels de la physique quantique est la notion d'opérateur dans un tel espace, notion généralisant celle de rotation d'un espace euclidien. La théorie des algèbres d'opérateurs a débuté vers 1930 par les travaux de von Neumann, qui a montré l'importance d'un certain type d'algèbres d'opérateurs, appelées aujourd'hui "algèbres de von Neumann", et qui a établi pour ces algèbres un théorème de décomposition en facteurs premiers assez analogue au théorème de décomposition bien connu pour les nombres entiers usuels. Dès l'origine, les facteurs avaient été classés en trois types: facteurs de type I, II, III. On a eu assez tôt une bonne compréhension des facteurs de type I et pas mal d'informations sur  ceux de type II, mais les facteurs de type III sont restés pendant longtemps beaucoup plus mystérieux. Même les exemples étaient rares et von Neumann disait, à propos de ce cas: "C'est le plus réfractaire de tous, et les outils pour l'étudier nous font défaut, au moins pour l'instant". La première réussite de Connes, qui lui a d'emblée valu la renommée internationale, a été une percée spectaculaire vers l'élucidation de la structure des facteurs de type III ; on peut dire qu'il est le premier à avoir acquis une connaissance concrète de ces objets, jusque-là assez énigmatiques, pris dans leur ensemble. Très grosso modo, les résultats de Connes ramènent l'étude des facteurs de type III à celle des facteurs de type II et de leurs automorphismes. L'oeuvre d'Alain Connes est celle d'un mathématicien très complet, capable de résoudre des problèmes difficiles, légués par le passé, mais aussi de transformer entièrement une discipline par l'introduction d'idées nouvelles, d'une grande originalité.

\parpic[l][t]{%
  \begin{minipage}{40mm}
    \fbox{\includegraphics[width=110px,height=140px]{img/medaillons/copernic.eps}}
  \end{minipage}
}
\textbf{Copernic, Nicolas} (1473-1543) étudie à l'Université de Cracovie à partir de 1491, il se rend ensuite en Italie pour y suivre des cours de droit canon à l'Université de Bologne. Il suit également les cours d'astronomie de Domenico Maria Novara, un des premiers scientifiques à remettre en cause les enseignements de Ptolémée. En 1500, il enseigne la mathématique à Rome, avant de retourner pour un an à Frauenburg où son oncle l'a nommé chanoine en 1497. Ayant obtenu l'autorisation de poursuivre ses études en Italie, il s'inscrit aux facultés de droit et de médecine de Padoue et obtient son doctorat en droit canon à Ferrare en 1503. Enfin, il retourne à Frauenburg où il fait construire un observatoire et entame ses recherches en astronomie. Il y demeurera jusqu'à sa mort. La cosmologie de l'époque est alors basée sur le système géocentrique de Ptolémée. La Terre se trouve immobile au centre de plusieurs sphères concentriques qui portent la Lune, Mercure, Vénus, le Soleil, Mars, Jupiter, Saturne et enfin les étoiles. Mais ce système ne convient pas à Copernic, qu'il trouve compliqué et bancal. Il consulte alors les auteurs de l'Antiquité (Cicéron, Aristarque de Samos, etc.) et constate que certains d'entre eux envisagent la rotation des planètes, dont la Terre, autour du Soleil, considéré comme fixe. Copernic démontre alors que la combinaison des mouvements de la Terre et des planètes explique parfaitement le mouvement apparent des planètes (dans le sens direct et rétrograde). De plus, il établit que leurs changements de diamètre apparent apparaissent comme une conséquence de leur révolution autour du Soleil. Ses recherches se poursuivront pendant 36 ans et il démontrera que la Lune est un satellite de la Terre et que l'axe de la Terre n'est pas fixe. Son oeuvre maîtresse \textit{De Revolutionibus orbium coelestium} est publiée en 1543 à Nuremberg et Copernic n'en reçoit les premiers exemplaires que quelques heures avant sa mort. Dans la dédicace qu'il fait au Pape Paul III, il présente son système comme une pure hypothèse, évitant ainsi la vindicte de l'Église. Adopté un siècle après sa mort après avoir été violemment rejeté, le système copernicien apporta une profonde révolution dans la conception du monde et plus généralement dans la pensée scientifique.

\parpic[l][t]{%
  \begin{minipage}{40mm}
    \fbox{\includegraphics[width=110px,height=140px]{img/medaillons/coriolis.eps}}
  \end{minipage}
}
\textbf{Coriolis, Gaspard} (1792-1843) était un ingénieur et mathématicien français qui mit en évidence les forces centrifuges composées, dites "forces de Coriolis". Cet ingénieur des Ponts et Chaussées est l'auteur d'importants travaux en mécanique. En 1835, il démontra que l'accélération d'un mobile dans un référentiel en rotation est soumis à une complémentaire (force de Coriolis) perpendiculaire au sens de déplacement du mobile dans ce référentiel. Bien que de faible intensité à la surface de la Terre, cette force, produite par la rotation de la planète, influence la direction des courants marins et aériens. Elle produit une déviation vers l'est et explique, par exemple, le mouvement circulaire des ouragans.

\parpic[l][t]{%
  \begin{minipage}{40mm}
    \fbox{\includegraphics[width=110px,height=140px]{img/medaillons/coulomb.eps}}
  \end{minipage}
}
\textbf{Coulomb, Charles Augustin} (1736-1806) était un physicien français, pionnier de la théorie de l'électricité. Né à Angoulême, il servit comme ingénieur militaire pour la France aux Antilles, mais se retira à Blois à la révolution française, pour continuer ses recherches sur le magnétisme, le frottement et l'électricité. En 1777, il inventa la balance de torsion qui permet de mesurer la force de l'attraction magnétique et électrique. Grâce à cette invention, Coulomb fut capable de formuler le principe, maintenant connu sous le nom de "loi de Coulomb", qui gouverne l'interaction entre les charges électriques. En 1779, Coulomb publia le traité \textit{Théorie des machines simples}, une analyse du frottement dans les machines. Après la révolution, Coulomb quitta sa retraite et aida le nouveau gouvernement à concevoir un système métrique pour les poids et mesures. L'unité utilisée pour exprimer la quantité de charge électrique, le "Coulomb", tient son nom du physicien.

\parpic[l][t]{%
  \begin{minipage}{40mm}
    \fbox{\includegraphics[width=110px,height=140px]{img/medaillons/cournot.eps}}
  \end{minipage}
}
\textbf{Cournot, Antoine Augustin} (1801-1877) étudia au collège de Gray de 1809 à 1816. Il obtient des prix d'excellence de mathématiques. Il entre en 1820 au collège Royal de Besançon et obtient le prix d'honneur de mathématiques spéciales. Avec deux mémoires et deux traductions de traités divers de mathématiques, il se fait remarquer par Poisson, qui le fait nommer en 1834 professeur d'analyse et de mécanique à la faculté des sciences de Lyon. Augustin Cournot est un savant, c'est-à-dire un homme de savoir étendu à tous les domaines de la science, un savant philosophe mais, qui par sa modestie, n'a pas connu la célébrité. Cournot fut d'abord un professeur et un vulgarisateur d'une grande clarté. Trois ouvrages mathématiques le distingue:\textit{ Traité élémentaire de la théorie des fonctions et du calcul infinitésimal} (1841); \textit{Exposition de la théorie des chances et des probabilités} (1843) ; \textit{De l'origine et des limites de la correspondance entre l'algèbre et la géométrie} (1847). Mais le génie de Cournot se situe dans l'introduction des probabilités en économie. Il est le précurseur des théories modernes en économie, reprises ensuite par Léon Walras qui dans sa notice autobiographique achevée en 1904, ainsi que dans plusieurs lettres, a rappelé le rôle primordial qu'ont joué dans le développement de sa pensée, d'une part, l'oeuvre d'Antoine Augustin Cournot et d'autre part, celle de son père, l'économiste et philosophe Auguste Walras qui fut le condisciple d'Augustin Cournot à l'École Normale.

\parpic[l][t]{%
  \begin{minipage}{40mm}
    \fbox{\includegraphics[width=110px,height=140px]{img/medaillons/clausius.eps}}
  \end{minipage}
}
\textbf{Clausius, Rudolf} (1822-1888) était l'un des plus grands physiciens du 19ème siècle. Il est connu principalement pour sa contribution à l'étude de la thermodynamique. Le premier, ce savant allemand formula ce que l'on a coutume d'appeler le "deuxième principe" et proposa une définition claire de l'entropie. Il est aussi l'un des principaux créateurs de la théorie cinétique des gaz. Né à Köslin, en Poméranie, Clausius fréquenta les Universités de Berlin, puis de Halle dont il sortit diplômé en 1848. Professeur jusqu'à sa mort, il fut titulaire de la chaire de physique de l'École Royale d'Artillerie et du Génie à Berlin (1850-1855), puis, simultanément, à l'Université et à l'École Polytechnique de Zürich (1855-1867), ensuite à l'Université de Würzburg (1867-1869), enfin à celle de Bonn, de 1869 à sa mort. Sa première publication, en 1850 dans les \textit{Annalen der Physik} de Poggendorff, attira largement l'attention. Il cherchait à y concilier l'idée de l'équivalence entre le travail et la chaleur. Clausius fit remarquer que l'hypothèse de la conservation de la chaleur dans le processus de transfert n'était pas une partie essentielle de la théorie de Carnot. Il établit en fait que, dans une machine idéale, la quantité de chaleur prise à la chaudière doit toujours être supérieure à celle qui est cédée au condenseur, et ce d'une quantité exactement équivalente au travail fourni. Cette importante synthèse effectuée, Clausius, dans la même publication, énonça ce que nous appelons aujourd'hui le "deuxième principe de la thermodynamique". C'était la généralisation de la nécessité, déjà établie par Carnot, de la présence, non seulement d'un corps chaud (la chaudière), mais aussi d'un corps froid (le condenseur) pour qu'un travail soit fourni par une machine à vapeur. En 1854, Clausius, poussant plus avant les vues exprimées dès 1850, proposa le premier énoncé clair du concept de l'entropie. Il cherchait à mesurer l'aptitude de l'énergie calorifique de n'importe quel système réel non idéal à fournir du travail. Dans le cas de la conduction thermique le long d'un barreau solide, par exemple, la chaleur passe de l'extrémité chaude à l'extrémité froide sans fournir aucun travail, bien que ce transfert s'accompagne d'une diminution de l'aptitude de l'extrémité chaude à servir par la suite de source potentielle de travail. Cette diminution survient parce qu'à la fin du processus l'énergie calorifique est détenue par un corps situé à une température inférieure à celle de l'état initial. Elle n'a donc pas été perdue, mais seulement dégradée puisque, d'après le deuxième principe de la thermodynamique, on ne peut retrouver la température initiale qu'avec l'aide d'un travail extérieur. Les dernières contributions majeures de Clausius à la science datent de 1857 et 1858 et sont relatives à la théorie cinétique des gaz. Bien qu'il ne soit pas le premier à avoir conçu cette dernière, déjà proposée et discutée par Joule et Krönig notamment, il prend rang avec Maxwell parmi ses fondateurs. Il introduisit le concept du libre parcours moyen et établit l'importante distinction entre l'énergie de translation et l'énergie interne d'une particule de gaz. De plus, on lui reconnaît généralement le mérite d'avoir, par ses travaux théoriques, jeté un pont entre la théorie atomique et la thermodynamique.

\parpic[l][t]{%
  \begin{minipage}{40mm}
    \fbox{\includegraphics[width=110px,height=140px]{img/medaillons/curiepierre.eps}}
  \end{minipage}
}
\textbf{Curie, Pierre} (1859-1906) est considéré comme un des pionniers de la chimie/physique sur la radioactivité. C'est même lors d'une thèse publiée en 1898 que le terme "radioactivité" fut employé pour la première fois par sa femme Marie et lui. L'éducation de Pierre commença à un très jeune âge par son père, qui était médecin général. Les Curie avaient l'habitude de fréquenter la campagne et les environs de Paris les dimanches ; Pierre, lors de ses promenades, apprit rapidement tous les noms de plantes et d'animaux. Étant donné que l'école n'était pas obligatoire à cette époque (pas avant 1881 où la loi Ferry l'a rendue obligatoire), Pierre reçut son éducation à la maison, en compagnie de sa mère, ensuite avec son frère et par après, avec des précepteurs et finalement, seul. À l'âge de 14 ans, l'éducation de Pierre fut confiée à M. Bazille qui lui enseigna la mathématique élémentaire et spéciale, ceci développa énormément les capacités mentales de Pierre qui avait clairement un intérêt pour la mathématique. À l'âge de 16 ans, il fut reçu bachelier en sciences. En 1877, il obtint la licence en sciences physiques de l'école de pharmacie. Dans les années qui suivront, il étudiera les cristaux et le magnétisme, ce qui le mènera éventuellement à la découverte de la piézo-électricité. En 1877, il prit un poste comme préparateur où il fut payé la somme de 1200 francs par année. Il devint par après démonstrateur d'expériences de physique pour les laboratoires jusqu'en 1882 où il devint directeur de tous les travaux pratiques aux écoles de physique et de chimie industrielle. Pierre épousa sa femme Marie Sklodowska en 1895 et ils eurent ensemble deux enfants, Irène et Êve. Pierre Curie gagna en 1903, avec sa femme, le prix Nobel de physique pour leurs travaux sur les substances radioactives et leurs découvertes de deux nouveaux éléments: le radium et le polonium.

\parpic[l][t]{%
  \begin{minipage}{40mm}
    \fbox{\includegraphics[width=110px,height=140px]{img/medaillons/curiemarie.eps}}
  \end{minipage}
}
\textbf{Curie, Marie} (1867-1934) était une chimiste et physicienne née à Varsovie et décédée en Haute-Savoie. Fille d'un père professeur de mathématiques et de physique et d'une mère institutrice, elle est la benjamine d'une famille de 4 soeurs. Entre 1876 et 1878 elle perd une soeur et sa mère. Elle se réfugie alors dans les études où elle excelle dans toutes les matières, et où la note maximale lui est accordée. Elle obtient ainsi son diplôme de fin d'études secondaires avec la médaille d'or en 1883. Elle souhaite poursuivre des études supérieures et enseigner, mais ces études sont interdites aux femmes. Lorsque sa soeur aînée, Bronia, part faire des études de médecine à Paris, Marie s'engage comme gouvernante en province en espérant économiser pour la rejoindre, tout en ayant initialement pour objectif de revenir en Pologne pour enseigner. Au bout de trois ans, elle regagne Varsovie, où un cousin lui permet d'entrer dans un laboratoire. En 1891, elle part pour Paris, où elle est hébergée par sa sœur et son beau-frère. La même année, elle s'inscrit pour des études de physique à la faculté des sciences de Paris. Trois ans plus tard, elle obtient sa licence en sciences physiques, en étant première de sa promotion. Pendant l'été, une bourse d'études lui est accordée, qui lui permet de poursuivre ses études à Paris. Un an plus tard, elle obtient sa licence en sciences mathématiques, en étant seconde. Elle hésite alors à retourner en Pologne. Lors d'une soirée elle rencontre Pierre Curie (son future époux), qui est chef des travaux de physique à l'École municipale de physique et de chimie industrielles et étudie également le magnétisme, avec lequel elle va travailler. Marie reçoit (avec son époux Pierre Curie) une moitié du prix Nobel de physique de 1903 (l'autre moitié est remise à Henri Becquerel) pour leurs recherches sur les radiations. En 1911, elle obtient le prix Nobel de chimie pour ses travaux sur le polonium et le radium.

\phantomsection
\addcontentsline{toc}{section}{D}
\label{sec:D}

\parpic[l][t]{%
  \begin{minipage}{40mm}
    \fbox{\includegraphics[width=110px,height=140px]{img/medaillons/dalton.eps}}
  \end{minipage}
}
\textbf{Dalton, John} (1766-1844) était un chimiste et physicien britannique, qui développa la théorie atomique sur laquelle fut fondée la science physique moderne. Dalton commença en 1787 une série d'observations météorologiques qu'il poursuivit pendant 57 ans, accumulant quelque $200'000$ observations et mesures du temps dans la région de Manchester! L'intérêt de Dalton pour la météorologie le conduisit à étudier différents phénomènes ainsi que les instruments utilisés pour les mesurer. Il fut le premier à prouver la validité de l'idée selon laquelle la pluie est précipitée par une baisse de température, non par un changement de la pression atmosphérique. Dalton arriva à sa théorie atomique par une étude des propriétés physiques de l'air atmosphérique et des autres gaz. Au cours de ses recherches, il découvrit la loi des pressions partielles des gaz mélangés, souvent connue comme la "loi de Dalton", selon laquelle la pression totale exercée par un mélange de gaz est égale à la somme des pressions individuelles qu'exercerait chacun des gaz s'il occupait seul le volume entier.

\parpic[l][t]{%
  \begin{minipage}{40mm}
    \fbox{\includegraphics[width=110px,height=140px]{img/medaillons/davinci.eps}}
  \end{minipage}
}
\textbf{Da Vinci, Leonardo} (1452-1519) était un peintre, sculpteur, architecte et homme de sciences italien. Homme d'esprit universel, à la fois artiste, scientifique, inventeur et philosophe, Leonardo incarna l'esprit universaliste de la Renaissance et demeure l'un des grands hommes de cette époque. À 5 ans, son père ayant noté ses dons pour le dessin, le place comme apprenti dans l'atelier de Verrocchio, à Florence. Il entre à 20 ans à la Guilde des peintres, et débute sa carrière de peintre par des oeuvres immédiatement remarquables comme \textit{La vierge à l'oeillet}, ou \textit{L'Annonciation} (1473). Il améliore la technique du sfumato (impression de brume) à un point de raffinement jamais atteint avant lui. En 1481, le monastère de San Donato lui commande \textit{L'Adoration des Mages}, mais Leonardo, vexé de pas être choisi pour la décoration de la chapelle Sixtine à Rome, ne terminera jamais ce tableau et quitte Florence pour Milan. Après la réalisation de \textit{La Vierge aux rochers}, pour la chapelle San Francesco Grande, et celle de la statue équestre de Francesco Sforza, il trouve la gloire dans toute l'Italie. En 1495, les Dominicains de Sainte-Marie-des-Grâces lui commandent \textit{La Cène}. En 1498, il réalise le plafond du palais Sforza. De cette époque, datent aussi \textit{La Joconde} et \textit{La Bataille d'Anghiari}. Leonardo réalise aussi une grande quantité d'études sur la zoologie, la botanique, l'anatomie, la géologie. Il imagine de multiples appareils et machines, dont la première machine volante, qui resteront au stade de dessins. Plus qu'en tant que scientifique proprement dit, Leonardo da Vinci a impressionné ses contemporains et les générations suivantes par son approche méthodique du savoir, du savoir apprendre, du savoir observer, du savoir analyser. La démarche qu'il déploya dans l'ensemble des activités qu'il abordait, aussi bien en art qu'en technique (les deux ne se distinguant d'ailleurs pas dans son esprit), procédait d'une accumulation préalable d'observations détaillées, de savoirs disséminés ça et là, qui tendait vers un surpassement de ce qui existait déjà, avec la perfection pour objectif. Bon nombre des croquis, notes et traités de Leonardo da Vinci ne sont pas à proprement parler des trouvailles originales, mais sont le résultat de recherches effectuées dans un souci encyclopédique, avant l'heure. En 1516, il rejoint la cour de François Ier, où il participe à des projets d'urbanisme.De Léonard de Vinci, subsistent aujourd'hui 7'000 notes et dessins, et quarante oeuvres attestées, dont huit ont disparu.

\parpic[l][t]{%
  \begin{minipage}{40mm}
    \fbox{\includegraphics[width=110px,height=140px]{img/medaillons/dantzig.eps}}
  \end{minipage}
}
\textbf{Dantzig, George Bernard} (1914-2005) était un mathématicien né à Portland et décédé à Stanford, inventeur du fameux algorithme du simplexe en optimisation linéaire. Son père, Tobias, était un mathématicien russe qui avait étudié avec Poincaré à Paris. Il a épousé une collègue de la Sorbonne, Anja Ourisson, et le couple a émigré aux États-Unis. Il est l'acteur principal d'une histoire fameuse en mathématique. Dans l'un de ses cours de doctorat à l'Université de Berkeley, le professeur Jerzy Neyman a proposé deux problèmes dits "ouverts" en statistiques (pour rappel un problème ouvert est un problème qui bien qu'ayant été formulé, n'a pas encore été résolu). De tels problèmes sont d'une difficulté importante et demandent des recherches pouvant s'étaler sur plusieurs années. Dantzig était en retard et croyait qu'il s'agissait de devoirs. Sans prendre plusieurs années mais bien quelques jours, il les a résolus. Il a reçu son doctorat de Berkeley en 1946. Six ans plus tard, il était engagé pour faire de la recherche mathématique à la RAND Corporation, où il implante l'algorithme du simplexe dans les ordinateurs. En 1960, l'Université de Berkeley l'engage pour enseigner l'informatique, pour éventuellement devenir le responsable du centre de recherche opérationnelle. Six ans plus tard, il occupe un poste similaire à l'Université Stanford, poste qu'il occupe jusqu'à sa retraite pendant les années 1990. En plus de ses travaux sur l'algorithme du simplexe et l'optimisation linéaire, il a aussi travaillé sur les méthodes de décomposition des problèmes de grande taille, l'analyse de sensibilité, les méthodes de résolution matricielles avec pivot, l'optimisation non linéaire et l'optimisation stochastique.

\parpic[l][t]{%
  \begin{minipage}{40mm}
    \fbox{\includegraphics[width=110px,height=140px]{img/medaillons/debye.eps}}
  \end{minipage}
}
\textbf{Debye, Peter Joseph Wilhelm} (1884-1966) était un physicien et chimiste né à Maastricht et décédé à New-York. Debye s'inscrit en 1901 à l'Université d'Aix-la-Chapelle en Allemagne. Il y étudie les mathématiques et la physique classique et en sort en 1905 titulaire d'un diplôme d'électrotechnique. En 1907, il produit sa première publication scientifique, une solution mathématique élégante d'un problème mettant en jeu des courants de Foucault. Il étudie à Aix-la-Chapelle sous la direction d'Arnold Sommerfeld. En 1906, il accompagne Sommerfeld à Münich comme assistant. Il y obtient son doctorat en 1908 avec une thèse sur la pression de radiation. En 1910, il démontre la loi de Planck par une méthode dont Max Planck reconnut qu'elle était plus simple que la sienne. En 1911, Debye est nommé professeur à Zürich. Il se rend ensuite à Utrecht en 1912, à Göttingen en 1913, il est de retour à Zürich en 1920, se rend à Leipzig en 1927 et à Berlin en 1934 où il devient directeur de la \textit{Société Kaiser Wilhelm} qui prendra en 1938 le nom de \textit{Société Max Planck}. En 1912, il étend la théorie d'Albert Einstein de la chaleur spécifique aux basses températures en incluant des contributions des phonons de basses fréquences (modèle de Debye). En 1913, il étend la théorie de Niels Bohr de la structure atomique en introduisant des orbites elliptiques, un concept également proposé par Arnold Sommerfeld. Debye profite en 1938 d'une proposition de conférence à l'Université de Cornell à Ithaca pour se rendre aux États-Unis et ensuite rester à l'Université de Cornell, où il devient professeur, puis pendant 10 ans est directeur du département de chimie. Il demeure à Cornell le reste de sa carrière. Il prend sa retraite en 1952, mais continue ses recherches jusqu'à sa mort.

\parpic[l][t]{%
  \begin{minipage}{40mm}
    \fbox{\includegraphics[width=110px,height=140px]{img/medaillons/descartes.eps}}
  \end{minipage}
}
\textbf{Descartes, René} (1596-1650) était un philosophe, scientifique et mathématicien français, fondateur du rationalisme moderne. Né à La Haye, d'un père conseiller au parlement de Rennes, Descartes reçut, de 1607 à 1614, l'enseignement, décisif pour lui, des pères jésuites du Collège royal de La Flèche. Cette expérience le conduisit à proposer une refondation des sciences, critiquant l'absence de fondement de l'enseignement professé. Il reçut une formation de juriste en 1616 puis entra dans la carrière militaire en 1618, entreprit des voyages, mêla vie scientifique et vie mondaine, avant de se consacrer pleinement à la philosophie. Il passa sa vie entre la France et les Pays-Bas, fuyant les villes, fréquentant les bibliothèques et rencontrant les esprits les plus illustres de son temps, notamment Bérulle, Fermat, Gassendi, Hobbes et Pascal. Il mourut d'une pneumonie à Stockholm, léguant à la postérité une oeuvre entourée de légendes et imprégnée d'un esprit nouveau.

\parpic[l][t]{%
  \begin{minipage}{40mm}
    \fbox{\includegraphics[width=110px,height=140px]{img/medaillons/dirac.eps}}
  \end{minipage}
}
\textbf{Dirac, Paul Adrien Maurice} (1902-1984) Né à Bristol, Dirac fait ses études aux Universités de Bristol et de Cambridge. En 1926, pour son doctorat (la première thèse au monde ayant pour sujet la mécanique quantique!), il introduit un formalisme général pour la physique quantique peu après Heisenberg mais indépendamment (il y retrouve la non-commutativité des opérateurs position et quantité de mouvement). En 1928, il élabore une théorie relativiste pour décrire les propriétés de l'électron. Celle-ci le conduit à postuler l'existence d'une particule identique à l'électron dans tous ses aspects mais de charge opposée, c'est-à-dire positive et devant s'annihiler en même temps que l'électron négatif lors d'une collision avec celui-ci. La théorie de Dirac est confirmée en 1932 quand le physicien Carl Anderson découvre le positron. Dirac contribue aussi, avec Fermi, au développement de la statistique dite de Fermi-Dirac, décrivant le comportement collectif des particules de spin demi-entier. En 1933, Dirac partage le prix Nobel de physique avec le physicien autrichien Erwin Schrödinger. En 1939, il devient membre de la Société royale. Il est professeur de mathématiques à Cambridge de 1932 à 1968, professeur de physique à l'Université d'État de Floride de 1971 jusqu'à sa mort, et membre de l'Institute of Advanced Studies (IAS) périodiquement entre 1934 et 1959.

\parpic[l][t]{%
  \begin{minipage}{40mm}
    \fbox{\includegraphics[width=110px,height=140px]{img/medaillons/dirichlet.eps}}
  \end{minipage}
}
\textbf{Dirichlet (-Lejeune), Peter Gustav} (1805-1859) Né à Düren (Allemagne), Dirichlet est un élève brillant, qui achève ses études secondaires à 16 ans. Devant la faible qualité des formations universitaires allemandes à cette époque, Dirichlet décide de partir étudier à Paris, emportant avec lui les\textit{ Disquisitiones Arithmeticae} de Gauss comme une bible. Dans la capitale française, sa situation personnelle est facilitée par le général Foy, un ancien grand général des campagnes napoléoniennes, dont il devient le précepteur des enfants, et qui se montrera bienveillant avec lui. Dirichlet rencontre alors quelques-uns des plus grands mathématiciens, dont Legendre, Poisson, Laplace et Fourier. Ce dernier surtout impressionnera beaucoup Dirichlet, et sera à l'origine de l'intérêt qu'il portera aux séries trigonométriques et à la physique-mathématique. C'est à Paris que Dirichlet rédige sa première contribution d'importance aux mathématiques, étant à l'initiative en 1825 de la preuve du cas $n=5$ dans le grand théorème de Fermat, preuve achevée par Legendre dans la foulée. Fin 1825, le général Foy décède, et Dirichlet décide de retourner en Allemagne. Il enseigne d'abord à l'Université de Breslau, au lycée militaire de Berlin, puis à l'Université de Berlin à partir de 1829, où il restera 27 ans durant. Parmi ses élèves, on retiendra les noms de Kronecker et Riemann. Dirichlet est décrit comme un bon professeur, mais non exempt de défauts. Il donne l'apparence de quelqu'un de sale, toujours affublé d'un cigare et d'un café, visiblement peu préoccupé de l'image qu'il donne. On dit aussi de lui qu'il était très souvent en retard. En 1848, son maître et ami Karl Jacobi est diagnostiqué comme étant malade du diabète. Dirichlet l'accompagne dans un voyage de 18 mois en Italie. De retour en Allemagne, Dirichlet commence à être lassé des lourdes charges d'enseignement qu'il doit assumer. À la mort de Gauss, il prend sa succession à Göttingen. C'est malheureusement pour peu de temps, car lui-même s'éteint en 1859 des suites d'un malaise cardiaque. L'éventail des travaux de Dirichlet illustre la profondeur de la culture mathématique allemande au début de son âge d'or. On lui doit le premier énoncé d'une condition suffisante de convergence d'une série de Fourier (dans le cas des fonctions continues par morceaux), le théorème de la progression arithmétique, le prolongement des fonctions harmoniques définies sur la frontière d'un ouvert et toute une classe d'équations aux dérivées partielles porte le nom de "problème de Dirichlet".  Nous lui devons aussi de très nombreuses contributions en arithmétique, où il existe le théorème des unités de Dirichlet, les séries de Dirichlet, etc.

\parpic[l][t]{%
  \begin{minipage}{40mm}
    \fbox{\includegraphics[width=110px,height=140px]{img/medaillons/doppler.eps}}
  \end{minipage}
}
\textbf{Doppler, Christian} (1803-1853) était un mathématicien et physicien autrichien, célèbre pour sa découverte de l'effet Doppler. Après avoir étudié à l'Université de Vienne, Doppler devient assistant-professeur dans cet établissement en 1829. Ce poste n'étant pas renouvelé, il envisage un temps une émigration vers les États-Unis. Il renonce à quitter son pays après avoir été nommé à Prague en 1837, puis à l'école polytechnique de Vienne en 1849. En 1850, il fonde l'Institut de Physique de l'Université de Vienne dont il est seul professeur et le premier directeur. Atteint d'une affection pulmonaire, la tuberculose, il quitte ses fonctions en 1852. Son travail scientifique est varié : optique, astronomie, électricité... Sa publication la plus célèbre a été présentée en 1842 à l'académie royale des sciences de Bohème et a pour titre \textit{Sur la lumière colorée des étoiles doubles et d'autres étoiles du ciel}, utilisant l'effet Doppler. Ses calculs étaient erronés, le décalage réel de la fréquence lumineuse étant trop faible pour pouvoir être détecté à l'époque. En 1846, Doppler publie une correction de son travail initial où il tient compte des vitesses relatives de la source de lumière et de l'observateur.

\parpic[l][t]{%
  \begin{minipage}{40mm}
    \fbox{\includegraphics[width=110px,height=140px]{img/medaillons/drude.eps}}
  \end{minipage}
}
\textbf{Drude, Paul Karl Ludwig} (1863-1906) était un physicien né à Braunschweig et décédé à Berlin. Drude a commencé ses études en mathématiques à l'Université de Göttingen, mais s'est ensuite dirigé vers la physique. Il y termine son doctorat en 1887 et rédige un mémoire portant sur la réflexion et la diffraction de la lumière dans les cristaux. En 1894 il est nommé professeur à l'Université de Leipzig. En 1900 il obtient le poste d'éditeur de la revue scientifique \textit{Annalen der Physik}. La même année, il développé un modèle (le modèle de Drude) expliquant les propriétés thermiques, électriques et optiques de la matière qui sera repris en 1933 par Arnold Sommerfeld et Hans Bethe et deviendra alors le modèle Drude-Sommerfeld. Il enseigne à l'Université de Giessen de 1901 à 1905 et est promu directeur du département de physique de l'Université de Berlin. En 1906 il devient membre de l'académie de Berlin.

\phantomsection
\addcontentsline{toc}{section}{E}
\label{sec:E}

\parpic[l][t]{%
  \begin{minipage}{40mm}
    \fbox{\includegraphics[width=110px,height=140px]{img/medaillons/einstein.eps}}
  \end{minipage}
}
\textbf{Einstein, Albert} (1879-1955) Né à Ulm et mort à Princeton c'était un physicien théoricien qui fut successivement allemand, puis apatride (1896), suisse (1901), et enfin sous la double nationalité helvético-américaine (1940). Il publie sa théorie de la Relativité Restreinte en 1905, et une théorie de la gravitation dite Relativité Générale en 1915. Il contribue largement au développement de la mécanique quantique et de la cosmologie, et reçoit le prix Nobel de physique de 1921 pour son explication de l'effet photoélectrique. Son travail est notamment connu pour l'équation d'équivalence qui établit une équivalence entre la matière et l'énergie d'un système. Il est aussi connu pour son hypothèse audacieuse sur la nature corpusculaire de la lumière. Mais il a également contribué au développement de nombre d'autres théories (physique quantique y comprise). En 1905, Einstein obtint son doctorat de l'Université de Zürich pour une thèse théorique sur les dimensions des molécules. Il publia également trois articles théoriques d'une importance capitale pour le développement de la physique du 20ème siècle. Dans le premier de ces articles, sur le mouvement brownien, il fit des prédictions importantes sur le mouvement des particules distribuées aléatoirement dans un fluide.  Pendant le reste de sa vie, Einstein consacra énormément de temps à généraliser encore plus sa théorie de la Relativité Générale. Il visait une théorie de champ unifié, qui ne fut pas complètement couronnée de succès, et fit de nombreuses tentatives pour décrire l'interaction électromagnétique et l'interaction gravitationnelle dans un modèle commun.

\parpic[l][t]{%
  \begin{minipage}{40mm}
    \fbox{\includegraphics[width=110px,height=140px]{img/medaillons/erdos.eps}}
  \end{minipage}
}
\textbf{Erdös, Paul} (1913-1996) était le plus prolifique des mathématiciens du 20ème siècle, avec environ 1'500 articles publiés (il faut remonter à Euler pour obtenir un tel volume). Plus que quelqu'un qui bâtissait des théories, il résolvait des problèmes, le plus souvent avec élégance et simplicité. Erdös est né à Budapest. Ses deux parents étaient professeurs de mathématiques dans le secondaire. Alors que Erdös était âgé d'à peine un an, son père fut fait prisonnier par les Russes et déporté en Sibérie. Ces événements ont contribué au développement d'une relation très forte mère/fils, qui influera beaucoup sur le cours de la vie de Paul Erdös. C'est à l'âge de 19 ans, alors qu'il commence ses études à l'université et qu'il se fait connaître des milieux mathématiques. Il publie en effet une nouvelle démonstration du postulat de Bertrand, qui affirme qu'il existe un nombre premier entre $n$ et $2n$, pour tout $n$. Deux ans plus tard, il obtient son doctorat (à 21 ans), puis s'en va faire un post-doc à Manchester. Comme Erdös est d'origine juive, il ne peut retourner en Hongrie à la fin des années 30, et il émigre aux États-Unis. Après quelques visites en Europe aux rescapés de sa famille après l'Holocauste, il a des problèmes aux États-Unis avec le MacCarthysme, et il se voit interdit de séjour sur le territoire américain. Erdös est donc contraint de poser ses valises en Israël. Avec ses 1'500 articles, les contributions de Erdös aux mathématiques sont nombreuses: en théorie des nombres, en combinatoire, en mathématiques discrètes, il fut un maître. Erdös avait une exceptionnelle aptitude à s'entourer des mathématiciens les plus compétents pour résoudre ses conjectures. Il en résulte que Erdös a eu beaucoup de collaborateurs: 500 mathématiciens environ ont écrit un article en commun avec lui. Les mathématiciens se sont amusés à définir un nombre de Erdös: tout mathématicien qui a publié un papier en commun avec Erdös a un nombre de Erdös égal à 1. Toute personne qui a publié un article en commun avec une personne qui a un nombre de Erdös égal à 1 a un nombre de Erdös égal à 2. Et ainsi de suite... Albert Einstein est l'un d'entre eux: son nombre de Erdös est 2.  Pourtant, parmi toutes ces collaborations, une au moins a mal tourné, et c'est d'autant plus regrettable qu'elle concerne le plus grand succès d'Erdös. À la fin du 19ème siècle, Hadamard et de La Vallée Poussin avaient démontré le théorème des nombres premiers, à savoir que le nombre de nombres premiers inférieurs ou égaux à $n$ est équivalent, quand $n$ est grand, à $n/\ln(n)$. Leur démonstration est particulièrement rude! En 1949, Atle Selberg trouve une inégalité qu'il pense pouvoir être une étape importante vers une démonstration élémentaire du théorème des nombres premiers. Elle est présentée à Erdös, qui trouve la clef manquante pour boucler la preuve. Un article coécrit de plus aurait sans doute été la solution la plus appropriée pour mesurer les apports de chacun. Mais, à la suite d'un malentendu lié à l'envoi de cartes postales triomphales d'Erdös, Selberg craint qu'Erdös ne tire la couverture à lui. Il publie seul une preuve complète. Il recevra la médaille Fields en 1950, alors qu'Erdös devra se contenter du prix Wolf en 1984. La vie d'Erdös fut vraiment étrange. Il n'avait pas de maison, pas d'épouse, les contingences matérielles étaient pénibles pour lui. Il voyageait en solitaire, accompagné de deux valises qui portaient toutes ses affaires, allant d'université en université, habitant à l'hôtel ou chez un ami mathématicien... Il est par ailleurs l'auteur de nombreux "erdosismes", comme cette phrase célèbre: "un mathématicien est une machine à transformer le café en théorème". Lui-même dopé à toutes sortes d'amphétamines! Jusqu'à la fin de sa vie, Erdös ne ralentira pas son activité mathématique. Mourir signifiait pour lui arrêter de faire des mathématiques. Il décède à Varsovie, en plein congrès.

\parpic[l][t]{%
  \begin{minipage}{40mm}
    \fbox{\includegraphics[width=110px,height=140px]{img/medaillons/erlang.eps}}
  \end{minipage}
}
\textbf{Erlang, Agner Krarup} (1878-1979) était un mathématicien danois ayant beaucoup travaillé sur la théorie des files d'attente, et la gestion des réseaux téléphoniques. Erlang s'est attelé, sur la base notamment des travaux de Poisson dont la loi des événements rares a trouvé toute sa dimension appliquée aux réseaux de télécommunications, à l'élaboration d'un modèle mathématique pour le dimensionnement des réseaux télécoms sur une approche statistique afin de parvenir à des coûts d'exploitation de nature à permettre un marché de masse.\\\\\\

\parpic[l][t]{%
  \begin{minipage}{40mm}
    \fbox{\includegraphics[width=110px,height=140px]{img/medaillons/euclide.eps}}
  \end{minipage}
}
\textbf{Euclide} (3ème siècle av. J.-C.) On ne sait que très peu de choses sur la vie d'Euclide. Il semble qu'il ait enseigné la mathématique à Alexandrie à la demande de Ptolémée Ier. Il apparaîtrait donc comme le fondateur de la célèbre École d'Alexandrie qui influença les travaux d'Archimède. En revanche, les théories d'Euclide sont connues et constituent une référence dans l'histoire des mathématiques. L'oeuvre maîtresse d'Euclide est incontestablement les \textit{Éléments}. Cet ouvrage représente une synthèse remarquable de résultats mathématiques et a marqué de son empreinte la discipline tout entière. Il est composé de treize livres. Les quatre premiers traitent de géométrie dans le plan avec les définitions du point, de la droite et de la surface. Ils exposent également le calcul d'aires de différents polygones. Le livre V contient les premières notions d'analyse. Le sixième aborde la similitude des figures et donne la résolution des équations du second degré à l'aide de constructions géométriques. Les livres VII, VIII, et IX portent sur l'arithmétique. Le X étudie les nombres irrationnels et enfin les trois derniers (XI, XII, XIII) abordent la géométrie dans l'espace. Euclide a, en outre, rédigé des ouvrages sur l'analyse géométrique, l'optique et l'astronomie. Représentation parfaite de l'exposé scientifique, les \textit{Éléments} sont composés de différentes propositions classées en deux groupes: les hypothèses et les axiomes. Parmi les 5 axiomes, on trouve le célèbre postulat d'Euclide: "par tout point du plan passe une et une seule droite parallèle à une autre droite." Cet axiome constitue le fondement de la géométrie euclidienne, en opposition aux géométries non-euclidiennes apparues quelque 2000 ans plus tard.

\parpic[l][t]{%
  \begin{minipage}{40mm}
    \fbox{\includegraphics[width=110px,height=140px]{img/medaillons/euler.eps}}
  \end{minipage}
}
\textbf{Euler, Leonhard} (1707-1783) était un mathématicien suisse, physicien, ingénieur et philosophe, et l'un des fondateurs des méthodes de calcul différentiel et intégral. Leonhard Euler naquit à Bâle, de Paul Euler, pasteur des Églises réformées et de Marguerite Brucker, fille de pasteur. Il eut deux jeunes sœurs du nom d'Anna Maria et de Maria Magdalena. Peu de temps après la naissance de Leonhard, la famille Euler déménagea de Bâle pour rejoindre la ville de Riehen, où Euler passa la plupart de son enfance. Paul Euler était un ami de la famille Bernoulli. Jean Bernoulli, alors considéré comme le principal mathématicien européen, pourrait être celui ayant eu la plus grande influence sur le jeune Leonhard. L'éducation officielle d'Euler commença tôt à Bâle, où il fut envoyé vivre avec sa grand-mère maternelle. À l'âge de 13 ans, il s'inscrivit à l'Université de Bâle, et en 1723, obtint son Master of Philosophy grâce à une dissertation qui comparait la philosophie de Descartes à celle de Newton. À cette époque, il recevait tous les samedis après-midi des leçons de Jean Bernoulli, qui découvrit rapidement chez son nouvel élève un incroyable talent pour les mathématiques. Euler commença alors à étudier la théologie, le grec et l'hébreu à la demande de son père, afin de devenir un pasteur, mais Jean Bernoulli convainquit Paul Euler que Leonhard était destiné à devenir un grand mathématicien. Euler fut le premier à traiter de manière analytique et complète l'algèbre, la théorie des équations, la trigonométrie et la géométrie analytique. Dans ce travail, il traita le sujet du développement des séries de fonctions et formula la règle selon laquelle seules les séries infinies convergentes pouvaient être correctement évaluées. Il discuta aussi des surfaces à trois dimensions et prouva que les sections coniques sont représentées par l'équation générale du second degré à deux dimensions. D'autres travaux traitent du calcul, dont le calcul des variations, la théorie des nombres, les nombres imaginaires et transcendants, l'algèbre déterminée et indéterminée, la théorie des graphes. Euler apporta ses contributions dans les domaines de l'astronomie, de la mécanique analytique (calcul variationnel), l'hydrodynamique, l'optique et l'acoustique. Euler est considéré comme un éminent mathématicien du 18ème siècle et l'un des plus grands et des plus prolifiques de tous les temps et a introduit (ou a contribué à introduire ou à rendre d'usage) une grande partie des notations encore utilisées en ce début de 21ème siècle (symboles pour la somme, fonction, logarithme, exponentiel, etc.).

\phantomsection
\addcontentsline{toc}{section}{F}
\label{sec:F}

\parpic[l][t]{%
  \begin{minipage}{40mm}
    \fbox{\includegraphics[width=110px,height=140px]{img/medaillons/faraday.eps}}
  \end{minipage}
}
\textbf{Faraday, Michael} (1791-1867) était un scientifique anglais qui a contribué aux domaines de l'électromagnétisme et l'électrochimie. Le jeune Faraday, qui était le troisième des 4 enfants, n'ayant que l'éducation scolaire la plus élémentaire, a dû se former en autodidacte. À 14 ans, il devient apprenti chez un relieur local. Au cours de ses 7 années d'apprentissage il a pu lire beaucoup de livres. A cette époque, il a également développé un intérêt pour la science, en particulier pour l'électricité. Faraday a été particulièrement inspiré par les livre \textit{Conversations on Chemistry} de Jane Marcet. Ses principales découvertes comprennent l'induction électromagnétique, le diamagnétisme et l'électrolyse. C'est par ses recherches sur le champ magnétique autour d'un conducteur parcouru par un courant continu que Faraday a établi la base pour le concept de champ électromagnétique en physique. Faraday a également établi que le magnétisme pourrait affecter rayons de lumière et qu'il y avait une relation sous-jacente entre les deux phénomènes. Il a découvert le principe même de l'induction électromagnétique, en même temps que Joseph Henry, diamagnétisme, et les lois de l'électrolyse. Ses inventions de dispositifs électromagnétiques rotatifs ont formé la base de la technologie des moteurs électriques, et c'est en grande partie grâce à ses efforts que l'électricité est devenue pratique pour une utilisation dans la technologie.

\parpic[l][t]{%
  \begin{minipage}{40mm}
    \fbox{\includegraphics[width=110px,height=140px]{img/medaillons/feigenbaum.eps}}
  \end{minipage}
}
\textbf{Feigenbaum, Mitchell} (1994-) est né à New York, d'immigrants polonais et ukrainiens. Il a fréquenté l'école secondaire Samuel J. Tilden, à Brooklyn, New York, et le City College de New York. En 1964, il a commencé ses études supérieures au Massachusetts Institute of Technology (M.I.T.). En s'inscrivant à des études supérieures en génie électrique, il a changé son domaine à la physique. Il a terminé son doctorat en 1970 pour une thèse sur les relations de dispersion. Après des courtes positions à l'Université de Cornell et au Virginia Polytechnic Institute et à l'université d'État, il s'est vu offrir un poste à plus long terme au Laboratoire National de Los Alamos au Nouveau-Mexique pour étudier la turbulence dans les fluides. Bien que ce groupe de chercheurs ait finalement été incapable de démêler la théorie actuellement intraitable des fluides turbulents, ses recherches l'ont amené à étudier des mappages chaotiques. Certains mappages mathématiques impliquant un seul paramètre linéaire présentent le comportement apparemment aléatoire appelé "chaos" lorsque le paramètre se situe dans certaines plages. Lorsque le paramètre est augmenté vers cette région, le mappage subit des bifurcations à des valeurs précises du paramètre. Au début, il y a un point stable, puis bifurquant vers une oscillation entre deux valeurs, puis bifurquant à nouveau pour osciller entre quatre valeurs et ainsi de suite. En 1975, le Dr Feigenbaum, utilisant la petite calculatrice HP-65 qui lui avait été délivrée, a découvert que le rapport de la différence entre les valeurs auxquelles se produisent de telles bifurcations successives de doublement de période tend vers une constante d'environ $4.66692$. Il a été capable de fournir une preuve mathématique de ce fait, et il a ensuite montré que le même comportement, avec la même constante mathématique, se produirait dans une large classe de fonctions mathématiques, avant le début du chaos. Pour la première fois, ce résultat universel a permis aux mathématiciens de faire leurs premiers pas pour démêler le comportement aléatoire apparemment insoluble des systèmes chaotiques. Ce "rapport de convergence" est maintenant connu comme la première "constante de Feigenbaum".

\parpic[l][t]{%
  \begin{minipage}{40mm}
    \fbox{\includegraphics[width=110px,height=140px]{img/medaillons/fermat.eps}}
  \end{minipage}
}
\textbf{Fermat, Pierre de} (1601-1665) était un mathématicien français, surnommé le "prince des mathématiques", auteur d'un célèbre théorème sans démonstration en arithmétique (grand théorème de Fermat). Il fut à l'origine du "principe de Fermat" (optique) et avec son ami Blaise Pascal de celui du calcul des probabilités. Il créa également la théorie des nombres et fit dans ce domaine différentes découvertes. Ainsi, certains le considèrent comme le père de la théorie moderne. Il devança le calcul différentiel par ses travaux sur le calcul infinitésimal. Il laissa à la postérité le soin de démontrer un théorème (le fameux "grand théorème de Fermat" déjà mentionné précédemment) sur lequel les mathématiciens se sont acharnés pendant plus de trois siècles. Ce n'est qu'en 1993 que le chercheur britannique Andrew Wiles en proposa une démonstration.

\parpic[l][t]{%
  \begin{minipage}{40mm}
    \fbox{\includegraphics[width=110px,height=140px]{img/medaillons/fermi.eps}}
  \end{minipage}
}
\textbf{Fermi, Enrico} (1901-1954) était un physicien italien connu pour la réalisation de la première réaction nucléaire contrôlée. Très jeune, Enrico Fermi fait preuve d'une mémoire exceptionnelle et d'une grande intelligence, qui lui permettent d'exceller dans les études. Enrico, profondément marqué par le décès d'un de ses très jeune frère, se jette alors dans l'étude de la physique pour surmonter sa douleur. Bon élève, il se passionne très vite pour la physique et les mathématiques et commence à étudier divers ouvrages qu'il achète et qui traitent de mécanique, d'optique, d'astronomie et d'acoustique. Un ami de son père, l'ingénieur Adolfo Amidei, qui prend conscience des qualités hors du commun du jeune Fermi, lui prête divers ouvrages traitant de mathématiques. Ainsi, à 17 ans, Enrico Fermi maîtrise la géométrie analytique, la géométrie projective, le calcul infinitésimal, le calcul intégral et la mécanique rationnelle. À partir de 1918, Fermi étudie à l'Université de Pise au sein de l'École normale supérieure de Pise. Comme à son habitude, il étudie seul divers problèmes de physique mathématique et consulte des ouvrages de Poincaré, de Poisson ou d'Appell. À partir de 1919, il s'intéresse aux nouvelles théories comme la relativité ou la physique atomique, ainsi il acquiert une grande connaissance de théories telles que la relativité restreinte, la théorie du corps noir ou encore le modèle de l'hydrogène de Bohr. Ainsi Enrico Fermi, le seul à l'université au fait de ces théories, en arrive, sur l'insistance de ses professeurs, à donner des conférences où il expose aux professeurs et aux assistants les dernières découvertes de physique atomique. En 1922, après 4 ans passés à l'université, Enrico Fermi publie son premier article qui traite de la Relativité Générale. Dans une communauté scientifique italienne hostile aux travaux d'Einstein, il est l'un des rares avec Levi-Civita à défendre la théorie de la relativité. En 1922, Fermi obtient son diplôme de fin d'études après avoir présenté un mémoire sur la diffraction des rayons X. Il fréquente ensuite divers physiciens de haut rang dans l'Italie de l'époque, avant de devenir, pendant 2 ans, conférencier à l'Université de Florence. En 1926, il devient professeur de physique théorique à l'Université La Sapienza de Rome. C'est durant cette période qu'il développe la théorie statistique quantique que l'on appellera plus tard la "statistique de Fermi-Dirac". À partir de 1932, il se tourne plus précisément vers la physique nucléaire, et c'est cette même année qu'il rédige un article sur la radioactivité bêta. En 1934, il développe sa théorie sur l'émission de rayonnement bêta en y incluant le neutron postulé en 1930 par Wolfgang Pauli, qu'il rebaptise neutrino (le nom neutron étant déjà utilisé pour une autre particule), et s'oriente vers la création d'isotopes radioactifs artificiels par bombardement de neutrons lents (raison pour laquelle il reçut le prix Nobel en 1938).

\parpic[l][t]{%
  \begin{minipage}{40mm}
    \fbox{\includegraphics[width=110px,height=140px]{img/medaillons/feynman.eps}}
  \end{minipage}
}
\textbf{Feynman, Richard Phillips} (1918-1988) était né à Far Rockaway dans le Queens, quartier de New York (États-Unis) de parents d'origine polonaise et russe. Son père, qui l'encourageait à poser des questions et à remettre en cause les choses communément admises, l'a durablement influencé. De sa mère, il tient un solide sens de l'humour qui ne l'a jamais quitté. Feynman est l'un des physiciens les plus influents de la seconde moitié du 20ème siècle, en raison notamment de ses travaux sur l'électrodynamique quantique relativiste, les quarks et l'hélium superfluide. Durant sa dernière année à l'école secondaire de Far Rockaway, Feynman remporta le championnat de mathématiques de l'Université de New York. Il reçut donc une bourse pour étudier au Massachusetts Institute of Technology (M.I.T.) où il reçut son baccalauréat en 1939 après s'être orienté d'abord en électronique, puis en mathématiques, et enfin avoir assisté à tous les cours de physique offerts y compris pendant sa seconde année un cours de physique théorique réservé aux étudiants de maîtrise. Feynman obtient un score remarquable aux examens d'entrée de l'Université de Princeton en mathématiques et en physique, mais il eut une note très faible dans la partie littéraire de l'examen. Durant ses études à l'Institute for Advanced Study de Princeton (IAS) (créé depuis peu et dirigé par Albert Einstein), Feynman travailla sous la direction de John Wheeler sur le principe de moindre action appliqué à la mécanique quantique. Il établit ici les bases des diagrammes de Feynman et de l'approche de la mécanique quantique par les intégrales de chemin. Il obtint son doctorat en 1942. Il reformula entièrement la mécanique quantique à l'aide de son intégrale de chemin qui généralise le principe de moindre action de la mécanique classique et inventa les diagrammes qui portent son nom et qui sont désormais largement utilisés en théorie quantique des champs (dont l'électrodynamique quantique fait partie). Musicien, pédagogue remarquable, rédacteur de nombreux ouvrages de vulgarisation, il a aussi été impliqué dans le développement de la bombe atomique américaine. Après la seconde guerre mondiale, il enseigna à l'Université de Cornell puis au Caltech où il effectua des travaux fondamentaux notamment dans la théorie de la superfluidité et des quarks. Sin-Itiro Tomonaga, Julian Schwinger et lui sont colauréats du prix Nobel de physique de 1965 pour leurs travaux en électrodynamique quantique.

\parpic[l][t]{%
  \begin{minipage}{40mm}
    \fbox{\includegraphics[width=110px,height=140px]{img/medaillons/fisher.eps}}
  \end{minipage}
}
\textbf{Fisher, Ronald Aymler} (1890-1962) Né à Londres était un biologiste et statisticien britannique, qui a énormément contribué à fonder les statistiques modernes. Ses travaux sur les statistiques lui valurent la médaille Darwin en 1948, la médaille Copley en 1955 et la médaille d'argent Darwin-Wallace en 1958. Dans le domaine des statistiques, il a introduit de nombreux concepts clés tels que le maximum de vraisemblance, l'information de Fisher et l'analyse de la variance (ANOVA). Il est considéré comme un grand précurseur de Shannon. Il est également un des fondateurs de la génétique moderne et un grand continuateur de Darwin, en particulier grâce à son utilisation des méthodes statistiques, incontournables dans la génétique des populations. Il a ainsi contribué à la formalisation mathématique du principe de sélection naturelle. Il est d'abord attiré par la physique et obtient en 1912 une licence d'astronomie à l'Université de Cambridge. De 1915 à 1919, il enseigne les mathématiques à Londres dans des écoles privées. En 1919, il est engagé à la station expérimentale de Rothamsted pour analyser l'effet des précipitations sur le rendement du blé où il travaille jusqu'en 1933. Dans son article de 1922 \textit{On the mathematical foundations of theoritical statistics}, il définit une quinzaine de notions fondamentales en statistiques dont la notion de convergence, d'efficacité, de vraisemblance et de statistique suffisante. Il propose l'estimateur du maximum de vraisemblance en 1922 après avoir présenté une première version en 1912. Il introduit aussi en 1924 l'analyse de la variance. En 1925 il publie des innovations en séries temporelles et en analyse des corrélations multiples.

\parpic[l][t]{%
  \begin{minipage}{40mm}
    \fbox{\includegraphics[width=110px,height=140px]{img/medaillons/foucault.eps}}
  \end{minipage}
}
\textbf{Foucault, Leon} (1819-1868) était un physicien français célèbre pour sa démonstration du mouvement de la Terre par la rotation du plan d'oscillation du pendule. Né à Paris, il travailla avec le physicien français Armand Fizeau sur la détermination de la vitesse de la lumière. Foucault prouva, de façon indépendante, que la vitesse de la lumière dans l'air était plus élevée que dans l'eau. En 1851, il fit une démonstration spectaculaire de la rotation de la Terre en suspendant un pendule à un long câble attaché à la coupole du Panthéon à Paris. Le mouvement du pendule mis en évidence la rotation de la Terre sur son axe. En 1855 il découvre que la force nécessaire à la rotation d'un disque de cuivre augmente quand il doit tourner avec sa jante entre les pôles d'un aimant, le disque chauffant dans le même temps du fait des "courants de Foucault" induits dans le métal. Il conçut également une méthode de mesure de la courbure des miroirs de télescopes. Il développa d'autres instruments dont un prisme polarisateur et une forme de gyroscope qui est à la base du gyrocompas moderne.

\parpic[l][t]{%
  \begin{minipage}{40mm}
    \fbox{\includegraphics[width=110px,height=140px]{img/medaillons/fourier.eps}}
  \end{minipage}
}
\textbf{Fourier, Joseph} (1768-1830) était un physicien et mathématicien français connu pour la découverte des séries trigonométriques et des transformées qui portent son nom. Fourier est orphelin de père et de mère à 10 ans. L'organiste d'Auxerre, Joseph Pallais, le fait entrer dans le pensionnat qu'il dirige. Recommandé par l'évêque d'Auxerre, il fait ses études à l'école militaire d'Auxerre tenue alors par les Bénédictins de la Congrégation de Saint-Maur. Destiné à l'état monastique, il préfère s'adonner aux sciences pour lesquelles il remporte la plupart des premiers prix. Élève brillant, il y est promu professeur dès l'âge de 16 ans et peut dès lors commencer ses recherches personnelles. Il intègre l'école normale supérieure à 26 ans, où il a entre autres comme professeurs Joseph-Louis Lagrange, Gaspard Monge et Pierre-Simon de Laplace, auquel il succède à la chaire à Polytechnique en 1797. Fourier a contribué à la résolution numérique des équations et à la diffusion de la chaleur dont une des lois porte son nom. Ses travaux ont une implication directe dans la convergence des séries et leur somme infinie. Il participa, avec Monge, à la campagne d'Égypte en tant qu'observateur scientifique. Anobli sous Napoléon, il fut professeur à l'école polytechnique, secrétaire de l'institut d'Égypte et préfet de l'Isère. Il fut aussi élu à l'académie des sciences et à l'académie française. On le considère comme l'un des fondateurs, avec le français Poisson et le suisse Daniel Bernoulli, de ce que l'on appelle aujourd'hui la "physique-mathématique".

\parpic[l][t]{%
  \begin{minipage}{40mm}
    \fbox{\includegraphics[width=110px,height=140px]{img/medaillons/fraunhofer.eps}}
  \end{minipage}
}
\textbf{Fraunhofer, Joseph von} (1787-1826) était un opticien et physicien allemand, né à Straubing. Fraunhofer apporta de nombreuses améliorations à la fabrication du verre optique, au meulage et au polissage des lentilles et à la construction des télescopes et d'autres instruments d'optique. Joseph Fraunhofer était le onzième enfant d'un souffleur de verre. Il avait 11 ans à la mort de ses parents, aussi son tuteur l'envoya-t-il à Münich en apprentissage pour 6 ans afin qu'il apprenne la miroiterie. C'est là, qu'en 1801, il faillit trouver la mort dans l'effondrement de l'atelier. À la fin de son apprentissage en 1806, il eut la possibilité de poursuivre une formation d'opticien dans l'institut de mécanique Reichenbach. Les ateliers furent transférés en 1807 à Benediktbeuern, et Fraunhofer y fut nommé contremaître. Là, il mit au point de nouvelles machines à polir les miroirs et de nouveaux types de verres optiques (le verre flint achromatique), qui apportèrent une amélioration décisive à la qualité des lentilles. Fraunhofer inventa aussi de nombreux instruments scientifiques. Son nom est associé à des lignes fixes et noires dans le spectre solaire, appelées les "lignes Fraunhofer", qu'il fut le premier à décrire en détail. Ses recherches dans le domaine de la réfraction et de la dispersion de la lumière aboutirent à l'invention du spectroscope et au développement de la spectroscopie.

\parpic[l][t]{%
  \begin{minipage}{40mm}
    \fbox{\includegraphics[width=110px,height=140px]{img/medaillons/fresnel.eps}}
  \end{minipage}
}
\textbf{Fresnel, Augustin Jean} (1788-1827) était un physicien français, fondateur de l'optique moderne. Il proposa une explication de tous les phénomènes optiques dans le cadre de la théorie ondulatoire de la lumière. Il commença par réaliser de nombreuses expériences sur les interférences lumineuses, pour lesquelles il forgea la notion de longueur d'onde, et calcula les intégrales dites "Intégrales de Fresnel". Il fut le premier à prouver que deux faisceaux de lumière polarisés dans des plans différents n'ont aucun effet d'interférence. Il déduisit très justement de cette expérience que le mouvement ondulatoire de la lumière polarisée est transversal et non longitudinal (comme celui du son) ainsi qu'on le croyait avant lui. En outre, il fut le premier à produire une lumière polarisée circulaire. Pour expliquer la propagation des ondes lumineuses, Fresnel eut recours à la notion d'éther, malheureusement contradictoire avec d'autres expériences. Cette théorie sera abandonnée avec la relativité, mais les relations dites "relations de Fresnel" sur la réfraction sont toujours utilisées. Dans le domaine de l'optique appliquée, Fresnel conçut la lentille à échelons utilisée pour accroître le pouvoir éclairant des phares. De son vivant, les travaux scientifiques de Fresnel n'étaient connus que d'un petit groupe de scientifiques et certains de ses articles ne furent publiés qu'après sa mort.

\phantomsection
\addcontentsline{toc}{section}{G}
\label{sec:G}

\parpic[l][t]{%
  \begin{minipage}{40mm}
    \fbox{\includegraphics[width=110px,height=140px]{img/medaillons/galilee.eps}}
  \end{minipage}
}
\textbf{Galileo, Galilei} (1564-1642) était un physicien et astronome italien né à Pise à l'origine de la révolution scientifique du 17ème siècle. Ses théories ainsi que celles de l'astronome allemand Johannes Kepler servirent de fondement aux travaux du physicien britannique sir Isaac Newton sur la loi de l'attraction universelle. Sa principale contribution à l'astronomie fut l'amélioration considérable (quand la technique marchait...) de la lunette astronomique (ce qui lui a permis de procéder à des observations qui ont bouleversé la discipline) et la découverte des taches solaires, des montagnes et des vallées lunaires, des quatre plus grands satellites de Jupiter et des phases de Vénus. En physique, il découvrit la loi de la chute des corps et les mouvements paraboliques des projectiles. Ses études sur les oscillations du pendule pesant l'ont amené à inventer le pulsomètre. Cet appareil permettait d'aider à la mesure du pouls et fournissait un étalon de temps, qui n'existait pas à l'époque. Il débute aussi ses études sur la chute des corps. Dans l'histoire de la culture, Galilée est le symbole de la bataille livrée contre les autorités religieuses pour la liberté de la recherche (il avait cependant très bonne réputation et de très bonnes relations auprès des instances religieuses ce qui a aidé...). Dans le domaine des mathématiques et de la physique, il a contribué à faire avancer les connaissances relatives à la cinématique et la dynamique, jetant ainsi les fondements des sciences mécaniques. Il est de ce fait considéré comme le fondateur de la physique moderne.

\parpic[l][t]{%
  \begin{minipage}{40mm}
    \fbox{\includegraphics[width=110px,height=140px]{img/medaillons/galois.eps}}
  \end{minipage}
}
\textbf{Galois, Evariste} (1811-1832) était un mathématicien français, qui a donné son nom à une branche des mathématiques: la "théorie de Galois". Sa vie est tellement mythique qu'il est parfois difficile de démêler le mythe et la réalité. Dès 1827-1828, la fureur des mathématiques domine. Galois lit Legendre, Lagrange, Euler, Gauss, Jacobi. Le professeur, Louis-Paul-Émile Richard, admire le génie mathématique de son élève et garde les copies qu'il confiera à un autre de ses élèves: Charles Hermite. C'est l'époque où il publie son premier article dans les\textit{Annales mathématiques} de Joseph Gergonne (il démontre un théorème sur les fractions continues périodiques). Il rédige aussi un premier mémoire sur la théorie des équations, envoyé à l'académie des Sciences, perdu par Cauchy. Il échoue au concours d'entrée à Polytechnique. On raconte qu'il a jeté le chiffon à effacer la craie à la tête de son examinateur devant la stupidité des questions posées. Sur les conseils de son professeur, Galois entre à l'école préparatoire (future école Normale). Il rédige le résultat de ses recherches dans un mémoire - \textit{Conditions pour qu'une équation soit résoluble par radicaux} - afin de concourir au grand prix de mathématiques de l'académie des Sciences. Fourier emporte le manuscrit chez lui et meurt peu après: le manuscrit est perdu, et le grand prix est décerné à Abel (mort l'année précédente), et à Jacobi. Pour des raisons politiques, Galois se retrouve en prison, où il y continue ses travaux. Libéré en 1832, il s'éprend en 1832 d'une femme, avec qui il rompt la même année. On ne sait trop pourquoi, mais un duel semble en résulter quelques jours plus tard. La nuit précédente, le 29 mai, Galois rassemble ses dernières découvertes dans une splendide lettre adressée à son ami Auguste Chevalier. De cette lettre naquit la légende selon laquelle Galois fit ses découvertes majeures en une seule nuit, pris par la fièvre de la mort. La matinée du 30 mai, Galois, abandonné, grièvement blessé, est relevé par un paysan et conduit à l'hôpital Cochin. Il meurt le 31 mai 1832 dans les bras de son jeune frère est est enterré dans la fosse commune du cimetière de Montparnasse. Les travaux de Galois sont redécouverts une dizaine d'années plus tard par Liouville, qui en 1843 annonce à l'académie des Sciences qu'il vient de trouver dans les papiers de Galois une solution aussi exacte que profonde au problème de la résolubilité par radicaux. Ce n'est qu'en 1846 qu'il publie les textes sans y joindre de commentaires. À partir de 1850, les écrits de Galois sont enfin accessibles par les meilleurs mathématiciens.

\parpic[l][t]{%
  \begin{minipage}{40mm}
    \fbox{\includegraphics[width=110px,height=140px]{img/medaillons/gamow.eps}}
  \end{minipage}
}
\textbf{Gamow, George} (1904-1968) était un physicien théorique, astronome, cosmologiste et auteur/vulgarisateur scientifique américano-russe né à Odessa en Ukraine. Gamow vient en 1928 à Göttingen, où il utilise la physique quantique pour faire une théorie quantique de la radioactivité alpha. Deux mois plus tard, il rejoint Niels Bohr à Copenhague. Il émet l'idée d'un noyau atomique se comportant comme un fluide nucléaire, modèle repris presque une décennie plus tard par Bohr. En 1929, il obtient une nouvelle bourse et il rejoint Ernest Rutherford à l'Université de Cambridge. Il développe l'idée de l'effet tunnel afin de faire interagir des protons pour obtenir des noyaux de numéro atomique plus élevé. Il y rencontre John Cockcroft, qui construit peu après le premier accélérateur de particules, parvenant ainsi à valider le modèle de Gamow en réussissant une transmutation du lithium. Professeur à Washington en 1934, Gamow collabore avec Edward Teller pour formuler la théorie de l'émission bêta (1936). S'intéressant ensuite à l'astrophysique, Gamow et Teller donnent un modèle de la structure interne des étoiles géantes rouges (1942). En 1954, c'est vers la biochimie qu'il se tourne, proposant le concept de code génétique déterminé par l'ordre des composants de l'ADN. En 1956, il est nommé professeur de physique à Boulder (Colorado).

\parpic[l][t]{%
  \begin{minipage}{40mm}
    \fbox{\includegraphics[width=110px,height=140px]{img/medaillons/gauss.eps}}
  \end{minipage}
}
\textbf{Gauss, Carl Friedrich} (1777-1855) était un mathématicien allemand, qui a apporté des contributions essentielles à la plupart des branches des sciences exactes et appliquées. À l'âge de 17 ans, il essaya de trouver une solution au problème classique de construction d'un polygone à sept côtés, à la règle et au compas. Il réussit à prouver l'impossibilité de cette construction et poursuivit sa démarche en donnant des méthodes de construction de polygones à 17, 257, et 65'537 côtés. Plus généralement, il prouva que la construction, à la règle et au compas, d'un polygone régulier à nombre impair de côtés n'était possible que si le nombre de côtés est un des nombres premiers 3, 5, 17, 257, et 65'537, ou un produit de ces nombres. Pour sa thèse de doctorat, il démontra que toute équation algébrique a au moins une racine. Ce théorème, dont la démonstration avait résisté aux mathématiciens les plus célèbres, est encore appelé le "théorème fondamental de l'algèbre" ou "théorème de d'Alembert-Gauss". Gauss tourna ensuite son attention vers le domaine de l'astronomie pour laquel il élabora également une nouvelle méthode de calcul des orbites des corps célestes, en développant une théorie des erreurs d'observation connue sous le nom de "méthode des moindres carrés". En probabilités, son nom est attaché à la loi Normale (dite aussi "loi de Laplace-Gauss"), dont la répartition est décrite par la fameuse courbe en cloche ou courbe de Gauss. On lui doit aussi des travaux en géodésie. Avec le physicien allemand Wilhelm Eduard Weber, Gauss fit, à partir de 1831, des recherches approfondies dans le domaine du magnétisme et de l'électricité. Il fit aussi des recherches en optique, en particulier sur les systèmes de lentilles. Pour revenir aux mathématiques, il fut le premier, en étudiant la série hypergéométrique, à donner des conditions rigoureuses de convergence d'une série. Il étudia des généralisations fructueuses de la loi de réciprocité quadratique et dégagea leurs liens avec la théorie des fonctions elliptiques. Son mémoire de 1828 sur la théorie intrinsèque des surfaces fut le point de départ d'une théorie générale des espaces courbes (travaux de Riemann et de ses successeurs). Signalons aussi l'étude arithmétique des entiers de Gauss (de la forme $a+\mathrm{i}b$) qui repose sur une présentation géométrique des nombres complexes comme points du plan.

\parpic[l][t]{%
  \begin{minipage}{40mm}
    \fbox{\includegraphics[width=110px,height=140px]{img/medaillons/gibbs.eps}}
  \end{minipage}
}
\textbf{Gibbs, Josiah Willard} (1839-1903) était un physicien et mathématicien né et décédé à New Haven dans le Connecticut (après y avoir passé presque toute son existence en célibataire). Issu d'une famille de lettrés, il poursuit des études de latin et de physique, puis il entreprend une carrière de professeur de physique-mathématique au Yale College. Il séjourne successivement à Paris, à Berlin où il suit les leçons de Heinrich Gustav Magnus et à Heidelberg où il rencontre Gustav Kirchhoff et Herman Ludwig Helmholtz. Il laisse le souvenir d'un savant d'une modestie proverbiale et d'une extraordinaire puissance d'investigation scientifique. Son oeuvre remarquablement compacte fut d'abord peu connue. Aujourd'hui, elle est considérée comme un monument au sein des contributions scientifiques du 19ème siècle. Les deux principales publications datent de 1877 et de 1902. La première s'intitule \textit{On the Equilibrium of Heterogeneous Substances} et est comparée, en importance, à la chimie pondérale créée par Antoine Laurent Lavoisier. La seconde, jugée plus originale encore, est intitulée \textit{Elementary Principles in Statistical Mechanics}, et est comparée, pour son génie, à la mécanique analytique de Joseph Louis Lagrange. Bien que les exposés de Gibbs se distinguent par une exceptionnelle clarté, et la façon dont l'idée essentielle y est toujours soigneusement dégagée, le premier des deux mémoires n'a guère retenu tout d'abord l'attention des chimistes de son époque, peu accoutumés au langage rigoureux des sciences exactes. La richesse des méthodes thermodynamiques sur lesquelles il s'appuie en a fait cependant une base unifiée de la théorie physico-chimique des états d'équilibre et de leur stabilité. La plupart des lois qui se rapportent à cette discipline, et qui portèrent d'abord d'autres noms, furent redécouvertes ultérieurement au sein de ce premier mémoire. Il en est ainsi, par exemple, de la loi des phases donnant la variance des systèmes en équilibre, longtemps attribuée à Bakkuis Roozeboom (également des lois dites "loi de Van't Hoff" et aussi "loi de Le Chatelier"), relatives aux déplacements d'équilibre à température constante et à pression constante. Il en est encore de même, des critères de stabilité de l'équilibre, dont le théorème de modération dit "théorème de Braun et Le Chatelier". En bref, la plupart des propriétés qui relèvent à présent de la thermodynamique chimique des états d'équilibre, telles que la pression osmotique, l'influence de la tension superficielle, celle des déformations élastiques, la loi relative à l'entropie des mélanges gazeux et le paradoxe de Gibbs associé, ont ce même mémoire pour origine. Gibbs conduit à développer, dans deux communications antérieures à la précédente, un exposé complet des diagrammes et des surfaces thermodynamiques qui contribua largement à la diffusion de leur emploi auprès des praticiens. La théorie de Gibbs utilise pour la première fois la notion d'ensemble ainsi que la distinction entre un ensemble canonique et un ensemble microcanonique de même qu'entre un grand et un petit ensemble. Elle introduit aussi le concept d'espace des phases, caractérisé par les coordonnées et les quantités de mouvement de chaque élément. Elle établit, à partir de l'équation de Liouville, la loi de conservation de l'élément d'extension en phase, ainsi que celle de densité et de probabilité de l'état statistique. Il réalise finalement un accord formel mais remarquable avec les lois macroscopiques de la thermodynamique, régissant le comportement des milieux matériels en équilibre. Les développements actuels de la mécanique statistique constituent encore, sur plus d'un point, des prolongements de la méthode de Gibbs. Il définit pour les réactions chimiques deux quantités très utiles, à savoir l'enthalpie qui représente la chaleur d'une réaction à pression constante, et l'enthalpie libre qui détermine si oui ou non une réaction peut procéder de façon spontanée à température et pression constante. Cette dernière quantité est maintenant nommée "énergie de Gibbs" en son honneur (ou comme anglicisme énergie libre de Gibbs). L'emploi du point pour désigner un produit scalaire, celui de la croix de Saint-André pour un produit vectoriel et l'adoption des opérateurs vectoriels différentiels del ($\nabla()$) et nabla ($\vec{\nabla}\times$) proviendraient de Gibbs.

\parpic[l][t]{
  \begin{minipage}{40mm}
    \fbox{\includegraphics[width=110px,height=140px]{img/medaillons/godel.eps}}
  \end{minipage}
}
\textbf{Gödel, Kurt} (1906-1978) Né à Brünn et décédé à Stanford, était le mathématicien et logicien austro-américain, qui de tout le 20ème siècle, a le plus révolutionné les fondements logiques des mathématiques. Il était un homme tellement obsédé par la logique qu'on raconte que, alors qu'il cherchait à obtenir sa naturalisation américaine, il osa démontrer devant le juge la contradiction de certains articles de la constitution des États-Unis. Sa thèse, et surtout un article publié en 1931 sous le titre \textit{Über formal unentscheidbare Sätze der Principia Mathematica und verwandter Systeme} (sur l'indécidabilité formelle des \textit{Principia Mathematica} et de systèmes équivalents), donneront à Gödel une réputation internationale. Gödel met fin aux espoirs de Hilbert d'axiomatiser totalement la mathématique, et de n'en faire qu'une suite de déductions mécaniques ne laissant aucune place à l'intuition. Ainsi, Gödel montre qu'il existe des propositions vraies sur les nombres entiers, mais que l'on ne sait pas démontrer. Il montre même que, si on ajoute d'autres axiomes, on trouvera toujours des propositions vraies indécidables (qu'on ne sait pas démontrer). Il prouve notamment que l'hypothèse du continu et l'axiome du choix ne sont pas en contradiction avec les autres axiomes de la théorie des ensembles. Puis il s'oriente vers la relativité, étant en relation directe à Princeton avec son ami Einstein. Il est notamment connu des physiciens pour avoir démontré que le voyage vers le passé est possible dans le cadre des équations de la Relativité Générale.

\parpic[l][t]{%
  \begin{minipage}{40mm}
    \fbox{\includegraphics[width=110px,height=140px]{img/medaillons/goeppertmayer.eps}}
  \end{minipage}
}
\textbf{Göpper-Meyer, Maria} (1906-1972) était une physicienne américaine d'origine allemande, prix Nobel en 1963, pour son étude de la structure nucléaire. Elle était mariée à un physicien, le spécialiste de la physique du solide Joseph Mayer (1904-1983). Mais, dans ce couple, chacun travaillait de son côté et dans sa spécialité. Goeppert-Mayer obtint son doctorat à l'Université de Göttingen, en Allemagne. Elle enseigna dans de nombreuses institutions avant de rentrer à l'Université de Californie à San Diego, en 1960. En 1963, elle partagea avec H.D.Jensen et E.Wigner le prix Nobel de physique, et fut citée par le comité Nobel pour son oeuvre indépendante à la fin des années 1940. Elle démontra que le noyau atomique possède un nombre de neutrons et de protons bien définis: elle introduisit un modèle structural du noyau atomique en couches. Ce modèle développé en détail à partir de 1948 supposait que la forte interaction entre le mouvement de rotation intrinsèque (quantifié par le spin) des nucléons et leur mouvement orbital était responsable de la structure des niveaux d'énergie des noyaux. De nombreuses conséquences déduites de cette hypothèse se révélèrent vérifiées par les mesures expérimentales. Quelques années plus tard, James Rainwater, Aage Bohr et Ben R. Mottelson (tous trois prix Nobel de physique 1975) complétaient la théorie en tenant compte du couplage entre les mouvements des nucléons de la couche externe et le mouvement collectif du coeur nucléaire.

\parpic[l][t]{%
  \begin{minipage}{40mm}
    \fbox{\includegraphics[width=110px,height=140px]{img/medaillons/gosset.eps}}
  \end{minipage}
}
\textbf{Gosset, William Sealy} (1876-1937) connu sous le pseudonyme de "Student" c'était un statisticien anglais. Employé de la fameuse brasserie Guinness pour stabiliser le goût de la bière, il a ainsi inventé le "test de Student" utilisé de manière standard dans de très nombreux domaines de l'industrie ou de l'économie. Il a aussi déterminé en 1908 l'origine de la distribution  expérimentale qu'il obtenait dans le cadre de son travail et après avoir suivi un cours de statistique avec Karl Pearson, il obtint son fameux résultat qu'il publia sous le pseudonyme de Student avec la loi qui porte son nom et son test.\\

\parpic[l][t]{%
  \begin{minipage}{40mm}
    \fbox{\includegraphics[width=110px,height=140px]{img/medaillons/gottlob.eps}}
  \end{minipage}
}
\textbf{Gottlob, Frege Friedrich Ludwig} (1848-1925) était un mathématicien et philosophe allemand, initiateur de la logique moderne. Frege est né à Wismar en 1848, et fit ses études aux Universités de Iéna et de Göttingen, où il obtint son doctorat de philosophie en 1873. De 1879 à 1917, il fut professeur à la faculté de philosophie d'Iéna. Ses travaux concernent notamment la logique mathématique et ses applications. Confronté à l'ambiguïté du langage ordinaire et à l'imperfection des systèmes logiques disponibles, il inventa de nombreuses notations symboliques, comme les quantificateurs et les variables, posant alors les bases de la logique mathématique moderne. Il est ainsi le premier à avoir présenté une théorie cohérente du calcul des prédicats et du calcul des propositions. Il fut aussi le premier à faire dériver l'arithmétique de la logique. Il définit ainsi notamment la suite des nombres entiers à partir de l'ensemble vide, en appliquant quelques règles simples.

\parpic[l][t]{%
  \begin{minipage}{40mm}
    \fbox{\includegraphics[width=110px,height=140px]{img/medaillons/grothendieck.eps}}
  \end{minipage}
}
\textbf{Grothendieck, Alexander} (1928-2014) était né à Berlin d'un père anarchiste russe, tué par les nazis, et d'une mère femme de lettres, réfugiée en France. Il passe sa licence à la faculté des sciences de Montpellier, puis passe une année en 1948-1949 à l'École Normale Supérieure à Paris, avant de migrer en 1949 à l'Université de Nancy. Il y devient l'élève, en analyse fonctionnelle, de Schwartz et Dieudonné. Ce dernier le trouve un peu prétentieux, et lui propose de travailler sur des questions que ni Schwartz, ni lui n'ont su résoudre. Voilà ce qu'en dit Schwartz dans son autobiographie: "Dieudonné, avec l'agressivité (toujours passagère), dont il était capable, lui passa un savon mémorable, arguant qu'on ne devait pas travailler de cette manière, en généralisant pour le plaisir de généraliser. [...] L'article s'achevait sur 14 questions, des problèmes que nous n'avions pas su résoudre, Dieudonné et moi. Dieudonné lui [Grothendieck] proposa de réfléchir à certains d'entre eux qu'il choisirait. Nous ne le revîmes plus pendant quelques semaines. Lorsqu'il avait réapparu, il avait trouvé la solution de la moitié d'entre eux!".  Rapidement, Grothendieck rédige sa thèse intitulée \textit{Produits tensoriels topologiques et espaces nucléaires}, et devient le spécialiste mondial de la théorie des espaces vectoriels topologiques. Il devient aussi membre du célèbre groupe Bourbaki auprès de ses aînés.  Au début des années 1960, il obtient une charge au tout récent Institut des Hautes Études Scientifiques (IHES), et son centre d'intérêt s'oriente vers la géométrie algébrique. Il y réalise des travaux gigantesques, qui lui valent la médaille Fields en 1966. Toutefois, Grothendieck refuse de se rendre en URSS pour la recevoir, afin de protester contre la répression de l'insurrection hongroise en 1956. On la lui remet plus tard, mais il l'offre au Viêtnam, afin qu'il utilise son or. Il y enseigne d'ailleurs plusieurs semaines sous les bombardements américains. Vers la fin des années 60, Grothendieck, qui a perdu l'habitude de rédiger (Dieudonné a rédigé des années durant son séminaire), devient de moins en moins clair. Il ne pardonnera jamais aux autres mathématiciens de ne pas le comprendre et de dénaturer ainsi ses idées. Si ses relations avec la communauté mathématique n'avaient jamais été faciles (il travaillait énormément en solitaire, ses journées faisaient 27 ou 28 heures, de sorte que parfois il lui arrivait de se décaler - Il méprisait légèrement Dieudonné, séquelle du premier coup de gueule de ce dernier - ses prises de becs avec Weil causèrent son départ de Bourbaki...), elles sont plus tendues que jamais... Il abandonne peu à peu la mathématique et quitte l'IHES après une dispute interne sur des financements militaires, pour se retirer dans sa maison de l'Hérault, où il se consacre à la méditation et à l'écologie. Il écrit vers 1985 une sorte d'autobiographie, \textit{Récoltes et semailles}, qui ne trouve pas d'éditeur. Ceux qui ont pu la lire sont unanimes pour dire qu'elle contenait de nombreuses attaques contre la communauté des mathématiciens.

\phantomsection
\addcontentsline{toc}{section}{H}
\label{sec:H}

\parpic[l][t]{%
  \begin{minipage}{40mm}
    \fbox{\includegraphics[width=110px,height=140px]{img/medaillons/hall.eps}}
  \end{minipage}
}
\textbf{Hall, Edwin Herbert} (1855-1938) était un physicien né dans le Main et décédé à Cambridge (U.S.A.). Hall a fait ses études de premier cycle au Bowdoin College, obtenant son diplôme en 1875. Il fait ses études supérieures et de recherche, et obtint son doctorat (1880), à l'Université Johns Hopkins, où ses expériences ont été effectuées. L'effet Hall a été découvert par Hall en 1879, alors qu'il travaillait sur sa thèse de doctorat en physique. Hall a été nommé professeur de physique à Harvard en 1895. Il était aussi anecdotiquement connu pour donner des conférences sans chaussures et a écrit de nombreux livres sur la physique.\\

\parpic[l][t]{%
  \begin{minipage}{40mm}
    \fbox{\includegraphics[width=110px,height=140px]{img/medaillons/hamilton.eps}}
  \end{minipage}
}
\textbf{Hamilton, William Rowan} (1805-1865) était un mathématicien, physicien et astronome irlandais (né et mort à Dublin) qui fut l'objet de son vivant des plus grands honneurs, on l'appelait le "Lagrange irlandais", et même le "Newton irlandais", et pourtant son oeuvre était peu connue et rarement étudiée. Il est connu pour sa découverte des quaternions, mais il contribua aussi au développement de l'optique, de la dynamique et de l'algèbre. Ses recherches se révélèrent importantes pour le développement de la mécanique quantique. Les travaux mathématiques de Hamilton incluent l'étude de l'optique géométrique, l'adaptation des méthodes dynamiques aux systèmes optiques, l'application des quaternions et des vecteurs aux problèmes de mécanique et géométriques, les possibilités de résolution des équations polynomiales, notamment l'équation générale du cinquième degré, les opérateurs linéaires, dont il prouve un résultat concernant ces opérateurs dans l'espace des quaternions et qui est un cas spécial du théorème de Cayley-Hamilton. Sa carrière scientifique fut prédestinée par des études à Trinity College, à Dublin, où, à l'âge de 19 ans, il terminait un travail remarquable sur l'optique. À 23 ans, il devint professeur d'astronomie à Dublin et astronome royal à l'observatoire de Dunsink. Il restera toute sa vie fidèle à Dublin et à son observatoire. Hamilton s'efforce de donner aux principes fondamentaux de la mécanique une forme simple permettant d'édifier toute une théorie déductive. Pour cela, il modifie les principes de variations antérieurs, notamment le principe de moindre action, et introduit ce qu'on appelle de nos jours le "principe de Hamilton". Indiquons enfin qu'on lui doit la forme dite "canonique" des équations de la dynamique qui n'apporte rien de nouveau à la physique mais fournit une méthode plus puissante pour résoudre les équations du mouvement. Dans ses travaux des années 1832 à 1835 Hamilton attache une grande importance à l'interprétation géométrique des nombres complexes, et c'est à partir de là qu'il cherche un calcul algébrique qui s'interpréterait dans l'espace à trois dimensions. Il n'arrive à ce but qu'en 1843, en construisant les quaternions. Dans les années qui suivent cette découverte, il se consacre à son développement et à sa diffusion, en lui trouvant des applications à divers domaines des mathématiques et de la physique. Les quaternions de Hamilton constituent un des premiers systèmes de vecteurs et ont, par leurs conséquences théoriques, beaucoup contribué à l'élaboration de l'algèbre et de la physique quantique du 20ème siècle.

\parpic[l][t]{%
  \begin{minipage}{40mm}
    \fbox{\includegraphics[width=110px,height=140px]{img/medaillons/hawking.eps}}
  \end{minipage}
}
\textbf{Hawking, Stephen} (1942-2018) Né à Oxford, était un physicien théoricien et cosmologiste britannique. Au même titre qu'Albert Einstein, Hawking n'aurait pas été particulièrement brillant à la petite école, mais son goût pour les sciences physiques le mène à l'Université d'Oxford, un lieu d'après lui d'ennui relatif d'où il sort avec les honneurs. Après avoir obtenu son diplôme B.A. à Oxford en 1962, il est resté pour étudier l'astronomie. Il a décidé d'arrêter quand il trouva que l'étude des taches solaires ne l'attirait pas et qu'il était plus intéressé par la théorie que par l'observation. Il a quitté Oxford, avec les honneurs, pour Trinity Hall où il a participé à l'étude de l'astronomie théorique et la cosmologie théorique. L'Université de Cambridge est un tout autre monde: d'un côté, Hawking y débute son passionnant doctorat sur la Relativité Générale, de l'autre, sa maladie se déclare. Malgré cette difficulté, l'étude des singularités, concept physique et astronomique récent, permet au chercheur de développer différentes théories, qui le mèneront du Big Bang aux Trous Noirs. En premier lieu, Roger Penrose et lui construisent la structure mathématique répondant à la question d'une singularité comme origine de l'Univers. Ensuite, à partir des années 1970, Hawking approfondit ses recherches sur les densités infinies locales, et ses études sur les Trous Noirs ont fait progresser bien d'autres domaines. Enfin, la "théorie du tout", visant à unifier les quatre forces physiques, est au centre des dernières recherches de Hawking. Le but est de démontrer que l'Univers peut être décrit par un modèle mathématique stable, déterminé par les lois physiques connues, en vertu du principe de croissance finie mais non bornée, modèle auquel Hawking a donné beaucoup de crédit. Son handicap lourd ne saurait expliquer à lui seul le grand succès de ses recherches ; Hawking a cherché à vulgariser son travail, et son livre \textit{Une brève histoire du temps} est l'un des plus grands succès de littérature scientifique. En 2001, paraît son deuxième ouvrage, \textit{L'univers dans une coquille de noix} qui vulgarise le dernier état de ses réflexions, en abordant la supergravité et la supersymétrie, la théorie quantique et théorie-M, l'holographie et la dualité, la théorie des supercordes et des $p$-branes... Il s'interroge également sur la possibilité de voyager dans le temps et sur l'existence d'univers multiples.

\parpic[l][t]{%
  \begin{minipage}{40mm}
    \fbox{\includegraphics[width=110px,height=140px]{img/medaillons/hausdorff.eps}}
  \end{minipage}
}
\textbf{Hausdorff, Felix} (1868-1942) La renommée du mathématicien allemand Felix Hausdorff repose surtout sur son ouvrage \textit{Grundzüge der Mengenlehre} (1914), qui en fit le fondateur de la topologie et de la théorie des espaces métriques. Né à Breslau dans une famille de marchands aisés, Hausdorff fit ses études secondaires à Leipzig, puis étudia la mathématique et l'astronomie à Leipzig, Fribourg-en-Brisgau et Berlin. En 1891, il obtint son doctorat à Leipzig et y enseigna de 1896 à 1902. Durant toute cette époque, Hausdorff, tout en publiant plusieurs mémoires d'astronomie, d'optique et de mathématiques, s'intéressa surtout à la philosophie, la littérature et l'art. De 1910 à 1935, il était professeur de mathématiques à l'Université de Bonn, à l'exception des années 1913-1921, où il enseignait à Greifswald. Depuis sa retraite forcée, en 1935, les travaux de Hausdorff ne furent plus publiés en Allemagne. Juif, Hausdorff risqua le camp de concentration et, lorsqu'en 1942 l'internement devint imminent, il se suicida à Bonn, avec sa femme et sa belle-soeur. Les contributions de Hausdorff au développement des mathématiques se situent dans plusieurs domaines. Son étude approfondie des séries déboucha sur la démonstration de théorèmes sur les méthodes de sommation et les coefficients de Fourier (1921). Considérant les propriétés d'ensembles numériques, il introduisit une classe importante de mesures. Il a étudié, en théorie générale des ensembles, les ensembles partiellement ordonnés et a obtenu plusieurs théorèmes sur les ensembles ordonnés (1906-1909). En théorie descriptive des ensembles, il a démontré le théorème sur la cardinalité des ensembles boréliens (1916). Outre des résultats isolés mais profonds en topologie et en théorie des ensembles, Hausdorff a surtout, par ses \textit{Grundzüge der Mengenlehre}, posé les fondements d'une discipline. Hausdorff développe une théorie des espaces topologiques et métriques englobant parfaitement les résultats antérieurs. Il choisit de construire sa théorie des espaces abstraits sur la notion de voisinage. Il ajouta bon nombre de résultats nouveaux à la théorie des espaces métriques, dont le plus profond est le théorème affirmant que chaque espace métrique peut être étendu d'une manière unique à un espace métrique complet. Hausdorff était un professeur méthodique, mais ses cours, au contenu riche et rigoureusement structuré, passèrent au-dessus du niveau de ses auditeurs.

\parpic[l][t]{%
  \begin{minipage}{40mm}
    \fbox{\includegraphics[width=110px,height=140px]{img/medaillons/heaviside.eps}}
  \end{minipage}
}
\textbf{Heaviside, Oliver} (1850-1925) est né à Camden Town en Angleterre et est mort à Torquay dans le Devon (situé aussi en Angleterre). C'est là qu'il a vécu les 25 dernières années de sa vie. Il est issu d'une famille assez pauvre. Il a attrapé la scarlatine quand il était un enfant en bas âge, ce qui a affecté son audition, il est resté partiellement sourd. Ce qui a eu un impact sur sa vie rendant son enfance difficile surtout au niveau des relations avec les autres enfants. Il a compensé par la timidité et le sarcasme. Cependant, malgré tout, son rendement académique était plutôt élevé. On peut même dire qu'à 16 ans, c'était un étudiant supérieur, mais il a échoué dans la géométrie d'Euclide. Il a détesté devoir déduire un fait d'autres. Le primat de la preuve rigoureuse en arithmétique, idée fortement détestée par Heaviside en fit le sujet où il était le plus faible. Bien qu'ayant interrompu ces études à 16 ans, il a continué à s'instruire par lui-même. Il a appris le code Morse, étudié l'électricité et d'autres langues en particulier le Danois et l'Allemand. Il était autodidacte. En 1868, après avoir quitté ses études, Heaviside est allé au Danemark et il est devenu opérateur de télégraphe. Il a progressé rapidement dans sa profession et il est revenu en Angleterre en 1871. C'est son travail qui l'a incité à étudier l'électricité. Il a donc lu le nouveau traité de Maxwell sur l'électricité et le magnétisme. Après avoir lu ce traité, il a apporté des changements à sa vie.  Il a arrêté de travailler et il s'est enfermé dans une chambre de la maison familiale pour travailler sur la théorie de Maxwell. Heaviside a réduit la théorie de Maxwell et c'est à partir de ce moment que la théorie électrique a pris sa forme moderne. Maxwell avait écrit 20 équations à 20 variables. Heaviside réduit ces 20 équations en les remplaçant par 4 équations à 2 variables. Aujourd'hui, nous appelons ces équations: "Les 4 équations de Maxwell", oubliant qu'elles sont en fait les équations de Heaviside! Cependant, c'est Hertz qui a obtenu le crédit pour cela, mais il admet que ses idées lui sont venues de Heaviside.

\parpic[l][t]{%
  \begin{minipage}{40mm}
    \fbox{\includegraphics[width=110px,height=140px]{img/medaillons/heisenberg.eps}}
  \end{minipage}
}
\textbf{Heisenberg, Werner Karl} (1901-1976) Né à Wurtzbourg et décédé à Münich était un physicien allemand. Il fut le fondateur des concepts théoriques rigoureux de la mécanique quantique. Il est lauréat du prix Nobel de physique de 1932. Il fréquente le prestigieux Maximiliangymnasieum où Max Planck avait étudié 40 ans plus tôt. A l'âge de 12 ans, il se mit à apprendre le calcul intégral et plus tard, passionné par les mathématiques, il suivit en auditeur libre plusieurs cours de l'Université de Münich, notamment sur les méthodes mathématiques de la physique moderne. Il accomplit ses études de physique dans le délai record de 3 ans, et soutint sa thèse (qu'il faillit louper à cause de lacunes en physique expérimentale élémentaire) sous la direction d'Arnold Sommerfeld avec lequel il élabora une théorie expliquant l'effet Zeeman anormal à l'âge de 20 ans qui attira l'attention des grands physiciens européens (il était considéré aussi brillant que Pauli qui lui-même était déjà considéré comme plus génial qu'Einstein). Dès 1924 il devenait l'assistant de Max Born à Göttingen puis il travailla avec Niels Bohr à Copenhague. C'est au cours des années suivantes qu'avec Max Born et Pascual Jordan, il jeta les bases théoriques de la mécanique quantique. Heisenberg fut recruté en 1927 comme professeur à l'Université de Leipzig âgé seulement de 26 ans. Il fit de cet établissement l'un des hauts-lieux de la physique théorique (et en particulier de la physique nucléaire) en Europe. Il développa la première formalisation de la mécanique quantique, en 1925, en même temps qu'Erwin Schrödinger. Toutefois le formalisme mathématique était différent. Heisenberg adopta une formalisation matricielle complexe (alors qu'il ne savait pas ce qu'était une matrice comme la majorité des physiciens de son temps...) qui faisait naturellement émerger la non commutativité alors que Schrödinger utilisa une approche par les équations différentielles (simple équation d'ondre). Pour cette raison, on crut d'abord que les deux théories étaient distinctes, mais l'année suivante, Schrödinger établit l'équivalence mathématique des deux formulations. Son principe d'incertitude, découvert en 1927, affirme que la détermination de certains couples de valeurs, par exemple la position et la quantité de mouvement, ne peut se faire avec une précision infinie. À partir de 1929, il travailla avec Wolfgang Pauli à l'élaboration de la théorie quantique des champs. Après la découverte du neutron par James Chadwick en 1932, Heisenberg proposa le modèle proton-neutron du noyau atomique, et s'en servit pour expliquer le spin nucléaire des isotopes.

\parpic[l][t]{%
  \begin{minipage}{40mm}
    \fbox{\includegraphics[width=110px,height=140px]{img/medaillons/hemlholtz.eps}}
  \end{minipage}
}
\textbf{Helmholtz, Hermann Ludwig Ferdinand von} (1821-1894) Né à Potsdam et mort à Berlin. Il n'est guère de domaines des sciences de la nature auxquels Helmholtz n'ait consacré quelques recherches. On pourrait répéter à son endroit, ce qu'il disait lui-même de Friedrich von Humboldt dans sa célèbre conférence inaugurale du colloque scientifique d'Innsbruck (sur le but et les progrès de la science de la Nature, 1869): "Il avait réussi à dominer toutes les sciences de la Nature à son époque et à pénétrer jusqu'en chacune de leurs spécialités." Même si Helmholtz ajoute que dans la seconde moitié du 19ème siècle ce savoir encyclopédique est désormais impossible, et qu'il faut se résigner à besogner dans un secteur étroitement délimité, il suffit de jeter un regard sur l'ensemble de ses travaux pour constater qu'il s'est préoccupé de matières aussi différentes que la thermodynamique, l'hydrodynamique, l'électrodynamique et la théorie de l'électricité, la physique météorologique, la physiologie, et plus particulièrement la théorie de l'acoustique et l'optique physiologique. Pourvu de dons remarquables pour la vulgarisation des résultats scientifiques les plus récents, il écrivit de nombreux articles et prononça maintes conférences où les exposés scientifiques populaires voisinent avec des préoccupations esthétiques ou philosophiques. Son nom reste surtout attaché à la formulation du principe de la conservation de l'énergie, qui fait de lui l'un des pères de l'énergétique, même si certaines de ses assertions peuvent sembler d'un mécanisme intransigeant et ont pu le faire considérer par certains comme le dernier tenant de la physique galiléenne. Son nom est aussi lié également à quelques inventions notoires comme celle de l'ophtalmoscope ou des résonateurs sphériques. Sur la fin de sa vie, Helmholtz reconnaîtra l'importance et l'universalité d'un autre principe physique, le principe de moindre action, qu'il appliquera, en particulier, à l'électrodynamique.

\parpic[l][t]{%
  \begin{minipage}{40mm}
    \fbox{\includegraphics[width=110px,height=140px]{img/medaillons/henry.eps}}
  \end{minipage}
}
\textbf{Henry, Joseph} (1797-1878) était un scientifique américain né à New-York et décédé à Washington. Ses parents étaient pauvres, et le père d'Henry est mort alors qu'il était encore jeune. Pour le reste de son enfance, Henry a vécu avec sa grand-mère à New York. Il a fréquenté une école qui allait plus tard être nommée Joseph Henry Elementary School, à son honneur. Après l'école, il a travaillé dans un magasin général, et à l'âge de 13 ans, il est devenu apprenti horloger et orfèvre. Son intérêt pour la science s'est éveillé à l'âge de 16 ans par un livre de conférences sur des sujets scientifiques intitulé \textit{Popular Lectures on Experimental Philosophy}. En 1819, il entra à l'académie Albany, où il a suivi des cours gratuits. Il était si pauvre qu'il devait subvenir à ses besoins avec des postes d'enseignement et de tutorat privé. Il avait l'intention d'étudier la médecine, mais en 1824 il a été nommé assistant ingénieur de l'inspection nationale des routes en cours de construction entre la rivière Hudson et le lac Érié. Dès lors, il a été inspiré d'une carrière soit dans le génie civil ou mécanique. Henry a tellement excellé dans ses études (à tel point, qu'il aidait souvent ses professeurs à enseigner la science) que, en 1826, il est nommé professeur de mathématiques et de philosophie naturelle à l'académie Albany. Certains de ses travaux de recherche les plus importants ont été réalisé dans pendant qu'il était à ce poste. Sa curiosité sur le magnétisme terrestre l'a amené à faire des expériences avec le magnétisme en général. Il a été le premier à enrouler un fil isolé étroitement autour d'un noyau de fer dans le but de faire un électro-aimant puissant. Pendant la construction des électro-aimants, Henry a découvert le phénomène électromagnétique d'auto-induction. Il a également découvert l'inductance mutuelle indépendamment de Michael Faraday, mais puisque Faraday a publié ses résultats en premier, il est devenu le découvreur officiellement reconnu du phénomène. En utilisant son principe électromagnétique nouvellement développé, Henry en 1831 a créé l'une des premières machines à utiliser l'électromagnétisme pour le mouvement. Ce fut le premier ancêtre du moteur à courant continu. L'unité SI de l'inductance, la Henry, est nommé en son honneur.

\parpic[l][t]{%
  \begin{minipage}{40mm}
    \fbox{\includegraphics[width=110px,height=140px]{img/medaillons/hermite.eps}}
  \end{minipage}
}
\textbf{Hermite, Charles} (1822-1901) Né à Dieuze, il publia ses premiers travaux alors qu'il était encore élève à l'École polytechnique, et à 30 ans, il était déjà considéré comme un des meilleurs mathématiciens de son temps. Il fut successivement professeur à l'école Polytechnique, au Collège de France et enfin à la Sorbonne à partir de 1869 où son enseignement et sa volumineuse correspondance eurent une influence considérable. Il vécut à Paris jusqu'à sa mort. Il avait été élu membre de l'académie des Sciences à 34 ans. En algèbre, Hermite prit une part active aux premiers développements de la théorie des invariants, inaugurée par Arthur Cayley et James Joseph Sylvester, il acheva, entre autres, la détermination des invariants des formes binaires du 5 degré, commencée par Sylvester, et découvrit la loi de réciprocité entre covariants de formes binaires de degrés différents. On lui doit aussi un procédé d'interpolation améliorant la méthode de Lagrange en tenant compte des valeurs des dérivées premières, et la découverte de la famille de polynômes orthogonaux qui portent son nom. Les travaux d'analyse d'Hermite portent la marque de son tempérament d'algébriste. Son sujet de prédilection pendant toute sa vie a été la théorie des fonctions elliptiques et des fonctions abéliennes, dont il aimait particulièrement explorer les liens cachés avec l'algèbre et la théorie des nombres. Un de ses résultats qui frappa le plus ses contemporains est la résolution de l'équation du cinquème degré à l'aide des fonctions elliptiques. Sa virtuosité dans les calculs des fonctions lui permit d'obtenir directement les remarquables relations sur les nombres de classes d'idéaux des corps quadratiques, que Kronecker avait déduites de la multiplication complexe. Il fut un des pionniers dans l'étude des fonctions abéliennes, où il développa la théorie de la transformation et rencontra à cette occasion pour la première fois le groupe symplectique. Enfin, le plus célèbre des mémoires d'Hermite est celui où, en 1872, il démontra la transcendance du nombre $e$ ; il y avait été conduit par ses recherches sur les fractions continuées algébriques, et sa méthode est restée presque la seule dont on dispose encore aujourd'hui pour aborder les problèmes de transcendance.

\parpic[l][t]{%
  \begin{minipage}{40mm}
    \fbox{\includegraphics[width=110px,height=140px]{img/medaillons/hertz.eps}}
  \end{minipage}
}
\textbf{Hertz, Heinrich Rudolf} (1857-1894) était un physicien allemand né à Hambourg et décédé à Bonn. Il fit ses études à l'Université de Berlin. En 1879, il est l'élève de Gustav Kirchhoff et de Hermann von Helmholtz à l'institut de physique de Berlin. Il devient maître de conférence à l'Université de Kiel en 1883 où il effectue des recherches sur l'électromagnétisme. De 1885 à 1889, à l'origine de la télégraphie sans fil, il fut professeur de physique à l'école technique de Karlsruhe, et, à partir de 1889, enseigna la physique à l'Université de Bonn. Hertz clarifia et étendit la théorie électromagnétique de la lumière proposée par le physicien anglais James Maxwell, en 1884. Il prouva que l'électricité pouvait être transmise par des ondes électromagnétiques qui se déplacent à la vitesse de la lumière et possèdent de nombreuses autres propriétés de la lumière. Ses expérimentations avec ces ondes aboutirent au développement du télégraphe sans fil et de la radio. L'unité de fréquence, une période par seconde, fut dénommée le "Hertz".

\parpic[l][t]{%
  \begin{minipage}{40mm}
    \fbox{\includegraphics[width=110px,height=140px]{img/medaillons/hilbert.eps}}
  \end{minipage}
}
\textbf{Hilbert, David} (1862-1943) Né à Königsberg et décédé à Göttingen fut un étudiant de Lindemann sous la supervision duquel il obtint sa thèse en 1885 et eut pour camarade Herman Minkowski, avec qui il resta lié par une profonde amitié. Bien que les intérêts mathématiques de Hilbert furent vastes, il préféra travailler à un sujet à la fois. Ses principaux domaines d'intérêts furent: jusqu'en 1892, la théorie algébrique des invariants; de 1892 à 1899 la théorie algébrique des nombres; de 1899 à 1905, le calcul des variations; de 1901 à 1912, les équations intégrales; de 1912 à 1917, les fondements mathématiques de la physique. Vers 1910, Hilbert soutient les efforts d'Emmy Noether, mathématicienne de premier ordre, qui souhaite enseigner à l'Université de Göttingen. Pour déjouer le système établi contre les femmes, Hilbert prête son nom à Noether qui peut ainsi annoncer l'horaire de ses cours sans entacher la réputation de l'université. De 1917 jusqu'à la fin de sa vie, il s'occupa de la logique mathématique. Il donna une impulsion décisive à l'essor des recherches sur les fondements des mathématiques. Au congrès international des mathématiques de 1900, Hilbert présenta une liste de $23$ problèmes dont plusieurs ne sont pas encore résolus aujourd'hui en ce début de 21ème siècle. Il a adopté et défendu avec vigueur les idées de Georg Cantor en théorie des ensembles et sur les nombres transfinis. Il est aussi connu comme l'un des fondateurs de la théorie de la démonstration, de la logique mathématique et a clairement distingué les mathématiques des métamathématiques. Il est considéré par plusieurs comme le plus grand mathématicien du 20ème siècle.

\parpic[l][t]{%
  \begin{minipage}{40mm}
    \fbox{\includegraphics[width=110px,height=140px]{img/medaillons/hotelling.eps}}
  \end{minipage}
}
\textbf{Hotelling, Harold} (1895-1973) était un statisticien mathématicien et un théoricien économique influent, connu pour la distribution $T^2$ de Hotelling dans les statistiques. Il a été professeur agrégé de mathématiques à l'Université de Stanford de 1927 à 1931, membre de la faculté de l'Université de Columbia de 1931 à 1946 et professeur de statistiques mathématiques à l'Université de Caroline du Nord à Chapel Hill de 1946 jusqu'à sa mort. Une rue à Chapel Hill porte d'ailleurs son nom. En 1972, il a reçu le prix de la Caroline du Nord pour ses contributions à la science. Hotelling est connu des statisticiens en raison de la distribution $T^2$ de Hotelling qui est une généralisation de la distribution $T$ de Student dans un contexte multivarié, et son utilisation dans les tests d'hypothèses statistiques et les régions de confiance. Il a également introduit l'analyse de corrélation canonique. Aux Etats-Unis d'Amérique (USA), Harold Hotelling est connu pour son leadership dans le métier de statisticien, notamment pour sa vision d'un département de statistiques à l'université, qui a convaincu de nombreuses universités de créer des départements statistiques.

\parpic[l][t]{%
  \begin{minipage}{40mm}
    \fbox{\includegraphics[width=110px,height=140px]{img/medaillons/hoyle.eps}}
  \end{minipage}
}
\textbf{Hoyle, Fred} (1915-2001) Né à Bingley, dans le Yorkshire et décédé à Bournemouth, il atudie la mathématique et la physique théorique à Cambridge de 1933 à 1939. Lorsque les hostilités de la deuxième guerre mondiale éclatent, il s'engage dans la Royal Navy pour travailler au développement du radar au centre de recherche de Witley. Il y rencontre les deux physiciens Hermann Bondi et Thomas Gold. Tous trois passionnés de cosmologie, ils considèrent avec scepticisme le modèle standard de l'Univers de l'époque (d'un point de vue philosophique il leur est inacceptable). À l'époque, le modèle standard achoppait de plus à une difficulté sérieuse: d'après les estimations de Hubble, l'âge de l'Univers devait être d'environ 2 milliards d'années, or les données géologiques conduisaient à un âge de la Terre d'au moins 4 milliards d'années. Pendant la guerre, et dans les quelques années qui suivent la fin des hostilités, Hoyle publie plusieurs études sur la théorie de l'accrétion et sur la théorie de la structure stellaire, en particulier pour les étoiles géantes et les naines blanches. La guerre terminée, les trois hommes retournent à Cambridge, où Hoyle obtient une chaire de mathématiques. En 1948, ils exposent leur théorie dans deux articles, l'un de Bondi et Gold, l'autre de Hoyle. En 1963, le premier quasar est découvert. Sa luminosité intrinsèque est très supérieure à celle de tout autre objet céleste connu: il est cent fois plus lumineux que n'importe quelle galaxie! En 1962, Hoyle et William A. Fowler avaient proposé une théorie qui pouvait rendre compte de la luminosité énorme des quasars ; il s'agissait de la théorie des étoiles supermassives. Des considérations théoriques permettent de démontrer que des étoiles normales  de masses supérieures à environ 60 masses solaires seraient le siège d'instabilités violentes dues à la pression de radiation et à la génération de l'énergie nucléaire. Cette hypothèse est corroborée par le fait que l'on n'observe pas d'étoiles normales au-delà de la limite d'instabilité. En dépit de cet argument, Hoyle et Fowler proposaient le concept d'étoile supermassive, étoile qui serait supportée presque entièrement par la pression de radiation. Ainsi, pour atteindre la luminosité caractéristique d'un quasar, l'étoile supermassive doit avoir une masse de l'ordre de 100 millions de masses solaires. Lorsque la densité devient suffisamment élevée, une étoile supermassive de moins de 1 million de masses solaires explose, tandis qu'une étoile plus massive subit un effondrement cataclysmique et forme des trous noirs supermassifs. Ces deux possibilités sont très importantes pour comprendre les quasars, et elles ont été étudiées par de nombreux chercheurs. Une autre explication du phénomène quasar, suggérée pour la première fois par Donald Lynden-Bell, suppose l'accrétion de matière dans un Trou Noir supermassif situé au centre d'une galaxie (hypothèse actuellement adoptée par consensus de la communauté scientifique).

\parpic[l][t]{%
  \begin{minipage}{40mm}
    \fbox{\includegraphics[width=110px,height=140px]{img/medaillons/huygens.eps}}
  \end{minipage}
}
\textbf{Huygens, Christian} (1629-1695) était un astronome, mathématicien et physicien néerlandais. Ses découvertes scientifiques, nombreuses et originales, lui valurent une large reconnaissance et les honneurs parmi les personnalités scientifiques du 17ème siècle. Avec son \textit{Traité de la lumière} (1690), il est à l'origine de la théorie ondulatoire de la lumière (qui plus tard prit son nom): chaque point d'ondes en mouvement est lui-même source de nouvelles ondes. Il se penche très vite, dès 1652 sur les règles exposées par Descartes dans les \textit{Principes de la philosophie}. Prenant appui sur la conservation cartésienne de la quantité de mouvement $p=mv$, il a l'idée de résoudre algébriquement le problème du choc en comparant les quantités $mv^2$ qui ne sont introduites que pour le bien du calcul, sans signification physique particulière. Découvrant alors que ces quantités se conservent avant et après le choc, il peut écrire les règles dans le cas général, ce que Descartes n'avait pu faire incluant donc conservation de la quantité de mouvement et de l'énergie cinétique. En 1655, il inventa une méthode de meulage et de polissage des lentilles d'optique. La définition plus fine ainsi obtenue lui permit de découvrir un satellite de Saturne et de fournir la première description précise des anneaux de Saturne. La nécessité de disposer d'une mesure exacte du temps pour l'observation du ciel l'amena à appliquer les lois du pendule composé pour régler les mouvements des horloges et montres. En 1656, il conçut une lunette de télescope qui porte son nom. Entre 1658 et 1659, Huygens travaille à la théorie du pendule oscillant. Il a en effet l'idée de réguler des horloges au moyen d'un pendule, afin de rendre la mesure du temps plus précise. Il découvre la formule de l'isochronisme rigoureux en 1659: lorsque l'extrémité du pendule parcourt un arc de cycloïde, la période d'oscillation est constante quelle que soit l'amplitude. Dans Horologium oscillatorium (1673), il détermina la véritable relation existant entre la longueur d'un pendule et la durée d'oscillation, et présenta ses théories sur la force centrifuge des mouvements circulaires, qui aidèrent le physicien anglais Isaac Newton à formuler les lois de la gravité. En 1673, Huygens et son jeune assistant Denis Papin, mettent en évidence le principe des moteurs à combustion interne, qui conduiront au xixe siècle à l'invention de l'automobile. En 1678, il découvrit la polarisation de la lumière par double réfraction sur la calcite.

\phantomsection
\addcontentsline{toc}{section}{I}
\label{sec:I}

\parpic[l][t]{%
  \begin{minipage}{40mm}
    \fbox{\includegraphics[width=110px,height=140px]{img/medaillons/ibn.eps}}
  \end{minipage}
}
\textbf{Ibn Al Haytham} (965-1039) était un mathématicien, un philosophe et un physicien Arabe. Il est l'un des pères de la physique quantitative et de l'optique moderne, le pionnier de la méthode scientifique moderne et le fondateur de la physique expérimentale et certains, pour ces raisons, l'ont décrit comme le premier scientifique. Al Haytham commença sa carrière de scientifique dans sa ville natale de Bassorah (Irak). Il fut cependant convoqué par le calife Hakim qui voulait maîtriser les inondations du Nil qui frappaient l'Égypte année après année. Après avoir mené une expédition en plein désert pour remonter jusqu'à la source du fameux fleuve, Alhazen se rendit compte que ce projet était pratiquement impossible. De retour au Caire, il craignait que le calife qui était furieux de son échec ne se vengeât et décida donc de feindre la folie. Le calife se borna à l'assigner à résidence. Alhazen profita de ce loisir forcé pour écrire plusieurs livres sur des sujets variés comme l'astronomie, la médecine, la mathématique, la méthode scientifique et l'optique. Le nombre exact de ses écrits n'est pas connu avec certitude mais on parle d'un nombre entre 80 et 200. Peu de ces ouvrages, en effet, ont survécu jusqu'à nos jours. Quelques-uns d'entre eux, ceux sur la cosmologie et ses traités sur l'optique notamment, n'ont survécu que grâce à leur traduction latine. La plupart de ses recherches concernaient l'optique géométrique et physiologique. Contrairement à une croyance populaire, il a été le premier à expliquer pourquoi le Soleil et la Lune semblent plus gros (on a cru longtemps que c'était Ptolémée), il établit aussi que la lumière de la Lune vient du Soleil. C'est aussi lui qui a contredit Ptolémée sur le fait que l'oeil émettrait de la lumière. Selon lui, si l'oeil était conçu de cette façon on pourrait voir la nuit. Il a compris que la lumière du Soleil était diffusée par les objets et ensuite entrait dans l'oeil. En astronomie il a tenté de mesurer la hauteur de l'atmosphère et a trouvé que le phénomène du crépuscule est dû à un phénomène de réfraction. Il parla également de l'attraction des masses et on croit qu'il connaissait l'accélération gravitationnelle. Alhazen a devancé de quelques siècles plusieurs découvertes faites par des scientifiques occidentaux pendant la Renaissance. Il fut un des premiers à se servir d'une méthode d'analyse scientifique et influença grandement des scientifiques comme Roger Bacon et Johannes Kepler.

\phantomsection
\addcontentsline{toc}{section}{J}
\label{sec:J}

\parpic[l][t]{%
  \begin{minipage}{40mm}
    \fbox{\includegraphics[width=110px,height=140px]{img/medaillons/jacobi.eps}}
  \end{minipage}
}
\textbf{Jacobi, Carl} (1804-1851) Né à Potsdam et décédé à Berlin fut, avec N. H. Abel, le fondateur de la théorie des fonctions elliptiques dont il donna de nombreuses applications aux branches les plus diverses des mathématiques. On lui doit également des exposés de mécanique théorique où il reprend les résultats de W. R. Hamilton, et des applications de la théorie des équations différentielles à la dynamique. À son entrée au gymnase, en 1816, Jacobi avait déjà achevé le cycle des études secondaires et, assez réfractaire à l'enseignement traditionnel, il étudia directement les oeuvres des grands mathématiciens, particulièrement celles d'Euler et de Lagrange. Inscrit en 1821 à l'Université de Berlin, il y apprit la philologie et la mathématique, à laquelle il se consacra bientôt uniquement. En 1825 il était docteur en philosophie avec une thèse où il démontrait ou généralisait certaines formules de Lagrange. Il enseigna à Berlin pendant une année environ, puis à Koenigsberg où il fut transféré par décision ministérielle. Fin 1827, il fut nommé professeur extraordinaire à l'université de cette ville où il entra en contact avec l'astronome Friedrich Wilhelm Bessel. Pensionné par le gouvernement de Prusse, il fut, après un voyage en Italie, en 1843, nommé académicien à Berlin, dispensé de tout enseignement mais autorisé à traiter, à l'université, tout sujet qui lui conviendrait. Présenté comme candidat aux élections de mai 1848, il fut persécuté un temps pour ses opinions libérales. Jacobi consacra de nombreux travaux à la transformation des intégrales et apporta une contribution essentielle à la théorie des équations différentielles et des équations aux dérivées partielles. C'est à cela que se rattachent ses apports au calcul des variations, à la dynamique des solides et à la mécanique céleste – problème des trois corps, perturbations des mouvements planétaires. L'algèbre lui doit d'importantes recherches sur les formes quadratiques et une exposition devenue classique de la théorie des déterminants, prélude au mémoire sur les déterminants fonctionnels appelés de nos jours "jacobiens". Il perfectionne la théorie de l'élimination et enseigne à représenter les racines d'une équation algébrique par des intégrales définies ou par des séries. Il étudie les points communs aux courbes et aux surfaces algébriques, et trouve directement le nombre des tangentes doubles d'une courbe plane, établi déjà par J. Plücker en utilisant la dualité.

\parpic[l][t]{%
  \begin{minipage}{40mm}
    \fbox{\includegraphics[width=110px,height=140px]{img/medaillons/jordan_camille.jpg}}
  \end{minipage}
}
\textbf{Jordan, Camille} (1838-1921) Né à Lyon et décédé à Paris fut le spécialiste indiscuté de la théorie des groupes pendant toute la fin du 19ème siècle et on lui doit de très nombreux résultats, tant sur les groupes finis que sur les groupes dits classiques, dont il fut le premier à mesurer toute l'importance. Ses cours d'analyse contribuèrent au développement de la théorie des fonctions de variable réelle. En 1855, à 17 ans, il est reçu premier à l'École Polytechnique et sort de l'École des Mines en 1861. Il sera, du moins en titre, ingénieur chargé de la surveillance des carrières de Paris jusqu'en 1885, ce qui n'empêchera pas une intense activité de recherche mathématique. Nommé examinateur à l'École Polytechnique en 1873, puis professeur en 1876, il entre à l'Académie des Sciences en 1881 puis succède à Joseph Liouville au Collège de France deux années plus tard. De 1885 à 1921, il assume la direction du \textit{Journal de mathématiques pures et appliquées} fondé par Liouville. Malgré les efforts de Liouville, l'oeuvre d'Évariste Galois était restée à peu près totalement inconnue du monde des mathématiques (seul Leopold Kronecker avait utilisé certains de ses résultats), et c'est à Jordan, avec son \textit{Traité des substitutions et des équations algébriques}, publié à Paris en 1870, que l'on doit le premier exposé systématique de théorie des groupes, enrichi de dix années de recherches personnelles. Il s'y limite aux groupes finis, plus précisément aux groupes de permutations, et introduit de nombreux concepts nouveaux. Dans des mémoires ultérieurs, Jordan étudie en détail, essentiellement du point de vue des facteurs de composition, le groupe linéaire et les groupes orthogonaux et symplectiques sur un corps premier. Les études de Jordan sur le groupe linéaire font intervenir des considérations sur la réduction des matrices, et, en particulier, la forme dite "forme de Jordan". Indiquons enfin les efforts de Jordan pour déterminer tous les groupes résolubles finis en réponse au problème, posé par Niels Henrik Abel, de rechercher toutes les équations de degré donné résolubles par radicaux. En plus des résultats donnés ci-dessus relatifs au groupe linéaire, on doit à Jordan un exposé complet de la géométrie euclidienne réelle à $n$ dimensions par des méthodes entièrement analytiques. L'enseignement de Jordan à l'École Polytechnique, puis au Collège de France, l'amène à préciser de nombreuses notions de la théorie des fonctions de variable réelle et son \textit{Cours d'analyse de l'École polytechnique}, dont la première édition date de 1880, contribuera à former des générations de mathématiciens. On lui doit aussi la notion de fonction à variation bornée, qui lui permet de donner une définition correcte de la longueur d'une courbe et d'obtenir sous sa forme générale le théorème de convergence des séries de Fourier ; mais le résultat le plus célèbre est celui qui affirme qu'une courbe fermée simple (dite, de nos jours, "courbe de Jordan") sépare le plan en deux régions. Signalons enfin, pour terminer, que Jordan, précurseur de Poincaré, a écrit plusieurs mémoires d'Analysis situs, c'est-à-dire sur la topologie combinatoire. On lui doit une démonstration, devenue classique, du théorème d'Euler sur les polyèdres et le fait que deux surfaces de même genre sont applicables l'une sur l'autre (ce qui, comme l'a montré Poincaré, n'est pas vrai en général pour les hypersurfaces).

\parpic[l][t]{%
  \begin{minipage}{40mm}
    \fbox{\includegraphics[width=110px,height=140px]{img/medaillons/jordanp.eps}}
  \end{minipage}
}
\textbf{Jordan, Pascual} (1902-1980) Né à Hanovre et décédé à Hamburg il était un physicien théoricien allemand, professeur à l'Université de Göttingen. Jordan passa dès 1921 une partie de ses études à l'Université Technique de Hanovre où il étudia un mélange de zoologie, de mathématique et physique. En 1923 il se spécialisa lors de son entrée à l'Université de Göttingen, qui était alors à son zénith tant du point de vue de la mathématique que de la physique. À Göttingen, Jordan devint assistant de Richard Courant et surtout de Max Born qui l'influença grandement. Il contribua ainsi de façon décisive à la fondation de la mécanique quantique et de la théorie quantique des champs. En raison de son affiliation au parti nazi, il fut cependant isolé de la communauté physicienne. En 1925, avec Max Born, Jordan écrit la relation de commutation canonique entre la quantité de mouvement et la position. Dans ce même article, il propose également l'idée qu'il faut aussi quantifier le champ électromagnétique, ouvrant la voie à la théorie quantique des champs. En 1925 également, avec Max Born et Werner Heisenberg, Jordan développe la formulation matricielle de Heisenberg de la mécanique quantique. Ils introduisent les transformations canoniques, la théorie des perturbations, le traitement des systèmes dégénérés, et enfin la fameuse relation de commutation canonique des composantes du moment cinétique.

\parpic[l][t]{%
  \begin{minipage}{40mm}
    \fbox{\includegraphics[width=110px,height=140px]{img/medaillons/joule.eps}}
  \end{minipage}
}
\textbf{Joule, James Prescott} (1818-1889) était un physicien britannique, né à Salford, dans le Lancashire et décédé à Sale (Angleterre). Il fut l'un des plus grands physiciens de son époque. Joule est célèbre pour ses travaux de recherche en électricité et en thermodynamique. Au cours de ses recherches sur la chaleur émise dans un circuit électrique, il formula la loi, connue sous le nom de "loi de Joule", sur la chaleur électrique, qui indique que la quantité de chaleur produite chaque seconde dans un conducteur par le passage du courant électrique est proportionnelle à la résistance électrique du conducteur et au carré du courant électrique. Joule a vérifié expérimentalement la loi de la conservation de l'énergie dans son étude sur la transformation de l'énergie mécanique en énergie thermique (relation entre joules et calories: il faut $1$ calorie, soit $4.18$ joules pour augmenter $1$ gramme d'eau de $1$ degré).

\phantomsection
\addcontentsline{toc}{section}{K}
\label{sec:K}

\parpic[l][t]{%
  \begin{minipage}{40mm}
    \fbox{\includegraphics[width=110px,height=140px]{img/medaillons/kepler.eps}}
  \end{minipage}
}
\textbf{Kepler, Johannes} (1571-1630) était astronome et physicien allemand, célèbre pour sa formulation et sa vérification des trois lois du mouvement planétaire. Ces lois sont maintenant connues sous le nom de "lois de Kepler". Son principal traité contient les formulations de deux des lois du mouvement planétaire. La première stipule que les planètes se déplacent selon des orbites elliptiques avec le Soleil comme foyer et la seconde, ou "loi des aires", énonce que la ligne imaginaire que l'on tracerait entre le Soleil et une planète balaie des aires identiques d'une ellipse pendant des intervalles de temps égaux; en d'autres termes, plus la planète se rapproche du Soleil, plus elle se déplace rapidement. Un autre traité contient une autre découverte sur le mouvement planétaire: le cube de la distance entre une planète et le Soleil divisé par la période orbitale de cette planète au carré est une constante et est la même pour toutes les planètes. Le mathématicien et physicien anglais Isaac Newton se reposa fortement sur les théories et les observations de Kepler pour formuler sa théorie de la force gravitationnelle. Kepler apporta également sa contribution dans le domaine de l'optique et développa en mathématiques un système infinitésimal qui fut le précurseur du calcul infinitésimal.

\parpic[l][t]{%
  \begin{minipage}{40mm}
    \fbox{\includegraphics[width=110px,height=140px]{img/medaillons/keynes.eps}}
  \end{minipage}
}
\textbf{Keynes, John Maynard} (1883-1946) était un économiste britannique. Il est le fondateur du "keynésianisme", doctrine économique qui encourage l'intervention de l'État au sein de l'économie, pour assurer le plein-emploi. Keynes est né dans une famille d'universitaires. À 7 ans, il entra à Perse School. Deux ans plus tard, il entrait en classe préparatoire à St Faith's. Avec les années, il se montra très prometteur et en 1894, il termina premier de sa classe et reçut un prix pour la première fois en mathématiques. Un an plus tard, il intègre l'Eton College où il brille et gagne, en 1899 et en 1900, le prix de mathématiques. En 1901, il finit premier en mathématiques, histoire et anglais. En 1902, il gagne sa place pour le King's college de Cambridge. Keynes est sans aucun doute une importante figure de l'histoire de la science économique qu'il révolutionna avec son oeuvre principale, la \textit{Théorie générale de l'emploi, de l'intérêt et de la monnaie} parue en 1936. L'ouvrage est considéré comme le traité de science sociale le plus influent du 20ème siècle dans la mesure où il a rapidement et continuellement modifié la façon dont le Monde a considéré l'économie et le rôle du pouvoir politique dans la société. Certains estiment qu'aucun autre ouvrage n'a eu une telle importance depuis en Europe, bien que l'ouvrage de Friedrich Hayek qui lui valut son Prix Nobel, \textit{The Road to Serfdom}, fasse la démonstration fulgurante des limites de la théorie keynésienne. Avec la \textit{Théorie générale}, Keynes a développé une théorie qui pouvait expliquer le niveau de la production et par conséquent de l'emploi ; le facteur déterminant étant la demande. Parmi les concepts révolutionnaires apportés par Keynes, on retiendra surtout: ceux de l'équilibre de sous-emploi où le chômage est possible pour un niveau donné de la demande effective, l'absence de mécanisme de régulation par les prix afin de résorber le chômage, une théorie de la monnaie fondée sur la préférence pour la liquidité, l'introduction de l'incertain et des prévisions, la notion d'efficacité marginale de l'investissement brisant la loi de Say (et renversant donc le lien de causalité épargne-investissement). Ces concepts accréditent la possibilité de politiques interventionnistes pour éliminer les récessions et freiner les surchauffes économiques. L'ensemble de ces concepts constitue ce qu'on appelle aujourd'hui la "macroéconomie".

\parpic[l][t]{%
  \begin{minipage}{40mm}
    \fbox{\includegraphics[width=110px,height=140px]{img/medaillons/kirchhoff.eps}}
  \end{minipage}
}
\textbf{Kirchhoff, Gustav Robert} (1824-1887) Né à Könisberg (aujourd'hui Kaliningrad en Russie) et décédé à Berlin, Kirchhoff étudie la physique-mathématique auprès de Franz Neumann. Après un doctorat en 1847, il devient conférencier à l'Université de Berlin avant d'obtenir, en 1850, le poste de professeur de physique extraordinaire à l'Université de Breslau. C'est là qu'il fait la connaissance du chimiste Robert Wilhelm Bunsen, avec qui il sera amené à travailler de nombreuses années. Leur collaboration se poursuivra en effet au-delà de 1854, date à laquelle Kirchhoff est nommé à professeur de physique à l'Université de Heidelberg. Élu vice-recteur de cette même université en 1865, il finit par accepter une chaire de physique théorique à Berlin en 1875. Kirchhoff est encore étudiant lorsqu'il commence à s'intéresser aux problèmes liés à l'électricité. En 1845, il établit la notion de potentiel électrique et énonce les lois de réseaux qui portent son nom (loi des noeuds, loi des mailles). Il généralise la loi d'Ohm sur le courant électrique à des conducteurs à trois dimensions et, plus tard, montre que le passage du courant à travers un conducteur se fait à la vitesse de la lumière. Sa rencontre avec Bunsen aboutit à la naissance de la spectroscopie. Ensemble, les deux chercheurs découvrent le caractère spécifique du spectre de la lumière émise par chaque corps chimique. Grâce à ce nouvel outil d'analyse, ils dépistent deux éléments encore inconnus: le césium (1860) et le rubidium (1861). La mise au point du spectroscope à prisme, pour analyser la lumière de substances en combustion, permet également à Kirchhoff d'établir la loi du rayonnement: le rapport des pouvoirs d'émission et d'absorption d'un corps, indépendant des propriétés de ce corps, est fonction de la température et de la longueur d'onde. Le pouvoir d'émission est ainsi proportionnel à celui du "corps noir", défini par Kirchhoff (1862) comme le corps parfaitement absorbant. Cette loi, qui explique notamment la présence des raies sombres d'absorption (dites "raies de Fraunhofer") dans le spectre de rayonnement solaire, marque le début d'une nouvelle ère en astrophysique et annonce l'avènement de la théorie des quanta de Planck.

\parpic[l][t]{%
  \begin{minipage}{40mm}
    \fbox{\includegraphics[width=110px,height=140px]{img/medaillons/klein.eps}}
  \end{minipage}
}
\textbf{Klein, Felix} (1849-1925)  fit ses études à Bonn, à Göttingen et à Berlin. En 1872, il devint professeur de mathématiques à l'Université d'Erlangen, où son cours inaugural fut l'énoncé des grandes lignes de son fameux programme d'Erlangen. Il enseigna ensuite à Münich (1875-1880), puis à l'université de Leipzig (1880-1886) et enfin à Göttingen (1886-1913). À partir de 1872, il édita les \textit{Mathematische Annalen} de Göttingen et fonda, en 1895, la grande \textit{Encyclopédie mathématique}, dont il supervisa la rédaction jusqu'à sa mort, à Göttingen. Il fut le chef incontesté de l'école mathématique allemande, et son influence fut très grande (il donna de nombreuses conférences à l'étranger, dont les États-Unis), notamment sur le développement de la géométrie, grâce à son programme d'Erlangen. Avec le texte, publié dans son ouvrage \textit{Gesammelte mathematische Abhandlungen} (1921-1923), Klein donne une définition de la géométrie englobant aussi bien la géométrie classique (c'est-à-dire euclidienne) que la géométrie projective, les géométries non euclidiennes, etc., mettant fin aux controverses stériles entre partisans de la géométrie synthétique et ceux de la géométrie analytique. Pour lui, une géométrie est l'étude des propriétés invariantes par un groupe donné de transformations: ainsi les théorèmes de géométrie classique sont l'expression de relations entre invariants du groupe des similitudes ; ceux de la géométrie projective entre covariants du groupe projectif. On doit aussi à Klein d'importants travaux sur l'équation différentielle hypergéométrique, sur les fonctions abéliennes, sur le groupe de l'icosaèdre régulier (\textit{Lectures on the Icosahedron}, 1914), sur les fonctions elliptiques, à partir desquelles il dégage la notion de fonction modulaire (\textit{Vorlesungen über die Theorie der automorphen Funktionen}, 1897-1902).

\parpic[l][t]{%
  \begin{minipage}{40mm}
    \fbox{\includegraphics[width=110px,height=140px]{img/medaillons/kolmogorov.eps}}
  \end{minipage}
}
\textbf{Kolmogorov, Andrei} (1903-1987) était un mathématicien russe dont les apports en mathématiques sont considérables. Kolmogorov est né à Tambov. Sa mère célibataire mourut à sa naissance et il fut élevé par sa tante avec les économies de son grand-père. On pense que son père fut tué lors de la guerre civile russe. Kolmogorov fut scolarisé à l'école du village de sa tante, et ses premiers efforts littéraires et articles mathématiques furent imprimés dans le journal de l'école. Adolescent, il conçut des machines à mouvement perpétuel, cachant tellement bien leurs défauts intrinsèques que ses professeurs d'enseignement secondaire n'arrivaient pas à les découvrir. En 1910, sa tante l'adopta et ils déménagèrent à Moscou, où il intégra un Gymnasium et y obtint son diplôme en 1920. Après avoir terminé ses études secondaires, il suit les cours à l'Université de Moscou et à l'institut Mendeleïev. Il étudie non seulement les mathématiques, mais aussi l'histoire russe et la métallurgie. En 1922, Kolmogorov publie ses premiers résultats concernant la théorie des ensembles et, en 1923 il publie ses travaux sur la théorie de l'intégration, sur l'analyse de Fourier et pour la première fois sur la théorie des probabilités et commence à devenir connu à l'étranger. Après la fin de ses études supérieures en 1925, il commence son doctorat auprès de Nikolaï Louzine, qu'il termine en 1929. En 1931, il reçoit une chaire de professeur à l'Université de Moscou. En 1933, paraît en allemand son manuel des \textit{Fondements de la théorie des probabilités} dans lequel il présente son axiomatisation du calcul des probabilités ainsi qu'une manière adaptée à traiter les processus stochastiques. La même année, il devient directeur de l'Institut de mathématiques de l'Université de Moscou. En 1934, il publie son travail sur la cohomologie et obtient, grâce à cette thèse, le titre de docteur en mathématique et en physique. Il obtient des récompenses des autorités soviétiques, comme l'Ordre de la science socialiste (1940), le prix Staline (1941) et plusieurs fois le prix Lénine. En 1941, il élabore une théorie fameuse de la turbulence des fluides. En 1953 et 1954, il décrit la théorie KAM (Kolmogorov-Arnold-Moser) de stabilité des systèmes dynamiques (un système mécanique complexe exactement soluble est stable si on le perturbe seulement un tout petit peu). Il introduit également la notion d'entropie métrique pour les systèmes dynamiques mesurés. En 1955, il devient docteur honoris causa de la Sorbonne. En 1962, il reçoit le Prix Balzan pour les mathématiques.

\parpic[l][t]{%
  \begin{minipage}{40mm}
    \fbox{\includegraphics[width=110px,height=140px]{img/medaillons/kronecker.eps}}
  \end{minipage}
}
\textbf{Kronecker, Leopold} (1823-1891) était un mathématicien allemand qui nous apparaît comme l'un des plus grands arithméticiens du 19ème siècle et l'un des fondateurs de la grande théorie des nombres algébriques. Ses travaux sur le corps de classes dans un cas particulier ont préparé ceux de Hilbert. Né à Liegnitz, dans une famille de riches commerçants, Kronecker suivit au gymnase les cours d'Ernst Kummer, qu'il devait retrouver plus tard comme professeur à l'Université de Breslau, puis comme collègue à Berlin, et qui, avec Peter Gustav Lejeune-Dirichlet, devait avoir l'influence la plus profonde sur le développement de sa pensée. Après avoir soutenu, en 1845, une thèse très originale sur les unités des corps cyclotomiques, il s'occupa pendant plusieurs années des affaires familiales, et ne put se livrer entièrement de nouveau aux recherches mathématiques qu'à partir de 1853. Élu, en 1860, membre de l'Académie des sciences de Berlin, il donna, à partir de cette époque, des cours libres à cette université, où il fut nommé professeur titulaire en 1883 et où il acheva sa vie. Bien que maniant avec virtuosité toutes les ressources de l'analyse (comme le montrent ses travaux sur les fonctions elliptiques, les séries de Dirichlet ou encore sa formule intégrale donnant le nombre des racines d'un système d'équations dans un espace à n dimensions), Kronecker est avant tout un algébriste et un arithméticien. Vers la fin de sa vie, il professait une doctrine tendant à rejeter l'infini actuel des mathématiques en ne gardant comme valable que ce qui pouvait être uniquement fondé sur les nombres entiers (ses polémiques avec Cantor sont restées célèbres). En algèbre, Kronecker fut l'un des animateurs les plus actifs du groupe de mathématiciens qui, dans les années 1860-1890, achevèrent de mettre sur pied l'algèbre linéaire et multilinéaire inaugurée par Arthur Cayley et Hermann Grassmann aux alentours de 1845. C'est ainsi qu'il reprit et compléta les travaux de Karl Weierstrass et fut l'un des premiers à comprendre et à utiliser les travaux d'Évariste Galois (publiés en 1846).

\phantomsection
\addcontentsline{toc}{section}{L}
\label{sec:L}

\parpic[l][t]{%
  \begin{minipage}{40mm}
    \fbox{\includegraphics[width=110px,height=140px]{img/medaillons/lagrange.eps}}
  \end{minipage}
}
\textbf{Lagrange, Joseph Louis} (1736-1813) Né à Turin et décédé à Paris, était comme l'un des plus grands mathématicien et astronome du 18ème siècle. Élève brillant issu d'un milieu aisé, il étudie au collège de Turin. Il prend goût pour les mathématiques par hasard à l'âge de 17 ans après la lecture d'un mémoire de Edmund Halley portant sur les applications de l'algèbre en optique. Le sujet l'intéresse au plus haut point. Dès lors, il se passionne pour les mathématiques qu'il étudie seul et assidûment. Il devient rapidement un mathématicien confirmé et ses premiers résultats ne se font pas attendre. Dans une lettre adressée à Leonhard Euler il jette les bases du calcul variationnel. Cet échange est le début d'une longue correspondance entre les deux hommes. Lagrange a alors 19 ans et enseigne à l'école d'artillerie de Turin où il fut nommé en 1755. Il fonde en 1758 l'Académie des Sciences de Turin qui publiera ses premiers résultats sur l'application du calcul variationnel à des problèmes de mécanique (propagation du son, corde vibrante...). En 1764, ses travaux sur les librations de la Lune (petites variations de son orbite) sont récompensés par le Grand Prix de l'Académie des Sciences de Paris. Il introduisit de nouvelles méthodes pour le calcul des variations et pour l'étude des équations différentielles, qui lui permirent de donner un exposé systématique de la mécanique dans son célèbre ouvrage Mécanique analytique (1788). Il travailla sur la théorie additive des nombres. On lui doit le théorème sur la décomposition d'un entier en quatre carrés. Dans l'étude des équations algébriques, il introduisit des concepts qui conduiront à la théorie des groupes développée plus tard par Abel et Galois. En physique, en précisant le principe de moindre action, avec le calcul des variations, vers 1756, il invente la fonction de Lagrange, qui vérifie les équations de Lagrange, puis développe la mécanique analytique, vers 1788, pour laquelle il introduit les multiplicateurs de Lagrange. Il entreprend aussi des recherches importantes sur le problème des trois corps en astronomie, un de ses résultats étant la mise en évidence des points de libration (dits points de Lagrange) (1772).

\parpic[l][t]{%
  \begin{minipage}{40mm}
    \fbox{\includegraphics[width=110px,height=140px]{img/medaillons/langevin.eps}}
  \end{minipage}
}
\textbf{Langevin, Paul} (1872-1946) était un physicien français né et décédé à Paris. Très jeune, Langevin manifeste des dons exceptionnels. Encouragé par ses instituteurs, il parcourt rapidement les divers échelons de l'enseignement obligatoire avant d'entrer à 16 ans à l'École Supérieure de Physique et de Chimie Industrielle de la Ville de Paris. Langevin y suit les cours et l'enseignement de laboratoire de Pierre Curie, avec lequel il se lie d'amitié. À sa sortie de l'école, il renonce à la carrière d'ingénieur et décide, sur les conseils de Pierre Curie, de se consacrer à la recherche et à l'enseignement. Aussi, se présente-t-il à l'École Normale Supérieure où il est reçu premier en 1894. En 1897, il bénéficie d'une bourse pour aller travailler un an au Cavendish Laboratory de Cambridge, haut lieu de la science européenne où se trouvent alors E. Rutherford et J. J. Thomson. De retour en France, il soutient sa thèse en 1902, est nommé professeur suppléant, puis professeur au Collège de France. En 1904, il succède à Pierre Curie à l'École de Physique et de Chimie, dont il devient directeur en 1925. L'oeuvre de Langevin se situe dans cette longue période de transition qui, de 1900 à 1930, mène de la physique classique à la physique moderne, dominée par la théorie de la relativité et la théorie quantique. Ses premiers travaux (sur l'ionisation des gaz) l'amènent à élaborer en 1905 son modèle théorique majeur, qui devrait par la suite servir de base à de nombreuses autres explications des propriétés macroscopiques de la matière, dans lequel les électrons à l'intérieur des atomes décrivent des orbites fermées, conférant ainsi aux atomes des propriétés analogues à celles de petits aimants. En 1906 il aboutit au résultat étonnant selon lequel l'inertie serait une propriété de l'énergie..., du moins dans le cas de l'électron. Ce n'est que quelques mois plus tard qu'il lira le mémoire d'Einstein sur la théorie de la relativité restreinte à laquelle il va consacrer son enseignement dans ses cours au Collège de France. Langevin est aussi à l'origine des fameux Congrès Solvay qui, dès 1911, réunirent périodiquement tous les grands noms de la physique, et où furent largement discutés les concepts de la théorie quantique. C'est d'ailleurs grâce à Langevin que les travaux de son élève Louis de Broglie sur la mécanique ondulatoire connurent la diffusion qu'ils méritaient: d'abord étonné, Langevin fut très vite convaincu de la justesse des idées de De Broglie et inscrivit immédiatement la nouvelle mécanique ondulatoire au programme de son cours au Collège de France. Fidèle à l'idéal de clarté pédagogique qui fut toujours le sien, Langevin a par ailleurs effectué, sur les concepts encore en gestation de la théorie quantique, un travail d'analyse et de refonte épistémologiques dont on mesure aujourd'hui l'importance.

\parpic[l][t]{%
  \begin{minipage}{40mm}
    \fbox{\includegraphics[width=110px,height=140px]{img/medaillons/langevin.eps}}
  \end{minipage}
}
\textbf{Laplace, Pierre Simon} (1749-1827) Né à Beaumont-en-Auge et décédé à Paris, fils de cultivateur, Laplace s'initia aux mathématiques à l'École Militaire de cette petite ville. Il y commença son enseignement. Il doit cette éducation à ses voisins aisés qui avaient détecté son intelligence exceptionnelle. À 18 ans, il arrive à Paris avec une lettre de recommandation pour rencontrer le mathématicien d'Alembert, mais ce dernier refuse de rencontrer l'inconnu. Mais Laplace insiste: il envoie à d'Alembert un article qu'il a écrit sur la mécanique classique. D'Alembert en est si impressionné qu'il est tout heureux de patronner Laplace. Il lui obtient un poste d'enseignement en mathématiques. L'oeuvre la plus importante de Laplace concerne le calcul des probabilités, les équations différentielles (laplacien) et la mécanique céleste. Il établit aussi, grâce à ses travaux avec Lavoisier entre 1782 et 1784 la relation des transformations adiabatiques d'un gaz, ainsi que deux lois fondamentales de l'électromagnétisme. En mécanique, c'est avec le mathématicien Joseph-Louis de Lagrange, que Laplace résume ses travaux et réunit ceux de Newton, Halley, Clairaut, d'Alembert et Euler, concernant la gravitation universelle (particulièrement le problème de stabilité du système solaire), dans les 5 volumes de sa \textit{Mécanique céleste} (1798-1825). On rapporte (mais c'est très probablement une légende) que, feuilletant la \textit{Mécanique céleste}, Napoléon fit remarquer à Laplace qu'il n'y était nulle part fait mention de Dieu. "Je n'ai pas eu besoin de cette hypothèse", rétorqua le savant qui n'était par ailleurs pas modeste (se considérant - probablement à juste titre - comme le meilleur mathématicien de sa génération). Il aussi un des premiers scientifiques à concevoir l'existence des Trous Noirs et la notion de "collapsus gravitationnel" (effondrement gravitationnel).

\parpic[l][t]{%
  \begin{minipage}{40mm}
    \fbox{\includegraphics[width=110px,height=140px]{img/medaillons/anonymous.eps}}
  \end{minipage}
}
\textbf{Laurent, Pierre Alphonse} (1813-1854) était un mathématicien français né à Paris et qui s'est rendu célèbre pour la découverte de la série de Laurent dans le domaine de l'analyse complexe et qui a un grand impact dans le calcul de certaines intégrales en physique. Il est entré à l'École Polytechnique de Paris en 1830. Laurent a été diplômé en 1832 comme un des meilleurs élèves de l'année et entra dans le corps d'ingénierie comme lieutenant. Pendant la gestion de ses projets de développement du port du Havre, Laurent écrivait sa première publication mathématique sur les séries de Laurent. Cette recherche était contenue dans un mémoire soumis au Grand prix de l'Académie des sciences en 1843, mais sa candidature étant trop tardive, l'article n'a jamais été inscrit au prix. Cependant, Cauchy fit une référence dans ses travaux au papier de Laurent 3 mois plus tard. Le même problème se réitéra pour une autre publication importante de Laurent quelques mois plus tard. Après ces événements, Laurent, déçu changea de domaine de recherche pour se concentrer sur la physique (mathématique appliquée). Cauchy lui propose un poste vacant à l'Académie des Sciences en 1846, mais sa candidature ne fut pas retenue. Laurent est décédé à Paris, à l'âge de 41 ans. Ses écrits n'ont été publiés qu'après son décès.

\parpic[l][t]{%
  \begin{minipage}{40mm}
    \fbox{\includegraphics[width=110px,height=140px]{img/medaillons/lavoisier.eps}}
  \end{minipage}
}
\textbf{Lavoisier, Antoine Laurent} (1743-1794) était un chimiste français, considéré comme le fondateur de la chimie moderne. Lavoisier naquit à Paris et fit ses études au collège Mazarin. Il fut élu membre de l'Académie des Sciences en 1768. Il occupa de nombreux postes, y compris celui de directeur des Poudreries Nationales en 1776, de membre de la Commission pour l'Établissement du Nouveau Système de Poids et Mesures en 1790 et de secrétaire de la Trésorerie en 1791. Il tenta d'introduire des réformes dans le système monétaire et fiscal français, ainsi que dans le système agricole. Lavoisier fut l'un des premiers à réaliser des expériences chimiques réellement quantitatives. Il montra qu'en dépit du changement d'état de la matière au cours d'une réaction chimique, la quantité de matière restait constante entre le début et la fin de chaque réaction. Ces expérimentations ont fourni des preuves en faveur de la loi de la conservation de la matière. Lavoisier fit également des recherches sur la composition de l'eau, dont il appela les composants "oxygène" et "hydrogène". L'une des plus importantes expériences de Lavoisier concerna la nature de la combustion (ou brûlage). Il démontra ainsi que le processus de combustion implique la présence d'oxygène. Il démontra également le rôle de l'oxygène dans la respiration chez les animaux et chez les végétaux. Les explications de Lavoisier sur la combustion remplacèrent la doctrine du phlogistique. Celle-ci postulait en effet qu'une substance se dégageait, le "phlogiston", lorsque la matière se consumait. Étant l'un des vingt-huit fermiers généraux, Lavoisier est bêtement stigmatisé comme traître par les révolutionnaires en 1794 et guillotiné lors de la terreur à Paris le 8 mai 1794, à l'âge de 50 ans, en même temps que l'ensemble de ses collègues.

\parpic[l][t]{%
  \begin{minipage}{40mm}
    \fbox{\includegraphics[width=110px,height=140px]{img/medaillons/lebesgue.eps}}
  \end{minipage}
}
\textbf{Lebesgue, Henri Léon} (1875-1941) Né à Beauvais est décédé à Paris est un ancien élève de l'École Normale Supérieur, il eut Émile Borel comme professeur (à qui l'on doit les premiers travaux importants en théorie de la mesure). Après quelques années au lycée de Nancy, Lebesgue enseignera à Rennes. C'est pendant cette période qu'il se fera connaître par son élégante théorie de la mesure. Professeur à la Sorbonne puis au collège de France, il sera élu à l'Académie des Sciences en 1922. Par sa théorie des fonctions mesurables (1901) s'appuyant sur les tribus boréliennes  (du nom du mathématicien Émile Borel), Lebesgue a profondément remanié et généralisé le calcul intégral. Sa théorie de l'intégration (1902-1904) répond aux besoins des physiciens en permettant la recherche et l'existence de primitives pour des fonctions "irrégulières". On lui doit aussi la transformée de Fourier établie dans la fin des années 1930. Il est nommé professeur à la Sorbonne en 1910, puis au Collège de France en 1921. Il donne également des cours à l'École Supérieure de Physique et de Chimie Industrielles de la ville de Paris de 1927 à 1937 et à l'École Normale Supérieure de Sèvres.

\parpic[l][t]{%
  \begin{minipage}{40mm}
    \fbox{\includegraphics[width=110px,height=140px]{img/medaillons/lee.eps}}
  \end{minipage}
}
\textbf{Lee, Tsung-Dao} (1926-) Né à Shanghai (Chine), Lee Tsung-Dao est le fils d'un homme d'affaires. La guerre sino-japonaise de 1937-1945 lui fit quitter l'Université Kweichow dans la province du Zhejiang, pour rejoindre celle de Kunming, dans le Yunnan, où il rencontra Yang Chen-Ning, dont il sera longtemps l'ami et le collaborateur. Une bourse du gouvernement chinois lui permit de terminer ses études à l'Université de Chicago (États-Unis), où il soutint sa thèse sur le contenu en hydrogène des naines blanches, en 1950. Membre de l'Institute for Advanced Study de Princeton (New Jersey) de 1951 à 1953, il devint bientôt, à 29 ans, le plus jeune professeur de l'Université Columbia, à New York.  En 1956, les physiciens étaient en butte à une énigme surgie du dépouillement des données fournies par l'accélérateur de particules du laboratoire national de Brookhaven, près de New York: deux particules, appelées "tau" et "thêta", semblaient avoir même masse et mêmes interactions nucléaires, mais différaient par leurs produits de désintégration. Lee et Yang proposèrent qu'elles n'étaient qu'une seule particule, maintenant notée "K0", et que l'interaction faible responsable de leur désintégration ne respectait pas la symétrie de parité. Ils en conclurent qu'il était indispensable de soumettre à vérification expérimentale le fait que l'interaction faible distingue la droite de la gauche. Six mois suffirent à l'équipe du National Bureau of Standards de Washington, mobilisée par la physicienne chinoise Wu Chien-Shiung, pour montrer que des noyaux radioactifs de Cobalt 60 polarisés émettaient plus d'électrons dans une direction que dans la direction opposée. Confirmée rapidement par plusieurs autres groupes expérimentaux, cette violation de la symétrie miroir valut à Lee Tsung-Dao et à Yang Chen-Ning de se partager le prix Nobel de physique 1957.

\parpic[l][t]{%
  \begin{minipage}{40mm}
    \fbox{\includegraphics[width=110px,height=140px]{img/medaillons/anonymous.eps}}
  \end{minipage}
}
\textbf{Legendre, Adrien Marie} (1752-1833) était un mathématicien français né à Paris et décédé à Auteil. Il occupe la chaire de Mathématiques de l'École Militaire de Paris de 1775 à 1780. En 1783, il devient membre de l'Académie des Sciences. En 1787, il est nommé commissaire chargé des opérations géodésiques. Les centres d'intérêts de Legendre étaient variés: analyse, théorie des nombres, géométrie, statistiques (méthodes des moindres carrés) et mécanique (transformation de Legendre en mécanique analytique et thermodynamique). Environ un siècle avant que l'on en obtienne les preuves, il conjectura le théorème des nombres premiers (distribution asymptotique des nombres premiers) ainsi que la loi de réciprocité quadratique (expression d'un nombre premier comme un carré modulo un autre nombre premier). Toute sa vie, il s'intéressa aux intégrales elliptiques, dont les travaux allaient finalement donner naissance aux courbes elliptiques, sujet très étudié par les mathématiciens contemporains. Il laisse en héritage à la communauté mathématique du 19ème siècle un traité de géométrie élémentaire, qui s'avère très précieux dans le monde de l'enseignement..

\parpic[l][t]{%
  \begin{minipage}{40mm}
    \fbox{\includegraphics[width=110px,height=140px]{img/medaillons/leibniz.eps}}
  \end{minipage}
}
\textbf{Leibniz, Gottfried Wilhelm} (1646-1716) Né à Leipzig et décédé à Hanovre était philosophe, juriste et mathématicien considéré comme un des plus brillants esprits du 17ème siècle. Fils d'un jurisconsulte il obtient son baccalauréat en philosophie ancienne en 1663 et écrira un peu plus tard une théorie des probabilités en Droit. Il entre ensuite à l'université de Leipzig et en 1666 obtient son doctorat en Droit... En 1669, il devient conseiller à la Chancellerie de l'Électorat de Mayence. Il est envoyé en 1672 à Paris, en mission diplomatique dit-on, pour convaincre Louis XIV de porter ses conquêtes vers l'Égypte plutôt que l'Allemagne. Il y reste jusqu'en 1676 et y rencontre les grands savants de l'époque. C'est pendant cette période que Leibniz travail sur ses travaux scientifiques. En 1676 il est nommé bibliothécaire du Brunswick-Lunebourg et s'y occupe aussi de mathématique, de physique, de religion et de diplomatie. Leibniz contribua aux mathématiques en découvrant, en 1675, les principes fondamentaux du calcul infinitésimal. Cette découverte fut réalisée indépendamment des découvertes de Newton, qui inventa son système de calcul en 1666. Le système de Leibniz fut publié en 1684, celui de Newton en 1687, date à laquelle la méthode de notation imaginée par Leibniz fut adoptée et on le considère aussi comme un pionnier du développement de la logique mathématique.

\parpic[l][t]{%
  \begin{minipage}{40mm}
    \fbox{\includegraphics[width=110px,height=140px]{img/medaillons/landau.eps}}
  \end{minipage}
}
\textbf{Landau, Lev Davidovich} (1908-1968) Né en Azerbaïdjan et décédé à Moscou, il était le fils d'un ingénieur et médecin. Après avoir achevé ses études au Département de Physique de l'Université de Léningrad à l'âge 19 ans, il commence sa carrière scientifique à l'Institut Physico-technique de Léningrad. De 1932 à 1937 il est le chef du Département Théorique de l'Institut Physico-technique Ukrainien à Kharkov et dès 1937 il est nommé chef du Département Théorique de l'Institut pour les Problèmes Physiques de l'Académie des Sciences de l'URSS à Moscou. Le travail de Landau couvre toutes les branches de physique théorique, aux limites de la mécanique liquide à la théorie des champs quantiques. Une grande partie de ses papiers se réfère à la théorie de l'état condensé. Ils ont commencé en 1936 par une formulation d'une théorie générale des transitions de phase du deuxième ordre. Après la découverte de Kapitsa, en 1938, de la superfluidité de l'hélium liquide, Landau a engagé la vaste recherche qui l'a mené à la construction de la théorie complète des liquides quantiques aux températures très basses. Parmi ses écrits, couvrant une vaste gamme de thèmes liés aux phénomènes physiques, on relève plus de cent articles et de nombreux livres, dont le célèbre \textit{Cours de physique théorique}, publié en 1943 avec E.M.Lifchitz. Landau a dominé toute la physique théorique de 1930 à 1965. Il avait créé un ensemble d'examens de physique théorique, appelé le "Minimum théorique" que les étudiants ou chercheurs confirmés devaient  passer pour entrer dans son groupe de recherche, examen qui incluait des problèmes dans toutes les branches des mathématiques.

\parpic[l][t]{%
  \begin{minipage}{40mm}
    \fbox{\includegraphics[width=110px,height=140px]{img/medaillons/levicivita.eps}}
  \end{minipage}
}
\textbf{Levi-Civita, Tullio} (1873-1941) Né à Padoue et décédé à Rome il fut diplômé en 1892 de la faculté de mathématiques de l'Université de Padoue. En 1894, il obtint un diplôme d'enseignement au Collège d'enseignement de la faculté des sciences de Pavie. En 1898, il fut nommé à la tête de la chaire de mécanique analytique et céleste de Padoue et il y rencontra Libera Trevisani, une de ses élèves, avec qui il se maria en 1914. Il resta à Padoue jusqu'en 1918, puis fut nommé à la chaire d'analyse supérieure à l'Université de Rome, où il prit 2 ans plus tard la chaire de mécanique. Avant tout physicien, ses travaux s'orientent principalement vers l'électromagnétisme et les théories de Lorentz et de Maxwell. En 1900, Ricci et lui publièrent \textit{La Théorie des tenseurs dans les méthodes de calcul différentiel et leurs applications} qu'Einstein utilisa afin de mieux maîtriser le calcul tensoriel, un outil-clef pour Einstein dans le développement de sa théorie de la Relativité Générale. Levi-Civita discuta aussi d'une série de problèmes à propos du champ gravitationnel statique dans sa correspondance avec Einstein entre les années 1915-1917. Leur correspondance tournait autour de la formulation variationnelle des équations de champs gravitationnelles et leurs propriétés covariantes, et la définition de l'énergie gravitationnelle et de l'existence d'ondes gravitationnelles. En 1933 Levi-Civita contribua aussi aux équations de la mécanique quantique de Dirac.

\parpic[l][t]{%
  \begin{minipage}{40mm}
    \fbox{\includegraphics[width=110px,height=140px]{img/medaillons/lie.eps}}
  \end{minipage}
}
\textbf{Lie, Sophus} (1842-1899) était un mathématicien norvégien qui fit ses études à l'Université de Christiana. Il donna des leçons particulières pour gagner sa vie, et passa avec Klein l'hiver 1869-1870 à Berlin, l'été 1870 à Paris. En 1872, une chaire de mathématiques fut créée pour lui à Christiana, et en 1886, il succéda à Klein à Leipzig. Outre des travaux en géométrie projective de l'espace, on retient surtout de Lie l'étude de structures algébriques nouvelles qu'il applique à la géométrie, jusqu'à la création de toutes pièces de la théorie des groupes et algèbres qui portent son nom. Dans la notion de groupe et d'algèbre de Lie, interviennent des propriétés de continuité (groupe topologique), annonçant la nouvelle branche importante des mathématiques que sera la topologie. Les travaux de Lie, dans ce domaine, seront principalement poursuivis par Élie Cartan.

\parpic[l][t]{%
  \begin{minipage}{40mm}
    \fbox{\includegraphics[width=110px,height=140px]{img/medaillons/lindemann.eps}}
  \end{minipage}
}
\textbf{Lindemann, Ferdinand} (1852-1939) Né à Hanovre et décédé à Münich Lindemann a été le premier mathématicien à démontrer la transcendance de $\pi$. Quand Ferdinand était âgé de 2 ans, il se déplaça à Schwerin où il passa ses années d'enfance et sa scolarité primaire. Comme il était de pratique à cette époque en Allemagne pendant la seconde moitié du 19ème, Lindemann se déplaça fréquemment d'une université à l'autre. Il commença ses études à Göttingen en 1870 et y fut grandement influence par Clebsch. Plus tard, Lindemann qui avait établi des très bonnes relations avec Clebsch rédigea à nouveau les notes de géométrie de ce dernier après son décès pour leur publication en 1876. Ensuite, Lindemann étudia à Erlangen à Münich où il effectua son travail de doctorat sous la direction de Klein sur les géométries non-euclidiennes et ses applications à la physique. Après avoir obtenu son doctorat, Lindemann fit des visites importantes à des centres de mathématiques anglais et français. En Angleterre, il visita Oxford, Cambridge et Londres, alors qu'en France, il passa la majeure partie de son temps à Paris où il fut grandement influencé par Chasles, Bertrand, Jordan et Hermite. Lorsqu'il retourna en Allemagne, Lindemann travailla sur des publications permettant sa réintégration et sa reconnaissance dans le domaine scientifique Allemand. Ce fut enfin en 1877 qu'il fut nominé professeur extraordinaire à l'Université de Würzburg et professeur ordinaire à l'université de Freiburg en 1879. Le principal travail de Lindemann porta sur la géométrie et l'analyse. En 1873, alors que Lindemann venait d'avoir obtenu son doctorat, Hermite démontra la transcendance du nombre d'Euler $e$. Peu de temps après, Lindemann rencontra Hermite à Paris et discuta des méthodes utilisées pour la démonstration. Ainsi, utilisant un raisonnement similaire, Lindemann démontra en 1882 la transcendance de $\pi$ (sur la base que le nombre d'Euler est lui-même transcendant).

\parpic[l][t]{%
  \begin{minipage}{40mm}
    \fbox{\includegraphics[width=110px,height=140px]{img/medaillons/liouville.eps}}
  \end{minipage}
}
\textbf{Liouville, Joseph} (1809-1882) Né à St-Omer et décédé à Paris il fut un artisan des mathématiques déployant une activité considérable dans l'enseignement et la diffusion des idées mathématiques de son temps. Il est le fondateur du \textit{Journal de mathématiques pures et appliquées} appelé traditionnellement "Journal de Liouville". Ses principaux travaux portent sur l'analyse et on lui doit un important théorème sur l'approximation des irrationnels algébriques. L'élection de Joseph Liouville à l'Assemblée constituante de 1848 est seule à rompre l'unité d'une carrière toute scientifique: sorti de l'École Polytechnique en 1827, il y revenait en 1833 comme répétiteur puis professeur d'analyse. Dès sa 31ème année, il était élu à l'Académie des Sciences, dans la section d'astronomie. Il fut un des meilleurs professeurs de son temps, et ses cours, à Polytechnique et au Collège de France, prirent une grande part de son activité. Liouville fonda le \textit{Journal de mathématiques pures et appliquées} en 1836 et le géra pendant 39 ans. Ses tâches d'académicien et d'éditeur lui ôtèrent la liberté d'esprit nécessaire à une recherche approfondie ce pour quoi il se plaint. Mais il mit à profit l'une et l'autre tâche pour aider plusieurs jeunes mathématiciens de grand avenir, par exemple C. Hermite et C. Jordan, par des rapports élogieux devant l'Académie, ou par la publication de leurs travaux dans son journal. Quant à lui, il y publia surtout de courtes notes sur un grand nombre de questions: analyse, arithmétique, géométrie, mécanique, astronomie. Il partage avec A. Cauchy le mérite d'avoir soumis l'analyse à une règle de rigueur souvent transgressée au 18ème siècle, et ce mérite est d'autant plus grand que le langage mathématique de son temps n'aidait guère à la rigueur.

\parpic[l][t]{%
  \begin{minipage}{40mm}
    \fbox{\includegraphics[width=110px,height=140px]{img/medaillons/lobatchevski.eps}}
  \end{minipage}
}
\textbf{Lobachevsky, Nikolai Ivanovich} (1792-1856) était un mathématicien russe né à Nijni-Novgorod et décédé à Kazan. Lobatchevski étudia à l'Université de Kazan, où il enseigna à partir de 1812 et occupa la chaire de mathématiques pures de 1822 à 1846. Sous l'influence de Gauss et de Laplace, ses premiers travaux sont: \textit{Théorie du mouvement elliptique des corps célestes} et \textit{De la solution de l'équation algébrique complexe simple}. Mais ses principales recherches concernent la géométrie. Son premier ouvrage, \textit{Géométrie} (1823), jugé trop révolutionnaire (il utilisait le système métrique), ne pourra être publié de son vivant. En 1826, Lobatchevski expose devant ses collègues de l'université un mémoire qui montre qu'il fut l'un des premiers mathématiciens à être convaincu de la possibilité d'une géométrie différente de celle d'Euclide. Malgré le scepticisme de ses collègues, il continue l'étude de cette nouvelle géométrie (où le postulat d'Euclide est remplacé par le postulat suivant, dit "postulat de Lobatchevski": par tout point extérieur à une droite, il passe une infinité de parallèles à cette droite) et consacre sa vie de mathématicien à essayer de convaincre le monde scientifique. Il publie successivement \textit{Éléments de géométrie} (1829), \textit{Nouveaux Éléments de géométrie avec la théorie complète des parallèles} (1838) et \textit{Pangéométrie} (1855). Mais la pleine reconnaissance de la valeur de ses travaux ne viendra qu'après sa mort (lorsque Eugenio Beltrami, en 1868, construira un modèle de la géométrie de Lobatchevski: la pseudo-sphère). En plus de ses recherches mathématiques, Lobatchevski fut l'animateur de l'Université de Kazan: recteur de 1827 à 1846, il eut la charge de la bibliothèque de l'université, mit en place son observatoire, organisa son muséum et dirigea la construction de nouveaux locaux universitaires.

\parpic[l][t]{%
  \begin{minipage}{40mm}
    \fbox{\includegraphics[width=110px,height=140px]{img/medaillons/lorentz.eps}}
  \end{minipage}
}
\textbf{Lorentz, Hendrik} (1853-1928) Né à Arnhem et décédé à Haarlem (Pays-Bas) il avait amélioré la théorie électromagnétique de Maxwell dans sa thèse doctorale sur la théorie de la réflexion et la réfraction de la lumière qu'il présenta en 1875. Il fut nommé professeur de physique-mathématique à l'Université de Leyde en 1878. Il est resté dans cet établissement jusqu'en 1912 où Ehrenfest a été nommé à sa place. Lorentz est ensuite nommé directeur de recherche à l'Institut de Teyler, Haarlem. Il a maintenu une position honorifique à Leyde, où il a continué à donner quelques cours. Avant que l'existence des électrons ait été confirmée, Lorentz a proposé que les vagues de lumière étaient dues aux oscillations d'une charge électrique dans l'atome. Lorentz a développé sa théorie mathématique de l'électron pour lequel il a reçu le prix Nobel en 1902. Le prix Nobel a été attribué conjointement à Lorentz et à Pieter Zeeman, un étudiant de Lorentz. Zeeman avait vérifié expérimentalement le travail théorique de Lorentz sur la structure atomique, démontrant l'effet d'un champ magnétique fort sur les oscillations en mesurant le changement de la longueur d'onde de la lumière produite. Lorentz est également célèbre pour son travail sur la contraction de Fitzgerald-Lorentz, qui est une contraction dans la longueur d'un objet aux vitesses relativistes. Les transformations de Lorentz, qu'il a présentées en 1904, forment la base de la théorie de Relativité Restreinte d'Einstein qui a l'époque était appelée "théorie de la relativité d'Einstein-Lorentz". Elles décrivent l'augmentation de la masse, du rapetissement de la longueur, et de la dilatation de temps d'un corps se déplaçant aux vitesses proches de celle de la lumière. Lorentz était président de la première conférence de Solvay tenue à Bruxelles en automne de 1911. Cette conférence avait pour sujet les deux approches de la théorie atomique, à savoir la théorie classique et la physique quantique. Cependant, Lorentz n'a jamais entièrement accepté la théorie quantique et a toujours espéré qu'il serait possible de l'incorporer de nouveau dans l'approche classique.

\parpic[l][t]{%
  \begin{minipage}{40mm}
    \fbox{\includegraphics[width=110px,height=140px]{img/medaillons/lucas.eps}}
  \end{minipage}
}
\textbf{Lucas, Edward} (1842-1891) Né à Guildford et décédé à Bath est un pasteur anglican, qui s'inquiéta de la croissance trop importante de la population en Angleterre au début de la révolution industrielle (de 1750 à 1900). Sa crainte tournait autour de l'idée que la progression démographique est plus rapide que l'augmentation des ressources, d'où une paupérisation de la population. Les anciens régulateurs démographiques (les guerres et les épidémies) ne jouant plus leurs rôles, il imagine de nouveaux obstacles, comme la limitation de la taille des familles et le recul de l'âge du mariage. Ces propositions ne sont appliquées à ce jour, toutes les deux, qu'en Chine, qui est en effet obligée de limiter sévèrement sa démographie. Les prévisions sinistres de Malthus sont dans la réalité mises à mal, car il n'imaginait pas une si grande augmentation des ressources et des rendements agricoles (révolution verte: chimie appliquée à l'agronomie ce qui n'est pas pour autant bénéfique...); les nouveaux moyens d'échanges internationaux de biens de subsistance (contribuant à la pollution des océans, au déforestage et au passage à la paupérisation des régions productrices...); le fait que le trop plein d'individus émigrerait vers les États-Unis ou les colonies. En revanche, si les prévisions de Malthus ne sont pas au rendez-vous, sa théorie garde tous ses droits. Il est exact que la population est en croissance dans certains pays (Arabie Saoudite: six enfants par femme) il est aussi exact (et heureux) que les progrès de l'hygiène et de la médecine augmentent la taille de la population, il est exact que les ressources renouvelables sur Terre sont limitées, in fine par l'énergie solaire que reçoit celle-ci, qui elle-même détermine la biomasse, sauf découverte scientifique majeure... et dans ces conditions, la mathématique est formelle: il ne sera pas possible à la population Terrestre d'augmenter indéfiniment, et la régulation devra intervenir à un moment ou à un autre, et d'une manière ou d'une autre (ne serait que par la pollution non gérée par certaines populations dont le sens des responsabilités et des priorités est très relatif...)!

\phantomsection
\addcontentsline{toc}{section}{M}
\label{sec:M}

\parpic[l][t]{%
  \begin{minipage}{40mm}
    \fbox{\includegraphics[width=110px,height=140px]{img/medaillons/malthus.eps}}
  \end{minipage}
}
\textbf{Malthus, Thomas Robert} (1766-1834) Né à Guildford et décédé à Bath est un pasteur anglican, qui s'inquiéta de la croissance trop importante de la population en Angleterre, au début de la révolution industrielle (de 1750 à 1900). Sa crainte tournait autour de l'idée que la progression démographique est plus rapide que l'augmentation des ressources, d'où une paupérisation de la population. Les anciens régulateurs démographiques (les guerres et les épidémies) ne jouant plus leurs rôles, il imagine de nouveaux obstacles, comme la limitation de la taille des familles et le recul de l'âge du mariage. Ces propositions ne sont appliquées à ce jour, toutes les deux, qu'en Chine, qui est en effet obligée de limiter sévèrement sa démographie. Les prévisions sinistres de Malthus sont dans la réalité mises à mal, car il n'imaginait pas une si grande augmentation des ressources et des rendements agricoles (révolution verte: chimie appliqué à l'agronomie ce qui n'est pas pour autant bénéfique...); les nouveaux moyens d'échanges internationaux de biens de subsistance (contribuant à la pollution des océans au passage...); le fait que le trop plein d'individus émigrerait vers les États-Unis ou les colonies. En revanche, si les prévisions de Malthus ne sont pas au rendez-vous, sa théorie garde tous ses droits. Il est exact que la population est en croissance dans certains pays (Arabie Saoudite: 6 enfants par femme) il est aussi exact (et heureux) que les progrès de l'hygiène et de la médecine augmentent la taille de la population, il est exact que les ressources renouvelables sur Terre sont limitées, in fine par l'énergie solaire que reçoit celle-ci, qui elle-même détermine la biomasse, sauf découverte scientifique majeure... et dans ces conditions, la mathématique est formelle: il ne sera pas possible à la population Terrestre d'augmenter indéfiniment, et la régulation devra intervenir à un moment ou à un autre, et d'une manière ou d'une autre!

\parpic[l][t]{%
  \begin{minipage}{40mm}
    \fbox{\includegraphics[width=110px,height=140px]{img/medaillons/marconi.eps}}
  \end{minipage}
}
\textbf{Marconi, Guglielmo} (1874-1937) Né et décédé à Rome, c'était un physicien, inventeur et homme d'affaires italien. Il est, avec Karl Ferdinand Braun, colauréat du prix Nobel de physique de 1909 en reconnaissance de leurs contributions au développement de la télégraphie sans fil (on peut considérer qu'il est l'origine des appareils de transmission/réception d'ondes électromagnétiques et donc de la radio et télévision hertzienne). Marconi est né dans une famille aisée, second fils de Giuseppe Marconi, un propriétaire italien, et d'une mère irlandaise, Annie Jameson, petite-fille du fondateur de la Distillerie Jameson Whiskey. Il a fait ses études à Bologne dans le laboratoire d'Augusto Righi, à Florence, à l'Institut Cavallero et, plus tard, à Livourne. Il fait en 1985 des expériences sur les ondes découvertes par Heinrich Rudolf Hertz sept ans auparavant. Il reproduit le matériel utilisé par Hertz en l'améliorant avec un cohéreur de Branly pour augmenter la sensibilité et l'antenne de Alexandre Popov. Après ses toutes premières expériences en Italie, il réalise dans les Alpes suisses à Salvan une liaison de 1.5 [km] durant l'été 1895. L'année d'après, faute d'être suivi par ses compatriotes, il part pour l'Angleterre, poursuit ses expériences et dépose un brevet. En 1897 il effectue la première communication en morse à plus de 13 [km] entre Lavernock (Pays de Galles) et Brean (Angleterre) par-dessus le Canal de Bristol. L'année d'après, il ouvre la première usine de radios au monde, à Chelmsford, Angleterre. Au début du 20ème siècle le nom Marconi est (malheureusement) plutôt connu comme étant le propriétaire du groupe de cinéma Pathé (qui s'appelle en réalité Pathé-Marconi).

\parpic[l][t]{%
  \begin{minipage}{40mm}
    \fbox{\includegraphics[width=110px,height=140px]{img/medaillons/mandelbrot.eps}}
  \end{minipage}
}
\textbf{Mandelbrot, Benoit} (1924-2010) est né à Varsovie et décédé à Cambridge. Sa famille a quitté la Pologne pour Paris afin de fuir la menace hitlérienne. C'est à Paris qu'il fut initié aux mathématiques par deux oncles dont un était professeur au Collège de France. L'invasion allemande force la famille à se réfugier ensuite à Brive-la-Gaillarde. Après avoir fréquenté le lycée Edmond-Perrier de Tulle, il poursuit ses études au lycée du Parc, à Lyon. Après avoir quitté l'École polytechnique (promotion 1944), où il a suivi les cours d'un spécialiste du calcul des probabilités (Paul Lévy), il s'intéresse aux phénomènes d'information, les idées de Claude Shannon étant alors en plein essor. Mandelbrot fit ses études en France et aux États-Unis et obtint son doctorat de mathématiques à l'Université de Paris en 1952. Il enseigna l'économie à l'Université Harvard, l'ingénierie à Yale, la physiologie à la faculté de médecine et la mathématique à Paris et à Genève. À partir de 1958, il travailla pour IBM au centre de recherche Thomas B. Watson à New York sur la transmission optimale dans les milieux bruités, tout en poursuivant son travail sur des objets étranges jusque-là assez négligés par les mathématiciens : les objets à complexité récursivement définie, comme la courbe de Von Koch, auxquels il pressent une unité: la géométrie fractale. La géométrie fractale se distingue par son approche plus abstraite de la dimension qu'elle ne l'est dans la géométrie traditionnelle. Elle trouve de plus en plus d'applications dans différents domaines de la science et de la technologie.



\parpic[l][t]{%
  \begin{minipage}{40mm}
    \fbox{\includegraphics[width=110px,height=140px]{img/medaillons/markov.eps}}
  \end{minipage}
}
\textbf{Markov, Andrei Andreyevich} (1856-1922) était un mathématicien russe spécialiste de la théorie des nombres, de la théorie des probabilités et de l'analyse mathématique né à Riazan et décédé à Petrograd. Issu d'une la famille d'un petit fonctionnaire du gouvernement, il fait ses études à l'Université de Saint-Pétersbourg et reçoit une médaille d'or pour son mémoire \textit{De l'intégration des équations différentielles par la méthode des fractions continues} (1878). Professeur à l'Université de Saint-Pétersbourg en 1886, il devient membre de l'Académie des Sciences en 1896. Les recherches de Markov continuent l'oeuvre de ses devanciers de l'école mathématique pétersbourgeoise: P. L. Tchebychev, E. I. Zolotarev et A. N. Korkin. Sa thèse \textit{Des formes quadratiques bilinéaires de déterminant positif} (1880) inaugure ses travaux dans le domaine de la théorie des nombres. En analyse, ses recherches concernent les fractions continues, les limites d'intégrales, la convergence des séries et la théorie de l'approximation. On lui doit une solution simple de la détermination de la limite supérieure de la dérivée d'un polynôme (inégalité de Markov). Après 1910, se tournant vers la théorie des probabilités, il démontre de façon rigoureuse, sous des conditions assez générales, le théorème central limite relatif à une somme de variables aléatoires indépendantes identiquement distribuées. Cherchant à généraliser ce théorème aux variables aléatoires dépendantes, il est amené à considérer la notion importante d'événements en chaînes, appelés depuis chaînes de Markov, et il établit une série de lois, fondement de la théorie des processus de Markov. Il étend plusieurs résultats classiques concernant des événements indépendants à certains types de chaînes. Ses travaux sont à l'origine de la théorie moderne des processus stochastiques. Markov s'intéressait aussi aux applications de la théorie des probabilités, et il a justifié de façon probabiliste la méthode des moindres carrés.

\parpic[l][t]{%
  \begin{minipage}{40mm}
    \fbox{\includegraphics[width=110px,height=140px]{img/medaillons/markowitz.eps}}
  \end{minipage}
}
\textbf{Markowitz, Harry Maurice} (1927 -) Né à Chicago, professeur à la City University de New York est connu pour avoir développé la théorie dite du "choix des portefeuilles pour le placement des fortunes". Markowitz ne se doutait pas que son article de jeunesse publié en 1952 dans le \textit{Journal of Finance}, puis développé dans un livre paru en 1959, \textit{Portofolio Selection: Efficient diversification}, jetterait les bases de la théorie moderne du portefeuille et de son utilisation par un grand nombre de praticiens. Plus précisément, Markowitz a montré que l'investisseur cherche à optimiser ses choix en tenant compte non seulement de la rentabilité attendue de ses placements, mais aussi du risque de son portefeuille qu'il définit mathématiquement par la variance de sa rentabilité. Appliquant des théorèmes classiques du calcul statistique et des techniques probabilistes, il a ainsi démontré qu'un portefeuille composé de plusieurs titres est toujours moins risqué qu'un portefeuille composé d'un seul titre, quand bien même il s'agirait du moins risqué d'entre eux. La mise en oeuvre du modèle de Markowitz a très vite posé des problèmes d'ordre pratique. Alors que le volume des statistiques nécessaires au calcul augmentait rapidement avec le nombre de titres retenus (avec $100$ titres, le nombre de statistiques nécessaires était de $3'150$, mais il passait à $20'300$ pour $200$ titres et à $125'750$ pour $300$ titres !), la collecte des informations et leur traitement devenaient presque impossibles avec les ordinateurs disponibles dans les années 1960, entraînant de surcroît des coûts de traitement prohibitifs. C'est la raison pour laquelle William F. Sharpe cherchera une méthode de sélection des portefeuilles efficients plus simple. Markowitz et Sharpe seront alors reconnus comme les pères fondateurs de la gestion de portefeuilles et du corps doctrinal sur lequel elle se fonde. Le prix Nobel de sciences économiques leur sera décerné ainsi qu'à Merton Miller en 1990.

\parpic[l][t]{%
  \begin{minipage}{40mm}
    \fbox{\includegraphics[width=110px,height=140px]{img/medaillons/markx.eps}}
  \end{minipage}
}
\textbf{Marx, Karl} (1818-1883) Né à Trèves et décédé à Londres il entra à l'Université de Bonn puis à celle de Berlin, après avoir terminé le Lycée de Trèves. Il étudia à Berlin le droit, mais surtout l'histoire et la philosophie. Marx a ensuite contribué à parachever les trois principaux courants d'idées du 19ème siècle: la philosophie classique allemande, l'économie politique classique anglaise et le socialisme français. La théorie sociale de Marx a pour objectif de dévoiler la loi économique de la société capitaliste où ce qui domine est la production des marchandises en recherchant l'origine de la forme monétaire de la valeur. Ainsi pour Marx, l'argent (en tant que produit suprême du développement de l'échange et de la production marchande) estompe et dissimule le caractère et le lien social du travail individuel. À un certain degré du développement de la production des marchandises, l'argent se transforme aussi en capital. Ainsi, la séquence de la circulation des marchandises était: M (marchandise) - A (argent) - M (marchandise), c'est-à-dire vente d'une marchandise pour l'achat d'une autre. La séquence générale du capital est par contre A-M-A, c'est-à-dire l'achat pour la vente (avec un profit). C'est cet accroissement de la valeur primitive de l'argent, donc sa transformation en capital, que Marx appelle "plus-value" et qui ne peut provenir de la circulation des marchandises, car celle-ci ne connaît que l'échange d'équivalents; elle ne peut provenir non plus d'une majoration des prix, étant donné que les pertes et les profits réciproques des acheteurs et des vendeurs s'équilibrent à grande échelle. Pour obtenir de la plus-value il faut selon Marx une marchandise dont le processus de consommation fût en même temps un processus de création de valeur. Or, cette marchandise est la force de travail humaine. Le possesseur d'argent achète la force de travail à sa valeur, déterminée, comme celle de toute autre marchandise, par le temps de travail socialement nécessaire à sa production. Ayant acheté la force de travail, le possesseur d'argent est en droit de la consommer, c'est-à-dire de l'obliger à travailler toute la journée, disons, $8$ heures. Or, en $5$ heures (temps de travail nécessaire), l'ouvrier crée un produit qui couvre les frais de son entretien, et, pendant les $3$ autres heures (temps de travail supplémentaire), il crée un produit supplémentaire, non rétribué par le capitaliste, et qui est la plus-value. Aussi, pour exprimer le degré d'exploitation de la force de travail par le capital, faut-il comparer la plus-value non pas au coût total de production, mais uniquement au coût variable de la main d'oeuvre humaine.

\parpic[l][t]{%
  \begin{minipage}{40mm}
    \fbox{\includegraphics[width=110px,height=140px]{img/medaillons/maxwell.eps}}
  \end{minipage}
}
\textbf{Maxwell, James Clerk} (1831-1879) est né à Edimbourg et décédé à Glenlair (Écosse). Brillant élève au collège, James Clerk Maxwell poursuit des études de mathématiques à l'Université de Cambridge. Il obtient une chaire de philosophie naturelle à Aberdeen à l'âge de 25 ans. Puis, de 1860 à 1865, il occupe le poste de professeur au King's College de Londres. À la suite de ces 5 années d'enseignement, il décide de se retirer dans sa propriété de Glenair, en Écosse. Il y restera 5 autres années qu'il emploiera à étudier. En 1871, Maxwell est nommé directeur du laboratoire Cavendish que vient de fonder le duc du Devonshire. Il n'aura alors de cesse de le développer afin qu'il devienne le centre de formation scientifique le plus illustre. Dès le début de sa carrière, Maxwell s'intéresse à la dynamique des gaz. Après avoir prouvé mathématiquement que les anneaux de Saturne sont constitués de particules distinctes, il étudie la répartition des vitesses des molécules gazeuses (conforme à loi de Gauss). En 1860, il montre que l'énergie cinétique de ces molécules ne dépend que de leur nature. Mais ce sont ses recherches en électromagnétisme qui font de Maxwell un des savants les plus célèbres du 19ème siècle. En se basant sur les travaux de Faraday, il introduit dès 1862 la notion de champ. Puis, il montre qu'un champ magnétique peut être créé par la variation d'un champ électrique (Faraday avait alors découvert l'induction, phénomène par lequel la variation d'un champ électrique crée un champ magnétique). Son enseignement purement mathématique va alors lui permettre d'élaborer les célèbres équations différentielles décrivant la nature des champs électromagnétiques dans l'espace et le temps. Il les expose dans son \textit{Traité d'électricité et de magnétisme} publié en 1873. Maxwell, en élaborant les théories de l'électromagnétisme, a également défini la lumière en tant qu'onde électromagnétique, ouvrant ainsi la voie aux recherches d'autres physiciens comme Heinrich Rudolph Hertz.

\parpic[l][t]{%
  \begin{minipage}{40mm}
    \fbox{\includegraphics[width=110px,height=140px]{img/medaillons/mcfadden.eps}}
  \end{minipage}
}
\textbf{McFadden, Daniel} (1937 -) Né à Raleigh (Caroline du Nord) est un économétricien ayant reçu en 2000, avec James Heckman, le prix Nobel d'économie pour son apport aux théories et méthodes de l'analyse des choix discrets. Il obtient un Bachelor de Science en physique à l'âge de 19 ans à l'Université du Minnesota puis un doctorat de philosophie en sciences du comportement (économie) 5 ans plus tard en 1962. En 1964, il intègre l'Université de Berkeley et focalise ses recherches sur les comportements de choix, et sur les liens entre la théorie économique et les mesures économiques. En 1975, il est récompensé par la médaille John Bates Clark. En 1977, il se rend au Massachusetts Institute of Technology (M.I.T.), mais retourne à Berkeley en 1991, car le M.I.T. n'avait pas de département de statistiques. Après son retour, il fonde le laboratoire d'économétrie, qui est dévoué à l'informatique statistique appliquée à l'économie. McFadden a développé en microéconométrie des théories et des méthodes d'analyse des comportements par choix discrets (par exemple, les données sur les métiers et les lieux de résidence des individus) et est connu pour être à l'origine du coefficient pseudo-R de la régression logistique probit. À partir de sa théorie économique sur les choix discrets, McFadden a développé de nouvelles méthodes statistiques qui ont eu une influence décisive sur la recherche théorique, mais qui sont aussi largement utilisées par le marketing.

\parpic[l][t]{%
  \begin{minipage}{40mm}
    \fbox{\includegraphics[width=110px,height=140px]{img/medaillons/meitner.eps}}
  \end{minipage}
}
\textbf{Meitner, Lise} (1878-1960) était une physicienne née à Vienne et décédée à Cambridge. En 1899, Lise commença une préparation accélérée de 2 ans pour son entrée à l'Université, afin de se présenter à cet examen. Elle fut reçue et entra à l'Université de Vienne en 1901, à l'âge de 22 ans. Après la première année, au cours de laquelle Lise suivit de nombreux cours en physique, chimie, mathématiques et botanique, elle se concentra sur la physique. Dès la seconde année, elle choisit de suivre tous les cours donnés par Ludwig Boltzmann ; cela témoigne de la fascination que ce grand physicien théoricien exerçait sur ses étudiants, avec qui il développait des liens intellectuels mais aussi personnels. Elle obtint son doctorat en 1905. Lise demeura à Vienne durant l'année qui suivit son doctorat. En tant que femme, elle ne pouvait espérer une carrière académique, mais continua malgré tout la recherche. Elle rencontra Paul Ehrenfest, ancien étudiant de Boltzmann, qui attira son attention sur les articles publiés par Lord Rayleigh. L'un d'eux décrivait un effet d'optique que Rayleigh ne parvenait pas à expliquer. Lise trouva l'explication théorique et en dériva de nouvelles observations. Lise partit pour Berlin en 1907 afin de suivre les cours de Max Planck. Lise et Otto Hahn étudièrent la radioactivité et ils devinrent réputés pour leurs travaux, notamment pour la découverte du protactinium en 1918. Indépendamment de ses travaux avec Hahn, Lise mena des recherches pionnières en physique nucléaire. Elle se consacra d'abord à l'étude des spectres de rayonnements bêta et gamma. En 1923, elle découvrit ainsi la transition non-radiative connue comme l'effet Auger, appelé en l'honneur de Pierre Auger, un scientifique français qui le découvre indépendamment 2 ans plus tard. Elle découvrit également l'émission de paires électron-positron lors de la désintégration bêta plus. Elle effectua différentes mesures de la masse du neutron. En 1939 elle joue un rôle majeur dans la découverte de la fission nucléaire, dont elle fournit avec son neveu Otto Frisch la première explication théorique en 1939 en employant le modèle de la goutte liquide de Niels Bohr. Raison pour laquelle elle est considérée comme la "mère de la bombe nucléaire" par les médias de l'époque.

\parpic[l][t]{%
  \begin{minipage}{40mm}
    \fbox{\includegraphics[width=110px,height=140px]{img/medaillons/mendel.eps}}
  \end{minipage}
}
\textbf{Mendel, Gregor Johann} (1822-1884) Né à Heinzendorf et décédé à Brno, a été moine dans le monastère de Brno (en Moravie). Mendel est communément reconnu comme un botaniste père fondateur de la génétique. Il est à l'origine de ce qui est aujourd'hui appelé les "lois de Mendel", qui définissent la manière dont les gènes se transmettent de génération en génération. Mendel naît dans une famille de paysans. Doué pour les études, mais de tendance dépressive qui lui vaudra de multiples indispositions dans la suite de sa carrière, le jeune garçon est très vite remarqué par le curé du village qui décide de l'envoyer poursuivre ses études loin de chez lui. Mendel part en 1851 pour suivre les cours, en tant qu'auditeur libre, de l'Institut de Physique de Christian Doppler. Il y étudie, en plus des matières obligatoires: la botanique, la physiologie végétale, l'entomologie et la paléontologie. Durant 2 années, il acquiert les bases méthodologiques qui lui permettront de réaliser plus tard ses expériences. Au cours de son séjour à Vienne, Mendel est amené à s'intéresser aux théories de Franz Unger, professeur de physiologie végétale. Celui-ci préconise l'étude expérimentale pour comprendre l'apparition des caractères nouveaux chez les végétaux au cours de générations successives. Il espère ainsi résoudre le problème que pose l'hybridation chez les végétaux. De retour au monastère, Mendel installe un jardin expérimental dans la cour et dans la serre, en accord avec son abbé, et met sur pied un plan d'expériences visant à comprendre les lois de l'origine et de la formation des hybrides. Il choisit pour cela le pois qui a l'avantage d'être facilement cultivable avec de nombreuses variétés décrites. En 1865, il expose à la Société des Sciences Naturelles de Brünn et publie en 1866 les résultats de ses études. Après 10 années de travaux minutieux, Mendel a ainsi posé les bases théoriques de la génétique et de l'hérédité moderne. Son travail ne va pas susciter d'enthousiasme auprès de ses contemporains qui ont du mal à comprendre la formalisation mathématique de ses expériences. Très peu de scientifiques de son temps vont citer son travail et Mendel ne reçoit guère de réponses auprès des différents correspondants à qui il envoie un tiré-à-part. Parmi ces derniers, seul Karl Wilhelm von Nägeli, professeur de botanique à Münich, lui écrit, doutant d'ailleurs de certaines de ses conclusions. En 1868, Mendel est élu supérieur de son couvent à la mort de l'abbé.

\parpic[l][t]{%
  \begin{minipage}{40mm}
    \fbox{\includegraphics[width=110px,height=140px]{img/medaillons/mendeleiev.eps}}
  \end{minipage}
}
\textbf{Mendeleev, Dmitri Ivanovich} (1834-1907) Né à Tobolsk et décédé à Saint-Pétersbourg, c'était chimiste russe surtout connu pour sa classification périodique des éléments publiée en 1869. Il montra en effet que les propriétés chimiques des éléments dépendaient directement de leur poids atomique et qu'elles étaient des fonctions périodiques de ce poids. Il entre à l'âge de 14 ans au lycée de Tobolsk, après la mort de son père. En 1849, la famille devenue pauvre s'installe à Saint-Pétersbourg et Mendeleïev entre à l'Université en 1850. Après avoir reçu son diplôme, il contracte la tuberculose ce qui l'oblige à se déplacer dans la péninsule criméenne près de la Mer Noire en 1855, où il devient responsable des sciences du lycée local. Il revient complètement guéri à Saint-Pétersbourg en 1856 où il étudie la chimie et fut diplômé en 1856. À 25 ans, il vient travailler à Heidelberg avec des savants comme Robert Bunsen et Gustav Kirchhoff. À Heidelberg, il rencontra le chimiste italien Stanislao Cannizzaro, dont les idées sur le poids atomique influencèrent sa réflexion. Mendeleïev retourna à Saint-Pétersbourg et enseigna la chimie à l'Institut Technique en 1863. En 1864, il soutient sa thèse de doctorat intitulée \textit{Considérations sur la combinaison de l'alcool et de l'eau}. En 1867, il est nommé professeur de chimie minérale (toujours à l'Université de Saint-Pétersbourg) et y fut enfin nommé professeur de chimie générale en 1866.

\parpic[l][t]{%
  \begin{minipage}{40mm}
    \fbox{\includegraphics[width=110px,height=140px]{img/medaillons/merton.eps}}
  \end{minipage}
}
\textbf{Merton, Robert Cox} (1944-) a reçu le prix Nobel d'Économie en 1997, en même temps que son compatriote Myron Scholes, pour avoir élaboré la méthode d'évaluation des instruments financiers dérivés. Cette méthode d'évaluation a certainement accéléré la croissance rapide des marchés des instruments financiers dérivés depuis les années 1980 et permis l'amélioration de la gestion des risques attachés à ces nouveaux produits financiers. Merton a sans conteste contribué à ouvrir une voie nouvelle dans le champ des sciences économiques et fortement influencé les deux autres lauréats. Né en 1944 à New York, il quitte le California Institute of Technology avec un Master en mathématiques appliquées. Il obtient par la suite un doctorat en sciences économiques au Massachusetts Institute of Technology (M.I.T.) de Cambridge, sous la direction de Paul Samuelson (Prix Nobel d'Économie 1970) et se spécialise dans les problèmes d'application des méthodes probabilistes à l'évolution aléatoire des cours des actifs financiers. En 1988, il occupe la chaire George Fischer Backer de professeur en Business Administration à la Harvard Business School de Cambridge. Le travail novateur de Merton date du début des années 1970, période pendant laquelle il élabore une méthode originale de calcul de la valeur des instruments dérivés. L'échec de sa méthode appliquée à la gestion d'un fonds de placement à risques américain (Long-Term Capital Management) en 1998, a quelque peu terni sa réputation de spécialiste de la finance internationale. Mais Merton avait lui-même déclaré à une chaîne de télévision américaine, au lendemain de l'attribution de sa récompense que c'est une mauvaise interprétation de penser que l'on peut éliminer les risques simplement parce qu'on les comprend et qu'on les mesure.

\parpic[l][t]{%
  \begin{minipage}{40mm}
    \fbox{\includegraphics[width=110px,height=140px]{img/medaillons/minkowski.eps}}
  \end{minipage}
}
\textbf{Minkowski, Hermann} (1864-1909) Né à Alexotas et décédé à Göttingen, c'était un physicien-mathématicien qui a étudié aux Universités de Berlin et de Königsberg. Il fit des études secondaires au lycée de Königsberg où il se fait remarquer par ses résultats en mathématiques et reçoit son doctorat en 1885 dans la même ville. Il a ensuite enseigné dans plusieurs universités, à Bonn, Königsberg et à Zürich. À Zürich, Einstein était un des étudiants dans plusieurs des cours qu'il a donnés. Minkowski a accepté une chaire en 1902 à l'Université de Göttingen, où il est resté pour le reste de sa vie. À Göttingen, il apprit la physique-mathématique de Hilbert, il a participé à une conférence sur la théorie de l'électron en 1905 et appris les derniers résultats dans la théorie dans l'électrodynamique. En 1907, Minkowski s'est rendu compte que le travail de Lorentz et d'Einstein pourrait mieux être compris dans un espace non-euclidien. Il a considéré l'espace et le temps, qui a été autrefois pensé pour être indépendant, d'être couplé ensemble dans un continuum d'espace-temps quadridimensionnel. Minkowski a établi un traitement quadridimensionnel de l'électrodynamique. Ce continuum d'espace-temps a fourni un cadre pour tout le travail mathématique postérieur dans la relativité. Ces idées ont été employées par Albert Einstein en développant la théorie Générale de Relativité. Minkowski était principalement intéressé par la mathématique pure et a passé beaucoup de son temps en étudiant les formes quadratiques et les fractions continues. Son travail le plus original était cependant sa \textit{Géométrie des nombres}. Cette étude a mené à des travaux sur les corps convexes et aux questions au sujet des problèmes d'emballage (les manières dans lesquelles des figures d'une forme donnée peuvent être placées dans une autre figure donnée).

\parpic[l][t]{%
  \begin{minipage}{40mm}
    \fbox{\includegraphics[width=110px,height=140px]{img/medaillons/mobius.eps}}
  \end{minipage}
}
\textbf{Möbius, August Ferdinand} (1790-1868) était un mathématicien et astronome allemand né à Schulpforta et mort à Leipzig (Allmagne). Möbius fit ses études à Leipzig, à Göttingen (sous la direction de Gauss) et à Halle. En 1815, il devint professeur d'astronomie à Leipzig, puis directeur de l'observatoire de cette ville, après en avoir dirigé la construction. On lui doit plusieurs ouvrages d'astronomie théorique, notamment \textit{De computandis occultationibus fixarum per planetas} (1815). Ses travaux mathématiques concernent principalement la géométrie et furent, pour la plupart, publiés dans le \textit{Journal des mathématiques pures et appliquées} de Crelle, de 1828 à 1858, comme compléments à son ouvrage fondamental \textit{Der barycentrische Calcul} (1827). En introduisant un nouveau système de coordonnées, Möbius y étudie les transformations géométriques, principalement la transformation projective. Son ouvrage eut une très grande importance dans le développement de la géométrie projective. Étudiant la statique sous l'angle de la géométrie, Möbius développa également la théorie des complexes linéaires de droites (Lehrbuch der Statik, 1837). On peut considérer Möbius comme un des pionniers de la topologie, avec la découverte, publiée dans un mémoire à l'Académie des sciences française, du fameux "ruban de Möbius", surface n'ayant qu'un seul côté.

\parpic[l][t]{%
  \begin{minipage}{40mm}
    \fbox{\includegraphics[width=110px,height=140px]{img/medaillons/monge.eps}}
  \end{minipage}
}
\textbf{Monge, Gaspard} (1746-1818) Né à Beaune et décédé à Paris Monge est fils d'un marchand forain. Il suit d'abord le collège de Beaune, puis va ensuite collège de Lyon, où il enseigne dès l'âge de 16 ans les sciences physiques. Un officier du génie, qui avait vu un plan de la ville de Beaune fait par Monge à l'aide de nouvelles méthodes d'observation et de construction graphique, le recommande au commandant de l'École Militaire de Mézières. Mais il ne peut y être admis à cause de son origine roturière et n'est accepté que dans une annexe technique de l'école. Ses talents scientifiques sont reconnus lorsqu'un jour il dresse le plan de fortifications à l'aide d'une méthode bien plus rapide que les méthodes connues jusque-là. Il est alors admis à l'École Militaire comme professeur de mathématiques et continue ses recherches, arrivant à la méthode générale de représentation géométrique connue depuis lors sous le nom de géométrie descriptive. Mais ses découvertes, considérées comme secret militaire de grande valeur, ne peuvent être publiées. En 1780, il vient à Paris enseigner l'hydrodynamique. Il entre aussitôt à l'Académie des Sciences, où il fait une communication sur les lignes de courbure tracées sur une surface (problème déjà étudié par Euler en 1760). En 1786, il publie son célèbre \textit{Traité élémentaire de la statique} et fonde peu après l'École Polytechnique, où il aura l'occasion de donner des leçons de géométrie descriptive et de publier ses travaux jusque-là inconnus. Chargé de mission en Italie, Monge rencontre Bonaparte et se charge du recrutement des savants pour l'expédition d'Égypte. Revenu en France, il reprend son enseignement à l'École Polytechnique, devient sénateur et est anobli . Mais la Restauration le privera de tous ses titres, le rayera de la liste des membres de l'Institut et lui enlèvera son poste d'enseignant. En 1989, ses cendres ont été transférées au Panthéon. Toutes ses recherches mêlent étroitement la géométrie pure, l'analyse infinitésimale et la géométrie analytique, lui permettant, par exemple, de lier chaque famille de surfaces à une équation aux dérivées partielles et, par là, de trouver les solutions d'équations différentielles à l'aide de sa théorie des surfaces. L'influence de Monge s'exerça par son enseignement oral, et la plupart des mathématiciens français du 19ème siècle ont été ses élèves.

\phantomsection
\addcontentsline{toc}{section}{N}
\label{sec:N}

\parpic[l][t]{%
  \begin{minipage}{40mm}
    \fbox{\includegraphics[width=110px,height=140px]{img/medaillons/napier.eps}}
  \end{minipage}
}
\textbf{Napier, John} (1550-1617) Né et décédé à Merchiston, il était  théologien, physicien, astronome et mathématicien. Comme c'était la pratique courante pour les membres de la noblesse à l'époque, John Napier n'est pas entré à l'école avant l'âge de 13 ans. Il n'est pas resté à l'école très longtemps, cependant. On croit qu'il a quitté l'école en Écosse et a peut-être voyagé en Europe continentale afin de mieux poursuivre ses études. En 1571, Napier, âgé de 21 ans, est retourné en Écosse, et a acheté un château à Gartness en 1574. A la mort de son père en 1608, Napier et sa famille ont emménagé dans Merchiston Castle à Edimbourg, où il résida le reste de sa vie. La mathématique n'était pas son activité principale mais il ne manquait pas d'idées pour simplifier les calculs. Il établit quelques formules de trigonométrie sphérique, popularisa l'usage du point pour la notation anglo-saxonne des nombres décimaux mais surtout inventa les logarithmes. Son objectif était de simplifier les calculs trigonométriques nécessaires en astronomie. Il s'attacha à définir le logarithme d'un sinus en s'appuyant sur des considérations mécaniques de points mobiles et sur le lien entre les progressions arithmétique et géométrique.

\parpic[l][t]{%
  \begin{minipage}{40mm}
    \fbox{\includegraphics[width=110px,height=140px]{img/medaillons/navier.eps}}
  \end{minipage}
}
\textbf{Navier, Henri} (1785-1836) Né à Dijon et décédé à Paris, c'était un ingénieur, mathématicien et économiste surtout connu pour ses travaux sur l'hydrodynamique. Henri devient orphelin à 9 ans, après la mort de son père, avocat réputé et ancien député durant la Révolution. Son oncle, ingénieur du Corps des Ponts et Chaussées s'occupe de son éducation à Paris, le considère comme son fils avant de l'adopter avec sa femme, également proche parente du jeune Henri. L'ingénieur cantonnier le pousse à se présenter à l'École Polytechnique. Bien qu'étant parmi les derniers reçus en 1802, il y réussit sa scolarité et son classement lui permet d'intégrer le corps des Ponts et Chaussées. Il est nommé ingénieur ordinaire des Ponts et Chaussées en 1808. Plus tard, il deviendra inspecteur divisionnaire de ce corps et, semble-t-il quelque temps, inspecteur général à l'instar de son oncle. De 1819 à 1835, il assure le cours de mécanique appliquée de l'École Nationale des Ponts et Chaussées (il y est titularisé en 1830 à la suite de la retraite d'Eisenmann). Au début des années 1820, il explore avec Augustin-Louis Cauchy les facettes de la théorie mathématique de l'élasticité, ce qui lui permet de proposer des équations sur le mouvement des fluides newtoniens.

\parpic[l][t]{%
  \begin{minipage}{40mm}
    \fbox{\includegraphics[width=110px,height=140px]{img/medaillons/nash.eps}}
  \end{minipage}
}
\textbf{Nash, John} (1928-2015) Né en Virginie Occidentale, fils de John Nash Sr., ingénieur, et Virginia Martin, enseignante. Jeune, il passait beaucoup de temps à lire et à faire des expériences dans sa chambre qu'il avait convertie en laboratoire. De 1945 à 1948, Nash a étudié au Carnegie Institute of Technology à Pittsburgh, dans l'intention de devenir ingénieur comme son père. À la place, il y a développé une passion durable pour les mathématiques, et en particulier pour la théorie des nombres, les équations diophantiennes, la mécanique quantique et la théorie de la relativité. Il fut admis en troisième cycle, à 20 ans, dans toutes les universités qu'il avait sollicitées: Harvard, Princeton... Il choisit d'aller à Princeton. Ayant un intérêt pour l'économie, Nash se mit à étudier la théorie des jeux, domaine qu'avait défriché John von Neumann, un des grands noms de Princeton, un peu plus d'une décennie auparavant. C'est sur ce sujet qu'il décida de faire sa thèse et qu'il obtint le prix Nobel d'Économie en 1994. Durant l'été 1950, Nash fut employé comme consultant à la RAND, institut top-secret qui employait de la matière grise pour mettre au point diverses stratégies de statu quo soit de victoire, en cas de conflit faisant appel à l'arme nucléaire. À la suite, Nash se mit à étudier les variétés lisses compactes, ce qui fit l'objet d'un papier. Il devint ensuite assistant au M.I.T à la rentrée 1951-52, âgé de 23 ans seulement. Il avait vraiment un tempérament de problem-solver et releva ainsi le pari de résoudre une question de Waren Ambros: est-il possible de plonger une variété riemannienne quelconque dans un espace euclidien? Nash trouve une méthode fondamentale originale pour y arriver. Nash devint malade après quelques problèmes privés et professionnels, mais il attribua sa maladie à sa tentative de résoudre les contradictions de la physique quantique. D'autant plus que peu de temps auparavant, il avait réalisé des travaux sur les équations différentielles partielles elliptiques non linéaires qui lui valurent beaucoup d'admiration autour de lui, mais dont il dut finalement partager la paternité avec un jeune italien qui avait énoncé, indépendamment et quelques semaines avant lui, des résultats similaires: ceci leur valu de ne pas obtenir la médaille Fields en 1958...

\parpic[l][t]{%
  \begin{minipage}{40mm}
    \fbox{\includegraphics[width=110px,height=140px]{img/medaillons/newton.eps}}
  \end{minipage}
}
\textbf{Newton, Isaac} (1642-1727) était un mathématicien et physicien anglais, considéré comme l'un des plus grands scientifiques de l'histoire. Newton nait dans le Lincolnshire (Angleterre), de parents paysans et décédera à Londres. À 5 ans, il fréquente l'école primaire de Skillington, puis à 12 ans celle de Grantham. Il y reste quatre années jusqu'à ce que sa mère le rappelle à Woolsthorpe pour qu'il devienne fermier et qu'il apprenne à administrer son domaine. Pourtant, sa mère, s'apercevant que son fils était plus doué pour la mécanique que pour le bétail, l'autorisa à retourner à l'école pour peut-être pouvoir entrer un jour à l'université. À 17 ans, Newton tombe amoureux d'une camarade de classe, mademoiselle Clara Storey. On l'autorise à la fréquenter et même à se fiancer avec elle, mais il doit terminer ses études avant de se marier. Finalement, le mariage ne se fit pas et Newton restera alors célibataire toute sa vie. À 18 ans, il entre alors au Trinity College de Cambridge (il y restera 7 ans), où il se fait remarquer par son maître, Isaac Barrow. Il a également comme professeur Henry More qui l'influencera dans sa conception de l'espace absolu. À Cambridge, il étudie l'arithmétique, la géométrie dans les \textit{Éléments} d'Euclide et la trigonométrie, mais s'intéresse particulièrement à l'astronomie, à l'alchimie et à la théologie. Il devient à 25 ans bachelier des arts, mais est contraint de suspendre ses études pendant deux années suite à l'apparition de la peste qui s'est abattue sur la ville en 1665 ; il retourne alors dans sa région natale. C'est à cette période que Newton progresse fortement en mathématiques, physique et surtout en optique. Il laissa d'importantes contributions à de nombreuses branches de la science. Ses découvertes et théories furent à la base d'une grande partie des progrès scientifiques réalisés après lui. Newton fut l'un des inventeurs de la branche des mathématiques appelée "calcul infinitésimal" (l'autre inventeur fut le mathématicien allemand Gottfried Wilhelm Leibniz). Il éclaircit également les mystères de la lumière et de l'optique, formula trois lois sur le mouvement et en déduisit la loi de la gravitation universelle en sa basant sur les lois de Kepler. Il parvint au raisonnement selon lequel la lumière est un mélange de différents rayons de couleurs différentes, et qu'en raison des phénomènes de réflexion et de réfraction, ses couleurs apparaissent en composants séparés. Newton mit en évidence sa théorie des couleurs en faisant passer de la lumière au travers d'un prisme, qui scinde le faisceau lumineux en couleurs séparées. En 1696, il quitte Cambridge pour devenir d'abord gardien de la Royal Mint puis maître de la monnaie dès l'année suivante. En 1699, il est nommé membre du conseil de la Royal Society et y est élu président en 1703. Il garde cette place jusqu'à son décès.

\parpic[l][t]{%
  \begin{minipage}{40mm}
    \fbox{\includegraphics[width=110px,height=140px]{img/medaillons/neumann.eps}}
  \end{minipage}
}
\textbf{Neumann Von, John} (1903-1957) était un mathématicien né à Budapest (Hongrie) et décédé à Washington. Neumann est un enfant prodige: à 6 ans, il converse avec son père en grec ancien et peut mentalement faire la division d'un nombre à huit chiffres. Une anecdote rapporte qu'à 8 ans, il a déjà lu les 44 volumes de l'\textit{Histoire universelle} de la bibliothèque familiale et qu'il les a entièrement mémorisés : doté d'une mémoire absolue, il sera capable de citer de mémoire des pages entières de livres lus des années auparavant. Il entre au lycée à Budapest en 1911. C'est âgé d'à peine 23 ans qu'il reçoit son doctorat en mathématiques (avec des mineures en physique expérimentale et en chimie) de l'Université de Budapest. En parallèle, il obtient un diplôme en génie chimique de l'École Polytechnique Fédérale de Zürich (à la demande de son père, désireux que son fils s'investisse dans un secteur plus rémunérateur que les mathématiques). Il est intéressant de noter que von Neumann n'a mis les pieds dans ces deux universités que pour les examens. Entre 1926 et 1930, il est privatdozent à Berlin et à Hambourg. Il travaille également à Göttingen avec Robert Oppenheimer sous la direction de David Hilbert. Durant cette période, l'une des plus fécondes de sa vie, il côtoie également Werner Heisenberg et Kurt Gödel. En 1930, von Neumann est professeur-invité à l'université de Princeton. Puis, de 1933 à sa mort en 1957, il est professeur de mathématiques à la faculté de l'Institute for Advanced Study qui vient d'être créée (Neumann émigra aux États-Unis en 1933). Il y rejoint donc Albert Einstein et Kurt Gödel. Il rédigea un important ouvrage sur la mathématique appliquée et effectua un travail majeur dans l'axiomatisation de la physique quantique (c'est lui qui réalisa système quantique peut être considéré comme un point dans un espace de Hilbert et qui introduisit les opérateurs linéaires). Il participa durant la deuxième guerre mondiale au développement théorique de la bombe atomique et à l'étude des ondes de choc. Ces travaux mathématiques sur les calculs ultra-rapides pour les simulations de la bombe H, contribua de façon non négligeable au développement de l'informatique (il est aussi à l'origine des méthodes de Monte-Carlo). Il contribua également à la théorie des jeux où certains de ces résultats eurent une grande influence sur l'économie.

\parpic[l][t]{%
  \begin{minipage}{40mm}
    \fbox{\includegraphics[width=110px,height=140px]{img/medaillons/abel.eps}}
  \end{minipage}
}
\textbf{Niels, Abel} (1802-1829) était un mathématicien norvégien né à Frindoë et décédé à Froland. Son père était un éminent homme politique norvégien, mais à la fin de sa vie, il tomba en disgrâce, et quand il mourut en 1820, c'est Abel qui dut supporter toute la charge de la famille. Son père, éduqua lui-même Abel jusqu'en 1815, puis l'envoya au collège paroissial d'Oslo. Dans ce lycée, le latin, le grec et la religion étaient enseignés à l'ancienne, avec punitions et châtiments corporels. La situation évolua en 1817 à la suite du renvoi d'un professeur consécutif au décès d'un élève: le lycée recruta un jeune enseignant ouvert aux idées nouvelles et instruit de mathématiques qui en découvrant l'intérêt de Niels pour les mathématiques, lui trouva une bourse pour l'université. Grâce à l'aide financière de ses professeurs, il parvient donc toutefois à poursuivre ses études et à faire ses premières découvertes. Mais ses mémoires sont perdus par Cauchy et mésestimés par Gauss. Après son doctorat, Abel ne parvint pas à trouver un poste et ses conditions de vie devinrent de plus en plus précaires et sa santé se fragilisa: il fut ainsi atteint de la tuberculose. Malgré des déplacements à Paris et à Berlin, ses travaux ne sont toujours pas perçus à leur juste valeur. Dans ses dernières semaines, il n'a plus assez de force pour quitter son lit. Il décède à même pas 27 ans, alors qu'un ami venait juste de lui trouver un poste à Berlin.  C'est Jacobi qui comprendra tout le génie de ce jeune mathématicien. Abel avait notamment démontré, à l'âge de 19 ans dans un article en français intitulé \textit{Mémoire sur les équations algébriques, où l'on démontre l'impossibilité de la résolution de l'équation générale du cinquième degré}, l'impossibilité de résoudre par radicaux les équations algébriques de degré 5, ce que son contemporain Galois généralisera à tout degré. À titre posthume, Abel recevra en 1830 le grand prix de Mathématiques de l'Institut de France.

\parpic[l][t]{%
  \begin{minipage}{40mm}
    \fbox{\includegraphics[width=110px,height=140px]{img/medaillons/noether.eps}}
  \end{minipage}
}
\textbf{Nöther, Emmy} (1882 -1935) Née à Erlangen et décédée à Princeton, Emmy envisage d'abord d'enseigner le français et l'anglais après avoir passé les examens requis, mais étudie finalement les mathématiques à l'Université d'Erlangen où son père donne des conférences. Durant le semestre d'hiver 1903-1904, elle étudie à l'Université de Göttingen et assiste aux cours de l'astronome Karl Schwarzschild et des mathématiciens Hermann Minkowski, Felix Klein et David Hilbert. Après avoir achevé sa thèse en 1907 elle travaille bénévolement à l'Institut de Mathématiques d'Erlangen pendant 7 ans. En 1915, elle est invitée par David Hilbert et Felix Klein à rejoindre le très renommé département de mathématiques de l'université de Göttingen jusqu'en 1933. En 1935, elle est opérée en raison d'un kyste ovarien et, malgré des signes de rétablissement, meurt 4 jours plus tard à l'âge de 53 ans. Elle reste dans l'histoire des mathématiques comme la fondatrice principale de l'Algèbre abstraite, ou algèbre moderne, qui est une des branches essentielles des mathématiques contemporaines. Cette algèbre abstraite prend de la hauteur par rapport aux calculs menés dans divers ensembles, munis de diverses opérations, et montre ce que ces calculs ont en commun. En physique, le théorème de Noether explique le lien fondamental entre la symétrie et les lois de conservation. Ses idées dans  ont contribué aussi au progrès de la physique, en particulier dans la théorie de la Relativité. Malgré toutes ses qualités, elle eut des difficultés à mener une carrière normale de professeur d'université, car elle était une femme, dans un milieu exclusivement masculin. Elle bénéficia cependant de l'estime et de l'appui de David Hilbert, d'Albert Einstein et de Felix Klein.

\phantomsection
\addcontentsline{toc}{section}{O}
\label{sec:O}

\parpic[l][t]{%
  \begin{minipage}{40mm}
    \fbox{\includegraphics[width=110px,height=140px]{img/medaillons/ohm.eps}}
  \end{minipage}
}
\textbf{Ohm, Georg Simon} (1789-1854) était un physicien né à Erlangen et décédé à Münich. Bien que ses parents n'aient pas faits d'études supérieures, le père de Ohm était un homme respecté et un autodidacte qui a lui-même donné à son fils une excellente éducation. Depuis sa plus jeune enfance Georg reçoit de son père des enseignements de très bon niveau en physique, mathématiques, chimie et philosophie. Georg fréquente le lycée d'Erlangen de 11 à 15 ans et il y reçoit une éducation scientifique très restreinte, contrastant avec les enseignements de son père. En 1805, à l'âge de 15 ans, Ohm entre à l'Université d'Erlangen. Comme Ohm est dissipé, son père en colère devant le gâchis de ses possibilités, l'envoie en Suisse où, en 1806, il prend un poste de professeur de mathématiques dans une école de Gottstadt bei Nydau. Ohm quitte son poste d'enseignant à Gottstadt bei Nydau en 1809 pour devenir précepteur à Neuchâtel pendant 2 ans. Puis, en 1811 il retourne à l'Université d'Erlangen. Ses études lui furent utiles pour obtenir son doctorat de l'Université d'Erlangen la même année et rejoindre immédiatement l'équipe enseignante comme maître de conférence en mathématiques. Le roi Frédéric-Guillaume III de Prusse lui offre un poste au lycée jésuite de Cologne en 1817. Grâce à la réputation de cette école dans l'enseignement des sciences, Ohm se retrouve à enseigner aussi bien les mathématiques que la physique. Le laboratoire de physique étant bien équipé, il se consacre à des expérimentations. Ce qui est actuellement connu sous le nom de "loi d'Ohm" est apparu en 1827 dans le livre \textit{Die galvanische Kette, mathematisch bearbeitet} dans lequel il fournit une théorie complète de l'électricité. Il entre à l'École Polytechnique de Nuremberg en 1833 et en 1852 devient professeur de physique expérimentale à l'Université de Münich, où il meurt un peu plus tard.

\parpic[l][t]{%
  \begin{minipage}{40mm}
    \fbox{\includegraphics[width=110px,height=140px]{img/medaillons/oppenheimer.eps}}
  \end{minipage}
}
\textbf{Oppenheimer, J. Robert} (1904-1967) était un physicien né à New-York et décédé à Princeton. Il a été le directeur scientifique du projet Manhattan et y dirigea donc la mise au point des premières bombes atomiques. Entré à Harvard avec une année de retard à cause d'une attaque de colite ulcéreuse, il profite de cette période pour se rendre, avec son ancien professeur d'anglais, au Nouveau-Mexique. Il y devint amateur de promenades à cheval ainsi que des montagnes et plateaux de cette région. À son retour, il obtient son diplôme de chimie en 3 ans. Percy Bridgman lui fait découvrir la physique expérimentale. C'est durant ses études au laboratoire Cavendish d'Ernest Rutherford à Cambridge, qu'il réalise qu'il maîtrise mieux la théorie que la pratique en raison de sa maladresse. En 1926, il poursuit ses études sous la direction de Max Born à l'Université de Göttingen, et obtient son doctorat à l'âge de 22 ans. À Göttingen, il publie des articles sur la théorie quantique. En 1927, il retourne à Harvard puis l'année suivante à l'Institut de Technologie de Californie. Il est aussi connu pour sa contribution à la théorie quantique et à la théorie de la relativité, et pour ses études sur les rayons cosmiques, les positrons et les étoiles à neutrons. Il fait des recherches importantes en astrophysique, en physique nucléaire, et en spectroscopie. Il découvre alors l'approximation de Born-Oppenheimer.

\parpic[l][t]{%
  \begin{minipage}{40mm}
    \fbox{\includegraphics[width=110px,height=140px]{img/medaillons/ostrogradsky.eps}}
  \end{minipage}
}
\textbf{Ostrogradsky, Mikhail Vasilyevich} (1801-1862) était un physicien et mathématicien ukrainien. Il commença ses études de mathématiques à l'Université de Kharkov, et les continua ensuite à Paris où il fut en contact étroit avec les célèbres mathématiciens français Cauchy, Binet, Fourier et Poisson. De retour dans sa patrie, il enseigna à l'École des Cadets de la Marine, à l'Académie du Génie Nicolas et à l'École d'Artillerie de Saint-Pétersbourg. Il est célèbre en particulier pour avoir établi le théorème de flux-divergence, qui permet d'exprimer l'intégrale sur un volume (ou intégrale triple) de la divergence d'un champ vectoriel comme l'intégrale de surface (intégrale double étendue à la superficie qui entoure ce volume) du flux défini par ce champ. Il fut élu à l'Académie Américaine des Arts et des Aciences en 1834, à l'Académie des Sciences de Turin en 1841, et à l'Académie des Sciences de Rome en 1853. Enfin il fut élu membre correspondant de l'Académie des Sciences de Paris en 1856. Les travaux scientifiques d'Ostrogradski sont dans le droit fil des principes professés à cette époque à l'École Polytechnique dans les domaines de l'analyse, et des mathématiques appliquées. En physique mathématique, il imagina une synthèse grandiose qui embrasserait l'hydromécanique, la théorie de l'élasticité, la théorie de la chaleur, et la théorie de l'électricité dans le cadre d'une seule méthode homogène.

\phantomsection
\addcontentsline{toc}{section}{P}
\label{sec:P}

\parpic[l][t]{%
  \begin{minipage}{40mm}
    \fbox{\includegraphics[width=110px,height=140px]{img/medaillons/pareto.eps}}
  \end{minipage}
}
\textbf{Pareto, Vilfredo} (1848-1923) était un économiste et sociologue italien, dont la contribution la plus célèbre à la théorie économique est la définition du concept d'optimum économique. Né à Paris d'un père italien en exil et d'une mère française, il retourna en Italie à l'âge de 10 ans. Il fit ses études à l'Université de Turin et devint ingénieur. En 1893, il fut nommé à la chaire d'Économie Politique de l'Université de Lausanne (il décédera à Céligny en Suisse), où il succédait à Léon Walras. Parmi ses travaux figure l'analyse des anticipations des agents économiques. Celles-ci, n'étant pas indépendantes les unes des autres, peuvent susciter des mouvements d'opinion pessimistes qui génèrent des crises. Pareto est également le père de la notion d'optimum. L'économie est à un optimum lorsqu'on ne peut améliorer la situation d'un agent sans détériorer celle d'au moins un autre agent. Ce concept est très utilisé en économie, car il permet de prendre en compte la non-additivité des utilités des différents agents. La concurrence permet d'atteindre l'optimum au sens de Pareto. Pareto a également intégré les courbes d'indifférence (formalisées par Francis Edgeworth) à la logique walrassienne d'équilibre général. Le travail sociologique de Pareto fut plus discuté. Dans le Traité de sociologie générale, paru en 1916, il présenta sa théorie des élites, selon laquelle le pouvoir d'État est dans toutes les sociétés l'objet d'un combat entre les seules élites. Cette thèse discréditait les démocraties, et contribua implicitement au développement du fascisme alors montant en Italie.

\parpic[l][t]{%
  \begin{minipage}{40mm}
    \fbox{\includegraphics[width=110px,height=140px]{img/medaillons/pascal.eps}}
  \end{minipage}
}
\textbf{Pascal, Blaise} (1623-1662) était un mathématicien, physicien, théologien, mystique, philosophe, moraliste et polémiste né à Clermont et décédé à Paris. Enfant précoce (à 11 ans, il compose un court\textit{ Traité des sons des corps vibrants} et aurait démontré la 32e proposition du Ier livre d’Euclide, à 16 ans il écrit un traité sur les \textit{Coniques}), il est éduqué par son père qui était mathématicien. Les tout premiers travaux de Pascal concernent les sciences naturelles et appliquées. Il contribua de manière importante à l’étude des fluides. Il a clarifié les concepts de pression et de vide, en étendant le travail de Torricelli. L'étendue des domaines d'intérêt et du génie de Pascal est impressionnante: inventeur de la machine à calculer, concepteur des premiers transports en commun en France, artisan de l'assèchement des marais poitevins il fut également l'un des plus brillants prosateurs de la langue française et l'une des plus grandes figures du 17ème siècle français.

\parpic[l][t]{%
  \begin{minipage}{40mm}
    \fbox{\includegraphics[width=110px,height=140px]{img/medaillons/pauli.eps}}
  \end{minipage}
}
\textbf{Pauli, Wolfgang} (1900-1958) était un physicien autrichien, né à Vienne et décédé à Zürich, connu pour sa définition du principe d'exclusion en mécanique quantique, ce qui lui valut le prix Nobel de Physique de 1945. Pauli est né d'un père professeur des universités et d'une mère journaliste et juriste. Au lycée à Vienne, Pauli était considéré comme un enfant prodige en mathématiques. À partir de 1919, il commence ses études de physique à l'Université de Mïnich avec pour professeur Arnold Sommerfeld. Depuis 1898, Sommerfeld était en charge d'écrire le cinquième volume de la \textit{Enzyklopädie der mathematischen Wissenschaften} (20'000 pages), consacré à la physique. Il requiert dans un premier temps la collaboration d'Albert Einstein pour rédiger l'article sur la relativité, mais ce dernier refuse. Sommerfeld fait alors appel à Pauli, dont la relativité était la spécialité lors de son inscription aux cours de Sommerfeld. C'est ainsi qu'à 21 ans, Pauli publie son article de synthèse des théories de la relativité restreinte et de la relativité générale. En 1921, il obtient son doctorat avec pour sujet l'atome d'hydrogène qui montra clairement la limite du modèle de l'atome de Bohr, auquel il travaillera en tant qu'assistant de Max Born à Göttingen entre 1921 et 1922. Pendant les années 1922 et 1923, il travailla aux côtés de Niels Bohr à Copenhague. Entre 1923 et 1928, il enseigna à Hambourg avant de partir à l'ETH de Zürich, où il obtint un poste de professeur de physique théorique. À partir de 1935, il est parti pour les États-Unis, où il occupe des postes de professeur invité, notamment à l'Institute for Advanced Study à Princeton durant les années 1935-1936, mais aussi à l'Université du Michigan, en 1931 et 1941, et l'Université Purdue, en 1942. En 1946, il obtient la citoyenneté américaine, mais revient la même année à l'ETH de Zürich, où une place de professeur lui avait été gardée. En 1949, il devient citoyen suisse. Dans les années 1950, il retourne régulièrement à Princeton afin de donner des cours en tant que professeur invité. Dans les dernières années de sa vie, il participa à la fondation du CERN. Il meurt d'un ulcère gastro-duodénal.

\parpic[l][t]{%
  \begin{minipage}{40mm}
    \fbox{\includegraphics[width=110px,height=140px]{img/medaillons/pearson.eps}}
  \end{minipage}
}
\textbf{Pearson, Karl} (1857-1936) était un mathématicien britannique né à Londres et décédé à Surrey, considéré comme un des fondateurs des statistiques modernes. L'analyse statistique connaît un grand développement à la fin du 19ème siècle au Royaume-Uni et Karl Pearson domine ses contemporains par l'étendue et la variété de ses contributions bien que s'étant intéressés aux statistiques seulement à partir de l'âge de 33 ans. Il développe des méthodes d'analyse pour l'étude de la sélection naturelle et de l'eugénisme dont il est un ardent promoteur. Ses principales contributions sont la création du test du d'indépendance du Khi-deux destiné à estimer si les écarts observés dans un ensemble de variables par rapport aux valeurs théoriques peuvent être attribués ou non à un échantillonnage au hasard et la définition du coefficient de corrélation. Il reçoit la médaille Darwin (biologie) en 1898. Pearson était aussi consultant dans les entreprises. Il a entre autres donné des cours à William S. Gosset qui introduisit la loi de Student en 1910. Il est l'un des fondateurs de la revue \textit{Biometrika} dont il a été l'éditeur pendant 36 ans et qu'il a hissé au rang de meilleure revue de statistiques mathématiques.

\parpic[l][t]{%
  \begin{minipage}{40mm}
    \fbox{\includegraphics[width=110px,height=140px]{img/medaillons/penrose.eps}}
  \end{minipage}
}
\textbf{Penrose, Roger} (1931-) est un physicien et mathématicien né à Colchester (Angleterre). Penrose obtient son diplôme en mathématiques de l'Université du Collège de Londres et son doctorat à l'Université de Cambridge avec une thèse sur les méthodes tensorielles en géométrie algébrique. Entre 1964 et 1973, il enseigne la mathématique au Birkbeck College de Londres et rencontre le célèbre physicien Stephen W. Hawking avec lequel il travailla sur une théorie de l'origine de l'Univers en apportant sa contribution mathématique à la théorie de la Relativité Générale appliquée à la cosmologie et à l'étude des Trous Noirs. En 1965, à Cambridge, il prouve que des singularités gravitationnelles peuvent être formées à partir de l'effondrement gravitationnel d'étoiles massives en fin de vie. En 1971, Prenrose découvre les réseaux de spin qui devaient plus tard former la géométrie de l'espace-temps dans la théorie quantique à boucles. Professeur à Oxford, il reçut, avec Hawking, le prix Wolf 1988 pour la physique

\parpic[l][t]{%
  \begin{minipage}{40mm}
    \fbox{\includegraphics[width=110px,height=140px]{img/medaillons/picard.eps}}
  \end{minipage}
}
\textbf{Picard, Charles-Emile} (1856-1941) Né et décédé à Paris il fait ses études classiques au lycée de Vanves dès 1864, puis au lycée Napoléon (futur lycée Henry IV) de 1868 à 1874 où il se révèle excellent élève, mais peu attiré par la mathématique. Il obtient cette année-là le baccalauréat ès lettres puis l'année suivante le baccalauréat ès sciences. Il est reçu second à l'École Polytechnique, et premier à l'École Normale Supérieure. Finalement, passionné par les sciences, il opte pour cette dernière, où il prépare l'Agrégation qu'il réussit en 1877. Après divers postes d'assistant à Paris et Toulouse, il devient en 1881 Maître de Conférences à l'École Normale Supérieure. Son nom est déjà célèbre dans le cercle des mathématiciens, car il a démontré un théorème important sur les singularités des fonctions holomorphe qui lui vaudra une nomination pour devenir membre de l'Académie des Sciences. Il est cependant trop jeune, et son élection est reportée en 1889. En 1885, Picard devient professeur à la Sorbonne, où il occupe la chaire de Calcul Différentiel. Là encore, son jeune âge est une gêne (il faut avoir au minimum 30 ans pour occuper un tel poste) et il faut utiliser une procédure astucieuse pour contourner la législation. Plus tard, Picard occupera la chaire d'analyse et d'algèbre, et il exercera aussi à l'École Centrale des Arts et Manufacture (1894-1937): il y forme à la mécanique plus de 10'000 ingénieurs, et est, selon Hadamard, un excellent professeur. Les travaux de Picard sont ardus, et ouvrent la voie à de nouvelles recherches. Il est le premier à utiliser le théorème du point fixe dans une méthode d'approximations successives qui permet de résoudre des équations aux dérivées partielles. On lui doit également des travaux en géométrie algébrique, comme des recherches plus appliquées sur l'élasticité ou la chaleur. Il est aussi l'un des premiers défenseurs des théories d'Albert Einstein. Son \textit{Traité d'Analyse} constitua longtemps une référence, et Picard fut aussi philosophe et historien des sciences. Parmi les distinctions que Picard a reçues, citons qu'il présida le Congrès International des Mathématiciens, qu'il fut élu membre de l'Académie Française en 1924, et qu'il reçut la médaille d'or Mittag-Leffler en 1937.

\parpic[l][t]{%
  \begin{minipage}{40mm}
    \fbox{\includegraphics[width=110px,height=140px]{img/medaillons/planck.eps}}
  \end{minipage}
}
\textbf{Planck, Max} (1858-1947) était un physicien allemand né à Kiel et décédé à Göttingen considéré comme le fondateur de la physique quantique. Après avoir obtenu son baccalauréat à 17 ans à Münich où son père enseigne, Max Planck poursuit ses études de physique à Berlin. Passionné par la thermodynamique, il soutient une thèse de doctorat sur le second principe de la thermodynamique et la notion d'entropie 1879, notion qui restera le moteur explicatif de la majorité de ses recherches. L'année suivante, il devient maître de conférence à l'Université de Münich puis obtient la chaire de physique de l'Université de Kiel en 1885. Quatre ans plus tard, il est professeur de physique à l'Université de Berlin, poste qu'il occupera pendant près de 40 ans. En 1930 il prend la direction de l'Institut Kaiser Wilhem pour la recherche scientifique qui portera bientôt son nom. Amorcées par sa thèse de doctorat, les recherches de Planck en thermodynamique se portent rapidement sur le corps noir. Entité purement théorique, le corps noir absorbe toutes les radiations qu'il reçoit (le noir de carbone, en absorbant $97\%$ du rayonnement, se rapproche de cet idéal). Pour tenter d'expliquer ce phénomène, Planck élabore une nouvelle théorie. Il émet l'hypothèse que l'énergie d'un rayonnement ne peut être émise ou absorbée par la matière que par quantités finies, les quanta. Il montre alors que ces "paquets d'énergie" ont pour valeur $h \nu$, où $\nu$ est la fréquence du rayonnement et $h$ une constante universelle (la "constante de Planck"). En exposant sa théorie à la Société Allemande dePhysique en 1900, à Berlin, Planck ne sait pas encore qu'il vient d'inventer une nouvelle branche de la physique: la physique quantique. Sa découverte entraînera alors la création du modèle de l'atome par Niels Bohr, l'élaboration de la mécanique ondulatoire par Louis de Broglie , l'explication du phénomène photoélectrique par Albert Einstein ou encore la découverte du principe d'incertitude par Werner Heisenberg. Considéré comme l'un des plus célèbres physiciens, Planck recevra le prix Nobel en 1918.

\parpic[l][t]{%
  \begin{minipage}{40mm}
    \fbox{\includegraphics[width=110px,height=140px]{img/medaillons/poincare.eps}}
  \end{minipage}
}
\textbf{Poincaré, Henri} (1854-1912) était un mathématicien et physicien français né à Nancy et décédé à Paris dont on a dit qu'il était le dernier savant susceptible de connaître la totalité des mathématiques de son temps. Élève d'exception au Lycée Impérial de Nancy, il obtient en 1871, le baccalauréat ès lettres, mention Bien, et la même année son baccalauréat ès sciences. Il se classe premier au concours d'entrée à l'École Polytechnique en 1873, puis à l'École des Mines de Paris, comme ingénieur du Corps des Mines, en 1875. Il est licencié ès sciences en 1876. Nommé ingénieur des mines de troisième classe en 1879 à Vesoul, il obtient, la même année le doctorat ès sciences Mathématiques à la Faculté des Sciences de Paris, et devient chargé de cours d'Analyse à la faculté des sciences de Caen. Les premiers travaux de Poincaré portent sur les fonctions automorphes ou fuchsiennes, la théorie qualitative des équations différentielles et la théorie des fonctions. Dans une série de six articles publiés à partir de 1894, il est le créateur de la topologie algébrique, science en pleine expansion au 20ème siècle et dans laquelle plusieurs conjectures dues à Poincaré restent ouvertes. Il s'est vivement intéressé à la mécanique céleste: \textit{Les Méthodes nouvelles de la mécanique céleste}, trois volumes parus entre 1892 et 1899, annoncent les recherches modernes sur les systèmes dynamiques et le chaos. En physique-mathématique, il dégagea les propriétés du groupe de Poincaré-Lorenz, qui allaient quelques mois plus tard conduire à l'article fondamental d'Albert Einstein sur la Relativité Restreinte.

\parpic[l][t]{%
  \begin{minipage}{40mm}
    \fbox{\includegraphics[width=110px,height=140px]{img/medaillons/poisson.eps}}
  \end{minipage}
}
\textbf{Poisson, Siméon Denis} (1781-1840) était un mathématicien français dont les travaux ont porté sur les intégrales définies, la théorie électromagnétique et le calcul des probabilités. Sa famille le força à faire des études de médecine qu'il abandonna, en 1798, pour aller étudier la mathématique à l'École Polytechnique, où il fut l'élève de Laplace et  Lagrange, qui devinrent l'un et l'autre ses amis. Il enseigna à l'École Polytechnique à partir de 1802 et en 1808, il fut nommé astronome du Bureau des Longitudes et, à sa création, en 1809, professeur à la Faculté des Sciences. Les travaux les plus importants de Poisson portent sur les applications des mathématiques à la physique et à la mécanique. Son \textit{Traité de mécanique} a été l'ouvrage de référence en mécanique pendant de nombreuses années. Un mémoire, publié en 1812, contient les lois les plus usuelles de l'électrostatique et la théorie selon laquelle l'électricité est constituée de deux fluides dont les éléments semblables se repoussent, tandis que les éléments différents s'attirent. En mathématiques pures, il a publié une série d'articles sur les intégrales définies, et ses recherches sur les séries de Fourier ont annoncé celles de Dirichlet et de Riemann sur ce sujet. C'est dans l'ouvrage \textit{Recherches sur la probabilité des jugements}... (1837), qui est un livre important sur le calcul des probabilités, qu'apparaît pour la première fois la distribution de Poisson (ou "loi de Poisson"). Obtenue initialement comme une approximation de la loi binomiale de Bernoulli, elle est devenue fondamentale dans de très nombreux problèmes. Les autres publications de Poisson comprennent\textit{ Théorie nouvelle de l'action} (1831) et \textit{Théorie mathématique de la chaleur} (1835). Le nom de Poisson est attaché à de nombreuses notions mathématiques et physiques (intégrale et équation de Poisson en théorie du potentiel, crochets de Poisson dans la théorie des équations différentielles, rapport de Poisson en élasticité et constante de Poisson en électricité).

\parpic[l][t]{%
  \begin{minipage}{40mm}
    \fbox{\includegraphics[width=110px,height=140px]{img/medaillons/poynting.eps}}
  \end{minipage}
}
\textbf{Poynting, John Henry} (1852-1912) était un physicien né dans le Lancashire et décédé à Birmingham qui a travaillé, en autres, sur les ondes électromagnétiques. Il a défini ce que l'on appelle le "vecteur de Poynting" qui représente la puissance par unité de surface que transporte une onde électromagnétique et la direction de ce flux d'énergie. Poynting suivi l'école élémentaire dans un école dirigée par son père. De 1867 à 1872 il suivit les cours du collège d'Owen (aujourd'hui Université de Manchester) où il eut comme professeur Osborne Reynolds. De 1872 à 1875 il est étudiant à l'Université de Cambridge où il obtint les honneurs en mathématiques. À la fin des années 1870 il travailla au laboratoire Cavendish sous les ordres de James Clerk Maxwell. En 1903 il fut le premier à réaliser que la radiation solaire pouvait attirer les petites particules vers le Soleil, effet reconnu plus tard sous le nom d'effet Poynting-Robertson. Pendant l'année 1884, il analysa les prix des bourses de commerce, notamment ceux du blé, de la soie, et du coton, à l'aide de méthodes statistiques. Il fut professeur de physique au Mason Science College (qui devint plus tard l'Université de Birmingham) jusqu'à sa mort

\phantomsection
\addcontentsline{toc}{section}{R}
\label{sec:R}

\parpic[l][t]{%
  \begin{minipage}{40mm}
    \fbox{\includegraphics[width=110px,height=140px]{img/medaillons/ramanujan.eps}}
  \end{minipage}
}
\textbf{Ramanujan, Srivanasa} (1887-1920) était né à Erode, un petit village situé $400$ [km] au sud de Madras, dans une famille pauvre de la caste des Brahmanes. Il passe son enfance dans la ville de Kumbakonam, où son père exerce le métier de comptable chez un drapier. À partir de l'âge de 5 ans, il fréquente différentes écoles primaires avant de pouvoir intégrer la Town High School en 1898. En 1900, il commence à développer ses propres mathématiques en se basant sur son premier livre de mathématiques, \textit{La Trigonométrie plane}. Il définit seul une méthode pour résoudre les équations du 3e degré, puis du 4e, puis il tente aussi de résoudre celles du 5e degré, ignorant qu'elles ne peuvent être résolues par les radicaux. On est alors en 1902 et c'est à cette époque que Ramanujan se procure le second (et dernier!) livre dans lequel il puisera ses connaissances mathématiques de bases, \textit{Synopsis of elementary results in pure mathematics}, compilation d'environ $6'000$ théorèmes et autres formules par G.S. Carr. Ce livre étant essentiellement un livre de résultats, la plupart sans démonstrations, influencera le style futur de Ramanujan, qui n'a laissé que très peu de preuves de ses propres résultats. À 17 ans, sa démarche est déjà celle d'un chercheur en mathématiques. Comme ses résultats scolaires sont bons, il reçoit une bourse lui permettant d'entrer au Government College de Kumbakonam en 1904. Cependant, il consacre trop de temps à ses recherches en mathématiques et néglige les autres matières, ce qui lui vaut la suppression de cette bourse l'année suivante. Sans argent, il part, à l'insu de ses parents, pour la ville de Vizagapatnam où il poursuit ses travaux sur les séries hypergéométriques et les relations entre intégrales et séries. En 1906, il retourne à nouveau au lycée, à Madras cette fois-ci, avec l'idée de passer un examen lui permettant d'entrer à l'université. Il assiste quelques mois aux cours puis tombe malade. Au cours de l'examen, il réussit seulement en mathématiques et échoue partout ailleurs, ce qui lui interdit l'entrée à l'Université de Madras. Dans les années qui suivent, il continue alors de développer seul ses idées, sans aucune aide extérieure et sans connaissance des thèmes de recherche possibles, en dehors de ceux découlant des notions abordées dans le livre de Carr. Ramanujan étudie ainsi les fractions continues et les séries divergentes en 1908. Il tombe alors de nouveau très malade et doit subir, en 1909, une opération dont il aura du mal à se remettre. Il commence alors de poser et de résoudre des problèmes mathématiques dans le journal de la \textit{Société Indienne de Mathématiques} (SIM). En 1910, il développe des relations sur les équations modulaires elliptiques. Un an plus tard, la publication d'un article brillant sur les nombres de Bernoulli dans ce même journal lui vaut la reconnaissance de son travail par ses pairs. Bien qu'il ne possède aucun diplôme universitaire, il acquiert la réputation de génie des mathématiques dans la région de Madras. La même année, il rencontre le fondateur de la SIM, qui lui permet d'obtenir un emploi temporaire chez un comptable de Madras et lui conseille de contacter Ramachandra Rao, un mécène membre de la SIM. Grâce à cette lettre, Ramanujan obtient le poste et commence son travail en 1912. Il a alors la chance d'être entouré de personnes ayant une formation en mathématiques et qui s'intéressent à son travail. Le chef comptable du port de Madras est un mathématicien qui publie un article sur le travail de Ramanujan en 1913, \textit{On the distribution of primes}. D'autre part, un professeur du Madras Engineering College est intéressé par les capacités de Ramanujan. Ayant lui-même fait ses études à Londres, il écrit à un de ses professeurs de mathématiques, à qui il envoie de quelques résultats de Ramanujan. L'Université de Madras allouera plus tard une bourse à Ramanujan en 1913 et en 1914, Hardy le fait venir au Trinity College, à Cambridge. C'est le début d'une extraordinaire collaboration entre les deux hommes. En 1916, il obtient le titre de docteur de l'Université de Cambridge, malgré qu'il ne possède pas les diplômes requis pour préparer une thèse. En 1918, Ramanujan est élu membre de la Cambridge Philosophical Society. Trois jours plus tard, probablement le plus grand honneur de toute sa carrière, son nom apparaît sur la liste des élections des membres de la "Royal Society of London". Il a été proposé par une liste impressionnante d'éminents mathématiciens. Son élection a effectivement lieu en 1918 et il est également élu membre du Trinity College pour 6 ans. Ramanujan repart pour l'Inde en 1919. Cependant, son état de santé déjà très mauvais ne cesse de se dégrader. Il meurt l'année suivante probablement à cause de graves carences alimentaires. Ramanujan a laissé derrière lui un grand nombre de cahiers non-publiés (les fameux Carnets de Ramanujan), remplis de théorèmes que les mathématiciens continuent d'étudier. Aujourd'hui, ses travaux ont bien sûr des applications en physique théorique.

\parpic[l][t]{%
  \begin{minipage}{40mm}
    \fbox{\includegraphics[width=110px,height=140px]{img/medaillons/riccicurbastro.eps}}
  \end{minipage}
}
\textbf{Ricci-Curbastro, Gregorio} (1853-1925) Né à Lugo et décédé à Boulogne il était un mathématicien spécialiste de la géométrie différentielle et l'un des pères du calcul tensoriel. Après des études de philosophie et de mathématiques, Ricci soutient sa thèse de doctorat à l'Université de Pise. En 1880, il sera nommé professeur de physique-mathématique à l'Université de Padoue. Levi-Civita fut son élève et contribua avec Ricci à l'élaboration de son calcul différentiel absolu (1900) visant à expliciter en mécanique, dans des espaces abstraits (variétés différentiables), des relations indépendantes du système de coordonnées utilisé, inhérentes au phénomène étudié (invariants différentiels). Associée à la géométrie différentielle de Gauss et de Riemann, le célèbre physicien Albert Einstein trouva, dans cette nouvelle approche de la mécanique qu'il nomma "calcul tensoriel" (1916), les outils mathématiques nécessaires à sa théorie de la Relativité Générale.

\parpic[l][t]{%
  \begin{minipage}{40mm}
    \fbox{\includegraphics[width=110px,height=140px]{img/medaillons/riemann.eps}}
  \end{minipage}
}
\textbf{Riemann, Georg Friedrich Bernhard} (1826-1866) était un mathématicien allemand. Au lycée, Riemann étudie la Bible intensivement, mais il est distrait par les mathématiques. Il essaie même de prouver, mathématiquement, l'exactitude de la Genèse. Ses professeurs sont surpris par ses capacités à résoudre des problèmes complexes en mathématique. En 1846, grâce à l'argent de sa famille, il commence à étudier la philosophie et la théologie pour devenir prêtre afin de financer sa famille. En 1847, son père l'autorise à étudier les mathématiques. Il étudie d'abord à l'Université de Göttingen où il rencontre Carl Friedrich Gauss, puis à l'université de Berlin, où il a entre autres comme professeurs Jacobi, Steiner et Dirichlet. Dans sa thèse, présentée en 1851 sous la direction de Gauss, Riemann met au point la théorie des fonctions d'une variable complexe. En 1854 il donne un exposé qui jette les bases de la géométrie différentielle. Il y introduit la bonne façon d'étendre à $n$ dimensions les résultats de Gauss lui-même sur les surfaces. Cette présentation a profondément changé la conception de la notion de géométrie, notamment en ouvrant la voie aux géométries non euclidiennes et à la théorie de la Relativité Générale. On lui doit également d'importants travaux sur les intégrales, poursuivant ceux de Cauchy, qui ont donné entre autres ce qu'on appelle aujourd'hui les "intégrales de Riemann". Intéressé par la dynamique des gaz, il jette les bases de l'analyse des équations aux dérivées partielles de type hyperbolique. Il succédera à Dirichlet sur la chaire de Gauss en 1859. À 39 ans, il fut emporté par la tuberculose.

\phantomsection
\addcontentsline{toc}{section}{S}
\label{sec:S}

\parpic[l][t]{%
  \begin{minipage}{40mm}
    \fbox{\includegraphics[width=110px,height=140px]{img/medaillons/salam.eps}}
  \end{minipage}
}
\textbf{Salam, Abdus} (1926-1996) était un physicien pakistanais, ayant reçu le prix Nobel de physique en 1979 pour ses travaux sur l'interaction électrofaible, synthèse de l'électromagnétisme et de l'interaction faible. Né à Jhang Sadar, il étudie au Government College à Lahore. À l'âge de 14 ans, Salam obtint les meilleures notes jamais enregistrées pour l'examen d'entrée à l'Université du Punjab. Persécuté par la majorité musulmane de son pays pour son appartenance religieuse (ahmadiste), il doit fuir son pays. Réfugié en Grande-Bretagne, il obtient en 1952, un doctorat en mathématiques et en physique de l'Université de Cambridge. Sa thèse doctorale fut une étude fondamentale en électrodynamique quantique. Ses travaux le rendirent célèbre internationalement. Il retourna au Government College de Lahore en tant que professeur de mathématiques, garda cette fonction de 1951 à 1954, puis retourna ensuite à Cambridge en tant conférencier en Mathématiques. Il enseigne dans ces établissements, puis en 1957, est nommé professeur de physique théorique à l'Imperial College de Londres. Il y demeura jusqu'à sa retraite. En 1959, il devient le plus jeune membre de la Royal Society à l'âge de 33 ans. Durant les années 1960, Salam joua un rôle important dans l'établissement de l'agence de recherche nucléaire du Pakistan, et de l'agence de recherche spatiale du Pakistan, de laquelle il fut le directeur fondateur. En 1964, il devient directeur du Centre international de Physique Théorique de Trieste, nouvellement créé. Cette même année, il est lauréat de la Médaille Hughes. En 1967, avec le physicien Steven Weinberg, Salam propose une théorie permettant d'unifier les interactions électromagnétiques et faibles entre particules élémentaires, théorie qui sera confirmée par l'expérience. Salam sera ainsi le premier musulman à obtenir le prix Nobel de physique en 1979, conjointement aux physiciens Sheldon Lee Glashow et Weinberg.

\parpic[l][t]{%
  \begin{minipage}{40mm}
    \fbox{\includegraphics[width=110px,height=140px]{img/medaillons/samuelson.eps}}
  \end{minipage}
}
\textbf{Samuelson, Paul} (1915-2009) était un économiste américain, prix Nobel d'économie en 1970 et chef de file de l'école qu'il appela la "synthèse néo-classique", qui entendait reprendre à son compte à la fois les théories de Keynes en macroéconomie et les enseignements néoclassiques en microéconomie. Samuelson est considéré comme un des père de la microéconomie traditionnelle actuelle et serait l'un des pionniers économistes à généraliser, dans un cadre économique, l'usage des modèles mathématiques mis en place pour l'analyse thermodynamique. Ainsi, il aurait aidé à faire passer l'économique de discipline principalement littéraire en un domaine du savoir hautement mathématique, formalisé et axiomatisé.

\parpic[l][t]{%
  \begin{minipage}{40mm}
    \fbox{\includegraphics[width=110px,height=140px]{img/medaillons/savart.eps}}
  \end{minipage}
}
\textbf{Savart, Felix} (1791-1841) était un médecin chirurgien et physicien né en Ardennes et décédé à Paris. Inventeur du sonomètre, d'une roue dentée qui porte son nom et du polariscope. Il jeta les bases de la physique moléculaire. Avec le physicien Jean-Baptiste Biot, il mesura le champ magnétique créé par un courant et formula la "loi de Biot-Savart". Il étudia également les propriétés des cordes vibrantes. Il fut membre de l'Académie des Sciences, élu en 1827, et titulaire de la Chaire de Physique Générale et Expérimentale du Collège de France, nommé en 1836, succédant à André-Marie Ampère. Il est élu membre étranger de la Royal Society en 1839. Son nom a été donné à une unité de mesure des intervalles musicaux : le savart.

\parpic[l][t]{%
  \begin{minipage}{40mm}
    \fbox{\includegraphics[width=110px,height=140px]{img/medaillons/say.eps}}
  \end{minipage}
}
\textbf{Say, Jean-Baptiste} (1767-1832)  était un économiste, journaliste et industriel français né à Lyon et décédé à Paris. Il est issu d'une famille de négociants nîmois ayant émigré à Amsterdam puis à Genève. C'est au cours d'un voyage en Grande-Bretagne, où la révolution industrielle est en cours, qu'il adoptera les idées libérales et en particulier les théories d'Adam Smith dont il sera un ardent défenseur de retour en France. En 1789, il publie la brochure: la \textit{Liberté de la presse}. En 1792, il participe aux campagnes militaires de la révolution française en Champagne. D'abord employé dans une banque, il dirigea ensuite une filature de coton à Auchy-lès-Hesdin, dans le Pas-de-Calais. Ses nombreux ouvrages d'économie politique firent qu'il fut nommé professeur au Conservatoire National des Arts et Métiers en 1821, puis au Collège de France en 1830. La "loi de Say", ou "loi des débouchés", stipule que plus les producteurs sont nombreux et les productions multiples, plus les débouchés sont faciles, variés et vastes. Dans une économie où la concurrence est libre et parfaite, les crises de surproduction sont impossibles. Il ne peut y avoir de déséquilibre global dans les économies de marché et de libre entreprise, il y a un équilibrage spontané des flux économiques (production, consommation, épargne, investissement). Cette loi est parfois réduite à tort à la formule: toute offre crée sa propre demande. Un meilleur résumé de cette approche serait: on dépense que l'argent qu'on a gagné. L'économie de l'offre, dans la tradition de Say, s'oppose à l'économie de la demande, qui est celle de Malthus et plus tard de Keynes.

\parpic[l][t]{%
  \begin{minipage}{40mm}
    \fbox{\includegraphics[width=110px,height=140px]{img/medaillons/schaefer.eps}}
  \end{minipage}
}
\textbf{Schaefer, Milner Baily} (1912-1970) Né au Wyoming et décédé à San Diego a étudié à l'Université de Washington où il avait obtenu un baccalauréat en science en 1935. Dès l'obtention du baccalauréat, il a travaillé au Département des Pêches de l'état de Washington à Seattle. De 1937 à 1942, il a travaillé à la Comission des Pêches du Saumon du Pacifique à Westminster, Colombie-Britannique. Il a servi dans la marine durant la guerre et par la suite, il a occupé divers postes en tant que biologiste spécialiste des pêches. Après avoir terminé son doctorat à l'Université de Washington, en 1950, Schaefer est devenu directeur des enquêtes de l'IATTC (Inter-American Tropical Tuna Commission), une commission internationale des pêches. Pendant les dix années qui ont suivi, il a travaillé sur la théorique de la dynamique de la pêche et mis au point un modèle de population des espèces marines qui est connu sous le nom de "modèle de Schaefer". Au courant des années 1950, Schaefer est devenu de plus en plus impliqué dans plusieurs comités, groupes et organisations concernés par les ressources marines, en particulier la pêche et tous les aspects de l'océanographie. Durant cette période, il a donné des cours sur la dynamique et l'exploitation des populations de poissons. En 1962, il démissionna de son poste de directeur des enquêtes à la IATTC pour occuper le poste de directeur de l'Institut des Ressources marines de l'Université de Californie tout en agissant comme conseiller scientifique pour l'IATTC.


\parpic[l][t]{%
  \begin{minipage}{40mm}
    \fbox{\includegraphics[width=110px,height=140px]{img/medaillons/scholes.eps}}
  \end{minipage}
}
\textbf{Scholes, Myron} (1941-) Né à Ontaria, il présente son doctorat en 1969 à l'Université de Chicago. Il occupe en 1988 la chaire Frank E. Buck de professeur de finance au Graduate School of Business de l'Université Stanford (Californie) où il dirige également des recherches pour l'Institution Hoover. Il a reçu le prix Nobel d'économie en 1997 pour avoir élaboré, avec Fischer Black, une méthode d'évaluation des instruments financiers dérivés (résultat mathématique novateur pour estimer les risques liés aux options sur actions) ayant ouvert de nouveaux horizons au champ des évaluations économiques. Le colauréat de Myron Scholes, Robert Merton, a joué un rôle très important dans l'élaboration de cette méthode d'évaluation ainsi que dans les applications qu'elle a permises pour améliorer la gestion des risques attachés aux nouveaux produits financiers. Déjà en 1900, Louis Bachelier, présentait à la Sorbonne une thèse de doctorat au titre visionnaire: \textit{Théorie de la spéculation}. Dans les années 1960, des auteurs tels James Boness et Paul Samuelson (Prix Nobel d'économie en 1970) proposaient des modèles pour déterminer les prix d'équilibre des options. Leurs hypothèses ne se sont pas révélées suffisamment réalistes pour entraîner des applications, mais des améliorations apportées à ces modèles au début des années 1970 ont permis d'obtenir des résultats plus satisfaisants. C'est en 1973 que Scholes et Black mettent leurs compétences en commun et proposent la première version de la formule de calcul du prix des options qui leur vaudra le prix Nobel. Si Myron Scholes et Fischer Black ont eu l'intuition fondamentale de la démonstration, ils ont pris pour base de recherche le modèle d'équilibre des actifs financiers (ou Capital Asset Pricing Model, dit C.A.P.M.) de leur compatriote William Sharpe récompensé à ce titre par le jury du Nobel en 1990 (les deux autres lauréats étaient Harry Markowitz et Merton Miller).

\parpic[l][t]{%
  \begin{minipage}{40mm}
    \fbox{\includegraphics[width=110px,height=140px]{img/medaillons/schrodinger.eps}}
  \end{minipage}
}
\textbf{Schrödinger, Erwin} (1887-1961) Né et décédé à Vienne il entre au gymnase de cette même ville en 1898. Presque depuis son premier jour de classe jusqu'à son départ du lycée huit ans plus tard, Schrödinger fut un excellent élève. Il était toujours premier de sa classe en travaillant dur chez lui entre les quatre murs de son bureau personnel.. Il poursuit ses études à l'Université d'Iéna (Allemagne). En 1920, il est nommé professeur à la Haute École Technique de Stuttgart puis à l'Université de Breslau l'année suivante. En 1927, il succède à Max Planck à l'Université de Berlin. Israélite, il quitte le pays à l'avènement du national-socialisme pour se rendre à Oxford où il obtient une chaire en 1933. Sept ans plus tard, il devient professeur de physique théorique à Dublin à l'Institut des Hautes Études de l'État Libre d'Irlande. Il ne rentrera en Autriche qu'en 1956. Schrödinger comme son contemporain Albert Einstein avait horreur d'apprendre par coeur et d'être forcé de retenir des faits inutiles. Les premiers travaux de Schrödinger portent sur l'étude des couleurs et la théorie des quanta. Mais il est avant tout reconnu pour ses recherches en mécanique ondulatoire, discipline développée par le Français Louis de Broglie. L'équation de Schrödinger, élaborée en 1926, permet de calculer la fonction d'onde d'une particule se déplaçant dans un champ. En établissant cette équation de propagation, il donne à la mécanique quantique un outil intuitif aujourd'hui indispensable (au contraire de l'approche matricielle et abstraite de Heisenberg) qu'Einstein qualifia d'idée de génie! Avec celle de Werner Heisenberg, la théorie de Schrödinger constitue ainsi la base de la mécanique quantique. En 1933, Schrödinger partage le prix Nobel de physique avec Paul Dirac pour leur contribution au développement de cette nouvelle discipline. Schrödinger essaiera également d'appliquer sa théorie à la biologie et à la génétique dans ses ouvrages \textit{What is life} (1944) et \textit{Science and Humanism} (1951).

\parpic[l][t]{%
  \begin{minipage}{40mm}
    \fbox{\includegraphics[width=110px,height=140px]{img/medaillons/schwartz.eps}}
  \end{minipage}
}
\textbf{Schwartz, Lawrence} (1915-2002) était un mathématicien français né et décédé à Paris. Ses travaux sont principalement relatifs à l'analyse. Ancien élève de l'École Normale Supérieure, Laurent Schwartz a enseigné de 1959 à 1960 et de 1963 à 1983 à l'École Polytechnique. En 1975, il est élu membre de l'Académie des Sciences. Sa thèse (1943) porte sur l'approximation et l'étude des sommes d'exponentielles. La théorie des distributions, dont l'idée initiale remonte à 1945, lui a valu la médaille Fields en 1950. Le langage et les notations de Schwartz pour les distributions ont été adoptées par les mathématiciens et constituent le cadre naturel de la théorie des équations aux dérivées partielles. De 1959 à 1962, Schwartz se consacre à la physique théorique: l'emploi des distributions lui permet une formulation mathématique correcte de la théorie des particules élémentaires. Il a aussi effectué des recherches sur les mesures de Radon et sur les espaces topologiques quelconques ; il a écrit diverses publications sur les probabilités cylindriques et les désintégrations de mesures.

\parpic[l][t]{%
  \begin{minipage}{40mm}
    \fbox{\includegraphics[width=110px,height=140px]{img/medaillons/schwarzschild.eps}}
  \end{minipage}
}
\textbf{Schwarzschild, Karl} (1873-1916) était un astronome mathématicien et physicien né à Francfort et décédé à Postdam qui prédit l'existence des Trous Noirs. Sa curiosité pour les étoiles se manifesta dès ses premières années scolaires, lorsqu'il construisit un petit télescope. Témoin de cet intérêt, son père le présenta à un ami mathématicien qui avait un observatoire privé. Schwarszchild apprit à utiliser un télescope et étudia des mathématiques plus avancées qu'à l'école. Il devint célèbre avec ses deux premiers articles sur la théorie des orbites publiés à l'âge de 16 ans alors qu'il était encore au collège. Il étudia à l'Université de Strasbourg, puis de Münich, et obtint son doctorat à l'âge de 23 ans pour des travaux sur les théories de Henri Poincaré. Il fut alors engagé en tant qu'assistant à l'Observatoire Kuffner à Ottakring. Il se consacra principalement à la photométrie: il accomplit un travail de pionnier pour améliorer les plaques photographiques et implanter leur utilisation en astronomie, ainsi que dans l'étude spectrale des étoiles. De 1901 à 1909, il officia comme professeur au prestigieux institut de Göttingen, où il eut l'occasion de travailler avec des personnalités telles que David Hilbert et Hermann Minkowski. Il occupa ensuite un poste à l'Observatoire d'astrophysique de Potsdam en 1909. Schwarzschild est surtout connu pour ses contributions théoriques, tant en physique du Soleil qu'en Relativité Générale, ou en cinématique stellaire, ainsi que dans divers domaines de l'astrophysique. En 1916, il détermina une grandeur, dite "rayon de Schwarzschild", dans le cadre de la théorie de la relativité, énoncée peu de temps avant par Albert Einstein. Lorsqu'une étoile suffisamment massive explose en supernova, la contraction gravitationnelle produit ce que l'on appelle un "Trou Noir": rien, pas même la lumière, ne peut sortir de ce champ de gravitation intense. Lorsque le rayon d'une masse gazeuse devient inférieur au "rayon de Schwarzschild" pour cette masse, elle s'effondre en Trou Noir.

\parpic[l][t]{%
  \begin{minipage}{40mm}
    \fbox{\includegraphics[width=110px,height=140px]{img/medaillons/shannon.eps}}
  \end{minipage}
}
\textbf{Shannon, Claude Elwood} (1916-2001) Né au Migichan et décédé au Massachusetts c'était un mathématicien spécialiste en mathématiques appliquées et un ingénieur électricien, qui développa la théorie de la communication, aujourd'hui connue sous le nom de la "théorie de l'information". Shannon suivit les cours de l'Université du Michigan et obtint en 1940 son doctorat de l'Institut de Technologie du Massachusetts (M.I.T.), de la faculté duquel il devint un membre, en 1956, après avoir travaillé aux laboratoires de téléphone Bell. En 1949, Shannon publia la \textit{Théorie mathématique de la communication}, un article dans lequel il présenta son concept initial pour une théorie unificatrice de la transmission et du traitement des informations. Les informations, selon cette théorie, incluent toutes formes de messages transmis, y compris ceux envoyés le long des canaux nerveux des organismes vivants. La théorie de l'information est aujourd'hui importante dans de nombreux domaines.


\parpic[l][t]{%
  \begin{minipage}{40mm}
    \fbox{\includegraphics[width=110px,height=140px]{img/medaillons/sharpe.eps}}
  \end{minipage}
}
\textbf{Sharpe, William Forsyth} (1934-) est un économiste né à Boston. L'Académie Royale des Sciences de Suède a décerné en 1990 le prix Nobel de sciences économiques à 3 professeurs américains: Harry Markowitz, Merton Miller et William Sharpe. Même si les travaux récompensés étaient déjà anciens et se situent pour l'essentiel entre 1950 et 1970, l'Académie a jugé que les lauréats étaient des novateurs dans le domaine de la théorie de l'économie financière et du financement des entreprises. Ils ont en effet tous contribué à faire sortir de l'ombre de quelques universités américaines, une nouvelle discipline: la finance. C'était la première fois que l'Académie Royale de Suède récompensait des travaux traitant des marchés boursiers et de la gestion de portefeuilles plutôt que des grands équilibres économiques. William Sharpe, de l'Université Stanford, fut récompensé pour son modèle d'équilibre des actifs financiers et pour ses travaux sur la théorie de la formation des prix des avoirs financiers. Il s'est aussi engagé dans ses recherches dans la voie ouverte par Harry Markowitz. Ce dernier avait en effet élaboré une procédure complexe de sélection des titres boursiers afin d'optimiser un portefeuille de placements. Mais la mise en oeuvre de ce modèle a très vite posé des problèmes d'ordre pratique, au point que la collecte des informations nécessaires et leur traitement devenaient presque impossibles avec les ordinateurs disponibles dans les années 1960. C'est la raison pour laquelle William Sharpe se mit à chercher une méthode de sélection des portefeuilles efficients plus simple. Il découvre que les variations de la rentabilité de chaque titre sont liées, linéairement, à la variation du marché dans son ensemble, mesurée par l'indice du marché concerné (par exemple l'indice Standard \& Poor 500 aux États-Unis, ou le C.A.C. 40 en France). Le nombre de statistiques nécessaires s'en est trouvé fortement réduit: $302$ statistiques au lieu de $3'150$ dans le modèle Markowitz pour $100$ titres, $602$ au lieu de $20'300$ pour $200$ titres et $10'002$ au lieu de $125'750$ pour $300$ titres, le calcul fut aussitôt facilité. C'est à partir de ce concept, simple en apparence, que Sharpe découvre ensuite le fameux coefficient Bêta reliant la rentabilité d'un titre à celle de l'indice du marché et constituant une mesure du risque associé à la volatilité du marché. Au-delà de leur apport pratique, les travaux de Sharpe ont contribué de façon décisive à la formulation d'une théorie de la formation des cours des actifs financiers plus connue sous le nom de "modèle C.A.P." (Capital Asset Pricing) ou, en français, de "Modèle d'équilibre des actifs financiers" (MEDAF).

\parpic[l][t]{%
  \begin{minipage}{40mm}
    \fbox{\includegraphics[width=110px,height=140px]{img/medaillons/smith.eps}}
  \end{minipage}
}
\textbf{Smith, Adam} (1723-1790) Né à Kirkcaldy et décédé à Edimbourg, en Écosse, c'était un économiste et philosophe. Il étudia aux Universités de Glasgow et Oxford. De 1748 à 1751, il enseigna la rhétorique et les belles-lettres à Édimbourg. Durant cette période, il se lia avec le philosophe David Hume, dont la pensée exerça une grande influence sur les conceptions de Smith en matière d'éthique et d'économie. Smith fut nommé professeur de logique en 1751 puis professeur de philosophie morale en 1752 à l'Université de Glasgow. Plus tard, il rassembla les cours d'éthique qu'il dispensait et les publia dans sa première oeuvre maîtresse intitulée \textit{Theory of Moral Sentiments}, en 1759. En 1763, il démissionna de son poste de professeur pour accompagner le duc de Buccleuch dans un voyage de 18 mois en France et en Suisse, en qualité de précepteur. De 1766 à 1776, il vécut à Kirkcaldy où il travailla à son ouvrage fondamental, la \textit{The Wealth of Nations}. Smith fut ensuite nommé commissaire des douanes à Édimbourg en 1778, poste qu'il occupa jusqu'à sa mort. En 1787, il fut également nommé recteur de l'université de Glasgow. Son célèbre traité \textit{An Inquiry into the Nature and Causes of the Wealth of Nations} (1776), première étude tentant de décrire la nature du capital et le développement historique de l'industrie et des échanges entre les pays européens, lui valut d'être considéré comme le père de la science économique moderne. La \textit{The Wealth of Nations} constitue le premier essai traitant de l'histoire de la science économique qui considère l'économie politique comme une discipline autonome, distincte de la science politique, de l'éthique et de la jurisprudence. Smith y propose une analyse du processus de production et de répartition de la richesse, et démontre que les sources principales de tout revenu, c'est-à-dire les formes fondamentales dans lesquelles la richesse est distribuée, sont les rentes, les salaires et les profits. \textit{The Wealth of Nations} affirme contre les physiocrates le principe selon lequel le travail est la source de toute richesse, et présente le développement de l'industrie comme une source d'accroissement de la production. Pour Smith, théoricien du capitalisme libéral, le progrès économique et moral procède de la concurrence, la production et les échanges de biens ne pouvant être stimulés, et en conséquence le niveau de vie général amélioré, que lorsque les gouvernements régulent et contrôlent au minimum les activités industrielles et commerciales individuelles. Pour décrire cette situation, il parle d'un ordre naturel réglé par la "main invisible", qui fait naturellement converger la somme des intérêts individuels vers l'intérêt général. En conséquence, trop d'intervention de l'État dans ce contexte de libre concurrence ne pourrait être que néfaste.

\parpic[l][t]{%
  \begin{minipage}{40mm}
    \fbox{\includegraphics[width=110px,height=140px]{img/medaillons/sommerfeld.eps}}
  \end{minipage}
}
\textbf{Sommerfeld, Arnold} (1868-1951) était un physicien allemand né à Königsberg et décédé à Münch. Il étudia les mathématiques et les sciences naturelles à l'Université de Königsberg où il reçut son doctorat en 1891. Il occupa successivement les chaires de mathématiques à Clausthal (1897), de mathématiques appliquées à Aix-la-Chapelle (1900) et de physique théorique à MÜnich (1906-1931). En 1897, il commença, avec C. F. Klein, un traité en 4 volumes sur le gyroscope, qu'il mit 13 ans à terminer et, à la même époque, fit également des recherches dans d'autres domaines de physique appliquée et d'ingénierie, comme la friction, la lubrification et la radio. On lui doit une amélioration du modèle de Bohr (1916) introduisant des orbites elliptiques et des corrections relativistes. Ce nouveau modèle, qui implique une dépendance de l'énergie vis-à-vis du deuxième nombre quantique, permet d'expliquer la structure fine des raies spectrales émises par les atomes. Sommerfeld introduisit d'ailleurs la fameuse "constante de structure fine". Il s'intéressa également après Drude et Lorenz au modèle des électrons libres qui explique certaines propriétés des métaux, en particulier la conduction, en considérant un comportement quantique des électrons. Il participa ainsi aux développements de la théorie des bandes en physique du solide, formulant en 1928 l'idée selon laquelle les électrons occupent des états quantifiés dans la matière.

\parpic[l][t]{%
  \begin{minipage}{40mm}
    \fbox{\includegraphics[width=110px,height=140px]{img/medaillons/stokes.eps}}
  \end{minipage}
}
\textbf{Stokes, George Gabriel} (1819-1903) était un mathématicien et physicien né en Irlande et décédé à Cambridge. En 1841, il reçoit son diplôme avec mention d'honneur de l'Université de Cambridge et entame une carrière de chercheur. Influencé par son ancien professeur, il se consacre à l'étude des fluides visqueux. Il publie en 1845 le résultat de ses travaux sur les mouvements des fluides dans sa thèse \textit{On the theories of the internal friction of fluids in motion}. Son approche mathématique décrivant l'écoulement d'un fluide newtonien imcompressible dans un espace tridimensionel, en ajoutant une force de viscosité à partir des équations d'Euler (Principes généraux du mouvement des fluides, 1755), est à l'origine des équations de Navier-Stokes. L'ensemble de ses recherches est synthétisé par son traité \textit{Report on recent research in Hydrodynamics}, paru en 1846, texte fondateur de l'hydrodynamique. Il devient dès 1849 professeur à la chaire de mathématique de cette même université. Élu en 1851 à la Royal Society, il en sera le président de 1885 à 1890. Les trois derniers postes cités avaient été occupés par Isaac Newton. Il est lauréat du prix Smith en 1841, de la médaille Rumford en 1852 et de la médaille Copley en 1893.

\parpic[l][t]{%
  \begin{minipage}{40mm}
    \fbox{\includegraphics[width=110px,height=140px]{img/medaillons/stefan.eps}}
  \end{minipage}
}
\textbf{Stefan, Josef} (1835-1893) était un physicien autrichien né à Sankt Peter près de Klagenfurt et décédé à Vienne. Les travaux originaux de Stefan comprennent la théorie cinétique des gaz, l'hydrodynamique et surtout la théorie du rayonnement. Après des études à l'Université de Vienne où il obtient son doctorat en 1858, nommé Privatdozent de physique-mathématique, il devient professeur de physique en 1863, puis directeur de l'Institut de Physique (1866). Membre de l'Académie des Sciences de Vienne, il en est le secrétaire à partir de 1875. Avant les travaux de Stefan, G. R. Kirchhoff avait déjà décrit les propriétés du "corps parfaitement noir", susceptible d'absorber la totalité du rayonnement incident et d'émettre un spectre étendu de longueurs d'ondes. Stefan démontre empiriquement en 1879 que l'intensité du rayonnement du corps noir est proportionnelle à la quatrième puissance de sa température absolue, relation connue depuis sous le nom de "loi de Stefan-Boltzmann", Boltzmann l'ayant déduite en 1884 de considérations thermodynamiques. Cette loi constitue l'une des premières étapes importantes qui ont conduit à l'interprétation du rayonnement du corps noir et à la théorie quantique du rayonnement.

\parpic[l][t]{%
  \begin{minipage}{40mm}
    \fbox{\includegraphics[width=110px,height=140px]{img/medaillons/sturm.eps}}
  \end{minipage}
}
\textbf{Sturm, Charles François} (1803-1855) Après avoir été étudiant à l'Université de Genève (sa ville natale), Sturm se rend, pour être précepteur dans la famille De Broglie, à Paris, où il fréquente les plus grands savants de l'époque et où il se fixe définitivement à partir de 1825. Il détermine en 1826 la vitesse de propagation du son dans l'eau, ce qui lui vaut, l'année suivante, le grand prix de mathématiques proposé pour le meilleur mémoire sur la compressibilité des liquides. En 1829, il énonce le célèbre théorème qui porte son nom, essentiel pour l'étude des propriétés des racines d'une équation algébrique et qui précise le nombre de racines réelles d'une équation numérique comprises entre deux limites données. Il publie la démonstration de ce théorème en 1835. À partir de 1830, en liaison avec son ami Liouville, il aborde le problème de la théorie générale des oscillations et étudie des équations différentielles du second ordre (problème de Sturm-Liouville) dans plusieurs articles, dont \textit{Sur les équations différentielles linéaires du second ordre} (1836) et\textit{ Sur une classe d'équations à différences partielles} (1836). Les méthodes employées seront à l'origine de nombreux travaux et découvertes mathématiques. Il est élu en 1836 à l'Académie des Sciences et travaille à l'École Polytechnique. Succédant à Poisson, il enseigne, à partir de 1840, à la faculté des sciences de Paris (chaire de mécanique). Ses \textit{Cours d'analyse de l'École polytechnique} (1857-1863) et ses \textit{Cours de mécanique de l'École polytechnique} (1861) seront publiés après son décès à Paris

\phantomsection
\addcontentsline{toc}{section}{T}
\label{sec:T}

\parpic[l][t]{%
  \begin{minipage}{40mm}
    \fbox{\includegraphics[width=110px,height=140px]{img/medaillons/taylor.eps}}
  \end{minipage}
}
\textbf{Taylor, Brook} (1685-1731) était un mathématicien anglais né à Edmonton et décédé à Londres, célèbre pour ses contributions au développement du calcul infinitésimal. Taylor fit ses études au collège Saint John, à Cambridge. Il obtint, en 1708, une remarquable solution du problème du centre d'oscillation, qui pourtant demeura inédite jusqu'en 1714 lorsque son droit de priorité lui fut contesté par Jean Bernoulli. L'ouvrage de Taylor, \textit{Methodus incrementorum directa et inversa} (1715), ajoute aux mathématiques supérieures un nouveau chapitre, que l'on appelle de nos jours le "calcul des différences finies". Entre autres applications ingénieuses, il s'en sert pour déterminer la forme du mouvement d'une corde vibrante en le réduisant avec succès aux principes de la mécanique. Le même ouvrage contient la célèbre formule connue sous le nom de "théorème de Taylor", dont l'importance n'apparut qu'en 1772, quand Louis de Lagrange réalisa sa puissance et en fit le principe fondamental du calcul différentiel. Dans son essai \textit{Linear Perspective}, Taylor pose les principes de l'art sous une forme originale et plus générale qu'aucun de ses prédécesseurs. Mais l'ouvrage souffrit de la confusion et du manque de clarté qui affectaient la plupart de ses écrits. Taylor fut élu membre de la Royal Society en 1712. Il siégea la même année au comité chargé de régler les querelles de priorité entre Newton et Leibniz et fut secrétaire de la société de 1714 à 1718. À partir de 1715, ses recherches prirent une orientation philosophique et religieuse.

\parpic[l][t]{%
  \begin{minipage}{40mm}
    \fbox{\includegraphics[width=110px,height=140px]{img/medaillons/teller.eps}}
  \end{minipage}
}
\textbf{Teller, Edward} (1908-2003) était un physicien nucléaire né à Budapest et décédé à Stanford. Il quitte Budapest en 1926 pour aller à Karlsruhe (Allemagne), afin d'étudier la chimie, mais très vite une affinité se créera avec la nouvelle théorie de la physique quantique ce qui l'amènera à étudier à l'Université de Leipzig où il obtiendra son doctorat à l'âge de 22 ans. Teller obtint ce titre sous la direction de Werner Heisenberg qui participa plus tard activement dans le camp des nationalistes allemands lors de la seconde guerre mondiale. En 1935, Teller s'expatria aux États-Unis et ses compétences dans la physique de pointe l'amenèrent à se faire beaucoup de relations et une très bonne réputation dans la communauté scientifique. Il fut ainsi nommé professeur dans de nombreuses universités américaines et travailla en 1942 au projet Manhattan où il mena des travaux très importants qui permirent de créer la première bombe nucléaire à fission. Le travail effectué, Teller soutint la continuité du travail pour la recherche d'une bombe thermonucléaire par peur de l'avancée des Russes dans ce domaine (Teller était anticommuniste et très bon ami de Landau qui se fit arrêter par la police communiste). Teller réussit à convaincre l'administration américaine à financer les recherches pour une bombe à hydrogène et mena les travaux avec succès qui fait qu'on le considère aujourd'hui comme le père de la bombe H.

\parpic[l][t]{%
  \begin{minipage}{40mm}
    \fbox{\includegraphics[width=110px,height=140px]{img/medaillons/tesla.eps}}
  \end{minipage}
}
\textbf{Tesla, Nikola} (1856-1943) était un inventeur et ingénieur serbe de génie dans le domaine de l'électricité décédé à New York. Il est souvent considéré comme l'un des plus grands scientifiques dans l'histoire de la technologie, pour avoir déposé plus de 900 brevets (qui sont pour la plupart repris au compte de Thomas Edison) traitant de nouvelles méthodes pour aborder la conversion de l'énergie. En 1875, il entre à l'École Polytechnique de Graz, en Autriche, où il étudie la mathématique, la physique et la mécanique. Une bourse lui est attribuée par l'administration des Confins Militaires (Vojna Krajina), le mettant à l'abri des problèmes d'argent. Ceci ne l'empêche cependant pas de travailler avec acharnement pour assimiler le programme des deux premières années d'études en un an. L'année suivante, la suppression des Confins Militaires retire toute aide financière à Tesla, hormis celle, très maigre, que peut lui apporter son père, ce qui ne lui permet pas d'achever sa seconde année d'études. On lui doit le moteur électrique asynchrone, l'alternateur polyphasé, le montage triphasé en étoile, la commutatrice. Tesla découvre le principe de la réflexion des ondes sur les objets en 1900, il étudie et publie, malgré des problèmes financiers, les bases de ce qui deviendra presque trois décennies plus tard le radar.

\parpic[l][t]{%
  \begin{minipage}{40mm}
    \fbox{\includegraphics[width=110px,height=140px]{img/medaillons/thom.eps}}
  \end{minipage}
}
\textbf{Thom, René} (1923-2002) était un mathématicien français auteur d'importants travaux en topologie différentielle. Né à Montbéliard et décédé à Bures-sur-Yvette, Thom fut élève de l'École Normale Supérieure. En 1958, il a reçu la médaille Fields pour sa théorie du cobordisme (relation d'équivalence entre variétés différentielles compactes). Dans une communication au colloque de Strasbourg (1951), Thom établit que, si les zéros d'un idéal polynomial forment une variété, c'est une variété bordante, et sa thèse, \textit{Espaces fibrés en sphères et carrés de Steenrod} (1951), contient déjà en germe les principales méthodes cobordistes. C'est dans le dernier chapitre d'un mémoire de 1954 (\textit{Quelques Propriétés globales des variétés différentiables}) que la théorie du cobordisme est exposée pour la première fois. Après 1955, Thom a surtout étudié les espaces feuilletés et les ensembles et morphismes stratifiés. On lui doit des résultats sur les approximations des transformations différentiables et leurs singularités, les comparaisons de structures différentiables sur une variété triangulée et une théorie de Morse pour les variétés feuilletées. Il est également l'un des premiers à avoir utilisé les techniques de "chirurgie" des variétés. Depuis 1969, Thom s'est consacré aux applications de la topologie aux phénomènes de la vie. Pour décrire la naissance et l'évolution des formes, il a élaboré une mathématique spécifique: sa théorie des catastrophes est une théorie des singularités de certaines équations différentielles. Concrètement, elle permet, à partir de phénomènes observés, de remonter à leurs causes inconnues, au moins partiellement. Thom a donné un exposé de ses travaux dans l'ouvrage Stabilité structurelle et morphogenèse (1973).

\parpic[l][t]{%
  \begin{minipage}{40mm}
    \fbox{\includegraphics[width=110px,height=140px]{img/medaillons/thales.eps}}
  \end{minipage}
}
\textbf{Thalès de Milet} ($\sim$624 av. J.-C. - $\sim$524 av. J.-C.) était le premier mathématicien dont l'histoire ait retenu le nom. Il est né à Milet, en Asie mineure, sur les côtes méditerranéennes de l'actuelle Turquie. Plus qu'un simple mathématicien, Thalès était un savant universel, curieux de tout, astronome et philosophe, très observateur. On ne démontrait pas ce qu'on avançait à l'époque de Thalès, on ne faisait que remarquer certaines propriétés. Mais la façon qu'avait Thalès de réfléchir, d'analyser des situations, d'en rechercher les causes font de lui le précurseur des scientifiques (il s'en tenait à l'observation et à l'expérimentation). Une de ses grandes interrogations était l'eau, et les causes de la pluie. Il avait remarqué que l'air se transformait en pluie, et il en cherchait désespérément les réponses. Thalès a formulé plusieurs propriétés géométriques qu'il tenait peut-être des Égyptiens et dont les premières traces de démonstration connues sont bien ultérieures mais, ce faisant, il pose les premiers jalons du raisonnement sur des figures géométriques idéales grâce auquel il obtint plusieurs résultats connus sous le nom de "théorèmes de Thalès". Mais le fait d'armes de Thalès est sans conteste la prévision d'une éclipse du Soleil, probablement celle du 8 mai 585 avant notre ère. On lui doit notamment la première connaissance de l'électricité, grâce à deux expériences. Il remarqua d'abord que l'ambre avait la propriété d'attirer les matériaux légers. Une autre expérience réalisée en Magnésie..., vers -600, lui permet de mettre en évidence les propriétés d'aimantation de l'oxyde de Fer.

\parpic[l][t]{%
  \begin{minipage}{40mm}
    \fbox{\includegraphics[width=110px,height=140px]{img/medaillons/turing.eps}}
  \end{minipage}
}
\textbf{Turing, Alan} (1912-1954) Par ses travaux théoriques dans les domaines de la logique et des probabilités, Turing est considéré, sinon comme le fondateur des ordinateurs, en tout cas, comme l'un des pères spirituels de l'intelligence artificielle. Né à Paddington (Londres) Turing connaît une scolarité sans éclat malgré un esprit brillant et de nettes dispositions pour les sciences. En 1928, à la Sherborne School où il est entré deux ans plus tôt, il fait une rencontre qui provoque en lui un déclic et l'amène à s'intéresser réellement à la science et plus exactement aux mathématiques. De 1931 à 1934, Alan Turing est étudiant en mathématiques au King's College de l'Université de Cambridge. Au cours de cette période, il prend connaissance des travaux de John von Neumann sur la mécanique quantique. Stimulé par ces recherches, il se lance dans l'étude de problèmes de probabilités et de logique. C'est aussi au King's College qu'il rencontre des théoriciens de l'économie comme John Keynes. Diplôme en poche, il apprend à l'été 1936 les avancées de Max Newman concernant l'élaboration d'une théorie mathématique sur l'incomplétude de Gödel et la question de la décidabilité de Hilbert. Si pour beaucoup de propositions, il est facile de trouver un algorithme, qu'en est-il de celles pour lesquelles l'algorithme, pas assez rigoureux, est insuffisant à valider la proposition? Doit-on en déduire qu'elles ne peuvent être validée? C'est désormais dans ce sens que vont s'orienter les recherches de Turing. En 1936 il reçoit le prix Smith pour ses travaux sur les probabilités et le concept de la "Machine de Turing". Ce concept constitue la base de toutes les théories sur les automates et plus généralement celle de la théorie de la calculabilité. Il s'agit en fait de formaliser le principe d'algorithme, représenté par une succession d'instructions – agissant en séquence sur des données d'entrée – susceptible de fournir un résultat. Cette formalisation oblige Turing à développer la notion de calculabilité et à déterminer des classes de problèmes "décidables". Cela le conduit à introduire une nouvelle classe de fonctions: les "fonctions calculables au sens de Turing". Au cours de son doctorat à l'Université de Princeton, de 1936 à 1938, Turing conçoit l'idée de la construction d'un ordinateur. De retour à Cambridge, il poursuit ses études mathématiques et s'intéresse à la fonction zêta de Riemann. La seconde Guerre Mondiale lui offre bientôt l'opportunité de mettre en pratique ses théories. C'est au département des communications du Ministère des affaires étrangères britannique qu'il se retrouve confronté au secret d'Enigma, nom de code de la machine utilisée par la marine allemande pour communiquer avec leurs sous-marins. Le cryptage utilisé par les Nazis échappait toujours aux modes d'investigation classiques. Mais avec la collaboration de W. G. Welchman, Turing réussit à percer le code en appliquant sa nouvelle méthode et, de façon indirecte, contribue ainsi à la victoire de la bataille de l'Atlantique. La guerre achevée, Turing intègre le National Physical Laboratory de Grande-Bretagne où il entreprend, en concurrence avec les projets américains, de créer le premier ordinateur. Les avancées technologiques lui laissent entrevoir la réalisation de cet objectif dans un avenir proche. En 1948, grâce à Newman, il obtient un poste de chargé de cours en mathématiques à l'Université de Manchester qu'il occupera jusqu'à la fin de sa vie. Deux ans plus tard, il participe avec Frederic Williams et Tom Kilburn à la réalisation d'un calculateur électronique, le Mark I, et conçoit à cette occasion un manuel de programmation. Dans la foulée, il publie \textit{Can a machine think ?} dans lequel il fait la synthèse des bases mathématiques et conceptuelles de l'ordinateur électronique programmable et résume sa philosophie de la "machine intelligente". Il énonce également le célèbre "Test de Turing" qui se résume à une expérience dans laquelle un homme tient une conversation avec une machine. Comment dans ce cas, un observateur, par l'unique analyse des messages échangés, pourra-t-il distinguer l'homme de la machine? Turing était convaincu que tout n'était qu'un problème d'information et que le développement des technologies permettrait d'ici 50 ans aux machines de tenir en échec l'être humain au moins cinq minutes. Turing se suicida par empoisonnement au cyanure à cause des pressions homophobes qu'il subissait au Royaume-Uni.

\phantomsection
\addcontentsline{toc}{section}{V}
\label{sec:V}

\parpic[l][t]{%
  \begin{minipage}{40mm}
    \fbox{\includegraphics[width=110px,height=140px]{img/medaillons/vanderwaals.eps}}
  \end{minipage}
}
\textbf{Van Der Waals, Johannes Diderik} (1837-1923) était un physicien néerlandais né à Leyde et décédé à Amsterdam. Van Der Waals fut tout d'abord instituteur dès l'âge de 20 ans avant de devenir, à la suite d'efforts solitaires, professeur dans l'enseignement moyen (1863). Il fréquenta les cours de l'Université de Leyde de 1862 à 1865 et enseigna la physique à Deventer et à La Haye (1866). En 1873, il fut reçu docteur par l'Université de Leyde, après la défense d'une dissertation intitulée: \textit{Over de continuiteit van den gas en vloeistoftoestand} qui contient la présentation de l'équation d'état qui porte son nom et conduit à des résultats beaucoup plus satisfaisants que l'équation classique des gaz parfaits au voisinage de la zone de liquéfaction. Cette étude contribua d'une façon décisive à accréditer l'idée de l'existence de forces intermoléculaires d'attraction et à déterminer le rôle du volume d'encombrement moléculaire dans le comportement des gaz à haute pression, deux concepts encore mal assurés à l'époque. Le succès rapide de la nouvelle théorie est illustré par les multiples traductions de la dissertation originale qui suivirent sa présentation. On sait à présent que l'équation de Van der Waals est encore imparfaite et qu'il serait téméraire de vouloir lui conserver le nom "d'équation des gaz réels" qui lui fut naguère attribué. En effet, des équations d'état encore mieux appropriées permettent d'atteindre aujourd'hui une approximation plus complète qui sont en général déduites de considérations de cinétique moléculaire fondées sur le théorème du viriel des forces. De 1877 à 1907, date de sa retraite, Van der Waals occupa la chaire de physique à l'Université d'Amsterdam. C'est pendant cette période qu'il fit connaître sa loi dite "loi des états correspondants" (1880). Cette équation d'état unique pour tous les corps purs contribua largement, elle aussi, à sa renommée, car elle servit par la suite de guide aux essais préalables à la liquéfaction de l'hydrogène et de l'hélium. D'un autre point de vue, cette contribution de Van der Waals est également considérée comme l'une des premières tentatives pour exprimer des lois de la physique en fonction de variables réduites. Parmi les autres travaux de Van der Waals, citons une contribution à la théorie moléculaire des mélanges binaires et l'étude de la capillarité. Le prix Nobel de physique lui a été décerné en 1910 pour ses travaux concernant l'équation de l'état d'agrégation des gaz et des liquides.

\parpic[l][t]{
  \begin{minipage}{40mm}
    \fbox{\includegraphics[width=110px,height=140px]{img/medaillons/viete.eps}}
  \end{minipage}
}
\textbf{Viète, François} (1540-1603) Né à Fontenay-le-Comte et décédé à Paris. Viète est célèbre aujourd'hui en tant qu'inventeur de l'algèbre moderne. Or, à son époque, il était plus connu comme maître des requêtes et conseiller privé d'Henri IV que comme mathématicien. Toute sa vie est en effet marquée par cette dualité d'une carrière politique brillante et d'un ardent travail de cabinet sur les plus hauts problèmes posés par la mathématique de son siècle. Son oeuvre scientifique a beaucoup souffert de ses nombreuses occupations politiques et du peu de temps qu'elles lui laissaient. Il reste néanmoins que la contribution de Viète au développement des mathématiques à la fin du 16ème siècle est fort importante. Elle se caractérise par l'introduction systématique de la représentation littérale dans les problèmes algébriques, tant pour les inconnues que pour les quantités connues, ce qui présente le principal avantage de traiter le cas général et non les cas particuliers et de s'intéresser à la structure des problèmes plutôt qu'à leur expression. Dans sa jeunesse Viète est l'élève des franciscains, au collège des Cordeliers. Il poursuit ses études de droit à la faculté de Poitiers et entra dans la vie active comme avocat. Il est nommé conseiller au parlement de Bretagne en 1573, il y séjourne en fait assez peu, occupé qu'il est par ses travaux mathématiques et les missions confidentielles que lui confie le roi. On retrouve ensuite sa trace à Paris en 1579 où il publie le Canon mathematicus, accompagné du Liber singularis. Nommé maître des requêtes de l'hôtel du roi en 1580, il est démis de sa fonction en 1585, à la suite de conflits de personnes. En 1589, il est à Tours et prépare la publication de son oeuvre scientifique. Il s'occupe également de cryptographie statistique pour le compte du roi. Il regagne Paris avec ce dernier et est nommé conseiller privé. Viète décédera après une assez longue période de déclin du à la maladie.


\phantomsection
\addcontentsline{toc}{section}{W}
\label{sec:W}

\parpic[l][t]{
  \begin{minipage}{40mm}
    \fbox{\includegraphics[width=110px,height=140px]{img/medaillons/walras.eps}}
  \end{minipage}
}
\textbf{Walras, Leon} (1834-1910) était un économiste français né à Évreux (France) et décédé à Clarens (Suisse). Il est le fils d'Auguste Walras, un économiste français dont la pensée influencera beaucoup celle de son fils, dans le domaine de la réforme sociale en général et foncière en particulier. Il étudie au collège de Caen en 1844, puis au lycée de Douai en 1850. Il est diplômé bachelier-ès-lettres en 1851 et bachelier-ès-sciences en 1853. La même année, il n'est pas déclaré admissible à l'École polytechnique et ce aussi lors d'un second essai. En 1854, il est reçu élève externe à l'École des Mines de Paris, mais il n'a pas d’intérêt pour la formation d'ingénieur et il abandonne cette école. Nommé professeur à l'université de Lausanne, Walras dénonça à partir des années 1870, les théories économiques libérales alors enseignées dans les universités, qu'il jugeait incapables de rendre compte des problèmes économiques de son temps. Dans ses Éléments d'économie politique pure (1874), sa critique vise en particulier les théories de la valeur travail et de la rente foncière mais à travers lui c'est tout l'héritage classique qu'il remet en cause (notamment celui d'Adam Smith). Influencé par le mathématicien Antoine Cournot, il est l'un des premiers à introduire de manière systématique le calcul mathématique en économie. Walras place l'entreprise au coeur de l'économie et s'intéresse à son action dans le cadre d'une concurrence entre agents, ainsi que dans celui d'une interdépendance de tous les marchés économiques: les marchés des produits (biens et services) et ceux des facteurs de production (notamment la terre, le travail et les capitaux). Il se demande comment se fixent les prix et les quantités de façon simultanée, et pose le problème de l'équilibre général, c'est-à-dire de la stabilité des équilibres sur tous les marchés. L'attention portée à cette question caractérise les membres de l'École de Lausanne, en particulier le successeur de Walras, Vilfredo Pareto. Avec l'Autrichien Carl Menger et le Britannique Stanley Jevons, qu'il ne connaissait pas au moment où il s'engageait sur cette voie, il est considéré comme l'un des fondateurs du courant néoclassique et du marginalisme.

\parpic[l][t]{
  \begin{minipage}{40mm}
    \fbox{\includegraphics[width=110px,height=140px]{img/medaillons/weber.eps}}
  \end{minipage}
}
\textbf{Weber, Wilhelm} (1804-1891) était un physicien allemand né à Wittenber et décédé à Göttingen qui se spécialisa en électrodynamique. Weber écrivit, en 1824, un traité sur le mouvement ondulatoire avec son frère aîné, Ernst Heinrich Weber, anatomiste réputé, et étudia, avec son frère cadet Eduard Friedrich Weber le mécanisme de la marche (1836). À Göttingen, il collabora avec Carl Friedrich Gauss pour l'étude du géomagnétisme, et il relia leurs laboratoires par un télégraphe électrique: ce fut l'une des premières transmissions par télégraphe que l'on connaisse. Sa réalisation majeure fut celle qu'il mena à Leipzig, avec F.W.G. Kohlrausch: il détermina le rapport des unités de charge électrostatiques et électrodynamiques (la constante de Weber) qui se révéla être l'équivalent d'une vitesse, et fut utilisé plus tard par James Clerk Maxwell pour renforcer sa théorie sur l'électromagnétisme.

\parpic[l][t]{
  \begin{minipage}{40mm}
    \fbox{\includegraphics[width=110px,height=140px]{img/medaillons/weierstrass.eps}}
  \end{minipage}
}
\textbf{Weierstrass, Karl Theodor Wilhelm} (1815-1897) était un mathématicien allemand, qui donna à la théorie des fonctions sa forme moderne en précisant en particulier le formalisme des limites et est considéré à ce titre comme le père de l'analyse moderne. Né à Ostenfelde, il fit ses études à Bonn et à Münster où il fut instituteur. C'est là qu'il s'intéressa aux mathématiques, et plus particulièrement à l'étude des fonctions elliptiques. Pendant de nombreuses années, Weierstrass travailla dans l'ombre pour établir sa théorie des fonctions de variable complexe, qui repose sur les développements en série entière. En 1854, il publia un mémoire sur les intégrales abéliennes et sur l'inversion des intégrales hyperelliptiques, qui établit sa réputation comme mathématicien et lui valut un doctorat honoraire de l'Université de Königsberg. Nommé professeur à l'Université de Berlin, il enseigna de 1864 à sa mort. Il a peu publié de son vivant et sa réputation est venue principalement de l'influence de ses cours à Berlin. Ceux-ci furent suivis par de nombreux mathématiciens et établirent la théorie des fonctions sur des bases de rigueur auxquelles son nom reste attaché, la "rigueur weierstrassienne". Il est aussi connu pour avoir rendu public un exemple de fonction continue nulle part dérivable (fonction de Weierstrass).

\parpic[l][t]{
  \begin{minipage}{40mm}
    \fbox{\includegraphics[width=110px,height=140px]{img/medaillons/weyl.eps}}
  \end{minipage}
}
\textbf{Weyl, Hermann} (1885-1955)  est un des mathématiciens les plus influents du 20ème siècle, l'un des premiers à combiner la Relativité Générale avec les lois de l'électromagnétisme. Ses recherches en mathématiques portèrent essentiellement sur la topologie et la géométrie. Il effectua des recherches en mécanique quantique et en théorie des nombres. Né à Elmshorn à proximité de Hambourg en Allemagne, Weyl étudia de 1904 à 1908 à Göttingen et à Münich, principalement intéressé par la mathématique et la physique. Son doctorat fut soutenu à Göttingen sous la direction de Hilbert et Minkowski. En 1910, il obtint un poste d'enseignant comme lecteur privé à Göttingen. Il enseigna la mathématique à l'École Polytechnique Fédérale de Zürich en Suisse en 1913. C'est à Princeton qu'il travailla avec Einstein. Weyl rechercha une unification de la gravitation et de l'électromagnétisme. Cette recherche donna des explications de la violation de la non-conservation de la parité, une caractéristique des interactions faibles. Weyl continua à travailler à l'IAS (Institude of Advanced Studies) jusqu'à sa retraite en 1952 ; il mourut à Zürich. En 1918, il introduit la notion de jauge, première étape de ce qui deviendra la théorie de jauge. En réalité, sa vision était une tentative non réussie de modéliser les champs électromagnétiques et gravitationnels comme des propriétés géométriques de l'espace-temps. Ces travaux se révélèrent fondamentaux pour comprendre la symétrie des lois de la mécanique quantique. Il en posa les bases, donnant naissance aux spineurs, devenus relativement familiers autour des années 1930.

\parpic[l][t]{
  \begin{minipage}{40mm}
    \fbox{\includegraphics[width=110px,height=140px]{img/medaillons/weinberg.eps}}
  \end{minipage}
}
\textbf{Weinberg, Steven} (1933-) Né à New York il débuta ses études à New York même puis à l'Université Cornell (dans l'État de New York) et soutint, en 1957 à Princeton, sa thèse sur les effets de l'interaction forte dans les processus dominés par l'interaction faible. Chercheur à l'Université de Californie à Berkeley de 1959 à 1966, il s'intéressa à de multiples problèmes en théorie quantique des champs, en physique des particules et en astrophysique. Professeur à Harvard à partir de 1973, il contribua de façon décisive à la compréhension moderne des interactions fondamentales. Il rejoignit l'Université du Texas à Austin en 1982. L'unification des forces fondamentales a sous-tendu les efforts des physiciens modernes depuis Newton, Maxwell et Einstein qui, après avoir uni l'espace et le temps, tenta, mais en vain, d'englober en une seule théorie gravitation et électromagnétisme. La découverte, au début du 20ème siècle, des deux forces nucléaires, les interactions faible et forte, donna un nouvel élan à ces tentatives. En 1967, Weinberg et le physicien pakistanais Abdus Salam proposèrent, indépendamment, que l'électromagnétisme et l'interaction nucléaire faible soient issus d'une même interaction électrofaible, dont la symétrie de jauge est spontanément brisée et dont le vecteur est un triplet de bosons massifs et le photon. Quelques années plus tard des expériences au CERN de Genève apportaient les premières confirmations du modèle de Weinberg-Salam. Le prix Nobel de physique 1979 (partagé avec l'Américain Sheldon Lee Glashow, pour l'importance de ses travaux de précurseur) récompensa les deux auteurs de ce qu'on appelle maintenant le "modèle standard" des interactions électrofaibles. Pédagogue, Weinberg est l'auteur de plusieurs cours de physique de haut niveau, tant sur la gravitation que sur la théorie des champs. Vulgarisateur de talent, son livre\textit{ Les Trois Premières Minutes de l'Univers} fut un succès mondial.

\parpic[l][t]{
  \begin{minipage}{40mm}
    \fbox{\includegraphics[width=110px,height=140px]{img/medaillons/wilcoxon.eps}}
  \end{minipage}
}
\textbf{Wilcoxon, Frank} (1892-1965) était un chimiste et statisticien connu pour le développement de tests statistiques très répandus! Frank Wilcoxon est né de parents américains à County Cork en Irelande. Il a grandi a Catskill, New York mais a suivi une partie de sa scolarité en Angleterre. En 1917, il est diplômé du collège militaire de Pennsilvanie avec une licence. Après la première guerre mondiale, il débute des cours de maîtrise à l'Université de Rugters où il obtint sa maîtrise en chimie en 1922 et ensuite à l'Université de Cornell où il obtint son doctorat en chimie physique en 1924. Wilcoxon commença sa carrière de chercheur à l'Institut Boyce Thompson en 1925 et y resta jusqu'en 1941. Ensuite, il prit un poste dans la compagnie Atlas Powder où mis en place et dirigea le laboratoire de contrôle avant de joindre la compagnie chimique American Cyanimid en 1943. Pendant cette période il développa un intérêt pour la statistique inférentielle à travers les lectures des textes de R.A. Fisher de 1925. Il prit sa retraite en 1957. Pendant sa carrière, Wilcoxon publia 70 articles, le plus connu étant celui contenant les deux tests statistiques qui portent ce nom: le test de la somme des rangs de Wilcoxon et le test de la somme des rangs signés de Wilcoxon. Il s'agit d'alternatives non paramétriques aux tests-$T$ de Student. Wilcoxon mourut après une brève maladie.

\parpic[l][t]{
  \begin{minipage}{40mm}
    \fbox{\includegraphics[width=110px,height=140px]{img/medaillons/witten.eps}}
  \end{minipage}
}
\textbf{Witten, Edward} (1951-) est un mathématicien et physicien, lauréat de la médaille Fields en 1990. Né à Baltimore (Maryland), Witten fait ses études supérieures à l'Université Brandeis à Waltham (Massachusetts), puis à l'Université de Princeton (New Jersey), où il soutient sa thèse de doctorat en physique en 1974. Chercheur à l'Université Harvard de 1976 à 1980, il enseigne ensuite à l'Université de Princeton, puis devient membre de l'Institute for Advanced Study de Princeton en 1987. Après des travaux en physique théorique des particules élémentaires, Witten axe ses recherches sur la physique-mathématique et contribue en particulier de façon déterminante au développement des théories des supercordes dans l'espoir que celles-ci pourraient émerger vers une compréhension de l'interaction gravitationnelle au niveau quantique. En mathématiques, il a contribué à l'étude de la théorie de Morse, démontrant les inégalités classiques de Morse en reliant les points critiques à l'homologie. En 1987, il démontre une suite infinie de théorèmes de rigidité sur l'espace des solutions d'équations différentielles, telles que l'équation de Rarita-Schwinger, rencontrées en physique. En théorie des noeuds, il a montré en 1989 qu'on peut interpréter les invariants de noeuds de Vaughan Jones comme des intégrales de Feynman pour une théorie de jauge tridimensionnelle. Il a, de plus, exploré les relations entre la théorie quantique des champs et la topologie différentielle des variétés bi- ou tridimensionnelles. Les progrès récents dans la compréhension des modèles bidimensionnels de la gravitation sont largement dus à l'influence des idées originales de Witten.

\phantomsection
\addcontentsline{toc}{section}{Y}
\label{sec:Y}

\parpic[l][t]{
  \begin{minipage}{40mm}
    \fbox{\includegraphics[width=110px,height=140px]{img/medaillons/yang.eps}}
  \end{minipage}
}
\textbf{Yang, Chen-Ning} (1922-) Professeur à l'Université Chinoise de Hong Kong et à l'Université de Tsinghua à Pékin, professeur émérite de l'Université de New York à Stony Brook, Yang est l'un des plus grands physiciens théoriciens de la seconde moitié du 20ème siècle. Il obtient son Master of Science à l'Université de Tsinghua en 1944. Il s'inscrit en 1946 à l'Université de Chicago que Fermi venait de rejoindre. Plus tard, il décide de se consacrer à la physique théorique et, en 1949, il soutient sa thèse avec un travail sur la phénoménologie des réactions nucléaires. Sa carrière débute à l'Institute for Advanced Studies à Princeton en 1949. En 1965, il refuse de succéder à Oppenheimer comme directeur, mais il décide en 1966 de sortir de sa tour d'ivoire et finit par accepter la chaire Einstein et le poste de directeur de l'Institut de Physique Théorique de la toute nouvelle Université de New York à Stony Brook. À partir de 1971 il s'engage très activement dans le rétablissement des relations scientifiques entre la Chine et les États-Unis et s'implique dans la création de nouveaux instituts de recherche, en particulier à Nankin. Les contributions de Yang se caractérisent par leur profondeur, par l'ampleur et la variété de leur spectre, de la phénoménologie des particules à la théorie quantique des champs, en passant par la mécanique statistique ainsi que par différentes incursions en physique de la matière condensée. Ses travaux sur la brisure de la symétrie par réflexion d'espace (ou violation de la parité) dans les interactions faibles constituent un exemple parfait d'analyse phénoménologique d'une expérience en contradiction avec les idées reçues, à savoir l'absence d'une orientation privilégiée de l'espace dans les lois de la physique. Son grand mérite porte sur deux points: d'une part, il met en évidence le fait que l'hypothèse en question n'avait pas été testée pour les interactions faibles et, d'autre part, il a imaginé tout un ensemble de tests nouveaux pour l'invariance par réflexion d'espace. Ce bond en avant de la théorie des interactions faibles a permis d'aboutir, avec l'introduction des champs de Yang-Mills, au modèle standard électrofaible. L'idée de Yang fut de généraliser l'invariance de jauge aux groupes des rotations dans un espace abstrait à 3 dimensions censé décrire les degrés de liberté interne des champs de matière. Les champs de Yang-Mills s'imposèrent comme outil fondamental pour la construction d'une théorie prédictive de l'ensemble des interactions faibles, fortes et électromagnétiques, événement décisif qui engagea la révolution de la physique des années 1970. L'ensemble de ses travaux ont eu un impact considérable en physique théorique. Près de 20 ans après la publication de son article avec Mills, Yang a donné une reformulation précise de la théorie des champs de Yang-Mills dans le cadre rigoureux des espaces fibrés. L'analogie avec la théorie de la gravitation devient ainsi apparente et les notions de courbure et de transport parallèle s'introduisent naturellement. Des solutions particulières des équations de Yang-Mills, comme celle découverte par Gerard't Hooft, sont utilisées par les mathématiciens pour explorer les propriétés des variétés différentielles à quatre dimensions. Yang a reçu de nombreux prix scientifiques, dont le prix Nobel de physique en 1957 qu'il a partagé avec Tsung-Dao Lee. Ce prix prestigieux leur a été accordé pour leurs travaux sur les lois de la parité dans le domaine des particules élémentaires. Ces travaux fondamentaux sont particulièrement importants parce qu'ils ont montré que la symétrie droite-gauche des particules élémentaires, universellement admise à l'époque, était tout simplement incorrecte, ce qui fut ensuite prouvé expérimentalement.

\parpic[l][t]{
  \begin{minipage}{40mm}
    \fbox{\includegraphics[width=110px,height=140px]{img/medaillons/yukawa.eps}}
  \end{minipage}
}
\textbf{Yukawa Hideki} (1907-1981) était un physicien japonais, né et décédé à Tokyo, il était le cinquième de sept enfants qui devinrent tous des universitaires renommés. Il fut très vite porté vers la mathématique et la philosophie. Admis au département de physique de l'Université de Kyoto en 1926, grand lecteur, Yukawa se passionna vite pour les nouvelles conceptions philosophiques accompagnant la relativité et la théorie des quanta, conceptions qu'il avait découvertes en particulier dans les ouvrages de Max Planck. En marge de ses études, il eut connaissance des développements contemporains de la physique quantique qui aboutirent à sa formulation bien établie vers la fin des années 1920. Il obtint son diplôme à l'Université de Kyoto en 1929 et commença, dès lors, des recherches personnelles dans la double direction de la physique quantique relativiste et de la physique nucléaire qui se dessinait alors. Il s'attacha tout d'abord au problème de la liaison nucléaire électron-proton, le neutron étant une particule encore inconnue, puis à la théorie quantique des champs. Tout en enseignant la physique quantique, Yukawa poursuivait ses recherches sur les problèmes de la physique des noyaux. En 1934, il s'attaqua au problème de la force nucléaire, que la théorie de Fermi était impuissante à résoudre. Il reprit une idée qu'il avait déjà considérée lors de ses premiers travaux, celle d'une force d'échange, transmise entre le neutron et le proton par une particule nouvelle associée à un champ nouveau, dont il se proposait de déduire les propriétés à partir de la force nucléaire. C'est en octobre 1934 qu'il découvrit la solution, en obtenant une relation entre la masse de cette particule d'échange hypothétique et la portée de l'action des forces nucléaires. La particule de Yukawa, le méson, devait avoir une masse valant $200$ fois celle de l'électron. Il fallait supposer que ces mésons étaient de spin entier ou nul, qu'ils obéissaient à la statistique de Bose-Einstein et qu'ils étaient pourvus de charges positive et négative. Ce travail n'attira pas l'attention jusqu'au jour où d'autres chercheurs annoncèrent la découverte d'une particule nouvelle dans le rayonnement cosmique, ayant la masse prévue par Yukawa. Il apparut toutefois que l'interaction de ce méson avec la matière était trop peu intense pour qu'il puisse être la particule d'échange des forces nucléaires. La théorie des deux mésons pallia la difficulté. Il avait découvert entre-temps le mécanisme de désintégration du noyau par capture d'un électron orbital, en appliquant la théorie de Fermi. Il fut le premier Japonais à recevoir le prix Nobel de Physique, en 1949, pour sa théorie mésique des forces nucléaires. Yukawa fonda l'Institut de Recherches de Physique Fondamentale de l'Université de Kyoto et le dirigea jusqu'à sa retraite, en 1970. Il ne se cantonna pas dans une activité de physicien: il écrivit des essais sur la créativité scientifique et milita en faveur de la paix, signant l'appel d'Albert Einstein et de Bertrand Russell contre l'utilisation des armes atomiques.

\parpic[l][t]{
  \begin{minipage}{40mm}
    \fbox{\includegraphics[width=110px,height=140px]{img/medaillons/young.eps}}
  \end{minipage}
}
\textbf{Young, Thomas} (1773-1829) était un physicien, médecin et égyptologue britannique né à Milverton et décédé à Londres, surtout connu pour ses découvertes en optique (phénomènes d'interférence), en élasticité des matériaux et en médecine (explication de la vision colorée).À l'âge de 14 ans il se débrouille déjà dans plus d'une dizaine de langues étrangères. Young commence à étudier la médecine en 1792 à Londres, part en 1794 pour Édimbourg, puis un an plus tard pour Göttingen, où il obtient le titre de docteur en physique en 1796. En 1799, il commença à pratiquer la médecine à Londres. À partir de 1802, et jusqu'à sa mort, il occupa le poste de secrétaire de la Royal Society. En 1811, Young fut nommé à l'hôpital Saint-George de Londres. Il fit partie de plusieurs commissions scientifiques officielles et, à partir de 1818, il fut nommé secrétaire du Bureau des longitudes et éditeur de l'Almanach nautique. En optique, Young découvrit le phénomène des interférences, et contribua ainsi à établir le caractère ondulatoire de la lumière. Il fut le premier à décrire et à mesurer l'astigmatisme et à trouver une explication physiologique à la sensation de couleur. Young est également connu pour ses travaux sur les théories de la capillarité et de l'élasticité. Il contribua également au déchiffrage des hiéroglyphes inscrits sur la pierre de Rosette. Ses écrits comportent d'importants travaux en médecine, en égyptologie et en physique.

\phantomsection
\addcontentsline{toc}{section}{Z}
\label{sec:Z}

\parpic[l][t]{
  \begin{minipage}{40mm}
    \fbox{\includegraphics[width=110px,height=140px]{img/medaillons/zeeman.eps}}
  \end{minipage}
}
\textbf{Zeeman, Pieter} (1865-1943) était un physicien né Zonnemaire et décédé à Amsterdam. Il commença à s'intéresser très jeune à la science. En 1883 lors d'aurores boréales visibles aux Pays-Bas, Zeeman, alors étudiant au collège, fit une description et un dessin détaillé du phénomène qui fut publié dans la revue \textit{Nature}. Après avoir passé ses examens d'entrée en 1885, il étudia la physique à l'Université de Leiden sous la direction de Hendrik Lorentz. En 1890, avant même de terminer sa thèse, il devint l'assistant de Lorentz. Celà lui permit de participer à un programme de recherche sur l'effet Kerr. En 1893 il soumit sa thèse sur l'effet Kerr, la réflexion de la lumière polarisée sur une surface magnétisée. Après avoir obtenu son doctorat il partit pour un semestre à l'institut F. Kohlrausch à Strasbourg. En 1895, après son retour de Strasbourg, Zeeman devint Privatdozent en mathématiques et physique à Leiden. En 1896, trois ans après avoir soumis sa thèse sur l'effet Kerr, il désobéit aux ordres directs de ses supérieurs et utilisa l'équipement du laboratoire pour mesure la séparation des lignes du spectres sous un champ magnétique intense. Il a été licencié pour ses efforts... mais il fut récompensé plus tard: il obtint le prix Nobel de Physique en 1902 pour sa découvert ce qui est connu aujourd'hui sous le nom d'effet Zeeman. En plus de son travail de thèse, dans l'étude de l'effet d'un champ magnétique sur une source de lumière. Grâce à sa découverte, Zeeman si vit offrir un poste d'assistant professeur à Amsterdam en 1897. En 1900 s'ensuivit la place de professeur à l'Université d'Amsterdam. En 1902 avec son mentor Lorentz il se vit attribuer le Prix Nobel de physique pour l'effet Zeeman. Cinq années plus tard, en 1908, il succéda à Van der Waals comme professeur à temps plein et directeur à l'Institut de Physique à Asterdam. Il se retira encore en tant que professeur en 1935.